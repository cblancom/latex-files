\chapter{Enhanced Natural Gas Flow Predictions Using Physics-Guided Neural Networks} \label{cap:non_linealcensnet}

\section{Introduction to Physics-Informed Neural Networks (PINNs)}

Physics-Informed Neural Networks (PINNs) represent a class of neural networks where physical laws are incorporated into the learning process, guiding the model to respect these constraints. Unlike traditional neural networks, where the loss function is typically based on the discrepancy between predicted and actual data, PINNs introduce additional terms in the loss function that penalize the model for deviating from known physical principles.

In this case, the physical constraints are derived from the gas balance and the Weymouth equations, which describe the flow and pressure behavior within the gas transportation network. These constraints are integrated into our neural network as additional loss terms. Specifically, we define two layers within the network: one that calculates the error in gas balance and another that calculates the error in the Weymouth equation. The outputs of these layers are then used to adjust the network's predictions, ensuring that they adhere to the physical laws governing the system.

The inclusion of these physics-informed layers allows the network to achieve better generalization, as it is not only trained on the data but also guided by the underlying physical laws. This approach can be seen as a specialized form of regularization, where the model is penalized if its predictions do not satisfy the physical constraints. The overall loss function can be expressed as:


\begin{equation}
   \mathcal{L}(\Theta) = \mathcal{L}_{\text{data}}(\Theta) + \lambda_1 \mathcal{L}_{\text{balance}}(\Theta) + \lambda_2 \mathcal{L}_{\text{weymouth}}(\Theta),     
    \label{eq:PINN_basic_definition}
\end{equation}


% \[
% \mathcal{L}(\Theta) = \mathcal{L}_{\text{data}}(\Theta) + \lambda_1 \mathcal{L}_{\text{balance}}(\Theta) + \lambda_2 \mathcal{L}_{\text{weymouth}}(\Theta),
% \]

where \( \mathcal{L}_{\text{data}}(\Theta) \) represents the traditional data-driven loss, \( \mathcal{L}_{\text{balance}}(\Theta) \) is the loss associated with the gas balance constraint, and \( \mathcal{L}_{\text{weymouth}}(\Theta) \) is the loss associated with the Weymouth equation constraint. The parameters \( \lambda_1 \) and \( \lambda_2 \) control the importance of each physical constraint in the learning process.


In this section, we incorporate the physical laws of the gas balance and Weymouth equations to guide the model's training process. The gas balance equation, represented by \cref{eq:gas_balance}, ensures that the flow into and out of each node in the network adheres to the principle of mass conservation. The Weymouth equation, referred to as \cref{eq:weymouth_cons}, establishes a relationship between the flow and pressure differences across pipelines. These two equations will be the foundation for introducing physics-based constraints into the neural network, ensuring the model's predictions respect the physical behavior of gas flow within the system.


\section{Experimental Setup}

In this chapter, we build upon the experimental setup outlined in \cref{sec:LinealCensnet_ExperimentalSetup}, maintaining the same general approach while incorporating new elements that account for the physics of the natural gas system. The samples are generated using the nonlinear natural gas network optimization model from \cref{cap:optimization_mpcc}. In this process, a power-interconnected system was considered, but since this study focuses on the gas system, the power system remained constant without any variation. As in the previous setup, noise is introduced into the base values of two gas networks: a small-scale test network of 8 nodes and the more extensive Colombian natural gas transportation system. The noise levels, ranging from 5\% to 25\%, simulate various operating conditions, providing diverse training data.

While the GNN-based model from \cref{cap:lienal-censnet} was designed as a fast alternative to the optimization-based model, this chapter introduces physics-informed elements into the network architecture. Specifically, the model now includes loss terms based on the gas balance and Weymouth equations to ensure the predicted flows comply with the physical laws governing gas transportation. These constraints, integrated through additional layers in the model, guide the learning process, penalizing deviations from the gas balance equation (\cref{eq:gas_balance}) and the Weymouth equation (\cref{eq:weymouth_cons}). The modified model maintains the same structural components, such as input channels, convolutional layers, and loss functions for node and edge predictions, with the difference that the balance equation and the Weymouth equation are now considered loss functions. 


\begin{figure}
    \centering
    \setlength\figurewidth{1\textwidth}        
    \setlength\figureheight{0.5\textwidth}
    \resizebox{\figurewidth}{\figureheight}{\begin{tikzpicture}[shorten >=1pt, ->, draw=black!50, node distance=1.5cm and 3.5cm, align=center]

    % Styles
    \tikzstyle{input} = [rectangle, draw, fill=orange!30, minimum width=3cm, minimum height=1cm]
    \tikzstyle{dense} = [rectangle, draw, fill=blue!30, minimum width=3cm, minimum height=1cm]
    \tikzstyle{conv} = [rectangle, draw, fill=green!30, minimum width=3cm, minimum height=1cm]
    \tikzstyle{output} = [rectangle, draw, fill=purple!30, minimum width=3cm, minimum height=1cm]
    \tikzstyle{loss} = [rectangle, draw, fill=red!30, minimum width=3cm, minimum height=1cm]
    \tikzstyle{arrow} = [->, thick]

    % Input Layer
    \node[input] (node_features) at (0,0) {\(\mathbf{X}\)};
    \node[input] (node_laplacian) [below of=node_features] {\(\mathbf{L}_v\)};
    \node[input] (edge_laplacian) [below of=node_laplacian] {\(\mathbf{L}_e\)};
    \node[input] (incidence_matrix) [below of=edge_laplacian] {\(\mathbf{T}\)};
    \node[input] (edge_features) [below of=incidence_matrix] {\(\mathbf{E}\)};

    % Normalization and Pre-dense Layer
    \node[dense] (norm_pre_dense) [right of=edge_laplacian, xshift=3cm] {Normalization \\ \& Pre-dense Layers};

    % Convolutional Layers
    \node[conv] (conv_layers) [right of=norm_pre_dense, xshift=3cm] {CensNet Blocks \\ (Convolutional Layers)};

    % Post-dense Layer
    \node[dense] (post_dense) [right of=conv_layers, xshift=3cm] {Post-dense Layers};

    % Outputs
    \node[output] (node_output) [right of=post_dense, xshift=3cm, yshift=3cm] {\(\hat{\mathbf{X}}_v\)};
    \node[output] (edge_output) [right of=post_dense, xshift=3cm, yshift=1cm] {\(\hat{\mathbf{X}}_e\)};
    \node[output] (balance_output) [right of=post_dense, xshift=3cm, yshift=-1cm] {\(\mathcal{J}_{\text{balance}}\)};
    \node[output] (weymouth_output) [right of=post_dense, xshift=3cm, yshift=-3cm] {\(\mathcal{J}_{\text{Weymouth}}\)};

    % Losses
    \node[loss] (node_loss) [right of=node_output, xshift=3cm] {Node Loss};
    \node[loss] (edge_loss) [right of=edge_output, xshift=3cm] {Edge Loss};
    \node[loss] (balance_loss) [right of=balance_output, xshift=3cm] {Balance Loss};
    \node[loss] (weymouth_loss) [right of=weymouth_output, xshift=3cm] {Weymouth Loss};

    % Arrows
    \draw[arrow] (node_features) -- (norm_pre_dense);
    \draw[arrow] (node_laplacian) -- (norm_pre_dense);
    \draw[arrow] (edge_laplacian) -- (norm_pre_dense);
    \draw[arrow] (incidence_matrix) -- (norm_pre_dense);
    \draw[arrow] (edge_features) -- (norm_pre_dense);

    \draw[arrow] (norm_pre_dense) -- (conv_layers);
    \draw[arrow] (conv_layers) -- (post_dense);

    \draw[arrow] (post_dense) -- (node_output);
    \draw[arrow] (post_dense) -- (edge_output);
    \draw[arrow] (post_dense) -- (balance_output);
    \draw[arrow] (post_dense) -- (weymouth_output);

    \draw[arrow] (node_output) -- (node_loss);
    \draw[arrow] (edge_output) -- (edge_loss);
    \draw[arrow] (balance_output) -- (balance_loss);
    \draw[arrow] (weymouth_output) -- (weymouth_loss);

\end{tikzpicture}

}
    % \begin{tikzpicture}[shorten >=1pt, ->, draw=black!50, node distance=1.5cm and 3.5cm, align=center]

    % Styles
    \tikzstyle{input} = [circle, draw, fill=green!50, minimum size=2em]
    \tikzstyle{hidden} = [circle, draw, fill=blue!50, minimum size=2em]
    \tikzstyle{output} = [circle, draw, fill=red!50, minimum size=2em]
    \tikzstyle{connection} = [->, thick]

    % Network Stage Labels
    \node[align=center] at (0,-0.4) {Input \\ Layer};
    \node[align=center] at (6,0.4) {Hidden \\ Layers};
    \node[align=center] at (12,-1.2) {Output \\ Layer};

    % Input Layer
    \foreach \i in {1,2,3}
        \node[input] (I\i) at (0,-\i*1.5) {$x_\i$};

    % Hidden Layer 1
    \foreach \i in {1,2,3,4}
        \node[hidden] (H1\i) at (3,-\i*1.5+0.75) {$z^{(1)}_\i$};

    % Hidden Layer 2
    \foreach \i in {1,2,3,4}
        \node[hidden] (H2\i) at (6,-\i*1.5+0.75) {$z^{(2)}_\i$};

    % Hidden Layer 3
    \foreach \i in {1,2,3,4}
        \node[hidden] (H3\i) at (9,-\i*1.5+0.75) {$z^{(3)}_\i$};

    % Output Layer
    \foreach \i in {1,2}
        \node[output] (O\i) at (12,-\i*1.5-0.75) {$\hat{y}_\i$};

    % Connections from Input to Hidden Layer 1
    \foreach \i in {1,2,3}
        \foreach \j in {1,2,3,4}
            \draw[connection] (I\i) -- (H1\j);

    % Connections from Hidden Layer 1 to Hidden Layer 2
    \foreach \i in {1,2,3,4}
        \foreach \j in {1,2,3,4}
            \draw[connection] (H1\i) -- (H2\j);

    % Connections from Hidden Layer 2 to Hidden Layer 3
    \foreach \i in {1,2,3,4}
        \foreach \j in {1,2,3,4}
            \draw[connection] (H2\i) -- (H3\j);

    % Connections from Hidden Layer 3 to Output Layer
    \foreach \i in {1,2,3,4}
        \foreach \j in {1,2}
            \draw[connection] (H3\i) -- (O\j);

\end{tikzpicture}

    \caption{General outline of the CensNet-based model used.}
        \label{fig:nonlineal_model_description}
\end{figure}

\section{Results}


In this section, we present the results of the proposed model, which now incorporates physical constraints from the natural gas system. The focus remains on the relationship between the predicted outputs and the actual observed values, evaluating the model's performance across the 8-node test network and the Colombian natural gas transportation system. By incorporating physics-based constraints, the goal is to assess the model's ability to predict critical parameters under various operational conditions while ensuring that the physical laws governing gas flow are respected.

\subsection{Case Study I: 8-node Network}



In this chapter, we begin with experiments that account for both node and edge losses, as it was found that considering only the node loss did not produce adequate results. The best parameters identified for this experiment were $N channels=25$, $N layers =4$, and $N dense = 11$. These settings yielded a total loss of 6.816, with a node loss of 2.794 and an edge loss of 4.021.

The results corresponding to the nodes, shown in \cref{fig:results_nonlineal_dummy_base_node}, exhibit a similar behavior to that observed in \cref{cap:lienal-censnet}, demonstrating that the model accurately captures the injection pattern at the nodes. The correlation between the actual and predicted values is also strong, as indicated by an $R^2$ of 0.983.

Edge flows show some variation, as seen in \cref{fig:results_nonlineal_dummy_base_f}, mainly when predicting the flows through the first pipeline connected to the injection field, where slight deviations from the actual flow values were observed. However, the model performed well overall, achieving an $R^2$ of 0.983 for the edge flows. While the first pipeline presents some prediction challenges, the accuracy in predicting flows across the rest of the pipelines remains high, demonstrating the model's ability to handle the complexity of gas transportation in this nonlinear system.


\begin{figure}
    \centering
    \setlength\figurewidth{.53\textwidth}        
    \setlength\figureheight{0.36\textwidth} 
    \subfloat[Actual vs predicted nodal flows.] 
    {\label{fig:results_nonlineal_dummy_base_node}\resizebox{\figurewidth}{\figureheight}{% This file was created with tikzplotlib v0.10.1.
\begin{tikzpicture}

\definecolor{darkgray176}{RGB}{176,176,176}
\definecolor{lightgray204}{RGB}{204,204,204}

\begin{axis}[
colorbar,
colorbar style={ylabel={node_id}},
colormap={mymap}{[1pt]
 rgb(0pt)=(0.12156862745098,0.466666666666667,0.705882352941177);
  rgb(1pt)=(1,0.498039215686275,0.0549019607843137);
  rgb(2pt)=(0.172549019607843,0.627450980392157,0.172549019607843);
  rgb(3pt)=(0.83921568627451,0.152941176470588,0.156862745098039);
  rgb(4pt)=(0.580392156862745,0.403921568627451,0.741176470588235);
  rgb(5pt)=(0.549019607843137,0.337254901960784,0.294117647058824);
  rgb(6pt)=(0.890196078431372,0.466666666666667,0.76078431372549);
  rgb(7pt)=(0.498039215686275,0.498039215686275,0.498039215686275);
  rgb(8pt)=(0.737254901960784,0.741176470588235,0.133333333333333);
  rgb(9pt)=(0.0901960784313725,0.745098039215686,0.811764705882353)
},
legend cell align={left},
legend style={
  fill opacity=0.8,
  draw opacity=1,
  text opacity=1,
  at={(0.03,0.97)},
  anchor=north west,
  draw=lightgray204
},
point meta max=7,
point meta min=0,
tick align=outside,
tick pos=left,
title={yn_test-y_pred},
x grid style={darkgray176},
xlabel={yn_test},
xmajorgrids,
xmin=-2.43954548935, xmax=51.23045527635,
xtick style={color=black},
y grid style={darkgray176},
ylabel={y_pred},
ymajorgrids,
ymin=-2.19038361012936, ymax=45.0462195843458,
ytick style={color=black}
]
\addplot [
  colormap={mymap}{[1pt]
 rgb(0pt)=(0.12156862745098,0.466666666666667,0.705882352941177);
  rgb(1pt)=(1,0.498039215686275,0.0549019607843137);
  rgb(2pt)=(0.172549019607843,0.627450980392157,0.172549019607843);
  rgb(3pt)=(0.83921568627451,0.152941176470588,0.156862745098039);
  rgb(4pt)=(0.580392156862745,0.403921568627451,0.741176470588235);
  rgb(5pt)=(0.549019607843137,0.337254901960784,0.294117647058824);
  rgb(6pt)=(0.890196078431372,0.466666666666667,0.76078431372549);
  rgb(7pt)=(0.498039215686275,0.498039215686275,0.498039215686275);
  rgb(8pt)=(0.737254901960784,0.741176470588235,0.133333333333333);
  rgb(9pt)=(0.0901960784313725,0.745098039215686,0.811764705882353)
},
  only marks,
  scatter,
  scatter src=explicit
]
table [x=x, y=y, meta=colordata]{%
x  y  colordata
39.565898645 38.7153625488281 0.0
0 0.0123127698898315 1.0
0 0.071319580078125 2.0
0 -0.0128939747810364 3.0
0 0.0256100296974182 4.0
0 -0.0238697528839111 5.0
0 0.00411289930343628 6.0
0 -0.0127748847007751 7.0
42.743257181 39.4932670593262 0.0
0 0.00441712141036987 1.0
0 0.0637285709381104 2.0
0 -0.0179623365402222 3.0
0 0.0215394496917725 4.0
0 -0.0140223503112793 5.0
0 1.97887420654297e-05 6.0
0 -0.0135558843612671 7.0
39.367180751 38.7526664733887 0.0
0 0.0138788819313049 1.0
0 0.0361418128013611 2.0
0 -0.0201342701911926 3.0
0 0.00914174318313599 4.0
0 -0.0191978812217712 5.0
0 0.000798225402832031 6.0
0 -0.00776267051696777 7.0
39.605393107 39.5196304321289 0.0
0 0.00788706541061401 1.0
0 0.0466488599777222 2.0
0 -0.0212488174438477 3.0
0 0.0412870049476624 4.0
0 -0.0301215648651123 5.0
0 0.00819826126098633 6.0
0 -0.0149205923080444 7.0
43.937345536 38.9456558227539 0.0
0 0.0131928324699402 1.0
0 0.0427597761154175 2.0
0 -0.0308229327201843 3.0
0 0.0441254377365112 4.0
0 -0.0282912254333496 5.0
0 0.00409531593322754 6.0
0 -0.0128695964813232 7.0
31.061989594 39.310115814209 0.0
0 0.0139701366424561 1.0
0 0.0382223725318909 2.0
0 -0.0262283682823181 3.0
0 0.0097421407699585 4.0
0 -0.0335473418235779 5.0
0 0.00768345594406128 6.0
0 -0.00622838735580444 7.0
36.357435273 42.0804176330566 0.0
0 0.00529545545578003 1.0
0 0.0581262111663818 2.0
0 -0.0150731205940247 3.0
0 0.0298828482627869 4.0
0 -0.028703510761261 5.0
0 0.00751966238021851 6.0
0 -0.0148077011108398 7.0
38.444969617 39.1996421813965 0.0
0 0.00592869520187378 1.0
0 0.0615658760070801 2.0
0 -0.0213605761528015 3.0
0 0.0231988430023193 4.0
0 -0.026212215423584 5.0
0 0.00326085090637207 6.0
0 -0.0112636089324951 7.0
35.498620524 39.3897895812988 0.0
0 0.00891655683517456 1.0
0 0.0605899095535278 2.0
0 -0.018476665019989 3.0
0 0.00775080919265747 4.0
0 -0.024048924446106 5.0
0 0.0055384635925293 6.0
0 -0.0057908296585083 7.0
36.520998279 39.447883605957 0.0
0 0.00435954332351685 1.0
0 0.0819315910339355 2.0
0 -0.0160262584686279 3.0
0 0.0174445509910583 4.0
0 -0.0244802236557007 5.0
0 0.00764340162277222 6.0
0 -0.00894832611083984 7.0
36.717272212 39.199634552002 0.0
0 0.00995087623596191 1.0
0 0.066163182258606 2.0
0 -0.0166041851043701 3.0
0 0.0310661196708679 4.0
0 -0.0306993126869202 5.0
0 0.0094258189201355 6.0
0 -0.0108351707458496 7.0
32.629629006 39.185905456543 0.0
0 0.0115001201629639 1.0
0 0.0270580649375916 2.0
0 -0.0159772038459778 3.0
0 0.00784498453140259 4.0
0 -0.0299507975578308 5.0
0 0.00833970308303833 6.0
0 -0.00938189029693604 7.0
37.75267434 39.1989135742188 0.0
0 0.0082511305809021 1.0
0 0.06497722864151 2.0
0 -0.0128266215324402 3.0
0 0.00258684158325195 4.0
0 -0.018075704574585 5.0
0 -0.000989556312561035 6.0
0 -0.0049208402633667 7.0
38.800291347 40.1666564941406 0.0
0 0.0118288397789001 1.0
0 0.0622610449790955 2.0
0 -0.0261341333389282 3.0
0 0.0260022878646851 4.0
0 -0.0306769013404846 5.0
0 0.0043455958366394 6.0
0 -0.0105564594268799 7.0
38.252729618 39.3290748596191 0.0
0 0.00866585969924927 1.0
0 0.0685842037200928 2.0
0 -0.0130652189254761 3.0
0 0.0246871113777161 4.0
0 -0.0263183116912842 5.0
0 0.00541287660598755 6.0
0 -0.0122238397598267 7.0
43.273596047 39.3154678344727 0.0
0 0.00989145040512085 1.0
0 0.073027491569519 2.0
0 -0.0273558497428894 3.0
0 0.0494243502616882 4.0
0 -0.0284897089004517 5.0
0 0.00563955307006836 6.0
0 -0.0137518644332886 7.0
34.027431486 38.857494354248 0.0
0 0.00671476125717163 1.0
0 0.0619230270385742 2.0
0 -0.0194013118743896 3.0
0 0.0136286020278931 4.0
0 -0.029705822467804 5.0
0 0.003440260887146 6.0
0 -0.00797498226165771 7.0
41.154171391 39.8500900268555 0.0
0 0.00906610488891602 1.0
0 0.0560967326164246 2.0
0 -0.0276532769203186 3.0
0 0.0497760772705078 4.0
0 -0.0301504731178284 5.0
0 0.00726139545440674 6.0
0 -0.0125929117202759 7.0
39.930826408 39.8617324829102 0.0
0 0.00576150417327881 1.0
0 0.0628125667572021 2.0
0 -0.0210210680961609 3.0
0 0.0447211265563965 4.0
0 -0.0306063890457153 5.0
0 0.00760847330093384 6.0
0 -0.0124554634094238 7.0
45.969703631 39.1456565856934 0.0
0 0.0107769966125488 1.0
0 0.0357969403266907 2.0
0 -0.0225849151611328 3.0
0 0.043922483921051 4.0
0 -0.0197867751121521 5.0
0 0.00242942571640015 6.0
0 -0.0128521323204041 7.0
38.398120221 39.2407455444336 0.0
0 0.00990653038024902 1.0
0 0.0493940114974976 2.0
0 -0.0257954001426697 3.0
0 0.035546600818634 4.0
0 -0.0293555855751038 5.0
0 0.00442242622375488 6.0
0 -0.0125865340232849 7.0
28.366017306 39.314582824707 0.0
0 1.03116035461426e-05 1.0
0 0.0335801839828491 2.0
0 -0.00797528028488159 3.0
0 -0.0116368532180786 4.0
0 -0.0308080911636353 5.0
0 0.00525516271591187 6.0
0 -0.00420612096786499 7.0
39.079818979 39.1686325073242 0.0
0 0.011313796043396 1.0
0 0.0414641499519348 2.0
0 -0.0202111601829529 3.0
0 0.0133368968963623 4.0
0 -0.0244540572166443 5.0
0 0.00453060865402222 6.0
0 -0.0085756778717041 7.0
40.466329383 39.3136444091797 0.0
0 -0.00223791599273682 1.0
0 0.0511056780815125 2.0
0 -0.0211246013641357 3.0
0 0.0184855461120605 4.0
0 -0.0249439477920532 5.0
0 0.00523895025253296 6.0
0 -0.00943392515182495 7.0
38.293116853 38.996208190918 0.0
0 0.00625330209732056 1.0
0 0.069944441318512 2.0
0 -0.0136637687683105 3.0
0 0.000947535037994385 4.0
0 -0.0120516419410706 5.0
0 0.000261187553405762 6.0
0 -0.00625407695770264 7.0
43.346089497 38.9964256286621 0.0
0 0.0137380957603455 1.0
0 0.0395742058753967 2.0
0 -0.0222710371017456 3.0
0 0.0477991700172424 4.0
0 -0.030639111995697 5.0
0 0.00966173410415649 6.0
0 -0.0138188004493713 7.0
34.547871154 39.3231048583984 0.0
0 0.00445806980133057 1.0
0 0.0469914674758911 2.0
0 -0.0166859030723572 3.0
0 0.0288587212562561 4.0
0 -0.0295593738555908 5.0
0 0.00691384077072144 6.0
0 -0.0114007592201233 7.0
41.927050143 39.338737487793 0.0
0 0.00693070888519287 1.0
0 0.0592769980430603 2.0
0 -0.0221498608589172 3.0
0 0.0644987821578979 4.0
0 -0.029447615146637 5.0
0 0.0115132927894592 6.0
0 -0.0177930593490601 7.0
44.548236377 39.3082656860352 0.0
0 0.0133536458015442 1.0
0 0.0384448766708374 2.0
0 -0.0213861465454102 3.0
0 0.0355319976806641 4.0
0 -0.0190299153327942 5.0
0 0.00214254856109619 6.0
0 -0.0123872756958008 7.0
34.958415487 39.313117980957 0.0
0 0.00151073932647705 1.0
0 0.0384351015090942 2.0
0 -0.0213229656219482 3.0
0 0.00255352258682251 4.0
0 -0.025343120098114 5.0
0 0.00335896015167236 6.0
0 -0.00606870651245117 7.0
41.619773298 39.9483871459961 0.0
0 -0.00310182571411133 1.0
0 0.0696684718132019 2.0
0 -0.0155016183853149 3.0
0 0.0176678895950317 4.0
0 -0.0168115496635437 5.0
0 0.000762403011322021 6.0
0 -0.0129686594009399 7.0
35.768623912 39.1677360534668 0.0
0 -0.00235003232955933 1.0
0 0.0565180778503418 2.0
0 -0.0165175199508667 3.0
0 0.0183342695236206 4.0
0 -0.0309199094772339 5.0
0 0.00655144453048706 6.0
0 -0.0103568434715271 7.0
36.03587308 37.5861854553223 0.0
0 -0.00157850980758667 1.0
0 0.0492555499076843 2.0
0 -0.0198881030082703 3.0
0 0.0342521667480469 4.0
0 -0.0291762351989746 5.0
0 0.0110065340995789 6.0
0 -0.011877179145813 7.0
42.671159584 39.3272399902344 0.0
0 0.0113665461540222 1.0
0 0.0392718911170959 2.0
0 -0.0274774432182312 3.0
0 0.052215039730072 4.0
0 -0.0293958783149719 5.0
0 0.00858646631240845 6.0
0 -0.0139873623847961 7.0
32.270518391 39.3328552246094 0.0
0 0.0110946297645569 1.0
0 0.05193692445755 2.0
0 -0.0178480744361877 3.0
0 0.00724279880523682 4.0
0 -0.0301480889320374 5.0
0 0.00648993253707886 6.0
0 -0.00789618492126465 7.0
36.239834941 41.1309661865234 0.0
0 0.00345695018768311 1.0
0 0.0747806429862976 2.0
0 -0.0145701169967651 3.0
0 0.0049513578414917 4.0
0 -0.0192160606384277 5.0
0 0.00256305932998657 6.0
0 -0.00866025686264038 7.0
38.292005181 39.2797698974609 0.0
0 0.00284379720687866 1.0
0 0.0393624305725098 2.0
0 -0.0206927061080933 3.0
0 0.0245354771614075 4.0
0 -0.028574526309967 5.0
0 0.00577658414840698 6.0
0 -0.0118468999862671 7.0
41.142171126 39.369083404541 0.0
0 0.000563442707061768 1.0
0 0.0569069981575012 2.0
0 -0.0239884257316589 3.0
0 0.0219588279724121 4.0
0 -0.0221072435379028 5.0
0 0.0049242377281189 6.0
0 -0.0127769112586975 7.0
30.660234751 38.9642333984375 0.0
0 -0.00240546464920044 1.0
0 0.0487861633300781 2.0
0 -0.00998640060424805 3.0
0 0.00200992822647095 4.0
0 -0.0323919057846069 5.0
0 0.00852972269058228 6.0
0 -0.00369340181350708 7.0
42.776716986 41.5339431762695 0.0
0 0.00600254535675049 1.0
0 0.0594097375869751 2.0
0 -0.0193828344345093 3.0
0 0.0313163995742798 4.0
0 -0.0243908762931824 5.0
0 0.0065605640411377 6.0
0 -0.016025185585022 7.0
39.136657955 39.7308387756348 0.0
0 0.00334018468856812 1.0
0 0.099915623664856 2.0
0 -0.0114416480064392 3.0
0 0.0245475769042969 4.0
0 -0.0192947387695312 5.0
0 0.00287395715713501 6.0
0 -0.0121320486068726 7.0
40.593075664 39.7711791992188 0.0
0 0.00522077083587646 1.0
0 0.0596650242805481 2.0
0 -0.0162090659141541 3.0
0 0.045604944229126 4.0
0 -0.0273600816726685 5.0
0 0.00408452749252319 6.0
0 -0.0147544145584106 7.0
38.455960057 40.2578964233398 0.0
0 0.00629007816314697 1.0
0 0.0590381622314453 2.0
0 -0.0165366530418396 3.0
0 0.043002724647522 4.0
0 -0.0271528959274292 5.0
0 0.00735443830490112 6.0
0 -0.0150219798088074 7.0
40.029822307 39.2045021057129 0.0
0 0.0109648704528809 1.0
0 0.0333818793296814 2.0
0 -0.0219238996505737 3.0
0 0.0246446132659912 4.0
0 -0.0267536044120789 5.0
0 0.00693017244338989 6.0
0 -0.0108383893966675 7.0
39.721089806 39.6830825805664 0.0
0 0.00558066368103027 1.0
0 0.0897490382194519 2.0
0 -0.0145083665847778 3.0
0 0.0387337803840637 4.0
0 -0.0278629660606384 5.0
0 0.00985139608383179 6.0
0 -0.0144490599632263 7.0
46.012802781 39.0627212524414 0.0
0 0.00166887044906616 1.0
0 0.0386090874671936 2.0
0 -0.0324397087097168 3.0
0 0.0550388693809509 4.0
0 -0.028700590133667 5.0
0 0.00663286447525024 6.0
0 -0.0132797360420227 7.0
43.791041158 39.1844825744629 0.0
0 0.0124698877334595 1.0
0 0.0309100747108459 2.0
0 -0.0285602211952209 3.0
0 0.046930730342865 4.0
0 -0.0275624394416809 5.0
0 0.00615853071212769 6.0
0 -0.0148841142654419 7.0
31.257332424 39.1901626586914 0.0
0 0.0106835961341858 1.0
0 0.0625916123390198 2.0
0 -0.0122222304344177 3.0
0 0.000936150550842285 4.0
0 -0.0310866236686707 5.0
0 0.00312197208404541 6.0
0 -0.00706321001052856 7.0
38.98847291 39.2609672546387 0.0
0 0.0101978778839111 1.0
0 0.0407465100288391 2.0
0 -0.0224083065986633 3.0
0 0.0104560256004333 4.0
0 -0.0216193795204163 5.0
0 0.00195193290710449 6.0
0 -0.00701618194580078 7.0
38.691218499 38.9683609008789 0.0
0 0.00895541906356812 1.0
0 0.0564865469932556 2.0
0 -0.0135195851325989 3.0
0 0.0024409294128418 4.0
0 -0.0234909653663635 5.0
0 0.0044097900390625 6.0
0 -0.00805675983428955 7.0
39.033211971 38.7177581787109 0.0
0 0.00353097915649414 1.0
0 0.0651161074638367 2.0
0 -0.0148383378982544 3.0
0 0.0438032746315002 4.0
0 -0.0298718214035034 5.0
0 0.00918209552764893 6.0
0 -0.0137563943862915 7.0
37.697547813 39.1868095397949 0.0
0 0.00985407829284668 1.0
0 0.0413744449615479 2.0
0 -0.0205171704292297 3.0
0 0.00754988193511963 4.0
0 -0.0230441689491272 5.0
0 0.00145184993743896 6.0
0 -0.0105559825897217 7.0
35.277541339 39.2073364257812 0.0
0 0.0143079161643982 1.0
0 0.0378283262252808 2.0
0 -0.0201423168182373 3.0
0 0.00532984733581543 4.0
0 -0.0268043279647827 5.0
0 0.00716966390609741 6.0
0 -0.0073121190071106 7.0
36.119763966 39.4090423583984 0.0
0 0.0113481879234314 1.0
0 0.0379583835601807 2.0
0 -0.0301769375801086 3.0
0 0.0023917555809021 4.0
0 -0.0140483379364014 5.0
0 -0.00150942802429199 6.0
0 -0.00346070528030396 7.0
33.14490305 39.51171875 0.0
0 0.00569027662277222 1.0
0 0.0744785666465759 2.0
0 -0.0195279121398926 3.0
0 0.00108557939529419 4.0
0 -0.0206176042556763 5.0
0 0.00269472599029541 6.0
0 -0.00410282611846924 7.0
34.486800814 40.0056648254395 0.0
0 0.0040428638458252 1.0
0 0.0613293647766113 2.0
0 -0.0154016017913818 3.0
0 0.0307614207267761 4.0
0 -0.0287876129150391 5.0
0 0.0124437212944031 6.0
0 -0.0129595398902893 7.0
40.933468207 40.1773338317871 0.0
0 0.0079658031463623 1.0
0 0.0732653141021729 2.0
0 -0.0176567435264587 3.0
0 0.0293158888816833 4.0
0 -0.0187020897865295 5.0
0 0.00174516439437866 6.0
0 -0.0126280188560486 7.0
35.993417928 39.3662071228027 0.0
0 0.00715410709381104 1.0
0 0.0595747232437134 2.0
0 -0.0184565782546997 3.0
0 0.0182121396064758 4.0
0 -0.0294560790061951 5.0
0 0.00316703319549561 6.0
0 -0.0110476613044739 7.0
39.331666923 39.7296485900879 0.0
0 0.00837588310241699 1.0
0 0.0545561909675598 2.0
0 -0.0260232090950012 3.0
0 0.0253786444664001 4.0
0 -0.0261372923851013 5.0
0 0.0041089653968811 6.0
0 -0.0101762413978577 7.0
37.559548496 39.3728485107422 0.0
0 0.00726288557052612 1.0
0 0.0635012984275818 2.0
0 -0.0159628987312317 3.0
0 0.0196371674537659 4.0
0 -0.0268263816833496 5.0
0 0.00435435771942139 6.0
0 -0.0115820169448853 7.0
41.796902482 39.2759208679199 0.0
0 0.0100352764129639 1.0
0 0.0439282655715942 2.0
0 -0.0254364013671875 3.0
0 0.0484634637832642 4.0
0 -0.0301735997200012 5.0
0 0.00479131937026978 6.0
0 -0.0128893852233887 7.0
35.679590823 39.2148361206055 0.0
0 0.0133700370788574 1.0
0 0.0399370193481445 2.0
0 -0.0224254131317139 3.0
0 0.0077054500579834 4.0
0 -0.0263437032699585 5.0
0 0.00416111946105957 6.0
0 -0.00710749626159668 7.0
33.227547292 38.8616104125977 0.0
0 0.0111078023910522 1.0
0 0.061786949634552 2.0
0 -0.0128821134567261 3.0
0 0.00850886106491089 4.0
0 -0.0278047919273376 5.0
0 0.00242877006530762 6.0
0 -0.00745308399200439 7.0
28.008071739 39.3282775878906 0.0
0 0.00727421045303345 1.0
0 0.0578927397727966 2.0
0 -0.0101944804191589 3.0
0 -0.00735962390899658 4.0
0 -0.0292975306510925 5.0
0 0.0059630274772644 6.0
0 0.00163012742996216 7.0
33.478498841 39.2475204467773 0.0
0 0.0053410530090332 1.0
0 0.0993016362190247 2.0
0 -0.00882583856582642 3.0
0 0.0095442533493042 4.0
0 -0.0248811841011047 5.0
0 0.00372958183288574 6.0
0 -0.00759077072143555 7.0
33.12294682 39.1935424804688 0.0
0 0.0134350061416626 1.0
0 0.0337791442871094 2.0
0 -0.0293073058128357 3.0
0 0.0119555592536926 4.0
0 -0.03260737657547 5.0
0 0.00758576393127441 6.0
0 -0.00859177112579346 7.0
34.970228384 39.3222885131836 0.0
0 0.0112384557723999 1.0
0 0.0673366785049438 2.0
0 -0.0168597102165222 3.0
0 0.00697022676467896 4.0
0 -0.0265811085700989 5.0
0 0.00594782829284668 6.0
0 0.0834335684776306 7.0
36.637966041 39.6626014709473 0.0
0 0.0114080905914307 1.0
0 0.039689302444458 2.0
0 -0.0293136239051819 3.0
0 0.0113623738288879 4.0
0 -0.028203547000885 5.0
0 0.00562357902526855 6.0
0 -0.00684285163879395 7.0
38.712447295 39.2186279296875 0.0
0 0.00919914245605469 1.0
0 0.0359517931938171 2.0
0 -0.0268067121505737 3.0
0 0.0361806750297546 4.0
0 -0.0292361378669739 5.0
0 0.00580942630767822 6.0
0 -0.0120308995246887 7.0
29.08402242 39.358585357666 0.0
0 0.00405758619308472 1.0
0 0.0620464682579041 2.0
0 -0.00910604000091553 3.0
0 -0.00566118955612183 4.0
0 -0.0290875434875488 5.0
0 0.00550538301467896 6.0
0 -0.00326275825500488 7.0
40.741903011 39.7179641723633 0.0
0 0.0086064338684082 1.0
0 0.0604866147041321 2.0
0 -0.0193439722061157 3.0
0 0.0264756679534912 4.0
0 -0.0243978500366211 5.0
0 0.0040895938873291 6.0
0 -0.0132392048835754 7.0
44.423927978 39.2226028442383 0.0
0 0.0112552642822266 1.0
0 0.0406379103660583 2.0
0 -0.0212950110435486 3.0
0 0.0372845530509949 4.0
0 -0.0296989679336548 5.0
0 0.00628405809402466 6.0
0 -0.0137932300567627 7.0
37.279929011 42.0506362915039 0.0
0 0.000679612159729004 1.0
0 0.0922307968139648 2.0
0 -0.0117670297622681 3.0
0 0.00707334280014038 4.0
0 -0.0226204395294189 5.0
0 0.001944899559021 6.0
0 -0.0140491127967834 7.0
39.065196566 39.0746231079102 0.0
0 0.00692451000213623 1.0
0 0.0894067883491516 2.0
0 -0.0125413537025452 3.0
0 0.0433874726295471 4.0
0 -0.0262654423713684 5.0
0 0.00937050580978394 6.0
0 -0.0140787959098816 7.0
34.346712911 39.6062927246094 0.0
0 0.00730884075164795 1.0
0 0.0575341582298279 2.0
0 -0.0125131011009216 3.0
0 0.00180590152740479 4.0
0 -0.0254272222518921 5.0
0 0.00626826286315918 6.0
0 -0.00440073013305664 7.0
44.832836202 39.1740646362305 0.0
0 0.00540620088577271 1.0
0 0.048359215259552 2.0
0 -0.0217028856277466 3.0
0 0.0456007122993469 4.0
0 -0.0261398553848267 5.0
0 0.00735974311828613 6.0
0 -0.0143005847930908 7.0
37.693005916 39.4430923461914 0.0
0 0.00866585969924927 1.0
0 0.0588068962097168 2.0
0 -0.0298804044723511 3.0
0 0.012173056602478 4.0
0 -0.0230597257614136 5.0
0 0.00236654281616211 6.0
0 -0.00678610801696777 7.0
35.516030206 39.2848510742188 0.0
0 0.00840789079666138 1.0
0 0.0401602387428284 2.0
0 -0.0224798321723938 3.0
0 0.00727975368499756 4.0
0 -0.0262488722801208 5.0
0 0.00442099571228027 6.0
0 -0.00669002532958984 7.0
45.449957523 39.3815727233887 0.0
0 0.00223982334136963 1.0
0 0.0652434825897217 2.0
0 -0.0192306637763977 3.0
0 0.0608529448509216 4.0
0 -0.027153491973877 5.0
0 0.00786775350570679 6.0
0 -0.0163425207138062 7.0
36.52210397 40.1063613891602 0.0
0 0.0123141407966614 1.0
0 0.0607444047927856 2.0
0 -0.0281732082366943 3.0
0 0.00769400596618652 4.0
0 -0.0218368172645569 5.0
0 0.00279814004898071 6.0
0 0.214010000228882 7.0
40.809333833 39.2384757995605 0.0
0 0.0117188096046448 1.0
0 0.039944052696228 2.0
0 -0.0229901075363159 3.0
0 0.0258876085281372 4.0
0 -0.0261783003807068 5.0
0 0.0060725212097168 6.0
0 -0.00992476940155029 7.0
43.708815025 39.8203887939453 0.0
0 0.00554096698760986 1.0
0 0.0680207014083862 2.0
0 -0.0179526805877686 3.0
0 0.0337607264518738 4.0
0 -0.0186030864715576 5.0
0 0.00325441360473633 6.0
0 -0.0130414366722107 7.0
35.746307575 39.0528678894043 0.0
0 0.00616163015365601 1.0
0 0.031990647315979 2.0
0 -0.0340034961700439 3.0
0 0.0250157713890076 4.0
0 -0.0317110419273376 5.0
0 0.00777506828308105 6.0
0 -0.00998783111572266 7.0
32.565260407 39.287166595459 0.0
0 0.0112178325653076 1.0
0 0.0362849831581116 2.0
0 -0.0211132764816284 3.0
0 0.000355720520019531 4.0
0 -0.0266507863998413 5.0
0 0.00528323650360107 6.0
0 -0.00296187400817871 7.0
37.459437569 39.7067375183105 0.0
0 0.00287795066833496 1.0
0 0.0870211720466614 2.0
0 -0.0136895775794983 3.0
0 0.0324925780296326 4.0
0 -0.0261927247047424 5.0
0 0.00662326812744141 6.0
0 -0.0170584321022034 7.0
41.600868777 39.6331176757812 0.0
0 0.00617289543151855 1.0
0 0.0472753047943115 2.0
0 -0.0370604395866394 3.0
0 0.0207778215408325 4.0
0 -0.0186801552772522 5.0
0 0.000774919986724854 6.0
0 -0.00782781839370728 7.0
43.088648289 42.4566268920898 0.0
0 0.00171780586242676 1.0
0 0.0445231199264526 2.0
0 -0.0263301730155945 3.0
0 0.0577532052993774 4.0
0 -0.0282729268074036 5.0
0 0.00841760635375977 6.0
0 -0.0169897675514221 7.0
30.268152677 39.3095664978027 0.0
0 0.0115619301795959 1.0
0 0.0627694725990295 2.0
0 -0.0172976851463318 3.0
0 0.000941872596740723 4.0
0 -0.0301180481910706 5.0
0 0.00371986627578735 6.0
0 -0.00358611345291138 7.0
38.454045888 38.8899307250977 0.0
0 0.0116772055625916 1.0
0 0.0517902970314026 2.0
0 -0.0236786603927612 3.0
0 0.0169258117675781 4.0
0 -0.0281082987785339 5.0
0 0.0070832371711731 6.0
0 -0.00702857971191406 7.0
36.654056864 39.1972312927246 0.0
0 0.00971025228500366 1.0
0 0.0378752946853638 2.0
0 -0.021247386932373 3.0
0 0.0270963311195374 4.0
0 -0.029135525226593 5.0
0 0.00854533910751343 6.0
0 -0.0112012028694153 7.0
36.990278128 39.4844970703125 0.0
0 0.010595440864563 1.0
0 0.0305523872375488 2.0
0 -0.0160865783691406 3.0
0 0.0211124420166016 4.0
0 -0.0285171270370483 5.0
0 0.00439399480819702 6.0
0 -0.0110663771629333 7.0
39.45539179 39.3758239746094 0.0
0 0.00812649726867676 1.0
0 0.0442739725112915 2.0
0 -0.0175419449806213 3.0
0 0.0141112208366394 4.0
0 -0.027271032333374 5.0
0 0.00580936670303345 6.0
0 -0.00898617506027222 7.0
41.49009831 39.2659378051758 0.0
0 0.00471115112304688 1.0
0 0.0437565445899963 2.0
0 -0.0283921957015991 3.0
0 0.0177510380744934 4.0
0 -0.018986701965332 5.0
0 0.00044095516204834 6.0
0 -0.00875365734100342 7.0
42.195848764 39.7440223693848 0.0
0 0.00705784559249878 1.0
0 0.0813563466072083 2.0
0 -0.0185672044754028 3.0
0 0.0338060855865479 4.0
0 -0.0192969441413879 5.0
0 0.00195407867431641 6.0
0 -0.0113794207572937 7.0
36.580423956 39.3618698120117 0.0
0 0.0103769898414612 1.0
0 0.0483078956604004 2.0
0 -0.0236889123916626 3.0
0 0.0567827224731445 4.0
0 -0.028109610080719 5.0
0 0.0141017436981201 6.0
0 -0.0142652988433838 7.0
41.112612846 38.2719535827637 0.0
0 0.00170391798019409 1.0
0 0.105811357498169 2.0
0 -0.0118167400360107 3.0
0 0.036638617515564 4.0
0 -0.0218501687049866 5.0
0 0.00543737411499023 6.0
0 -0.0120397210121155 7.0
41.165032305 39.1872482299805 0.0
0 0.0157047510147095 1.0
0 0.0349633693695068 2.0
0 -0.0216887593269348 3.0
0 0.0505847930908203 4.0
0 -0.0271188616752625 5.0
0 0.00774049758911133 6.0
0 -0.0159221291542053 7.0
28.236329769 39.7965545654297 0.0
0 0.013361930847168 1.0
0 0.0601737499237061 2.0
0 -0.00644588470458984 3.0
0 -0.00728082656860352 4.0
0 -0.0303657054901123 5.0
0 0.00617170333862305 6.0
0 -0.00426137447357178 7.0
46.563722235 39.247730255127 0.0
0 0.0073997974395752 1.0
0 0.0409015417098999 2.0
0 -0.0347752571105957 3.0
0 0.0438136458396912 4.0
0 -0.0267916321754456 5.0
0 0.00308680534362793 6.0
0 -0.01231449842453 7.0
38.976039446 38.2965316772461 0.0
0 0.000528395175933838 1.0
0 0.025409460067749 2.0
0 -0.0173540711402893 3.0
0 0.00302207469940186 4.0
0 -0.0253949165344238 5.0
0 0.00395089387893677 6.0
0 -0.00843304395675659 7.0
43.902220165 39.8411254882812 0.0
0 0.0118625164031982 1.0
0 0.0401428937911987 2.0
0 -0.0226224660873413 3.0
0 0.0575634241104126 4.0
0 -0.0248637795448303 5.0
0 0.00626331567764282 6.0
0 -0.016412615776062 7.0
37.104870385 38.8935470581055 0.0
0 0.00900518894195557 1.0
0 0.0604310631752014 2.0
0 -0.0258265733718872 3.0
0 0.0378197431564331 4.0
0 -0.0289685726165771 5.0
0 0.00780671834945679 6.0
0 -0.0103233456611633 7.0
48.094555135 39.2413864135742 0.0
0 0.0141285061836243 1.0
0 0.0317863821983337 2.0
0 -0.0394392609596252 3.0
0 0.0513074994087219 4.0
0 -0.0285180807113647 5.0
0 0.0042346715927124 6.0
0 -0.0118893384933472 7.0
39.863550951 38.9883651733398 0.0
0 0.00595420598983765 1.0
0 0.0344380140304565 2.0
0 -0.0326076745986938 3.0
0 0.0386518836021423 4.0
0 -0.0331050753593445 5.0
0 0.00429987907409668 6.0
0 -0.0127902626991272 7.0
38.158424706 38.903678894043 0.0
0 0.0116057395935059 1.0
0 0.0537588000297546 2.0
0 -0.0285190343856812 3.0
0 0.0557814836502075 4.0
0 -0.029060959815979 5.0
0 0.00855320692062378 6.0
0 -0.0132871866226196 7.0
37.671552644 39.2374725341797 0.0
0 0.006736159324646 1.0
0 0.0430566668510437 2.0
0 -0.0213061571121216 3.0
0 0.0102139711380005 4.0
0 -0.0230097770690918 5.0
0 0.0043870210647583 6.0
0 -0.00778210163116455 7.0
38.99000687 39.3680267333984 0.0
0 0.00496220588684082 1.0
0 0.0836588740348816 2.0
0 -0.0154605507850647 3.0
0 0.0217403173446655 4.0
0 -0.0192769169807434 5.0
0 0.00258815288543701 6.0
0 -0.0127060413360596 7.0
32.824279197 39.2818450927734 0.0
0 0.0120186805725098 1.0
0 0.0409272313117981 2.0
0 -0.0234582424163818 3.0
0 0.00576364994049072 4.0
0 -0.0319589376449585 5.0
0 0.00346595048904419 6.0
0 -0.00725382566452026 7.0
39.752392398 39.3454055786133 0.0
0 0.00749439001083374 1.0
0 0.0421807169914246 2.0
0 -0.0276132225990295 3.0
0 0.00937461853027344 4.0
0 -0.0163375735282898 5.0
0 -0.000748932361602783 6.0
0 -0.00666379928588867 7.0
36.014997988 39.1476974487305 0.0
0 0.0123811960220337 1.0
0 0.0417296886444092 2.0
0 -0.0282233953475952 3.0
0 0.0180580019950867 4.0
0 -0.032294750213623 5.0
0 0.00644737482070923 6.0
0 -0.00982779264450073 7.0
42.307205606 39.2100219726562 0.0
0 0.0106115341186523 1.0
0 0.0420219302177429 2.0
0 -0.030609130859375 3.0
0 0.0188378095626831 4.0
0 -0.0219831466674805 5.0
0 0.00154978036880493 6.0
0 -0.00879871845245361 7.0
41.624369934 37.9989166259766 0.0
0 0.00612103939056396 1.0
0 0.0464754104614258 2.0
0 -0.0220385789871216 3.0
0 0.0484929084777832 4.0
0 -0.0285543203353882 5.0
0 0.00955373048782349 6.0
0 -0.0139495730400085 7.0
40.927860926 38.1056976318359 0.0
0 0.00858557224273682 1.0
0 0.0353962779045105 2.0
0 -0.0306072235107422 3.0
0 0.0340653657913208 4.0
0 -0.0286934375762939 5.0
0 0.00470423698425293 6.0
0 -0.0121994614601135 7.0
47.441321812 39.2602005004883 0.0
0 0.00840264558792114 1.0
0 0.0400999188423157 2.0
0 -0.0347042083740234 3.0
0 0.0447063446044922 4.0
0 -0.0227860808372498 5.0
0 0.00289112329483032 6.0
0 -0.0108776092529297 7.0
39.379505619 39.218936920166 0.0
0 0.00225824117660522 1.0
0 0.0381470918655396 2.0
0 -0.0227357745170593 3.0
0 0.0566052198410034 4.0
0 -0.0273053050041199 5.0
0 0.00671166181564331 6.0
0 -0.0176942348480225 7.0
42.095557299 39.302864074707 0.0
0 0.000669658184051514 1.0
0 0.0900809168815613 2.0
0 -0.0144271850585938 3.0
0 0.046349048614502 4.0
0 -0.0227258205413818 5.0
0 0.00791877508163452 6.0
0 -0.0147258043289185 7.0
35.151188225 39.5698585510254 0.0
0 0.00752902030944824 1.0
0 0.0650426149368286 2.0
0 -0.0100555419921875 3.0
0 -0.00416684150695801 4.0
0 -0.0206260681152344 5.0
0 -0.000355720520019531 6.0
0 -0.00151586532592773 7.0
37.713935371 39.5196762084961 0.0
0 0.000767946243286133 1.0
0 0.0639498233795166 2.0
0 -0.0160928964614868 3.0
0 0.0237512588500977 4.0
0 -0.0243517756462097 5.0
0 0.0049559473991394 6.0
0 -0.00916051864624023 7.0
39.66774326 39.3000450134277 0.0
0 0.00557941198348999 1.0
0 0.0528358221054077 2.0
0 -0.0230751037597656 3.0
0 0.0496906638145447 4.0
0 -0.0281819701194763 5.0
0 0.00943386554718018 6.0
0 -0.0162165760993958 7.0
41.240556194 39.485954284668 0.0
0 0.0104520320892334 1.0
0 0.0391672253608704 2.0
0 -0.0241261720657349 3.0
0 0.0442981123924255 4.0
0 -0.029579222202301 5.0
0 0.00497233867645264 6.0
0 -0.01584392786026 7.0
40.3201946 39.4521560668945 0.0
0 0.0161011219024658 1.0
0 0.0323320627212524 2.0
0 -0.026906430721283 3.0
0 0.021517276763916 4.0
0 -0.0247675180435181 5.0
0 0.00135600566864014 6.0
0 -0.0140462517738342 7.0
39.561151059 39.2782707214355 0.0
0 -0.00219696760177612 1.0
0 0.0362330079078674 2.0
0 -0.0246546864509583 3.0
0 0.0312585830688477 4.0
0 -0.0289401412010193 5.0
0 0.00624489784240723 6.0
0 -0.012248694896698 7.0
43.566431947 39.6282119750977 0.0
0 0.0138700008392334 1.0
0 0.025522768497467 2.0
0 -0.0197473764419556 3.0
0 0.0512849688529968 4.0
0 -0.0301445722579956 5.0
0 0.00931829214096069 6.0
0 -0.0174809098243713 7.0
37.929366562 39.231330871582 0.0
0 0.010800302028656 1.0
0 0.043013334274292 2.0
0 -0.0228856801986694 3.0
0 0.0199460387229919 4.0
0 -0.0271242260932922 5.0
0 0.00468665361404419 6.0
0 -0.011076033115387 7.0
39.745668641 41.7682800292969 0.0
0 0.00947737693786621 1.0
0 0.0982784032821655 2.0
0 -0.0173757672309875 3.0
0 0.00384366512298584 4.0
0 -0.0101626515388489 5.0
0 -0.000672698020935059 6.0
0 -0.00756984949111938 7.0
37.173794625 39.3144912719727 0.0
0 0.00357526540756226 1.0
0 0.0787873268127441 2.0
0 -0.0133687853813171 3.0
0 0.00534951686859131 4.0
0 -0.0198338031768799 5.0
0 0.00434726476669312 6.0
0 -0.00697380304336548 7.0
27.124866015 39.1525268554688 0.0
0 0.00913196802139282 1.0
0 0.0650781393051147 2.0
0 -0.00544387102127075 3.0
0 -0.00907778739929199 4.0
0 -0.0324002504348755 5.0
0 0.00577831268310547 6.0
0 -0.00207573175430298 7.0
27.411746904 36.7663421630859 0.0
0 0.0113141536712646 1.0
0 0.0197427868843079 2.0
0 -0.0141464471817017 3.0
0 -0.00583177804946899 4.0
0 -0.0341864824295044 5.0
0 0.00935220718383789 6.0
0 0.000265181064605713 7.0
38.866617317 39.1861763000488 0.0
0 0.00661951303482056 1.0
0 0.056182324886322 2.0
0 -0.021723210811615 3.0
0 0.0445036292076111 4.0
0 -0.029047966003418 5.0
0 0.00600475072860718 6.0
0 -0.0126184225082397 7.0
42.383057354 39.06201171875 0.0
0 0.0104770064353943 1.0
0 0.0411927103996277 2.0
0 -0.0221678614616394 3.0
0 0.0380172729492188 4.0
0 -0.0284179449081421 5.0
0 0.00465226173400879 6.0
0 -0.0151697993278503 7.0
47.643751036 39.1715316772461 0.0
0 0.000791609287261963 1.0
0 0.0480496883392334 2.0
0 -0.0215412378311157 3.0
0 0.053810715675354 4.0
0 -0.0269570350646973 5.0
0 0.00569742918014526 6.0
0 -0.0143246054649353 7.0
38.439399957 37.768684387207 0.0
0 0.00674319267272949 1.0
0 0.064213752746582 2.0
0 -0.0178337693214417 3.0
0 0.0179169774055481 4.0
0 -0.0244011878967285 5.0
0 0.0028071403503418 6.0
0 -0.0114209651947021 7.0
40.263562371 39.1669425964355 0.0
0 0.0111555457115173 1.0
0 0.0356557369232178 2.0
0 -0.0216387510299683 3.0
0 0.0576434135437012 4.0
0 -0.0272286534309387 5.0
0 0.0108609199523926 6.0
0 -0.0150966048240662 7.0
41.519528397 39.2469940185547 0.0
0 0.00689941644668579 1.0
0 0.0410262942314148 2.0
0 -0.0277458429336548 3.0
0 0.0269702672958374 4.0
0 -0.0244253873825073 5.0
0 0.0042043924331665 6.0
0 -0.0120133757591248 7.0
40.414570474 39.2424240112305 0.0
0 0.0107041597366333 1.0
0 0.0481968522071838 2.0
0 -0.0277416110038757 3.0
0 0.0130128264427185 4.0
0 -0.02088862657547 5.0
0 0.0020906925201416 6.0
0 -0.00745880603790283 7.0
37.834181221 39.3338584899902 0.0
0 0.00449800491333008 1.0
0 0.0605843067169189 2.0
0 -0.0180467963218689 3.0
0 0.0176889896392822 4.0
0 -0.0260162949562073 5.0
0 0.0046238899230957 6.0
0 -0.0100391507148743 7.0
37.201121556 39.261905670166 0.0
0 0.0136870741844177 1.0
0 0.0373289585113525 2.0
0 -0.0291894674301147 3.0
0 0.0102047920227051 4.0
0 -0.027827262878418 5.0
0 0.0071987509727478 6.0
0 -0.00730001926422119 7.0
42.94444514 39.2730560302734 0.0
0 0.0106481313705444 1.0
0 0.0675126910209656 2.0
0 -0.0177286267280579 3.0
0 0.0313223600387573 4.0
0 -0.0179081559181213 5.0
0 0.00385010242462158 6.0
0 -0.0127323865890503 7.0
47.309600034 39.3127517700195 0.0
0 0.0105430483818054 1.0
0 0.0398989319801331 2.0
0 -0.0270081758499146 3.0
0 0.0480376482009888 4.0
0 -0.0255812406539917 5.0
0 0.00514119863510132 6.0
0 -0.0150682926177979 7.0
34.703660747 39.2433204650879 0.0
0 0.0071418285369873 1.0
0 0.0537604689598083 2.0
0 -0.020726203918457 3.0
0 0.00321263074874878 4.0
0 -0.021537184715271 5.0
0 0.00369763374328613 6.0
0 -0.00550210475921631 7.0
39.355679833 39.1759719848633 0.0
0 0.00848031044006348 1.0
0 0.0391460657119751 2.0
0 -0.0200943946838379 3.0
0 0.0308890342712402 4.0
0 -0.0281250476837158 5.0
0 0.00596320629119873 6.0
0 -0.0112216472625732 7.0
36.760838135 39.1233863830566 0.0
0 0.00670671463012695 1.0
0 0.0653762817382812 2.0
0 -0.0138376355171204 3.0
0 0.0073549747467041 4.0
0 -0.0216763615608215 5.0
0 0.00479042530059814 6.0
0 -0.00692027807235718 7.0
45.817767369 39.8167724609375 0.0
0 0.0150848031044006 1.0
0 0.0327014923095703 2.0
0 -0.0391030311584473 3.0
0 0.056266725063324 4.0
0 -0.0295649170875549 5.0
0 0.00774598121643066 6.0
0 -0.0133306980133057 7.0
37.456387944 38.9908180236816 0.0
0 0.00394445657730103 1.0
0 0.0999663472175598 2.0
0 -0.00932472944259644 3.0
0 0.0233494639396667 4.0
0 -0.026712954044342 5.0
0 0.00659197568893433 6.0
0 -0.0119200348854065 7.0
36.979299113 39.0708389282227 0.0
0 0.00956428050994873 1.0
0 0.069067120552063 2.0
0 -0.0124732851982117 3.0
0 0.00505238771438599 4.0
0 -0.0217453241348267 5.0
0 0.00202125310897827 6.0
0 -0.00531578063964844 7.0
44.357392157 39.4652366638184 0.0
0 0.00713968276977539 1.0
0 0.0625491738319397 2.0
0 -0.0172478556632996 3.0
0 0.037219226360321 4.0
0 -0.0215816497802734 5.0
0 0.00404489040374756 6.0
0 -0.0137313008308411 7.0
36.812079299 39.2094459533691 0.0
0 0.00176447629928589 1.0
0 0.0411084890365601 2.0
0 -0.0210328102111816 3.0
0 0.0358702540397644 4.0
0 -0.026917576789856 5.0
0 0.00472712516784668 6.0
0 -0.0162766575813293 7.0
43.172254743 39.3419456481934 0.0
0 0.000443518161773682 1.0
0 0.0612084865570068 2.0
0 -0.0245235562324524 3.0
0 0.0217550992965698 4.0
0 -0.0219128727912903 5.0
0 0.00296598672866821 6.0
0 -0.0115439891815186 7.0
47.109370567 39.1553535461426 0.0
0 0.00755888223648071 1.0
0 0.0679612159729004 2.0
0 -0.0227341055870056 3.0
0 0.0571240186691284 4.0
0 -0.0264320373535156 5.0
0 0.0023043155670166 6.0
0 -0.0164520740509033 7.0
35.871355662 35.5522994995117 0.0
0 0.00543594360351562 1.0
0 0.0244265198707581 2.0
0 -0.0247161984443665 3.0
0 0.0317372679710388 4.0
0 -0.0295183062553406 5.0
0 0.0111661553382874 6.0
0 -0.0115025639533997 7.0
42.169433911 39.2534637451172 0.0
0 0.0117353200912476 1.0
0 0.062811553478241 2.0
0 -0.0239560604095459 3.0
0 0.0422621369361877 4.0
0 -0.0282317399978638 5.0
0 0.00753867626190186 6.0
0 -0.0137905478477478 7.0
38.001249401 39.3316230773926 0.0
0 0.00929278135299683 1.0
0 0.0384349226951599 2.0
0 -0.0226041078567505 3.0
0 0.0314591526985168 4.0
0 -0.0318945050239563 5.0
0 0.00869399309158325 6.0
0 -0.0126127600669861 7.0
36.026296153 39.0085372924805 0.0
0 0.00563478469848633 1.0
0 0.0343274474143982 2.0
0 -0.0304900407791138 3.0
0 0.0181832313537598 4.0
0 -0.0314008593559265 5.0
0 0.00474011898040771 6.0
0 -0.00925129652023315 7.0
33.200213159 39.6004867553711 0.0
0 0.00608885288238525 1.0
0 0.0842117667198181 2.0
0 -0.0121983289718628 3.0
0 0.00233906507492065 4.0
0 -0.0259609818458557 5.0
0 0.00225377082824707 6.0
0 -0.0079498291015625 7.0
36.881947952 39.2454299926758 0.0
0 -0.00684928894042969 1.0
0 0.0569502115249634 2.0
0 -0.0175040364265442 3.0
0 0.0261569619178772 4.0
0 -0.0287220478057861 5.0
0 0.00709050893783569 6.0
0 -0.0138422846794128 7.0
28.363012012 39.311351776123 0.0
0 0.00370430946350098 1.0
0 0.0378327369689941 2.0
0 -0.0167774558067322 3.0
0 -0.00418192148208618 4.0
0 -0.0319845080375671 5.0
0 0.00403189659118652 6.0
0 -0.00433927774429321 7.0
39.574467426 39.1792831420898 0.0
0 0.0124419927597046 1.0
0 0.0643044710159302 2.0
0 -0.0187584757804871 3.0
0 0.0552770495414734 4.0
0 -0.0264163017272949 5.0
0 0.00504046678543091 6.0
0 -0.0174887776374817 7.0
36.22040148 39.3068237304688 0.0
0 0.00884103775024414 1.0
0 0.0396783351898193 2.0
0 -0.0248955488204956 3.0
0 0.0143897533416748 4.0
0 -0.02907395362854 5.0
0 0.00538623332977295 6.0
0 -0.00855058431625366 7.0
31.086432141 39.3125267028809 0.0
0 -0.00659275054931641 1.0
0 0.0630043745040894 2.0
0 -0.0123117566108704 3.0
0 -0.004280686378479 4.0
0 -0.0248154997825623 5.0
0 0.00149953365325928 6.0
0 -0.00380051136016846 7.0
36.274353063 39.3318557739258 0.0
0 0.00835269689559937 1.0
0 0.0660974383354187 2.0
0 -0.0127261281013489 3.0
0 0.000513792037963867 4.0
0 -0.0210362672805786 5.0
0 0.00450301170349121 6.0
0 -0.00523066520690918 7.0
31.65795009 40.0534744262695 0.0
0 0.00283342599868774 1.0
0 0.0983990430831909 2.0
0 -0.0014115571975708 3.0
0 0.0108137130737305 4.0
0 -0.0277411937713623 5.0
0 0.0125419497489929 6.0
0 -0.00817984342575073 7.0
37.059171479 39.6106262207031 0.0
0 0.0132450461387634 1.0
0 0.0257991552352905 2.0
0 -0.0181992650032043 3.0
0 0.0418131947517395 4.0
0 -0.0292995572090149 5.0
0 0.0161434412002563 6.0
0 -0.014696478843689 7.0
45.308862357 39.1841506958008 0.0
0 0.00910359621047974 1.0
0 0.0663491487503052 2.0
0 -0.0206829309463501 3.0
0 0.0444293618202209 4.0
0 -0.0230956673622131 5.0
0 0.0029442310333252 6.0
0 -0.0143704414367676 7.0
32.988279909 39.4551620483398 0.0
0 0.00968480110168457 1.0
0 0.0575180649757385 2.0
0 -0.0297290086746216 3.0
0 0.00657224655151367 4.0
0 -0.0277795195579529 5.0
0 0.0065266489982605 6.0
0 -0.00360023975372314 7.0
41.812200593 42.8991012573242 0.0
0 -0.000670015811920166 1.0
0 0.0709576010704041 2.0
0 -0.022020161151886 3.0
0 0.0297814011573792 4.0
0 -0.0287713408470154 5.0
0 0.006084144115448 6.0
0 -0.0115088224411011 7.0
34.159578007 38.4319915771484 0.0
0 0.0111063122749329 1.0
0 0.034248411655426 2.0
0 -0.0333222150802612 3.0
0 0.00719785690307617 4.0
0 -0.0321474075317383 5.0
0 0.00268566608428955 6.0
0 -0.00621837377548218 7.0
41.353058204 39.204532623291 0.0
0 0.00952857732772827 1.0
0 0.081007719039917 2.0
0 -0.0175600051879883 3.0
0 0.046159565448761 4.0
0 -0.0286591649055481 5.0
0 0.00810974836349487 6.0
0 -0.0123917460441589 7.0
35.663236644 39.2133636474609 0.0
0 0.00283634662628174 1.0
0 0.0574272274971008 2.0
0 -0.0158212780952454 3.0
0 0.0198498964309692 4.0
0 -0.0298638343811035 5.0
0 0.00515240430831909 6.0
0 -0.0104875564575195 7.0
40.444916123 39.1653671264648 0.0
0 0.00414353609085083 1.0
0 0.0910453796386719 2.0
0 -0.0167463421821594 3.0
0 0.038258969783783 4.0
0 -0.0234010815620422 5.0
0 0.00739949941635132 6.0
0 -0.0134220123291016 7.0
39.900416914 39.0888595581055 0.0
0 0.00893539190292358 1.0
0 0.0359787344932556 2.0
0 -0.0211640000343323 3.0
0 0.0231063961982727 4.0
0 -0.0259357690811157 5.0
0 0.00329744815826416 6.0
0 -0.0172143578529358 7.0
34.405324849 39.97607421875 0.0
0 0.0172434449195862 1.0
0 0.0376632213592529 2.0
0 -0.0179058313369751 3.0
0 0.0187467336654663 4.0
0 -0.0309320092201233 5.0
0 0.00770348310470581 6.0
0 -0.00930380821228027 7.0
33.074256004 39.249397277832 0.0
0 0.00738769769668579 1.0
0 0.0685418844223022 2.0
0 -0.0165396332740784 3.0
0 0.00878924131393433 4.0
0 -0.028192400932312 5.0
0 0.00620543956756592 6.0
0 -0.0068475604057312 7.0
40.036170308 41.1524620056152 0.0
0 0.00465995073318481 1.0
0 0.0542076826095581 2.0
0 -0.0169073343276978 3.0
0 0.0155050754547119 4.0
0 -0.0213350653648376 5.0
0 0.00354379415512085 6.0
0 -0.00880926847457886 7.0
44.453206241 39.2687759399414 0.0
0 0.0084221363067627 1.0
0 0.0385976433753967 2.0
0 -0.0236493945121765 3.0
0 0.0401104688644409 4.0
0 -0.0222529768943787 5.0
0 0.0013420581817627 6.0
0 -0.0154792070388794 7.0
33.85004541 39.3401641845703 0.0
0 0.0100825428962708 1.0
0 0.0304266810417175 2.0
0 -0.0158665776252747 3.0
0 -0.00110459327697754 4.0
0 -0.0240499973297119 5.0
0 0.00319308042526245 6.0
0 -0.00285923480987549 7.0
30.537877308 38.9493637084961 0.0
0 0.00709092617034912 1.0
0 0.0584277510643005 2.0
0 -0.010155200958252 3.0
0 0.00216293334960938 4.0
0 -0.0305030345916748 5.0
0 0.00729340314865112 6.0
0 -0.00616723299026489 7.0
27.727440942 38.7583274841309 0.0
0 0.0101714730262756 1.0
0 0.0332983732223511 2.0
0 -0.00794166326522827 3.0
0 -0.0133890509605408 4.0
0 -0.0270724296569824 5.0
0 0.00212264060974121 6.0
0 0.000737905502319336 7.0
35.630182897 39.3510971069336 0.0
0 0.00258451700210571 1.0
0 0.0429811477661133 2.0
0 -0.0208121538162231 3.0
0 0.00156229734420776 4.0
0 -0.0212336182594299 5.0
0 0.00289946794509888 6.0
0 -0.00715601444244385 7.0
31.038251519 39.2236862182617 0.0
0 0.00821179151535034 1.0
0 0.0344059467315674 2.0
0 -0.0368871092796326 3.0
0 0.00533008575439453 4.0
0 -0.0332455635070801 5.0
0 0.00354135036468506 6.0
0 0.527037620544434 7.0
31.867011967 39.251651763916 0.0
0 0.00107324123382568 1.0
0 0.054533839225769 2.0
0 -0.0123904943466187 3.0
0 0.000221490859985352 4.0
0 -0.0262243151664734 5.0
0 0.00323355197906494 6.0
0 -0.00364047288894653 7.0
30.754831646 39.2847442626953 0.0
0 0.000662744045257568 1.0
0 0.0500643253326416 2.0
0 -0.0136899352073669 3.0
0 -0.00537914037704468 4.0
0 -0.0222152471542358 5.0
0 0.00284117460250854 6.0
0 -0.00049436092376709 7.0
38.95156428 39.6339340209961 0.0
0 0.0100151300430298 1.0
0 0.0536243319511414 2.0
0 -0.0311551094055176 3.0
0 0.0400863289833069 4.0
0 -0.0307068228721619 5.0
0 0.00652027130126953 6.0
0 -0.0129345059394836 7.0
33.384480936 40.8775634765625 0.0
0 -0.00124102830886841 1.0
0 0.082556426525116 2.0
0 -0.0175243020057678 3.0
0 0.00131016969680786 4.0
0 -0.0186545848846436 5.0
0 0.0023646354675293 6.0
0 -0.00465917587280273 7.0
37.291199282 39.228099822998 0.0
0 -0.00408190488815308 1.0
0 0.0565981864929199 2.0
0 -0.0146593451499939 3.0
0 -0.00024259090423584 4.0
0 -0.0163029432296753 5.0
0 -0.000132083892822266 6.0
0 -0.00570875406265259 7.0
33.571811016 39.1781425476074 0.0
0 -0.00461328029632568 1.0
0 0.0580087304115295 2.0
0 -0.0127177834510803 3.0
0 0.00326383113861084 4.0
0 -0.0277678370475769 5.0
0 0.00513482093811035 6.0
0 -0.00657427310943604 7.0
44.048761746 39.1108169555664 0.0
0 0.0133167505264282 1.0
0 0.0345013737678528 2.0
0 -0.0266736149787903 3.0
0 0.0249786972999573 4.0
0 -0.0194141864776611 5.0
0 0.0016477108001709 6.0
0 -0.0132050514221191 7.0
38.09281546 39.2019882202148 0.0
0 0.0076175332069397 1.0
0 0.0517597794532776 2.0
0 -0.0209031105041504 3.0
0 0.0179016590118408 4.0
0 -0.026101291179657 5.0
0 0.00557214021682739 6.0
0 -0.00983411073684692 7.0
42.616606924 39.1387519836426 0.0
0 0.0139750242233276 1.0
0 0.0358852744102478 2.0
0 -0.0247586965560913 3.0
0 0.0312950015068054 4.0
0 -0.0261248350143433 5.0
0 0.00321567058563232 6.0
0 -0.0164498090744019 7.0
41.080227074 39.3054122924805 0.0
0 0.00673854351043701 1.0
0 0.0645447969436646 2.0
0 -0.0166183114051819 3.0
0 0.0444710850715637 4.0
0 -0.0268169045448303 5.0
0 0.00533783435821533 6.0
0 -0.01416015625 7.0
36.859087813 39.3295364379883 0.0
0 0.011201024055481 1.0
0 0.0334835648536682 2.0
0 -0.0205079913139343 3.0
0 0.00591433048248291 4.0
0 -0.0269621610641479 5.0
0 0.00424182415008545 6.0
0 -0.00879001617431641 7.0
39.316268876 39.2730102539062 0.0
0 0.0123557448387146 1.0
0 0.0479531288146973 2.0
0 -0.026253879070282 3.0
0 0.0314406752586365 4.0
0 -0.0275577306747437 5.0
0 0.00751090049743652 6.0
0 -0.0127378702163696 7.0
40.023071967 39.1557769775391 0.0
0 0.00954633951187134 1.0
0 0.0609726905822754 2.0
0 -0.0256583690643311 3.0
0 0.0391594171524048 4.0
0 -0.0301165580749512 5.0
0 0.00344556570053101 6.0
0 -0.0116539001464844 7.0
32.691039566 39.2794570922852 0.0
0 -0.0015331506729126 1.0
0 0.0595820546150208 2.0
0 -0.0134975910186768 3.0
0 0.000495076179504395 4.0
0 -0.02903151512146 5.0
0 0.00216841697692871 6.0
0 -0.00742316246032715 7.0
39.35017223 39.2273406982422 0.0
0 0.0103467106819153 1.0
0 0.0433016419410706 2.0
0 -0.0213442444801331 3.0
0 0.0215471982955933 4.0
0 -0.0266906023025513 5.0
0 0.00573867559432983 6.0
0 -0.0110342502593994 7.0
38.62577922 39.0582504272461 0.0
0 0.00206905603408813 1.0
0 0.0952557921409607 2.0
0 -0.0130048990249634 3.0
0 0.0412141084671021 4.0
0 -0.0248604416847229 5.0
0 0.00641036033630371 6.0
0 -0.0157877802848816 7.0
43.36321018 39.2144470214844 0.0
0 0.000869214534759521 1.0
0 0.0585317015647888 2.0
0 -0.0207402110099792 3.0
0 0.0382145643234253 4.0
0 -0.0245130062103271 5.0
0 0.00293725728988647 6.0
0 -0.0152141451835632 7.0
41.711370519 39.4875259399414 0.0
0 0.0113811492919922 1.0
0 0.0472018122673035 2.0
0 -0.0314878225326538 3.0
0 0.0203774571418762 4.0
0 -0.0202274322509766 5.0
0 0.00135642290115356 6.0
0 -0.0107417106628418 7.0
41.401748585 39.4137382507324 0.0
0 0.00584369897842407 1.0
0 0.0841087698936462 2.0
0 -0.0202708840370178 3.0
0 0.0407012104988098 4.0
0 -0.026205837726593 5.0
0 0.00837677717208862 6.0
0 -0.0136880874633789 7.0
45.411303661 39.0654220581055 0.0
0 0.0100109577178955 1.0
0 0.0425238609313965 2.0
0 -0.0348400473594666 3.0
0 0.0446252226829529 4.0
0 -0.0273600816726685 5.0
0 0.00585973262786865 6.0
0 -0.011528491973877 7.0
46.173622738 39.1124382019043 0.0
0 0.00331169366836548 1.0
0 0.0585666298866272 2.0
0 -0.0199772119522095 3.0
0 0.0386195778846741 4.0
0 -0.0185463428497314 5.0
0 0.000947475433349609 6.0
0 -0.0127515196800232 7.0
43.970500892 39.0283279418945 0.0
0 0.0122954845428467 1.0
0 0.0489515066146851 2.0
0 -0.0201903581619263 3.0
0 0.047145664691925 4.0
0 -0.0270442962646484 5.0
0 0.00498276948928833 6.0
0 -0.0150191783905029 7.0
41.561964524 39.1679534912109 0.0
0 0.00369572639465332 1.0
0 0.0706163048744202 2.0
0 -0.0154618620872498 3.0
0 0.0192911028862 4.0
0 -0.0254248380661011 5.0
0 0.00374150276184082 6.0
0 -0.00939935445785522 7.0
42.043570834 39.0953025817871 0.0
0 0.0122132301330566 1.0
0 0.0394858121871948 2.0
0 -0.0237278342247009 3.0
0 0.0304210186004639 4.0
0 -0.0229068994522095 5.0
0 0.00413888692855835 6.0
0 -0.0118696093559265 7.0
42.728187908 39.2809371948242 0.0
0 0.00305062532424927 1.0
0 0.0476612448692322 2.0
0 -0.0216407179832458 3.0
0 0.0432797074317932 4.0
0 -0.0265682935714722 5.0
0 0.0038069486618042 6.0
0 -0.0142127871513367 7.0
46.647822945 39.149055480957 0.0
0 0.0102046728134155 1.0
0 0.0686874389648438 2.0
0 -0.0248903036117554 3.0
0 0.0577712059020996 4.0
0 -0.0270435810089111 5.0
0 0.00424551963806152 6.0
0 -0.0131389498710632 7.0
34.157298462 39.2264099121094 0.0
0 0.00863933563232422 1.0
0 0.0604138970375061 2.0
0 -0.0150264501571655 3.0
0 0.00885719060897827 4.0
0 -0.0287423133850098 5.0
0 0.00353121757507324 6.0
0 -0.00855332612991333 7.0
38.08940786 39.6085052490234 0.0
0 0.00380009412765503 1.0
0 0.0974803566932678 2.0
0 -0.0130190849304199 3.0
0 -0.00274521112442017 4.0
0 -0.0061413049697876 5.0
0 -0.00102388858795166 6.0
0 -0.00624740123748779 7.0
48.253456605 39.9512825012207 0.0
0 0.0059741735458374 1.0
0 0.0713126659393311 2.0
0 -0.0189449787139893 3.0
0 0.059030294418335 4.0
0 -0.0232948064804077 5.0
0 0.00563764572143555 6.0
0 -0.0141817331314087 7.0
41.303512955 39.4999008178711 0.0
0 0.00800210237503052 1.0
0 0.0413928627967834 2.0
0 -0.0225116014480591 3.0
0 0.0409272313117981 4.0
0 -0.0284388065338135 5.0
0 0.00299149751663208 6.0
0 -0.0151891708374023 7.0
34.11247552 39.3124771118164 0.0
0 0.0101791024208069 1.0
0 0.0430975556373596 2.0
0 -0.0161041021347046 3.0
0 -0.00142985582351685 4.0
0 -0.0245183706283569 5.0
0 0.00401866436004639 6.0
0 -0.00263869762420654 7.0
39.346899321 39.1841049194336 0.0
0 0.0121107697486877 1.0
0 0.0421473383903503 2.0
0 -0.0220503211021423 3.0
0 0.0479787588119507 4.0
0 -0.0291696786880493 5.0
0 0.00787287950515747 6.0
0 -0.015704333782196 7.0
42.355081321 39.1616287231445 0.0
0 0.00376284122467041 1.0
0 0.0752883553504944 2.0
0 -0.0166050791740417 3.0
0 0.0418010354042053 4.0
0 -0.0204254984855652 5.0
0 0.00545191764831543 6.0
0 -0.0175421833992004 7.0
46.025284619 39.0997200012207 0.0
0 0.0107316970825195 1.0
0 0.0411190390586853 2.0
0 -0.0379164218902588 3.0
0 0.0383503437042236 4.0
0 -0.0208945274353027 5.0
0 0.000841259956359863 6.0
0 -0.012586236000061 7.0
48.790909787 39.3467254638672 0.0
0 0.00544178485870361 1.0
0 0.0473642945289612 2.0
0 -0.0263350009918213 3.0
0 0.0592371225357056 4.0
0 -0.0242483615875244 5.0
0 0.0057494044303894 6.0
0 -0.0136021971702576 7.0
41.698940502 39.3436889648438 0.0
0 0.0103285908699036 1.0
0 0.0436311960220337 2.0
0 -0.0286429524421692 3.0
0 0.0448897480964661 4.0
0 -0.0313480496406555 5.0
0 0.00959122180938721 6.0
0 -0.0129580497741699 7.0
47.902972781 39.1910247802734 0.0
0 0.0104819536209106 1.0
0 0.0497010946273804 2.0
0 -0.0263906717300415 3.0
0 0.0467828512191772 4.0
0 -0.0187551379203796 5.0
0 0.00208765268325806 6.0
0 -0.0136069655418396 7.0
41.37717087 39.0758895874023 0.0
0 0.00883805751800537 1.0
0 0.0678055286407471 2.0
0 -0.0207234621047974 3.0
0 0.029201865196228 4.0
0 -0.0210285186767578 5.0
0 0.00388187170028687 6.0
0 -0.0106710195541382 7.0
32.285684902 39.1620063781738 0.0
0 0.00987410545349121 1.0
0 0.0596850514411926 2.0
0 -0.0258843302726746 3.0
0 -0.00052642822265625 4.0
0 -0.0234203934669495 5.0
0 0.00188684463500977 6.0
0 -0.00331372022628784 7.0
32.344206851 39.1839065551758 0.0
0 0.00693005323410034 1.0
0 0.0404759049415588 2.0
0 -0.0222252607345581 3.0
0 0.0102599263191223 4.0
0 -0.0320175290107727 5.0
0 0.00594037771224976 6.0
0 -0.00761276483535767 7.0
34.184112331 39.1595001220703 0.0
0 0.000822484493255615 1.0
0 0.0435900092124939 2.0
0 -0.0234014391899109 3.0
0 0.016313374042511 4.0
0 -0.0279474854469299 5.0
0 0.00414592027664185 6.0
0 -0.0104918479919434 7.0
34.663221808 39.3465347290039 0.0
0 -0.000472068786621094 1.0
0 0.0439326167106628 2.0
0 -0.0197044610977173 3.0
0 0.0183951854705811 4.0
0 -0.0307208895683289 5.0
0 0.00468224287033081 6.0
0 -0.00934088230133057 7.0
36.873774543 37.5921669006348 0.0
0 0.0069698691368103 1.0
0 0.0686353445053101 2.0
0 -0.0163471102714539 3.0
0 0.0109089612960815 4.0
0 -0.0253394842147827 5.0
0 0.00617462396621704 6.0
0 -0.0072709321975708 7.0
40.211352908 41.326587677002 0.0
0 0.00541871786117554 1.0
0 0.0645110607147217 2.0
0 -0.01763916015625 3.0
0 0.0509659647941589 4.0
0 -0.027746319770813 5.0
0 0.0062638521194458 6.0
0 -0.0150855183601379 7.0
46.160689867 39.1345748901367 0.0
0 0.00654143095016479 1.0
0 0.0590987801551819 2.0
0 -0.0207692384719849 3.0
0 0.0336949825286865 4.0
0 -0.0170185565948486 5.0
0 0.000889062881469727 6.0
0 -0.0137842893600464 7.0
32.727702676 39.3031883239746 0.0
0 -0.00463449954986572 1.0
0 0.0640050768852234 2.0
0 -0.0136813521385193 3.0
0 0.00972533226013184 4.0
0 -0.0276746153831482 5.0
0 0.00688004493713379 6.0
0 -0.00769346952438354 7.0
40.640985417 39.6990623474121 0.0
0 0.00356471538543701 1.0
0 0.0938190817832947 2.0
0 -0.0121647715568542 3.0
0 0.0314699411392212 4.0
0 -0.0217582583427429 5.0
0 0.00673258304595947 6.0
0 -0.0126134753227234 7.0
34.611451594 39.2935600280762 0.0
0 0.000468611717224121 1.0
0 0.0598671436309814 2.0
0 -0.0131622552871704 3.0
0 0.00562536716461182 4.0
0 -0.0259057879447937 5.0
0 0.00558769702911377 6.0
0 -0.00558269023895264 7.0
43.311280807 39.3564987182617 0.0
0 0.00210869312286377 1.0
0 0.0423729419708252 2.0
0 -0.0237454771995544 3.0
0 0.0202456712722778 4.0
0 -0.0209356546401978 5.0
0 0.00234854221343994 6.0
0 -0.0102128982543945 7.0
36.406364057 39.175666809082 0.0
0 0.0102632641792297 1.0
0 0.0409222841262817 2.0
0 -0.0188425183296204 3.0
0 0.00562071800231934 4.0
0 -0.0216001868247986 5.0
0 0.00233513116836548 6.0
0 -0.00800949335098267 7.0
31.685630326 40.3169288635254 0.0
0 0.00647974014282227 1.0
0 0.101521492004395 2.0
0 -0.00225663185119629 3.0
0 0.00412189960479736 4.0
0 -0.0269104242324829 5.0
0 0.00584179162979126 6.0
0 -0.00681865215301514 7.0
40.165364432 39.4257736206055 0.0
0 0.0061718225479126 1.0
0 0.105965197086334 2.0
0 -0.0121121406555176 3.0
0 0.0475353002548218 4.0
0 -0.0240756273269653 5.0
0 0.00451123714447021 6.0
0 -0.0161829590797424 7.0
42.886465647 39.1565208435059 0.0
0 0.00445181131362915 1.0
0 0.058378279209137 2.0
0 -0.0202310085296631 3.0
0 0.0512354969978333 4.0
0 -0.0279179811477661 5.0
0 0.00581037998199463 6.0
0 -0.0149148106575012 7.0
46.942592332 40.7634811401367 0.0
0 0.00733822584152222 1.0
0 0.0581620931625366 2.0
0 -0.0207223296165466 3.0
0 0.0556744337081909 4.0
0 -0.0250055193901062 5.0
0 0.00304579734802246 6.0
0 -0.0171255469322205 7.0
38.417030568 39.1845207214355 0.0
0 -0.000441789627075195 1.0
0 0.105224907398224 2.0
0 -0.00935876369476318 3.0
0 0.0345804691314697 4.0
0 -0.0238878726959229 5.0
0 0.00515294075012207 6.0
0 -0.0114102363586426 7.0
39.82159543 39.3340148925781 0.0
0 -0.00290924310684204 1.0
0 0.0601381659507751 2.0
0 -0.0178883075714111 3.0
0 0.0185866355895996 4.0
0 -0.0235638618469238 5.0
0 0.00506407022476196 6.0
0 -0.00952208042144775 7.0
42.375846269 40.3157577514648 0.0
0 0.00914698839187622 1.0
0 0.0349929332733154 2.0
0 -0.0344812273979187 3.0
0 0.013907790184021 4.0
0 -0.0226353406906128 5.0
0 0.00246036052703857 6.0
0 -0.010317862033844 7.0
41.909868924 39.2151527404785 0.0
0 0.00799280405044556 1.0
0 0.0414422750473022 2.0
0 -0.022936224937439 3.0
0 0.0615046620368958 4.0
0 -0.0280951261520386 5.0
0 0.0132050514221191 6.0
0 -0.0195711851119995 7.0
38.515403814 39.1933288574219 0.0
0 0.00757837295532227 1.0
0 0.0305657386779785 2.0
0 -0.0187130570411682 3.0
0 0.00982528924942017 4.0
0 -0.0262295603752136 5.0
0 0.00381177663803101 6.0
0 -0.0142884850502014 7.0
32.187464773 40.187915802002 0.0
0 0.00200361013412476 1.0
0 0.064132034778595 2.0
0 -0.0111357569694519 3.0
0 -0.000443577766418457 4.0
0 -0.025220513343811 5.0
0 0.0035330057144165 6.0
0 -0.00435435771942139 7.0
43.553340498 39.1683349609375 0.0
0 0.00320440530776978 1.0
0 0.0983927845954895 2.0
0 -0.0131569504737854 3.0
0 0.0549652576446533 4.0
0 -0.0220618844032288 5.0
0 0.00318139791488647 6.0
0 -0.0133134126663208 7.0
38.661685118 39.4650688171387 0.0
0 0.00699847936630249 1.0
0 0.0637048482894897 2.0
0 -0.0227053165435791 3.0
0 0.053022027015686 4.0
0 -0.0268428325653076 5.0
0 0.00573450326919556 6.0
0 -0.0151920914649963 7.0
47.185976948 38.8878593444824 0.0
0 0.00848960876464844 1.0
0 0.0649888515472412 2.0
0 -0.0193367600440979 3.0
0 0.0486640930175781 4.0
0 -0.0238392949104309 5.0
0 0.00460058450698853 6.0
0 -0.0163248777389526 7.0
41.073649825 39.4418640136719 0.0
0 0.01736980676651 1.0
0 0.0300519466400146 2.0
0 -0.0189222693443298 3.0
0 0.0178244709968567 4.0
0 -0.0293866395950317 5.0
0 0.00656688213348389 6.0
0 -0.00890588760375977 7.0
39.086114846 39.2041931152344 0.0
0 0.00746828317642212 1.0
0 0.0617871284484863 2.0
0 -0.0160078406333923 3.0
0 0.0467209815979004 4.0
0 -0.0276315808296204 5.0
0 0.00437235832214355 6.0
0 -0.0158487558364868 7.0
38.072309151 39.6437606811523 0.0
0 0.00959813594818115 1.0
0 0.0610776543617249 2.0
0 -0.0135322213172913 3.0
0 0.00896525382995605 4.0
0 -0.023659884929657 5.0
0 0.00476402044296265 6.0
0 -0.00752836465835571 7.0
41.585402374 39.4675216674805 0.0
0 0.0107553005218506 1.0
0 0.0416585206985474 2.0
0 -0.0218178033828735 3.0
0 0.0383657217025757 4.0
0 -0.0271846652030945 5.0
0 0.00441157817840576 6.0
0 -0.0129627585411072 7.0
42.317317809 39.0025901794434 0.0
0 0.0110228657722473 1.0
0 0.0656244158744812 2.0
0 -0.0249553322792053 3.0
0 0.0582363605499268 4.0
0 -0.0284330248832703 5.0
0 0.00378590822219849 6.0
0 -0.0140643119812012 7.0
42.931477585 39.7166442871094 0.0
0 0.00569617748260498 1.0
0 0.0841277241706848 2.0
0 -0.0175623297691345 3.0
0 0.0348663926124573 4.0
0 -0.0192890167236328 5.0
0 0.00405728816986084 6.0
0 -0.0120517611503601 7.0
46.096274411 39.246166229248 0.0
0 -0.00434160232543945 1.0
0 0.0482379794120789 2.0
0 -0.0227041244506836 3.0
0 0.0343828797340393 4.0
0 -0.0184457898139954 5.0
0 0.000456690788269043 6.0
0 -0.0140078067779541 7.0
39.955261554 39.2918701171875 0.0
0 0.0116240978240967 1.0
0 0.0501741766929626 2.0
0 -0.0237830281257629 3.0
0 0.0337701439857483 4.0
0 -0.0290184617042542 5.0
0 0.00481361150741577 6.0
0 -0.0128209590911865 7.0
45.415961846 39.1588554382324 0.0
0 0.00861090421676636 1.0
0 0.0404647588729858 2.0
0 -0.0227407813072205 3.0
0 0.0548158288002014 4.0
0 -0.0274008512496948 5.0
0 0.00865215063095093 6.0
0 -0.0156762003898621 7.0
44.275746657 39.1352005004883 0.0
0 0.00998026132583618 1.0
0 0.0705584287643433 2.0
0 -0.0192747116088867 3.0
0 0.0429856181144714 4.0
0 -0.0221502780914307 5.0
0 0.00686490535736084 6.0
0 -0.0132014751434326 7.0
34.766238966 39.1581649780273 0.0
0 0.0108010172843933 1.0
0 0.066469132900238 2.0
0 -0.0145248174667358 3.0
0 0.0109792947769165 4.0
0 -0.0286319255828857 5.0
0 0.00657302141189575 6.0
0 -0.00772237777709961 7.0
39.697519492 40.0110397338867 0.0
0 0.00621956586837769 1.0
0 0.0532541871070862 2.0
0 -0.0158854722976685 3.0
0 0.0149997472763062 4.0
0 -0.0200769305229187 5.0
0 0.00282829999923706 6.0
0 -0.0118199586868286 7.0
36.459952599 39.2017936706543 0.0
0 0.00732451677322388 1.0
0 0.0604976415634155 2.0
0 -0.016093373298645 3.0
0 0.00341081619262695 4.0
0 -0.0244671106338501 5.0
0 0.00652408599853516 6.0
0 -0.00836789608001709 7.0
44.856515586 39.419303894043 0.0
0 0.00477176904678345 1.0
0 0.0616564154624939 2.0
0 -0.0213950872421265 3.0
0 0.0443920493125916 4.0
0 -0.0205050706863403 5.0
0 0.00288867950439453 6.0
0 -0.0148175954818726 7.0
41.157913956 39.2678260803223 0.0
0 0.00864797830581665 1.0
0 0.0603436231613159 2.0
0 -0.0151415467262268 3.0
0 0.0148828029632568 4.0
0 -0.0174505710601807 5.0
0 0.00234842300415039 6.0
0 -0.00891876220703125 7.0
32.092085792 42.0078506469727 0.0
0 -0.00147992372512817 1.0
0 0.0986078977584839 2.0
0 -0.00447070598602295 3.0
0 0.00738292932510376 4.0
0 -0.027155876159668 5.0
0 0.00754690170288086 6.0
0 -0.00817745923995972 7.0
30.327518425 39.2771072387695 0.0
0 0.000197649002075195 1.0
0 0.05263751745224 2.0
0 -0.00881850719451904 3.0
0 -0.00964021682739258 4.0
0 -0.0268546342849731 5.0
0 0.00452160835266113 6.0
0 -0.00201511383056641 7.0
36.482977278 39.6762313842773 0.0
0 0.00489264726638794 1.0
0 0.0885087847709656 2.0
0 -0.0142347812652588 3.0
0 0.0247588753700256 4.0
0 -0.028696596622467 5.0
0 0.00561767816543579 6.0
0 -0.0122042894363403 7.0
40.491170914 40.3549003601074 0.0
0 0.000296533107757568 1.0
0 0.0906490683555603 2.0
0 -0.0146321058273315 3.0
0 0.0381223559379578 4.0
0 -0.0233554840087891 5.0
0 0.00532442331314087 6.0
0 -0.012855052947998 7.0
41.188898334 40.0176544189453 0.0
0 -0.00143259763717651 1.0
0 0.0951639413833618 2.0
0 -0.0130124092102051 3.0
0 0.0215404033660889 4.0
0 -0.0166791677474976 5.0
0 0.00263166427612305 6.0
0 -0.00915664434432983 7.0
37.756949904 39.1947593688965 0.0
0 0.0107933282852173 1.0
0 0.0656872391700745 2.0
0 -0.0142897963523865 3.0
0 0.0106445550918579 4.0
0 -0.0203924179077148 5.0
0 0.00396895408630371 6.0
0 -0.00912606716156006 7.0
26.9626458 39.3925285339355 0.0
0 0.0110253691673279 1.0
0 0.0370810627937317 2.0
0 -0.00381720066070557 3.0
0 -0.016218900680542 4.0
0 -0.0292441248893738 5.0
0 0.00169575214385986 6.0
0 -0.00216329097747803 7.0
44.154680866 39.3581161499023 0.0
0 0.00145381689071655 1.0
0 0.0635524392127991 2.0
0 -0.0199567079544067 3.0
0 0.029373824596405 4.0
0 -0.0141153931617737 5.0
0 0.000345468521118164 6.0
0 -0.0163211822509766 7.0
36.49920361 39.3040390014648 0.0
0 0.00980162620544434 1.0
0 0.046657919883728 2.0
0 -0.0151821374893188 3.0
0 0.0272837281227112 4.0
0 -0.0288670659065247 5.0
0 0.00800848007202148 6.0
0 -0.0117116570472717 7.0
36.136187007 39.2891960144043 0.0
0 0.0114787220954895 1.0
0 0.0657549500465393 2.0
0 -0.0165747404098511 3.0
0 0.0228102207183838 4.0
0 -0.0275917053222656 5.0
0 0.00385874509811401 6.0
0 -0.0123043656349182 7.0
43.805905709 39.2161407470703 0.0
0 0.00916695594787598 1.0
0 0.0414623022079468 2.0
0 -0.0270732641220093 3.0
0 0.0342727899551392 4.0
0 -0.0226308107376099 5.0
0 0.00324547290802002 6.0
0 -0.012790858745575 7.0
43.400673478 38.5996551513672 0.0
0 0.00959324836730957 1.0
0 0.07652348279953 2.0
0 -0.0166631937026978 3.0
0 0.0510177612304688 4.0
0 -0.0256300568580627 5.0
0 0.00306117534637451 6.0
0 -0.0166022777557373 7.0
35.708898932 39.7909126281738 0.0
0 0.00258690118789673 1.0
0 0.101891338825226 2.0
0 -0.0113672614097595 3.0
0 0.0187270045280457 4.0
0 -0.026531994342804 5.0
0 0.00352787971496582 6.0
0 -0.010202944278717 7.0
42.195375979 39.2065582275391 0.0
0 0.00936859846115112 1.0
0 0.0624610185623169 2.0
0 -0.0231862664222717 3.0
0 0.0374898910522461 4.0
0 -0.0261919498443604 5.0
0 0.00701963901519775 6.0
0 -0.0106062889099121 7.0
38.219132252 39.2476501464844 0.0
0 0.00796747207641602 1.0
0 0.060630202293396 2.0
0 -0.0189728736877441 3.0
0 0.00981682538986206 4.0
0 -0.0244853496551514 5.0
0 0.00241279602050781 6.0
0 -0.00450032949447632 7.0
37.34785904 40.7463417053223 0.0
0 0.00804460048675537 1.0
0 0.0386269092559814 2.0
0 -0.0224984884262085 3.0
0 0.0120346546173096 4.0
0 -0.0248059034347534 5.0
0 0.0053831934928894 6.0
0 -0.00742852687835693 7.0
41.413317083 39.1686630249023 0.0
0 0.0103911161422729 1.0
0 0.0369403958320618 2.0
0 -0.0325966477394104 3.0
0 0.0260374546051025 4.0
0 -0.0254538059234619 5.0
0 0.00346553325653076 6.0
0 -0.0124757289886475 7.0
39.408474012 41.2074737548828 0.0
0 0.00463259220123291 1.0
0 0.0393736362457275 2.0
0 -0.0281912088394165 3.0
0 0.0336098074913025 4.0
0 -0.0318562984466553 5.0
0 0.00816363096237183 6.0
0 -0.0106446146965027 7.0
44.621947793 39.146858215332 0.0
0 0.0129572153091431 1.0
0 0.0404406189918518 2.0
0 -0.0323220491409302 3.0
0 0.0574005246162415 4.0
0 -0.0294423699378967 5.0
0 0.0104334950447083 6.0
0 -0.0138011574745178 7.0
35.042314673 41.5502014160156 0.0
0 0.0104658603668213 1.0
0 0.0302814841270447 2.0
0 -0.0129294991493225 3.0
0 0.00461137294769287 4.0
0 -0.0273417234420776 5.0
0 0.00530403852462769 6.0
0 -0.00574469566345215 7.0
44.668685987 37.9078559875488 0.0
0 0.00188130140304565 1.0
0 0.0569884777069092 2.0
0 -0.0238006114959717 3.0
0 0.0405648350715637 4.0
0 -0.0205771923065186 5.0
0 0.00241458415985107 6.0
0 -0.0139407515525818 7.0
44.271188235 39.1885566711426 0.0
0 0.0127209424972534 1.0
0 0.0644296407699585 2.0
0 -0.0185632109642029 3.0
0 0.0303701162338257 4.0
0 -0.0184800624847412 5.0
0 0.00196337699890137 6.0
0 -0.0148364305496216 7.0
43.329615715 39.419849395752 0.0
0 0.00306707620620728 1.0
0 0.0421066284179688 2.0
0 -0.0237050652503967 3.0
0 0.0451827645301819 4.0
0 -0.0286803841590881 5.0
0 0.00414633750915527 6.0
0 -0.0128758549690247 7.0
37.727034893 39.9785270690918 0.0
0 0.00373846292495728 1.0
0 0.103487849235535 2.0
0 -0.00995522737503052 3.0
0 0.0277867317199707 4.0
0 -0.0272249579429626 5.0
0 0.00663357973098755 6.0
0 -0.0104453563690186 7.0
43.665312113 39.5566329956055 0.0
0 0.00762540102005005 1.0
0 0.0750016570091248 2.0
0 -0.0351837277412415 3.0
0 0.0328378081321716 4.0
0 -0.0215832591056824 5.0
0 0.00253617763519287 6.0
0 -0.0119035840034485 7.0
40.656348383 39.3315620422363 0.0
0 0.00649392604827881 1.0
0 0.0533838868141174 2.0
0 -0.0215229988098145 3.0
0 0.0549891591072083 4.0
0 -0.0267018675804138 5.0
0 0.0048946738243103 6.0
0 -0.0185717940330505 7.0
30.759139555 39.0844345092773 0.0
0 0.0121983289718628 1.0
0 0.0359728336334229 2.0
0 -0.0149562954902649 3.0
0 -0.00471550226211548 4.0
0 -0.028002142906189 5.0
0 0.0035325288772583 6.0
0 -0.00117355585098267 7.0
36.474229652 40.8910751342773 0.0
0 0.00264835357666016 1.0
0 0.0991296172142029 2.0
0 -0.00933718681335449 3.0
0 0.00034099817276001 4.0
0 -0.0150880813598633 5.0
0 -0.000272393226623535 6.0
0 -0.00282025337219238 7.0
33.340721195 39.4016265869141 0.0
0 0.00756460428237915 1.0
0 0.0755277276039124 2.0
0 -0.0127190351486206 3.0
0 0.00348019599914551 4.0
0 -0.0280179381370544 5.0
0 0.00535142421722412 6.0
0 -0.00605171918869019 7.0
34.143349268 41.1993026733398 0.0
0 0.0105020999908447 1.0
0 0.0738517045974731 2.0
0 -0.0288435220718384 3.0
0 0.00792866945266724 4.0
0 -0.022797167301178 5.0
0 0.00235277414321899 6.0
0 -0.00532907247543335 7.0
40.763829757 40.4009170532227 0.0
0 0.00627636909484863 1.0
0 0.0678209066390991 2.0
0 -0.0207937955856323 3.0
0 0.0571699142456055 4.0
0 -0.0280265212059021 5.0
0 0.00891619920730591 6.0
0 -0.0133203268051147 7.0
29.788224958 39.3779296875 0.0
0 -0.00283634662628174 1.0
0 0.0589134097099304 2.0
0 -0.011816680431366 3.0
0 0.000213921070098877 4.0
0 -0.03299880027771 5.0
0 0.00723862648010254 6.0
0 -0.00425004959106445 7.0
34.675869554 39.2007484436035 0.0
0 0.00924175977706909 1.0
0 0.062283456325531 2.0
0 -0.0111774206161499 3.0
0 0.00481492280960083 4.0
0 -0.0254631042480469 5.0
0 0.00442290306091309 6.0
0 -0.00621861219406128 7.0
40.723855828 39.139232635498 0.0
0 0.004505455493927 1.0
0 0.0670126080513 2.0
0 -0.0169151425361633 3.0
0 0.0427116751670837 4.0
0 -0.0239702463150024 5.0
0 0.00283890962600708 6.0
0 -0.0185911655426025 7.0
41.414838352 39.2057685852051 0.0
0 0.002247154712677 1.0
0 0.059136688709259 2.0
0 -0.0206869840621948 3.0
0 0.0594696998596191 4.0
0 -0.0269739031791687 5.0
0 0.0113312602043152 6.0
0 -0.0169715285301208 7.0
33.391441214 39.5404624938965 0.0
0 0.00687164068222046 1.0
0 0.068155825138092 2.0
0 -0.0151040554046631 3.0
0 0.00336939096450806 4.0
0 -0.0272725820541382 5.0
0 0.00364166498184204 6.0
0 -0.0084540843963623 7.0
36.660645385 39.1693992614746 0.0
0 0.0112992525100708 1.0
0 0.0539188981056213 2.0
0 -0.0432652831077576 3.0
0 0.0196741223335266 4.0
0 -0.0308671593666077 5.0
0 0.00485384464263916 6.0
0 -0.00823885202407837 7.0
33.547567166 41.1571044921875 0.0
0 0.00413417816162109 1.0
0 0.0913617014884949 2.0
0 -0.0106722712516785 3.0
0 0.0163525938987732 4.0
0 -0.0269976854324341 5.0
0 0.00591397285461426 6.0
0 -0.00831866264343262 7.0
36.717923495 39.3228797912598 0.0
0 -0.00490498542785645 1.0
0 0.0684824585914612 2.0
0 -0.0163276791572571 3.0
0 0.0187729001045227 4.0
0 -0.0261236429214478 5.0
0 0.00353103876113892 6.0
0 -0.0108597874641418 7.0
44.575737494 40.2921981811523 0.0
0 0.00726407766342163 1.0
0 0.0921012759208679 2.0
0 -0.0169275999069214 3.0
0 0.0439640879631042 4.0
0 -0.0148934721946716 5.0
0 0.00246167182922363 6.0
0 -0.0144742131233215 7.0
43.091610321 39.251880645752 0.0
0 0.00294780731201172 1.0
0 0.0471649765968323 2.0
0 -0.0214514136314392 3.0
0 0.0389942526817322 4.0
0 -0.0256434082984924 5.0
0 0.00726276636123657 6.0
0 -0.0136053562164307 7.0
42.647069973 39.2061424255371 0.0
0 0.00906306505203247 1.0
0 0.0546945333480835 2.0
0 -0.0212883949279785 3.0
0 0.0380060076713562 4.0
0 -0.0249283909797668 5.0
0 0.00426143407821655 6.0
0 -0.0141628384590149 7.0
36.240225339 39.1404800415039 0.0
0 0.00918090343475342 1.0
0 0.0593395233154297 2.0
0 -0.0215871930122375 3.0
0 0.0104026794433594 4.0
0 -0.0252101421356201 5.0
0 0.00523948669433594 6.0
0 -0.00701224803924561 7.0
31.876756203 39.139762878418 0.0
0 -0.000991940498352051 1.0
0 0.0431054830551147 2.0
0 -0.0127851963043213 3.0
0 -0.00101745128631592 4.0
0 -0.0230093598365784 5.0
0 0.00262880325317383 6.0
0 -0.00305753946304321 7.0
38.711049659 39.3581237792969 0.0
0 0.00102627277374268 1.0
0 0.0828373432159424 2.0
0 -0.0169112682342529 3.0
0 0.00627750158309937 4.0
0 -0.0123506188392639 5.0
0 -0.000407099723815918 6.0
0 -0.00948333740234375 7.0
33.592635672 39.4242515563965 0.0
0 0.00957810878753662 1.0
0 0.0583613514900208 2.0
0 -0.016112208366394 3.0
0 0.00602555274963379 4.0
0 -0.027643084526062 5.0
0 0.00605815649032593 6.0
0 -0.00576615333557129 7.0
31.939497021 39.1622772216797 0.0
0 0.00103104114532471 1.0
0 0.0396385192871094 2.0
0 -0.0136800408363342 3.0
0 -0.00527811050415039 4.0
0 -0.0252284407615662 5.0
0 0.00408285856246948 6.0
0 -0.00103867053985596 7.0
32.605677369 39.2626190185547 0.0
0 0.00559169054031372 1.0
0 0.0369923114776611 2.0
0 -0.0197780132293701 3.0
0 0.000530064105987549 4.0
0 -0.0273765325546265 5.0
0 0.00718575716018677 6.0
0 -0.00460433959960938 7.0
35.02561425 39.1090469360352 0.0
0 0.0119553208351135 1.0
0 0.0453833937644958 2.0
0 -0.0258418917655945 3.0
0 0.0193158388137817 4.0
0 -0.0308746695518494 5.0
0 0.0087929368019104 6.0
0 -0.007557213306427 7.0
29.155020371 39.2253952026367 0.0
0 0.0101317167282104 1.0
0 0.0378549098968506 2.0
0 -0.0250255465507507 3.0
0 0.00145792961120605 4.0
0 -0.0328030586242676 5.0
0 0.00670236349105835 6.0
0 0.182495445013046 7.0
29.9790999 38.9883193969727 0.0
0 -0.00512790679931641 1.0
0 0.0625416040420532 2.0
0 -0.00465649366378784 3.0
0 -0.00537872314453125 4.0
0 -0.0296611785888672 5.0
0 0.00600504875183105 6.0
0 -0.00503420829772949 7.0
40.412374428 39.2049255371094 0.0
0 -0.000606119632720947 1.0
0 0.0611275434494019 2.0
0 -0.0168014168739319 3.0
0 0.00644654035568237 4.0
0 -0.0177046060562134 5.0
0 0.00162214040756226 6.0
0 -0.0136350393295288 7.0
44.799669607 39.0750770568848 0.0
0 0.00750100612640381 1.0
0 0.0563011765480042 2.0
0 -0.0221543312072754 3.0
0 0.0368587374687195 4.0
0 -0.0180134177207947 5.0
0 0.000889480113983154 6.0
0 -0.0161266922950745 7.0
35.691930085 39.3859329223633 0.0
0 0.00633901357650757 1.0
0 0.0712013244628906 2.0
0 -0.0153719782829285 3.0
0 0.00453132390975952 4.0
0 -0.0222364664077759 5.0
0 0.00377142429351807 6.0
0 -0.00545322895050049 7.0
37.293002382 39.293083190918 0.0
0 0.00778055191040039 1.0
0 0.047900378704071 2.0
0 -0.0283905863761902 3.0
0 0.0200028419494629 4.0
0 -0.0298619866371155 5.0
0 0.00535517930984497 6.0
0 -0.011212944984436 7.0
31.749050635 39.1932945251465 0.0
0 0.00823521614074707 1.0
0 0.0382753014564514 2.0
0 -0.0209130644798279 3.0
0 0.00449585914611816 4.0
0 -0.028892993927002 5.0
0 0.00504481792449951 6.0
0 -0.00455832481384277 7.0
44.438599316 38.0644073486328 0.0
0 0.00910055637359619 1.0
0 0.0278371572494507 2.0
0 -0.0197588205337524 3.0
0 0.0566070675849915 4.0
0 -0.0264150500297546 5.0
0 0.00440853834152222 6.0
0 -0.0190314054489136 7.0
30.584762345 39.4195861816406 0.0
0 0.00186467170715332 1.0
0 0.0638080835342407 2.0
0 -0.0128000378608704 3.0
0 -0.00119149684906006 4.0
0 -0.0299015045166016 5.0
0 0.00222301483154297 6.0
0 -0.00576400756835938 7.0
36.167711467 40.7079963684082 0.0
0 0.00583952665328979 1.0
0 0.0441656112670898 2.0
0 -0.0209394693374634 3.0
0 0.00305551290512085 4.0
0 -0.019661009311676 5.0
0 -0.000128626823425293 6.0
0 -0.0033729076385498 7.0
45.589023165 39.2716369628906 0.0
0 0.0118179321289062 1.0
0 0.0400128960609436 2.0
0 -0.0317465662956238 3.0
0 0.0473290085792542 4.0
0 -0.0295619964599609 5.0
0 0.00329500436782837 6.0
0 -0.0119420289993286 7.0
33.954863975 39.3449859619141 0.0
0 -0.00191313028335571 1.0
0 0.057437539100647 2.0
0 -0.0151700377464294 3.0
0 0.00851249694824219 4.0
0 -0.0251034498214722 5.0
0 0.00245237350463867 6.0
0 -0.0112330913543701 7.0
39.104021217 39.2743453979492 0.0
0 0.00290310382843018 1.0
0 0.0589911937713623 2.0
0 -0.018439769744873 3.0
0 0.0172427892684937 4.0
0 -0.020279586315155 5.0
0 0.00379747152328491 6.0
0 -0.0103532075881958 7.0
41.615139375 39.283073425293 0.0
0 0.00299423933029175 1.0
0 0.0724921226501465 2.0
0 -0.0180994868278503 3.0
0 0.058605432510376 4.0
0 -0.0211931467056274 5.0
0 0.00603270530700684 6.0
0 -0.0181660652160645 7.0
29.251209169 39.1626129150391 0.0
0 0.0102241635322571 1.0
0 0.0657804012298584 2.0
0 -0.0218889713287354 3.0
0 0.000598728656768799 4.0
0 -0.0287654995918274 5.0
0 0.00124531984329224 6.0
0 -0.00544106960296631 7.0
39.145261601 39.434383392334 0.0
0 0.0106642246246338 1.0
0 0.0615565180778503 2.0
0 -0.0195338726043701 3.0
0 0.0287595987319946 4.0
0 -0.0294601917266846 5.0
0 0.0090785026550293 6.0
0 -0.0129284858703613 7.0
43.724223514 39.9443969726562 0.0
0 0.00348567962646484 1.0
0 0.0611519813537598 2.0
0 -0.0157661437988281 3.0
0 0.0497835278511047 4.0
0 -0.0242002606391907 5.0
0 0.00416934490203857 6.0
0 -0.016603946685791 7.0
30.694427948 39.1826820373535 0.0
0 0.00585633516311646 1.0
0 0.0551376342773438 2.0
0 -0.00917786359786987 3.0
0 -0.00169706344604492 4.0
0 -0.0302058458328247 5.0
0 0.00556594133377075 6.0
0 -0.00630462169647217 7.0
46.899446079 39.3772010803223 0.0
0 0.00522828102111816 1.0
0 0.119101643562317 2.0
0 -0.0105419754981995 3.0
0 0.105043053627014 4.0
0 -0.0214530229568481 5.0
0 0.00253283977508545 6.0
0 -0.0160475373268127 7.0
30.801741874 39.1385078430176 0.0
0 0.0105280876159668 1.0
0 0.0485381484031677 2.0
0 -0.0218062996864319 3.0
0 0.00451433658599854 4.0
0 -0.0322909355163574 5.0
0 0.00279700756072998 6.0
0 -0.00666153430938721 7.0
40.388900757 39.6161880493164 0.0
0 0.0105310678482056 1.0
0 0.0530675649642944 2.0
0 -0.0235248208045959 3.0
0 0.0383188128471375 4.0
0 -0.0301554203033447 5.0
0 0.00359994173049927 6.0
0 -0.0129978656768799 7.0
35.801686131 39.3320503234863 0.0
0 0.00162678956985474 1.0
0 0.0328916311264038 2.0
0 -0.0161664485931396 3.0
0 0.0023643970489502 4.0
0 -0.0239839553833008 5.0
0 0.00194603204727173 6.0
0 -0.00305426120758057 7.0
43.275430302 39.1461601257324 0.0
0 0.00677180290222168 1.0
0 0.066714346408844 2.0
0 -0.0179618000984192 3.0
0 0.0459111332893372 4.0
0 -0.0245304107666016 5.0
0 0.00858467817306519 6.0
0 -0.013133704662323 7.0
36.674533885 39.2840042114258 0.0
0 0.00746917724609375 1.0
0 0.0606139898300171 2.0
0 -0.0169159770011902 3.0
0 0.0125620365142822 4.0
0 -0.0260381698608398 5.0
0 0.00515270233154297 6.0
0 -0.00862163305282593 7.0
42.291929941 39.0463104248047 0.0
0 0.011316180229187 1.0
0 0.0493571162223816 2.0
0 -0.0247151851654053 3.0
0 0.0330401062965393 4.0
0 -0.0230516195297241 5.0
0 0.00425034761428833 6.0
0 -0.0135107636451721 7.0
37.80686855 38.719539642334 0.0
0 0.00790625810623169 1.0
0 0.0712704658508301 2.0
0 -0.0141570568084717 3.0
0 0.00576889514923096 4.0
0 -0.0159686803817749 5.0
0 -0.000587940216064453 6.0
0 -0.00549298524856567 7.0
36.296445808 38.5794677734375 0.0
0 0.0126692056655884 1.0
0 0.0595385432243347 2.0
0 -0.0157817602157593 3.0
0 0.00248444080352783 4.0
0 -0.0212079882621765 5.0
0 0.00206708908081055 6.0
0 -0.00864028930664062 7.0
26.786347985 41.6884956359863 0.0
0 1.05500221252441e-05 1.0
0 0.0941086411476135 2.0
0 -0.0172778964042664 3.0
0 -0.00798177719116211 4.0
0 -0.0263301730155945 5.0
0 0.00166332721710205 6.0
0 0.325894445180893 7.0
38.715518937 39.3310127258301 0.0
0 -0.00107306241989136 1.0
0 0.0533255338668823 2.0
0 -0.0208864808082581 3.0
0 0.0355126857757568 4.0
0 -0.0311377644538879 5.0
0 0.00435280799865723 6.0
0 -0.0146806836128235 7.0
35.505063002 39.0710105895996 0.0
0 0.0100252032279968 1.0
0 0.0440443158149719 2.0
0 -0.0252014994621277 3.0
0 0.00799816846847534 4.0
0 -0.0276076197624207 5.0
0 0.00622522830963135 6.0
0 -0.00744861364364624 7.0
44.227649071 39.2503204345703 0.0
0 -0.00350522994995117 1.0
0 0.0830144882202148 2.0
0 -0.0194167494773865 3.0
0 0.0467356443405151 4.0
0 -0.0179108381271362 5.0
0 0.00280278921127319 6.0
0 -0.0161828398704529 7.0
45.773124255 39.1983108520508 0.0
0 0.0102026462554932 1.0
0 0.0390667915344238 2.0
0 -0.0257700681686401 3.0
0 0.0395534634590149 4.0
0 -0.0186536312103271 5.0
0 0.00253766775131226 6.0
0 -0.0136967897415161 7.0
38.615282752 39.6247634887695 0.0
0 0.00875389575958252 1.0
0 0.0653161406517029 2.0
0 -0.0225114226341248 3.0
0 0.0422231554985046 4.0
0 -0.0281192064285278 5.0
0 0.00611317157745361 6.0
0 -0.0130001306533813 7.0
42.131124836 39.6604461669922 0.0
0 0.00814712047576904 1.0
0 0.0577985048294067 2.0
0 -0.0264377593994141 3.0
0 0.0277695655822754 4.0
0 -0.0232623219490051 5.0
0 0.00406908988952637 6.0
0 -0.0118699073791504 7.0
38.030369766 39.4241714477539 0.0
0 0.00730973482131958 1.0
0 0.0649347305297852 2.0
0 -0.0156338810920715 3.0
0 0.028175950050354 4.0
0 -0.0283916592597961 5.0
0 0.00388896465301514 6.0
0 -0.0123919248580933 7.0
32.694870002 39.2483978271484 0.0
0 0.00167417526245117 1.0
0 0.0581066608428955 2.0
0 -0.0132492780685425 3.0
0 -0.00167709589004517 4.0
0 -0.0220803618431091 5.0
0 0.000322103500366211 6.0
0 -0.00424700975418091 7.0
41.58883184 39.7600479125977 0.0
0 0.0111787915229797 1.0
0 0.0836202502250671 2.0
0 -0.0201756954193115 3.0
0 0.014514684677124 4.0
0 -0.0171653032302856 5.0
0 -0.000364601612091064 6.0
0 -0.00911998748779297 7.0
33.622063045 40.3607978820801 0.0
0 0.00406837463378906 1.0
0 0.0649199485778809 2.0
0 -0.0115718841552734 3.0
0 0.0118268728256226 4.0
0 -0.0288293957710266 5.0
0 0.00606423616409302 6.0
0 -0.00943201780319214 7.0
40.101738506 39.3203430175781 0.0
0 0.00397443771362305 1.0
0 0.0380821228027344 2.0
0 -0.0269546508789062 3.0
0 0.0460603833198547 4.0
0 -0.0289177894592285 5.0
0 0.0054628849029541 6.0
0 -0.0136521458625793 7.0
39.347750494 39.5650939941406 0.0
0 0.0116128921508789 1.0
0 0.0399914383888245 2.0
0 -0.0279043912887573 3.0
0 0.0306953191757202 4.0
0 -0.0280945897102356 5.0
0 0.00678908824920654 6.0
0 -0.0120279788970947 7.0
46.240008463 39.3097534179688 0.0
0 0.00658661127090454 1.0
0 0.0587683320045471 2.0
0 -0.0242289304733276 3.0
0 0.0355812907218933 4.0
0 -0.0214058756828308 5.0
0 0.00193798542022705 6.0
0 -0.0121954083442688 7.0
43.885613972 39.8051948547363 0.0
0 0.0136743783950806 1.0
0 0.0521467924118042 2.0
0 -0.0285190343856812 3.0
0 0.0310320258140564 4.0
0 -0.0195971727371216 5.0
0 0.000680863857269287 6.0
0 -0.0110752582550049 7.0
44.877286055 39.2165946960449 0.0
0 0.00419837236404419 1.0
0 0.0352147817611694 2.0
0 -0.0227287411689758 3.0
0 0.062357485294342 4.0
0 -0.0253933668136597 5.0
0 0.00503700971603394 6.0
0 -0.0205078125 7.0
44.281024849 39.100700378418 0.0
0 0.0138094425201416 1.0
0 0.0409066081047058 2.0
0 -0.0262832045555115 3.0
0 0.0474932789802551 4.0
0 -0.0281514525413513 5.0
0 0.00751256942749023 6.0
0 -0.0144731402397156 7.0
39.443195334 39.5614204406738 0.0
0 0.00306022167205811 1.0
0 0.0668423175811768 2.0
0 -0.0195117592811584 3.0
0 0.0196396708488464 4.0
0 -0.0204795002937317 5.0
0 0.00201261043548584 6.0
0 -0.013425886631012 7.0
44.50998012 37.5527420043945 0.0
0 0.00241619348526001 1.0
0 0.0602812767028809 2.0
0 -0.0234832167625427 3.0
0 0.0416800379753113 4.0
0 -0.0199583768844604 5.0
0 0.00301408767700195 6.0
0 -0.0121491551399231 7.0
36.079576822 39.6479034423828 0.0
0 0.0111091732978821 1.0
0 0.0342797040939331 2.0
0 -0.0194944739341736 3.0
0 0.0268609523773193 4.0
0 -0.0297252535820007 5.0
0 0.00870823860168457 6.0
0 -0.0126831531524658 7.0
43.881758876 39.3958206176758 0.0
0 0.0129520893096924 1.0
0 0.0298763513565063 2.0
0 -0.0417283773422241 3.0
0 0.0476492047309875 4.0
0 -0.0290360450744629 5.0
0 0.00537043809890747 6.0
0 -0.013556182384491 7.0
30.108818284 39.1312637329102 0.0
0 0.00127166509628296 1.0
0 0.0570345520973206 2.0
0 -0.0248159170150757 3.0
0 0.00391620397567749 4.0
0 -0.0310516953468323 5.0
0 0.00319862365722656 6.0
0 -0.0063481330871582 7.0
34.22051219 39.2516632080078 0.0
0 0.00860375165939331 1.0
0 0.059270441532135 2.0
0 -0.0170743465423584 3.0
0 0.0202519297599792 4.0
0 -0.030121386051178 5.0
0 0.00781601667404175 6.0
0 -0.0109772086143494 7.0
37.402197971 38.8370780944824 0.0
0 0.0108410120010376 1.0
0 0.0724775195121765 2.0
0 -0.0135229825973511 3.0
0 0.0213173627853394 4.0
0 -0.0238214731216431 5.0
0 0.00379276275634766 6.0
0 -0.00984638929367065 7.0
35.453722409 39.3546600341797 0.0
0 0.00394952297210693 1.0
0 0.0594062805175781 2.0
0 -0.0146932005882263 3.0
0 0.00734615325927734 4.0
0 -0.0265988707542419 5.0
0 0.0027625560760498 6.0
0 -0.00827115774154663 7.0
32.872381155 39.2892608642578 0.0
0 0.010620653629303 1.0
0 0.0609287023544312 2.0
0 -0.0179244875907898 3.0
0 0.00361537933349609 4.0
0 -0.0287152528762817 5.0
0 0.00480866432189941 6.0
0 -0.0058131217956543 7.0
33.577714316 39.4748306274414 0.0
0 -0.00118172168731689 1.0
0 0.0713554620742798 2.0
0 -0.0126799345016479 3.0
0 0.016187310218811 4.0
0 -0.029799222946167 5.0
0 0.00732976198196411 6.0
0 -0.00867438316345215 7.0
43.986168802 39.2046051025391 0.0
0 0.0113417506217957 1.0
0 0.0563039183616638 2.0
0 -0.0383703708648682 3.0
0 0.0311552882194519 4.0
0 -0.0199617147445679 5.0
0 0.000735759735107422 6.0
0 -0.0111079216003418 7.0
39.701527205 39.3187294006348 0.0
0 0.0148863196372986 1.0
0 0.0385289788246155 2.0
0 -0.0257417559623718 3.0
0 0.0237706899642944 4.0
0 -0.0290372967720032 5.0
0 0.00385552644729614 6.0
0 -0.0131402611732483 7.0
36.319422317 40.3933143615723 0.0
0 0.0118604898452759 1.0
0 0.0607532262802124 2.0
0 -0.0213486552238464 3.0
0 0.0343860387802124 4.0
0 -0.0309789180755615 5.0
0 0.00633883476257324 6.0
0 -0.0120580196380615 7.0
39.54652972 39.2991943359375 0.0
0 0.0112186670303345 1.0
0 0.0607815980911255 2.0
0 -0.0222302675247192 3.0
0 0.0338104367256165 4.0
0 -0.0278091430664062 5.0
0 0.00617092847824097 6.0
0 -0.0118278861045837 7.0
46.3288844 39.1624336242676 0.0
0 0.00678884983062744 1.0
0 0.0581302642822266 2.0
0 -0.0223013758659363 3.0
0 0.0426577925682068 4.0
0 -0.0176820755004883 5.0
0 0.00250375270843506 6.0
0 -0.0138975381851196 7.0
39.359671461 39.2167892456055 0.0
0 0.0114791989326477 1.0
0 0.0373216867446899 2.0
0 -0.0210093259811401 3.0
0 0.0348582863807678 4.0
0 -0.0306043028831482 5.0
0 0.00945502519607544 6.0
0 -0.0132561922073364 7.0
43.813870988 39.6971778869629 0.0
0 0.00339007377624512 1.0
0 0.0766655206680298 2.0
0 -0.0190767049789429 3.0
0 0.0360779166221619 4.0
0 -0.0164307951927185 5.0
0 0.00113844871520996 6.0
0 -0.0129740834236145 7.0
39.146221876 38.7799415588379 0.0
0 0.0131780505180359 1.0
0 0.0397027730941772 2.0
0 -0.0297419428825378 3.0
0 0.0398967862129211 4.0
0 -0.0307811498641968 5.0
0 0.00434565544128418 6.0
0 -0.0123134255409241 7.0
34.753353014 39.8708610534668 0.0
0 -0.000608384609222412 1.0
0 0.115673184394836 2.0
0 -0.00372225046157837 3.0
0 0.0298271775245667 4.0
0 -0.0289943218231201 5.0
0 0.00512754917144775 6.0
0 -0.0145651698112488 7.0
36.858915008 39.2941970825195 0.0
0 0.00608658790588379 1.0
0 0.0364677906036377 2.0
0 -0.0228100419044495 3.0
0 0.0330633521080017 4.0
0 -0.0306199789047241 5.0
0 0.0058516263961792 6.0
0 -0.0124207735061646 7.0
45.609388725 39.0649108886719 0.0
0 0.0138698220252991 1.0
0 0.0337219834327698 2.0
0 -0.0362816452980042 3.0
0 0.0378729701042175 4.0
0 -0.0258591175079346 5.0
0 0.00363445281982422 6.0
0 -0.0116531848907471 7.0
37.221502314 39.0854949951172 0.0
0 0.00831472873687744 1.0
0 0.0745607614517212 2.0
0 -0.0210608243942261 3.0
0 0.0341051816940308 4.0
0 -0.0302374362945557 5.0
0 0.00336241722106934 6.0
0 -0.0136440992355347 7.0
39.351241848 39.079216003418 0.0
0 0.0129474401473999 1.0
0 0.0575575828552246 2.0
0 -0.0210944414138794 3.0
0 0.0328056812286377 4.0
0 -0.0277805328369141 5.0
0 0.00349217653274536 6.0
0 -0.0115208029747009 7.0
44.354286139 39.4327774047852 0.0
0 0.00980257987976074 1.0
0 0.0603073239326477 2.0
0 -0.0212836861610413 3.0
0 0.0488134622573853 4.0
0 -0.0223110914230347 5.0
0 0.00612592697143555 6.0
0 -0.0144089460372925 7.0
41.143843969 39.2214050292969 0.0
0 0.00443631410598755 1.0
0 0.0381275415420532 2.0
0 -0.0224068760871887 3.0
0 0.0291399955749512 4.0
0 -0.027215301990509 5.0
0 0.00449514389038086 6.0
0 -0.0142806768417358 7.0
32.436363672 40.0223999023438 0.0
0 0.00905328989028931 1.0
0 0.0224776268005371 2.0
0 -0.0241499543190002 3.0
0 0.00383108854293823 4.0
0 -0.031097412109375 5.0
0 0.00539493560791016 6.0
0 -0.00590020418167114 7.0
35.980268826 39.2067642211914 0.0
0 0.00584536790847778 1.0
0 0.101477026939392 2.0
0 -0.0109339952468872 3.0
0 0.0248546004295349 4.0
0 -0.0253920555114746 5.0
0 0.0057300329208374 6.0
0 -0.0112724304199219 7.0
39.361696951 39.1074066162109 0.0
0 8.82148742675781e-06 1.0
0 0.0373160839080811 2.0
0 -0.0219159722328186 3.0
0 0.00553721189498901 4.0
0 -0.0202788710594177 5.0
0 0.0008392333984375 6.0
0 -0.00792026519775391 7.0
38.339957501 39.3969459533691 0.0
0 0.00817453861236572 1.0
0 0.0689851045608521 2.0
0 -0.0219828486442566 3.0
0 0.0383086204528809 4.0
0 -0.0291087031364441 5.0
0 0.00392496585845947 6.0
0 -0.0140419602394104 7.0
38.437078026 39.1383895874023 0.0
0 0.00185298919677734 1.0
0 0.086402416229248 2.0
0 -0.015372633934021 3.0
0 0.0599687695503235 4.0
0 -0.0258504748344421 5.0
0 0.00392729043960571 6.0
0 -0.0180753469467163 7.0
31.680301161 39.6670150756836 0.0
0 0.00319278240203857 1.0
0 0.0746790766716003 2.0
0 -0.00584864616394043 3.0
0 -0.00392329692840576 4.0
0 -0.0279016494750977 5.0
0 0.00421690940856934 6.0
0 -0.00840109586715698 7.0
36.698769132 39.5708198547363 0.0
0 0.0115896463394165 1.0
0 0.0365105271339417 2.0
0 -0.0163916349411011 3.0
0 0.00393229722976685 4.0
0 -0.0225992798805237 5.0
0 0.002402663230896 6.0
0 -0.00545710325241089 7.0
40.942334922 39.1316337585449 0.0
0 0.0144314765930176 1.0
0 0.0406362414360046 2.0
0 -0.0247728824615479 3.0
0 0.0245896577835083 4.0
0 -0.0239621996879578 5.0
0 0.00334888696670532 6.0
0 -0.0101450085639954 7.0
41.496981537 39.2279319763184 0.0
0 0.00682604312896729 1.0
0 0.0635049939155579 2.0
0 -0.0187178254127502 3.0
0 0.0420576930046082 4.0
0 -0.0189405679702759 5.0
0 0.00453871488571167 6.0
0 -0.015053927898407 7.0
33.577466065 37.6259269714355 0.0
0 0.00761991739273071 1.0
0 0.0568365454673767 2.0
0 -0.0344957709312439 3.0
0 0.0208529233932495 4.0
0 -0.0321889519691467 5.0
0 0.00291848182678223 6.0
0 -0.00838345289230347 7.0
38.410697595 39.4420318603516 0.0
0 0.0113469362258911 1.0
0 0.0686453580856323 2.0
0 -0.0142496228218079 3.0
0 0.00417876243591309 4.0
0 -0.0155113935470581 5.0
0 -0.000855326652526855 6.0
0 -0.00508135557174683 7.0
40.390522112 39.2720718383789 0.0
0 0.000581443309783936 1.0
0 0.0393170714378357 2.0
0 -0.0221692323684692 3.0
0 0.0482469201087952 4.0
0 -0.0264850854873657 5.0
0 0.00661689043045044 6.0
0 -0.0178845524787903 7.0
38.02724699 38.9787788391113 0.0
0 0.00272881984710693 1.0
0 0.0378914475440979 2.0
0 -0.0236902236938477 3.0
0 0.00943964719772339 4.0
0 -0.0208128094673157 5.0
0 0.00278943777084351 6.0
0 -0.00711840391159058 7.0
34.831322295 38.5845375061035 0.0
0 0.00436198711395264 1.0
0 0.0335705280303955 2.0
0 -0.0219966173171997 3.0
0 0.00454270839691162 4.0
0 -0.0286591649055481 5.0
0 0.00389528274536133 6.0
0 -0.008575439453125 7.0
33.152476832 39.2792053222656 0.0
0 0.000803530216217041 1.0
0 0.0964491367340088 2.0
0 -0.00595211982727051 3.0
0 -0.00248456001281738 4.0
0 -0.0183451175689697 5.0
0 0.000454962253570557 6.0
0 -0.00583839416503906 7.0
37.76181476 39.1988182067871 0.0
0 0.00701266527175903 1.0
0 0.0517593026161194 2.0
0 -0.0180467367172241 3.0
0 0.0260825753211975 4.0
0 -0.0297691822052002 5.0
0 0.00736707448959351 6.0
0 -0.0125783681869507 7.0
46.821650967 39.1685638427734 0.0
0 0.0124068260192871 1.0
0 0.0685530304908752 2.0
0 -0.0276700854301453 3.0
0 0.0559806823730469 4.0
0 -0.0227029323577881 5.0
0 0.00737607479095459 6.0
0 -0.0143417716026306 7.0
41.183790175 39.3380546569824 0.0
0 0.00915086269378662 1.0
0 0.043701171875 2.0
0 -0.0248683094978333 3.0
0 0.0265670418739319 4.0
0 -0.0222071409225464 5.0
0 0.00195217132568359 6.0
0 -0.0123401284217834 7.0
36.78498864 39.2007217407227 0.0
0 0.0107810497283936 1.0
0 0.0372036695480347 2.0
0 -0.0210047960281372 3.0
0 0.0306706428527832 4.0
0 -0.0301961898803711 5.0
0 0.00700825452804565 6.0
0 -0.0121517777442932 7.0
37.689494945 40.0001411437988 0.0
0 0.00271826982498169 1.0
0 0.0959305763244629 2.0
0 -0.0141395926475525 3.0
0 0.0551460385322571 4.0
0 -0.0244983434677124 5.0
0 0.00532084703445435 6.0
0 -0.0170699954032898 7.0
41.308881097 39.8917083740234 0.0
0 0.013401985168457 1.0
0 0.0613721013069153 2.0
0 -0.0210734605789185 3.0
0 0.0379018187522888 4.0
0 -0.0249822735786438 5.0
0 0.00856763124465942 6.0
0 -0.0117775201797485 7.0
42.253011184 38.8202743530273 0.0
0 0.00312644243240356 1.0
0 0.0345909595489502 2.0
0 -0.0317898988723755 3.0
0 0.0441475510597229 4.0
0 -0.0303978323936462 5.0
0 0.00695013999938965 6.0
0 -0.0131359100341797 7.0
40.047660493 41.552059173584 0.0
0 0.00905096530914307 1.0
0 0.101017355918884 2.0
0 -0.0131558179855347 3.0
0 0.0354183316230774 4.0
0 -0.018403947353363 5.0
0 0.00396114587783813 6.0
0 -0.0143160223960876 7.0
46.604605733 39.235408782959 0.0
0 0.00629884004592896 1.0
0 0.105640709400177 2.0
0 -0.015018105506897 3.0
0 0.0429240465164185 4.0
0 -0.0164434313774109 5.0
0 0.00407266616821289 6.0
0 -0.0142799615859985 7.0
45.106295542 39.4154434204102 0.0
0 0.00659686326980591 1.0
0 0.0552960634231567 2.0
0 -0.0203244686126709 3.0
0 0.0602788925170898 4.0
0 -0.0267068147659302 5.0
0 0.00430470705032349 6.0
0 -0.018653392791748 7.0
38.977055099 39.2278289794922 0.0
0 0.00663143396377563 1.0
0 0.0948184132575989 2.0
0 -0.0122199654579163 3.0
0 0.0315920114517212 4.0
0 -0.0249831080436707 5.0
0 0.00514775514602661 6.0
0 -0.0134918689727783 7.0
38.935365883 39.1499710083008 0.0
0 0.00999718904495239 1.0
0 0.0386626124382019 2.0
0 -0.0293843746185303 3.0
0 0.0103581547737122 4.0
0 -0.0220971703529358 5.0
0 0.00317615270614624 6.0
0 -0.00762468576431274 7.0
42.741409918 39.2675628662109 0.0
0 0.00993281602859497 1.0
0 0.0399733185768127 2.0
0 -0.031740128993988 3.0
0 0.0284399390220642 4.0
0 -0.0234741568565369 5.0
0 0.00261366367340088 6.0
0 -0.00984954833984375 7.0
31.310794576 39.3400192260742 0.0
0 0.0062834620475769 1.0
0 0.038298487663269 2.0
0 -0.0212872624397278 3.0
0 0.00529992580413818 4.0
0 -0.0331861972808838 5.0
0 0.00680768489837646 6.0
0 -0.00564706325531006 7.0
};
\addlegendentry{$R^2$=0.983}
\end{axis}

\end{tikzpicture}
}}
    \subfloat[Actual vs predicted edge flows.] 
    {\label{fig:results_nonlineal_dummy_base_f}\resizebox{\figurewidth}{\figureheight}{% This file was created with tikzplotlib v0.10.1.
\begin{tikzpicture}

\definecolor{darkgray176}{RGB}{176,176,176}
\definecolor{lightgray204}{RGB}{204,204,204}

\begin{axis}[
colorbar,
colorbar style={ylabel={edge_id}},
colormap={mymap}{[1pt]
 rgb(0pt)=(0.12156862745098,0.466666666666667,0.705882352941177);
  rgb(1pt)=(1,0.498039215686275,0.0549019607843137);
  rgb(2pt)=(0.172549019607843,0.627450980392157,0.172549019607843);
  rgb(3pt)=(0.83921568627451,0.152941176470588,0.156862745098039);
  rgb(4pt)=(0.580392156862745,0.403921568627451,0.741176470588235);
  rgb(5pt)=(0.549019607843137,0.337254901960784,0.294117647058824);
  rgb(6pt)=(0.890196078431372,0.466666666666667,0.76078431372549);
  rgb(7pt)=(0.498039215686275,0.498039215686275,0.498039215686275);
  rgb(8pt)=(0.737254901960784,0.741176470588235,0.133333333333333);
  rgb(9pt)=(0.0901960784313725,0.745098039215686,0.811764705882353)
},
legend cell align={left},
legend style={
  fill opacity=0.8,
  draw opacity=1,
  text opacity=1,
  at={(0.03,0.97)},
  anchor=north west,
  draw=lightgray204
},
point meta max=7,
point meta min=0,
tick align=outside,
tick pos=left,
title={ye_test-ye_pred},
x grid style={darkgray176},
xlabel={ye_test},
xmajorgrids,
xmin=-15.25776872815, xmax=51.84084685915,
xtick style={color=black},
y grid style={darkgray176},
ylabel={ye_pred},
ymajorgrids,
ymin=-14.0268879413605, ymax=52.5269442081451,
ytick style={color=black}
]
\addplot [
  colormap={mymap}{[1pt]
 rgb(0pt)=(0.12156862745098,0.466666666666667,0.705882352941177);
  rgb(1pt)=(1,0.498039215686275,0.0549019607843137);
  rgb(2pt)=(0.172549019607843,0.627450980392157,0.172549019607843);
  rgb(3pt)=(0.83921568627451,0.152941176470588,0.156862745098039);
  rgb(4pt)=(0.580392156862745,0.403921568627451,0.741176470588235);
  rgb(5pt)=(0.549019607843137,0.337254901960784,0.294117647058824);
  rgb(6pt)=(0.890196078431372,0.466666666666667,0.76078431372549);
  rgb(7pt)=(0.498039215686275,0.498039215686275,0.498039215686275);
  rgb(8pt)=(0.737254901960784,0.741176470588235,0.133333333333333);
  rgb(9pt)=(0.0901960784313725,0.745098039215686,0.811764705882353)
},
  only marks,
  scatter,
  scatter src=explicit
]
table [x=x, y=y, meta=colordata]{%
x  y  colordata
39.565898635 38.596305847168 0.0
18.050517226 19.9765090942383 1.0
21.515381409 19.5053291320801 2.0
-8.8987516789 -7.5859842300415 3.0
12.61662972 11.9246826171875 4.0
26.949268915 26.8885326385498 5.0
39.565898645 40.6012878417969 6.0
39.565898645 39.244514465332 7.0
42.743257171 38.6418075561523 0.0
22.448801828 20.5235271453857 1.0
20.294455342 20.9817905426025 2.0
-4.2455956832 -6.40148591995239 3.0
16.048859649 15.7577791213989 4.0
26.694397521 26.0581035614014 5.0
42.743257181 41.7566375732422 6.0
42.743257181 41.8745651245117 7.0
39.367180747 38.091423034668 0.0
18.970258592 19.4065704345703 1.0
20.39692216 19.8818187713623 2.0
-4.9776642323 -5.26164960861206 3.0
15.419257923 14.7249002456665 4.0
23.947922829 23.1789817810059 5.0
39.367180751 38.5191268920898 6.0
39.367180751 39.1420822143555 7.0
39.605393097 37.8412399291992 0.0
22.082745928 20.4279651641846 1.0
17.522647169 18.7483291625977 2.0
-7.2517903354 -9.21843147277832 3.0
10.270856823 9.68563938140869 4.0
29.334536274 28.5466556549072 5.0
39.605393107 39.5127182006836 6.0
39.605393107 40.0177230834961 7.0
43.937345526 38.6058807373047 0.0
21.865843593 21.8540668487549 1.0
22.071501933 20.891975402832 2.0
-8.3469765086 -7.86768054962158 3.0
13.724525414 13.045955657959 4.0
30.212820112 28.7595252990723 5.0
43.937345536 43.1537170410156 6.0
43.937345536 43.2702102661133 7.0
31.061989584 37.880500793457 0.0
16.203677385 15.8516836166382 1.0
14.858312199 15.0089178085327 2.0
-5.5704045507 -6.62269258499146 3.0
9.2879076384 9.24989891052246 4.0
21.774081945 20.9252452850342 5.0
31.061989594 31.7036800384521 6.0
31.061989594 31.6152458190918 7.0
36.357435265 40.6078720092773 0.0
19.38173709 18.5710964202881 1.0
16.975698176 16.8981056213379 2.0
-8.3424496148 -9.10635566711426 3.0
8.6332485532 8.03556060791016 4.0
27.724186714 27.9642848968506 5.0
36.357435273 35.7285537719727 6.0
36.357435273 35.4694442749023 7.0
38.444969607 38.5469589233398 0.0
21.080570389 19.3154773712158 1.0
17.364399218 18.5943260192871 2.0
-5.0980379196 -7.51685047149658 3.0
12.266361289 11.6779813766479 4.0
26.178608318 25.8334617614746 5.0
38.444969617 38.1791000366211 6.0
38.444969617 38.3308563232422 7.0
35.498620518 38.3295974731445 0.0
17.900031823 17.3579730987549 1.0
17.598588698 18.0279216766357 2.0
-4.651704216 -6.18915939331055 3.0
12.946884476 12.3227348327637 4.0
22.551736046 22.3548603057861 5.0
35.498620524 35.0162658691406 6.0
35.498620524 35.7486877441406 7.0
36.52099827 39.1421966552734 0.0
18.961240079 18.6394557952881 1.0
17.559758192 18.268310546875 2.0
-5.3314768445 -6.87822914123535 3.0
12.228281339 11.798867225647 4.0
24.292716932 24.457181930542 5.0
36.520998279 36.9719772338867 6.0
36.520998279 36.2696228027344 7.0
36.717272204 38.9959335327148 0.0
19.763035348 19.2362728118896 1.0
16.954236857 18.1576652526855 2.0
-7.0689091682 -8.69227981567383 3.0
9.88532768 9.23490810394287 4.0
26.831944525 26.86110496521 5.0
36.717272212 37.3186569213867 6.0
36.717272212 37.7003936767578 7.0
32.629628996 38.8705673217773 0.0
17.103127104 16.7790718078613 1.0
15.526501892 15.2252159118652 2.0
-7.1999965109 -8.01188087463379 3.0
8.3265053708 8.09377098083496 4.0
24.303123625 24.2568454742432 5.0
32.629629006 33.6453323364258 6.0
32.629629006 33.813346862793 7.0
37.75267433 38.2975540161133 0.0
17.533699782 18.3296546936035 1.0
20.218974548 19.5139389038086 2.0
-3.919745576 -4.89840984344482 3.0
16.299228962 15.4050731658936 4.0
21.453445368 20.1247844696045 5.0
37.75267434 37.2333374023438 6.0
37.75267434 37.7951812744141 7.0
38.800291337 38.8440856933594 0.0
21.414385846 19.40407371521 1.0
17.385905491 18.7268810272217 2.0
-6.3164463541 -8.4114408493042 3.0
11.069459127 10.6674652099609 4.0
27.73083221 27.2714900970459 5.0
38.800291347 37.6164703369141 6.0
38.800291347 38.5971755981445 7.0
38.252729609 38.4748001098633 0.0
20.199036122 19.4157733917236 1.0
18.053693488 18.6936721801758 2.0
-6.9495443437 -8.02486419677734 3.0
11.104149136 10.5527362823486 4.0
27.148580475 27.2325859069824 5.0
38.252729618 38.1705474853516 6.0
38.252729618 38.0124740600586 7.0
43.273596037 40.0211029052734 0.0
20.811072897 22.2805271148682 1.0
22.46252314 20.9060554504395 2.0
-10.185324283 -9.22338104248047 3.0
12.277198847 11.4580049514771 4.0
30.99639719 30.8943367004395 5.0
43.273596047 43.1003189086914 6.0
43.273596047 43.7081604003906 7.0
34.027431477 37.5528106689453 0.0
16.463058834 17.5791893005371 1.0
17.564372643 16.3833618164062 2.0
-7.8238749469 -7.02876377105713 3.0
9.7404976863 9.56681632995605 4.0
24.28693379 23.8013954162598 5.0
34.027431486 34.5615615844727 6.0
34.027431486 33.9703750610352 7.0
41.154171382 38.8431701660156 0.0
20.969763 20.984806060791 1.0
20.184408383 19.1587142944336 2.0
-9.8031418559 -9.94899749755859 3.0
10.381266519 9.61151504516602 4.0
30.772904865 29.7687873840332 5.0
41.154171391 41.1792373657227 6.0
41.154171391 40.7767715454102 7.0
39.9308264078757 38.3818206787109 0.0
21.2554356118761 20.7513122558594 1.0
18.6753908028755 19.0751094818115 2.0
-8.41978314337461 -9.43831157684326 3.0
10.2556076588756 9.64951229095459 4.0
29.6752187558758 28.8787422180176 5.0
39.930826408 39.7014236450195 6.0
39.930826408 40.0568618774414 7.0
45.969703622 38.6865692138672 0.0
25.855804141 22.8934497833252 1.0
20.113899483 22.2165870666504 2.0
-4.729541314 -7.56897640228271 3.0
15.38435816 14.5867805480957 4.0
30.585345464 29.0078372955322 5.0
45.969703631 44.1232299804688 6.0
45.969703631 45.2887420654297 7.0
38.398120212 38.6374282836914 0.0
19.546425793 19.6057834625244 1.0
18.851694421 18.1635780334473 2.0
-9.8922695857 -9.75459480285645 3.0
8.9594248259 8.33925437927246 4.0
29.438695387 28.486421585083 5.0
38.398120221 38.1621627807617 6.0
38.398120221 38.5771560668945 7.0
28.366017296 39.6057739257812 0.0
14.995271371 14.1404008865356 1.0
13.370745924 13.0689744949341 2.0
-4.4280592982 -6.7935733795166 3.0
8.9426866162 8.9682502746582 4.0
19.42333068 19.1228618621826 5.0
28.366017306 29.9161052703857 6.0
28.366017306 28.6994247436523 7.0
39.07981897 37.8760223388672 0.0
18.202914308 19.484001159668 1.0
20.876904662 19.5376033782959 2.0
-6.6281228283 -6.22390174865723 3.0
14.248781824 13.701379776001 4.0
24.831037146 24.1412754058838 5.0
39.079818979 38.8075942993164 6.0
39.079818979 39.393684387207 7.0
40.466329374 38.3118438720703 0.0
19.842708002 19.766731262207 1.0
20.623621372 19.9568691253662 2.0
-6.0436252005 -6.6066198348999 3.0
14.579996162 13.9918174743652 4.0
25.886333212 24.9481201171875 5.0
40.466329383 39.8389892578125 6.0
40.466329383 40.9878768920898 7.0
38.293116843 37.9650497436523 0.0
18.225197444 18.1354637145996 1.0
20.067919399 19.4418029785156 2.0
-4.6459628917 -5.61400127410889 3.0
15.421956498 15.0479040145874 4.0
22.871160346 22.6656131744385 5.0
38.293116853 37.6979522705078 6.0
38.293116853 37.6028518676758 7.0
43.346089487 38.575080871582 0.0
21.23189324 21.8341217041016 1.0
22.114196248 20.7341327667236 2.0
-9.387268107 -8.84220218658447 3.0
12.726928131 11.9266443252563 4.0
30.619161357 29.7657775878906 5.0
43.346089497 42.8919219970703 6.0
43.346089497 43.2442321777344 7.0
34.547871144 38.1383056640625 0.0
19.900789633 18.2591228485107 1.0
14.647081512 16.0518455505371 2.0
-6.6924528176 -8.45735645294189 3.0
7.9546286844 7.69030857086182 4.0
26.59324246 26.038480758667 5.0
34.547871154 34.2125091552734 6.0
34.547871154 34.9699249267578 7.0
41.927050143208 38.3602600097656 0.0
21.8340017142739 21.933032989502 1.0
20.0930484299516 19.568078994751 2.0
-12.0156472013624 -10.9687099456787 3.0
8.07740122783537 8.01268196105957 4.0
33.8496486756856 32.4716110229492 5.0
41.927050143 41.8528671264648 6.0
41.927050143 42.3923263549805 7.0
44.548236367 38.2497406005859 0.0
22.097953544 21.9802989959717 1.0
22.450282823 21.7488422393799 2.0
-6.9116853021 -6.82119369506836 3.0
15.538597511 15.3108968734741 4.0
29.009638856 27.8976154327393 5.0
44.548236377 43.4468307495117 6.0
44.548236377 43.7278213500977 7.0
34.958415477 39.1437835693359 0.0
19.046265323 17.0440578460693 1.0
15.912150156 17.5630054473877 2.0
-3.947719982 -6.61912107467651 3.0
11.964430164 11.5081748962402 4.0
22.993985314 22.4576072692871 5.0
34.958415487 34.7578430175781 6.0
34.958415487 35.6662673950195 7.0
41.619773288 38.0942611694336 0.0
18.036088777 20.262638092041 1.0
23.583684511 20.2230663299561 2.0
-9.1140217471 -7.23081398010254 3.0
14.469662754 13.9196548461914 4.0
27.150110534 27.2458477020264 5.0
41.619773298 39.2246780395508 6.0
41.619773298 40.2623062133789 7.0
35.768623904 38.4565124511719 0.0
18.854929214 18.1294422149658 1.0
16.913694692 17.2556228637695 2.0
-6.8203969536 -8.47326946258545 3.0
10.093297731 9.59994029998779 4.0
25.675326176 25.5502910614014 5.0
35.768623912 36.0705490112305 6.0
35.768623912 36.3270874023438 7.0
36.035873071 36.6593170166016 0.0
18.197202866 18.8770313262939 1.0
17.838670207 17.1606254577637 2.0
-9.8802007315 -9.52300262451172 3.0
7.958469467 7.84091186523438 4.0
28.077403606 28.0217590332031 5.0
36.03587308 35.13916015625 6.0
36.03587308 36.5958099365234 7.0
42.671159575 38.0124359130859 0.0
22.542382477 21.8425483703613 1.0
20.128777099 20.0145320892334 2.0
-9.3171954196 -10.1334800720215 3.0
10.811581671 10.0967769622803 4.0
31.859577906 31.1639347076416 5.0
42.671159584 41.8028335571289 6.0
42.671159584 42.3421096801758 7.0
32.270518381 38.76953125 0.0
16.341237488 17.2153034210205 1.0
15.929280893 15.8293142318726 2.0
-7.2560391874 -7.5936803817749 3.0
8.6732416955 8.44317626953125 4.0
23.597276685 23.5255069732666 5.0
32.270518391 32.6548080444336 6.0
32.270518391 33.550910949707 7.0
36.239834932 41.2490692138672 0.0
18.511196273 18.1738815307617 1.0
17.72863866 18.0583553314209 2.0
-5.3875241311 -7.16932010650635 3.0
12.34111452 11.7859210968018 4.0
23.898720413 23.8287391662598 5.0
36.239834941 35.6087036132812 6.0
36.239834941 35.4636993408203 7.0
38.292005172 37.9891510009766 0.0
20.144840979 19.4964828491211 1.0
18.147164195 18.4614276885986 2.0
-6.9715526507 -7.96281623840332 3.0
11.175611535 10.6878089904785 4.0
27.116393638 26.6137714385986 5.0
38.292005181 38.5519485473633 6.0
38.292005181 39.5151824951172 7.0
41.142171116 39.3999252319336 0.0
21.697070846 20.3426818847656 1.0
19.44510027 20.4383220672607 2.0
-5.49061917 -7.43276405334473 3.0
13.95448109 13.2350759506226 4.0
27.187690026 26.9425754547119 5.0
41.142171126 40.9181289672852 6.0
41.142171126 41.5554046630859 7.0
30.660234741 37.4738845825195 0.0
16.571808951 15.7378721237183 1.0
14.08842579 14.9877004623413 2.0
-3.6405774968 -5.70759725570679 3.0
10.447848283 10.1889791488647 4.0
20.212386458 19.7445640563965 5.0
30.660234751 32.1174278259277 6.0
30.660234751 32.041820526123 7.0
42.776716976 39.9827499389648 0.0
18.943122536 21.1849422454834 1.0
23.83359444 20.9559173583984 2.0
-9.5992659914 -7.79702854156494 3.0
14.234328438 13.7019624710083 4.0
28.542388537 28.1028633117676 5.0
42.776716986 41.7659683227539 6.0
42.776716986 42.5332946777344 7.0
39.136657948 38.8526840209961 0.0
21.244730642 19.9374752044678 1.0
17.891927309 19.648380279541 2.0
-5.2069844714 -7.2206449508667 3.0
12.684942831 11.9592514038086 4.0
26.45171512 26.3674869537354 5.0
39.136657955 37.942253112793 6.0
39.136657955 38.3528442382812 7.0
40.593075656 38.2235794067383 0.0
19.181149666 21.0331974029541 1.0
21.411925992 19.1207962036133 2.0
-11.788434371 -9.97361087799072 3.0
9.6234916129 9.05599689483643 4.0
30.969584045 30.2422485351562 5.0
40.593075664 39.523551940918 6.0
40.593075664 40.1297454833984 7.0
38.455960047 38.7767868041992 0.0
19.606731652 20.1121311187744 1.0
18.849228395 18.2991676330566 2.0
-10.066988594 -9.78773784637451 3.0
8.7822397903 8.13940715789795 4.0
29.673720256 29.5408210754395 5.0
38.455960057 38.0539016723633 6.0
38.455960057 38.3448028564453 7.0
40.029822299 38.7315826416016 0.0
19.696760735 20.0480403900146 1.0
20.333061566 19.6599903106689 2.0
-6.9313004298 -7.38136005401611 3.0
13.401761128 12.8033199310303 4.0
26.628061173 25.8108005523682 5.0
40.029822307 39.3719940185547 6.0
40.029822307 40.4371185302734 7.0
39.721089796 38.3653564453125 0.0
21.982867862 20.3135223388672 1.0
17.738221933 19.1263008117676 2.0
-6.2167396454 -8.72383594512939 3.0
11.521482278 11.0404500961304 4.0
28.199607518 27.3751811981201 5.0
39.721089806 40.900260925293 6.0
39.721089806 39.7707290649414 7.0
46.012802771 38.5219421386719 0.0
23.852238991 22.91357421875 1.0
22.16056378 21.7913818359375 2.0
-9.7843125098 -10.1084747314453 3.0
12.37625126 11.3134632110596 4.0
33.636551511 32.5210723876953 5.0
46.012802781 45.08203125 6.0
46.012802781 46.1660232543945 7.0
43.79104115 38.6123657226562 0.0
23.810489927 22.027738571167 1.0
19.980551224 20.673770904541 2.0
-8.8972279263 -10.03688621521 3.0
11.083323289 9.92902183532715 4.0
32.707717862 31.3464870452881 5.0
43.791041158 42.2916946411133 6.0
43.791041158 43.6183166503906 7.0
31.257332414 38.2450332641602 0.0
14.677686479 15.6956052780151 1.0
16.579645934 14.9001684188843 2.0
-7.7653359569 -7.08528423309326 3.0
8.8143099674 8.48609733581543 4.0
22.443022446 22.779369354248 5.0
31.257332424 30.916130065918 6.0
31.257332424 30.941858291626 7.0
38.988472902 38.1238403320312 0.0
18.706030748 19.4086837768555 1.0
20.282442155 19.8146343231201 2.0
-4.4931154738 -4.96176481246948 3.0
15.789326673 14.8181753158569 4.0
23.19914623 22.3707904815674 5.0
38.98847291 39.0290298461914 6.0
38.98847291 39.5676727294922 7.0
38.691218489 38.2635803222656 0.0
19.181746937 18.3563499450684 1.0
19.509471553 18.8885822296143 2.0
-4.305791265 -6.0208797454834 3.0
15.203680278 14.929160118103 4.0
23.487538212 22.4304141998291 5.0
38.691218499 37.3509140014648 6.0
38.691218499 38.6561508178711 7.0
39.033211962 37.2093505859375 0.0
21.751284426 20.4594287872314 1.0
17.281927536 18.6953716278076 2.0
-7.1228573229 -9.43244361877441 3.0
10.159070204 9.23776054382324 4.0
28.874141758 28.742280960083 5.0
39.033211971 38.8525695800781 6.0
39.033211971 39.3592529296875 7.0
37.697547803 38.6268463134766 0.0
18.641388049 18.8006534576416 1.0
19.056159754 18.699592590332 2.0
-6.2754402376 -6.98374080657959 3.0
12.780719506 12.4110231399536 4.0
24.916828297 24.5393657684326 5.0
37.697547813 36.8487167358398 6.0
37.697547813 37.9475479125977 7.0
35.277541329 38.1812057495117 0.0
16.862419132 17.8773555755615 1.0
18.415122198 17.9661350250244 2.0
-6.2131742864 -6.36807012557983 3.0
12.201947903 11.7862014770508 4.0
23.075593428 23.2379150390625 5.0
35.277541339 34.7342910766602 6.0
35.277541339 36.3307418823242 7.0
36.1197639664031 38.3135223388672 0.0
19.1517629624013 17.7718601226807 1.0
16.9680010143999 18.865629196167 2.0
-0.473187581125011 -3.97485685348511 3.0
16.4948134334014 14.9833126068115 4.0
19.6249505434022 19.2428321838379 5.0
36.119763966 36.810417175293 6.0
36.119763966 37.0653457641602 7.0
33.14490304 38.2044525146484 0.0
16.831426445 16.1671714782715 1.0
16.313476595 16.9022331237793 2.0
-3.637680404 -4.73959493637085 3.0
12.675796181 12.6736574172974 4.0
20.469106859 20.0596942901611 5.0
33.14490305 33.2030410766602 6.0
33.14490305 32.5096092224121 7.0
34.4868008142505 38.2473602294922 0.0
18.3228058422505 18.7227478027344 1.0
16.1639949822505 16.799503326416 2.0
-8.00476657755046 -8.97555541992188 3.0
8.15922840445046 7.85224437713623 4.0
26.3275723771958 27.5033168792725 5.0
34.486800814 34.5710830688477 6.0
34.486800814 35.8238296508789 7.0
40.933468199 38.9690628051758 0.0
20.889274427 20.8757381439209 1.0
20.044193774 20.2090339660645 2.0
-6.533500242 -7.47461318969727 3.0
13.510693524 12.9050893783569 4.0
27.422774677 27.0413875579834 5.0
40.933468207 41.4710693359375 6.0
40.933468207 40.4933700561523 7.0
35.993417919 38.3794860839844 0.0
18.964188614 18.5946922302246 1.0
17.029229306 17.2728900909424 2.0
-7.4575389686 -8.26856327056885 3.0
9.5716903284 9.30193614959717 4.0
26.421727592 26.5541229248047 5.0
35.993417928 36.0943832397461 6.0
35.993417928 36.9887390136719 7.0
39.331666914 38.3532333374023 0.0
19.408314017 19.5726890563965 1.0
19.923352898 19.2500877380371 2.0
-7.2428639103 -7.33345890045166 3.0
12.680488979 12.0278520584106 4.0
26.651177937 25.8608589172363 5.0
39.331666923 39.1381759643555 6.0
39.331666923 39.3521347045898 7.0
37.559548486 38.3558502197266 0.0
18.054642609 19.0616474151611 1.0
19.504905877 18.4792575836182 2.0
-8.2806473226 -8.43041896820068 3.0
11.224258545 10.6311960220337 4.0
26.335289942 26.4604682922363 5.0
37.559548496 37.2250823974609 6.0
37.559548496 37.6933441162109 7.0
41.796902472 38.2650909423828 0.0
20.074580702 21.5512237548828 1.0
21.72232177 19.6002368927002 2.0
-11.24258785 -9.91202449798584 3.0
10.479733911 9.6037425994873 4.0
31.317168562 30.1995830535889 5.0
41.796902482 41.5065841674805 6.0
41.796902482 41.99755859375 7.0
35.679590813 38.4417953491211 0.0
15.932888769 17.7633152008057 1.0
19.746702044 17.7697124481201 2.0
-7.4161334139 -6.28019094467163 3.0
12.330568621 12.0021047592163 4.0
23.349022193 23.0090293884277 5.0
35.679590823 35.9085006713867 6.0
35.679590823 36.6312637329102 7.0
33.227547284 37.9404754638672 0.0
19.156810973 17.3763046264648 1.0
14.070736312 15.8583564758301 2.0
-5.2205843651 -6.91761207580566 3.0
8.8501519384 8.57953262329102 4.0
24.377395346 24.0266704559326 5.0
33.227547292 32.8059196472168 6.0
33.227547292 33.2664070129395 7.0
28.00807173 38.808479309082 0.0
12.909757219 14.271240234375 1.0
15.098314511 14.2345685958862 2.0
-4.3989930962 -3.41055345535278 3.0
10.699321405 10.7542247772217 4.0
17.308750325 17.332633972168 5.0
28.008071739 29.8767280578613 6.0
28.008071739 29.2562599182129 7.0
33.4784988410636 39.5744171142578 0.0
17.4596045520637 17.5532684326172 1.0
16.0188942990637 16.7638511657715 2.0
-6.65519022176347 -7.79201221466064 3.0
9.36370407706344 8.58208847045898 4.0
24.1147947730637 23.8467273712158 5.0
33.478498841 34.1995620727539 6.0
33.478498841 32.6516494750977 7.0
33.1229468201162 38.7150039672852 0.0
16.2738036511163 16.6416358947754 1.0
16.8491431771162 15.7469615936279 2.0
-8.48238354211593 -8.02114486694336 3.0
8.36675963501607 8.30586051940918 4.0
24.7561871931169 24.5005741119385 5.0
33.12294682 33.6349487304688 6.0
33.12294682 33.6500625610352 7.0
34.970228375 38.9385833740234 0.0
17.461716438 17.4915294647217 1.0
17.508511937 18.2084255218506 2.0
-3.5027478521 -4.3347749710083 3.0
14.005764076 13.2965602874756 4.0
20.964464299 20.1009483337402 5.0
34.970228384 35.6356582641602 6.0
34.970228384 35.9989166259766 7.0
36.637966033 38.3001022338867 0.0
18.611959282 17.4169902801514 1.0
18.026006753 17.929515838623 2.0
-5.1789935919 -6.43232107162476 3.0
12.847013153 12.1779184341431 4.0
23.790952882 23.1638679504395 5.0
36.637966041 36.7235488891602 6.0
36.637966041 36.484733581543 7.0
38.712447285 38.7288208007812 0.0
19.874849154 20.1867618560791 1.0
18.837598131 18.1591987609863 2.0
-9.656315418 -9.04708290100098 3.0
9.181282703 8.90895366668701 4.0
29.531164582 29.003870010376 5.0
38.712447295 38.2031784057617 6.0
38.712447295 39.4380950927734 7.0
29.0840224199388 38.1819839477539 0.0
12.9348822379388 14.5154609680176 1.0
16.1491401919388 14.3640298843384 2.0
-6.2158408037388 -5.854407787323 3.0
9.93329938853882 9.7581787109375 4.0
19.1507230409388 19.3183727264404 5.0
29.08402242 29.7144298553467 6.0
29.08402242 29.3537063598633 7.0
40.7419030105752 38.7526321411133 0.0
19.6019574718567 20.2419586181641 1.0
21.139945535663 20.0191478729248 2.0
-8.29196876971742 -8.19618988037109 3.0
12.8479767652648 12.1341876983643 4.0
27.8939262422712 28.183198928833 5.0
40.741903011 40.4545211791992 6.0
40.741903011 40.4759216308594 7.0
44.4239279780974 38.7700424194336 0.0
22.6860882760974 21.8470287322998 1.0
21.7378397120974 21.6257266998291 2.0
-6.03315836089742 -7.33268356323242 3.0
15.7046813510974 14.6819858551025 4.0
28.7192466370973 27.9537601470947 5.0
44.423927978 43.4490966796875 6.0
44.423927978 44.0868301391602 7.0
37.279929002 40.722297668457 0.0
17.697440733 18.6612033843994 1.0
19.582488269 17.9926910400391 2.0
-8.4860638308 -8.0453052520752 3.0
11.096424428 10.6737976074219 4.0
26.183504574 26.7601871490479 5.0
37.279929011 35.4391937255859 6.0
37.279929011 35.7633285522461 7.0
39.065196556 38.2746047973633 0.0
22.911627527 20.3264713287354 1.0
16.153569029 18.2511138916016 2.0
-7.035333012 -9.65429401397705 3.0
9.1182360067 8.52085018157959 4.0
29.946960549 29.2601470947266 5.0
39.065196566 38.0067367553711 6.0
39.065196566 38.2126693725586 7.0
34.346712901 38.617561340332 0.0
17.419062718 16.6762275695801 1.0
16.927650183 17.3127841949463 2.0
-3.8964434363 -5.28463315963745 3.0
13.031206737 12.3719930648804 4.0
21.315506165 21.3649024963379 5.0
34.346712911 34.0277557373047 6.0
34.346712911 34.495849609375 7.0
44.832836192 37.8687896728516 0.0
22.86238329 21.5256099700928 1.0
21.970452902 21.2041835784912 2.0
-8.0950157692 -8.58045959472656 3.0
13.875437123 13.1890459060669 4.0
30.957399069 29.8986225128174 5.0
44.832836202 43.7193450927734 6.0
44.832836202 43.844123840332 7.0
37.693005907 37.8074417114258 0.0
18.767233493 18.4735507965088 1.0
18.925772415 18.8588104248047 2.0
-4.4039460583 -4.7475790977478 3.0
14.521826348 14.3328046798706 4.0
23.17117956 22.4735546112061 5.0
37.693005916 37.4263381958008 6.0
37.693005916 37.8930740356445 7.0
35.516030196 38.2577285766602 0.0
17.238388415 17.4624156951904 1.0
18.277641781 17.627254486084 2.0
-5.9879556024 -6.29086351394653 3.0
12.289686168 11.9834842681885 4.0
23.226344027 22.9670333862305 5.0
35.516030206 35.4134750366211 6.0
35.516030206 36.0930252075195 7.0
45.449957515 36.9237747192383 0.0
24.470260824 22.9747180938721 1.0
20.979696692 21.4246616363525 2.0
-8.6547786371 -10.2861642837524 3.0
12.324918046 11.495792388916 4.0
33.12503947 32.4793243408203 5.0
45.449957523 45.0430755615234 6.0
45.449957523 44.0608062744141 7.0
36.522103961 38.1218719482422 0.0
17.787809101 17.765209197998 1.0
18.734294862 18.7959289550781 2.0
-3.7955939001 -4.28736639022827 3.0
14.938700953 14.4545946121216 4.0
21.583403009 20.6723518371582 5.0
36.52210397 35.9171676635742 6.0
36.52210397 36.7773666381836 7.0
40.809333824 38.1619262695312 0.0
21.825363738 20.3745574951172 1.0
18.983970087 20.2993831634521 2.0
-4.5096782574 -6.41374826431274 3.0
14.474291821 13.794828414917 4.0
26.335042004 25.4069137573242 5.0
40.809333833 40.2103576660156 6.0
40.809333833 40.7029113769531 7.0
43.708815015 39.1823577880859 0.0
22.06122806 21.4146289825439 1.0
21.647586954 21.0050201416016 2.0
-7.2998869135 -8.47407913208008 3.0
14.347700031 13.7081642150879 4.0
29.361114984 29.4641246795654 5.0
43.708815025 43.1477203369141 6.0
43.708815025 43.0359878540039 7.0
35.746307566 38.1672134399414 0.0
17.418825391 18.5653247833252 1.0
18.327482175 17.4033260345459 2.0
-8.2684434585 -7.56141662597656 3.0
10.059038708 9.9279899597168 4.0
25.687268859 25.0303840637207 5.0
35.746307575 37.0441131591797 6.0
35.746307575 36.9562149047852 7.0
32.565260397 38.3837738037109 0.0
15.347203209 16.2376251220703 1.0
17.218057188 16.839334487915 2.0
-4.4968959364 -4.29552936553955 3.0
12.721161242 12.2814855575562 4.0
19.844099155 19.8660144805908 5.0
32.565260407 33.5784378051758 6.0
32.565260407 34.2695922851562 7.0
37.459437559 39.1984710693359 0.0
20.974106733 19.2715492248535 1.0
16.485330826 17.0758037567139 2.0
-7.8025517563 -9.46932220458984 3.0
8.6827790601 8.27703666687012 4.0
28.776658499 28.5797557830811 5.0
37.459437569 36.4917373657227 6.0
37.459437569 37.0200958251953 7.0
41.6008687772188 37.8575820922852 0.0
18.8099504026802 19.9241485595703 1.0
22.7909183777866 20.1553192138672 2.0
-6.50941037372392 -5.38870763778687 3.0
16.2815080036091 15.0416536331177 4.0
25.319360776445 23.9512557983398 5.0
41.600868777 39.8447265625 6.0
41.600868777 40.6090469360352 7.0
43.08864828 42.9987716674805 0.0
20.75366021 23.2980003356934 1.0
22.334988069 22.0826263427734 2.0
-9.5492158301 -9.45747566223145 3.0
12.785772229 11.5652866363525 4.0
30.30287605 32.2585906982422 5.0
43.088648289 45.6551971435547 6.0
43.088648289 46.1906280517578 7.0
30.268152668 37.1409225463867 0.0
15.014233699 15.0411100387573 1.0
15.25391897 14.9551963806152 2.0
-5.5931438946 -5.8237099647522 3.0
9.6607750663 9.57966232299805 4.0
20.607377602 20.4172859191895 5.0
30.268152677 30.5712718963623 6.0
30.268152677 30.7838001251221 7.0
38.454045879 37.2824249267578 0.0
17.159579151 18.6700572967529 1.0
21.29446673 18.9856376647949 2.0
-7.5807931052 -5.8023157119751 3.0
13.713673616 13.2957019805908 4.0
24.740372265 23.9335842132568 5.0
38.454045888 38.1710433959961 6.0
38.454045888 38.3021545410156 7.0
36.654056854 38.5471496582031 0.0
19.439010257 18.9937343597412 1.0
17.215046598 17.5202980041504 2.0
-7.9864590963 -8.61619663238525 3.0
9.2285874923 8.71558570861816 4.0
27.425469363 27.1988697052002 5.0
36.654056864 36.5646362304688 6.0
36.654056864 37.8646697998047 7.0
36.990278119 38.5366821289062 0.0
18.644416453 18.7880783081055 1.0
18.345861667 17.3624744415283 2.0
-7.9977544937 -7.63241291046143 3.0
10.348107164 9.8026123046875 4.0
26.642170956 25.8676433563232 5.0
36.990278128 36.491813659668 6.0
36.990278128 37.627555847168 7.0
39.45539178 38.8359832763672 0.0
17.953554537 19.0125274658203 1.0
21.501837243 18.6644802093506 2.0
-6.7799927019 -5.81520938873291 3.0
14.721844531 14.5281772613525 4.0
24.733547249 23.6711483001709 5.0
39.45539179 38.500602722168 6.0
39.45539179 39.6684875488281 7.0
41.490098301 38.646110534668 0.0
18.418139969 20.1088981628418 1.0
23.071958334 20.5167922973633 2.0
-6.7020850264 -5.84434700012207 3.0
16.369873299 15.1859064102173 4.0
25.120225004 24.2188415527344 5.0
41.49009831 40.7906951904297 6.0
41.49009831 40.9147186279297 7.0
42.195848756 40.1018142700195 0.0
18.431626231 21.4116287231445 1.0
23.764222526 20.5440311431885 2.0
-9.6629820221 -7.10689353942871 3.0
14.101240496 13.4462242126465 4.0
28.094608262 27.11061668396 5.0
42.195848764 42.3811187744141 6.0
42.195848764 41.2984848022461 7.0
36.580423947 37.9035873413086 0.0
18.088597645 21.2339611053467 1.0
18.491826302 19.0799865722656 2.0
-10.331066357 -9.98916530609131 3.0
8.1607599346 7.98917293548584 4.0
28.419664012 30.2720394134521 5.0
36.580423956 39.9783248901367 6.0
36.580423956 40.3662643432617 7.0
41.112612837 38.6265029907227 0.0
21.264748673 21.0978622436523 1.0
19.847864165 19.9049644470215 2.0
-7.8539892402 -8.99751663208008 3.0
11.993874916 11.0752277374268 4.0
29.118737922 29.3703079223633 5.0
41.112612846 39.4206314086914 6.0
41.112612846 39.5524520874023 7.0
41.165032295 38.2608489990234 0.0
20.777866565 21.3963184356689 1.0
20.387165729 19.1337871551514 2.0
-11.013080697 -10.121452331543 3.0
9.3740850227 8.92656707763672 4.0
31.790947272 30.5620822906494 5.0
41.165032305 40.1152267456055 6.0
41.165032305 41.0985946655273 7.0
28.23632976 38.1864700317383 0.0
14.450875753 14.5610046386719 1.0
13.785454008 13.6108846664429 2.0
-5.4067157656 -6.53684329986572 3.0
8.3787382334 8.26820278167725 4.0
19.857591527 20.2184524536133 5.0
28.236329769 29.2296028137207 6.0
28.236329769 29.4686431884766 7.0
46.563722225 38.1880493164062 0.0
24.514076121 22.7633590698242 1.0
22.049646104 21.9152851104736 2.0
-6.9458736065 -7.58726024627686 3.0
15.103772488 14.7064065933228 4.0
31.459949737 29.7712516784668 5.0
46.563722235 44.9567184448242 6.0
46.563722235 45.9777069091797 7.0
38.976039438 34.1136169433594 0.0
20.47408062 18.8379173278809 1.0
18.50195882 19.1683197021484 2.0
-3.6791774969 -6.09064865112305 3.0
14.822781314 14.9086608886719 4.0
24.153258125 23.3799839019775 5.0
38.976039446 39.2413177490234 6.0
38.976039446 39.4658050537109 7.0
43.902220155 39.1786575317383 0.0
23.085534006 23.4488563537598 1.0
20.816686149 22.4787120819092 2.0
-6.0778252495 -9.12560844421387 3.0
14.738860889 14.0046243667603 4.0
29.163359266 31.5139141082764 5.0
43.902220165 46.4647827148438 6.0
43.902220165 47.53662109375 7.0
37.104870377 37.707893371582 0.0
21.928667997 19.5249156951904 1.0
15.176202381 17.5614700317383 2.0
-7.1829261037 -9.80313682556152 3.0
7.9932762681 7.98862075805664 4.0
29.11159411 28.3439083099365 5.0
37.104870385 37.2066040039062 6.0
37.104870385 37.9261703491211 7.0
48.094555135036 38.6682662963867 0.0
24.9788607525581 25.1454029083252 1.0
23.1156943831877 22.4715099334717 2.0
-7.58647576369851 -7.68739700317383 3.0
15.5292186191899 14.3461151123047 4.0
32.5653365171039 30.5723094940186 5.0
48.094555135 44.4077377319336 6.0
48.094555135 46.8743133544922 7.0
39.863550950908 38.1398620605469 0.0
20.2748443148297 20.5085353851318 1.0
19.5887066378693 18.5262336730957 2.0
-9.95882058391343 -9.65867233276367 3.0
9.62988605424657 9.29629039764404 4.0
30.233664897769 29.5470676422119 5.0
39.863550951 39.6430892944336 6.0
39.863550951 40.0736465454102 7.0
38.158424696 37.6681747436523 0.0
21.093616444 22.0732917785645 1.0
17.064808252 20.5325775146484 2.0
-6.6865614213 -10.2466697692871 3.0
10.378246821 9.59789752960205 4.0
27.780177875 31.4843349456787 5.0
38.158424706 42.7158203125 6.0
38.158424706 43.3326950073242 7.0
37.671552635 38.2944183349609 0.0
18.060105754 18.5829219818115 1.0
19.611446882 18.7748870849609 2.0
-6.6868904195 -6.5537691116333 3.0
12.924556453 12.4624843597412 4.0
24.746996183 24.1268634796143 5.0
37.671552644 36.9315948486328 6.0
37.671552644 38.0746917724609 7.0
38.9900068701748 38.3024597167969 0.0
19.4317892501748 19.9019870758057 1.0
19.5582176301748 19.5858058929443 2.0
-6.7795985865748 -7.9151029586792 3.0
12.7786190431748 12.0273065567017 4.0
26.2113877989032 26.9956111907959 5.0
38.99000687 39.3139495849609 6.0
38.99000687 38.7906112670898 7.0
32.824279187 38.8112869262695 0.0
17.524252685 16.0635318756104 1.0
15.300026502 15.6275568008423 2.0
-5.6991008156 -6.74575519561768 3.0
9.6009256767 9.46850776672363 4.0
23.223353511 23.2582511901855 5.0
32.824279197 32.8469848632812 6.0
32.824279197 32.7363967895508 7.0
39.752392388 38.4769897460938 0.0
19.337645943 19.1063632965088 1.0
20.414746445 19.9582710266113 2.0
-4.1151033929 -5.25276899337769 3.0
16.299643042 15.4891633987427 4.0
23.452749346 22.6432285308838 5.0
39.752392398 39.1198425292969 6.0
39.752392398 39.4828796386719 7.0
36.0149979881443 38.0255355834961 0.0
19.5282530962186 17.835147857666 1.0
16.4867448931817 17.4042663574219 2.0
-5.97886461861353 -7.98457145690918 3.0
10.5078802751384 9.96859836578369 4.0
25.5071177152005 25.1389141082764 5.0
36.014997988 35.8599395751953 6.0
36.014997988 36.1396865844727 7.0
42.307205596 38.211555480957 0.0
21.688253633 20.0025787353516 1.0
20.618951963 20.7712383270264 2.0
-4.4760898368 -5.28428745269775 3.0
16.142862116 15.6773233413696 4.0
26.164343479 25.2863941192627 5.0
42.307205606 40.986946105957 6.0
42.307205606 41.2777633666992 7.0
41.624369925 38.2369766235352 0.0
21.269456313 21.4407863616943 1.0
20.354913614 20.0920639038086 2.0
-9.726585677 -9.24065017700195 3.0
10.628327928 9.91324234008789 4.0
30.996041999 29.5769729614258 5.0
41.624369934 39.8592681884766 6.0
41.624369934 41.3274841308594 7.0
40.927860916 37.5762405395508 0.0
20.558611306 20.8498229980469 1.0
20.36924961 19.8419551849365 2.0
-7.6689328495 -7.97664070129395 3.0
12.700316751 12.0926990509033 4.0
28.227544165 27.9054050445557 5.0
40.927860926 39.9788818359375 6.0
40.927860926 40.8771896362305 7.0
47.441321803 38.3906173706055 0.0
22.43048805 23.1683750152588 1.0
25.010833753 22.8570461273193 2.0
-8.9760978683 -7.52433204650879 3.0
16.034735875 15.56809425354 4.0
31.406585928 30.2101974487305 5.0
47.441321812 45.6768646240234 6.0
47.441321812 46.8501434326172 7.0
39.37950561 38.0905303955078 0.0
19.873167771 21.6447582244873 1.0
19.506337839 19.4344139099121 2.0
-9.8415530061 -10.8011245727539 3.0
9.6647848227 8.9887752532959 4.0
29.714720787 31.1449851989746 5.0
39.379505619 41.7300796508789 6.0
39.379505619 42.3433303833008 7.0
42.095557289 38.8531799316406 0.0
22.277305426 21.7924900054932 1.0
19.818251863 19.8177433013916 2.0
-7.6874055589 -9.42489433288574 3.0
12.130846294 11.3354415893555 4.0
29.964710995 29.3569068908691 5.0
42.095557299 41.9169235229492 6.0
42.095557299 40.2805328369141 7.0
35.151188218 38.4096450805664 0.0
17.118344744 16.9611892700195 1.0
18.032843477 18.1718235015869 2.0
-2.0060677241 -3.1261191368103 3.0
16.026775747 15.6545400619507 4.0
19.124412475 19.1798362731934 5.0
35.151188225 34.8697280883789 6.0
35.151188225 34.7039413452148 7.0
37.713935362 38.1939315795898 0.0
20.380527895 19.2701034545898 1.0
17.333407469 18.2382144927979 2.0
-5.6423114193 -7.66351127624512 3.0
11.691096041 11.0271034240723 4.0
26.022839323 25.0261936187744 5.0
37.713935371 37.0099563598633 6.0
37.713935371 37.2810211181641 7.0
39.667743251 37.5596237182617 0.0
20.027544136 20.8216609954834 1.0
19.640199117 18.7193450927734 2.0
-10.376032212 -10.334132194519 3.0
9.2641668955 8.83496761322021 4.0
30.403576357 30.1484069824219 5.0
39.66774326 40.1139984130859 6.0
39.66774326 40.4533233642578 7.0
41.240556184 38.3862609863281 0.0
23.347458936 21.1435432434082 1.0
17.893097248 19.3683052062988 2.0
-7.252142064 -9.32804203033447 3.0
10.640955174 10.0189075469971 4.0
30.59960101 29.6779499053955 5.0
41.240556194 40.7343292236328 6.0
41.240556194 41.4971542358398 7.0
40.32019459 38.6126174926758 0.0
19.177093862 20.7322006225586 1.0
21.143100728 20.89084815979 2.0
-7.2609061044 -6.85930156707764 3.0
13.882194613 13.1603937149048 4.0
26.437999977 28.0188465118408 5.0
40.3201946 39.28173828125 6.0
40.3201946 40.6734466552734 7.0
39.5611510585585 39.3859558105469 0.0
19.6094718095582 19.8197803497314 1.0
19.9516792585585 19.0167236328125 2.0
-8.44726993515885 -8.88727855682373 3.0
11.5044093235587 10.6241550445557 4.0
28.0567416170478 28.0198516845703 5.0
39.561151059 38.9889526367188 6.0
39.561151059 39.7982940673828 7.0
43.5664319470049 39.2536926269531 0.0
22.3670487340048 21.7380771636963 1.0
21.1993832230048 20.2213344573975 2.0
-8.61702915780503 -8.69567966461182 3.0
12.582354065005 12.0542526245117 4.0
30.9840778920045 29.7219352722168 5.0
43.566431947 41.4726181030273 6.0
43.566431947 42.7189102172852 7.0
37.929366556 38.362907409668 0.0
20.105495873 19.1329498291016 1.0
17.823870687 18.7470550537109 2.0
-5.9768423324 -7.61201190948486 3.0
11.847028349 11.3475866317749 4.0
26.082338211 25.8750991821289 5.0
37.929366562 37.3192825317383 6.0
37.929366562 38.3035202026367 7.0
39.7456686413092 42.3975677490234 0.0
20.6016083083092 19.5039939880371 1.0
19.1440603423092 20.3956241607666 2.0
-3.0966467648096 -5.32644462585449 3.0
16.0474135773093 15.4909830093384 4.0
23.6982550733092 23.5807266235352 5.0
39.745668641 38.9904098510742 6.0
39.745668641 38.6185531616211 7.0
37.1737946256367 38.6771087646484 0.0
18.9810950589955 18.28173828125 1.0
18.1926995622865 18.5749950408936 2.0
-4.83110587561615 -6.28658008575439 3.0
13.3615936860386 12.8778429031372 4.0
23.8122009348139 23.6073837280273 5.0
37.173794625 36.0051956176758 6.0
37.173794625 36.3588485717773 7.0
27.124866007 40.3398284912109 0.0
13.623671484 13.8152093887329 1.0
13.501194525 12.9474258422852 2.0
-4.5998142153 -5.25400972366333 3.0
8.9013803006 9.02766990661621 4.0
18.223485707 18.1296672821045 5.0
27.124866015 28.8868579864502 6.0
27.124866015 28.4412879943848 7.0
27.411746895 36.7463607788086 0.0
13.763683287 13.9854421615601 1.0
13.648063611 13.749698638916 2.0
-4.2527245718 -4.60786962509155 3.0
9.3953390304 9.39307498931885 4.0
18.016407867 18.6928081512451 5.0
27.411746904 29.2872047424316 6.0
27.411746904 28.4803886413574 7.0
38.866617309 38.1587142944336 0.0
22.319478804 20.197925567627 1.0
16.547138508 17.9812850952148 2.0
-7.9092899105 -10.3619441986084 3.0
8.6378485894 8.25391960144043 4.0
30.228768722 29.4518909454346 5.0
38.866617317 38.6341781616211 6.0
38.866617317 39.0581817626953 7.0
42.3830573535328 38.3150253295898 0.0
20.5951530026372 20.8092937469482 1.0
21.7879043479541 19.9050922393799 2.0
-9.3946766066605 -8.83049869537354 3.0
12.3932277407315 11.7097043991089 4.0
29.98982960914 29.5242958068848 5.0
42.383057354 41.7898635864258 6.0
42.383057354 41.9013290405273 7.0
47.6437510358734 38.1060256958008 0.0
23.1576165238734 23.1068058013916 1.0
24.4861345218735 22.5069370269775 2.0
-8.74385685767346 -8.68577289581299 3.0
15.7422776638735 14.8399543762207 4.0
31.9014733818732 30.6543979644775 5.0
47.643751036 46.4351348876953 6.0
47.643751036 46.9786758422852 7.0
38.439399947 36.4208679199219 0.0
19.975660163 19.6007099151611 1.0
18.463739784 18.9128589630127 2.0
-6.4266906816 -7.36500644683838 3.0
12.037049093 11.4773578643799 4.0
26.402350854 26.1536636352539 5.0
38.439399957 37.2386703491211 6.0
38.439399957 38.7358169555664 7.0
40.263562361 38.7521286010742 0.0
20.42306719 21.6324996948242 1.0
19.840495171 19.0169887542725 2.0
-11.677045299 -10.7362909317017 3.0
8.1634498621 7.76362562179565 4.0
32.100112499 31.1445732116699 5.0
40.263562371 39.7273941040039 6.0
40.263562371 40.8870162963867 7.0
41.519528387 38.4955291748047 0.0
22.877320367 20.5484390258789 1.0
18.642208021 20.2042045593262 2.0
-5.2280057864 -7.99882888793945 3.0
13.414202225 12.5762281417847 4.0
28.105326163 27.7521781921387 5.0
41.519528397 41.0124359130859 6.0
41.519528397 41.6467742919922 7.0
40.414570464 38.545280456543 0.0
17.791289494 19.2980270385742 1.0
22.623280971 20.0405139923096 2.0
-6.6442159501 -5.92895364761353 3.0
15.979065012 14.6843795776367 4.0
24.435505453 23.8397598266602 5.0
40.414570474 39.4348983764648 6.0
40.414570474 40.1382675170898 7.0
37.834181215 38.0458221435547 0.0
18.770560656 19.026725769043 1.0
19.063620563 18.743106842041 2.0
-7.0281211763 -7.5380687713623 3.0
12.035499382 11.4961061477661 4.0
25.798681838 25.4843254089355 5.0
37.834181221 37.5501403808594 6.0
37.834181221 38.2561798095703 7.0
37.201121546 38.5927581787109 0.0
17.811506649 18.0195503234863 1.0
19.389614897 18.8371429443359 2.0
-6.0990519868 -6.29935646057129 3.0
13.290562901 12.8052835464478 4.0
23.910558645 23.9273529052734 5.0
37.201121556 36.8999633789062 6.0
37.201121556 38.0799560546875 7.0
42.944445132 37.6338272094727 0.0
21.59677927 21.4080848693848 1.0
21.347665865 21.0880851745605 2.0
-6.9735110037 -7.91884708404541 3.0
14.374154853 13.9020223617554 4.0
28.570290281 28.1455478668213 5.0
42.94444514 42.7225570678711 6.0
42.94444514 42.663444519043 7.0
47.3096000339538 38.3098220825195 0.0
24.7818404359387 23.2113876342773 1.0
22.5277596000527 22.52197265625 2.0
-6.75460229154402 -7.96077537536621 3.0
15.7731573079881 15.2921838760376 4.0
31.5364427279756 30.5247344970703 5.0
47.309600034 45.4152984619141 6.0
47.309600034 46.799430847168 7.0
34.703660739 37.7339782714844 0.0
16.95243502 16.7246932983398 1.0
17.75122572 17.4133987426758 2.0
-4.712400117 -5.60176753997803 3.0
13.038825595 12.5838718414307 4.0
21.664835145 21.2091007232666 5.0
34.703660747 34.5719451904297 6.0
34.703660747 35.0426940917969 7.0
39.355679823 38.3111801147461 0.0
18.719972615 20.0054531097412 1.0
20.635707207 19.1246032714844 2.0
-8.5340541425 -7.71829414367676 3.0
12.101653055 11.4348659515381 4.0
27.254026768 26.2859840393066 5.0
39.355679833 39.5544891357422 6.0
39.355679833 40.2200546264648 7.0
36.760838125 39.3952713012695 0.0
18.098408908 18.4492740631104 1.0
18.662429217 18.7476558685303 2.0
-4.9002916105 -5.8651704788208 3.0
13.762137597 13.3496942520142 4.0
22.998700528 22.7466430664062 5.0
36.760838135 36.7012939453125 6.0
36.760838135 37.1171569824219 7.0
45.817767369632 39.4295349121094 0.0
23.4066861136316 24.7768516540527 1.0
22.4110812656318 22.5875473022461 2.0
-7.99255969573223 -8.99110126495361 3.0
14.4185215696319 13.5078659057617 4.0
31.3992456710555 31.421703338623 5.0
45.817767369 46.1188812255859 6.0
45.817767369 47.965446472168 7.0
37.456387934 38.718879699707 0.0
18.714257418 19.2436027526855 1.0
18.742130516 18.3056735992432 2.0
-7.2626807055 -7.89689064025879 3.0
11.4794498 10.9147844314575 4.0
25.976938133 25.7266731262207 5.0
37.456387944 38.0156631469727 6.0
37.456387944 37.4311981201172 7.0
36.979299104 38.0109329223633 0.0
17.001867984 18.2078247070312 1.0
19.977431121 19.2218017578125 2.0
-4.4589079828 -4.98952913284302 3.0
15.518523129 14.8784132003784 4.0
21.460775976 21.1240882873535 5.0
36.979299113 37.1302108764648 6.0
36.979299113 37.1038970947266 7.0
44.357392148 38.3760147094727 0.0
23.57148554 21.6354026794434 1.0
20.785906608 21.3629627227783 2.0
-6.1950554121 -7.60999011993408 3.0
14.590851187 14.170747756958 4.0
29.766540962 28.0877113342285 5.0
44.357392157 43.3780136108398 6.0
44.357392157 43.6659927368164 7.0
36.8120792993302 38.4635009765625 0.0
19.8640110823302 19.7038173675537 1.0
16.9480682263302 17.6865768432617 2.0
-8.51793333553018 -9.63223648071289 3.0
8.43013489113018 7.97719144821167 4.0
28.381944393825 28.8917350769043 5.0
36.812079299 37.8954010009766 6.0
36.812079299 38.8740615844727 7.0
43.172254734 39.1506042480469 0.0
21.994700089 21.287561416626 1.0
21.177554646 21.6107559204102 2.0
-4.8220170211 -6.68520212173462 3.0
16.355537616 15.6073360443115 4.0
26.816717119 26.7941513061523 5.0
43.172254743 43.32275390625 6.0
43.172254743 43.4117965698242 7.0
47.109370558 39.7863998413086 0.0
25.034746172 23.357873916626 1.0
22.074624387 22.3467330932617 2.0
-8.7769581235 -9.06822299957275 3.0
13.297666255 12.5269060134888 4.0
33.811704304 32.8016815185547 5.0
47.109370567 46.5079803466797 6.0
47.109370567 46.2982406616211 7.0
35.871355653 37.1250152587891 0.0
18.076771786 18.8049793243408 1.0
17.794583868 17.2032375335693 2.0
-9.6137759762 -9.29005146026611 3.0
8.1808078825 7.89497995376587 4.0
27.690547772 27.5203189849854 5.0
35.871355662 35.1652069091797 6.0
35.871355662 36.8341217041016 7.0
42.1694339111434 38.8314437866211 0.0
22.6030080131434 21.3081550598145 1.0
19.5664259071434 20.3241405487061 2.0
-7.61628576964338 -9.15699577331543 3.0
11.9501401371434 11.008526802063 4.0
30.2192937831435 29.7380466461182 5.0
42.169433911 41.5776596069336 6.0
42.169433911 42.0681686401367 7.0
38.001249391 38.1650390625 0.0
19.442391644 19.7656803131104 1.0
18.558857748 18.629207611084 2.0
-7.4577746389 -8.1876916885376 3.0
11.101083099 10.5302896499634 4.0
26.900166292 26.8108882904053 5.0
38.001249401 37.9244537353516 6.0
38.001249401 38.9150009155273 7.0
36.026296144 38.3478240966797 0.0
18.544058535 18.6628570556641 1.0
17.482237609 17.9380531311035 2.0
-7.0335899837 -7.57428932189941 3.0
10.448647616 9.87599658966064 4.0
25.577648528 24.9330654144287 5.0
36.026296153 37.2376403808594 6.0
36.026296153 37.703010559082 7.0
33.200213149 40.287467956543 0.0
14.957525483 17.025318145752 1.0
18.242687666 16.2773590087891 2.0
-8.613844647 -7.50124549865723 3.0
9.6288430093 9.21061706542969 4.0
23.57137014 23.7718601226807 5.0
33.200213159 33.4814682006836 6.0
33.200213159 32.6489105224609 7.0
36.881947947 39.2395782470703 0.0
19.760267083 18.9527988433838 1.0
17.12168087 17.8010845184326 2.0
-6.5510496361 -8.39307975769043 3.0
10.570631229 9.94493865966797 4.0
26.311316723 26.6371574401855 5.0
36.881947952 36.7554473876953 6.0
36.881947952 37.4116897583008 7.0
28.363012003 38.7270050048828 0.0
14.617095463 14.1622161865234 1.0
13.745916541 13.3465538024902 2.0
-5.3741086073 -6.76098442077637 3.0
8.3718079243 8.29381370544434 4.0
19.991204079 19.9495868682861 5.0
28.363012012 29.9768486022949 6.0
28.363012012 29.0645999908447 7.0
39.574467416 37.6061553955078 0.0
20.649388383 22.4414443969727 1.0
18.925079033 20.6889381408691 2.0
-8.4547238017 -10.4059810638428 3.0
10.470355222 9.51583576202393 4.0
29.104112194 32.0760040283203 5.0
39.574467426 43.910514831543 6.0
39.574467426 43.7261199951172 7.0
36.220401471 38.3949584960938 0.0
17.995668566 18.2893886566162 1.0
18.224732906 17.837739944458 2.0
-6.9105415737 -7.24199485778809 3.0
11.314191323 11.0085277557373 4.0
24.906210149 24.359188079834 5.0
36.22040148 36.4723129272461 6.0
36.22040148 37.1749725341797 7.0
31.086432133 39.1634368896484 0.0
16.0910287 15.4654598236084 1.0
14.995403435 14.9265394210815 2.0
-4.3439743018 -5.34590101242065 3.0
10.651429126 10.6104555130005 4.0
20.43500301 20.3031311035156 5.0
31.086432141 31.1367130279541 6.0
31.086432141 31.2280387878418 7.0
36.274353053 38.4463577270508 0.0
18.420887917 17.6843662261963 1.0
17.853465137 18.4291648864746 2.0
-3.6202808126 -5.21385526657104 3.0
14.233184315 13.9288187026978 4.0
22.041168739 22.1156845092773 5.0
36.274353063 35.6696166992188 6.0
36.274353063 36.3318481445312 7.0
31.65795008 39.2199935913086 0.0
17.489254099 16.4708976745605 1.0
14.168695981 14.7580490112305 2.0
-6.0427493715 -7.45183753967285 3.0
8.1259465996 7.81711769104004 4.0
23.53200348 23.9427928924561 5.0
31.65795009 32.1667098999023 6.0
31.65795009 30.7517910003662 7.0
37.059171469 39.3203582763672 0.0
19.339994356 19.4019470214844 1.0
17.719177113 16.8399124145508 2.0
-9.4011683011 -9.28989315032959 3.0
8.318008802 8.35234451293945 4.0
28.741162667 28.0324687957764 5.0
37.059171479 37.2253952026367 6.0
37.059171479 37.5436477661133 7.0
45.308862348 39.220817565918 0.0
24.95721568 22.3412780761719 1.0
20.351646669 21.7738075256348 2.0
-7.323343681 -9.45879745483398 3.0
13.028302978 11.908106803894 4.0
32.28055937 31.4420166015625 5.0
45.308862357 44.7452087402344 6.0
45.308862357 44.6438064575195 7.0
32.988279899 39.2047119140625 0.0
17.965429309 16.0819435119629 1.0
15.02285059 16.926342010498 2.0
-2.6394240327 -4.85493850708008 3.0
12.383426547 11.7717523574829 4.0
20.604853352 20.5145568847656 5.0
32.988279909 33.3665161132812 6.0
32.988279909 33.1205062866211 7.0
41.812200584 40.716796875 0.0
21.127130287 20.7889308929443 1.0
20.685070296 20.4194183349609 2.0
-5.5457377677 -6.51394557952881 3.0
15.139332519 14.2325305938721 4.0
26.672868065 26.1188449859619 5.0
41.812200593 41.8736572265625 6.0
41.812200593 41.9870223999023 7.0
34.159577997 37.265266418457 0.0
15.979883142 16.5733184814453 1.0
18.179694855 16.5619583129883 2.0
-6.0058060763 -4.70657825469971 3.0
12.173888768 12.3701992034912 4.0
21.985689228 21.2810325622559 5.0
34.159578007 33.7586975097656 6.0
34.159578007 33.9638290405273 7.0
41.353058194 40.8923721313477 0.0
20.665722502 21.6208305358887 1.0
20.687335692 20.076301574707 2.0
-9.525258586 -9.4038724899292 3.0
11.162077096 10.2773103713989 4.0
30.190981098 29.7053661346436 5.0
41.353058204 42.2395629882812 6.0
41.353058204 41.7193908691406 7.0
35.663236634 38.0364151000977 0.0
18.766917562 18.0305480957031 1.0
16.896319073 16.7553672790527 2.0
-7.670552873 -8.74667644500732 3.0
9.2257661902 8.80641746520996 4.0
26.437470445 25.8856792449951 5.0
35.663236644 35.025276184082 6.0
35.663236644 35.8400115966797 7.0
40.444916113 39.8431777954102 0.0
18.421748417 20.8138008117676 1.0
22.023167696 19.4187850952148 2.0
-9.937900394 -9.13444900512695 3.0
12.085267292 11.1179285049438 4.0
28.359648821 28.2522449493408 5.0
40.444916123 40.682243347168 6.0
40.444916123 39.7233963012695 7.0
39.900416904 39.0326843261719 0.0
21.000367331 20.732442855835 1.0
18.900049573 19.8056240081787 2.0
-6.1981317567 -8.53449058532715 3.0
12.701917806 12.1823568344116 4.0
27.198499098 27.9057292938232 5.0
39.900416914 40.4196701049805 6.0
39.900416914 41.707633972168 7.0
34.40532484 38.8125534057617 0.0
18.680133752 17.6248817443848 1.0
15.725191089 16.2882823944092 2.0
-6.5675447079 -8.02764225006104 3.0
9.157646372 8.86200714111328 4.0
25.247678469 24.9007549285889 5.0
34.405324849 33.9266052246094 6.0
34.405324849 33.9994049072266 7.0
33.074255995 39.7259750366211 0.0
16.81819884 17.1022033691406 1.0
16.256057155 16.4729804992676 2.0
-5.8048358004 -6.80570220947266 3.0
10.451221345 10.1145238876343 4.0
22.62303465 22.4940090179443 5.0
33.074256004 33.1487350463867 6.0
33.074256004 33.1579895019531 7.0
40.036170298 40.0705032348633 0.0
18.427862329 19.8924922943115 1.0
21.60830797 20.1940879821777 2.0
-6.8170168406 -6.05037450790405 3.0
14.791291119 14.3158731460571 4.0
25.244879179 24.3987693786621 5.0
40.036170308 39.1054306030273 6.0
40.036170308 40.0101089477539 7.0
44.453206231 38.9618377685547 0.0
21.51525667 22.0500297546387 1.0
22.937949561 21.89235496521 2.0
-6.654859187 -7.4339656829834 3.0
16.283090365 15.3504457473755 4.0
28.170115867 28.1608390808105 5.0
44.453206241 44.2733459472656 6.0
44.453206241 44.8532028198242 7.0
33.8500454 39.1324462890625 0.0
15.317438725 16.899019241333 1.0
18.532606675 17.4372062683105 2.0
-3.771729854 -3.66815042495728 3.0
14.760876811 14.0392112731934 4.0
19.089168589 18.587818145752 5.0
33.85004541 33.9061126708984 6.0
33.85004541 34.3162841796875 7.0
30.537877299 37.9304428100586 0.0
15.743236653 15.7335357666016 1.0
14.794640646 14.6241807937622 2.0
-6.2614673178 -7.16267967224121 3.0
8.5331733191 8.3601188659668 4.0
22.00470398 22.2800445556641 5.0
30.537877308 30.127908706665 6.0
30.537877308 30.27783203125 7.0
27.727440934 36.8406219482422 0.0
13.298433958 13.9568004608154 1.0
14.429006978 13.708288192749 2.0
-4.8732361991 -4.95635652542114 3.0
9.5557707702 9.69101238250732 4.0
18.171670165 18.6766796112061 5.0
27.727440942 29.0367889404297 6.0
27.727440942 28.8692474365234 7.0
35.630182891 38.8659591674805 0.0
17.416160369 17.4986476898193 1.0
18.214022525 17.9752025604248 2.0
-5.5362398817 -6.49297046661377 3.0
12.677782636 12.2979021072388 4.0
22.952400258 23.0043239593506 5.0
35.630182897 35.8207244873047 6.0
35.630182897 36.9947814941406 7.0
31.038251509 38.4938354492188 0.0
16.216885208 15.719181060791 1.0
14.821366301 15.8819465637207 2.0
-5.1281710605 -5.38061714172363 3.0
9.6931952301 9.57784557342529 4.0
21.345056279 21.1744575500488 5.0
31.038251519 35.3713836669922 6.0
31.038251519 33.4895095825195 7.0
31.867011958 38.5478668212891 0.0
14.888086344 16.0857772827148 1.0
16.978925614 16.0145263671875 2.0
-5.7710515718 -5.4265308380127 3.0
11.207874033 10.7744665145874 4.0
20.659137926 20.0649967193604 5.0
31.867011967 32.6286582946777 6.0
31.867011967 32.7651405334473 7.0
30.754831636 38.8118896484375 0.0
14.067592922 15.3162145614624 1.0
16.687238713 15.7319660186768 2.0
-4.2836008115 -4.33065414428711 3.0
12.403637892 12.184775352478 4.0
18.351193744 18.0475749969482 5.0
30.754831646 31.9293422698975 6.0
30.754831646 31.8863906860352 7.0
38.951564273 38.2272415161133 0.0
20.789904648 19.7062282562256 1.0
18.161659627 17.7537155151367 2.0
-9.14497609 -9.92282485961914 3.0
9.0166835299 8.43372058868408 4.0
29.934880745 29.3792514801025 5.0
38.95156428 38.0840377807617 6.0
38.95156428 38.6076202392578 7.0
33.384480929 39.1025085449219 0.0
14.280681926 16.4918975830078 1.0
19.103799006 16.9669761657715 2.0
-6.8894608583 -5.55620813369751 3.0
12.214338142 11.8145179748535 4.0
21.170142791 21.3958950042725 5.0
33.384480936 32.6654891967773 6.0
33.384480936 31.9091911315918 7.0
37.291199278 38.8030319213867 0.0
18.976122996 17.4957332611084 1.0
18.315076287 18.4499969482422 2.0
-3.6314782733 -5.38515138626099 3.0
14.68359801 14.2914695739746 4.0
22.607601274 22.0485286712646 5.0
37.291199282 36.0691604614258 6.0
37.291199282 36.6016616821289 7.0
33.5718110164568 38.6800003051758 0.0
16.2455196066318 16.9113616943359 1.0
17.3262914102844 16.5193481445312 2.0
-6.34124959778268 -6.53616285324097 3.0
10.9850418123699 10.714958190918 4.0
22.5867692042782 22.6549491882324 5.0
33.571811016 34.1067886352539 6.0
33.571811016 34.3091583251953 7.0
44.048761737 38.4862060546875 0.0
22.194046458 21.6261215209961 1.0
21.85471528 21.7322368621826 2.0
-6.0219898767 -7.19724655151367 3.0
15.832725394 14.8925008773804 4.0
28.216036344 28.0335731506348 5.0
44.048761746 42.1424026489258 6.0
44.048761746 43.2741317749023 7.0
38.092815451 38.2420806884766 0.0
20.42584869 19.3928108215332 1.0
17.666966762 19.0491485595703 2.0
-5.2714738638 -7.20650196075439 3.0
12.395492889 11.8210115432739 4.0
25.697322563 25.2565593719482 5.0
38.09281546 37.8123321533203 6.0
38.09281546 38.8998031616211 7.0
42.616606914 38.5621795654297 0.0
22.043217204 21.0991725921631 1.0
20.57338971 20.2528762817383 2.0
-9.1142862064 -9.46041584014893 3.0
11.459103494 10.7132511138916 4.0
31.157503421 30.0427093505859 5.0
42.616606924 40.8032150268555 6.0
42.616606924 41.7413864135742 7.0
41.080227066 38.7166213989258 0.0
21.412108977 21.3951625823975 1.0
19.66811809 19.7237911224365 2.0
-8.9308117628 -9.66688823699951 3.0
10.737306318 9.92698860168457 4.0
30.342920748 29.7334499359131 5.0
41.080227074 41.443359375 6.0
41.080227074 41.4077987670898 7.0
36.859087804 38.433349609375 0.0
16.308519924 18.2599468231201 1.0
20.550567882 18.3344230651855 2.0
-8.0027710471 -6.93605136871338 3.0
12.547796826 12.0907773971558 4.0
24.31129098 23.9222564697266 5.0
36.859087813 37.3687133789062 6.0
36.859087813 37.9873733520508 7.0
39.316268866 38.0415191650391 0.0
19.479900065 19.8143157958984 1.0
19.836368801 19.0760173797607 2.0
-7.370707565 -7.87371635437012 3.0
12.465661226 11.8045310974121 4.0
26.85060764 26.5419731140137 5.0
39.316268876 39.4646453857422 6.0
39.316268876 39.6341781616211 7.0
40.023071957 38.8050231933594 0.0
21.374119573 20.7730007171631 1.0
18.648952384 18.9011993408203 2.0
-8.5239396023 -9.63755702972412 3.0
10.125012772 9.36878681182861 4.0
29.898059185 29.2027492523193 5.0
40.023071967 40.0522842407227 6.0
40.023071967 40.5714340209961 7.0
32.691039556 38.3575057983398 0.0
16.99402987 16.4254302978516 1.0
15.697009686 15.8058452606201 2.0
-6.1678662484 -7.45424652099609 3.0
9.5291434282 9.1234302520752 4.0
23.161896128 23.3277568817139 5.0
32.691039566 32.5299263000488 6.0
32.691039566 32.8125190734863 7.0
39.350172222 38.2304534912109 0.0
19.79571531 19.6176052093506 1.0
19.554456913 19.3827495574951 2.0
-6.9332448751 -7.61796092987061 3.0
12.62121203 12.0058012008667 4.0
26.728960194 26.4710826873779 5.0
39.35017223 38.9909057617188 6.0
39.35017223 39.605842590332 7.0
38.6257792205169 39.7686233520508 0.0
20.6764320695262 20.0450859069824 1.0
17.9493471475253 18.4269847869873 2.0
-8.74948254057215 -9.06752014160156 3.0
9.19986460693105 8.63833236694336 4.0
29.4259146106884 28.8499183654785 5.0
38.62577922 40.032356262207 6.0
38.62577922 38.3957214355469 7.0
43.363210173 38.3566284179688 0.0
22.264479357 21.3504753112793 1.0
21.09873082 20.8455333709717 2.0
-7.8566163895 -8.34503364562988 3.0
13.242114424 12.4946718215942 4.0
30.121095753 29.5680236816406 5.0
43.36321018 42.360481262207 6.0
43.36321018 42.765510559082 7.0
41.7113705187954 38.410530090332 0.0
19.9560794530664 20.5620975494385 1.0
21.7552910618007 20.6483879089355 2.0
-6.27295647675057 -6.27053594589233 3.0
15.4823345851289 14.6441297531128 4.0
26.2290359296866 25.2980918884277 5.0
41.711370519 40.6120376586914 6.0
41.711370519 41.5839080810547 7.0
41.4017485851654 40.012321472168 0.0
21.9191533447582 21.2350120544434 1.0
19.4825952431145 20.0624504089355 2.0
-7.53340700226978 -9.23404026031494 3.0
11.9491882404739 11.171706199646 4.0
29.4525603463383 29.4800243377686 5.0
41.401748585 41.8496017456055 6.0
41.401748585 40.5664215087891 7.0
45.411303660968 38.6011657714844 0.0
22.8793399812672 22.183198928833 1.0
22.5319636798771 21.0902271270752 2.0
-8.35657341793646 -8.23290729522705 3.0
14.1753902620561 13.4885587692261 4.0
31.2359133986188 30.6988735198975 5.0
45.411303661 42.7552947998047 6.0
45.411303661 44.0513076782227 7.0
46.173622728 38.2765502929688 0.0
22.478207281 22.1792011260986 1.0
23.695415447 21.4904689788818 2.0
-7.2335265538 -8.16776943206787 3.0
16.461888883 16.5144691467285 4.0
29.711733845 28.9169616699219 5.0
46.173622738 44.499885559082 6.0
46.173622738 45.0760192871094 7.0
43.970500883 38.4109878540039 0.0
23.319439955 22.0423831939697 1.0
20.651060929 20.3925151824951 2.0
-9.6037663307 -10.1343755722046 3.0
11.047294589 10.3460187911987 4.0
32.923206295 31.1921272277832 5.0
43.970500892 42.2292098999023 6.0
43.970500892 43.0829010009766 7.0
41.561964514 38.8353042602539 0.0
22.423587267 20.4294242858887 1.0
19.138377247 20.6321563720703 2.0
-2.7143772539 -6.24344968795776 3.0
16.423999984 16.1530227661133 4.0
25.13796453 24.6569118499756 5.0
41.561964524 41.7213363647461 6.0
41.561964524 41.050407409668 7.0
42.043570825 38.1755065917969 0.0
21.184612309 21.3031368255615 1.0
20.858958517 20.6786994934082 2.0
-7.1973175688 -7.39851951599121 3.0
13.66164094 13.0213384628296 4.0
28.381929887 27.8728427886963 5.0
42.043570834 41.2108306884766 6.0
42.043570834 41.688117980957 7.0
42.728187899 38.2817077636719 0.0
22.937429443 21.7080574035645 1.0
19.790758457 20.6395950317383 2.0
-7.3221221307 -8.65526103973389 3.0
12.468636317 11.8406934738159 4.0
30.259551583 29.384672164917 5.0
42.728187908 41.8752899169922 6.0
42.728187908 42.7050247192383 7.0
46.647822935 38.5361404418945 0.0
24.419972425 23.8600158691406 1.0
22.22785051 22.0308856964111 2.0
-9.7005341836 -9.42664051055908 3.0
12.527316316 11.4878358840942 4.0
34.120506619 33.8525314331055 5.0
46.647822945 46.0634155273438 6.0
46.647822945 47.0327758789062 7.0
34.157298452 38.6374435424805 0.0
16.904613725 17.4004230499268 1.0
17.252684728 16.6335182189941 2.0
-7.2828539724 -7.40594291687012 3.0
9.9698307451 9.54412651062012 4.0
24.187467707 23.9261646270752 5.0
34.157298462 33.9298858642578 6.0
34.157298462 34.4733428955078 7.0
38.08940785 40.2585372924805 0.0
20.151317931 18.6608867645264 1.0
17.93808992 19.6084251403809 2.0
-2.5516772312 -5.72512340545654 3.0
15.386412679 14.8432426452637 4.0
22.702995171 22.112154006958 5.0
38.08940786 39.3014678955078 6.0
38.08940786 38.1493377685547 7.0
48.253456595 39.4123382568359 0.0
26.02387816 23.8605499267578 1.0
22.229578435 22.6421127319336 2.0
-8.9125381923 -9.442946434021 3.0
13.317040233 12.2097778320312 4.0
34.936416362 33.5578842163086 5.0
48.253456605 47.5990447998047 6.0
48.253456605 47.9421081542969 7.0
41.303512945 38.0241088867188 0.0
21.842523524 20.9907550811768 1.0
19.460989421 19.5410175323486 2.0
-8.4785475097 -9.25720405578613 3.0
10.982441901 10.3447942733765 4.0
30.321071043 29.6855010986328 5.0
41.303512955 40.9467391967773 6.0
41.303512955 41.3833389282227 7.0
34.11247551 38.0281600952148 0.0
16.407323353 16.5559768676758 1.0
17.705152157 17.6839389801025 2.0
-3.4961419723 -4.1192774772644 3.0
14.209010175 13.9205007553101 4.0
19.903465335 19.9902420043945 5.0
34.11247552 33.7182388305664 6.0
34.11247552 34.1492080688477 7.0
39.346899311 38.4550399780273 0.0
20.280952021 20.4826831817627 1.0
19.06594729 18.3528308868408 2.0
-10.212334112 -9.99576377868652 3.0
8.8536131687 8.21247959136963 4.0
30.493286142 29.7933006286621 5.0
39.346899321 39.6096343994141 6.0
39.346899321 39.724723815918 7.0
42.355081312 37.3127059936523 0.0
22.237080125 22.240650177002 1.0
20.118001187 21.6637630462646 2.0
-4.2250518369 -8.08274555206299 3.0
15.892949341 15.4623432159424 4.0
26.462131971 29.0546703338623 5.0
42.355081321 45.5446014404297 6.0
42.355081321 45.1520080566406 7.0
46.02528461 37.5210113525391 0.0
20.767999035 22.6121349334717 1.0
25.257285575 21.6847133636475 2.0
-9.1629445671 -6.89923095703125 3.0
16.094340999 15.181357383728 4.0
29.930943612 28.2912178039551 5.0
46.025284619 43.1628112792969 6.0
46.025284619 44.9817733764648 7.0
48.790909777 37.9999694824219 0.0
25.145056622 23.4483261108398 1.0
23.645853155 22.4899921417236 2.0
-9.2179915631 -9.22602367401123 3.0
14.427861582 13.5308208465576 4.0
34.363048195 32.7184677124023 5.0
48.790909787 47.4653244018555 6.0
48.790909787 47.6438446044922 7.0
41.698940493 38.5718154907227 0.0
20.507660886 21.1916675567627 1.0
21.191279609 19.8242645263672 2.0
-9.1543537316 -8.90018558502197 3.0
12.036925868 11.3282794952393 4.0
29.662014626 28.7836875915527 5.0
41.698940502 40.8714141845703 6.0
41.698940502 41.4614944458008 7.0
47.902972774 37.9236907958984 0.0
23.393159022 23.3439140319824 1.0
24.509813754 23.0763282775879 2.0
-8.5476809737 -7.97155380249023 3.0
15.962132774 15.1462869644165 4.0
31.940840003 31.0478076934814 5.0
47.902972781 46.0086822509766 6.0
47.902972781 46.6090774536133 7.0
41.377170862 38.4424133300781 0.0
20.577519401 20.8493041992188 1.0
20.799651462 20.52956199646 2.0
-6.2821130824 -6.64414548873901 3.0
14.517538372 13.8285627365112 4.0
26.859632492 26.135570526123 5.0
41.37717087 41.5638656616211 6.0
41.37717087 41.2836227416992 7.0
32.285684892 38.3121643066406 0.0
14.316873201 15.6851854324341 1.0
17.968811691 16.3081245422363 2.0
-5.3227711546 -4.38966703414917 3.0
12.646040526 12.6627941131592 4.0
19.639644366 19.7672290802002 5.0
32.285684902 32.7703018188477 6.0
32.285684902 32.6642761230469 7.0
32.344206843 37.9958114624023 0.0
16.160894185 16.8645553588867 1.0
16.183312659 15.6117248535156 2.0
-7.3310896054 -7.66491508483887 3.0
8.852223045 8.37063026428223 4.0
23.491983799 23.2982540130615 5.0
32.344206851 33.25244140625 6.0
32.344206851 33.7069854736328 7.0
34.184112321 38.4180679321289 0.0
18.400062549 17.376672744751 1.0
15.784049772 16.0339813232422 2.0
-7.6583985283 -8.91202545166016 3.0
8.125651234 7.94446277618408 4.0
26.058461087 25.9154930114746 5.0
34.184112331 34.4313507080078 6.0
34.184112331 34.9890899658203 7.0
34.663221798 39.060661315918 0.0
18.206197344 17.9848823547363 1.0
16.457024454 16.6225280761719 2.0
-6.9582970888 -7.89465618133545 3.0
9.4987273548 9.08846759796143 4.0
25.164494443 24.9151744842529 5.0
34.663221808 34.8955383300781 6.0
34.663221808 36.0787048339844 7.0
36.873774533 36.4842071533203 0.0
18.579406772 18.2184753417969 1.0
18.294367761 18.4318828582764 2.0
-5.3733020341 -6.18012809753418 3.0
12.921065717 12.6412115097046 4.0
23.952708817 23.7828521728516 5.0
36.873774543 36.1619033813477 6.0
36.873774543 36.9672317504883 7.0
40.211352898 40.8574752807617 0.0
22.061687235 21.8373050689697 1.0
18.149665663 20.0331058502197 2.0
-7.4872031869 -10.1407079696655 3.0
10.662462467 9.82650661468506 4.0
29.548890431 31.1030902862549 5.0
40.211352908 42.4312973022461 6.0
40.211352908 42.6530380249023 7.0
46.160689864 38.6199569702148 0.0
24.176307096 22.1570148468018 1.0
21.984382774 22.1123676300049 2.0
-5.8472747438 -7.55188083648682 3.0
16.137108027 15.5524559020996 4.0
30.023581843 29.1270599365234 5.0
46.160689867 44.5380935668945 6.0
46.160689867 45.6408157348633 7.0
32.727702668 39.0109100341797 0.0
17.796794822 16.9581928253174 1.0
14.930907847 15.1722841262817 2.0
-5.7416146893 -6.93614101409912 3.0
9.1892931492 8.79934120178223 4.0
23.53840952 22.9070301055908 5.0
32.727702676 32.644401550293 6.0
32.727702676 33.5182876586914 7.0
40.640985417073 40.9353637695312 0.0
21.2248663380953 20.5563144683838 1.0
19.4161190831271 19.8506031036377 2.0
-6.12795167285201 -7.54850482940674 3.0
13.2881674100867 12.5912218093872 4.0
27.3528180102152 27.0946311950684 5.0
40.640985417 40.7124404907227 6.0
40.640985417 39.9781875610352 7.0
34.611451584 38.7155685424805 0.0
16.03184427 17.1442260742188 1.0
18.579607315 17.3146991729736 2.0
-5.6371786801 -5.41832494735718 3.0
12.942428625 12.5042181015015 4.0
21.66902296 21.6537895202637 5.0
34.611451594 35.0280303955078 6.0
34.611451594 35.4404067993164 7.0
43.311280798 39.3773040771484 0.0
19.747206987 20.760814666748 1.0
23.564073813 21.1719360351562 2.0
-7.5093463576 -5.80857801437378 3.0
16.054727447 15.5904731750488 4.0
27.256553353 26.2792015075684 5.0
43.311280807 41.630485534668 6.0
43.311280807 42.7580184936523 7.0
36.406364048 38.5013809204102 0.0
19.818637935 18.2263565063477 1.0
16.587726114 18.2509441375732 2.0
-3.8896932395 -6.24281692504883 3.0
12.698032865 12.240761756897 4.0
23.708331184 23.5217761993408 5.0
36.406364057 35.6973571777344 6.0
36.406364057 37.0864334106445 7.0
31.68563032 42.1182861328125 0.0
16.740289757 16.6609230041504 1.0
14.945340567 15.3443574905396 2.0
-6.0451472911 -7.26226711273193 3.0
8.9001932697 8.35330390930176 4.0
22.785437054 22.8738708496094 5.0
31.685630326 32.1779098510742 6.0
31.685630326 31.5273990631104 7.0
40.165364423 41.7417984008789 0.0
21.225216477 21.3776092529297 1.0
18.940147947 19.0668811798096 2.0
-9.5161355118 -9.72500038146973 3.0
9.4240124269 8.83594512939453 4.0
30.741351997 30.0516490936279 5.0
40.165364432 40.1582717895508 6.0
40.165364432 39.4980697631836 7.0
42.886465637 38.1925201416016 0.0
31.633244398 22.099588394165 1.0
11.253221239 20.3598155975342 2.0
0 -10.0018396377563 3.0
11.253221239 10.3876953125 4.0
31.633244398 30.4753379821777 5.0
42.886465647 42.8859329223633 6.0
42.886465647 43.4424133300781 7.0
46.942592322 39.7968139648438 0.0
23.694673267 23.3517818450928 1.0
23.247919055 22.2436904907227 2.0
-9.8881851606 -9.55183219909668 3.0
13.359733884 12.4381628036499 4.0
33.582858438 32.7497406005859 5.0
46.942592332 45.6073760986328 6.0
46.942592332 46.5753326416016 7.0
38.417030558 40.2535781860352 0.0
20.711241226 20.4441623687744 1.0
17.705789332 18.2222671508789 2.0
-7.7084466149 -9.50608062744141 3.0
9.9973427073 9.2020206451416 4.0
28.419687851 28.4029159545898 5.0
38.417030568 38.0373611450195 6.0
38.417030568 37.2130508422852 7.0
39.82159542 38.9206085205078 0.0
19.816428679 19.6259670257568 1.0
20.005166741 19.7445392608643 2.0
-5.1739908699 -6.13262510299683 3.0
14.831175861 14.4361400604248 4.0
24.990419559 24.3615665435791 5.0
39.82159543 38.943603515625 6.0
39.82159543 40.2914505004883 7.0
42.375846264 39.1178283691406 0.0
20.538210367 20.5903358459473 1.0
21.837635902 21.1823558807373 2.0
-5.3545430676 -6.22136545181274 3.0
16.483092829 15.3301210403442 4.0
25.89275344 25.6335678100586 5.0
42.375846269 40.8890533447266 6.0
42.375846269 42.2571411132812 7.0
41.909868914 38.1749801635742 0.0
21.249762031 21.4168834686279 1.0
20.660106883 18.9072437286377 2.0
-10.971309543 -10.2181768417358 3.0
9.6887973304 9.57851219177246 4.0
32.221071584 30.601146697998 5.0
41.909868924 41.6845245361328 6.0
41.909868924 42.7259216308594 7.0
38.5154038135067 38.4745330810547 0.0
19.3171244035071 18.9394989013672 1.0
19.1982794195071 17.5154647827148 2.0
-8.28760453680882 -8.4829626083374 3.0
10.9106748825063 10.6974229812622 4.0
27.6047289405071 26.8294239044189 5.0
38.515403814 37.6025390625 6.0
38.515403814 39.1212463378906 7.0
32.187464764 37.9287490844727 0.0
15.688410189 16.0790748596191 1.0
16.499054576 16.0943984985352 2.0
-5.8055743047 -5.66058874130249 3.0
10.693480262 10.3485889434814 4.0
21.493984503 21.5010547637939 5.0
32.187464773 31.9919776916504 6.0
32.187464773 31.9350719451904 7.0
43.5533404982843 39.243034362793 0.0
25.257472916292 22.6826000213623 1.0
18.2958675832873 20.0144195556641 2.0
-8.52565235008034 -10.6790180206299 3.0
9.77021523347734 8.71351432800293 4.0
33.7831252662979 33.5288772583008 5.0
43.553340498 42.7242431640625 6.0
43.553340498 41.6242828369141 7.0
38.661685108 38.3978805541992 0.0
21.023698145 22.1767997741699 1.0
17.637986964 20.4815006256104 2.0
-6.4647611773 -9.44525146484375 3.0
11.173225776 10.1704559326172 4.0
27.488459332 30.011926651001 5.0
38.661685118 43.3254928588867 6.0
38.661685118 43.201042175293 7.0
47.185976942 37.8694458007812 0.0
20.88210062 22.9890632629395 1.0
26.303876326 22.8522891998291 2.0
-9.8059530494 -7.71758937835693 3.0
16.497923272 16.7459602355957 4.0
30.688053675 29.5526866912842 5.0
47.185976948 46.2387313842773 6.0
47.185976948 46.8695526123047 7.0
41.0736498246764 38.6558151245117 0.0
17.9299575176762 20.1273136138916 1.0
23.1436923166761 20.1229667663574 2.0
-7.36188695227673 -5.97925853729248 3.0
15.7818053646765 15.4023761749268 4.0
25.2918444696756 24.1534004211426 5.0
41.073649825 39.4811935424805 6.0
41.073649825 40.7490234375 7.0
39.086114836 38.8619689941406 0.0
22.054277289 20.55735206604 1.0
17.031837547 18.5319519042969 2.0
-8.5092532605 -9.80507850646973 3.0
8.5225842768 8.27036571502686 4.0
30.563530559 29.8800678253174 5.0
39.086114846 38.731803894043 6.0
39.086114846 38.9209671020508 7.0
38.072309142 38.6814727783203 0.0
19.059692214 18.7046546936035 1.0
19.012616928 19.1962356567383 2.0
-4.5092621515 -6.01113700866699 3.0
14.503354768 14.0734481811523 4.0
23.568954375 23.2052192687988 5.0
38.072309151 37.7062454223633 6.0
38.072309151 37.8989028930664 7.0
41.585402364 38.2527008056641 0.0
21.878346027 21.2242889404297 1.0
19.707056337 20.1088809967041 2.0
-7.856520423 -8.65490341186523 3.0
11.850535904 11.2097215652466 4.0
29.73486646 29.1271724700928 5.0
41.585402374 41.1975479125977 6.0
41.585402374 41.5237426757812 7.0
42.3173178092696 38.4215469360352 0.0
23.2468941623901 22.1233310699463 1.0
19.0704236443393 18.9379272460938 2.0
-10.4383552532401 -11.0017137527466 3.0
8.63206839120634 8.19525623321533 4.0
33.6852494163203 32.4750366210938 5.0
42.317317809 42.1171417236328 6.0
42.317317809 42.6608352661133 7.0
42.931477575 39.3991851806641 0.0
22.005453055 21.6870460510254 1.0
20.926024519 21.0156536102295 2.0
-5.8784626661 -7.5215368270874 3.0
15.047561843 14.1241235733032 4.0
27.883915732 27.3795833587646 5.0
42.931477585 43.399299621582 6.0
42.931477585 43.0428009033203 7.0
46.096274403 39.2615356445312 0.0
25.32088965 22.5405292510986 1.0
20.775384755 22.2419261932373 2.0
-5.4662065709 -8.19584560394287 3.0
15.309178177 14.5654554367065 4.0
30.787096229 30.0127086639404 5.0
46.096274411 44.4457626342773 6.0
46.096274411 45.9870376586914 7.0
39.955261544 38.6042404174805 0.0
19.931762754 20.3806228637695 1.0
20.02349879 19.182035446167 2.0
-8.9110310329 -8.76078701019287 3.0
11.112467747 10.7508745193481 4.0
28.842793797 28.0816440582275 5.0
39.955261554 39.1785736083984 6.0
39.955261554 39.9708480834961 7.0
45.415961836 38.7581024169922 0.0
23.097461473 22.6445693969727 1.0
22.318500363 21.4108543395996 2.0
-8.5889928185 -9.13318157196045 3.0
13.729507535 12.8146114349365 4.0
31.686454301 30.696325302124 5.0
45.415961846 43.9062652587891 6.0
45.415961846 45.0336608886719 7.0
44.275746647 38.3189926147461 0.0
22.654009807 22.217716217041 1.0
21.62173684 21.3062992095947 2.0
-7.7547516702 -8.74852466583252 3.0
13.86698516 12.9053621292114 4.0
30.408761488 29.4190692901611 5.0
44.275746657 43.9858322143555 6.0
44.275746657 44.1960830688477 7.0
34.766238956 37.287353515625 0.0
17.943458551 17.6131191253662 1.0
16.822780406 17.1192417144775 2.0
-5.6356736314 -7.00773429870605 3.0
11.187106764 10.7948007583618 4.0
23.579132192 23.2619380950928 5.0
34.766238966 34.7213668823242 6.0
34.766238966 34.5344619750977 7.0
39.697519492335 38.7826766967773 0.0
21.2162061883195 19.4346141815186 1.0
18.481313305488 19.3157329559326 2.0
-4.67528909066386 -7.0928783416748 3.0
13.8060242148741 13.4178695678711 4.0
25.8914952793298 25.2283058166504 5.0
39.697519492 38.5190505981445 6.0
39.697519492 39.5111618041992 7.0
36.4599525996002 38.5531311035156 0.0
18.3065166475976 18.3537063598633 1.0
18.1534359526018 18.5036849975586 2.0
-5.15104061002709 -6.58333730697632 3.0
13.0023953426082 12.6136074066162 4.0
23.457557257596 23.4695262908936 5.0
36.459952599 36.2533340454102 6.0
36.459952599 37.4287109375 7.0
44.856515577 38.2222518920898 0.0
22.473710733 22.5195598602295 1.0
22.382804843 21.8416500091553 2.0
-8.0040806232 -7.90747356414795 3.0
14.37872421 13.7032289505005 4.0
30.477791367 29.1743850708008 5.0
44.856515586 44.1223602294922 6.0
44.856515586 44.4332046508789 7.0
41.1579139562054 38.4082717895508 0.0
18.5564288232053 19.9055194854736 1.0
22.601485143205 20.1454486846924 2.0
-7.31439675300561 -6.26758575439453 3.0
15.2870883902055 14.8041162490845 4.0
25.8708255762049 24.740291595459 5.0
41.157913956 39.4463882446289 6.0
41.157913956 40.2281723022461 7.0
32.092085782 42.7318878173828 0.0
17.272160967 16.5270881652832 1.0
14.819924815 15.0818977355957 2.0
-6.6255124723 -7.97371292114258 3.0
8.1944123328 7.85368251800537 4.0
23.897673449 24.0389785766602 5.0
32.092085792 31.7450618743896 6.0
32.092085792 30.3968200683594 7.0
30.327518415 38.4048843383789 0.0
13.541286609 14.6094779968262 1.0
16.786231807 15.1072158813477 2.0
-5.5133467024 -5.7421088218689 3.0
11.272885094 11.1871385574341 4.0
19.054633321 19.3803768157959 5.0
30.327518425 30.3478355407715 6.0
30.327518425 30.4830265045166 7.0
36.482977268 39.0758666992188 0.0
19.553260998 19.031644821167 1.0
16.929716271 17.6167831420898 2.0
-7.3162847454 -9.2794132232666 3.0
9.613431516 8.96224021911621 4.0
26.869545753 27.4428234100342 5.0
36.482977278 36.0645523071289 6.0
36.482977278 36.1489791870117 7.0
40.491170904 39.51025390625 0.0
20.509684159 20.8911056518555 1.0
19.981486745 19.167610168457 2.0
-8.9848735316 -10.0219697952271 3.0
10.996613204 10.0886106491089 4.0
29.4945577 30.1088333129883 5.0
40.491170914 39.7505264282227 6.0
40.491170914 39.4850921630859 7.0
41.188898324 38.781135559082 0.0
17.482572735 20.9087905883789 1.0
23.70632559 20.5050888061523 2.0
-7.86829602 -6.30735492706299 3.0
15.838029561 15.0808620452881 4.0
25.350868764 24.5638866424561 5.0
41.188898334 39.6158218383789 6.0
41.188898334 40.3856430053711 7.0
37.756949894 37.3375091552734 0.0
19.043754327 18.7420654296875 1.0
18.713195567 19.0237312316895 2.0
-5.5841962186 -6.77661228179932 3.0
13.128999338 12.5939464569092 4.0
24.627950556 24.4968547821045 5.0
37.756949904 38.0179138183594 6.0
37.756949904 37.8417816162109 7.0
26.962645791 38.6975555419922 0.0
14.231795944 13.5606527328491 1.0
12.730849849 12.0478553771973 2.0
-4.3614404899 -5.80632019042969 3.0
8.3694093501 8.35910892486572 4.0
18.593236442 18.5433235168457 5.0
26.9626458 29.0926151275635 6.0
26.9626458 27.8304920196533 7.0
44.1546808660895 38.0645523071289 0.0
20.845479755763 21.6654720306396 1.0
23.3092011082446 22.5813045501709 2.0
-6.97221748920505 -7.73952484130859 3.0
16.3369836194072 15.5416536331177 4.0
27.8176971034536 29.6074733734131 5.0
44.154680866 44.8336486816406 6.0
44.154680866 45.3158950805664 7.0
36.4992036 38.1218414306641 0.0
18.499566849 18.9169368743896 1.0
17.999636751 17.2083702087402 2.0
-8.164709678 -8.1932954788208 3.0
9.834927063 9.57297229766846 4.0
26.664276537 25.952974319458 5.0
36.49920361 36.0940704345703 6.0
36.49920361 37.3394546508789 7.0
36.136186998 39.2956619262695 0.0
19.096538919 19.0549125671387 1.0
17.039648079 17.2332229614258 2.0
-8.7939144312 -8.95334053039551 3.0
8.2457336392 8.1334285736084 4.0
27.89045336 27.9088649749756 5.0
36.136187007 35.604606628418 6.0
36.136187007 36.6808471679688 7.0
43.805905699 38.4082794189453 0.0
19.467800315 21.7324333190918 1.0
24.338105385 21.3228511810303 2.0
-10.07714477 -7.99375820159912 3.0
14.260960606 13.4028282165527 4.0
29.544945095 29.3950824737549 5.0
43.805905709 42.9095916748047 6.0
43.805905709 43.1864471435547 7.0
43.400673471 37.3553619384766 0.0
21.450145846 21.3319797515869 1.0
21.950527627 20.0062026977539 2.0
-10.718319514 -9.52839851379395 3.0
11.232208106 10.443018913269 4.0
32.168465368 30.1349048614502 5.0
43.400673478 44.3151473999023 6.0
43.400673478 42.2234802246094 7.0
35.708898922 41.3468627929688 0.0
19.860917727 18.4082355499268 1.0
15.847981195 17.1911163330078 2.0
-6.8010445999 -8.64494705200195 3.0
9.0469365848 8.26922416687012 4.0
26.661962337 26.3812522888184 5.0
35.708898932 35.3210830688477 6.0
35.708898932 34.2125473022461 7.0
42.19537597 38.6839065551758 0.0
21.611035583 21.2544460296631 1.0
20.584340387 20.4304485321045 2.0
-7.818182446 -8.66779899597168 3.0
12.766157932 11.9916820526123 4.0
29.429218038 28.7422046661377 5.0
42.195375979 41.3631820678711 6.0
42.195375979 42.2733154296875 7.0
38.219132248 38.8931350708008 0.0
18.124133621 18.5880088806152 1.0
20.094998634 19.4918537139893 2.0
-3.9675986496 -4.18937349319458 3.0
16.127399981 15.6467847824097 4.0
22.091732274 21.4282341003418 5.0
38.219132252 37.8382568359375 6.0
38.219132252 38.0583267211914 7.0
37.34785903 39.6710510253906 0.0
19.494183125 18.5862197875977 1.0
17.853675906 18.7472915649414 2.0
-4.6743476987 -6.32650375366211 3.0
13.179328197 12.6402902603149 4.0
24.168530833 23.53444480896 5.0
37.34785904 37.4297256469727 6.0
37.34785904 37.857795715332 7.0
41.413317073 38.4210968017578 0.0
22.502810393 20.7934989929199 1.0
18.91050668 20.2211971282959 2.0
-4.5479233593 -6.43856811523438 3.0
14.362583311 13.6872663497925 4.0
27.050733762 26.306791305542 5.0
41.413317083 40.7909088134766 6.0
41.413317083 41.0473098754883 7.0
39.408474003 40.3025207519531 0.0
19.787258075 19.7200088500977 1.0
19.621215928 18.9290428161621 2.0
-8.5467597404 -8.02848339080811 3.0
11.074456178 10.4783992767334 4.0
28.334017825 27.7507095336914 5.0
39.408474012 39.2541885375977 6.0
39.408474012 39.7915420532227 7.0
44.6219477930596 37.9476928710938 0.0
23.5295769810594 22.9357490539551 1.0
21.0923708210596 21.2818946838379 2.0
-9.44677516695988 -9.62642574310303 3.0
11.6455956540597 10.8971891403198 4.0
32.9763521480588 31.588306427002 5.0
44.621947793 43.3761825561523 6.0
44.621947793 44.4126815795898 7.0
35.042314664 40.4611358642578 0.0
17.377708596 17.5871887207031 1.0
17.66460607 17.1801624298096 2.0
-4.7609303256 -5.11108732223511 3.0
12.903675736 12.5969161987305 4.0
22.13863893 21.5433940887451 5.0
35.042314673 34.7478256225586 6.0
35.042314673 34.9639511108398 7.0
44.668685977 36.4160079956055 0.0
21.700098462 22.3417587280273 1.0
22.968587516 21.7795524597168 2.0
-8.4663400465 -8.06090354919434 3.0
14.50224746 13.6951875686646 4.0
30.166438518 29.3170509338379 5.0
44.668685987 43.0581436157227 6.0
44.668685987 44.2628860473633 7.0
44.271188228 37.6531753540039 0.0
21.134860564 21.8332481384277 1.0
23.136327666 22.3055229187012 2.0
-6.9953395415 -6.5513596534729 3.0
16.140988117 15.0220232009888 4.0
28.130200113 27.2446060180664 5.0
44.271188235 44.1387329101562 6.0
44.271188235 43.6082611083984 7.0
43.3296157145675 39.0664367675781 0.0
21.8144364805029 21.656759262085 1.0
21.5151792385451 20.2940578460693 2.0
-10.3071776780671 -9.62773132324219 3.0
11.2080015596363 10.501633644104 4.0
32.1216141588196 30.5324611663818 5.0
43.329615715 41.9489364624023 6.0
43.329615715 42.7121810913086 7.0
37.727034885 40.9678497314453 0.0
16.740902708 19.1518745422363 1.0
20.986132178 18.1882343292236 2.0
-10.156378742 -7.44830799102783 3.0
10.829753428 10.1057720184326 4.0
26.897281458 26.3376235961914 5.0
37.727034893 37.7822113037109 6.0
37.727034893 36.8564376831055 7.0
43.665312103 38.6473083496094 0.0
21.432332029 21.7133026123047 1.0
22.232980074 21.0747184753418 2.0
-7.2001290055 -6.87165451049805 3.0
15.032851059 14.5441732406616 4.0
28.632461044 28.4710445404053 5.0
43.665312113 42.2714920043945 6.0
43.665312113 43.1181564331055 7.0
40.6563483834553 38.2322006225586 0.0
20.9212538534144 22.2874374389648 1.0
19.7350945303589 20.575080871582 2.0
-7.59696943049863 -9.7576961517334 3.0
12.1381250996025 11.3042478561401 4.0
28.5182231244438 30.6564598083496 5.0
40.656348383 42.8358840942383 6.0
40.656348383 43.5187072753906 7.0
30.759139545 38.3696823120117 0.0
13.433268738 15.0255012512207 1.0
17.325870807 15.4236707687378 2.0
-5.9875785844 -4.64086389541626 3.0
11.338292213 11.294243812561 4.0
19.420847332 19.5131206512451 5.0
30.759139555 31.2909145355225 6.0
30.759139555 31.389928817749 7.0
36.474229645 40.226318359375 0.0
14.989459641 17.8264293670654 1.0
21.484770007 19.1009178161621 2.0
-5.0219280476 -3.41641426086426 3.0
16.462841952 16.4279747009277 4.0
20.011387696 18.6752967834473 5.0
36.474229652 35.9620742797852 6.0
36.474229652 35.4279861450195 7.0
33.340721185 38.9816360473633 0.0
15.596021276 16.8473777770996 1.0
17.744699909 16.3874359130859 2.0
-7.4570294442 -6.50424766540527 3.0
10.287670455 9.99333381652832 4.0
23.05305073 23.0990505218506 5.0
33.340721195 34.0192718505859 6.0
33.340721195 32.3213653564453 7.0
34.143349258 40.9910354614258 0.0
15.417033592 17.2636260986328 1.0
18.726315666 17.2722606658936 2.0
-5.8334622114 -4.53218936920166 3.0
12.892853445 13.1548357009888 4.0
21.250495814 20.2220420837402 5.0
34.143349268 33.7969512939453 6.0
34.143349268 34.1047210693359 7.0
40.763829748 38.9162445068359 0.0
20.589059268 21.5375442504883 1.0
20.174770481 18.90452003479 2.0
-11.800756688 -10.6801910400391 3.0
8.374013783 8.09120941162109 4.0
32.389815966 31.1681251525879 5.0
40.763829757 41.1318054199219 6.0
40.763829757 41.0230712890625 7.0
29.788224949 39.1653823852539 0.0
14.329239328 15.0021514892578 1.0
15.458985621 14.4682674407959 2.0
-6.2969425219 -6.21877908706665 3.0
9.1620430892 8.91473579406738 4.0
20.62618186 20.7115249633789 5.0
29.788224958 30.6734504699707 6.0
29.788224958 29.9834861755371 7.0
34.675869546 37.6262817382812 0.0
17.321679699 17.2443523406982 1.0
17.354189848 16.9586791992188 2.0
-5.026706589 -5.94963693618774 3.0
12.32748325 11.9344501495361 4.0
22.348386296 21.9416027069092 5.0
34.675869554 35.6226425170898 6.0
34.675869554 35.7264633178711 7.0
40.7238558279895 39.4912109375 0.0
19.1061702059935 21.142370223999 1.0
21.6176856259914 20.7369403839111 2.0
-8.60840523789577 -7.55541706085205 3.0
13.0092803879935 12.2754936218262 4.0
27.7145752997044 28.849515914917 5.0
40.723855828 42.8596496582031 6.0
40.723855828 42.3968048095703 7.0
41.414838344 37.8283233642578 0.0
21.79665744 21.0733413696289 1.0
19.618180905 18.9981441497803 2.0
-10.700829677 -10.7724571228027 3.0
8.9173512199 8.26488208770752 4.0
32.497487126 30.8184375762939 5.0
41.414838352 41.5673294067383 6.0
41.414838352 41.5150299072266 7.0
33.391441204 37.6373596191406 0.0
16.946662464 16.966480255127 1.0
16.44477874 16.3719825744629 2.0
-6.5811351715 -7.87235927581787 3.0
9.8636435585 9.08365631103516 4.0
23.527797645 23.7306652069092 5.0
33.391441214 33.2765502929688 6.0
33.391441214 32.7680435180664 7.0
36.660645376 37.8180084228516 0.0
17.87289688 18.109561920166 1.0
18.787748497 17.614595413208 2.0
-6.6425323884 -6.07526779174805 3.0
12.1452161 11.9890327453613 4.0
24.515429278 23.5778293609619 5.0
36.660645385 36.2939300537109 6.0
36.660645385 36.242317199707 7.0
33.547567157 39.7165679931641 0.0
16.605704713 17.7154903411865 1.0
16.941862445 15.9527902603149 2.0
-8.1253896728 -7.4156379699707 3.0
8.8164727625 8.34013557434082 4.0
24.731094395 24.3980903625488 5.0
33.547567166 34.0128707885742 6.0
33.547567166 32.6779708862305 7.0
36.717923485 39.2869567871094 0.0
19.502688668 18.7698841094971 1.0
17.215234818 17.8815155029297 2.0
-6.0032006006 -7.53859424591064 3.0
11.212034207 10.7419080734253 4.0
25.505889278 25.0586738586426 5.0
36.717923495 37.1936569213867 6.0
36.717923495 37.5066909790039 7.0
44.575737489 40.5854110717773 0.0
24.24689991 22.124979019165 1.0
20.328837584 22.1266841888428 2.0
-6.0388557339 -8.2476863861084 3.0
14.289981845 13.3454523086548 4.0
30.285755649 29.336555480957 5.0
44.575737494 45.5420227050781 6.0
44.575737494 43.1909103393555 7.0
43.091610313 38.3678588867188 0.0
22.329030269 21.560863494873 1.0
20.762580045 20.8383655548096 2.0
-7.1251861541 -8.47113418579102 3.0
13.637393882 12.9377851486206 4.0
29.454216432 28.8676280975342 5.0
43.091610321 42.3122711181641 6.0
43.091610321 43.2684097290039 7.0
42.647069964 37.9135208129883 0.0
21.774885881 21.5169792175293 1.0
20.872184083 20.7233734130859 2.0
-7.6954983452 -8.57610130310059 3.0
13.176685728 12.4285440444946 4.0
29.470384236 29.0644855499268 5.0
42.647069973 41.9329528808594 6.0
42.647069973 42.6456298828125 7.0
36.24022533 38.1860504150391 0.0
17.188554164 17.9934139251709 1.0
19.051671166 18.1627101898193 2.0
-6.1891251621 -6.36701965332031 3.0
12.862545995 12.3815546035767 4.0
23.377679336 22.9789581298828 5.0
36.240225339 36.4092025756836 6.0
36.240225339 36.9738311767578 7.0
31.876756195 38.5105590820312 0.0
14.446454361 16.0429267883301 1.0
17.430301835 15.9513492584229 2.0
-5.0545683645 -4.60816955566406 3.0
12.375733461 12.1305341720581 4.0
19.501022735 18.7645626068115 5.0
31.876756203 32.8788146972656 6.0
31.876756203 33.262825012207 7.0
38.711049649 38.8478851318359 0.0
19.068925854 18.7625312805176 1.0
19.642123794 19.1813526153564 2.0
-5.0802825502 -6.32479190826416 3.0
14.561841234 13.9412775039673 4.0
24.149208415 23.7705402374268 5.0
38.711049659 37.8623046875 6.0
38.711049659 38.0536041259766 7.0
33.592635662 38.2101058959961 0.0
16.042625471 16.7317085266113 1.0
17.550010192 16.7213745117188 2.0
-5.9813560668 -6.13964128494263 3.0
11.568654115 11.3106861114502 4.0
22.023981548 21.8307762145996 5.0
33.592635672 34.0051498413086 6.0
33.592635672 33.7525329589844 7.0
31.9394970213296 38.4451141357422 0.0
15.7221064593445 15.8866930007935 1.0
16.2173905613348 16.2083702087402 2.0
-3.67606552949996 -4.31400108337402 3.0
12.5413250313355 12.2592344284058 4.0
19.3981719893576 19.1507167816162 5.0
31.939497021 32.6907539367676 6.0
31.939497021 33.176887512207 7.0
32.605677366 39.1546630859375 0.0
15.053063258 16.0942420959473 1.0
17.552614114 16.665340423584 2.0
-5.3475005701 -5.25166940689087 3.0
12.20511354 11.8578281402588 4.0
20.400563832 20.0710620880127 5.0
32.605677369 33.5073165893555 6.0
32.605677369 33.7071304321289 7.0
35.02561424 37.8257446289062 0.0
18.375955543 17.9033374786377 1.0
16.649658698 17.3728160858154 2.0
-5.1216145805 -6.55123424530029 3.0
11.528044108 11.2901048660278 4.0
23.497570133 22.6965236663818 5.0
35.02561425 35.544075012207 6.0
35.02561425 35.8606414794922 7.0
29.155020361 38.2448348999023 0.0
13.928860816 14.4688167572021 1.0
15.226159546 14.524450302124 2.0
-5.2043069028 -4.99999237060547 3.0
10.021852634 10.0219116210938 4.0
19.133167728 19.0475940704346 5.0
29.155020371 30.7879123687744 6.0
29.155020371 30.5458068847656 7.0
29.979099891 36.8306045532227 0.0
16.392071584 14.8867559432983 1.0
13.587028308 14.4228944778442 2.0
-4.5603535805 -7.07263088226318 3.0
9.026674718 8.6599006652832 4.0
20.952425173 21.9011154174805 5.0
29.9790999 31.8999938964844 6.0
29.9790999 29.1347827911377 7.0
40.412374419 38.1481323242188 0.0
21.632869342 19.2404441833496 1.0
18.779505078 19.3067245483398 2.0
-4.6354756287 -7.43406772613525 3.0
14.14402944 13.6475553512573 4.0
26.26834498 26.1334705352783 5.0
40.412374428 39.5415802001953 6.0
40.412374428 39.9898986816406 7.0
44.7996696066759 37.9980392456055 0.0
24.3565352806759 21.7528915405273 1.0
20.4431343366759 21.9079704284668 2.0
-4.33926525477587 -6.89923667907715 3.0
16.1038690816759 15.8342761993408 4.0
28.6958005009671 27.6428508758545 5.0
44.799669607 44.8251876831055 6.0
44.799669607 44.7814636230469 7.0
35.691930078 39.358154296875 0.0
18.173171081 17.9495601654053 1.0
17.518759 18.1219100952148 2.0
-4.3926482071 -5.22031116485596 3.0
13.126110785 12.507830619812 4.0
22.565819295 22.0303707122803 5.0
35.691930085 36.1713790893555 6.0
35.691930085 36.1316299438477 7.0
37.293002373 37.5319519042969 0.0
17.018201783 18.5514183044434 1.0
20.274800591 17.8870372772217 2.0
-8.9983650782 -7.6791410446167 3.0
11.276435503 10.9807929992676 4.0
26.016566871 25.8867511749268 5.0
37.293002382 36.7497329711914 6.0
37.293002382 37.4043655395508 7.0
31.749050626 38.4727096557617 0.0
16.61083934 15.8307504653931 1.0
15.138211288 15.9538488388062 2.0
-3.9083704158 -5.61944961547852 3.0
11.229840863 10.8547801971436 4.0
20.519209764 19.7830390930176 5.0
31.749050635 32.5288543701172 6.0
31.749050635 32.7002868652344 7.0
44.43859931 38.7235336303711 0.0
22.567248059 22.5149822235107 1.0
21.871351255 20.7959957122803 2.0
-10.274790847 -9.63494777679443 3.0
11.596560402 10.814507484436 4.0
32.842038912 31.5965003967285 5.0
44.438599316 41.4154891967773 6.0
44.438599316 43.2081680297852 7.0
30.584762336 36.7992782592773 0.0
14.448986708 15.5870504379272 1.0
16.13577563 14.6688270568848 2.0
-7.0107275854 -6.67455816268921 3.0
9.1250480359 9.00351905822754 4.0
21.459714302 21.4524440765381 5.0
30.584762345 32.0151710510254 6.0
30.584762345 30.9605445861816 7.0
36.167711461 40.1857681274414 0.0
15.702748451 17.8237361907959 1.0
20.464963013 18.4717121124268 2.0
-4.5943051912 -3.68921661376953 3.0
15.870657815 15.2465696334839 4.0
20.297053649 19.1244277954102 5.0
36.167711467 36.386474609375 6.0
36.167711467 36.5115127563477 7.0
45.589023155 38.3743133544922 0.0
22.43199889 22.9356861114502 1.0
23.157024265 21.4268608093262 2.0
-9.6703339461 -8.2329273223877 3.0
13.486690309 12.9158134460449 4.0
32.102332846 30.6000347137451 5.0
45.589023165 44.1303176879883 6.0
45.589023165 45.0658645629883 7.0
33.954863965 38.4884490966797 0.0
16.569146941 17.1004333496094 1.0
17.385717024 15.977596282959 2.0
-8.6921868392 -9.10738182067871 3.0
8.6935301746 8.16158390045166 4.0
25.26133379 25.3978309631348 5.0
33.954863975 32.5750885009766 6.0
33.954863975 33.2103843688965 7.0
39.104021209 38.1976928710938 0.0
19.726585997 19.6325626373291 1.0
19.377435214 19.6350040435791 2.0
-5.8707549184 -7.07136917114258 3.0
13.506680288 12.9698495864868 4.0
25.597340923 25.1903820037842 5.0
39.104021217 39.1459655761719 6.0
39.104021217 39.5250015258789 7.0
41.6151393745108 39.4013900756836 0.0
21.6144474948183 23.5700511932373 1.0
20.0006918800463 22.4477863311768 2.0
-5.68615976863137 -9.3006649017334 3.0
14.3145321112399 13.2300615310669 4.0
27.3006071309532 31.0678310394287 5.0
41.615139375 47.0245132446289 6.0
41.615139375 46.963737487793 7.0
29.251209159 37.5830535888672 0.0
16.305978933 14.7973585128784 1.0
12.945230226 13.9417028427124 2.0
-4.8421763952 -6.6864709854126 3.0
8.1030538205 7.87594175338745 4.0
21.148155339 21.0361042022705 5.0
29.251209169 30.5941467285156 6.0
29.251209169 29.7554321289062 7.0
39.145261593 38.5635375976562 0.0
20.853611949 19.8906574249268 1.0
18.291649647 18.9076118469238 2.0
-6.3281983222 -7.84693908691406 3.0
11.963451316 11.5609912872314 4.0
27.181810279 26.8756942749023 5.0
39.145261601 38.62890625 6.0
39.145261601 39.4422607421875 7.0
43.724223504 38.7131423950195 0.0
24.240212841 21.9978733062744 1.0
19.484010663 20.526388168335 2.0
-6.7543223688 -9.00374126434326 3.0
12.729688284 12.0810670852661 4.0
30.99453522 29.9828624725342 5.0
43.724223514 42.9493255615234 6.0
43.724223514 43.0419464111328 7.0
30.694427938 38.4703216552734 0.0
15.962392584 15.455174446106 1.0
14.732035355 14.5811805725098 2.0
-5.8529078341 -7.19793224334717 3.0
8.8791275119 8.61276054382324 4.0
21.815300427 22.3543128967285 5.0
30.694427948 30.4355545043945 6.0
30.694427948 30.8030529022217 7.0
46.899446071 41.2384796142578 0.0
22.810606072 23.9734020233154 1.0
24.08884 22.4933032989502 2.0
-12.207831656 -9.55136775970459 3.0
11.881008335 10.8042459487915 4.0
35.018437738 33.4036178588867 5.0
46.899446079 48.6947555541992 6.0
46.899446079 46.1362533569336 7.0
30.801741864 37.6285858154297 0.0
14.581609416 15.9768629074097 1.0
16.220132448 14.6345262527466 2.0
-7.6078809929 -6.72684097290039 3.0
8.6122514454 8.90434455871582 4.0
22.189490419 21.7099018096924 5.0
30.801741874 31.9071769714355 6.0
30.801741874 31.7369689941406 7.0
40.3889007570329 38.0543670654297 0.0
21.8864546730297 20.3379878997803 1.0
18.5024460860314 18.9957790374756 2.0
-8.35977776734346 -9.02691745758057 3.0
10.1426683190367 9.57519721984863 4.0
30.2462322648819 29.0913619995117 5.0
40.388900757 39.9377670288086 6.0
40.388900757 40.1006011962891 7.0
35.801686121 37.8537368774414 0.0
19.152651215 17.5728244781494 1.0
16.649034906 17.9854488372803 2.0
-1.2326560583 -3.47315120697021 3.0
15.416378838 14.7052917480469 4.0
20.385307283 19.0002346038818 5.0
35.801686131 35.7912902832031 6.0
35.801686131 36.2294082641602 7.0
43.275430292 37.515022277832 0.0
21.49768275 21.9793605804443 1.0
21.777747542 20.6276302337646 2.0
-8.5057364073 -8.83036231994629 3.0
13.272011125 12.5774688720703 4.0
30.003419167 29.1209125518799 5.0
43.275430302 43.2871246337891 6.0
43.275430302 43.1338653564453 7.0
36.6745338859035 38.2283172607422 0.0
17.2258453439028 18.5150699615479 1.0
19.4486885479032 18.3444747924805 2.0
-7.45674864010256 -7.02922916412354 3.0
11.9919399079021 11.4609842300415 4.0
24.6825939849038 24.3843479156494 5.0
36.674533885 36.7487487792969 6.0
36.674533885 37.4470138549805 7.0
42.2919299411196 38.5026702880859 0.0
20.8701371011196 20.9765014648438 1.0
21.4217928491193 20.4122142791748 2.0
-7.8990978219197 -8.14122104644775 3.0
13.5226950281196 12.5381097793579 4.0
28.769234923119 28.1013698577881 5.0
42.291929941 40.9769821166992 6.0
42.291929941 41.6912384033203 7.0
37.80686854 37.7174224853516 0.0
18.84529386 18.549186706543 1.0
18.961574682 19.310604095459 2.0
-3.0668288295 -4.97530174255371 3.0
15.894745843 15.2099599838257 4.0
21.912122698 21.1462707519531 5.0
37.80686855 38.2931594848633 6.0
37.80686855 37.5266723632812 7.0
36.296445798 38.1344528198242 0.0
19.144127702 17.7966156005859 1.0
17.152318097 18.110237121582 2.0
-4.4799623055 -6.88712787628174 3.0
12.672355782 12.2650785446167 4.0
23.624090017 23.5981693267822 5.0
36.296445808 35.3136367797852 6.0
36.296445808 36.5476531982422 7.0
26.786347975 42.6801834106445 0.0
13.739106858 13.5130414962769 1.0
13.047241117 13.2145490646362 2.0
-5.0192652199 -5.87567567825317 3.0
8.0279758877 7.60976409912109 4.0
18.758372088 18.9660835266113 5.0
26.786347985 28.9723720550537 6.0
26.786347985 28.2391414642334 7.0
38.715518927 38.507682800293 0.0
20.971478582 19.7143726348877 1.0
17.744040345 18.1030349731445 2.0
-7.5775969749 -9.30812072753906 3.0
10.16644336 9.681884765625 4.0
28.549075567 27.826286315918 5.0
38.715518937 37.4333877563477 6.0
38.715518937 38.1795654296875 7.0
35.505062995 38.513427734375 0.0
16.975132375 17.2819538116455 1.0
18.529930624 17.7054233551025 2.0
-6.1934391103 -6.34982109069824 3.0
12.336491507 12.0390348434448 4.0
23.168571492 23.276496887207 5.0
35.505063002 35.3272552490234 6.0
35.505063002 35.8840560913086 7.0
44.227649061 38.7107391357422 0.0
23.455343466 22.3802471160889 1.0
20.772305597 21.5785312652588 2.0
-7.272984789 -9.04497814178467 3.0
13.499320798 12.4648637771606 4.0
30.728328264 30.2675457000732 5.0
44.227649071 44.5139617919922 6.0
44.227649071 43.0474395751953 7.0
45.773124246 38.4521026611328 0.0
24.134680774 22.6842613220215 1.0
21.638443473 22.3891849517822 2.0
-6.2883820036 -7.2651834487915 3.0
15.350061461 14.9112253189087 4.0
30.423062786 29.5425624847412 5.0
45.773124255 43.8929138183594 6.0
45.773124255 44.469352722168 7.0
38.6152827522378 39.1381759643555 0.0
18.6636500942378 21.1879348754883 1.0
19.9516326682378 20.1070251464844 2.0
-8.33253161363777 -9.0809211730957 3.0
11.6191010542378 10.7593660354614 4.0
26.9961816644691 29.2537994384766 5.0
38.615282752 41.3080062866211 6.0
38.615282752 41.2782516479492 7.0
42.131124826 38.4854736328125 0.0
21.88265512 20.588752746582 1.0
20.248469706 20.2322254180908 2.0
-6.067617241 -7.61580181121826 3.0
14.180852455 13.8058958053589 4.0
27.950272371 27.9883079528809 5.0
42.131124836 40.829704284668 6.0
42.131124836 41.1903457641602 7.0
38.030369756 38.8973083496094 0.0
19.698625144 19.6794357299805 1.0
18.331744611 18.5008449554443 2.0
-8.0645153211 -8.2222261428833 3.0
10.26722928 9.75477409362793 4.0
27.763140475 27.2713928222656 5.0
38.030369766 38.3273696899414 6.0
38.030369766 38.1484527587891 7.0
32.694869995 38.2904281616211 0.0
14.429732866 16.1712837219238 1.0
18.265137131 16.580057144165 2.0
-5.8301144142 -4.95665502548218 3.0
12.435022709 12.3130111694336 4.0
20.259847288 19.9836120605469 5.0
32.694870002 33.1489868164062 6.0
32.694870002 33.3169479370117 7.0
41.58883183 41.6641159057617 0.0
19.184872754 20.3355979919434 1.0
22.403959077 20.9348182678223 2.0
-6.437067104 -5.56141662597656 3.0
15.966891964 15.7411727905273 4.0
25.621939867 24.977367401123 5.0
41.58883184 41.1224365234375 6.0
41.58883184 41.019416809082 7.0
33.622063035 38.832878112793 0.0
16.212159516 17.3043212890625 1.0
17.409903519 15.6869668960571 2.0
-8.2486263793 -7.31462860107422 3.0
9.16127713 8.79054832458496 4.0
24.460785905 24.6647968292236 5.0
33.622063045 33.4089508056641 6.0
33.622063045 32.7049369812012 7.0
40.101738497 39.5207138061523 0.0
22.613544165 20.7906303405762 1.0
17.488194333 18.8444232940674 2.0
-8.294425066 -10.4181060791016 3.0
9.1937692583 8.37910270690918 4.0
30.90796924 30.3288879394531 5.0
40.101738506 39.3720703125 6.0
40.101738506 40.6680603027344 7.0
39.347750484 38.0299377441406 0.0
19.631600055 20.4716777801514 1.0
19.716150429 19.8641872406006 2.0
-7.2110260491 -7.61504364013672 3.0
12.50512437 11.6787710189819 4.0
26.842626114 27.9090576171875 5.0
39.347750494 40.1459808349609 6.0
39.347750494 41.011962890625 7.0
46.240008453 38.8345947265625 0.0
23.277257726 22.0742149353027 1.0
22.962750727 22.1802196502686 2.0
-6.5077704351 -7.46328449249268 3.0
16.454980282 15.5035314559937 4.0
29.785028171 28.4560871124268 5.0
46.240008463 44.5308380126953 6.0
46.240008463 45.5188598632812 7.0
43.885613963 38.0525588989258 0.0
23.043389811 21.3101081848145 1.0
20.842224153 21.1093502044678 2.0
-5.1243468199 -6.55911016464233 3.0
15.717877324 15.2683639526367 4.0
28.16773664 27.1201992034912 5.0
43.885613972 43.3039321899414 6.0
43.885613972 43.0225067138672 7.0
44.877286045 38.8000869750977 0.0
23.85937511 22.7260208129883 1.0
21.017910935 20.5420475006104 2.0
-10.548874854 -10.5124816894531 3.0
10.469036071 9.67226791381836 4.0
34.408249974 32.6380271911621 5.0
44.877286055 43.9720916748047 6.0
44.877286055 44.9793243408203 7.0
44.28102484 38.0077743530273 0.0
23.476033214 22.1387367248535 1.0
20.804991626 21.0905380249023 2.0
-7.5751269732 -9.15844440460205 3.0
13.229864644 12.7179937362671 4.0
31.051160197 30.1925029754639 5.0
44.281024849 43.0857620239258 6.0
44.281024849 43.9228057861328 7.0
39.443195324 37.3810882568359 0.0
17.235953612 19.3434944152832 1.0
22.207241712 18.8450679779053 2.0
-9.075573722 -7.60132884979248 3.0
13.13166798 12.7153692245483 4.0
26.311527344 25.5991306304932 5.0
39.443195334 38.8532257080078 6.0
39.443195334 38.8245544433594 7.0
44.5099801195094 36.8591995239258 0.0
20.7192442195094 22.3555526733398 1.0
23.7907359105096 21.7317199707031 2.0
-9.37764966100969 -7.63202953338623 3.0
14.4130862495096 13.609694480896 4.0
30.0968938805093 28.47580909729 5.0
44.50998012 42.7778625488281 6.0
44.50998012 43.8332748413086 7.0
36.079576813 38.4415283203125 0.0
18.198560719 18.7957248687744 1.0
17.881016095 17.4483032226562 2.0
-8.6789161085 -9.28349304199219 3.0
9.202099978 8.59134674072266 4.0
26.877476836 27.5283660888672 5.0
36.079576822 36.163818359375 6.0
36.079576822 36.8237457275391 7.0
43.881758866 39.027214050293 0.0
21.090009056 25.6206130981445 1.0
22.79174981 23.5201053619385 2.0
-6.6710954927 -8.04027843475342 3.0
16.120654308 14.9540529251099 4.0
27.761104558 30.4392051696777 5.0
43.881758876 46.893798828125 6.0
43.881758876 49.5017700195312 7.0
30.108818274 38.0872573852539 0.0
17.112174764 14.8953723907471 1.0
12.996643511 14.1075763702393 2.0
-4.8964528045 -7.41787242889404 3.0
8.1001906976 8.07676124572754 4.0
22.008627577 22.1093921661377 5.0
30.108818284 30.7107715606689 6.0
30.108818284 29.7873954772949 7.0
34.2205121900012 38.3739700317383 0.0
18.8110553650029 17.6508178710938 1.0
15.4094568300067 16.343204498291 2.0
-7.05168531729669 -9.02465915679932 3.0
8.35777151259514 8.10700225830078 4.0
25.8627406830075 26.1589450836182 5.0
34.22051219 33.3663787841797 6.0
34.22051219 34.2508392333984 7.0
37.402197962 37.5958633422852 0.0
19.960548139 19.5640735626221 1.0
17.441649823 18.4205322265625 2.0
-5.9560955294 -7.01253890991211 3.0
11.485554285 10.9263305664062 4.0
25.916643678 25.4087047576904 5.0
37.402197971 38.1326751708984 6.0
37.402197971 37.9016265869141 7.0
35.453722399 38.0886306762695 0.0
16.625051972 17.825475692749 1.0
18.828670427 17.2883014678955 2.0
-7.5918246106 -7.05069732666016 3.0
11.236845807 10.8335590362549 4.0
24.216876592 23.7833080291748 5.0
35.453722409 34.7095642089844 6.0
35.453722409 35.6290512084961 7.0
32.872381145 38.9959030151367 0.0
15.679111706 16.1414489746094 1.0
17.193269439 16.6131019592285 2.0
-6.3769976884 -6.32965803146362 3.0
10.816271741 10.4307107925415 4.0
22.056109405 22.1622524261475 5.0
32.872381155 32.7730331420898 6.0
32.872381155 33.1625518798828 7.0
33.5777143162235 38.2724685668945 0.0
24.3043372519187 17.6195983886719 1.0
9.27337706863705 16.3525238037109 2.0
-1.413226005e-07 -8.37569046020508 3.0
9.27337692732672 8.6875524520874 4.0
24.3043373936238 24.3537464141846 5.0
33.577714316 33.2163162231445 6.0
33.577714316 33.0355224609375 7.0
43.986168792 39.041877746582 0.0
19.73252431 21.5856075286865 1.0
24.253644482 20.8663463592529 2.0
-7.8587048092 -5.91611337661743 3.0
16.394939663 15.5781421661377 4.0
27.591229129 26.2266521453857 5.0
43.986168802 41.7158050537109 6.0
43.986168802 42.3371353149414 7.0
39.701527196 38.9045562744141 0.0
21.433905404 19.840934753418 1.0
18.267621793 19.1758861541748 2.0
-6.9530592881 -8.71702003479004 3.0
11.314562495 10.6607789993286 4.0
28.386964701 28.2417583465576 5.0
39.701527205 39.0857620239258 6.0
39.701527205 39.897834777832 7.0
36.319422307 37.8450088500977 0.0
18.261170173 19.0830535888672 1.0
18.058252134 17.1586532592773 2.0
-9.3076661868 -8.92712688446045 3.0
8.7505859373 8.37872791290283 4.0
27.56883637 27.2055644989014 5.0
36.319422317 36.7306365966797 6.0
36.319422317 37.2687530517578 7.0
39.546529711 38.6017227172852 0.0
21.883828824 20.6165561676025 1.0
17.662700888 19.2193470001221 2.0
-6.2516235274 -7.78132343292236 3.0
11.411077352 10.8421440124512 4.0
28.13545236 27.5925903320312 5.0
39.54652972 39.2355422973633 6.0
39.54652972 39.8413543701172 7.0
46.328884399 38.8549118041992 0.0
20.182008475 22.8751964569092 1.0
26.146875933 22.5814533233643 2.0
-10.473669833 -8.06719207763672 3.0
15.673206099 14.8391761779785 4.0
30.655678309 29.4269580841064 5.0
46.3288844 45.0779724121094 6.0
46.3288844 46.2499313354492 7.0
39.359671453 38.194091796875 0.0
20.208221774 20.3423404693604 1.0
19.151449681 19.1097717285156 2.0
-7.7565035076 -8.42878341674805 3.0
11.394946165 10.7688608169556 4.0
27.964725289 27.7421894073486 5.0
39.359671461 39.5275573730469 6.0
39.359671461 40.1919708251953 7.0
43.813870978 38.8582763671875 0.0
22.861832963 22.0358791351318 1.0
20.952038016 21.2789440155029 2.0
-6.6064627953 -8.3498363494873 3.0
14.345575211 13.5406513214111 4.0
29.468295768 29.1588001251221 5.0
43.813870988 43.1231842041016 6.0
43.813870988 42.5593948364258 7.0
39.146221866 37.9765167236328 0.0
21.274977313 20.3604354858398 1.0
17.871244553 18.3191699981689 2.0
-8.221666714 -9.06628322601318 3.0
9.6495778292 9.37703990936279 4.0
29.496644037 28.4865341186523 5.0
39.146221876 38.9574432373047 6.0
39.146221876 39.3885269165039 7.0
34.753353004 43.2433700561523 0.0
19.306587872 19.1949615478516 1.0
15.446765133 17.0121841430664 2.0
-7.4077231802 -8.44002819061279 3.0
8.0390419425 7.87218189239502 4.0
26.714311062 26.8391723632812 5.0
34.753353014 35.7559432983398 6.0
34.753353014 35.3633346557617 7.0
36.8589150083515 39.3040237426758 0.0
18.5606086093515 19.5789794921875 1.0
18.2983064083517 18.169942855835 2.0
-8.423068234052 -8.68423557281494 3.0
9.87523817465195 9.27243614196777 4.0
26.9836767778432 27.2485866546631 5.0
36.858915008 37.5935211181641 6.0
36.858915008 38.6410980224609 7.0
45.609388715 38.4172973632812 0.0
20.431056468 22.9536399841309 1.0
25.178332247 22.4702548980713 2.0
-8.8685199312 -6.5566611289978 3.0
16.309812306 15.7383069992065 4.0
29.299576409 28.2491645812988 5.0
45.609388725 43.5644912719727 6.0
45.609388725 45.0015563964844 7.0
37.221502305 38.3516006469727 0.0
19.795691461 19.6236991882324 1.0
17.425810844 17.979154586792 2.0
-8.0004149655 -8.99427700042725 3.0
9.4253958691 8.80390930175781 4.0
27.796106436 27.9256324768066 5.0
37.221502314 37.9484024047852 6.0
37.221502314 37.553581237793 7.0
39.3512418478291 37.6393203735352 0.0
21.5003115557496 20.3177757263184 1.0
17.8509302889164 18.7557277679443 2.0
-7.9913457065862 -8.93670558929443 3.0
9.85958458276425 9.30246639251709 4.0
29.4916572628094 28.8016185760498 5.0
39.351241848 38.7210159301758 6.0
39.351241848 39.3560943603516 7.0
44.354286129 38.2987594604492 0.0
23.778468789 22.5706043243408 1.0
20.57581734 21.4556121826172 2.0
-6.7304183908 -8.56882190704346 3.0
13.845398939 13.29296875 4.0
30.50888719 29.5489082336426 5.0
44.354286139 43.3493347167969 6.0
44.354286139 44.251838684082 7.0
41.1438439687938 38.1038208007812 0.0
20.0722524531884 20.0741729736328 1.0
21.0715915195 19.5480842590332 2.0
-7.89338141352269 -8.26870822906494 3.0
13.1782101064826 12.4874887466431 4.0
27.9656338666077 27.7842121124268 5.0
41.143843969 40.4757919311523 6.0
41.143843969 40.8191757202148 7.0
32.436363662 39.0113754272461 0.0
18.29560079 16.6161632537842 1.0
14.140762872 16.010425567627 2.0
-4.3766657499 -6.25706720352173 3.0
9.7640971121 9.62257289886475 4.0
22.67226655 22.4643726348877 5.0
32.436363672 33.1113319396973 6.0
32.436363672 33.7142486572266 7.0
35.980268817 38.7608642578125 0.0
18.371260578 19.0526332855225 1.0
17.609008241 17.4489593505859 2.0
-7.5074447835 -7.72255992889404 3.0
10.101563449 9.86926555633545 4.0
25.87870537 25.642749786377 5.0
35.980268826 38.5067367553711 6.0
35.980268826 37.0414428710938 7.0
39.361696941 38.5839385986328 0.0
18.798616943 18.5381984710693 1.0
20.563079997 19.60231590271 2.0
-4.5015726852 -5.77445220947266 3.0
16.061507302 15.1822214126587 4.0
23.300189639 22.9885711669922 5.0
39.361696951 39.0705871582031 6.0
39.361696951 39.5996551513672 7.0
38.339957491 38.8900909423828 0.0
20.325182647 21.0611763000488 1.0
18.014774844 20.0291538238525 2.0
-6.00621 -8.46483707427979 3.0
12.008564835 11.5466804504395 4.0
26.331392656 29.179708480835 5.0
38.339957501 41.6321411132812 6.0
38.339957501 41.3705215454102 7.0
38.4370780266379 38.1817092895508 0.0
19.1112284596379 22.9684066772461 1.0
19.3258495766379 20.8143291473389 2.0
-8.41787501523793 -10.4658393859863 3.0
10.9079745616379 9.77639198303223 4.0
27.5291034382175 32.4787940979004 5.0
38.437078026 43.5633163452148 6.0
38.437078026 42.662971496582 7.0
31.680301152 38.6968383789062 0.0
16.922816031 15.7942514419556 1.0
14.757485122 14.8312072753906 2.0
-6.2787127987 -8.25482273101807 3.0
8.4787723148 7.88749504089355 4.0
23.201528838 23.5726051330566 5.0
31.680301161 30.23606300354 6.0
31.680301161 30.2849559783936 7.0
36.698769122 39.2011032104492 0.0
19.594432034 17.6511535644531 1.0
17.104337088 18.1099872589111 2.0
-2.472193454 -4.85581254959106 3.0
14.632143624 14.4233636856079 4.0
22.066625498 21.2976608276367 5.0
36.698769132 35.2066955566406 6.0
36.698769132 36.1286392211914 7.0
40.942334912 38.3624801635742 0.0
22.219375241 19.7867298126221 1.0
18.722959671 19.6671524047852 2.0
-5.2427967213 -7.02317523956299 3.0
13.48016294 12.8304309844971 4.0
27.462171973 26.1630973815918 5.0
40.942334922 40.4672698974609 6.0
40.942334922 40.4371871948242 7.0
41.496981527 37.8840942382812 0.0
18.951043784 22.03586769104 1.0
22.545937743 21.4198608398438 2.0
-7.7537930795 -8.08576393127441 3.0
14.792144654 14.3052701950073 4.0
26.704836873 29.4385719299316 5.0
41.496981537 44.5245513916016 6.0
41.496981537 44.6415786743164 7.0
33.577466055 36.7898712158203 0.0
17.755510174 17.1780185699463 1.0
15.821955881 15.8019666671753 2.0
-7.358985858 -7.47995376586914 3.0
8.4629700135 8.25488758087158 4.0
25.114496042 24.0903701782227 5.0
33.577466065 34.1602630615234 6.0
33.577466065 33.8193740844727 7.0
38.410697594746 39.6828231811523 0.0
20.0282722997678 18.9050846099854 1.0
18.3824252977607 20.1255741119385 2.0
-2.25462323941058 -4.23089838027954 3.0
16.1278020587646 15.6055974960327 4.0
22.2828955387576 21.443187713623 5.0
38.410697595 38.4636077880859 6.0
38.410697595 38.7058334350586 7.0
40.390522102 39.0937118530273 0.0
23.101316011 20.7243633270264 1.0
17.289206093 18.6004695892334 2.0
-8.5131515481 -10.4230661392212 3.0
8.7760545352 8.47414588928223 4.0
31.614467568 30.8545036315918 5.0
40.390522112 39.4214401245117 6.0
40.390522112 40.6173629760742 7.0
38.02724698 38.0636672973633 0.0
17.092159812 18.586088180542 1.0
20.935087168 19.0112609863281 2.0
-7.0096758313 -6.10312986373901 3.0
13.925411327 13.6782245635986 4.0
24.101835653 23.475118637085 5.0
38.02724699 38.3910369873047 6.0
38.02724699 38.6760025024414 7.0
34.831322289 38.1769027709961 0.0
17.981428364 17.1401195526123 1.0
16.849893928 17.1651268005371 2.0
-5.7085713019 -7.30678272247314 3.0
11.14132262 10.6464405059814 4.0
23.689999673 23.5980854034424 5.0
34.831322295 34.5875854492188 6.0
34.831322295 35.0507965087891 7.0
33.152476822 39.5353088378906 0.0
16.395481675 16.7693042755127 1.0
16.756995148 16.0176315307617 2.0
-6.4008279506 -6.76018333435059 3.0
10.356167188 9.77749252319336 4.0
22.796309635 22.7096328735352 5.0
33.152476832 33.6477966308594 6.0
33.152476832 31.664514541626 7.0
37.7618147597387 38.8365478515625 0.0
20.5714887837879 19.5433959960938 1.0
17.1903259800247 18.5501022338867 2.0
-5.43010176849644 -7.48969841003418 3.0
11.7602242121808 11.0746173858643 4.0
26.0015905517818 25.8585243225098 5.0
37.76181476 37.0658569335938 6.0
37.76181476 38.1935043334961 7.0
46.821650957 38.3199157714844 0.0
22.453957792 23.5474395751953 1.0
24.367693165 22.5898189544678 2.0
-10.797760543 -9.05830097198486 3.0
13.569932612 12.4644193649292 4.0
33.251718344 31.6772708892822 5.0
46.821650967 46.1562271118164 6.0
46.821650967 46.6109848022461 7.0
41.1837901754312 38.6112060546875 0.0
19.2503118354314 20.2322578430176 1.0
21.9334783494314 20.0590953826904 2.0
-7.5443023473312 -6.67366218566895 3.0
14.3891760014314 13.9186019897461 4.0
26.7946141824314 26.1711616516113 5.0
41.183790175 40.1880493164062 6.0
41.183790175 40.5314788818359 7.0
36.784988632 38.8896408081055 0.0
21.035700431 19.03271484375 1.0
15.749288201 17.5402450561523 2.0
-6.0648376105 -8.63184070587158 3.0
9.6844505818 9.24567317962646 4.0
27.100538051 26.8921699523926 5.0
36.78498864 36.8820114135742 6.0
36.78498864 37.6822967529297 7.0
37.6894949442102 38.8402252197266 0.0
17.6138134052102 22.3506031036377 1.0
20.0756815502102 20.9995899200439 2.0
-8.78806436921022 -9.89967250823975 3.0
11.2876171802102 10.2714033126831 4.0
26.401877737898 31.1973533630371 5.0
37.689494945 43.9602508544922 6.0
37.689494945 42.8615112304688 7.0
41.308881088 37.9395294189453 0.0
21.148817904 21.0546417236328 1.0
20.160063185 20.1850166320801 2.0
-6.8065142938 -8.00381851196289 3.0
13.353548882 12.6052093505859 4.0
27.955332207 27.4679794311523 5.0
41.308881097 41.778434753418 6.0
41.308881097 41.7641677856445 7.0
42.253011175 37.220344543457 0.0
21.557015891 21.945873260498 1.0
20.695995284 20.3488464355469 2.0
-9.3835381575 -9.3173999786377 3.0
11.312457118 10.7231931686401 4.0
30.940554058 30.3450355529785 5.0
42.253011184 41.5468368530273 6.0
42.253011184 42.3424377441406 7.0
40.047660483 39.439697265625 0.0
19.392839472 21.3383140563965 1.0
20.654821011 20.470890045166 2.0
-7.4809136797 -7.925124168396 3.0
13.173907322 12.2221212387085 4.0
26.873753161 27.2404041290283 5.0
40.047660493 43.3388900756836 6.0
40.047660493 40.7363052368164 7.0
46.604605723 40.7205352783203 0.0
22.954120862 22.8051605224609 1.0
23.650484861 23.0985088348389 2.0
-7.9204617962 -7.7554407119751 3.0
15.730023055 15.3602828979492 4.0
30.874582668 29.926815032959 5.0
46.604605733 47.6629104614258 6.0
46.604605733 46.2474212646484 7.0
45.106295532 38.264518737793 0.0
23.2126055 22.8173294067383 1.0
21.893690031 21.1630477905273 2.0
-10.542381522 -10.2076473236084 3.0
11.3513085 10.5679950714111 4.0
33.754987032 32.8176498413086 5.0
45.106295542 43.587158203125 6.0
45.106295542 44.4005889892578 7.0
38.977055089 39.6535949707031 0.0
19.684728957 19.9886054992676 1.0
19.292326132 18.5582733154297 2.0
-8.6831116468 -9.32623386383057 3.0
10.609214475 9.9409351348877 4.0
28.367840614 28.3870124816895 5.0
38.977055099 39.1203765869141 6.0
38.977055099 38.0659027099609 7.0
38.935365873 38.247673034668 0.0
21.088579762 18.8313083648682 1.0
17.846786112 19.5821418762207 2.0
-3.2985252296 -5.88558101654053 3.0
14.548260873 13.8563652038574 4.0
24.387105001 23.9787139892578 5.0
38.935365883 38.69384765625 6.0
38.935365883 39.1697158813477 7.0
42.74140991 38.2397308349609 0.0
20.9458217 20.9357070922852 1.0
21.795588212 20.7361640930176 2.0
-6.3794891462 -5.80559158325195 3.0
15.416099058 14.8113632202148 4.0
27.325310854 25.7390193939209 5.0
42.741409918 42.1773986816406 6.0
42.741409918 42.1405639648438 7.0
31.310794567 38.8892364501953 0.0
16.880318068 15.570930480957 1.0
14.4304765 14.8110074996948 2.0
-5.6188756586 -6.82168197631836 3.0
8.8116008333 8.73148727416992 4.0
22.499193736 22.5182342529297 5.0
31.310794576 31.7474250793457 6.0
31.310794576 31.7730731964111 7.0
};
\addlegendentry{$R^2$=0.983}
\end{axis}

\end{tikzpicture}
}}
    
    \caption{Model results using only the loss associated with nodal flow predictions in the 8-node network.}
    \label{fig:dummy_base_results}
\end{figure}


