\chapter{Enhanced Natural Gas Flow Predictions Using Physics-Guided Neural Networks} \label{cap:non_linealcensnet}

\section{Introduction to Physics-Informed Neural Networks (PINNs)}

Physics-Informed Neural Networks (PINNs) represent a class of neural networks where physical laws are incorporated into the learning process, guiding the model to respect these constraints. Unlike traditional neural networks, where the loss function is typically based on the discrepancy between predicted and actual data, PINNs introduce additional terms in the loss function that penalize the model for deviating from known physical principles.

In this case, the physical constraints are derived from the gas balance and the Weymouth equations, which describe the flow and pressure behavior within the gas transportation network. These constraints are integrated into our neural network as additional loss terms. Specifically, we define two layers within the network: one that calculates the error in gas balance and another that calculates the error in the Weymouth equation. The outputs of these layers are then used to adjust the network's predictions, ensuring that they adhere to the physical laws governing the system.

The inclusion of these physics-informed layers allows the network to achieve better generalization, as it is not only trained on the data but also guided by the underlying physical laws. This approach can be seen as a specialized form of regularization, where the model is penalized if its predictions do not satisfy the physical constraints. The overall loss function can be expressed as:


\begin{equation}
   \mathcal{L}(\Theta) = \mathcal{L}_{\text{data}}(\Theta) + \lambda_1 \mathcal{L}_{\text{balance}}(\Theta) + \lambda_2 \mathcal{L}_{\text{weymouth}}(\Theta),     
    \label{eq:PINN_basic_definition}
\end{equation}


% \[
% \mathcal{L}(\Theta) = \mathcal{L}_{\text{data}}(\Theta) + \lambda_1 \mathcal{L}_{\text{balance}}(\Theta) + \lambda_2 \mathcal{L}_{\text{weymouth}}(\Theta),
% \]

where \( \mathcal{L}_{\text{data}}(\Theta) \) represents the traditional data-driven loss, \( \mathcal{L}_{\text{balance}}(\Theta) \) is the loss associated with the gas balance constraint, and \( \mathcal{L}_{\text{weymouth}}(\Theta) \) is the loss associated with the Weymouth equation constraint. The parameters \( \lambda_1 \) and \( \lambda_2 \) control the importance of each physical constraint in the learning process.


In this section, we incorporate the physical laws of the gas balance and Weymouth equations to guide the model's training process. The gas balance equation, represented by \cref{eq:gas_balance}, ensures that the flow into and out of each node in the network adheres to the principle of mass conservation. The Weymouth equation, referred to as \cref{eq:weymouth_cons}, establishes a relationship between the flow and pressure differences across pipelines. These two equations will be the foundation for introducing physics-based constraints into the neural network, ensuring the model's predictions respect the physical behavior of gas flow within the system.


\section{Experimental Setup}

In this chapter, we build upon the experimental setup outlined in \cref{sec:LinealCensnet_ExperimentalSetup}, maintaining the same general approach while incorporating new elements that account for the physics of the natural gas system. The samples are generated using the nonlinear natural gas network optimization model from \cref{cap:optimization_mpcc}. In this process, a power-interconnected system was considered, but since this study focuses on the gas system, the power system remained constant without any variation. As in the previous setup, noise is introduced into the base values of two gas networks: a small-scale test network of 8 nodes and the more extensive Colombian natural gas transportation system. The noise levels, ranging from 5\% to 25\%, simulate various operating conditions, providing diverse training data.

While the GNN-based model from \cref{cap:lienal-censnet} was designed as a fast alternative to the optimization-based model, this chapter introduces physics-informed elements into the network architecture. Specifically, the model now includes loss terms based on the gas balance and Weymouth equations to ensure the predicted flows comply with the physical laws governing gas transportation. These constraints, integrated through additional layers in the model, guide the learning process, penalizing deviations from the gas balance equation (\cref{eq:gas_balance}) and the Weymouth equation (\cref{eq:weymouth_cons}). The modified model maintains the same structural components, such as input channels, convolutional layers, and loss functions for node and edge predictions, with the difference that the balance equation and the Weymouth equation are now considered loss functions. 


\begin{figure}
    \centering
    \setlength\figurewidth{1\textwidth}        
    \setlength\figureheight{0.5\textwidth}
    \resizebox{\figurewidth}{\figureheight}{\begin{tikzpicture}[shorten >=1pt, ->, draw=black!50, node distance=1.5cm and 3.5cm, align=center]

    % Styles
    \tikzstyle{input} = [rectangle, draw, fill=orange!30, minimum width=3cm, minimum height=1cm]
    \tikzstyle{dense} = [rectangle, draw, fill=blue!30, minimum width=3cm, minimum height=1cm]
    \tikzstyle{conv} = [rectangle, draw, fill=green!30, minimum width=3cm, minimum height=1cm]
    \tikzstyle{output} = [rectangle, draw, fill=purple!30, minimum width=3cm, minimum height=1cm]
    \tikzstyle{loss} = [rectangle, draw, fill=red!30, minimum width=3cm, minimum height=1cm]
    \tikzstyle{arrow} = [->, thick]

    % Input Layer
    \node[input] (node_features) at (0,0) {\(\mathbf{X}\)};
    \node[input] (node_laplacian) [below of=node_features] {\(\mathbf{L}_v\)};
    \node[input] (edge_laplacian) [below of=node_laplacian] {\(\mathbf{L}_e\)};
    \node[input] (incidence_matrix) [below of=edge_laplacian] {\(\mathbf{T}\)};
    \node[input] (edge_features) [below of=incidence_matrix] {\(\mathbf{E}\)};

    % Normalization and Pre-dense Layer
    \node[dense] (norm_pre_dense) [right of=edge_laplacian, xshift=3cm] {Normalization \\ \& Pre-dense Layers};

    % Convolutional Layers
    \node[conv] (conv_layers) [right of=norm_pre_dense, xshift=3cm] {CensNet Blocks \\ (Convolutional Layers)};

    % Post-dense Layer
    \node[dense] (post_dense) [right of=conv_layers, xshift=3cm] {Post-dense Layers};

    % Outputs
    \node[output] (node_output) [right of=post_dense, xshift=3cm, yshift=3cm] {\(\hat{\mathbf{X}}_v\)};
    \node[output] (edge_output) [right of=post_dense, xshift=3cm, yshift=1cm] {\(\hat{\mathbf{X}}_e\)};
    \node[output] (balance_output) [right of=post_dense, xshift=3cm, yshift=-1cm] {\(\mathcal{J}_{\text{balance}}\)};
    \node[output] (weymouth_output) [right of=post_dense, xshift=3cm, yshift=-3cm] {\(\mathcal{J}_{\text{Weymouth}}\)};

    % Losses
    \node[loss] (node_loss) [right of=node_output, xshift=3cm] {Node Loss};
    \node[loss] (edge_loss) [right of=edge_output, xshift=3cm] {Edge Loss};
    \node[loss] (balance_loss) [right of=balance_output, xshift=3cm] {Balance Loss};
    \node[loss] (weymouth_loss) [right of=weymouth_output, xshift=3cm] {Weymouth Loss};

    % Arrows
    \draw[arrow] (node_features) -- (norm_pre_dense);
    \draw[arrow] (node_laplacian) -- (norm_pre_dense);
    \draw[arrow] (edge_laplacian) -- (norm_pre_dense);
    \draw[arrow] (incidence_matrix) -- (norm_pre_dense);
    \draw[arrow] (edge_features) -- (norm_pre_dense);

    \draw[arrow] (norm_pre_dense) -- (conv_layers);
    \draw[arrow] (conv_layers) -- (post_dense);

    \draw[arrow] (post_dense) -- (node_output);
    \draw[arrow] (post_dense) -- (edge_output);
    \draw[arrow] (post_dense) -- (balance_output);
    \draw[arrow] (post_dense) -- (weymouth_output);

    \draw[arrow] (node_output) -- (node_loss);
    \draw[arrow] (edge_output) -- (edge_loss);
    \draw[arrow] (balance_output) -- (balance_loss);
    \draw[arrow] (weymouth_output) -- (weymouth_loss);

\end{tikzpicture}

}
    % \begin{tikzpicture}[shorten >=1pt, ->, draw=black!50, node distance=1.5cm and 3.5cm, align=center]

    % Styles
    \tikzstyle{input} = [circle, draw, fill=green!50, minimum size=2em]
    \tikzstyle{hidden} = [circle, draw, fill=blue!50, minimum size=2em]
    \tikzstyle{output} = [circle, draw, fill=red!50, minimum size=2em]
    \tikzstyle{connection} = [->, thick]

    % Network Stage Labels
    \node[align=center] at (0,-0.4) {Input \\ Layer};
    \node[align=center] at (6,0.4) {Hidden \\ Layers};
    \node[align=center] at (12,-1.2) {Output \\ Layer};

    % Input Layer
    \foreach \i in {1,2,3}
        \node[input] (I\i) at (0,-\i*1.5) {$x_\i$};

    % Hidden Layer 1
    \foreach \i in {1,2,3,4}
        \node[hidden] (H1\i) at (3,-\i*1.5+0.75) {$z^{(1)}_\i$};

    % Hidden Layer 2
    \foreach \i in {1,2,3,4}
        \node[hidden] (H2\i) at (6,-\i*1.5+0.75) {$z^{(2)}_\i$};

    % Hidden Layer 3
    \foreach \i in {1,2,3,4}
        \node[hidden] (H3\i) at (9,-\i*1.5+0.75) {$z^{(3)}_\i$};

    % Output Layer
    \foreach \i in {1,2}
        \node[output] (O\i) at (12,-\i*1.5-0.75) {$\hat{y}_\i$};

    % Connections from Input to Hidden Layer 1
    \foreach \i in {1,2,3}
        \foreach \j in {1,2,3,4}
            \draw[connection] (I\i) -- (H1\j);

    % Connections from Hidden Layer 1 to Hidden Layer 2
    \foreach \i in {1,2,3,4}
        \foreach \j in {1,2,3,4}
            \draw[connection] (H1\i) -- (H2\j);

    % Connections from Hidden Layer 2 to Hidden Layer 3
    \foreach \i in {1,2,3,4}
        \foreach \j in {1,2,3,4}
            \draw[connection] (H2\i) -- (H3\j);

    % Connections from Hidden Layer 3 to Output Layer
    \foreach \i in {1,2,3,4}
        \foreach \j in {1,2}
            \draw[connection] (H3\i) -- (O\j);

\end{tikzpicture}

    \caption{General outline of the CensNet-based model used.}
        \label{fig:nonlineal_model_description}
\end{figure}

\section{Results}


In this section, we present the results of the proposed model, which now incorporates physical constraints from the natural gas system. The focus remains on the relationship between the predicted outputs and the actual observed values, evaluating the model's performance across the 8-node test network and the Colombian natural gas transportation system. By incorporating physics-based constraints, the goal is to assess the model's ability to predict critical parameters under various operational conditions while ensuring that the physical laws governing gas flow are respected.

\subsection{Case Study I: 8-node Network}



In this chapter, we begin with experiments that account for both node and edge losses, as it was found that considering only the node loss did not produce adequate results. The best parameters identified for this experiment were $N channels=25$, $N layers =4$, and $N dense = 11$. These settings yielded a total loss of 6.816, with a node loss of 2.794 and an edge loss of 4.021.

The results corresponding to the nodes, shown in \cref{fig:results_nonlineal_dummy_base_node}, exhibit a similar behavior to that observed in 
\cref{fig:results_dummy_node_base_f}, demonstrating that the model accurately captures the injection pattern at the nodes. The correlation between the actual and predicted values is also strong, as indicated by an $R^2$ of 0.983.

Edge flows show some variation, as seen in \cref{fig:results_nonlineal_dummy_base_f}, mainly when predicting the flows through the first pipeline connected to the injection field, where slight deviations from the actual flow values were observed. However, the model performed well overall, achieving an $R^2$ of 0.983 for the edge flows. While the first pipeline presents some prediction challenges, the accuracy in predicting flows across the rest of the pipelines remains high, demonstrating the model's ability to handle the complexity of gas transportation in this nonlinear system.


\begin{figure}
    \centering
    \setlength\figurewidth{.53\textwidth}        
    \setlength\figureheight{0.36\textwidth} 
    \subfloat[Actual vs predicted nodal flows.] 
    {\label{fig:results_nonlineal_dummy_base_node}\resizebox{\figurewidth}{\figureheight}{% This file was created with tikzplotlib v0.10.1.
\begin{tikzpicture}

\definecolor{darkgray176}{RGB}{176,176,176}
\definecolor{lightgray204}{RGB}{204,204,204}

\begin{axis}[
colorbar,
colorbar style={ylabel={node_id}},
colormap={mymap}{[1pt]
 rgb(0pt)=(0.12156862745098,0.466666666666667,0.705882352941177);
  rgb(1pt)=(1,0.498039215686275,0.0549019607843137);
  rgb(2pt)=(0.172549019607843,0.627450980392157,0.172549019607843);
  rgb(3pt)=(0.83921568627451,0.152941176470588,0.156862745098039);
  rgb(4pt)=(0.580392156862745,0.403921568627451,0.741176470588235);
  rgb(5pt)=(0.549019607843137,0.337254901960784,0.294117647058824);
  rgb(6pt)=(0.890196078431372,0.466666666666667,0.76078431372549);
  rgb(7pt)=(0.498039215686275,0.498039215686275,0.498039215686275);
  rgb(8pt)=(0.737254901960784,0.741176470588235,0.133333333333333);
  rgb(9pt)=(0.0901960784313725,0.745098039215686,0.811764705882353)
},
legend cell align={left},
legend style={
  fill opacity=0.8,
  draw opacity=1,
  text opacity=1,
  at={(0.03,0.97)},
  anchor=north west,
  draw=lightgray204
},
point meta max=7,
point meta min=0,
tick align=outside,
tick pos=left,
title={yn_test-y_pred},
x grid style={darkgray176},
xlabel={yn_test},
xmajorgrids,
xmin=-2.43954548935, xmax=51.23045527635,
xtick style={color=black},
y grid style={darkgray176},
ylabel={y_pred},
ymajorgrids,
ymin=-2.19038361012936, ymax=45.0462195843458,
ytick style={color=black}
]
\addplot [
  colormap={mymap}{[1pt]
 rgb(0pt)=(0.12156862745098,0.466666666666667,0.705882352941177);
  rgb(1pt)=(1,0.498039215686275,0.0549019607843137);
  rgb(2pt)=(0.172549019607843,0.627450980392157,0.172549019607843);
  rgb(3pt)=(0.83921568627451,0.152941176470588,0.156862745098039);
  rgb(4pt)=(0.580392156862745,0.403921568627451,0.741176470588235);
  rgb(5pt)=(0.549019607843137,0.337254901960784,0.294117647058824);
  rgb(6pt)=(0.890196078431372,0.466666666666667,0.76078431372549);
  rgb(7pt)=(0.498039215686275,0.498039215686275,0.498039215686275);
  rgb(8pt)=(0.737254901960784,0.741176470588235,0.133333333333333);
  rgb(9pt)=(0.0901960784313725,0.745098039215686,0.811764705882353)
},
  only marks,
  scatter,
  scatter src=explicit
]
table [x=x, y=y, meta=colordata]{%
x  y  colordata
39.565898645 38.7153625488281 0.0
0 0.0123127698898315 1.0
0 0.071319580078125 2.0
0 -0.0128939747810364 3.0
0 0.0256100296974182 4.0
0 -0.0238697528839111 5.0
0 0.00411289930343628 6.0
0 -0.0127748847007751 7.0
42.743257181 39.4932670593262 0.0
0 0.00441712141036987 1.0
0 0.0637285709381104 2.0
0 -0.0179623365402222 3.0
0 0.0215394496917725 4.0
0 -0.0140223503112793 5.0
0 1.97887420654297e-05 6.0
0 -0.0135558843612671 7.0
39.367180751 38.7526664733887 0.0
0 0.0138788819313049 1.0
0 0.0361418128013611 2.0
0 -0.0201342701911926 3.0
0 0.00914174318313599 4.0
0 -0.0191978812217712 5.0
0 0.000798225402832031 6.0
0 -0.00776267051696777 7.0
39.605393107 39.5196304321289 0.0
0 0.00788706541061401 1.0
0 0.0466488599777222 2.0
0 -0.0212488174438477 3.0
0 0.0412870049476624 4.0
0 -0.0301215648651123 5.0
0 0.00819826126098633 6.0
0 -0.0149205923080444 7.0
43.937345536 38.9456558227539 0.0
0 0.0131928324699402 1.0
0 0.0427597761154175 2.0
0 -0.0308229327201843 3.0
0 0.0441254377365112 4.0
0 -0.0282912254333496 5.0
0 0.00409531593322754 6.0
0 -0.0128695964813232 7.0
31.061989594 39.310115814209 0.0
0 0.0139701366424561 1.0
0 0.0382223725318909 2.0
0 -0.0262283682823181 3.0
0 0.0097421407699585 4.0
0 -0.0335473418235779 5.0
0 0.00768345594406128 6.0
0 -0.00622838735580444 7.0
36.357435273 42.0804176330566 0.0
0 0.00529545545578003 1.0
0 0.0581262111663818 2.0
0 -0.0150731205940247 3.0
0 0.0298828482627869 4.0
0 -0.028703510761261 5.0
0 0.00751966238021851 6.0
0 -0.0148077011108398 7.0
38.444969617 39.1996421813965 0.0
0 0.00592869520187378 1.0
0 0.0615658760070801 2.0
0 -0.0213605761528015 3.0
0 0.0231988430023193 4.0
0 -0.026212215423584 5.0
0 0.00326085090637207 6.0
0 -0.0112636089324951 7.0
35.498620524 39.3897895812988 0.0
0 0.00891655683517456 1.0
0 0.0605899095535278 2.0
0 -0.018476665019989 3.0
0 0.00775080919265747 4.0
0 -0.024048924446106 5.0
0 0.0055384635925293 6.0
0 -0.0057908296585083 7.0
36.520998279 39.447883605957 0.0
0 0.00435954332351685 1.0
0 0.0819315910339355 2.0
0 -0.0160262584686279 3.0
0 0.0174445509910583 4.0
0 -0.0244802236557007 5.0
0 0.00764340162277222 6.0
0 -0.00894832611083984 7.0
36.717272212 39.199634552002 0.0
0 0.00995087623596191 1.0
0 0.066163182258606 2.0
0 -0.0166041851043701 3.0
0 0.0310661196708679 4.0
0 -0.0306993126869202 5.0
0 0.0094258189201355 6.0
0 -0.0108351707458496 7.0
32.629629006 39.185905456543 0.0
0 0.0115001201629639 1.0
0 0.0270580649375916 2.0
0 -0.0159772038459778 3.0
0 0.00784498453140259 4.0
0 -0.0299507975578308 5.0
0 0.00833970308303833 6.0
0 -0.00938189029693604 7.0
37.75267434 39.1989135742188 0.0
0 0.0082511305809021 1.0
0 0.06497722864151 2.0
0 -0.0128266215324402 3.0
0 0.00258684158325195 4.0
0 -0.018075704574585 5.0
0 -0.000989556312561035 6.0
0 -0.0049208402633667 7.0
38.800291347 40.1666564941406 0.0
0 0.0118288397789001 1.0
0 0.0622610449790955 2.0
0 -0.0261341333389282 3.0
0 0.0260022878646851 4.0
0 -0.0306769013404846 5.0
0 0.0043455958366394 6.0
0 -0.0105564594268799 7.0
38.252729618 39.3290748596191 0.0
0 0.00866585969924927 1.0
0 0.0685842037200928 2.0
0 -0.0130652189254761 3.0
0 0.0246871113777161 4.0
0 -0.0263183116912842 5.0
0 0.00541287660598755 6.0
0 -0.0122238397598267 7.0
43.273596047 39.3154678344727 0.0
0 0.00989145040512085 1.0
0 0.073027491569519 2.0
0 -0.0273558497428894 3.0
0 0.0494243502616882 4.0
0 -0.0284897089004517 5.0
0 0.00563955307006836 6.0
0 -0.0137518644332886 7.0
34.027431486 38.857494354248 0.0
0 0.00671476125717163 1.0
0 0.0619230270385742 2.0
0 -0.0194013118743896 3.0
0 0.0136286020278931 4.0
0 -0.029705822467804 5.0
0 0.003440260887146 6.0
0 -0.00797498226165771 7.0
41.154171391 39.8500900268555 0.0
0 0.00906610488891602 1.0
0 0.0560967326164246 2.0
0 -0.0276532769203186 3.0
0 0.0497760772705078 4.0
0 -0.0301504731178284 5.0
0 0.00726139545440674 6.0
0 -0.0125929117202759 7.0
39.930826408 39.8617324829102 0.0
0 0.00576150417327881 1.0
0 0.0628125667572021 2.0
0 -0.0210210680961609 3.0
0 0.0447211265563965 4.0
0 -0.0306063890457153 5.0
0 0.00760847330093384 6.0
0 -0.0124554634094238 7.0
45.969703631 39.1456565856934 0.0
0 0.0107769966125488 1.0
0 0.0357969403266907 2.0
0 -0.0225849151611328 3.0
0 0.043922483921051 4.0
0 -0.0197867751121521 5.0
0 0.00242942571640015 6.0
0 -0.0128521323204041 7.0
38.398120221 39.2407455444336 0.0
0 0.00990653038024902 1.0
0 0.0493940114974976 2.0
0 -0.0257954001426697 3.0
0 0.035546600818634 4.0
0 -0.0293555855751038 5.0
0 0.00442242622375488 6.0
0 -0.0125865340232849 7.0
28.366017306 39.314582824707 0.0
0 1.03116035461426e-05 1.0
0 0.0335801839828491 2.0
0 -0.00797528028488159 3.0
0 -0.0116368532180786 4.0
0 -0.0308080911636353 5.0
0 0.00525516271591187 6.0
0 -0.00420612096786499 7.0
39.079818979 39.1686325073242 0.0
0 0.011313796043396 1.0
0 0.0414641499519348 2.0
0 -0.0202111601829529 3.0
0 0.0133368968963623 4.0
0 -0.0244540572166443 5.0
0 0.00453060865402222 6.0
0 -0.0085756778717041 7.0
40.466329383 39.3136444091797 0.0
0 -0.00223791599273682 1.0
0 0.0511056780815125 2.0
0 -0.0211246013641357 3.0
0 0.0184855461120605 4.0
0 -0.0249439477920532 5.0
0 0.00523895025253296 6.0
0 -0.00943392515182495 7.0
38.293116853 38.996208190918 0.0
0 0.00625330209732056 1.0
0 0.069944441318512 2.0
0 -0.0136637687683105 3.0
0 0.000947535037994385 4.0
0 -0.0120516419410706 5.0
0 0.000261187553405762 6.0
0 -0.00625407695770264 7.0
43.346089497 38.9964256286621 0.0
0 0.0137380957603455 1.0
0 0.0395742058753967 2.0
0 -0.0222710371017456 3.0
0 0.0477991700172424 4.0
0 -0.030639111995697 5.0
0 0.00966173410415649 6.0
0 -0.0138188004493713 7.0
34.547871154 39.3231048583984 0.0
0 0.00445806980133057 1.0
0 0.0469914674758911 2.0
0 -0.0166859030723572 3.0
0 0.0288587212562561 4.0
0 -0.0295593738555908 5.0
0 0.00691384077072144 6.0
0 -0.0114007592201233 7.0
41.927050143 39.338737487793 0.0
0 0.00693070888519287 1.0
0 0.0592769980430603 2.0
0 -0.0221498608589172 3.0
0 0.0644987821578979 4.0
0 -0.029447615146637 5.0
0 0.0115132927894592 6.0
0 -0.0177930593490601 7.0
44.548236377 39.3082656860352 0.0
0 0.0133536458015442 1.0
0 0.0384448766708374 2.0
0 -0.0213861465454102 3.0
0 0.0355319976806641 4.0
0 -0.0190299153327942 5.0
0 0.00214254856109619 6.0
0 -0.0123872756958008 7.0
34.958415487 39.313117980957 0.0
0 0.00151073932647705 1.0
0 0.0384351015090942 2.0
0 -0.0213229656219482 3.0
0 0.00255352258682251 4.0
0 -0.025343120098114 5.0
0 0.00335896015167236 6.0
0 -0.00606870651245117 7.0
41.619773298 39.9483871459961 0.0
0 -0.00310182571411133 1.0
0 0.0696684718132019 2.0
0 -0.0155016183853149 3.0
0 0.0176678895950317 4.0
0 -0.0168115496635437 5.0
0 0.000762403011322021 6.0
0 -0.0129686594009399 7.0
35.768623912 39.1677360534668 0.0
0 -0.00235003232955933 1.0
0 0.0565180778503418 2.0
0 -0.0165175199508667 3.0
0 0.0183342695236206 4.0
0 -0.0309199094772339 5.0
0 0.00655144453048706 6.0
0 -0.0103568434715271 7.0
36.03587308 37.5861854553223 0.0
0 -0.00157850980758667 1.0
0 0.0492555499076843 2.0
0 -0.0198881030082703 3.0
0 0.0342521667480469 4.0
0 -0.0291762351989746 5.0
0 0.0110065340995789 6.0
0 -0.011877179145813 7.0
42.671159584 39.3272399902344 0.0
0 0.0113665461540222 1.0
0 0.0392718911170959 2.0
0 -0.0274774432182312 3.0
0 0.052215039730072 4.0
0 -0.0293958783149719 5.0
0 0.00858646631240845 6.0
0 -0.0139873623847961 7.0
32.270518391 39.3328552246094 0.0
0 0.0110946297645569 1.0
0 0.05193692445755 2.0
0 -0.0178480744361877 3.0
0 0.00724279880523682 4.0
0 -0.0301480889320374 5.0
0 0.00648993253707886 6.0
0 -0.00789618492126465 7.0
36.239834941 41.1309661865234 0.0
0 0.00345695018768311 1.0
0 0.0747806429862976 2.0
0 -0.0145701169967651 3.0
0 0.0049513578414917 4.0
0 -0.0192160606384277 5.0
0 0.00256305932998657 6.0
0 -0.00866025686264038 7.0
38.292005181 39.2797698974609 0.0
0 0.00284379720687866 1.0
0 0.0393624305725098 2.0
0 -0.0206927061080933 3.0
0 0.0245354771614075 4.0
0 -0.028574526309967 5.0
0 0.00577658414840698 6.0
0 -0.0118468999862671 7.0
41.142171126 39.369083404541 0.0
0 0.000563442707061768 1.0
0 0.0569069981575012 2.0
0 -0.0239884257316589 3.0
0 0.0219588279724121 4.0
0 -0.0221072435379028 5.0
0 0.0049242377281189 6.0
0 -0.0127769112586975 7.0
30.660234751 38.9642333984375 0.0
0 -0.00240546464920044 1.0
0 0.0487861633300781 2.0
0 -0.00998640060424805 3.0
0 0.00200992822647095 4.0
0 -0.0323919057846069 5.0
0 0.00852972269058228 6.0
0 -0.00369340181350708 7.0
42.776716986 41.5339431762695 0.0
0 0.00600254535675049 1.0
0 0.0594097375869751 2.0
0 -0.0193828344345093 3.0
0 0.0313163995742798 4.0
0 -0.0243908762931824 5.0
0 0.0065605640411377 6.0
0 -0.016025185585022 7.0
39.136657955 39.7308387756348 0.0
0 0.00334018468856812 1.0
0 0.099915623664856 2.0
0 -0.0114416480064392 3.0
0 0.0245475769042969 4.0
0 -0.0192947387695312 5.0
0 0.00287395715713501 6.0
0 -0.0121320486068726 7.0
40.593075664 39.7711791992188 0.0
0 0.00522077083587646 1.0
0 0.0596650242805481 2.0
0 -0.0162090659141541 3.0
0 0.045604944229126 4.0
0 -0.0273600816726685 5.0
0 0.00408452749252319 6.0
0 -0.0147544145584106 7.0
38.455960057 40.2578964233398 0.0
0 0.00629007816314697 1.0
0 0.0590381622314453 2.0
0 -0.0165366530418396 3.0
0 0.043002724647522 4.0
0 -0.0271528959274292 5.0
0 0.00735443830490112 6.0
0 -0.0150219798088074 7.0
40.029822307 39.2045021057129 0.0
0 0.0109648704528809 1.0
0 0.0333818793296814 2.0
0 -0.0219238996505737 3.0
0 0.0246446132659912 4.0
0 -0.0267536044120789 5.0
0 0.00693017244338989 6.0
0 -0.0108383893966675 7.0
39.721089806 39.6830825805664 0.0
0 0.00558066368103027 1.0
0 0.0897490382194519 2.0
0 -0.0145083665847778 3.0
0 0.0387337803840637 4.0
0 -0.0278629660606384 5.0
0 0.00985139608383179 6.0
0 -0.0144490599632263 7.0
46.012802781 39.0627212524414 0.0
0 0.00166887044906616 1.0
0 0.0386090874671936 2.0
0 -0.0324397087097168 3.0
0 0.0550388693809509 4.0
0 -0.028700590133667 5.0
0 0.00663286447525024 6.0
0 -0.0132797360420227 7.0
43.791041158 39.1844825744629 0.0
0 0.0124698877334595 1.0
0 0.0309100747108459 2.0
0 -0.0285602211952209 3.0
0 0.046930730342865 4.0
0 -0.0275624394416809 5.0
0 0.00615853071212769 6.0
0 -0.0148841142654419 7.0
31.257332424 39.1901626586914 0.0
0 0.0106835961341858 1.0
0 0.0625916123390198 2.0
0 -0.0122222304344177 3.0
0 0.000936150550842285 4.0
0 -0.0310866236686707 5.0
0 0.00312197208404541 6.0
0 -0.00706321001052856 7.0
38.98847291 39.2609672546387 0.0
0 0.0101978778839111 1.0
0 0.0407465100288391 2.0
0 -0.0224083065986633 3.0
0 0.0104560256004333 4.0
0 -0.0216193795204163 5.0
0 0.00195193290710449 6.0
0 -0.00701618194580078 7.0
38.691218499 38.9683609008789 0.0
0 0.00895541906356812 1.0
0 0.0564865469932556 2.0
0 -0.0135195851325989 3.0
0 0.0024409294128418 4.0
0 -0.0234909653663635 5.0
0 0.0044097900390625 6.0
0 -0.00805675983428955 7.0
39.033211971 38.7177581787109 0.0
0 0.00353097915649414 1.0
0 0.0651161074638367 2.0
0 -0.0148383378982544 3.0
0 0.0438032746315002 4.0
0 -0.0298718214035034 5.0
0 0.00918209552764893 6.0
0 -0.0137563943862915 7.0
37.697547813 39.1868095397949 0.0
0 0.00985407829284668 1.0
0 0.0413744449615479 2.0
0 -0.0205171704292297 3.0
0 0.00754988193511963 4.0
0 -0.0230441689491272 5.0
0 0.00145184993743896 6.0
0 -0.0105559825897217 7.0
35.277541339 39.2073364257812 0.0
0 0.0143079161643982 1.0
0 0.0378283262252808 2.0
0 -0.0201423168182373 3.0
0 0.00532984733581543 4.0
0 -0.0268043279647827 5.0
0 0.00716966390609741 6.0
0 -0.0073121190071106 7.0
36.119763966 39.4090423583984 0.0
0 0.0113481879234314 1.0
0 0.0379583835601807 2.0
0 -0.0301769375801086 3.0
0 0.0023917555809021 4.0
0 -0.0140483379364014 5.0
0 -0.00150942802429199 6.0
0 -0.00346070528030396 7.0
33.14490305 39.51171875 0.0
0 0.00569027662277222 1.0
0 0.0744785666465759 2.0
0 -0.0195279121398926 3.0
0 0.00108557939529419 4.0
0 -0.0206176042556763 5.0
0 0.00269472599029541 6.0
0 -0.00410282611846924 7.0
34.486800814 40.0056648254395 0.0
0 0.0040428638458252 1.0
0 0.0613293647766113 2.0
0 -0.0154016017913818 3.0
0 0.0307614207267761 4.0
0 -0.0287876129150391 5.0
0 0.0124437212944031 6.0
0 -0.0129595398902893 7.0
40.933468207 40.1773338317871 0.0
0 0.0079658031463623 1.0
0 0.0732653141021729 2.0
0 -0.0176567435264587 3.0
0 0.0293158888816833 4.0
0 -0.0187020897865295 5.0
0 0.00174516439437866 6.0
0 -0.0126280188560486 7.0
35.993417928 39.3662071228027 0.0
0 0.00715410709381104 1.0
0 0.0595747232437134 2.0
0 -0.0184565782546997 3.0
0 0.0182121396064758 4.0
0 -0.0294560790061951 5.0
0 0.00316703319549561 6.0
0 -0.0110476613044739 7.0
39.331666923 39.7296485900879 0.0
0 0.00837588310241699 1.0
0 0.0545561909675598 2.0
0 -0.0260232090950012 3.0
0 0.0253786444664001 4.0
0 -0.0261372923851013 5.0
0 0.0041089653968811 6.0
0 -0.0101762413978577 7.0
37.559548496 39.3728485107422 0.0
0 0.00726288557052612 1.0
0 0.0635012984275818 2.0
0 -0.0159628987312317 3.0
0 0.0196371674537659 4.0
0 -0.0268263816833496 5.0
0 0.00435435771942139 6.0
0 -0.0115820169448853 7.0
41.796902482 39.2759208679199 0.0
0 0.0100352764129639 1.0
0 0.0439282655715942 2.0
0 -0.0254364013671875 3.0
0 0.0484634637832642 4.0
0 -0.0301735997200012 5.0
0 0.00479131937026978 6.0
0 -0.0128893852233887 7.0
35.679590823 39.2148361206055 0.0
0 0.0133700370788574 1.0
0 0.0399370193481445 2.0
0 -0.0224254131317139 3.0
0 0.0077054500579834 4.0
0 -0.0263437032699585 5.0
0 0.00416111946105957 6.0
0 -0.00710749626159668 7.0
33.227547292 38.8616104125977 0.0
0 0.0111078023910522 1.0
0 0.061786949634552 2.0
0 -0.0128821134567261 3.0
0 0.00850886106491089 4.0
0 -0.0278047919273376 5.0
0 0.00242877006530762 6.0
0 -0.00745308399200439 7.0
28.008071739 39.3282775878906 0.0
0 0.00727421045303345 1.0
0 0.0578927397727966 2.0
0 -0.0101944804191589 3.0
0 -0.00735962390899658 4.0
0 -0.0292975306510925 5.0
0 0.0059630274772644 6.0
0 0.00163012742996216 7.0
33.478498841 39.2475204467773 0.0
0 0.0053410530090332 1.0
0 0.0993016362190247 2.0
0 -0.00882583856582642 3.0
0 0.0095442533493042 4.0
0 -0.0248811841011047 5.0
0 0.00372958183288574 6.0
0 -0.00759077072143555 7.0
33.12294682 39.1935424804688 0.0
0 0.0134350061416626 1.0
0 0.0337791442871094 2.0
0 -0.0293073058128357 3.0
0 0.0119555592536926 4.0
0 -0.03260737657547 5.0
0 0.00758576393127441 6.0
0 -0.00859177112579346 7.0
34.970228384 39.3222885131836 0.0
0 0.0112384557723999 1.0
0 0.0673366785049438 2.0
0 -0.0168597102165222 3.0
0 0.00697022676467896 4.0
0 -0.0265811085700989 5.0
0 0.00594782829284668 6.0
0 0.0834335684776306 7.0
36.637966041 39.6626014709473 0.0
0 0.0114080905914307 1.0
0 0.039689302444458 2.0
0 -0.0293136239051819 3.0
0 0.0113623738288879 4.0
0 -0.028203547000885 5.0
0 0.00562357902526855 6.0
0 -0.00684285163879395 7.0
38.712447295 39.2186279296875 0.0
0 0.00919914245605469 1.0
0 0.0359517931938171 2.0
0 -0.0268067121505737 3.0
0 0.0361806750297546 4.0
0 -0.0292361378669739 5.0
0 0.00580942630767822 6.0
0 -0.0120308995246887 7.0
29.08402242 39.358585357666 0.0
0 0.00405758619308472 1.0
0 0.0620464682579041 2.0
0 -0.00910604000091553 3.0
0 -0.00566118955612183 4.0
0 -0.0290875434875488 5.0
0 0.00550538301467896 6.0
0 -0.00326275825500488 7.0
40.741903011 39.7179641723633 0.0
0 0.0086064338684082 1.0
0 0.0604866147041321 2.0
0 -0.0193439722061157 3.0
0 0.0264756679534912 4.0
0 -0.0243978500366211 5.0
0 0.0040895938873291 6.0
0 -0.0132392048835754 7.0
44.423927978 39.2226028442383 0.0
0 0.0112552642822266 1.0
0 0.0406379103660583 2.0
0 -0.0212950110435486 3.0
0 0.0372845530509949 4.0
0 -0.0296989679336548 5.0
0 0.00628405809402466 6.0
0 -0.0137932300567627 7.0
37.279929011 42.0506362915039 0.0
0 0.000679612159729004 1.0
0 0.0922307968139648 2.0
0 -0.0117670297622681 3.0
0 0.00707334280014038 4.0
0 -0.0226204395294189 5.0
0 0.001944899559021 6.0
0 -0.0140491127967834 7.0
39.065196566 39.0746231079102 0.0
0 0.00692451000213623 1.0
0 0.0894067883491516 2.0
0 -0.0125413537025452 3.0
0 0.0433874726295471 4.0
0 -0.0262654423713684 5.0
0 0.00937050580978394 6.0
0 -0.0140787959098816 7.0
34.346712911 39.6062927246094 0.0
0 0.00730884075164795 1.0
0 0.0575341582298279 2.0
0 -0.0125131011009216 3.0
0 0.00180590152740479 4.0
0 -0.0254272222518921 5.0
0 0.00626826286315918 6.0
0 -0.00440073013305664 7.0
44.832836202 39.1740646362305 0.0
0 0.00540620088577271 1.0
0 0.048359215259552 2.0
0 -0.0217028856277466 3.0
0 0.0456007122993469 4.0
0 -0.0261398553848267 5.0
0 0.00735974311828613 6.0
0 -0.0143005847930908 7.0
37.693005916 39.4430923461914 0.0
0 0.00866585969924927 1.0
0 0.0588068962097168 2.0
0 -0.0298804044723511 3.0
0 0.012173056602478 4.0
0 -0.0230597257614136 5.0
0 0.00236654281616211 6.0
0 -0.00678610801696777 7.0
35.516030206 39.2848510742188 0.0
0 0.00840789079666138 1.0
0 0.0401602387428284 2.0
0 -0.0224798321723938 3.0
0 0.00727975368499756 4.0
0 -0.0262488722801208 5.0
0 0.00442099571228027 6.0
0 -0.00669002532958984 7.0
45.449957523 39.3815727233887 0.0
0 0.00223982334136963 1.0
0 0.0652434825897217 2.0
0 -0.0192306637763977 3.0
0 0.0608529448509216 4.0
0 -0.027153491973877 5.0
0 0.00786775350570679 6.0
0 -0.0163425207138062 7.0
36.52210397 40.1063613891602 0.0
0 0.0123141407966614 1.0
0 0.0607444047927856 2.0
0 -0.0281732082366943 3.0
0 0.00769400596618652 4.0
0 -0.0218368172645569 5.0
0 0.00279814004898071 6.0
0 0.214010000228882 7.0
40.809333833 39.2384757995605 0.0
0 0.0117188096046448 1.0
0 0.039944052696228 2.0
0 -0.0229901075363159 3.0
0 0.0258876085281372 4.0
0 -0.0261783003807068 5.0
0 0.0060725212097168 6.0
0 -0.00992476940155029 7.0
43.708815025 39.8203887939453 0.0
0 0.00554096698760986 1.0
0 0.0680207014083862 2.0
0 -0.0179526805877686 3.0
0 0.0337607264518738 4.0
0 -0.0186030864715576 5.0
0 0.00325441360473633 6.0
0 -0.0130414366722107 7.0
35.746307575 39.0528678894043 0.0
0 0.00616163015365601 1.0
0 0.031990647315979 2.0
0 -0.0340034961700439 3.0
0 0.0250157713890076 4.0
0 -0.0317110419273376 5.0
0 0.00777506828308105 6.0
0 -0.00998783111572266 7.0
32.565260407 39.287166595459 0.0
0 0.0112178325653076 1.0
0 0.0362849831581116 2.0
0 -0.0211132764816284 3.0
0 0.000355720520019531 4.0
0 -0.0266507863998413 5.0
0 0.00528323650360107 6.0
0 -0.00296187400817871 7.0
37.459437569 39.7067375183105 0.0
0 0.00287795066833496 1.0
0 0.0870211720466614 2.0
0 -0.0136895775794983 3.0
0 0.0324925780296326 4.0
0 -0.0261927247047424 5.0
0 0.00662326812744141 6.0
0 -0.0170584321022034 7.0
41.600868777 39.6331176757812 0.0
0 0.00617289543151855 1.0
0 0.0472753047943115 2.0
0 -0.0370604395866394 3.0
0 0.0207778215408325 4.0
0 -0.0186801552772522 5.0
0 0.000774919986724854 6.0
0 -0.00782781839370728 7.0
43.088648289 42.4566268920898 0.0
0 0.00171780586242676 1.0
0 0.0445231199264526 2.0
0 -0.0263301730155945 3.0
0 0.0577532052993774 4.0
0 -0.0282729268074036 5.0
0 0.00841760635375977 6.0
0 -0.0169897675514221 7.0
30.268152677 39.3095664978027 0.0
0 0.0115619301795959 1.0
0 0.0627694725990295 2.0
0 -0.0172976851463318 3.0
0 0.000941872596740723 4.0
0 -0.0301180481910706 5.0
0 0.00371986627578735 6.0
0 -0.00358611345291138 7.0
38.454045888 38.8899307250977 0.0
0 0.0116772055625916 1.0
0 0.0517902970314026 2.0
0 -0.0236786603927612 3.0
0 0.0169258117675781 4.0
0 -0.0281082987785339 5.0
0 0.0070832371711731 6.0
0 -0.00702857971191406 7.0
36.654056864 39.1972312927246 0.0
0 0.00971025228500366 1.0
0 0.0378752946853638 2.0
0 -0.021247386932373 3.0
0 0.0270963311195374 4.0
0 -0.029135525226593 5.0
0 0.00854533910751343 6.0
0 -0.0112012028694153 7.0
36.990278128 39.4844970703125 0.0
0 0.010595440864563 1.0
0 0.0305523872375488 2.0
0 -0.0160865783691406 3.0
0 0.0211124420166016 4.0
0 -0.0285171270370483 5.0
0 0.00439399480819702 6.0
0 -0.0110663771629333 7.0
39.45539179 39.3758239746094 0.0
0 0.00812649726867676 1.0
0 0.0442739725112915 2.0
0 -0.0175419449806213 3.0
0 0.0141112208366394 4.0
0 -0.027271032333374 5.0
0 0.00580936670303345 6.0
0 -0.00898617506027222 7.0
41.49009831 39.2659378051758 0.0
0 0.00471115112304688 1.0
0 0.0437565445899963 2.0
0 -0.0283921957015991 3.0
0 0.0177510380744934 4.0
0 -0.018986701965332 5.0
0 0.00044095516204834 6.0
0 -0.00875365734100342 7.0
42.195848764 39.7440223693848 0.0
0 0.00705784559249878 1.0
0 0.0813563466072083 2.0
0 -0.0185672044754028 3.0
0 0.0338060855865479 4.0
0 -0.0192969441413879 5.0
0 0.00195407867431641 6.0
0 -0.0113794207572937 7.0
36.580423956 39.3618698120117 0.0
0 0.0103769898414612 1.0
0 0.0483078956604004 2.0
0 -0.0236889123916626 3.0
0 0.0567827224731445 4.0
0 -0.028109610080719 5.0
0 0.0141017436981201 6.0
0 -0.0142652988433838 7.0
41.112612846 38.2719535827637 0.0
0 0.00170391798019409 1.0
0 0.105811357498169 2.0
0 -0.0118167400360107 3.0
0 0.036638617515564 4.0
0 -0.0218501687049866 5.0
0 0.00543737411499023 6.0
0 -0.0120397210121155 7.0
41.165032305 39.1872482299805 0.0
0 0.0157047510147095 1.0
0 0.0349633693695068 2.0
0 -0.0216887593269348 3.0
0 0.0505847930908203 4.0
0 -0.0271188616752625 5.0
0 0.00774049758911133 6.0
0 -0.0159221291542053 7.0
28.236329769 39.7965545654297 0.0
0 0.013361930847168 1.0
0 0.0601737499237061 2.0
0 -0.00644588470458984 3.0
0 -0.00728082656860352 4.0
0 -0.0303657054901123 5.0
0 0.00617170333862305 6.0
0 -0.00426137447357178 7.0
46.563722235 39.247730255127 0.0
0 0.0073997974395752 1.0
0 0.0409015417098999 2.0
0 -0.0347752571105957 3.0
0 0.0438136458396912 4.0
0 -0.0267916321754456 5.0
0 0.00308680534362793 6.0
0 -0.01231449842453 7.0
38.976039446 38.2965316772461 0.0
0 0.000528395175933838 1.0
0 0.025409460067749 2.0
0 -0.0173540711402893 3.0
0 0.00302207469940186 4.0
0 -0.0253949165344238 5.0
0 0.00395089387893677 6.0
0 -0.00843304395675659 7.0
43.902220165 39.8411254882812 0.0
0 0.0118625164031982 1.0
0 0.0401428937911987 2.0
0 -0.0226224660873413 3.0
0 0.0575634241104126 4.0
0 -0.0248637795448303 5.0
0 0.00626331567764282 6.0
0 -0.016412615776062 7.0
37.104870385 38.8935470581055 0.0
0 0.00900518894195557 1.0
0 0.0604310631752014 2.0
0 -0.0258265733718872 3.0
0 0.0378197431564331 4.0
0 -0.0289685726165771 5.0
0 0.00780671834945679 6.0
0 -0.0103233456611633 7.0
48.094555135 39.2413864135742 0.0
0 0.0141285061836243 1.0
0 0.0317863821983337 2.0
0 -0.0394392609596252 3.0
0 0.0513074994087219 4.0
0 -0.0285180807113647 5.0
0 0.0042346715927124 6.0
0 -0.0118893384933472 7.0
39.863550951 38.9883651733398 0.0
0 0.00595420598983765 1.0
0 0.0344380140304565 2.0
0 -0.0326076745986938 3.0
0 0.0386518836021423 4.0
0 -0.0331050753593445 5.0
0 0.00429987907409668 6.0
0 -0.0127902626991272 7.0
38.158424706 38.903678894043 0.0
0 0.0116057395935059 1.0
0 0.0537588000297546 2.0
0 -0.0285190343856812 3.0
0 0.0557814836502075 4.0
0 -0.029060959815979 5.0
0 0.00855320692062378 6.0
0 -0.0132871866226196 7.0
37.671552644 39.2374725341797 0.0
0 0.006736159324646 1.0
0 0.0430566668510437 2.0
0 -0.0213061571121216 3.0
0 0.0102139711380005 4.0
0 -0.0230097770690918 5.0
0 0.0043870210647583 6.0
0 -0.00778210163116455 7.0
38.99000687 39.3680267333984 0.0
0 0.00496220588684082 1.0
0 0.0836588740348816 2.0
0 -0.0154605507850647 3.0
0 0.0217403173446655 4.0
0 -0.0192769169807434 5.0
0 0.00258815288543701 6.0
0 -0.0127060413360596 7.0
32.824279197 39.2818450927734 0.0
0 0.0120186805725098 1.0
0 0.0409272313117981 2.0
0 -0.0234582424163818 3.0
0 0.00576364994049072 4.0
0 -0.0319589376449585 5.0
0 0.00346595048904419 6.0
0 -0.00725382566452026 7.0
39.752392398 39.3454055786133 0.0
0 0.00749439001083374 1.0
0 0.0421807169914246 2.0
0 -0.0276132225990295 3.0
0 0.00937461853027344 4.0
0 -0.0163375735282898 5.0
0 -0.000748932361602783 6.0
0 -0.00666379928588867 7.0
36.014997988 39.1476974487305 0.0
0 0.0123811960220337 1.0
0 0.0417296886444092 2.0
0 -0.0282233953475952 3.0
0 0.0180580019950867 4.0
0 -0.032294750213623 5.0
0 0.00644737482070923 6.0
0 -0.00982779264450073 7.0
42.307205606 39.2100219726562 0.0
0 0.0106115341186523 1.0
0 0.0420219302177429 2.0
0 -0.030609130859375 3.0
0 0.0188378095626831 4.0
0 -0.0219831466674805 5.0
0 0.00154978036880493 6.0
0 -0.00879871845245361 7.0
41.624369934 37.9989166259766 0.0
0 0.00612103939056396 1.0
0 0.0464754104614258 2.0
0 -0.0220385789871216 3.0
0 0.0484929084777832 4.0
0 -0.0285543203353882 5.0
0 0.00955373048782349 6.0
0 -0.0139495730400085 7.0
40.927860926 38.1056976318359 0.0
0 0.00858557224273682 1.0
0 0.0353962779045105 2.0
0 -0.0306072235107422 3.0
0 0.0340653657913208 4.0
0 -0.0286934375762939 5.0
0 0.00470423698425293 6.0
0 -0.0121994614601135 7.0
47.441321812 39.2602005004883 0.0
0 0.00840264558792114 1.0
0 0.0400999188423157 2.0
0 -0.0347042083740234 3.0
0 0.0447063446044922 4.0
0 -0.0227860808372498 5.0
0 0.00289112329483032 6.0
0 -0.0108776092529297 7.0
39.379505619 39.218936920166 0.0
0 0.00225824117660522 1.0
0 0.0381470918655396 2.0
0 -0.0227357745170593 3.0
0 0.0566052198410034 4.0
0 -0.0273053050041199 5.0
0 0.00671166181564331 6.0
0 -0.0176942348480225 7.0
42.095557299 39.302864074707 0.0
0 0.000669658184051514 1.0
0 0.0900809168815613 2.0
0 -0.0144271850585938 3.0
0 0.046349048614502 4.0
0 -0.0227258205413818 5.0
0 0.00791877508163452 6.0
0 -0.0147258043289185 7.0
35.151188225 39.5698585510254 0.0
0 0.00752902030944824 1.0
0 0.0650426149368286 2.0
0 -0.0100555419921875 3.0
0 -0.00416684150695801 4.0
0 -0.0206260681152344 5.0
0 -0.000355720520019531 6.0
0 -0.00151586532592773 7.0
37.713935371 39.5196762084961 0.0
0 0.000767946243286133 1.0
0 0.0639498233795166 2.0
0 -0.0160928964614868 3.0
0 0.0237512588500977 4.0
0 -0.0243517756462097 5.0
0 0.0049559473991394 6.0
0 -0.00916051864624023 7.0
39.66774326 39.3000450134277 0.0
0 0.00557941198348999 1.0
0 0.0528358221054077 2.0
0 -0.0230751037597656 3.0
0 0.0496906638145447 4.0
0 -0.0281819701194763 5.0
0 0.00943386554718018 6.0
0 -0.0162165760993958 7.0
41.240556194 39.485954284668 0.0
0 0.0104520320892334 1.0
0 0.0391672253608704 2.0
0 -0.0241261720657349 3.0
0 0.0442981123924255 4.0
0 -0.029579222202301 5.0
0 0.00497233867645264 6.0
0 -0.01584392786026 7.0
40.3201946 39.4521560668945 0.0
0 0.0161011219024658 1.0
0 0.0323320627212524 2.0
0 -0.026906430721283 3.0
0 0.021517276763916 4.0
0 -0.0247675180435181 5.0
0 0.00135600566864014 6.0
0 -0.0140462517738342 7.0
39.561151059 39.2782707214355 0.0
0 -0.00219696760177612 1.0
0 0.0362330079078674 2.0
0 -0.0246546864509583 3.0
0 0.0312585830688477 4.0
0 -0.0289401412010193 5.0
0 0.00624489784240723 6.0
0 -0.012248694896698 7.0
43.566431947 39.6282119750977 0.0
0 0.0138700008392334 1.0
0 0.025522768497467 2.0
0 -0.0197473764419556 3.0
0 0.0512849688529968 4.0
0 -0.0301445722579956 5.0
0 0.00931829214096069 6.0
0 -0.0174809098243713 7.0
37.929366562 39.231330871582 0.0
0 0.010800302028656 1.0
0 0.043013334274292 2.0
0 -0.0228856801986694 3.0
0 0.0199460387229919 4.0
0 -0.0271242260932922 5.0
0 0.00468665361404419 6.0
0 -0.011076033115387 7.0
39.745668641 41.7682800292969 0.0
0 0.00947737693786621 1.0
0 0.0982784032821655 2.0
0 -0.0173757672309875 3.0
0 0.00384366512298584 4.0
0 -0.0101626515388489 5.0
0 -0.000672698020935059 6.0
0 -0.00756984949111938 7.0
37.173794625 39.3144912719727 0.0
0 0.00357526540756226 1.0
0 0.0787873268127441 2.0
0 -0.0133687853813171 3.0
0 0.00534951686859131 4.0
0 -0.0198338031768799 5.0
0 0.00434726476669312 6.0
0 -0.00697380304336548 7.0
27.124866015 39.1525268554688 0.0
0 0.00913196802139282 1.0
0 0.0650781393051147 2.0
0 -0.00544387102127075 3.0
0 -0.00907778739929199 4.0
0 -0.0324002504348755 5.0
0 0.00577831268310547 6.0
0 -0.00207573175430298 7.0
27.411746904 36.7663421630859 0.0
0 0.0113141536712646 1.0
0 0.0197427868843079 2.0
0 -0.0141464471817017 3.0
0 -0.00583177804946899 4.0
0 -0.0341864824295044 5.0
0 0.00935220718383789 6.0
0 0.000265181064605713 7.0
38.866617317 39.1861763000488 0.0
0 0.00661951303482056 1.0
0 0.056182324886322 2.0
0 -0.021723210811615 3.0
0 0.0445036292076111 4.0
0 -0.029047966003418 5.0
0 0.00600475072860718 6.0
0 -0.0126184225082397 7.0
42.383057354 39.06201171875 0.0
0 0.0104770064353943 1.0
0 0.0411927103996277 2.0
0 -0.0221678614616394 3.0
0 0.0380172729492188 4.0
0 -0.0284179449081421 5.0
0 0.00465226173400879 6.0
0 -0.0151697993278503 7.0
47.643751036 39.1715316772461 0.0
0 0.000791609287261963 1.0
0 0.0480496883392334 2.0
0 -0.0215412378311157 3.0
0 0.053810715675354 4.0
0 -0.0269570350646973 5.0
0 0.00569742918014526 6.0
0 -0.0143246054649353 7.0
38.439399957 37.768684387207 0.0
0 0.00674319267272949 1.0
0 0.064213752746582 2.0
0 -0.0178337693214417 3.0
0 0.0179169774055481 4.0
0 -0.0244011878967285 5.0
0 0.0028071403503418 6.0
0 -0.0114209651947021 7.0
40.263562371 39.1669425964355 0.0
0 0.0111555457115173 1.0
0 0.0356557369232178 2.0
0 -0.0216387510299683 3.0
0 0.0576434135437012 4.0
0 -0.0272286534309387 5.0
0 0.0108609199523926 6.0
0 -0.0150966048240662 7.0
41.519528397 39.2469940185547 0.0
0 0.00689941644668579 1.0
0 0.0410262942314148 2.0
0 -0.0277458429336548 3.0
0 0.0269702672958374 4.0
0 -0.0244253873825073 5.0
0 0.0042043924331665 6.0
0 -0.0120133757591248 7.0
40.414570474 39.2424240112305 0.0
0 0.0107041597366333 1.0
0 0.0481968522071838 2.0
0 -0.0277416110038757 3.0
0 0.0130128264427185 4.0
0 -0.02088862657547 5.0
0 0.0020906925201416 6.0
0 -0.00745880603790283 7.0
37.834181221 39.3338584899902 0.0
0 0.00449800491333008 1.0
0 0.0605843067169189 2.0
0 -0.0180467963218689 3.0
0 0.0176889896392822 4.0
0 -0.0260162949562073 5.0
0 0.0046238899230957 6.0
0 -0.0100391507148743 7.0
37.201121556 39.261905670166 0.0
0 0.0136870741844177 1.0
0 0.0373289585113525 2.0
0 -0.0291894674301147 3.0
0 0.0102047920227051 4.0
0 -0.027827262878418 5.0
0 0.0071987509727478 6.0
0 -0.00730001926422119 7.0
42.94444514 39.2730560302734 0.0
0 0.0106481313705444 1.0
0 0.0675126910209656 2.0
0 -0.0177286267280579 3.0
0 0.0313223600387573 4.0
0 -0.0179081559181213 5.0
0 0.00385010242462158 6.0
0 -0.0127323865890503 7.0
47.309600034 39.3127517700195 0.0
0 0.0105430483818054 1.0
0 0.0398989319801331 2.0
0 -0.0270081758499146 3.0
0 0.0480376482009888 4.0
0 -0.0255812406539917 5.0
0 0.00514119863510132 6.0
0 -0.0150682926177979 7.0
34.703660747 39.2433204650879 0.0
0 0.0071418285369873 1.0
0 0.0537604689598083 2.0
0 -0.020726203918457 3.0
0 0.00321263074874878 4.0
0 -0.021537184715271 5.0
0 0.00369763374328613 6.0
0 -0.00550210475921631 7.0
39.355679833 39.1759719848633 0.0
0 0.00848031044006348 1.0
0 0.0391460657119751 2.0
0 -0.0200943946838379 3.0
0 0.0308890342712402 4.0
0 -0.0281250476837158 5.0
0 0.00596320629119873 6.0
0 -0.0112216472625732 7.0
36.760838135 39.1233863830566 0.0
0 0.00670671463012695 1.0
0 0.0653762817382812 2.0
0 -0.0138376355171204 3.0
0 0.0073549747467041 4.0
0 -0.0216763615608215 5.0
0 0.00479042530059814 6.0
0 -0.00692027807235718 7.0
45.817767369 39.8167724609375 0.0
0 0.0150848031044006 1.0
0 0.0327014923095703 2.0
0 -0.0391030311584473 3.0
0 0.056266725063324 4.0
0 -0.0295649170875549 5.0
0 0.00774598121643066 6.0
0 -0.0133306980133057 7.0
37.456387944 38.9908180236816 0.0
0 0.00394445657730103 1.0
0 0.0999663472175598 2.0
0 -0.00932472944259644 3.0
0 0.0233494639396667 4.0
0 -0.026712954044342 5.0
0 0.00659197568893433 6.0
0 -0.0119200348854065 7.0
36.979299113 39.0708389282227 0.0
0 0.00956428050994873 1.0
0 0.069067120552063 2.0
0 -0.0124732851982117 3.0
0 0.00505238771438599 4.0
0 -0.0217453241348267 5.0
0 0.00202125310897827 6.0
0 -0.00531578063964844 7.0
44.357392157 39.4652366638184 0.0
0 0.00713968276977539 1.0
0 0.0625491738319397 2.0
0 -0.0172478556632996 3.0
0 0.037219226360321 4.0
0 -0.0215816497802734 5.0
0 0.00404489040374756 6.0
0 -0.0137313008308411 7.0
36.812079299 39.2094459533691 0.0
0 0.00176447629928589 1.0
0 0.0411084890365601 2.0
0 -0.0210328102111816 3.0
0 0.0358702540397644 4.0
0 -0.026917576789856 5.0
0 0.00472712516784668 6.0
0 -0.0162766575813293 7.0
43.172254743 39.3419456481934 0.0
0 0.000443518161773682 1.0
0 0.0612084865570068 2.0
0 -0.0245235562324524 3.0
0 0.0217550992965698 4.0
0 -0.0219128727912903 5.0
0 0.00296598672866821 6.0
0 -0.0115439891815186 7.0
47.109370567 39.1553535461426 0.0
0 0.00755888223648071 1.0
0 0.0679612159729004 2.0
0 -0.0227341055870056 3.0
0 0.0571240186691284 4.0
0 -0.0264320373535156 5.0
0 0.0023043155670166 6.0
0 -0.0164520740509033 7.0
35.871355662 35.5522994995117 0.0
0 0.00543594360351562 1.0
0 0.0244265198707581 2.0
0 -0.0247161984443665 3.0
0 0.0317372679710388 4.0
0 -0.0295183062553406 5.0
0 0.0111661553382874 6.0
0 -0.0115025639533997 7.0
42.169433911 39.2534637451172 0.0
0 0.0117353200912476 1.0
0 0.062811553478241 2.0
0 -0.0239560604095459 3.0
0 0.0422621369361877 4.0
0 -0.0282317399978638 5.0
0 0.00753867626190186 6.0
0 -0.0137905478477478 7.0
38.001249401 39.3316230773926 0.0
0 0.00929278135299683 1.0
0 0.0384349226951599 2.0
0 -0.0226041078567505 3.0
0 0.0314591526985168 4.0
0 -0.0318945050239563 5.0
0 0.00869399309158325 6.0
0 -0.0126127600669861 7.0
36.026296153 39.0085372924805 0.0
0 0.00563478469848633 1.0
0 0.0343274474143982 2.0
0 -0.0304900407791138 3.0
0 0.0181832313537598 4.0
0 -0.0314008593559265 5.0
0 0.00474011898040771 6.0
0 -0.00925129652023315 7.0
33.200213159 39.6004867553711 0.0
0 0.00608885288238525 1.0
0 0.0842117667198181 2.0
0 -0.0121983289718628 3.0
0 0.00233906507492065 4.0
0 -0.0259609818458557 5.0
0 0.00225377082824707 6.0
0 -0.0079498291015625 7.0
36.881947952 39.2454299926758 0.0
0 -0.00684928894042969 1.0
0 0.0569502115249634 2.0
0 -0.0175040364265442 3.0
0 0.0261569619178772 4.0
0 -0.0287220478057861 5.0
0 0.00709050893783569 6.0
0 -0.0138422846794128 7.0
28.363012012 39.311351776123 0.0
0 0.00370430946350098 1.0
0 0.0378327369689941 2.0
0 -0.0167774558067322 3.0
0 -0.00418192148208618 4.0
0 -0.0319845080375671 5.0
0 0.00403189659118652 6.0
0 -0.00433927774429321 7.0
39.574467426 39.1792831420898 0.0
0 0.0124419927597046 1.0
0 0.0643044710159302 2.0
0 -0.0187584757804871 3.0
0 0.0552770495414734 4.0
0 -0.0264163017272949 5.0
0 0.00504046678543091 6.0
0 -0.0174887776374817 7.0
36.22040148 39.3068237304688 0.0
0 0.00884103775024414 1.0
0 0.0396783351898193 2.0
0 -0.0248955488204956 3.0
0 0.0143897533416748 4.0
0 -0.02907395362854 5.0
0 0.00538623332977295 6.0
0 -0.00855058431625366 7.0
31.086432141 39.3125267028809 0.0
0 -0.00659275054931641 1.0
0 0.0630043745040894 2.0
0 -0.0123117566108704 3.0
0 -0.004280686378479 4.0
0 -0.0248154997825623 5.0
0 0.00149953365325928 6.0
0 -0.00380051136016846 7.0
36.274353063 39.3318557739258 0.0
0 0.00835269689559937 1.0
0 0.0660974383354187 2.0
0 -0.0127261281013489 3.0
0 0.000513792037963867 4.0
0 -0.0210362672805786 5.0
0 0.00450301170349121 6.0
0 -0.00523066520690918 7.0
31.65795009 40.0534744262695 0.0
0 0.00283342599868774 1.0
0 0.0983990430831909 2.0
0 -0.0014115571975708 3.0
0 0.0108137130737305 4.0
0 -0.0277411937713623 5.0
0 0.0125419497489929 6.0
0 -0.00817984342575073 7.0
37.059171479 39.6106262207031 0.0
0 0.0132450461387634 1.0
0 0.0257991552352905 2.0
0 -0.0181992650032043 3.0
0 0.0418131947517395 4.0
0 -0.0292995572090149 5.0
0 0.0161434412002563 6.0
0 -0.014696478843689 7.0
45.308862357 39.1841506958008 0.0
0 0.00910359621047974 1.0
0 0.0663491487503052 2.0
0 -0.0206829309463501 3.0
0 0.0444293618202209 4.0
0 -0.0230956673622131 5.0
0 0.0029442310333252 6.0
0 -0.0143704414367676 7.0
32.988279909 39.4551620483398 0.0
0 0.00968480110168457 1.0
0 0.0575180649757385 2.0
0 -0.0297290086746216 3.0
0 0.00657224655151367 4.0
0 -0.0277795195579529 5.0
0 0.0065266489982605 6.0
0 -0.00360023975372314 7.0
41.812200593 42.8991012573242 0.0
0 -0.000670015811920166 1.0
0 0.0709576010704041 2.0
0 -0.022020161151886 3.0
0 0.0297814011573792 4.0
0 -0.0287713408470154 5.0
0 0.006084144115448 6.0
0 -0.0115088224411011 7.0
34.159578007 38.4319915771484 0.0
0 0.0111063122749329 1.0
0 0.034248411655426 2.0
0 -0.0333222150802612 3.0
0 0.00719785690307617 4.0
0 -0.0321474075317383 5.0
0 0.00268566608428955 6.0
0 -0.00621837377548218 7.0
41.353058204 39.204532623291 0.0
0 0.00952857732772827 1.0
0 0.081007719039917 2.0
0 -0.0175600051879883 3.0
0 0.046159565448761 4.0
0 -0.0286591649055481 5.0
0 0.00810974836349487 6.0
0 -0.0123917460441589 7.0
35.663236644 39.2133636474609 0.0
0 0.00283634662628174 1.0
0 0.0574272274971008 2.0
0 -0.0158212780952454 3.0
0 0.0198498964309692 4.0
0 -0.0298638343811035 5.0
0 0.00515240430831909 6.0
0 -0.0104875564575195 7.0
40.444916123 39.1653671264648 0.0
0 0.00414353609085083 1.0
0 0.0910453796386719 2.0
0 -0.0167463421821594 3.0
0 0.038258969783783 4.0
0 -0.0234010815620422 5.0
0 0.00739949941635132 6.0
0 -0.0134220123291016 7.0
39.900416914 39.0888595581055 0.0
0 0.00893539190292358 1.0
0 0.0359787344932556 2.0
0 -0.0211640000343323 3.0
0 0.0231063961982727 4.0
0 -0.0259357690811157 5.0
0 0.00329744815826416 6.0
0 -0.0172143578529358 7.0
34.405324849 39.97607421875 0.0
0 0.0172434449195862 1.0
0 0.0376632213592529 2.0
0 -0.0179058313369751 3.0
0 0.0187467336654663 4.0
0 -0.0309320092201233 5.0
0 0.00770348310470581 6.0
0 -0.00930380821228027 7.0
33.074256004 39.249397277832 0.0
0 0.00738769769668579 1.0
0 0.0685418844223022 2.0
0 -0.0165396332740784 3.0
0 0.00878924131393433 4.0
0 -0.028192400932312 5.0
0 0.00620543956756592 6.0
0 -0.0068475604057312 7.0
40.036170308 41.1524620056152 0.0
0 0.00465995073318481 1.0
0 0.0542076826095581 2.0
0 -0.0169073343276978 3.0
0 0.0155050754547119 4.0
0 -0.0213350653648376 5.0
0 0.00354379415512085 6.0
0 -0.00880926847457886 7.0
44.453206241 39.2687759399414 0.0
0 0.0084221363067627 1.0
0 0.0385976433753967 2.0
0 -0.0236493945121765 3.0
0 0.0401104688644409 4.0
0 -0.0222529768943787 5.0
0 0.0013420581817627 6.0
0 -0.0154792070388794 7.0
33.85004541 39.3401641845703 0.0
0 0.0100825428962708 1.0
0 0.0304266810417175 2.0
0 -0.0158665776252747 3.0
0 -0.00110459327697754 4.0
0 -0.0240499973297119 5.0
0 0.00319308042526245 6.0
0 -0.00285923480987549 7.0
30.537877308 38.9493637084961 0.0
0 0.00709092617034912 1.0
0 0.0584277510643005 2.0
0 -0.010155200958252 3.0
0 0.00216293334960938 4.0
0 -0.0305030345916748 5.0
0 0.00729340314865112 6.0
0 -0.00616723299026489 7.0
27.727440942 38.7583274841309 0.0
0 0.0101714730262756 1.0
0 0.0332983732223511 2.0
0 -0.00794166326522827 3.0
0 -0.0133890509605408 4.0
0 -0.0270724296569824 5.0
0 0.00212264060974121 6.0
0 0.000737905502319336 7.0
35.630182897 39.3510971069336 0.0
0 0.00258451700210571 1.0
0 0.0429811477661133 2.0
0 -0.0208121538162231 3.0
0 0.00156229734420776 4.0
0 -0.0212336182594299 5.0
0 0.00289946794509888 6.0
0 -0.00715601444244385 7.0
31.038251519 39.2236862182617 0.0
0 0.00821179151535034 1.0
0 0.0344059467315674 2.0
0 -0.0368871092796326 3.0
0 0.00533008575439453 4.0
0 -0.0332455635070801 5.0
0 0.00354135036468506 6.0
0 0.527037620544434 7.0
31.867011967 39.251651763916 0.0
0 0.00107324123382568 1.0
0 0.054533839225769 2.0
0 -0.0123904943466187 3.0
0 0.000221490859985352 4.0
0 -0.0262243151664734 5.0
0 0.00323355197906494 6.0
0 -0.00364047288894653 7.0
30.754831646 39.2847442626953 0.0
0 0.000662744045257568 1.0
0 0.0500643253326416 2.0
0 -0.0136899352073669 3.0
0 -0.00537914037704468 4.0
0 -0.0222152471542358 5.0
0 0.00284117460250854 6.0
0 -0.00049436092376709 7.0
38.95156428 39.6339340209961 0.0
0 0.0100151300430298 1.0
0 0.0536243319511414 2.0
0 -0.0311551094055176 3.0
0 0.0400863289833069 4.0
0 -0.0307068228721619 5.0
0 0.00652027130126953 6.0
0 -0.0129345059394836 7.0
33.384480936 40.8775634765625 0.0
0 -0.00124102830886841 1.0
0 0.082556426525116 2.0
0 -0.0175243020057678 3.0
0 0.00131016969680786 4.0
0 -0.0186545848846436 5.0
0 0.0023646354675293 6.0
0 -0.00465917587280273 7.0
37.291199282 39.228099822998 0.0
0 -0.00408190488815308 1.0
0 0.0565981864929199 2.0
0 -0.0146593451499939 3.0
0 -0.00024259090423584 4.0
0 -0.0163029432296753 5.0
0 -0.000132083892822266 6.0
0 -0.00570875406265259 7.0
33.571811016 39.1781425476074 0.0
0 -0.00461328029632568 1.0
0 0.0580087304115295 2.0
0 -0.0127177834510803 3.0
0 0.00326383113861084 4.0
0 -0.0277678370475769 5.0
0 0.00513482093811035 6.0
0 -0.00657427310943604 7.0
44.048761746 39.1108169555664 0.0
0 0.0133167505264282 1.0
0 0.0345013737678528 2.0
0 -0.0266736149787903 3.0
0 0.0249786972999573 4.0
0 -0.0194141864776611 5.0
0 0.0016477108001709 6.0
0 -0.0132050514221191 7.0
38.09281546 39.2019882202148 0.0
0 0.0076175332069397 1.0
0 0.0517597794532776 2.0
0 -0.0209031105041504 3.0
0 0.0179016590118408 4.0
0 -0.026101291179657 5.0
0 0.00557214021682739 6.0
0 -0.00983411073684692 7.0
42.616606924 39.1387519836426 0.0
0 0.0139750242233276 1.0
0 0.0358852744102478 2.0
0 -0.0247586965560913 3.0
0 0.0312950015068054 4.0
0 -0.0261248350143433 5.0
0 0.00321567058563232 6.0
0 -0.0164498090744019 7.0
41.080227074 39.3054122924805 0.0
0 0.00673854351043701 1.0
0 0.0645447969436646 2.0
0 -0.0166183114051819 3.0
0 0.0444710850715637 4.0
0 -0.0268169045448303 5.0
0 0.00533783435821533 6.0
0 -0.01416015625 7.0
36.859087813 39.3295364379883 0.0
0 0.011201024055481 1.0
0 0.0334835648536682 2.0
0 -0.0205079913139343 3.0
0 0.00591433048248291 4.0
0 -0.0269621610641479 5.0
0 0.00424182415008545 6.0
0 -0.00879001617431641 7.0
39.316268876 39.2730102539062 0.0
0 0.0123557448387146 1.0
0 0.0479531288146973 2.0
0 -0.026253879070282 3.0
0 0.0314406752586365 4.0
0 -0.0275577306747437 5.0
0 0.00751090049743652 6.0
0 -0.0127378702163696 7.0
40.023071967 39.1557769775391 0.0
0 0.00954633951187134 1.0
0 0.0609726905822754 2.0
0 -0.0256583690643311 3.0
0 0.0391594171524048 4.0
0 -0.0301165580749512 5.0
0 0.00344556570053101 6.0
0 -0.0116539001464844 7.0
32.691039566 39.2794570922852 0.0
0 -0.0015331506729126 1.0
0 0.0595820546150208 2.0
0 -0.0134975910186768 3.0
0 0.000495076179504395 4.0
0 -0.02903151512146 5.0
0 0.00216841697692871 6.0
0 -0.00742316246032715 7.0
39.35017223 39.2273406982422 0.0
0 0.0103467106819153 1.0
0 0.0433016419410706 2.0
0 -0.0213442444801331 3.0
0 0.0215471982955933 4.0
0 -0.0266906023025513 5.0
0 0.00573867559432983 6.0
0 -0.0110342502593994 7.0
38.62577922 39.0582504272461 0.0
0 0.00206905603408813 1.0
0 0.0952557921409607 2.0
0 -0.0130048990249634 3.0
0 0.0412141084671021 4.0
0 -0.0248604416847229 5.0
0 0.00641036033630371 6.0
0 -0.0157877802848816 7.0
43.36321018 39.2144470214844 0.0
0 0.000869214534759521 1.0
0 0.0585317015647888 2.0
0 -0.0207402110099792 3.0
0 0.0382145643234253 4.0
0 -0.0245130062103271 5.0
0 0.00293725728988647 6.0
0 -0.0152141451835632 7.0
41.711370519 39.4875259399414 0.0
0 0.0113811492919922 1.0
0 0.0472018122673035 2.0
0 -0.0314878225326538 3.0
0 0.0203774571418762 4.0
0 -0.0202274322509766 5.0
0 0.00135642290115356 6.0
0 -0.0107417106628418 7.0
41.401748585 39.4137382507324 0.0
0 0.00584369897842407 1.0
0 0.0841087698936462 2.0
0 -0.0202708840370178 3.0
0 0.0407012104988098 4.0
0 -0.026205837726593 5.0
0 0.00837677717208862 6.0
0 -0.0136880874633789 7.0
45.411303661 39.0654220581055 0.0
0 0.0100109577178955 1.0
0 0.0425238609313965 2.0
0 -0.0348400473594666 3.0
0 0.0446252226829529 4.0
0 -0.0273600816726685 5.0
0 0.00585973262786865 6.0
0 -0.011528491973877 7.0
46.173622738 39.1124382019043 0.0
0 0.00331169366836548 1.0
0 0.0585666298866272 2.0
0 -0.0199772119522095 3.0
0 0.0386195778846741 4.0
0 -0.0185463428497314 5.0
0 0.000947475433349609 6.0
0 -0.0127515196800232 7.0
43.970500892 39.0283279418945 0.0
0 0.0122954845428467 1.0
0 0.0489515066146851 2.0
0 -0.0201903581619263 3.0
0 0.047145664691925 4.0
0 -0.0270442962646484 5.0
0 0.00498276948928833 6.0
0 -0.0150191783905029 7.0
41.561964524 39.1679534912109 0.0
0 0.00369572639465332 1.0
0 0.0706163048744202 2.0
0 -0.0154618620872498 3.0
0 0.0192911028862 4.0
0 -0.0254248380661011 5.0
0 0.00374150276184082 6.0
0 -0.00939935445785522 7.0
42.043570834 39.0953025817871 0.0
0 0.0122132301330566 1.0
0 0.0394858121871948 2.0
0 -0.0237278342247009 3.0
0 0.0304210186004639 4.0
0 -0.0229068994522095 5.0
0 0.00413888692855835 6.0
0 -0.0118696093559265 7.0
42.728187908 39.2809371948242 0.0
0 0.00305062532424927 1.0
0 0.0476612448692322 2.0
0 -0.0216407179832458 3.0
0 0.0432797074317932 4.0
0 -0.0265682935714722 5.0
0 0.0038069486618042 6.0
0 -0.0142127871513367 7.0
46.647822945 39.149055480957 0.0
0 0.0102046728134155 1.0
0 0.0686874389648438 2.0
0 -0.0248903036117554 3.0
0 0.0577712059020996 4.0
0 -0.0270435810089111 5.0
0 0.00424551963806152 6.0
0 -0.0131389498710632 7.0
34.157298462 39.2264099121094 0.0
0 0.00863933563232422 1.0
0 0.0604138970375061 2.0
0 -0.0150264501571655 3.0
0 0.00885719060897827 4.0
0 -0.0287423133850098 5.0
0 0.00353121757507324 6.0
0 -0.00855332612991333 7.0
38.08940786 39.6085052490234 0.0
0 0.00380009412765503 1.0
0 0.0974803566932678 2.0
0 -0.0130190849304199 3.0
0 -0.00274521112442017 4.0
0 -0.0061413049697876 5.0
0 -0.00102388858795166 6.0
0 -0.00624740123748779 7.0
48.253456605 39.9512825012207 0.0
0 0.0059741735458374 1.0
0 0.0713126659393311 2.0
0 -0.0189449787139893 3.0
0 0.059030294418335 4.0
0 -0.0232948064804077 5.0
0 0.00563764572143555 6.0
0 -0.0141817331314087 7.0
41.303512955 39.4999008178711 0.0
0 0.00800210237503052 1.0
0 0.0413928627967834 2.0
0 -0.0225116014480591 3.0
0 0.0409272313117981 4.0
0 -0.0284388065338135 5.0
0 0.00299149751663208 6.0
0 -0.0151891708374023 7.0
34.11247552 39.3124771118164 0.0
0 0.0101791024208069 1.0
0 0.0430975556373596 2.0
0 -0.0161041021347046 3.0
0 -0.00142985582351685 4.0
0 -0.0245183706283569 5.0
0 0.00401866436004639 6.0
0 -0.00263869762420654 7.0
39.346899321 39.1841049194336 0.0
0 0.0121107697486877 1.0
0 0.0421473383903503 2.0
0 -0.0220503211021423 3.0
0 0.0479787588119507 4.0
0 -0.0291696786880493 5.0
0 0.00787287950515747 6.0
0 -0.015704333782196 7.0
42.355081321 39.1616287231445 0.0
0 0.00376284122467041 1.0
0 0.0752883553504944 2.0
0 -0.0166050791740417 3.0
0 0.0418010354042053 4.0
0 -0.0204254984855652 5.0
0 0.00545191764831543 6.0
0 -0.0175421833992004 7.0
46.025284619 39.0997200012207 0.0
0 0.0107316970825195 1.0
0 0.0411190390586853 2.0
0 -0.0379164218902588 3.0
0 0.0383503437042236 4.0
0 -0.0208945274353027 5.0
0 0.000841259956359863 6.0
0 -0.012586236000061 7.0
48.790909787 39.3467254638672 0.0
0 0.00544178485870361 1.0
0 0.0473642945289612 2.0
0 -0.0263350009918213 3.0
0 0.0592371225357056 4.0
0 -0.0242483615875244 5.0
0 0.0057494044303894 6.0
0 -0.0136021971702576 7.0
41.698940502 39.3436889648438 0.0
0 0.0103285908699036 1.0
0 0.0436311960220337 2.0
0 -0.0286429524421692 3.0
0 0.0448897480964661 4.0
0 -0.0313480496406555 5.0
0 0.00959122180938721 6.0
0 -0.0129580497741699 7.0
47.902972781 39.1910247802734 0.0
0 0.0104819536209106 1.0
0 0.0497010946273804 2.0
0 -0.0263906717300415 3.0
0 0.0467828512191772 4.0
0 -0.0187551379203796 5.0
0 0.00208765268325806 6.0
0 -0.0136069655418396 7.0
41.37717087 39.0758895874023 0.0
0 0.00883805751800537 1.0
0 0.0678055286407471 2.0
0 -0.0207234621047974 3.0
0 0.029201865196228 4.0
0 -0.0210285186767578 5.0
0 0.00388187170028687 6.0
0 -0.0106710195541382 7.0
32.285684902 39.1620063781738 0.0
0 0.00987410545349121 1.0
0 0.0596850514411926 2.0
0 -0.0258843302726746 3.0
0 -0.00052642822265625 4.0
0 -0.0234203934669495 5.0
0 0.00188684463500977 6.0
0 -0.00331372022628784 7.0
32.344206851 39.1839065551758 0.0
0 0.00693005323410034 1.0
0 0.0404759049415588 2.0
0 -0.0222252607345581 3.0
0 0.0102599263191223 4.0
0 -0.0320175290107727 5.0
0 0.00594037771224976 6.0
0 -0.00761276483535767 7.0
34.184112331 39.1595001220703 0.0
0 0.000822484493255615 1.0
0 0.0435900092124939 2.0
0 -0.0234014391899109 3.0
0 0.016313374042511 4.0
0 -0.0279474854469299 5.0
0 0.00414592027664185 6.0
0 -0.0104918479919434 7.0
34.663221808 39.3465347290039 0.0
0 -0.000472068786621094 1.0
0 0.0439326167106628 2.0
0 -0.0197044610977173 3.0
0 0.0183951854705811 4.0
0 -0.0307208895683289 5.0
0 0.00468224287033081 6.0
0 -0.00934088230133057 7.0
36.873774543 37.5921669006348 0.0
0 0.0069698691368103 1.0
0 0.0686353445053101 2.0
0 -0.0163471102714539 3.0
0 0.0109089612960815 4.0
0 -0.0253394842147827 5.0
0 0.00617462396621704 6.0
0 -0.0072709321975708 7.0
40.211352908 41.326587677002 0.0
0 0.00541871786117554 1.0
0 0.0645110607147217 2.0
0 -0.01763916015625 3.0
0 0.0509659647941589 4.0
0 -0.027746319770813 5.0
0 0.0062638521194458 6.0
0 -0.0150855183601379 7.0
46.160689867 39.1345748901367 0.0
0 0.00654143095016479 1.0
0 0.0590987801551819 2.0
0 -0.0207692384719849 3.0
0 0.0336949825286865 4.0
0 -0.0170185565948486 5.0
0 0.000889062881469727 6.0
0 -0.0137842893600464 7.0
32.727702676 39.3031883239746 0.0
0 -0.00463449954986572 1.0
0 0.0640050768852234 2.0
0 -0.0136813521385193 3.0
0 0.00972533226013184 4.0
0 -0.0276746153831482 5.0
0 0.00688004493713379 6.0
0 -0.00769346952438354 7.0
40.640985417 39.6990623474121 0.0
0 0.00356471538543701 1.0
0 0.0938190817832947 2.0
0 -0.0121647715568542 3.0
0 0.0314699411392212 4.0
0 -0.0217582583427429 5.0
0 0.00673258304595947 6.0
0 -0.0126134753227234 7.0
34.611451594 39.2935600280762 0.0
0 0.000468611717224121 1.0
0 0.0598671436309814 2.0
0 -0.0131622552871704 3.0
0 0.00562536716461182 4.0
0 -0.0259057879447937 5.0
0 0.00558769702911377 6.0
0 -0.00558269023895264 7.0
43.311280807 39.3564987182617 0.0
0 0.00210869312286377 1.0
0 0.0423729419708252 2.0
0 -0.0237454771995544 3.0
0 0.0202456712722778 4.0
0 -0.0209356546401978 5.0
0 0.00234854221343994 6.0
0 -0.0102128982543945 7.0
36.406364057 39.175666809082 0.0
0 0.0102632641792297 1.0
0 0.0409222841262817 2.0
0 -0.0188425183296204 3.0
0 0.00562071800231934 4.0
0 -0.0216001868247986 5.0
0 0.00233513116836548 6.0
0 -0.00800949335098267 7.0
31.685630326 40.3169288635254 0.0
0 0.00647974014282227 1.0
0 0.101521492004395 2.0
0 -0.00225663185119629 3.0
0 0.00412189960479736 4.0
0 -0.0269104242324829 5.0
0 0.00584179162979126 6.0
0 -0.00681865215301514 7.0
40.165364432 39.4257736206055 0.0
0 0.0061718225479126 1.0
0 0.105965197086334 2.0
0 -0.0121121406555176 3.0
0 0.0475353002548218 4.0
0 -0.0240756273269653 5.0
0 0.00451123714447021 6.0
0 -0.0161829590797424 7.0
42.886465647 39.1565208435059 0.0
0 0.00445181131362915 1.0
0 0.058378279209137 2.0
0 -0.0202310085296631 3.0
0 0.0512354969978333 4.0
0 -0.0279179811477661 5.0
0 0.00581037998199463 6.0
0 -0.0149148106575012 7.0
46.942592332 40.7634811401367 0.0
0 0.00733822584152222 1.0
0 0.0581620931625366 2.0
0 -0.0207223296165466 3.0
0 0.0556744337081909 4.0
0 -0.0250055193901062 5.0
0 0.00304579734802246 6.0
0 -0.0171255469322205 7.0
38.417030568 39.1845207214355 0.0
0 -0.000441789627075195 1.0
0 0.105224907398224 2.0
0 -0.00935876369476318 3.0
0 0.0345804691314697 4.0
0 -0.0238878726959229 5.0
0 0.00515294075012207 6.0
0 -0.0114102363586426 7.0
39.82159543 39.3340148925781 0.0
0 -0.00290924310684204 1.0
0 0.0601381659507751 2.0
0 -0.0178883075714111 3.0
0 0.0185866355895996 4.0
0 -0.0235638618469238 5.0
0 0.00506407022476196 6.0
0 -0.00952208042144775 7.0
42.375846269 40.3157577514648 0.0
0 0.00914698839187622 1.0
0 0.0349929332733154 2.0
0 -0.0344812273979187 3.0
0 0.013907790184021 4.0
0 -0.0226353406906128 5.0
0 0.00246036052703857 6.0
0 -0.010317862033844 7.0
41.909868924 39.2151527404785 0.0
0 0.00799280405044556 1.0
0 0.0414422750473022 2.0
0 -0.022936224937439 3.0
0 0.0615046620368958 4.0
0 -0.0280951261520386 5.0
0 0.0132050514221191 6.0
0 -0.0195711851119995 7.0
38.515403814 39.1933288574219 0.0
0 0.00757837295532227 1.0
0 0.0305657386779785 2.0
0 -0.0187130570411682 3.0
0 0.00982528924942017 4.0
0 -0.0262295603752136 5.0
0 0.00381177663803101 6.0
0 -0.0142884850502014 7.0
32.187464773 40.187915802002 0.0
0 0.00200361013412476 1.0
0 0.064132034778595 2.0
0 -0.0111357569694519 3.0
0 -0.000443577766418457 4.0
0 -0.025220513343811 5.0
0 0.0035330057144165 6.0
0 -0.00435435771942139 7.0
43.553340498 39.1683349609375 0.0
0 0.00320440530776978 1.0
0 0.0983927845954895 2.0
0 -0.0131569504737854 3.0
0 0.0549652576446533 4.0
0 -0.0220618844032288 5.0
0 0.00318139791488647 6.0
0 -0.0133134126663208 7.0
38.661685118 39.4650688171387 0.0
0 0.00699847936630249 1.0
0 0.0637048482894897 2.0
0 -0.0227053165435791 3.0
0 0.053022027015686 4.0
0 -0.0268428325653076 5.0
0 0.00573450326919556 6.0
0 -0.0151920914649963 7.0
47.185976948 38.8878593444824 0.0
0 0.00848960876464844 1.0
0 0.0649888515472412 2.0
0 -0.0193367600440979 3.0
0 0.0486640930175781 4.0
0 -0.0238392949104309 5.0
0 0.00460058450698853 6.0
0 -0.0163248777389526 7.0
41.073649825 39.4418640136719 0.0
0 0.01736980676651 1.0
0 0.0300519466400146 2.0
0 -0.0189222693443298 3.0
0 0.0178244709968567 4.0
0 -0.0293866395950317 5.0
0 0.00656688213348389 6.0
0 -0.00890588760375977 7.0
39.086114846 39.2041931152344 0.0
0 0.00746828317642212 1.0
0 0.0617871284484863 2.0
0 -0.0160078406333923 3.0
0 0.0467209815979004 4.0
0 -0.0276315808296204 5.0
0 0.00437235832214355 6.0
0 -0.0158487558364868 7.0
38.072309151 39.6437606811523 0.0
0 0.00959813594818115 1.0
0 0.0610776543617249 2.0
0 -0.0135322213172913 3.0
0 0.00896525382995605 4.0
0 -0.023659884929657 5.0
0 0.00476402044296265 6.0
0 -0.00752836465835571 7.0
41.585402374 39.4675216674805 0.0
0 0.0107553005218506 1.0
0 0.0416585206985474 2.0
0 -0.0218178033828735 3.0
0 0.0383657217025757 4.0
0 -0.0271846652030945 5.0
0 0.00441157817840576 6.0
0 -0.0129627585411072 7.0
42.317317809 39.0025901794434 0.0
0 0.0110228657722473 1.0
0 0.0656244158744812 2.0
0 -0.0249553322792053 3.0
0 0.0582363605499268 4.0
0 -0.0284330248832703 5.0
0 0.00378590822219849 6.0
0 -0.0140643119812012 7.0
42.931477585 39.7166442871094 0.0
0 0.00569617748260498 1.0
0 0.0841277241706848 2.0
0 -0.0175623297691345 3.0
0 0.0348663926124573 4.0
0 -0.0192890167236328 5.0
0 0.00405728816986084 6.0
0 -0.0120517611503601 7.0
46.096274411 39.246166229248 0.0
0 -0.00434160232543945 1.0
0 0.0482379794120789 2.0
0 -0.0227041244506836 3.0
0 0.0343828797340393 4.0
0 -0.0184457898139954 5.0
0 0.000456690788269043 6.0
0 -0.0140078067779541 7.0
39.955261554 39.2918701171875 0.0
0 0.0116240978240967 1.0
0 0.0501741766929626 2.0
0 -0.0237830281257629 3.0
0 0.0337701439857483 4.0
0 -0.0290184617042542 5.0
0 0.00481361150741577 6.0
0 -0.0128209590911865 7.0
45.415961846 39.1588554382324 0.0
0 0.00861090421676636 1.0
0 0.0404647588729858 2.0
0 -0.0227407813072205 3.0
0 0.0548158288002014 4.0
0 -0.0274008512496948 5.0
0 0.00865215063095093 6.0
0 -0.0156762003898621 7.0
44.275746657 39.1352005004883 0.0
0 0.00998026132583618 1.0
0 0.0705584287643433 2.0
0 -0.0192747116088867 3.0
0 0.0429856181144714 4.0
0 -0.0221502780914307 5.0
0 0.00686490535736084 6.0
0 -0.0132014751434326 7.0
34.766238966 39.1581649780273 0.0
0 0.0108010172843933 1.0
0 0.066469132900238 2.0
0 -0.0145248174667358 3.0
0 0.0109792947769165 4.0
0 -0.0286319255828857 5.0
0 0.00657302141189575 6.0
0 -0.00772237777709961 7.0
39.697519492 40.0110397338867 0.0
0 0.00621956586837769 1.0
0 0.0532541871070862 2.0
0 -0.0158854722976685 3.0
0 0.0149997472763062 4.0
0 -0.0200769305229187 5.0
0 0.00282829999923706 6.0
0 -0.0118199586868286 7.0
36.459952599 39.2017936706543 0.0
0 0.00732451677322388 1.0
0 0.0604976415634155 2.0
0 -0.016093373298645 3.0
0 0.00341081619262695 4.0
0 -0.0244671106338501 5.0
0 0.00652408599853516 6.0
0 -0.00836789608001709 7.0
44.856515586 39.419303894043 0.0
0 0.00477176904678345 1.0
0 0.0616564154624939 2.0
0 -0.0213950872421265 3.0
0 0.0443920493125916 4.0
0 -0.0205050706863403 5.0
0 0.00288867950439453 6.0
0 -0.0148175954818726 7.0
41.157913956 39.2678260803223 0.0
0 0.00864797830581665 1.0
0 0.0603436231613159 2.0
0 -0.0151415467262268 3.0
0 0.0148828029632568 4.0
0 -0.0174505710601807 5.0
0 0.00234842300415039 6.0
0 -0.00891876220703125 7.0
32.092085792 42.0078506469727 0.0
0 -0.00147992372512817 1.0
0 0.0986078977584839 2.0
0 -0.00447070598602295 3.0
0 0.00738292932510376 4.0
0 -0.027155876159668 5.0
0 0.00754690170288086 6.0
0 -0.00817745923995972 7.0
30.327518425 39.2771072387695 0.0
0 0.000197649002075195 1.0
0 0.05263751745224 2.0
0 -0.00881850719451904 3.0
0 -0.00964021682739258 4.0
0 -0.0268546342849731 5.0
0 0.00452160835266113 6.0
0 -0.00201511383056641 7.0
36.482977278 39.6762313842773 0.0
0 0.00489264726638794 1.0
0 0.0885087847709656 2.0
0 -0.0142347812652588 3.0
0 0.0247588753700256 4.0
0 -0.028696596622467 5.0
0 0.00561767816543579 6.0
0 -0.0122042894363403 7.0
40.491170914 40.3549003601074 0.0
0 0.000296533107757568 1.0
0 0.0906490683555603 2.0
0 -0.0146321058273315 3.0
0 0.0381223559379578 4.0
0 -0.0233554840087891 5.0
0 0.00532442331314087 6.0
0 -0.012855052947998 7.0
41.188898334 40.0176544189453 0.0
0 -0.00143259763717651 1.0
0 0.0951639413833618 2.0
0 -0.0130124092102051 3.0
0 0.0215404033660889 4.0
0 -0.0166791677474976 5.0
0 0.00263166427612305 6.0
0 -0.00915664434432983 7.0
37.756949904 39.1947593688965 0.0
0 0.0107933282852173 1.0
0 0.0656872391700745 2.0
0 -0.0142897963523865 3.0
0 0.0106445550918579 4.0
0 -0.0203924179077148 5.0
0 0.00396895408630371 6.0
0 -0.00912606716156006 7.0
26.9626458 39.3925285339355 0.0
0 0.0110253691673279 1.0
0 0.0370810627937317 2.0
0 -0.00381720066070557 3.0
0 -0.016218900680542 4.0
0 -0.0292441248893738 5.0
0 0.00169575214385986 6.0
0 -0.00216329097747803 7.0
44.154680866 39.3581161499023 0.0
0 0.00145381689071655 1.0
0 0.0635524392127991 2.0
0 -0.0199567079544067 3.0
0 0.029373824596405 4.0
0 -0.0141153931617737 5.0
0 0.000345468521118164 6.0
0 -0.0163211822509766 7.0
36.49920361 39.3040390014648 0.0
0 0.00980162620544434 1.0
0 0.046657919883728 2.0
0 -0.0151821374893188 3.0
0 0.0272837281227112 4.0
0 -0.0288670659065247 5.0
0 0.00800848007202148 6.0
0 -0.0117116570472717 7.0
36.136187007 39.2891960144043 0.0
0 0.0114787220954895 1.0
0 0.0657549500465393 2.0
0 -0.0165747404098511 3.0
0 0.0228102207183838 4.0
0 -0.0275917053222656 5.0
0 0.00385874509811401 6.0
0 -0.0123043656349182 7.0
43.805905709 39.2161407470703 0.0
0 0.00916695594787598 1.0
0 0.0414623022079468 2.0
0 -0.0270732641220093 3.0
0 0.0342727899551392 4.0
0 -0.0226308107376099 5.0
0 0.00324547290802002 6.0
0 -0.012790858745575 7.0
43.400673478 38.5996551513672 0.0
0 0.00959324836730957 1.0
0 0.07652348279953 2.0
0 -0.0166631937026978 3.0
0 0.0510177612304688 4.0
0 -0.0256300568580627 5.0
0 0.00306117534637451 6.0
0 -0.0166022777557373 7.0
35.708898932 39.7909126281738 0.0
0 0.00258690118789673 1.0
0 0.101891338825226 2.0
0 -0.0113672614097595 3.0
0 0.0187270045280457 4.0
0 -0.026531994342804 5.0
0 0.00352787971496582 6.0
0 -0.010202944278717 7.0
42.195375979 39.2065582275391 0.0
0 0.00936859846115112 1.0
0 0.0624610185623169 2.0
0 -0.0231862664222717 3.0
0 0.0374898910522461 4.0
0 -0.0261919498443604 5.0
0 0.00701963901519775 6.0
0 -0.0106062889099121 7.0
38.219132252 39.2476501464844 0.0
0 0.00796747207641602 1.0
0 0.060630202293396 2.0
0 -0.0189728736877441 3.0
0 0.00981682538986206 4.0
0 -0.0244853496551514 5.0
0 0.00241279602050781 6.0
0 -0.00450032949447632 7.0
37.34785904 40.7463417053223 0.0
0 0.00804460048675537 1.0
0 0.0386269092559814 2.0
0 -0.0224984884262085 3.0
0 0.0120346546173096 4.0
0 -0.0248059034347534 5.0
0 0.0053831934928894 6.0
0 -0.00742852687835693 7.0
41.413317083 39.1686630249023 0.0
0 0.0103911161422729 1.0
0 0.0369403958320618 2.0
0 -0.0325966477394104 3.0
0 0.0260374546051025 4.0
0 -0.0254538059234619 5.0
0 0.00346553325653076 6.0
0 -0.0124757289886475 7.0
39.408474012 41.2074737548828 0.0
0 0.00463259220123291 1.0
0 0.0393736362457275 2.0
0 -0.0281912088394165 3.0
0 0.0336098074913025 4.0
0 -0.0318562984466553 5.0
0 0.00816363096237183 6.0
0 -0.0106446146965027 7.0
44.621947793 39.146858215332 0.0
0 0.0129572153091431 1.0
0 0.0404406189918518 2.0
0 -0.0323220491409302 3.0
0 0.0574005246162415 4.0
0 -0.0294423699378967 5.0
0 0.0104334950447083 6.0
0 -0.0138011574745178 7.0
35.042314673 41.5502014160156 0.0
0 0.0104658603668213 1.0
0 0.0302814841270447 2.0
0 -0.0129294991493225 3.0
0 0.00461137294769287 4.0
0 -0.0273417234420776 5.0
0 0.00530403852462769 6.0
0 -0.00574469566345215 7.0
44.668685987 37.9078559875488 0.0
0 0.00188130140304565 1.0
0 0.0569884777069092 2.0
0 -0.0238006114959717 3.0
0 0.0405648350715637 4.0
0 -0.0205771923065186 5.0
0 0.00241458415985107 6.0
0 -0.0139407515525818 7.0
44.271188235 39.1885566711426 0.0
0 0.0127209424972534 1.0
0 0.0644296407699585 2.0
0 -0.0185632109642029 3.0
0 0.0303701162338257 4.0
0 -0.0184800624847412 5.0
0 0.00196337699890137 6.0
0 -0.0148364305496216 7.0
43.329615715 39.419849395752 0.0
0 0.00306707620620728 1.0
0 0.0421066284179688 2.0
0 -0.0237050652503967 3.0
0 0.0451827645301819 4.0
0 -0.0286803841590881 5.0
0 0.00414633750915527 6.0
0 -0.0128758549690247 7.0
37.727034893 39.9785270690918 0.0
0 0.00373846292495728 1.0
0 0.103487849235535 2.0
0 -0.00995522737503052 3.0
0 0.0277867317199707 4.0
0 -0.0272249579429626 5.0
0 0.00663357973098755 6.0
0 -0.0104453563690186 7.0
43.665312113 39.5566329956055 0.0
0 0.00762540102005005 1.0
0 0.0750016570091248 2.0
0 -0.0351837277412415 3.0
0 0.0328378081321716 4.0
0 -0.0215832591056824 5.0
0 0.00253617763519287 6.0
0 -0.0119035840034485 7.0
40.656348383 39.3315620422363 0.0
0 0.00649392604827881 1.0
0 0.0533838868141174 2.0
0 -0.0215229988098145 3.0
0 0.0549891591072083 4.0
0 -0.0267018675804138 5.0
0 0.0048946738243103 6.0
0 -0.0185717940330505 7.0
30.759139555 39.0844345092773 0.0
0 0.0121983289718628 1.0
0 0.0359728336334229 2.0
0 -0.0149562954902649 3.0
0 -0.00471550226211548 4.0
0 -0.028002142906189 5.0
0 0.0035325288772583 6.0
0 -0.00117355585098267 7.0
36.474229652 40.8910751342773 0.0
0 0.00264835357666016 1.0
0 0.0991296172142029 2.0
0 -0.00933718681335449 3.0
0 0.00034099817276001 4.0
0 -0.0150880813598633 5.0
0 -0.000272393226623535 6.0
0 -0.00282025337219238 7.0
33.340721195 39.4016265869141 0.0
0 0.00756460428237915 1.0
0 0.0755277276039124 2.0
0 -0.0127190351486206 3.0
0 0.00348019599914551 4.0
0 -0.0280179381370544 5.0
0 0.00535142421722412 6.0
0 -0.00605171918869019 7.0
34.143349268 41.1993026733398 0.0
0 0.0105020999908447 1.0
0 0.0738517045974731 2.0
0 -0.0288435220718384 3.0
0 0.00792866945266724 4.0
0 -0.022797167301178 5.0
0 0.00235277414321899 6.0
0 -0.00532907247543335 7.0
40.763829757 40.4009170532227 0.0
0 0.00627636909484863 1.0
0 0.0678209066390991 2.0
0 -0.0207937955856323 3.0
0 0.0571699142456055 4.0
0 -0.0280265212059021 5.0
0 0.00891619920730591 6.0
0 -0.0133203268051147 7.0
29.788224958 39.3779296875 0.0
0 -0.00283634662628174 1.0
0 0.0589134097099304 2.0
0 -0.011816680431366 3.0
0 0.000213921070098877 4.0
0 -0.03299880027771 5.0
0 0.00723862648010254 6.0
0 -0.00425004959106445 7.0
34.675869554 39.2007484436035 0.0
0 0.00924175977706909 1.0
0 0.062283456325531 2.0
0 -0.0111774206161499 3.0
0 0.00481492280960083 4.0
0 -0.0254631042480469 5.0
0 0.00442290306091309 6.0
0 -0.00621861219406128 7.0
40.723855828 39.139232635498 0.0
0 0.004505455493927 1.0
0 0.0670126080513 2.0
0 -0.0169151425361633 3.0
0 0.0427116751670837 4.0
0 -0.0239702463150024 5.0
0 0.00283890962600708 6.0
0 -0.0185911655426025 7.0
41.414838352 39.2057685852051 0.0
0 0.002247154712677 1.0
0 0.059136688709259 2.0
0 -0.0206869840621948 3.0
0 0.0594696998596191 4.0
0 -0.0269739031791687 5.0
0 0.0113312602043152 6.0
0 -0.0169715285301208 7.0
33.391441214 39.5404624938965 0.0
0 0.00687164068222046 1.0
0 0.068155825138092 2.0
0 -0.0151040554046631 3.0
0 0.00336939096450806 4.0
0 -0.0272725820541382 5.0
0 0.00364166498184204 6.0
0 -0.0084540843963623 7.0
36.660645385 39.1693992614746 0.0
0 0.0112992525100708 1.0
0 0.0539188981056213 2.0
0 -0.0432652831077576 3.0
0 0.0196741223335266 4.0
0 -0.0308671593666077 5.0
0 0.00485384464263916 6.0
0 -0.00823885202407837 7.0
33.547567166 41.1571044921875 0.0
0 0.00413417816162109 1.0
0 0.0913617014884949 2.0
0 -0.0106722712516785 3.0
0 0.0163525938987732 4.0
0 -0.0269976854324341 5.0
0 0.00591397285461426 6.0
0 -0.00831866264343262 7.0
36.717923495 39.3228797912598 0.0
0 -0.00490498542785645 1.0
0 0.0684824585914612 2.0
0 -0.0163276791572571 3.0
0 0.0187729001045227 4.0
0 -0.0261236429214478 5.0
0 0.00353103876113892 6.0
0 -0.0108597874641418 7.0
44.575737494 40.2921981811523 0.0
0 0.00726407766342163 1.0
0 0.0921012759208679 2.0
0 -0.0169275999069214 3.0
0 0.0439640879631042 4.0
0 -0.0148934721946716 5.0
0 0.00246167182922363 6.0
0 -0.0144742131233215 7.0
43.091610321 39.251880645752 0.0
0 0.00294780731201172 1.0
0 0.0471649765968323 2.0
0 -0.0214514136314392 3.0
0 0.0389942526817322 4.0
0 -0.0256434082984924 5.0
0 0.00726276636123657 6.0
0 -0.0136053562164307 7.0
42.647069973 39.2061424255371 0.0
0 0.00906306505203247 1.0
0 0.0546945333480835 2.0
0 -0.0212883949279785 3.0
0 0.0380060076713562 4.0
0 -0.0249283909797668 5.0
0 0.00426143407821655 6.0
0 -0.0141628384590149 7.0
36.240225339 39.1404800415039 0.0
0 0.00918090343475342 1.0
0 0.0593395233154297 2.0
0 -0.0215871930122375 3.0
0 0.0104026794433594 4.0
0 -0.0252101421356201 5.0
0 0.00523948669433594 6.0
0 -0.00701224803924561 7.0
31.876756203 39.139762878418 0.0
0 -0.000991940498352051 1.0
0 0.0431054830551147 2.0
0 -0.0127851963043213 3.0
0 -0.00101745128631592 4.0
0 -0.0230093598365784 5.0
0 0.00262880325317383 6.0
0 -0.00305753946304321 7.0
38.711049659 39.3581237792969 0.0
0 0.00102627277374268 1.0
0 0.0828373432159424 2.0
0 -0.0169112682342529 3.0
0 0.00627750158309937 4.0
0 -0.0123506188392639 5.0
0 -0.000407099723815918 6.0
0 -0.00948333740234375 7.0
33.592635672 39.4242515563965 0.0
0 0.00957810878753662 1.0
0 0.0583613514900208 2.0
0 -0.016112208366394 3.0
0 0.00602555274963379 4.0
0 -0.027643084526062 5.0
0 0.00605815649032593 6.0
0 -0.00576615333557129 7.0
31.939497021 39.1622772216797 0.0
0 0.00103104114532471 1.0
0 0.0396385192871094 2.0
0 -0.0136800408363342 3.0
0 -0.00527811050415039 4.0
0 -0.0252284407615662 5.0
0 0.00408285856246948 6.0
0 -0.00103867053985596 7.0
32.605677369 39.2626190185547 0.0
0 0.00559169054031372 1.0
0 0.0369923114776611 2.0
0 -0.0197780132293701 3.0
0 0.000530064105987549 4.0
0 -0.0273765325546265 5.0
0 0.00718575716018677 6.0
0 -0.00460433959960938 7.0
35.02561425 39.1090469360352 0.0
0 0.0119553208351135 1.0
0 0.0453833937644958 2.0
0 -0.0258418917655945 3.0
0 0.0193158388137817 4.0
0 -0.0308746695518494 5.0
0 0.0087929368019104 6.0
0 -0.007557213306427 7.0
29.155020371 39.2253952026367 0.0
0 0.0101317167282104 1.0
0 0.0378549098968506 2.0
0 -0.0250255465507507 3.0
0 0.00145792961120605 4.0
0 -0.0328030586242676 5.0
0 0.00670236349105835 6.0
0 0.182495445013046 7.0
29.9790999 38.9883193969727 0.0
0 -0.00512790679931641 1.0
0 0.0625416040420532 2.0
0 -0.00465649366378784 3.0
0 -0.00537872314453125 4.0
0 -0.0296611785888672 5.0
0 0.00600504875183105 6.0
0 -0.00503420829772949 7.0
40.412374428 39.2049255371094 0.0
0 -0.000606119632720947 1.0
0 0.0611275434494019 2.0
0 -0.0168014168739319 3.0
0 0.00644654035568237 4.0
0 -0.0177046060562134 5.0
0 0.00162214040756226 6.0
0 -0.0136350393295288 7.0
44.799669607 39.0750770568848 0.0
0 0.00750100612640381 1.0
0 0.0563011765480042 2.0
0 -0.0221543312072754 3.0
0 0.0368587374687195 4.0
0 -0.0180134177207947 5.0
0 0.000889480113983154 6.0
0 -0.0161266922950745 7.0
35.691930085 39.3859329223633 0.0
0 0.00633901357650757 1.0
0 0.0712013244628906 2.0
0 -0.0153719782829285 3.0
0 0.00453132390975952 4.0
0 -0.0222364664077759 5.0
0 0.00377142429351807 6.0
0 -0.00545322895050049 7.0
37.293002382 39.293083190918 0.0
0 0.00778055191040039 1.0
0 0.047900378704071 2.0
0 -0.0283905863761902 3.0
0 0.0200028419494629 4.0
0 -0.0298619866371155 5.0
0 0.00535517930984497 6.0
0 -0.011212944984436 7.0
31.749050635 39.1932945251465 0.0
0 0.00823521614074707 1.0
0 0.0382753014564514 2.0
0 -0.0209130644798279 3.0
0 0.00449585914611816 4.0
0 -0.028892993927002 5.0
0 0.00504481792449951 6.0
0 -0.00455832481384277 7.0
44.438599316 38.0644073486328 0.0
0 0.00910055637359619 1.0
0 0.0278371572494507 2.0
0 -0.0197588205337524 3.0
0 0.0566070675849915 4.0
0 -0.0264150500297546 5.0
0 0.00440853834152222 6.0
0 -0.0190314054489136 7.0
30.584762345 39.4195861816406 0.0
0 0.00186467170715332 1.0
0 0.0638080835342407 2.0
0 -0.0128000378608704 3.0
0 -0.00119149684906006 4.0
0 -0.0299015045166016 5.0
0 0.00222301483154297 6.0
0 -0.00576400756835938 7.0
36.167711467 40.7079963684082 0.0
0 0.00583952665328979 1.0
0 0.0441656112670898 2.0
0 -0.0209394693374634 3.0
0 0.00305551290512085 4.0
0 -0.019661009311676 5.0
0 -0.000128626823425293 6.0
0 -0.0033729076385498 7.0
45.589023165 39.2716369628906 0.0
0 0.0118179321289062 1.0
0 0.0400128960609436 2.0
0 -0.0317465662956238 3.0
0 0.0473290085792542 4.0
0 -0.0295619964599609 5.0
0 0.00329500436782837 6.0
0 -0.0119420289993286 7.0
33.954863975 39.3449859619141 0.0
0 -0.00191313028335571 1.0
0 0.057437539100647 2.0
0 -0.0151700377464294 3.0
0 0.00851249694824219 4.0
0 -0.0251034498214722 5.0
0 0.00245237350463867 6.0
0 -0.0112330913543701 7.0
39.104021217 39.2743453979492 0.0
0 0.00290310382843018 1.0
0 0.0589911937713623 2.0
0 -0.018439769744873 3.0
0 0.0172427892684937 4.0
0 -0.020279586315155 5.0
0 0.00379747152328491 6.0
0 -0.0103532075881958 7.0
41.615139375 39.283073425293 0.0
0 0.00299423933029175 1.0
0 0.0724921226501465 2.0
0 -0.0180994868278503 3.0
0 0.058605432510376 4.0
0 -0.0211931467056274 5.0
0 0.00603270530700684 6.0
0 -0.0181660652160645 7.0
29.251209169 39.1626129150391 0.0
0 0.0102241635322571 1.0
0 0.0657804012298584 2.0
0 -0.0218889713287354 3.0
0 0.000598728656768799 4.0
0 -0.0287654995918274 5.0
0 0.00124531984329224 6.0
0 -0.00544106960296631 7.0
39.145261601 39.434383392334 0.0
0 0.0106642246246338 1.0
0 0.0615565180778503 2.0
0 -0.0195338726043701 3.0
0 0.0287595987319946 4.0
0 -0.0294601917266846 5.0
0 0.0090785026550293 6.0
0 -0.0129284858703613 7.0
43.724223514 39.9443969726562 0.0
0 0.00348567962646484 1.0
0 0.0611519813537598 2.0
0 -0.0157661437988281 3.0
0 0.0497835278511047 4.0
0 -0.0242002606391907 5.0
0 0.00416934490203857 6.0
0 -0.016603946685791 7.0
30.694427948 39.1826820373535 0.0
0 0.00585633516311646 1.0
0 0.0551376342773438 2.0
0 -0.00917786359786987 3.0
0 -0.00169706344604492 4.0
0 -0.0302058458328247 5.0
0 0.00556594133377075 6.0
0 -0.00630462169647217 7.0
46.899446079 39.3772010803223 0.0
0 0.00522828102111816 1.0
0 0.119101643562317 2.0
0 -0.0105419754981995 3.0
0 0.105043053627014 4.0
0 -0.0214530229568481 5.0
0 0.00253283977508545 6.0
0 -0.0160475373268127 7.0
30.801741874 39.1385078430176 0.0
0 0.0105280876159668 1.0
0 0.0485381484031677 2.0
0 -0.0218062996864319 3.0
0 0.00451433658599854 4.0
0 -0.0322909355163574 5.0
0 0.00279700756072998 6.0
0 -0.00666153430938721 7.0
40.388900757 39.6161880493164 0.0
0 0.0105310678482056 1.0
0 0.0530675649642944 2.0
0 -0.0235248208045959 3.0
0 0.0383188128471375 4.0
0 -0.0301554203033447 5.0
0 0.00359994173049927 6.0
0 -0.0129978656768799 7.0
35.801686131 39.3320503234863 0.0
0 0.00162678956985474 1.0
0 0.0328916311264038 2.0
0 -0.0161664485931396 3.0
0 0.0023643970489502 4.0
0 -0.0239839553833008 5.0
0 0.00194603204727173 6.0
0 -0.00305426120758057 7.0
43.275430302 39.1461601257324 0.0
0 0.00677180290222168 1.0
0 0.066714346408844 2.0
0 -0.0179618000984192 3.0
0 0.0459111332893372 4.0
0 -0.0245304107666016 5.0
0 0.00858467817306519 6.0
0 -0.013133704662323 7.0
36.674533885 39.2840042114258 0.0
0 0.00746917724609375 1.0
0 0.0606139898300171 2.0
0 -0.0169159770011902 3.0
0 0.0125620365142822 4.0
0 -0.0260381698608398 5.0
0 0.00515270233154297 6.0
0 -0.00862163305282593 7.0
42.291929941 39.0463104248047 0.0
0 0.011316180229187 1.0
0 0.0493571162223816 2.0
0 -0.0247151851654053 3.0
0 0.0330401062965393 4.0
0 -0.0230516195297241 5.0
0 0.00425034761428833 6.0
0 -0.0135107636451721 7.0
37.80686855 38.719539642334 0.0
0 0.00790625810623169 1.0
0 0.0712704658508301 2.0
0 -0.0141570568084717 3.0
0 0.00576889514923096 4.0
0 -0.0159686803817749 5.0
0 -0.000587940216064453 6.0
0 -0.00549298524856567 7.0
36.296445808 38.5794677734375 0.0
0 0.0126692056655884 1.0
0 0.0595385432243347 2.0
0 -0.0157817602157593 3.0
0 0.00248444080352783 4.0
0 -0.0212079882621765 5.0
0 0.00206708908081055 6.0
0 -0.00864028930664062 7.0
26.786347985 41.6884956359863 0.0
0 1.05500221252441e-05 1.0
0 0.0941086411476135 2.0
0 -0.0172778964042664 3.0
0 -0.00798177719116211 4.0
0 -0.0263301730155945 5.0
0 0.00166332721710205 6.0
0 0.325894445180893 7.0
38.715518937 39.3310127258301 0.0
0 -0.00107306241989136 1.0
0 0.0533255338668823 2.0
0 -0.0208864808082581 3.0
0 0.0355126857757568 4.0
0 -0.0311377644538879 5.0
0 0.00435280799865723 6.0
0 -0.0146806836128235 7.0
35.505063002 39.0710105895996 0.0
0 0.0100252032279968 1.0
0 0.0440443158149719 2.0
0 -0.0252014994621277 3.0
0 0.00799816846847534 4.0
0 -0.0276076197624207 5.0
0 0.00622522830963135 6.0
0 -0.00744861364364624 7.0
44.227649071 39.2503204345703 0.0
0 -0.00350522994995117 1.0
0 0.0830144882202148 2.0
0 -0.0194167494773865 3.0
0 0.0467356443405151 4.0
0 -0.0179108381271362 5.0
0 0.00280278921127319 6.0
0 -0.0161828398704529 7.0
45.773124255 39.1983108520508 0.0
0 0.0102026462554932 1.0
0 0.0390667915344238 2.0
0 -0.0257700681686401 3.0
0 0.0395534634590149 4.0
0 -0.0186536312103271 5.0
0 0.00253766775131226 6.0
0 -0.0136967897415161 7.0
38.615282752 39.6247634887695 0.0
0 0.00875389575958252 1.0
0 0.0653161406517029 2.0
0 -0.0225114226341248 3.0
0 0.0422231554985046 4.0
0 -0.0281192064285278 5.0
0 0.00611317157745361 6.0
0 -0.0130001306533813 7.0
42.131124836 39.6604461669922 0.0
0 0.00814712047576904 1.0
0 0.0577985048294067 2.0
0 -0.0264377593994141 3.0
0 0.0277695655822754 4.0
0 -0.0232623219490051 5.0
0 0.00406908988952637 6.0
0 -0.0118699073791504 7.0
38.030369766 39.4241714477539 0.0
0 0.00730973482131958 1.0
0 0.0649347305297852 2.0
0 -0.0156338810920715 3.0
0 0.028175950050354 4.0
0 -0.0283916592597961 5.0
0 0.00388896465301514 6.0
0 -0.0123919248580933 7.0
32.694870002 39.2483978271484 0.0
0 0.00167417526245117 1.0
0 0.0581066608428955 2.0
0 -0.0132492780685425 3.0
0 -0.00167709589004517 4.0
0 -0.0220803618431091 5.0
0 0.000322103500366211 6.0
0 -0.00424700975418091 7.0
41.58883184 39.7600479125977 0.0
0 0.0111787915229797 1.0
0 0.0836202502250671 2.0
0 -0.0201756954193115 3.0
0 0.014514684677124 4.0
0 -0.0171653032302856 5.0
0 -0.000364601612091064 6.0
0 -0.00911998748779297 7.0
33.622063045 40.3607978820801 0.0
0 0.00406837463378906 1.0
0 0.0649199485778809 2.0
0 -0.0115718841552734 3.0
0 0.0118268728256226 4.0
0 -0.0288293957710266 5.0
0 0.00606423616409302 6.0
0 -0.00943201780319214 7.0
40.101738506 39.3203430175781 0.0
0 0.00397443771362305 1.0
0 0.0380821228027344 2.0
0 -0.0269546508789062 3.0
0 0.0460603833198547 4.0
0 -0.0289177894592285 5.0
0 0.0054628849029541 6.0
0 -0.0136521458625793 7.0
39.347750494 39.5650939941406 0.0
0 0.0116128921508789 1.0
0 0.0399914383888245 2.0
0 -0.0279043912887573 3.0
0 0.0306953191757202 4.0
0 -0.0280945897102356 5.0
0 0.00678908824920654 6.0
0 -0.0120279788970947 7.0
46.240008463 39.3097534179688 0.0
0 0.00658661127090454 1.0
0 0.0587683320045471 2.0
0 -0.0242289304733276 3.0
0 0.0355812907218933 4.0
0 -0.0214058756828308 5.0
0 0.00193798542022705 6.0
0 -0.0121954083442688 7.0
43.885613972 39.8051948547363 0.0
0 0.0136743783950806 1.0
0 0.0521467924118042 2.0
0 -0.0285190343856812 3.0
0 0.0310320258140564 4.0
0 -0.0195971727371216 5.0
0 0.000680863857269287 6.0
0 -0.0110752582550049 7.0
44.877286055 39.2165946960449 0.0
0 0.00419837236404419 1.0
0 0.0352147817611694 2.0
0 -0.0227287411689758 3.0
0 0.062357485294342 4.0
0 -0.0253933668136597 5.0
0 0.00503700971603394 6.0
0 -0.0205078125 7.0
44.281024849 39.100700378418 0.0
0 0.0138094425201416 1.0
0 0.0409066081047058 2.0
0 -0.0262832045555115 3.0
0 0.0474932789802551 4.0
0 -0.0281514525413513 5.0
0 0.00751256942749023 6.0
0 -0.0144731402397156 7.0
39.443195334 39.5614204406738 0.0
0 0.00306022167205811 1.0
0 0.0668423175811768 2.0
0 -0.0195117592811584 3.0
0 0.0196396708488464 4.0
0 -0.0204795002937317 5.0
0 0.00201261043548584 6.0
0 -0.013425886631012 7.0
44.50998012 37.5527420043945 0.0
0 0.00241619348526001 1.0
0 0.0602812767028809 2.0
0 -0.0234832167625427 3.0
0 0.0416800379753113 4.0
0 -0.0199583768844604 5.0
0 0.00301408767700195 6.0
0 -0.0121491551399231 7.0
36.079576822 39.6479034423828 0.0
0 0.0111091732978821 1.0
0 0.0342797040939331 2.0
0 -0.0194944739341736 3.0
0 0.0268609523773193 4.0
0 -0.0297252535820007 5.0
0 0.00870823860168457 6.0
0 -0.0126831531524658 7.0
43.881758876 39.3958206176758 0.0
0 0.0129520893096924 1.0
0 0.0298763513565063 2.0
0 -0.0417283773422241 3.0
0 0.0476492047309875 4.0
0 -0.0290360450744629 5.0
0 0.00537043809890747 6.0
0 -0.013556182384491 7.0
30.108818284 39.1312637329102 0.0
0 0.00127166509628296 1.0
0 0.0570345520973206 2.0
0 -0.0248159170150757 3.0
0 0.00391620397567749 4.0
0 -0.0310516953468323 5.0
0 0.00319862365722656 6.0
0 -0.0063481330871582 7.0
34.22051219 39.2516632080078 0.0
0 0.00860375165939331 1.0
0 0.059270441532135 2.0
0 -0.0170743465423584 3.0
0 0.0202519297599792 4.0
0 -0.030121386051178 5.0
0 0.00781601667404175 6.0
0 -0.0109772086143494 7.0
37.402197971 38.8370780944824 0.0
0 0.0108410120010376 1.0
0 0.0724775195121765 2.0
0 -0.0135229825973511 3.0
0 0.0213173627853394 4.0
0 -0.0238214731216431 5.0
0 0.00379276275634766 6.0
0 -0.00984638929367065 7.0
35.453722409 39.3546600341797 0.0
0 0.00394952297210693 1.0
0 0.0594062805175781 2.0
0 -0.0146932005882263 3.0
0 0.00734615325927734 4.0
0 -0.0265988707542419 5.0
0 0.0027625560760498 6.0
0 -0.00827115774154663 7.0
32.872381155 39.2892608642578 0.0
0 0.010620653629303 1.0
0 0.0609287023544312 2.0
0 -0.0179244875907898 3.0
0 0.00361537933349609 4.0
0 -0.0287152528762817 5.0
0 0.00480866432189941 6.0
0 -0.0058131217956543 7.0
33.577714316 39.4748306274414 0.0
0 -0.00118172168731689 1.0
0 0.0713554620742798 2.0
0 -0.0126799345016479 3.0
0 0.016187310218811 4.0
0 -0.029799222946167 5.0
0 0.00732976198196411 6.0
0 -0.00867438316345215 7.0
43.986168802 39.2046051025391 0.0
0 0.0113417506217957 1.0
0 0.0563039183616638 2.0
0 -0.0383703708648682 3.0
0 0.0311552882194519 4.0
0 -0.0199617147445679 5.0
0 0.000735759735107422 6.0
0 -0.0111079216003418 7.0
39.701527205 39.3187294006348 0.0
0 0.0148863196372986 1.0
0 0.0385289788246155 2.0
0 -0.0257417559623718 3.0
0 0.0237706899642944 4.0
0 -0.0290372967720032 5.0
0 0.00385552644729614 6.0
0 -0.0131402611732483 7.0
36.319422317 40.3933143615723 0.0
0 0.0118604898452759 1.0
0 0.0607532262802124 2.0
0 -0.0213486552238464 3.0
0 0.0343860387802124 4.0
0 -0.0309789180755615 5.0
0 0.00633883476257324 6.0
0 -0.0120580196380615 7.0
39.54652972 39.2991943359375 0.0
0 0.0112186670303345 1.0
0 0.0607815980911255 2.0
0 -0.0222302675247192 3.0
0 0.0338104367256165 4.0
0 -0.0278091430664062 5.0
0 0.00617092847824097 6.0
0 -0.0118278861045837 7.0
46.3288844 39.1624336242676 0.0
0 0.00678884983062744 1.0
0 0.0581302642822266 2.0
0 -0.0223013758659363 3.0
0 0.0426577925682068 4.0
0 -0.0176820755004883 5.0
0 0.00250375270843506 6.0
0 -0.0138975381851196 7.0
39.359671461 39.2167892456055 0.0
0 0.0114791989326477 1.0
0 0.0373216867446899 2.0
0 -0.0210093259811401 3.0
0 0.0348582863807678 4.0
0 -0.0306043028831482 5.0
0 0.00945502519607544 6.0
0 -0.0132561922073364 7.0
43.813870988 39.6971778869629 0.0
0 0.00339007377624512 1.0
0 0.0766655206680298 2.0
0 -0.0190767049789429 3.0
0 0.0360779166221619 4.0
0 -0.0164307951927185 5.0
0 0.00113844871520996 6.0
0 -0.0129740834236145 7.0
39.146221876 38.7799415588379 0.0
0 0.0131780505180359 1.0
0 0.0397027730941772 2.0
0 -0.0297419428825378 3.0
0 0.0398967862129211 4.0
0 -0.0307811498641968 5.0
0 0.00434565544128418 6.0
0 -0.0123134255409241 7.0
34.753353014 39.8708610534668 0.0
0 -0.000608384609222412 1.0
0 0.115673184394836 2.0
0 -0.00372225046157837 3.0
0 0.0298271775245667 4.0
0 -0.0289943218231201 5.0
0 0.00512754917144775 6.0
0 -0.0145651698112488 7.0
36.858915008 39.2941970825195 0.0
0 0.00608658790588379 1.0
0 0.0364677906036377 2.0
0 -0.0228100419044495 3.0
0 0.0330633521080017 4.0
0 -0.0306199789047241 5.0
0 0.0058516263961792 6.0
0 -0.0124207735061646 7.0
45.609388725 39.0649108886719 0.0
0 0.0138698220252991 1.0
0 0.0337219834327698 2.0
0 -0.0362816452980042 3.0
0 0.0378729701042175 4.0
0 -0.0258591175079346 5.0
0 0.00363445281982422 6.0
0 -0.0116531848907471 7.0
37.221502314 39.0854949951172 0.0
0 0.00831472873687744 1.0
0 0.0745607614517212 2.0
0 -0.0210608243942261 3.0
0 0.0341051816940308 4.0
0 -0.0302374362945557 5.0
0 0.00336241722106934 6.0
0 -0.0136440992355347 7.0
39.351241848 39.079216003418 0.0
0 0.0129474401473999 1.0
0 0.0575575828552246 2.0
0 -0.0210944414138794 3.0
0 0.0328056812286377 4.0
0 -0.0277805328369141 5.0
0 0.00349217653274536 6.0
0 -0.0115208029747009 7.0
44.354286139 39.4327774047852 0.0
0 0.00980257987976074 1.0
0 0.0603073239326477 2.0
0 -0.0212836861610413 3.0
0 0.0488134622573853 4.0
0 -0.0223110914230347 5.0
0 0.00612592697143555 6.0
0 -0.0144089460372925 7.0
41.143843969 39.2214050292969 0.0
0 0.00443631410598755 1.0
0 0.0381275415420532 2.0
0 -0.0224068760871887 3.0
0 0.0291399955749512 4.0
0 -0.027215301990509 5.0
0 0.00449514389038086 6.0
0 -0.0142806768417358 7.0
32.436363672 40.0223999023438 0.0
0 0.00905328989028931 1.0
0 0.0224776268005371 2.0
0 -0.0241499543190002 3.0
0 0.00383108854293823 4.0
0 -0.031097412109375 5.0
0 0.00539493560791016 6.0
0 -0.00590020418167114 7.0
35.980268826 39.2067642211914 0.0
0 0.00584536790847778 1.0
0 0.101477026939392 2.0
0 -0.0109339952468872 3.0
0 0.0248546004295349 4.0
0 -0.0253920555114746 5.0
0 0.0057300329208374 6.0
0 -0.0112724304199219 7.0
39.361696951 39.1074066162109 0.0
0 8.82148742675781e-06 1.0
0 0.0373160839080811 2.0
0 -0.0219159722328186 3.0
0 0.00553721189498901 4.0
0 -0.0202788710594177 5.0
0 0.0008392333984375 6.0
0 -0.00792026519775391 7.0
38.339957501 39.3969459533691 0.0
0 0.00817453861236572 1.0
0 0.0689851045608521 2.0
0 -0.0219828486442566 3.0
0 0.0383086204528809 4.0
0 -0.0291087031364441 5.0
0 0.00392496585845947 6.0
0 -0.0140419602394104 7.0
38.437078026 39.1383895874023 0.0
0 0.00185298919677734 1.0
0 0.086402416229248 2.0
0 -0.015372633934021 3.0
0 0.0599687695503235 4.0
0 -0.0258504748344421 5.0
0 0.00392729043960571 6.0
0 -0.0180753469467163 7.0
31.680301161 39.6670150756836 0.0
0 0.00319278240203857 1.0
0 0.0746790766716003 2.0
0 -0.00584864616394043 3.0
0 -0.00392329692840576 4.0
0 -0.0279016494750977 5.0
0 0.00421690940856934 6.0
0 -0.00840109586715698 7.0
36.698769132 39.5708198547363 0.0
0 0.0115896463394165 1.0
0 0.0365105271339417 2.0
0 -0.0163916349411011 3.0
0 0.00393229722976685 4.0
0 -0.0225992798805237 5.0
0 0.002402663230896 6.0
0 -0.00545710325241089 7.0
40.942334922 39.1316337585449 0.0
0 0.0144314765930176 1.0
0 0.0406362414360046 2.0
0 -0.0247728824615479 3.0
0 0.0245896577835083 4.0
0 -0.0239621996879578 5.0
0 0.00334888696670532 6.0
0 -0.0101450085639954 7.0
41.496981537 39.2279319763184 0.0
0 0.00682604312896729 1.0
0 0.0635049939155579 2.0
0 -0.0187178254127502 3.0
0 0.0420576930046082 4.0
0 -0.0189405679702759 5.0
0 0.00453871488571167 6.0
0 -0.015053927898407 7.0
33.577466065 37.6259269714355 0.0
0 0.00761991739273071 1.0
0 0.0568365454673767 2.0
0 -0.0344957709312439 3.0
0 0.0208529233932495 4.0
0 -0.0321889519691467 5.0
0 0.00291848182678223 6.0
0 -0.00838345289230347 7.0
38.410697595 39.4420318603516 0.0
0 0.0113469362258911 1.0
0 0.0686453580856323 2.0
0 -0.0142496228218079 3.0
0 0.00417876243591309 4.0
0 -0.0155113935470581 5.0
0 -0.000855326652526855 6.0
0 -0.00508135557174683 7.0
40.390522112 39.2720718383789 0.0
0 0.000581443309783936 1.0
0 0.0393170714378357 2.0
0 -0.0221692323684692 3.0
0 0.0482469201087952 4.0
0 -0.0264850854873657 5.0
0 0.00661689043045044 6.0
0 -0.0178845524787903 7.0
38.02724699 38.9787788391113 0.0
0 0.00272881984710693 1.0
0 0.0378914475440979 2.0
0 -0.0236902236938477 3.0
0 0.00943964719772339 4.0
0 -0.0208128094673157 5.0
0 0.00278943777084351 6.0
0 -0.00711840391159058 7.0
34.831322295 38.5845375061035 0.0
0 0.00436198711395264 1.0
0 0.0335705280303955 2.0
0 -0.0219966173171997 3.0
0 0.00454270839691162 4.0
0 -0.0286591649055481 5.0
0 0.00389528274536133 6.0
0 -0.008575439453125 7.0
33.152476832 39.2792053222656 0.0
0 0.000803530216217041 1.0
0 0.0964491367340088 2.0
0 -0.00595211982727051 3.0
0 -0.00248456001281738 4.0
0 -0.0183451175689697 5.0
0 0.000454962253570557 6.0
0 -0.00583839416503906 7.0
37.76181476 39.1988182067871 0.0
0 0.00701266527175903 1.0
0 0.0517593026161194 2.0
0 -0.0180467367172241 3.0
0 0.0260825753211975 4.0
0 -0.0297691822052002 5.0
0 0.00736707448959351 6.0
0 -0.0125783681869507 7.0
46.821650967 39.1685638427734 0.0
0 0.0124068260192871 1.0
0 0.0685530304908752 2.0
0 -0.0276700854301453 3.0
0 0.0559806823730469 4.0
0 -0.0227029323577881 5.0
0 0.00737607479095459 6.0
0 -0.0143417716026306 7.0
41.183790175 39.3380546569824 0.0
0 0.00915086269378662 1.0
0 0.043701171875 2.0
0 -0.0248683094978333 3.0
0 0.0265670418739319 4.0
0 -0.0222071409225464 5.0
0 0.00195217132568359 6.0
0 -0.0123401284217834 7.0
36.78498864 39.2007217407227 0.0
0 0.0107810497283936 1.0
0 0.0372036695480347 2.0
0 -0.0210047960281372 3.0
0 0.0306706428527832 4.0
0 -0.0301961898803711 5.0
0 0.00700825452804565 6.0
0 -0.0121517777442932 7.0
37.689494945 40.0001411437988 0.0
0 0.00271826982498169 1.0
0 0.0959305763244629 2.0
0 -0.0141395926475525 3.0
0 0.0551460385322571 4.0
0 -0.0244983434677124 5.0
0 0.00532084703445435 6.0
0 -0.0170699954032898 7.0
41.308881097 39.8917083740234 0.0
0 0.013401985168457 1.0
0 0.0613721013069153 2.0
0 -0.0210734605789185 3.0
0 0.0379018187522888 4.0
0 -0.0249822735786438 5.0
0 0.00856763124465942 6.0
0 -0.0117775201797485 7.0
42.253011184 38.8202743530273 0.0
0 0.00312644243240356 1.0
0 0.0345909595489502 2.0
0 -0.0317898988723755 3.0
0 0.0441475510597229 4.0
0 -0.0303978323936462 5.0
0 0.00695013999938965 6.0
0 -0.0131359100341797 7.0
40.047660493 41.552059173584 0.0
0 0.00905096530914307 1.0
0 0.101017355918884 2.0
0 -0.0131558179855347 3.0
0 0.0354183316230774 4.0
0 -0.018403947353363 5.0
0 0.00396114587783813 6.0
0 -0.0143160223960876 7.0
46.604605733 39.235408782959 0.0
0 0.00629884004592896 1.0
0 0.105640709400177 2.0
0 -0.015018105506897 3.0
0 0.0429240465164185 4.0
0 -0.0164434313774109 5.0
0 0.00407266616821289 6.0
0 -0.0142799615859985 7.0
45.106295542 39.4154434204102 0.0
0 0.00659686326980591 1.0
0 0.0552960634231567 2.0
0 -0.0203244686126709 3.0
0 0.0602788925170898 4.0
0 -0.0267068147659302 5.0
0 0.00430470705032349 6.0
0 -0.018653392791748 7.0
38.977055099 39.2278289794922 0.0
0 0.00663143396377563 1.0
0 0.0948184132575989 2.0
0 -0.0122199654579163 3.0
0 0.0315920114517212 4.0
0 -0.0249831080436707 5.0
0 0.00514775514602661 6.0
0 -0.0134918689727783 7.0
38.935365883 39.1499710083008 0.0
0 0.00999718904495239 1.0
0 0.0386626124382019 2.0
0 -0.0293843746185303 3.0
0 0.0103581547737122 4.0
0 -0.0220971703529358 5.0
0 0.00317615270614624 6.0
0 -0.00762468576431274 7.0
42.741409918 39.2675628662109 0.0
0 0.00993281602859497 1.0
0 0.0399733185768127 2.0
0 -0.031740128993988 3.0
0 0.0284399390220642 4.0
0 -0.0234741568565369 5.0
0 0.00261366367340088 6.0
0 -0.00984954833984375 7.0
31.310794576 39.3400192260742 0.0
0 0.0062834620475769 1.0
0 0.038298487663269 2.0
0 -0.0212872624397278 3.0
0 0.00529992580413818 4.0
0 -0.0331861972808838 5.0
0 0.00680768489837646 6.0
0 -0.00564706325531006 7.0
};
\addlegendentry{$R^2$=0.983}
\end{axis}

\end{tikzpicture}
}}
    \subfloat[Actual vs predicted edge flows.] 
    {\label{fig:results_nonlineal_dummy_base_f}\resizebox{\figurewidth}{\figureheight}{% This file was created with tikzplotlib v0.10.1.
\begin{tikzpicture}

\definecolor{darkgray176}{RGB}{176,176,176}
\definecolor{lightgray204}{RGB}{204,204,204}

\begin{axis}[
colorbar,
colorbar style={ylabel={edge_id}},
colormap={mymap}{[1pt]
 rgb(0pt)=(0.12156862745098,0.466666666666667,0.705882352941177);
  rgb(1pt)=(1,0.498039215686275,0.0549019607843137);
  rgb(2pt)=(0.172549019607843,0.627450980392157,0.172549019607843);
  rgb(3pt)=(0.83921568627451,0.152941176470588,0.156862745098039);
  rgb(4pt)=(0.580392156862745,0.403921568627451,0.741176470588235);
  rgb(5pt)=(0.549019607843137,0.337254901960784,0.294117647058824);
  rgb(6pt)=(0.890196078431372,0.466666666666667,0.76078431372549);
  rgb(7pt)=(0.498039215686275,0.498039215686275,0.498039215686275);
  rgb(8pt)=(0.737254901960784,0.741176470588235,0.133333333333333);
  rgb(9pt)=(0.0901960784313725,0.745098039215686,0.811764705882353)
},
legend cell align={left},
legend style={
  fill opacity=0.8,
  draw opacity=1,
  text opacity=1,
  at={(0.03,0.97)},
  anchor=north west,
  draw=lightgray204
},
point meta max=7,
point meta min=0,
tick align=outside,
tick pos=left,
title={ye_test-ye_pred},
x grid style={darkgray176},
xlabel={ye_test},
xmajorgrids,
xmin=-15.25776872815, xmax=51.84084685915,
xtick style={color=black},
y grid style={darkgray176},
ylabel={ye_pred},
ymajorgrids,
ymin=-14.0268879413605, ymax=52.5269442081451,
ytick style={color=black}
]
\addplot [
  colormap={mymap}{[1pt]
 rgb(0pt)=(0.12156862745098,0.466666666666667,0.705882352941177);
  rgb(1pt)=(1,0.498039215686275,0.0549019607843137);
  rgb(2pt)=(0.172549019607843,0.627450980392157,0.172549019607843);
  rgb(3pt)=(0.83921568627451,0.152941176470588,0.156862745098039);
  rgb(4pt)=(0.580392156862745,0.403921568627451,0.741176470588235);
  rgb(5pt)=(0.549019607843137,0.337254901960784,0.294117647058824);
  rgb(6pt)=(0.890196078431372,0.466666666666667,0.76078431372549);
  rgb(7pt)=(0.498039215686275,0.498039215686275,0.498039215686275);
  rgb(8pt)=(0.737254901960784,0.741176470588235,0.133333333333333);
  rgb(9pt)=(0.0901960784313725,0.745098039215686,0.811764705882353)
},
  only marks,
  scatter,
  scatter src=explicit
]
table [x=x, y=y, meta=colordata]{%
x  y  colordata
39.565898635 38.596305847168 0.0
18.050517226 19.9765090942383 1.0
21.515381409 19.5053291320801 2.0
-8.8987516789 -7.5859842300415 3.0
12.61662972 11.9246826171875 4.0
26.949268915 26.8885326385498 5.0
39.565898645 40.6012878417969 6.0
39.565898645 39.244514465332 7.0
42.743257171 38.6418075561523 0.0
22.448801828 20.5235271453857 1.0
20.294455342 20.9817905426025 2.0
-4.2455956832 -6.40148591995239 3.0
16.048859649 15.7577791213989 4.0
26.694397521 26.0581035614014 5.0
42.743257181 41.7566375732422 6.0
42.743257181 41.8745651245117 7.0
39.367180747 38.091423034668 0.0
18.970258592 19.4065704345703 1.0
20.39692216 19.8818187713623 2.0
-4.9776642323 -5.26164960861206 3.0
15.419257923 14.7249002456665 4.0
23.947922829 23.1789817810059 5.0
39.367180751 38.5191268920898 6.0
39.367180751 39.1420822143555 7.0
39.605393097 37.8412399291992 0.0
22.082745928 20.4279651641846 1.0
17.522647169 18.7483291625977 2.0
-7.2517903354 -9.21843147277832 3.0
10.270856823 9.68563938140869 4.0
29.334536274 28.5466556549072 5.0
39.605393107 39.5127182006836 6.0
39.605393107 40.0177230834961 7.0
43.937345526 38.6058807373047 0.0
21.865843593 21.8540668487549 1.0
22.071501933 20.891975402832 2.0
-8.3469765086 -7.86768054962158 3.0
13.724525414 13.045955657959 4.0
30.212820112 28.7595252990723 5.0
43.937345536 43.1537170410156 6.0
43.937345536 43.2702102661133 7.0
31.061989584 37.880500793457 0.0
16.203677385 15.8516836166382 1.0
14.858312199 15.0089178085327 2.0
-5.5704045507 -6.62269258499146 3.0
9.2879076384 9.24989891052246 4.0
21.774081945 20.9252452850342 5.0
31.061989594 31.7036800384521 6.0
31.061989594 31.6152458190918 7.0
36.357435265 40.6078720092773 0.0
19.38173709 18.5710964202881 1.0
16.975698176 16.8981056213379 2.0
-8.3424496148 -9.10635566711426 3.0
8.6332485532 8.03556060791016 4.0
27.724186714 27.9642848968506 5.0
36.357435273 35.7285537719727 6.0
36.357435273 35.4694442749023 7.0
38.444969607 38.5469589233398 0.0
21.080570389 19.3154773712158 1.0
17.364399218 18.5943260192871 2.0
-5.0980379196 -7.51685047149658 3.0
12.266361289 11.6779813766479 4.0
26.178608318 25.8334617614746 5.0
38.444969617 38.1791000366211 6.0
38.444969617 38.3308563232422 7.0
35.498620518 38.3295974731445 0.0
17.900031823 17.3579730987549 1.0
17.598588698 18.0279216766357 2.0
-4.651704216 -6.18915939331055 3.0
12.946884476 12.3227348327637 4.0
22.551736046 22.3548603057861 5.0
35.498620524 35.0162658691406 6.0
35.498620524 35.7486877441406 7.0
36.52099827 39.1421966552734 0.0
18.961240079 18.6394557952881 1.0
17.559758192 18.268310546875 2.0
-5.3314768445 -6.87822914123535 3.0
12.228281339 11.798867225647 4.0
24.292716932 24.457181930542 5.0
36.520998279 36.9719772338867 6.0
36.520998279 36.2696228027344 7.0
36.717272204 38.9959335327148 0.0
19.763035348 19.2362728118896 1.0
16.954236857 18.1576652526855 2.0
-7.0689091682 -8.69227981567383 3.0
9.88532768 9.23490810394287 4.0
26.831944525 26.86110496521 5.0
36.717272212 37.3186569213867 6.0
36.717272212 37.7003936767578 7.0
32.629628996 38.8705673217773 0.0
17.103127104 16.7790718078613 1.0
15.526501892 15.2252159118652 2.0
-7.1999965109 -8.01188087463379 3.0
8.3265053708 8.09377098083496 4.0
24.303123625 24.2568454742432 5.0
32.629629006 33.6453323364258 6.0
32.629629006 33.813346862793 7.0
37.75267433 38.2975540161133 0.0
17.533699782 18.3296546936035 1.0
20.218974548 19.5139389038086 2.0
-3.919745576 -4.89840984344482 3.0
16.299228962 15.4050731658936 4.0
21.453445368 20.1247844696045 5.0
37.75267434 37.2333374023438 6.0
37.75267434 37.7951812744141 7.0
38.800291337 38.8440856933594 0.0
21.414385846 19.40407371521 1.0
17.385905491 18.7268810272217 2.0
-6.3164463541 -8.4114408493042 3.0
11.069459127 10.6674652099609 4.0
27.73083221 27.2714900970459 5.0
38.800291347 37.6164703369141 6.0
38.800291347 38.5971755981445 7.0
38.252729609 38.4748001098633 0.0
20.199036122 19.4157733917236 1.0
18.053693488 18.6936721801758 2.0
-6.9495443437 -8.02486419677734 3.0
11.104149136 10.5527362823486 4.0
27.148580475 27.2325859069824 5.0
38.252729618 38.1705474853516 6.0
38.252729618 38.0124740600586 7.0
43.273596037 40.0211029052734 0.0
20.811072897 22.2805271148682 1.0
22.46252314 20.9060554504395 2.0
-10.185324283 -9.22338104248047 3.0
12.277198847 11.4580049514771 4.0
30.99639719 30.8943367004395 5.0
43.273596047 43.1003189086914 6.0
43.273596047 43.7081604003906 7.0
34.027431477 37.5528106689453 0.0
16.463058834 17.5791893005371 1.0
17.564372643 16.3833618164062 2.0
-7.8238749469 -7.02876377105713 3.0
9.7404976863 9.56681632995605 4.0
24.28693379 23.8013954162598 5.0
34.027431486 34.5615615844727 6.0
34.027431486 33.9703750610352 7.0
41.154171382 38.8431701660156 0.0
20.969763 20.984806060791 1.0
20.184408383 19.1587142944336 2.0
-9.8031418559 -9.94899749755859 3.0
10.381266519 9.61151504516602 4.0
30.772904865 29.7687873840332 5.0
41.154171391 41.1792373657227 6.0
41.154171391 40.7767715454102 7.0
39.9308264078757 38.3818206787109 0.0
21.2554356118761 20.7513122558594 1.0
18.6753908028755 19.0751094818115 2.0
-8.41978314337461 -9.43831157684326 3.0
10.2556076588756 9.64951229095459 4.0
29.6752187558758 28.8787422180176 5.0
39.930826408 39.7014236450195 6.0
39.930826408 40.0568618774414 7.0
45.969703622 38.6865692138672 0.0
25.855804141 22.8934497833252 1.0
20.113899483 22.2165870666504 2.0
-4.729541314 -7.56897640228271 3.0
15.38435816 14.5867805480957 4.0
30.585345464 29.0078372955322 5.0
45.969703631 44.1232299804688 6.0
45.969703631 45.2887420654297 7.0
38.398120212 38.6374282836914 0.0
19.546425793 19.6057834625244 1.0
18.851694421 18.1635780334473 2.0
-9.8922695857 -9.75459480285645 3.0
8.9594248259 8.33925437927246 4.0
29.438695387 28.486421585083 5.0
38.398120221 38.1621627807617 6.0
38.398120221 38.5771560668945 7.0
28.366017296 39.6057739257812 0.0
14.995271371 14.1404008865356 1.0
13.370745924 13.0689744949341 2.0
-4.4280592982 -6.7935733795166 3.0
8.9426866162 8.9682502746582 4.0
19.42333068 19.1228618621826 5.0
28.366017306 29.9161052703857 6.0
28.366017306 28.6994247436523 7.0
39.07981897 37.8760223388672 0.0
18.202914308 19.484001159668 1.0
20.876904662 19.5376033782959 2.0
-6.6281228283 -6.22390174865723 3.0
14.248781824 13.701379776001 4.0
24.831037146 24.1412754058838 5.0
39.079818979 38.8075942993164 6.0
39.079818979 39.393684387207 7.0
40.466329374 38.3118438720703 0.0
19.842708002 19.766731262207 1.0
20.623621372 19.9568691253662 2.0
-6.0436252005 -6.6066198348999 3.0
14.579996162 13.9918174743652 4.0
25.886333212 24.9481201171875 5.0
40.466329383 39.8389892578125 6.0
40.466329383 40.9878768920898 7.0
38.293116843 37.9650497436523 0.0
18.225197444 18.1354637145996 1.0
20.067919399 19.4418029785156 2.0
-4.6459628917 -5.61400127410889 3.0
15.421956498 15.0479040145874 4.0
22.871160346 22.6656131744385 5.0
38.293116853 37.6979522705078 6.0
38.293116853 37.6028518676758 7.0
43.346089487 38.575080871582 0.0
21.23189324 21.8341217041016 1.0
22.114196248 20.7341327667236 2.0
-9.387268107 -8.84220218658447 3.0
12.726928131 11.9266443252563 4.0
30.619161357 29.7657775878906 5.0
43.346089497 42.8919219970703 6.0
43.346089497 43.2442321777344 7.0
34.547871144 38.1383056640625 0.0
19.900789633 18.2591228485107 1.0
14.647081512 16.0518455505371 2.0
-6.6924528176 -8.45735645294189 3.0
7.9546286844 7.69030857086182 4.0
26.59324246 26.038480758667 5.0
34.547871154 34.2125091552734 6.0
34.547871154 34.9699249267578 7.0
41.927050143208 38.3602600097656 0.0
21.8340017142739 21.933032989502 1.0
20.0930484299516 19.568078994751 2.0
-12.0156472013624 -10.9687099456787 3.0
8.07740122783537 8.01268196105957 4.0
33.8496486756856 32.4716110229492 5.0
41.927050143 41.8528671264648 6.0
41.927050143 42.3923263549805 7.0
44.548236367 38.2497406005859 0.0
22.097953544 21.9802989959717 1.0
22.450282823 21.7488422393799 2.0
-6.9116853021 -6.82119369506836 3.0
15.538597511 15.3108968734741 4.0
29.009638856 27.8976154327393 5.0
44.548236377 43.4468307495117 6.0
44.548236377 43.7278213500977 7.0
34.958415477 39.1437835693359 0.0
19.046265323 17.0440578460693 1.0
15.912150156 17.5630054473877 2.0
-3.947719982 -6.61912107467651 3.0
11.964430164 11.5081748962402 4.0
22.993985314 22.4576072692871 5.0
34.958415487 34.7578430175781 6.0
34.958415487 35.6662673950195 7.0
41.619773288 38.0942611694336 0.0
18.036088777 20.262638092041 1.0
23.583684511 20.2230663299561 2.0
-9.1140217471 -7.23081398010254 3.0
14.469662754 13.9196548461914 4.0
27.150110534 27.2458477020264 5.0
41.619773298 39.2246780395508 6.0
41.619773298 40.2623062133789 7.0
35.768623904 38.4565124511719 0.0
18.854929214 18.1294422149658 1.0
16.913694692 17.2556228637695 2.0
-6.8203969536 -8.47326946258545 3.0
10.093297731 9.59994029998779 4.0
25.675326176 25.5502910614014 5.0
35.768623912 36.0705490112305 6.0
35.768623912 36.3270874023438 7.0
36.035873071 36.6593170166016 0.0
18.197202866 18.8770313262939 1.0
17.838670207 17.1606254577637 2.0
-9.8802007315 -9.52300262451172 3.0
7.958469467 7.84091186523438 4.0
28.077403606 28.0217590332031 5.0
36.03587308 35.13916015625 6.0
36.03587308 36.5958099365234 7.0
42.671159575 38.0124359130859 0.0
22.542382477 21.8425483703613 1.0
20.128777099 20.0145320892334 2.0
-9.3171954196 -10.1334800720215 3.0
10.811581671 10.0967769622803 4.0
31.859577906 31.1639347076416 5.0
42.671159584 41.8028335571289 6.0
42.671159584 42.3421096801758 7.0
32.270518381 38.76953125 0.0
16.341237488 17.2153034210205 1.0
15.929280893 15.8293142318726 2.0
-7.2560391874 -7.5936803817749 3.0
8.6732416955 8.44317626953125 4.0
23.597276685 23.5255069732666 5.0
32.270518391 32.6548080444336 6.0
32.270518391 33.550910949707 7.0
36.239834932 41.2490692138672 0.0
18.511196273 18.1738815307617 1.0
17.72863866 18.0583553314209 2.0
-5.3875241311 -7.16932010650635 3.0
12.34111452 11.7859210968018 4.0
23.898720413 23.8287391662598 5.0
36.239834941 35.6087036132812 6.0
36.239834941 35.4636993408203 7.0
38.292005172 37.9891510009766 0.0
20.144840979 19.4964828491211 1.0
18.147164195 18.4614276885986 2.0
-6.9715526507 -7.96281623840332 3.0
11.175611535 10.6878089904785 4.0
27.116393638 26.6137714385986 5.0
38.292005181 38.5519485473633 6.0
38.292005181 39.5151824951172 7.0
41.142171116 39.3999252319336 0.0
21.697070846 20.3426818847656 1.0
19.44510027 20.4383220672607 2.0
-5.49061917 -7.43276405334473 3.0
13.95448109 13.2350759506226 4.0
27.187690026 26.9425754547119 5.0
41.142171126 40.9181289672852 6.0
41.142171126 41.5554046630859 7.0
30.660234741 37.4738845825195 0.0
16.571808951 15.7378721237183 1.0
14.08842579 14.9877004623413 2.0
-3.6405774968 -5.70759725570679 3.0
10.447848283 10.1889791488647 4.0
20.212386458 19.7445640563965 5.0
30.660234751 32.1174278259277 6.0
30.660234751 32.041820526123 7.0
42.776716976 39.9827499389648 0.0
18.943122536 21.1849422454834 1.0
23.83359444 20.9559173583984 2.0
-9.5992659914 -7.79702854156494 3.0
14.234328438 13.7019624710083 4.0
28.542388537 28.1028633117676 5.0
42.776716986 41.7659683227539 6.0
42.776716986 42.5332946777344 7.0
39.136657948 38.8526840209961 0.0
21.244730642 19.9374752044678 1.0
17.891927309 19.648380279541 2.0
-5.2069844714 -7.2206449508667 3.0
12.684942831 11.9592514038086 4.0
26.45171512 26.3674869537354 5.0
39.136657955 37.942253112793 6.0
39.136657955 38.3528442382812 7.0
40.593075656 38.2235794067383 0.0
19.181149666 21.0331974029541 1.0
21.411925992 19.1207962036133 2.0
-11.788434371 -9.97361087799072 3.0
9.6234916129 9.05599689483643 4.0
30.969584045 30.2422485351562 5.0
40.593075664 39.523551940918 6.0
40.593075664 40.1297454833984 7.0
38.455960047 38.7767868041992 0.0
19.606731652 20.1121311187744 1.0
18.849228395 18.2991676330566 2.0
-10.066988594 -9.78773784637451 3.0
8.7822397903 8.13940715789795 4.0
29.673720256 29.5408210754395 5.0
38.455960057 38.0539016723633 6.0
38.455960057 38.3448028564453 7.0
40.029822299 38.7315826416016 0.0
19.696760735 20.0480403900146 1.0
20.333061566 19.6599903106689 2.0
-6.9313004298 -7.38136005401611 3.0
13.401761128 12.8033199310303 4.0
26.628061173 25.8108005523682 5.0
40.029822307 39.3719940185547 6.0
40.029822307 40.4371185302734 7.0
39.721089796 38.3653564453125 0.0
21.982867862 20.3135223388672 1.0
17.738221933 19.1263008117676 2.0
-6.2167396454 -8.72383594512939 3.0
11.521482278 11.0404500961304 4.0
28.199607518 27.3751811981201 5.0
39.721089806 40.900260925293 6.0
39.721089806 39.7707290649414 7.0
46.012802771 38.5219421386719 0.0
23.852238991 22.91357421875 1.0
22.16056378 21.7913818359375 2.0
-9.7843125098 -10.1084747314453 3.0
12.37625126 11.3134632110596 4.0
33.636551511 32.5210723876953 5.0
46.012802781 45.08203125 6.0
46.012802781 46.1660232543945 7.0
43.79104115 38.6123657226562 0.0
23.810489927 22.027738571167 1.0
19.980551224 20.673770904541 2.0
-8.8972279263 -10.03688621521 3.0
11.083323289 9.92902183532715 4.0
32.707717862 31.3464870452881 5.0
43.791041158 42.2916946411133 6.0
43.791041158 43.6183166503906 7.0
31.257332414 38.2450332641602 0.0
14.677686479 15.6956052780151 1.0
16.579645934 14.9001684188843 2.0
-7.7653359569 -7.08528423309326 3.0
8.8143099674 8.48609733581543 4.0
22.443022446 22.779369354248 5.0
31.257332424 30.916130065918 6.0
31.257332424 30.941858291626 7.0
38.988472902 38.1238403320312 0.0
18.706030748 19.4086837768555 1.0
20.282442155 19.8146343231201 2.0
-4.4931154738 -4.96176481246948 3.0
15.789326673 14.8181753158569 4.0
23.19914623 22.3707904815674 5.0
38.98847291 39.0290298461914 6.0
38.98847291 39.5676727294922 7.0
38.691218489 38.2635803222656 0.0
19.181746937 18.3563499450684 1.0
19.509471553 18.8885822296143 2.0
-4.305791265 -6.0208797454834 3.0
15.203680278 14.929160118103 4.0
23.487538212 22.4304141998291 5.0
38.691218499 37.3509140014648 6.0
38.691218499 38.6561508178711 7.0
39.033211962 37.2093505859375 0.0
21.751284426 20.4594287872314 1.0
17.281927536 18.6953716278076 2.0
-7.1228573229 -9.43244361877441 3.0
10.159070204 9.23776054382324 4.0
28.874141758 28.742280960083 5.0
39.033211971 38.8525695800781 6.0
39.033211971 39.3592529296875 7.0
37.697547803 38.6268463134766 0.0
18.641388049 18.8006534576416 1.0
19.056159754 18.699592590332 2.0
-6.2754402376 -6.98374080657959 3.0
12.780719506 12.4110231399536 4.0
24.916828297 24.5393657684326 5.0
37.697547813 36.8487167358398 6.0
37.697547813 37.9475479125977 7.0
35.277541329 38.1812057495117 0.0
16.862419132 17.8773555755615 1.0
18.415122198 17.9661350250244 2.0
-6.2131742864 -6.36807012557983 3.0
12.201947903 11.7862014770508 4.0
23.075593428 23.2379150390625 5.0
35.277541339 34.7342910766602 6.0
35.277541339 36.3307418823242 7.0
36.1197639664031 38.3135223388672 0.0
19.1517629624013 17.7718601226807 1.0
16.9680010143999 18.865629196167 2.0
-0.473187581125011 -3.97485685348511 3.0
16.4948134334014 14.9833126068115 4.0
19.6249505434022 19.2428321838379 5.0
36.119763966 36.810417175293 6.0
36.119763966 37.0653457641602 7.0
33.14490304 38.2044525146484 0.0
16.831426445 16.1671714782715 1.0
16.313476595 16.9022331237793 2.0
-3.637680404 -4.73959493637085 3.0
12.675796181 12.6736574172974 4.0
20.469106859 20.0596942901611 5.0
33.14490305 33.2030410766602 6.0
33.14490305 32.5096092224121 7.0
34.4868008142505 38.2473602294922 0.0
18.3228058422505 18.7227478027344 1.0
16.1639949822505 16.799503326416 2.0
-8.00476657755046 -8.97555541992188 3.0
8.15922840445046 7.85224437713623 4.0
26.3275723771958 27.5033168792725 5.0
34.486800814 34.5710830688477 6.0
34.486800814 35.8238296508789 7.0
40.933468199 38.9690628051758 0.0
20.889274427 20.8757381439209 1.0
20.044193774 20.2090339660645 2.0
-6.533500242 -7.47461318969727 3.0
13.510693524 12.9050893783569 4.0
27.422774677 27.0413875579834 5.0
40.933468207 41.4710693359375 6.0
40.933468207 40.4933700561523 7.0
35.993417919 38.3794860839844 0.0
18.964188614 18.5946922302246 1.0
17.029229306 17.2728900909424 2.0
-7.4575389686 -8.26856327056885 3.0
9.5716903284 9.30193614959717 4.0
26.421727592 26.5541229248047 5.0
35.993417928 36.0943832397461 6.0
35.993417928 36.9887390136719 7.0
39.331666914 38.3532333374023 0.0
19.408314017 19.5726890563965 1.0
19.923352898 19.2500877380371 2.0
-7.2428639103 -7.33345890045166 3.0
12.680488979 12.0278520584106 4.0
26.651177937 25.8608589172363 5.0
39.331666923 39.1381759643555 6.0
39.331666923 39.3521347045898 7.0
37.559548486 38.3558502197266 0.0
18.054642609 19.0616474151611 1.0
19.504905877 18.4792575836182 2.0
-8.2806473226 -8.43041896820068 3.0
11.224258545 10.6311960220337 4.0
26.335289942 26.4604682922363 5.0
37.559548496 37.2250823974609 6.0
37.559548496 37.6933441162109 7.0
41.796902472 38.2650909423828 0.0
20.074580702 21.5512237548828 1.0
21.72232177 19.6002368927002 2.0
-11.24258785 -9.91202449798584 3.0
10.479733911 9.6037425994873 4.0
31.317168562 30.1995830535889 5.0
41.796902482 41.5065841674805 6.0
41.796902482 41.99755859375 7.0
35.679590813 38.4417953491211 0.0
15.932888769 17.7633152008057 1.0
19.746702044 17.7697124481201 2.0
-7.4161334139 -6.28019094467163 3.0
12.330568621 12.0021047592163 4.0
23.349022193 23.0090293884277 5.0
35.679590823 35.9085006713867 6.0
35.679590823 36.6312637329102 7.0
33.227547284 37.9404754638672 0.0
19.156810973 17.3763046264648 1.0
14.070736312 15.8583564758301 2.0
-5.2205843651 -6.91761207580566 3.0
8.8501519384 8.57953262329102 4.0
24.377395346 24.0266704559326 5.0
33.227547292 32.8059196472168 6.0
33.227547292 33.2664070129395 7.0
28.00807173 38.808479309082 0.0
12.909757219 14.271240234375 1.0
15.098314511 14.2345685958862 2.0
-4.3989930962 -3.41055345535278 3.0
10.699321405 10.7542247772217 4.0
17.308750325 17.332633972168 5.0
28.008071739 29.8767280578613 6.0
28.008071739 29.2562599182129 7.0
33.4784988410636 39.5744171142578 0.0
17.4596045520637 17.5532684326172 1.0
16.0188942990637 16.7638511657715 2.0
-6.65519022176347 -7.79201221466064 3.0
9.36370407706344 8.58208847045898 4.0
24.1147947730637 23.8467273712158 5.0
33.478498841 34.1995620727539 6.0
33.478498841 32.6516494750977 7.0
33.1229468201162 38.7150039672852 0.0
16.2738036511163 16.6416358947754 1.0
16.8491431771162 15.7469615936279 2.0
-8.48238354211593 -8.02114486694336 3.0
8.36675963501607 8.30586051940918 4.0
24.7561871931169 24.5005741119385 5.0
33.12294682 33.6349487304688 6.0
33.12294682 33.6500625610352 7.0
34.970228375 38.9385833740234 0.0
17.461716438 17.4915294647217 1.0
17.508511937 18.2084255218506 2.0
-3.5027478521 -4.3347749710083 3.0
14.005764076 13.2965602874756 4.0
20.964464299 20.1009483337402 5.0
34.970228384 35.6356582641602 6.0
34.970228384 35.9989166259766 7.0
36.637966033 38.3001022338867 0.0
18.611959282 17.4169902801514 1.0
18.026006753 17.929515838623 2.0
-5.1789935919 -6.43232107162476 3.0
12.847013153 12.1779184341431 4.0
23.790952882 23.1638679504395 5.0
36.637966041 36.7235488891602 6.0
36.637966041 36.484733581543 7.0
38.712447285 38.7288208007812 0.0
19.874849154 20.1867618560791 1.0
18.837598131 18.1591987609863 2.0
-9.656315418 -9.04708290100098 3.0
9.181282703 8.90895366668701 4.0
29.531164582 29.003870010376 5.0
38.712447295 38.2031784057617 6.0
38.712447295 39.4380950927734 7.0
29.0840224199388 38.1819839477539 0.0
12.9348822379388 14.5154609680176 1.0
16.1491401919388 14.3640298843384 2.0
-6.2158408037388 -5.854407787323 3.0
9.93329938853882 9.7581787109375 4.0
19.1507230409388 19.3183727264404 5.0
29.08402242 29.7144298553467 6.0
29.08402242 29.3537063598633 7.0
40.7419030105752 38.7526321411133 0.0
19.6019574718567 20.2419586181641 1.0
21.139945535663 20.0191478729248 2.0
-8.29196876971742 -8.19618988037109 3.0
12.8479767652648 12.1341876983643 4.0
27.8939262422712 28.183198928833 5.0
40.741903011 40.4545211791992 6.0
40.741903011 40.4759216308594 7.0
44.4239279780974 38.7700424194336 0.0
22.6860882760974 21.8470287322998 1.0
21.7378397120974 21.6257266998291 2.0
-6.03315836089742 -7.33268356323242 3.0
15.7046813510974 14.6819858551025 4.0
28.7192466370973 27.9537601470947 5.0
44.423927978 43.4490966796875 6.0
44.423927978 44.0868301391602 7.0
37.279929002 40.722297668457 0.0
17.697440733 18.6612033843994 1.0
19.582488269 17.9926910400391 2.0
-8.4860638308 -8.0453052520752 3.0
11.096424428 10.6737976074219 4.0
26.183504574 26.7601871490479 5.0
37.279929011 35.4391937255859 6.0
37.279929011 35.7633285522461 7.0
39.065196556 38.2746047973633 0.0
22.911627527 20.3264713287354 1.0
16.153569029 18.2511138916016 2.0
-7.035333012 -9.65429401397705 3.0
9.1182360067 8.52085018157959 4.0
29.946960549 29.2601470947266 5.0
39.065196566 38.0067367553711 6.0
39.065196566 38.2126693725586 7.0
34.346712901 38.617561340332 0.0
17.419062718 16.6762275695801 1.0
16.927650183 17.3127841949463 2.0
-3.8964434363 -5.28463315963745 3.0
13.031206737 12.3719930648804 4.0
21.315506165 21.3649024963379 5.0
34.346712911 34.0277557373047 6.0
34.346712911 34.495849609375 7.0
44.832836192 37.8687896728516 0.0
22.86238329 21.5256099700928 1.0
21.970452902 21.2041835784912 2.0
-8.0950157692 -8.58045959472656 3.0
13.875437123 13.1890459060669 4.0
30.957399069 29.8986225128174 5.0
44.832836202 43.7193450927734 6.0
44.832836202 43.844123840332 7.0
37.693005907 37.8074417114258 0.0
18.767233493 18.4735507965088 1.0
18.925772415 18.8588104248047 2.0
-4.4039460583 -4.7475790977478 3.0
14.521826348 14.3328046798706 4.0
23.17117956 22.4735546112061 5.0
37.693005916 37.4263381958008 6.0
37.693005916 37.8930740356445 7.0
35.516030196 38.2577285766602 0.0
17.238388415 17.4624156951904 1.0
18.277641781 17.627254486084 2.0
-5.9879556024 -6.29086351394653 3.0
12.289686168 11.9834842681885 4.0
23.226344027 22.9670333862305 5.0
35.516030206 35.4134750366211 6.0
35.516030206 36.0930252075195 7.0
45.449957515 36.9237747192383 0.0
24.470260824 22.9747180938721 1.0
20.979696692 21.4246616363525 2.0
-8.6547786371 -10.2861642837524 3.0
12.324918046 11.495792388916 4.0
33.12503947 32.4793243408203 5.0
45.449957523 45.0430755615234 6.0
45.449957523 44.0608062744141 7.0
36.522103961 38.1218719482422 0.0
17.787809101 17.765209197998 1.0
18.734294862 18.7959289550781 2.0
-3.7955939001 -4.28736639022827 3.0
14.938700953 14.4545946121216 4.0
21.583403009 20.6723518371582 5.0
36.52210397 35.9171676635742 6.0
36.52210397 36.7773666381836 7.0
40.809333824 38.1619262695312 0.0
21.825363738 20.3745574951172 1.0
18.983970087 20.2993831634521 2.0
-4.5096782574 -6.41374826431274 3.0
14.474291821 13.794828414917 4.0
26.335042004 25.4069137573242 5.0
40.809333833 40.2103576660156 6.0
40.809333833 40.7029113769531 7.0
43.708815015 39.1823577880859 0.0
22.06122806 21.4146289825439 1.0
21.647586954 21.0050201416016 2.0
-7.2998869135 -8.47407913208008 3.0
14.347700031 13.7081642150879 4.0
29.361114984 29.4641246795654 5.0
43.708815025 43.1477203369141 6.0
43.708815025 43.0359878540039 7.0
35.746307566 38.1672134399414 0.0
17.418825391 18.5653247833252 1.0
18.327482175 17.4033260345459 2.0
-8.2684434585 -7.56141662597656 3.0
10.059038708 9.9279899597168 4.0
25.687268859 25.0303840637207 5.0
35.746307575 37.0441131591797 6.0
35.746307575 36.9562149047852 7.0
32.565260397 38.3837738037109 0.0
15.347203209 16.2376251220703 1.0
17.218057188 16.839334487915 2.0
-4.4968959364 -4.29552936553955 3.0
12.721161242 12.2814855575562 4.0
19.844099155 19.8660144805908 5.0
32.565260407 33.5784378051758 6.0
32.565260407 34.2695922851562 7.0
37.459437559 39.1984710693359 0.0
20.974106733 19.2715492248535 1.0
16.485330826 17.0758037567139 2.0
-7.8025517563 -9.46932220458984 3.0
8.6827790601 8.27703666687012 4.0
28.776658499 28.5797557830811 5.0
37.459437569 36.4917373657227 6.0
37.459437569 37.0200958251953 7.0
41.6008687772188 37.8575820922852 0.0
18.8099504026802 19.9241485595703 1.0
22.7909183777866 20.1553192138672 2.0
-6.50941037372392 -5.38870763778687 3.0
16.2815080036091 15.0416536331177 4.0
25.319360776445 23.9512557983398 5.0
41.600868777 39.8447265625 6.0
41.600868777 40.6090469360352 7.0
43.08864828 42.9987716674805 0.0
20.75366021 23.2980003356934 1.0
22.334988069 22.0826263427734 2.0
-9.5492158301 -9.45747566223145 3.0
12.785772229 11.5652866363525 4.0
30.30287605 32.2585906982422 5.0
43.088648289 45.6551971435547 6.0
43.088648289 46.1906280517578 7.0
30.268152668 37.1409225463867 0.0
15.014233699 15.0411100387573 1.0
15.25391897 14.9551963806152 2.0
-5.5931438946 -5.8237099647522 3.0
9.6607750663 9.57966232299805 4.0
20.607377602 20.4172859191895 5.0
30.268152677 30.5712718963623 6.0
30.268152677 30.7838001251221 7.0
38.454045879 37.2824249267578 0.0
17.159579151 18.6700572967529 1.0
21.29446673 18.9856376647949 2.0
-7.5807931052 -5.8023157119751 3.0
13.713673616 13.2957019805908 4.0
24.740372265 23.9335842132568 5.0
38.454045888 38.1710433959961 6.0
38.454045888 38.3021545410156 7.0
36.654056854 38.5471496582031 0.0
19.439010257 18.9937343597412 1.0
17.215046598 17.5202980041504 2.0
-7.9864590963 -8.61619663238525 3.0
9.2285874923 8.71558570861816 4.0
27.425469363 27.1988697052002 5.0
36.654056864 36.5646362304688 6.0
36.654056864 37.8646697998047 7.0
36.990278119 38.5366821289062 0.0
18.644416453 18.7880783081055 1.0
18.345861667 17.3624744415283 2.0
-7.9977544937 -7.63241291046143 3.0
10.348107164 9.8026123046875 4.0
26.642170956 25.8676433563232 5.0
36.990278128 36.491813659668 6.0
36.990278128 37.627555847168 7.0
39.45539178 38.8359832763672 0.0
17.953554537 19.0125274658203 1.0
21.501837243 18.6644802093506 2.0
-6.7799927019 -5.81520938873291 3.0
14.721844531 14.5281772613525 4.0
24.733547249 23.6711483001709 5.0
39.45539179 38.500602722168 6.0
39.45539179 39.6684875488281 7.0
41.490098301 38.646110534668 0.0
18.418139969 20.1088981628418 1.0
23.071958334 20.5167922973633 2.0
-6.7020850264 -5.84434700012207 3.0
16.369873299 15.1859064102173 4.0
25.120225004 24.2188415527344 5.0
41.49009831 40.7906951904297 6.0
41.49009831 40.9147186279297 7.0
42.195848756 40.1018142700195 0.0
18.431626231 21.4116287231445 1.0
23.764222526 20.5440311431885 2.0
-9.6629820221 -7.10689353942871 3.0
14.101240496 13.4462242126465 4.0
28.094608262 27.11061668396 5.0
42.195848764 42.3811187744141 6.0
42.195848764 41.2984848022461 7.0
36.580423947 37.9035873413086 0.0
18.088597645 21.2339611053467 1.0
18.491826302 19.0799865722656 2.0
-10.331066357 -9.98916530609131 3.0
8.1607599346 7.98917293548584 4.0
28.419664012 30.2720394134521 5.0
36.580423956 39.9783248901367 6.0
36.580423956 40.3662643432617 7.0
41.112612837 38.6265029907227 0.0
21.264748673 21.0978622436523 1.0
19.847864165 19.9049644470215 2.0
-7.8539892402 -8.99751663208008 3.0
11.993874916 11.0752277374268 4.0
29.118737922 29.3703079223633 5.0
41.112612846 39.4206314086914 6.0
41.112612846 39.5524520874023 7.0
41.165032295 38.2608489990234 0.0
20.777866565 21.3963184356689 1.0
20.387165729 19.1337871551514 2.0
-11.013080697 -10.121452331543 3.0
9.3740850227 8.92656707763672 4.0
31.790947272 30.5620822906494 5.0
41.165032305 40.1152267456055 6.0
41.165032305 41.0985946655273 7.0
28.23632976 38.1864700317383 0.0
14.450875753 14.5610046386719 1.0
13.785454008 13.6108846664429 2.0
-5.4067157656 -6.53684329986572 3.0
8.3787382334 8.26820278167725 4.0
19.857591527 20.2184524536133 5.0
28.236329769 29.2296028137207 6.0
28.236329769 29.4686431884766 7.0
46.563722225 38.1880493164062 0.0
24.514076121 22.7633590698242 1.0
22.049646104 21.9152851104736 2.0
-6.9458736065 -7.58726024627686 3.0
15.103772488 14.7064065933228 4.0
31.459949737 29.7712516784668 5.0
46.563722235 44.9567184448242 6.0
46.563722235 45.9777069091797 7.0
38.976039438 34.1136169433594 0.0
20.47408062 18.8379173278809 1.0
18.50195882 19.1683197021484 2.0
-3.6791774969 -6.09064865112305 3.0
14.822781314 14.9086608886719 4.0
24.153258125 23.3799839019775 5.0
38.976039446 39.2413177490234 6.0
38.976039446 39.4658050537109 7.0
43.902220155 39.1786575317383 0.0
23.085534006 23.4488563537598 1.0
20.816686149 22.4787120819092 2.0
-6.0778252495 -9.12560844421387 3.0
14.738860889 14.0046243667603 4.0
29.163359266 31.5139141082764 5.0
43.902220165 46.4647827148438 6.0
43.902220165 47.53662109375 7.0
37.104870377 37.707893371582 0.0
21.928667997 19.5249156951904 1.0
15.176202381 17.5614700317383 2.0
-7.1829261037 -9.80313682556152 3.0
7.9932762681 7.98862075805664 4.0
29.11159411 28.3439083099365 5.0
37.104870385 37.2066040039062 6.0
37.104870385 37.9261703491211 7.0
48.094555135036 38.6682662963867 0.0
24.9788607525581 25.1454029083252 1.0
23.1156943831877 22.4715099334717 2.0
-7.58647576369851 -7.68739700317383 3.0
15.5292186191899 14.3461151123047 4.0
32.5653365171039 30.5723094940186 5.0
48.094555135 44.4077377319336 6.0
48.094555135 46.8743133544922 7.0
39.863550950908 38.1398620605469 0.0
20.2748443148297 20.5085353851318 1.0
19.5887066378693 18.5262336730957 2.0
-9.95882058391343 -9.65867233276367 3.0
9.62988605424657 9.29629039764404 4.0
30.233664897769 29.5470676422119 5.0
39.863550951 39.6430892944336 6.0
39.863550951 40.0736465454102 7.0
38.158424696 37.6681747436523 0.0
21.093616444 22.0732917785645 1.0
17.064808252 20.5325775146484 2.0
-6.6865614213 -10.2466697692871 3.0
10.378246821 9.59789752960205 4.0
27.780177875 31.4843349456787 5.0
38.158424706 42.7158203125 6.0
38.158424706 43.3326950073242 7.0
37.671552635 38.2944183349609 0.0
18.060105754 18.5829219818115 1.0
19.611446882 18.7748870849609 2.0
-6.6868904195 -6.5537691116333 3.0
12.924556453 12.4624843597412 4.0
24.746996183 24.1268634796143 5.0
37.671552644 36.9315948486328 6.0
37.671552644 38.0746917724609 7.0
38.9900068701748 38.3024597167969 0.0
19.4317892501748 19.9019870758057 1.0
19.5582176301748 19.5858058929443 2.0
-6.7795985865748 -7.9151029586792 3.0
12.7786190431748 12.0273065567017 4.0
26.2113877989032 26.9956111907959 5.0
38.99000687 39.3139495849609 6.0
38.99000687 38.7906112670898 7.0
32.824279187 38.8112869262695 0.0
17.524252685 16.0635318756104 1.0
15.300026502 15.6275568008423 2.0
-5.6991008156 -6.74575519561768 3.0
9.6009256767 9.46850776672363 4.0
23.223353511 23.2582511901855 5.0
32.824279197 32.8469848632812 6.0
32.824279197 32.7363967895508 7.0
39.752392388 38.4769897460938 0.0
19.337645943 19.1063632965088 1.0
20.414746445 19.9582710266113 2.0
-4.1151033929 -5.25276899337769 3.0
16.299643042 15.4891633987427 4.0
23.452749346 22.6432285308838 5.0
39.752392398 39.1198425292969 6.0
39.752392398 39.4828796386719 7.0
36.0149979881443 38.0255355834961 0.0
19.5282530962186 17.835147857666 1.0
16.4867448931817 17.4042663574219 2.0
-5.97886461861353 -7.98457145690918 3.0
10.5078802751384 9.96859836578369 4.0
25.5071177152005 25.1389141082764 5.0
36.014997988 35.8599395751953 6.0
36.014997988 36.1396865844727 7.0
42.307205596 38.211555480957 0.0
21.688253633 20.0025787353516 1.0
20.618951963 20.7712383270264 2.0
-4.4760898368 -5.28428745269775 3.0
16.142862116 15.6773233413696 4.0
26.164343479 25.2863941192627 5.0
42.307205606 40.986946105957 6.0
42.307205606 41.2777633666992 7.0
41.624369925 38.2369766235352 0.0
21.269456313 21.4407863616943 1.0
20.354913614 20.0920639038086 2.0
-9.726585677 -9.24065017700195 3.0
10.628327928 9.91324234008789 4.0
30.996041999 29.5769729614258 5.0
41.624369934 39.8592681884766 6.0
41.624369934 41.3274841308594 7.0
40.927860916 37.5762405395508 0.0
20.558611306 20.8498229980469 1.0
20.36924961 19.8419551849365 2.0
-7.6689328495 -7.97664070129395 3.0
12.700316751 12.0926990509033 4.0
28.227544165 27.9054050445557 5.0
40.927860926 39.9788818359375 6.0
40.927860926 40.8771896362305 7.0
47.441321803 38.3906173706055 0.0
22.43048805 23.1683750152588 1.0
25.010833753 22.8570461273193 2.0
-8.9760978683 -7.52433204650879 3.0
16.034735875 15.56809425354 4.0
31.406585928 30.2101974487305 5.0
47.441321812 45.6768646240234 6.0
47.441321812 46.8501434326172 7.0
39.37950561 38.0905303955078 0.0
19.873167771 21.6447582244873 1.0
19.506337839 19.4344139099121 2.0
-9.8415530061 -10.8011245727539 3.0
9.6647848227 8.9887752532959 4.0
29.714720787 31.1449851989746 5.0
39.379505619 41.7300796508789 6.0
39.379505619 42.3433303833008 7.0
42.095557289 38.8531799316406 0.0
22.277305426 21.7924900054932 1.0
19.818251863 19.8177433013916 2.0
-7.6874055589 -9.42489433288574 3.0
12.130846294 11.3354415893555 4.0
29.964710995 29.3569068908691 5.0
42.095557299 41.9169235229492 6.0
42.095557299 40.2805328369141 7.0
35.151188218 38.4096450805664 0.0
17.118344744 16.9611892700195 1.0
18.032843477 18.1718235015869 2.0
-2.0060677241 -3.1261191368103 3.0
16.026775747 15.6545400619507 4.0
19.124412475 19.1798362731934 5.0
35.151188225 34.8697280883789 6.0
35.151188225 34.7039413452148 7.0
37.713935362 38.1939315795898 0.0
20.380527895 19.2701034545898 1.0
17.333407469 18.2382144927979 2.0
-5.6423114193 -7.66351127624512 3.0
11.691096041 11.0271034240723 4.0
26.022839323 25.0261936187744 5.0
37.713935371 37.0099563598633 6.0
37.713935371 37.2810211181641 7.0
39.667743251 37.5596237182617 0.0
20.027544136 20.8216609954834 1.0
19.640199117 18.7193450927734 2.0
-10.376032212 -10.334132194519 3.0
9.2641668955 8.83496761322021 4.0
30.403576357 30.1484069824219 5.0
39.66774326 40.1139984130859 6.0
39.66774326 40.4533233642578 7.0
41.240556184 38.3862609863281 0.0
23.347458936 21.1435432434082 1.0
17.893097248 19.3683052062988 2.0
-7.252142064 -9.32804203033447 3.0
10.640955174 10.0189075469971 4.0
30.59960101 29.6779499053955 5.0
41.240556194 40.7343292236328 6.0
41.240556194 41.4971542358398 7.0
40.32019459 38.6126174926758 0.0
19.177093862 20.7322006225586 1.0
21.143100728 20.89084815979 2.0
-7.2609061044 -6.85930156707764 3.0
13.882194613 13.1603937149048 4.0
26.437999977 28.0188465118408 5.0
40.3201946 39.28173828125 6.0
40.3201946 40.6734466552734 7.0
39.5611510585585 39.3859558105469 0.0
19.6094718095582 19.8197803497314 1.0
19.9516792585585 19.0167236328125 2.0
-8.44726993515885 -8.88727855682373 3.0
11.5044093235587 10.6241550445557 4.0
28.0567416170478 28.0198516845703 5.0
39.561151059 38.9889526367188 6.0
39.561151059 39.7982940673828 7.0
43.5664319470049 39.2536926269531 0.0
22.3670487340048 21.7380771636963 1.0
21.1993832230048 20.2213344573975 2.0
-8.61702915780503 -8.69567966461182 3.0
12.582354065005 12.0542526245117 4.0
30.9840778920045 29.7219352722168 5.0
43.566431947 41.4726181030273 6.0
43.566431947 42.7189102172852 7.0
37.929366556 38.362907409668 0.0
20.105495873 19.1329498291016 1.0
17.823870687 18.7470550537109 2.0
-5.9768423324 -7.61201190948486 3.0
11.847028349 11.3475866317749 4.0
26.082338211 25.8750991821289 5.0
37.929366562 37.3192825317383 6.0
37.929366562 38.3035202026367 7.0
39.7456686413092 42.3975677490234 0.0
20.6016083083092 19.5039939880371 1.0
19.1440603423092 20.3956241607666 2.0
-3.0966467648096 -5.32644462585449 3.0
16.0474135773093 15.4909830093384 4.0
23.6982550733092 23.5807266235352 5.0
39.745668641 38.9904098510742 6.0
39.745668641 38.6185531616211 7.0
37.1737946256367 38.6771087646484 0.0
18.9810950589955 18.28173828125 1.0
18.1926995622865 18.5749950408936 2.0
-4.83110587561615 -6.28658008575439 3.0
13.3615936860386 12.8778429031372 4.0
23.8122009348139 23.6073837280273 5.0
37.173794625 36.0051956176758 6.0
37.173794625 36.3588485717773 7.0
27.124866007 40.3398284912109 0.0
13.623671484 13.8152093887329 1.0
13.501194525 12.9474258422852 2.0
-4.5998142153 -5.25400972366333 3.0
8.9013803006 9.02766990661621 4.0
18.223485707 18.1296672821045 5.0
27.124866015 28.8868579864502 6.0
27.124866015 28.4412879943848 7.0
27.411746895 36.7463607788086 0.0
13.763683287 13.9854421615601 1.0
13.648063611 13.749698638916 2.0
-4.2527245718 -4.60786962509155 3.0
9.3953390304 9.39307498931885 4.0
18.016407867 18.6928081512451 5.0
27.411746904 29.2872047424316 6.0
27.411746904 28.4803886413574 7.0
38.866617309 38.1587142944336 0.0
22.319478804 20.197925567627 1.0
16.547138508 17.9812850952148 2.0
-7.9092899105 -10.3619441986084 3.0
8.6378485894 8.25391960144043 4.0
30.228768722 29.4518909454346 5.0
38.866617317 38.6341781616211 6.0
38.866617317 39.0581817626953 7.0
42.3830573535328 38.3150253295898 0.0
20.5951530026372 20.8092937469482 1.0
21.7879043479541 19.9050922393799 2.0
-9.3946766066605 -8.83049869537354 3.0
12.3932277407315 11.7097043991089 4.0
29.98982960914 29.5242958068848 5.0
42.383057354 41.7898635864258 6.0
42.383057354 41.9013290405273 7.0
47.6437510358734 38.1060256958008 0.0
23.1576165238734 23.1068058013916 1.0
24.4861345218735 22.5069370269775 2.0
-8.74385685767346 -8.68577289581299 3.0
15.7422776638735 14.8399543762207 4.0
31.9014733818732 30.6543979644775 5.0
47.643751036 46.4351348876953 6.0
47.643751036 46.9786758422852 7.0
38.439399947 36.4208679199219 0.0
19.975660163 19.6007099151611 1.0
18.463739784 18.9128589630127 2.0
-6.4266906816 -7.36500644683838 3.0
12.037049093 11.4773578643799 4.0
26.402350854 26.1536636352539 5.0
38.439399957 37.2386703491211 6.0
38.439399957 38.7358169555664 7.0
40.263562361 38.7521286010742 0.0
20.42306719 21.6324996948242 1.0
19.840495171 19.0169887542725 2.0
-11.677045299 -10.7362909317017 3.0
8.1634498621 7.76362562179565 4.0
32.100112499 31.1445732116699 5.0
40.263562371 39.7273941040039 6.0
40.263562371 40.8870162963867 7.0
41.519528387 38.4955291748047 0.0
22.877320367 20.5484390258789 1.0
18.642208021 20.2042045593262 2.0
-5.2280057864 -7.99882888793945 3.0
13.414202225 12.5762281417847 4.0
28.105326163 27.7521781921387 5.0
41.519528397 41.0124359130859 6.0
41.519528397 41.6467742919922 7.0
40.414570464 38.545280456543 0.0
17.791289494 19.2980270385742 1.0
22.623280971 20.0405139923096 2.0
-6.6442159501 -5.92895364761353 3.0
15.979065012 14.6843795776367 4.0
24.435505453 23.8397598266602 5.0
40.414570474 39.4348983764648 6.0
40.414570474 40.1382675170898 7.0
37.834181215 38.0458221435547 0.0
18.770560656 19.026725769043 1.0
19.063620563 18.743106842041 2.0
-7.0281211763 -7.5380687713623 3.0
12.035499382 11.4961061477661 4.0
25.798681838 25.4843254089355 5.0
37.834181221 37.5501403808594 6.0
37.834181221 38.2561798095703 7.0
37.201121546 38.5927581787109 0.0
17.811506649 18.0195503234863 1.0
19.389614897 18.8371429443359 2.0
-6.0990519868 -6.29935646057129 3.0
13.290562901 12.8052835464478 4.0
23.910558645 23.9273529052734 5.0
37.201121556 36.8999633789062 6.0
37.201121556 38.0799560546875 7.0
42.944445132 37.6338272094727 0.0
21.59677927 21.4080848693848 1.0
21.347665865 21.0880851745605 2.0
-6.9735110037 -7.91884708404541 3.0
14.374154853 13.9020223617554 4.0
28.570290281 28.1455478668213 5.0
42.94444514 42.7225570678711 6.0
42.94444514 42.663444519043 7.0
47.3096000339538 38.3098220825195 0.0
24.7818404359387 23.2113876342773 1.0
22.5277596000527 22.52197265625 2.0
-6.75460229154402 -7.96077537536621 3.0
15.7731573079881 15.2921838760376 4.0
31.5364427279756 30.5247344970703 5.0
47.309600034 45.4152984619141 6.0
47.309600034 46.799430847168 7.0
34.703660739 37.7339782714844 0.0
16.95243502 16.7246932983398 1.0
17.75122572 17.4133987426758 2.0
-4.712400117 -5.60176753997803 3.0
13.038825595 12.5838718414307 4.0
21.664835145 21.2091007232666 5.0
34.703660747 34.5719451904297 6.0
34.703660747 35.0426940917969 7.0
39.355679823 38.3111801147461 0.0
18.719972615 20.0054531097412 1.0
20.635707207 19.1246032714844 2.0
-8.5340541425 -7.71829414367676 3.0
12.101653055 11.4348659515381 4.0
27.254026768 26.2859840393066 5.0
39.355679833 39.5544891357422 6.0
39.355679833 40.2200546264648 7.0
36.760838125 39.3952713012695 0.0
18.098408908 18.4492740631104 1.0
18.662429217 18.7476558685303 2.0
-4.9002916105 -5.8651704788208 3.0
13.762137597 13.3496942520142 4.0
22.998700528 22.7466430664062 5.0
36.760838135 36.7012939453125 6.0
36.760838135 37.1171569824219 7.0
45.817767369632 39.4295349121094 0.0
23.4066861136316 24.7768516540527 1.0
22.4110812656318 22.5875473022461 2.0
-7.99255969573223 -8.99110126495361 3.0
14.4185215696319 13.5078659057617 4.0
31.3992456710555 31.421703338623 5.0
45.817767369 46.1188812255859 6.0
45.817767369 47.965446472168 7.0
37.456387934 38.718879699707 0.0
18.714257418 19.2436027526855 1.0
18.742130516 18.3056735992432 2.0
-7.2626807055 -7.89689064025879 3.0
11.4794498 10.9147844314575 4.0
25.976938133 25.7266731262207 5.0
37.456387944 38.0156631469727 6.0
37.456387944 37.4311981201172 7.0
36.979299104 38.0109329223633 0.0
17.001867984 18.2078247070312 1.0
19.977431121 19.2218017578125 2.0
-4.4589079828 -4.98952913284302 3.0
15.518523129 14.8784132003784 4.0
21.460775976 21.1240882873535 5.0
36.979299113 37.1302108764648 6.0
36.979299113 37.1038970947266 7.0
44.357392148 38.3760147094727 0.0
23.57148554 21.6354026794434 1.0
20.785906608 21.3629627227783 2.0
-6.1950554121 -7.60999011993408 3.0
14.590851187 14.170747756958 4.0
29.766540962 28.0877113342285 5.0
44.357392157 43.3780136108398 6.0
44.357392157 43.6659927368164 7.0
36.8120792993302 38.4635009765625 0.0
19.8640110823302 19.7038173675537 1.0
16.9480682263302 17.6865768432617 2.0
-8.51793333553018 -9.63223648071289 3.0
8.43013489113018 7.97719144821167 4.0
28.381944393825 28.8917350769043 5.0
36.812079299 37.8954010009766 6.0
36.812079299 38.8740615844727 7.0
43.172254734 39.1506042480469 0.0
21.994700089 21.287561416626 1.0
21.177554646 21.6107559204102 2.0
-4.8220170211 -6.68520212173462 3.0
16.355537616 15.6073360443115 4.0
26.816717119 26.7941513061523 5.0
43.172254743 43.32275390625 6.0
43.172254743 43.4117965698242 7.0
47.109370558 39.7863998413086 0.0
25.034746172 23.357873916626 1.0
22.074624387 22.3467330932617 2.0
-8.7769581235 -9.06822299957275 3.0
13.297666255 12.5269060134888 4.0
33.811704304 32.8016815185547 5.0
47.109370567 46.5079803466797 6.0
47.109370567 46.2982406616211 7.0
35.871355653 37.1250152587891 0.0
18.076771786 18.8049793243408 1.0
17.794583868 17.2032375335693 2.0
-9.6137759762 -9.29005146026611 3.0
8.1808078825 7.89497995376587 4.0
27.690547772 27.5203189849854 5.0
35.871355662 35.1652069091797 6.0
35.871355662 36.8341217041016 7.0
42.1694339111434 38.8314437866211 0.0
22.6030080131434 21.3081550598145 1.0
19.5664259071434 20.3241405487061 2.0
-7.61628576964338 -9.15699577331543 3.0
11.9501401371434 11.008526802063 4.0
30.2192937831435 29.7380466461182 5.0
42.169433911 41.5776596069336 6.0
42.169433911 42.0681686401367 7.0
38.001249391 38.1650390625 0.0
19.442391644 19.7656803131104 1.0
18.558857748 18.629207611084 2.0
-7.4577746389 -8.1876916885376 3.0
11.101083099 10.5302896499634 4.0
26.900166292 26.8108882904053 5.0
38.001249401 37.9244537353516 6.0
38.001249401 38.9150009155273 7.0
36.026296144 38.3478240966797 0.0
18.544058535 18.6628570556641 1.0
17.482237609 17.9380531311035 2.0
-7.0335899837 -7.57428932189941 3.0
10.448647616 9.87599658966064 4.0
25.577648528 24.9330654144287 5.0
36.026296153 37.2376403808594 6.0
36.026296153 37.703010559082 7.0
33.200213149 40.287467956543 0.0
14.957525483 17.025318145752 1.0
18.242687666 16.2773590087891 2.0
-8.613844647 -7.50124549865723 3.0
9.6288430093 9.21061706542969 4.0
23.57137014 23.7718601226807 5.0
33.200213159 33.4814682006836 6.0
33.200213159 32.6489105224609 7.0
36.881947947 39.2395782470703 0.0
19.760267083 18.9527988433838 1.0
17.12168087 17.8010845184326 2.0
-6.5510496361 -8.39307975769043 3.0
10.570631229 9.94493865966797 4.0
26.311316723 26.6371574401855 5.0
36.881947952 36.7554473876953 6.0
36.881947952 37.4116897583008 7.0
28.363012003 38.7270050048828 0.0
14.617095463 14.1622161865234 1.0
13.745916541 13.3465538024902 2.0
-5.3741086073 -6.76098442077637 3.0
8.3718079243 8.29381370544434 4.0
19.991204079 19.9495868682861 5.0
28.363012012 29.9768486022949 6.0
28.363012012 29.0645999908447 7.0
39.574467416 37.6061553955078 0.0
20.649388383 22.4414443969727 1.0
18.925079033 20.6889381408691 2.0
-8.4547238017 -10.4059810638428 3.0
10.470355222 9.51583576202393 4.0
29.104112194 32.0760040283203 5.0
39.574467426 43.910514831543 6.0
39.574467426 43.7261199951172 7.0
36.220401471 38.3949584960938 0.0
17.995668566 18.2893886566162 1.0
18.224732906 17.837739944458 2.0
-6.9105415737 -7.24199485778809 3.0
11.314191323 11.0085277557373 4.0
24.906210149 24.359188079834 5.0
36.22040148 36.4723129272461 6.0
36.22040148 37.1749725341797 7.0
31.086432133 39.1634368896484 0.0
16.0910287 15.4654598236084 1.0
14.995403435 14.9265394210815 2.0
-4.3439743018 -5.34590101242065 3.0
10.651429126 10.6104555130005 4.0
20.43500301 20.3031311035156 5.0
31.086432141 31.1367130279541 6.0
31.086432141 31.2280387878418 7.0
36.274353053 38.4463577270508 0.0
18.420887917 17.6843662261963 1.0
17.853465137 18.4291648864746 2.0
-3.6202808126 -5.21385526657104 3.0
14.233184315 13.9288187026978 4.0
22.041168739 22.1156845092773 5.0
36.274353063 35.6696166992188 6.0
36.274353063 36.3318481445312 7.0
31.65795008 39.2199935913086 0.0
17.489254099 16.4708976745605 1.0
14.168695981 14.7580490112305 2.0
-6.0427493715 -7.45183753967285 3.0
8.1259465996 7.81711769104004 4.0
23.53200348 23.9427928924561 5.0
31.65795009 32.1667098999023 6.0
31.65795009 30.7517910003662 7.0
37.059171469 39.3203582763672 0.0
19.339994356 19.4019470214844 1.0
17.719177113 16.8399124145508 2.0
-9.4011683011 -9.28989315032959 3.0
8.318008802 8.35234451293945 4.0
28.741162667 28.0324687957764 5.0
37.059171479 37.2253952026367 6.0
37.059171479 37.5436477661133 7.0
45.308862348 39.220817565918 0.0
24.95721568 22.3412780761719 1.0
20.351646669 21.7738075256348 2.0
-7.323343681 -9.45879745483398 3.0
13.028302978 11.908106803894 4.0
32.28055937 31.4420166015625 5.0
45.308862357 44.7452087402344 6.0
45.308862357 44.6438064575195 7.0
32.988279899 39.2047119140625 0.0
17.965429309 16.0819435119629 1.0
15.02285059 16.926342010498 2.0
-2.6394240327 -4.85493850708008 3.0
12.383426547 11.7717523574829 4.0
20.604853352 20.5145568847656 5.0
32.988279909 33.3665161132812 6.0
32.988279909 33.1205062866211 7.0
41.812200584 40.716796875 0.0
21.127130287 20.7889308929443 1.0
20.685070296 20.4194183349609 2.0
-5.5457377677 -6.51394557952881 3.0
15.139332519 14.2325305938721 4.0
26.672868065 26.1188449859619 5.0
41.812200593 41.8736572265625 6.0
41.812200593 41.9870223999023 7.0
34.159577997 37.265266418457 0.0
15.979883142 16.5733184814453 1.0
18.179694855 16.5619583129883 2.0
-6.0058060763 -4.70657825469971 3.0
12.173888768 12.3701992034912 4.0
21.985689228 21.2810325622559 5.0
34.159578007 33.7586975097656 6.0
34.159578007 33.9638290405273 7.0
41.353058194 40.8923721313477 0.0
20.665722502 21.6208305358887 1.0
20.687335692 20.076301574707 2.0
-9.525258586 -9.4038724899292 3.0
11.162077096 10.2773103713989 4.0
30.190981098 29.7053661346436 5.0
41.353058204 42.2395629882812 6.0
41.353058204 41.7193908691406 7.0
35.663236634 38.0364151000977 0.0
18.766917562 18.0305480957031 1.0
16.896319073 16.7553672790527 2.0
-7.670552873 -8.74667644500732 3.0
9.2257661902 8.80641746520996 4.0
26.437470445 25.8856792449951 5.0
35.663236644 35.025276184082 6.0
35.663236644 35.8400115966797 7.0
40.444916113 39.8431777954102 0.0
18.421748417 20.8138008117676 1.0
22.023167696 19.4187850952148 2.0
-9.937900394 -9.13444900512695 3.0
12.085267292 11.1179285049438 4.0
28.359648821 28.2522449493408 5.0
40.444916123 40.682243347168 6.0
40.444916123 39.7233963012695 7.0
39.900416904 39.0326843261719 0.0
21.000367331 20.732442855835 1.0
18.900049573 19.8056240081787 2.0
-6.1981317567 -8.53449058532715 3.0
12.701917806 12.1823568344116 4.0
27.198499098 27.9057292938232 5.0
39.900416914 40.4196701049805 6.0
39.900416914 41.707633972168 7.0
34.40532484 38.8125534057617 0.0
18.680133752 17.6248817443848 1.0
15.725191089 16.2882823944092 2.0
-6.5675447079 -8.02764225006104 3.0
9.157646372 8.86200714111328 4.0
25.247678469 24.9007549285889 5.0
34.405324849 33.9266052246094 6.0
34.405324849 33.9994049072266 7.0
33.074255995 39.7259750366211 0.0
16.81819884 17.1022033691406 1.0
16.256057155 16.4729804992676 2.0
-5.8048358004 -6.80570220947266 3.0
10.451221345 10.1145238876343 4.0
22.62303465 22.4940090179443 5.0
33.074256004 33.1487350463867 6.0
33.074256004 33.1579895019531 7.0
40.036170298 40.0705032348633 0.0
18.427862329 19.8924922943115 1.0
21.60830797 20.1940879821777 2.0
-6.8170168406 -6.05037450790405 3.0
14.791291119 14.3158731460571 4.0
25.244879179 24.3987693786621 5.0
40.036170308 39.1054306030273 6.0
40.036170308 40.0101089477539 7.0
44.453206231 38.9618377685547 0.0
21.51525667 22.0500297546387 1.0
22.937949561 21.89235496521 2.0
-6.654859187 -7.4339656829834 3.0
16.283090365 15.3504457473755 4.0
28.170115867 28.1608390808105 5.0
44.453206241 44.2733459472656 6.0
44.453206241 44.8532028198242 7.0
33.8500454 39.1324462890625 0.0
15.317438725 16.899019241333 1.0
18.532606675 17.4372062683105 2.0
-3.771729854 -3.66815042495728 3.0
14.760876811 14.0392112731934 4.0
19.089168589 18.587818145752 5.0
33.85004541 33.9061126708984 6.0
33.85004541 34.3162841796875 7.0
30.537877299 37.9304428100586 0.0
15.743236653 15.7335357666016 1.0
14.794640646 14.6241807937622 2.0
-6.2614673178 -7.16267967224121 3.0
8.5331733191 8.3601188659668 4.0
22.00470398 22.2800445556641 5.0
30.537877308 30.127908706665 6.0
30.537877308 30.27783203125 7.0
27.727440934 36.8406219482422 0.0
13.298433958 13.9568004608154 1.0
14.429006978 13.708288192749 2.0
-4.8732361991 -4.95635652542114 3.0
9.5557707702 9.69101238250732 4.0
18.171670165 18.6766796112061 5.0
27.727440942 29.0367889404297 6.0
27.727440942 28.8692474365234 7.0
35.630182891 38.8659591674805 0.0
17.416160369 17.4986476898193 1.0
18.214022525 17.9752025604248 2.0
-5.5362398817 -6.49297046661377 3.0
12.677782636 12.2979021072388 4.0
22.952400258 23.0043239593506 5.0
35.630182897 35.8207244873047 6.0
35.630182897 36.9947814941406 7.0
31.038251509 38.4938354492188 0.0
16.216885208 15.719181060791 1.0
14.821366301 15.8819465637207 2.0
-5.1281710605 -5.38061714172363 3.0
9.6931952301 9.57784557342529 4.0
21.345056279 21.1744575500488 5.0
31.038251519 35.3713836669922 6.0
31.038251519 33.4895095825195 7.0
31.867011958 38.5478668212891 0.0
14.888086344 16.0857772827148 1.0
16.978925614 16.0145263671875 2.0
-5.7710515718 -5.4265308380127 3.0
11.207874033 10.7744665145874 4.0
20.659137926 20.0649967193604 5.0
31.867011967 32.6286582946777 6.0
31.867011967 32.7651405334473 7.0
30.754831636 38.8118896484375 0.0
14.067592922 15.3162145614624 1.0
16.687238713 15.7319660186768 2.0
-4.2836008115 -4.33065414428711 3.0
12.403637892 12.184775352478 4.0
18.351193744 18.0475749969482 5.0
30.754831646 31.9293422698975 6.0
30.754831646 31.8863906860352 7.0
38.951564273 38.2272415161133 0.0
20.789904648 19.7062282562256 1.0
18.161659627 17.7537155151367 2.0
-9.14497609 -9.92282485961914 3.0
9.0166835299 8.43372058868408 4.0
29.934880745 29.3792514801025 5.0
38.95156428 38.0840377807617 6.0
38.95156428 38.6076202392578 7.0
33.384480929 39.1025085449219 0.0
14.280681926 16.4918975830078 1.0
19.103799006 16.9669761657715 2.0
-6.8894608583 -5.55620813369751 3.0
12.214338142 11.8145179748535 4.0
21.170142791 21.3958950042725 5.0
33.384480936 32.6654891967773 6.0
33.384480936 31.9091911315918 7.0
37.291199278 38.8030319213867 0.0
18.976122996 17.4957332611084 1.0
18.315076287 18.4499969482422 2.0
-3.6314782733 -5.38515138626099 3.0
14.68359801 14.2914695739746 4.0
22.607601274 22.0485286712646 5.0
37.291199282 36.0691604614258 6.0
37.291199282 36.6016616821289 7.0
33.5718110164568 38.6800003051758 0.0
16.2455196066318 16.9113616943359 1.0
17.3262914102844 16.5193481445312 2.0
-6.34124959778268 -6.53616285324097 3.0
10.9850418123699 10.714958190918 4.0
22.5867692042782 22.6549491882324 5.0
33.571811016 34.1067886352539 6.0
33.571811016 34.3091583251953 7.0
44.048761737 38.4862060546875 0.0
22.194046458 21.6261215209961 1.0
21.85471528 21.7322368621826 2.0
-6.0219898767 -7.19724655151367 3.0
15.832725394 14.8925008773804 4.0
28.216036344 28.0335731506348 5.0
44.048761746 42.1424026489258 6.0
44.048761746 43.2741317749023 7.0
38.092815451 38.2420806884766 0.0
20.42584869 19.3928108215332 1.0
17.666966762 19.0491485595703 2.0
-5.2714738638 -7.20650196075439 3.0
12.395492889 11.8210115432739 4.0
25.697322563 25.2565593719482 5.0
38.09281546 37.8123321533203 6.0
38.09281546 38.8998031616211 7.0
42.616606914 38.5621795654297 0.0
22.043217204 21.0991725921631 1.0
20.57338971 20.2528762817383 2.0
-9.1142862064 -9.46041584014893 3.0
11.459103494 10.7132511138916 4.0
31.157503421 30.0427093505859 5.0
42.616606924 40.8032150268555 6.0
42.616606924 41.7413864135742 7.0
41.080227066 38.7166213989258 0.0
21.412108977 21.3951625823975 1.0
19.66811809 19.7237911224365 2.0
-8.9308117628 -9.66688823699951 3.0
10.737306318 9.92698860168457 4.0
30.342920748 29.7334499359131 5.0
41.080227074 41.443359375 6.0
41.080227074 41.4077987670898 7.0
36.859087804 38.433349609375 0.0
16.308519924 18.2599468231201 1.0
20.550567882 18.3344230651855 2.0
-8.0027710471 -6.93605136871338 3.0
12.547796826 12.0907773971558 4.0
24.31129098 23.9222564697266 5.0
36.859087813 37.3687133789062 6.0
36.859087813 37.9873733520508 7.0
39.316268866 38.0415191650391 0.0
19.479900065 19.8143157958984 1.0
19.836368801 19.0760173797607 2.0
-7.370707565 -7.87371635437012 3.0
12.465661226 11.8045310974121 4.0
26.85060764 26.5419731140137 5.0
39.316268876 39.4646453857422 6.0
39.316268876 39.6341781616211 7.0
40.023071957 38.8050231933594 0.0
21.374119573 20.7730007171631 1.0
18.648952384 18.9011993408203 2.0
-8.5239396023 -9.63755702972412 3.0
10.125012772 9.36878681182861 4.0
29.898059185 29.2027492523193 5.0
40.023071967 40.0522842407227 6.0
40.023071967 40.5714340209961 7.0
32.691039556 38.3575057983398 0.0
16.99402987 16.4254302978516 1.0
15.697009686 15.8058452606201 2.0
-6.1678662484 -7.45424652099609 3.0
9.5291434282 9.1234302520752 4.0
23.161896128 23.3277568817139 5.0
32.691039566 32.5299263000488 6.0
32.691039566 32.8125190734863 7.0
39.350172222 38.2304534912109 0.0
19.79571531 19.6176052093506 1.0
19.554456913 19.3827495574951 2.0
-6.9332448751 -7.61796092987061 3.0
12.62121203 12.0058012008667 4.0
26.728960194 26.4710826873779 5.0
39.35017223 38.9909057617188 6.0
39.35017223 39.605842590332 7.0
38.6257792205169 39.7686233520508 0.0
20.6764320695262 20.0450859069824 1.0
17.9493471475253 18.4269847869873 2.0
-8.74948254057215 -9.06752014160156 3.0
9.19986460693105 8.63833236694336 4.0
29.4259146106884 28.8499183654785 5.0
38.62577922 40.032356262207 6.0
38.62577922 38.3957214355469 7.0
43.363210173 38.3566284179688 0.0
22.264479357 21.3504753112793 1.0
21.09873082 20.8455333709717 2.0
-7.8566163895 -8.34503364562988 3.0
13.242114424 12.4946718215942 4.0
30.121095753 29.5680236816406 5.0
43.36321018 42.360481262207 6.0
43.36321018 42.765510559082 7.0
41.7113705187954 38.410530090332 0.0
19.9560794530664 20.5620975494385 1.0
21.7552910618007 20.6483879089355 2.0
-6.27295647675057 -6.27053594589233 3.0
15.4823345851289 14.6441297531128 4.0
26.2290359296866 25.2980918884277 5.0
41.711370519 40.6120376586914 6.0
41.711370519 41.5839080810547 7.0
41.4017485851654 40.012321472168 0.0
21.9191533447582 21.2350120544434 1.0
19.4825952431145 20.0624504089355 2.0
-7.53340700226978 -9.23404026031494 3.0
11.9491882404739 11.171706199646 4.0
29.4525603463383 29.4800243377686 5.0
41.401748585 41.8496017456055 6.0
41.401748585 40.5664215087891 7.0
45.411303660968 38.6011657714844 0.0
22.8793399812672 22.183198928833 1.0
22.5319636798771 21.0902271270752 2.0
-8.35657341793646 -8.23290729522705 3.0
14.1753902620561 13.4885587692261 4.0
31.2359133986188 30.6988735198975 5.0
45.411303661 42.7552947998047 6.0
45.411303661 44.0513076782227 7.0
46.173622728 38.2765502929688 0.0
22.478207281 22.1792011260986 1.0
23.695415447 21.4904689788818 2.0
-7.2335265538 -8.16776943206787 3.0
16.461888883 16.5144691467285 4.0
29.711733845 28.9169616699219 5.0
46.173622738 44.499885559082 6.0
46.173622738 45.0760192871094 7.0
43.970500883 38.4109878540039 0.0
23.319439955 22.0423831939697 1.0
20.651060929 20.3925151824951 2.0
-9.6037663307 -10.1343755722046 3.0
11.047294589 10.3460187911987 4.0
32.923206295 31.1921272277832 5.0
43.970500892 42.2292098999023 6.0
43.970500892 43.0829010009766 7.0
41.561964514 38.8353042602539 0.0
22.423587267 20.4294242858887 1.0
19.138377247 20.6321563720703 2.0
-2.7143772539 -6.24344968795776 3.0
16.423999984 16.1530227661133 4.0
25.13796453 24.6569118499756 5.0
41.561964524 41.7213363647461 6.0
41.561964524 41.050407409668 7.0
42.043570825 38.1755065917969 0.0
21.184612309 21.3031368255615 1.0
20.858958517 20.6786994934082 2.0
-7.1973175688 -7.39851951599121 3.0
13.66164094 13.0213384628296 4.0
28.381929887 27.8728427886963 5.0
42.043570834 41.2108306884766 6.0
42.043570834 41.688117980957 7.0
42.728187899 38.2817077636719 0.0
22.937429443 21.7080574035645 1.0
19.790758457 20.6395950317383 2.0
-7.3221221307 -8.65526103973389 3.0
12.468636317 11.8406934738159 4.0
30.259551583 29.384672164917 5.0
42.728187908 41.8752899169922 6.0
42.728187908 42.7050247192383 7.0
46.647822935 38.5361404418945 0.0
24.419972425 23.8600158691406 1.0
22.22785051 22.0308856964111 2.0
-9.7005341836 -9.42664051055908 3.0
12.527316316 11.4878358840942 4.0
34.120506619 33.8525314331055 5.0
46.647822945 46.0634155273438 6.0
46.647822945 47.0327758789062 7.0
34.157298452 38.6374435424805 0.0
16.904613725 17.4004230499268 1.0
17.252684728 16.6335182189941 2.0
-7.2828539724 -7.40594291687012 3.0
9.9698307451 9.54412651062012 4.0
24.187467707 23.9261646270752 5.0
34.157298462 33.9298858642578 6.0
34.157298462 34.4733428955078 7.0
38.08940785 40.2585372924805 0.0
20.151317931 18.6608867645264 1.0
17.93808992 19.6084251403809 2.0
-2.5516772312 -5.72512340545654 3.0
15.386412679 14.8432426452637 4.0
22.702995171 22.112154006958 5.0
38.08940786 39.3014678955078 6.0
38.08940786 38.1493377685547 7.0
48.253456595 39.4123382568359 0.0
26.02387816 23.8605499267578 1.0
22.229578435 22.6421127319336 2.0
-8.9125381923 -9.442946434021 3.0
13.317040233 12.2097778320312 4.0
34.936416362 33.5578842163086 5.0
48.253456605 47.5990447998047 6.0
48.253456605 47.9421081542969 7.0
41.303512945 38.0241088867188 0.0
21.842523524 20.9907550811768 1.0
19.460989421 19.5410175323486 2.0
-8.4785475097 -9.25720405578613 3.0
10.982441901 10.3447942733765 4.0
30.321071043 29.6855010986328 5.0
41.303512955 40.9467391967773 6.0
41.303512955 41.3833389282227 7.0
34.11247551 38.0281600952148 0.0
16.407323353 16.5559768676758 1.0
17.705152157 17.6839389801025 2.0
-3.4961419723 -4.1192774772644 3.0
14.209010175 13.9205007553101 4.0
19.903465335 19.9902420043945 5.0
34.11247552 33.7182388305664 6.0
34.11247552 34.1492080688477 7.0
39.346899311 38.4550399780273 0.0
20.280952021 20.4826831817627 1.0
19.06594729 18.3528308868408 2.0
-10.212334112 -9.99576377868652 3.0
8.8536131687 8.21247959136963 4.0
30.493286142 29.7933006286621 5.0
39.346899321 39.6096343994141 6.0
39.346899321 39.724723815918 7.0
42.355081312 37.3127059936523 0.0
22.237080125 22.240650177002 1.0
20.118001187 21.6637630462646 2.0
-4.2250518369 -8.08274555206299 3.0
15.892949341 15.4623432159424 4.0
26.462131971 29.0546703338623 5.0
42.355081321 45.5446014404297 6.0
42.355081321 45.1520080566406 7.0
46.02528461 37.5210113525391 0.0
20.767999035 22.6121349334717 1.0
25.257285575 21.6847133636475 2.0
-9.1629445671 -6.89923095703125 3.0
16.094340999 15.181357383728 4.0
29.930943612 28.2912178039551 5.0
46.025284619 43.1628112792969 6.0
46.025284619 44.9817733764648 7.0
48.790909777 37.9999694824219 0.0
25.145056622 23.4483261108398 1.0
23.645853155 22.4899921417236 2.0
-9.2179915631 -9.22602367401123 3.0
14.427861582 13.5308208465576 4.0
34.363048195 32.7184677124023 5.0
48.790909787 47.4653244018555 6.0
48.790909787 47.6438446044922 7.0
41.698940493 38.5718154907227 0.0
20.507660886 21.1916675567627 1.0
21.191279609 19.8242645263672 2.0
-9.1543537316 -8.90018558502197 3.0
12.036925868 11.3282794952393 4.0
29.662014626 28.7836875915527 5.0
41.698940502 40.8714141845703 6.0
41.698940502 41.4614944458008 7.0
47.902972774 37.9236907958984 0.0
23.393159022 23.3439140319824 1.0
24.509813754 23.0763282775879 2.0
-8.5476809737 -7.97155380249023 3.0
15.962132774 15.1462869644165 4.0
31.940840003 31.0478076934814 5.0
47.902972781 46.0086822509766 6.0
47.902972781 46.6090774536133 7.0
41.377170862 38.4424133300781 0.0
20.577519401 20.8493041992188 1.0
20.799651462 20.52956199646 2.0
-6.2821130824 -6.64414548873901 3.0
14.517538372 13.8285627365112 4.0
26.859632492 26.135570526123 5.0
41.37717087 41.5638656616211 6.0
41.37717087 41.2836227416992 7.0
32.285684892 38.3121643066406 0.0
14.316873201 15.6851854324341 1.0
17.968811691 16.3081245422363 2.0
-5.3227711546 -4.38966703414917 3.0
12.646040526 12.6627941131592 4.0
19.639644366 19.7672290802002 5.0
32.285684902 32.7703018188477 6.0
32.285684902 32.6642761230469 7.0
32.344206843 37.9958114624023 0.0
16.160894185 16.8645553588867 1.0
16.183312659 15.6117248535156 2.0
-7.3310896054 -7.66491508483887 3.0
8.852223045 8.37063026428223 4.0
23.491983799 23.2982540130615 5.0
32.344206851 33.25244140625 6.0
32.344206851 33.7069854736328 7.0
34.184112321 38.4180679321289 0.0
18.400062549 17.376672744751 1.0
15.784049772 16.0339813232422 2.0
-7.6583985283 -8.91202545166016 3.0
8.125651234 7.94446277618408 4.0
26.058461087 25.9154930114746 5.0
34.184112331 34.4313507080078 6.0
34.184112331 34.9890899658203 7.0
34.663221798 39.060661315918 0.0
18.206197344 17.9848823547363 1.0
16.457024454 16.6225280761719 2.0
-6.9582970888 -7.89465618133545 3.0
9.4987273548 9.08846759796143 4.0
25.164494443 24.9151744842529 5.0
34.663221808 34.8955383300781 6.0
34.663221808 36.0787048339844 7.0
36.873774533 36.4842071533203 0.0
18.579406772 18.2184753417969 1.0
18.294367761 18.4318828582764 2.0
-5.3733020341 -6.18012809753418 3.0
12.921065717 12.6412115097046 4.0
23.952708817 23.7828521728516 5.0
36.873774543 36.1619033813477 6.0
36.873774543 36.9672317504883 7.0
40.211352898 40.8574752807617 0.0
22.061687235 21.8373050689697 1.0
18.149665663 20.0331058502197 2.0
-7.4872031869 -10.1407079696655 3.0
10.662462467 9.82650661468506 4.0
29.548890431 31.1030902862549 5.0
40.211352908 42.4312973022461 6.0
40.211352908 42.6530380249023 7.0
46.160689864 38.6199569702148 0.0
24.176307096 22.1570148468018 1.0
21.984382774 22.1123676300049 2.0
-5.8472747438 -7.55188083648682 3.0
16.137108027 15.5524559020996 4.0
30.023581843 29.1270599365234 5.0
46.160689867 44.5380935668945 6.0
46.160689867 45.6408157348633 7.0
32.727702668 39.0109100341797 0.0
17.796794822 16.9581928253174 1.0
14.930907847 15.1722841262817 2.0
-5.7416146893 -6.93614101409912 3.0
9.1892931492 8.79934120178223 4.0
23.53840952 22.9070301055908 5.0
32.727702676 32.644401550293 6.0
32.727702676 33.5182876586914 7.0
40.640985417073 40.9353637695312 0.0
21.2248663380953 20.5563144683838 1.0
19.4161190831271 19.8506031036377 2.0
-6.12795167285201 -7.54850482940674 3.0
13.2881674100867 12.5912218093872 4.0
27.3528180102152 27.0946311950684 5.0
40.640985417 40.7124404907227 6.0
40.640985417 39.9781875610352 7.0
34.611451584 38.7155685424805 0.0
16.03184427 17.1442260742188 1.0
18.579607315 17.3146991729736 2.0
-5.6371786801 -5.41832494735718 3.0
12.942428625 12.5042181015015 4.0
21.66902296 21.6537895202637 5.0
34.611451594 35.0280303955078 6.0
34.611451594 35.4404067993164 7.0
43.311280798 39.3773040771484 0.0
19.747206987 20.760814666748 1.0
23.564073813 21.1719360351562 2.0
-7.5093463576 -5.80857801437378 3.0
16.054727447 15.5904731750488 4.0
27.256553353 26.2792015075684 5.0
43.311280807 41.630485534668 6.0
43.311280807 42.7580184936523 7.0
36.406364048 38.5013809204102 0.0
19.818637935 18.2263565063477 1.0
16.587726114 18.2509441375732 2.0
-3.8896932395 -6.24281692504883 3.0
12.698032865 12.240761756897 4.0
23.708331184 23.5217761993408 5.0
36.406364057 35.6973571777344 6.0
36.406364057 37.0864334106445 7.0
31.68563032 42.1182861328125 0.0
16.740289757 16.6609230041504 1.0
14.945340567 15.3443574905396 2.0
-6.0451472911 -7.26226711273193 3.0
8.9001932697 8.35330390930176 4.0
22.785437054 22.8738708496094 5.0
31.685630326 32.1779098510742 6.0
31.685630326 31.5273990631104 7.0
40.165364423 41.7417984008789 0.0
21.225216477 21.3776092529297 1.0
18.940147947 19.0668811798096 2.0
-9.5161355118 -9.72500038146973 3.0
9.4240124269 8.83594512939453 4.0
30.741351997 30.0516490936279 5.0
40.165364432 40.1582717895508 6.0
40.165364432 39.4980697631836 7.0
42.886465637 38.1925201416016 0.0
31.633244398 22.099588394165 1.0
11.253221239 20.3598155975342 2.0
0 -10.0018396377563 3.0
11.253221239 10.3876953125 4.0
31.633244398 30.4753379821777 5.0
42.886465647 42.8859329223633 6.0
42.886465647 43.4424133300781 7.0
46.942592322 39.7968139648438 0.0
23.694673267 23.3517818450928 1.0
23.247919055 22.2436904907227 2.0
-9.8881851606 -9.55183219909668 3.0
13.359733884 12.4381628036499 4.0
33.582858438 32.7497406005859 5.0
46.942592332 45.6073760986328 6.0
46.942592332 46.5753326416016 7.0
38.417030558 40.2535781860352 0.0
20.711241226 20.4441623687744 1.0
17.705789332 18.2222671508789 2.0
-7.7084466149 -9.50608062744141 3.0
9.9973427073 9.2020206451416 4.0
28.419687851 28.4029159545898 5.0
38.417030568 38.0373611450195 6.0
38.417030568 37.2130508422852 7.0
39.82159542 38.9206085205078 0.0
19.816428679 19.6259670257568 1.0
20.005166741 19.7445392608643 2.0
-5.1739908699 -6.13262510299683 3.0
14.831175861 14.4361400604248 4.0
24.990419559 24.3615665435791 5.0
39.82159543 38.943603515625 6.0
39.82159543 40.2914505004883 7.0
42.375846264 39.1178283691406 0.0
20.538210367 20.5903358459473 1.0
21.837635902 21.1823558807373 2.0
-5.3545430676 -6.22136545181274 3.0
16.483092829 15.3301210403442 4.0
25.89275344 25.6335678100586 5.0
42.375846269 40.8890533447266 6.0
42.375846269 42.2571411132812 7.0
41.909868914 38.1749801635742 0.0
21.249762031 21.4168834686279 1.0
20.660106883 18.9072437286377 2.0
-10.971309543 -10.2181768417358 3.0
9.6887973304 9.57851219177246 4.0
32.221071584 30.601146697998 5.0
41.909868924 41.6845245361328 6.0
41.909868924 42.7259216308594 7.0
38.5154038135067 38.4745330810547 0.0
19.3171244035071 18.9394989013672 1.0
19.1982794195071 17.5154647827148 2.0
-8.28760453680882 -8.4829626083374 3.0
10.9106748825063 10.6974229812622 4.0
27.6047289405071 26.8294239044189 5.0
38.515403814 37.6025390625 6.0
38.515403814 39.1212463378906 7.0
32.187464764 37.9287490844727 0.0
15.688410189 16.0790748596191 1.0
16.499054576 16.0943984985352 2.0
-5.8055743047 -5.66058874130249 3.0
10.693480262 10.3485889434814 4.0
21.493984503 21.5010547637939 5.0
32.187464773 31.9919776916504 6.0
32.187464773 31.9350719451904 7.0
43.5533404982843 39.243034362793 0.0
25.257472916292 22.6826000213623 1.0
18.2958675832873 20.0144195556641 2.0
-8.52565235008034 -10.6790180206299 3.0
9.77021523347734 8.71351432800293 4.0
33.7831252662979 33.5288772583008 5.0
43.553340498 42.7242431640625 6.0
43.553340498 41.6242828369141 7.0
38.661685108 38.3978805541992 0.0
21.023698145 22.1767997741699 1.0
17.637986964 20.4815006256104 2.0
-6.4647611773 -9.44525146484375 3.0
11.173225776 10.1704559326172 4.0
27.488459332 30.011926651001 5.0
38.661685118 43.3254928588867 6.0
38.661685118 43.201042175293 7.0
47.185976942 37.8694458007812 0.0
20.88210062 22.9890632629395 1.0
26.303876326 22.8522891998291 2.0
-9.8059530494 -7.71758937835693 3.0
16.497923272 16.7459602355957 4.0
30.688053675 29.5526866912842 5.0
47.185976948 46.2387313842773 6.0
47.185976948 46.8695526123047 7.0
41.0736498246764 38.6558151245117 0.0
17.9299575176762 20.1273136138916 1.0
23.1436923166761 20.1229667663574 2.0
-7.36188695227673 -5.97925853729248 3.0
15.7818053646765 15.4023761749268 4.0
25.2918444696756 24.1534004211426 5.0
41.073649825 39.4811935424805 6.0
41.073649825 40.7490234375 7.0
39.086114836 38.8619689941406 0.0
22.054277289 20.55735206604 1.0
17.031837547 18.5319519042969 2.0
-8.5092532605 -9.80507850646973 3.0
8.5225842768 8.27036571502686 4.0
30.563530559 29.8800678253174 5.0
39.086114846 38.731803894043 6.0
39.086114846 38.9209671020508 7.0
38.072309142 38.6814727783203 0.0
19.059692214 18.7046546936035 1.0
19.012616928 19.1962356567383 2.0
-4.5092621515 -6.01113700866699 3.0
14.503354768 14.0734481811523 4.0
23.568954375 23.2052192687988 5.0
38.072309151 37.7062454223633 6.0
38.072309151 37.8989028930664 7.0
41.585402364 38.2527008056641 0.0
21.878346027 21.2242889404297 1.0
19.707056337 20.1088809967041 2.0
-7.856520423 -8.65490341186523 3.0
11.850535904 11.2097215652466 4.0
29.73486646 29.1271724700928 5.0
41.585402374 41.1975479125977 6.0
41.585402374 41.5237426757812 7.0
42.3173178092696 38.4215469360352 0.0
23.2468941623901 22.1233310699463 1.0
19.0704236443393 18.9379272460938 2.0
-10.4383552532401 -11.0017137527466 3.0
8.63206839120634 8.19525623321533 4.0
33.6852494163203 32.4750366210938 5.0
42.317317809 42.1171417236328 6.0
42.317317809 42.6608352661133 7.0
42.931477575 39.3991851806641 0.0
22.005453055 21.6870460510254 1.0
20.926024519 21.0156536102295 2.0
-5.8784626661 -7.5215368270874 3.0
15.047561843 14.1241235733032 4.0
27.883915732 27.3795833587646 5.0
42.931477585 43.399299621582 6.0
42.931477585 43.0428009033203 7.0
46.096274403 39.2615356445312 0.0
25.32088965 22.5405292510986 1.0
20.775384755 22.2419261932373 2.0
-5.4662065709 -8.19584560394287 3.0
15.309178177 14.5654554367065 4.0
30.787096229 30.0127086639404 5.0
46.096274411 44.4457626342773 6.0
46.096274411 45.9870376586914 7.0
39.955261544 38.6042404174805 0.0
19.931762754 20.3806228637695 1.0
20.02349879 19.182035446167 2.0
-8.9110310329 -8.76078701019287 3.0
11.112467747 10.7508745193481 4.0
28.842793797 28.0816440582275 5.0
39.955261554 39.1785736083984 6.0
39.955261554 39.9708480834961 7.0
45.415961836 38.7581024169922 0.0
23.097461473 22.6445693969727 1.0
22.318500363 21.4108543395996 2.0
-8.5889928185 -9.13318157196045 3.0
13.729507535 12.8146114349365 4.0
31.686454301 30.696325302124 5.0
45.415961846 43.9062652587891 6.0
45.415961846 45.0336608886719 7.0
44.275746647 38.3189926147461 0.0
22.654009807 22.217716217041 1.0
21.62173684 21.3062992095947 2.0
-7.7547516702 -8.74852466583252 3.0
13.86698516 12.9053621292114 4.0
30.408761488 29.4190692901611 5.0
44.275746657 43.9858322143555 6.0
44.275746657 44.1960830688477 7.0
34.766238956 37.287353515625 0.0
17.943458551 17.6131191253662 1.0
16.822780406 17.1192417144775 2.0
-5.6356736314 -7.00773429870605 3.0
11.187106764 10.7948007583618 4.0
23.579132192 23.2619380950928 5.0
34.766238966 34.7213668823242 6.0
34.766238966 34.5344619750977 7.0
39.697519492335 38.7826766967773 0.0
21.2162061883195 19.4346141815186 1.0
18.481313305488 19.3157329559326 2.0
-4.67528909066386 -7.0928783416748 3.0
13.8060242148741 13.4178695678711 4.0
25.8914952793298 25.2283058166504 5.0
39.697519492 38.5190505981445 6.0
39.697519492 39.5111618041992 7.0
36.4599525996002 38.5531311035156 0.0
18.3065166475976 18.3537063598633 1.0
18.1534359526018 18.5036849975586 2.0
-5.15104061002709 -6.58333730697632 3.0
13.0023953426082 12.6136074066162 4.0
23.457557257596 23.4695262908936 5.0
36.459952599 36.2533340454102 6.0
36.459952599 37.4287109375 7.0
44.856515577 38.2222518920898 0.0
22.473710733 22.5195598602295 1.0
22.382804843 21.8416500091553 2.0
-8.0040806232 -7.90747356414795 3.0
14.37872421 13.7032289505005 4.0
30.477791367 29.1743850708008 5.0
44.856515586 44.1223602294922 6.0
44.856515586 44.4332046508789 7.0
41.1579139562054 38.4082717895508 0.0
18.5564288232053 19.9055194854736 1.0
22.601485143205 20.1454486846924 2.0
-7.31439675300561 -6.26758575439453 3.0
15.2870883902055 14.8041162490845 4.0
25.8708255762049 24.740291595459 5.0
41.157913956 39.4463882446289 6.0
41.157913956 40.2281723022461 7.0
32.092085782 42.7318878173828 0.0
17.272160967 16.5270881652832 1.0
14.819924815 15.0818977355957 2.0
-6.6255124723 -7.97371292114258 3.0
8.1944123328 7.85368251800537 4.0
23.897673449 24.0389785766602 5.0
32.092085792 31.7450618743896 6.0
32.092085792 30.3968200683594 7.0
30.327518415 38.4048843383789 0.0
13.541286609 14.6094779968262 1.0
16.786231807 15.1072158813477 2.0
-5.5133467024 -5.7421088218689 3.0
11.272885094 11.1871385574341 4.0
19.054633321 19.3803768157959 5.0
30.327518425 30.3478355407715 6.0
30.327518425 30.4830265045166 7.0
36.482977268 39.0758666992188 0.0
19.553260998 19.031644821167 1.0
16.929716271 17.6167831420898 2.0
-7.3162847454 -9.2794132232666 3.0
9.613431516 8.96224021911621 4.0
26.869545753 27.4428234100342 5.0
36.482977278 36.0645523071289 6.0
36.482977278 36.1489791870117 7.0
40.491170904 39.51025390625 0.0
20.509684159 20.8911056518555 1.0
19.981486745 19.167610168457 2.0
-8.9848735316 -10.0219697952271 3.0
10.996613204 10.0886106491089 4.0
29.4945577 30.1088333129883 5.0
40.491170914 39.7505264282227 6.0
40.491170914 39.4850921630859 7.0
41.188898324 38.781135559082 0.0
17.482572735 20.9087905883789 1.0
23.70632559 20.5050888061523 2.0
-7.86829602 -6.30735492706299 3.0
15.838029561 15.0808620452881 4.0
25.350868764 24.5638866424561 5.0
41.188898334 39.6158218383789 6.0
41.188898334 40.3856430053711 7.0
37.756949894 37.3375091552734 0.0
19.043754327 18.7420654296875 1.0
18.713195567 19.0237312316895 2.0
-5.5841962186 -6.77661228179932 3.0
13.128999338 12.5939464569092 4.0
24.627950556 24.4968547821045 5.0
37.756949904 38.0179138183594 6.0
37.756949904 37.8417816162109 7.0
26.962645791 38.6975555419922 0.0
14.231795944 13.5606527328491 1.0
12.730849849 12.0478553771973 2.0
-4.3614404899 -5.80632019042969 3.0
8.3694093501 8.35910892486572 4.0
18.593236442 18.5433235168457 5.0
26.9626458 29.0926151275635 6.0
26.9626458 27.8304920196533 7.0
44.1546808660895 38.0645523071289 0.0
20.845479755763 21.6654720306396 1.0
23.3092011082446 22.5813045501709 2.0
-6.97221748920505 -7.73952484130859 3.0
16.3369836194072 15.5416536331177 4.0
27.8176971034536 29.6074733734131 5.0
44.154680866 44.8336486816406 6.0
44.154680866 45.3158950805664 7.0
36.4992036 38.1218414306641 0.0
18.499566849 18.9169368743896 1.0
17.999636751 17.2083702087402 2.0
-8.164709678 -8.1932954788208 3.0
9.834927063 9.57297229766846 4.0
26.664276537 25.952974319458 5.0
36.49920361 36.0940704345703 6.0
36.49920361 37.3394546508789 7.0
36.136186998 39.2956619262695 0.0
19.096538919 19.0549125671387 1.0
17.039648079 17.2332229614258 2.0
-8.7939144312 -8.95334053039551 3.0
8.2457336392 8.1334285736084 4.0
27.89045336 27.9088649749756 5.0
36.136187007 35.604606628418 6.0
36.136187007 36.6808471679688 7.0
43.805905699 38.4082794189453 0.0
19.467800315 21.7324333190918 1.0
24.338105385 21.3228511810303 2.0
-10.07714477 -7.99375820159912 3.0
14.260960606 13.4028282165527 4.0
29.544945095 29.3950824737549 5.0
43.805905709 42.9095916748047 6.0
43.805905709 43.1864471435547 7.0
43.400673471 37.3553619384766 0.0
21.450145846 21.3319797515869 1.0
21.950527627 20.0062026977539 2.0
-10.718319514 -9.52839851379395 3.0
11.232208106 10.443018913269 4.0
32.168465368 30.1349048614502 5.0
43.400673478 44.3151473999023 6.0
43.400673478 42.2234802246094 7.0
35.708898922 41.3468627929688 0.0
19.860917727 18.4082355499268 1.0
15.847981195 17.1911163330078 2.0
-6.8010445999 -8.64494705200195 3.0
9.0469365848 8.26922416687012 4.0
26.661962337 26.3812522888184 5.0
35.708898932 35.3210830688477 6.0
35.708898932 34.2125473022461 7.0
42.19537597 38.6839065551758 0.0
21.611035583 21.2544460296631 1.0
20.584340387 20.4304485321045 2.0
-7.818182446 -8.66779899597168 3.0
12.766157932 11.9916820526123 4.0
29.429218038 28.7422046661377 5.0
42.195375979 41.3631820678711 6.0
42.195375979 42.2733154296875 7.0
38.219132248 38.8931350708008 0.0
18.124133621 18.5880088806152 1.0
20.094998634 19.4918537139893 2.0
-3.9675986496 -4.18937349319458 3.0
16.127399981 15.6467847824097 4.0
22.091732274 21.4282341003418 5.0
38.219132252 37.8382568359375 6.0
38.219132252 38.0583267211914 7.0
37.34785903 39.6710510253906 0.0
19.494183125 18.5862197875977 1.0
17.853675906 18.7472915649414 2.0
-4.6743476987 -6.32650375366211 3.0
13.179328197 12.6402902603149 4.0
24.168530833 23.53444480896 5.0
37.34785904 37.4297256469727 6.0
37.34785904 37.857795715332 7.0
41.413317073 38.4210968017578 0.0
22.502810393 20.7934989929199 1.0
18.91050668 20.2211971282959 2.0
-4.5479233593 -6.43856811523438 3.0
14.362583311 13.6872663497925 4.0
27.050733762 26.306791305542 5.0
41.413317083 40.7909088134766 6.0
41.413317083 41.0473098754883 7.0
39.408474003 40.3025207519531 0.0
19.787258075 19.7200088500977 1.0
19.621215928 18.9290428161621 2.0
-8.5467597404 -8.02848339080811 3.0
11.074456178 10.4783992767334 4.0
28.334017825 27.7507095336914 5.0
39.408474012 39.2541885375977 6.0
39.408474012 39.7915420532227 7.0
44.6219477930596 37.9476928710938 0.0
23.5295769810594 22.9357490539551 1.0
21.0923708210596 21.2818946838379 2.0
-9.44677516695988 -9.62642574310303 3.0
11.6455956540597 10.8971891403198 4.0
32.9763521480588 31.588306427002 5.0
44.621947793 43.3761825561523 6.0
44.621947793 44.4126815795898 7.0
35.042314664 40.4611358642578 0.0
17.377708596 17.5871887207031 1.0
17.66460607 17.1801624298096 2.0
-4.7609303256 -5.11108732223511 3.0
12.903675736 12.5969161987305 4.0
22.13863893 21.5433940887451 5.0
35.042314673 34.7478256225586 6.0
35.042314673 34.9639511108398 7.0
44.668685977 36.4160079956055 0.0
21.700098462 22.3417587280273 1.0
22.968587516 21.7795524597168 2.0
-8.4663400465 -8.06090354919434 3.0
14.50224746 13.6951875686646 4.0
30.166438518 29.3170509338379 5.0
44.668685987 43.0581436157227 6.0
44.668685987 44.2628860473633 7.0
44.271188228 37.6531753540039 0.0
21.134860564 21.8332481384277 1.0
23.136327666 22.3055229187012 2.0
-6.9953395415 -6.5513596534729 3.0
16.140988117 15.0220232009888 4.0
28.130200113 27.2446060180664 5.0
44.271188235 44.1387329101562 6.0
44.271188235 43.6082611083984 7.0
43.3296157145675 39.0664367675781 0.0
21.8144364805029 21.656759262085 1.0
21.5151792385451 20.2940578460693 2.0
-10.3071776780671 -9.62773132324219 3.0
11.2080015596363 10.501633644104 4.0
32.1216141588196 30.5324611663818 5.0
43.329615715 41.9489364624023 6.0
43.329615715 42.7121810913086 7.0
37.727034885 40.9678497314453 0.0
16.740902708 19.1518745422363 1.0
20.986132178 18.1882343292236 2.0
-10.156378742 -7.44830799102783 3.0
10.829753428 10.1057720184326 4.0
26.897281458 26.3376235961914 5.0
37.727034893 37.7822113037109 6.0
37.727034893 36.8564376831055 7.0
43.665312103 38.6473083496094 0.0
21.432332029 21.7133026123047 1.0
22.232980074 21.0747184753418 2.0
-7.2001290055 -6.87165451049805 3.0
15.032851059 14.5441732406616 4.0
28.632461044 28.4710445404053 5.0
43.665312113 42.2714920043945 6.0
43.665312113 43.1181564331055 7.0
40.6563483834553 38.2322006225586 0.0
20.9212538534144 22.2874374389648 1.0
19.7350945303589 20.575080871582 2.0
-7.59696943049863 -9.7576961517334 3.0
12.1381250996025 11.3042478561401 4.0
28.5182231244438 30.6564598083496 5.0
40.656348383 42.8358840942383 6.0
40.656348383 43.5187072753906 7.0
30.759139545 38.3696823120117 0.0
13.433268738 15.0255012512207 1.0
17.325870807 15.4236707687378 2.0
-5.9875785844 -4.64086389541626 3.0
11.338292213 11.294243812561 4.0
19.420847332 19.5131206512451 5.0
30.759139555 31.2909145355225 6.0
30.759139555 31.389928817749 7.0
36.474229645 40.226318359375 0.0
14.989459641 17.8264293670654 1.0
21.484770007 19.1009178161621 2.0
-5.0219280476 -3.41641426086426 3.0
16.462841952 16.4279747009277 4.0
20.011387696 18.6752967834473 5.0
36.474229652 35.9620742797852 6.0
36.474229652 35.4279861450195 7.0
33.340721185 38.9816360473633 0.0
15.596021276 16.8473777770996 1.0
17.744699909 16.3874359130859 2.0
-7.4570294442 -6.50424766540527 3.0
10.287670455 9.99333381652832 4.0
23.05305073 23.0990505218506 5.0
33.340721195 34.0192718505859 6.0
33.340721195 32.3213653564453 7.0
34.143349258 40.9910354614258 0.0
15.417033592 17.2636260986328 1.0
18.726315666 17.2722606658936 2.0
-5.8334622114 -4.53218936920166 3.0
12.892853445 13.1548357009888 4.0
21.250495814 20.2220420837402 5.0
34.143349268 33.7969512939453 6.0
34.143349268 34.1047210693359 7.0
40.763829748 38.9162445068359 0.0
20.589059268 21.5375442504883 1.0
20.174770481 18.90452003479 2.0
-11.800756688 -10.6801910400391 3.0
8.374013783 8.09120941162109 4.0
32.389815966 31.1681251525879 5.0
40.763829757 41.1318054199219 6.0
40.763829757 41.0230712890625 7.0
29.788224949 39.1653823852539 0.0
14.329239328 15.0021514892578 1.0
15.458985621 14.4682674407959 2.0
-6.2969425219 -6.21877908706665 3.0
9.1620430892 8.91473579406738 4.0
20.62618186 20.7115249633789 5.0
29.788224958 30.6734504699707 6.0
29.788224958 29.9834861755371 7.0
34.675869546 37.6262817382812 0.0
17.321679699 17.2443523406982 1.0
17.354189848 16.9586791992188 2.0
-5.026706589 -5.94963693618774 3.0
12.32748325 11.9344501495361 4.0
22.348386296 21.9416027069092 5.0
34.675869554 35.6226425170898 6.0
34.675869554 35.7264633178711 7.0
40.7238558279895 39.4912109375 0.0
19.1061702059935 21.142370223999 1.0
21.6176856259914 20.7369403839111 2.0
-8.60840523789577 -7.55541706085205 3.0
13.0092803879935 12.2754936218262 4.0
27.7145752997044 28.849515914917 5.0
40.723855828 42.8596496582031 6.0
40.723855828 42.3968048095703 7.0
41.414838344 37.8283233642578 0.0
21.79665744 21.0733413696289 1.0
19.618180905 18.9981441497803 2.0
-10.700829677 -10.7724571228027 3.0
8.9173512199 8.26488208770752 4.0
32.497487126 30.8184375762939 5.0
41.414838352 41.5673294067383 6.0
41.414838352 41.5150299072266 7.0
33.391441204 37.6373596191406 0.0
16.946662464 16.966480255127 1.0
16.44477874 16.3719825744629 2.0
-6.5811351715 -7.87235927581787 3.0
9.8636435585 9.08365631103516 4.0
23.527797645 23.7306652069092 5.0
33.391441214 33.2765502929688 6.0
33.391441214 32.7680435180664 7.0
36.660645376 37.8180084228516 0.0
17.87289688 18.109561920166 1.0
18.787748497 17.614595413208 2.0
-6.6425323884 -6.07526779174805 3.0
12.1452161 11.9890327453613 4.0
24.515429278 23.5778293609619 5.0
36.660645385 36.2939300537109 6.0
36.660645385 36.242317199707 7.0
33.547567157 39.7165679931641 0.0
16.605704713 17.7154903411865 1.0
16.941862445 15.9527902603149 2.0
-8.1253896728 -7.4156379699707 3.0
8.8164727625 8.34013557434082 4.0
24.731094395 24.3980903625488 5.0
33.547567166 34.0128707885742 6.0
33.547567166 32.6779708862305 7.0
36.717923485 39.2869567871094 0.0
19.502688668 18.7698841094971 1.0
17.215234818 17.8815155029297 2.0
-6.0032006006 -7.53859424591064 3.0
11.212034207 10.7419080734253 4.0
25.505889278 25.0586738586426 5.0
36.717923495 37.1936569213867 6.0
36.717923495 37.5066909790039 7.0
44.575737489 40.5854110717773 0.0
24.24689991 22.124979019165 1.0
20.328837584 22.1266841888428 2.0
-6.0388557339 -8.2476863861084 3.0
14.289981845 13.3454523086548 4.0
30.285755649 29.336555480957 5.0
44.575737494 45.5420227050781 6.0
44.575737494 43.1909103393555 7.0
43.091610313 38.3678588867188 0.0
22.329030269 21.560863494873 1.0
20.762580045 20.8383655548096 2.0
-7.1251861541 -8.47113418579102 3.0
13.637393882 12.9377851486206 4.0
29.454216432 28.8676280975342 5.0
43.091610321 42.3122711181641 6.0
43.091610321 43.2684097290039 7.0
42.647069964 37.9135208129883 0.0
21.774885881 21.5169792175293 1.0
20.872184083 20.7233734130859 2.0
-7.6954983452 -8.57610130310059 3.0
13.176685728 12.4285440444946 4.0
29.470384236 29.0644855499268 5.0
42.647069973 41.9329528808594 6.0
42.647069973 42.6456298828125 7.0
36.24022533 38.1860504150391 0.0
17.188554164 17.9934139251709 1.0
19.051671166 18.1627101898193 2.0
-6.1891251621 -6.36701965332031 3.0
12.862545995 12.3815546035767 4.0
23.377679336 22.9789581298828 5.0
36.240225339 36.4092025756836 6.0
36.240225339 36.9738311767578 7.0
31.876756195 38.5105590820312 0.0
14.446454361 16.0429267883301 1.0
17.430301835 15.9513492584229 2.0
-5.0545683645 -4.60816955566406 3.0
12.375733461 12.1305341720581 4.0
19.501022735 18.7645626068115 5.0
31.876756203 32.8788146972656 6.0
31.876756203 33.262825012207 7.0
38.711049649 38.8478851318359 0.0
19.068925854 18.7625312805176 1.0
19.642123794 19.1813526153564 2.0
-5.0802825502 -6.32479190826416 3.0
14.561841234 13.9412775039673 4.0
24.149208415 23.7705402374268 5.0
38.711049659 37.8623046875 6.0
38.711049659 38.0536041259766 7.0
33.592635662 38.2101058959961 0.0
16.042625471 16.7317085266113 1.0
17.550010192 16.7213745117188 2.0
-5.9813560668 -6.13964128494263 3.0
11.568654115 11.3106861114502 4.0
22.023981548 21.8307762145996 5.0
33.592635672 34.0051498413086 6.0
33.592635672 33.7525329589844 7.0
31.9394970213296 38.4451141357422 0.0
15.7221064593445 15.8866930007935 1.0
16.2173905613348 16.2083702087402 2.0
-3.67606552949996 -4.31400108337402 3.0
12.5413250313355 12.2592344284058 4.0
19.3981719893576 19.1507167816162 5.0
31.939497021 32.6907539367676 6.0
31.939497021 33.176887512207 7.0
32.605677366 39.1546630859375 0.0
15.053063258 16.0942420959473 1.0
17.552614114 16.665340423584 2.0
-5.3475005701 -5.25166940689087 3.0
12.20511354 11.8578281402588 4.0
20.400563832 20.0710620880127 5.0
32.605677369 33.5073165893555 6.0
32.605677369 33.7071304321289 7.0
35.02561424 37.8257446289062 0.0
18.375955543 17.9033374786377 1.0
16.649658698 17.3728160858154 2.0
-5.1216145805 -6.55123424530029 3.0
11.528044108 11.2901048660278 4.0
23.497570133 22.6965236663818 5.0
35.02561425 35.544075012207 6.0
35.02561425 35.8606414794922 7.0
29.155020361 38.2448348999023 0.0
13.928860816 14.4688167572021 1.0
15.226159546 14.524450302124 2.0
-5.2043069028 -4.99999237060547 3.0
10.021852634 10.0219116210938 4.0
19.133167728 19.0475940704346 5.0
29.155020371 30.7879123687744 6.0
29.155020371 30.5458068847656 7.0
29.979099891 36.8306045532227 0.0
16.392071584 14.8867559432983 1.0
13.587028308 14.4228944778442 2.0
-4.5603535805 -7.07263088226318 3.0
9.026674718 8.6599006652832 4.0
20.952425173 21.9011154174805 5.0
29.9790999 31.8999938964844 6.0
29.9790999 29.1347827911377 7.0
40.412374419 38.1481323242188 0.0
21.632869342 19.2404441833496 1.0
18.779505078 19.3067245483398 2.0
-4.6354756287 -7.43406772613525 3.0
14.14402944 13.6475553512573 4.0
26.26834498 26.1334705352783 5.0
40.412374428 39.5415802001953 6.0
40.412374428 39.9898986816406 7.0
44.7996696066759 37.9980392456055 0.0
24.3565352806759 21.7528915405273 1.0
20.4431343366759 21.9079704284668 2.0
-4.33926525477587 -6.89923667907715 3.0
16.1038690816759 15.8342761993408 4.0
28.6958005009671 27.6428508758545 5.0
44.799669607 44.8251876831055 6.0
44.799669607 44.7814636230469 7.0
35.691930078 39.358154296875 0.0
18.173171081 17.9495601654053 1.0
17.518759 18.1219100952148 2.0
-4.3926482071 -5.22031116485596 3.0
13.126110785 12.507830619812 4.0
22.565819295 22.0303707122803 5.0
35.691930085 36.1713790893555 6.0
35.691930085 36.1316299438477 7.0
37.293002373 37.5319519042969 0.0
17.018201783 18.5514183044434 1.0
20.274800591 17.8870372772217 2.0
-8.9983650782 -7.6791410446167 3.0
11.276435503 10.9807929992676 4.0
26.016566871 25.8867511749268 5.0
37.293002382 36.7497329711914 6.0
37.293002382 37.4043655395508 7.0
31.749050626 38.4727096557617 0.0
16.61083934 15.8307504653931 1.0
15.138211288 15.9538488388062 2.0
-3.9083704158 -5.61944961547852 3.0
11.229840863 10.8547801971436 4.0
20.519209764 19.7830390930176 5.0
31.749050635 32.5288543701172 6.0
31.749050635 32.7002868652344 7.0
44.43859931 38.7235336303711 0.0
22.567248059 22.5149822235107 1.0
21.871351255 20.7959957122803 2.0
-10.274790847 -9.63494777679443 3.0
11.596560402 10.814507484436 4.0
32.842038912 31.5965003967285 5.0
44.438599316 41.4154891967773 6.0
44.438599316 43.2081680297852 7.0
30.584762336 36.7992782592773 0.0
14.448986708 15.5870504379272 1.0
16.13577563 14.6688270568848 2.0
-7.0107275854 -6.67455816268921 3.0
9.1250480359 9.00351905822754 4.0
21.459714302 21.4524440765381 5.0
30.584762345 32.0151710510254 6.0
30.584762345 30.9605445861816 7.0
36.167711461 40.1857681274414 0.0
15.702748451 17.8237361907959 1.0
20.464963013 18.4717121124268 2.0
-4.5943051912 -3.68921661376953 3.0
15.870657815 15.2465696334839 4.0
20.297053649 19.1244277954102 5.0
36.167711467 36.386474609375 6.0
36.167711467 36.5115127563477 7.0
45.589023155 38.3743133544922 0.0
22.43199889 22.9356861114502 1.0
23.157024265 21.4268608093262 2.0
-9.6703339461 -8.2329273223877 3.0
13.486690309 12.9158134460449 4.0
32.102332846 30.6000347137451 5.0
45.589023165 44.1303176879883 6.0
45.589023165 45.0658645629883 7.0
33.954863965 38.4884490966797 0.0
16.569146941 17.1004333496094 1.0
17.385717024 15.977596282959 2.0
-8.6921868392 -9.10738182067871 3.0
8.6935301746 8.16158390045166 4.0
25.26133379 25.3978309631348 5.0
33.954863975 32.5750885009766 6.0
33.954863975 33.2103843688965 7.0
39.104021209 38.1976928710938 0.0
19.726585997 19.6325626373291 1.0
19.377435214 19.6350040435791 2.0
-5.8707549184 -7.07136917114258 3.0
13.506680288 12.9698495864868 4.0
25.597340923 25.1903820037842 5.0
39.104021217 39.1459655761719 6.0
39.104021217 39.5250015258789 7.0
41.6151393745108 39.4013900756836 0.0
21.6144474948183 23.5700511932373 1.0
20.0006918800463 22.4477863311768 2.0
-5.68615976863137 -9.3006649017334 3.0
14.3145321112399 13.2300615310669 4.0
27.3006071309532 31.0678310394287 5.0
41.615139375 47.0245132446289 6.0
41.615139375 46.963737487793 7.0
29.251209159 37.5830535888672 0.0
16.305978933 14.7973585128784 1.0
12.945230226 13.9417028427124 2.0
-4.8421763952 -6.6864709854126 3.0
8.1030538205 7.87594175338745 4.0
21.148155339 21.0361042022705 5.0
29.251209169 30.5941467285156 6.0
29.251209169 29.7554321289062 7.0
39.145261593 38.5635375976562 0.0
20.853611949 19.8906574249268 1.0
18.291649647 18.9076118469238 2.0
-6.3281983222 -7.84693908691406 3.0
11.963451316 11.5609912872314 4.0
27.181810279 26.8756942749023 5.0
39.145261601 38.62890625 6.0
39.145261601 39.4422607421875 7.0
43.724223504 38.7131423950195 0.0
24.240212841 21.9978733062744 1.0
19.484010663 20.526388168335 2.0
-6.7543223688 -9.00374126434326 3.0
12.729688284 12.0810670852661 4.0
30.99453522 29.9828624725342 5.0
43.724223514 42.9493255615234 6.0
43.724223514 43.0419464111328 7.0
30.694427938 38.4703216552734 0.0
15.962392584 15.455174446106 1.0
14.732035355 14.5811805725098 2.0
-5.8529078341 -7.19793224334717 3.0
8.8791275119 8.61276054382324 4.0
21.815300427 22.3543128967285 5.0
30.694427948 30.4355545043945 6.0
30.694427948 30.8030529022217 7.0
46.899446071 41.2384796142578 0.0
22.810606072 23.9734020233154 1.0
24.08884 22.4933032989502 2.0
-12.207831656 -9.55136775970459 3.0
11.881008335 10.8042459487915 4.0
35.018437738 33.4036178588867 5.0
46.899446079 48.6947555541992 6.0
46.899446079 46.1362533569336 7.0
30.801741864 37.6285858154297 0.0
14.581609416 15.9768629074097 1.0
16.220132448 14.6345262527466 2.0
-7.6078809929 -6.72684097290039 3.0
8.6122514454 8.90434455871582 4.0
22.189490419 21.7099018096924 5.0
30.801741874 31.9071769714355 6.0
30.801741874 31.7369689941406 7.0
40.3889007570329 38.0543670654297 0.0
21.8864546730297 20.3379878997803 1.0
18.5024460860314 18.9957790374756 2.0
-8.35977776734346 -9.02691745758057 3.0
10.1426683190367 9.57519721984863 4.0
30.2462322648819 29.0913619995117 5.0
40.388900757 39.9377670288086 6.0
40.388900757 40.1006011962891 7.0
35.801686121 37.8537368774414 0.0
19.152651215 17.5728244781494 1.0
16.649034906 17.9854488372803 2.0
-1.2326560583 -3.47315120697021 3.0
15.416378838 14.7052917480469 4.0
20.385307283 19.0002346038818 5.0
35.801686131 35.7912902832031 6.0
35.801686131 36.2294082641602 7.0
43.275430292 37.515022277832 0.0
21.49768275 21.9793605804443 1.0
21.777747542 20.6276302337646 2.0
-8.5057364073 -8.83036231994629 3.0
13.272011125 12.5774688720703 4.0
30.003419167 29.1209125518799 5.0
43.275430302 43.2871246337891 6.0
43.275430302 43.1338653564453 7.0
36.6745338859035 38.2283172607422 0.0
17.2258453439028 18.5150699615479 1.0
19.4486885479032 18.3444747924805 2.0
-7.45674864010256 -7.02922916412354 3.0
11.9919399079021 11.4609842300415 4.0
24.6825939849038 24.3843479156494 5.0
36.674533885 36.7487487792969 6.0
36.674533885 37.4470138549805 7.0
42.2919299411196 38.5026702880859 0.0
20.8701371011196 20.9765014648438 1.0
21.4217928491193 20.4122142791748 2.0
-7.8990978219197 -8.14122104644775 3.0
13.5226950281196 12.5381097793579 4.0
28.769234923119 28.1013698577881 5.0
42.291929941 40.9769821166992 6.0
42.291929941 41.6912384033203 7.0
37.80686854 37.7174224853516 0.0
18.84529386 18.549186706543 1.0
18.961574682 19.310604095459 2.0
-3.0668288295 -4.97530174255371 3.0
15.894745843 15.2099599838257 4.0
21.912122698 21.1462707519531 5.0
37.80686855 38.2931594848633 6.0
37.80686855 37.5266723632812 7.0
36.296445798 38.1344528198242 0.0
19.144127702 17.7966156005859 1.0
17.152318097 18.110237121582 2.0
-4.4799623055 -6.88712787628174 3.0
12.672355782 12.2650785446167 4.0
23.624090017 23.5981693267822 5.0
36.296445808 35.3136367797852 6.0
36.296445808 36.5476531982422 7.0
26.786347975 42.6801834106445 0.0
13.739106858 13.5130414962769 1.0
13.047241117 13.2145490646362 2.0
-5.0192652199 -5.87567567825317 3.0
8.0279758877 7.60976409912109 4.0
18.758372088 18.9660835266113 5.0
26.786347985 28.9723720550537 6.0
26.786347985 28.2391414642334 7.0
38.715518927 38.507682800293 0.0
20.971478582 19.7143726348877 1.0
17.744040345 18.1030349731445 2.0
-7.5775969749 -9.30812072753906 3.0
10.16644336 9.681884765625 4.0
28.549075567 27.826286315918 5.0
38.715518937 37.4333877563477 6.0
38.715518937 38.1795654296875 7.0
35.505062995 38.513427734375 0.0
16.975132375 17.2819538116455 1.0
18.529930624 17.7054233551025 2.0
-6.1934391103 -6.34982109069824 3.0
12.336491507 12.0390348434448 4.0
23.168571492 23.276496887207 5.0
35.505063002 35.3272552490234 6.0
35.505063002 35.8840560913086 7.0
44.227649061 38.7107391357422 0.0
23.455343466 22.3802471160889 1.0
20.772305597 21.5785312652588 2.0
-7.272984789 -9.04497814178467 3.0
13.499320798 12.4648637771606 4.0
30.728328264 30.2675457000732 5.0
44.227649071 44.5139617919922 6.0
44.227649071 43.0474395751953 7.0
45.773124246 38.4521026611328 0.0
24.134680774 22.6842613220215 1.0
21.638443473 22.3891849517822 2.0
-6.2883820036 -7.2651834487915 3.0
15.350061461 14.9112253189087 4.0
30.423062786 29.5425624847412 5.0
45.773124255 43.8929138183594 6.0
45.773124255 44.469352722168 7.0
38.6152827522378 39.1381759643555 0.0
18.6636500942378 21.1879348754883 1.0
19.9516326682378 20.1070251464844 2.0
-8.33253161363777 -9.0809211730957 3.0
11.6191010542378 10.7593660354614 4.0
26.9961816644691 29.2537994384766 5.0
38.615282752 41.3080062866211 6.0
38.615282752 41.2782516479492 7.0
42.131124826 38.4854736328125 0.0
21.88265512 20.588752746582 1.0
20.248469706 20.2322254180908 2.0
-6.067617241 -7.61580181121826 3.0
14.180852455 13.8058958053589 4.0
27.950272371 27.9883079528809 5.0
42.131124836 40.829704284668 6.0
42.131124836 41.1903457641602 7.0
38.030369756 38.8973083496094 0.0
19.698625144 19.6794357299805 1.0
18.331744611 18.5008449554443 2.0
-8.0645153211 -8.2222261428833 3.0
10.26722928 9.75477409362793 4.0
27.763140475 27.2713928222656 5.0
38.030369766 38.3273696899414 6.0
38.030369766 38.1484527587891 7.0
32.694869995 38.2904281616211 0.0
14.429732866 16.1712837219238 1.0
18.265137131 16.580057144165 2.0
-5.8301144142 -4.95665502548218 3.0
12.435022709 12.3130111694336 4.0
20.259847288 19.9836120605469 5.0
32.694870002 33.1489868164062 6.0
32.694870002 33.3169479370117 7.0
41.58883183 41.6641159057617 0.0
19.184872754 20.3355979919434 1.0
22.403959077 20.9348182678223 2.0
-6.437067104 -5.56141662597656 3.0
15.966891964 15.7411727905273 4.0
25.621939867 24.977367401123 5.0
41.58883184 41.1224365234375 6.0
41.58883184 41.019416809082 7.0
33.622063035 38.832878112793 0.0
16.212159516 17.3043212890625 1.0
17.409903519 15.6869668960571 2.0
-8.2486263793 -7.31462860107422 3.0
9.16127713 8.79054832458496 4.0
24.460785905 24.6647968292236 5.0
33.622063045 33.4089508056641 6.0
33.622063045 32.7049369812012 7.0
40.101738497 39.5207138061523 0.0
22.613544165 20.7906303405762 1.0
17.488194333 18.8444232940674 2.0
-8.294425066 -10.4181060791016 3.0
9.1937692583 8.37910270690918 4.0
30.90796924 30.3288879394531 5.0
40.101738506 39.3720703125 6.0
40.101738506 40.6680603027344 7.0
39.347750484 38.0299377441406 0.0
19.631600055 20.4716777801514 1.0
19.716150429 19.8641872406006 2.0
-7.2110260491 -7.61504364013672 3.0
12.50512437 11.6787710189819 4.0
26.842626114 27.9090576171875 5.0
39.347750494 40.1459808349609 6.0
39.347750494 41.011962890625 7.0
46.240008453 38.8345947265625 0.0
23.277257726 22.0742149353027 1.0
22.962750727 22.1802196502686 2.0
-6.5077704351 -7.46328449249268 3.0
16.454980282 15.5035314559937 4.0
29.785028171 28.4560871124268 5.0
46.240008463 44.5308380126953 6.0
46.240008463 45.5188598632812 7.0
43.885613963 38.0525588989258 0.0
23.043389811 21.3101081848145 1.0
20.842224153 21.1093502044678 2.0
-5.1243468199 -6.55911016464233 3.0
15.717877324 15.2683639526367 4.0
28.16773664 27.1201992034912 5.0
43.885613972 43.3039321899414 6.0
43.885613972 43.0225067138672 7.0
44.877286045 38.8000869750977 0.0
23.85937511 22.7260208129883 1.0
21.017910935 20.5420475006104 2.0
-10.548874854 -10.5124816894531 3.0
10.469036071 9.67226791381836 4.0
34.408249974 32.6380271911621 5.0
44.877286055 43.9720916748047 6.0
44.877286055 44.9793243408203 7.0
44.28102484 38.0077743530273 0.0
23.476033214 22.1387367248535 1.0
20.804991626 21.0905380249023 2.0
-7.5751269732 -9.15844440460205 3.0
13.229864644 12.7179937362671 4.0
31.051160197 30.1925029754639 5.0
44.281024849 43.0857620239258 6.0
44.281024849 43.9228057861328 7.0
39.443195324 37.3810882568359 0.0
17.235953612 19.3434944152832 1.0
22.207241712 18.8450679779053 2.0
-9.075573722 -7.60132884979248 3.0
13.13166798 12.7153692245483 4.0
26.311527344 25.5991306304932 5.0
39.443195334 38.8532257080078 6.0
39.443195334 38.8245544433594 7.0
44.5099801195094 36.8591995239258 0.0
20.7192442195094 22.3555526733398 1.0
23.7907359105096 21.7317199707031 2.0
-9.37764966100969 -7.63202953338623 3.0
14.4130862495096 13.609694480896 4.0
30.0968938805093 28.47580909729 5.0
44.50998012 42.7778625488281 6.0
44.50998012 43.8332748413086 7.0
36.079576813 38.4415283203125 0.0
18.198560719 18.7957248687744 1.0
17.881016095 17.4483032226562 2.0
-8.6789161085 -9.28349304199219 3.0
9.202099978 8.59134674072266 4.0
26.877476836 27.5283660888672 5.0
36.079576822 36.163818359375 6.0
36.079576822 36.8237457275391 7.0
43.881758866 39.027214050293 0.0
21.090009056 25.6206130981445 1.0
22.79174981 23.5201053619385 2.0
-6.6710954927 -8.04027843475342 3.0
16.120654308 14.9540529251099 4.0
27.761104558 30.4392051696777 5.0
43.881758876 46.893798828125 6.0
43.881758876 49.5017700195312 7.0
30.108818274 38.0872573852539 0.0
17.112174764 14.8953723907471 1.0
12.996643511 14.1075763702393 2.0
-4.8964528045 -7.41787242889404 3.0
8.1001906976 8.07676124572754 4.0
22.008627577 22.1093921661377 5.0
30.108818284 30.7107715606689 6.0
30.108818284 29.7873954772949 7.0
34.2205121900012 38.3739700317383 0.0
18.8110553650029 17.6508178710938 1.0
15.4094568300067 16.343204498291 2.0
-7.05168531729669 -9.02465915679932 3.0
8.35777151259514 8.10700225830078 4.0
25.8627406830075 26.1589450836182 5.0
34.22051219 33.3663787841797 6.0
34.22051219 34.2508392333984 7.0
37.402197962 37.5958633422852 0.0
19.960548139 19.5640735626221 1.0
17.441649823 18.4205322265625 2.0
-5.9560955294 -7.01253890991211 3.0
11.485554285 10.9263305664062 4.0
25.916643678 25.4087047576904 5.0
37.402197971 38.1326751708984 6.0
37.402197971 37.9016265869141 7.0
35.453722399 38.0886306762695 0.0
16.625051972 17.825475692749 1.0
18.828670427 17.2883014678955 2.0
-7.5918246106 -7.05069732666016 3.0
11.236845807 10.8335590362549 4.0
24.216876592 23.7833080291748 5.0
35.453722409 34.7095642089844 6.0
35.453722409 35.6290512084961 7.0
32.872381145 38.9959030151367 0.0
15.679111706 16.1414489746094 1.0
17.193269439 16.6131019592285 2.0
-6.3769976884 -6.32965803146362 3.0
10.816271741 10.4307107925415 4.0
22.056109405 22.1622524261475 5.0
32.872381155 32.7730331420898 6.0
32.872381155 33.1625518798828 7.0
33.5777143162235 38.2724685668945 0.0
24.3043372519187 17.6195983886719 1.0
9.27337706863705 16.3525238037109 2.0
-1.413226005e-07 -8.37569046020508 3.0
9.27337692732672 8.6875524520874 4.0
24.3043373936238 24.3537464141846 5.0
33.577714316 33.2163162231445 6.0
33.577714316 33.0355224609375 7.0
43.986168792 39.041877746582 0.0
19.73252431 21.5856075286865 1.0
24.253644482 20.8663463592529 2.0
-7.8587048092 -5.91611337661743 3.0
16.394939663 15.5781421661377 4.0
27.591229129 26.2266521453857 5.0
43.986168802 41.7158050537109 6.0
43.986168802 42.3371353149414 7.0
39.701527196 38.9045562744141 0.0
21.433905404 19.840934753418 1.0
18.267621793 19.1758861541748 2.0
-6.9530592881 -8.71702003479004 3.0
11.314562495 10.6607789993286 4.0
28.386964701 28.2417583465576 5.0
39.701527205 39.0857620239258 6.0
39.701527205 39.897834777832 7.0
36.319422307 37.8450088500977 0.0
18.261170173 19.0830535888672 1.0
18.058252134 17.1586532592773 2.0
-9.3076661868 -8.92712688446045 3.0
8.7505859373 8.37872791290283 4.0
27.56883637 27.2055644989014 5.0
36.319422317 36.7306365966797 6.0
36.319422317 37.2687530517578 7.0
39.546529711 38.6017227172852 0.0
21.883828824 20.6165561676025 1.0
17.662700888 19.2193470001221 2.0
-6.2516235274 -7.78132343292236 3.0
11.411077352 10.8421440124512 4.0
28.13545236 27.5925903320312 5.0
39.54652972 39.2355422973633 6.0
39.54652972 39.8413543701172 7.0
46.328884399 38.8549118041992 0.0
20.182008475 22.8751964569092 1.0
26.146875933 22.5814533233643 2.0
-10.473669833 -8.06719207763672 3.0
15.673206099 14.8391761779785 4.0
30.655678309 29.4269580841064 5.0
46.3288844 45.0779724121094 6.0
46.3288844 46.2499313354492 7.0
39.359671453 38.194091796875 0.0
20.208221774 20.3423404693604 1.0
19.151449681 19.1097717285156 2.0
-7.7565035076 -8.42878341674805 3.0
11.394946165 10.7688608169556 4.0
27.964725289 27.7421894073486 5.0
39.359671461 39.5275573730469 6.0
39.359671461 40.1919708251953 7.0
43.813870978 38.8582763671875 0.0
22.861832963 22.0358791351318 1.0
20.952038016 21.2789440155029 2.0
-6.6064627953 -8.3498363494873 3.0
14.345575211 13.5406513214111 4.0
29.468295768 29.1588001251221 5.0
43.813870988 43.1231842041016 6.0
43.813870988 42.5593948364258 7.0
39.146221866 37.9765167236328 0.0
21.274977313 20.3604354858398 1.0
17.871244553 18.3191699981689 2.0
-8.221666714 -9.06628322601318 3.0
9.6495778292 9.37703990936279 4.0
29.496644037 28.4865341186523 5.0
39.146221876 38.9574432373047 6.0
39.146221876 39.3885269165039 7.0
34.753353004 43.2433700561523 0.0
19.306587872 19.1949615478516 1.0
15.446765133 17.0121841430664 2.0
-7.4077231802 -8.44002819061279 3.0
8.0390419425 7.87218189239502 4.0
26.714311062 26.8391723632812 5.0
34.753353014 35.7559432983398 6.0
34.753353014 35.3633346557617 7.0
36.8589150083515 39.3040237426758 0.0
18.5606086093515 19.5789794921875 1.0
18.2983064083517 18.169942855835 2.0
-8.423068234052 -8.68423557281494 3.0
9.87523817465195 9.27243614196777 4.0
26.9836767778432 27.2485866546631 5.0
36.858915008 37.5935211181641 6.0
36.858915008 38.6410980224609 7.0
45.609388715 38.4172973632812 0.0
20.431056468 22.9536399841309 1.0
25.178332247 22.4702548980713 2.0
-8.8685199312 -6.5566611289978 3.0
16.309812306 15.7383069992065 4.0
29.299576409 28.2491645812988 5.0
45.609388725 43.5644912719727 6.0
45.609388725 45.0015563964844 7.0
37.221502305 38.3516006469727 0.0
19.795691461 19.6236991882324 1.0
17.425810844 17.979154586792 2.0
-8.0004149655 -8.99427700042725 3.0
9.4253958691 8.80390930175781 4.0
27.796106436 27.9256324768066 5.0
37.221502314 37.9484024047852 6.0
37.221502314 37.553581237793 7.0
39.3512418478291 37.6393203735352 0.0
21.5003115557496 20.3177757263184 1.0
17.8509302889164 18.7557277679443 2.0
-7.9913457065862 -8.93670558929443 3.0
9.85958458276425 9.30246639251709 4.0
29.4916572628094 28.8016185760498 5.0
39.351241848 38.7210159301758 6.0
39.351241848 39.3560943603516 7.0
44.354286129 38.2987594604492 0.0
23.778468789 22.5706043243408 1.0
20.57581734 21.4556121826172 2.0
-6.7304183908 -8.56882190704346 3.0
13.845398939 13.29296875 4.0
30.50888719 29.5489082336426 5.0
44.354286139 43.3493347167969 6.0
44.354286139 44.251838684082 7.0
41.1438439687938 38.1038208007812 0.0
20.0722524531884 20.0741729736328 1.0
21.0715915195 19.5480842590332 2.0
-7.89338141352269 -8.26870822906494 3.0
13.1782101064826 12.4874887466431 4.0
27.9656338666077 27.7842121124268 5.0
41.143843969 40.4757919311523 6.0
41.143843969 40.8191757202148 7.0
32.436363662 39.0113754272461 0.0
18.29560079 16.6161632537842 1.0
14.140762872 16.010425567627 2.0
-4.3766657499 -6.25706720352173 3.0
9.7640971121 9.62257289886475 4.0
22.67226655 22.4643726348877 5.0
32.436363672 33.1113319396973 6.0
32.436363672 33.7142486572266 7.0
35.980268817 38.7608642578125 0.0
18.371260578 19.0526332855225 1.0
17.609008241 17.4489593505859 2.0
-7.5074447835 -7.72255992889404 3.0
10.101563449 9.86926555633545 4.0
25.87870537 25.642749786377 5.0
35.980268826 38.5067367553711 6.0
35.980268826 37.0414428710938 7.0
39.361696941 38.5839385986328 0.0
18.798616943 18.5381984710693 1.0
20.563079997 19.60231590271 2.0
-4.5015726852 -5.77445220947266 3.0
16.061507302 15.1822214126587 4.0
23.300189639 22.9885711669922 5.0
39.361696951 39.0705871582031 6.0
39.361696951 39.5996551513672 7.0
38.339957491 38.8900909423828 0.0
20.325182647 21.0611763000488 1.0
18.014774844 20.0291538238525 2.0
-6.00621 -8.46483707427979 3.0
12.008564835 11.5466804504395 4.0
26.331392656 29.179708480835 5.0
38.339957501 41.6321411132812 6.0
38.339957501 41.3705215454102 7.0
38.4370780266379 38.1817092895508 0.0
19.1112284596379 22.9684066772461 1.0
19.3258495766379 20.8143291473389 2.0
-8.41787501523793 -10.4658393859863 3.0
10.9079745616379 9.77639198303223 4.0
27.5291034382175 32.4787940979004 5.0
38.437078026 43.5633163452148 6.0
38.437078026 42.662971496582 7.0
31.680301152 38.6968383789062 0.0
16.922816031 15.7942514419556 1.0
14.757485122 14.8312072753906 2.0
-6.2787127987 -8.25482273101807 3.0
8.4787723148 7.88749504089355 4.0
23.201528838 23.5726051330566 5.0
31.680301161 30.23606300354 6.0
31.680301161 30.2849559783936 7.0
36.698769122 39.2011032104492 0.0
19.594432034 17.6511535644531 1.0
17.104337088 18.1099872589111 2.0
-2.472193454 -4.85581254959106 3.0
14.632143624 14.4233636856079 4.0
22.066625498 21.2976608276367 5.0
36.698769132 35.2066955566406 6.0
36.698769132 36.1286392211914 7.0
40.942334912 38.3624801635742 0.0
22.219375241 19.7867298126221 1.0
18.722959671 19.6671524047852 2.0
-5.2427967213 -7.02317523956299 3.0
13.48016294 12.8304309844971 4.0
27.462171973 26.1630973815918 5.0
40.942334922 40.4672698974609 6.0
40.942334922 40.4371871948242 7.0
41.496981527 37.8840942382812 0.0
18.951043784 22.03586769104 1.0
22.545937743 21.4198608398438 2.0
-7.7537930795 -8.08576393127441 3.0
14.792144654 14.3052701950073 4.0
26.704836873 29.4385719299316 5.0
41.496981537 44.5245513916016 6.0
41.496981537 44.6415786743164 7.0
33.577466055 36.7898712158203 0.0
17.755510174 17.1780185699463 1.0
15.821955881 15.8019666671753 2.0
-7.358985858 -7.47995376586914 3.0
8.4629700135 8.25488758087158 4.0
25.114496042 24.0903701782227 5.0
33.577466065 34.1602630615234 6.0
33.577466065 33.8193740844727 7.0
38.410697594746 39.6828231811523 0.0
20.0282722997678 18.9050846099854 1.0
18.3824252977607 20.1255741119385 2.0
-2.25462323941058 -4.23089838027954 3.0
16.1278020587646 15.6055974960327 4.0
22.2828955387576 21.443187713623 5.0
38.410697595 38.4636077880859 6.0
38.410697595 38.7058334350586 7.0
40.390522102 39.0937118530273 0.0
23.101316011 20.7243633270264 1.0
17.289206093 18.6004695892334 2.0
-8.5131515481 -10.4230661392212 3.0
8.7760545352 8.47414588928223 4.0
31.614467568 30.8545036315918 5.0
40.390522112 39.4214401245117 6.0
40.390522112 40.6173629760742 7.0
38.02724698 38.0636672973633 0.0
17.092159812 18.586088180542 1.0
20.935087168 19.0112609863281 2.0
-7.0096758313 -6.10312986373901 3.0
13.925411327 13.6782245635986 4.0
24.101835653 23.475118637085 5.0
38.02724699 38.3910369873047 6.0
38.02724699 38.6760025024414 7.0
34.831322289 38.1769027709961 0.0
17.981428364 17.1401195526123 1.0
16.849893928 17.1651268005371 2.0
-5.7085713019 -7.30678272247314 3.0
11.14132262 10.6464405059814 4.0
23.689999673 23.5980854034424 5.0
34.831322295 34.5875854492188 6.0
34.831322295 35.0507965087891 7.0
33.152476822 39.5353088378906 0.0
16.395481675 16.7693042755127 1.0
16.756995148 16.0176315307617 2.0
-6.4008279506 -6.76018333435059 3.0
10.356167188 9.77749252319336 4.0
22.796309635 22.7096328735352 5.0
33.152476832 33.6477966308594 6.0
33.152476832 31.664514541626 7.0
37.7618147597387 38.8365478515625 0.0
20.5714887837879 19.5433959960938 1.0
17.1903259800247 18.5501022338867 2.0
-5.43010176849644 -7.48969841003418 3.0
11.7602242121808 11.0746173858643 4.0
26.0015905517818 25.8585243225098 5.0
37.76181476 37.0658569335938 6.0
37.76181476 38.1935043334961 7.0
46.821650957 38.3199157714844 0.0
22.453957792 23.5474395751953 1.0
24.367693165 22.5898189544678 2.0
-10.797760543 -9.05830097198486 3.0
13.569932612 12.4644193649292 4.0
33.251718344 31.6772708892822 5.0
46.821650967 46.1562271118164 6.0
46.821650967 46.6109848022461 7.0
41.1837901754312 38.6112060546875 0.0
19.2503118354314 20.2322578430176 1.0
21.9334783494314 20.0590953826904 2.0
-7.5443023473312 -6.67366218566895 3.0
14.3891760014314 13.9186019897461 4.0
26.7946141824314 26.1711616516113 5.0
41.183790175 40.1880493164062 6.0
41.183790175 40.5314788818359 7.0
36.784988632 38.8896408081055 0.0
21.035700431 19.03271484375 1.0
15.749288201 17.5402450561523 2.0
-6.0648376105 -8.63184070587158 3.0
9.6844505818 9.24567317962646 4.0
27.100538051 26.8921699523926 5.0
36.78498864 36.8820114135742 6.0
36.78498864 37.6822967529297 7.0
37.6894949442102 38.8402252197266 0.0
17.6138134052102 22.3506031036377 1.0
20.0756815502102 20.9995899200439 2.0
-8.78806436921022 -9.89967250823975 3.0
11.2876171802102 10.2714033126831 4.0
26.401877737898 31.1973533630371 5.0
37.689494945 43.9602508544922 6.0
37.689494945 42.8615112304688 7.0
41.308881088 37.9395294189453 0.0
21.148817904 21.0546417236328 1.0
20.160063185 20.1850166320801 2.0
-6.8065142938 -8.00381851196289 3.0
13.353548882 12.6052093505859 4.0
27.955332207 27.4679794311523 5.0
41.308881097 41.778434753418 6.0
41.308881097 41.7641677856445 7.0
42.253011175 37.220344543457 0.0
21.557015891 21.945873260498 1.0
20.695995284 20.3488464355469 2.0
-9.3835381575 -9.3173999786377 3.0
11.312457118 10.7231931686401 4.0
30.940554058 30.3450355529785 5.0
42.253011184 41.5468368530273 6.0
42.253011184 42.3424377441406 7.0
40.047660483 39.439697265625 0.0
19.392839472 21.3383140563965 1.0
20.654821011 20.470890045166 2.0
-7.4809136797 -7.925124168396 3.0
13.173907322 12.2221212387085 4.0
26.873753161 27.2404041290283 5.0
40.047660493 43.3388900756836 6.0
40.047660493 40.7363052368164 7.0
46.604605723 40.7205352783203 0.0
22.954120862 22.8051605224609 1.0
23.650484861 23.0985088348389 2.0
-7.9204617962 -7.7554407119751 3.0
15.730023055 15.3602828979492 4.0
30.874582668 29.926815032959 5.0
46.604605733 47.6629104614258 6.0
46.604605733 46.2474212646484 7.0
45.106295532 38.264518737793 0.0
23.2126055 22.8173294067383 1.0
21.893690031 21.1630477905273 2.0
-10.542381522 -10.2076473236084 3.0
11.3513085 10.5679950714111 4.0
33.754987032 32.8176498413086 5.0
45.106295542 43.587158203125 6.0
45.106295542 44.4005889892578 7.0
38.977055089 39.6535949707031 0.0
19.684728957 19.9886054992676 1.0
19.292326132 18.5582733154297 2.0
-8.6831116468 -9.32623386383057 3.0
10.609214475 9.9409351348877 4.0
28.367840614 28.3870124816895 5.0
38.977055099 39.1203765869141 6.0
38.977055099 38.0659027099609 7.0
38.935365873 38.247673034668 0.0
21.088579762 18.8313083648682 1.0
17.846786112 19.5821418762207 2.0
-3.2985252296 -5.88558101654053 3.0
14.548260873 13.8563652038574 4.0
24.387105001 23.9787139892578 5.0
38.935365883 38.69384765625 6.0
38.935365883 39.1697158813477 7.0
42.74140991 38.2397308349609 0.0
20.9458217 20.9357070922852 1.0
21.795588212 20.7361640930176 2.0
-6.3794891462 -5.80559158325195 3.0
15.416099058 14.8113632202148 4.0
27.325310854 25.7390193939209 5.0
42.741409918 42.1773986816406 6.0
42.741409918 42.1405639648438 7.0
31.310794567 38.8892364501953 0.0
16.880318068 15.570930480957 1.0
14.4304765 14.8110074996948 2.0
-5.6188756586 -6.82168197631836 3.0
8.8116008333 8.73148727416992 4.0
22.499193736 22.5182342529297 5.0
31.310794576 31.7474250793457 6.0
31.310794576 31.7730731964111 7.0
};
\addlegendentry{$R^2$=0.983}
\end{axis}

\end{tikzpicture}
}}
    
    \caption{Model results using only the loss associated with nodal flow predictions in the 8-node network.}
    \label{fig:dummy_base_results}
\end{figure}


The second part of this experiment involves the additional loss associated with the gas balance, building upon the previous setup that considered both node and edge losses. The hyperparameter optimization yielded the best parameters: $N channels =61$, $N layers =2$, and $N dense=2$. These settings resulted in a total loss of 10.041, with a node loss of 2.850, an edge loss of 6.414, and a balance loss of 0.776.

The prediction behavior at the nodes, as shown in \cref{fig:results_nonlineal_dummy_node_base_f_bal}, remained consistent with the results obtained in the previous experiment, where the balance loss was not included. The model accurately captured the gas injection pattern, with an $R^2$ of 0.983 for node flow predictions, identical to the earlier case.

Similarly, the prediction of edge flows, shown in \cref{fig:results_nonlineal_dummy_edge_base_f_bal}, followed the same general trend as before, although a slight decrease in accuracy was observed, reflected by an $R^2$ of 0.973. While this represents a minor reduction in performance compared to the previous experiment, the model still demonstrated a strong ability to predict gas flows through the edges, maintaining a high level of accuracy.

\begin{figure}
    \centering
    \setlength\figurewidth{.53\textwidth}        
    \setlength\figureheight{0.36\textwidth} 
    \subfloat[Actual vs predicted nodal flows.] 
    {\label{fig:results_nonlineal_dummy_node_base_f_bal}\resizebox{\figurewidth}{\figureheight}{% This file was created with tikzplotlib v0.10.1.
\begin{tikzpicture}

\definecolor{darkgray176}{RGB}{176,176,176}
\definecolor{lightgray204}{RGB}{204,204,204}

\begin{axis}[
colorbar,
colorbar style={ylabel={node id}},
colormap={mymap}{[1pt]
 rgb(0pt)=(0.12156862745098,0.466666666666667,0.705882352941177);
  rgb(1pt)=(1,0.498039215686275,0.0549019607843137);
  rgb(2pt)=(0.172549019607843,0.627450980392157,0.172549019607843);
  rgb(3pt)=(0.83921568627451,0.152941176470588,0.156862745098039);
  rgb(4pt)=(0.580392156862745,0.403921568627451,0.741176470588235);
  rgb(5pt)=(0.549019607843137,0.337254901960784,0.294117647058824);
  rgb(6pt)=(0.890196078431372,0.466666666666667,0.76078431372549);
  rgb(7pt)=(0.498039215686275,0.498039215686275,0.498039215686275);
  rgb(8pt)=(0.737254901960784,0.741176470588235,0.133333333333333);
  rgb(9pt)=(0.0901960784313725,0.745098039215686,0.811764705882353)
},
legend cell align={left},
legend style={
  fill opacity=0.8,
  draw opacity=1,
  text opacity=1,
  at={(0.03,0.97)},
  anchor=north west,
  draw=lightgray204
},
point meta max=7,
point meta min=0,
tick align=outside,
tick pos=left,
title={},
x grid style={darkgray176},
xlabel={True},
xmajorgrids,
xmin=-2.43954548935, xmax=51.23045527635,
xtick={0,10,20,30,40,50}, 
xticklabels={0,10,20,30,40,$f_n$}, 
xtick style={color=black},
y grid style={darkgray176},
ylabel={Predicted},
ymajorgrids,
ymin=-4.67627371549606, ymax=42.686191213131,
ytick={0,10,20,30,40}, 
yticklabels={0,10,20,30,$f_n$}, 
ytick style={color=black}
]
\addplot [
  colormap={mymap}{[1pt]
 rgb(0pt)=(0.12156862745098,0.466666666666667,0.705882352941177);
  rgb(1pt)=(1,0.498039215686275,0.0549019607843137);
  rgb(2pt)=(0.172549019607843,0.627450980392157,0.172549019607843);
  rgb(3pt)=(0.83921568627451,0.152941176470588,0.156862745098039);
  rgb(4pt)=(0.580392156862745,0.403921568627451,0.741176470588235);
  rgb(5pt)=(0.549019607843137,0.337254901960784,0.294117647058824);
  rgb(6pt)=(0.890196078431372,0.466666666666667,0.76078431372549);
  rgb(7pt)=(0.498039215686275,0.498039215686275,0.498039215686275);
  rgb(8pt)=(0.737254901960784,0.741176470588235,0.133333333333333);
  rgb(9pt)=(0.0901960784313725,0.745098039215686,0.811764705882353)
},
  only marks,
  scatter,
  scatter src=explicit
]
table [x=x, y=y, meta=colordata]{%
x  y  colordata
39.565898645 38.7807159423828 0.0
0 0.0390294790267944 1.0
0 0.687381327152252 2.0
0 0.150869160890579 3.0
0 0.0302326381206512 4.0
0 -0.0184029042720795 5.0
0 0.0203696489334106 6.0
0 0.0104289948940277 7.0
42.743257181 39.911994934082 0.0
0 -0.050416499376297 1.0
0 1.11706292629242 2.0
0 0.665011584758759 3.0
0 -0.0553262531757355 4.0
0 0.104341715574265 5.0
0 0.000722020864486694 6.0
0 -0.0158502161502838 7.0
39.367180751 39.2427749633789 0.0
0 0.0295336842536926 1.0
0 0.622588634490967 2.0
0 0.21805602312088 3.0
0 -0.128991514444351 4.0
0 0.102068245410919 5.0
0 0.0319209396839142 6.0
0 -0.0390597283840179 7.0
39.605393107 38.4911880493164 0.0
0 0.0556691884994507 1.0
0 0.640954494476318 2.0
0 -0.0553026497364044 3.0
0 0.115547597408295 4.0
0 -0.101797610521317 5.0
0 0.0163717865943909 6.0
0 0.117020338773727 7.0
43.937345536 38.4560508728027 0.0
0 0.0386702120304108 1.0
0 1.32853770256042 2.0
0 0.306313931941986 3.0
0 0.095480740070343 4.0
0 0.0786909461021423 5.0
0 0.0125740468502045 6.0
0 0.0155363082885742 7.0
31.061989594 38.1250534057617 0.0
0 0.050401359796524 1.0
0 -1.48016977310181 2.0
0 -0.591398477554321 3.0
0 -0.0786186754703522 4.0
0 -0.202126950025558 5.0
0 0.0457426905632019 6.0
0 -0.0259649455547333 7.0
36.357435273 38.6034545898438 0.0
0 0.0115385055541992 1.0
0 -0.49808344244957 2.0
0 -0.368038982152939 3.0
0 0.120659798383713 4.0
0 0.0127613544464111 5.0
0 -0.00104102492332458 6.0
0 0.0507091581821442 7.0
38.444969617 38.7404708862305 0.0
0 -0.0528589189052582 1.0
0 0.550376653671265 2.0
0 -0.0229515433311462 3.0
0 -0.00640097260475159 4.0
0 -0.0821091830730438 5.0
0 0.0600452721118927 6.0
0 -0.00556716322898865 7.0
35.498620524 38.0814628601074 0.0
0 0.0222894549369812 1.0
0 -0.352378875017166 2.0
0 -0.198789328336716 3.0
0 -0.164021879434586 4.0
0 -0.0660082995891571 5.0
0 0.0521028637886047 6.0
0 -0.0196513235569 7.0
36.520998279 39.1113510131836 0.0
0 -0.0385665595531464 1.0
0 0.191477179527283 2.0
0 -0.219058126211166 3.0
0 -0.0586468875408173 4.0
0 -0.046745091676712 5.0
0 0.0614476203918457 6.0
0 -0.0647366344928741 7.0
36.717272212 38.4550170898438 0.0
0 -0.0505353510379791 1.0
0 0.578135550022125 2.0
0 -0.240118831396103 3.0
0 0.101209104061127 4.0
0 -0.0214777588844299 5.0
0 0.0732402503490448 6.0
0 -0.00158435106277466 7.0
32.629629006 39.3894577026367 0.0
0 -0.0220723152160645 1.0
0 -1.4050704240799 2.0
0 -0.717639207839966 3.0
0 0.0596912205219269 4.0
0 -0.104845851659775 5.0
0 0.00380429625511169 6.0
0 -0.047752171754837 7.0
37.75267434 35.0604705810547 0.0
0 -0.0386965572834015 1.0
0 -0.483706146478653 2.0
0 0.116963267326355 3.0
0 -0.259949237108231 4.0
0 -0.059516578912735 5.0
0 0.025240957736969 6.0
0 -0.0507874190807343 7.0
38.800291347 37.6808929443359 0.0
0 0.0540114641189575 1.0
0 0.733562111854553 2.0
0 -0.0290373265743256 3.0
0 0.0834235548973083 4.0
0 0.00605052709579468 5.0
0 0.0006142258644104 6.0
0 0.0391792058944702 7.0
38.252729618 38.6630706787109 0.0
0 0.0380311906337738 1.0
0 0.485172003507614 2.0
0 -0.0532437860965729 3.0
0 0.0433309674263 4.0
0 -0.052877277135849 5.0
0 0.0241513550281525 6.0
0 0.0370899140834808 7.0
43.273596047 39.6313591003418 0.0
0 -0.0453270971775055 1.0
0 1.40773439407349 2.0
0 0.362457275390625 3.0
0 0.136399984359741 4.0
0 0.0658433437347412 5.0
0 -0.029952198266983 6.0
0 0.178600668907166 7.0
34.027431486 39.1221122741699 0.0
0 0.0232710242271423 1.0
0 -0.696680545806885 2.0
0 -0.538212418556213 3.0
0 0.000645667314529419 4.0
0 -0.134311765432358 5.0
0 -0.0366456210613251 6.0
0 -0.0378514230251312 7.0
41.154171391 39.0699729919434 0.0
0 0.00734546780586243 1.0
0 0.803988695144653 2.0
0 0.111234694719315 3.0
0 0.140818446874619 4.0
0 -0.103350549936295 5.0
0 0.0438562333583832 6.0
0 0.0681416094303131 7.0
39.930826408 38.1136665344238 0.0
0 -0.0125821828842163 1.0
0 0.364738523960114 2.0
0 0.0268895328044891 3.0
0 0.145027428865433 4.0
0 -0.0421799123287201 5.0
0 0.0325004756450653 6.0
0 0.0422276258468628 7.0
45.969703631 37.899543762207 0.0
0 0.0031130313873291 1.0
0 2.4158718585968 2.0
0 0.788434028625488 3.0
0 0.0883679687976837 4.0
0 0.125244677066803 5.0
0 0.00427520275115967 6.0
0 0.0258832275867462 7.0
38.398120221 37.6985626220703 0.0
0 -0.0212984383106232 1.0
0 0.680952966213226 2.0
0 -0.215259343385696 3.0
0 0.10072335600853 4.0
0 -0.0387965142726898 5.0
0 -0.0199668705463409 6.0
0 0.130744218826294 7.0
28.366017306 39.4864692687988 0.0
0 -0.0634640157222748 1.0
0 -1.09842574596405 2.0
0 -0.86850118637085 3.0
0 -0.242926090955734 4.0
0 -0.177257031202316 5.0
0 -0.0320635139942169 6.0
0 -0.0255875289440155 7.0
39.079818979 38.2649459838867 0.0
0 -0.00661382079124451 1.0
0 0.129336774349213 2.0
0 0.235322862863541 3.0
0 -0.0668224394321442 4.0
0 0.0806121826171875 5.0
0 0.0267211496829987 6.0
0 -0.0465520322322845 7.0
40.466329383 39.195671081543 0.0
0 -0.0196767747402191 1.0
0 1.07860553264618 2.0
0 0.429133206605911 3.0
0 -0.0737719237804413 4.0
0 0.0188650488853455 5.0
0 0.0279071033000946 6.0
0 -0.0180650353431702 7.0
38.293116853 38.33984375 0.0
0 -0.0387757122516632 1.0
0 0.502089083194733 2.0
0 0.112023651599884 3.0
0 -0.269812732934952 4.0
0 -0.041276603937149 5.0
0 -0.0002765953540802 6.0
0 -0.0504074394702911 7.0
43.346089497 39.4515991210938 0.0
0 -0.0439045131206512 1.0
0 1.26294469833374 2.0
0 0.504219770431519 3.0
0 0.153600782155991 4.0
0 0.0344879329204559 5.0
0 0.0195392668247223 6.0
0 0.0326660871505737 7.0
34.547871154 39.4455642700195 0.0
0 -0.0062001645565033 1.0
0 -0.817020416259766 2.0
0 -0.73096239566803 3.0
0 0.14644929766655 4.0
0 -0.110113710165024 5.0
0 -0.0494408309459686 6.0
0 0.457633286714554 7.0
41.927050143 39.2945899963379 0.0
0 -0.036404937505722 1.0
0 1.07193267345428 2.0
0 0.13761043548584 3.0
0 0.296372592449188 4.0
0 0.0686601102352142 5.0
0 -0.0449710786342621 6.0
0 0.385504633188248 7.0
44.548236377 38.1797866821289 0.0
0 0.00879600644111633 1.0
0 1.44248878955841 2.0
0 0.750518977642059 3.0
0 0.0611172616481781 4.0
0 0.127862185239792 5.0
0 -0.00452166795730591 6.0
0 0.0121083557605743 7.0
34.958415487 39.6784706115723 0.0
0 -0.014015793800354 1.0
0 0.205074638128281 2.0
0 -0.277805835008621 3.0
0 -0.137114852666855 4.0
0 0.0318798124790192 5.0
0 0.0158077776432037 6.0
0 0.0155801475048065 7.0
41.619773298 37.7818336486816 0.0
0 -0.0277369022369385 1.0
0 0.819820761680603 2.0
0 0.308770775794983 3.0
0 -0.0267046391963959 4.0
0 0.0859882533550262 5.0
0 0.017451137304306 6.0
0 0.0918035209178925 7.0
35.768623912 38.4961814880371 0.0
0 -0.115918964147568 1.0
0 0.13489043712616 2.0
0 -0.310698360204697 3.0
0 0.00500822067260742 4.0
0 -0.136931449174881 5.0
0 0.0267219245433807 6.0
0 -0.0023399293422699 7.0
36.03587308 35.9560890197754 0.0
0 -0.00718718767166138 1.0
0 -0.0409426391124725 2.0
0 -0.476732939481735 3.0
0 0.136874735355377 4.0
0 -0.0456999838352203 5.0
0 0.0140937864780426 6.0
0 0.0560321509838104 7.0
42.671159584 38.1353454589844 0.0
0 0.029904842376709 1.0
0 1.0791552066803 2.0
0 0.242302224040031 3.0
0 0.147656917572021 4.0
0 -0.108803302049637 5.0
0 0.0390794277191162 6.0
0 0.13880866765976 7.0
32.270518391 38.8226852416992 0.0
0 -0.0637445747852325 1.0
0 -0.273482471704483 2.0
0 -0.753962635993958 3.0
0 0.00228622555732727 4.0
0 -0.215895861387253 5.0
0 0.014941930770874 6.0
0 -0.0458433926105499 7.0
36.239834941 38.291431427002 0.0
0 0.0435711443424225 1.0
0 0.184087783098221 2.0
0 -0.162597090005875 3.0
0 -0.0698451697826385 4.0
0 -0.0356323421001434 5.0
0 0.0374956727027893 6.0
0 -0.0381764471530914 7.0
38.292005181 38.0417289733887 0.0
0 -0.0120866000652313 1.0
0 0.320809781551361 2.0
0 0.102484226226807 3.0
0 0.0978689193725586 4.0
0 0.0252014994621277 5.0
0 0.0185411870479584 6.0
0 -0.0216188132762909 7.0
41.142171126 37.4512329101562 0.0
0 -0.0192840695381165 1.0
0 1.58292961120605 2.0
0 0.355861037969589 3.0
0 -0.0993565618991852 4.0
0 -0.0570743978023529 5.0
0 0.0116469264030457 6.0
0 0.0640183687210083 7.0
30.660234751 36.3844757080078 0.0
0 -0.13076850771904 1.0
0 -0.796864628791809 2.0
0 -0.901338219642639 3.0
0 -0.160846918821335 4.0
0 -0.11530002951622 5.0
0 0.0484358966350555 6.0
0 -0.0526761114597321 7.0
42.776716986 37.6662216186523 0.0
0 0.00617566704750061 1.0
0 1.3254154920578 2.0
0 0.48811212182045 3.0
0 0.0404703915119171 4.0
0 0.0840141475200653 5.0
0 -0.0165799856185913 6.0
0 0.0299546122550964 7.0
39.136657955 36.8314590454102 0.0
0 0.0690381824970245 1.0
0 0.229831755161285 2.0
0 0.21285405755043 3.0
0 0.00548025965690613 4.0
0 -0.0438046157360077 5.0
0 0.0150558948516846 6.0
0 -0.05980584025383 7.0
40.593075664 38.2367744445801 0.0
0 0.0331628620624542 1.0
0 0.89752459526062 2.0
0 0.0872803628444672 3.0
0 0.226244300603867 4.0
0 0.0155229866504669 5.0
0 -0.0634270012378693 6.0
0 0.0814640522003174 7.0
38.455960057 36.9916305541992 0.0
0 0.0138240158557892 1.0
0 0.159217864274979 2.0
0 -0.165718168020248 3.0
0 0.187530010938644 4.0
0 -0.100364953279495 5.0
0 -0.0115776658058167 6.0
0 0.06383416056633 7.0
40.029822307 38.4322891235352 0.0
0 0.0194771885871887 1.0
0 0.22092741727829 2.0
0 0.187972158193588 3.0
0 0.0653082132339478 4.0
0 0.0552197992801666 5.0
0 0.0304805338382721 6.0
0 -0.0350183546543121 7.0
39.721089806 37.3876647949219 0.0
0 -0.0653609335422516 1.0
0 0.323364734649658 2.0
0 0.0702760517597198 3.0
0 0.118442565202713 4.0
0 -0.118920892477036 5.0
0 0.047224223613739 6.0
0 0.00167503952980042 7.0
46.012802781 36.8888168334961 0.0
0 -0.0497294366359711 1.0
0 2.43094372749329 2.0
0 0.65622740983963 3.0
0 0.259718418121338 4.0
0 -0.062612384557724 5.0
0 0.0556562542915344 6.0
0 0.195971816778183 7.0
43.791041158 38.1203804016113 0.0
0 -0.0211540758609772 1.0
0 1.49984383583069 2.0
0 0.506956517696381 3.0
0 0.220430135726929 4.0
0 0.0478436946868896 5.0
0 0.0152004659175873 6.0
0 0.222033962607384 7.0
31.257332424 38.6225852966309 0.0
0 0.0679140388965607 1.0
0 -0.949542641639709 2.0
0 -0.870958209037781 3.0
0 -0.131458550691605 4.0
0 -0.089064747095108 5.0
0 -0.0745481550693512 6.0
0 -0.0132040083408356 7.0
38.98847291 38.532096862793 0.0
0 0.0268354713916779 1.0
0 0.235600620508194 2.0
0 0.280920475721359 3.0
0 -0.173323839902878 4.0
0 0.0665885806083679 5.0
0 0.0264682173728943 6.0
0 -0.0502153933048248 7.0
38.691218499 39.2169380187988 0.0
0 0.00567257404327393 1.0
0 0.863517880439758 2.0
0 0.125500619411469 3.0
0 -0.161862164735794 4.0
0 0.0928739309310913 5.0
0 -0.043161004781723 6.0
0 -0.0429538190364838 7.0
39.033211971 36.7828598022461 0.0
0 0.0172908008098602 1.0
0 0.57524973154068 2.0
0 -0.0312075912952423 3.0
0 0.157713979482651 4.0
0 -0.0222493410110474 5.0
0 0.0507160425186157 6.0
0 0.00584086775779724 7.0
37.697547813 38.1581649780273 0.0
0 0.0089167058467865 1.0
0 -0.209341913461685 2.0
0 -0.00371652841567993 3.0
0 -0.0635068118572235 4.0
0 0.119619578123093 5.0
0 0.00166866183280945 6.0
0 -0.0253732800483704 7.0
35.277541339 37.7204627990723 0.0
0 0.0371574759483337 1.0
0 -0.194770604372025 2.0
0 -0.337696880102158 3.0
0 -0.152936786413193 4.0
0 0.0365249514579773 5.0
0 0.048222690820694 6.0
0 0.00518062710762024 7.0
36.119763966 38.167236328125 0.0
0 0.097140371799469 1.0
0 -0.7493497133255 2.0
0 -0.291569620370865 3.0
0 -0.299669057130814 4.0
0 -0.0874418914318085 5.0
0 0.0340142846107483 6.0
0 -0.031021922826767 7.0
33.14490305 37.8338584899902 0.0
0 0.0297432243824005 1.0
0 -0.327435106039047 2.0
0 -0.484725922346115 3.0
0 -0.257625609636307 4.0
0 0.100316435098648 5.0
0 0.027045339345932 6.0
0 -0.0110397636890411 7.0
34.486800814 37.8205528259277 0.0
0 0.0309855341911316 1.0
0 -0.300163894891739 2.0
0 -0.770080208778381 3.0
0 0.111288994550705 4.0
0 -0.042367547750473 5.0
0 0.0198244750499725 6.0
0 0.765416741371155 7.0
40.933468207 37.683479309082 0.0
0 0.0355637967586517 1.0
0 0.535225331783295 2.0
0 0.325543463230133 3.0
0 0.0114805996417999 4.0
0 0.0126529335975647 5.0
0 0.0559045970439911 6.0
0 -0.0187086760997772 7.0
35.993417928 38.083194732666 0.0
0 -0.0178670287132263 1.0
0 0.119696497917175 2.0
0 -0.296382457017899 3.0
0 0.0717155039310455 4.0
0 -0.00309836864471436 5.0
0 -0.155146926641464 6.0
0 0.0193229019641876 7.0
39.331666923 37.9243850708008 0.0
0 0.0407545864582062 1.0
0 0.364667505025864 2.0
0 0.126727938652039 3.0
0 0.00185272097587585 4.0
0 0.0141611397266388 5.0
0 0.0386544167995453 6.0
0 -0.00926795601844788 7.0
37.559548496 38.5975303649902 0.0
0 -0.0236560106277466 1.0
0 0.379308342933655 2.0
0 -0.0921030342578888 3.0
0 0.0181429088115692 4.0
0 -0.0813889801502228 5.0
0 0.0223363637924194 6.0
0 0.00696524977684021 7.0
41.796902482 38.1648368835449 0.0
0 0.0240066945552826 1.0
0 0.579645931720734 2.0
0 0.27534356713295 3.0
0 0.197276324033737 4.0
0 -0.0174721479415894 5.0
0 0.0053427517414093 6.0
0 0.05475252866745 7.0
35.679590823 39.3994674682617 0.0
0 0.0173686146736145 1.0
0 -0.645057320594788 2.0
0 -0.247907489538193 3.0
0 -0.107515126466751 4.0
0 -0.0145440697669983 5.0
0 0.0369995534420013 6.0
0 -0.0448894798755646 7.0
33.227547292 38.3330612182617 0.0
0 0.0414885282516479 1.0
0 -0.6172696352005 2.0
0 -0.53005576133728 3.0
0 0.0210317969322205 4.0
0 -0.122390359640121 5.0
0 -0.0780129134654999 6.0
0 -0.0318295657634735 7.0
28.008071739 36.8347015380859 0.0
0 -0.0659752190113068 1.0
0 -1.50339066982269 2.0
0 -0.835333347320557 3.0
0 -0.179858654737473 4.0
0 -0.0593057572841644 5.0
0 0.0216876864433289 6.0
0 -0.0344645082950592 7.0
33.478498841 39.2734527587891 0.0
0 -0.00456690788269043 1.0
0 0.128661543130875 2.0
0 -0.457116097211838 3.0
0 -0.00399994850158691 4.0
0 -0.202548414468765 5.0
0 0.014797180891037 6.0
0 -0.0447780787944794 7.0
33.12294682 38.7541351318359 0.0
0 0.0206910371780396 1.0
0 -1.53891277313232 2.0
0 -0.110004633665085 3.0
0 0.106542080640793 4.0
0 -0.0846746265888214 5.0
0 0.0164613127708435 6.0
0 0.139023810625076 7.0
34.970228384 38.1511154174805 0.0
0 -0.0587297976016998 1.0
0 -0.199275702238083 2.0
0 -0.161985844373703 3.0
0 -0.177202314138412 4.0
0 -0.0924865901470184 5.0
0 0.0430388450622559 6.0
0 -0.0227261781692505 7.0
36.637966041 38.3036193847656 0.0
0 0.0842869877815247 1.0
0 -0.213710814714432 2.0
0 -0.241906493902206 3.0
0 -0.111238032579422 4.0
0 -0.00995227694511414 5.0
0 0.0523355305194855 6.0
0 -0.0118197202682495 7.0
38.712447295 38.5916404724121 0.0
0 -0.0291334688663483 1.0
0 -0.416705995798111 2.0
0 -0.0996390879154205 3.0
0 0.201502233743668 4.0
0 0.0976060926914215 5.0
0 -0.0894133150577545 6.0
0 0.0599584877490997 7.0
29.08402242 39.0198707580566 0.0
0 -0.0401305258274078 1.0
0 -1.22153043746948 2.0
0 -0.829167485237122 3.0
0 -0.214747399091721 4.0
0 -0.163329213857651 5.0
0 0.0369897186756134 6.0
0 -0.0179602801799774 7.0
40.741903011 38.2931861877441 0.0
0 0.0252460241317749 1.0
0 0.742085099220276 2.0
0 0.254266440868378 3.0
0 0.0381401479244232 4.0
0 0.00101622939109802 5.0
0 0.0231049954891205 6.0
0 0.00546154379844666 7.0
44.423927978 39.4454383850098 0.0
0 -0.0384130775928497 1.0
0 1.22883677482605 2.0
0 0.85334312915802 3.0
0 -0.0104032754898071 4.0
0 0.0395295917987823 5.0
0 0.00850361585617065 6.0
0 0.0143651366233826 7.0
37.279929011 36.647274017334 0.0
0 -0.010259598493576 1.0
0 0.332952916622162 2.0
0 -0.123368114233017 3.0
0 0.000702321529388428 4.0
0 -0.0155894160270691 5.0
0 -0.0436192452907562 6.0
0 0.0290978252887726 7.0
39.065196566 37.4629821777344 0.0
0 0.049974650144577 1.0
0 0.680323362350464 2.0
0 -0.0716843903064728 3.0
0 0.189993619918823 4.0
0 -0.117139309644699 5.0
0 0.0215922296047211 6.0
0 0.0831213891506195 7.0
34.346712911 37.6793212890625 0.0
0 -0.0433056056499481 1.0
0 -0.39701172709465 2.0
0 -0.400398701429367 3.0
0 -0.2439184486866 4.0
0 -0.0233840942382812 5.0
0 0.027268648147583 6.0
0 -0.0515764057636261 7.0
44.832836202 38.8109855651855 0.0
0 0.0152344405651093 1.0
0 1.22444725036621 2.0
0 0.730147182941437 3.0
0 0.107961148023605 4.0
0 0.0149390399456024 5.0
0 0.0336912870407104 6.0
0 0.062531977891922 7.0
37.693005916 37.6150360107422 0.0
0 0.0220876634120941 1.0
0 -0.440151542425156 2.0
0 -0.188348561525345 3.0
0 -0.114401310682297 4.0
0 0.089824378490448 5.0
0 0.0318648219108582 6.0
0 -0.0199694335460663 7.0
35.516030206 38.4178314208984 0.0
0 -0.00835976004600525 1.0
0 -0.237524002790451 2.0
0 -0.210521310567856 3.0
0 -0.127708286046982 4.0
0 -0.032683938741684 5.0
0 0.0390874743461609 6.0
0 -0.0336330831050873 7.0
45.449957523 38.5656623840332 0.0
0 0.0403198003768921 1.0
0 1.5034499168396 2.0
0 0.561776518821716 3.0
0 0.169393539428711 4.0
0 -0.0620675384998322 5.0
0 0.0325442850589752 6.0
0 0.268288493156433 7.0
36.52210397 38.1124572753906 0.0
0 0.137504905462265 1.0
0 -0.316785305738449 2.0
0 -0.115488797426224 3.0
0 -0.209489375352859 4.0
0 0.0372112989425659 5.0
0 0.0265770852565765 6.0
0 -0.019342452287674 7.0
40.809333833 37.7631149291992 0.0
0 0.00586244463920593 1.0
0 0.604465842247009 2.0
0 0.435493290424347 3.0
0 -0.0384324491024017 4.0
0 0.0747904479503632 5.0
0 0.0200978517532349 6.0
0 -0.0244668424129486 7.0
43.708815025 38.3452262878418 0.0
0 -0.0199667513370514 1.0
0 1.50199556350708 2.0
0 0.563172519207001 3.0
0 0.0812050402164459 4.0
0 0.0553604364395142 5.0
0 0.0222965776920319 6.0
0 0.0550247132778168 7.0
35.746307575 37.474666595459 0.0
0 -0.0209079384803772 1.0
0 -0.284441977739334 2.0
0 0.808690190315247 3.0
0 0.0225182473659515 4.0
0 -0.108197122812271 5.0
0 0.0459221005439758 6.0
0 0.00485715270042419 7.0
32.565260407 38.9089851379395 0.0
0 0.0147016644477844 1.0
0 -1.59205603599548 2.0
0 -0.741749405860901 3.0
0 -0.176766604185104 4.0
0 0.0963587164878845 5.0
0 0.0278893709182739 6.0
0 -0.0598211586475372 7.0
37.459437569 39.2277297973633 0.0
0 0.0203363597393036 1.0
0 0.170728087425232 2.0
0 -0.229952603578568 3.0
0 0.141425639390945 4.0
0 -0.124369829893112 5.0
0 -0.0667640864849091 6.0
0 0.13677379488945 7.0
41.600868777 38.3704528808594 0.0
0 0.0155590772628784 1.0
0 0.31630951166153 2.0
0 0.0742815434932709 3.0
0 -0.0556133687496185 4.0
0 -0.0749392211437225 5.0
0 0.019318014383316 6.0
0 -0.0254901051521301 7.0
43.088648289 39.7514724731445 0.0
0 0.0622588396072388 1.0
0 2.25177073478699 2.0
0 0.664883434772491 3.0
0 0.155922055244446 4.0
0 -0.016082227230072 5.0
0 0.0271314084529877 6.0
0 0.549206733703613 7.0
30.268152677 37.5501136779785 0.0
0 0.0727604627609253 1.0
0 -1.45856297016144 2.0
0 -0.814110636711121 3.0
0 -0.119223266839981 4.0
0 -0.0137538909912109 5.0
0 -0.0091170072555542 6.0
0 -0.0409217774868011 7.0
38.454045888 38.1196022033691 0.0
0 0.0433622300624847 1.0
0 -0.495599001646042 2.0
0 -0.261224418878555 3.0
0 -0.084888368844986 4.0
0 -0.0160107016563416 5.0
0 0.0460114479064941 6.0
0 0.301980286836624 7.0
36.654056864 38.6733169555664 0.0
0 -0.0273533761501312 1.0
0 -0.146266371011734 2.0
0 -0.26607158780098 3.0
0 0.138693243265152 4.0
0 -0.0813786685466766 5.0
0 0.0228560268878937 6.0
0 0.00858831405639648 7.0
36.990278128 39.1845741271973 0.0
0 0.0103622674942017 1.0
0 -0.531781196594238 2.0
0 -0.099296361207962 3.0
0 0.0681612193584442 4.0
0 -0.0359629094600677 5.0
0 -0.0263132154941559 6.0
0 -0.00112473964691162 7.0
39.45539179 38.6768646240234 0.0
0 -0.0159789323806763 1.0
0 0.196908175945282 2.0
0 0.254363477230072 3.0
0 -0.0752551257610321 4.0
0 0.127245515584946 5.0
0 -0.01190385222435 6.0
0 -0.0521171391010284 7.0
41.49009831 38.0361518859863 0.0
0 -0.0247929692268372 1.0
0 0.679000079631805 2.0
0 0.143650144338608 3.0
0 -0.140485316514969 4.0
0 -0.00272154808044434 5.0
0 0.0192124247550964 6.0
0 -0.0202018916606903 7.0
42.195848764 39.5463218688965 0.0
0 -0.116303771734238 1.0
0 1.06640100479126 2.0
0 0.323663234710693 3.0
0 0.0407457649707794 4.0
0 0.0164220333099365 5.0
0 -0.0114176571369171 6.0
0 -0.0151205062866211 7.0
36.580423956 38.1617317199707 0.0
0 0.0518950223922729 1.0
0 0.395745515823364 2.0
0 -0.16905489563942 3.0
0 0.155609756708145 4.0
0 0.124004304409027 5.0
0 0.0228832066059113 6.0
0 0.794449329376221 7.0
41.112612846 38.4576683044434 0.0
0 -0.0296303331851959 1.0
0 0.654075622558594 2.0
0 0.268121600151062 3.0
0 0.147404670715332 4.0
0 -0.0951347053050995 5.0
0 0.0246095657348633 6.0
0 0.0428214073181152 7.0
41.165032305 38.9587745666504 0.0
0 -0.019788384437561 1.0
0 0.672592759132385 2.0
0 0.127809315919876 3.0
0 0.262057244777679 4.0
0 0.063584953546524 5.0
0 -0.0249413549900055 6.0
0 0.122444689273834 7.0
28.236329769 38.4605178833008 0.0
0 0.0753072798252106 1.0
0 -1.39139699935913 2.0
0 -1.02668583393097 3.0
0 -0.189141482114792 4.0
0 -0.227541357278824 5.0
0 0.00152045488357544 6.0
0 -0.0386983454227448 7.0
46.563722235 38.7650184631348 0.0
0 0.00204434990882874 1.0
0 1.90834128856659 2.0
0 0.397077351808548 3.0
0 0.0509653687477112 4.0
0 0.0636164844036102 5.0
0 0.0296604633331299 6.0
0 0.126455754041672 7.0
38.976039446 37.0834159851074 0.0
0 -0.189598470926285 1.0
0 0.485970318317413 2.0
0 0.246950104832649 3.0
0 -0.0911610424518585 4.0
0 0.156733632087708 5.0
0 -0.0148750245571136 6.0
0 -0.0220912992954254 7.0
43.902220165 37.4947509765625 0.0
0 -0.0290753543376923 1.0
0 1.51115584373474 2.0
0 0.871913909912109 3.0
0 0.00967380404472351 4.0
0 0.0643483996391296 5.0
0 0.0104776918888092 6.0
0 0.991599798202515 7.0
37.104870385 37.9987869262695 0.0
0 0.00953388214111328 1.0
0 -0.0887704789638519 2.0
0 -0.223830670118332 3.0
0 0.178489476442337 4.0
0 -0.100368469953537 5.0
0 -0.0130640268325806 6.0
0 0.0484529733657837 7.0
48.094555135 38.767692565918 0.0
0 -0.0317751467227936 1.0
0 1.0752100944519 2.0
0 0.293733447790146 3.0
0 0.118210524320602 4.0
0 0.0745170712471008 5.0
0 -0.0430496037006378 6.0
0 0.0753486156463623 7.0
39.863550951 37.6261024475098 0.0
0 -0.0757074058055878 1.0
0 -0.0272449553012848 2.0
0 -0.0089246928691864 3.0
0 0.12600103020668 4.0
0 0.0628152191638947 5.0
0 -0.0369543135166168 6.0
0 0.0870952606201172 7.0
38.158424706 37.6236915588379 0.0
0 0.044694572687149 1.0
0 1.18428325653076 2.0
0 0.199639320373535 3.0
0 0.11383855342865 4.0
0 0.0622396767139435 5.0
0 0.0427625179290771 6.0
0 1.25306129455566 7.0
37.671552644 39.1274795532227 0.0
0 -0.0410472452640533 1.0
0 0.385402947664261 2.0
0 0.007271409034729 3.0
0 -0.0817926824092865 4.0
0 0.00759559869766235 5.0
0 0.0393317937850952 6.0
0 -0.0112757682800293 7.0
38.99000687 37.980655670166 0.0
0 0.0217977166175842 1.0
0 0.103707075119019 2.0
0 5.09321689605713e-05 3.0
0 -0.0148172974586487 4.0
0 0.0349729657173157 5.0
0 0.0342704057693481 6.0
0 0.637595534324646 7.0
32.824279197 39.4129791259766 0.0
0 -0.0136662125587463 1.0
0 -1.88641762733459 2.0
0 -0.650291919708252 3.0
0 -0.123429208993912 4.0
0 -0.00144591927528381 5.0
0 -0.0440123379230499 6.0
0 0.0570438504219055 7.0
39.752392398 38.5725631713867 0.0
0 -0.0111989974975586 1.0
0 0.595271527767181 2.0
0 0.0929049849510193 3.0
0 -0.220001190900803 4.0
0 -0.0512186586856842 5.0
0 0.0272365808486938 6.0
0 -0.000194281339645386 7.0
36.014997988 37.7037048339844 0.0
0 0.0152795016765594 1.0
0 -0.863515973091125 2.0
0 -0.431773692369461 3.0
0 -0.000909775495529175 4.0
0 0.0267556309700012 5.0
0 0.0286760032176971 6.0
0 0.0164675712585449 7.0
42.307205606 38.8233261108398 0.0
0 0.0386487245559692 1.0
0 0.736997604370117 2.0
0 0.266140103340149 3.0
0 -0.0733543932437897 4.0
0 0.0666691958904266 5.0
0 -0.016712099313736 6.0
0 0.0214889347553253 7.0
41.624369934 37.9998245239258 0.0
0 0.0513029992580414 1.0
0 0.855541110038757 2.0
0 0.177130371332169 3.0
0 0.195241242647171 4.0
0 -0.0104012787342072 5.0
0 0.0370656847953796 6.0
0 0.0925264954566956 7.0
40.927860926 37.1402893066406 0.0
0 0.00572183728218079 1.0
0 0.290425300598145 2.0
0 0.137520641088486 3.0
0 0.0423173308372498 4.0
0 -9.94503498077393e-05 5.0
0 0.0494304597377777 6.0
0 0.0158121585845947 7.0
47.441321812 37.9771041870117 0.0
0 0.0167296528816223 1.0
0 2.02034187316895 2.0
0 0.812424898147583 3.0
0 0.0763177871704102 4.0
0 0.0299682915210724 5.0
0 0.0127643346786499 6.0
0 0.0731853842735291 7.0
39.379505619 37.938549041748 0.0
0 -0.0245366096496582 1.0
0 1.41832768917084 2.0
0 0.131325006484985 3.0
0 0.254668831825256 4.0
0 -0.0543899834156036 5.0
0 0.0101538002490997 6.0
0 0.665782928466797 7.0
42.095557299 38.3962821960449 0.0
0 -0.0747672021389008 1.0
0 1.00226056575775 2.0
0 0.19634011387825 3.0
0 0.0223001837730408 4.0
0 -0.0892376005649567 5.0
0 0.04888916015625 6.0
0 0.177871137857437 7.0
35.151188225 35.6708030700684 0.0
0 -0.0405570566654205 1.0
0 -0.331354051828384 2.0
0 -0.220967501401901 3.0
0 -0.339792817831039 4.0
0 -0.15724715590477 5.0
0 -0.0158500671386719 6.0
0 -0.0898257791996002 7.0
37.713935371 39.654182434082 0.0
0 -0.0525208413600922 1.0
0 0.171201288700104 2.0
0 -0.267337948083878 3.0
0 -0.0395719707012177 4.0
0 0.000200718641281128 5.0
0 0.0342182219028473 6.0
0 0.477345496416092 7.0
39.66774326 38.4856300354004 0.0
0 0.0448462665081024 1.0
0 0.553324460983276 2.0
0 -0.0149554014205933 3.0
0 0.209031611680984 4.0
0 -0.0414699614048004 5.0
0 0.00676849484443665 6.0
0 0.0406777560710907 7.0
41.240556194 38.4853019714355 0.0
0 0.0538685619831085 1.0
0 1.12830090522766 2.0
0 0.10261270403862 3.0
0 0.13481479883194 4.0
0 -0.0418825447559357 5.0
0 -0.0111183524131775 6.0
0 0.308449804782867 7.0
40.3201946 37.7155876159668 0.0
0 0.0169031918048859 1.0
0 0.498244881629944 2.0
0 0.142458260059357 3.0
0 -0.0535257756710052 4.0
0 0.142902702093124 5.0
0 -0.102736681699753 6.0
0 0.755209684371948 7.0
39.561151059 39.644832611084 0.0
0 -0.0234904587268829 1.0
0 1.46332097053528 2.0
0 -0.0523240268230438 3.0
0 0.0839137732982635 4.0
0 -0.0588851273059845 5.0
0 0.0288496911525726 6.0
0 0.22881506383419 7.0
43.566431947 39.4640083312988 0.0
0 0.0399808585643768 1.0
0 1.29068672657013 2.0
0 0.532536029815674 3.0
0 0.188563615083694 4.0
0 0.126278102397919 5.0
0 0.00426667928695679 6.0
0 0.0509714484214783 7.0
37.929366562 38.4710311889648 0.0
0 -0.00857755541801453 1.0
0 0.141826003789902 2.0
0 -0.0480257570743561 3.0
0 -0.0151157975196838 4.0
0 0.0396130383014679 5.0
0 0.0388723313808441 6.0
0 0.00711089372634888 7.0
39.745668641 38.8787117004395 0.0
0 -0.110410064458847 1.0
0 0.69560694694519 2.0
0 0.249093636870384 3.0
0 -0.255401700735092 4.0
0 -0.00379183888435364 5.0
0 -0.0113239586353302 6.0
0 -0.0136144161224365 7.0
37.173794625 39.0844612121582 0.0
0 -0.00929167866706848 1.0
0 0.466004550457001 2.0
0 -0.152718156576157 3.0
0 -0.125254124403 4.0
0 0.0770134329795837 5.0
0 0.0361621677875519 6.0
0 0.0382488369941711 7.0
27.124866015 37.9581871032715 0.0
0 -0.0366770327091217 1.0
0 -1.64606404304504 2.0
0 -1.08955979347229 3.0
0 -0.15421935915947 4.0
0 -0.0576819479465485 5.0
0 -0.0151661336421967 6.0
0 -0.0385214388370514 7.0
27.411746904 36.2912559509277 0.0
0 0.150603234767914 1.0
0 -2.49726963043213 2.0
0 -0.757160544395447 3.0
0 -0.083183616399765 4.0
0 -0.0913078486919403 5.0
0 0.0631610751152039 6.0
0 -0.071320503950119 7.0
38.866617317 38.2941513061523 0.0
0 0.0107827484607697 1.0
0 0.189093977212906 2.0
0 -0.0647148191928864 3.0
0 0.182923078536987 4.0
0 -0.0943621098995209 5.0
0 0.00114411115646362 6.0
0 0.0570804476737976 7.0
42.383057354 38.6173133850098 0.0
0 0.00381612777709961 1.0
0 0.638069868087769 2.0
0 0.306482255458832 3.0
0 0.103446811437607 4.0
0 0.12553808093071 5.0
0 -0.0336205661296844 6.0
0 0.0903995633125305 7.0
47.643751036 38.3346748352051 0.0
0 -0.0253780484199524 1.0
0 1.99526822566986 2.0
0 0.779180765151978 3.0
0 0.0374946594238281 4.0
0 -0.0360651314258575 5.0
0 -0.00773927569389343 6.0
0 0.181904256343842 7.0
38.439399957 36.3941841125488 0.0
0 -7.3164701461792e-05 1.0
0 0.126699447631836 2.0
0 0.00508546829223633 3.0
0 0.0116435289382935 4.0
0 0.0294660627841949 5.0
0 0.0205676853656769 6.0
0 0.0195218622684479 7.0
40.263562371 37.735897064209 0.0
0 0.0167442560195923 1.0
0 0.355074405670166 2.0
0 0.0521688163280487 3.0
0 0.303222239017487 4.0
0 0.0544809401035309 5.0
0 -0.0123229324817657 6.0
0 0.0577684044837952 7.0
41.519528397 37.3909568786621 0.0
0 -0.0152777135372162 1.0
0 1.05946779251099 2.0
0 0.224622294306755 3.0
0 0.0237171351909637 4.0
0 0.0700708031654358 5.0
0 0.0521140694618225 6.0
0 0.0989196002483368 7.0
40.414570474 38.1181983947754 0.0
0 -0.00688076019287109 1.0
0 1.06584918498993 2.0
0 0.217888981103897 3.0
0 -0.188512653112411 4.0
0 0.00862035155296326 5.0
0 0.0263164639472961 6.0
0 0.00175878405570984 7.0
37.834181221 39.3150177001953 0.0
0 -0.0370447337627411 1.0
0 0.434993386268616 2.0
0 -0.0400767624378204 3.0
0 -0.00664922595024109 4.0
0 -0.0327106416225433 5.0
0 0.0241530239582062 6.0
0 -0.0180577039718628 7.0
37.201121556 39.3342323303223 0.0
0 0.0162084996700287 1.0
0 -0.0937548577785492 2.0
0 -0.124076098203659 3.0
0 -0.138864248991013 4.0
0 -0.0525325238704681 5.0
0 0.0507732331752777 6.0
0 -0.0179467797279358 7.0
42.94444514 39.2299156188965 0.0
0 -0.0206248164176941 1.0
0 1.52534472942352 2.0
0 0.489934653043747 3.0
0 0.0232639014720917 4.0
0 0.0665787160396576 5.0
0 0.0138692259788513 6.0
0 0.0569679141044617 7.0
47.309600034 38.1150207519531 0.0
0 -0.00159695744514465 1.0
0 1.94079053401947 2.0
0 0.997515320777893 3.0
0 0.0708155333995819 4.0
0 0.0969352126121521 5.0
0 -0.00578242540359497 6.0
0 0.102289944887161 7.0
34.703660747 37.6564407348633 0.0
0 -0.0183762013912201 1.0
0 -0.0385021269321442 2.0
0 -0.359519153833389 3.0
0 -0.234431058168411 4.0
0 -0.00115764141082764 5.0
0 0.0349386632442474 6.0
0 -0.0334479510784149 7.0
39.355679833 38.1761627197266 0.0
0 -0.0257196426391602 1.0
0 0.22646476328373 2.0
0 0.166127771139145 3.0
0 0.0796016156673431 4.0
0 -0.0561510026454926 5.0
0 0.0364807546138763 6.0
0 -0.0462780296802521 7.0
36.760838135 37.2655982971191 0.0
0 -0.0591689646244049 1.0
0 -0.0142973959445953 2.0
0 -0.0953355133533478 3.0
0 -0.109275013208389 4.0
0 0.0874241292476654 5.0
0 0.020307719707489 6.0
0 -0.0506404936313629 7.0
45.817767369 40.027400970459 0.0
0 -0.0194887816905975 1.0
0 1.89090776443481 2.0
0 0.573336720466614 3.0
0 0.0960702300071716 4.0
0 0.0280004143714905 5.0
0 0.0144957900047302 6.0
0 0.468617230653763 7.0
37.456387944 36.4002723693848 0.0
0 -0.0752773582935333 1.0
0 -0.10630401968956 2.0
0 -0.0921538770198822 3.0
0 -0.00148990750312805 4.0
0 -0.0818144381046295 5.0
0 0.0117003619670868 6.0
0 -0.028637021780014 7.0
36.979299113 38.1515235900879 0.0
0 0.0563787519931793 1.0
0 0.0542124509811401 2.0
0 -0.0742001235485077 3.0
0 -0.217196494340897 4.0
0 0.00240477919578552 5.0
0 0.00830808281898499 6.0
0 -0.0522963106632233 7.0
44.357392157 38.419548034668 0.0
0 0.0115028619766235 1.0
0 1.10325157642365 2.0
0 0.564593374729156 3.0
0 0.0716415643692017 4.0
0 0.146089643239975 5.0
0 -0.0372441709041595 6.0
0 0.0760943293571472 7.0
36.812079299 38.2496376037598 0.0
0 -0.0474378764629364 1.0
0 0.265174746513367 2.0
0 -0.26232185959816 3.0
0 0.121325582265854 4.0
0 -0.187197238206863 5.0
0 -0.0921118557453156 6.0
0 0.472335964441299 7.0
43.172254743 37.0546760559082 0.0
0 -0.0823991000652313 1.0
0 1.84136998653412 2.0
0 0.607471585273743 3.0
0 -0.132171362638474 4.0
0 0.00435701012611389 5.0
0 -0.0264386832714081 6.0
0 -0.0203844308853149 7.0
47.109370567 37.6512832641602 0.0
0 -0.0590590536594391 1.0
0 1.42517638206482 2.0
0 0.728706359863281 3.0
0 0.157940238714218 4.0
0 0.0638044476509094 5.0
0 -0.0336466133594513 6.0
0 0.276160657405853 7.0
35.871355662 36.4623374938965 0.0
0 0.0455839931964874 1.0
0 -0.553479433059692 2.0
0 -0.478126853704453 3.0
0 0.124936252832413 4.0
0 0.0284014940261841 5.0
0 0.0302327573299408 6.0
0 0.0506866872310638 7.0
42.169433911 38.3862686157227 0.0
0 -0.0307396948337555 1.0
0 1.2697639465332 2.0
0 0.396995067596436 3.0
0 0.0912122428417206 4.0
0 0.0171624720096588 5.0
0 0.055271327495575 6.0
0 0.107793897390366 7.0
38.001249401 37.8324165344238 0.0
0 0.0480708181858063 1.0
0 0.614314913749695 2.0
0 -0.177175372838974 3.0
0 0.062270313501358 4.0
0 -0.120931535959244 5.0
0 0.0479724705219269 6.0
0 -0.010226309299469 7.0
36.026296153 37.1007804870605 0.0
0 -0.0420711934566498 1.0
0 -0.180716961622238 2.0
0 -0.324820667505264 3.0
0 0.0193078219890594 4.0
0 0.0366161465644836 5.0
0 0.0280156433582306 6.0
0 0.0148287713527679 7.0
33.200213159 38.2770462036133 0.0
0 -0.0743725001811981 1.0
0 -0.357539921998978 2.0
0 -0.557262778282166 3.0
0 -0.0617791712284088 4.0
0 -0.0777322947978973 5.0
0 -0.0437784492969513 6.0
0 -0.0314547717571259 7.0
36.881947952 39.0467338562012 0.0
0 -0.0701363384723663 1.0
0 0.012144148349762 2.0
0 -0.28552708029747 3.0
0 0.05924391746521 4.0
0 -0.076420396566391 5.0
0 -0.0187683403491974 6.0
0 -0.0176985263824463 7.0
28.363012012 39.0927276611328 0.0
0 -0.00139400362968445 1.0
0 -1.99324083328247 2.0
0 -0.79026186466217 3.0
0 -0.190730661153793 4.0
0 -0.157469302415848 5.0
0 -0.012580543756485 6.0
0 0.00833883881568909 7.0
39.574467426 38.4098281860352 0.0
0 0.0191673636436462 1.0
0 1.24807858467102 2.0
0 0.451621234416962 3.0
0 0.157072812318802 4.0
0 0.017247349023819 5.0
0 0.024133563041687 6.0
0 0.748751044273376 7.0
36.22040148 38.6228370666504 0.0
0 0.00733864307403564 1.0
0 0.110966920852661 2.0
0 -0.25114044547081 3.0
0 -0.0147051811218262 4.0
0 -0.0157572031021118 5.0
0 0.0247698724269867 6.0
0 -0.0310802161693573 7.0
31.086432141 38.4686050415039 0.0
0 -0.0864501297473907 1.0
0 -0.323137134313583 2.0
0 -0.769401788711548 3.0
0 -0.222982496023178 4.0
0 -0.194812506437302 5.0
0 -0.0365482270717621 6.0
0 -0.0456169545650482 7.0
36.274353063 38.2614440917969 0.0
0 0.0227864980697632 1.0
0 -0.0732143819332123 2.0
0 -0.216033488512039 3.0
0 -0.238992422819138 4.0
0 0.138484001159668 5.0
0 -0.00576600432395935 6.0
0 -0.052438348531723 7.0
31.65795009 37.3841094970703 0.0
0 -0.00667080283164978 1.0
0 -0.762511730194092 2.0
0 -0.904087901115417 3.0
0 -0.0682811439037323 4.0
0 0.1079061627388 5.0
0 0.0406116247177124 6.0
0 -0.0271545350551605 7.0
37.059171479 40.4634246826172 0.0
0 -0.00166568160057068 1.0
0 -0.0740306675434113 2.0
0 -0.341471821069717 3.0
0 0.217308938503265 4.0
0 0.0831453800201416 5.0
0 0.0239531695842743 6.0
0 0.0144567787647247 7.0
45.308862357 38.3497314453125 0.0
0 -0.113188654184341 1.0
0 1.87428522109985 2.0
0 0.823745489120483 3.0
0 0.250378608703613 4.0
0 -0.000454843044281006 5.0
0 0.0195232629776001 6.0
0 0.118268579244614 7.0
32.988279909 39.1378402709961 0.0
0 0.0558353364467621 1.0
0 -1.09624397754669 2.0
0 -0.438990563154221 3.0
0 -0.155533879995346 4.0
0 -0.0604645907878876 5.0
0 0.0744709968566895 6.0
0 -0.0298620164394379 7.0
41.812200593 36.485652923584 0.0
0 0.0192826688289642 1.0
0 0.873306274414062 2.0
0 0.37354364991188 3.0
0 -0.137822240591049 4.0
0 -0.0134514570236206 5.0
0 0.0126524567604065 6.0
0 -0.00441074371337891 7.0
34.159578007 37.5444107055664 0.0
0 0.0284624099731445 1.0
0 -0.93839704990387 2.0
0 -0.0382331311702728 3.0
0 -0.150419384241104 4.0
0 0.141619712114334 5.0
0 0.0323673188686371 6.0
0 -0.0238885879516602 7.0
41.353058204 37.8061637878418 0.0
0 -0.112448900938034 1.0
0 1.15681684017181 2.0
0 0.285093575716019 3.0
0 0.154003202915192 4.0
0 -0.0827841460704803 5.0
0 0.06011962890625 6.0
0 0.0518067181110382 7.0
35.663236644 37.9432258605957 0.0
0 -0.0379081070423126 1.0
0 -0.455278187990189 2.0
0 -0.507125496864319 3.0
0 0.0645866096019745 4.0
0 -0.0541451275348663 5.0
0 -0.01790452003479 6.0
0 0.0704671144485474 7.0
40.444916123 38.9146385192871 0.0
0 -0.117452591657639 1.0
0 0.255349457263947 2.0
0 0.171948492527008 3.0
0 0.0304437577724457 4.0
0 -0.207073301076889 5.0
0 0.0702624022960663 6.0
0 0.0355806350708008 7.0
39.900416914 38.0579566955566 0.0
0 -0.0163217186927795 1.0
0 0.892715334892273 2.0
0 0.271589607000351 3.0
0 0.117226392030716 4.0
0 0.0842187702655792 5.0
0 0.00175079703330994 6.0
0 0.645577371120453 7.0
34.405324849 39.8503265380859 0.0
0 0.000544577836990356 1.0
0 -1.12554478645325 2.0
0 -0.508710980415344 3.0
0 0.0280264616012573 4.0
0 -0.0375227034091949 5.0
0 0.00852659344673157 6.0
0 -0.00253975391387939 7.0
33.074256004 37.9996719360352 0.0
0 -0.0352363288402557 1.0
0 -0.503911256790161 2.0
0 -0.551686882972717 3.0
0 -0.0681971609592438 4.0
0 -0.152587622404099 5.0
0 0.0278448760509491 6.0
0 -0.0462107956409454 7.0
40.036170308 37.6946907043457 0.0
0 -0.00733113288879395 1.0
0 0.578685104846954 2.0
0 0.374883115291595 3.0
0 -0.0704929530620575 4.0
0 0.0697660446166992 5.0
0 0.00218597054481506 6.0
0 -0.0471406280994415 7.0
44.453206241 39.5736465454102 0.0
0 -0.0255559980869293 1.0
0 0.939143598079681 2.0
0 0.736413061618805 3.0
0 -0.0962874591350555 4.0
0 0.0136998891830444 5.0
0 0.0218074917793274 6.0
0 0.747741460800171 7.0
33.85004541 37.7245712280273 0.0
0 0.0845398902893066 1.0
0 -0.949512600898743 2.0
0 -0.374567121267319 3.0
0 -0.312933057546616 4.0
0 -0.0025181770324707 5.0
0 0.0222087204456329 6.0
0 -0.0311493575572968 7.0
30.537877308 37.1994972229004 0.0
0 0.0254012644290924 1.0
0 -1.03290295600891 2.0
0 -0.776633262634277 3.0
0 0.0167989134788513 4.0
0 -0.126987010240555 5.0
0 0.00126594305038452 6.0
0 -0.0617137849330902 7.0
27.727440942 36.8510055541992 0.0
0 0.0297664999961853 1.0
0 -2.06722569465637 2.0
0 -1.13284981250763 3.0
0 -0.25198683142662 4.0
0 -0.0489442646503448 5.0
0 -0.0509573519229889 6.0
0 -0.0501252114772797 7.0
35.630182897 38.1855735778809 0.0
0 0.0481047034263611 1.0
0 0.10670080780983 2.0
0 -0.180422455072403 3.0
0 -0.219018191099167 4.0
0 -0.0441862046718597 5.0
0 0.0434234738349915 6.0
0 -0.0052679181098938 7.0
31.038251519 37.6291275024414 0.0
0 -0.0428708493709564 1.0
0 -1.25102818012238 2.0
0 -0.261221379041672 3.0
0 -0.19379398226738 4.0
0 -0.0904238522052765 5.0
0 0.0363373756408691 6.0
0 0.011374831199646 7.0
31.867011967 38.2035369873047 0.0
0 -0.0631146728992462 1.0
0 -1.1464307308197 2.0
0 -0.788671135902405 3.0
0 -0.141707509756088 4.0
0 0.0837773382663727 5.0
0 0.0163076817989349 6.0
0 -0.0526300966739655 7.0
30.754831646 38.1887359619141 0.0
0 -0.07967808842659 1.0
0 -0.652634263038635 2.0
0 -0.738018870353699 3.0
0 -0.286301881074905 4.0
0 0.00292879343032837 5.0
0 0.015402227640152 6.0
0 -0.0429352819919586 7.0
38.95156428 38.7958602905273 0.0
0 -0.0419107973575592 1.0
0 0.155907899141312 2.0
0 -0.173488110303879 3.0
0 0.10295894742012 4.0
0 -0.103485852479935 5.0
0 0.01809161901474 6.0
0 0.136276841163635 7.0
33.384480936 37.9227447509766 0.0
0 0.0150998830795288 1.0
0 -0.243328005075455 2.0
0 -0.452688008546829 3.0
0 -0.172503858804703 4.0
0 -0.134050875902176 5.0
0 0.0292517840862274 6.0
0 -0.0586743652820587 7.0
37.291199282 38.9341888427734 0.0
0 -0.108635812997818 1.0
0 0.0588495135307312 2.0
0 -0.318713456392288 3.0
0 -0.171762138605118 4.0
0 -0.0459371507167816 5.0
0 -0.0197703242301941 6.0
0 0.0891872346401215 7.0
33.571811016 38.2020492553711 0.0
0 -0.0987163484096527 1.0
0 -0.384162276983261 2.0
0 -0.540098547935486 3.0
0 -0.1099514067173 4.0
0 -0.0638063848018646 5.0
0 0.0183091461658478 6.0
0 -0.0478018820285797 7.0
44.048761746 37.7593994140625 0.0
0 -0.0110270082950592 1.0
0 1.95915877819061 2.0
0 0.598145246505737 3.0
0 -0.0021020770072937 4.0
0 0.158022940158844 5.0
0 0.00139260292053223 6.0
0 0.03123739361763 7.0
38.09281546 37.6316261291504 0.0
0 -0.00922444462776184 1.0
0 0.314308404922485 2.0
0 0.103777587413788 3.0
0 -0.025994747877121 4.0
0 0.02729532122612 5.0
0 0.0437546670436859 6.0
0 -0.00579750537872314 7.0
42.616606924 38.7480392456055 0.0
0 0.00982773303985596 1.0
0 1.35112166404724 2.0
0 0.52734375 3.0
0 0.215642362833023 4.0
0 0.109922915697098 5.0
0 -0.0103373527526855 6.0
0 0.210887998342514 7.0
41.080227074 39.0791244506836 0.0
0 -0.103679627180099 1.0
0 1.23018682003021 2.0
0 0.18121525645256 3.0
0 0.159860521554947 4.0
0 -0.073690265417099 5.0
0 0.0177755355834961 6.0
0 0.054710179567337 7.0
36.859087813 38.3836212158203 0.0
0 0.00271973013877869 1.0
0 -0.502426624298096 2.0
0 -0.176560193300247 3.0
0 -0.0797180831432343 4.0
0 0.127713650465012 5.0
0 0.0254793167114258 6.0
0 -0.00120639801025391 7.0
39.316268876 38.2179145812988 0.0
0 0.0368964374065399 1.0
0 0.155881583690643 2.0
0 0.101623028516769 3.0
0 -0.0187805891036987 4.0
0 -0.115480273962021 5.0
0 0.059354692697525 6.0
0 -0.0118778347969055 7.0
40.023071967 38.0203552246094 0.0
0 -0.0413940846920013 1.0
0 0.614411532878876 2.0
0 0.100418031215668 3.0
0 0.147619634866714 4.0
0 -0.0671749413013458 5.0
0 0.00286826491355896 6.0
0 0.0452431440353394 7.0
32.691039566 38.9748382568359 0.0
0 -0.0639322698116302 1.0
0 -0.483811885118484 2.0
0 -0.615338921546936 3.0
0 -0.114204376935959 4.0
0 -0.206632941961288 5.0
0 -0.0462221801280975 6.0
0 -0.00471556186676025 7.0
39.35017223 38.4057350158691 0.0
0 0.0206992924213409 1.0
0 0.636324465274811 2.0
0 0.150375515222549 3.0
0 0.00909125804901123 4.0
0 0.0252905786037445 5.0
0 0.0354829728603363 6.0
0 0.0287934541702271 7.0
38.62577922 37.842643737793 0.0
0 -0.0967571437358856 1.0
0 0.242527693510056 2.0
0 -0.170598357915878 3.0
0 0.194054871797562 4.0
0 -0.101286798715591 5.0
0 -0.0785422027111053 6.0
0 0.0627144575119019 7.0
43.36321018 38.0939865112305 0.0
0 -0.0420911014080048 1.0
0 1.7789466381073 2.0
0 0.635403990745544 3.0
0 0.0472777783870697 4.0
0 -0.0623894035816193 5.0
0 0.00916212797164917 6.0
0 0.0791584253311157 7.0
41.711370519 38.9771423339844 0.0
0 0.036515086889267 1.0
0 1.13988053798676 2.0
0 0.338052958250046 3.0
0 -0.0668222010135651 4.0
0 0.0587421953678131 5.0
0 0.0445996820926666 6.0
0 -0.00359398126602173 7.0
41.401748585 38.4273071289062 0.0
0 -0.0956786572933197 1.0
0 0.901749014854431 2.0
0 0.242853701114655 3.0
0 0.083494633436203 4.0
0 -0.043611079454422 5.0
0 0.046334832906723 6.0
0 0.0650734901428223 7.0
45.411303661 38.003231048584 0.0
0 -0.00961020588874817 1.0
0 1.49413466453552 2.0
0 0.322367906570435 3.0
0 0.149843961000443 4.0
0 0.0180788934230804 5.0
0 0.0314652919769287 6.0
0 0.0883066356182098 7.0
46.173622738 38.1381454467773 0.0
0 -0.0366036593914032 1.0
0 1.60181355476379 2.0
0 0.727382779121399 3.0
0 -0.00960439443588257 4.0
0 0.0522462427616119 5.0
0 -0.00315910577774048 6.0
0 0.0583402812480927 7.0
43.970500892 38.1963043212891 0.0
0 0.0040382444858551 1.0
0 1.48432624340057 2.0
0 0.476128935813904 3.0
0 0.266087293624878 4.0
0 0.146459251642227 5.0
0 -0.0167936980724335 6.0
0 0.218426644802094 7.0
41.561964524 39.2812652587891 0.0
0 -0.0904538929462433 1.0
0 1.23714959621429 2.0
0 0.301080226898193 3.0
0 -0.243707567453384 4.0
0 -0.0189048945903778 5.0
0 -0.0246134102344513 6.0
0 -0.0325981676578522 7.0
42.043570834 38.6131019592285 0.0
0 0.0198627114295959 1.0
0 0.936121821403503 2.0
0 0.548407196998596 3.0
0 0.0809520781040192 4.0
0 0.103752732276917 5.0
0 0.0280997455120087 6.0
0 0.0150052309036255 7.0
42.728187908 39.1055641174316 0.0
0 -0.0263045728206635 1.0
0 1.3396680355072 2.0
0 0.463964998722076 3.0
0 0.147686511278152 4.0
0 0.137156277894974 5.0
0 0.0027942955493927 6.0
0 0.0432316660881042 7.0
46.647822945 35.7333488464355 0.0
0 -0.0376095473766327 1.0
0 1.23744320869446 2.0
0 0.761270046234131 3.0
0 0.199429661035538 4.0
0 0.0967973172664642 5.0
0 0.0386351644992828 6.0
0 0.152585655450821 7.0
34.157298462 38.7215156555176 0.0
0 -0.0494354069232941 1.0
0 -0.588896989822388 2.0
0 -0.458181589841843 3.0
0 -0.0131537020206451 4.0
0 -0.0639188587665558 5.0
0 -0.0187208652496338 6.0
0 0.00246196985244751 7.0
38.08940786 39.0311470031738 0.0
0 -0.123813420534134 1.0
0 0.401577711105347 2.0
0 0.161241829395294 3.0
0 -0.271874755620956 4.0
0 -0.26535376906395 5.0
0 -0.00220274925231934 6.0
0 -0.0487962067127228 7.0
48.253456605 39.5874290466309 0.0
0 0.0059228241443634 1.0
0 2.70820069313049 2.0
0 1.15645575523376 3.0
0 0.271321952342987 4.0
0 0.0752136409282684 5.0
0 0.0294584035873413 6.0
0 0.202552199363708 7.0
41.303512955 38.0428428649902 0.0
0 0.0590474903583527 1.0
0 0.912361204624176 2.0
0 0.177195280790329 3.0
0 0.174091666936874 4.0
0 0.0512797832489014 5.0
0 -0.117112427949905 6.0
0 0.0729454457759857 7.0
34.11247552 37.6589698791504 0.0
0 0.00990846753120422 1.0
0 -0.883583068847656 2.0
0 -0.44401428103447 3.0
0 -0.21386781334877 4.0
0 0.0862202942371368 5.0
0 0.00982129573822021 6.0
0 -0.038503497838974 7.0
39.346899321 38.4597854614258 0.0
0 0.00103646516799927 1.0
0 -0.0384722054004669 2.0
0 -0.101907104253769 3.0
0 0.211742401123047 4.0
0 -0.0145973563194275 5.0
0 -0.0120106041431427 6.0
0 0.06675124168396 7.0
42.355081321 38.3021545410156 0.0
0 -0.0239836573600769 1.0
0 1.12133157253265 2.0
0 0.511757493019104 3.0
0 -0.0629777610301971 4.0
0 -0.00896915793418884 5.0
0 -0.0179072022438049 6.0
0 1.23955190181732 7.0
46.025284619 37.8800239562988 0.0
0 0.0430606007575989 1.0
0 1.75078570842743 2.0
0 0.466012895107269 3.0
0 0.0157944858074188 4.0
0 0.0693248808383942 5.0
0 0.0297353863716125 6.0
0 0.045110285282135 7.0
48.790909787 38.3815879821777 0.0
0 0.0190810263156891 1.0
0 2.17850279808044 2.0
0 0.890224933624268 3.0
0 0.105059653520584 4.0
0 0.00432798266410828 5.0
0 0.0323373675346375 6.0
0 0.346641480922699 7.0
41.698940502 38.1170120239258 0.0
0 -0.0344004333019257 1.0
0 0.769968032836914 2.0
0 0.180048644542694 3.0
0 0.0918790698051453 4.0
0 -0.0771873295307159 5.0
0 0.0668909847736359 6.0
0 0.0473715662956238 7.0
47.902972781 38.2283363342285 0.0
0 0.033897191286087 1.0
0 1.98975229263306 2.0
0 1.02678966522217 3.0
0 0.0690054595470428 4.0
0 0.107927531003952 5.0
0 0.000703901052474976 6.0
0 0.115879774093628 7.0
41.37717087 38.5143356323242 0.0
0 -0.0481623709201813 1.0
0 1.04471075534821 2.0
0 0.385396629571915 3.0
0 -0.0281484127044678 4.0
0 -0.052816241979599 5.0
0 0.0287574529647827 6.0
0 -0.00953912734985352 7.0
32.285684902 38.7779731750488 0.0
0 0.0145005881786346 1.0
0 -0.669342041015625 2.0
0 -0.549390435218811 3.0
0 -0.360265344381332 4.0
0 -0.0430789291858673 5.0
0 0.0364625453948975 6.0
0 -0.0122133493423462 7.0
32.344206851 38.2265777587891 0.0
0 -0.0124004781246185 1.0
0 -0.97749137878418 2.0
0 -0.690160632133484 3.0
0 0.0260143578052521 4.0
0 -0.133681267499924 5.0
0 0.0115867555141449 6.0
0 -0.0628142654895782 7.0
34.184112331 38.155158996582 0.0
0 -0.040008157491684 1.0
0 -0.0863779485225677 2.0
0 -0.534904360771179 3.0
0 0.0593609511852264 4.0
0 -0.20420977473259 5.0
0 -0.072703093290329 6.0
0 0.0245271027088165 7.0
34.663221808 39.39794921875 0.0
0 -0.0427985489368439 1.0
0 -0.048856645822525 2.0
0 -0.525167465209961 3.0
0 0.0340587496757507 4.0
0 -0.20629957318306 5.0
0 -0.00329220294952393 6.0
0 -0.0314189493656158 7.0
36.873774543 37.3240737915039 0.0
0 0.0201607644557953 1.0
0 -0.165877312421799 2.0
0 -0.186139076948166 3.0
0 -0.102504998445511 4.0
0 0.0374492108821869 5.0
0 0.0431866049766541 6.0
0 0.00611570477485657 7.0
40.211352908 37.9330902099609 0.0
0 0.0415265262126923 1.0
0 1.19823801517487 2.0
0 0.222903400659561 3.0
0 0.148825585842133 4.0
0 -0.0605229437351227 5.0
0 0.0166619420051575 6.0
0 0.547482430934906 7.0
46.160689867 38.6688270568848 0.0
0 -0.0712141692638397 1.0
0 2.29835295677185 2.0
0 0.810410797595978 3.0
0 0.0173055231571198 4.0
0 0.0360356569290161 5.0
0 -0.00155183672904968 6.0
0 0.119778126478195 7.0
32.727702676 39.568531036377 0.0
0 -0.103499680757523 1.0
0 -0.26927986741066 2.0
0 -0.730385780334473 3.0
0 0.0436151921749115 4.0
0 -0.142500311136246 5.0
0 -0.118752390146255 6.0
0 -0.0538096725940704 7.0
40.640985417 37.2762565612793 0.0
0 -0.0957179367542267 1.0
0 0.854291141033173 2.0
0 0.196943730115891 3.0
0 0.00918278098106384 4.0
0 0.0750411152839661 5.0
0 0.0241033434867859 6.0
0 0.00547999143600464 7.0
34.611451594 38.1916923522949 0.0
0 -0.063610702753067 1.0
0 -0.0553494989871979 2.0
0 -0.450922340154648 3.0
0 -0.209692567586899 4.0
0 -0.076292484998703 5.0
0 0.0147295296192169 6.0
0 -0.0768562853336334 7.0
43.311280807 38.8516845703125 0.0
0 -0.0451030433177948 1.0
0 1.47674536705017 2.0
0 0.375233709812164 3.0
0 -0.0272530317306519 4.0
0 0.177070021629333 5.0
0 -0.0408309996128082 6.0
0 0.0355225503444672 7.0
36.406364057 38.3874282836914 0.0
0 -0.0153038203716278 1.0
0 0.110938787460327 2.0
0 -0.111588209867477 3.0
0 -0.13426598906517 4.0
0 -0.0724171698093414 5.0
0 0.0250198841094971 6.0
0 -0.027804970741272 7.0
31.685630326 39.6303329467773 0.0
0 -0.13125017285347 1.0
0 -0.228804618120193 2.0
0 -0.722043871879578 3.0
0 -0.0206847190856934 4.0
0 -0.300405591726303 5.0
0 0.0201678574085236 6.0
0 -0.0485210716724396 7.0
40.165364432 38.4950637817383 0.0
0 -0.0817812383174896 1.0
0 0.927523732185364 2.0
0 0.143401116132736 3.0
0 0.207614004611969 4.0
0 -0.0343461930751801 5.0
0 -0.0622668564319611 6.0
0 0.0807114243507385 7.0
42.886465647 37.4482688903809 0.0
0 -0.0394355952739716 1.0
0 0.699548006057739 2.0
0 0.466985911130905 3.0
0 0.175778776407242 4.0
0 0.0194632112979889 5.0
0 0.0290476977825165 6.0
0 0.0914503037929535 7.0
46.942592332 37.7354469299316 0.0
0 0.0275440216064453 1.0
0 2.10751414299011 2.0
0 0.774399161338806 3.0
0 0.205049246549606 4.0
0 0.0602129995822906 5.0
0 0.0105682611465454 6.0
0 0.1920245885849 7.0
38.417030568 37.4178428649902 0.0
0 -0.0982111990451813 1.0
0 -0.136385053396225 2.0
0 -0.15933683514595 3.0
0 0.143467843532562 4.0
0 -0.011839896440506 5.0
0 -0.0525763332843781 6.0
0 0.0159859955310822 7.0
39.82159543 38.9404106140137 0.0
0 -0.0483744442462921 1.0
0 0.612968921661377 2.0
0 0.236331522464752 3.0
0 -0.076745480298996 4.0
0 0.10900205373764 5.0
0 -0.00824412703514099 6.0
0 -0.0635391771793365 7.0
42.375846269 37.6523284912109 0.0
0 0.05009526014328 1.0
0 1.34776341915131 2.0
0 0.452313452959061 3.0
0 -0.111139327287674 4.0
0 0.0513196289539337 5.0
0 -0.00302368402481079 6.0
0 0.0126669704914093 7.0
41.909868924 38.3426971435547 0.0
0 0.00291362404823303 1.0
0 1.23349702358246 2.0
0 0.158749401569366 3.0
0 0.273139595985413 4.0
0 0.00535199046134949 5.0
0 0.0228085815906525 6.0
0 0.111535340547562 7.0
38.515403814 38.042594909668 0.0
0 -0.0424048006534576 1.0
0 0.604826629161835 2.0
0 0.083654373884201 3.0
0 0.138631850481033 4.0
0 -0.0851323306560516 5.0
0 -0.101000338792801 6.0
0 0.0108802020549774 7.0
32.187464773 37.4040603637695 0.0
0 -0.0568129122257233 1.0
0 -0.472006767988205 2.0
0 -0.697058439254761 3.0
0 -0.236449629068375 4.0
0 -0.159503191709518 5.0
0 0.0274175405502319 6.0
0 0.0138149559497833 7.0
43.553340498 37.4506187438965 0.0
0 -0.105107218027115 1.0
0 1.40850567817688 2.0
0 0.407425582408905 3.0
0 0.336433589458466 4.0
0 -0.0867978632450104 5.0
0 -0.0703405439853668 6.0
0 0.178111165761948 7.0
38.661685118 38.6310806274414 0.0
0 -0.0389821231365204 1.0
0 1.03920638561249 2.0
0 0.202074378728867 3.0
0 -0.0670200288295746 4.0
0 -0.0213088691234589 5.0
0 0.0362934172153473 6.0
0 1.63852119445801 7.0
47.185976948 37.5965805053711 0.0
0 0.0528658628463745 1.0
0 1.94761955738068 2.0
0 0.941468000411987 3.0
0 0.0532013773918152 4.0
0 0.0487551391124725 5.0
0 -0.0156641006469727 6.0
0 0.055486649274826 7.0
41.073649825 40.5333518981934 0.0
0 0.00157725811004639 1.0
0 0.421325594186783 2.0
0 0.243464469909668 3.0
0 -0.0866519510746002 4.0
0 0.0972997844219208 5.0
0 -0.00864279270172119 6.0
0 -0.0155076086521149 7.0
39.086114846 38.5569610595703 0.0
0 0.039866179227829 1.0
0 0.344873815774918 2.0
0 -0.0556918084621429 3.0
0 0.192990154027939 4.0
0 0.0190714597702026 5.0
0 -0.0978608429431915 6.0
0 0.077458381652832 7.0
38.072309151 38.4070243835449 0.0
0 0.0318323075771332 1.0
0 0.24251501262188 2.0
0 0.0446723997592926 3.0
0 -0.141942650079727 4.0
0 0.0646858811378479 5.0
0 0.0108446478843689 6.0
0 -0.0301295816898346 7.0
41.585402374 38.5946922302246 0.0
0 0.0345306396484375 1.0
0 0.989545583724976 2.0
0 0.32552182674408 3.0
0 0.124163776636124 4.0
0 0.0361418128013611 5.0
0 0.0114926993846893 6.0
0 0.0634678304195404 7.0
42.317317809 39.3280372619629 0.0
0 -0.0364948213100433 1.0
0 1.3198094367981 2.0
0 0.24762262403965 3.0
0 0.231325089931488 4.0
0 -0.151039451360703 5.0
0 -0.0564407408237457 6.0
0 0.294820934534073 7.0
42.931477585 38.8375015258789 0.0
0 -0.007822185754776 1.0
0 1.33065056800842 2.0
0 0.469849854707718 3.0
0 -0.0169491171836853 4.0
0 -0.0622221529483795 5.0
0 0.020101398229599 6.0
0 -0.0211827456951141 7.0
46.096274411 39.3640480041504 0.0
0 -0.0776444375514984 1.0
0 2.3633770942688 2.0
0 0.96755987405777 3.0
0 0.12633353471756 4.0
0 0.0986304879188538 5.0
0 -0.043506532907486 6.0
0 0.116615653038025 7.0
39.955261554 38.8154487609863 0.0
0 -0.0615434944629669 1.0
0 1.14393925666809 2.0
0 0.028632789850235 3.0
0 0.0751593112945557 4.0
0 0.0736317038536072 5.0
0 0.00632503628730774 6.0
0 0.0711969137191772 7.0
45.415961846 38.0297393798828 0.0
0 -0.00147143006324768 1.0
0 1.08115267753601 2.0
0 0.524784505367279 3.0
0 0.116655737161636 4.0
0 0.0527068972587585 5.0
0 0.00537669658660889 6.0
0 0.105265349149704 7.0
44.275746657 38.8387184143066 0.0
0 -0.0242552757263184 1.0
0 1.85828602313995 2.0
0 0.581170618534088 3.0
0 0.0213092863559723 4.0
0 -0.0175756216049194 5.0
0 0.0286931693553925 6.0
0 0.124256521463394 7.0
34.766238966 38.1128082275391 0.0
0 0.0543892383575439 1.0
0 -0.266553848981857 2.0
0 -0.366740852594376 3.0
0 -0.0241129100322723 4.0
0 -0.0405623018741608 5.0
0 0.0145333111286163 6.0
0 -0.0475595891475677 7.0
39.697519492 37.697639465332 0.0
0 -0.015803337097168 1.0
0 0.218629330396652 2.0
0 0.257490932941437 3.0
0 -0.0330057442188263 4.0
0 0.134436964988708 5.0
0 0.0268847346305847 6.0
0 -0.0132358968257904 7.0
36.459952599 38.6323699951172 0.0
0 0.0211819112300873 1.0
0 0.197260141372681 2.0
0 -0.268608063459396 3.0
0 -0.142322152853012 4.0
0 0.0323430001735687 5.0
0 0.0126661956310272 6.0
0 -0.041635125875473 7.0
44.856515586 37.2520065307617 0.0
0 0.0149086117744446 1.0
0 1.62149810791016 2.0
0 0.539750277996063 3.0
0 0.0856549739837646 4.0
0 0.150924175977707 5.0
0 -0.0499078929424286 6.0
0 0.232294961810112 7.0
41.157913956 37.1700439453125 0.0
0 0.00199732184410095 1.0
0 0.817769110202789 2.0
0 0.361072510480881 3.0
0 -0.109636455774307 4.0
0 0.102562814950943 5.0
0 0.000998586416244507 6.0
0 0.0135476291179657 7.0
32.092085792 37.6975212097168 0.0
0 0.0115975439548492 1.0
0 -0.494477957487106 2.0
0 -0.75693690776825 3.0
0 0.00283017754554749 4.0
0 -0.181776911020279 5.0
0 -0.0136788189411163 6.0
0 -0.0334043204784393 7.0
30.327518425 37.7932929992676 0.0
0 -0.0022423267364502 1.0
0 -0.661984324455261 2.0
0 -0.786434769630432 3.0
0 -0.341766327619553 4.0
0 -0.0785493552684784 5.0
0 0.0176271796226501 6.0
0 -0.0240576565265656 7.0
36.482977278 38.4943885803223 0.0
0 0.0603820085525513 1.0
0 0.351719677448273 2.0
0 -0.250766724348068 3.0
0 0.0755843222141266 4.0
0 -0.17302480340004 5.0
0 0.0246903896331787 6.0
0 0.0179241597652435 7.0
40.491170914 37.3554725646973 0.0
0 -0.0537866652011871 1.0
0 0.592685103416443 2.0
0 0.16993847489357 3.0
0 0.154788136482239 4.0
0 -0.198535650968552 5.0
0 0.0412156581878662 6.0
0 0.061668336391449 7.0
41.188898334 37.4446601867676 0.0
0 0.0641465783119202 1.0
0 1.00775384902954 2.0
0 0.361176401376724 3.0
0 -0.160997003316879 4.0
0 0.0213830769062042 5.0
0 0.0135297179222107 6.0
0 -0.0392268598079681 7.0
37.756949904 38.3852882385254 0.0
0 0.0180520415306091 1.0
0 0.360069066286087 2.0
0 -0.0541398823261261 3.0
0 -0.0934447944164276 4.0
0 -0.018984854221344 5.0
0 0.032959520816803 6.0
0 -0.0301582515239716 7.0
26.9626458 38.9474906921387 0.0
0 -0.0164003670215607 1.0
0 -2.52343440055847 2.0
0 -1.24311494827271 3.0
0 -0.229741603136063 4.0
0 -0.174796789884567 5.0
0 -0.141753822565079 6.0
0 -0.00551623106002808 7.0
44.154680866 37.9782905578613 0.0
0 -0.0132213532924652 1.0
0 2.09981751441956 2.0
0 0.896598219871521 3.0
0 -0.0662078559398651 4.0
0 -0.0453117191791534 5.0
0 -0.0161224901676178 6.0
0 0.660722374916077 7.0
36.49920361 37.8903465270996 0.0
0 0.0277237296104431 1.0
0 -0.070603996515274 2.0
0 -0.258313089609146 3.0
0 0.144510328769684 4.0
0 -0.0824920833110809 5.0
0 -0.0046362578868866 6.0
0 -0.045390397310257 7.0
36.136187007 37.2441101074219 0.0
0 -0.0494689047336578 1.0
0 -0.119914144277573 2.0
0 -0.245296031236649 3.0
0 0.136627346277237 4.0
0 -0.0263318717479706 5.0
0 -0.120538324117661 6.0
0 0.0513081252574921 7.0
43.805905709 37.6064910888672 0.0
0 -0.00211063027381897 1.0
0 1.45155453681946 2.0
0 0.645930528640747 3.0
0 0.0512465834617615 4.0
0 0.121807545423508 5.0
0 0.0136411786079407 6.0
0 0.0816564559936523 7.0
43.400673478 38.8980445861816 0.0
0 0.0158452391624451 1.0
0 1.13670432567596 2.0
0 0.366032093763351 3.0
0 0.168810278177261 4.0
0 0.0462433695793152 5.0
0 -0.0523947179317474 6.0
0 0.163033008575439 7.0
35.708898932 37.9456520080566 0.0
0 -0.0445531904697418 1.0
0 -0.385639935731888 2.0
0 -0.520099759101868 3.0
0 0.0287808477878571 4.0
0 -0.120472639799118 5.0
0 -0.0265864729881287 6.0
0 0.345603227615356 7.0
42.195375979 38.216667175293 0.0
0 -0.00962898135185242 1.0
0 0.999915719032288 2.0
0 0.410657674074173 3.0
0 0.127200424671173 4.0
0 0.051592230796814 5.0
0 0.0424105226993561 6.0
0 0.0474206805229187 7.0
38.219132252 38.7597618103027 0.0
0 0.043500542640686 1.0
0 -0.197195023298264 2.0
0 -0.117157727479935 3.0
0 -0.178760081529617 4.0
0 -0.0284276306629181 5.0
0 0.00494316220283508 6.0
0 -0.0603722631931305 7.0
37.34785904 37.7450065612793 0.0
0 0.0345292687416077 1.0
0 -0.195403426885605 2.0
0 -0.0394544899463654 3.0
0 -0.031347244977951 4.0
0 0.00192913413047791 5.0
0 0.0521170198917389 6.0
0 -0.0399884283542633 7.0
41.413317083 37.7437400817871 0.0
0 -0.00287318229675293 1.0
0 0.454920917749405 2.0
0 -0.0866118967533112 3.0
0 -0.0267677009105682 4.0
0 0.0935737192630768 5.0
0 -0.00997713208198547 6.0
0 0.0145072042942047 7.0
39.408474012 39.8284454345703 0.0
0 -0.00922167301177979 1.0
0 0.713748455047607 2.0
0 -0.0575031936168671 3.0
0 0.0940714180469513 4.0
0 -0.0446329414844513 5.0
0 0.0123689472675323 6.0
0 0.0332256853580475 7.0
44.621947793 39.0112457275391 0.0
0 0.0436350405216217 1.0
0 1.84085643291473 2.0
0 0.390873372554779 3.0
0 0.215181916952133 4.0
0 0.0667123198509216 5.0
0 0.0563094019889832 6.0
0 0.138874411582947 7.0
35.042314673 38.1724815368652 0.0
0 -0.0105720460414886 1.0
0 -0.79607355594635 2.0
0 -0.283095389604568 3.0
0 -0.122635692358017 4.0
0 0.0896082818508148 5.0
0 0.0139018297195435 6.0
0 -0.0645895898342133 7.0
44.668685987 36.5688705444336 0.0
0 0.046678751707077 1.0
0 1.19571590423584 2.0
0 0.7161585688591 3.0
0 0.069768875837326 4.0
0 0.0859482884407043 5.0
0 0.0339719653129578 6.0
0 0.0440485775470734 7.0
44.271188235 38.0241165161133 0.0
0 0.0804373621940613 1.0
0 1.51207756996155 2.0
0 0.741229951381683 3.0
0 -0.0489716827869415 4.0
0 0.0216118097305298 5.0
0 -0.0368122756481171 6.0
0 -0.0123284757137299 7.0
43.329615715 38.3723564147949 0.0
0 0.0298317074775696 1.0
0 0.942001461982727 2.0
0 0.437874913215637 3.0
0 0.215020149946213 4.0
0 0.105014622211456 5.0
0 -0.0753577053546906 6.0
0 0.139167070388794 7.0
37.727034893 38.5219993591309 0.0
0 -0.0415655076503754 1.0
0 0.329666048288345 2.0
0 -0.145508319139481 3.0
0 -0.00646528601646423 4.0
0 -0.146019786596298 5.0
0 0.0358109772205353 6.0
0 0.134538948535919 7.0
43.665312113 38.608699798584 0.0
0 0.0744343996047974 1.0
0 1.12881779670715 2.0
0 0.508457839488983 3.0
0 -0.0131702721118927 4.0
0 0.0587190985679626 5.0
0 0.0218559205532074 6.0
0 0.0377456247806549 7.0
40.656348383 38.6891136169434 0.0
0 0.00886136293411255 1.0
0 1.32877564430237 2.0
0 0.355734348297119 3.0
0 0.125860601663589 4.0
0 -0.0358519852161407 5.0
0 0.0370421707630157 6.0
0 0.938868701457977 7.0
30.759139555 38.2037582397461 0.0
0 0.0305066406726837 1.0
0 -1.90346574783325 2.0
0 -0.745392441749573 3.0
0 -0.212602943181992 4.0
0 0.142933219671249 5.0
0 -0.0840743482112885 6.0
0 -0.0429741442203522 7.0
36.474229652 36.1768798828125 0.0
0 -0.0363980829715729 1.0
0 -0.32817742228508 2.0
0 -0.0605570375919342 3.0
0 -0.318716615438461 4.0
0 0.0763510763645172 5.0
0 0.00516554713249207 6.0
0 -0.0406336486339569 7.0
33.340721195 37.345458984375 0.0
0 0.0044693648815155 1.0
0 -0.704030275344849 2.0
0 -0.578238964080811 3.0
0 -0.0658557713031769 4.0
0 -0.0980494916439056 5.0
0 0.00311154127120972 6.0
0 0.0010753870010376 7.0
34.143349268 39.5640068054199 0.0
0 -0.146076530218124 1.0
0 -0.212895840406418 2.0
0 -0.421862155199051 3.0
0 -0.194014459848404 4.0
0 -0.0227789580821991 5.0
0 0.0302191078662872 6.0
0 -0.0404661595821381 7.0
40.763829757 37.7475166320801 0.0
0 0.0260895490646362 1.0
0 0.561390519142151 2.0
0 0.0562894940376282 3.0
0 0.248619094491005 4.0
0 -0.0363609492778778 5.0
0 0.0195097327232361 6.0
0 0.0611528754234314 7.0
29.788224958 38.1629333496094 0.0
0 -0.0317889750003815 1.0
0 -0.526137471199036 2.0
0 -0.81742000579834 3.0
0 -0.1181580722332 4.0
0 -0.267964214086533 5.0
0 0.0140591263771057 6.0
0 -0.0407598912715912 7.0
34.675869554 38.0525550842285 0.0
0 0.0116169452667236 1.0
0 -0.358101338148117 2.0
0 -0.308020502328873 3.0
0 -0.115295976400375 4.0
0 -0.0640860497951508 5.0
0 0.0396854281425476 6.0
0 -0.0647275745868683 7.0
40.723855828 37.553596496582 0.0
0 -0.0900666415691376 1.0
0 0.735757112503052 2.0
0 0.312197625637054 3.0
0 -0.00840896368026733 4.0
0 0.179304838180542 5.0
0 -0.0221403539180756 6.0
0 1.07449674606323 7.0
41.414838352 38.4477310180664 0.0
0 -0.00724098086357117 1.0
0 0.942968368530273 2.0
0 -0.0255630314350128 3.0
0 0.259004056453705 4.0
0 -0.0230595767498016 5.0
0 0.0212758183479309 6.0
0 0.12984973192215 7.0
33.391441214 38.3951873779297 0.0
0 0.00356423854827881 1.0
0 -0.486273437738419 2.0
0 -0.546425461769104 3.0
0 -0.0667296350002289 4.0
0 -0.0886664092540741 5.0
0 -0.0229828655719757 6.0
0 -0.0332169830799103 7.0
36.660645385 38.5063705444336 0.0
0 0.0503027737140656 1.0
0 -0.166006475687027 2.0
0 -0.492953270673752 3.0
0 -0.0631861984729767 4.0
0 -0.0594987571239471 5.0
0 0.0403631329536438 6.0
0 -0.0109817385673523 7.0
33.547567166 36.2871208190918 0.0
0 -0.0424629151821136 1.0
0 -0.511932849884033 2.0
0 -0.538069009780884 3.0
0 0.0561772584915161 4.0
0 -0.0944549143314362 5.0
0 -0.0129253566265106 6.0
0 -0.044609934091568 7.0
36.717923495 37.5891952514648 0.0
0 -0.0569951236248016 1.0
0 0.433856815099716 2.0
0 -0.145985990762711 3.0
0 0.0161522328853607 4.0
0 -0.0458091199398041 5.0
0 0.0211796164512634 6.0
0 -0.0610634982585907 7.0
44.575737494 37.9427909851074 0.0
0 -0.0779495537281036 1.0
0 1.13770520687103 2.0
0 0.755800724029541 3.0
0 0.0854301750659943 4.0
0 0.0610607266426086 5.0
0 0.0279401540756226 6.0
0 0.0519422590732574 7.0
43.091610321 37.9852752685547 0.0
0 -0.0148956775665283 1.0
0 1.45558130741119 2.0
0 0.517203330993652 3.0
0 0.0697386264801025 4.0
0 0.0251313447952271 5.0
0 0.0122534930706024 6.0
0 0.0513400733470917 7.0
42.647069973 38.1647644042969 0.0
0 0.026370644569397 1.0
0 0.985974192619324 2.0
0 0.474388986825943 3.0
0 0.0735975503921509 4.0
0 0.0807190537452698 5.0
0 0.0336315631866455 6.0
0 0.0535813271999359 7.0
36.240225339 38.0172386169434 0.0
0 0.00182679295539856 1.0
0 -0.322965651750565 2.0
0 -0.180555611848831 3.0
0 -0.11571404337883 4.0
0 -0.0312249362468719 5.0
0 0.0536815822124481 6.0
0 -0.0244355797767639 7.0
31.876756203 38.047534942627 0.0
0 -0.0760480463504791 1.0
0 -0.605665683746338 2.0
0 -0.63357150554657 3.0
0 -0.196373790502548 4.0
0 -0.132960051298141 5.0
0 0.0264699459075928 6.0
0 -0.0375917255878448 7.0
38.711049659 38.3277473449707 0.0
0 -0.082956463098526 1.0
0 0.314717054367065 2.0
0 0.0106884837150574 3.0
0 -0.114188641309738 4.0
0 0.177716970443726 5.0
0 -0.0630269348621368 6.0
0 -0.0461783707141876 7.0
33.592635672 38.6184577941895 0.0
0 0.0170921981334686 1.0
0 -0.991007447242737 2.0
0 -0.457338064908981 3.0
0 -0.0971750915050507 4.0
0 -0.0442380607128143 5.0
0 0.0500286221504211 6.0
0 -0.0457682311534882 7.0
31.939497021 37.9781112670898 0.0
0 -0.036555677652359 1.0
0 -0.954605579376221 2.0
0 -0.644748091697693 3.0
0 -0.232822090387344 4.0
0 0.0553779006004333 5.0
0 0.0410000383853912 6.0
0 -0.0444577038288116 7.0
32.605677369 39.5351333618164 0.0
0 -0.0364867150783539 1.0
0 -0.771359086036682 2.0
0 -0.496018558740616 3.0
0 -0.212856858968735 4.0
0 0.0338919162750244 5.0
0 0.0422211885452271 6.0
0 -0.0207336246967316 7.0
35.02561425 38.6220054626465 0.0
0 0.0380437672138214 1.0
0 -0.67377781867981 2.0
0 -0.414541691541672 3.0
0 -0.0236606001853943 4.0
0 -0.0811810195446014 5.0
0 0.061872810125351 6.0
0 -0.0499819815158844 7.0
29.155020371 37.719654083252 0.0
0 0.0363408923149109 1.0
0 -1.65941691398621 2.0
0 -0.588409900665283 3.0
0 -0.123387604951859 4.0
0 -0.135929316282272 5.0
0 0.0416273772716522 6.0
0 -0.0379381477832794 7.0
29.9790999 36.9570579528809 0.0
0 -0.106330007314682 1.0
0 -0.684025645256042 2.0
0 -0.898196339607239 3.0
0 -0.136256605386734 4.0
0 -0.208820372819901 5.0
0 -0.0121319890022278 6.0
0 -0.0185496211051941 7.0
40.412374428 38.5639915466309 0.0
0 -0.0498742163181305 1.0
0 1.0072158575058 2.0
0 0.185395926237106 3.0
0 -0.0645168125629425 4.0
0 0.120518863201141 5.0
0 -0.0201872885227203 6.0
0 0.0381520688533783 7.0
44.799669607 38.4441261291504 0.0
0 0.0158124268054962 1.0
0 1.4548259973526 2.0
0 0.904554128646851 3.0
0 -0.0659115612506866 4.0
0 0.0744893550872803 5.0
0 -0.0100348293781281 6.0
0 0.626753211021423 7.0
35.691930085 38.5408325195312 0.0
0 -0.123791068792343 1.0
0 0.0393179357051849 2.0
0 -0.212616533041 3.0
0 -0.105817764997482 4.0
0 -0.014313817024231 5.0
0 -0.00438547134399414 6.0
0 -0.0488525927066803 7.0
37.293002382 38.0975761413574 0.0
0 -0.0187940895557404 1.0
0 0.162140607833862 2.0
0 -0.190025717020035 3.0
0 -0.0210875272750854 4.0
0 -0.0955968201160431 5.0
0 0.0346626043319702 6.0
0 0.0385762751102448 7.0
31.749050635 38.5369911193848 0.0
0 -0.0186328291893005 1.0
0 -1.49082660675049 2.0
0 -0.520395040512085 3.0
0 -0.156350165605545 4.0
0 -0.134665697813034 5.0
0 0.0391263663768768 6.0
0 -0.0328096449375153 7.0
44.438599316 37.3588523864746 0.0
0 0.0242540240287781 1.0
0 0.722833037376404 2.0
0 0.653316378593445 3.0
0 0.179726451635361 4.0
0 0.0651595890522003 5.0
0 -0.001056969165802 6.0
0 0.252179801464081 7.0
30.584762345 38.2101287841797 0.0
0 -0.0237400829792023 1.0
0 -0.811002373695374 2.0
0 -0.871310710906982 3.0
0 -0.107119709253311 4.0
0 -0.234345942735672 5.0
0 -0.123825222253799 6.0
0 -0.0316235721111298 7.0
36.167711467 38.5791511535645 0.0
0 0.0340359508991241 1.0
0 -0.452563315629959 2.0
0 -0.102622121572495 3.0
0 -0.270025640726089 4.0
0 -0.0254765152931213 5.0
0 0.00123226642608643 6.0
0 -0.0275563299655914 7.0
45.589023165 38.1438293457031 0.0
0 -0.0141007006168365 1.0
0 1.30477404594421 2.0
0 0.587352752685547 3.0
0 0.128600150346756 4.0
0 0.0809884369373322 5.0
0 0.0339106917381287 6.0
0 0.107214957475662 7.0
33.954863975 39.0133056640625 0.0
0 0.0257588028907776 1.0
0 -0.0055515468120575 2.0
0 -0.535663843154907 3.0
0 -0.0198977589607239 4.0
0 -0.139153510332108 5.0
0 -0.0979550778865814 6.0
0 0.00653794407844543 7.0
39.104021217 38.1776351928711 0.0
0 -0.0274868905544281 1.0
0 0.677122414112091 2.0
0 0.154235780239105 3.0
0 -0.0352371037006378 4.0
0 0.00314909219741821 5.0
0 0.0325975716114044 6.0
0 -0.0361160337924957 7.0
41.615139375 38.7671165466309 0.0
0 -0.113191515207291 1.0
0 2.13644957542419 2.0
0 0.684418678283691 3.0
0 0.0583910942077637 4.0
0 0.0100042521953583 5.0
0 0.0205450057983398 6.0
0 1.5006468296051 7.0
29.251209169 37.7931900024414 0.0
0 0.0553510189056396 1.0
0 -1.43917727470398 2.0
0 -0.787691712379456 3.0
0 -0.131267160177231 4.0
0 -0.165975958108902 5.0
0 -0.0916138589382172 6.0
0 -0.0317428410053253 7.0
39.145261601 38.0190353393555 0.0
0 0.0402939021587372 1.0
0 0.417411386966705 2.0
0 -0.041892021894455 3.0
0 0.0244126319885254 4.0
0 0.0497766137123108 5.0
0 0.04117152094841 6.0
0 0.0333792269229889 7.0
43.724223514 37.6033897399902 0.0
0 -0.00352194905281067 1.0
0 0.787213325500488 2.0
0 0.545513987541199 3.0
0 0.0987031161785126 4.0
0 0.0947729051113129 5.0
0 0.0212439894676208 6.0
0 0.244300663471222 7.0
30.694427948 37.9997901916504 0.0
0 -0.0114031732082367 1.0
0 -1.23192465305328 2.0
0 -0.735457539558411 3.0
0 -0.159554272890091 4.0
0 -0.246383160352707 5.0
0 0.0188727974891663 6.0
0 -0.00112038850784302 7.0
46.899446079 38.5695190429688 0.0
0 -0.0413410365581512 1.0
0 1.43464004993439 2.0
0 0.697018146514893 3.0
0 0.215936809778214 4.0
0 0.0414231717586517 5.0
0 -0.144466668367386 6.0
0 0.360958993434906 7.0
30.801741874 37.759822845459 0.0
0 0.0572121441364288 1.0
0 -1.35387516021729 2.0
0 -0.796643495559692 3.0
0 -0.0481330454349518 4.0
0 -0.167438954114914 5.0
0 -0.141192644834518 6.0
0 -0.0626432597637177 7.0
40.388900757 38.2041625976562 0.0
0 0.110538125038147 1.0
0 0.403992831707001 2.0
0 0.0104900896549225 3.0
0 0.128271251916885 4.0
0 0.099056750535965 5.0
0 -0.14397069811821 6.0
0 0.148137181997299 7.0
35.801686131 37.6284790039062 0.0
0 -0.0298397243022919 1.0
0 -0.630969405174255 2.0
0 -0.279015928506851 3.0
0 -0.2440445125103 4.0
0 -0.160833269357681 5.0
0 -0.00386640429496765 6.0
0 -0.0821110904216766 7.0
43.275430302 38.4462547302246 0.0
0 -0.0452731549739838 1.0
0 1.47296226024628 2.0
0 0.404370784759521 3.0
0 0.120460838079453 4.0
0 -0.0321696698665619 5.0
0 0.0412383079528809 6.0
0 0.0210069417953491 7.0
36.674533885 38.1683540344238 0.0
0 -0.0198332369327545 1.0
0 0.0434493124485016 2.0
0 -0.151685565710068 3.0
0 -0.0396496951580048 4.0
0 -0.0304886996746063 5.0
0 0.0324015319347382 6.0
0 -0.0376475751399994 7.0
42.291929941 38.3409881591797 0.0
0 0.0166175663471222 1.0
0 0.664012312889099 2.0
0 0.448504447937012 3.0
0 0.000885277986526489 4.0
0 0.0409136712551117 5.0
0 0.0544681549072266 6.0
0 0.0781054198741913 7.0
37.80686855 38.165599822998 0.0
0 0.000339716672897339 1.0
0 0.0803843438625336 2.0
0 0.0179984569549561 3.0
0 -0.192803055047989 4.0
0 -0.0495445430278778 5.0
0 0.0222902595996857 6.0
0 -0.0609360039234161 7.0
36.296445808 38.6877937316895 0.0
0 0.0722845494747162 1.0
0 0.101001054048538 2.0
0 -0.211300760507584 3.0
0 -0.134661883115768 4.0
0 -0.0158767104148865 5.0
0 0.018459677696228 6.0
0 -0.0292125642299652 7.0
26.786347985 39.156867980957 0.0
0 -0.0440907776355743 1.0
0 -0.854947686195374 2.0
0 -0.824981927871704 3.0
0 -0.269544452428818 4.0
0 -0.203250795602798 5.0
0 -0.0995990931987762 6.0
0 -0.00718522071838379 7.0
38.715518937 39.6762733459473 0.0
0 -0.0177838504314423 1.0
0 0.0190917253494263 2.0
0 -0.102657645940781 3.0
0 0.0845089852809906 4.0
0 -0.025517076253891 5.0
0 -0.0160001516342163 6.0
0 0.0368534326553345 7.0
35.505063002 38.5767517089844 0.0
0 -0.0392483174800873 1.0
0 -0.042005866765976 2.0
0 -0.332822173833847 3.0
0 -0.150412708520889 4.0
0 -0.0506062805652618 5.0
0 0.0465732514858246 6.0
0 -0.0216045379638672 7.0
44.227649071 38.6124382019043 0.0
0 -0.0556440651416779 1.0
0 1.51173865795135 2.0
0 0.602035522460938 3.0
0 0.101921945810318 4.0
0 0.00212427973747253 5.0
0 0.043914794921875 6.0
0 0.0723752379417419 7.0
45.773124255 37.8079261779785 0.0
0 -0.00628188252449036 1.0
0 1.91340351104736 2.0
0 0.799879014492035 3.0
0 0.0965557396411896 4.0
0 0.157332271337509 5.0
0 -0.0102691352367401 6.0
0 0.0616367757320404 7.0
38.615282752 39.4313087463379 0.0
0 -0.114446491003036 1.0
0 0.934215605258942 2.0
0 0.0862705409526825 3.0
0 -0.016563892364502 4.0
0 -0.120141416788101 5.0
0 0.045153021812439 6.0
0 0.970207214355469 7.0
42.131124836 37.7904624938965 0.0
0 -0.083805650472641 1.0
0 0.827475011348724 2.0
0 0.366186708211899 3.0
0 -0.0131983160972595 4.0
0 0.0837361514568329 5.0
0 0.032332718372345 6.0
0 0.0425986051559448 7.0
38.030369766 38.6188583374023 0.0
0 -0.0990695655345917 1.0
0 0.219958573579788 2.0
0 -0.215440064668655 3.0
0 0.0096551775932312 4.0
0 0.020585834980011 5.0
0 -0.0309785306453705 6.0
0 0.225936010479927 7.0
32.694870002 38.3865432739258 0.0
0 -0.0770018994808197 1.0
0 -0.829386472702026 2.0
0 -0.613612294197083 3.0
0 -0.210865706205368 4.0
0 0.133636832237244 5.0
0 -0.0772944986820221 6.0
0 -0.0590320527553558 7.0
41.58883184 38.3973236083984 0.0
0 -0.128785043954849 1.0
0 1.06935596466064 2.0
0 0.367453455924988 3.0
0 -0.154054492712021 4.0
0 0.070822149515152 5.0
0 0.00624606013298035 6.0
0 0.00464242696762085 7.0
33.622063045 38.4418678283691 0.0
0 0.059673011302948 1.0
0 -0.20069882273674 2.0
0 -0.584372758865356 3.0
0 0.00648900866508484 4.0
0 -0.221666604280472 5.0
0 0.0171381533145905 6.0
0 -0.0309703052043915 7.0
40.101738506 39.8272895812988 0.0
0 -0.0290087163448334 1.0
0 0.781081318855286 2.0
0 0.00807499885559082 3.0
0 0.197212487459183 4.0
0 -0.0316996276378632 5.0
0 -0.00504487752914429 6.0
0 0.0848739743232727 7.0
39.347750494 38.6376914978027 0.0
0 0.0326572954654694 1.0
0 0.804118990898132 2.0
0 0.125025063753128 3.0
0 -0.0326739251613617 4.0
0 0.10558295249939 5.0
0 0.0238703787326813 6.0
0 0.710553288459778 7.0
46.240008463 38.5155944824219 0.0
0 0.016793817281723 1.0
0 1.78604078292847 2.0
0 0.709984958171844 3.0
0 -0.00222867727279663 4.0
0 -0.0117977857589722 5.0
0 0.0118048489093781 6.0
0 0.133130848407745 7.0
43.885613972 38.5607833862305 0.0
0 0.144348740577698 1.0
0 1.20529139041901 2.0
0 0.548233389854431 3.0
0 -0.0139211416244507 4.0
0 0.072644054889679 5.0
0 0.028004378080368 6.0
0 0.0384495556354523 7.0
44.877286055 38.0248336791992 0.0
0 -0.0356482565402985 1.0
0 1.71227157115936 2.0
0 0.706928431987762 3.0
0 0.262192785739899 4.0
0 -0.0840410888195038 5.0
0 -0.0494904816150665 6.0
0 0.353560835123062 7.0
44.281024849 38.3742942810059 0.0
0 0.00309759378433228 1.0
0 1.11830914020538 2.0
0 0.425123661756516 3.0
0 0.157504469156265 4.0
0 0.0652454495429993 5.0
0 0.0349218547344208 6.0
0 0.0837924480438232 7.0
39.443195334 37.8026885986328 0.0
0 0.0102928578853607 1.0
0 0.479854375123978 2.0
0 0.0985898971557617 3.0
0 -0.00969091057777405 4.0
0 0.0315072536468506 5.0
0 0.0164314806461334 6.0
0 -0.0353882610797882 7.0
44.50998012 37.3463973999023 0.0
0 -0.0169224739074707 1.0
0 1.55456435680389 2.0
0 0.65507310628891 3.0
0 0.0836113691329956 4.0
0 0.0818211734294891 5.0
0 0.024893194437027 6.0
0 0.0226562321186066 7.0
36.079576822 38.4664878845215 0.0
0 0.0161034464836121 1.0
0 -0.201319307088852 2.0
0 -0.414338618516922 3.0
0 0.115859717130661 4.0
0 -0.0871318280696869 5.0
0 0.0253755152225494 6.0
0 -0.00856471061706543 7.0
43.881758876 39.8910179138184 0.0
0 0.000328242778778076 1.0
0 2.5189905166626 2.0
0 0.721178710460663 3.0
0 -0.0502881705760956 4.0
0 0.0319193601608276 5.0
0 0.00809100270271301 6.0
0 1.04075026512146 7.0
30.108818284 36.8491630554199 0.0
0 0.0331783890724182 1.0
0 -1.30730521678925 2.0
0 -0.688471436500549 3.0
0 -0.102159172296524 4.0
0 -0.102010935544968 5.0
0 -0.0722479522228241 6.0
0 -0.0163693428039551 7.0
34.22051219 38.1842460632324 0.0
0 0.0286427736282349 1.0
0 -0.142994672060013 2.0
0 -0.612900495529175 3.0
0 0.108315467834473 4.0
0 -0.0700349509716034 5.0
0 0.0143690705299377 6.0
0 0.0128897726535797 7.0
37.402197971 38.680118560791 0.0
0 -0.0362415015697479 1.0
0 0.197019547224045 2.0
0 -0.014576643705368 3.0
0 0.0224778354167938 4.0
0 -0.116527050733566 5.0
0 0.00502640008926392 6.0
0 -0.0639179050922394 7.0
35.453722409 37.709156036377 0.0
0 0.0167311728000641 1.0
0 -0.498890548944473 2.0
0 -0.359484046697617 3.0
0 -0.0516423881053925 4.0
0 0.133619040250778 5.0
0 -0.105374187231064 6.0
0 -0.0220082104206085 7.0
32.872381155 38.2971305847168 0.0
0 -0.0990553200244904 1.0
0 -0.581633448600769 2.0
0 -0.6550532579422 3.0
0 -0.181791335344315 4.0
0 0.00491622090339661 5.0
0 0.0304801464080811 6.0
0 0.0215885937213898 7.0
33.577714316 38.3698043823242 0.0
0 -0.0947717130184174 1.0
0 -0.35871633887291 2.0
0 -0.521745920181274 3.0
0 -0.0596606433391571 4.0
0 -0.2835473716259 5.0
0 0.0445408523082733 6.0
0 -0.0287639796733856 7.0
43.986168802 36.6260032653809 0.0
0 -0.0137883424758911 1.0
0 0.789835572242737 2.0
0 0.0551254451274872 3.0
0 -0.0614887177944183 4.0
0 0.0450031757354736 5.0
0 0.0283835232257843 6.0
0 0.0161790251731873 7.0
39.701527205 38.6733818054199 0.0
0 0.00615948438644409 1.0
0 0.312890410423279 2.0
0 0.0880099534988403 3.0
0 0.0749066472053528 4.0
0 0.0680590569972992 5.0
0 0.0277470648288727 6.0
0 0.12348136305809 7.0
36.319422317 37.5694847106934 0.0
0 0.0374801456928253 1.0
0 -0.200310617685318 2.0
0 -0.324823468923569 3.0
0 0.138300895690918 4.0
0 -0.111760228872299 5.0
0 0.00110459327697754 6.0
0 -0.0119470953941345 7.0
39.54652972 38.2449569702148 0.0
0 0.000582635402679443 1.0
0 0.911799907684326 2.0
0 0.150873184204102 3.0
0 0.114919692277908 4.0
0 -0.0575626790523529 5.0
0 0.0141554474830627 6.0
0 0.0129421949386597 7.0
46.3288844 37.925853729248 0.0
0 -0.0233434438705444 1.0
0 2.22350239753723 2.0
0 0.787264823913574 3.0
0 0.0597258508205414 4.0
0 0.0760122835636139 5.0
0 0.0166820883750916 6.0
0 0.0685180425643921 7.0
39.359671461 37.9191856384277 0.0
0 0.0144284963607788 1.0
0 0.382854104042053 2.0
0 0.047181248664856 3.0
0 0.111428499221802 4.0
0 -0.136802405118942 5.0
0 0.0461002290248871 6.0
0 -0.0169160068035126 7.0
43.813870988 39.2016334533691 0.0
0 -0.0669676959514618 1.0
0 1.01597583293915 2.0
0 0.603542625904083 3.0
0 0.0454472303390503 4.0
0 0.0888770818710327 5.0
0 0.0193924605846405 6.0
0 0.0604300796985626 7.0
39.146221876 38.3304100036621 0.0
0 0.00792747735977173 1.0
0 0.081896036863327 2.0
0 -0.258961766958237 3.0
0 0.15278372168541 4.0
0 0.110046029090881 5.0
0 -0.0817656219005585 6.0
0 0.0343771576881409 7.0
34.753353014 39.0619125366211 0.0
0 -0.046236664056778 1.0
0 -0.0753102004528046 2.0
0 -0.59538745880127 3.0
0 0.0783649981021881 4.0
0 -0.118464201688766 5.0
0 -0.102223187685013 6.0
0 0.849960565567017 7.0
36.858915008 39.1625213623047 0.0
0 -0.0396778285503387 1.0
0 0.0380693376064301 2.0
0 -0.324180334806442 3.0
0 0.1348916888237 4.0
0 -0.0331349074840546 5.0
0 0.0169941186904907 6.0
0 0.785203039646149 7.0
45.609388725 38.9880332946777 0.0
0 0.0353558361530304 1.0
0 1.59201753139496 2.0
0 0.772328853607178 3.0
0 0.0325125157833099 4.0
0 0.0979205965995789 5.0
0 -0.012559562921524 6.0
0 0.0146353542804718 7.0
37.221502314 38.2829971313477 0.0
0 0.0486613214015961 1.0
0 0.0222895443439484 2.0
0 -0.236235290765762 3.0
0 0.0832694172859192 4.0
0 -0.179926663637161 5.0
0 -0.000369280576705933 6.0
0 0.0240909457206726 7.0
39.351241848 39.3557205200195 0.0
0 0.00766286253929138 1.0
0 0.791463017463684 2.0
0 0.149800419807434 3.0
0 0.17218890786171 4.0
0 -0.136720567941666 5.0
0 -0.0365296304225922 6.0
0 0.0993119776248932 7.0
44.354286139 37.1266403198242 0.0
0 0.0412713587284088 1.0
0 1.00926780700684 2.0
0 0.619926035404205 3.0
0 0.0956478714942932 4.0
0 0.0557312667369843 5.0
0 0.0265247821807861 6.0
0 0.0306362807750702 7.0
41.143843969 37.8796691894531 0.0
0 -0.0202117264270782 1.0
0 0.2995845079422 2.0
0 0.158109277486801 3.0
0 0.0175772309303284 4.0
0 0.118915557861328 5.0
0 -0.0271309614181519 6.0
0 0.0417793393135071 7.0
32.436363672 38.1157417297363 0.0
0 0.108734726905823 1.0
0 -1.27567446231842 2.0
0 -0.549058079719543 3.0
0 -0.0272566676139832 4.0
0 -0.0400669872760773 5.0
0 -0.00269854068756104 6.0
0 -0.0622613728046417 7.0
35.980268826 37.698486328125 0.0
0 0.0312800109386444 1.0
0 -0.228605955839157 2.0
0 -0.232096940279007 3.0
0 0.079499214887619 4.0
0 -0.0715079009532928 5.0
0 -0.0308017432689667 6.0
0 -0.0663217008113861 7.0
39.361696951 37.887378692627 0.0
0 -0.0494515597820282 1.0
0 0.941659092903137 2.0
0 0.132000416517258 3.0
0 -0.231270343065262 4.0
0 -0.0369683802127838 5.0
0 -0.000379502773284912 6.0
0 -0.0363234579563141 7.0
38.339957501 37.8423919677734 0.0
0 -0.0435488522052765 1.0
0 0.856434226036072 2.0
0 0.0123614072799683 3.0
0 0.00704970955848694 4.0
0 -0.00164932012557983 5.0
0 0.00370559096336365 6.0
0 1.02121913433075 7.0
38.437078026 38.6074333190918 0.0
0 -0.0941479504108429 1.0
0 1.2054854631424 2.0
0 0.252515971660614 3.0
0 -0.00246089696884155 4.0
0 -0.0997379720211029 5.0
0 -0.000387191772460938 6.0
0 1.91817104816437 7.0
31.680301161 38.4038047790527 0.0
0 -0.0730888545513153 1.0
0 -0.84307324886322 2.0
0 -0.814336657524109 3.0
0 -0.111711353063583 4.0
0 -0.129891246557236 5.0
0 -0.0222685933113098 6.0
0 0.0536773800849915 7.0
36.698769132 37.6468200683594 0.0
0 -0.0206885039806366 1.0
0 -1.07486426830292 2.0
0 -0.233874589204788 3.0
0 -0.125756651163101 4.0
0 0.0849790275096893 5.0
0 0.0256407558917999 6.0
0 -0.00269070267677307 7.0
40.942334922 39.9749908447266 0.0
0 -0.0464788377285004 1.0
0 0.395676225423813 2.0
0 0.258812487125397 3.0
0 -0.0820527970790863 4.0
0 0.0334739685058594 5.0
0 0.04819056391716 6.0
0 0.167959988117218 7.0
41.496981537 38.2411727905273 0.0
0 0.0266203284263611 1.0
0 1.07317578792572 2.0
0 0.489969700574875 3.0
0 0.0126970112323761 4.0
0 0.071982204914093 5.0
0 0.00535589456558228 6.0
0 1.19555735588074 7.0
33.577466065 36.9342727661133 0.0
0 0.0257668495178223 1.0
0 -0.872485160827637 2.0
0 -0.600973010063171 3.0
0 0.0307832658290863 4.0
0 -0.130819052457809 5.0
0 -0.0528099238872528 6.0
0 0.252137720584869 7.0
38.410697595 38.6224250793457 0.0
0 -0.00403627753257751 1.0
0 0.720973789691925 2.0
0 0.185471773147583 3.0
0 -0.251736789941788 4.0
0 -0.0119940042495728 5.0
0 0.00169548392295837 6.0
0 -0.0403356850147247 7.0
40.390522112 38.9434013366699 0.0
0 -0.029240757226944 1.0
0 0.960174560546875 2.0
0 0.0452598035335541 3.0
0 0.217796981334686 4.0
0 0.00940781831741333 5.0
0 -0.0717208683490753 6.0
0 0.150417774915695 7.0
38.02724699 37.4662666320801 0.0
0 -0.0275631844997406 1.0
0 -0.033369392156601 2.0
0 0.0609936118125916 3.0
0 -0.129786878824234 4.0
0 0.0923448503017426 5.0
0 0.0408604443073273 6.0
0 -0.0108247399330139 7.0
34.831322295 37.452766418457 0.0
0 0.0305184125900269 1.0
0 -0.726799964904785 2.0
0 -0.413708180189133 3.0
0 -0.130264669656754 4.0
0 -0.0193396806716919 5.0
0 0.0329628586769104 6.0
0 0.0216392278671265 7.0
33.152476832 37.5823402404785 0.0
0 -0.0851857364177704 1.0
0 -0.492479592561722 2.0
0 -0.533522844314575 3.0
0 -0.156130462884903 4.0
0 -0.0507475435733795 5.0
0 -0.0676204264163971 6.0
0 -0.00113150477409363 7.0
37.76181476 37.6931953430176 0.0
0 -0.0328963100910187 1.0
0 0.0305260121822357 2.0
0 -0.148641437292099 3.0
0 0.0454874634742737 4.0
0 -0.0299809873104095 5.0
0 0.0128882825374603 6.0
0 -0.0412571728229523 7.0
46.821650967 38.2775344848633 0.0
0 0.0286263525485992 1.0
0 2.38094902038574 2.0
0 0.88955020904541 3.0
0 0.119852334260941 4.0
0 -0.0178356766700745 5.0
0 0.0176417827606201 6.0
0 0.277017861604691 7.0
41.183790175 39.9020881652832 0.0
0 0.0129043459892273 1.0
0 0.583121180534363 2.0
0 0.373299151659012 3.0
0 -0.0544111430644989 4.0
0 0.0406754314899445 5.0
0 0.0261692106723785 6.0
0 0.00669565796852112 7.0
36.78498864 39.3653030395508 0.0
0 -0.0371750891208649 1.0
0 -0.953216314315796 2.0
0 -0.268789261579514 3.0
0 0.192337304353714 4.0
0 0.0294080078601837 5.0
0 -0.0443116724491119 6.0
0 -0.0373700559139252 7.0
37.689494945 38.365119934082 0.0
0 -0.0834979712963104 1.0
0 0.859630525112152 2.0
0 0.17061522603035 3.0
0 0.0302331149578094 4.0
0 -0.0961403548717499 5.0
0 0.0186133980751038 6.0
0 1.78847455978394 7.0
41.308881097 38.3512306213379 0.0
0 0.0252665579319 1.0
0 0.376072943210602 2.0
0 0.397322714328766 3.0
0 0.0447943806648254 4.0
0 0.00966653227806091 5.0
0 0.045295238494873 6.0
0 -0.0105388164520264 7.0
42.253011184 37.1834526062012 0.0
0 -0.044936865568161 1.0
0 0.974548578262329 2.0
0 0.356156378984451 3.0
0 0.15157362818718 4.0
0 -0.0186652541160583 5.0
0 0.0262832045555115 6.0
0 0.0775788426399231 7.0
40.047660493 38.1944694519043 0.0
0 -0.00892540812492371 1.0
0 0.367306411266327 2.0
0 0.20391383767128 3.0
0 -0.0751852095127106 4.0
0 -0.0315669476985931 5.0
0 0.0341484248638153 6.0
0 1.01710641384125 7.0
46.604605733 39.2786560058594 0.0
0 -0.0175401270389557 1.0
0 1.24532508850098 2.0
0 1.09441924095154 3.0
0 0.0243585705757141 4.0
0 0.151715606451035 5.0
0 -0.0473507344722748 6.0
0 0.0906441509723663 7.0
45.106295542 38.6322135925293 0.0
0 0.0235176384449005 1.0
0 1.48195803165436 2.0
0 0.60973459482193 3.0
0 0.190087854862213 4.0
0 0.033701479434967 5.0
0 -0.0120623111724854 6.0
0 0.358897000551224 7.0
38.977055099 38.8568954467773 0.0
0 0.0586048364639282 1.0
0 0.560654103755951 2.0
0 -0.00396722555160522 3.0
0 0.0706243216991425 4.0
0 -0.0230173170566559 5.0
0 0.0161882340908051 6.0
0 0.0698666870594025 7.0
38.935365883 37.9343338012695 0.0
0 -0.0153950154781342 1.0
0 0.517863631248474 2.0
0 0.108974307775497 3.0
0 -0.139427989721298 4.0
0 0.0881659984588623 5.0
0 0.0347747206687927 6.0
0 0.00762933492660522 7.0
42.741409918 38.9490966796875 0.0
0 -0.0100374221801758 1.0
0 1.01003503799438 2.0
0 0.518382370471954 3.0
0 -0.0338847935199738 4.0
0 0.0538663268089294 5.0
0 0.0239258706569672 6.0
0 0.000382006168365479 7.0
31.310794576 38.2825126647949 0.0
0 -0.0116852521896362 1.0
0 -1.68343067169189 2.0
0 -0.827005386352539 3.0
0 -0.0348661243915558 4.0
0 -0.0645361244678497 5.0
0 0.010288268327713 6.0
0 0.013467937707901 7.0
};
\addlegendentry{$R^2$=0.983}
\end{axis}

\end{tikzpicture}
}}
    \subfloat[Actual vs predicted edge flows.] 
    {\label{fig:results_nonlineal_dummy_edge_base_f_bal}\resizebox{\figurewidth}{\figureheight}{% This file was created with tikzplotlib v0.10.1.
\begin{tikzpicture}

\definecolor{darkgray176}{RGB}{176,176,176}
\definecolor{lightgray204}{RGB}{204,204,204}

\begin{axis}[
colorbar,
colorbar style={ylabel={edge id}},
colormap={mymap}{[1pt]
 rgb(0pt)=(0.12156862745098,0.466666666666667,0.705882352941177);
  rgb(1pt)=(1,0.498039215686275,0.0549019607843137);
  rgb(2pt)=(0.172549019607843,0.627450980392157,0.172549019607843);
  rgb(3pt)=(0.83921568627451,0.152941176470588,0.156862745098039);
  rgb(4pt)=(0.580392156862745,0.403921568627451,0.741176470588235);
  rgb(5pt)=(0.549019607843137,0.337254901960784,0.294117647058824);
  rgb(6pt)=(0.890196078431372,0.466666666666667,0.76078431372549);
  rgb(7pt)=(0.498039215686275,0.498039215686275,0.498039215686275);
  rgb(8pt)=(0.737254901960784,0.741176470588235,0.133333333333333);
  rgb(9pt)=(0.0901960784313725,0.745098039215686,0.811764705882353)
},
legend cell align={left},
legend style={
  fill opacity=0.8,
  draw opacity=1,
  text opacity=1,
  at={(0.03,0.97)},
  anchor=north west,
  draw=lightgray204
},
point meta max=7,
point meta min=0,
tick align=outside,
tick pos=left,
title={ye test-ye pred},
x grid style={darkgray176},
xlabel={ye test},
xmajorgrids,
xmin=-15.25776872815, xmax=51.84084685915,
xtick style={color=black},
y grid style={darkgray176},
ylabel={ye pred},
ymajorgrids,
ymin=-13.1978090763092, ymax=47.7495201587677,
ytick style={color=black}
]
\addplot [
  colormap={mymap}{[1pt]
 rgb(0pt)=(0.12156862745098,0.466666666666667,0.705882352941177);
  rgb(1pt)=(1,0.498039215686275,0.0549019607843137);
  rgb(2pt)=(0.172549019607843,0.627450980392157,0.172549019607843);
  rgb(3pt)=(0.83921568627451,0.152941176470588,0.156862745098039);
  rgb(4pt)=(0.580392156862745,0.403921568627451,0.741176470588235);
  rgb(5pt)=(0.549019607843137,0.337254901960784,0.294117647058824);
  rgb(6pt)=(0.890196078431372,0.466666666666667,0.76078431372549);
  rgb(7pt)=(0.498039215686275,0.498039215686275,0.498039215686275);
  rgb(8pt)=(0.737254901960784,0.741176470588235,0.133333333333333);
  rgb(9pt)=(0.0901960784313725,0.745098039215686,0.811764705882353)
},
  only marks,
  scatter,
  scatter src=explicit
]
table [x=x, y=y, meta=colordata]{%
x  y  colordata
39.565898635 39.3345794677734 0.0
18.050517226 20.0723266601562 1.0
21.515381409 19.9729766845703 2.0
-8.8987516789 -6.97377347946167 3.0
12.61662972 12.5903730392456 4.0
26.949268915 27.140079498291 5.0
39.565898645 39.2012176513672 6.0
39.565898645 39.2268257141113 7.0
42.743257171 40.1491851806641 0.0
22.448801828 20.9024467468262 1.0
20.294455342 21.3691215515137 2.0
-4.2455956832 -5.29506778717041 3.0
16.048859649 15.9766044616699 4.0
26.694397521 26.7978935241699 5.0
42.743257181 39.4642486572266 6.0
42.743257181 40.6815414428711 7.0
39.367180747 39.7446403503418 0.0
18.970258592 19.7440013885498 1.0
20.39692216 20.324197769165 2.0
-4.9776642323 -4.62439155578613 3.0
15.419257923 15.4482841491699 4.0
23.947922829 24.1330471038818 5.0
39.367180751 39.0624656677246 6.0
39.367180751 38.7107276916504 7.0
39.605393097 38.7040519714355 0.0
22.082745928 20.5085277557373 1.0
17.522647169 18.8980751037598 2.0
-7.2517903354 -8.44517135620117 3.0
10.270856823 10.3232707977295 4.0
29.334536274 29.1012153625488 5.0
39.605393107 38.2746734619141 6.0
39.605393107 38.7892150878906 7.0
43.937345526 38.6753387451172 0.0
21.865843593 22.4252281188965 1.0
22.071501933 21.6168231964111 2.0
-8.3469765086 -8.00339221954346 3.0
13.724525414 13.7125625610352 4.0
30.212820112 30.6568183898926 5.0
43.937345536 38.2790412902832 6.0
43.937345536 41.6920509338379 7.0
31.061989584 38.3475532531738 0.0
16.203677385 17.0252819061279 1.0
14.858312199 15.8402662277222 2.0
-5.5704045507 -5.98298263549805 3.0
9.2879076384 9.30077171325684 4.0
21.774081945 21.8559398651123 5.0
31.061989594 38.3677368164062 6.0
31.061989594 33.9937629699707 7.0
36.357435265 38.6616439819336 0.0
19.38173709 19.6196308135986 1.0
16.975698176 17.7391376495361 2.0
-8.3424496148 -8.50375556945801 3.0
8.6332485532 8.76476764678955 4.0
27.724186714 27.7909984588623 5.0
36.357435273 38.1068954467773 6.0
36.357435273 37.1308975219727 7.0
38.444969607 38.7742691040039 0.0
21.080570389 20.0594997406006 1.0
17.364399218 19.3066329956055 2.0
-5.0980379196 -6.44902372360229 3.0
12.266361289 12.3819894790649 4.0
26.178608318 26.3588123321533 5.0
38.444969617 38.7529830932617 6.0
38.444969617 38.9205131530762 7.0
35.498620518 38.3921432495117 0.0
17.900031823 18.4577865600586 1.0
17.598588698 18.5077953338623 2.0
-4.651704216 -4.93939304351807 3.0
12.946884476 12.9111051559448 4.0
22.551736046 22.5877056121826 5.0
35.498620524 38.5943946838379 6.0
35.498620524 37.113468170166 7.0
36.52099827 39.3547134399414 0.0
18.961240079 18.6721839904785 1.0
17.559758192 18.4015712738037 2.0
-5.3314768445 -6.08781337738037 3.0
12.228281339 12.1524457931519 4.0
24.292716932 24.400505065918 5.0
36.520998279 38.9182395935059 6.0
36.520998279 36.993049621582 7.0
36.717272204 38.2971420288086 0.0
19.763035348 19.7459144592285 1.0
16.954236857 18.3168601989746 2.0
-7.0689091682 -8.10307121276855 3.0
9.88532768 9.79075908660889 4.0
26.831944525 27.1091442108154 5.0
36.717272212 38.3257904052734 6.0
36.717272212 37.7552604675293 7.0
32.629628996 38.9628829956055 0.0
17.103127104 17.9149131774902 1.0
15.526501892 16.2193450927734 2.0
-7.1999965109 -7.32573366165161 3.0
8.3265053708 8.4040060043335 4.0
24.303123625 24.4697704315186 5.0
32.629629006 38.7401962280273 6.0
32.629629006 34.9978981018066 7.0
37.75267433 35.9422569274902 0.0
17.533699782 18.5568466186523 1.0
20.218974548 19.9474334716797 2.0
-3.919745576 -3.33978843688965 3.0
16.299228962 16.2989921569824 4.0
21.453445368 21.4073009490967 5.0
37.75267434 36.116138458252 6.0
37.75267434 37.7508125305176 7.0
38.800291337 38.081241607666 0.0
21.414385846 20.3076782226562 1.0
17.385905491 18.9737663269043 2.0
-6.3164463541 -7.50169229507446 3.0
11.069459127 11.1304721832275 4.0
27.73083221 27.9864082336426 5.0
38.800291347 38.3804702758789 6.0
38.800291347 38.5848579406738 7.0
38.252729609 38.851806640625 0.0
20.199036122 19.9070777893066 1.0
18.053693488 19.025505065918 2.0
-6.9495443437 -7.48799180984497 3.0
11.104149136 11.0629940032959 4.0
27.148580475 27.201000213623 5.0
38.252729618 38.3880424499512 6.0
38.252729618 38.5338172912598 7.0
43.273596037 39.3659133911133 0.0
20.811072897 21.8572235107422 1.0
22.46252314 20.4961032867432 2.0
-10.185324283 -8.94838523864746 3.0
12.277198847 12.1719207763672 4.0
30.99639719 31.4776496887207 5.0
43.273596047 38.5025787353516 6.0
43.273596047 40.7984161376953 7.0
34.027431477 39.1298141479492 0.0
16.463058834 17.8683471679688 1.0
17.564372643 16.9279403686523 2.0
-7.8238749469 -6.75731611251831 3.0
9.7404976863 9.94193744659424 4.0
24.28693379 24.5775299072266 5.0
34.027431486 38.2427711486816 6.0
34.027431486 35.4309234619141 7.0
41.154171382 39.3738174438477 0.0
20.969763 21.9758605957031 1.0
20.184408383 19.918664932251 2.0
-9.8031418559 -9.11959552764893 3.0
10.381266519 10.2539939880371 4.0
30.772904865 30.8099784851074 5.0
41.154171391 39.0281867980957 6.0
41.154171391 40.6942100524902 7.0
39.9308264078757 37.9129028320312 0.0
21.2554356118761 21.1868114471436 1.0
18.6753908028755 19.3683986663818 2.0
-8.41978314337461 -8.74329853057861 3.0
10.2556076588756 10.1733961105347 4.0
29.6752187558758 29.8780746459961 5.0
39.930826408 37.6298484802246 6.0
39.930826408 39.9071197509766 7.0
45.969703622 37.8526992797852 0.0
25.855804141 23.0218162536621 1.0
20.113899483 22.2923698425293 2.0
-4.729541314 -7.19895124435425 3.0
15.38435816 15.2290802001953 4.0
30.585345464 30.6686973571777 5.0
45.969703631 38.238151550293 6.0
45.969703631 42.9410247802734 7.0
38.398120212 37.4333763122559 0.0
19.546425793 20.4437122344971 1.0
18.851694421 18.1665153503418 2.0
-9.8922695857 -8.78772068023682 3.0
8.9594248259 9.00230312347412 4.0
29.438695387 29.348180770874 5.0
38.398120221 37.3549690246582 6.0
38.398120221 38.7551422119141 7.0
28.366017296 39.3723602294922 0.0
14.995271371 15.6443758010864 1.0
13.370745924 14.8143091201782 2.0
-4.4280592982 -5.00721645355225 3.0
8.9426866162 9.01937770843506 4.0
19.42333068 19.6219120025635 5.0
28.366017306 39.357234954834 6.0
28.366017306 32.1149940490723 7.0
39.07981897 38.4662818908691 0.0
18.202914308 19.5859107971191 1.0
20.876904662 19.9735527038574 2.0
-6.6281228283 -5.587730884552 3.0
14.248781824 14.2133045196533 4.0
24.831037146 24.9802474975586 5.0
39.079818979 38.4765892028809 6.0
39.079818979 38.6683807373047 7.0
40.466329374 39.5043487548828 0.0
19.842708002 20.6467437744141 1.0
20.623621372 21.0924472808838 2.0
-6.0436252005 -5.89025402069092 3.0
14.579996162 14.6062860488892 4.0
25.886333212 26.1635684967041 5.0
40.466329383 38.972469329834 6.0
40.466329383 40.1097068786621 7.0
38.293116843 38.8729972839355 0.0
18.225197444 19.2173194885254 1.0
20.067919399 20.1265621185303 2.0
-4.6459628917 -4.19734621047974 3.0
15.421956498 15.3532686233521 4.0
22.871160346 23.0721111297607 5.0
38.293116853 38.5435180664062 6.0
38.293116853 38.078498840332 7.0
43.346089487 39.773853302002 0.0
21.23189324 22.0176811218262 1.0
22.114196248 21.120735168457 2.0
-9.387268107 -8.68057250976562 3.0
12.726928131 12.5109510421753 4.0
30.619161357 30.812967300415 5.0
43.346089497 39.6920547485352 6.0
43.346089497 41.3932151794434 7.0
34.547871144 39.5555458068848 0.0
19.900789633 18.1978607177734 1.0
14.647081512 15.9535398483276 2.0
-6.6924528176 -7.65826749801636 3.0
7.9546286844 8.11317253112793 4.0
26.59324246 25.9812564849854 5.0
34.547871154 38.9658393859863 6.0
34.547871154 34.7629699707031 7.0
41.927050143208 39.4558563232422 0.0
21.8340017142739 22.0362205505371 1.0
20.0930484299516 19.2269287109375 2.0
-12.0156472013624 -10.4274759292603 3.0
8.07740122783537 8.29088973999023 4.0
33.8496486756856 34.3163719177246 5.0
41.927050143 38.983081817627 6.0
41.927050143 40.4694290161133 7.0
44.548236367 38.1956176757812 0.0
22.097953544 22.1178379058838 1.0
22.450282823 22.1396999359131 2.0
-6.9116853021 -6.79388952255249 3.0
15.538597511 15.4395246505737 4.0
29.009638856 29.3877658843994 5.0
44.548236377 38.1344604492188 6.0
44.548236377 42.0627021789551 7.0
34.958415477 39.1969909667969 0.0
19.046265323 18.4219799041748 1.0
15.912150156 17.8373336791992 2.0
-3.947719982 -5.31135702133179 3.0
11.964430164 12.0062065124512 4.0
22.993985314 23.2450084686279 5.0
34.958415487 39.09033203125 6.0
34.958415487 36.5466918945312 7.0
41.619773288 37.5924377441406 0.0
18.036088777 20.8765716552734 1.0
23.583684511 20.827356338501 2.0
-9.1140217471 -6.01758623123169 3.0
14.469662754 14.2635269165039 4.0
27.150110534 26.9405460357666 5.0
41.619773298 37.3148612976074 6.0
41.619773298 40.6650199890137 7.0
35.768623904 39.1770133972168 0.0
18.854929214 18.8277683258057 1.0
16.913694692 17.602222442627 2.0
-6.8203969536 -7.22317361831665 3.0
10.093297731 10.0157594680786 4.0
25.675326176 25.6959228515625 5.0
35.768623912 38.3223838806152 6.0
35.768623912 36.8563003540039 7.0
36.035873071 36.1648788452148 0.0
18.197202866 19.4080810546875 1.0
17.838670207 17.2279510498047 2.0
-9.8802007315 -8.85068321228027 3.0
7.958469467 8.09082126617432 4.0
28.077403606 28.1332855224609 5.0
36.03587308 36.0396308898926 6.0
36.03587308 36.8168792724609 7.0
42.671159575 38.5748901367188 0.0
22.542382477 22.094331741333 1.0
20.128777099 20.17578125 2.0
-9.3171954196 -9.38669872283936 3.0
10.811581671 10.6329412460327 4.0
31.859577906 31.5848865509033 5.0
42.671159584 38.4652061462402 6.0
42.671159584 41.0942649841309 7.0
32.270518381 38.7154083251953 0.0
16.341237488 17.4038562774658 1.0
15.929280893 16.0235004425049 2.0
-7.2560391874 -6.93990278244019 3.0
8.6732416955 8.73000335693359 4.0
23.597276685 23.7827129364014 5.0
32.270518391 38.4286079406738 6.0
32.270518391 34.5629577636719 7.0
36.239834932 38.6521148681641 0.0
18.511196273 18.595043182373 1.0
17.72863866 18.3012752532959 2.0
-5.3875241311 -5.58633518218994 3.0
12.34111452 12.3703098297119 4.0
23.898720413 23.9559860229492 5.0
36.239834941 38.3725662231445 6.0
36.239834941 37.0857048034668 7.0
38.292005172 38.408016204834 0.0
20.144840979 20.0303077697754 1.0
18.147164195 18.9869403839111 2.0
-6.9715526507 -7.55614423751831 3.0
11.175611535 11.186408996582 4.0
27.116393638 27.4978694915771 5.0
38.292005181 38.1750984191895 6.0
38.292005181 38.6253662109375 7.0
41.142171116 37.6595153808594 0.0
21.697070846 20.7462921142578 1.0
19.44510027 20.6265621185303 2.0
-5.49061917 -6.55220651626587 3.0
13.95448109 13.8942441940308 4.0
27.187690026 27.1193714141846 5.0
41.142171126 37.6438407897949 6.0
41.142171126 40.1901016235352 7.0
30.660234741 36.8207015991211 0.0
16.571808951 16.4452457427979 1.0
14.08842579 16.0057201385498 2.0
-3.6405774968 -5.00668239593506 3.0
10.447848283 10.5123109817505 4.0
20.212386458 20.5225582122803 5.0
30.660234751 37.3039360046387 6.0
30.660234751 33.7859230041504 7.0
42.776716976 38.0745239257812 0.0
18.943122536 21.00315284729 1.0
23.83359444 20.9342308044434 2.0
-9.5992659914 -7.30014276504517 3.0
14.234328438 14.0723857879639 4.0
28.542388537 28.5507755279541 5.0
42.776716986 37.6589508056641 6.0
42.776716986 40.5874557495117 7.0
39.136657948 37.3209190368652 0.0
21.244730642 19.6619319915771 1.0
17.891927309 19.1688365936279 2.0
-5.2069844714 -6.68005132675171 3.0
12.684942831 12.6854581832886 4.0
26.45171512 26.6918659210205 5.0
39.136657955 36.6887626647949 6.0
39.136657955 38.1578178405762 7.0
40.593075656 38.2428550720215 0.0
19.181149666 21.2314033508301 1.0
21.411925992 19.176477432251 2.0
-11.788434371 -9.28981685638428 3.0
9.6234916129 9.64438819885254 4.0
30.969584045 31.2156734466553 5.0
40.593075664 37.8778877258301 6.0
40.593075664 39.719367980957 7.0
38.455960047 37.3077201843262 0.0
19.606731652 20.1573219299316 1.0
18.849228395 17.8166522979736 2.0
-10.066988594 -9.06369876861572 3.0
8.7822397903 8.78284549713135 4.0
29.673720256 29.7491626739502 5.0
38.455960057 37.4087829589844 6.0
38.455960057 37.9487571716309 7.0
40.029822299 38.315113067627 0.0
19.696760735 20.4950580596924 1.0
20.333061566 20.0638446807861 2.0
-6.9313004298 -6.60739231109619 3.0
13.401761128 13.1824064254761 4.0
26.628061173 26.802770614624 5.0
40.029822307 37.9573822021484 6.0
40.029822307 39.7206687927246 7.0
39.721089796 38.5319404602051 0.0
21.982867862 20.3762035369873 1.0
17.738221933 18.9817981719971 2.0
-6.2167396454 -7.77141904830933 3.0
11.521482278 11.4240837097168 4.0
28.199607518 28.3973579406738 5.0
39.721089806 37.7835426330566 6.0
39.721089806 38.4864463806152 7.0
46.012802771 37.5924758911133 0.0
23.852238991 23.197639465332 1.0
22.16056378 21.3634700775146 2.0
-9.7843125098 -9.45822906494141 3.0
12.37625126 12.1921653747559 4.0
33.636551511 33.0815315246582 5.0
46.012802781 37.578784942627 6.0
46.012802781 43.3273773193359 7.0
43.79104115 37.987907409668 0.0
23.810489927 22.4961891174316 1.0
19.980551224 20.6266708374023 2.0
-8.8972279263 -9.46152400970459 3.0
11.083323289 11.1469554901123 4.0
32.707717862 32.3605308532715 5.0
43.791041158 37.9134521484375 6.0
43.791041158 41.6670799255371 7.0
31.257332414 38.7568817138672 0.0
14.677686479 16.7751502990723 1.0
16.579645934 15.6296014785767 2.0
-7.7653359569 -6.30360507965088 3.0
8.8143099674 8.96261310577393 4.0
22.443022446 22.6259880065918 5.0
31.257332424 38.6366806030273 6.0
31.257332424 33.5342979431152 7.0
38.988472902 38.8736724853516 0.0
18.706030748 19.2559146881104 1.0
20.282442155 20.3734226226807 2.0
-4.4931154738 -4.18191003799438 3.0
15.789326673 15.8503284454346 4.0
23.19914623 23.2987194061279 5.0
38.98847291 38.4399070739746 6.0
38.98847291 38.3524551391602 7.0
38.691218489 39.7889823913574 0.0
19.181746937 19.4423675537109 1.0
19.509471553 19.8143577575684 2.0
-4.305791265 -4.58238554000854 3.0
15.203680278 15.0802068710327 4.0
23.487538212 23.5581016540527 5.0
38.691218499 38.9600257873535 6.0
38.691218499 38.2042007446289 7.0
39.033211962 37.3370933532715 0.0
21.751284426 20.5173416137695 1.0
17.281927536 18.8593463897705 2.0
-7.1228573229 -8.53970527648926 3.0
10.159070204 10.2152242660522 4.0
28.874141758 29.0438957214355 5.0
39.033211971 37.0781898498535 6.0
39.033211971 38.7087020874023 7.0
37.697547803 38.48779296875 0.0
18.641388049 19.2011413574219 1.0
19.056159754 18.8110523223877 2.0
-6.2754402376 -5.89070272445679 3.0
12.780719506 12.8012113571167 4.0
24.916828297 24.9200630187988 5.0
37.697547813 38.3790435791016 6.0
37.697547813 37.4092025756836 7.0
35.277541329 38.3713493347168 0.0
16.862419132 18.0923957824707 1.0
18.415122198 17.8501148223877 2.0
-6.2131742864 -5.56184816360474 3.0
12.201947903 12.1626577377319 4.0
23.075593428 23.0515003204346 5.0
35.277541339 38.0776405334473 6.0
35.277541339 36.2151145935059 7.0
36.1197639664031 37.6640892028809 0.0
19.1517629624013 17.7198619842529 1.0
16.9680010143999 18.9016609191895 2.0
-0.473187581125011 -2.44807291030884 3.0
16.4948134334014 16.5538864135742 4.0
19.6249505434022 19.9452838897705 5.0
36.119763966 37.7177848815918 6.0
36.119763966 36.1134834289551 7.0
33.14490304 38.0710563659668 0.0
16.831426445 17.2661914825439 1.0
16.313476595 17.4312877655029 2.0
-3.637680404 -4.01696443557739 3.0
12.675796181 12.8616161346436 4.0
20.469106859 20.6613864898682 5.0
33.14490305 37.3677406311035 6.0
33.14490305 35.0627975463867 7.0
34.4868008142505 38.0146179199219 0.0
18.3228058422505 18.440408706665 1.0
16.1639949822505 16.3285999298096 2.0
-8.00476657755046 -7.83313322067261 3.0
8.15922840445046 8.17999839782715 4.0
26.3275723771958 26.1861038208008 5.0
34.486800814 37.4829254150391 6.0
34.486800814 35.4492607116699 7.0
40.933468199 37.9780616760254 0.0
20.889274427 20.9033012390137 1.0
20.044193774 20.4612979888916 2.0
-6.533500242 -6.75594282150269 3.0
13.510693524 13.5270853042603 4.0
27.422774677 27.5301609039307 5.0
40.933468207 38.4501419067383 6.0
40.933468207 40.0449295043945 7.0
35.993417919 38.6016731262207 0.0
18.964188614 19.3762035369873 1.0
17.029229306 17.9529457092285 2.0
-7.4575389686 -7.48704767227173 3.0
9.5716903284 9.73286914825439 4.0
26.421727592 26.6882953643799 5.0
35.993417928 38.4622802734375 6.0
35.993417928 37.4206352233887 7.0
39.331666914 37.8897171020508 0.0
19.408314017 20.0682792663574 1.0
19.923352898 19.5476264953613 2.0
-7.2428639103 -6.75395727157593 3.0
12.680488979 12.7047252655029 4.0
26.651177937 26.875373840332 5.0
39.331666923 37.8529281616211 6.0
39.331666923 38.8823280334473 7.0
37.559548486 38.6692161560059 0.0
18.054642609 19.4285068511963 1.0
19.504905877 18.6113243103027 2.0
-8.2806473226 -7.0376935005188 3.0
11.224258545 11.1863851547241 4.0
26.335289942 26.435188293457 5.0
37.559548496 38.7164154052734 6.0
37.559548496 37.8561325073242 7.0
41.796902472 38.4606056213379 0.0
20.074580702 21.4977321624756 1.0
21.72232177 19.7847480773926 2.0
-11.24258785 -9.37717437744141 3.0
10.479733911 10.632363319397 4.0
31.317168562 31.7291469573975 5.0
41.796902482 38.5771293640137 6.0
41.796902482 40.2876586914062 7.0
35.679590813 38.7638244628906 0.0
15.932888769 18.4350833892822 1.0
19.746702044 18.3978633880615 2.0
-7.4161334139 -5.51938199996948 3.0
12.330568621 12.368106842041 4.0
23.349022193 23.4517936706543 5.0
35.679590823 38.576717376709 6.0
35.679590823 36.8794898986816 7.0
33.227547284 39.1693649291992 0.0
19.156810973 18.0659141540527 1.0
14.070736312 16.5076522827148 2.0
-5.2205843651 -7.02160024642944 3.0
8.8501519384 9.13248634338379 4.0
24.377395346 24.6919269561768 5.0
33.227547292 38.2826118469238 6.0
33.227547292 35.2505340576172 7.0
28.00807173 36.4160575866699 0.0
12.909757219 14.991003036499 1.0
15.098314511 15.0629119873047 2.0
-4.3989930962 -3.74069404602051 3.0
10.699321405 10.6914510726929 4.0
17.308750325 17.6976013183594 5.0
28.008071739 37.0195045471191 6.0
28.008071739 32.188232421875 7.0
33.4784988410636 39.1758308410645 0.0
17.4596045520637 18.0232162475586 1.0
16.0188942990637 16.8305435180664 2.0
-6.65519022176347 -6.72232246398926 3.0
9.36370407706344 9.48214721679688 4.0
24.1147947730637 24.4770679473877 5.0
33.478498841 38.7359771728516 6.0
33.478498841 35.5594367980957 7.0
33.1229468201162 38.9232940673828 0.0
16.2738036511163 17.3595848083496 1.0
16.8491431771162 15.6062183380127 2.0
-8.48238354211593 -6.91449499130249 3.0
8.36675963501607 8.46172714233398 4.0
24.7561871931169 24.052282333374 5.0
33.12294682 38.3879432678223 6.0
33.12294682 34.2443237304688 7.0
34.970228375 38.5084228515625 0.0
17.461716438 17.8510589599609 1.0
17.508511937 18.5217342376709 2.0
-3.5027478521 -4.1580662727356 3.0
14.005764076 13.9433917999268 4.0
20.964464299 21.0939502716064 5.0
34.970228384 38.4107856750488 6.0
34.970228384 36.1648330688477 7.0
36.637966033 38.6374244689941 0.0
18.611959282 19.2935981750488 1.0
18.026006753 19.0067119598389 2.0
-5.1789935919 -5.35238265991211 3.0
12.847013153 12.7913417816162 4.0
23.790952882 23.8577136993408 5.0
36.637966041 38.6813545227051 6.0
36.637966041 37.9695892333984 7.0
38.712447285 38.5915870666504 0.0
19.874849154 20.2247123718262 1.0
18.837598131 18.267333984375 2.0
-9.656315418 -8.9950008392334 3.0
9.181282703 9.41797637939453 4.0
29.531164582 29.7869091033936 5.0
38.712447295 38.5354385375977 6.0
38.712447295 37.9218940734863 7.0
29.0840224199388 39.2318649291992 0.0
12.9348822379388 15.6936225891113 1.0
16.1491401919388 15.2472724914551 2.0
-6.2158408037388 -4.69085073471069 3.0
9.93329938853882 9.80230140686035 4.0
19.1507230409388 19.3257083892822 5.0
29.08402242 39.148868560791 6.0
29.08402242 32.6611442565918 7.0
40.7419030105752 38.275016784668 0.0
19.6019574718567 20.5910205841064 1.0
21.139945535663 20.1964664459229 2.0
-8.29196876971742 -7.23989915847778 3.0
12.8479767652648 12.8541069030762 4.0
27.8939262422712 28.1106586456299 5.0
40.741903011 38.3861923217773 6.0
40.741903011 39.8275299072266 7.0
44.4239279780974 39.4227561950684 0.0
22.6860882760974 22.0330467224121 1.0
21.7378397120974 22.3923454284668 2.0
-6.03315836089742 -6.54087162017822 3.0
15.7046813510974 15.6339740753174 4.0
28.7192466370973 28.7417755126953 5.0
44.423927978 39.3079032897949 6.0
44.423927978 42.2716102600098 7.0
37.279929002 37.0849723815918 0.0
17.697440733 18.9515953063965 1.0
19.582488269 18.1912441253662 2.0
-8.4860638308 -6.93210077285767 3.0
11.096424428 11.066460609436 4.0
26.183504574 26.207612991333 5.0
37.279929011 36.5893783569336 6.0
37.279929011 36.9259719848633 7.0
39.065196556 37.6107864379883 0.0
22.911627527 20.5795345306396 1.0
16.153569029 18.16015625 2.0
-7.035333012 -8.8982982635498 3.0
9.1182360067 8.98701477050781 4.0
29.946960549 29.6531620025635 5.0
39.065196566 37.0505256652832 6.0
39.065196566 38.3482322692871 7.0
34.346712901 38.2734069824219 0.0
17.419062718 17.6810474395752 1.0
16.927650183 18.191068649292 2.0
-3.8964434363 -4.58442687988281 3.0
13.031206737 13.079906463623 4.0
21.315506165 21.4151992797852 5.0
34.346712911 37.4645690917969 6.0
34.346712911 35.9237213134766 7.0
44.832836192 39.2960395812988 0.0
22.86238329 22.4896030426025 1.0
21.970452902 21.9800281524658 2.0
-8.0950157692 -8.13814163208008 3.0
13.875437123 13.7864475250244 4.0
30.957399069 30.8783569335938 5.0
44.832836202 38.9298706054688 6.0
44.832836202 42.3807144165039 7.0
37.693005907 37.5732040405273 0.0
18.767233493 18.5866279602051 1.0
18.925772415 18.8119029998779 2.0
-4.4039460583 -4.41044807434082 3.0
14.521826348 14.445707321167 4.0
23.17117956 23.2283763885498 5.0
37.693005916 37.2829170227051 6.0
37.693005916 37.1398887634277 7.0
35.516030196 38.3085136413574 0.0
17.238388415 18.4299869537354 1.0
18.277641781 18.3606700897217 2.0
-5.9879556024 -5.43432140350342 3.0
12.289686168 12.2548675537109 4.0
23.226344027 23.2853546142578 5.0
35.516030206 38.1332244873047 6.0
35.516030206 36.8967552185059 7.0
45.449957515 38.7880096435547 0.0
24.470260824 23.4024410247803 1.0
20.979696692 21.6329116821289 2.0
-8.6547786371 -9.27630043029785 3.0
12.324918046 12.0158672332764 4.0
33.12503947 32.7615356445312 5.0
45.449957523 38.5583076477051 6.0
45.449957523 42.8902702331543 7.0
36.522103961 38.1988182067871 0.0
17.787809101 18.4845123291016 1.0
18.734294862 18.9984836578369 2.0
-3.7955939001 -3.68893098831177 3.0
14.938700953 14.894232749939 4.0
21.583403009 21.5477275848389 5.0
36.52210397 38.84375 6.0
36.52210397 38.1775741577148 7.0
40.809333824 37.9338455200195 0.0
21.825363738 20.607536315918 1.0
18.983970087 20.5572547912598 2.0
-4.5096782574 -6.15444087982178 3.0
14.474291821 14.4591627120972 4.0
26.335042004 26.8023624420166 5.0
40.809333833 37.758861541748 6.0
40.809333833 39.7419319152832 7.0
43.708815015 38.486457824707 0.0
22.06122806 21.6443004608154 1.0
21.647586954 21.4025249481201 2.0
-7.2998869135 -7.1194109916687 3.0
14.347700031 14.2060632705688 4.0
29.361114984 29.1562423706055 5.0
43.708815025 38.5050506591797 6.0
43.708815025 41.577320098877 7.0
35.746307566 36.6662445068359 0.0
17.418825391 18.8358345031738 1.0
18.327482175 16.9211444854736 2.0
-8.2684434585 -7.0301947593689 3.0
10.059038708 10.0344371795654 4.0
25.687268859 25.8386669158936 5.0
35.746307575 37.3261489868164 6.0
35.746307575 35.7919540405273 7.0
32.565260397 38.7693405151367 0.0
15.347203209 16.1670074462891 1.0
17.218057188 16.5169124603271 2.0
-4.4968959364 -4.1200213432312 3.0
12.721161242 12.8064947128296 4.0
19.844099155 20.0769023895264 5.0
32.565260407 38.3116569519043 6.0
32.565260407 33.7367134094238 7.0
37.459437559 38.8559761047363 0.0
20.974106733 19.7705554962158 1.0
16.485330826 17.6194324493408 2.0
-7.8025517563 -8.54379558563232 3.0
8.6827790601 8.62260341644287 4.0
28.776658499 28.5269012451172 5.0
37.459437569 37.9959449768066 6.0
37.459437569 36.8755950927734 7.0
41.6008687772188 38.5121154785156 0.0
18.8099504026802 20.4718456268311 1.0
22.7909183777866 20.4170742034912 2.0
-6.50941037372392 -4.63936138153076 3.0
16.2815080036091 16.0889644622803 4.0
25.319360776445 25.4699001312256 5.0
41.600868777 38.4078712463379 6.0
41.600868777 39.5987968444824 7.0
43.08864828 40.2582893371582 0.0
20.75366021 23.4074840545654 1.0
22.334988069 22.1483421325684 2.0
-9.5492158301 -9.31879234313965 3.0
12.785772229 12.5565137863159 4.0
30.30287605 32.5383758544922 5.0
43.088648289 39.3353843688965 6.0
43.088648289 43.2654113769531 7.0
30.268152668 37.7504234313965 0.0
15.014233699 16.4552745819092 1.0
15.25391897 15.566933631897 2.0
-5.5931438946 -5.28544473648071 3.0
9.6607750663 9.73861312866211 4.0
20.607377602 20.8413162231445 5.0
30.268152677 37.8697319030762 6.0
30.268152677 33.3416061401367 7.0
38.454045879 38.5938377380371 0.0
17.159579151 18.7904586791992 1.0
21.29446673 18.953577041626 2.0
-7.5807931052 -5.21518087387085 3.0
13.713673616 13.7108497619629 4.0
24.740372265 24.2856369018555 5.0
38.454045888 37.9366683959961 6.0
38.454045888 37.3657188415527 7.0
36.654056854 38.4793968200684 0.0
19.439010257 19.6138877868652 1.0
17.215046598 17.9176235198975 2.0
-7.9864590963 -8.25198745727539 3.0
9.2285874923 9.20231437683105 4.0
27.425469363 27.5615634918213 5.0
36.654056864 38.2055969238281 6.0
36.654056864 37.5632095336914 7.0
36.990278119 38.6199913024902 0.0
18.644416453 19.4465351104736 1.0
18.345861667 18.2143821716309 2.0
-7.9977544937 -7.60477876663208 3.0
10.348107164 10.4911279678345 4.0
26.642170956 26.9650650024414 5.0
36.990278128 38.4152755737305 6.0
36.990278128 37.2942047119141 7.0
39.45539178 38.480110168457 0.0
17.953554537 19.569149017334 1.0
21.501837243 20.1177310943604 2.0
-6.7799927019 -5.43611812591553 3.0
14.721844531 14.7498035430908 4.0
24.733547249 24.9646320343018 5.0
39.45539179 38.1169509887695 6.0
39.45539179 38.340202331543 7.0
41.490098301 38.4543647766113 0.0
18.418139969 20.3530082702637 1.0
23.071958334 20.9710445404053 2.0
-6.7020850264 -4.39486265182495 3.0
16.369873299 16.2571315765381 4.0
25.120225004 25.1643371582031 5.0
41.49009831 38.4754981994629 6.0
41.49009831 40.3983116149902 7.0
42.195848756 39.3670310974121 0.0
18.431626231 20.7523174285889 1.0
23.764222526 20.7349338531494 2.0
-9.6629820221 -6.98153924942017 3.0
14.101240496 14.0796232223511 4.0
28.094608262 28.2827053070068 5.0
42.195848764 39.1276321411133 6.0
42.195848764 40.1832847595215 7.0
36.580423947 38.2521629333496 0.0
18.088597645 20.7810020446777 1.0
18.491826302 18.2555904388428 2.0
-10.331066357 -9.50990867614746 3.0
8.1607599346 8.26523780822754 4.0
28.419664012 30.5462245941162 5.0
36.580423956 37.7904472351074 6.0
36.580423956 38.8716621398926 7.0
41.112612837 38.4932670593262 0.0
21.264748673 20.5060844421387 1.0
19.847864165 19.4797763824463 2.0
-7.8539892402 -7.81238889694214 3.0
11.993874916 11.7778568267822 4.0
29.118737922 29.0861740112305 5.0
41.112612846 38.1174507141113 6.0
41.112612846 39.3634147644043 7.0
41.165032295 38.8824081420898 0.0
20.777866565 21.4228210449219 1.0
20.387165729 19.1577224731445 2.0
-11.013080697 -9.77507495880127 3.0
9.3740850227 9.38996982574463 4.0
31.790947272 31.9651985168457 5.0
41.165032305 38.6067962646484 6.0
41.165032305 39.5214729309082 7.0
28.23632976 38.3263969421387 0.0
14.450875753 15.4980268478394 1.0
13.785454008 14.5968532562256 2.0
-5.4067157656 -5.52440738677979 3.0
8.3787382334 8.54624652862549 4.0
19.857591527 20.1096858978271 5.0
28.236329769 38.0705490112305 6.0
28.236329769 32.2906875610352 7.0
46.563722225 39.2035179138184 0.0
24.514076121 23.6944484710693 1.0
22.049646104 22.8095092773438 2.0
-6.9458736065 -7.63333177566528 3.0
15.103772488 14.9180755615234 4.0
31.459949737 31.2645606994629 5.0
46.563722235 38.5593070983887 6.0
46.563722235 43.8995361328125 7.0
38.976039438 35.4340286254883 0.0
20.47408062 19.5564193725586 1.0
18.50195882 19.8086585998535 2.0
-3.6791774969 -4.89018774032593 3.0
14.822781314 14.7713775634766 4.0
24.153258125 24.2811431884766 5.0
38.976039446 35.6982078552246 6.0
38.976039446 38.737476348877 7.0
43.902220155 38.7376441955566 0.0
23.085534006 23.8216934204102 1.0
20.816686149 22.9141082763672 2.0
-6.0778252495 -7.79718637466431 3.0
14.738860889 14.5385389328003 4.0
29.163359266 30.942253112793 5.0
43.902220165 38.7427215576172 6.0
43.902220165 43.9215660095215 7.0
37.104870377 38.0620384216309 0.0
21.928667997 20.4577560424805 1.0
15.176202381 17.8632850646973 2.0
-7.1829261037 -9.02495956420898 3.0
7.9932762681 8.19883918762207 4.0
29.11159411 29.3611602783203 5.0
37.104870385 37.6764793395996 6.0
37.104870385 37.9350891113281 7.0
48.094555135036 38.377498626709 0.0
24.9788607525581 23.4296455383301 1.0
23.1156943831877 21.875244140625 2.0
-7.58647576369851 -8.10384464263916 3.0
15.5292186191899 14.9413146972656 4.0
32.5653365171039 32.5541191101074 5.0
48.094555135 38.5407600402832 6.0
48.094555135 42.0341110229492 7.0
39.863550950908 37.8991851806641 0.0
20.2748443148297 20.9967517852783 1.0
19.5887066378693 18.4880752563477 2.0
-9.95882058391343 -8.87155246734619 3.0
9.62988605424657 9.76962852478027 4.0
30.233664897769 30.2971458435059 5.0
39.863550951 37.8148078918457 6.0
39.863550951 38.7496719360352 7.0
38.158424696 38.0357513427734 0.0
21.093616444 22.5340824127197 1.0
17.064808252 20.2541580200195 2.0
-6.6865614213 -8.4762659072876 3.0
10.378246821 10.3176116943359 4.0
27.780177875 30.832986831665 5.0
38.158424706 37.627254486084 6.0
38.158424706 41.4271507263184 7.0
37.671552635 39.4075622558594 0.0
18.060105754 19.2845287322998 1.0
19.611446882 19.2346115112305 2.0
-6.6868904195 -5.8576397895813 3.0
12.924556453 12.8451051712036 4.0
24.746996183 24.8283843994141 5.0
37.671552644 39.0014915466309 6.0
37.671552644 38.2057991027832 7.0
38.9900068701748 37.9956092834473 0.0
19.4317892501748 19.3694972991943 1.0
19.5582176301748 18.8778381347656 2.0
-6.7795985865748 -6.12608671188354 3.0
12.7786190431748 12.6215867996216 4.0
26.2113877989032 26.1713752746582 5.0
38.99000687 37.54833984375 6.0
38.99000687 37.9050254821777 7.0
32.824279187 39.2786102294922 0.0
17.524252685 17.4746742248535 1.0
15.300026502 16.2563152313232 2.0
-5.6991008156 -6.06389188766479 3.0
9.6009256767 9.7673921585083 4.0
23.223353511 23.2494812011719 5.0
32.824279197 39.2704544067383 6.0
32.824279197 34.6441116333008 7.0
39.752392388 38.9533920288086 0.0
19.337645943 19.948917388916 1.0
20.414746445 20.5483150482178 2.0
-4.1151033929 -3.62446928024292 3.0
16.299643042 16.1282444000244 4.0
23.452749346 23.4986915588379 5.0
39.752392398 38.972225189209 6.0
39.752392398 39.7438812255859 7.0
36.0149979881443 37.9607276916504 0.0
19.5282530962186 18.882869720459 1.0
16.4867448931817 17.5272083282471 2.0
-5.97886461861353 -6.82782602310181 3.0
10.5078802751384 10.5064334869385 4.0
25.5071177152005 25.5986003875732 5.0
36.014997988 38.1831130981445 6.0
36.014997988 36.9101829528809 7.0
42.307205596 39.0336723327637 0.0
21.688253633 20.767240524292 1.0
20.618951963 20.9607639312744 2.0
-4.4760898368 -4.81245183944702 3.0
16.142862116 15.9063291549683 4.0
26.164343479 26.1734790802002 5.0
42.307205606 38.9007835388184 6.0
42.307205606 40.5248908996582 7.0
41.624369925 38.6179847717285 0.0
21.269456313 21.4670906066895 1.0
20.354913614 19.8164291381836 2.0
-9.726585677 -9.30029582977295 3.0
10.628327928 10.443039894104 4.0
30.996041999 30.8735256195068 5.0
41.624369934 37.8871917724609 6.0
41.624369934 40.2387847900391 7.0
40.927860916 35.8812370300293 0.0
20.558611306 20.9596500396729 1.0
20.36924961 20.0280017852783 2.0
-7.6689328495 -7.40433931350708 3.0
12.700316751 12.6950750350952 4.0
28.227544165 28.355396270752 5.0
40.927860926 36.4459648132324 6.0
40.927860926 40.0307006835938 7.0
47.441321803 38.0800666809082 0.0
22.43048805 23.4143447875977 1.0
25.010833753 23.1378707885742 2.0
-8.9760978683 -7.34187269210815 3.0
16.034735875 15.7536697387695 4.0
31.406585928 31.4281730651855 5.0
47.441321812 38.3270378112793 6.0
47.441321812 43.9404296875 7.0
39.37950561 37.8171272277832 0.0
19.873167771 22.0693702697754 1.0
19.506337839 19.6082153320312 2.0
-9.8415530061 -9.18755531311035 3.0
9.6647848227 9.54823112487793 4.0
29.714720787 31.6812686920166 5.0
39.379505619 37.9542198181152 6.0
39.379505619 40.7324714660645 7.0
42.095557289 38.8284034729004 0.0
22.277305426 21.4153785705566 1.0
19.818251863 20.144250869751 2.0
-7.6874055589 -8.13358211517334 3.0
12.130846294 11.9622735977173 4.0
29.964710995 29.6125869750977 5.0
42.095557299 38.9218292236328 6.0
42.095557299 40.4712829589844 7.0
35.151188218 37.1212425231934 0.0
17.118344744 17.2001037597656 1.0
18.032843477 18.9425220489502 2.0
-2.0060677241 -2.57813787460327 3.0
16.026775747 16.1403694152832 4.0
19.124412475 19.4037742614746 5.0
35.151188225 36.9953117370605 6.0
35.151188225 36.0798759460449 7.0
37.713935362 40.0173988342285 0.0
20.380527895 19.0674839019775 1.0
17.333407469 18.2290306091309 2.0
-5.6423114193 -6.31645441055298 3.0
11.691096041 11.6885919570923 4.0
26.022839323 25.4497661590576 5.0
37.713935371 39.3742713928223 6.0
37.713935371 37.1830177307129 7.0
39.667743251 38.9768142700195 0.0
20.027544136 20.7649402618408 1.0
19.640199117 18.611141204834 2.0
-10.376032212 -9.30700874328613 3.0
9.2641668955 9.18335151672363 4.0
30.403576357 30.3999137878418 5.0
39.66774326 38.4366874694824 6.0
39.66774326 39.1105499267578 7.0
41.240556184 38.036808013916 0.0
23.347458936 21.3232231140137 1.0
17.893097248 19.3496608734131 2.0
-7.252142064 -8.6467809677124 3.0
10.640955174 10.7499237060547 4.0
30.59960101 30.3054752349854 5.0
41.240556194 37.6214332580566 6.0
41.240556194 39.7050170898438 7.0
40.32019459 38.1608657836914 0.0
19.177093862 20.319314956665 1.0
21.143100728 19.7744941711426 2.0
-7.2609061044 -6.21839952468872 3.0
13.882194613 13.8020782470703 4.0
26.437999977 27.6244144439697 5.0
40.3201946 37.7762718200684 6.0
40.3201946 39.3257637023926 7.0
39.5611510585585 38.8541374206543 0.0
19.6094718095582 20.2713298797607 1.0
19.9516792585585 19.0029792785645 2.0
-8.44726993515885 -7.36919069290161 3.0
11.5044093235587 11.4572906494141 4.0
28.0567416170478 28.0949554443359 5.0
39.561151059 38.6165237426758 6.0
39.561151059 39.1585845947266 7.0
43.5664319470049 39.1574745178223 0.0
22.3670487340048 22.1274738311768 1.0
21.1993832230048 20.8127517700195 2.0
-8.61702915780503 -8.6732816696167 3.0
12.582354065005 12.5103702545166 4.0
30.9840778920045 31.1338195800781 5.0
43.566431947 39.1498832702637 6.0
43.566431947 41.062858581543 7.0
37.929366556 38.3504753112793 0.0
20.105495873 19.7503261566162 1.0
17.823870687 18.8944702148438 2.0
-5.9768423324 -6.63515901565552 3.0
11.847028349 11.8233337402344 4.0
26.082338211 26.2165622711182 5.0
37.929366562 38.0031280517578 6.0
37.929366562 38.4383735656738 7.0
39.7456686413092 39.403694152832 0.0
20.6016083083092 19.3843326568604 1.0
19.1440603423092 20.2893848419189 2.0
-3.0966467648096 -4.1475658416748 3.0
16.0474135773093 16.0188045501709 4.0
23.6982550733092 23.8910846710205 5.0
39.745668641 37.8839416503906 6.0
39.745668641 38.3215293884277 7.0
37.1737946256367 39.048526763916 0.0
18.9810950589955 18.8391036987305 1.0
18.1926995622865 18.8903579711914 2.0
-4.83110587561615 -5.20935201644897 3.0
13.3615936860386 13.3734664916992 4.0
23.8122009348139 23.6739768981934 5.0
37.173794625 38.4979820251465 6.0
37.173794625 37.3299331665039 7.0
27.124866007 38.0027160644531 0.0
13.623671484 14.69162940979 1.0
13.501194525 13.9001789093018 2.0
-4.5998142153 -4.82250928878784 3.0
8.9013803006 9.11277866363525 4.0
18.223485707 18.6316566467285 5.0
27.124866015 38.0920600891113 6.0
27.124866015 30.7318325042725 7.0
27.411746895 36.4466209411621 0.0
13.763683287 15.0503730773926 1.0
13.648063611 14.3679933547974 2.0
-4.2527245718 -5.00184726715088 3.0
9.3953390304 9.41952419281006 4.0
18.016407867 18.3906002044678 5.0
27.411746904 36.9876365661621 6.0
27.411746904 31.6319103240967 7.0
38.866617309 38.6117477416992 0.0
22.319478804 21.1366901397705 1.0
16.547138508 18.5790233612061 2.0
-7.9092899105 -9.23157501220703 3.0
8.6378485894 8.73214912414551 4.0
30.228768722 30.2495307922363 5.0
38.866617317 38.8179168701172 6.0
38.866617317 39.3734474182129 7.0
42.3830573535328 39.1804695129395 0.0
20.5951530026372 21.3306922912598 1.0
21.7879043479541 20.3441467285156 2.0
-9.3946766066605 -8.17381954193115 3.0
12.3932277407315 12.4019460678101 4.0
29.98982960914 29.8850975036621 5.0
42.383057354 38.9546279907227 6.0
42.383057354 40.5749549865723 7.0
47.6437510358734 38.3083801269531 0.0
23.1576165238734 23.1414394378662 1.0
24.4861345218735 22.9273834228516 2.0
-8.74385685767346 -7.80543756484985 3.0
15.7422776638735 15.4929323196411 4.0
31.9014733818732 31.598258972168 5.0
47.643751036 38.0872688293457 6.0
47.643751036 43.9125709533691 7.0
38.439399947 37.4916038513184 0.0
19.975660163 19.7153568267822 1.0
18.463739784 18.9899520874023 2.0
-6.4266906816 -6.67466688156128 3.0
12.037049093 12.2286500930786 4.0
26.402350854 26.6145095825195 5.0
38.439399957 36.7280158996582 6.0
38.439399957 38.1363372802734 7.0
40.263562361 38.045482635498 0.0
20.42306719 21.3184661865234 1.0
19.840495171 18.4450702667236 2.0
-11.677045299 -10.0718717575073 3.0
8.1634498621 8.20311260223389 4.0
32.100112499 32.1801643371582 5.0
40.263562371 38.044807434082 6.0
40.263562371 39.3574295043945 7.0
41.519528387 37.5042839050293 0.0
22.877320367 21.155387878418 1.0
18.642208021 20.3848133087158 2.0
-5.2280057864 -6.87348508834839 3.0
13.414202225 13.3601036071777 4.0
28.105326163 27.8187923431396 5.0
41.519528397 37.6002235412598 6.0
41.519528397 40.6245193481445 7.0
40.414570464 37.9433250427246 0.0
17.791289494 19.865571975708 1.0
22.623280971 20.8138084411621 2.0
-6.6442159501 -4.51150989532471 3.0
15.979065012 15.8835477828979 4.0
24.435505453 24.4548778533936 5.0
40.414570474 37.9164009094238 6.0
40.414570474 39.7159996032715 7.0
37.834181215 39.5017852783203 0.0
18.770560656 19.462869644165 1.0
19.063620563 19.0076484680176 2.0
-7.0281211763 -6.64159631729126 3.0
12.035499382 11.9407434463501 4.0
25.798681838 25.9067630767822 5.0
37.834181221 39.0348434448242 6.0
37.834181221 38.1482238769531 7.0
37.201121546 39.1870536804199 0.0
17.811506649 19.022705078125 1.0
19.389614897 19.2139911651611 2.0
-6.0990519868 -5.35982942581177 3.0
13.290562901 13.3291130065918 4.0
23.910558645 24.0227470397949 5.0
37.201121556 38.8553619384766 6.0
37.201121556 37.7886085510254 7.0
42.944445132 39.4572257995605 0.0
21.59677927 21.4524440765381 1.0
21.347665865 21.3313312530518 2.0
-6.9735110037 -6.9572319984436 3.0
14.374154853 14.3313226699829 4.0
28.570290281 28.570291519165 5.0
42.94444514 39.1150207519531 6.0
42.94444514 41.0948867797852 7.0
47.3096000339538 38.265438079834 0.0
24.7818404359387 23.4610900878906 1.0
22.5277596000527 23.3615837097168 2.0
-6.75460229154402 -7.54383993148804 3.0
15.7731573079881 15.6812047958374 4.0
31.5364427279756 31.5587615966797 5.0
47.309600034 38.2595748901367 6.0
47.309600034 43.8409271240234 7.0
34.703660739 38.7271347045898 0.0
16.95243502 17.9484767913818 1.0
17.75122572 18.2813262939453 2.0
-4.712400117 -4.47102975845337 3.0
13.038825595 13.0531692504883 4.0
21.664835145 21.6078281402588 5.0
34.703660747 38.6646690368652 6.0
34.703660747 36.1299095153809 7.0
39.355679823 38.3557243347168 0.0
18.719972615 20.1084289550781 1.0
20.635707207 19.5234413146973 2.0
-8.5340541425 -7.48537015914917 3.0
12.101653055 12.206148147583 4.0
27.254026768 27.671178817749 5.0
39.355679833 38.1757392883301 6.0
39.355679833 39.1593589782715 7.0
36.760838125 37.506965637207 0.0
18.098408908 18.640775680542 1.0
18.662429217 18.9092063903809 2.0
-4.9002916105 -5.02941751480103 3.0
13.762137597 13.7477617263794 4.0
22.998700528 23.104040145874 5.0
36.760838135 37.3901901245117 6.0
36.760838135 37.2106246948242 7.0
45.817767369632 39.7615127563477 0.0
23.4066861136316 24.2815818786621 1.0
22.4110812656318 23.1132621765137 2.0
-7.99255969573223 -8.75945091247559 3.0
14.4185215696319 14.2004833221436 4.0
31.3992456710555 32.5994338989258 5.0
45.817767369 39.4497947692871 6.0
45.817767369 44.2532119750977 7.0
37.456387934 37.737247467041 0.0
18.714257418 19.1406517028809 1.0
18.742130516 18.2788791656494 2.0
-7.2626807055 -7.01530694961548 3.0
11.4794498 11.4433069229126 4.0
25.976938133 26.2366275787354 5.0
37.456387944 37.0947189331055 6.0
37.456387944 36.872444152832 7.0
36.979299104 38.4553413391113 0.0
17.001867984 18.3720970153809 1.0
19.977431121 19.42698097229 2.0
-4.4589079828 -3.83373832702637 3.0
15.518523129 15.3631496429443 4.0
21.460775976 21.4305591583252 5.0
36.979299113 38.3092765808105 6.0
36.979299113 37.2735710144043 7.0
44.357392148 38.4727249145508 0.0
23.57148554 21.8489475250244 1.0
20.785906608 21.4757652282715 2.0
-6.1950554121 -7.26749086380005 3.0
14.590851187 14.5565643310547 4.0
29.766540962 29.5890350341797 5.0
44.357392157 38.4141464233398 6.0
44.357392157 41.5349617004395 7.0
36.8120792993302 38.3463172912598 0.0
19.8640110823302 19.9659557342529 1.0
16.9480682263302 17.6436195373535 2.0
-8.51793333553018 -8.45174503326416 3.0
8.43013489113018 8.55601119995117 4.0
28.381944393825 28.9872570037842 5.0
36.812079299 38.0929679870605 6.0
36.812079299 37.6994094848633 7.0
43.172254734 36.3969192504883 0.0
21.994700089 21.1066112518311 1.0
21.177554646 21.8962478637695 2.0
-4.8220170211 -5.34043407440186 3.0
16.355537616 16.2393436431885 4.0
26.816717119 26.8951416015625 5.0
43.172254743 37.2419090270996 6.0
43.172254743 41.4006843566895 7.0
47.109370558 38.3245162963867 0.0
25.034746172 23.8183498382568 1.0
22.074624387 22.2811031341553 2.0
-8.7769581235 -8.85605621337891 3.0
13.297666255 13.2704277038574 4.0
33.811704304 33.5030364990234 5.0
47.109370567 38.5997467041016 6.0
47.109370567 43.6489028930664 7.0
35.871355653 36.1391105651855 0.0
18.076771786 19.3992118835449 1.0
17.794583868 17.1314334869385 2.0
-9.6137759762 -8.6608304977417 3.0
8.1808078825 8.26788425445557 4.0
27.690547772 27.7614498138428 5.0
35.871355662 36.6153602600098 6.0
35.871355662 36.9288024902344 7.0
42.1694339111434 38.9470367431641 0.0
22.6030080131434 21.7849102020264 1.0
19.5664259071434 20.5925121307373 2.0
-7.61628576964338 -8.42886543273926 3.0
11.9501401371434 11.8825168609619 4.0
30.2192937831435 30.0559196472168 5.0
42.169433911 38.180965423584 6.0
42.169433911 40.9437446594238 7.0
38.001249391 37.9973373413086 0.0
19.442391644 19.6870040893555 1.0
18.558857748 18.4691886901855 2.0
-7.4577746389 -7.60242509841919 3.0
11.101083099 10.9167470932007 4.0
26.900166292 27.0291309356689 5.0
38.001249401 37.8973541259766 6.0
38.001249401 37.9358291625977 7.0
36.026296144 37.834156036377 0.0
18.544058535 19.1647186279297 1.0
17.482237609 17.6753730773926 2.0
-7.0335899837 -6.9518027305603 3.0
10.448647616 10.5112190246582 4.0
25.577648528 25.7897281646729 5.0
36.026296153 37.3978881835938 6.0
36.026296153 36.7877044677734 7.0
33.200213149 38.2262687683105 0.0
14.957525483 17.4856433868408 1.0
18.242687666 16.649564743042 2.0
-8.613844647 -6.40061712265015 3.0
9.6288430093 9.77003288269043 4.0
23.57137014 23.8824157714844 5.0
33.200213159 38.5693969726562 6.0
33.200213159 35.0668182373047 7.0
36.881947947 39.8000144958496 0.0
19.760267083 19.2064819335938 1.0
17.12168087 17.7888221740723 2.0
-6.5510496361 -7.51888227462769 3.0
10.570631229 10.364785194397 4.0
26.311316723 26.4026145935059 5.0
36.881947952 39.1503143310547 6.0
36.881947952 37.0003318786621 7.0
28.363012003 39.1219253540039 0.0
14.617095463 15.8446378707886 1.0
13.745916541 14.4695587158203 2.0
-5.3741086073 -5.28941774368286 3.0
8.3718079243 8.50251007080078 4.0
19.991204079 20.054931640625 5.0
28.363012012 38.8123588562012 6.0
28.363012012 32.2619972229004 7.0
39.574467416 38.4331016540527 0.0
20.649388383 22.8117408752441 1.0
18.925079033 20.9025974273682 2.0
-8.4547238017 -9.2232837677002 3.0
10.470355222 10.4297094345093 4.0
29.104112194 31.7213916778564 5.0
39.574467426 38.2710723876953 6.0
39.574467426 42.0457229614258 7.0
36.220401471 38.5217552185059 0.0
17.995668566 19.0029716491699 1.0
18.224732906 18.2967529296875 2.0
-6.9105415737 -6.45233821868896 3.0
11.314191323 11.3048276901245 4.0
24.906210149 25.0654830932617 5.0
36.22040148 38.1886749267578 6.0
36.22040148 37.3958930969238 7.0
31.086432133 39.1134452819824 0.0
16.0910287 16.6368618011475 1.0
14.995403435 16.5084400177002 2.0
-4.3439743018 -4.76894569396973 3.0
10.651429126 10.7659616470337 4.0
20.43500301 20.7076358795166 5.0
31.086432141 39.1620941162109 6.0
31.086432141 34.1228904724121 7.0
36.274353053 38.1587753295898 0.0
18.420887917 18.227180480957 1.0
17.853465137 18.9322662353516 2.0
-3.6202808126 -4.39880275726318 3.0
14.233184315 14.2248706817627 4.0
22.041168739 22.0639228820801 5.0
36.274353063 37.732795715332 6.0
36.274353063 36.895809173584 7.0
31.65795008 37.3974723815918 0.0
17.489254099 16.7293682098389 1.0
14.168695981 14.8930864334106 2.0
-6.0427493715 -7.42318201065063 3.0
8.1259465996 8.10333061218262 4.0
23.53200348 23.6273975372314 5.0
31.65795009 36.9578399658203 6.0
31.65795009 33.0307197570801 7.0
37.059171469 40.2969055175781 0.0
19.339994356 19.6622295379639 1.0
17.719177113 17.264289855957 2.0
-9.4011683011 -9.12735080718994 3.0
8.318008802 8.29968452453613 4.0
28.741162667 29.0508136749268 5.0
37.059171479 39.3399505615234 6.0
37.059171479 37.0677528381348 7.0
45.308862348 38.3506889343262 0.0
24.95721568 22.4607753753662 1.0
20.351646669 21.7557640075684 2.0
-7.323343681 -8.3251428604126 3.0
13.028302978 13.0641050338745 4.0
32.28055937 31.9004364013672 5.0
45.308862357 38.0416564941406 6.0
45.308862357 42.3728561401367 7.0
32.988279899 39.0228614807129 0.0
17.965429309 17.2364673614502 1.0
15.02285059 16.9845314025879 2.0
-2.6394240327 -4.33430576324463 3.0
12.383426547 12.4037923812866 4.0
20.604853352 20.7507858276367 5.0
32.988279909 38.6048202514648 6.0
32.988279909 34.9824638366699 7.0
41.812200584 37.9049949645996 0.0
21.127130287 20.6354084014893 1.0
20.685070296 21.1698837280273 2.0
-5.5457377677 -5.76394939422607 3.0
15.139332519 15.1024026870728 4.0
26.672868065 26.6041221618652 5.0
41.812200593 37.2795524597168 6.0
41.812200593 40.5468635559082 7.0
34.159577997 37.9061012268066 0.0
15.979883142 17.7400856018066 1.0
18.179694855 17.1720199584961 2.0
-6.0058060763 -4.72562742233276 3.0
12.173888768 12.1953344345093 4.0
21.985689228 21.9894180297852 5.0
34.159578007 37.5464248657227 6.0
34.159578007 35.3343238830566 7.0
41.353058194 38.2025909423828 0.0
20.665722502 21.3835315704346 1.0
20.687335692 20.1754627227783 2.0
-9.525258586 -8.8731164932251 3.0
11.162077096 11.1456699371338 4.0
30.190981098 30.2903385162354 5.0
41.353058204 38.3067436218262 6.0
41.353058204 40.1993560791016 7.0
35.663236634 38.9613342285156 0.0
18.766917562 18.7335109710693 1.0
16.896319073 17.0505180358887 2.0
-7.670552873 -7.69027376174927 3.0
9.2257661902 9.16448020935059 4.0
26.437470445 26.2822685241699 5.0
35.663236644 38.6389656066895 6.0
35.663236644 36.1357612609863 7.0
40.444916113 38.5851821899414 0.0
18.421748417 20.3712253570557 1.0
22.023167696 19.5753860473633 2.0
-9.937900394 -7.78186178207397 3.0
12.085267292 11.9894781112671 4.0
28.359648821 28.1582260131836 5.0
40.444916123 38.2193603515625 6.0
40.444916123 38.9356880187988 7.0
39.900416904 38.339225769043 0.0
21.000367331 20.8128719329834 1.0
18.900049573 19.7326850891113 2.0
-6.1981317567 -6.85677194595337 3.0
12.701917806 12.7586240768433 4.0
27.198499098 28.5842571258545 5.0
39.900416914 38.541576385498 6.0
39.900416914 39.5500373840332 7.0
34.40532484 40.0965919494629 0.0
18.680133752 18.8859977722168 1.0
15.725191089 17.118221282959 2.0
-6.5675447079 -7.43034219741821 3.0
9.157646372 9.11384391784668 4.0
25.247678469 25.3310470581055 5.0
34.405324849 39.9153175354004 6.0
34.405324849 36.1260604858398 7.0
33.074255995 37.7230491638184 0.0
16.81819884 17.4982872009277 1.0
16.256057155 16.6835250854492 2.0
-5.8048358004 -5.8877739906311 3.0
10.451221345 10.4451055526733 4.0
22.62303465 22.790412902832 5.0
33.074256004 38.1820373535156 6.0
33.074256004 35.1139717102051 7.0
40.036170298 37.8572654724121 0.0
18.427862329 19.9349575042725 1.0
21.60830797 20.3506088256836 2.0
-6.8170168406 -5.54568243026733 3.0
14.791291119 14.7280750274658 4.0
25.244879179 25.5118217468262 5.0
40.036170308 37.3348388671875 6.0
40.036170308 39.253044128418 7.0
44.453206231 39.3662605285645 0.0
21.51525667 22.3724098205566 1.0
22.937949561 22.7719058990479 2.0
-6.654859187 -5.72617721557617 3.0
16.283090365 16.1743602752686 4.0
28.170115867 28.4922256469727 5.0
44.453206241 39.2769737243652 6.0
44.453206241 43.0232200622559 7.0
33.8500454 38.2779655456543 0.0
15.317438725 17.1191787719727 1.0
18.532606675 18.3790493011475 2.0
-3.771729854 -3.20301723480225 3.0
14.760876811 14.8654623031616 4.0
19.089168589 19.3875637054443 5.0
33.85004541 37.9120826721191 6.0
33.85004541 35.5749015808105 7.0
30.537877299 37.8488311767578 0.0
15.743236653 16.6234226226807 1.0
14.794640646 15.1962947845459 2.0
-6.2614673178 -6.23832082748413 3.0
8.5331733191 8.48759269714355 4.0
22.00470398 22.0531063079834 5.0
30.537877308 37.7914619445801 6.0
30.537877308 33.0207977294922 7.0
27.727440934 37.6216011047363 0.0
13.298433958 15.1315011978149 1.0
14.429006978 14.6653671264648 2.0
-4.8732361991 -4.28905010223389 3.0
9.5557707702 9.7619104385376 4.0
18.171670165 18.5045604705811 5.0
27.727440942 37.1800384521484 6.0
27.727440942 31.4279823303223 7.0
35.630182891 38.3329277038574 0.0
17.416160369 18.3649158477783 1.0
18.214022525 18.601619720459 2.0
-5.5362398817 -5.13091468811035 3.0
12.677782636 12.7241449356079 4.0
22.952400258 23.0846748352051 5.0
35.630182897 38.0397872924805 6.0
35.630182897 36.9646224975586 7.0
31.038251509 36.9757652282715 0.0
16.216885208 16.952278137207 1.0
14.821366301 15.8470649719238 2.0
-5.1281710605 -5.32798194885254 3.0
9.6931952301 9.84314346313477 4.0
21.345056279 21.4381351470947 5.0
31.038251519 36.9899101257324 6.0
31.038251519 34.1666870117188 7.0
31.867011958 38.7021217346191 0.0
14.888086344 16.2196197509766 1.0
16.978925614 16.0588855743408 2.0
-5.7710515718 -4.62554502487183 3.0
11.207874033 11.4314613342285 4.0
20.659137926 20.721134185791 5.0
31.867011967 38.455379486084 6.0
31.867011967 33.3669853210449 7.0
30.754831636 38.3041725158691 0.0
14.067592922 15.9600868225098 1.0
16.687238713 16.6951732635498 2.0
-4.2836008115 -3.62969493865967 3.0
12.403637892 12.4398860931396 4.0
18.351193744 18.8905506134033 5.0
30.754831646 37.7824630737305 6.0
30.754831646 33.3293609619141 7.0
38.951564273 38.9221000671387 0.0
20.789904648 20.4890651702881 1.0
18.161659627 18.036548614502 2.0
-9.14497609 -8.85071659088135 3.0
9.0166835299 9.09578227996826 4.0
29.934880745 29.6688098907471 5.0
38.95156428 38.8319511413574 6.0
38.95156428 38.2851409912109 7.0
33.384480929 37.9672470092773 0.0
14.280681926 17.1481494903564 1.0
19.103799006 17.4260501861572 2.0
-6.8894608583 -4.66559171676636 3.0
12.214338142 12.1998014450073 4.0
21.170142791 21.1825637817383 5.0
33.384480936 37.8986511230469 6.0
33.384480936 34.808521270752 7.0
37.291199278 39.3691215515137 0.0
18.976122996 17.7962970733643 1.0
18.315076287 18.4106788635254 2.0
-3.6314782733 -3.84028816223145 3.0
14.68359801 14.6078224182129 4.0
22.607601274 21.9764194488525 5.0
37.291199282 38.5237426757812 6.0
37.291199282 36.0630683898926 7.0
33.5718110164568 38.8134803771973 0.0
16.2455196066318 17.4075813293457 1.0
17.3262914102844 16.9650688171387 2.0
-6.34124959778268 -5.72014236450195 3.0
10.9850418123699 10.9390592575073 4.0
22.5867692042782 22.7267665863037 5.0
33.571811016 38.651065826416 6.0
33.571811016 35.2567138671875 7.0
44.048761737 38.2008895874023 0.0
22.194046458 21.8874683380127 1.0
21.85471528 22.1075420379639 2.0
-6.0219898767 -6.244553565979 3.0
15.832725394 15.7312002182007 4.0
28.216036344 28.3096771240234 5.0
44.048761746 37.8605003356934 6.0
44.048761746 41.7067680358887 7.0
38.092815451 37.8056907653809 0.0
20.42584869 19.7785148620605 1.0
17.666966762 19.2065086364746 2.0
-5.2714738638 -6.39794921875 3.0
12.395492889 12.3522872924805 4.0
25.697322563 25.8050994873047 5.0
38.09281546 37.8364639282227 6.0
38.09281546 38.4982719421387 7.0
42.616606914 39.3548049926758 0.0
22.043217204 21.752613067627 1.0
20.57338971 20.5958099365234 2.0
-9.1142862064 -8.33427429199219 3.0
11.459103494 11.5397100448608 4.0
31.157503421 30.5504360198975 5.0
42.616606924 38.5783843994141 6.0
42.616606924 40.9488677978516 7.0
41.080227066 39.4208221435547 0.0
21.412108977 21.1633720397949 1.0
19.66811809 19.6458835601807 2.0
-8.9308117628 -8.77218341827393 3.0
10.737306318 10.7332286834717 4.0
30.342920748 30.4963665008545 5.0
41.080227074 39.0475959777832 6.0
41.080227074 40.0405654907227 7.0
36.859087804 38.5008506774902 0.0
16.308519924 18.9337844848633 1.0
20.550567882 18.7108459472656 2.0
-8.0027710471 -5.88641977310181 3.0
12.547796826 12.5460395812988 4.0
24.31129098 24.4554481506348 5.0
36.859087813 37.8426208496094 6.0
36.859087813 37.348274230957 7.0
39.316268866 38.4526557922363 0.0
19.479900065 20.0125102996826 1.0
19.836368801 19.4864063262939 2.0
-7.370707565 -7.02618741989136 3.0
12.465661226 12.4006481170654 4.0
26.85060764 27.018009185791 5.0
39.316268876 38.4067573547363 6.0
39.316268876 38.9094161987305 7.0
40.023071957 38.4276542663574 0.0
21.374119573 21.3113040924072 1.0
18.648952384 19.5782604217529 2.0
-8.5239396023 -8.73092555999756 3.0
10.125012772 10.3349542617798 4.0
29.898059185 30.1562576293945 5.0
40.023071967 38.5255165100098 6.0
40.023071967 40.1709480285645 7.0
32.691039556 38.9717903137207 0.0
16.99402987 17.4871578216553 1.0
15.697009686 16.3900203704834 2.0
-6.1678662484 -6.07196140289307 3.0
9.5291434282 9.67775726318359 4.0
23.161896128 23.2176322937012 5.0
32.691039566 38.633358001709 6.0
32.691039566 34.7509994506836 7.0
39.350172222 38.6460380554199 0.0
19.79571531 20.2547664642334 1.0
19.554456913 19.8143711090088 2.0
-6.9332448751 -6.87013483047485 3.0
12.62121203 12.4979839324951 4.0
26.728960194 26.7693386077881 5.0
39.35017223 38.2516288757324 6.0
39.35017223 39.1650619506836 7.0
38.6257792205169 38.4796257019043 0.0
20.6764320695262 19.9304809570312 1.0
17.9493471475253 17.9891796112061 2.0
-8.74948254057215 -8.63763618469238 3.0
9.19986460693105 9.25247764587402 4.0
29.4259146106884 29.3816986083984 5.0
38.62577922 38.2519950866699 6.0
38.62577922 37.5677795410156 7.0
43.363210173 38.331844329834 0.0
22.264479357 22.0126209259033 1.0
21.09873082 21.5122699737549 2.0
-7.8566163895 -7.92946290969849 3.0
13.242114424 13.1891174316406 4.0
30.121095753 30.1695938110352 5.0
43.36321018 38.181697845459 6.0
43.36321018 41.6432609558105 7.0
41.7113705187954 38.492000579834 0.0
19.9560794530664 20.8143310546875 1.0
21.7552910618007 21.1297817230225 2.0
-6.27295647675057 -5.38473606109619 3.0
15.4823345851289 15.4585943222046 4.0
26.2290359296866 26.262523651123 5.0
41.711370519 38.5355796813965 6.0
41.711370519 40.411994934082 7.0
41.4017485851654 38.4023933410645 0.0
21.9191533447582 20.7927055358887 1.0
19.4825952431145 19.5519771575928 2.0
-7.53340700226978 -8.09284496307373 3.0
11.9491882404739 11.7435369491577 4.0
29.4525603463383 29.2129650115967 5.0
41.401748585 38.0895957946777 6.0
41.401748585 39.6945724487305 7.0
45.411303660968 37.900806427002 0.0
22.8793399812672 23.0925197601318 1.0
22.5319636798771 22.1451930999756 2.0
-8.35657341793646 -7.9482741355896 3.0
14.1753902620561 13.895414352417 4.0
31.2359133986188 30.9234485626221 5.0
45.411303661 37.9091148376465 6.0
45.411303661 43.2660865783691 7.0
46.173622728 38.9038887023926 0.0
22.478207281 22.3913154602051 1.0
23.695415447 22.5334300994873 2.0
-7.2335265538 -6.46610116958618 3.0
16.461888883 16.3160305023193 4.0
29.711733845 29.6540660858154 5.0
46.173622738 38.4396858215332 6.0
46.173622738 42.9612846374512 7.0
43.970500883 38.5951766967773 0.0
23.319439955 22.50270652771 1.0
20.651060929 20.5110015869141 2.0
-9.6037663307 -9.2838830947876 3.0
11.047294589 11.0010118484497 4.0
32.923206295 32.4175186157227 5.0
43.970500892 38.4553871154785 6.0
43.970500892 41.82177734375 7.0
41.561964514 39.0163803100586 0.0
22.423587267 20.1328048706055 1.0
19.138377247 20.8865184783936 2.0
-2.7143772539 -4.84830856323242 3.0
16.423999984 16.1324405670166 4.0
25.13796453 25.0001106262207 5.0
41.561964524 38.3462409973145 6.0
41.561964524 39.6517333984375 7.0
42.043570825 38.2575302124023 0.0
21.184612309 21.1398735046387 1.0
20.858958517 20.9147262573242 2.0
-7.1973175688 -7.081139087677 3.0
13.66164094 13.7134799957275 4.0
28.381929887 28.630973815918 5.0
42.043570834 37.9104347229004 6.0
42.043570834 40.7702941894531 7.0
42.728187899 39.3955192565918 0.0
22.937429443 21.8731517791748 1.0
19.790758457 20.5174007415771 2.0
-7.3221221307 -8.2036247253418 3.0
12.468636317 12.5039081573486 4.0
30.259551583 30.652042388916 5.0
42.728187908 38.989143371582 6.0
42.728187908 41.0130577087402 7.0
46.647822935 36.1177825927734 0.0
24.419972425 23.8842372894287 1.0
22.22785051 22.0999279022217 2.0
-9.7005341836 -9.20944690704346 3.0
12.527316316 12.5433397293091 4.0
34.120506619 34.1122817993164 5.0
46.647822945 36.1165199279785 6.0
46.647822945 43.371711730957 7.0
34.157298452 39.0583457946777 0.0
16.904613725 18.264087677002 1.0
17.252684728 17.2979602813721 2.0
-7.2828539724 -6.47605800628662 3.0
9.9698307451 10.1908960342407 4.0
24.187467707 24.5378665924072 5.0
34.157298462 39.0790748596191 6.0
34.157298462 36.0433464050293 7.0
38.08940785 38.9064445495605 0.0
20.151317931 18.890682220459 1.0
17.93808992 19.8142032623291 2.0
-2.5516772312 -3.8943190574646 3.0
15.386412679 15.4161424636841 4.0
22.702995171 22.8279647827148 5.0
38.08940786 38.388053894043 6.0
38.08940786 37.7570495605469 7.0
48.253456595 39.7466049194336 0.0
26.02387816 24.5818042755127 1.0
22.229578435 23.3479824066162 2.0
-8.9125381923 -9.63110065460205 3.0
13.317040233 13.182201385498 4.0
34.936416362 34.3858337402344 5.0
48.253456605 39.1636276245117 6.0
48.253456605 44.9791870117188 7.0
41.303512945 37.8351287841797 0.0
21.842523524 21.3165035247803 1.0
19.460989421 19.5027008056641 2.0
-8.4785475097 -8.54372024536133 3.0
10.982441901 11.0276985168457 4.0
30.321071043 30.3419189453125 5.0
41.303512955 37.5850410461426 6.0
41.303512955 39.8128929138184 7.0
34.11247551 38.2778854370117 0.0
16.407323353 17.0558700561523 1.0
17.705152157 17.7634963989258 2.0
-3.4961419723 -3.50868511199951 3.0
14.209010175 14.2257833480835 4.0
19.903465335 19.9693183898926 5.0
34.11247552 38.0238151550293 6.0
34.11247552 35.1724319458008 7.0
39.346899311 38.8418846130371 0.0
20.280952021 20.4650230407715 1.0
19.06594729 18.2519931793213 2.0
-10.212334112 -9.54252624511719 3.0
8.8536131687 8.93718910217285 4.0
30.493286142 30.6021213531494 5.0
39.346899321 38.7209014892578 6.0
39.346899321 38.2946014404297 7.0
42.355081312 38.2149353027344 0.0
22.237080125 21.6500663757324 1.0
20.118001187 21.5789527893066 2.0
-4.2250518369 -5.86720561981201 3.0
15.892949341 15.7463598251343 4.0
26.462131971 27.5548362731934 5.0
42.355081321 37.8650093078613 6.0
42.355081321 40.9994506835938 7.0
46.02528461 38.1543426513672 0.0
20.767999035 22.4715404510498 1.0
25.257285575 22.5083560943604 2.0
-9.1629445671 -6.82663869857788 3.0
16.094340999 15.9072780609131 4.0
29.930943612 29.9363441467285 5.0
46.025284619 37.9036674499512 6.0
46.025284619 42.7907257080078 7.0
48.790909777 38.3799858093262 0.0
25.145056622 24.565523147583 1.0
23.645853155 23.6248645782471 2.0
-9.2179915631 -8.74100685119629 3.0
14.427861582 14.2924747467041 4.0
34.363048195 33.8121604919434 5.0
48.790909787 38.231761932373 6.0
48.790909787 44.8305053710938 7.0
41.698940493 38.3337364196777 0.0
20.507660886 21.3365058898926 1.0
21.191279609 20.2005290985107 2.0
-9.1543537316 -8.15841484069824 3.0
12.036925868 11.9397983551025 4.0
29.662014626 29.6006832122803 5.0
41.698940502 38.6391830444336 6.0
41.698940502 40.2691268920898 7.0
47.902972774 38.3395004272461 0.0
23.393159022 23.7867240905762 1.0
24.509813754 23.6689224243164 2.0
-8.5476809737 -7.65032434463501 3.0
15.962132774 15.6468839645386 4.0
31.940840003 32.1224250793457 5.0
47.902972781 38.0790710449219 6.0
47.902972781 44.2539825439453 7.0
41.377170862 38.5026092529297 0.0
20.577519401 20.6898593902588 1.0
20.799651462 20.8835391998291 2.0
-6.2821130824 -6.11445331573486 3.0
14.517538372 14.3948059082031 4.0
26.859632492 26.9460678100586 5.0
41.37717087 38.5189819335938 6.0
41.37717087 40.0224914550781 7.0
32.285684892 38.6919708251953 0.0
14.316873201 16.6548652648926 1.0
17.968811691 17.0451602935791 2.0
-5.3227711546 -3.91176891326904 3.0
12.646040526 12.7483263015747 4.0
19.639644366 19.8384857177734 5.0
32.285684902 38.6419792175293 6.0
32.285684902 34.2981719970703 7.0
32.344206843 38.2900505065918 0.0
16.160894185 17.5494174957275 1.0
16.183312659 16.2927093505859 2.0
-7.3310896054 -6.79985570907593 3.0
8.852223045 9.0007848739624 4.0
23.491983799 23.6876468658447 5.0
32.344206851 38.2080459594727 6.0
32.344206851 35.1000061035156 7.0
34.184112321 38.2591438293457 0.0
18.400062549 18.517240524292 1.0
15.784049772 16.5578651428223 2.0
-7.6583985283 -7.84811735153198 3.0
8.125651234 8.25148487091064 4.0
26.058461087 26.1528873443604 5.0
34.184112331 37.9949722290039 6.0
34.184112331 36.001277923584 7.0
34.663221798 39.7613983154297 0.0
18.206197344 18.4781150817871 1.0
16.457024454 17.2514896392822 2.0
-6.9582970888 -7.38293886184692 3.0
9.4987273548 9.62739181518555 4.0
25.164494443 25.5369186401367 5.0
34.663221808 39.0562591552734 6.0
34.663221808 35.9800186157227 7.0
36.873774533 37.7823791503906 0.0
18.579406772 18.5500831604004 1.0
18.294367761 18.7208843231201 2.0
-5.3733020341 -5.44784212112427 3.0
12.921065717 12.9975357055664 4.0
23.952708817 24.1360092163086 5.0
36.873774543 36.9180526733398 6.0
36.873774543 36.534839630127 7.0
40.211352898 38.3067932128906 0.0
22.061687235 22.1006622314453 1.0
18.149665663 19.881383895874 2.0
-7.4872031869 -8.86136245727539 3.0
10.662462467 10.4395351409912 4.0
29.548890431 31.0889053344727 5.0
40.211352908 37.9759407043457 6.0
40.211352908 41.0227279663086 7.0
46.160689864 38.4829216003418 0.0
24.176307096 22.913537979126 1.0
21.984382774 23.0707225799561 2.0
-5.8472747438 -6.77477502822876 3.0
16.137108027 15.9860591888428 4.0
30.023581843 29.8622074127197 5.0
46.160689867 38.1598777770996 6.0
46.160689867 43.194637298584 7.0
32.727702668 39.220760345459 0.0
17.796794822 17.620325088501 1.0
14.930907847 16.2459030151367 2.0
-5.7416146893 -6.78174829483032 3.0
9.1892931492 9.08954906463623 4.0
23.53840952 23.9127254486084 5.0
32.727702676 38.8151741027832 6.0
32.727702676 34.2955589294434 7.0
40.640985417073 37.8235778808594 0.0
21.2248663380953 20.2969245910645 1.0
19.4161190831271 20.0406818389893 2.0
-6.12795167285201 -6.99463129043579 3.0
13.2881674100867 13.3721771240234 4.0
27.3528180102152 27.3285121917725 5.0
40.640985417 37.3972320556641 6.0
40.640985417 39.1115036010742 7.0
34.611451584 38.2627410888672 0.0
16.03184427 17.5202369689941 1.0
18.579607315 18.2030448913574 2.0
-5.6371786801 -4.93142366409302 3.0
12.942428625 12.9725513458252 4.0
21.66902296 21.9798030853271 5.0
34.611451594 38.0885467529297 6.0
34.611451594 35.8022689819336 7.0
43.311280798 39.5316696166992 0.0
19.747206987 20.7522964477539 1.0
23.564073813 21.3958969116211 2.0
-7.5093463576 -5.71678972244263 3.0
16.054727447 15.7949066162109 4.0
27.256553353 27.3104763031006 5.0
43.311280807 38.8658065795898 6.0
43.311280807 40.4640350341797 7.0
36.406364048 38.081615447998 0.0
19.818637935 18.7061958312988 1.0
16.587726114 18.5058536529541 2.0
-3.8896932395 -5.29627513885498 3.0
12.698032865 12.7912406921387 4.0
23.708331184 23.8667964935303 5.0
36.406364057 37.6333160400391 6.0
36.406364057 37.2790756225586 7.0
31.68563032 39.6092796325684 0.0
16.740289757 16.9900989532471 1.0
14.945340567 15.8076047897339 2.0
-6.0451472911 -6.54139375686646 3.0
8.9001932697 8.96568202972412 4.0
22.785437054 23.1350288391113 5.0
31.685630326 38.7310981750488 6.0
31.685630326 34.0025901794434 7.0
40.165364423 38.580322265625 0.0
21.225216477 20.9404487609863 1.0
18.940147947 18.8936939239502 2.0
-9.5161355118 -9.19925117492676 3.0
9.4240124269 9.39347839355469 4.0
30.741351997 31.0670490264893 5.0
40.165364432 37.7935218811035 6.0
40.165364432 39.0313529968262 7.0
42.886465637 38.3235359191895 0.0
31.633244398 22.0049629211426 1.0
11.253221239 20.4712867736816 2.0
0 -9.22647857666016 3.0
11.253221239 11.2555828094482 4.0
31.633244398 31.7547283172607 5.0
42.886465647 38.427661895752 6.0
42.886465647 41.1830558776855 7.0
46.942592322 38.0785293579102 0.0
23.694673267 23.5310745239258 1.0
23.247919055 22.1178245544434 2.0
-9.8881851606 -9.2271146774292 3.0
13.359733884 13.2550611495972 4.0
33.582858438 33.6203422546387 5.0
46.942592332 37.9169692993164 6.0
46.942592332 43.4059791564941 7.0
38.417030558 37.7453765869141 0.0
20.711241226 19.6721496582031 1.0
17.705789332 17.7730751037598 2.0
-7.7084466149 -8.1473217010498 3.0
9.9973427073 9.85224437713623 4.0
28.419687851 28.5753421783447 5.0
38.417030568 37.7708549499512 6.0
38.417030568 37.4430732727051 7.0
39.82159542 39.4618453979492 0.0
19.816428679 19.7194309234619 1.0
20.005166741 20.1412906646729 2.0
-5.1739908699 -5.43543148040771 3.0
14.831175861 14.8720455169678 4.0
24.990419559 25.2127742767334 5.0
39.82159543 38.7720108032227 6.0
39.82159543 38.7316818237305 7.0
42.375846264 37.9908447265625 0.0
20.538210367 20.8847198486328 1.0
21.837635902 21.8447952270508 2.0
-5.3545430676 -4.96765613555908 3.0
16.483092829 16.3122215270996 4.0
25.89275344 25.7551174163818 5.0
42.375846269 37.7852783203125 6.0
42.375846269 40.814338684082 7.0
41.909868914 38.7760963439941 0.0
21.249762031 21.7039051055908 1.0
20.660106883 19.2264232635498 2.0
-10.971309543 -9.84107398986816 3.0
9.6887973304 9.45884227752686 4.0
32.221071584 32.3416748046875 5.0
41.909868924 37.9846382141113 6.0
41.909868924 39.7842788696289 7.0
38.5154038135067 37.0633392333984 0.0
19.3171244035071 19.9166698455811 1.0
19.1982794195071 18.4847106933594 2.0
-8.28760453680882 -7.56388998031616 3.0
10.9106748825063 10.97434425354 4.0
27.6047289405071 27.9936294555664 5.0
38.515403814 37.1728668212891 6.0
38.515403814 38.3562850952148 7.0
32.187464764 37.2003707885742 0.0
15.688410189 16.5693244934082 1.0
16.499054576 16.353816986084 2.0
-5.8055743047 -5.20375919342041 3.0
10.693480262 10.8113737106323 4.0
21.493984503 21.6506900787354 5.0
32.187464773 36.7830543518066 6.0
32.187464773 33.755069732666 7.0
43.5533404982843 38.5904922485352 0.0
25.257472916292 22.945463180542 1.0
18.2958675832873 20.253438949585 2.0
-8.52565235008034 -9.53647232055664 3.0
9.77021523347734 9.77159690856934 4.0
33.7831252662979 33.2633628845215 5.0
43.553340498 38.7698822021484 6.0
43.553340498 42.0539360046387 7.0
38.661685108 38.7784614562988 0.0
21.023698145 21.5666103363037 1.0
17.637986964 20.0254650115967 2.0
-6.4647611773 -7.82008981704712 3.0
11.173225776 11.225682258606 4.0
27.488459332 29.4805145263672 5.0
38.661685118 38.7132415771484 6.0
38.661685118 40.6800689697266 7.0
47.185976942 38.1790504455566 0.0
20.88210062 22.8496646881104 1.0
26.303876326 23.3071422576904 2.0
-9.8059530494 -7.25625276565552 3.0
16.497923272 16.2354717254639 4.0
30.688053675 30.831090927124 5.0
47.185976948 37.6274604797363 6.0
47.185976948 43.4040832519531 7.0
41.0736498246764 39.7575569152832 0.0
17.9299575176762 19.7567939758301 1.0
23.1436923166761 20.3189125061035 2.0
-7.36188695227673 -5.27147197723389 3.0
15.7818053646765 15.6511077880859 4.0
25.2918444696756 25.1916675567627 5.0
41.073649825 39.1447143554688 6.0
41.073649825 38.9504127502441 7.0
39.086114836 38.4388236999512 0.0
22.054277289 20.9816703796387 1.0
17.031837547 18.4156074523926 2.0
-8.5092532605 -9.34933757781982 3.0
8.5225842768 8.65257263183594 4.0
30.563530559 30.7802848815918 5.0
39.086114846 38.2088165283203 6.0
39.086114846 38.6474571228027 7.0
38.072309142 38.7093811035156 0.0
19.059692214 19.2102489471436 1.0
19.012616928 19.5619335174561 2.0
-4.5092621515 -4.90857887268066 3.0
14.503354768 14.4329710006714 4.0
23.568954375 23.6224174499512 5.0
38.072309151 38.3668060302734 6.0
38.072309151 38.0536270141602 7.0
41.585402364 38.671875 0.0
21.878346027 21.5596027374268 1.0
19.707056337 20.2283916473389 2.0
-7.856520423 -8.20332813262939 3.0
11.850535904 11.8041658401489 4.0
29.73486646 29.8865242004395 5.0
41.585402374 38.5574913024902 6.0
41.585402374 40.4260215759277 7.0
42.3173178092696 39.2003211975098 0.0
23.2468941623901 22.4106006622314 1.0
19.0704236443393 19.6497783660889 2.0
-10.4383552532401 -10.0983200073242 3.0
8.63206839120634 8.88124656677246 4.0
33.6852494163203 33.5251350402832 5.0
42.317317809 38.5805435180664 6.0
42.317317809 40.4654769897461 7.0
42.931477575 38.6651496887207 0.0
22.005453055 21.1441173553467 1.0
20.926024519 21.2468852996826 2.0
-5.8784626661 -6.37744951248169 3.0
15.047561843 15.0249242782593 4.0
27.883915732 27.9067726135254 5.0
42.931477585 38.6978416442871 6.0
42.931477585 40.917179107666 7.0
46.096274403 38.9670715332031 0.0
25.32088965 22.9606437683105 1.0
20.775384755 22.5176906585693 2.0
-5.4662065709 -7.25039148330688 3.0
15.309178177 15.3176670074463 4.0
30.787096229 30.7462635040283 5.0
46.096274411 39.0156402587891 6.0
46.096274411 43.2713394165039 7.0
39.955261544 39.2081871032715 0.0
19.931762754 20.6051769256592 1.0
20.02349879 19.2760391235352 2.0
-8.9110310329 -8.05020523071289 3.0
11.112467747 11.2224378585815 4.0
28.842793797 29.0741901397705 5.0
39.955261554 38.7306327819824 6.0
39.955261554 39.167854309082 7.0
45.415961836 38.1488189697266 0.0
23.097461473 22.5320873260498 1.0
22.318500363 21.3838653564453 2.0
-8.5889928185 -8.62192058563232 3.0
13.729507535 13.4086790084839 4.0
31.686454301 31.4764919281006 5.0
45.415961846 38.3332252502441 6.0
45.415961846 42.1306228637695 7.0
44.275746647 39.0462417602539 0.0
22.654009807 22.4427623748779 1.0
21.62173684 21.8679695129395 2.0
-7.7547516702 -7.91963338851929 3.0
13.86698516 13.7218952178955 4.0
30.408761488 30.2719440460205 5.0
44.275746657 38.9389953613281 6.0
44.275746657 41.9607086181641 7.0
34.766238956 38.2070732116699 0.0
17.943458551 18.2756175994873 1.0
16.822780406 17.7036514282227 2.0
-5.6356736314 -6.07906579971313 3.0
11.187106764 11.2660207748413 4.0
23.579132192 23.75364112854 5.0
34.766238966 37.9689331054688 6.0
34.766238966 36.201286315918 7.0
39.697519492335 38.0046157836914 0.0
21.2162061883195 20.1794452667236 1.0
18.481313305488 19.8342056274414 2.0
-4.67528909066386 -5.95824527740479 3.0
13.8060242148741 13.7891788482666 4.0
25.8914952793298 25.9905738830566 5.0
39.697519492 38.1888656616211 6.0
39.697519492 39.2251167297363 7.0
36.4599525996002 38.825122833252 0.0
18.3065166475976 18.1653957366943 1.0
18.1534359526018 18.1191844940186 2.0
-5.15104061002709 -5.3004002571106 3.0
13.0023953426082 13.0544738769531 4.0
23.457557257596 23.4793033599854 5.0
36.459952599 38.2801170349121 6.0
36.459952599 36.6482162475586 7.0
44.856515577 38.0501403808594 0.0
22.473710733 21.8676338195801 1.0
22.382804843 21.3630294799805 2.0
-8.0040806232 -7.65929460525513 3.0
14.37872421 14.3311719894409 4.0
30.477791367 30.2109889984131 5.0
44.856515586 37.8568000793457 6.0
44.856515586 41.4204750061035 7.0
41.1579139562054 37.6138801574707 0.0
18.5564288232053 20.3360137939453 1.0
22.601485143205 20.8913230895996 2.0
-7.31439675300561 -5.53146505355835 3.0
15.2870883902055 15.2956819534302 4.0
25.8708255762049 25.9708251953125 5.0
41.157913956 37.835391998291 6.0
41.157913956 39.6958045959473 7.0
32.092085782 38.6101531982422 0.0
17.272160967 17.2664966583252 1.0
14.819924815 15.5446949005127 2.0
-6.6255124723 -7.0962233543396 3.0
8.1944123328 8.23289203643799 4.0
23.897673449 24.0657196044922 5.0
32.092085792 38.0145263671875 6.0
32.092085792 33.811897277832 7.0
30.327518415 38.0236701965332 0.0
13.541286609 16.0234603881836 1.0
16.786231807 16.1957130432129 2.0
-5.5133467024 -4.27234125137329 3.0
11.272885094 11.2488842010498 4.0
19.054633321 19.2343254089355 5.0
30.327518425 38.2287635803223 6.0
30.327518425 33.4151954650879 7.0
36.482977268 38.5723991394043 0.0
19.553260998 19.435131072998 1.0
16.929716271 17.7375602722168 2.0
-7.3162847454 -7.78918218612671 3.0
9.613431516 9.49156379699707 4.0
26.869545753 26.8538303375244 5.0
36.482977278 38.3571624755859 6.0
36.482977278 37.5654602050781 7.0
40.491170904 37.7951469421387 0.0
20.509684159 20.928861618042 1.0
19.981486745 19.4188385009766 2.0
-8.9848735316 -8.27414321899414 3.0
10.996613204 10.7758769989014 4.0
29.4945577 29.521240234375 5.0
40.491170914 37.9939041137695 6.0
40.491170914 39.7558059692383 7.0
41.188898324 37.5842971801758 0.0
17.482572735 20.1373691558838 1.0
23.70632559 21.0679702758789 2.0
-7.86829602 -5.21888446807861 3.0
15.838029561 15.720272064209 4.0
25.350868764 25.5014457702637 5.0
41.188898334 37.2636032104492 6.0
41.188898334 39.8098258972168 7.0
37.756949894 38.5852699279785 0.0
19.043754327 19.0069675445557 1.0
18.713195567 19.0543231964111 2.0
-5.5841962186 -5.73911094665527 3.0
13.128999338 13.0972671508789 4.0
24.627950556 24.7212047576904 5.0
37.756949904 38.5318603515625 6.0
37.756949904 37.7819633483887 7.0
26.962645791 38.7544555664062 0.0
14.231795944 14.9186735153198 1.0
12.730849849 13.966757774353 2.0
-4.3614404899 -4.63944721221924 3.0
8.3694093501 8.94529724121094 4.0
18.593236442 18.9010906219482 5.0
26.9626458 38.427433013916 6.0
26.9626458 31.0080108642578 7.0
44.1546808660895 38.7651214599609 0.0
20.845479755763 22.6547164916992 1.0
23.3092011082446 23.2143135070801 2.0
-6.97221748920505 -6.23361110687256 3.0
16.3369836194072 16.0898494720459 4.0
27.8176971034536 29.0703105926514 5.0
44.154680866 38.3661727905273 6.0
44.154680866 42.9674491882324 7.0
36.4992036 37.8188171386719 0.0
18.499566849 19.1346092224121 1.0
17.999636751 17.337121963501 2.0
-8.164709678 -7.83892965316772 3.0
9.834927063 9.7706127166748 4.0
26.664276537 27.0530452728271 5.0
36.49920361 37.5981636047363 6.0
36.49920361 36.6617050170898 7.0
36.136186998 37.7969665527344 0.0
19.096538919 19.4380626678467 1.0
17.039648079 17.3502979278564 2.0
-8.7939144312 -8.48283004760742 3.0
8.2457336392 8.44078254699707 4.0
27.89045336 28.2475204467773 5.0
36.136187007 37.266170501709 6.0
36.136187007 37.401538848877 7.0
43.805905699 37.4568061828613 0.0
19.467800315 21.7853813171387 1.0
24.338105385 21.5366630554199 2.0
-10.07714477 -7.46288824081421 3.0
14.260960606 14.0233745574951 4.0
29.544945095 29.5650730133057 5.0
43.805905709 37.548942565918 6.0
43.805905709 41.6801910400391 7.0
43.400673471 39.6040191650391 0.0
21.450145846 22.113037109375 1.0
21.950527627 20.4832782745361 2.0
-10.718319514 -9.06201457977295 3.0
11.232208106 11.2945995330811 4.0
32.168465368 32.0704879760742 5.0
43.400673478 39.1733894348145 6.0
43.400673478 40.9292221069336 7.0
35.708898922 38.3624496459961 0.0
19.860917727 18.2857570648193 1.0
15.847981195 16.7142066955566 2.0
-6.8010445999 -7.22702932357788 3.0
9.0469365848 9.09976482391357 4.0
26.661962337 25.879077911377 5.0
35.708898932 38.3006820678711 6.0
35.708898932 35.2741546630859 7.0
42.19537597 38.8802375793457 0.0
21.611035583 21.7268905639648 1.0
20.584340387 20.7553749084473 2.0
-7.818182446 -7.85692644119263 3.0
12.766157932 12.6171722412109 4.0
29.429218038 29.402551651001 5.0
42.195375979 38.3520317077637 6.0
42.195375979 41.3113861083984 7.0
38.219132248 38.7132949829102 0.0
18.124133621 18.3968467712402 1.0
20.094998634 19.4984836578369 2.0
-3.9675986496 -3.65563249588013 3.0
16.127399981 15.9047183990479 4.0
22.091732274 21.9588279724121 5.0
38.219132252 38.4102020263672 6.0
38.219132252 37.3089485168457 7.0
37.34785903 37.9731597900391 0.0
19.494183125 19.5167541503906 1.0
17.853675906 19.3473663330078 2.0
-4.6743476987 -5.48700428009033 3.0
13.179328197 13.0351839065552 4.0
24.168530833 24.2482719421387 5.0
37.34785904 37.6260528564453 6.0
37.34785904 38.5072937011719 7.0
41.413317073 37.3570365905762 0.0
22.502810393 20.6565856933594 1.0
18.91050668 19.9366588592529 2.0
-4.5479233593 -6.06594562530518 3.0
14.362583311 14.2619075775146 4.0
27.050733762 27.0849170684814 5.0
41.413317083 36.9146499633789 6.0
41.413317083 39.338077545166 7.0
39.408474003 40.027702331543 0.0
19.787258075 20.1469440460205 1.0
19.621215928 18.9557418823242 2.0
-8.5467597404 -7.96667242050171 3.0
11.074456178 10.9917440414429 4.0
28.334017825 28.2721538543701 5.0
39.408474012 39.5381736755371 6.0
39.408474012 38.6387329101562 7.0
44.6219477930596 38.9475250244141 0.0
23.5295769810594 22.6683521270752 1.0
21.0923708210596 20.885139465332 2.0
-9.44677516695988 -9.76819801330566 3.0
11.6455956540597 11.4896726608276 4.0
32.9763521480588 32.7810287475586 5.0
44.621947793 38.6793365478516 6.0
44.621947793 41.9113464355469 7.0
35.042314664 38.3472061157227 0.0
17.377708596 18.0098876953125 1.0
17.66460607 17.9887828826904 2.0
-4.7609303256 -5.00931310653687 3.0
12.903675736 12.8152532577515 4.0
22.13863893 22.3599720001221 5.0
35.042314673 37.8676338195801 6.0
35.042314673 35.9290466308594 7.0
44.668685977 37.2419815063477 0.0
21.700098462 22.4027042388916 1.0
22.968587516 22.0980243682861 2.0
-8.4663400465 -7.49271059036255 3.0
14.50224746 14.3927011489868 4.0
30.166438518 30.1903648376465 5.0
44.668685987 37.1828880310059 6.0
44.668685987 42.8977546691895 7.0
44.271188228 38.3534698486328 0.0
21.134860564 21.5860939025879 1.0
23.136327666 22.1121635437012 2.0
-6.9953395415 -6.15792608261108 3.0
16.140988117 16.1164531707764 4.0
28.130200113 28.2101402282715 5.0
44.271188235 38.0834770202637 6.0
44.271188235 41.6959190368652 7.0
43.3296157145675 38.5631256103516 0.0
21.8144364805029 22.0623760223389 1.0
21.5151792385451 20.3174686431885 2.0
-10.3071776780671 -9.12745380401611 3.0
11.2080015596363 11.2886629104614 4.0
32.1216141588196 31.8779468536377 5.0
43.329615715 38.3454933166504 6.0
43.329615715 40.9410018920898 7.0
37.727034885 38.638484954834 0.0
16.740902708 19.2074718475342 1.0
20.986132178 18.4543018341064 2.0
-10.156378742 -7.5487961769104 3.0
10.829753428 10.6884984970093 4.0
26.897281458 26.7575035095215 5.0
37.727034893 38.4260940551758 6.0
37.727034893 37.3893280029297 7.0
43.665312103 38.2827987670898 0.0
21.432332029 21.6314697265625 1.0
22.232980074 21.3344211578369 2.0
-7.2001290055 -6.50747919082642 3.0
15.032851059 14.8496322631836 4.0
28.632461044 28.6244316101074 5.0
43.665312113 37.7539863586426 6.0
43.665312113 41.150634765625 7.0
40.6563483834553 38.2883262634277 0.0
20.9212538534144 22.2465400695801 1.0
19.7350945303589 20.7602844238281 2.0
-7.59696943049863 -7.95373582839966 3.0
12.1381250996025 12.0895071029663 4.0
28.5182231244438 30.2104187011719 5.0
40.656348383 38.0420913696289 6.0
40.656348383 41.4780654907227 7.0
30.759139545 38.3072166442871 0.0
13.433268738 16.086181640625 1.0
17.325870807 16.069314956665 2.0
-5.9875785844 -4.42493057250977 3.0
11.338292213 11.2699136734009 4.0
19.420847332 19.6887302398682 5.0
30.759139555 38.3209800720215 6.0
30.759139555 33.4528694152832 7.0
36.474229645 37.4969329833984 0.0
14.989459641 17.5503273010254 1.0
21.484770007 19.5517921447754 2.0
-5.0219280476 -2.88324689865112 3.0
16.462841952 16.5241451263428 4.0
20.011387696 20.1928482055664 5.0
36.474229652 36.9158172607422 6.0
36.474229652 36.2849960327148 7.0
33.340721185 37.986743927002 0.0
15.596021276 17.2868957519531 1.0
17.744699909 16.5625686645508 2.0
-7.4570294442 -6.01919174194336 3.0
10.287670455 10.2125196456909 4.0
23.05305073 23.0605983734131 5.0
33.340721195 37.6939582824707 6.0
33.340721195 34.3272285461426 7.0
34.143349258 39.8136596679688 0.0
15.417033592 17.3361968994141 1.0
18.726315666 17.5084629058838 2.0
-5.8334622114 -4.23455429077148 3.0
12.892853445 12.9768857955933 4.0
21.250495814 21.3223304748535 5.0
34.143349268 38.922176361084 6.0
34.143349268 35.4409523010254 7.0
40.763829748 37.8036422729492 0.0
20.589059268 21.7240676879883 1.0
20.174770481 19.268009185791 2.0
-11.800756688 -10.216362953186 3.0
8.374013783 8.5294303894043 4.0
32.389815966 32.5669822692871 5.0
40.763829757 37.798526763916 6.0
40.763829757 40.0757904052734 7.0
29.788224949 38.4382171630859 0.0
14.329239328 16.1451072692871 1.0
15.458985621 15.3412637710571 2.0
-6.2969425219 -5.56965923309326 3.0
9.1620430892 9.1718864440918 4.0
20.62618186 20.728702545166 5.0
29.788224958 38.5615501403809 6.0
29.788224958 33.155517578125 7.0
34.675869546 38.7797241210938 0.0
17.321679699 18.2421398162842 1.0
17.354189848 18.4201259613037 2.0
-5.026706589 -5.26643371582031 3.0
12.32748325 12.3594636917114 4.0
22.348386296 22.5424747467041 5.0
34.675869554 38.6566352844238 6.0
34.675869554 36.5259704589844 7.0
40.7238558279895 37.8541221618652 0.0
19.1061702059935 20.8180713653564 1.0
21.6176856259914 20.1595478057861 2.0
-8.60840523789577 -7.10564851760864 3.0
13.0092803879935 12.997220993042 4.0
27.7145752997044 28.4251232147217 5.0
40.723855828 37.2337265014648 6.0
40.723855828 39.6796073913574 7.0
41.414838344 38.8182792663574 0.0
21.79665744 21.4206218719482 1.0
19.618180905 18.8540878295898 2.0
-10.700829677 -10.1526880264282 3.0
8.9173512199 8.80392169952393 4.0
32.497487126 32.341236114502 5.0
41.414838352 38.1456489562988 6.0
41.414838352 39.7276268005371 7.0
33.391441204 38.5349998474121 0.0
16.946662464 17.8222122192383 1.0
16.44477874 16.7978572845459 2.0
-6.5811351715 -6.30256128311157 3.0
9.8636435585 9.97660255432129 4.0
23.527797645 23.6111125946045 5.0
33.391441214 38.5937538146973 6.0
33.391441214 35.174186706543 7.0
36.660645376 38.5673980712891 0.0
17.87289688 19.1674098968506 1.0
18.787748497 18.3186130523682 2.0
-6.6425323884 -5.8053035736084 3.0
12.1452161 12.1920785903931 4.0
24.515429278 24.8411178588867 5.0
36.660645385 38.1461601257324 6.0
36.660645385 37.245304107666 7.0
33.547567157 37.9813652038574 0.0
16.605704713 17.9759273529053 1.0
16.941862445 16.5288963317871 2.0
-8.1253896728 -7.49967050552368 3.0
8.8164727625 8.8326997756958 4.0
24.731094395 25.0884532928467 5.0
33.547567166 37.5001068115234 6.0
33.547567166 34.8342514038086 7.0
36.717923485 38.0211410522461 0.0
19.502688668 19.2083435058594 1.0
17.215234818 18.4051914215088 2.0
-6.0032006006 -6.88339185714722 3.0
11.212034207 11.2874431610107 4.0
25.505889278 25.972806930542 5.0
36.717923495 37.9767570495605 6.0
36.717923495 37.3089218139648 7.0
44.575737489 37.8732223510742 0.0
24.24689991 22.2731647491455 1.0
20.328837584 21.8076343536377 2.0
-6.0388557339 -7.60254907608032 3.0
14.289981845 14.2327938079834 4.0
30.285755649 30.4391555786133 5.0
44.575737494 38.1661148071289 6.0
44.575737494 42.1611785888672 7.0
43.091610313 38.131534576416 0.0
22.329030269 21.6345500946045 1.0
20.762580045 21.051965713501 2.0
-7.1251861541 -7.6769700050354 3.0
13.637393882 13.5253868103027 4.0
29.454216432 29.4389019012451 5.0
43.091610321 37.9382514953613 6.0
43.091610321 41.2020378112793 7.0
42.647069964 38.2785568237305 0.0
21.774885881 21.6894512176514 1.0
20.872184083 20.8021678924561 2.0
-7.6954983452 -7.71259260177612 3.0
13.176685728 13.0901155471802 4.0
29.470384236 29.4799499511719 5.0
42.647069973 38.3412551879883 6.0
42.647069973 41.0602569580078 7.0
36.24022533 38.6551628112793 0.0
17.188554164 18.8818283081055 1.0
19.051671166 18.8205490112305 2.0
-6.1891251621 -5.29846858978271 3.0
12.862545995 12.7339458465576 4.0
23.377679336 23.4845695495605 5.0
36.240225339 39.0571975708008 6.0
36.240225339 37.7871971130371 7.0
31.876756195 37.8946952819824 0.0
14.446454361 16.7504482269287 1.0
17.430301835 17.4115371704102 2.0
-5.0545683645 -4.13169622421265 3.0
12.375733461 12.4499425888062 4.0
19.501022735 19.9124183654785 5.0
31.876756203 37.9031105041504 6.0
31.876756203 34.8857650756836 7.0
38.711049649 39.0694046020508 0.0
19.068925854 19.1226139068604 1.0
19.642123794 19.5599174499512 2.0
-5.0802825502 -5.01072645187378 3.0
14.561841234 14.4733190536499 4.0
24.149208415 24.4036884307861 5.0
38.711049659 38.2914772033691 6.0
38.711049659 37.544506072998 7.0
33.592635662 39.0207405090332 0.0
16.042625471 17.7420520782471 1.0
17.550010192 17.5223922729492 2.0
-5.9813560668 -5.27224731445312 3.0
11.568654115 11.5348405838013 4.0
22.023981548 22.1476860046387 5.0
33.592635672 38.7382926940918 6.0
33.592635672 35.8421974182129 7.0
31.9394970213296 38.2818374633789 0.0
15.7221064593445 16.5701427459717 1.0
16.2173905613348 16.9497032165527 2.0
-3.67606552949996 -3.85584878921509 3.0
12.5413250313355 12.3931245803833 4.0
19.3981719893576 19.5934658050537 5.0
31.939497021 37.99658203125 6.0
31.939497021 34.4708824157715 7.0
32.605677366 39.6545944213867 0.0
15.053063258 16.8182258605957 1.0
17.552614114 16.8343372344971 2.0
-5.3475005701 -4.4284291267395 3.0
12.20511354 12.0861310958862 4.0
20.400563832 20.3604335784912 5.0
32.605677369 39.0122146606445 6.0
32.605677369 34.7319755554199 7.0
35.02561424 39.1762199401855 0.0
18.375955543 18.5617561340332 1.0
16.649658698 17.933931350708 2.0
-5.1216145805 -6.00941038131714 3.0
11.528044108 11.5016431808472 4.0
23.497570133 23.7881946563721 5.0
35.02561425 39.0269508361816 6.0
35.02561425 36.4207153320312 7.0
29.155020361 37.7641563415527 0.0
13.928860816 15.9876508712769 1.0
15.226159546 15.3726282119751 2.0
-5.2043069028 -4.68897819519043 3.0
10.021852634 10.070650100708 4.0
19.133167728 19.3883056640625 5.0
29.155020371 37.7823295593262 6.0
29.155020371 32.8344879150391 7.0
29.979099891 35.9966773986816 0.0
16.392071584 16.2226486206055 1.0
13.587028308 15.0323486328125 2.0
-4.5603535805 -5.53596305847168 3.0
9.026674718 8.923171043396 4.0
20.952425173 20.7936782836914 5.0
29.9790999 36.8325042724609 6.0
29.9790999 32.9512901306152 7.0
40.412374419 39.0125999450684 0.0
21.632869342 20.2347812652588 1.0
18.779505078 20.165864944458 2.0
-4.6354756287 -5.86581945419312 3.0
14.14402944 14.2217636108398 4.0
26.26834498 26.4187030792236 5.0
40.412374428 38.3975982666016 6.0
40.412374428 39.3583679199219 7.0
44.7996696066759 38.7006759643555 0.0
24.3565352806759 22.194787979126 1.0
20.4431343366759 22.1857109069824 2.0
-4.33926525477587 -6.07624387741089 3.0
16.1038690816759 16.0274314880371 4.0
28.6958005009671 28.7368316650391 5.0
44.799669607 38.2276153564453 6.0
44.799669607 42.1971893310547 7.0
35.691930078 38.4926795959473 0.0
18.173171081 18.1872406005859 1.0
17.518759 18.5101547241211 2.0
-4.3926482071 -4.9055609703064 3.0
13.126110785 13.1647872924805 4.0
22.565819295 22.7568283081055 5.0
35.691930085 38.0520210266113 6.0
35.691930085 36.4406776428223 7.0
37.293002373 38.2175941467285 0.0
17.018201783 19.4093265533447 1.0
20.274800591 18.6651878356934 2.0
-8.9983650782 -6.96180295944214 3.0
11.276435503 11.2391748428345 4.0
26.016566871 26.0208625793457 5.0
37.293002382 37.857292175293 6.0
37.293002382 37.7943305969238 7.0
31.749050626 38.0218124389648 0.0
16.61083934 16.9624633789062 1.0
15.138211288 16.573881149292 2.0
-3.9083704158 -4.69251728057861 3.0
11.229840863 11.3628749847412 4.0
20.519209764 20.7576198577881 5.0
31.749050635 37.7182846069336 6.0
31.749050635 34.4254112243652 7.0
44.43859931 37.0469703674316 0.0
22.567248059 22.8323040008545 1.0
21.871351255 20.9507427215576 2.0
-10.274790847 -9.31620407104492 3.0
11.596560402 11.5659952163696 4.0
32.842038912 32.5918464660645 5.0
44.438599316 37.4995803833008 6.0
44.438599316 41.7696151733398 7.0
30.584762336 38.4174194335938 0.0
14.448986708 16.2740821838379 1.0
16.13577563 15.5234565734863 2.0
-7.0107275854 -5.78191757202148 3.0
9.1250480359 9.37574577331543 4.0
21.459714302 21.6913471221924 5.0
30.584762345 37.8574676513672 6.0
30.584762345 33.0694770812988 7.0
36.167711461 39.1198539733887 0.0
15.702748451 17.9952907562256 1.0
20.464963013 19.3852233886719 2.0
-4.5943051912 -3.1515679359436 3.0
15.870657815 15.8489856719971 4.0
20.297053649 20.5541362762451 5.0
36.167711467 38.6347007751465 6.0
36.167711467 37.102840423584 7.0
45.589023155 38.2959823608398 0.0
22.43199889 22.7925148010254 1.0
23.157024265 21.6703662872314 2.0
-9.6703339461 -8.5331974029541 3.0
13.486690309 13.4877252578735 4.0
32.102332846 32.2811393737793 5.0
45.589023165 38.3305435180664 6.0
45.589023165 42.6452903747559 7.0
33.954863965 39.1545906066895 0.0
16.569146941 18.4803066253662 1.0
17.385717024 16.8482513427734 2.0
-8.6921868392 -7.21514654159546 3.0
8.6935301746 8.79104232788086 4.0
25.26133379 25.4949645996094 5.0
33.954863975 38.7936134338379 6.0
33.954863975 36.1590461730957 7.0
39.104021209 38.4186553955078 0.0
19.726585997 19.7319431304932 1.0
19.377435214 19.7776069641113 2.0
-5.8707549184 -5.94842338562012 3.0
13.506680288 13.4467115402222 4.0
25.597340923 25.7061157226562 5.0
39.104021217 38.1387214660645 6.0
39.104021217 38.8953056335449 7.0
41.6151393745108 39.4605255126953 0.0
21.6144474948183 23.2610206604004 1.0
20.0006918800463 22.3324222564697 2.0
-5.68615976863137 -7.44690847396851 3.0
14.3145321112399 14.1787891387939 4.0
27.3006071309532 30.1197662353516 5.0
41.615139375 38.8681373596191 6.0
41.615139375 43.1649169921875 7.0
29.251209159 37.6545829772949 0.0
16.305978933 16.6048641204834 1.0
12.945230226 15.2248754501343 2.0
-4.8421763952 -5.61566686630249 3.0
8.1030538205 8.64456081390381 4.0
21.148155339 21.4315528869629 5.0
29.251209169 37.4464569091797 6.0
29.251209169 33.1458473205566 7.0
39.145261593 38.1697654724121 0.0
20.853611949 20.081937789917 1.0
18.291649647 19.0412845611572 2.0
-6.3281983222 -7.2891001701355 3.0
11.963451316 11.8616352081299 4.0
27.181810279 27.3477458953857 5.0
39.145261601 38.2401885986328 6.0
39.145261601 38.4910583496094 7.0
43.724223504 37.7768745422363 0.0
24.240212841 22.3241062164307 1.0
19.484010663 20.8815269470215 2.0
-6.7543223688 -8.31910037994385 3.0
12.729688284 12.6391706466675 4.0
30.99453522 30.8486366271973 5.0
43.724223514 37.884765625 6.0
43.724223514 41.6264266967773 7.0
30.694427938 38.1176147460938 0.0
15.962392584 16.6270313262939 1.0
14.732035355 15.3380336761475 2.0
-5.8529078341 -5.93337106704712 3.0
8.8791275119 8.83771896362305 4.0
21.815300427 21.7334136962891 5.0
30.694427948 37.9934768676758 6.0
30.694427948 33.3604965209961 7.0
46.899446071 38.6436882019043 0.0
22.810606072 23.2001914978027 1.0
24.08884 21.5602931976318 2.0
-12.207831656 -9.33517932891846 3.0
11.881008335 11.7105913162231 4.0
35.018437738 34.3398780822754 5.0
46.899446079 38.6103820800781 6.0
46.899446079 42.5004539489746 7.0
30.801741864 37.7746276855469 0.0
14.581609416 16.4901885986328 1.0
16.220132448 15.3019275665283 2.0
-7.6078809929 -6.2940468788147 3.0
8.6122514454 8.90657806396484 4.0
22.189490419 22.362361907959 5.0
30.801741874 37.6708908081055 6.0
30.801741874 33.1409111022949 7.0
40.3889007570329 38.3990745544434 0.0
21.8864546730297 20.8562088012695 1.0
18.5024460860314 19.0381507873535 2.0
-8.35977776734346 -8.62235546112061 3.0
10.1426683190367 10.3160648345947 4.0
30.2462322648819 29.876615524292 5.0
40.388900757 38.4628791809082 6.0
40.388900757 38.9935417175293 7.0
35.801686121 37.8508529663086 0.0
19.152651215 17.4389457702637 1.0
16.649034906 18.3284683227539 2.0
-1.2326560583 -3.14993762969971 3.0
15.416378838 15.5309925079346 4.0
20.385307283 20.6580429077148 5.0
35.801686131 38.1226081848145 6.0
35.801686131 35.5984802246094 7.0
43.275430292 38.9351806640625 0.0
21.49768275 22.1346187591553 1.0
21.777747542 21.1901092529297 2.0
-8.5057364073 -8.0211296081543 3.0
13.272011125 13.1058235168457 4.0
30.003419167 29.9724426269531 5.0
43.275430302 38.866512298584 6.0
43.275430302 41.4904594421387 7.0
36.6745338859035 38.3367919921875 0.0
17.2258453439028 18.8013954162598 1.0
19.4486885479032 18.627857208252 2.0
-7.45674864010256 -6.27988004684448 3.0
11.9919399079021 12.017541885376 4.0
24.6825939849038 24.8856868743896 5.0
36.674533885 38.2248344421387 6.0
36.674533885 37.3767623901367 7.0
42.2919299411196 37.5875625610352 0.0
20.8701371011196 21.2876930236816 1.0
21.4217928491193 20.6673679351807 2.0
-7.8990978219197 -7.34667825698853 3.0
13.5226950281196 13.4771938323975 4.0
28.769234923119 28.5805339813232 5.0
42.291929941 37.3599815368652 6.0
42.291929941 40.3604278564453 7.0
37.80686854 38.4056816101074 0.0
18.84529386 18.5401039123535 1.0
18.961574682 19.6283340454102 2.0
-3.0668288295 -3.70741987228394 3.0
15.894745843 15.907096862793 4.0
21.912122698 22.1396751403809 5.0
37.80686855 38.7179260253906 6.0
37.80686855 37.4704437255859 7.0
36.296445798 38.8396034240723 0.0
19.144127702 18.6647720336914 1.0
17.152318097 18.3849201202393 2.0
-4.4799623055 -5.2938404083252 3.0
12.672355782 12.6883163452148 4.0
23.624090017 23.57541847229 5.0
36.296445808 38.4032173156738 6.0
36.296445808 36.8623657226562 7.0
26.786347975 39.0824241638184 0.0
13.739106858 15.5152616500854 1.0
13.047241117 14.46364402771 2.0
-5.0192652199 -4.96618747711182 3.0
8.0279758877 8.28325080871582 4.0
18.758372088 19.1722412109375 5.0
26.786347985 38.8822364807129 6.0
26.786347985 31.68630027771 7.0
38.715518927 39.803783416748 0.0
20.971478582 20.418212890625 1.0
17.744040345 18.6807651519775 2.0
-7.5775969749 -8.07617950439453 3.0
10.16644336 10.2803039550781 4.0
28.549075567 28.6153907775879 5.0
38.715518937 39.2394905090332 6.0
38.715518937 38.5385589599609 7.0
35.505062995 39.2403678894043 0.0
16.975132375 18.4003238677979 1.0
18.529930624 18.3079319000244 2.0
-6.1934391103 -5.42330980300903 3.0
12.336491507 12.2231378555298 4.0
23.168571492 23.0345230102539 5.0
35.505063002 38.9632453918457 6.0
35.505063002 36.842945098877 7.0
44.227649061 38.8344802856445 0.0
23.455343466 22.3934555053711 1.0
20.772305597 21.5205249786377 2.0
-7.272984789 -8.10009860992432 3.0
13.499320798 13.4871215820312 4.0
30.728328264 30.7766151428223 5.0
44.227649071 38.0770072937012 6.0
44.227649071 41.8299903869629 7.0
45.773124246 37.7202339172363 0.0
24.134680774 22.321159362793 1.0
21.638443473 21.8957672119141 2.0
-6.2883820036 -7.22992277145386 3.0
15.350061461 15.3104095458984 4.0
30.423062786 30.5529003143311 5.0
45.773124255 37.2553977966309 6.0
45.773124255 42.3487434387207 7.0
38.6152827522378 39.477424621582 0.0
18.6636500942378 20.573938369751 1.0
19.9516326682378 19.4033470153809 2.0
-8.33253161363777 -7.28676462173462 3.0
11.6191010542378 11.4374580383301 4.0
26.9961816644691 28.1716995239258 5.0
38.615282752 39.3005790710449 6.0
38.615282752 39.5897064208984 7.0
42.131124826 38.1851119995117 0.0
21.88265512 21.0735111236572 1.0
20.248469706 20.705358505249 2.0
-6.067617241 -6.48887205123901 3.0
14.180852455 14.1790180206299 4.0
27.950272371 28.1024494171143 5.0
42.131124836 37.9523544311523 6.0
42.131124836 40.2436027526855 7.0
38.030369756 39.1824035644531 0.0
19.698625144 19.2150058746338 1.0
18.331744611 17.9667816162109 2.0
-8.0645153211 -7.47400903701782 3.0
10.26722928 10.4045381546021 4.0
27.763140475 27.5179481506348 5.0
38.030369766 38.5118560791016 6.0
38.030369766 37.5643844604492 7.0
32.694869995 38.5036087036133 0.0
14.429732866 16.6289081573486 1.0
18.265137131 16.9874668121338 2.0
-5.8301144142 -4.12526512145996 3.0
12.435022709 12.6043615341187 4.0
20.259847288 20.5650177001953 5.0
32.694870002 38.3232879638672 6.0
32.694870002 34.1418228149414 7.0
41.58883183 38.2969932556152 0.0
19.184872754 20.2061920166016 1.0
22.403959077 21.0583267211914 2.0
-6.437067104 -4.62938070297241 3.0
15.966891964 15.9271173477173 4.0
25.621939867 25.7042179107666 5.0
41.58883184 38.6637344360352 6.0
41.58883184 39.7629280090332 7.0
33.622063035 38.530574798584 0.0
16.212159516 17.9072532653809 1.0
17.409903519 16.844934463501 2.0
-8.2486263793 -7.34330320358276 3.0
9.16127713 9.19750308990479 4.0
24.460785905 24.6437816619873 5.0
33.622063045 37.9805374145508 6.0
33.622063045 35.4740943908691 7.0
40.101738497 39.2006187438965 0.0
22.613544165 21.4271278381348 1.0
17.488194333 18.9481239318848 2.0
-8.294425066 -9.33534049987793 3.0
9.1937692583 9.32270431518555 4.0
30.90796924 31.0978183746338 5.0
40.101738506 39.1336441040039 6.0
40.101738506 39.617359161377 7.0
39.347750484 38.9384841918945 0.0
19.631600055 20.7048988342285 1.0
19.716150429 20.1532783508301 2.0
-7.2110260491 -7.05266904830933 3.0
12.50512437 12.3973474502563 4.0
26.842626114 27.589786529541 5.0
39.347750494 38.1576271057129 6.0
39.347750494 39.5988540649414 7.0
46.240008453 38.6810073852539 0.0
23.277257726 23.4989967346191 1.0
22.962750727 23.6234474182129 2.0
-6.5077704351 -6.42988252639771 3.0
16.454980282 16.2094058990479 4.0
29.785028171 29.3643608093262 5.0
46.240008463 38.4073829650879 6.0
46.240008463 44.5637626647949 7.0
43.885613963 38.2784118652344 0.0
23.043389811 22.0986080169678 1.0
20.842224153 22.0082912445068 2.0
-5.1243468199 -5.82089614868164 3.0
15.717877324 15.6009302139282 4.0
28.16773664 28.2542762756348 5.0
43.885613972 38.2855911254883 6.0
43.885613972 42.091381072998 7.0
44.877286045 38.3865165710449 0.0
23.85937511 22.7580642700195 1.0
21.017910935 20.1526775360107 2.0
-10.548874854 -9.93652534484863 3.0
10.469036071 10.5815296173096 4.0
34.408249974 34.0433540344238 5.0
44.877286055 37.9927787780762 6.0
44.877286055 41.8243064880371 7.0
44.28102484 38.5324211120605 0.0
23.476033214 22.68678855896 1.0
20.804991626 21.4768867492676 2.0
-7.5751269732 -8.29131698608398 3.0
13.229864644 13.1623373031616 4.0
31.051160197 31.0129375457764 5.0
44.281024849 38.5806045532227 6.0
44.281024849 42.1327171325684 7.0
39.443195324 38.1847991943359 0.0
17.235953612 19.6798992156982 1.0
22.207241712 19.5363464355469 2.0
-9.075573722 -6.50760841369629 3.0
13.13166798 13.1101360321045 4.0
26.311527344 26.5249290466309 5.0
39.443195334 37.6124038696289 6.0
39.443195334 38.3225212097168 7.0
44.5099801195094 37.1273231506348 0.0
20.7192442195094 22.0590629577637 1.0
23.7907359105096 21.7909851074219 2.0
-9.37764966100969 -7.54417181015015 3.0
14.4130862495096 14.3207731246948 4.0
30.0968938805093 30.1416130065918 5.0
44.50998012 36.8482818603516 6.0
44.50998012 42.1727066040039 7.0
36.079576813 38.2234840393066 0.0
18.198560719 19.2656593322754 1.0
17.881016095 17.4695701599121 2.0
-8.6789161085 -8.1096658706665 3.0
9.202099978 9.118896484375 4.0
26.877476836 27.020824432373 5.0
36.079576822 38.0360488891602 6.0
36.079576822 37.1305694580078 7.0
43.881758866 39.3747177124023 0.0
21.090009056 23.5745143890381 1.0
22.79174981 23.4815635681152 2.0
-6.6710954927 -6.47954130172729 3.0
16.120654308 15.989706993103 4.0
27.761104558 30.1951484680176 5.0
43.881758876 39.0117988586426 6.0
43.881758876 43.9466247558594 7.0
30.108818274 36.8926582336426 0.0
17.112174764 16.9153232574463 1.0
12.996643511 15.2840881347656 2.0
-4.8964528045 -6.21089315414429 3.0
8.1001906976 8.40028953552246 4.0
22.008627577 22.3543682098389 5.0
30.108818284 36.9432106018066 6.0
30.108818284 33.2788925170898 7.0
34.2205121900012 37.8752822875977 0.0
18.8110553650029 18.5524215698242 1.0
15.4094568300067 16.4252243041992 2.0
-7.05168531729669 -7.78768491744995 3.0
8.35777151259514 8.34664630889893 4.0
25.8627406830075 25.8669109344482 5.0
34.22051219 37.9323921203613 6.0
34.22051219 35.668701171875 7.0
37.402197962 38.5551109313965 0.0
19.960548139 19.3140754699707 1.0
17.441649823 18.4927577972412 2.0
-5.9560955294 -7.0258355140686 3.0
11.485554285 11.5273351669312 4.0
25.916643678 26.3034286499023 5.0
37.402197971 37.9541931152344 6.0
37.402197971 37.5865859985352 7.0
35.453722399 38.3747253417969 0.0
16.625051972 18.2392101287842 1.0
18.828670427 17.5041103363037 2.0
-7.5918246106 -6.273681640625 3.0
11.236845807 11.1635589599609 4.0
24.216876592 24.3407764434814 5.0
35.453722409 38.4625778198242 6.0
35.453722409 35.9341163635254 7.0
32.872381145 39.074348449707 0.0
15.679111706 17.2336750030518 1.0
17.193269439 16.6686668395996 2.0
-6.3769976884 -5.32223606109619 3.0
10.816271741 10.8602409362793 4.0
22.056109405 22.08229637146 5.0
32.872381155 38.5046463012695 6.0
32.872381155 34.7904052734375 7.0
33.5777143162235 38.6191482543945 0.0
24.3043372519187 17.8207416534424 1.0
9.27337706863705 16.6267776489258 2.0
-1.413226005e-07 -7.02728605270386 3.0
9.27337692732672 9.24127864837646 4.0
24.3043373936238 24.6269416809082 5.0
33.577714316 38.0670509338379 6.0
33.577714316 35.0301551818848 7.0
43.986168792 36.937198638916 0.0
19.73252431 21.5083961486816 1.0
24.253644482 21.2181015014648 2.0
-7.8587048092 -5.47689723968506 3.0
16.394939663 16.1498260498047 4.0
27.591229129 27.7168655395508 5.0
43.986168802 37.1762313842773 6.0
43.986168802 41.6757850646973 7.0
39.701527196 38.2919845581055 0.0
21.433905404 20.7723426818848 1.0
18.267621793 19.425687789917 2.0
-6.9530592881 -7.59958505630493 3.0
11.314562495 11.3121290206909 4.0
28.386964701 28.1717796325684 5.0
39.701527205 38.5264472961426 6.0
39.701527205 39.3666076660156 7.0
36.319422307 37.7331275939941 0.0
18.261170173 19.0919361114502 1.0
18.058252134 17.4455738067627 2.0
-9.3076661868 -8.36457443237305 3.0
8.7505859373 8.94688892364502 4.0
27.56883637 27.8639831542969 5.0
36.319422317 38.0315895080566 6.0
36.319422317 36.9209632873535 7.0
39.546529711 38.4356460571289 0.0
21.883828824 20.4687423706055 1.0
17.662700888 19.2652797698975 2.0
-6.2516235274 -7.79578828811646 3.0
11.411077352 11.4421272277832 4.0
28.13545236 28.3675937652588 5.0
39.54652972 37.8579177856445 6.0
39.54652972 39.0443458557129 7.0
46.328884399 37.5195198059082 0.0
20.182008475 22.7009830474854 1.0
26.146875933 22.7080307006836 2.0
-10.473669833 -7.37232732772827 3.0
15.673206099 15.575855255127 4.0
30.655678309 30.4711780548096 5.0
46.3288844 37.3919486999512 6.0
46.3288844 43.2718124389648 7.0
39.359671453 37.9955558776855 0.0
20.208221774 20.3056735992432 1.0
19.151449681 19.2121753692627 2.0
-7.7565035076 -7.96970701217651 3.0
11.394946165 11.3201990127563 4.0
27.964725289 28.1223049163818 5.0
39.359671461 37.8209571838379 6.0
39.359671461 38.8423461914062 7.0
43.813870978 39.1599159240723 0.0
22.861832963 22.254322052002 1.0
20.952038016 21.7326011657715 2.0
-6.6064627953 -7.22918367385864 3.0
14.345575211 14.2311363220215 4.0
29.468295768 29.4399394989014 5.0
43.813870988 38.9048614501953 6.0
43.813870988 41.9718818664551 7.0
39.146221866 39.0258140563965 0.0
21.274977313 20.644718170166 1.0
17.871244553 18.4408435821533 2.0
-8.221666714 -8.69419384002686 3.0
9.6495778292 10.0071420669556 4.0
29.496644037 30.0180149078369 5.0
39.146221876 38.4662284851074 6.0
39.146221876 38.3873329162598 7.0
34.753353004 39.1427421569824 0.0
19.306587872 18.2827625274658 1.0
15.446765133 16.2944030761719 2.0
-7.4077231802 -7.63805055618286 3.0
8.0390419425 8.27563858032227 4.0
26.714311062 26.3839092254639 5.0
34.753353014 38.7827377319336 6.0
34.753353014 35.2345581054688 7.0
36.8589150083515 39.4706802368164 0.0
18.5606086093515 19.4125213623047 1.0
18.2983064083517 17.7765789031982 2.0
-8.423068234052 -7.53331518173218 3.0
9.87523817465195 9.90042686462402 4.0
26.9836767778432 27.1030578613281 5.0
36.858915008 38.9655838012695 6.0
36.858915008 37.0646896362305 7.0
45.609388715 39.0425567626953 0.0
20.431056468 22.1931304931641 1.0
25.178332247 22.656436920166 2.0
-8.8685199312 -6.40776920318604 3.0
16.309812306 16.1693744659424 4.0
29.299576409 29.5193862915039 5.0
45.609388725 38.3936195373535 6.0
45.609388725 42.3732833862305 7.0
37.221502305 38.3484802246094 0.0
19.795691461 19.8528137207031 1.0
17.425810844 18.0503482818604 2.0
-8.0004149655 -8.0742244720459 3.0
9.4253958691 9.6834888458252 4.0
27.796106436 27.9083557128906 5.0
37.221502314 38.0214576721191 6.0
37.221502314 37.7615051269531 7.0
39.3512418478291 39.2957038879395 0.0
21.5003115557496 20.8432102203369 1.0
17.8509302889164 19.1229019165039 2.0
-7.9913457065862 -8.4628210067749 3.0
9.85958458276425 9.9439697265625 4.0
29.4916572628094 29.4896602630615 5.0
39.351241848 38.7678680419922 6.0
39.351241848 39.442512512207 7.0
44.354286129 38.2162055969238 0.0
23.778468789 22.5338478088379 1.0
20.57581734 21.5574703216553 2.0
-6.7304183908 -7.99050569534302 3.0
13.845398939 13.756760597229 4.0
30.50888719 30.5417575836182 5.0
44.354286139 38.0869255065918 6.0
44.354286139 42.2690010070801 7.0
41.1438439687938 38.1556243896484 0.0
20.0722524531884 20.5898857116699 1.0
21.0715915195 19.8956928253174 2.0
-7.89338141352269 -7.13482618331909 3.0
13.1782101064826 13.0356712341309 4.0
27.9656338666077 28.0035648345947 5.0
41.143843969 38.1603202819824 6.0
41.143843969 39.794563293457 7.0
32.436363662 38.3456611633301 0.0
18.29560079 17.4481048583984 1.0
14.140762872 16.4804420471191 2.0
-4.3766657499 -6.01217126846313 3.0
9.7640971121 9.92094039916992 4.0
22.67226655 22.9372463226318 5.0
32.436363672 38.0900001525879 6.0
32.436363672 34.9349594116211 7.0
35.980268817 37.7474098205566 0.0
18.371260578 18.7088871002197 1.0
17.609008241 17.5887641906738 2.0
-7.5074447835 -7.42871522903442 3.0
10.101563449 10.1222381591797 4.0
25.87870537 26.2794914245605 5.0
35.980268826 37.0687408447266 6.0
35.980268826 36.1344223022461 7.0
39.361696941 37.561466217041 0.0
18.798616943 19.0980243682861 1.0
20.563079997 20.0402278900146 2.0
-4.5015726852 -4.05684280395508 3.0
16.061507302 15.9852266311646 4.0
23.300189639 23.1504173278809 5.0
39.361696951 37.5984420776367 6.0
39.361696951 38.9515914916992 7.0
38.339957491 37.9287757873535 0.0
20.325182647 20.5509910583496 1.0
18.014774844 19.6161155700684 2.0
-6.00621 -6.91498708724976 3.0
12.008564835 12.0412445068359 4.0
26.331392656 28.0150165557861 5.0
38.339957501 38.3887176513672 6.0
38.339957501 39.0284233093262 7.0
38.4370780266379 38.9979934692383 0.0
19.1112284596379 22.0945701599121 1.0
19.3258495766379 20.1875915527344 2.0
-8.41787501523793 -7.70697546005249 3.0
10.9079745616379 10.8547630310059 4.0
27.5291034382175 30.1355514526367 5.0
38.437078026 38.7288360595703 6.0
38.437078026 40.9472618103027 7.0
31.680301152 38.4310493469238 0.0
16.922816031 16.8353137969971 1.0
14.757485122 15.2912740707397 2.0
-6.2787127987 -6.42772340774536 3.0
8.4787723148 8.54596519470215 4.0
23.201528838 23.2188587188721 5.0
31.680301161 38.3872604370117 6.0
31.680301161 33.4328651428223 7.0
36.698769122 38.8383483886719 0.0
19.594432034 17.7835655212402 1.0
17.104337088 18.3008460998535 2.0
-2.472193454 -3.84929037094116 3.0
14.632143624 14.632981300354 4.0
22.066625498 21.6511745452881 5.0
36.698769132 38.5000343322754 6.0
36.698769132 35.8946838378906 7.0
40.942334912 39.9460792541504 0.0
22.219375241 20.6884422302246 1.0
18.722959671 20.2446804046631 2.0
-5.2427967213 -6.32739400863647 3.0
13.48016294 13.5754909515381 4.0
27.462171973 27.1797504425049 5.0
40.942334922 39.5808792114258 6.0
40.942334922 39.7870864868164 7.0
41.496981527 38.2445068359375 0.0
18.951043784 21.9174098968506 1.0
22.545937743 21.6434020996094 2.0
-7.7537930795 -6.4086012840271 3.0
14.792144654 14.6895732879639 4.0
26.704836873 28.3137054443359 5.0
41.496981537 38.265625 6.0
41.496981537 41.8146209716797 7.0
33.577466055 37.5693817138672 0.0
17.755510174 17.8776912689209 1.0
15.821955881 16.1289157867432 2.0
-7.358985858 -6.93504953384399 3.0
8.4629700135 8.82866096496582 4.0
25.114496042 24.9716320037842 5.0
33.577466065 37.1988296508789 6.0
33.577466065 34.797924041748 7.0
38.410697594746 38.8489418029785 0.0
20.0282722997678 18.9285411834717 1.0
18.3824252977607 20.0090808868408 2.0
-2.25462323941058 -3.35720491409302 3.0
16.1278020587646 15.9544897079468 4.0
22.2828955387576 22.3579692840576 5.0
38.410697595 38.710319519043 6.0
38.410697595 38.5886878967285 7.0
40.390522102 39.3337631225586 0.0
23.101316011 21.5135726928711 1.0
17.289206093 18.837438583374 2.0
-8.5131515481 -9.63582038879395 3.0
8.7760545352 8.85773277282715 4.0
31.614467568 31.832483291626 5.0
40.390522112 38.7059936523438 6.0
40.390522112 39.4735984802246 7.0
38.02724698 38.1036338806152 0.0
17.092159812 19.120288848877 1.0
20.935087168 19.2895698547363 2.0
-7.0096758313 -5.11860513687134 3.0
13.925411327 14.0265121459961 4.0
24.101835653 24.4289226531982 5.0
38.02724699 37.8691673278809 6.0
38.02724699 38.1358222961426 7.0
34.831322289 36.8072509765625 0.0
17.981428364 18.4415626525879 1.0
16.849893928 17.5926132202148 2.0
-5.7085713019 -5.93267583847046 3.0
11.14132262 11.1927518844604 4.0
23.689999673 23.7075290679932 5.0
34.831322295 36.6822929382324 6.0
34.831322295 36.1207618713379 7.0
33.152476822 38.7410354614258 0.0
16.395481675 17.4682922363281 1.0
16.756995148 16.7102336883545 2.0
-6.4008279506 -5.54494476318359 3.0
10.356167188 10.7550373077393 4.0
22.796309635 22.932674407959 5.0
33.152476832 39.0138130187988 6.0
33.152476832 34.7097129821777 7.0
37.7618147597387 37.9946594238281 0.0
20.5714887837879 19.3425712585449 1.0
17.1903259800247 18.2179393768311 2.0
-5.43010176849644 -6.86212682723999 3.0
11.7602242121808 11.6600589752197 4.0
26.0015905517818 25.9702701568604 5.0
37.76181476 37.9316368103027 6.0
37.76181476 37.3116416931152 7.0
46.821650957 38.4257621765137 0.0
22.453957792 23.3864841461182 1.0
24.367693165 22.5277557373047 2.0
-10.797760543 -9.41441059112549 3.0
13.569932612 13.3759241104126 4.0
33.251718344 32.9302749633789 5.0
46.821650967 38.3611297607422 6.0
46.821650967 43.4030952453613 7.0
41.1837901754312 39.306941986084 0.0
19.2503118354314 20.7185611724854 1.0
21.9334783494314 20.4899139404297 2.0
-7.5443023473312 -6.17099189758301 3.0
14.3891760014314 14.2149782180786 4.0
26.7946141824314 26.9228382110596 5.0
41.183790175 38.7546234130859 6.0
41.183790175 40.0614166259766 7.0
36.784988632 38.6524810791016 0.0
21.035700431 19.6585426330566 1.0
15.749288201 17.6597785949707 2.0
-6.0648376105 -7.91333246231079 3.0
9.6844505818 9.57503318786621 4.0
27.100538051 27.3205089569092 5.0
36.78498864 38.4926071166992 6.0
36.78498864 37.3708419799805 7.0
37.6894949442102 38.7406768798828 0.0
17.6138134052102 21.1670055389404 1.0
20.0756815502102 20.0744361877441 2.0
-8.78806436921022 -7.7686619758606 3.0
11.2876171802102 11.1525831222534 4.0
26.401877737898 29.2062187194824 5.0
37.689494945 38.8196334838867 6.0
37.689494945 40.10791015625 7.0
41.308881088 38.2769470214844 0.0
21.148817904 20.980188369751 1.0
20.160063185 20.5358371734619 2.0
-6.8065142938 -7.30223608016968 3.0
13.353548882 13.3990888595581 4.0
27.955332207 28.1741981506348 5.0
41.308881097 38.4388542175293 6.0
41.308881097 40.3364677429199 7.0
42.253011175 37.3778648376465 0.0
21.557015891 21.396484375 1.0
20.695995284 19.9566631317139 2.0
-9.3835381575 -8.94362545013428 3.0
11.312457118 11.2526245117188 4.0
30.940554058 30.9920768737793 5.0
42.253011184 37.2502593994141 6.0
42.253011184 40.585636138916 7.0
40.047660483 38.1495399475098 0.0
19.392839472 20.0742130279541 1.0
20.654821011 19.7708702087402 2.0
-7.4809136797 -6.60109424591064 3.0
13.173907322 13.1142282485962 4.0
26.873753161 26.9768104553223 5.0
40.047660493 38.281078338623 6.0
40.047660493 38.8249397277832 7.0
46.604605723 39.019947052002 0.0
22.954120862 22.5326461791992 1.0
23.650484861 22.7595176696777 2.0
-7.9204617962 -7.5112156867981 3.0
15.730023055 15.5815744400024 4.0
30.874582668 30.9756259918213 5.0
46.604605733 38.4075317382812 6.0
46.604605733 42.8081741333008 7.0
45.106295532 38.5554237365723 0.0
23.2126055 23.2300148010254 1.0
21.893690031 21.1572799682617 2.0
-10.542381522 -9.44595813751221 3.0
11.3513085 11.2740564346313 4.0
33.754987032 33.2581100463867 5.0
45.106295542 38.5765724182129 6.0
45.106295542 42.3174285888672 7.0
38.977055089 38.8600425720215 0.0
19.684728957 20.3399639129639 1.0
19.292326132 18.9574108123779 2.0
-8.6831116468 -8.02165222167969 3.0
10.609214475 10.5335245132446 4.0
28.367840614 28.3295516967773 5.0
38.977055099 38.544677734375 6.0
38.977055099 38.7534713745117 7.0
38.935365873 37.7276420593262 0.0
21.088579762 19.6334953308105 1.0
17.846786112 19.7528839111328 2.0
-3.2985252296 -4.86309766769409 3.0
14.548260873 14.5714502334595 4.0
24.387105001 24.4193744659424 5.0
38.935365883 37.3412399291992 6.0
38.935365883 38.5599403381348 7.0
42.74140991 39.1374816894531 0.0
20.9458217 21.1816730499268 1.0
21.795588212 21.305233001709 2.0
-6.3794891462 -5.77197694778442 3.0
15.416099058 15.4234342575073 4.0
27.325310854 27.4654064178467 5.0
42.741409918 39.1037139892578 6.0
42.741409918 40.7676734924316 7.0
31.310794567 38.2226676940918 0.0
16.880318068 16.7188167572021 1.0
14.4304765 15.3694944381714 2.0
-5.6188756586 -6.29079008102417 3.0
8.8116008333 8.95686054229736 4.0
22.499193736 22.5843410491943 5.0
31.310794576 38.616527557373 6.0
31.310794576 33.2752838134766 7.0
};
\addlegendentry{$R^2$=0.973}
\end{axis}

\end{tikzpicture}
}}
    
    \caption{Model results using only the loss associated with nodal flow predictions in the 8-node network.}
    \label{fig:dummy_base_results}
\end{figure}


In the following part of this experiment, we incorporated losses associated with node and edge flows, the gas balance, and the Weymouth equation. The hyperparameter optimization for this setup yielded the following best parameters: $N channels=17$, $N layers =1$, and $N dense =4$. These settings resulted in a total loss of 20.670, with the individual losses being a node loss of 3.000, an edge loss of 11.354, a balance loss of 2.724, and a Weymouth equation loss of 3.592.

As shown in \cref{fig:results_nonlineal_dummy_node_base_f_bal_wey}, the behavior of the node flow predictions remained consistent with the previous experiments, with an $R^2$ of 0.983. The model continued to accurately capture the gas injection patterns at the nodes.

However, the prediction accuracy for edge flows showed a notable deterioration, as seen in \cref{fig:results_nonlineal_dummy_edge_base_f_bal_wey}. The $R^2$ value for edge flow predictions dropped to 0.952. This decrease in performance is primarily due to the difficulties encountered in predicting flows along edges 1, 2, 6, and 7. Edges 1 and 2 correspond to pipelines that are part of a closed path in the network, while edges 6 and 7 correspond to compressors. These complexities in the network configuration likely contributed to the reduction in predictive accuracy for these specific edges.

\begin{figure}
    \centering
    \setlength\figurewidth{.53\textwidth}        
    \setlength\figureheight{0.36\textwidth} 
    \subfloat[Actual vs predicted nodal flows.] 
    {\label{fig:results_nonlineal_dummy_node_base_f_bal_wey}\resizebox{\figurewidth}{\figureheight}{% This file was created with tikzplotlib v0.10.1.
\begin{tikzpicture}

\definecolor{darkgray176}{RGB}{176,176,176}
\definecolor{lightgray204}{RGB}{204,204,204}

\begin{axis}[
colorbar,
colorbar style={ylabel={node id}},
colormap={mymap}{[1pt]
 rgb(0pt)=(0.12156862745098,0.466666666666667,0.705882352941177);
  rgb(1pt)=(1,0.498039215686275,0.0549019607843137);
  rgb(2pt)=(0.172549019607843,0.627450980392157,0.172549019607843);
  rgb(3pt)=(0.83921568627451,0.152941176470588,0.156862745098039);
  rgb(4pt)=(0.580392156862745,0.403921568627451,0.741176470588235);
  rgb(5pt)=(0.549019607843137,0.337254901960784,0.294117647058824);
  rgb(6pt)=(0.890196078431372,0.466666666666667,0.76078431372549);
  rgb(7pt)=(0.498039215686275,0.498039215686275,0.498039215686275);
  rgb(8pt)=(0.737254901960784,0.741176470588235,0.133333333333333);
  rgb(9pt)=(0.0901960784313725,0.745098039215686,0.811764705882353)
},
legend cell align={left},
legend style={
  fill opacity=0.8,
  draw opacity=1,
  text opacity=1,
  at={(0.03,0.97)},
  anchor=north west,
  draw=lightgray204
},
point meta max=7,
point meta min=0,
tick align=outside,
tick pos=left,
title={},
x grid style={darkgray176},
xlabel={True},
xmajorgrids,
xmin=-2.43954548935, xmax=51.23045527635,
xtick={0,10,20,30,40,50}, 
xticklabels={0,10,20,30,40,$f_n$},
xtick style={color=black},
y grid style={darkgray176},
ylabel={Predicted},
ymajorgrids,
ymin=-3.71705335378647, ymax=40.9747516512871,
ytick={0,10,20,30,40}, 
yticklabels={0,10,20,30,$f_n$},
ytick style={color=black}
]
\addplot [
  colormap={mymap}{[1pt]
 rgb(0pt)=(0.12156862745098,0.466666666666667,0.705882352941177);
  rgb(1pt)=(1,0.498039215686275,0.0549019607843137);
  rgb(2pt)=(0.172549019607843,0.627450980392157,0.172549019607843);
  rgb(3pt)=(0.83921568627451,0.152941176470588,0.156862745098039);
  rgb(4pt)=(0.580392156862745,0.403921568627451,0.741176470588235);
  rgb(5pt)=(0.549019607843137,0.337254901960784,0.294117647058824);
  rgb(6pt)=(0.890196078431372,0.466666666666667,0.76078431372549);
  rgb(7pt)=(0.498039215686275,0.498039215686275,0.498039215686275);
  rgb(8pt)=(0.737254901960784,0.741176470588235,0.133333333333333);
  rgb(9pt)=(0.0901960784313725,0.745098039215686,0.811764705882353)
},
  only marks,
  scatter,
  scatter src=explicit
]
table [x=x, y=y, meta=colordata]{%
x  y  colordata
39.565898645 37.2002334594727 0.0
0 0.169259428977966 1.0
0 -0.287830591201782 2.0
0 0.0633044242858887 3.0
0 0.701341867446899 4.0
0 -0.347691774368286 5.0
0 -0.173882246017456 6.0
0 1.38979136943817 7.0
42.743257181 37.4570503234863 0.0
0 -0.564504146575928 1.0
0 -0.218822479248047 2.0
0 -0.114206314086914 3.0
0 0.404315114021301 4.0
0 -0.266536474227905 5.0
0 -0.00889647006988525 6.0
0 0.1069495677948 7.0
39.367180751 38.0131530761719 0.0
0 0.109915494918823 1.0
0 0.394701361656189 2.0
0 -0.00679206848144531 3.0
0 0.188824534416199 4.0
0 -0.250398397445679 5.0
0 -0.0823309421539307 6.0
0 -0.105892300605774 7.0
39.605393107 38.0601196289062 0.0
0 0.16053032875061 1.0
0 -0.210066556930542 2.0
0 0.032062292098999 3.0
0 0.601880311965942 4.0
0 -0.182414531707764 5.0
0 -0.543129444122314 6.0
0 0.884384751319885 7.0
43.937345536 37.8871612548828 0.0
0 0.0309610366821289 1.0
0 -0.0989174842834473 2.0
0 -0.0100007057189941 3.0
0 0.470385551452637 4.0
0 -0.45317006111145 5.0
0 -0.419832706451416 6.0
0 0.36679220199585 7.0
31.061989594 36.8186836242676 0.0
0 0.236799597740173 1.0
0 -0.100792527198792 2.0
0 -0.180663824081421 3.0
0 0.0608124732971191 4.0
0 -0.369452953338623 5.0
0 0.0880031585693359 6.0
0 -0.354859113693237 7.0
36.357435273 37.8138961791992 0.0
0 -0.0249279737472534 1.0
0 -0.00934410095214844 2.0
0 0.424957990646362 3.0
0 0.290063738822937 4.0
0 -0.526926040649414 5.0
0 -0.685169458389282 6.0
0 0.556330680847168 7.0
38.444969617 37.033519744873 0.0
0 0.18208909034729 1.0
0 -0.282180547714233 2.0
0 -0.295518636703491 3.0
0 0.227351188659668 4.0
0 -0.705784559249878 5.0
0 0.0432302951812744 6.0
0 0.392435789108276 7.0
35.498620524 38.0038032531738 0.0
0 -0.594991445541382 1.0
0 -0.236080288887024 2.0
0 -0.240267276763916 3.0
0 0.19992995262146 4.0
0 -0.28917384147644 5.0
0 -0.165250778198242 6.0
0 0.0590331554412842 7.0
36.520998279 38.0805854797363 0.0
0 0.138278484344482 1.0
0 -0.305443525314331 2.0
0 -0.0594360828399658 3.0
0 0.0682514905929565 4.0
0 -0.212929010391235 5.0
0 -0.826930284500122 6.0
0 0.09907066822052 7.0
36.717272212 38.2084884643555 0.0
0 -0.0289492607116699 1.0
0 -0.269818067550659 2.0
0 0.254478693008423 3.0
0 0.636723756790161 4.0
0 -0.190581083297729 5.0
0 -0.749223947525024 6.0
0 1.22769320011139 7.0
32.629629006 38.5754737854004 0.0
0 -0.27463173866272 1.0
0 3.02901124954224 2.0
0 -0.156828165054321 3.0
0 0.267564535140991 4.0
0 -0.445150375366211 5.0
0 -0.0236518383026123 6.0
0 -0.323848485946655 7.0
37.75267434 37.3933334350586 0.0
0 0.22585916519165 1.0
0 -0.279909372329712 2.0
0 0.0331283807754517 3.0
0 -0.0457782745361328 4.0
0 -0.103767871856689 5.0
0 -0.151060581207275 6.0
0 -0.14793062210083 7.0
38.800291347 37.8225898742676 0.0
0 0.407272934913635 1.0
0 -0.287496328353882 2.0
0 -0.287057399749756 3.0
0 0.284600377082825 4.0
0 -0.628727912902832 5.0
0 -0.28039026260376 6.0
0 0.0975193977355957 7.0
38.252729618 37.5729293823242 0.0
0 -0.113675355911255 1.0
0 -0.255944490432739 2.0
0 0.0468313694000244 3.0
0 0.513723134994507 4.0
0 -0.47474479675293 5.0
0 -0.0117604732513428 6.0
0 0.811739683151245 7.0
43.273596047 37.7534561157227 0.0
0 -0.415119409561157 1.0
0 -0.531493425369263 2.0
0 -0.249844312667847 3.0
0 0.437668323516846 4.0
0 -0.423386335372925 5.0
0 -0.460921764373779 6.0
0 0.0678699016571045 7.0
34.027431486 37.8611373901367 0.0
0 0.234125137329102 1.0
0 -0.227123498916626 2.0
0 -0.125092625617981 3.0
0 0.53302001953125 4.0
0 -0.563767910003662 5.0
0 0.111242532730103 6.0
0 0.0460461378097534 7.0
41.154171391 37.5525779724121 0.0
0 -0.0613765716552734 1.0
0 -0.2198725938797 2.0
0 -0.203933954238892 3.0
0 0.716683149337769 4.0
0 -0.480995178222656 5.0
0 -0.218364000320435 6.0
0 2.00977516174316 7.0
39.930826408 38.4148864746094 0.0
0 -0.00104761123657227 1.0
0 -0.209811091423035 2.0
0 -0.142423033714294 3.0
0 0.691585421562195 4.0
0 -0.543980360031128 5.0
0 -0.0504612922668457 6.0
0 1.5702075958252 7.0
45.969703631 37.7824287414551 0.0
0 0.0189210176467896 1.0
0 -0.163213491439819 2.0
0 -0.176523208618164 3.0
0 0.622569441795349 4.0
0 -0.310530424118042 5.0
0 0.148032188415527 6.0
0 1.09274435043335 7.0
38.398120221 37.9468841552734 0.0
0 -0.099024772644043 1.0
0 -0.241862058639526 2.0
0 -0.537126779556274 3.0
0 0.747195899486542 4.0
0 -0.768521547317505 5.0
0 0.408243656158447 6.0
0 2.29147148132324 7.0
28.366017306 37.7088088989258 0.0
0 -0.103093862533569 1.0
0 -0.0715980529785156 2.0
0 -0.201879620552063 3.0
0 0.0942702293395996 4.0
0 -0.445433855056763 5.0
0 -0.37946605682373 6.0
0 -0.754090070724487 7.0
39.079818979 37.054874420166 0.0
0 0.285246849060059 1.0
0 -0.0196648836135864 2.0
0 0.0566463470458984 3.0
0 0.42071807384491 4.0
0 -0.267547130584717 5.0
0 -0.251262784004211 6.0
0 -0.139453887939453 7.0
40.466329383 38.1843872070312 0.0
0 0.233215093612671 1.0
0 -0.194013714790344 2.0
0 0.506468772888184 3.0
0 0.476369857788086 4.0
0 -0.125925540924072 5.0
0 -0.322431802749634 6.0
0 0.365741372108459 7.0
38.293116853 37.9485092163086 0.0
0 0.102312088012695 1.0
0 -0.358851671218872 2.0
0 0.306449174880981 3.0
0 0.278098106384277 4.0
0 -0.219168424606323 5.0
0 -0.375505924224854 6.0
0 -0.263507127761841 7.0
43.346089497 38.5111808776855 0.0
0 -0.106931924819946 1.0
0 0.0732955932617188 2.0
0 -0.0180624723434448 3.0
0 0.752818405628204 4.0
0 -0.0537310838699341 5.0
0 -0.159572601318359 6.0
0 0.644774913787842 7.0
34.547871154 38.1396369934082 0.0
0 -0.0819689035415649 1.0
0 -0.222944498062134 2.0
0 -0.00651192665100098 3.0
0 0.278984308242798 4.0
0 -0.66294264793396 5.0
0 0.0219395160675049 6.0
0 0.149653673171997 7.0
41.927050143 36.7811889648438 0.0
0 0.205381870269775 1.0
0 -0.252994418144226 2.0
0 -0.163344621658325 3.0
0 0.886784315109253 4.0
0 -0.409445524215698 5.0
0 -0.346015214920044 6.0
0 0.733158707618713 7.0
44.548236377 37.9944343566895 0.0
0 0.360582828521729 1.0
0 -0.129225969314575 2.0
0 -0.128922581672668 3.0
0 0.777518272399902 4.0
0 -0.364200830459595 5.0
0 -0.291475772857666 6.0
0 1.35816466808319 7.0
34.958415487 36.9152755737305 0.0
0 -0.28176212310791 1.0
0 -0.224816560745239 2.0
0 -0.0928323268890381 3.0
0 0.361677527427673 4.0
0 -0.718587636947632 5.0
0 0.0897564888000488 6.0
0 0.814304232597351 7.0
41.619773298 38.2806091308594 0.0
0 -0.078108549118042 1.0
0 -0.242096304893494 2.0
0 -0.251641511917114 3.0
0 0.393022894859314 4.0
0 -0.443493127822876 5.0
0 0.169111967086792 6.0
0 0.964947819709778 7.0
35.768623912 37.4543991088867 0.0
0 0.281252861022949 1.0
0 -0.279979467391968 2.0
0 -0.103221416473389 3.0
0 0.254816055297852 4.0
0 -0.437325000762939 5.0
0 -0.289105415344238 6.0
0 -0.25270140171051 7.0
36.03587308 38.4725112915039 0.0
0 -0.167118549346924 1.0
0 -0.197663426399231 2.0
0 -0.00300514698028564 3.0
0 0.729452967643738 4.0
0 -0.411173820495605 5.0
0 -0.185838460922241 6.0
0 1.39223408699036 7.0
42.671159584 37.8159523010254 0.0
0 -0.121634483337402 1.0
0 -0.0406523942947388 2.0
0 -0.124996423721313 3.0
0 0.784656465053558 4.0
0 -0.232450723648071 5.0
0 -0.313619375228882 6.0
0 2.25341725349426 7.0
32.270518391 38.0477104187012 0.0
0 -0.0846718549728394 1.0
0 -0.207898736000061 2.0
0 0.00776076316833496 3.0
0 0.528941512107849 4.0
0 -0.372263431549072 5.0
0 0.091058611869812 6.0
0 0.238791227340698 7.0
36.239834941 38.7916297912598 0.0
0 -0.0661487579345703 1.0
0 -0.291295528411865 2.0
0 -0.298757553100586 3.0
0 0.33805787563324 4.0
0 -0.296345710754395 5.0
0 0.0698119401931763 6.0
0 -0.27261757850647 7.0
38.292005181 38.3072814941406 0.0
0 0.218194961547852 1.0
0 -0.20565927028656 2.0
0 -0.0308597087860107 3.0
0 0.617907524108887 4.0
0 -0.446989297866821 5.0
0 0.123205184936523 6.0
0 0.387466073036194 7.0
41.142171126 37.4263038635254 0.0
0 0.128543376922607 1.0
0 -0.259983658790588 2.0
0 0.347197294235229 3.0
0 0.626899242401123 4.0
0 -0.0338377952575684 5.0
0 0.0709964036941528 6.0
0 1.5746808052063 7.0
30.660234751 37.9779396057129 0.0
0 0.126171588897705 1.0
0 -0.196113109588623 2.0
0 0.0847322940826416 3.0
0 0.201912641525269 4.0
0 -0.189703702926636 5.0
0 -0.291376352310181 6.0
0 -0.118277072906494 7.0
42.776716986 38.3894653320312 0.0
0 -0.149407505989075 1.0
0 -0.15449321269989 2.0
0 -0.156123399734497 3.0
0 0.739207029342651 4.0
0 -0.092564582824707 5.0
0 -0.238792419433594 6.0
0 0.037752628326416 7.0
39.136657955 38.646369934082 0.0
0 -0.210787653923035 1.0
0 -0.415425062179565 2.0
0 0.0127255916595459 3.0
0 0.599132776260376 4.0
0 -0.158288836479187 5.0
0 0.176143169403076 6.0
0 0.0326617956161499 7.0
40.593075664 38.2094993591309 0.0
0 0.0590794086456299 1.0
0 -0.232085347175598 2.0
0 -0.0500190258026123 3.0
0 0.707081913948059 4.0
0 -0.669234037399292 5.0
0 -0.197407722473145 6.0
0 0.505205750465393 7.0
38.455960057 38.2412452697754 0.0
0 -0.220564484596252 1.0
0 -0.24339485168457 2.0
0 -0.255217552185059 3.0
0 0.878998279571533 4.0
0 -0.393985748291016 5.0
0 -0.147870063781738 6.0
0 1.16284871101379 7.0
40.029822307 37.9086837768555 0.0
0 -0.00928378105163574 1.0
0 1.10280227661133 2.0
0 -0.171652555465698 3.0
0 0.461877584457397 4.0
0 -0.260985374450684 5.0
0 -0.0874053239822388 6.0
0 -0.0160727500915527 7.0
39.721089806 37.8744926452637 0.0
0 0.355368375778198 1.0
0 -0.387604713439941 2.0
0 0.141079425811768 3.0
0 0.872663557529449 4.0
0 0.105510354042053 5.0
0 -1.00087857246399 6.0
0 1.61192405223846 7.0
46.012802781 37.9997940063477 0.0
0 0.0809178352355957 1.0
0 -0.303426027297974 2.0
0 -0.49855899810791 3.0
0 0.654655933380127 4.0
0 -0.213323831558228 5.0
0 0.0454779863357544 6.0
0 0.562016248703003 7.0
43.791041158 37.6419792175293 0.0
0 0.121715784072876 1.0
0 1.26866388320923 2.0
0 -0.192611694335938 3.0
0 0.802588045597076 4.0
0 -0.419427394866943 5.0
0 -0.251010417938232 6.0
0 2.48242497444153 7.0
31.257332424 37.4250946044922 0.0
0 0.0881016254425049 1.0
0 -0.236838102340698 2.0
0 0.0862808227539062 3.0
0 0.285619020462036 4.0
0 -0.644461870193481 5.0
0 -0.227165341377258 6.0
0 -0.319159746170044 7.0
38.98847291 37.0705184936523 0.0
0 0.254436254501343 1.0
0 -0.164412379264832 2.0
0 0.0657204389572144 3.0
0 0.0437488555908203 4.0
0 -0.0892298221588135 5.0
0 -0.193382620811462 6.0
0 -0.279639959335327 7.0
38.691218499 38.1063842773438 0.0
0 0.0654633045196533 1.0
0 -0.172278046607971 2.0
0 -0.169105768203735 3.0
0 0.527813673019409 4.0
0 -0.0672010183334351 5.0
0 -0.136846542358398 6.0
0 0.469297170639038 7.0
39.033211971 37.1926803588867 0.0
0 0.0128399133682251 1.0
0 -0.290940761566162 2.0
0 0.0776848793029785 3.0
0 0.410050749778748 4.0
0 -0.177785754203796 5.0
0 -0.68099045753479 6.0
0 0.700372934341431 7.0
37.697547813 36.7911567687988 0.0
0 0.251331090927124 1.0
0 -0.107108950614929 2.0
0 0.0784635543823242 3.0
0 0.0997118949890137 4.0
0 -0.630835056304932 5.0
0 0.155158996582031 6.0
0 -0.300743341445923 7.0
35.277541339 37.3313446044922 0.0
0 -0.307245254516602 1.0
0 -0.174744606018066 2.0
0 -0.0172789096832275 3.0
0 0.318334817886353 4.0
0 -0.341958045959473 5.0
0 -0.231449842453003 6.0
0 0.505506277084351 7.0
36.119763966 38.0068588256836 0.0
0 -0.131386399269104 1.0
0 -0.187229037284851 2.0
0 -0.194747447967529 3.0
0 -0.144355058670044 4.0
0 -0.115407228469849 5.0
0 -0.417536020278931 6.0
0 -0.411046743392944 7.0
33.14490305 38.8048934936523 0.0
0 -0.256235837936401 1.0
0 -0.318185567855835 2.0
0 -0.1818608045578 3.0
0 0.0477951765060425 4.0
0 -0.605113983154297 5.0
0 -0.190538883209229 6.0
0 -0.364505529403687 7.0
34.486800814 37.2350311279297 0.0
0 0.0449519157409668 1.0
0 -0.228806734085083 2.0
0 0.115298509597778 3.0
0 0.67801034450531 4.0
0 -0.360451698303223 5.0
0 -0.0727962255477905 6.0
0 2.03513097763062 7.0
40.933468207 38.3564376831055 0.0
0 0.213807821273804 1.0
0 -0.293026208877563 2.0
0 0.0499045848846436 3.0
0 0.159296154975891 4.0
0 -0.383970499038696 5.0
0 0.0886210203170776 6.0
0 -0.0842075347900391 7.0
35.993417928 38.0652046203613 0.0
0 0.112274050712585 1.0
0 -0.231756687164307 2.0
0 0.207550287246704 3.0
0 0.369506120681763 4.0
0 -0.764149904251099 5.0
0 -0.0713533163070679 6.0
0 0.88979172706604 7.0
39.331666923 38.2533149719238 0.0
0 -0.000711202621459961 1.0
0 -0.197830080986023 2.0
0 -0.297441720962524 3.0
0 0.506584048271179 4.0
0 -0.506460428237915 5.0
0 0.0119237899780273 6.0
0 0.115601778030396 7.0
37.559548496 38.3615989685059 0.0
0 -0.0107724666595459 1.0
0 -0.282819032669067 2.0
0 -0.158299207687378 3.0
0 0.467667102813721 4.0
0 -0.562494277954102 5.0
0 -0.237583875656128 6.0
0 0.502041459083557 7.0
41.796902482 38.270881652832 0.0
0 -0.337889909744263 1.0
0 -0.186974167823792 2.0
0 -0.233812093734741 3.0
0 0.815750062465668 4.0
0 -0.644329786300659 5.0
0 -0.150741338729858 6.0
0 1.04526376724243 7.0
35.679590823 37.4580497741699 0.0
0 0.440304279327393 1.0
0 -0.11521315574646 2.0
0 -0.0468738079071045 3.0
0 0.455806970596313 4.0
0 -0.391058683395386 5.0
0 -0.283623456954956 6.0
0 -0.200713515281677 7.0
33.227547292 35.4333305358887 0.0
0 0.187448382377625 1.0
0 -0.211544752120972 2.0
0 -0.0847647190093994 3.0
0 0.31676459312439 4.0
0 -0.716625690460205 5.0
0 0.209605932235718 6.0
0 -0.148748159408569 7.0
28.008071739 37.5513954162598 0.0
0 -0.228009223937988 1.0
0 -0.227761507034302 2.0
0 -0.400337696075439 3.0
0 0.183823943138123 4.0
0 -0.30519437789917 5.0
0 -0.417527675628662 6.0
0 0.140562057495117 7.0
33.478498841 37.6189384460449 0.0
0 0.17590856552124 1.0
0 -0.343067646026611 2.0
0 -0.253837108612061 3.0
0 0.546959280967712 4.0
0 -0.545683622360229 5.0
0 0.118918657302856 6.0
0 0.222257852554321 7.0
33.12294682 38.1232376098633 0.0
0 -0.17867374420166 1.0
0 0.52698814868927 2.0
0 -0.227580308914185 3.0
0 0.269090294837952 4.0
0 -0.550087690353394 5.0
0 -0.472012281417847 6.0
0 -0.190004348754883 7.0
34.970228384 38.0959053039551 0.0
0 0.408636093139648 1.0
0 -0.339219093322754 2.0
0 -0.160905361175537 3.0
0 0.372968196868896 4.0
0 -0.155856132507324 5.0
0 -0.454294681549072 6.0
0 0.0878983736038208 7.0
36.637966041 37.7279052734375 0.0
0 -0.118524551391602 1.0
0 -0.200144052505493 2.0
0 -0.218913078308105 3.0
0 0.140674591064453 4.0
0 -0.408346652984619 5.0
0 -0.176931858062744 6.0
0 -0.271597146987915 7.0
38.712447295 37.0502815246582 0.0
0 0.113901615142822 1.0
0 0.194762229919434 2.0
0 -0.186304330825806 3.0
0 0.776475429534912 4.0
0 -0.66611123085022 5.0
0 0.0606635808944702 6.0
0 1.15091609954834 7.0
29.08402242 37.786922454834 0.0
0 -0.408588886260986 1.0
0 -0.279032468795776 2.0
0 -0.172401905059814 3.0
0 0.111711144447327 4.0
0 -0.56746768951416 5.0
0 0.138999700546265 6.0
0 -0.592000722885132 7.0
40.741903011 37.8244705200195 0.0
0 0.0978879928588867 1.0
0 -0.218857765197754 2.0
0 0.0109368562698364 3.0
0 0.498151421546936 4.0
0 -0.389167547225952 5.0
0 -0.306290864944458 6.0
0 -0.224243521690369 7.0
44.423927978 36.0485458374023 0.0
0 0.316178560256958 1.0
0 -0.0417205095291138 2.0
0 -0.171687841415405 3.0
0 0.630297064781189 4.0
0 0.157346725463867 5.0
0 -0.318689107894897 6.0
0 1.15694212913513 7.0
37.279929011 38.8538017272949 0.0
0 -0.247755169868469 1.0
0 -0.363473176956177 2.0
0 0.088719367980957 3.0
0 0.460721015930176 4.0
0 -0.534109115600586 5.0
0 0.346329808235168 6.0
0 -0.0850760936737061 7.0
39.065196566 37.7703857421875 0.0
0 -0.32534384727478 1.0
0 -0.362853288650513 2.0
0 0.0791245698928833 3.0
0 0.493427515029907 4.0
0 -0.216928243637085 5.0
0 -0.174487113952637 6.0
0 0.0718992948532104 7.0
34.346712911 38.8670425415039 0.0
0 0.0299710035324097 1.0
0 -0.255981087684631 2.0
0 0.154043674468994 3.0
0 0.236307621002197 4.0
0 -0.110009074211121 5.0
0 -0.338704586029053 6.0
0 -0.306875944137573 7.0
44.832836202 37.8019561767578 0.0
0 -0.148072957992554 1.0
0 -0.195420265197754 2.0
0 0.007476806640625 3.0
0 0.58234715461731 4.0
0 -0.205502390861511 5.0
0 -0.69054102897644 6.0
0 -0.0763163566589355 7.0
37.693005916 38.4473609924316 0.0
0 -0.0419975519180298 1.0
0 -0.211871027946472 2.0
0 -0.203723073005676 3.0
0 0.0342875719070435 4.0
0 -0.513449668884277 5.0
0 0.0778001546859741 6.0
0 0.00392615795135498 7.0
35.516030206 37.953125 0.0
0 -0.166070699691772 1.0
0 -0.21449089050293 2.0
0 -0.0362746715545654 3.0
0 0.312258839607239 4.0
0 -0.424062490463257 5.0
0 -0.172282099723816 6.0
0 -0.242353558540344 7.0
45.449957523 37.7031288146973 0.0
0 -0.425414800643921 1.0
0 -0.262605905532837 2.0
0 -0.0357750654220581 3.0
0 0.623458743095398 4.0
0 -0.354610443115234 5.0
0 -0.173225522041321 6.0
0 1.78730642795563 7.0
36.52210397 37.2987670898438 0.0
0 0.486531734466553 1.0
0 -0.22829806804657 2.0
0 -0.338537454605103 3.0
0 0.194444537162781 4.0
0 -0.265234708786011 5.0
0 -0.431441783905029 6.0
0 -0.107329845428467 7.0
40.809333833 38.2978134155273 0.0
0 -0.0640097856521606 1.0
0 -0.0859607458114624 2.0
0 -0.21967077255249 3.0
0 0.746559917926788 4.0
0 0.0143009424209595 5.0
0 -0.12559175491333 6.0
0 1.30185031890869 7.0
43.708815025 37.885124206543 0.0
0 0.0246310234069824 1.0
0 -0.290473461151123 2.0
0 -0.193700671195984 3.0
0 0.379313707351685 4.0
0 -0.429344892501831 5.0
0 -0.105638980865479 6.0
0 0.155070543289185 7.0
35.746307575 37.7581634521484 0.0
0 0.278895616531372 1.0
0 -0.131361484527588 2.0
0 -0.349656820297241 3.0
0 0.37863028049469 4.0
0 -0.506231784820557 5.0
0 -0.452111721038818 6.0
0 0.919278740882874 7.0
32.565260407 36.6600685119629 0.0
0 0.201853275299072 1.0
0 -0.154062151908875 2.0
0 -0.233282446861267 3.0
0 0.0915858745574951 4.0
0 -0.185634970664978 5.0
0 -0.168465375900269 6.0
0 0.172224044799805 7.0
37.459437569 36.0472869873047 0.0
0 -0.157306313514709 1.0
0 -0.40479588508606 2.0
0 0.676112174987793 3.0
0 0.624497175216675 4.0
0 -0.493572950363159 5.0
0 0.158000707626343 6.0
0 0.954303622245789 7.0
41.600868777 38.0224342346191 0.0
0 0.0223286151885986 1.0
0 -0.238004565238953 2.0
0 -0.306910991668701 3.0
0 0.289011240005493 4.0
0 -0.463261604309082 5.0
0 -0.0755749940872192 6.0
0 0.0598933696746826 7.0
43.088648289 38.4936981201172 0.0
0 -0.105807781219482 1.0
0 -0.0291221141815186 2.0
0 -0.164589166641235 3.0
0 0.788821995258331 4.0
0 -0.0164649486541748 5.0
0 -0.244705080986023 6.0
0 0.891664743423462 7.0
30.268152677 37.9713401794434 0.0
0 0.199033260345459 1.0
0 -0.233491897583008 2.0
0 -0.120284080505371 3.0
0 0.0919866561889648 4.0
0 -0.741365194320679 5.0
0 -0.0194754600524902 6.0
0 -0.18975043296814 7.0
38.454045888 36.7144660949707 0.0
0 -0.00435554981231689 1.0
0 -0.22428560256958 2.0
0 -0.198359847068787 3.0
0 0.671595573425293 4.0
0 -0.384537220001221 5.0
0 -0.647875785827637 6.0
0 0.853360891342163 7.0
36.654056864 38.2402610778809 0.0
0 0.0632588863372803 1.0
0 0.00261342525482178 2.0
0 0.0470747947692871 3.0
0 0.495208501815796 4.0
0 -0.314559459686279 5.0
0 -0.379502773284912 6.0
0 -0.0655057430267334 7.0
36.990278128 38.5124969482422 0.0
0 -0.163190364837646 1.0
0 -0.128012537956238 2.0
0 -0.179603338241577 3.0
0 0.679524660110474 4.0
0 -0.538313627243042 5.0
0 -0.0226702690124512 6.0
0 0.457101225852966 7.0
39.45539179 37.9153327941895 0.0
0 0.11719822883606 1.0
0 -0.0304499864578247 2.0
0 -0.177050113677979 3.0
0 0.667137980461121 4.0
0 -0.211448907852173 5.0
0 -0.718790769577026 6.0
0 0.375157237052917 7.0
41.49009831 37.8272361755371 0.0
0 -0.212602615356445 1.0
0 -0.243842959403992 2.0
0 -0.231322765350342 3.0
0 0.301067113876343 4.0
0 -0.387824773788452 5.0
0 -0.459747076034546 6.0
0 -0.0580298900604248 7.0
42.195848764 38.1985282897949 0.0
0 0.144180059432983 1.0
0 -1.17367482185364 2.0
0 -0.340133666992188 3.0
0 0.73195219039917 4.0
0 -0.276169061660767 5.0
0 -0.3075270652771 6.0
0 0.352790832519531 7.0
36.580423956 38.2524375915527 0.0
0 -0.12670886516571 1.0
0 -0.0990722179412842 2.0
0 -0.28669810295105 3.0
0 0.782248139381409 4.0
0 -0.362535715103149 5.0
0 -0.248741865158081 6.0
0 0.0430417060852051 7.0
41.112612846 38.2443580627441 0.0
0 -0.575524568557739 1.0
0 -1.68560767173767 2.0
0 -0.440459489822388 3.0
0 0.601357221603394 4.0
0 -0.232683300971985 5.0
0 -0.435184240341187 6.0
0 0.197087168693542 7.0
41.165032305 35.6932983398438 0.0
0 0.39403223991394 1.0
0 -0.205694198608398 2.0
0 -0.180716753005981 3.0
0 0.817940473556519 4.0
0 -0.395382165908813 5.0
0 -0.752209186553955 6.0
0 1.13123226165771 7.0
28.236329769 38.0303726196289 0.0
0 0.223541736602783 1.0
0 -0.203654527664185 2.0
0 -0.190118789672852 3.0
0 0.443485736846924 4.0
0 -0.274266481399536 5.0
0 0.0545015335083008 6.0
0 -0.574124097824097 7.0
46.563722235 38.1171379089355 0.0
0 -0.115152359008789 1.0
0 -0.0835323333740234 2.0
0 -0.0736225843429565 3.0
0 0.492484450340271 4.0
0 -0.606148719787598 5.0
0 -0.0312373638153076 6.0
0 1.63797974586487 7.0
38.976039446 37.3001518249512 0.0
0 -0.0724996328353882 1.0
0 -0.13895058631897 2.0
0 -0.199964284896851 3.0
0 0.361873269081116 4.0
0 -0.250664591789246 5.0
0 -0.314976930618286 6.0
0 0.411396026611328 7.0
43.902220165 37.8490180969238 0.0
0 0.322252511978149 1.0
0 -0.0589005947113037 2.0
0 -0.021243691444397 3.0
0 0.584638595581055 4.0
0 -0.267582654953003 5.0
0 -0.190856337547302 6.0
0 1.91274344921112 7.0
37.104870385 37.4096984863281 0.0
0 -0.014404296875 1.0
0 -0.249053835868835 2.0
0 -0.268474340438843 3.0
0 0.472056865692139 4.0
0 -0.593343496322632 5.0
0 0.351828455924988 6.0
0 0.851223587989807 7.0
48.094555135 35.1621360778809 0.0
0 -0.320187568664551 1.0
0 1.00059175491333 2.0
0 -0.465060234069824 3.0
0 0.560883402824402 4.0
0 -0.104988813400269 5.0
0 -0.142831325531006 6.0
0 0.966924071311951 7.0
39.863550951 37.3979759216309 0.0
0 0.181107878684998 1.0
0 -0.115323901176453 2.0
0 -0.420963287353516 3.0
0 0.438129425048828 4.0
0 -0.839054584503174 5.0
0 -0.100161790847778 6.0
0 0.273425817489624 7.0
38.158424706 38.0839614868164 0.0
0 -0.393306016921997 1.0
0 -0.191314220428467 2.0
0 -0.194749116897583 3.0
0 0.727922201156616 4.0
0 -0.387811183929443 5.0
0 -0.710214376449585 6.0
0 3.13079333305359 7.0
37.671552644 38.0547027587891 0.0
0 -0.182933807373047 1.0
0 -0.180825591087341 2.0
0 0.0170788764953613 3.0
0 0.481404304504395 4.0
0 -0.372659921646118 5.0
0 -0.151788473129272 6.0
0 0.268154621124268 7.0
38.99000687 38.1053276062012 0.0
0 -0.317291021347046 1.0
0 -0.329813241958618 2.0
0 -0.527798175811768 3.0
0 0.585848093032837 4.0
0 -0.543547391891479 5.0
0 0.242907166481018 6.0
0 1.35014855861664 7.0
32.824279197 38.3994750976562 0.0
0 0.142992854118347 1.0
0 0.00302422046661377 2.0
0 -0.236001372337341 3.0
0 0.207676887512207 4.0
0 -0.788823127746582 5.0
0 1.00016152858734 6.0
0 0.515289902687073 7.0
39.752392398 37.814998626709 0.0
0 -0.41579794883728 1.0
0 -0.191042900085449 2.0
0 -0.0516337156295776 3.0
0 0.156049966812134 4.0
0 -0.267496347427368 5.0
0 -0.0309159755706787 6.0
0 0.181396007537842 7.0
36.014997988 38.6865158081055 0.0
0 -0.0833034515380859 1.0
0 -0.215163707733154 2.0
0 -0.442822456359863 3.0
0 0.234899997711182 4.0
0 -0.559880971908569 5.0
0 -0.0211020708084106 6.0
0 0.181499719619751 7.0
42.307205606 38.2676124572754 0.0
0 -0.371938943862915 1.0
0 -0.191350817680359 2.0
0 -0.453990697860718 3.0
0 0.403597593307495 4.0
0 -0.0505291223526001 5.0
0 -0.749059915542603 6.0
0 0.444090008735657 7.0
41.624369934 38.391918182373 0.0
0 -0.159429550170898 1.0
0 -0.025769829750061 2.0
0 -0.0126388072967529 3.0
0 0.610337018966675 4.0
0 -0.178430080413818 5.0
0 -0.100505232810974 6.0
0 -0.0299859046936035 7.0
40.927860926 37.2393417358398 0.0
0 -0.0430448055267334 1.0
0 -0.20372211933136 2.0
0 -0.346153736114502 3.0
0 0.522186875343323 4.0
0 -0.420128583908081 5.0
0 -0.0751999616622925 6.0
0 1.12843024730682 7.0
47.441321812 37.930290222168 0.0
0 -0.134860038757324 1.0
0 -0.196523070335388 2.0
0 -0.268415212631226 3.0
0 0.569846153259277 4.0
0 -0.337485074996948 5.0
0 -0.746578693389893 6.0
0 0.364472627639771 7.0
39.379505619 37.9247169494629 0.0
0 0.124060273170471 1.0
0 -0.103296518325806 2.0
0 0.00258779525756836 3.0
0 0.714645862579346 4.0
0 -0.537508249282837 5.0
0 -0.212826251983643 6.0
0 0.950432777404785 7.0
42.095557299 38.4975166320801 0.0
0 -0.0162560939788818 1.0
0 -0.328069448471069 2.0
0 -0.355610609054565 3.0
0 0.577233672142029 4.0
0 -0.263373851776123 5.0
0 -0.205716371536255 6.0
0 1.59658932685852 7.0
35.151188225 38.5512580871582 0.0
0 -0.233904719352722 1.0
0 -0.311061143875122 2.0
0 -0.0664944648742676 3.0
0 0.399610757827759 4.0
0 0.18767237663269 5.0
0 -0.493221998214722 6.0
0 -0.0551391839981079 7.0
37.713935371 37.9170951843262 0.0
0 0.429975032806396 1.0
0 -0.279746770858765 2.0
0 -0.19325578212738 3.0
0 0.51653265953064 4.0
0 -0.578880786895752 5.0
0 -0.112882852554321 6.0
0 1.07182359695435 7.0
39.66774326 38.0369491577148 0.0
0 0.164271831512451 1.0
0 -0.194999098777771 2.0
0 -0.146594524383545 3.0
0 0.569705724716187 4.0
0 -0.355697393417358 5.0
0 -0.0985374450683594 6.0
0 -0.0333777666091919 7.0
41.240556194 37.3603286743164 0.0
0 0.0708317756652832 1.0
0 -0.103000164031982 2.0
0 -0.217048406600952 3.0
0 0.5730801820755 4.0
0 -0.439975500106812 5.0
0 -0.154386043548584 6.0
0 1.16957020759583 7.0
40.3201946 37.6391258239746 0.0
0 0.0594854354858398 1.0
0 1.2462922334671 2.0
0 -0.170354962348938 3.0
0 1.47491872310638 4.0
0 -0.32731294631958 5.0
0 0.306501150131226 6.0
0 1.33862972259521 7.0
39.561151059 37.8627166748047 0.0
0 0.163033604621887 1.0
0 -0.172850966453552 2.0
0 -0.413043737411499 3.0
0 0.523203611373901 4.0
0 -0.346947431564331 5.0
0 -0.603073835372925 6.0
0 -0.153909206390381 7.0
43.566431947 37.696460723877 0.0
0 -0.488623142242432 1.0
0 2.05566382408142 2.0
0 -0.173099756240845 3.0
0 0.725049018859863 4.0
0 -0.212450742721558 5.0
0 -0.357983589172363 6.0
0 0.25268816947937 7.0
37.929366562 38.0525894165039 0.0
0 -0.146291136741638 1.0
0 -0.157551765441895 2.0
0 -0.215641498565674 3.0
0 0.221796274185181 4.0
0 -0.579155921936035 5.0
0 0.134150981903076 6.0
0 0.0387586355209351 7.0
39.745668641 37.8475608825684 0.0
0 0.157349586486816 1.0
0 -0.35772705078125 2.0
0 0.175487279891968 3.0
0 0.0908188819885254 4.0
0 -0.304956197738647 5.0
0 -0.0755524635314941 6.0
0 -0.445888042449951 7.0
37.173794625 38.2042350769043 0.0
0 -0.420426607131958 1.0
0 -0.276220083236694 2.0
0 0.202169895172119 3.0
0 0.168111562728882 4.0
0 -0.353787660598755 5.0
0 -0.601686954498291 6.0
0 0.217365503311157 7.0
27.124866015 37.742301940918 0.0
0 -0.0649454593658447 1.0
0 -0.266653537750244 2.0
0 0.0126903057098389 3.0
0 -0.137218475341797 4.0
0 -0.370476007461548 5.0
0 -0.812102317810059 6.0
0 -0.153939843177795 7.0
27.411746904 37.727897644043 0.0
0 -0.222171902656555 1.0
0 2.40911555290222 2.0
0 -0.156563282012939 3.0
0 0.151611089706421 4.0
0 -0.072296142578125 5.0
0 -0.788114070892334 6.0
0 0.130670547485352 7.0
38.866617317 38.3996849060059 0.0
0 -0.293075561523438 1.0
0 -0.204826354980469 2.0
0 -0.160280466079712 3.0
0 0.311803460121155 4.0
0 -0.706974983215332 5.0
0 0.0995534658432007 6.0
0 0.0202170610427856 7.0
42.383057354 38.2683219909668 0.0
0 -0.108166694641113 1.0
0 0.00630033016204834 2.0
0 -0.133476376533508 3.0
0 0.575891971588135 4.0
0 -0.557399988174438 5.0
0 -0.0148487091064453 6.0
0 0.670676827430725 7.0
47.643751036 37.626033782959 0.0
0 -0.105839490890503 1.0
0 -0.24412477016449 2.0
0 -0.365706443786621 3.0
0 0.747549951076508 4.0
0 -0.00269722938537598 5.0
0 -0.39321231842041 6.0
0 1.16409766674042 7.0
38.439399957 38.276309967041 0.0
0 -0.140192151069641 1.0
0 -0.233856439590454 2.0
0 -0.125805974006653 3.0
0 0.400934457778931 4.0
0 -0.54006028175354 5.0
0 0.41241979598999 6.0
0 0.0536650419235229 7.0
40.263562371 38.1368598937988 0.0
0 -0.00548505783081055 1.0
0 -0.285221338272095 2.0
0 -0.191219210624695 3.0
0 0.713197231292725 4.0
0 -0.628902673721313 5.0
0 0.0258686542510986 6.0
0 0.144718527793884 7.0
41.519528397 36.9954719543457 0.0
0 -0.204070091247559 1.0
0 -0.232495546340942 2.0
0 -0.230618238449097 3.0
0 0.187067747116089 4.0
0 -0.521304368972778 5.0
0 0.0421425104141235 6.0
0 0.056272029876709 7.0
40.414570474 37.6742630004883 0.0
0 0.249840021133423 1.0
0 -0.202081322669983 2.0
0 -0.27094578742981 3.0
0 0.479709625244141 4.0
0 -0.210130214691162 5.0
0 -0.593431711196899 6.0
0 0.603255271911621 7.0
37.834181221 38.12353515625 0.0
0 0.202781915664673 1.0
0 -0.23875892162323 2.0
0 0.042421817779541 3.0
0 0.465792179107666 4.0
0 -0.362313508987427 5.0
0 -0.222111701965332 6.0
0 -0.101075530052185 7.0
37.201121556 38.2207489013672 0.0
0 0.210466384887695 1.0
0 -0.052794337272644 2.0
0 -0.0410865545272827 3.0
0 0.317895650863647 4.0
0 -0.196887254714966 5.0
0 -0.787958145141602 6.0
0 -0.282079458236694 7.0
42.94444514 38.1734352111816 0.0
0 0.101837873458862 1.0
0 -0.273547649383545 2.0
0 0.0924530029296875 3.0
0 0.604528188705444 4.0
0 -0.21881902217865 5.0
0 -0.545419692993164 6.0
0 1.44555866718292 7.0
47.309600034 38.2109489440918 0.0
0 -0.284559965133667 1.0
0 -0.095078706741333 2.0
0 0.0540890693664551 3.0
0 0.64516282081604 4.0
0 -0.183562994003296 5.0
0 -0.529608249664307 6.0
0 2.38354730606079 7.0
34.703660747 38.2554817199707 0.0
0 -0.391480684280396 1.0
0 -0.272840738296509 2.0
0 0.117748975753784 3.0
0 0.0425287485122681 4.0
0 -0.311640977859497 5.0
0 -0.521694660186768 6.0
0 -0.230281591415405 7.0
39.355679833 36.7197189331055 0.0
0 0.348154306411743 1.0
0 -0.0251661539077759 2.0
0 -0.146993637084961 3.0
0 0.796976268291473 4.0
0 -0.223500728607178 5.0
0 -0.210777521133423 6.0
0 0.154900431632996 7.0
36.760838135 38.1111221313477 0.0
0 -0.275828123092651 1.0
0 -0.241212606430054 2.0
0 0.0796482563018799 3.0
0 0.280863165855408 4.0
0 -0.251239538192749 5.0
0 -0.298120021820068 6.0
0 -0.107448101043701 7.0
45.817767369 35.7727851867676 0.0
0 0.0142240524291992 1.0
0 0.897656440734863 2.0
0 -0.0181832313537598 3.0
0 0.541900157928467 4.0
0 -0.417888641357422 5.0
0 -0.434842348098755 6.0
0 1.27660191059113 7.0
37.456387944 38.3562965393066 0.0
0 -0.0527043342590332 1.0
0 -0.421114206314087 2.0
0 0.0698046684265137 3.0
0 0.784707903862 4.0
0 -0.128528594970703 5.0
0 -0.275948762893677 6.0
0 0.603277683258057 7.0
36.979299113 38.2336235046387 0.0
0 -0.176790714263916 1.0
0 -0.256789207458496 2.0
0 -0.0261688232421875 3.0
0 0.00695478916168213 4.0
0 -0.361829996109009 5.0
0 -0.113266706466675 6.0
0 -0.246804237365723 7.0
44.357392157 38.2335395812988 0.0
0 -0.237964153289795 1.0
0 -0.239361763000488 2.0
0 -0.0125906467437744 3.0
0 0.583804130554199 4.0
0 -0.197205185890198 5.0
0 -0.660237073898315 6.0
0 0.42578399181366 7.0
36.812079299 37.9328498840332 0.0
0 -0.40088152885437 1.0
0 -0.23781430721283 2.0
0 -0.0828559398651123 3.0
0 0.712757706642151 4.0
0 -0.614667654037476 5.0
0 0.0789783000946045 6.0
0 0.813002467155457 7.0
43.172254743 37.0509452819824 0.0
0 0.0406835079193115 1.0
0 -0.290909051895142 2.0
0 -0.128551959991455 3.0
0 0.611831188201904 4.0
0 0.316948175430298 5.0
0 -0.441057443618774 6.0
0 -0.110261201858521 7.0
47.109370567 38.2008934020996 0.0
0 -0.155642628669739 1.0
0 -0.254766583442688 2.0
0 -0.144441723823547 3.0
0 0.36839485168457 4.0
0 -0.833066701889038 5.0
0 0.126165151596069 6.0
0 0.855830192565918 7.0
35.871355662 38.2685737609863 0.0
0 -0.860153436660767 1.0
0 1.73750734329224 2.0
0 -0.236164569854736 3.0
0 0.76594877243042 4.0
0 -0.464623689651489 5.0
0 -0.270623683929443 6.0
0 1.61037945747375 7.0
42.169433911 37.5183792114258 0.0
0 0.156482100486755 1.0
0 -0.197471022605896 2.0
0 0.0144407749176025 3.0
0 0.471382141113281 4.0
0 -0.415889024734497 5.0
0 0.0234358310699463 6.0
0 1.36403906345367 7.0
38.001249401 36.3900871276855 0.0
0 0.20686674118042 1.0
0 -0.182324409484863 2.0
0 -0.061806321144104 3.0
0 0.582828402519226 4.0
0 -0.213070750236511 5.0
0 0.104170083999634 6.0
0 0.510766863822937 7.0
36.026296153 37.8205070495605 0.0
0 0.173980951309204 1.0
0 -0.101333737373352 2.0
0 -0.036510705947876 3.0
0 0.418338298797607 4.0
0 -0.730125665664673 5.0
0 0.0947551727294922 6.0
0 1.19019913673401 7.0
33.200213159 38.213306427002 0.0
0 0.461779236793518 1.0
0 -0.623458623886108 2.0
0 -0.220470070838928 3.0
0 0.2958984375 4.0
0 -0.709063529968262 5.0
0 -0.0488817691802979 6.0
0 -0.289680242538452 7.0
36.881947952 37.946590423584 0.0
0 0.0671913623809814 1.0
0 -0.233146786689758 2.0
0 0.0898871421813965 3.0
0 0.138991117477417 4.0
0 -0.269469976425171 5.0
0 -0.141284823417664 6.0
0 -0.331530809402466 7.0
28.363012012 38.1172790527344 0.0
0 -0.302376508712769 1.0
0 -0.188489317893982 2.0
0 -0.13250470161438 3.0
0 0.0870229005813599 4.0
0 -0.706278085708618 5.0
0 0.227497339248657 6.0
0 -0.210738778114319 7.0
39.574467426 38.4519424438477 0.0
0 0.0737771987915039 1.0
0 -0.19246232509613 2.0
0 0.185916900634766 3.0
0 0.435808658599854 4.0
0 -0.636600494384766 5.0
0 0.2413010597229 6.0
0 0.61533784866333 7.0
36.22040148 38.082447052002 0.0
0 -0.218408107757568 1.0
0 -0.08050537109375 2.0
0 -0.0602649450302124 3.0
0 0.409972667694092 4.0
0 -0.471150875091553 5.0
0 -0.254328727722168 6.0
0 -0.0570566654205322 7.0
31.086432141 37.829288482666 0.0
0 0.282293558120728 1.0
0 -0.286970138549805 2.0
0 0.016819953918457 3.0
0 0.361631035804749 4.0
0 -0.498899698257446 5.0
0 0.475262761116028 6.0
0 -0.41057562828064 7.0
36.274353063 36.2512435913086 0.0
0 -0.0626630783081055 1.0
0 -0.255402088165283 2.0
0 0.0634975433349609 3.0
0 0.212900638580322 4.0
0 -0.177680253982544 5.0
0 -0.194230318069458 6.0
0 -0.23147976398468 7.0
31.65795009 38.2348785400391 0.0
0 -0.341755628585815 1.0
0 -0.403825283050537 2.0
0 0.058479905128479 3.0
0 0.298250079154968 4.0
0 -0.0428166389465332 5.0
0 -0.478070497512817 6.0
0 -0.42031455039978 7.0
37.059171479 37.265453338623 0.0
0 -0.0329396724700928 1.0
0 2.27190375328064 2.0
0 -0.177255630493164 3.0
0 0.943317532539368 4.0
0 -0.126359462738037 5.0
0 -0.628837823867798 6.0
0 1.51233804225922 7.0
45.308862357 38.2540664672852 0.0
0 -0.518732070922852 1.0
0 -0.313593626022339 2.0
0 0.161396741867065 3.0
0 0.587238550186157 4.0
0 -0.465566158294678 5.0
0 0.182194232940674 6.0
0 0.691593766212463 7.0
32.988279909 38.603084564209 0.0
0 0.00927960872650146 1.0
0 -0.178764820098877 2.0
0 -0.367156982421875 3.0
0 -0.0441553592681885 4.0
0 -0.272942066192627 5.0
0 -0.0473340749740601 6.0
0 -0.312782049179077 7.0
41.812200593 38.6239471435547 0.0
0 -0.0628968477249146 1.0
0 -0.254510402679443 2.0
0 -0.303938150405884 3.0
0 0.553947925567627 4.0
0 0.136734366416931 5.0
0 -0.277354717254639 6.0
0 0.752730131149292 7.0
34.159578007 36.5351448059082 0.0
0 -0.0750195980072021 1.0
0 -0.0482892990112305 2.0
0 -0.253533124923706 3.0
0 0.00406050682067871 4.0
0 -0.724410772323608 5.0
0 0.181339025497437 6.0
0 -0.0979443788528442 7.0
41.353058204 38.3780975341797 0.0
0 0.00579845905303955 1.0
0 -0.315656900405884 2.0
0 0.0896854400634766 3.0
0 0.632294774055481 4.0
0 -0.371396064758301 5.0
0 -0.32107138633728 6.0
0 1.4062272310257 7.0
35.663236644 38.0206108093262 0.0
0 0.124375820159912 1.0
0 -0.244772791862488 2.0
0 0.00568163394927979 3.0
0 0.413193821907043 4.0
0 -0.682523012161255 5.0
0 -0.0033947229385376 6.0
0 0.110496997833252 7.0
40.444916123 37.5294189453125 0.0
0 0.139635324478149 1.0
0 -1.45020341873169 2.0
0 -0.31969690322876 3.0
0 0.580675601959229 4.0
0 -0.278090715408325 5.0
0 -0.226438522338867 6.0
0 1.1192854642868 7.0
39.900416914 38.1729125976562 0.0
0 -0.176117181777954 1.0
0 -0.0639520883560181 2.0
0 -0.18073844909668 3.0
0 0.637501239776611 4.0
0 -0.387099266052246 5.0
0 0.108003616333008 6.0
0 -0.224622368812561 7.0
34.405324849 38.5740051269531 0.0
0 -0.175257682800293 1.0
0 1.26640570163727 2.0
0 -0.0109543800354004 3.0
0 0.316189646720886 4.0
0 -0.570112466812134 5.0
0 -0.266166210174561 6.0
0 0.761806011199951 7.0
33.074256004 37.5056419372559 0.0
0 -0.0157623291015625 1.0
0 -0.273559808731079 2.0
0 -0.401785135269165 3.0
0 0.129385232925415 4.0
0 -0.403735876083374 5.0
0 -0.373796463012695 6.0
0 0.0346194505691528 7.0
40.036170308 37.5796890258789 0.0
0 -0.0330996513366699 1.0
0 -0.0600931644439697 2.0
0 -0.200103521347046 3.0
0 0.761157333850861 4.0
0 -0.112207651138306 5.0
0 -0.147305488586426 6.0
0 0.574297904968262 7.0
44.453206241 37.6580047607422 0.0
0 -0.108207106590271 1.0
0 -0.0233310461044312 2.0
0 -0.443291902542114 3.0
0 0.704611539840698 4.0
0 -0.107490301132202 5.0
0 0.252450823783875 6.0
0 2.05543994903564 7.0
33.85004541 38.349063873291 0.0
0 -0.161499261856079 1.0
0 0.360306739807129 2.0
0 -0.198359966278076 3.0
0 0.150657653808594 4.0
0 0.00434708595275879 5.0
0 -0.119342923164368 6.0
0 -0.425829172134399 7.0
30.537877308 36.0110359191895 0.0
0 0.18684458732605 1.0
0 -0.267765283584595 2.0
0 0.186954498291016 3.0
0 0.225530385971069 4.0
0 -0.436043262481689 5.0
0 -0.185184359550476 6.0
0 -0.102651715278625 7.0
27.727440942 37.6201782226562 0.0
0 -0.179068326950073 1.0
0 -0.0497102737426758 2.0
0 -0.090623140335083 3.0
0 0.104896783828735 4.0
0 -0.641779661178589 5.0
0 0.726122856140137 6.0
0 -0.201827526092529 7.0
35.630182897 37.625316619873 0.0
0 0.48189377784729 1.0
0 -0.198164224624634 2.0
0 0.111599683761597 3.0
0 0.259189367294312 4.0
0 -0.267871618270874 5.0
0 0.152856230735779 6.0
0 -0.252257823944092 7.0
31.038251519 38.0266494750977 0.0
0 0.251136660575867 1.0
0 -0.16553795337677 2.0
0 -0.467737674713135 3.0
0 0.23598837852478 4.0
0 -0.709680318832397 5.0
0 -0.175470113754272 6.0
0 -0.0306671857833862 7.0
31.867011967 37.6021461486816 0.0
0 0.406722068786621 1.0
0 -0.198778748512268 2.0
0 -0.0643279552459717 3.0
0 0.42363977432251 4.0
0 -0.64085841178894 5.0
0 -0.154655933380127 6.0
0 0.235535740852356 7.0
30.754831646 37.1513214111328 0.0
0 -0.1203693151474 1.0
0 -0.321998834609985 2.0
0 0.149597883224487 3.0
0 0.293766021728516 4.0
0 -0.318053245544434 5.0
0 -0.12796425819397 6.0
0 -0.175993204116821 7.0
38.95156428 37.2363243103027 0.0
0 -0.165887475013733 1.0
0 -0.269863128662109 2.0
0 -0.398171186447144 3.0
0 0.556317925453186 4.0
0 -0.52601146697998 5.0
0 -0.483704805374146 6.0
0 1.38603091239929 7.0
33.384480936 37.9992790222168 0.0
0 -0.0824921131134033 1.0
0 -0.338196754455566 2.0
0 -0.137604594230652 3.0
0 0.231873512268066 4.0
0 -0.31235671043396 5.0
0 -0.29480242729187 6.0
0 -0.175278186798096 7.0
37.291199282 38.2186393737793 0.0
0 -0.295500755310059 1.0
0 -0.287079572677612 2.0
0 0.0368590354919434 3.0
0 0.279052495956421 4.0
0 -0.420699834823608 5.0
0 0.151886820793152 6.0
0 -0.250487565994263 7.0
33.571811016 37.460376739502 0.0
0 0.0137331485748291 1.0
0 -0.252887368202209 2.0
0 -0.114773154258728 3.0
0 0.385014295578003 4.0
0 -0.428219795227051 5.0
0 -0.0226427316665649 6.0
0 -0.297979593276978 7.0
44.048761746 38.1371765136719 0.0
0 -0.20075786113739 1.0
0 0.317384004592896 2.0
0 -0.200475215911865 3.0
0 0.390530109405518 4.0
0 -0.253287315368652 5.0
0 -0.199228644371033 6.0
0 0.144032955169678 7.0
38.09281546 38.0204162597656 0.0
0 -0.169054269790649 1.0
0 -0.197456955909729 2.0
0 -0.0735089778900146 3.0
0 0.349813580513 4.0
0 -0.327705383300781 5.0
0 -0.113048315048218 6.0
0 0.305657625198364 7.0
42.616606924 38.0626029968262 0.0
0 0.0226221084594727 1.0
0 -0.133760690689087 2.0
0 -0.190871953964233 3.0
0 0.472669243812561 4.0
0 -0.591968536376953 5.0
0 -0.00209259986877441 6.0
0 1.93639647960663 7.0
41.080227074 38.1298637390137 0.0
0 0.0167779922485352 1.0
0 -0.279508829116821 2.0
0 -0.0633440017700195 3.0
0 0.592018246650696 4.0
0 -0.510941743850708 5.0
0 0.0270812511444092 6.0
0 0.719547867774963 7.0
36.859087813 38.1102256774902 0.0
0 0.435817360877991 1.0
0 1.88217401504517 2.0
0 -0.19841742515564 3.0
0 0.602588295936584 4.0
0 -0.526391744613647 5.0
0 -0.270820617675781 6.0
0 0.798430562019348 7.0
39.316268876 37.7710723876953 0.0
0 -0.237595796585083 1.0
0 -0.216592192649841 2.0
0 -0.481058359146118 3.0
0 0.466518521308899 4.0
0 -0.0467357635498047 5.0
0 0.0946956872940063 6.0
0 0.238945841789246 7.0
40.023071967 38.2428855895996 0.0
0 -0.346796989440918 1.0
0 -0.210662484169006 2.0
0 -0.162182092666626 3.0
0 0.475986957550049 4.0
0 -0.847932815551758 5.0
0 0.249469876289368 6.0
0 0.244660139083862 7.0
32.691039566 38.0694274902344 0.0
0 0.467131853103638 1.0
0 -0.271533727645874 2.0
0 -0.0602177381515503 3.0
0 0.238240838050842 4.0
0 -0.7074875831604 5.0
0 0.148088216781616 6.0
0 -0.170773506164551 7.0
39.35017223 38.1511344909668 0.0
0 0.0106163024902344 1.0
0 -0.219057083129883 2.0
0 0.130796909332275 3.0
0 0.470374345779419 4.0
0 -0.315966606140137 5.0
0 -0.213358640670776 6.0
0 0.797488570213318 7.0
38.62577922 38.3594017028809 0.0
0 0.424865961074829 1.0
0 -0.56659984588623 2.0
0 0.0646703243255615 3.0
0 0.698847532272339 4.0
0 -0.41806960105896 5.0
0 -0.273507833480835 6.0
0 0.292388558387756 7.0
43.36321018 37.6704139709473 0.0
0 0.107958316802979 1.0
0 -0.270687341690063 2.0
0 0.689557194709778 3.0
0 0.639695525169373 4.0
0 -0.296959638595581 5.0
0 -0.202871799468994 6.0
0 1.22068452835083 7.0
41.711370519 36.7678413391113 0.0
0 0.628366708755493 1.0
0 -0.19009268283844 2.0
0 -0.168028354644775 3.0
0 0.178023099899292 4.0
0 -0.174020767211914 5.0
0 0.0061190128326416 6.0
0 -0.234395742416382 7.0
41.401748585 37.7634048461914 0.0
0 -0.0371618270874023 1.0
0 -0.794343233108521 2.0
0 -0.353441715240479 3.0
0 0.497199654579163 4.0
0 -0.154067635536194 5.0
0 -0.537550687789917 6.0
0 0.613951683044434 7.0
45.411303661 36.5366325378418 0.0
0 0.0483125448226929 1.0
0 -0.215537428855896 2.0
0 -0.182124376296997 3.0
0 0.295311450958252 4.0
0 -0.456730842590332 5.0
0 -0.15652871131897 6.0
0 0.103638052940369 7.0
46.173622738 38.1801376342773 0.0
0 -0.124875545501709 1.0
0 -0.268786907196045 2.0
0 -0.506638765335083 3.0
0 0.803557336330414 4.0
0 -0.167198061943054 5.0
0 -0.507945775985718 6.0
0 1.50985705852509 7.0
43.970500892 38.0304679870605 0.0
0 0.00630450248718262 1.0
0 -0.0428438186645508 2.0
0 -0.249543428421021 3.0
0 0.858311831951141 4.0
0 -0.688125610351562 5.0
0 -0.303299188613892 6.0
0 2.59589648246765 7.0
41.561964524 38.2210388183594 0.0
0 -0.42961859703064 1.0
0 -0.358009099960327 2.0
0 0.0545748472213745 3.0
0 0.268713116645813 4.0
0 0.0909918546676636 5.0
0 -0.341956853866577 6.0
0 -0.0393184423446655 7.0
42.043570834 36.5913314819336 0.0
0 0.169784784317017 1.0
0 -0.0279194116592407 2.0
0 -0.212491989135742 3.0
0 0.790816307067871 4.0
0 -0.235300898551941 5.0
0 -0.393880844116211 6.0
0 1.08267688751221 7.0
42.728187908 37.3719215393066 0.0
0 -0.14280366897583 1.0
0 -0.198686957359314 2.0
0 -0.218318462371826 3.0
0 0.584456205368042 4.0
0 -0.590444326400757 5.0
0 -0.503875017166138 6.0
0 0.882217407226562 7.0
46.647822945 38.157096862793 0.0
0 -0.292713165283203 1.0
0 -0.281616449356079 2.0
0 -0.0503835678100586 3.0
0 0.597112059593201 4.0
0 -0.655223846435547 5.0
0 0.0185105800628662 6.0
0 1.33927083015442 7.0
34.157298462 38.1393203735352 0.0
0 -0.442243814468384 1.0
0 -0.20483672618866 2.0
0 -0.0573487281799316 3.0
0 0.45450758934021 4.0
0 -0.737014532089233 5.0
0 -0.250651597976685 6.0
0 0.991172671318054 7.0
38.08940786 37.6850738525391 0.0
0 0.497749328613281 1.0
0 -0.686261415481567 2.0
0 0.169938802719116 3.0
0 0.426481246948242 4.0
0 0.0777859687805176 5.0
0 0.0582671165466309 6.0
0 -0.30633282661438 7.0
48.253456605 37.4551849365234 0.0
0 -0.194845199584961 1.0
0 -0.267076253890991 2.0
0 0.174766302108765 3.0
0 0.570026397705078 4.0
0 -0.205016136169434 5.0
0 -0.330666780471802 6.0
0 0.63228178024292 7.0
41.303512955 37.484619140625 0.0
0 -0.246476054191589 1.0
0 -0.145244002342224 2.0
0 -0.0911074876785278 3.0
0 0.417212247848511 4.0
0 -0.774739980697632 5.0
0 0.88127726316452 6.0
0 0.203672409057617 7.0
34.11247552 35.0377388000488 0.0
0 0.314133882522583 1.0
0 -0.0763225555419922 2.0
0 -0.129271507263184 3.0
0 0.0952231884002686 4.0
0 -0.190337896347046 5.0
0 -0.695054292678833 6.0
0 -0.293839693069458 7.0
39.346899321 37.6348037719727 0.0
0 0.0940911769866943 1.0
0 -0.0184941291809082 2.0
0 -0.147531867027283 3.0
0 0.842624723911285 4.0
0 -0.456753969192505 5.0
0 -0.0337381362915039 6.0
0 1.75469517707825 7.0
42.355081321 38.3144760131836 0.0
0 -0.431054353713989 1.0
0 -0.377702236175537 2.0
0 0.13573169708252 3.0
0 0.717527508735657 4.0
0 0.0787239074707031 5.0
0 -0.345009088516235 6.0
0 1.86493170261383 7.0
46.025284619 37.9888877868652 0.0
0 -0.032625675201416 1.0
0 -0.22252082824707 2.0
0 -0.332157611846924 3.0
0 0.51042366027832 4.0
0 -0.224066734313965 5.0
0 0.021633505821228 6.0
0 0.259620189666748 7.0
48.790909787 38.3799705505371 0.0
0 0.099947452545166 1.0
0 -0.236382126808167 2.0
0 0.0822820663452148 3.0
0 0.658200860023499 4.0
0 -0.42223596572876 5.0
0 -0.223820328712463 6.0
0 1.7103363275528 7.0
41.698940502 37.8754768371582 0.0
0 -0.565130472183228 1.0
0 -0.207319259643555 2.0
0 -0.208969593048096 3.0
0 0.567966461181641 4.0
0 -0.373228788375854 5.0
0 -0.658563375473022 6.0
0 1.03816711902618 7.0
47.902972781 38.3145179748535 0.0
0 0.0121076107025146 1.0
0 -0.173000931739807 2.0
0 -0.0094829797744751 3.0
0 0.671643257141113 4.0
0 -0.338266611099243 5.0
0 0.0511239767074585 6.0
0 1.68094718456268 7.0
41.37717087 38.2120704650879 0.0
0 -0.278881072998047 1.0
0 -0.313841104507446 2.0
0 -0.113258004188538 3.0
0 0.502711057662964 4.0
0 -0.143787860870361 5.0
0 -0.143849015235901 6.0
0 0.595085382461548 7.0
32.285684902 36.4002494812012 0.0
0 0.10178816318512 1.0
0 -0.285447359085083 2.0
0 -0.333153486251831 3.0
0 0.187896013259888 4.0
0 -0.536683797836304 5.0
0 0.052363395690918 6.0
0 -0.730260610580444 7.0
32.344206851 38.1654930114746 0.0
0 -0.232256174087524 1.0
0 -0.181212902069092 2.0
0 -0.173463940620422 3.0
0 0.343669772148132 4.0
0 -0.491954326629639 5.0
0 -0.358771562576294 6.0
0 0.176887512207031 7.0
34.184112331 37.4673194885254 0.0
0 0.253570079803467 1.0
0 -0.26023805141449 2.0
0 -0.32091212272644 3.0
0 0.337071657180786 4.0
0 -0.672094345092773 5.0
0 0.232657551765442 6.0
0 -0.238960027694702 7.0
34.663221808 36.8910217285156 0.0
0 -0.108678102493286 1.0
0 -0.237229943275452 2.0
0 0.0951861143112183 3.0
0 0.579286456108093 4.0
0 -0.428537368774414 5.0
0 0.142714023590088 6.0
0 0.465059518814087 7.0
36.873774543 38.1369094848633 0.0
0 -0.288044691085815 1.0
0 -0.292827129364014 2.0
0 -0.0229915380477905 3.0
0 0.253217458724976 4.0
0 -0.22378671169281 5.0
0 -0.824081420898438 6.0
0 -0.288119077682495 7.0
40.211352908 38.7613410949707 0.0
0 0.157359838485718 1.0
0 -0.250598549842834 2.0
0 -0.246178865432739 3.0
0 0.725827574729919 4.0
0 -0.460910558700562 5.0
0 0.334162354469299 6.0
0 2.05783224105835 7.0
46.160689867 38.2640686035156 0.0
0 -0.444269895553589 1.0
0 -0.252123713493347 2.0
0 0.201003670692444 3.0
0 0.558600187301636 4.0
0 -0.195149898529053 5.0
0 -0.283248662948608 6.0
0 1.13149356842041 7.0
32.727702676 37.7371978759766 0.0
0 0.412008166313171 1.0
0 -0.304033041000366 2.0
0 0.371429681777954 3.0
0 0.62620997428894 4.0
0 -0.296971559524536 5.0
0 -0.042356014251709 6.0
0 0.786261796951294 7.0
40.640985417 38.3238677978516 0.0
0 -0.454776763916016 1.0
0 -0.35515284538269 2.0
0 0.185338020324707 3.0
0 0.415790438652039 4.0
0 -0.201981663703918 5.0
0 -0.605022430419922 6.0
0 0.490357279777527 7.0
34.611451594 36.9063682556152 0.0
0 -0.180743932723999 1.0
0 -0.299197435379028 2.0
0 0.110708236694336 3.0
0 0.177433013916016 4.0
0 -0.058274507522583 5.0
0 -0.322599411010742 6.0
0 -0.0667726993560791 7.0
43.311280807 37.0737533569336 0.0
0 0.347077608108521 1.0
0 -0.236951112747192 2.0
0 0.0913729667663574 3.0
0 0.654915332794189 4.0
0 -0.0208683013916016 5.0
0 -0.783708095550537 6.0
0 -0.0336704254150391 7.0
36.406364057 37.1961898803711 0.0
0 0.0218989849090576 1.0
0 -0.0663164854049683 2.0
0 -0.267929315567017 3.0
0 0.425845503807068 4.0
0 -0.224127054214478 5.0
0 0.186677575111389 6.0
0 0.194292426109314 7.0
31.685630326 38.1030540466309 0.0
0 -0.0371747016906738 1.0
0 -0.84189772605896 2.0
0 -0.287739515304565 3.0
0 0.691764116287231 4.0
0 -0.240362644195557 5.0
0 1.15166592597961 6.0
0 0.677238345146179 7.0
40.165364432 37.7072143554688 0.0
0 -0.424236059188843 1.0
0 -0.38940167427063 2.0
0 0.203342199325562 3.0
0 0.86771547794342 4.0
0 -0.693302631378174 5.0
0 0.322751760482788 6.0
0 2.06195545196533 7.0
42.886465647 38.4553031921387 0.0
0 -0.335622549057007 1.0
0 -0.221626996994019 2.0
0 0.0108251571655273 3.0
0 0.690367460250854 4.0
0 -0.489032030105591 5.0
0 -0.227444171905518 6.0
0 0.808408498764038 7.0
46.942592332 38.3565979003906 0.0
0 -0.155924081802368 1.0
0 -0.19244384765625 2.0
0 -0.11479640007019 3.0
0 0.683144092559814 4.0
0 -0.423417568206787 5.0
0 0.126772165298462 6.0
0 0.364391207695007 7.0
38.417030568 38.5958595275879 0.0
0 -0.412194013595581 1.0
0 -1.56746363639832 2.0
0 -0.393146514892578 3.0
0 0.612778186798096 4.0
0 -0.470524549484253 5.0
0 -0.749729156494141 6.0
0 0.664360880851746 7.0
39.82159543 37.7675933837891 0.0
0 0.162184238433838 1.0
0 -0.250009536743164 2.0
0 0.191055059432983 3.0
0 0.469789028167725 4.0
0 -0.130026340484619 5.0
0 -0.393316507339478 6.0
0 -0.0851743221282959 7.0
42.375846269 38.0063858032227 0.0
0 -0.221886157989502 1.0
0 0.820373296737671 2.0
0 0.0515979528427124 3.0
0 0.194510817527771 4.0
0 -0.197316408157349 5.0
0 -0.374264001846313 6.0
0 -0.409883737564087 7.0
41.909868924 38.2319641113281 0.0
0 -0.127452373504639 1.0
0 -0.0510166883468628 2.0
0 -0.153699159622192 3.0
0 1.16750764846802 4.0
0 0.0890541076660156 5.0
0 -0.346464872360229 6.0
0 2.47983622550964 7.0
38.515403814 38.1327705383301 0.0
0 0.0817676782608032 1.0
0 -0.144984126091003 2.0
0 -0.159141063690186 3.0
0 0.909328401088715 4.0
0 -0.42089319229126 5.0
0 0.112806797027588 6.0
0 0.259241223335266 7.0
32.187464773 38.6779441833496 0.0
0 -0.181178450584412 1.0
0 -0.340678215026855 2.0
0 -0.0185315608978271 3.0
0 0.549932718276978 4.0
0 -0.384865045547485 5.0
0 0.340621590614319 6.0
0 0.107057332992554 7.0
43.553340498 37.9032783508301 0.0
0 -0.433886051177979 1.0
0 -0.411630868911743 2.0
0 -0.432826519012451 3.0
0 0.551971673965454 4.0
0 -0.960876703262329 5.0
0 0.97905045747757 6.0
0 1.38199663162231 7.0
38.661685118 37.9960861206055 0.0
0 0.470362186431885 1.0
0 -0.26541543006897 2.0
0 -0.182985782623291 3.0
0 0.801794111728668 4.0
0 -0.51978063583374 5.0
0 0.0906689167022705 6.0
0 2.07926487922668 7.0
47.185976948 37.746696472168 0.0
0 -0.139330506324768 1.0
0 -0.242756724357605 2.0
0 -0.0233882665634155 3.0
0 0.838750541210175 4.0
0 -0.181458711624146 5.0
0 -0.511241674423218 6.0
0 1.22140896320343 7.0
41.073649825 36.1723442077637 0.0
0 0.0185480117797852 1.0
0 0.859254360198975 2.0
0 -0.173502087593079 3.0
0 0.745954394340515 4.0
0 0.145409822463989 5.0
0 -0.520633935928345 6.0
0 0.0892437696456909 7.0
39.086114846 37.3074760437012 0.0
0 0.0461180210113525 1.0
0 -0.225179553031921 2.0
0 0.0406086444854736 3.0
0 0.551661729812622 4.0
0 -0.928939342498779 5.0
0 -0.0218620300292969 6.0
0 1.92167282104492 7.0
38.072309151 37.8041648864746 0.0
0 0.212536454200745 1.0
0 -0.215453267097473 2.0
0 0.0316927433013916 3.0
0 0.305475354194641 4.0
0 -0.189601182937622 5.0
0 -0.337292432785034 6.0
0 0.273344993591309 7.0
41.585402374 38.2893104553223 0.0
0 0.197662830352783 1.0
0 -0.0479346513748169 2.0
0 0.00308084487915039 3.0
0 0.624208450317383 4.0
0 -0.586333036422729 5.0
0 -0.155055284500122 6.0
0 1.35561680793762 7.0
42.317317809 36.1765098571777 0.0
0 0.390322923660278 1.0
0 -0.31341028213501 2.0
0 0.121731042861938 3.0
0 0.664822578430176 4.0
0 -0.658772706985474 5.0
0 2.12358784675598 6.0
0 1.5074474811554 7.0
42.931477585 38.326774597168 0.0
0 -0.522270679473877 1.0
0 -0.358250617980957 2.0
0 -0.338838815689087 3.0
0 0.463363528251648 4.0
0 0.0739814043045044 5.0
0 -0.282404661178589 6.0
0 -0.0297156572341919 7.0
46.096274411 37.9832344055176 0.0
0 0.311043739318848 1.0
0 -0.219649195671082 2.0
0 0.0627717971801758 3.0
0 0.55199146270752 4.0
0 -0.244248628616333 5.0
0 0.28298282623291 6.0
0 2.1391077041626 7.0
39.955261554 37.2480621337891 0.0
0 -0.399331331253052 1.0
0 -0.271701574325562 2.0
0 -0.0888369083404541 3.0
0 0.660822749137878 4.0
0 -0.685114622116089 5.0
0 0.0742917060852051 6.0
0 1.0435117483139 7.0
45.415961846 38.5576438903809 0.0
0 0.0911979675292969 1.0
0 -0.167261719703674 2.0
0 -0.138952016830444 3.0
0 0.551891088485718 4.0
0 -0.220184206962585 5.0
0 -0.519299507141113 6.0
0 0.954544425010681 7.0
44.275746657 38.4997711181641 0.0
0 -0.101448059082031 1.0
0 -0.329053640365601 2.0
0 0.181313991546631 3.0
0 0.776606440544128 4.0
0 -0.184151887893677 5.0
0 -0.0470395088195801 6.0
0 2.14544177055359 7.0
34.766238966 35.0451126098633 0.0
0 -0.152744054794312 1.0
0 -0.280553102493286 2.0
0 -0.0695915222167969 3.0
0 0.197344064712524 4.0
0 -0.285446643829346 5.0
0 -0.874872446060181 6.0
0 -0.250021576881409 7.0
39.697519492 38.3627395629883 0.0
0 -0.180624842643738 1.0
0 -0.0300378799438477 2.0
0 -0.122833013534546 3.0
0 0.312835693359375 4.0
0 -0.306505441665649 5.0
0 0.400923609733582 6.0
0 -0.251022577285767 7.0
36.459952599 38.2694358825684 0.0
0 0.131168961524963 1.0
0 -0.240277886390686 2.0
0 -0.159517168998718 3.0
0 0.456149339675903 4.0
0 0.000643253326416016 5.0
0 -0.381793260574341 6.0
0 -0.495996952056885 7.0
44.856515586 37.7952117919922 0.0
0 0.234313368797302 1.0
0 -0.218380451202393 2.0
0 -0.0191619396209717 3.0
0 0.561827182769775 4.0
0 -0.348015785217285 5.0
0 -0.529793262481689 6.0
0 0.0901684761047363 7.0
41.157913956 38.2749214172363 0.0
0 -0.00183713436126709 1.0
0 -0.239104986190796 2.0
0 0.128359794616699 3.0
0 0.803152918815613 4.0
0 -0.209905624389648 5.0
0 -0.834749937057495 6.0
0 1.38859724998474 7.0
32.092085792 38.9433059692383 0.0
0 -0.423834323883057 1.0
0 -0.410298347473145 2.0
0 -0.281436681747437 3.0
0 0.547096967697144 4.0
0 -0.308276414871216 5.0
0 0.0494062900543213 6.0
0 0.119788289070129 7.0
30.327518425 38.0563583374023 0.0
0 -0.159770965576172 1.0
0 -0.252401113510132 2.0
0 0.189219117164612 3.0
0 0.157899856567383 4.0
0 -0.336297988891602 5.0
0 -0.284708023071289 6.0
0 -0.617540597915649 7.0
36.482977278 38.3401107788086 0.0
0 -0.39614725112915 1.0
0 -0.290480613708496 2.0
0 -0.432143449783325 3.0
0 0.248198747634888 4.0
0 -0.492793798446655 5.0
0 0.093048095703125 6.0
0 -0.377017736434937 7.0
40.491170914 38.5105934143066 0.0
0 -0.513708353042603 1.0
0 -0.393326759338379 2.0
0 -0.37799596786499 3.0
0 0.612654685974121 4.0
0 -0.448976755142212 5.0
0 0.241763114929199 6.0
0 1.0147670507431 7.0
41.188898334 38.5119018554688 0.0
0 -0.142570018768311 1.0
0 -0.374892950057983 2.0
0 -0.160027980804443 3.0
0 0.605662822723389 4.0
0 -0.267756462097168 5.0
0 -0.451034784317017 6.0
0 0.66082239151001 7.0
37.756949904 38.3684539794922 0.0
0 -0.370019912719727 1.0
0 -0.272306680679321 2.0
0 -0.158122777938843 3.0
0 0.364170074462891 4.0
0 -0.278390645980835 5.0
0 0.00449693202972412 6.0
0 -0.374369144439697 7.0
26.9626458 35.2078018188477 0.0
0 0.1788489818573 1.0
0 -0.0188243389129639 2.0
0 -0.196877956390381 3.0
0 0.164721727371216 4.0
0 -0.607057332992554 5.0
0 0.408661603927612 6.0
0 -0.573544502258301 7.0
44.154680866 37.3342475891113 0.0
0 -0.0331640243530273 1.0
0 -0.290959596633911 2.0
0 0.416349053382874 3.0
0 0.629885315895081 4.0
0 -0.197506904602051 5.0
0 0.0431878566741943 6.0
0 2.16722226142883 7.0
36.49920361 37.9869003295898 0.0
0 0.219234704971313 1.0
0 -0.101433515548706 2.0
0 -0.192037582397461 3.0
0 0.892424821853638 4.0
0 -0.243679404258728 5.0
0 -0.295492649078369 6.0
0 0.962074279785156 7.0
36.136187007 38.236759185791 0.0
0 -0.214044809341431 1.0
0 -0.246180415153503 2.0
0 -0.192464351654053 3.0
0 0.831297993659973 4.0
0 -0.713846445083618 5.0
0 0.317479968070984 6.0
0 1.43978750705719 7.0
43.805905709 37.8661422729492 0.0
0 -0.295955419540405 1.0
0 -0.263381719589233 2.0
0 -0.0388337373733521 3.0
0 0.733456611633301 4.0
0 -0.306282758712769 5.0
0 0.217428207397461 6.0
0 1.83581149578094 7.0
43.400673478 38.6176300048828 0.0
0 -0.231053352355957 1.0
0 -0.307445764541626 2.0
0 0.0208721160888672 3.0
0 0.910418689250946 4.0
0 -0.665759563446045 5.0
0 -0.139532566070557 6.0
0 1.8643491268158 7.0
35.708898932 38.539924621582 0.0
0 0.0357470512390137 1.0
0 -1.20707893371582 2.0
0 -0.0829973220825195 3.0
0 0.365241050720215 4.0
0 -0.775676488876343 5.0
0 -0.311482667922974 6.0
0 0.344984173774719 7.0
42.195375979 38.1422386169434 0.0
0 -0.381463289260864 1.0
0 -0.248866200447083 2.0
0 -0.162406206130981 3.0
0 0.417815446853638 4.0
0 -0.422652959823608 5.0
0 -0.444005727767944 6.0
0 0.492365598678589 7.0
38.219132252 38.3212890625 0.0
0 -0.271913528442383 1.0
0 -0.209117531776428 2.0
0 0.012225866317749 3.0
0 -0.0179314613342285 4.0
0 -0.233347296714783 5.0
0 -0.386373996734619 6.0
0 0.280413150787354 7.0
37.34785904 38.4695472717285 0.0
0 -0.205513834953308 1.0
0 -0.132047414779663 2.0
0 -0.190691947937012 3.0
0 0.263391494750977 4.0
0 -0.0900624990463257 5.0
0 -0.0674309730529785 6.0
0 -0.119197010993958 7.0
41.413317083 38.0743064880371 0.0
0 -0.0180668830871582 1.0
0 -0.0649148225784302 2.0
0 -0.214722633361816 3.0
0 0.14849054813385 4.0
0 -0.341552972793579 5.0
0 0.13208794593811 6.0
0 -0.199143052101135 7.0
39.408474012 38.9237823486328 0.0
0 -0.181172609329224 1.0
0 -0.197943925857544 2.0
0 -0.416520833969116 3.0
0 0.656323432922363 4.0
0 -0.17142391204834 5.0
0 -0.428765773773193 6.0
0 0.323708772659302 7.0
44.621947793 37.8165702819824 0.0
0 0.0125672817230225 1.0
0 -0.225801348686218 2.0
0 -0.134201765060425 3.0
0 0.682062387466431 4.0
0 -0.233695983886719 5.0
0 -0.811933279037476 6.0
0 0.668441414833069 7.0
35.042314673 38.3691749572754 0.0
0 -0.352604866027832 1.0
0 1.18566226959229 2.0
0 -0.173362016677856 3.0
0 0.546821236610413 4.0
0 -0.241734743118286 5.0
0 0.00288629531860352 6.0
0 0.37858521938324 7.0
44.668685987 37.7239532470703 0.0
0 -0.144008755683899 1.0
0 -0.12114417552948 2.0
0 -0.395444393157959 3.0
0 0.445341110229492 4.0
0 -0.264827966690063 5.0
0 0.212258219718933 6.0
0 0.0823891162872314 7.0
44.271188235 38.3123207092285 0.0
0 -0.0954949855804443 1.0
0 -0.243334174156189 2.0
0 -0.171482443809509 3.0
0 0.660595178604126 4.0
0 0.323468208312988 5.0
0 -0.0115576982498169 6.0
0 -0.109113574028015 7.0
43.329615715 38.3195610046387 0.0
0 0.0784916877746582 1.0
0 -0.235922694206238 2.0
0 -0.0318349599838257 3.0
0 0.734532117843628 4.0
0 -0.707624197006226 5.0
0 -0.402095556259155 6.0
0 1.74201250076294 7.0
37.727034893 37.9385871887207 0.0
0 0.16304612159729 1.0
0 -0.373134136199951 2.0
0 0.152762651443481 3.0
0 1.00755560398102 4.0
0 -0.331214189529419 5.0
0 0.0969347953796387 6.0
0 1.48263847827911 7.0
43.665312113 36.6764450073242 0.0
0 0.263379096984863 1.0
0 -0.317164182662964 2.0
0 -0.414024114608765 3.0
0 0.407849788665771 4.0
0 -0.338751077651978 5.0
0 0.279090166091919 6.0
0 0.227195024490356 7.0
40.656348383 38.1192474365234 0.0
0 -0.0801599025726318 1.0
0 -0.202273011207581 2.0
0 -0.172312259674072 3.0
0 0.419043779373169 4.0
0 -0.359326124191284 5.0
0 0.0337018966674805 6.0
0 -0.101455211639404 7.0
30.759139555 38.1249809265137 0.0
0 -0.173085808753967 1.0
0 -0.135176301002502 2.0
0 -0.122214913368225 3.0
0 0.346059083938599 4.0
0 -0.446455717086792 5.0
0 -0.257526874542236 6.0
0 -0.169100165367126 7.0
36.474229652 38.922248840332 0.0
0 -0.334157705307007 1.0
0 -0.386319160461426 2.0
0 0.103130340576172 3.0
0 0.165078639984131 4.0
0 -0.165418863296509 5.0
0 -0.737318277359009 6.0
0 -0.119624972343445 7.0
33.340721195 37.4351921081543 0.0
0 0.203714370727539 1.0
0 -0.335776805877686 2.0
0 -0.00716829299926758 3.0
0 0.499643564224243 4.0
0 -0.427947759628296 5.0
0 -0.471636772155762 6.0
0 0.378044009208679 7.0
34.143349268 37.7863998413086 0.0
0 0.244834899902344 1.0
0 -1.09760499000549 2.0
0 -0.326842784881592 3.0
0 -0.13373327255249 4.0
0 -0.468292474746704 5.0
0 -0.259369850158691 6.0
0 -0.024770975112915 7.0
40.763829757 38.3444976806641 0.0
0 -0.0497903823852539 1.0
0 -0.242779970169067 2.0
0 -0.146228313446045 3.0
0 0.881098210811615 4.0
0 -0.579242467880249 5.0
0 -0.0702017545700073 6.0
0 1.26471793651581 7.0
29.788224958 37.9883575439453 0.0
0 0.374404430389404 1.0
0 -0.307290315628052 2.0
0 -0.337557792663574 3.0
0 0.363442897796631 4.0
0 -0.292948961257935 5.0
0 -0.51552414894104 6.0
0 -0.276032209396362 7.0
34.675869554 38.3907089233398 0.0
0 -0.412520170211792 1.0
0 -0.232771039009094 2.0
0 0.476433396339417 3.0
0 0.2444167137146 4.0
0 -0.246869802474976 5.0
0 -0.00949966907501221 6.0
0 -0.211910367012024 7.0
40.723855828 38.2133865356445 0.0
0 -0.60289478302002 1.0
0 -0.291898488998413 2.0
0 -0.0172886848449707 3.0
0 0.986126184463501 4.0
0 -0.471684455871582 5.0
0 0.182528972625732 6.0
0 1.83247339725494 7.0
41.414838352 36.9279441833496 0.0
0 0.0978062152862549 1.0
0 -0.264137029647827 2.0
0 0.030187726020813 3.0
0 0.85716050863266 4.0
0 -0.324277400970459 5.0
0 -0.248111009597778 6.0
0 0.70628547668457 7.0
33.391441214 37.8371467590332 0.0
0 0.20508337020874 1.0
0 -0.278693199157715 2.0
0 -0.0165634155273438 3.0
0 0.131724834442139 4.0
0 -0.672899723052979 5.0
0 -0.362652778625488 6.0
0 -0.275365591049194 7.0
36.660645385 37.6194000244141 0.0
0 -0.344926834106445 1.0
0 -0.276827335357666 2.0
0 -0.2389817237854 3.0
0 0.440268278121948 4.0
0 -0.493053197860718 5.0
0 -0.427257776260376 6.0
0 0.868929624557495 7.0
33.547567166 36.8476676940918 0.0
0 -0.116831183433533 1.0
0 -0.3283851146698 2.0
0 0.138549327850342 3.0
0 0.590574741363525 4.0
0 -0.556514263153076 5.0
0 0.131281614303589 6.0
0 0.0693552494049072 7.0
36.717923495 38.0243339538574 0.0
0 0.290438413619995 1.0
0 -0.295759916305542 2.0
0 0.124046087265015 3.0
0 0.389407157897949 4.0
0 -0.490115404129028 5.0
0 0.318592548370361 6.0
0 -0.148103713989258 7.0
44.575737494 37.7191505432129 0.0
0 -0.0402113199234009 1.0
0 -0.302358388900757 2.0
0 -0.314715623855591 3.0
0 0.65808379650116 4.0
0 -0.34340238571167 5.0
0 -0.0169692039489746 6.0
0 1.16347992420197 7.0
43.091610321 38.1561698913574 0.0
0 -0.0391066074371338 1.0
0 -0.210016250610352 2.0
0 0.0330873727798462 3.0
0 0.665749907493591 4.0
0 -0.0859036445617676 5.0
0 -0.501899003982544 6.0
0 1.00820183753967 7.0
42.647069973 37.7142486572266 0.0
0 -0.0849385261535645 1.0
0 -0.269834280014038 2.0
0 -0.0712094306945801 3.0
0 0.476293802261353 4.0
0 -0.488709211349487 5.0
0 -0.0724787712097168 6.0
0 1.0268292427063 7.0
36.240225339 38.5001258850098 0.0
0 -0.206627368927002 1.0
0 -0.208115935325623 2.0
0 -0.181745529174805 3.0
0 0.288426041603088 4.0
0 -0.414926290512085 5.0
0 -0.246529340744019 6.0
0 0.167505621910095 7.0
31.876756203 38.1138153076172 0.0
0 0.178369760513306 1.0
0 -0.207579612731934 2.0
0 0.0510427951812744 3.0
0 0.476535558700562 4.0
0 -0.21655261516571 5.0
0 0.0462149381637573 6.0
0 -0.0346536636352539 7.0
38.711049659 36.4012336730957 0.0
0 0.139508247375488 1.0
0 -0.344122409820557 2.0
0 0.118475317955017 3.0
0 0.224749565124512 4.0
0 -0.300268650054932 5.0
0 -0.139055728912354 6.0
0 -0.300618410110474 7.0
33.592635672 38.2685241699219 0.0
0 -0.0175764560699463 1.0
0 -0.222737669944763 2.0
0 -0.149013757705688 3.0
0 0.187649965286255 4.0
0 -0.380817413330078 5.0
0 -0.225193858146667 6.0
0 -0.228569507598877 7.0
31.939497021 37.98583984375 0.0
0 -0.079952597618103 1.0
0 -0.190449118614197 2.0
0 0.0421414375305176 3.0
0 0.205034971237183 4.0
0 -0.267570018768311 5.0
0 0.2071293592453 6.0
0 -0.206736922264099 7.0
32.605677369 38.0590896606445 0.0
0 -0.137702345848083 1.0
0 -0.031693696975708 2.0
0 -0.138607740402222 3.0
0 0.166922569274902 4.0
0 -0.31270956993103 5.0
0 0.0157430171966553 6.0
0 -0.343770742416382 7.0
35.02561425 38.3408012390137 0.0
0 0.00323343276977539 1.0
0 -0.226416707038879 2.0
0 -0.00420904159545898 3.0
0 0.174561738967896 4.0
0 -0.400297403335571 5.0
0 -0.895344257354736 6.0
0 0.207136273384094 7.0
29.155020371 38.1937141418457 0.0
0 -0.0298449993133545 1.0
0 -0.218515157699585 2.0
0 -0.146848678588867 3.0
0 0.170099496841431 4.0
0 -0.418640375137329 5.0
0 -0.531238555908203 6.0
0 -0.0539044141769409 7.0
29.9790999 38.0672225952148 0.0
0 0.385779738426208 1.0
0 -1.06227898597717 2.0
0 -0.140069484710693 3.0
0 -0.187403202056885 4.0
0 -0.388806819915771 5.0
0 -0.411620855331421 6.0
0 -0.433201551437378 7.0
40.412374428 38.244010925293 0.0
0 -0.332977771759033 1.0
0 -0.269173622131348 2.0
0 0.0742030143737793 3.0
0 0.471150875091553 4.0
0 -0.320183753967285 5.0
0 -0.350206136703491 6.0
0 1.05381751060486 7.0
44.799669607 38.1905097961426 0.0
0 -0.26012134552002 1.0
0 -0.21551525592804 2.0
0 -0.199522733688354 3.0
0 0.867998421192169 4.0
0 -0.20648181438446 5.0
0 -0.0870825052261353 6.0
0 2.03972721099854 7.0
35.691930085 38.0569152832031 0.0
0 0.144345760345459 1.0
0 -0.38860011100769 2.0
0 0.209831595420837 3.0
0 0.490484595298767 4.0
0 -0.136711597442627 5.0
0 -0.122484683990479 6.0
0 0.355215787887573 7.0
37.293002382 38.1684608459473 0.0
0 -0.17205822467804 1.0
0 -0.209368944168091 2.0
0 -0.236521601676941 3.0
0 0.551456212997437 4.0
0 -0.295182943344116 5.0
0 -0.337747097015381 6.0
0 1.21887362003326 7.0
31.749050635 38.1198654174805 0.0
0 -0.169233560562134 1.0
0 -0.0486161708831787 2.0
0 -0.318307161331177 3.0
0 0.257851362228394 4.0
0 -0.369430541992188 5.0
0 -0.139599084854126 6.0
0 -0.297379732131958 7.0
44.438599316 34.5469398498535 0.0
0 -1.55369925498962 1.0
0 1.69279062747955 2.0
0 -0.292229890823364 3.0
0 0.864450752735138 4.0
0 -0.514680147171021 5.0
0 0.252918481826782 6.0
0 2.57941627502441 7.0
30.584762345 37.0005874633789 0.0
0 0.482590317726135 1.0
0 -0.638201951980591 2.0
0 -0.247271180152893 3.0
0 0.461858749389648 4.0
0 -0.605194807052612 5.0
0 0.348551750183105 6.0
0 0.0632258653640747 7.0
36.167711467 38.6204872131348 0.0
0 -0.218040227890015 1.0
0 -0.0246405601501465 2.0
0 -0.158756494522095 3.0
0 0.317561149597168 4.0
0 -0.202215790748596 5.0
0 0.309308767318726 6.0
0 -0.145748019218445 7.0
45.589023165 37.4945220947266 0.0
0 0.3735511302948 1.0
0 -0.0205743312835693 2.0
0 -0.417247295379639 3.0
0 0.940840721130371 4.0
0 -0.653218984603882 5.0
0 0.284492492675781 6.0
0 2.23066401481628 7.0
33.954863975 36.9333190917969 0.0
0 -0.326806783676147 1.0
0 -0.234383702278137 2.0
0 -0.245264530181885 3.0
0 0.415324330329895 4.0
0 -0.89977240562439 5.0
0 0.288844108581543 6.0
0 0.441350817680359 7.0
39.104021217 38.1422920227051 0.0
0 0.0289782285690308 1.0
0 -0.246348738670349 2.0
0 0.0295751094818115 3.0
0 0.374989748001099 4.0
0 -0.304695844650269 5.0
0 -0.177971839904785 6.0
0 -0.206287622451782 7.0
41.615139375 37.2350997924805 0.0
0 0.0568931102752686 1.0
0 -0.305189371109009 2.0
0 0.179914474487305 3.0
0 0.576789379119873 4.0
0 0.0620708465576172 5.0
0 -0.145837545394897 6.0
0 0.390314936637878 7.0
29.251209169 35.6371688842773 0.0
0 0.00481986999511719 1.0
0 -0.262315988540649 2.0
0 -0.326462984085083 3.0
0 -0.0155401229858398 4.0
0 -0.898995637893677 5.0
0 1.65862441062927 6.0
0 -0.21510910987854 7.0
39.145261601 38.4009399414062 0.0
0 0.0464761257171631 1.0
0 -0.215338945388794 2.0
0 -0.0300884246826172 3.0
0 0.641082048416138 4.0
0 -0.167457222938538 5.0
0 -0.306601762771606 6.0
0 1.47311460971832 7.0
43.724223514 37.6296348571777 0.0
0 0.19312310218811 1.0
0 -0.216270565986633 2.0
0 -0.170636177062988 3.0
0 0.739034056663513 4.0
0 -0.577557325363159 5.0
0 0.162777304649353 6.0
0 2.51887583732605 7.0
30.694427948 36.8475379943848 0.0
0 0.256727457046509 1.0
0 -0.246504187583923 2.0
0 0.0644222497940063 3.0
0 0.270598888397217 4.0
0 -0.465023756027222 5.0
0 0.259483218193054 6.0
0 -0.0516172647476196 7.0
46.899446079 38.3975028991699 0.0
0 0.206971645355225 1.0
0 -0.957546472549438 2.0
0 0.179876208305359 3.0
0 1.15286254882812 4.0
0 -0.488067150115967 5.0
0 -0.152841687202454 6.0
0 2.48050665855408 7.0
30.801741874 38.2776565551758 0.0
0 -0.0857614278793335 1.0
0 -0.224750280380249 2.0
0 -0.0997546911239624 3.0
0 0.133341789245605 4.0
0 -0.565573215484619 5.0
0 0.465699434280396 6.0
0 -0.0878423452377319 7.0
40.388900757 37.2345504760742 0.0
0 0.459462761878967 1.0
0 -0.210881114006042 2.0
0 0.0624368190765381 3.0
0 0.41385817527771 4.0
0 -0.824301719665527 5.0
0 0.117761492729187 6.0
0 0.0247198343276978 7.0
35.801686131 36.9800682067871 0.0
0 0.389256596565247 1.0
0 -0.0972616672515869 2.0
0 -0.217176198959351 3.0
0 0.462990522384644 4.0
0 -0.0360031127929688 5.0
0 -0.0540966987609863 6.0
0 0.0920470952987671 7.0
43.275430302 38.3159103393555 0.0
0 0.184052467346191 1.0
0 -0.257354378700256 2.0
0 0.211697578430176 3.0
0 0.647692918777466 4.0
0 -0.177887201309204 5.0
0 -0.852500677108765 6.0
0 0.76116955280304 7.0
36.674533885 38.2551040649414 0.0
0 0.0362548828125 1.0
0 -0.239826440811157 2.0
0 0.0326749086380005 3.0
0 0.622209072113037 4.0
0 -0.316011905670166 5.0
0 -0.274505615234375 6.0
0 0.216386914253235 7.0
42.291929941 37.2988624572754 0.0
0 0.114857912063599 1.0
0 -0.298529148101807 2.0
0 -0.0993936061859131 3.0
0 0.538971185684204 4.0
0 -0.305231809616089 5.0
0 0.116536021232605 6.0
0 1.06308543682098 7.0
37.80686855 38.5046119689941 0.0
0 -0.332278251647949 1.0
0 -0.298213720321655 2.0
0 -0.280787229537964 3.0
0 0.235551357269287 4.0
0 -0.124516725540161 5.0
0 -0.383609533309937 6.0
0 0.00841474533081055 7.0
36.296445808 38.3198432922363 0.0
0 -0.0766794681549072 1.0
0 -0.287353277206421 2.0
0 0.0821750164031982 3.0
0 0.0544557571411133 4.0
0 -0.501044988632202 5.0
0 0.0159523487091064 6.0
0 -0.418131113052368 7.0
26.786347985 38.0044059753418 0.0
0 -0.123799562454224 1.0
0 -0.983284473419189 2.0
0 -0.0920861959457397 3.0
0 0.0574706792831421 4.0
0 -0.855048418045044 5.0
0 -0.0195510387420654 6.0
0 -0.345316171646118 7.0
38.715518937 36.4978561401367 0.0
0 0.35184121131897 1.0
0 -0.193002343177795 2.0
0 -0.122087955474854 3.0
0 0.336402058601379 4.0
0 -0.771129608154297 5.0
0 -0.540761947631836 6.0
0 0.730645179748535 7.0
35.505063002 38.0028953552246 0.0
0 -0.140863180160522 1.0
0 -0.197596311569214 2.0
0 -0.19202721118927 3.0
0 0.149364471435547 4.0
0 -0.288889169692993 5.0
0 -0.46938419342041 6.0
0 -0.331581830978394 7.0
44.227649071 38.0720024108887 0.0
0 0.263915061950684 1.0
0 -0.879394054412842 2.0
0 0.219429135322571 3.0
0 0.627953052520752 4.0
0 -0.311311006546021 5.0
0 -0.0484974384307861 6.0
0 2.0666389465332 7.0
45.773124255 36.4662437438965 0.0
0 0.0818374156951904 1.0
0 -0.322307348251343 2.0
0 -0.244776606559753 3.0
0 0.783931493759155 4.0
0 -0.0776770114898682 5.0
0 -0.517922163009644 6.0
0 1.17142295837402 7.0
38.615282752 37.1072196960449 0.0
0 -0.218107461929321 1.0
0 -0.604412078857422 2.0
0 -0.419475555419922 3.0
0 0.781610727310181 4.0
0 -0.497533082962036 5.0
0 0.161868333816528 6.0
0 1.84807538986206 7.0
42.131124836 37.6904602050781 0.0
0 0.327384233474731 1.0
0 -0.306174039840698 2.0
0 -0.322533369064331 3.0
0 0.33608078956604 4.0
0 -0.463847875595093 5.0
0 -0.171356201171875 6.0
0 0.582824468612671 7.0
38.030369766 38.1380462646484 0.0
0 0.0425829887390137 1.0
0 -0.278359413146973 2.0
0 -0.471272468566895 3.0
0 0.63836932182312 4.0
0 -0.663286209106445 5.0
0 0.222291707992554 6.0
0 0.520799875259399 7.0
32.694870002 37.5814323425293 0.0
0 0.520917415618896 1.0
0 -0.252016305923462 2.0
0 -0.0306954383850098 3.0
0 0.0623022317886353 4.0
0 -0.508613586425781 5.0
0 0.0894566774368286 6.0
0 -0.160390734672546 7.0
41.58883184 38.2783088684082 0.0
0 0.441120028495789 1.0
0 -0.501411199569702 2.0
0 -0.0456973314285278 3.0
0 0.479400873184204 4.0
0 -0.24712085723877 5.0
0 -0.439454317092896 6.0
0 0.464501857757568 7.0
33.622063045 38.3645362854004 0.0
0 -0.177744388580322 1.0
0 -0.270464897155762 2.0
0 0.0121710300445557 3.0
0 0.629596471786499 4.0
0 -0.41149377822876 5.0
0 0.201563954353333 6.0
0 -0.126306414604187 7.0
40.101738506 38.0910453796387 0.0
0 -0.177456855773926 1.0
0 -0.0536154508590698 2.0
0 -0.154243350028992 3.0
0 0.489054679870605 4.0
0 -0.645746946334839 5.0
0 -0.344412803649902 6.0
0 1.39563727378845 7.0
39.347750494 38.2497062683105 0.0
0 0.164800643920898 1.0
0 -0.0050736665725708 2.0
0 0.312014579772949 3.0
0 0.607566356658936 4.0
0 -0.328084230422974 5.0
0 -0.199849247932434 6.0
0 1.73740231990814 7.0
46.240008463 38.0902862548828 0.0
0 0.0495458841323853 1.0
0 -0.189947843551636 2.0
0 -0.196909427642822 3.0
0 0.37896466255188 4.0
0 -0.205937027931213 5.0
0 0.0257542133331299 6.0
0 1.01943290233612 7.0
43.885613972 37.9398040771484 0.0
0 -0.0155000686645508 1.0
0 -0.194031119346619 2.0
0 -0.0917253494262695 3.0
0 0.547953844070435 4.0
0 -0.490734577178955 5.0
0 0.109315633773804 6.0
0 1.67961609363556 7.0
44.877286055 37.8741302490234 0.0
0 0.285963177680969 1.0
0 -0.0767428874969482 2.0
0 -0.160351753234863 3.0
0 1.07300281524658 4.0
0 -0.343310594558716 5.0
0 0.344938039779663 6.0
0 1.49842500686646 7.0
44.281024849 38.3132934570312 0.0
0 0.125900268554688 1.0
0 -0.314591646194458 2.0
0 -0.0890324115753174 3.0
0 0.270407676696777 4.0
0 -0.333187341690063 5.0
0 -0.822698831558228 6.0
0 -0.00487363338470459 7.0
39.443195334 38.6590385437012 0.0
0 -0.467075109481812 1.0
0 -0.336113691329956 2.0
0 0.111684679985046 3.0
0 0.558037996292114 4.0
0 -0.337888240814209 5.0
0 -0.141319990158081 6.0
0 -0.105296730995178 7.0
44.50998012 38.1703338623047 0.0
0 -0.120327711105347 1.0
0 -0.219549298286438 2.0
0 -0.068474292755127 3.0
0 0.691309213638306 4.0
0 -0.359362363815308 5.0
0 -0.0211715698242188 6.0
0 0.188111305236816 7.0
36.079576822 35.4511222839355 0.0
0 0.0245699882507324 1.0
0 0.263631820678711 2.0
0 -0.159363746643066 3.0
0 0.571344137191772 4.0
0 -0.352806329727173 5.0
0 -0.467707872390747 6.0
0 0.267959952354431 7.0
43.881758876 38.3761672973633 0.0
0 -0.0888842344284058 1.0
0 1.01139199733734 2.0
0 -0.0382121801376343 3.0
0 0.686092853546143 4.0
0 -0.125787019729614 5.0
0 -0.47356104850769 6.0
0 1.94547319412231 7.0
30.108818284 38.1629486083984 0.0
0 -0.145466923713684 1.0
0 -0.283086061477661 2.0
0 -0.305297374725342 3.0
0 0.138000965118408 4.0
0 -0.883403539657593 5.0
0 0.144310355186462 6.0
0 -0.0441895723342896 7.0
34.22051219 37.0783309936523 0.0
0 0.228223323822021 1.0
0 -0.217992424964905 2.0
0 -0.0129565000534058 3.0
0 0.177544832229614 4.0
0 -0.506984710693359 5.0
0 -0.484919309616089 6.0
0 -0.18371570110321 7.0
37.402197971 38.3397674560547 0.0
0 0.049012303352356 1.0
0 -0.311206102371216 2.0
0 -0.011838436126709 3.0
0 0.666260600090027 4.0
0 -0.210198760032654 5.0
0 0.384181976318359 6.0
0 0.0427552461624146 7.0
35.453722409 38.0516510009766 0.0
0 -0.531721353530884 1.0
0 -0.241238832473755 2.0
0 -0.00494837760925293 3.0
0 0.416697978973389 4.0
0 -0.624515056610107 5.0
0 0.261401176452637 6.0
0 0.172392010688782 7.0
32.872381155 38.104076385498 0.0
0 -0.266526937484741 1.0
0 -0.225950837135315 2.0
0 -0.134938955307007 3.0
0 0.0758655071258545 4.0
0 -0.553425312042236 5.0
0 -0.130931854248047 6.0
0 -0.374738216400146 7.0
33.577714316 37.8498497009277 0.0
0 0.207655310630798 1.0
0 -1.45173406600952 2.0
0 -0.184754967689514 3.0
0 0.674961566925049 4.0
0 -0.353039979934692 5.0
0 -0.148146033287048 6.0
0 0.937549233436584 7.0
43.986168802 38.2138252258301 0.0
0 -0.212903022766113 1.0
0 -0.201224327087402 2.0
0 -0.430400609970093 3.0
0 0.593810677528381 4.0
0 -0.289870500564575 5.0
0 -0.0251104831695557 6.0
0 1.04479575157166 7.0
39.701527205 38.2806510925293 0.0
0 0.0796942710876465 1.0
0 0.208651661872864 2.0
0 -0.0893229246139526 3.0
0 0.309407591819763 4.0
0 -0.679608583450317 5.0
0 0.458317518234253 6.0
0 1.8895343542099 7.0
36.319422317 38.4299011230469 0.0
0 -0.404623746871948 1.0
0 -0.284709215164185 2.0
0 -0.388374328613281 3.0
0 0.697676420211792 4.0
0 -0.421231985092163 5.0
0 -0.4408118724823 6.0
0 0.393359065055847 7.0
39.54652972 37.8792152404785 0.0
0 -0.158182263374329 1.0
0 -0.224109649658203 2.0
0 -0.120337009429932 3.0
0 0.805721998214722 4.0
0 -0.167348623275757 5.0
0 -0.161333680152893 6.0
0 1.54570257663727 7.0
46.3288844 38.0248527526855 0.0
0 0.0665750503540039 1.0
0 -0.293342351913452 2.0
0 -0.0646724700927734 3.0
0 0.761084794998169 4.0
0 -0.0132515430450439 5.0
0 -0.120163440704346 6.0
0 0.698758244514465 7.0
39.359671461 37.4219665527344 0.0
0 0.131548285484314 1.0
0 -0.0573910474777222 2.0
0 -0.175915122032166 3.0
0 0.676822423934937 4.0
0 -0.0782581567764282 5.0
0 -0.484039783477783 6.0
0 0.70098876953125 7.0
43.813870988 38.1995582580566 0.0
0 0.0991947650909424 1.0
0 -0.296068906784058 2.0
0 -0.0536487102508545 3.0
0 0.288695216178894 4.0
0 -0.504193067550659 5.0
0 -0.0455477237701416 6.0
0 0.570007562637329 7.0
39.146221876 37.8919486999512 0.0
0 -0.0408947467803955 1.0
0 -0.0339300632476807 2.0
0 -0.162937164306641 3.0
0 0.490388989448547 4.0
0 -0.787501096725464 5.0
0 -0.412801742553711 6.0
0 1.00490784645081 7.0
34.753353014 38.2137413024902 0.0
0 -0.480408906936646 1.0
0 -0.5872642993927 2.0
0 -0.149401307106018 3.0
0 0.611805558204651 4.0
0 -0.542889595031738 5.0
0 -0.080970287322998 6.0
0 0.907220482826233 7.0
36.858915008 37.3996543884277 0.0
0 0.199391603469849 1.0
0 -0.127014517784119 2.0
0 0.0313220024108887 3.0
0 0.601784706115723 4.0
0 -0.543402433395386 5.0
0 -0.102582693099976 6.0
0 0.645191192626953 7.0
45.609388725 37.039680480957 0.0
0 0.372527241706848 1.0
0 0.701599478721619 2.0
0 -0.189462184906006 3.0
0 0.822058498859406 4.0
0 -0.190669178962708 5.0
0 -0.304826021194458 6.0
0 0.918221950531006 7.0
37.221502314 36.8256874084473 0.0
0 0.202775597572327 1.0
0 -0.274666547775269 2.0
0 -0.371158361434937 3.0
0 0.399163007736206 4.0
0 -0.71528959274292 5.0
0 0.148557066917419 6.0
0 0.83861231803894 7.0
39.351241848 37.3091011047363 0.0
0 0.356141686439514 1.0
0 -0.19051718711853 2.0
0 -0.0770614147186279 3.0
0 0.826613068580627 4.0
0 -0.617578029632568 5.0
0 0.704386949539185 6.0
0 2.37872886657715 7.0
44.354286139 38.1309776306152 0.0
0 -0.173447847366333 1.0
0 -0.210649728775024 2.0
0 -0.0472993850708008 3.0
0 0.342166185379028 4.0
0 -0.354728221893311 5.0
0 -0.401188611984253 6.0
0 0.270106196403503 7.0
41.143843969 37.994945526123 0.0
0 -0.158722400665283 1.0
0 -0.120663285255432 2.0
0 -0.313171625137329 3.0
0 0.442976355552673 4.0
0 -0.448485374450684 5.0
0 -0.0894771814346313 6.0
0 0.432830929756165 7.0
32.436363672 38.1416778564453 0.0
0 -0.369595766067505 1.0
0 3.12781167030334 2.0
0 -0.172399997711182 3.0
0 0.27475106716156 4.0
0 -0.347083806991577 5.0
0 0.104560494422913 6.0
0 -0.069507360458374 7.0
35.980268826 38.5133018493652 0.0
0 -0.289830684661865 1.0
0 -0.395947694778442 2.0
0 0.0918378829956055 3.0
0 0.683108448982239 4.0
0 -0.359841108322144 5.0
0 0.0467770099639893 6.0
0 0.023412823677063 7.0
39.361696951 38.0446891784668 0.0
0 0.310472965240479 1.0
0 -0.188063740730286 2.0
0 -0.268662452697754 3.0
0 0.257854223251343 4.0
0 0.149812459945679 5.0
0 -0.273763418197632 6.0
0 -0.392579793930054 7.0
38.339957501 38.1154136657715 0.0
0 -0.260786533355713 1.0
0 -0.303481578826904 2.0
0 -0.0351171493530273 3.0
0 0.364650726318359 4.0
0 -0.610440492630005 5.0
0 -0.343085289001465 6.0
0 0.149425148963928 7.0
38.437078026 38.2779846191406 0.0
0 -0.453246831893921 1.0
0 -0.367608070373535 2.0
0 -0.342670679092407 3.0
0 0.957582592964172 4.0
0 -0.589494466781616 5.0
0 -0.0839564800262451 6.0
0 3.17460894584656 7.0
31.680301161 37.4142723083496 0.0
0 -0.118972063064575 1.0
0 -0.313036203384399 2.0
0 -0.0933322906494141 3.0
0 0.22541618347168 4.0
0 -0.634193658828735 5.0
0 -0.211985349655151 6.0
0 0.314061760902405 7.0
36.698769132 35.7489013671875 0.0
0 0.390759944915771 1.0
0 0.290920495986938 2.0
0 -0.236075401306152 3.0
0 0.208208560943604 4.0
0 -0.0855264663696289 5.0
0 -0.336171388626099 6.0
0 -0.0649546384811401 7.0
40.942334922 38.2224273681641 0.0
0 -0.052689790725708 1.0
0 -0.124259471893311 2.0
0 -0.157662749290466 3.0
0 0.570372104644775 4.0
0 -0.451802492141724 5.0
0 0.0882890224456787 6.0
0 1.21227943897247 7.0
41.496981537 37.3857307434082 0.0
0 -0.160454511642456 1.0
0 -0.21258819103241 2.0
0 -0.206850051879883 3.0
0 0.924338221549988 4.0
0 -0.429945468902588 5.0
0 -0.200612545013428 6.0
0 2.71049237251282 7.0
33.577466065 37.2047996520996 0.0
0 0.212778329849243 1.0
0 -0.214462161064148 2.0
0 -0.391335725784302 3.0
0 0.55680525302887 4.0
0 -0.792884588241577 5.0
0 0.106175422668457 6.0
0 0.923413872718811 7.0
38.410697595 37.6725540161133 0.0
0 -0.116466999053955 1.0
0 -0.267976045608521 2.0
0 -0.400984525680542 3.0
0 0.330644607543945 4.0
0 -0.314404726028442 5.0
0 0.233519077301025 6.0
0 0.0261133909225464 7.0
40.390522112 37.9872894287109 0.0
0 -0.170293807983398 1.0
0 -0.182642698287964 2.0
0 -0.0806014537811279 3.0
0 0.620949745178223 4.0
0 -0.595688343048096 5.0
0 0.0639975070953369 6.0
0 2.47944211959839 7.0
38.02724699 37.945686340332 0.0
0 0.079007625579834 1.0
0 -0.119932770729065 2.0
0 -0.330389499664307 3.0
0 0.508744835853577 4.0
0 -0.60553240776062 5.0
0 -0.191479921340942 6.0
0 0.497171759605408 7.0
34.831322295 38.291633605957 0.0
0 -0.173713207244873 1.0
0 -0.120435476303101 2.0
0 -0.0277514457702637 3.0
0 0.193899154663086 4.0
0 -0.682255983352661 5.0
0 0.0701934099197388 6.0
0 0.406957387924194 7.0
33.152476832 38.2485389709473 0.0
0 -0.472808122634888 1.0
0 -1.29005885124207 2.0
0 -0.31211519241333 3.0
0 0.423447370529175 4.0
0 -0.86156439781189 5.0
0 0.369147777557373 6.0
0 0.804293870925903 7.0
37.76181476 37.9525451660156 0.0
0 -0.00865960121154785 1.0
0 -0.152747273445129 2.0
0 -0.0935730934143066 3.0
0 0.222189664840698 4.0
0 -0.11743175983429 5.0
0 -0.138501763343811 6.0
0 -0.149877786636353 7.0
46.821650967 38.1566047668457 0.0
0 0.313689708709717 1.0
0 -0.238319873809814 2.0
0 0.0724809169769287 3.0
0 1.02813804149628 4.0
0 -0.0534367561340332 5.0
0 0.13327431678772 6.0
0 1.8259289264679 7.0
41.183790175 36.10693359375 0.0
0 0.0860071182250977 1.0
0 -0.261086463928223 2.0
0 -0.47384786605835 3.0
0 0.67265796661377 4.0
0 -0.381690502166748 5.0
0 0.37049412727356 6.0
0 1.46408760547638 7.0
36.78498864 38.5453491210938 0.0
0 -0.179576396942139 1.0
0 0.691988229751587 2.0
0 -0.196476459503174 3.0
0 0.251587629318237 4.0
0 -0.424649953842163 5.0
0 -0.200451135635376 6.0
0 0.0333437919616699 7.0
37.689494945 38.4253845214844 0.0
0 -0.393067121505737 1.0
0 -0.382611989974976 2.0
0 -0.0758798122406006 3.0
0 0.816988468170166 4.0
0 -0.414283752441406 5.0
0 -0.363934278488159 6.0
0 1.29491460323334 7.0
41.308881097 38.4740142822266 0.0
0 -0.486352205276489 1.0
0 -0.209060788154602 2.0
0 -0.19727885723114 3.0
0 0.723197460174561 4.0
0 -0.170817136764526 5.0
0 -0.663132905960083 6.0
0 1.15917277336121 7.0
42.253011184 37.4200210571289 0.0
0 -0.237241983413696 1.0
0 -0.0737912654876709 2.0
0 -0.289545774459839 3.0
0 0.806572258472443 4.0
0 -0.321501970291138 5.0
0 -0.00569140911102295 6.0
0 1.46328949928284 7.0
40.047660493 37.7273864746094 0.0
0 0.488973259925842 1.0
0 -1.12654209136963 2.0
0 0.0607767105102539 3.0
0 0.770553648471832 4.0
0 -0.132360696792603 5.0
0 0.133897423744202 6.0
0 1.21539068222046 7.0
46.604605733 38.3181114196777 0.0
0 -0.37921142578125 1.0
0 -0.42083477973938 2.0
0 0.12788724899292 3.0
0 0.981654763221741 4.0
0 0.00140345096588135 5.0
0 -0.310238599777222 6.0
0 1.71037018299103 7.0
45.106295542 38.2241592407227 0.0
0 -0.0694361925125122 1.0
0 -0.179256558418274 2.0
0 0.0462667942047119 3.0
0 0.769319117069244 4.0
0 -0.585641145706177 5.0
0 0.118253946304321 6.0
0 2.64816617965698 7.0
38.977055099 37.7579154968262 0.0
0 0.0877690315246582 1.0
0 -0.308596611022949 2.0
0 -0.0645570755004883 3.0
0 0.433469533920288 4.0
0 -0.508968353271484 5.0
0 -0.406071424484253 6.0
0 1.14640176296234 7.0
38.935365883 38.1224174499512 0.0
0 -0.210243463516235 1.0
0 -0.0734668970108032 2.0
0 -0.0771884918212891 3.0
0 0.22931432723999 4.0
0 -0.324558258056641 5.0
0 -0.239970803260803 6.0
0 0.387653231620789 7.0
42.741409918 38.1832618713379 0.0
0 -0.217572927474976 1.0
0 -0.190828800201416 2.0
0 -0.165142178535461 3.0
0 0.644562602043152 4.0
0 -0.0560569763183594 5.0
0 -0.0670759677886963 6.0
0 1.09319579601288 7.0
31.310794576 38.1987800598145 0.0
0 -0.230072021484375 1.0
0 -0.250129342079163 2.0
0 -0.124576330184937 3.0
0 0.185495257377625 4.0
0 -0.436838626861572 5.0
0 -0.755523920059204 6.0
0 -0.323332548141479 7.0
};
\addlegendentry{$R^2$=0.982}
\end{axis}

\end{tikzpicture}
}}
    \subfloat[Actual vs predicted edge flows.] 
    {\label{fig:results_nonlineal_dummy_edge_base_f_bal_wey}\resizebox{\figurewidth}{\figureheight}{% This file was created with tikzplotlib v0.10.1.
\begin{tikzpicture}

\definecolor{darkgray176}{RGB}{176,176,176}
\definecolor{lightgray204}{RGB}{204,204,204}

\begin{axis}[
colorbar,
colorbar style={ylabel={edge id}},
colormap={mymap}{[1pt]
 rgb(0pt)=(0.12156862745098,0.466666666666667,0.705882352941177);
  rgb(1pt)=(1,0.498039215686275,0.0549019607843137);
  rgb(2pt)=(0.172549019607843,0.627450980392157,0.172549019607843);
  rgb(3pt)=(0.83921568627451,0.152941176470588,0.156862745098039);
  rgb(4pt)=(0.580392156862745,0.403921568627451,0.741176470588235);
  rgb(5pt)=(0.549019607843137,0.337254901960784,0.294117647058824);
  rgb(6pt)=(0.890196078431372,0.466666666666667,0.76078431372549);
  rgb(7pt)=(0.498039215686275,0.498039215686275,0.498039215686275);
  rgb(8pt)=(0.737254901960784,0.741176470588235,0.133333333333333);
  rgb(9pt)=(0.0901960784313725,0.745098039215686,0.811764705882353)
},
legend cell align={left},
legend style={
  fill opacity=0.8,
  draw opacity=1,
  text opacity=1,
  at={(0.03,0.97)},
  anchor=north west,
  draw=lightgray204
},
point meta max=7,
point meta min=0,
tick align=outside,
tick pos=left,
title={ye test-ye pred},
x grid style={darkgray176},
xlabel={ye test},
xmajorgrids,
xmin=-15.25776872815, xmax=51.84084685915,
xtick style={color=black},
y grid style={darkgray176},
ylabel={ye pred},
ymajorgrids,
ymin=-11.6745676040649, ymax=43.243258857727,
ytick style={color=black}
]
\addplot [
  colormap={mymap}{[1pt]
 rgb(0pt)=(0.12156862745098,0.466666666666667,0.705882352941177);
  rgb(1pt)=(1,0.498039215686275,0.0549019607843137);
  rgb(2pt)=(0.172549019607843,0.627450980392157,0.172549019607843);
  rgb(3pt)=(0.83921568627451,0.152941176470588,0.156862745098039);
  rgb(4pt)=(0.580392156862745,0.403921568627451,0.741176470588235);
  rgb(5pt)=(0.549019607843137,0.337254901960784,0.294117647058824);
  rgb(6pt)=(0.890196078431372,0.466666666666667,0.76078431372549);
  rgb(7pt)=(0.498039215686275,0.498039215686275,0.498039215686275);
  rgb(8pt)=(0.737254901960784,0.741176470588235,0.133333333333333);
  rgb(9pt)=(0.0901960784313725,0.745098039215686,0.811764705882353)
},
  only marks,
  scatter,
  scatter src=explicit
]
table [x=x, y=y, meta=colordata]{%
x  y  colordata
39.565898635 35.4653205871582 0.0
18.050517226 17.3756046295166 1.0
21.515381409 21.5058689117432 2.0
-8.8987516789 -7.41880321502686 3.0
12.61662972 13.174578666687 4.0
26.949268915 24.6183643341064 5.0
39.565898645 36.8705558776855 6.0
39.565898645 38.4876174926758 7.0
42.743257171 38.666618347168 0.0
22.448801828 18.2283744812012 1.0
20.294455342 16.9746837615967 2.0
-4.2455956832 -6.35125541687012 3.0
16.048859649 15.1489057540894 4.0
26.694397521 25.3805446624756 5.0
42.743257181 39.93994140625 6.0
42.743257181 38.0151405334473 7.0
39.367180747 38.0222969055176 0.0
18.970258592 20.9103965759277 1.0
20.39692216 22.6864452362061 2.0
-4.9776642323 -7.30985355377197 3.0
15.419257923 16.6070251464844 4.0
23.947922829 24.1785202026367 5.0
39.367180751 38.4962539672852 6.0
39.367180751 38.8252449035645 7.0
39.605393097 37.6728630065918 0.0
22.082745928 19.8698539733887 1.0
17.522647169 18.4970073699951 2.0
-7.2517903354 -6.71411561965942 3.0
10.270856823 10.6269388198853 4.0
29.334536274 27.0830173492432 5.0
39.605393107 37.9824638366699 6.0
39.605393107 38.4194946289062 7.0
43.937345526 39.7172431945801 0.0
21.865843593 20.1422176361084 1.0
22.071501933 22.2362079620361 2.0
-8.3469765086 -8.56262683868408 3.0
13.724525414 14.8643665313721 4.0
30.212820112 29.4784488677979 5.0
43.937345536 39.1973419189453 6.0
43.937345536 38.46826171875 7.0
31.061989584 37.4646873474121 0.0
16.203677385 19.0510768890381 1.0
14.858312199 18.1562156677246 2.0
-5.5704045507 -5.61095571517944 3.0
9.2879076384 9.98746395111084 4.0
21.774081945 21.4392986297607 5.0
31.061989594 37.8200721740723 6.0
31.061989594 37.8142585754395 7.0
36.357435265 37.29736328125 0.0
19.38173709 20.9259548187256 1.0
16.975698176 21.932596206665 2.0
-8.3424496148 -7.24617290496826 3.0
8.6332485532 10.7698469161987 4.0
27.724186714 26.1984100341797 5.0
36.357435273 37.4389686584473 6.0
36.357435273 38.7885589599609 7.0
38.444969607 34.1662254333496 0.0
21.080570389 19.7358665466309 1.0
17.364399218 17.5085773468018 2.0
-5.0980379196 -6.54931449890137 3.0
12.266361289 11.6689586639404 4.0
26.178608318 24.9278354644775 5.0
38.444969617 36.8102188110352 6.0
38.444969617 38.131175994873 7.0
35.498620518 39.0944938659668 0.0
17.900031823 18.6953296661377 1.0
17.598588698 18.4822673797607 2.0
-4.651704216 -5.25825500488281 3.0
12.946884476 12.9584341049194 4.0
22.551736046 23.0617561340332 5.0
35.498620524 39.8168487548828 6.0
35.498620524 37.9859809875488 7.0
36.52099827 37.8296165466309 0.0
18.961240079 19.2161827087402 1.0
17.559758192 18.9821796417236 2.0
-5.3314768445 -5.96808385848999 3.0
12.228281339 12.8238792419434 4.0
24.292716932 24.1209850311279 5.0
36.520998279 38.6563262939453 6.0
36.520998279 38.4122314453125 7.0
36.717272204 37.5162620544434 0.0
19.763035348 21.4768199920654 1.0
16.954236857 21.1295642852783 2.0
-7.0689091682 -6.11248445510864 3.0
9.88532768 11.1671953201294 4.0
26.831944525 25.1821346282959 5.0
36.717272212 38.3054389953613 6.0
36.717272212 38.8544960021973 7.0
32.629628996 36.7428092956543 0.0
17.103127104 20.8836612701416 1.0
15.526501892 22.1743087768555 2.0
-7.1999965109 -6.40161752700806 3.0
8.3265053708 10.3728590011597 4.0
24.303123625 24.7735805511475 5.0
32.629629006 36.9937705993652 6.0
32.629629006 38.9704780578613 7.0
37.75267433 37.3057136535645 0.0
17.533699782 18.6969738006592 1.0
20.218974548 21.6238555908203 2.0
-3.919745576 -5.51858329772949 3.0
16.299228962 17.2922649383545 4.0
21.453445368 21.9420394897461 5.0
37.75267434 37.4594764709473 6.0
37.75267434 38.5565376281738 7.0
38.800291337 36.8951759338379 0.0
21.414385846 20.1205978393555 1.0
17.385905491 17.9326210021973 2.0
-6.3164463541 -6.63514518737793 3.0
11.069459127 11.2369613647461 4.0
27.73083221 27.7883319854736 5.0
38.800291347 38.1491203308105 6.0
38.800291347 38.1593437194824 7.0
38.252729609 37.4027366638184 0.0
20.199036122 19.1417007446289 1.0
18.053693488 18.9128494262695 2.0
-6.9495443437 -6.66568946838379 3.0
11.104149136 11.0922813415527 4.0
27.148580475 26.2323360443115 5.0
38.252729618 37.8708381652832 6.0
38.252729618 38.306079864502 7.0
43.273596037 39.2056999206543 0.0
20.811072897 17.881778717041 1.0
22.46252314 21.8325176239014 2.0
-10.185324283 -7.62959575653076 3.0
12.277198847 13.3323535919189 4.0
30.99639719 28.748254776001 5.0
43.273596047 39.9000854492188 6.0
43.273596047 38.3842926025391 7.0
34.027431477 39.3840789794922 0.0
16.463058834 16.4280853271484 1.0
17.564372643 18.6851749420166 2.0
-7.8238749469 -7.8075532913208 3.0
9.7404976863 10.2903823852539 4.0
24.28693379 24.8640155792236 5.0
34.027431486 38.4053153991699 6.0
34.027431486 37.7713508605957 7.0
41.154171382 38.3997192382812 0.0
20.969763 18.6950950622559 1.0
20.184408383 19.7631797790527 2.0
-9.8031418559 -7.53774070739746 3.0
10.381266519 11.3606691360474 4.0
30.772904865 27.773645401001 5.0
41.154171391 37.9016342163086 6.0
41.154171391 37.8784332275391 7.0
39.9308264078757 38.2566032409668 0.0
21.2554356118761 18.9973220825195 1.0
18.6753908028755 18.9338874816895 2.0
-8.41978314337461 -6.19795846939087 3.0
10.2556076588756 10.6592283248901 4.0
29.6752187558758 25.6492652893066 5.0
39.930826408 38.1504592895508 6.0
39.930826408 37.9991455078125 7.0
45.969703622 38.673152923584 0.0
25.855804141 21.0030612945557 1.0
20.113899483 16.6713199615479 2.0
-4.729541314 -5.93174934387207 3.0
15.38435816 14.5208530426025 4.0
30.585345464 28.7320556640625 5.0
45.969703631 38.909366607666 6.0
45.969703631 37.9774436950684 7.0
38.398120212 36.842960357666 0.0
19.546425793 16.2819404602051 1.0
18.851694421 15.6090288162231 2.0
-9.8922695857 -7.01049327850342 3.0
8.9594248259 8.86262607574463 4.0
29.438695387 26.5986728668213 5.0
38.398120221 38.1087913513184 6.0
38.398120221 37.0829010009766 7.0
28.366017296 35.5328559875488 0.0
14.995271371 20.4606666564941 1.0
13.370745924 21.3311176300049 2.0
-4.4280592982 -5.04580402374268 3.0
8.9426866162 10.3953104019165 4.0
19.42333068 20.0415744781494 5.0
28.366017306 37.5268058776855 6.0
28.366017306 38.9739112854004 7.0
39.07981897 37.2415809631348 0.0
18.202914308 17.6860256195068 1.0
20.876904662 21.1985359191895 2.0
-6.6281228283 -6.97441244125366 3.0
14.248781824 14.703784942627 4.0
24.831037146 24.9639759063721 5.0
39.079818979 37.5848426818848 6.0
39.079818979 38.3440475463867 7.0
40.466329374 37.584400177002 0.0
19.842708002 20.2735328674316 1.0
20.623621372 21.8903045654297 2.0
-6.0436252005 -6.61271286010742 3.0
14.579996162 16.0458526611328 4.0
25.886333212 26.2317142486572 5.0
40.466329383 38.1175079345703 6.0
40.466329383 39.024528503418 7.0
38.293116843 36.7522659301758 0.0
18.225197444 20.2101020812988 1.0
20.067919399 22.7675037384033 2.0
-4.6459628917 -5.83757257461548 3.0
15.421956498 15.6366138458252 4.0
22.871160346 23.9102439880371 5.0
38.293116853 37.312255859375 6.0
38.293116853 39.0198364257812 7.0
43.346089487 39.0412139892578 0.0
21.23189324 18.4254016876221 1.0
22.114196248 21.7744655609131 2.0
-9.387268107 -7.61680889129639 3.0
12.726928131 12.4937238693237 4.0
30.619161357 29.5391025543213 5.0
43.346089497 37.9473190307617 6.0
43.346089497 38.5031623840332 7.0
34.547871144 38.0280075073242 0.0
19.900789633 20.764591217041 1.0
14.647081512 19.1177883148193 2.0
-6.6924528176 -7.11344003677368 3.0
7.9546286844 9.17579174041748 4.0
26.59324246 26.6384811401367 5.0
34.547871154 38.2960624694824 6.0
34.547871154 38.3977241516113 7.0
41.927050143208 35.3876533508301 0.0
21.8340017142739 16.672191619873 1.0
20.0930484299516 17.920804977417 2.0
-12.0156472013624 -8.13709545135498 3.0
8.07740122783537 9.05303287506104 4.0
33.8496486756856 29.4292640686035 5.0
41.927050143 37.0138893127441 6.0
41.927050143 37.8010787963867 7.0
44.548236367 37.9736862182617 0.0
22.097953544 18.7023639678955 1.0
22.450282823 20.1144523620605 2.0
-6.9116853021 -7.28447818756104 3.0
15.538597511 15.127100944519 4.0
29.009638856 27.0433349609375 5.0
44.548236377 37.7228050231934 6.0
44.548236377 38.2209701538086 7.0
34.958415477 38.6017303466797 0.0
19.046265323 20.0808353424072 1.0
15.912150156 17.3438911437988 2.0
-3.947719982 -5.67544555664062 3.0
11.964430164 11.364031791687 4.0
22.993985314 23.5479564666748 5.0
34.958415487 39.6270332336426 6.0
34.958415487 38.0484466552734 7.0
41.619773288 39.4502906799316 0.0
18.036088777 16.7410888671875 1.0
23.583684511 20.9694309234619 2.0
-9.1140217471 -6.45247220993042 3.0
14.469662754 15.533860206604 4.0
27.150110534 25.2907733917236 5.0
41.619773298 38.7763710021973 6.0
41.619773298 37.9865455627441 7.0
35.768623904 33.6828956604004 0.0
18.854929214 19.1714649200439 1.0
16.913694692 19.659252166748 2.0
-6.8203969536 -6.2099494934082 3.0
10.093297731 10.9762487411499 4.0
25.675326176 25.4097499847412 5.0
35.768623912 36.622257232666 6.0
35.768623912 38.401195526123 7.0
36.035873071 39.5147476196289 0.0
18.197202866 18.034688949585 1.0
17.838670207 20.5167541503906 2.0
-9.8802007315 -6.90202665328979 3.0
7.958469467 9.81205654144287 4.0
28.077403606 26.8622741699219 5.0
36.03587308 38.5354652404785 6.0
36.03587308 38.0508232116699 7.0
42.671159575 38.2762565612793 0.0
22.542382477 18.2093620300293 1.0
20.128777099 17.9436073303223 2.0
-9.3171954196 -6.64804077148438 3.0
10.811581671 11.2003002166748 4.0
31.859577906 27.7768325805664 5.0
42.671159584 38.0575981140137 6.0
42.671159584 37.6818923950195 7.0
32.270518381 37.0702438354492 0.0
16.341237488 18.2197017669678 1.0
15.929280893 20.0205383300781 2.0
-7.2560391874 -6.180100440979 3.0
8.6732416955 9.86216449737549 4.0
23.597276685 22.9922466278076 5.0
32.270518391 38.1774864196777 6.0
32.270518391 38.281078338623 7.0
36.239834932 40.7256011962891 0.0
18.511196273 16.8397960662842 1.0
17.72863866 17.1440944671631 2.0
-5.3875241311 -5.80819940567017 3.0
12.34111452 12.437572479248 4.0
23.898720413 24.85595703125 5.0
36.239834941 39.0537071228027 6.0
36.239834941 37.7124137878418 7.0
38.292005172 37.223274230957 0.0
20.144840979 18.8875827789307 1.0
18.147164195 18.4090003967285 2.0
-6.9715526507 -6.98413896560669 3.0
11.175611535 11.1395320892334 4.0
27.116393638 27.5400428771973 5.0
38.292005181 37.5089378356934 6.0
38.292005181 38.35302734375 7.0
41.142171116 36.6408348083496 0.0
21.697070846 21.4725742340088 1.0
19.44510027 19.843900680542 2.0
-5.49061917 -5.73292922973633 3.0
13.95448109 14.0705699920654 4.0
27.187690026 24.9555797576904 5.0
41.142171126 38.3615570068359 6.0
41.142171126 38.8515357971191 7.0
30.660234741 35.2469940185547 0.0
16.571808951 20.5061511993408 1.0
14.08842579 17.9669532775879 2.0
-3.6405774968 -5.44748020172119 3.0
10.447848283 10.8117570877075 4.0
20.212386458 21.7120456695557 5.0
30.660234751 37.6594505310059 6.0
30.660234751 38.490364074707 7.0
42.776716976 38.535343170166 0.0
18.943122536 17.0984287261963 1.0
23.83359444 21.3234806060791 2.0
-9.5992659914 -6.33389854431152 3.0
14.234328438 14.1220865249634 4.0
28.542388537 26.5163593292236 5.0
42.776716986 37.8880958557129 6.0
42.776716986 38.3730964660645 7.0
39.136657948 40.4559555053711 0.0
21.244730642 18.6975498199463 1.0
17.891927309 17.8959159851074 2.0
-5.2069844714 -6.79718017578125 3.0
12.684942831 11.5512933731079 4.0
26.45171512 26.5308933258057 5.0
39.136657955 39.6542663574219 6.0
39.136657955 38.0883674621582 7.0
40.593075656 38.9321594238281 0.0
19.181149666 16.5812149047852 1.0
21.411925992 20.7951469421387 2.0
-11.788434371 -8.55827903747559 3.0
9.6234916129 11.1027517318726 4.0
30.969584045 29.2816753387451 5.0
40.593075664 38.6607398986816 6.0
40.593075664 38.0511245727539 7.0
38.455960047 40.0799751281738 0.0
19.606731652 16.5889930725098 1.0
18.849228395 17.4694442749023 2.0
-10.066988594 -7.38510036468506 3.0
8.7822397903 9.59571361541748 4.0
29.673720256 28.2650299072266 5.0
38.455960057 39.345458984375 6.0
38.455960057 37.5471534729004 7.0
40.029822299 35.9868965148926 0.0
19.696760735 17.2216205596924 1.0
20.333061566 18.7908496856689 2.0
-6.9313004298 -6.32973909378052 3.0
13.401761128 14.0268583297729 4.0
26.628061173 27.1875228881836 5.0
40.029822307 37.0472946166992 6.0
40.029822307 37.7197799682617 7.0
39.721089796 37.9785041809082 0.0
21.982867862 20.3423328399658 1.0
17.738221933 18.6634120941162 2.0
-6.2167396454 -6.00539112091064 3.0
11.521482278 11.905387878418 4.0
28.199607518 26.0543575286865 5.0
39.721089806 37.5427932739258 6.0
39.721089806 38.499942779541 7.0
46.012802771 36.0985069274902 0.0
23.852238991 16.2848243713379 1.0
22.16056378 15.7384719848633 2.0
-9.7843125098 -6.66674089431763 3.0
12.37625126 13.0209121704102 4.0
33.636551511 29.6966228485107 5.0
46.012802781 37.8373908996582 6.0
46.012802781 37.0415802001953 7.0
43.79104115 37.1208152770996 0.0
23.810489927 19.2008628845215 1.0
19.980551224 19.2718296051025 2.0
-8.8972279263 -7.37373352050781 3.0
11.083323289 11.0026369094849 4.0
32.707717862 28.0619678497314 5.0
43.791041158 37.3797302246094 6.0
43.791041158 38.2214546203613 7.0
31.257332414 35.538501739502 0.0
14.677686479 17.7423572540283 1.0
16.579645934 22.5990123748779 2.0
-7.7653359569 -7.41388702392578 3.0
8.8143099674 11.033242225647 4.0
22.443022446 21.8734645843506 5.0
31.257332424 36.9123497009277 6.0
31.257332424 38.5729942321777 7.0
38.988472902 33.8966522216797 0.0
18.706030748 20.1319179534912 1.0
20.282442155 22.3169937133789 2.0
-4.4931154738 -6.36370992660522 3.0
15.789326673 17.1311988830566 4.0
23.19914623 23.0193576812744 5.0
38.98847291 36.4157600402832 6.0
38.98847291 38.8160400390625 7.0
38.691218489 38.3732147216797 0.0
19.181746937 20.690580368042 1.0
19.509471553 21.0897750854492 2.0
-4.305791265 -5.15237808227539 3.0
15.203680278 14.5934247970581 4.0
23.487538212 23.1051979064941 5.0
38.691218499 38.7862396240234 6.0
38.691218499 38.9283103942871 7.0
39.033211962 37.6705360412598 0.0
21.751284426 21.1437644958496 1.0
17.281927536 18.8009243011475 2.0
-7.1228573229 -6.86285448074341 3.0
10.159070204 10.7049341201782 4.0
28.874141758 27.0112285614014 5.0
39.033211971 37.8716926574707 6.0
39.033211971 38.4241104125977 7.0
37.697547803 36.0761299133301 0.0
18.641388049 19.510103225708 1.0
19.056159754 22.3363666534424 2.0
-6.2754402376 -7.43240833282471 3.0
12.780719506 13.7744407653809 4.0
24.916828297 24.5542697906494 5.0
37.697547813 36.9770927429199 6.0
37.697547813 38.6161003112793 7.0
35.277541329 39.0147132873535 0.0
16.862419132 18.6056060791016 1.0
18.415122198 20.8864612579346 2.0
-6.2131742864 -5.50057983398438 3.0
12.201947903 12.9447164535522 4.0
23.075593428 22.1663055419922 5.0
35.277541339 39.2546882629395 6.0
35.277541339 38.2692642211914 7.0
36.1197639664031 39.1642799377441 0.0
19.1517629624013 19.5875949859619 1.0
16.9680010143999 17.3812294006348 2.0
-0.473187581125011 -5.52647924423218 3.0
16.4948134334014 16.1167793273926 4.0
19.6249505434022 20.3062381744385 5.0
36.119763966 39.1540451049805 6.0
36.119763966 37.6578979492188 7.0
33.14490304 40.5488967895508 0.0
16.831426445 19.9976482391357 1.0
16.313476595 19.342903137207 2.0
-3.637680404 -5.99765729904175 3.0
12.675796181 13.3142366409302 4.0
20.469106859 21.713264465332 5.0
33.14490305 39.5158500671387 6.0
33.14490305 38.3032035827637 7.0
34.4868008142505 34.6715927124023 0.0
18.3228058422505 19.9605255126953 1.0
16.1639949822505 20.8186855316162 2.0
-8.00476657755046 -6.63871622085571 3.0
8.15922840445046 9.78540802001953 4.0
26.3275723771958 25.156867980957 5.0
34.486800814 36.8346557617188 6.0
34.486800814 38.5365905761719 7.0
40.933468199 38.6581535339355 0.0
20.889274427 21.2290420532227 1.0
20.044193774 21.7453689575195 2.0
-6.533500242 -7.73348331451416 3.0
13.510693524 15.6935615539551 4.0
27.422774677 26.9105186462402 5.0
40.933468207 38.9740524291992 6.0
40.933468207 38.8520431518555 7.0
35.993417919 35.9673538208008 0.0
18.964188614 21.0168151855469 1.0
17.029229306 21.9043769836426 2.0
-7.4575389686 -7.1358380317688 3.0
9.5716903284 11.0588760375977 4.0
26.421727592 25.1010398864746 5.0
35.993417928 37.8782768249512 6.0
35.993417928 38.9797172546387 7.0
39.331666914 37.9793167114258 0.0
19.408314017 16.9057846069336 1.0
19.923352898 17.782735824585 2.0
-7.2428639103 -7.28089427947998 3.0
12.680488979 12.3145294189453 4.0
26.651177937 26.5895519256592 5.0
39.331666923 38.0542449951172 6.0
39.331666923 37.6314277648926 7.0
37.559548486 38.9111251831055 0.0
18.054642609 17.6203556060791 1.0
19.504905877 20.8733539581299 2.0
-8.2806473226 -6.94800233840942 3.0
11.224258545 12.0297651290894 4.0
26.335289942 25.5811729431152 5.0
37.559548496 38.6911659240723 6.0
37.559548496 38.2905158996582 7.0
41.796902472 39.2585258483887 0.0
20.074580702 16.4526443481445 1.0
21.72232177 19.7667598724365 2.0
-11.24258785 -8.64604473114014 3.0
10.479733911 11.0139245986938 4.0
31.317168562 29.4223155975342 5.0
41.796902482 39.1358375549316 6.0
41.796902482 37.8080596923828 7.0
35.679590813 36.9355697631836 0.0
15.932888769 16.9123687744141 1.0
19.746702044 21.601375579834 2.0
-7.4161334139 -6.91284418106079 3.0
12.330568621 13.2260074615479 4.0
23.349022193 24.2337055206299 5.0
35.679590823 37.1244850158691 6.0
35.679590823 38.1030616760254 7.0
33.227547284 34.3156051635742 0.0
19.156810973 20.4620838165283 1.0
14.070736312 17.2766647338867 2.0
-5.2205843651 -6.5874924659729 3.0
8.8501519384 9.00404262542725 4.0
24.377395346 25.5241756439209 5.0
33.227547292 36.6083755493164 6.0
33.227547292 38.2035179138184 7.0
28.00807173 38.6212196350098 0.0
12.909757219 15.7199649810791 1.0
15.098314511 17.7205505371094 2.0
-4.3989930962 -4.68946933746338 3.0
10.699321405 11.2861490249634 4.0
17.308750325 20.8392677307129 5.0
28.008071739 38.8385848999023 6.0
28.008071739 37.5811386108398 7.0
33.4784988410636 36.802116394043 0.0
17.4596045520637 17.543004989624 1.0
16.0188942990637 17.9615440368652 2.0
-6.65519022176347 -6.66371440887451 3.0
9.36370407706344 9.64588642120361 4.0
24.1147947730637 25.0955944061279 5.0
33.478498841 37.6718788146973 6.0
33.478498841 38.0194320678711 7.0
33.1229468201162 39.239185333252 0.0
16.2738036511163 16.8806571960449 1.0
16.8491431771162 20.2441291809082 2.0
-8.48238354211593 -5.97146987915039 3.0
8.36675963501607 10.1942710876465 4.0
24.7561871931169 24.7002410888672 5.0
33.12294682 39.3540840148926 6.0
33.12294682 37.602611541748 7.0
34.970228375 37.2638130187988 0.0
17.461716438 19.4523582458496 1.0
17.508511937 19.3418807983398 2.0
-3.5027478521 -4.81587600708008 3.0
14.005764076 14.176157951355 4.0
20.964464299 22.3750553131104 5.0
34.970228384 37.5186462402344 6.0
34.970228384 38.3038063049316 7.0
36.637966033 38.1349296569824 0.0
18.611959282 19.6738548278809 1.0
18.026006753 19.80002784729 2.0
-5.1789935919 -5.54611206054688 3.0
12.847013153 13.0047369003296 4.0
23.790952882 25.0252285003662 5.0
36.637966041 37.7208976745605 6.0
36.637966041 38.1831436157227 7.0
38.712447285 35.8302726745605 0.0
19.874849154 17.494083404541 1.0
18.837598131 19.472827911377 2.0
-9.656315418 -7.67136573791504 3.0
9.181282703 9.56632614135742 4.0
29.531164582 28.0461978912354 5.0
38.712447295 36.8873558044434 6.0
38.712447295 37.9137840270996 7.0
29.0840224199388 39.7540245056152 0.0
12.9348822379388 16.7235908508301 1.0
16.1491401919388 21.3575019836426 2.0
-6.2158408037388 -5.37837934494019 3.0
9.93329938853882 10.9627370834351 4.0
19.1507230409388 19.9845485687256 5.0
29.08402242 39.8738212585449 6.0
29.08402242 38.2873725891113 7.0
40.7419030105752 37.5099678039551 0.0
19.6019574718567 16.9492511749268 1.0
21.139945535663 19.5421485900879 2.0
-8.29196876971742 -6.90576171875 3.0
12.8479767652648 12.9394598007202 4.0
27.8939262422712 26.4139614105225 5.0
40.741903011 37.6844329833984 6.0
40.741903011 38.103141784668 7.0
44.4239279780974 35.4817657470703 0.0
22.6860882760974 18.7149219512939 1.0
21.7378397120974 18.4618644714355 2.0
-6.03315836089742 -5.73828840255737 3.0
15.7046813510974 16.988452911377 4.0
28.7192466370973 26.1901931762695 5.0
44.423927978 36.6100845336914 6.0
44.423927978 38.103572845459 7.0
37.279929002 39.7993659973145 0.0
17.697440733 17.5055980682373 1.0
19.582488269 20.57568359375 2.0
-8.4860638308 -6.98848628997803 3.0
11.096424428 11.0373001098633 4.0
26.183504574 24.8732032775879 5.0
37.279929011 38.3934898376465 6.0
37.279929011 38.3291358947754 7.0
39.065196556 39.1243553161621 0.0
22.911627527 20.9004516601562 1.0
16.153569029 18.0072937011719 2.0
-7.035333012 -6.13697195053101 3.0
9.1182360067 9.82585620880127 4.0
29.946960549 29.2519931793213 5.0
39.065196566 37.9887580871582 6.0
39.065196566 38.3311882019043 7.0
34.346712901 37.7470741271973 0.0
17.419062718 19.9809494018555 1.0
16.927650183 20.2644786834717 2.0
-3.8964434363 -5.00055122375488 3.0
13.031206737 12.9184484481812 4.0
21.315506165 22.0776691436768 5.0
34.346712911 37.6354637145996 6.0
34.346712911 38.6239585876465 7.0
44.832836192 36.3437652587891 0.0
22.86238329 18.2994632720947 1.0
21.970452902 20.2435283660889 2.0
-8.0950157692 -6.97081327438354 3.0
13.875437123 14.2393379211426 4.0
30.957399069 29.6271419525146 5.0
44.832836202 37.0530090332031 6.0
44.832836202 38.3020668029785 7.0
37.693005907 39.5398483276367 0.0
18.767233493 19.1133327484131 1.0
18.925772415 19.3068504333496 2.0
-4.4039460583 -6.81012058258057 3.0
14.521826348 13.5349493026733 4.0
23.17117956 23.4890308380127 5.0
37.693005916 38.6137199401855 6.0
37.693005916 37.9704399108887 7.0
35.516030196 37.4118118286133 0.0
17.238388415 17.7522010803223 1.0
18.277641781 19.9813842773438 2.0
-5.9879556024 -6.2247142791748 3.0
12.289686168 12.4118461608887 4.0
23.226344027 23.9958438873291 5.0
35.516030206 38.0722198486328 6.0
35.516030206 38.1066436767578 7.0
45.449957515 39.6227760314941 0.0
24.470260824 20.8842582702637 1.0
20.979696692 19.3156452178955 2.0
-8.6547786371 -7.18789482116699 3.0
12.324918046 12.1942052841187 4.0
33.12503947 28.9034843444824 5.0
45.449957523 40.0229148864746 6.0
45.449957523 38.5502548217773 7.0
36.522103961 36.5871963500977 0.0
17.787809101 15.5215396881104 1.0
18.734294862 15.3080177307129 2.0
-3.7955939001 -4.77981615066528 3.0
14.938700953 14.6207628250122 4.0
21.583403009 23.6144771575928 5.0
36.52210397 37.4491806030273 6.0
36.52210397 37.2467880249023 7.0
40.809333824 38.2446632385254 0.0
21.825363738 19.7938346862793 1.0
18.983970087 17.2958889007568 2.0
-4.5096782574 -6.01210737228394 3.0
14.474291821 14.3636341094971 4.0
26.335042004 25.1934623718262 5.0
40.809333833 37.9128227233887 6.0
40.809333833 37.9802665710449 7.0
43.708815015 36.015625 0.0
22.06122806 18.6328430175781 1.0
21.647586954 19.3943347930908 2.0
-7.2998869135 -6.38581562042236 3.0
14.347700031 14.1700811386108 4.0
29.361114984 27.6807346343994 5.0
43.708815025 37.1772918701172 6.0
43.708815025 38.20751953125 7.0
35.746307566 32.6347389221191 0.0
17.418825391 16.9106388092041 1.0
18.327482175 18.4308776855469 2.0
-8.2684434585 -6.40064001083374 3.0
10.059038708 11.0958557128906 4.0
25.687268859 24.3468036651611 5.0
35.746307575 36.3313865661621 6.0
35.746307575 37.3810997009277 7.0
32.565260397 34.68701171875 0.0
15.347203209 16.5264835357666 1.0
17.218057188 18.5787410736084 2.0
-4.4968959364 -5.90532684326172 3.0
12.721161242 12.3210496902466 4.0
19.844099155 20.9270458221436 5.0
32.565260407 36.3100852966309 6.0
32.565260407 37.8091812133789 7.0
37.459437559 35.6610145568848 0.0
20.974106733 21.3789138793945 1.0
16.485330826 21.7622585296631 2.0
-7.8025517563 -6.79418182373047 3.0
8.6827790601 9.85436725616455 4.0
28.776658499 26.9120502471924 5.0
37.459437569 36.9886894226074 6.0
37.459437569 39.0450553894043 7.0
41.6008687772188 38.4161148071289 0.0
18.8099504026802 16.9404411315918 1.0
22.7909183777866 20.3920154571533 2.0
-6.50941037372392 -6.65179634094238 3.0
16.2815080036091 16.3526134490967 4.0
25.319360776445 25.8084678649902 5.0
41.600868777 38.0057640075684 6.0
41.600868777 37.384090423584 7.0
43.08864828 38.1198883056641 0.0
20.75366021 17.7513256072998 1.0
22.334988069 20.8185501098633 2.0
-9.5492158301 -7.06487464904785 3.0
12.785772229 14.3090744018555 4.0
30.30287605 29.6633243560791 5.0
43.088648289 37.722713470459 6.0
43.088648289 38.3106460571289 7.0
30.268152668 37.9134101867676 0.0
15.014233699 19.1410713195801 1.0
15.25391897 21.4970626831055 2.0
-5.5931438946 -6.88521575927734 3.0
9.6607750663 11.1947813034058 4.0
20.607377602 22.3481540679932 5.0
30.268152677 38.0093841552734 6.0
30.268152677 38.4102783203125 7.0
38.454045879 36.7790794372559 0.0
17.159579151 16.8621997833252 1.0
21.29446673 20.6027946472168 2.0
-7.5807931052 -6.34041500091553 3.0
13.713673616 14.2096071243286 4.0
24.740372265 24.3069190979004 5.0
38.454045888 37.5153007507324 6.0
38.454045888 37.8663749694824 7.0
36.654056854 36.6972846984863 0.0
19.439010257 19.8712711334229 1.0
17.215046598 20.8875598907471 2.0
-7.9864590963 -6.56668138504028 3.0
9.2285874923 10.5909986495972 4.0
27.425469363 27.5843372344971 5.0
36.654056864 37.3461532592773 6.0
36.654056864 38.6250457763672 7.0
36.990278119 39.6796607971191 0.0
18.644416453 18.7333354949951 1.0
18.345861667 21.5480861663818 2.0
-7.9977544937 -8.10555171966553 3.0
10.348107164 11.3741865158081 4.0
26.642170956 26.6637172698975 5.0
36.990278128 38.4804801940918 6.0
36.990278128 38.6059799194336 7.0
39.45539178 38.1850357055664 0.0
17.953554537 18.3077945709229 1.0
21.501837243 23.1326274871826 2.0
-6.7799927019 -7.15221738815308 3.0
14.721844531 15.8494625091553 4.0
24.733547249 24.8081092834473 5.0
39.45539179 38.0632934570312 6.0
39.45539179 38.9635200500488 7.0
41.490098301 38.1328582763672 0.0
18.418139969 17.1661739349365 1.0
23.071958334 21.1780757904053 2.0
-6.7020850264 -6.26236629486084 3.0
16.369873299 16.7049694061279 4.0
25.120225004 24.7541561126709 5.0
41.49009831 39.2599792480469 6.0
41.49009831 38.2299461364746 7.0
42.195848756 38.1313667297363 0.0
18.431626231 16.7935009002686 1.0
23.764222526 21.5665130615234 2.0
-9.6629820221 -7.97653102874756 3.0
14.101240496 14.8084964752197 4.0
28.094608262 27.7169704437256 5.0
42.195848764 38.7598915100098 6.0
42.195848764 38.3484497070312 7.0
36.580423947 37.5942420959473 0.0
18.088597645 16.3336277008057 1.0
18.491826302 17.2920608520508 2.0
-10.331066357 -7.89533233642578 3.0
8.1607599346 9.35499858856201 4.0
28.419664012 28.9472122192383 5.0
36.580423956 37.8751411437988 6.0
36.580423956 37.3316955566406 7.0
41.112612837 39.2198143005371 0.0
21.264748673 16.8600559234619 1.0
19.847864165 17.7608680725098 2.0
-7.8539892402 -5.55823802947998 3.0
11.993874916 11.7000770568848 4.0
29.118737922 27.9035739898682 5.0
41.112612846 39.8083763122559 6.0
41.112612846 37.7253952026367 7.0
41.165032295 36.2949714660645 0.0
20.777866565 18.5140781402588 1.0
20.387165729 22.0810794830322 2.0
-11.013080697 -8.45360279083252 3.0
9.3740850227 11.5126523971558 4.0
31.790947272 29.789493560791 5.0
41.165032305 36.829216003418 6.0
41.165032305 38.6028633117676 7.0
28.23632976 37.6028633117676 0.0
14.450875753 16.7008533477783 1.0
13.785454008 17.3421630859375 2.0
-5.4067157656 -5.08211231231689 3.0
8.3787382334 8.78112983703613 4.0
19.857591527 20.49245262146 5.0
28.236329769 38.1504440307617 6.0
28.236329769 37.7352828979492 7.0
46.563722225 38.8589248657227 0.0
24.514076121 20.639762878418 1.0
22.049646104 19.9430713653564 2.0
-6.9458736065 -7.1668848991394 3.0
15.103772488 14.6044998168945 4.0
31.459949737 28.6262683868408 5.0
46.563722235 38.8097953796387 6.0
46.563722235 38.232608795166 7.0
38.976039438 37.331958770752 0.0
20.47408062 20.6113166809082 1.0
18.50195882 18.8984298706055 2.0
-3.6791774969 -5.29188442230225 3.0
14.822781314 14.4449262619019 4.0
24.153258125 24.5462341308594 5.0
38.976039446 37.931396484375 6.0
38.976039446 38.5800628662109 7.0
43.902220155 38.9752731323242 0.0
23.085534006 21.0071392059326 1.0
20.816686149 19.0826930999756 2.0
-6.0778252495 -6.89632034301758 3.0
14.738860889 14.4391775131226 4.0
29.163359266 29.1278705596924 5.0
43.902220165 37.8686408996582 6.0
43.902220165 38.2332305908203 7.0
37.104870377 37.8965644836426 0.0
21.928667997 20.4923725128174 1.0
15.176202381 16.6892337799072 2.0
-7.1829261037 -6.325767993927 3.0
7.9932762681 7.56781196594238 4.0
29.11159411 28.4807929992676 5.0
37.104870385 37.7768020629883 6.0
37.104870385 37.8745918273926 7.0
48.094555135036 36.6954154968262 0.0
24.9788607525581 16.9844570159912 1.0
23.1156943831877 15.9990739822388 2.0
-7.58647576369851 -6.69425821304321 3.0
15.5292186191899 16.1874923706055 4.0
32.5653365171039 29.9328441619873 5.0
48.094555135 37.2086639404297 6.0
48.094555135 36.9048271179199 7.0
39.863550950908 34.5463714599609 0.0
20.2748443148297 16.5149993896484 1.0
19.5887066378693 17.5757522583008 2.0
-9.95882058391343 -7.76387214660645 3.0
9.62988605424657 10.168966293335 4.0
30.233664897769 28.1714630126953 5.0
39.863550951 36.8533096313477 6.0
39.863550951 37.2463760375977 7.0
38.158424696 38.6414070129395 0.0
21.093616444 20.2235279083252 1.0
17.064808252 17.5017166137695 2.0
-6.6865614213 -6.44503831863403 3.0
10.378246821 10.9876689910889 4.0
27.780177875 27.435697555542 5.0
38.158424706 39.252197265625 6.0
38.158424706 37.8200798034668 7.0
37.671552635 37.1185111999512 0.0
18.060105754 17.9331340789795 1.0
19.611446882 20.2718257904053 2.0
-6.6868904195 -6.38768577575684 3.0
12.924556453 12.9261560440063 4.0
24.746996183 25.5070629119873 5.0
37.671552644 37.890926361084 6.0
37.671552644 38.2298774719238 7.0
38.9900068701748 39.0634574890137 0.0
19.4317892501748 16.8800964355469 1.0
19.5582176301748 16.8932361602783 2.0
-6.7795985865748 -6.22614860534668 3.0
12.7786190431748 11.6259899139404 4.0
26.2113877989032 25.7219562530518 5.0
38.99000687 38.1114654541016 6.0
38.99000687 37.5810470581055 7.0
32.824279187 39.5273895263672 0.0
17.524252685 18.3989601135254 1.0
15.300026502 17.4014720916748 2.0
-5.6991008156 -6.1023006439209 3.0
9.6009256767 8.90410709381104 4.0
23.223353511 22.243070602417 5.0
32.824279197 38.6805038452148 6.0
32.824279197 37.6441268920898 7.0
39.752392388 38.1880378723145 0.0
19.337645943 19.8233051300049 1.0
20.414746445 20.7847709655762 2.0
-4.1151033929 -5.92124080657959 3.0
16.299643042 16.8373718261719 4.0
23.452749346 23.7517681121826 5.0
39.752392398 39.6352386474609 6.0
39.752392398 38.4018936157227 7.0
36.0149979881443 39.4967460632324 0.0
19.5282530962186 17.6583042144775 1.0
16.4867448931817 15.5640344619751 2.0
-5.97886461861353 -5.71810102462769 3.0
10.5078802751384 10.2762327194214 4.0
25.5071177152005 25.2489490509033 5.0
36.014997988 38.4412040710449 6.0
36.014997988 37.2755584716797 7.0
42.307205596 38.9080390930176 0.0
21.688253633 17.5894966125488 1.0
20.618951963 16.7494773864746 2.0
-4.4760898368 -4.95459270477295 3.0
16.142862116 16.0776596069336 4.0
26.164343479 26.5251483917236 5.0
42.307205606 38.8598213195801 6.0
42.307205606 37.4142379760742 7.0
41.624369925 40.0605888366699 0.0
21.269456313 18.062801361084 1.0
20.354913614 20.23903465271 2.0
-9.726585677 -7.0881462097168 3.0
10.628327928 10.953893661499 4.0
30.996041999 29.2324390411377 5.0
41.624369934 39.7845115661621 6.0
41.624369934 38.0927429199219 7.0
40.927860916 34.6504211425781 0.0
20.558611306 17.3395099639893 1.0
20.36924961 17.5542907714844 2.0
-7.6689328495 -7.17262411117554 3.0
12.700316751 12.6255922317505 4.0
28.227544165 26.0011386871338 5.0
40.927860926 36.7514572143555 6.0
40.927860926 37.3148918151855 7.0
47.441321803 38.6355781555176 0.0
22.43048805 17.814115524292 1.0
25.010833753 20.6058692932129 2.0
-8.9760978683 -6.90214490890503 3.0
16.034735875 16.3400611877441 4.0
31.406585928 30.1895618438721 5.0
47.441321812 38.5971832275391 6.0
47.441321812 37.953685760498 7.0
39.37950561 35.2110252380371 0.0
19.873167771 17.7960777282715 1.0
19.506337839 21.2623195648193 2.0
-9.8415530061 -8.30772590637207 3.0
9.6647848227 11.0898971557617 4.0
29.714720787 29.6827964782715 5.0
39.379505619 36.649829864502 6.0
39.379505619 38.371997833252 7.0
42.095557289 37.4206771850586 0.0
22.277305426 18.7350254058838 1.0
19.818251863 17.645751953125 2.0
-7.6874055589 -6.41407489776611 3.0
12.130846294 11.9272747039795 4.0
29.964710995 26.8799571990967 5.0
42.095557299 37.7611045837402 6.0
42.095557299 38.075927734375 7.0
35.151188218 40.11279296875 0.0
17.118344744 17.3597660064697 1.0
18.032843477 17.9787769317627 2.0
-2.0060677241 -4.80558300018311 3.0
16.026775747 15.3886995315552 4.0
19.124412475 20.3243427276611 5.0
35.151188225 39.4486274719238 6.0
35.151188225 37.9462242126465 7.0
37.713935362 36.5203628540039 0.0
20.380527895 20.3142642974854 1.0
17.333407469 18.1303882598877 2.0
-5.6423114193 -6.40474033355713 3.0
11.691096041 11.5552997589111 4.0
26.022839323 25.4703731536865 5.0
37.713935371 37.9535827636719 6.0
37.713935371 38.3385772705078 7.0
39.667743251 37.0361175537109 0.0
20.027544136 16.3190212249756 1.0
19.640199117 18.528938293457 2.0
-10.376032212 -7.25973796844482 3.0
9.2641668955 10.2391595840454 4.0
30.403576357 27.7896709442139 5.0
39.66774326 37.6248359680176 6.0
39.66774326 37.7986602783203 7.0
41.240556184 36.5898094177246 0.0
23.347458936 19.9264812469482 1.0
17.893097248 17.6132526397705 2.0
-7.252142064 -7.1680326461792 3.0
10.640955174 10.3806371688843 4.0
30.59960101 27.701416015625 5.0
41.240556194 37.1225357055664 6.0
41.240556194 38.1176872253418 7.0
40.32019459 38.9908714294434 0.0
19.177093862 17.2122211456299 1.0
21.143100728 17.6689205169678 2.0
-7.2609061044 -7.03615093231201 3.0
13.882194613 12.344443321228 4.0
26.437999977 27.017614364624 5.0
40.3201946 39.8475685119629 6.0
40.3201946 37.5685768127441 7.0
39.5611510585585 36.6382904052734 0.0
19.6094718095582 16.4475021362305 1.0
19.9516792585585 17.3122997283936 2.0
-8.44726993515885 -6.67057037353516 3.0
11.5044093235587 11.7370634078979 4.0
28.0567416170478 28.3895740509033 5.0
39.561151059 38.2710990905762 6.0
39.561151059 37.6170806884766 7.0
43.5664319470049 37.5773658752441 0.0
22.3670487340048 18.9176502227783 1.0
21.1993832230048 21.056676864624 2.0
-8.61702915780503 -7.84403991699219 3.0
12.582354065005 13.1009206771851 4.0
30.9840778920045 29.1656360626221 5.0
43.566431947 37.3903312683105 6.0
43.566431947 38.6609382629395 7.0
37.929366556 37.5912818908691 0.0
20.105495873 18.5613899230957 1.0
17.823870687 17.207799911499 2.0
-5.9768423324 -6.44722366333008 3.0
11.847028349 11.3310651779175 4.0
26.082338211 25.4126319885254 5.0
37.929366562 38.3185806274414 6.0
37.929366562 37.6977958679199 7.0
39.7456686413092 36.2982177734375 0.0
20.6016083083092 20.8071670532227 1.0
19.1440603423092 19.3741130828857 2.0
-3.0966467648096 -5.68187999725342 3.0
16.0474135773093 14.9708690643311 4.0
23.6982550733092 23.0108776092529 5.0
39.745668641 38.0655708312988 6.0
39.745668641 38.638614654541 7.0
37.1737946256367 38.4812965393066 0.0
18.9810950589955 21.3365249633789 1.0
18.1926995622865 21.7580661773682 2.0
-4.83110587561615 -5.65295124053955 3.0
13.3615936860386 14.1435194015503 4.0
23.8122009348139 23.9389019012451 5.0
37.173794625 38.6397743225098 6.0
37.173794625 38.8791961669922 7.0
27.124866007 38.4895133972168 0.0
13.623671484 18.7385673522949 1.0
13.501194525 20.3670616149902 2.0
-4.5998142153 -5.94470167160034 3.0
8.9013803006 10.6044492721558 4.0
18.223485707 20.3034267425537 5.0
27.124866015 38.6823997497559 6.0
27.124866015 38.3752555847168 7.0
27.411746895 37.6403579711914 0.0
13.763683287 18.3662776947021 1.0
13.648063611 19.373628616333 2.0
-4.2527245718 -4.94068241119385 3.0
9.3953390304 10.5482664108276 4.0
18.016407867 19.9672756195068 5.0
27.411746904 37.8583374023438 6.0
27.411746904 38.0805130004883 7.0
38.866617309 38.3166122436523 0.0
22.319478804 20.3048706054688 1.0
16.547138508 18.0170288085938 2.0
-7.9092899105 -7.22729778289795 3.0
8.6378485894 9.339524269104 4.0
30.228768722 29.3728523254395 5.0
38.866617317 38.1810111999512 6.0
38.866617317 38.091438293457 7.0
42.3830573535328 40.1779708862305 0.0
20.5951530026372 16.8205680847168 1.0
21.7879043479541 19.1644229888916 2.0
-9.3946766066605 -7.40978336334229 3.0
12.3932277407315 11.9388694763184 4.0
29.98982960914 27.5325908660889 5.0
42.383057354 39.2903289794922 6.0
42.383057354 37.7934913635254 7.0
47.6437510358734 35.6916046142578 0.0
23.1576165238734 16.6958179473877 1.0
24.4861345218735 18.0414352416992 2.0
-8.74385685767346 -6.27904272079468 3.0
15.7422776638735 16.5013828277588 4.0
31.9014733818732 29.0067462921143 5.0
47.643751036 37.0794296264648 6.0
47.643751036 37.7900238037109 7.0
38.439399947 39.0081939697266 0.0
19.975660163 17.5922451019287 1.0
18.463739784 17.0425319671631 2.0
-6.4266906816 -6.89442825317383 3.0
12.037049093 11.4043302536011 4.0
26.402350854 25.461145401001 5.0
38.439399957 39.26220703125 6.0
38.439399957 37.7406959533691 7.0
40.263562361 39.0080032348633 0.0
20.42306719 16.3928813934326 1.0
19.840495171 19.4860496520996 2.0
-11.677045299 -8.3868989944458 3.0
8.1634498621 9.66355228424072 4.0
32.100112499 29.9776630401611 5.0
40.263562371 38.9021797180176 6.0
40.263562371 37.7431983947754 7.0
41.519528387 38.0884056091309 0.0
22.877320367 19.9312705993652 1.0
18.642208021 17.150390625 2.0
-5.2280057864 -6.20232105255127 3.0
13.414202225 13.0618019104004 4.0
28.105326163 27.330041885376 5.0
41.519528397 38.9987487792969 6.0
41.519528397 37.8157119750977 7.0
40.414570464 34.7163963317871 0.0
17.791289494 16.6664752960205 1.0
22.623280971 20.157190322876 2.0
-6.6442159501 -5.33582353591919 3.0
15.979065012 16.4150505065918 4.0
24.435505453 23.9982471466064 5.0
40.414570474 37.0233993530273 6.0
40.414570474 37.9090805053711 7.0
37.834181215 37.173397064209 0.0
18.770560656 18.2496604919434 1.0
19.063620563 20.0592346191406 2.0
-7.0281211763 -6.68292713165283 3.0
12.035499382 12.3962545394897 4.0
25.798681838 26.1502437591553 5.0
37.834181221 37.8128433227539 6.0
37.834181221 38.3767204284668 7.0
37.201121546 38.6326789855957 0.0
17.811506649 17.8080368041992 1.0
19.389614897 20.1660423278809 2.0
-6.0990519868 -4.96530151367188 3.0
13.290562901 13.9513702392578 4.0
23.910558645 24.1085834503174 5.0
37.201121556 38.2230033874512 6.0
37.201121556 37.9799842834473 7.0
42.944445132 37.9054527282715 0.0
21.59677927 19.4499835968018 1.0
21.347665865 20.6282920837402 2.0
-6.9735110037 -6.47272300720215 3.0
14.374154853 14.6083469390869 4.0
28.570290281 26.2699642181396 5.0
42.94444514 38.1703605651855 6.0
42.94444514 38.5209693908691 7.0
47.3096000339538 38.7105979919434 0.0
24.7818404359387 20.8261642456055 1.0
22.5277596000527 19.639440536499 2.0
-6.75460229154402 -6.49394607543945 3.0
15.7731573079881 15.6479959487915 4.0
31.5364427279756 26.4816188812256 5.0
47.309600034 38.9719467163086 6.0
47.309600034 38.4407081604004 7.0
34.703660739 40.2283401489258 0.0
16.95243502 20.6027145385742 1.0
17.75122572 22.3910427093506 2.0
-4.712400117 -5.93302631378174 3.0
13.038825595 14.1036748886108 4.0
21.664835145 22.7707901000977 5.0
34.703660747 39.417236328125 6.0
34.703660747 38.9034118652344 7.0
39.355679823 35.0680656433105 0.0
18.719972615 16.4936618804932 1.0
20.635707207 19.5595722198486 2.0
-8.5340541425 -7.81641578674316 3.0
12.101653055 12.192193031311 4.0
27.254026768 27.49267578125 5.0
39.355679833 36.8188819885254 6.0
39.355679833 38.0126914978027 7.0
36.760838125 38.2377128601074 0.0
18.098408908 19.1853103637695 1.0
18.662429217 20.3234329223633 2.0
-4.9002916105 -5.89747619628906 3.0
13.762137597 13.9016780853271 4.0
22.998700528 23.5197811126709 5.0
36.760838135 38.4216613769531 6.0
36.760838135 38.447582244873 7.0
45.817767369632 37.6665916442871 0.0
23.4066861136316 20.5924530029297 1.0
22.4110812656318 20.9806900024414 2.0
-7.99255969573223 -7.38699531555176 3.0
14.4185215696319 14.8418226242065 4.0
31.3992456710555 29.7923221588135 5.0
45.817767369 37.2447662353516 6.0
45.817767369 38.2023506164551 7.0
37.456387934 39.2336158752441 0.0
18.714257418 17.690559387207 1.0
18.742130516 19.2366886138916 2.0
-7.2626807055 -6.48669672012329 3.0
11.4794498 11.5591011047363 4.0
25.976938133 24.6620082855225 5.0
37.456387944 38.3965873718262 6.0
37.456387944 38.259765625 7.0
36.979299104 38.5870246887207 0.0
17.001867984 18.2880764007568 1.0
19.977431121 21.2388553619385 2.0
-4.4589079828 -5.41479015350342 3.0
15.518523129 15.3102693557739 4.0
21.460775976 21.6501235961914 5.0
36.979299113 38.6614570617676 6.0
36.979299113 38.4019432067871 7.0
44.357392148 40.0601272583008 0.0
23.57148554 19.0946254730225 1.0
20.785906608 18.1794185638428 2.0
-6.1950554121 -6.80176782608032 3.0
14.590851187 14.3151111602783 4.0
29.766540962 28.9647846221924 5.0
44.357392157 39.7223777770996 6.0
44.357392157 38.1694412231445 7.0
36.8120792993302 39.7566604614258 0.0
19.8640110823302 19.234655380249 1.0
16.9480682263302 20.6766796112061 2.0
-8.51793333553018 -8.00967693328857 3.0
8.43013489113018 9.71043109893799 4.0
28.381944393825 28.6906681060791 5.0
36.812079299 39.1985778808594 6.0
36.812079299 38.5785751342773 7.0
43.172254734 37.4452056884766 0.0
21.994700089 17.6620769500732 1.0
21.177554646 17.9069385528564 2.0
-4.8220170211 -4.91870641708374 3.0
16.355537616 16.8805446624756 4.0
26.816717119 26.7317600250244 5.0
43.172254743 38.8187828063965 6.0
43.172254743 38.0640411376953 7.0
47.109370558 38.814868927002 0.0
25.034746172 19.9527626037598 1.0
22.074624387 20.0149459838867 2.0
-8.7769581235 -6.85673666000366 3.0
13.297666255 12.5867052078247 4.0
33.811704304 29.7457790374756 5.0
47.109370567 38.6435852050781 6.0
47.109370567 38.3257484436035 7.0
35.871355653 37.975715637207 0.0
18.076771786 16.8946857452393 1.0
17.794583868 18.2764835357666 2.0
-9.6137759762 -6.53797626495361 3.0
8.1808078825 9.58578491210938 4.0
27.690547772 26.4637470245361 5.0
35.871355662 37.9599914550781 6.0
35.871355662 37.319019317627 7.0
42.1694339111434 34.9261436462402 0.0
22.6030080131434 19.987964630127 1.0
19.5664259071434 19.242639541626 2.0
-7.61628576964338 -6.56450510025024 3.0
11.9501401371434 11.8839435577393 4.0
30.2192937831435 27.4134559631348 5.0
42.169433911 37.1177787780762 6.0
42.169433911 38.2599983215332 7.0
38.001249391 34.3312606811523 0.0
19.442391644 17.9249420166016 1.0
18.558857748 19.0504283905029 2.0
-7.4577746389 -6.47963428497314 3.0
11.101083099 10.639575958252 4.0
26.900166292 25.4920845031738 5.0
38.001249401 36.6131324768066 6.0
38.001249401 38.1094779968262 7.0
36.026296144 34.2754669189453 0.0
18.544058535 20.0352020263672 1.0
17.482237609 21.2008266448975 2.0
-7.0335899837 -7.13041257858276 3.0
10.448647616 11.5355739593506 4.0
25.577648528 24.3865280151367 5.0
36.026296153 36.7303695678711 6.0
36.026296153 38.3877487182617 7.0
33.200213149 35.4951210021973 0.0
14.957525483 16.2560729980469 1.0
18.242687666 20.9775104522705 2.0
-8.613844647 -7.26988506317139 3.0
9.6288430093 11.0639972686768 4.0
23.57137014 23.4200344085693 5.0
33.200213159 37.4874496459961 6.0
33.200213159 38.2910385131836 7.0
36.881947947 37.465705871582 0.0
19.760267083 20.5645656585693 1.0
17.12168087 20.083963394165 2.0
-6.5510496361 -6.32881307601929 3.0
10.570631229 11.6332015991211 4.0
26.311316723 25.1574764251709 5.0
36.881947952 38.5717811584473 6.0
36.881947952 38.6490745544434 7.0
28.363012003 39.3616371154785 0.0
14.617095463 18.5794944763184 1.0
13.745916541 19.3535118103027 2.0
-5.3741086073 -5.35430812835693 3.0
8.3718079243 9.50516605377197 4.0
19.991204079 20.801586151123 5.0
28.363012012 39.1959762573242 6.0
28.363012012 37.9662437438965 7.0
39.574467416 39.6277046203613 0.0
20.649388383 20.3861446380615 1.0
18.925079033 21.4070167541504 2.0
-8.4547238017 -7.67993927001953 3.0
10.470355222 11.36217212677 4.0
29.104112194 29.5989208221436 5.0
39.574467426 39.0266876220703 6.0
39.574467426 38.7337036132812 7.0
36.220401471 37.8718681335449 0.0
17.995668566 17.9903526306152 1.0
18.224732906 19.4818363189697 2.0
-6.9105415737 -6.72850561141968 3.0
11.314191323 11.7548751831055 4.0
24.906210149 25.6317291259766 5.0
36.22040148 38.6176986694336 6.0
36.22040148 38.0011329650879 7.0
31.086432133 34.5395927429199 0.0
16.0910287 18.2713642120361 1.0
14.995403435 18.2115154266357 2.0
-4.3439743018 -6.06010627746582 3.0
10.651429126 9.93178558349609 4.0
20.43500301 21.702917098999 5.0
31.086432141 37.5585861206055 6.0
31.086432141 38.2967491149902 7.0
36.274353053 35.5193634033203 0.0
18.420887917 18.7950248718262 1.0
17.853465137 18.6873111724854 2.0
-3.6202808126 -4.77694416046143 3.0
14.233184315 13.1600112915039 4.0
22.041168739 22.2741985321045 5.0
36.274353063 36.9859390258789 6.0
36.274353063 38.3580703735352 7.0
31.65795008 39.1103515625 0.0
17.489254099 19.1011295318604 1.0
14.168695981 18.1463451385498 2.0
-6.0427493715 -5.76578617095947 3.0
8.1259465996 9.35105609893799 4.0
23.53200348 22.4135150909424 5.0
31.65795009 38.3070411682129 6.0
31.65795009 38.2208633422852 7.0
37.059171469 38.0806655883789 0.0
19.339994356 17.360631942749 1.0
17.719177113 20.2771015167236 2.0
-9.4011683011 -6.69037199020386 3.0
8.318008802 10.2146921157837 4.0
28.741162667 26.2939968109131 5.0
37.059171479 37.6685371398926 6.0
37.059171479 38.3063812255859 7.0
45.308862348 39.4984741210938 0.0
24.95721568 21.0566253662109 1.0
20.351646669 19.7054500579834 2.0
-7.323343681 -7.38778591156006 3.0
13.028302978 12.2876348495483 4.0
32.28055937 30.0469646453857 5.0
45.308862357 39.0293121337891 6.0
45.308862357 38.6610450744629 7.0
32.988279899 39.2722434997559 0.0
17.965429309 19.0594024658203 1.0
15.02285059 16.1304512023926 2.0
-2.6394240327 -4.62847852706909 3.0
12.383426547 12.306004524231 4.0
20.604853352 21.5992603302002 5.0
32.988279909 38.4034690856934 6.0
32.988279909 37.3758430480957 7.0
41.812200584 40.7469940185547 0.0
21.127130287 17.0935840606689 1.0
20.685070296 17.030725479126 2.0
-5.5457377677 -5.02018690109253 3.0
15.139332519 16.496021270752 4.0
26.672868065 24.9826946258545 5.0
41.812200593 39.2973442077637 6.0
41.812200593 37.6023178100586 7.0
34.159577997 36.5237197875977 0.0
15.979883142 17.1097774505615 1.0
18.179694855 20.0799789428711 2.0
-6.0058060763 -6.84571695327759 3.0
12.173888768 12.1253242492676 4.0
21.985689228 22.745059967041 5.0
34.159578007 37.6194610595703 6.0
34.159578007 37.6643333435059 7.0
41.353058194 37.9218711853027 0.0
20.665722502 19.657527923584 1.0
20.687335692 21.8933563232422 2.0
-9.525258586 -7.57327747344971 3.0
11.162077096 12.188159942627 4.0
30.190981098 27.4297847747803 5.0
41.353058204 38.3164482116699 6.0
41.353058204 38.7013931274414 7.0
35.663236634 37.0223770141602 0.0
18.766917562 19.2966918945312 1.0
16.896319073 20.7640075683594 2.0
-7.670552873 -7.28912448883057 3.0
9.2257661902 10.2489175796509 4.0
26.437470445 26.7452220916748 5.0
35.663236644 37.2712364196777 6.0
35.663236644 38.4917945861816 7.0
40.444916113 36.7806930541992 0.0
18.421748417 17.6094818115234 1.0
22.023167696 21.7339935302734 2.0
-9.937900394 -6.94379043579102 3.0
12.085267292 13.0559673309326 4.0
28.359648821 25.8374824523926 5.0
40.444916123 37.510498046875 6.0
40.444916123 38.4287109375 7.0
39.900416904 39.561824798584 0.0
21.000367331 18.6583671569824 1.0
18.900049573 18.6498279571533 2.0
-6.1981317567 -7.3685474395752 3.0
12.701917806 12.1160717010498 4.0
27.198499098 27.9874000549316 5.0
39.900416914 39.4692611694336 6.0
39.900416914 38.3411903381348 7.0
34.40532484 40.3465423583984 0.0
18.680133752 20.5751361846924 1.0
15.725191089 19.8472232818604 2.0
-6.5675447079 -6.24158763885498 3.0
9.157646372 10.5586423873901 4.0
25.247678469 25.0553722381592 5.0
34.405324849 39.6702690124512 6.0
34.405324849 38.118293762207 7.0
33.074255995 38.9682807922363 0.0
16.81819884 16.5250034332275 1.0
16.256057155 16.9557018280029 2.0
-5.8048358004 -6.15396642684937 3.0
10.451221345 10.6875638961792 4.0
22.62303465 23.3383483886719 5.0
33.074256004 39.1054153442383 6.0
33.074256004 37.6281509399414 7.0
40.036170298 35.6023597717285 0.0
18.427862329 17.2975292205811 1.0
21.60830797 20.2734298706055 2.0
-6.8170168406 -6.75686025619507 3.0
14.791291119 14.3797836303711 4.0
25.244879179 25.2866992950439 5.0
40.036170308 36.8496704101562 6.0
40.036170308 38.207691192627 7.0
44.453206231 38.2669944763184 0.0
21.51525667 16.4036426544189 1.0
22.937949561 16.2548389434814 2.0
-6.654859187 -6.30569934844971 3.0
16.283090365 17.3473625183105 4.0
28.170115867 25.6836376190186 5.0
44.453206241 38.1628379821777 6.0
44.453206241 37.2434043884277 7.0
33.8500454 39.7613563537598 0.0
15.317438725 18.3624534606934 1.0
18.532606675 21.6233692169189 2.0
-3.771729854 -5.61658668518066 3.0
14.760876811 16.5066318511963 4.0
19.089168589 19.8828372955322 5.0
33.85004541 38.8879776000977 6.0
33.85004541 38.3865776062012 7.0
30.537877299 36.7850570678711 0.0
15.743236653 20.2103805541992 1.0
14.794640646 21.7704200744629 2.0
-6.2614673178 -6.10544776916504 3.0
8.5331733191 10.502703666687 4.0
22.00470398 23.8876972198486 5.0
30.537877308 37.3802719116211 6.0
30.537877308 38.7151412963867 7.0
27.727440934 38.0936508178711 0.0
13.298433958 18.0505313873291 1.0
14.429006978 20.9084587097168 2.0
-4.8732361991 -6.11902189254761 3.0
9.5557707702 10.1523923873901 4.0
18.171670165 19.3737983703613 5.0
27.727440942 38.0871047973633 6.0
27.727440942 38.331485748291 7.0
35.630182891 35.1338539123535 0.0
17.416160369 18.9028091430664 1.0
18.214022525 20.4124279022217 2.0
-5.5362398817 -5.61684417724609 3.0
12.677782636 13.7851343154907 4.0
22.952400258 22.8744201660156 5.0
35.630182897 37.1462669372559 6.0
35.630182897 38.5279006958008 7.0
31.038251509 33.3807983398438 0.0
16.216885208 16.6455249786377 1.0
14.821366301 16.0486469268799 2.0
-5.1281710605 -5.42746448516846 3.0
9.6931952301 9.41812515258789 4.0
21.345056279 22.1098098754883 5.0
31.038251519 36.6000366210938 6.0
31.038251519 37.2300720214844 7.0
31.867011958 33.202579498291 0.0
14.888086344 17.621057510376 1.0
16.978925614 20.5448951721191 2.0
-5.7710515718 -7.19498538970947 3.0
11.207874033 11.9303922653198 4.0
20.659137926 21.9857730865479 5.0
31.867011967 36.4212303161621 6.0
31.867011967 38.2883949279785 7.0
30.754831636 38.4820671081543 0.0
14.067592922 19.0547122955322 1.0
16.687238713 22.8461303710938 2.0
-4.2836008115 -6.31049394607544 3.0
12.403637892 13.1376867294312 4.0
18.351193744 20.5959892272949 5.0
30.754831646 39.4844512939453 6.0
30.754831646 38.871654510498 7.0
38.951564273 38.8101234436035 0.0
20.789904648 16.9238395690918 1.0
18.161659627 16.4314575195312 2.0
-9.14497609 -6.46928310394287 3.0
9.0166835299 9.47122001647949 4.0
29.934880745 27.5617809295654 5.0
38.95156428 40.0618705749512 6.0
38.95156428 37.3818550109863 7.0
33.384480929 38.2642669677734 0.0
14.280681926 16.9685535430908 1.0
19.103799006 22.3477458953857 2.0
-6.8894608583 -6.19771575927734 3.0
12.214338142 13.3826465606689 4.0
21.170142791 21.340461730957 5.0
33.384480936 38.0517997741699 6.0
33.384480936 38.3708686828613 7.0
37.291199278 38.7678489685059 0.0
18.976122996 19.6549739837646 1.0
18.315076287 19.2251129150391 2.0
-3.6314782733 -5.769202709198 3.0
14.68359801 13.7143478393555 4.0
22.607601274 22.8545246124268 5.0
37.291199282 38.8392524719238 6.0
37.291199282 38.459602355957 7.0
33.5718110164568 37.8135719299316 0.0
16.2455196066318 17.0999641418457 1.0
17.3262914102844 19.1486968994141 2.0
-6.34124959778268 -6.17118310928345 3.0
10.9850418123699 11.164589881897 4.0
22.5867692042782 22.981409072876 5.0
33.571811016 38.7851409912109 6.0
33.571811016 38.1783447265625 7.0
44.048761737 38.7022171020508 0.0
22.194046458 21.1126499176025 1.0
21.85471528 22.6992053985596 2.0
-6.0219898767 -6.91707420349121 3.0
15.832725394 16.0890579223633 4.0
28.216036344 26.933723449707 5.0
44.048761746 39.0128364562988 6.0
44.048761746 38.9169616699219 7.0
38.092815451 38.6273002624512 0.0
20.42584869 19.5031509399414 1.0
17.666966762 17.6601657867432 2.0
-5.2714738638 -6.12135696411133 3.0
12.395492889 12.151912689209 4.0
25.697322563 25.8764457702637 5.0
38.09281546 38.5205574035645 6.0
38.09281546 38.0732727050781 7.0
42.616606914 38.4376983642578 0.0
22.043217204 20.2253856658936 1.0
20.57338971 21.3311023712158 2.0
-9.1142862064 -7.6845064163208 3.0
11.459103494 12.3795433044434 4.0
31.157503421 27.7393169403076 5.0
42.616606924 39.1956596374512 6.0
42.616606924 38.6021957397461 7.0
41.080227066 37.0157890319824 0.0
21.412108977 18.4684085845947 1.0
19.66811809 19.1322727203369 2.0
-8.9308117628 -7.57399082183838 3.0
10.737306318 10.990288734436 4.0
30.342920748 28.2037944793701 5.0
41.080227074 38.0487861633301 6.0
41.080227074 38.2917442321777 7.0
36.859087804 37.1518630981445 0.0
16.308519924 16.9446182250977 1.0
20.550567882 21.6307849884033 2.0
-8.0027710471 -6.72001934051514 3.0
12.547796826 13.332706451416 4.0
24.31129098 23.4197864532471 5.0
36.859087813 37.2813301086426 6.0
36.859087813 38.1286964416504 7.0
39.316268866 38.2126045227051 0.0
19.479900065 15.9603338241577 1.0
19.836368801 16.2911548614502 2.0
-7.370707565 -6.07263088226318 3.0
12.465661226 13.1816377639771 4.0
26.85060764 25.2210559844971 5.0
39.316268876 38.3132858276367 6.0
39.316268876 37.3171691894531 7.0
40.023071957 38.8573913574219 0.0
21.374119573 18.377046585083 1.0
18.648952384 19.1282634735107 2.0
-8.5239396023 -7.83970546722412 3.0
10.125012772 10.4218111038208 4.0
29.898059185 29.0349903106689 5.0
40.023071967 38.986644744873 6.0
40.023071967 38.0671806335449 7.0
32.691039556 36.2697486877441 0.0
16.99402987 19.1154479980469 1.0
15.697009686 20.327672958374 2.0
-6.1678662484 -6.42944049835205 3.0
9.5291434282 10.3375549316406 4.0
23.161896128 23.4445819854736 5.0
32.691039566 37.2629356384277 6.0
32.691039566 38.4503898620605 7.0
39.350172222 37.387622833252 0.0
19.79571531 20.2490291595459 1.0
19.554456913 21.3270282745361 2.0
-6.9332448751 -6.5169825553894 3.0
12.62121203 13.0898675918579 4.0
26.728960194 26.1385059356689 5.0
39.35017223 37.6635932922363 6.0
39.35017223 38.6725959777832 7.0
38.6257792205169 35.910888671875 0.0
20.6764320695262 18.5485877990723 1.0
17.9493471475253 19.9325942993164 2.0
-8.74948254057215 -7.7156229019165 3.0
9.19986460693105 10.4040594100952 4.0
29.4259146106884 28.4760036468506 5.0
38.62577922 37.1810493469238 6.0
38.62577922 38.5134086608887 7.0
43.363210173 34.9958305358887 0.0
22.264479357 21.5803890228271 1.0
21.09873082 22.7854270935059 2.0
-7.8566163895 -7.66118907928467 3.0
13.242114424 13.809365272522 4.0
30.121095753 27.8011150360107 5.0
43.36321018 37.0335807800293 6.0
43.36321018 39.2744178771973 7.0
41.7113705187954 34.5380516052246 0.0
19.9560794530664 18.6088314056396 1.0
21.7552910618007 21.1725254058838 2.0
-6.27295647675057 -6.8930835723877 3.0
15.4823345851289 16.8441963195801 4.0
26.2290359296866 25.2371444702148 5.0
41.711370519 36.5812339782715 6.0
41.711370519 38.2609329223633 7.0
41.4017485851654 38.726692199707 0.0
21.9191533447582 18.3549346923828 1.0
19.4825952431145 17.6266326904297 2.0
-7.53340700226978 -5.80062866210938 3.0
11.9491882404739 12.1092176437378 4.0
29.4525603463383 27.7694702148438 5.0
41.401748585 39.0000228881836 6.0
41.401748585 37.9891586303711 7.0
45.411303660968 34.8692779541016 0.0
22.8793399812672 19.2991485595703 1.0
22.5319636798771 20.823148727417 2.0
-8.35657341793646 -6.4659366607666 3.0
14.1753902620561 14.11097240448 4.0
31.2359133986188 29.139835357666 5.0
45.411303661 36.9041976928711 6.0
45.411303661 38.1950416564941 7.0
46.173622728 37.9404335021973 0.0
22.478207281 16.4069652557373 1.0
23.695415447 16.9864826202393 2.0
-7.2335265538 -6.53816890716553 3.0
16.461888883 16.3021755218506 4.0
29.711733845 27.6196422576904 5.0
46.173622738 37.6282730102539 6.0
46.173622738 37.5303649902344 7.0
43.970500883 38.6664772033691 0.0
23.319439955 17.0189323425293 1.0
20.651060929 16.4767665863037 2.0
-9.6037663307 -7.15401172637939 3.0
11.047294589 10.7901926040649 4.0
32.923206295 28.2128067016602 5.0
43.970500892 38.6267585754395 6.0
43.970500892 37.4834747314453 7.0
41.561964514 38.938362121582 0.0
22.423587267 20.8568439483643 1.0
19.138377247 18.6268405914307 2.0
-2.7143772539 -5.22896766662598 3.0
16.423999984 15.6767549514771 4.0
25.13796453 24.7959842681885 5.0
41.561964524 39.0437469482422 6.0
41.561964524 38.5228996276855 7.0
42.043570825 34.6056327819824 0.0
21.184612309 16.6557102203369 1.0
20.858958517 16.8770961761475 2.0
-7.1973175688 -6.75646781921387 3.0
13.66164094 13.0737333297729 4.0
28.381929887 26.9604167938232 5.0
42.043570834 36.6866226196289 6.0
42.043570834 37.5439300537109 7.0
42.728187899 37.9417037963867 0.0
22.937429443 18.1662998199463 1.0
19.790758457 16.926342010498 2.0
-7.3221221307 -7.4279670715332 3.0
12.468636317 12.2546701431274 4.0
30.259551583 28.8183994293213 5.0
42.728187908 38.9333877563477 6.0
42.728187908 37.7406578063965 7.0
46.647822935 39.0414543151855 0.0
24.419972425 20.5212535858154 1.0
22.22785051 20.6525497436523 2.0
-9.7005341836 -7.72737884521484 3.0
12.527316316 13.1360559463501 4.0
34.120506619 29.7437191009521 5.0
46.647822945 39.6391525268555 6.0
46.647822945 38.539665222168 7.0
34.157298452 38.5819664001465 0.0
16.904613725 17.7336578369141 1.0
17.252684728 19.7643737792969 2.0
-7.2828539724 -7.29813289642334 3.0
9.9698307451 10.6002712249756 4.0
24.187467707 23.1277008056641 5.0
34.157298462 38.9928588867188 6.0
34.157298462 38.1195526123047 7.0
38.08940785 34.4805297851562 0.0
20.151317931 20.2890071868896 1.0
17.93808992 18.9348888397217 2.0
-2.5516772312 -5.35033512115479 3.0
15.386412679 16.437816619873 4.0
22.702995171 24.5431327819824 5.0
38.08940786 36.7943687438965 6.0
38.08940786 38.571117401123 7.0
48.253456595 35.6440849304199 0.0
26.02387816 20.5944747924805 1.0
22.229578435 19.3955402374268 2.0
-8.9125381923 -7.61924934387207 3.0
13.317040233 13.3688135147095 4.0
34.936416362 30.0503540039062 5.0
48.253456605 37.2490653991699 6.0
48.253456605 38.6128845214844 7.0
41.303512945 38.997444152832 0.0
21.842523524 18.8424205780029 1.0
19.460989421 19.637321472168 2.0
-8.4785475097 -7.82302665710449 3.0
10.982441901 10.2565088272095 4.0
30.321071043 28.28835105896 5.0
41.303512955 39.471794128418 6.0
41.303512955 38.1234130859375 7.0
34.11247551 34.8043518066406 0.0
16.407323353 17.1238861083984 1.0
17.705152157 18.3692054748535 2.0
-3.4961419723 -4.27990436553955 3.0
14.209010175 14.5651512145996 4.0
19.903465335 21.2219104766846 5.0
34.11247552 36.6649284362793 6.0
34.11247552 37.6765174865723 7.0
39.346899311 36.9189567565918 0.0
20.280952021 17.7573394775391 1.0
19.06594729 19.357852935791 2.0
-10.212334112 -7.72798728942871 3.0
8.8536131687 9.71210765838623 4.0
30.493286142 26.6863384246826 5.0
39.346899321 37.0829734802246 6.0
39.346899321 38.0853118896484 7.0
42.355081312 40.3727226257324 0.0
22.237080125 20.6085567474365 1.0
20.118001187 18.9643726348877 2.0
-4.2250518369 -5.76421642303467 3.0
15.892949341 15.6362218856812 4.0
26.462131971 26.5652847290039 5.0
42.355081321 39.8341255187988 6.0
42.355081321 38.5112571716309 7.0
46.02528461 39.333553314209 0.0
20.767999035 16.5162086486816 1.0
25.257285575 19.9766368865967 2.0
-9.1629445671 -7.81440830230713 3.0
16.094340999 16.6618881225586 4.0
29.930943612 27.8539047241211 5.0
46.025284619 39.6643447875977 6.0
46.025284619 37.6344261169434 7.0
48.790909777 38.2397956848145 0.0
25.145056622 21.2874984741211 1.0
23.645853155 21.935022354126 2.0
-9.2179915631 -7.02708625793457 3.0
14.427861582 15.1086835861206 4.0
34.363048195 29.5774612426758 5.0
48.790909787 38.2678718566895 6.0
48.790909787 38.8505020141602 7.0
41.698940493 39.187141418457 0.0
20.507660886 17.1788635253906 1.0
21.191279609 19.8359222412109 2.0
-9.1543537316 -6.76197481155396 3.0
12.036925868 12.6234407424927 4.0
29.662014626 26.6526031494141 5.0
41.698940502 39.7519950866699 6.0
41.698940502 37.9041481018066 7.0
47.902972774 39.0749473571777 0.0
23.393159022 19.0067977905273 1.0
24.509813754 20.7153282165527 2.0
-8.5476809737 -7.40678310394287 3.0
15.962132774 15.5545797348022 4.0
31.940840003 28.6378288269043 5.0
47.902972781 38.7142448425293 6.0
47.902972781 38.1966438293457 7.0
41.377170862 38.7197875976562 0.0
20.577519401 19.5597648620605 1.0
20.799651462 20.1840419769287 2.0
-6.2821130824 -6.78234958648682 3.0
14.517538372 15.7355527877808 4.0
26.859632492 26.6086616516113 5.0
41.37717087 39.0496406555176 6.0
41.37717087 38.3725395202637 7.0
32.285684892 34.8387641906738 0.0
14.316873201 16.8764095306396 1.0
17.968811691 20.3310852050781 2.0
-5.3227711546 -6.28601264953613 3.0
12.646040526 12.7847814559937 4.0
19.639644366 20.0413036346436 5.0
32.285684902 36.9405670166016 6.0
32.285684902 38.0453186035156 7.0
32.344206843 37.524097442627 0.0
16.160894185 16.8279609680176 1.0
16.183312659 19.3459796905518 2.0
-7.3310896054 -7.07427835464478 3.0
8.852223045 9.81633949279785 4.0
23.491983799 24.615535736084 5.0
32.344206851 38.1274299621582 6.0
32.344206851 37.9068145751953 7.0
34.184112321 33.9647979736328 0.0
18.400062549 18.154125213623 1.0
15.784049772 18.0120944976807 2.0
-7.6583985283 -6.57077312469482 3.0
8.125651234 8.92036151885986 4.0
26.058461087 26.0505714416504 5.0
34.184112331 36.6210975646973 6.0
34.184112331 37.9755592346191 7.0
34.663221798 38.3398475646973 0.0
18.206197344 19.6308727264404 1.0
16.457024454 19.6530456542969 2.0
-6.9582970888 -7.39903926849365 3.0
9.4987273548 10.1422443389893 4.0
25.164494443 25.2262382507324 5.0
34.663221808 39.7007331848145 6.0
34.663221808 38.5433044433594 7.0
36.873774533 38.9478340148926 0.0
18.579406772 19.0967044830322 1.0
18.294367761 20.2818202972412 2.0
-5.3733020341 -6.14883661270142 3.0
12.921065717 13.4256219863892 4.0
23.952708817 23.7879180908203 5.0
36.873774543 38.9723587036133 6.0
36.873774543 38.3319320678711 7.0
40.211352898 39.2230224609375 0.0
22.061687235 18.4162063598633 1.0
18.149665663 17.0960712432861 2.0
-7.4872031869 -6.53025531768799 3.0
10.662462467 9.88484382629395 4.0
29.548890431 28.937967300415 5.0
40.211352908 38.1462364196777 6.0
40.211352908 37.7331657409668 7.0
46.160689864 38.4753837585449 0.0
24.176307096 20.872989654541 1.0
21.984382774 20.1225337982178 2.0
-5.8472747438 -6.72783327102661 3.0
16.137108027 15.6272401809692 4.0
30.023581843 28.5064601898193 5.0
46.160689867 38.5231323242188 6.0
46.160689867 38.7388534545898 7.0
32.727702668 35.2151374816895 0.0
17.796794822 21.370044708252 1.0
14.930907847 20.2228393554688 2.0
-5.7416146893 -6.3030800819397 3.0
9.1892931492 10.2777500152588 4.0
23.53840952 23.4413299560547 5.0
32.727702676 37.2155494689941 6.0
32.727702676 38.9572448730469 7.0
40.640985417073 39.849910736084 0.0
21.2248663380953 21.3396530151367 1.0
19.4161190831271 20.794340133667 2.0
-6.12795167285201 -6.78942346572876 3.0
13.2881674100867 13.7047262191772 4.0
27.3528180102152 26.4716758728027 5.0
40.640985417 39.0562629699707 6.0
40.640985417 38.7816734313965 7.0
34.611451584 38.3984603881836 0.0
16.03184427 17.5221157073975 1.0
18.579607315 20.6536464691162 2.0
-5.6371786801 -6.10508060455322 3.0
12.942428625 12.9278697967529 4.0
21.66902296 21.356164932251 5.0
34.611451594 39.5638542175293 6.0
34.611451594 38.4655113220215 7.0
43.311280798 34.2838706970215 0.0
19.747206987 17.438850402832 1.0
23.564073813 21.5021286010742 2.0
-7.5093463576 -6.22211313247681 3.0
16.054727447 16.2760124206543 4.0
27.256553353 27.0686702728271 5.0
43.311280807 36.9938011169434 6.0
43.311280807 38.5341148376465 7.0
36.406364048 33.7295227050781 0.0
19.818637935 18.9442310333252 1.0
16.587726114 17.2003860473633 2.0
-3.8896932395 -5.77384185791016 3.0
12.698032865 12.4604072570801 4.0
23.708331184 23.5987510681152 5.0
36.406364057 36.5618133544922 6.0
36.406364057 37.962230682373 7.0
31.68563032 37.6358680725098 0.0
16.740289757 17.1892528533936 1.0
14.945340567 17.8820724487305 2.0
-6.0451472911 -5.75692653656006 3.0
8.9001932697 7.80824184417725 4.0
22.785437054 22.1846828460693 5.0
31.685630326 38.7874946594238 6.0
31.685630326 37.9096069335938 7.0
40.165364423 39.0904960632324 0.0
21.225216477 19.2164707183838 1.0
18.940147947 21.827766418457 2.0
-9.5161355118 -8.00538539886475 3.0
9.4240124269 10.1755542755127 4.0
30.741351997 26.6046562194824 5.0
40.165364432 39.7336044311523 6.0
40.165364432 38.7250556945801 7.0
42.886465637 39.4899482727051 0.0
31.633244398 17.4097709655762 1.0
11.253221239 21.3548221588135 2.0
0 -8.58141326904297 3.0
11.253221239 12.2181043624878 4.0
31.633244398 29.8330821990967 5.0
42.886465647 38.6242370605469 6.0
42.886465647 38.3472862243652 7.0
46.942592322 39.213249206543 0.0
23.694673267 17.5460948944092 1.0
23.247919055 18.4841766357422 2.0
-9.8881851606 -8.14958381652832 3.0
13.359733884 12.706485748291 4.0
33.582858438 29.6778583526611 5.0
46.942592332 38.2048301696777 6.0
46.942592332 37.9393730163574 7.0
38.417030558 39.695442199707 0.0
20.711241226 17.0683765411377 1.0
17.705789332 17.1958065032959 2.0
-7.7084466149 -6.59390497207642 3.0
9.9973427073 10.021556854248 4.0
28.419687851 27.8532428741455 5.0
38.417030568 38.9786949157715 6.0
38.417030568 37.7934150695801 7.0
39.82159542 36.3071708679199 0.0
19.816428679 19.6519145965576 1.0
20.005166741 21.1215915679932 2.0
-5.1739908699 -6.62862920761108 3.0
14.831175861 15.1213846206665 4.0
24.990419559 25.130973815918 5.0
39.82159543 37.7508163452148 6.0
39.82159543 38.7998161315918 7.0
42.375846264 39.8415489196777 0.0
20.538210367 20.3709735870361 1.0
21.837635902 21.7246589660645 2.0
-5.3545430676 -4.69664478302002 3.0
16.483092829 16.3289604187012 4.0
25.89275344 24.6845245361328 5.0
42.375846269 39.3446846008301 6.0
42.375846269 38.534236907959 7.0
41.909868914 38.3409690856934 0.0
21.249762031 18.9616279602051 1.0
20.660106883 21.9951686859131 2.0
-10.971309543 -7.54651260375977 3.0
9.6887973304 11.5055046081543 4.0
32.221071584 26.7947101593018 5.0
41.909868924 38.0737800598145 6.0
41.909868924 38.9064521789551 7.0
38.5154038135067 36.3999366760254 0.0
19.3171244035071 17.346170425415 1.0
19.1982794195071 19.3678302764893 2.0
-8.28760453680882 -7.64575290679932 3.0
10.9106748825063 11.0921430587769 4.0
27.6047289405071 27.5109024047852 5.0
38.515403814 37.2400932312012 6.0
38.515403814 38.4260368347168 7.0
32.187464764 40.2895965576172 0.0
15.688410189 17.2849273681641 1.0
16.499054576 18.6891613006592 2.0
-5.8055743047 -5.98850774765015 3.0
10.693480262 10.3741636276245 4.0
21.493984503 21.8292598724365 5.0
32.187464773 39.085132598877 6.0
32.187464773 37.9769668579102 7.0
43.5533404982843 38.7666854858398 0.0
25.257472916292 19.3649311065674 1.0
18.2958675832873 16.73366355896 2.0
-8.52565235008034 -6.60113000869751 3.0
9.77021523347734 9.0255823135376 4.0
33.7831252662979 29.8078918457031 5.0
43.553340498 39.6603584289551 6.0
43.553340498 37.9881782531738 7.0
38.661685108 34.8419151306152 0.0
21.023698145 19.7037200927734 1.0
17.637986964 17.4145698547363 2.0
-6.4647611773 -7.03553819656372 3.0
11.173225776 10.8373641967773 4.0
27.488459332 28.6406230926514 5.0
38.661685118 37.1198577880859 6.0
38.661685118 38.1369934082031 7.0
47.185976942 37.7625045776367 0.0
20.88210062 17.106071472168 1.0
26.303876326 21.5674457550049 2.0
-9.8059530494 -7.16569995880127 3.0
16.497923272 15.7508506774902 4.0
30.688053675 28.244743347168 5.0
47.185976948 38.1203880310059 6.0
47.185976948 38.2671699523926 7.0
41.0736498246764 38.2608222961426 0.0
17.9299575176762 16.2582855224609 1.0
23.1436923166761 19.4006061553955 2.0
-7.36188695227673 -6.06347846984863 3.0
15.7818053646765 15.6510610580444 4.0
25.2918444696756 25.2223072052002 5.0
41.073649825 37.4899520874023 6.0
41.073649825 37.9960746765137 7.0
39.086114836 34.7249450683594 0.0
22.054277289 20.8767223358154 1.0
17.031837547 21.1369781494141 2.0
-8.5092532605 -7.36842346191406 3.0
8.5225842768 10.0453176498413 4.0
30.563530559 26.458610534668 5.0
39.086114846 36.750545501709 6.0
39.086114846 38.6412467956543 7.0
38.072309142 38.6234550476074 0.0
19.059692214 19.3129920959473 1.0
19.012616928 19.3900184631348 2.0
-4.5092621515 -5.46946334838867 3.0
14.503354768 14.3380393981934 4.0
23.568954375 23.4154148101807 5.0
38.072309151 38.0421485900879 6.0
38.072309151 38.2661743164062 7.0
41.585402364 38.9092063903809 0.0
21.878346027 19.9680881500244 1.0
19.707056337 20.5680255889893 2.0
-7.856520423 -7.48246002197266 3.0
11.850535904 12.2141876220703 4.0
29.73486646 27.5755481719971 5.0
41.585402374 38.393009185791 6.0
41.585402374 38.3633346557617 7.0
42.3173178092696 34.7173500061035 0.0
23.2468941623901 21.7894153594971 1.0
19.0704236443393 21.3920230865479 2.0
-10.4383552532401 -5.59252595901489 3.0
8.63206839120634 9.52871417999268 4.0
33.6852494163203 29.7356128692627 5.0
42.317317809 36.836555480957 6.0
42.317317809 38.8939743041992 7.0
42.931477575 39.6010284423828 0.0
22.005453055 17.7157821655273 1.0
20.926024519 17.9823207855225 2.0
-5.8784626661 -6.58261632919312 3.0
15.047561843 14.7975568771362 4.0
27.883915732 26.8547840118408 5.0
42.931477585 39.391284942627 6.0
42.931477585 37.9523735046387 7.0
46.096274403 34.9248886108398 0.0
25.32088965 21.040979385376 1.0
20.775384755 18.0037059783936 2.0
-5.4662065709 -6.33879089355469 3.0
15.309178177 16.2221813201904 4.0
30.787096229 27.046537399292 5.0
46.096274411 37.457145690918 6.0
46.096274411 38.5175018310547 7.0
39.955261544 38.0565032958984 0.0
19.931762754 17.762414932251 1.0
20.02349879 19.8553256988525 2.0
-8.9110310329 -7.8418664932251 3.0
11.112467747 11.3818483352661 4.0
28.842793797 27.5776538848877 5.0
39.955261554 39.3833198547363 6.0
39.955261554 38.0312118530273 7.0
45.415961836 39.5297088623047 0.0
23.097461473 18.0270404815674 1.0
22.318500363 18.4816265106201 2.0
-8.5889928185 -6.86664295196533 3.0
13.729507535 13.9234113693237 4.0
31.686454301 28.6587562561035 5.0
45.415961846 38.5327033996582 6.0
45.415961846 37.7915153503418 7.0
44.275746647 38.8625373840332 0.0
22.654009807 20.8217697143555 1.0
21.62173684 21.3384380340576 2.0
-7.7547516702 -6.64871025085449 3.0
13.86698516 15.0281648635864 4.0
30.408761488 27.1864204406738 5.0
44.275746657 38.5459632873535 6.0
44.275746657 38.7896575927734 7.0
34.766238956 36.4577865600586 0.0
17.943458551 19.5047988891602 1.0
16.822780406 19.6932811737061 2.0
-5.6356736314 -6.13012552261353 3.0
11.187106764 12.0098743438721 4.0
23.579132192 24.2640991210938 5.0
34.766238966 37.2673683166504 6.0
34.766238966 38.3619270324707 7.0
39.697519492335 39.4813537597656 0.0
21.2162061883195 18.5607566833496 1.0
18.481313305488 16.9600372314453 2.0
-4.67528909066386 -6.03753757476807 3.0
13.8060242148741 13.9673175811768 4.0
25.8914952793298 25.4044914245605 5.0
39.697519492 38.7140808105469 6.0
39.697519492 37.9176368713379 7.0
36.4599525996002 36.9982566833496 0.0
18.3065166475976 17.3094387054443 1.0
18.1534359526018 17.8176383972168 2.0
-5.15104061002709 -4.83540773391724 3.0
13.0023953426082 13.0549306869507 4.0
23.457557257596 23.3491973876953 5.0
36.459952599 37.690975189209 6.0
36.459952599 37.9722137451172 7.0
44.856515577 34.7300796508789 0.0
22.473710733 18.2238788604736 1.0
22.382804843 20.5238056182861 2.0
-8.0040806232 -8.11902523040771 3.0
14.37872421 14.5894737243652 4.0
30.477791367 28.9276676177979 5.0
44.856515586 37.448486328125 6.0
44.856515586 38.3791732788086 7.0
41.1579139562054 39.5911979675293 0.0
18.5564288232053 18.4170455932617 1.0
22.601485143205 22.7831153869629 2.0
-7.31439675300561 -6.95280313491821 3.0
15.2870883902055 16.3563270568848 4.0
25.8708255762049 25.1820011138916 5.0
41.157913956 39.2486877441406 6.0
41.157913956 38.6349067687988 7.0
32.092085782 39.6106567382812 0.0
17.272160967 17.7487602233887 1.0
14.819924815 17.967004776001 2.0
-6.6255124723 -5.76021003723145 3.0
8.1944123328 8.48110485076904 4.0
23.897673449 24.2485523223877 5.0
32.092085792 38.28076171875 6.0
32.092085792 37.932487487793 7.0
30.327518415 37.3035430908203 0.0
13.541286609 17.8687286376953 1.0
16.786231807 22.4109535217285 2.0
-5.5133467024 -5.19699192047119 3.0
11.272885094 12.475170135498 4.0
19.054633321 19.5776023864746 5.0
30.327518425 38.6042671203613 6.0
30.327518425 38.7228317260742 7.0
36.482977268 39.0804862976074 0.0
19.553260998 17.0510520935059 1.0
16.929716271 17.1136207580566 2.0
-7.3162847454 -5.62727451324463 3.0
9.613431516 10.0466403961182 4.0
26.869545753 25.6699390411377 5.0
36.482977278 38.941349029541 6.0
36.482977278 37.7085418701172 7.0
40.491170904 39.5412368774414 0.0
20.509684159 17.3545360565186 1.0
19.981486745 18.8166675567627 2.0
-8.9848735316 -6.33952522277832 3.0
10.996613204 10.7380218505859 4.0
29.4945577 27.455171585083 5.0
40.491170914 38.6130294799805 6.0
40.491170914 38.0393409729004 7.0
41.188898324 40.3837280273438 0.0
17.482572735 16.7367057800293 1.0
23.70632559 21.5126457214355 2.0
-7.86829602 -6.49981355667114 3.0
15.838029561 16.4243545532227 4.0
25.350868764 24.3608722686768 5.0
41.188898334 39.7150726318359 6.0
41.188898334 38.2748756408691 7.0
37.756949894 39.3667259216309 0.0
19.043754327 17.7176990509033 1.0
18.713195567 18.132740020752 2.0
-5.5841962186 -6.03086280822754 3.0
13.128999338 12.5235013961792 4.0
24.627950556 24.1311492919922 5.0
37.756949904 39.0832366943359 6.0
37.756949904 38.017650604248 7.0
26.962645791 35.5660362243652 0.0
14.231795944 17.8442535400391 1.0
12.730849849 17.1239891052246 2.0
-4.3614404899 -5.87844753265381 3.0
8.3694093501 8.61323547363281 4.0
18.593236442 19.2884216308594 5.0
26.9626458 36.6992225646973 6.0
26.9626458 38.0589637756348 7.0
44.1546808660895 36.8254013061523 0.0
20.845479755763 20.5473022460938 1.0
23.3092011082446 23.6247158050537 2.0
-6.97221748920505 -7.235032081604 3.0
16.3369836194072 16.2490100860596 4.0
27.8176971034536 27.0591068267822 5.0
44.154680866 37.8936042785645 6.0
44.154680866 39.258186340332 7.0
36.4992036 37.8822746276855 0.0
18.499566849 17.9694080352783 1.0
17.999636751 19.6494178771973 2.0
-8.164709678 -7.19838857650757 3.0
9.834927063 10.9263486862183 4.0
26.664276537 25.8939685821533 5.0
36.49920361 37.7012672424316 6.0
36.49920361 38.3017616271973 7.0
36.136186998 38.2519340515137 0.0
19.096538919 17.2199687957764 1.0
17.039648079 16.7989940643311 2.0
-8.7939144312 -6.60872364044189 3.0
8.2457336392 8.49795722961426 4.0
27.89045336 25.4412384033203 5.0
36.136187007 38.5442771911621 6.0
36.136187007 37.7164764404297 7.0
43.805905699 38.6154136657715 0.0
19.467800315 17.3762302398682 1.0
24.338105385 21.5224094390869 2.0
-10.07714477 -6.81198024749756 3.0
14.260960606 15.1414966583252 4.0
29.544945095 26.3030300140381 5.0
43.805905709 38.6243667602539 6.0
43.805905709 38.094841003418 7.0
43.400673471 39.3181228637695 0.0
21.450145846 18.2652263641357 1.0
21.950527627 21.2074584960938 2.0
-10.718319514 -8.56122303009033 3.0
11.232208106 12.0700073242188 4.0
32.168465368 28.9847068786621 5.0
43.400673478 37.9904441833496 6.0
43.400673478 38.426212310791 7.0
35.708898922 38.1408500671387 0.0
19.860917727 20.6685256958008 1.0
15.847981195 20.8581523895264 2.0
-6.8010445999 -7.15414428710938 3.0
9.0469365848 10.237645149231 4.0
26.661962337 27.0262126922607 5.0
35.708898932 37.9282531738281 6.0
35.708898932 38.6784172058105 7.0
42.19537597 38.4190063476562 0.0
21.611035583 19.2420635223389 1.0
20.584340387 20.1553134918213 2.0
-7.818182446 -6.50506114959717 3.0
12.766157932 13.106032371521 4.0
29.429218038 29.4020175933838 5.0
42.195375979 38.5174560546875 6.0
42.195375979 38.1820487976074 7.0
38.219132248 37.6021690368652 0.0
18.124133621 19.546293258667 1.0
20.094998634 21.1196765899658 2.0
-3.9675986496 -5.88177299499512 3.0
16.127399981 15.5530233383179 4.0
22.091732274 23.7608242034912 5.0
38.219132252 37.8726539611816 6.0
38.219132252 38.3980522155762 7.0
37.34785903 40.0214080810547 0.0
19.494183125 18.3025646209717 1.0
17.853675906 17.163480758667 2.0
-4.6743476987 -5.38087558746338 3.0
13.179328197 14.8149833679199 4.0
24.168530833 25.8003540039062 5.0
37.34785904 38.7032775878906 6.0
37.34785904 37.4965934753418 7.0
41.413317073 34.2613334655762 0.0
22.502810393 19.7356395721436 1.0
18.91050668 16.6663055419922 2.0
-4.5479233593 -6.21697616577148 3.0
14.362583311 14.3824586868286 4.0
27.050733762 25.5736656188965 5.0
41.413317083 36.8319854736328 6.0
41.413317083 37.7006378173828 7.0
39.408474003 40.6333351135254 0.0
19.787258075 16.3774700164795 1.0
19.621215928 16.9697494506836 2.0
-8.5467597404 -5.77481174468994 3.0
11.074456178 11.2354145050049 4.0
28.334017825 27.9660263061523 5.0
39.408474012 38.9195213317871 6.0
39.408474012 37.2394981384277 7.0
44.6219477930596 39.2290802001953 0.0
23.5295769810594 18.1718940734863 1.0
21.0923708210596 19.1033954620361 2.0
-9.44677516695988 -7.15398597717285 3.0
11.6455956540597 12.2990732192993 4.0
32.9763521480588 29.7604579925537 5.0
44.621947793 39.2680435180664 6.0
44.621947793 37.8836784362793 7.0
35.042314664 37.6828575134277 0.0
17.377708596 19.7601642608643 1.0
17.66460607 21.2511615753174 2.0
-4.7609303256 -6.40326023101807 3.0
12.903675736 13.0322036743164 4.0
22.13863893 22.5032081604004 5.0
35.042314673 37.3393821716309 6.0
35.042314673 38.7120018005371 7.0
44.668685977 38.5372505187988 0.0
21.700098462 16.0450439453125 1.0
22.968587516 17.0630588531494 2.0
-8.4663400465 -7.04927968978882 3.0
14.50224746 15.8255062103271 4.0
30.166438518 27.8241519927979 5.0
44.668685987 38.3334655761719 6.0
44.668685987 37.2814140319824 7.0
44.271188228 39.2124252319336 0.0
21.134860564 17.3605632781982 1.0
23.136327666 19.8883266448975 2.0
-6.9953395415 -6.56080913543701 3.0
16.140988117 16.9699478149414 4.0
28.130200113 25.8115043640137 5.0
44.271188235 38.4050750732422 6.0
44.271188235 38.3365516662598 7.0
43.3296157145675 38.6363182067871 0.0
21.8144364805029 18.6364040374756 1.0
21.5151792385451 20.9786052703857 2.0
-10.3071776780671 -8.27692222595215 3.0
11.2080015596363 12.1965818405151 4.0
32.1216141588196 28.9941310882568 5.0
43.329615715 38.3849639892578 6.0
43.329615715 38.2598495483398 7.0
37.727034885 36.3503494262695 0.0
16.740902708 17.2993545532227 1.0
20.986132178 22.4434947967529 2.0
-10.156378742 -7.63459300994873 3.0
10.829753428 12.1094446182251 4.0
26.897281458 25.4926662445068 5.0
37.727034893 37.3625373840332 6.0
37.727034893 38.5631675720215 7.0
43.665312103 35.9822845458984 0.0
21.432332029 16.8152160644531 1.0
22.232980074 17.2242069244385 2.0
-7.2001290055 -6.2326545715332 3.0
15.032851059 15.6108980178833 4.0
28.632461044 27.0609607696533 5.0
43.665312113 37.1079406738281 6.0
43.665312113 37.575855255127 7.0
40.6563483834553 38.5425643920898 0.0
20.9212538534144 17.9228458404541 1.0
19.7350945303589 18.6338024139404 2.0
-7.59696943049863 -7.84156799316406 3.0
12.1381250996025 12.6853294372559 4.0
28.5182231244438 28.5860366821289 5.0
40.656348383 38.4834022521973 6.0
40.656348383 37.8940048217773 7.0
30.759139545 39.8242874145508 0.0
13.433268738 16.3226337432861 1.0
17.325870807 20.7528858184814 2.0
-5.9875785844 -5.78634834289551 3.0
11.338292213 12.0246229171753 4.0
19.420847332 21.169849395752 5.0
30.759139555 39.664249420166 6.0
30.759139555 37.8251075744629 7.0
36.474229645 39.5734100341797 0.0
14.989459641 17.0739307403564 1.0
21.484770007 22.8482685089111 2.0
-5.0219280476 -6.33901596069336 3.0
16.462841952 16.9426364898682 4.0
20.011387696 21.1584453582764 5.0
36.474229652 38.1857528686523 6.0
36.474229652 38.5920181274414 7.0
33.340721185 36.6191062927246 0.0
15.596021276 18.2751483917236 1.0
17.744699909 22.2481555938721 2.0
-7.4570294442 -6.47745037078857 3.0
10.287670455 12.1020812988281 4.0
23.05305073 23.6981258392334 5.0
33.340721195 36.9607200622559 6.0
33.340721195 38.5387916564941 7.0
34.143349258 35.800479888916 0.0
15.417033592 15.6841268539429 1.0
18.726315666 18.2945423126221 2.0
-5.8334622114 -6.55417633056641 3.0
12.892853445 13.0055017471313 4.0
21.250495814 21.4418239593506 5.0
34.143349268 38.0671691894531 6.0
34.143349268 37.7962532043457 7.0
40.763829748 39.1516189575195 0.0
20.589059268 17.4663600921631 1.0
20.174770481 20.7779884338379 2.0
-11.800756688 -8.55895233154297 3.0
8.374013783 9.9173755645752 4.0
32.389815966 29.5637683868408 5.0
40.763829757 38.6028060913086 6.0
40.763829757 38.1368026733398 7.0
29.788224949 35.7150421142578 0.0
14.329239328 17.0737781524658 1.0
15.458985621 19.3655891418457 2.0
-6.2969425219 -5.14855003356934 3.0
9.1620430892 10.4467115402222 4.0
20.62618186 22.2186908721924 5.0
29.788224958 38.079029083252 6.0
29.788224958 38.1271705627441 7.0
34.675869546 39.9288139343262 0.0
17.321679699 21.0032958984375 1.0
17.354189848 22.0540981292725 2.0
-5.026706589 -6.52810907363892 3.0
12.32748325 13.4520206451416 4.0
22.348386296 23.5548858642578 5.0
34.675869554 39.3473052978516 6.0
34.675869554 39.1159286499023 7.0
40.7238558279895 39.676399230957 0.0
19.1061702059935 17.7111377716064 1.0
21.6176856259914 21.1939105987549 2.0
-8.60840523789577 -8.13226890563965 3.0
13.0092803879935 12.7899503707886 4.0
27.7145752997044 27.1225490570068 5.0
40.723855828 39.02392578125 6.0
40.723855828 38.4413185119629 7.0
41.414838344 35.0238609313965 0.0
21.79665744 17.7927265167236 1.0
19.618180905 20.3057594299316 2.0
-10.700829677 -7.73186111450195 3.0
8.9173512199 10.3136720657349 4.0
32.497487126 29.6861228942871 5.0
41.414838352 36.6936492919922 6.0
41.414838352 38.3621711730957 7.0
33.391441204 37.6954917907715 0.0
16.946662464 19.0476036071777 1.0
16.44477874 20.8906707763672 2.0
-6.5811351715 -6.97586107254028 3.0
9.8636435585 11.1420335769653 4.0
23.527797645 23.9220523834229 5.0
33.391441214 37.8286399841309 6.0
33.391441214 38.5835113525391 7.0
36.660645376 37.5117073059082 0.0
17.87289688 18.784252166748 1.0
18.787748497 21.1154689788818 2.0
-6.6425323884 -7.03012275695801 3.0
12.1452161 12.9863157272339 4.0
24.515429278 23.3860759735107 5.0
36.660645385 37.5796508789062 6.0
36.660645385 38.2959289550781 7.0
33.547567157 36.6569404602051 0.0
16.605704713 18.3050479888916 1.0
16.941862445 21.7436714172363 2.0
-8.1253896728 -7.84177017211914 3.0
8.8164727625 10.4224920272827 4.0
24.731094395 25.4744510650635 5.0
33.547567166 37.0505676269531 6.0
33.547567166 38.5951652526855 7.0
36.717923485 35.8551902770996 0.0
19.502688668 19.6084232330322 1.0
17.215234818 18.9154472351074 2.0
-6.0032006006 -7.09572076797485 3.0
11.212034207 10.849720954895 4.0
25.505889278 25.4962882995605 5.0
36.717923495 37.8734893798828 6.0
36.717923495 38.6213989257812 7.0
44.575737489 39.0113372802734 0.0
24.24689991 18.5161914825439 1.0
20.328837584 16.7506427764893 2.0
-6.0388557339 -6.86410284042358 3.0
14.289981845 13.2420587539673 4.0
30.285755649 28.5042934417725 5.0
44.575737494 39.6898994445801 6.0
44.575737494 37.9387283325195 7.0
43.091610313 37.4515724182129 0.0
22.329030269 19.4624996185303 1.0
20.762580045 19.3551044464111 2.0
-7.1251861541 -6.52673435211182 3.0
13.637393882 13.7396125793457 4.0
29.454216432 28.1623153686523 5.0
43.091610321 38.0801200866699 6.0
43.091610321 38.4085083007812 7.0
42.647069964 36.2827110290527 0.0
21.774885881 18.8103733062744 1.0
20.872184083 19.3114471435547 2.0
-7.6954983452 -7.19486618041992 3.0
13.176685728 12.8811769485474 4.0
29.470384236 26.9542083740234 5.0
42.647069973 37.1617202758789 6.0
42.647069973 38.1011962890625 7.0
36.24022533 38.7500267028809 0.0
17.188554164 17.6467113494873 1.0
19.051671166 19.9954471588135 2.0
-6.1891251621 -6.09590673446655 3.0
12.862545995 13.2469730377197 4.0
23.377679336 23.6893405914307 5.0
36.240225339 38.4422721862793 6.0
36.240225339 38.070743560791 7.0
31.876756195 35.9190101623535 0.0
14.446454361 17.3462619781494 1.0
17.430301835 20.934757232666 2.0
-5.0545683645 -6.49335527420044 3.0
12.375733461 13.1319532394409 4.0
19.501022735 21.4917469024658 5.0
31.876756203 37.6297035217285 6.0
31.876756203 38.4817924499512 7.0
38.711049649 36.4316139221191 0.0
19.068925854 19.0153942108154 1.0
19.642123794 20.3911113739014 2.0
-5.0802825502 -6.89239931106567 3.0
14.561841234 14.660472869873 4.0
24.149208415 23.7649803161621 5.0
38.711049659 37.2930603027344 6.0
38.711049659 38.6595458984375 7.0
33.592635662 39.1254272460938 0.0
16.042625471 17.1367511749268 1.0
17.550010192 19.5258445739746 2.0
-5.9813560668 -6.0064582824707 3.0
11.568654115 12.0733289718628 4.0
22.023981548 22.7371234893799 5.0
33.592635672 38.873966217041 6.0
33.592635672 37.8708534240723 7.0
31.9394970213296 36.1707878112793 0.0
15.7221064593445 18.9186573028564 1.0
16.2173905613348 19.5127868652344 2.0
-3.67606552949996 -4.91719722747803 3.0
12.5413250313355 13.0996160507202 4.0
19.3981719893576 21.7066707611084 5.0
31.939497021 37.4598236083984 6.0
31.939497021 38.3520164489746 7.0
32.605677366 37.6998252868652 0.0
15.053063258 16.1318321228027 1.0
17.552614114 18.2213077545166 2.0
-5.3475005701 -4.61513900756836 3.0
12.20511354 11.5713329315186 4.0
20.400563832 21.2397594451904 5.0
32.605677369 38.4368629455566 6.0
32.605677369 37.6347846984863 7.0
35.02561424 39.7455940246582 0.0
18.375955543 20.5471267700195 1.0
16.649658698 20.2320442199707 2.0
-5.1216145805 -6.50403881072998 3.0
11.528044108 12.5048398971558 4.0
23.497570133 25.3339729309082 5.0
35.02561425 39.2523422241211 6.0
35.02561425 38.2831687927246 7.0
29.155020361 38.21875 0.0
13.928860816 18.0569610595703 1.0
15.226159546 20.0909557342529 2.0
-5.2043069028 -5.68940877914429 3.0
10.021852634 11.1943778991699 4.0
19.133167728 21.7673664093018 5.0
29.155020371 38.2261161804199 6.0
29.155020371 37.9722328186035 7.0
29.979099891 34.6037788391113 0.0
16.392071584 20.3025779724121 1.0
13.587028308 19.6305694580078 2.0
-4.5603535805 -4.44800615310669 3.0
9.026674718 10.2378225326538 4.0
20.952425173 20.5530548095703 5.0
29.9790999 37.4360542297363 6.0
29.9790999 38.6387023925781 7.0
40.412374419 38.1802024841309 0.0
21.632869342 19.6780796051025 1.0
18.779505078 18.1570529937744 2.0
-4.6354756287 -5.81484413146973 3.0
14.14402944 13.4902839660645 4.0
26.26834498 24.0640182495117 5.0
40.412374428 38.4475517272949 6.0
40.412374428 38.4524192810059 7.0
44.7996696066759 40.1845779418945 0.0
24.3565352806759 20.3579845428467 1.0
20.4431343366759 17.5501384735107 2.0
-4.33926525477587 -6.56163835525513 3.0
16.1038690816759 14.8861961364746 4.0
28.6958005009671 27.1913547515869 5.0
44.799669607 39.5273704528809 6.0
44.799669607 38.3375129699707 7.0
35.691930078 37.1548271179199 0.0
18.173171081 20.883581161499 1.0
17.518759 20.4196796417236 2.0
-4.3926482071 -6.05445432662964 3.0
13.126110785 12.9954280853271 4.0
22.565819295 23.509298324585 5.0
35.691930085 38.2849998474121 6.0
35.691930085 38.7726860046387 7.0
37.293002373 39.8247375488281 0.0
17.018201783 17.0743732452393 1.0
20.274800591 20.6157398223877 2.0
-8.9983650782 -6.45097732543945 3.0
11.276435503 12.1281795501709 4.0
26.016566871 23.7936420440674 5.0
37.293002382 39.5139808654785 6.0
37.293002382 37.8783378601074 7.0
31.749050626 39.9060325622559 0.0
16.61083934 17.6210594177246 1.0
15.138211288 16.5646133422852 2.0
-3.9083704158 -5.72663593292236 3.0
11.229840863 10.7420425415039 4.0
20.519209764 21.534143447876 5.0
31.749050635 39.1338157653809 6.0
31.749050635 37.4966812133789 7.0
44.43859931 35.9501800537109 0.0
22.567248059 17.1624965667725 1.0
21.871351255 18.1946582794189 2.0
-10.274790847 -7.6455020904541 3.0
11.596560402 12.2384643554688 4.0
32.842038912 27.445104598999 5.0
44.438599316 36.9843788146973 6.0
44.438599316 37.6335983276367 7.0
30.584762336 36.0443687438965 0.0
14.448986708 17.1661567687988 1.0
16.13577563 20.1180191040039 2.0
-7.0107275854 -6.82076025009155 3.0
9.1250480359 9.59850788116455 4.0
21.459714302 21.0373249053955 5.0
30.584762345 37.0255012512207 6.0
30.584762345 38.2627372741699 7.0
36.167711461 40.5016593933105 0.0
15.702748451 16.4308986663818 1.0
20.464963013 19.8333835601807 2.0
-4.5943051912 -6.14291191101074 3.0
15.870657815 16.385368347168 4.0
20.297053649 21.8605079650879 5.0
36.167711467 39.119686126709 6.0
36.167711467 37.7031364440918 7.0
45.589023155 36.1110229492188 0.0
22.43199889 16.4521350860596 1.0
23.157024265 16.1621913909912 2.0
-9.6703339461 -7.44725704193115 3.0
13.486690309 12.1208353042603 4.0
32.102332846 27.9177169799805 5.0
45.589023165 37.0966987609863 6.0
45.589023165 37.1847229003906 7.0
33.954863965 38.1003913879395 0.0
16.569146941 17.0386028289795 1.0
17.385717024 19.5538558959961 2.0
-8.6921868392 -7.13388633728027 3.0
8.6935301746 9.59525585174561 4.0
25.26133379 24.6805458068848 5.0
33.954863975 39.903491973877 6.0
33.954863975 38.1070175170898 7.0
39.104021209 37.1890602111816 0.0
19.726585997 18.6673240661621 1.0
19.377435214 19.6046199798584 2.0
-5.8707549184 -6.55767869949341 3.0
13.506680288 13.2063274383545 4.0
25.597340923 25.4118576049805 5.0
39.104021217 38.2189979553223 6.0
39.104021217 38.3823738098145 7.0
41.6151393745108 36.0034675598145 0.0
21.6144474948183 20.8185634613037 1.0
20.0006918800463 20.2191123962402 2.0
-5.68615976863137 -7.36462593078613 3.0
14.3145321112399 16.212366104126 4.0
27.3006071309532 29.1972198486328 5.0
41.615139375 37.324878692627 6.0
41.615139375 38.740364074707 7.0
29.251209159 36.5690612792969 0.0
16.305978933 19.4952411651611 1.0
12.945230226 16.8321228027344 2.0
-4.8421763952 -6.58975744247437 3.0
8.1030538205 8.1594123840332 4.0
21.148155339 22.5768966674805 5.0
29.251209169 37.0963096618652 6.0
29.251209169 37.8749160766602 7.0
39.145261593 38.9524688720703 0.0
20.853611949 19.3592910766602 1.0
18.291649647 18.0178546905518 2.0
-6.3281983222 -5.88606643676758 3.0
11.963451316 12.1013307571411 4.0
27.181810279 25.6368141174316 5.0
39.145261601 38.4512176513672 6.0
39.145261601 38.1817741394043 7.0
43.724223504 36.2134056091309 0.0
24.240212841 19.1630268096924 1.0
19.484010663 16.3680171966553 2.0
-6.7543223688 -6.72774839401245 3.0
12.729688284 11.391209602356 4.0
30.99453522 26.6184062957764 5.0
43.724223514 37.19091796875 6.0
43.724223514 37.8654594421387 7.0
30.694427938 36.2299652099609 0.0
15.962392584 18.8357257843018 1.0
14.732035355 19.7146396636963 2.0
-5.8529078341 -5.35205793380737 3.0
8.8791275119 9.8284330368042 4.0
21.815300427 21.3804016113281 5.0
30.694427948 37.0368576049805 6.0
30.694427948 38.3340644836426 7.0
46.899446071 38.4624633789062 0.0
22.810606072 18.7257556915283 1.0
24.08884 22.9906597137451 2.0
-12.207831656 -9.17830276489258 3.0
11.881008335 13.1736688613892 4.0
35.018437738 28.2085132598877 5.0
46.899446079 38.4089660644531 6.0
46.899446079 38.7839050292969 7.0
30.801741864 38.8928833007812 0.0
14.581609416 16.3869285583496 1.0
16.220132448 19.8372364044189 2.0
-7.6078809929 -7.20111179351807 3.0
8.6122514454 9.51728916168213 4.0
22.189490419 21.9634876251221 5.0
30.801741874 38.4225196838379 6.0
30.801741874 37.9269752502441 7.0
40.3889007570329 35.9056053161621 0.0
21.8864546730297 20.0105285644531 1.0
18.5024460860314 20.8980541229248 2.0
-8.35977776734346 -7.5291223526001 3.0
10.1426683190367 10.5267763137817 4.0
30.2462322648819 29.0382213592529 5.0
40.388900757 37.1913604736328 6.0
40.388900757 38.6142845153809 7.0
35.801686121 36.4369430541992 0.0
19.152651215 20.6031761169434 1.0
16.649034906 17.9702014923096 2.0
-1.2326560583 -5.91027688980103 3.0
15.416378838 14.7479400634766 4.0
20.385307283 22.1405830383301 5.0
35.801686131 36.8716278076172 6.0
35.801686131 38.5079727172852 7.0
43.275430292 37.2468223571777 0.0
21.49768275 20.1980228424072 1.0
21.777747542 22.2920093536377 2.0
-8.5057364073 -7.18509531021118 3.0
13.272011125 14.4264154434204 4.0
30.003419167 29.2678737640381 5.0
43.275430302 37.4689903259277 6.0
43.275430302 38.8737564086914 7.0
36.6745338859035 37.6527061462402 0.0
17.2258453439028 17.3092403411865 1.0
19.4486885479032 20.2807064056396 2.0
-7.45674864010256 -6.76329612731934 3.0
11.9919399079021 12.3733997344971 4.0
24.6825939849038 25.0338249206543 5.0
36.674533885 38.1266059875488 6.0
36.674533885 38.2824363708496 7.0
42.2919299411196 36.7328262329102 0.0
20.8701371011196 18.5836925506592 1.0
21.4217928491193 20.9584980010986 2.0
-7.8990978219197 -7.14548254013062 3.0
13.5226950281196 15.265739440918 4.0
28.769234923119 27.0314064025879 5.0
42.291929941 37.2983436584473 6.0
42.291929941 38.1323013305664 7.0
37.80686854 40.2857246398926 0.0
18.84529386 17.4930572509766 1.0
18.961574682 17.8232574462891 2.0
-3.0668288295 -6.16938638687134 3.0
15.894745843 15.6355495452881 4.0
21.912122698 23.3225402832031 5.0
37.80686855 38.9864883422852 6.0
37.80686855 37.9358596801758 7.0
36.296445798 39.6125221252441 0.0
19.144127702 20.5195503234863 1.0
17.152318097 19.877613067627 2.0
-4.4799623055 -5.82166814804077 3.0
12.672355782 12.5882196426392 4.0
23.624090017 23.4946041107178 5.0
36.296445808 38.7144050598145 6.0
36.296445808 38.4514007568359 7.0
26.786347975 36.9559135437012 0.0
13.739106858 20.7554378509521 1.0
13.047241117 22.3015441894531 2.0
-5.0192652199 -6.34451627731323 3.0
8.0279758877 10.2528371810913 4.0
18.758372088 21.4266090393066 5.0
26.786347985 37.5944671630859 6.0
26.786347985 38.8998832702637 7.0
38.715518927 34.8286323547363 0.0
20.971478582 19.3625221252441 1.0
17.744040345 19.3628635406494 2.0
-7.5775969749 -7.20511817932129 3.0
10.16644336 10.7594957351685 4.0
28.549075567 26.6323680877686 5.0
38.715518937 36.5846328735352 6.0
38.715518937 38.2582817077637 7.0
35.505062995 39.6642227172852 0.0
16.975132375 17.2259540557861 1.0
18.529930624 19.679723739624 2.0
-6.1934391103 -5.11766338348389 3.0
12.336491507 12.9192733764648 4.0
23.168571492 23.2860584259033 5.0
35.505063002 39.8746871948242 6.0
35.505063002 37.8663215637207 7.0
44.227649061 35.5304107666016 0.0
23.455343466 21.7359809875488 1.0
20.772305597 21.6844692230225 2.0
-7.272984789 -7.21561622619629 3.0
13.499320798 14.0473499298096 4.0
30.728328264 26.3004093170166 5.0
44.227649071 37.6820259094238 6.0
44.227649071 39.0220718383789 7.0
45.773124246 33.3483238220215 0.0
24.134680774 17.5405864715576 1.0
21.638443473 16.6813869476318 2.0
-6.2883820036 -6.71469783782959 3.0
15.350061461 14.868275642395 4.0
30.423062786 28.6510372161865 5.0
45.773124255 36.2077445983887 6.0
45.773124255 37.5494651794434 7.0
38.6152827522378 38.6166496276855 0.0
18.6636500942378 16.5856781005859 1.0
19.9516326682378 17.5414447784424 2.0
-8.33253161363777 -6.74179983139038 3.0
11.6191010542378 10.920280456543 4.0
26.9961816644691 27.5464878082275 5.0
38.615282752 39.9415397644043 6.0
38.615282752 37.6706619262695 7.0
42.131124826 36.1531066894531 0.0
21.88265512 19.1036434173584 1.0
20.248469706 18.0285205841064 2.0
-6.067617241 -6.29041004180908 3.0
14.180852455 13.7849254608154 4.0
27.950272371 26.3579730987549 5.0
42.131124836 37.8071632385254 6.0
42.131124836 38.1163177490234 7.0
38.030369756 36.5552940368652 0.0
19.698625144 16.3324356079102 1.0
18.331744611 16.5336265563965 2.0
-8.0645153211 -7.26559162139893 3.0
10.26722928 9.37832164764404 4.0
27.763140475 27.1530799865723 5.0
38.030369766 37.9865684509277 6.0
38.030369766 37.5903511047363 7.0
32.694869995 33.4845848083496 0.0
14.429732866 16.9723320007324 1.0
18.265137131 21.8640613555908 2.0
-5.8301144142 -6.87159776687622 3.0
12.435022709 13.1505746841431 4.0
20.259847288 20.6953105926514 5.0
32.694870002 36.3521499633789 6.0
32.694870002 38.4278526306152 7.0
41.58883183 36.142765045166 0.0
19.184872754 18.5600395202637 1.0
22.403959077 22.0343379974365 2.0
-6.437067104 -7.00483798980713 3.0
15.966891964 16.2597789764404 4.0
25.621939867 25.2923736572266 5.0
41.58883184 37.7245826721191 6.0
41.58883184 38.663158416748 7.0
33.622063035 39.5672225952148 0.0
16.212159516 16.8275699615479 1.0
17.409903519 20.8000202178955 2.0
-8.2486263793 -7.00642013549805 3.0
9.16127713 10.2425479888916 4.0
24.460785905 23.5319938659668 5.0
33.622063045 39.3157157897949 6.0
33.622063045 38.2448768615723 7.0
40.101738497 38.1461143493652 0.0
22.613544165 20.0844078063965 1.0
17.488194333 17.3468914031982 2.0
-8.294425066 -7.38509559631348 3.0
9.1937692583 9.55960655212402 4.0
30.90796924 28.0162487030029 5.0
40.101738506 38.9010162353516 6.0
40.101738506 37.8704414367676 7.0
39.347750484 38.8317527770996 0.0
19.631600055 20.4883460998535 1.0
19.716150429 22.4274349212646 2.0
-7.2110260491 -6.57867002487183 3.0
12.50512437 13.5865125656128 4.0
26.842626114 27.2362174987793 5.0
39.347750494 38.561222076416 6.0
39.347750494 38.7565956115723 7.0
46.240008453 37.4476089477539 0.0
23.277257726 19.0432968139648 1.0
22.962750727 19.3793659210205 2.0
-6.5077704351 -5.19791316986084 3.0
16.454980282 17.5948657989502 4.0
29.785028171 28.4659595489502 5.0
46.240008463 37.9011077880859 6.0
46.240008463 37.9929809570312 7.0
43.885613963 38.2233123779297 0.0
23.043389811 20.9264125823975 1.0
20.842224153 18.861536026001 2.0
-5.1243468199 -7.00564050674438 3.0
15.717877324 15.5973348617554 4.0
28.16773664 26.1577415466309 5.0
43.885613972 38.2060165405273 6.0
43.885613972 38.2629356384277 7.0
44.877286045 34.6564292907715 0.0
23.85937511 17.0195732116699 1.0
21.017910935 17.0872077941895 2.0
-10.548874854 -7.70529556274414 3.0
10.469036071 9.99593067169189 4.0
34.408249974 29.8998527526855 5.0
44.877286055 36.6510581970215 6.0
44.877286055 37.8367729187012 7.0
44.28102484 39.061939239502 0.0
23.476033214 19.8337802886963 1.0
20.804991626 19.3273811340332 2.0
-7.5751269732 -6.93829441070557 3.0
13.229864644 13.8864946365356 4.0
31.051160197 29.0546836853027 5.0
44.281024849 37.9993171691895 6.0
44.281024849 37.9134559631348 7.0
39.443195324 39.9367599487305 0.0
17.235953612 16.773681640625 1.0
22.207241712 21.8965015411377 2.0
-9.075573722 -7.77798175811768 3.0
13.13166798 13.5912866592407 4.0
26.311527344 25.0254669189453 5.0
39.443195334 39.4767417907715 6.0
39.443195334 38.5500259399414 7.0
44.5099801195094 38.8014030456543 0.0
20.7192442195094 17.0999717712402 1.0
23.7907359105096 20.1095504760742 2.0
-9.37764966100969 -8.29046058654785 3.0
14.4130862495096 14.3264951705933 4.0
30.0968938805093 29.0858402252197 5.0
44.50998012 38.6260414123535 6.0
44.50998012 37.9602203369141 7.0
36.079576813 34.539852142334 0.0
18.198560719 16.6852359771729 1.0
17.881016095 18.6660137176514 2.0
-8.6789161085 -6.21569490432739 3.0
9.202099978 10.2802906036377 4.0
26.877476836 26.5313968658447 5.0
36.079576822 36.6356811523438 6.0
36.079576822 37.7027702331543 7.0
43.881758866 39.6019668579102 0.0
21.090009056 20.3917980194092 1.0
22.79174981 22.136739730835 2.0
-6.6710954927 -6.30685901641846 3.0
16.120654308 16.9464378356934 4.0
27.761104558 29.2005558013916 5.0
43.881758876 39.0472640991211 6.0
43.881758876 38.4816970825195 7.0
30.108818274 38.3659324645996 0.0
17.112174764 19.5444087982178 1.0
12.996643511 16.6228466033936 2.0
-4.8964528045 -6.18853950500488 3.0
8.1001906976 8.62433338165283 4.0
22.008627577 22.4601058959961 5.0
30.108818284 38.0248794555664 6.0
30.108818284 37.6737213134766 7.0
34.2205121900012 33.8367500305176 0.0
18.8110553650029 20.2180995941162 1.0
15.4094568300067 20.2596454620361 2.0
-7.05168531729669 -5.87221336364746 3.0
8.35777151259514 10.0970964431763 4.0
25.8627406830075 25.5819244384766 5.0
34.22051219 36.5000762939453 6.0
34.22051219 38.3376274108887 7.0
37.402197962 38.6404113769531 0.0
19.960548139 18.6190757751465 1.0
17.441649823 18.259485244751 2.0
-5.9560955294 -6.9165153503418 3.0
11.485554285 11.1548700332642 4.0
25.916643678 26.2131881713867 5.0
37.402197971 38.6726913452148 6.0
37.402197971 38.2411842346191 7.0
35.453722399 39.1289176940918 0.0
16.625051972 17.860294342041 1.0
18.828670427 21.9131813049316 2.0
-7.5918246106 -7.54552936553955 3.0
11.236845807 11.9919881820679 4.0
24.216876592 24.7641162872314 5.0
35.453722409 39.6575241088867 6.0
35.453722409 38.4016227722168 7.0
32.872381145 37.790771484375 0.0
15.679111706 18.2003936767578 1.0
17.193269439 20.691743850708 2.0
-6.3769976884 -5.92192554473877 3.0
10.816271741 11.9368648529053 4.0
22.056109405 21.9461345672607 5.0
32.872381155 38.6675758361816 6.0
32.872381155 38.2420806884766 7.0
33.5777143162235 38.8967933654785 0.0
24.3043372519187 17.3272399902344 1.0
9.27337706863705 20.8468036651611 2.0
-1.413226005e-07 -6.66348314285278 3.0
9.27337692732672 10.5178842544556 4.0
24.3043373936238 23.2891674041748 5.0
33.577714316 39.6319046020508 6.0
33.577714316 38.4198341369629 7.0
43.986168792 38.436393737793 0.0
19.73252431 16.4545841217041 1.0
24.253644482 19.8476085662842 2.0
-7.8587048092 -6.15474605560303 3.0
16.394939663 16.649076461792 4.0
27.591229129 24.1902828216553 5.0
43.986168802 38.4593124389648 6.0
43.986168802 37.5851364135742 7.0
39.701527196 39.4896507263184 0.0
21.433905404 20.127857208252 1.0
18.267621793 18.17995262146 2.0
-6.9530592881 -6.28958129882812 3.0
11.314562495 11.3697576522827 4.0
28.386964701 25.4871158599854 5.0
39.701527205 39.1962242126465 6.0
39.701527205 37.8663597106934 7.0
36.319422307 40.0301780700684 0.0
18.261170173 16.0857753753662 1.0
18.058252134 17.6932373046875 2.0
-9.3076661868 -7.49587249755859 3.0
8.7505859373 9.28429985046387 4.0
27.56883637 26.9862957000732 5.0
36.319422317 40.1858100891113 6.0
36.319422317 37.6850852966309 7.0
39.546529711 37.6465759277344 0.0
21.883828824 19.5343742370605 1.0
17.662700888 17.3124408721924 2.0
-6.2516235274 -6.57799339294434 3.0
11.411077352 11.1157789230347 4.0
28.13545236 26.2918682098389 5.0
39.54652972 38.7156677246094 6.0
39.54652972 38.0377922058105 7.0
46.328884399 35.2986755371094 0.0
20.182008475 16.5791320800781 1.0
26.146875933 21.2157039642334 2.0
-10.473669833 -7.14548397064209 3.0
15.673206099 17.3581409454346 4.0
30.655678309 29.5013866424561 5.0
46.3288844 37.0872116088867 6.0
46.3288844 38.1457557678223 7.0
39.359671453 34.9806060791016 0.0
20.208221774 18.566822052002 1.0
19.151449681 20.0679302215576 2.0
-7.7565035076 -6.66325950622559 3.0
11.394946165 11.9818820953369 4.0
27.964725289 25.7796630859375 5.0
39.359671461 36.5393562316895 6.0
39.359671461 38.4158210754395 7.0
43.813870978 37.9543342590332 0.0
22.861832963 20.7741146087646 1.0
20.952038016 20.9173946380615 2.0
-6.6064627953 -7.35905265808105 3.0
14.345575211 15.4241228103638 4.0
29.468295768 28.3699188232422 5.0
43.813870988 38.7194709777832 6.0
43.813870988 38.6816825866699 7.0
39.146221866 38.9421272277832 0.0
21.274977313 19.2592067718506 1.0
17.871244553 18.3061752319336 2.0
-8.221666714 -7.79278373718262 3.0
9.6495778292 10.2620697021484 4.0
29.496644037 27.6393814086914 5.0
39.146221876 39.8716812133789 6.0
39.146221876 37.7367210388184 7.0
34.753353004 39.5494613647461 0.0
19.306587872 18.8013820648193 1.0
15.446765133 17.7048168182373 2.0
-7.4077231802 -7.51777839660645 3.0
8.0390419425 8.81339740753174 4.0
26.714311062 26.405891418457 5.0
34.753353014 39.1735382080078 6.0
34.753353014 38.1327209472656 7.0
36.8589150083515 36.248950958252 0.0
18.5606086093515 19.4830989837646 1.0
18.2983064083517 21.6777877807617 2.0
-8.423068234052 -8.2694149017334 3.0
9.87523817465195 11.4467935562134 4.0
26.9836767778432 28.2866687774658 5.0
36.858915008 37.520320892334 6.0
36.858915008 38.5040702819824 7.0
45.609388715 36.3589172363281 0.0
20.431056468 16.7664966583252 1.0
25.178332247 20.4939517974854 2.0
-8.8685199312 -7.22541809082031 3.0
16.309812306 15.5593004226685 4.0
29.299576409 27.2352237701416 5.0
45.609388725 37.098518371582 6.0
45.609388725 38.0064582824707 7.0
37.221502305 36.574577331543 0.0
19.795691461 18.0077419281006 1.0
17.425810844 17.3137683868408 2.0
-8.0004149655 -7.484787940979 3.0
9.4253958691 9.58867835998535 4.0
27.796106436 25.8432331085205 5.0
37.221502314 37.4349632263184 6.0
37.221502314 37.8264274597168 7.0
39.3512418478291 36.4378700256348 0.0
21.5003115557496 19.2800197601318 1.0
17.8509302889164 18.071590423584 2.0
-7.9913457065862 -6.50579738616943 3.0
9.85958458276425 9.3059606552124 4.0
29.4916572628094 26.7413444519043 5.0
39.351241848 37.3937568664551 6.0
39.351241848 38.1073112487793 7.0
44.354286129 38.0149116516113 0.0
23.778468789 20.3683109283447 1.0
20.57581734 19.7280883789062 2.0
-6.7304183908 -6.3310809135437 3.0
13.845398939 14.077956199646 4.0
30.50888719 28.6794376373291 5.0
44.354286139 38.6585540771484 6.0
44.354286139 38.2267532348633 7.0
41.1438439687938 39.8776512145996 0.0
20.0722524531884 16.1948013305664 1.0
21.0715915195 17.5669994354248 2.0
-7.89338141352269 -6.64840078353882 3.0
13.1782101064826 12.6746397018433 4.0
27.9656338666077 26.1149482727051 5.0
41.143843969 39.1461944580078 6.0
41.143843969 37.5298004150391 7.0
32.436363662 38.2723236083984 0.0
18.29560079 20.8478183746338 1.0
14.140762872 17.3991928100586 2.0
-4.3766657499 -5.92985916137695 3.0
9.7640971121 9.4856595993042 4.0
22.67226655 23.675199508667 5.0
32.436363672 37.9255790710449 6.0
32.436363672 38.1044578552246 7.0
35.980268817 39.950138092041 0.0
18.371260578 17.9045333862305 1.0
17.609008241 19.4868717193604 2.0
-7.5074447835 -7.45249271392822 3.0
10.101563449 10.7137460708618 4.0
25.87870537 26.0074882507324 5.0
35.980268826 38.4578857421875 6.0
35.980268826 38.3808441162109 7.0
39.361696941 35.3204956054688 0.0
18.798616943 16.0567970275879 1.0
20.563079997 17.0013408660889 2.0
-4.5015726852 -4.29186153411865 3.0
16.061507302 16.6354999542236 4.0
23.300189639 22.5916538238525 5.0
39.361696951 36.9872207641602 6.0
39.361696951 37.6248931884766 7.0
38.339957491 39.5491752624512 0.0
20.325182647 20.5378723144531 1.0
18.014774844 20.0390720367432 2.0
-6.00621 -7.59119987487793 3.0
12.008564835 12.2722463607788 4.0
26.331392656 28.1390132904053 5.0
38.339957501 39.1344451904297 6.0
38.339957501 38.5156402587891 7.0
38.4370780266379 38.9714126586914 0.0
19.1112284596379 18.0756320953369 1.0
19.3258495766379 19.0748462677002 2.0
-8.41787501523793 -7.86053085327148 3.0
10.9079745616379 11.2380199432373 4.0
27.5291034382175 27.4263381958008 5.0
38.437078026 39.0664939880371 6.0
38.437078026 38.1275825500488 7.0
31.680301152 39.3503379821777 0.0
16.922816031 18.6008491516113 1.0
14.757485122 19.322681427002 2.0
-6.2787127987 -5.86661338806152 3.0
8.4787723148 9.44736671447754 4.0
23.201528838 22.0356750488281 5.0
31.680301161 40.0940399169922 6.0
31.680301161 38.3107147216797 7.0
36.698769122 35.8100967407227 0.0
19.594432034 18.5553550720215 1.0
17.104337088 16.7621841430664 2.0
-2.472193454 -5.5091552734375 3.0
14.632143624 14.1751136779785 4.0
22.066625498 21.8878898620605 5.0
36.698769132 36.5412788391113 6.0
36.698769132 37.7671699523926 7.0
40.942334912 39.1882553100586 0.0
22.219375241 19.6039447784424 1.0
18.722959671 16.7714729309082 2.0
-5.2427967213 -6.73885297775269 3.0
13.48016294 12.8476734161377 4.0
27.462171973 26.9065914154053 5.0
40.942334922 38.3204917907715 6.0
40.942334922 37.7555274963379 7.0
41.496981527 36.8102378845215 0.0
18.951043784 17.6200046539307 1.0
22.545937743 20.1670207977295 2.0
-7.7537930795 -7.00982189178467 3.0
14.792144654 14.9391307830811 4.0
26.704836873 27.2329120635986 5.0
41.496981537 37.461856842041 6.0
41.496981537 37.9791030883789 7.0
33.577466055 38.2830924987793 0.0
17.755510174 17.7207145690918 1.0
15.821955881 16.7386608123779 2.0
-7.358985858 -7.42938899993896 3.0
8.4629700135 8.84808731079102 4.0
25.114496042 25.0298595428467 5.0
33.577466065 38.156379699707 6.0
33.577466065 37.4351921081543 7.0
38.410697594746 38.4117965698242 0.0
20.0282722997678 17.3123779296875 1.0
18.3824252977607 16.3509311676025 2.0
-2.25462323941058 -4.4073314666748 3.0
16.1278020587646 14.7081031799316 4.0
22.2828955387576 24.3371124267578 5.0
38.410697595 39.2803382873535 6.0
38.410697595 37.6869697570801 7.0
40.390522102 37.7898597717285 0.0
23.101316011 21.3158073425293 1.0
17.289206093 18.576868057251 2.0
-8.5131515481 -6.48823976516724 3.0
8.7760545352 9.59349536895752 4.0
31.614467568 26.9253063201904 5.0
40.390522112 38.6680793762207 6.0
40.390522112 38.5638771057129 7.0
38.02724698 35.0099182128906 0.0
17.092159812 16.1066265106201 1.0
20.935087168 18.6067218780518 2.0
-7.0096758313 -6.87712049484253 3.0
13.925411327 14.1699085235596 4.0
24.101835653 24.6244106292725 5.0
38.02724699 37.0739860534668 6.0
38.02724699 37.5205917358398 7.0
34.831322289 40.2558555603027 0.0
17.981428364 19.6494388580322 1.0
16.849893928 19.897289276123 2.0
-5.7085713019 -5.74042844772339 3.0
11.14132262 11.5214385986328 4.0
23.689999673 22.3983592987061 5.0
34.831322295 39.211051940918 6.0
34.831322295 38.0100135803223 7.0
33.152476822 40.236213684082 0.0
16.395481675 18.2237682342529 1.0
16.756995148 19.9681777954102 2.0
-6.4008279506 -6.9058518409729 3.0
10.356167188 10.8340177536011 4.0
22.796309635 23.0421161651611 5.0
33.152476832 39.4895133972168 6.0
33.152476832 38.3559875488281 7.0
37.7618147597387 36.1664962768555 0.0
20.5714887837879 19.650899887085 1.0
17.1903259800247 17.8782920837402 2.0
-5.43010176849644 -5.88047933578491 3.0
11.7602242121808 11.8923149108887 4.0
26.0015905517818 24.9457721710205 5.0
37.76181476 37.3111228942871 6.0
37.76181476 38.0984725952148 7.0
46.821650957 36.1463203430176 0.0
22.453957792 17.4700126647949 1.0
24.367693165 20.6113338470459 2.0
-10.797760543 -7.36479377746582 3.0
13.569932612 14.7609729766846 4.0
33.251718344 29.5737895965576 5.0
46.821650967 37.3573951721191 6.0
46.821650967 38.3540534973145 7.0
41.1837901754312 34.9524383544922 0.0
19.2503118354314 16.4692707061768 1.0
21.9334783494314 17.3463039398193 2.0
-7.5443023473312 -6.83179664611816 3.0
14.3891760014314 14.9632349014282 4.0
26.7946141824314 24.3297309875488 5.0
41.183790175 36.5516738891602 6.0
41.183790175 37.4594230651855 7.0
36.784988632 37.9996452331543 0.0
21.035700431 19.484842300415 1.0
15.749288201 16.2459335327148 2.0
-6.0648376105 -6.62789392471313 3.0
9.6844505818 10.1088352203369 4.0
27.100538051 27.7328853607178 5.0
36.78498864 37.6034240722656 6.0
36.78498864 37.5380592346191 7.0
37.6894949442102 39.2356758117676 0.0
17.6138134052102 17.4938850402832 1.0
20.0756815502102 22.1048755645752 2.0
-8.78806436921022 -8.25920677185059 3.0
11.2876171802102 12.2995519638062 4.0
26.401877737898 29.6799945831299 5.0
37.689494945 38.779109954834 6.0
37.689494945 38.5182495117188 7.0
41.308881088 40.0853843688965 0.0
21.148817904 18.0840511322021 1.0
20.160063185 17.7822418212891 2.0
-6.8065142938 -6.5239725112915 3.0
13.353548882 13.4922513961792 4.0
27.955332207 26.4369106292725 5.0
41.308881097 39.8244323730469 6.0
41.308881097 37.8349723815918 7.0
42.253011175 38.4190902709961 0.0
21.557015891 16.6394500732422 1.0
20.695995284 16.8388385772705 2.0
-9.3835381575 -6.6953649520874 3.0
11.312457118 10.7796087265015 4.0
30.940554058 27.3102111816406 5.0
42.253011184 38.6306419372559 6.0
42.253011184 37.4408302307129 7.0
40.047660483 37.6135749816895 0.0
19.392839472 18.5748271942139 1.0
20.654821011 21.1591815948486 2.0
-7.4809136797 -7.22601890563965 3.0
13.173907322 14.5550680160522 4.0
26.873753161 26.8438014984131 5.0
40.047660493 37.8558959960938 6.0
40.047660493 38.6015586853027 7.0
46.604605723 40.1663360595703 0.0
22.954120862 18.439811706543 1.0
23.650484861 20.3242454528809 2.0
-7.9204617962 -6.31084442138672 3.0
15.730023055 15.1422176361084 4.0
30.874582668 28.3543529510498 5.0
46.604605733 39.4593963623047 6.0
46.604605733 38.3992462158203 7.0
45.106295532 39.8661231994629 0.0
23.2126055 19.7311840057373 1.0
21.893690031 21.4793262481689 2.0
-10.542381522 -8.23345947265625 3.0
11.3513085 12.1904249191284 4.0
33.754987032 27.7502689361572 5.0
45.106295542 39.1266441345215 6.0
45.106295542 38.511116027832 7.0
38.977055089 37.6408767700195 0.0
19.684728957 19.0667247772217 1.0
19.292326132 21.0522518157959 2.0
-8.6831116468 -7.07795333862305 3.0
10.609214475 11.8523740768433 4.0
28.367840614 26.4774742126465 5.0
38.977055099 37.7126121520996 6.0
38.977055099 38.4745559692383 7.0
38.935365873 37.8397178649902 0.0
21.088579762 20.5365447998047 1.0
17.846786112 17.5018978118896 2.0
-3.2985252296 -5.49281644821167 3.0
14.548260873 14.0151863098145 4.0
24.387105001 24.2035694122314 5.0
38.935365883 38.2132110595703 6.0
38.935365883 37.9515838623047 7.0
42.74140991 38.5098495483398 0.0
20.9458217 18.3005981445312 1.0
21.795588212 19.49462890625 2.0
-6.3794891462 -7.00051498413086 3.0
15.416099058 16.6546745300293 4.0
27.325310854 26.5448570251465 5.0
42.741409918 38.7497138977051 6.0
42.741409918 37.9633445739746 7.0
31.310794567 38.7914810180664 0.0
16.880318068 19.6016578674316 1.0
14.4304765 18.3839340209961 2.0
-5.6188756586 -5.73224449157715 3.0
8.8116008333 9.96028900146484 4.0
22.499193736 23.6642322540283 5.0
31.310794576 39.1023826599121 6.0
31.310794576 37.9795036315918 7.0
};
\addlegendentry{$R^2$=0.952}
\end{axis}

\end{tikzpicture}
}}
    
    \caption{Model results using only the loss associated with nodal flow predictions in the 8-node network.}
    \label{fig:dummy_base_results}
\end{figure}



In the subsequent experiment, the losses associated with node flows and the physical equations—namely, the gas balance and the Weymouth equation—were considered. The hyperparameter optimization process resulted in the best parameters being $N channels=18$, $N layers=1$, and $N dense=5$. These settings led to a total loss of 10.270, with a node loss of 3.976, a balance loss of 4.747, and a Weymouth equation loss of 1.547.

The prediction at the nodes, shown in \cref{fig:results_nonlineal_dummy_node_base_bal_wey}, remained largely consistent with previous experiments, though there was a slight decrease in accuracy, with the $R^2$ value dropping to 0.976. This minor reduction indicates that the model continues to perform well in predicting gas injection patterns at the nodes.

However, the prediction accuracy for edge flows, as seen in \cref{fig:results_nonlineal_dummy_edge_base_bal_wey}, experienced another decline. The $R^2$ value dropped to 0.899, reflecting increased difficulties in predicting flows through the compressors and the pipeline connected to the injection field. 


\begin{figure}
    \centering
    \setlength\figurewidth{.53\textwidth}        
    \setlength\figureheight{0.36\textwidth} 
    \subfloat[Actual vs predicted nodal flows.] 
    {\label{fig:results_nonlineal_dummy_node_base_bal_wey}\resizebox{\figurewidth}{\figureheight}{% This file was created with tikzplotlib v0.10.1.
\begin{tikzpicture}

\definecolor{darkgray176}{RGB}{176,176,176}
\definecolor{lightgray204}{RGB}{204,204,204}

\begin{axis}[
colorbar,
colorbar style={ylabel={node id}},
colormap={mymap}{[1pt]
 rgb(0pt)=(0.12156862745098,0.466666666666667,0.705882352941177);
  rgb(1pt)=(1,0.498039215686275,0.0549019607843137);
  rgb(2pt)=(0.172549019607843,0.627450980392157,0.172549019607843);
  rgb(3pt)=(0.83921568627451,0.152941176470588,0.156862745098039);
  rgb(4pt)=(0.580392156862745,0.403921568627451,0.741176470588235);
  rgb(5pt)=(0.549019607843137,0.337254901960784,0.294117647058824);
  rgb(6pt)=(0.890196078431372,0.466666666666667,0.76078431372549);
  rgb(7pt)=(0.498039215686275,0.498039215686275,0.498039215686275);
  rgb(8pt)=(0.737254901960784,0.741176470588235,0.133333333333333);
  rgb(9pt)=(0.0901960784313725,0.745098039215686,0.811764705882353)
},
legend cell align={left},
legend style={
  fill opacity=0.8,
  draw opacity=1,
  text opacity=1,
  at={(0.03,0.97)},
  anchor=north west,
  draw=lightgray204
},
point meta max=7,
point meta min=0,
tick align=outside,
tick pos=left,
title={yn test-y pred},
x grid style={darkgray176},
xlabel={yn test},
xmajorgrids,
xmin=-2.43954548935, xmax=51.23045527635,
xtick style={color=black},
y grid style={darkgray176},
ylabel={y pred},
ymajorgrids,
ymin=-3.00428009033203, ymax=41.4792022705078,
ytick style={color=black}
]
\addplot [
  colormap={mymap}{[1pt]
 rgb(0pt)=(0.12156862745098,0.466666666666667,0.705882352941177);
  rgb(1pt)=(1,0.498039215686275,0.0549019607843137);
  rgb(2pt)=(0.172549019607843,0.627450980392157,0.172549019607843);
  rgb(3pt)=(0.83921568627451,0.152941176470588,0.156862745098039);
  rgb(4pt)=(0.580392156862745,0.403921568627451,0.741176470588235);
  rgb(5pt)=(0.549019607843137,0.337254901960784,0.294117647058824);
  rgb(6pt)=(0.890196078431372,0.466666666666667,0.76078431372549);
  rgb(7pt)=(0.498039215686275,0.498039215686275,0.498039215686275);
  rgb(8pt)=(0.737254901960784,0.741176470588235,0.133333333333333);
  rgb(9pt)=(0.0901960784313725,0.745098039215686,0.811764705882353)
},
  only marks,
  scatter,
  scatter src=explicit
]
table [x=x, y=y, meta=colordata]{%
x  y  colordata
39.565898645 35.1491203308105 0.0
0 0.200160503387451 1.0
0 -0.304188013076782 2.0
0 0.288060188293457 3.0
0 0.959013938903809 4.0
0 0.282236337661743 5.0
0 0.161908149719238 6.0
0 2.45821094512939 7.0
42.743257181 38.9043083190918 0.0
0 -0.870528697967529 1.0
0 -0.333846807479858 2.0
0 -0.242161989212036 3.0
0 0.81291127204895 4.0
0 0.293648242950439 5.0
0 0.107356548309326 6.0
0 1.40890288352966 7.0
39.367180751 37.5996971130371 0.0
0 -0.602354049682617 1.0
0 -0.302783250808716 2.0
0 -0.055495023727417 3.0
0 0.663342952728271 4.0
0 0.293450832366943 5.0
0 0.219899654388428 6.0
0 0.990182161331177 7.0
39.605393107 37.6227607727051 0.0
0 -0.438477039337158 1.0
0 -0.303226470947266 2.0
0 -0.321555376052856 3.0
0 0.984273910522461 4.0
0 0.353354930877686 5.0
0 0.203123092651367 6.0
0 2.41648626327515 7.0
43.937345536 37.8707847595215 0.0
0 -0.0981721878051758 1.0
0 -0.282657623291016 2.0
0 0.258069753646851 3.0
0 0.880074977874756 4.0
0 0.241923093795776 5.0
0 0.149406909942627 6.0
0 2.19368028640747 7.0
31.061989594 36.3803291320801 0.0
0 -0.385033369064331 1.0
0 -0.302149057388306 2.0
0 0.037606954574585 3.0
0 0.585165500640869 4.0
0 0.412700414657593 5.0
0 0.202289581298828 6.0
0 0.733374357223511 7.0
36.357435273 34.8341293334961 0.0
0 0.19524884223938 1.0
0 -0.241751194000244 2.0
0 0.421350717544556 3.0
0 0.825200796127319 4.0
0 0.35273289680481 5.0
0 0.218162536621094 6.0
0 1.85719728469849 7.0
38.444969617 35.0095252990723 0.0
0 0.132223844528198 1.0
0 -0.343241453170776 2.0
0 0.347116708755493 3.0
0 0.827474117279053 4.0
0 0.25757622718811 5.0
0 0.166465759277344 6.0
0 1.64691567420959 7.0
35.498620524 38.6316223144531 0.0
0 -0.694920063018799 1.0
0 -0.36021900177002 2.0
0 0.224963665008545 3.0
0 0.639010429382324 4.0
0 0.317108869552612 5.0
0 0.157165288925171 6.0
0 0.936028718948364 7.0
36.520998279 37.8719711303711 0.0
0 -0.447102069854736 1.0
0 -0.350897550582886 2.0
0 0.297503232955933 3.0
0 0.475533962249756 4.0
0 0.400114297866821 5.0
0 0.195269346237183 6.0
0 0.434977531433105 7.0
36.717272212 36.8782081604004 0.0
0 -0.312676191329956 1.0
0 -0.329649448394775 2.0
0 0.767141103744507 3.0
0 0.934754371643066 4.0
0 0.477460861206055 5.0
0 0.199023008346558 6.0
0 2.61246132850647 7.0
32.629629006 34.9397010803223 0.0
0 0.44486927986145 1.0
0 -0.311789035797119 2.0
0 -0.0227086544036865 3.0
0 0.508869647979736 4.0
0 0.387642860412598 5.0
0 0.237492322921753 6.0
0 0.51690936088562 7.0
37.75267434 35.6338195800781 0.0
0 -0.306561231613159 1.0
0 -0.268002986907959 2.0
0 0.331413745880127 3.0
0 0.409496784210205 4.0
0 0.39671802520752 5.0
0 0.276199817657471 6.0
0 0.153796195983887 7.0
38.800291347 37.7029609680176 0.0
0 -0.876469612121582 1.0
0 -0.361846446990967 2.0
0 0.333401679992676 3.0
0 0.750264406204224 4.0
0 0.194632053375244 5.0
0 0.181176424026489 6.0
0 1.44466543197632 7.0
38.252729618 35.9988670349121 0.0
0 -0.197893619537354 1.0
0 -0.311842441558838 2.0
0 0.191397190093994 3.0
0 0.85358738899231 4.0
0 0.214818239212036 5.0
0 0.164434432983398 6.0
0 1.9861273765564 7.0
43.273596047 38.2937240600586 0.0
0 -0.557488918304443 1.0
0 -0.300804853439331 2.0
0 0.599737882614136 3.0
0 0.835059881210327 4.0
0 0.286216497421265 5.0
0 0.169748067855835 6.0
0 1.900306224823 7.0
34.027431486 36.0399169921875 0.0
0 -0.403566598892212 1.0
0 -0.240749359130859 2.0
0 -0.32008957862854 3.0
0 0.576428413391113 4.0
0 0.137766599655151 5.0
0 0.174072980880737 6.0
0 0.72962498664856 7.0
41.154171391 35.4760971069336 0.0
0 0.156408786773682 1.0
0 -0.223593711853027 2.0
0 -0.0161685943603516 3.0
0 1.18227171897888 4.0
0 0.407424449920654 5.0
0 0.193575859069824 6.0
0 3.60193109512329 7.0
39.930826408 35.4365081787109 0.0
0 0.0579378604888916 1.0
0 -0.27394700050354 2.0
0 0.248442649841309 3.0
0 1.19060230255127 4.0
0 0.270642757415771 5.0
0 0.207311153411865 6.0
0 3.40486097335815 7.0
45.969703631 37.2330551147461 0.0
0 -0.288480997085571 1.0
0 -0.271238565444946 2.0
0 -0.231120824813843 3.0
0 1.05145597457886 4.0
0 0.25397801399231 5.0
0 0.125441789627075 6.0
0 2.84950256347656 7.0
38.398120221 36.79443359375 0.0
0 -0.156545639038086 1.0
0 -0.294591426849365 2.0
0 -0.329975605010986 3.0
0 1.07668590545654 4.0
0 0.12549901008606 5.0
0 0.256502628326416 6.0
0 3.29018211364746 7.0
28.366017306 37.6010513305664 0.0
0 -0.485724449157715 1.0
0 -0.263768196105957 2.0
0 -0.355943441390991 3.0
0 0.482065916061401 4.0
0 0.303758859634399 5.0
0 0.215090751647949 6.0
0 0.218419075012207 7.0
39.079818979 35.9179420471191 0.0
0 -0.150002002716064 1.0
0 -0.244218826293945 2.0
0 -0.193658351898193 3.0
0 0.620914697647095 4.0
0 0.33909273147583 5.0
0 0.166008949279785 6.0
0 0.880293846130371 7.0
40.466329383 37.9851837158203 0.0
0 -0.655858039855957 1.0
0 -0.361320734024048 2.0
0 0.229216337203979 3.0
0 0.744308471679688 4.0
0 0.453869581222534 5.0
0 0.276443719863892 6.0
0 1.52835202217102 7.0
38.293116853 35.4321365356445 0.0
0 0.0802001953125 1.0
0 -0.222990989685059 2.0
0 0.303955078125 3.0
0 0.52009916305542 4.0
0 0.223268508911133 5.0
0 0.12163257598877 6.0
0 0.374756574630737 7.0
43.346089497 35.9670143127441 0.0
0 0.170412540435791 1.0
0 -0.199920177459717 2.0
0 -0.082261323928833 3.0
0 0.929552316665649 4.0
0 0.373592853546143 5.0
0 0.160814523696899 6.0
0 2.39286422729492 7.0
34.547871154 37.6202163696289 0.0
0 -0.604975700378418 1.0
0 -0.232476711273193 2.0
0 -0.259604215621948 3.0
0 0.658791542053223 4.0
0 0.243122577667236 5.0
0 0.207667589187622 6.0
0 0.986487627029419 7.0
41.927050143 35.723934173584 0.0
0 -0.284560441970825 1.0
0 -0.279213428497314 2.0
0 -0.347092628479004 3.0
0 1.08193302154541 4.0
0 0.319298505783081 5.0
0 0.206420421600342 6.0
0 2.9623806476593 7.0
44.548236377 35.8919410705566 0.0
0 -0.0944139957427979 1.0
0 -0.250997304916382 2.0
0 -0.156850814819336 3.0
0 1.05530524253845 4.0
0 0.196887731552124 5.0
0 0.124454975128174 6.0
0 2.82484817504883 7.0
34.958415487 38.8891906738281 0.0
0 -0.686660766601562 1.0
0 -0.306885480880737 2.0
0 -0.249858379364014 3.0
0 0.722409009933472 4.0
0 0.101306915283203 5.0
0 0.156508922576904 6.0
0 1.53749775886536 7.0
41.619773298 36.1163101196289 0.0
0 0.115781545639038 1.0
0 -0.228307723999023 2.0
0 -0.115780591964722 3.0
0 0.865060567855835 4.0
0 0.136262893676758 5.0
0 0.134879589080811 6.0
0 2.00606036186218 7.0
35.768623912 35.4121627807617 0.0
0 -0.0348520278930664 1.0
0 -0.294477939605713 2.0
0 0.180348634719849 3.0
0 0.639548301696777 4.0
0 0.345688343048096 5.0
0 0.182987928390503 6.0
0 0.928183794021606 7.0
36.03587308 35.4803657531738 0.0
0 0.188814878463745 1.0
0 -0.144060373306274 2.0
0 -0.155117750167847 3.0
0 0.920083284378052 4.0
0 0.472607612609863 5.0
0 0.182946443557739 6.0
0 2.51964092254639 7.0
42.671159584 35.5550880432129 0.0
0 0.0306406021118164 1.0
0 -0.183652400970459 2.0
0 -0.399832248687744 3.0
0 1.24317002296448 4.0
0 0.390200138092041 5.0
0 0.180903673171997 6.0
0 3.80771398544312 7.0
32.270518391 36.7594604492188 0.0
0 -0.484694004058838 1.0
0 -0.19990611076355 2.0
0 -0.367530584335327 3.0
0 0.690913200378418 4.0
0 0.358729839324951 5.0
0 0.175649166107178 6.0
0 1.13808107376099 7.0
36.239834941 36.591869354248 0.0
0 0.0848326683044434 1.0
0 -0.202228784561157 2.0
0 -0.359315156936646 3.0
0 0.564240694046021 4.0
0 0.336480617523193 5.0
0 0.190918207168579 6.0
0 0.536912679672241 7.0
38.292005181 36.9821014404297 0.0
0 -0.272171497344971 1.0
0 -0.292556047439575 2.0
0 -0.220717430114746 3.0
0 0.757697582244873 4.0
0 0.146747827529907 5.0
0 0.169679164886475 6.0
0 1.51842951774597 7.0
41.142171126 36.9610900878906 0.0
0 -0.185033798217773 1.0
0 -0.277701616287231 2.0
0 -0.0819008350372314 3.0
0 0.945342779159546 4.0
0 0.211397409439087 5.0
0 0.146326541900635 6.0
0 2.54395413398743 7.0
30.660234751 36.8818702697754 0.0
0 -0.141919374465942 1.0
0 -0.341428518295288 2.0
0 -0.0679891109466553 3.0
0 0.559625864028931 4.0
0 0.355998277664185 5.0
0 0.175096273422241 6.0
0 0.699920177459717 7.0
42.776716986 35.2879104614258 0.0
0 0.284000635147095 1.0
0 -0.075392484664917 2.0
0 -0.345176696777344 3.0
0 0.760550975799561 4.0
0 0.332157373428345 5.0
0 0.147163867950439 6.0
0 1.40705394744873 7.0
39.136657955 37.7565689086914 0.0
0 -0.460649967193604 1.0
0 -0.189444780349731 2.0
0 -0.383573532104492 3.0
0 0.634093284606934 4.0
0 0.128233671188354 5.0
0 0.158735275268555 6.0
0 0.849921941757202 7.0
40.593075664 36.6438331604004 0.0
0 0.0568764209747314 1.0
0 -0.207026481628418 2.0
0 -0.295063257217407 3.0
0 0.931962728500366 4.0
0 0.17394495010376 5.0
0 0.182347536087036 6.0
0 2.25140857696533 7.0
38.455960057 37.3759269714355 0.0
0 -0.163788318634033 1.0
0 -0.134499073028564 2.0
0 -0.381940126419067 3.0
0 1.01044631004333 4.0
0 0.272303342819214 5.0
0 0.183182716369629 6.0
0 2.62317323684692 7.0
40.029822307 35.164966583252 0.0
0 0.379606962203979 1.0
0 -0.421205282211304 2.0
0 -0.21974778175354 3.0
0 0.703233480453491 4.0
0 0.346462488174438 5.0
0 0.161199331283569 6.0
0 1.20190477371216 7.0
39.721089806 35.9332504272461 0.0
0 -0.437535047531128 1.0
0 -0.249340772628784 2.0
0 -0.440981388092041 3.0
0 1.0745906829834 4.0
0 0.365663766860962 5.0
0 0.23167085647583 6.0
0 3.04691410064697 7.0
46.012802781 37.4608688354492 0.0
0 -0.395824432373047 1.0
0 -0.237329721450806 2.0
0 -0.564579486846924 3.0
0 1.0185854434967 4.0
0 0.34703254699707 5.0
0 0.17690372467041 6.0
0 2.62358450889587 7.0
43.791041158 35.7060089111328 0.0
0 0.0177044868469238 1.0
0 -0.266382455825806 2.0
0 -0.243668079376221 3.0
0 1.27132415771484 4.0
0 0.269634962081909 5.0
0 0.194462537765503 6.0
0 4.08839130401611 7.0
31.257332424 35.1832504272461 0.0
0 0.165177345275879 1.0
0 -0.206424236297607 2.0
0 -0.100756883621216 3.0
0 0.612746715545654 4.0
0 0.129507541656494 5.0
0 0.178511619567871 6.0
0 0.750536441802979 7.0
38.98847291 35.3365058898926 0.0
0 0.0526750087738037 1.0
0 -0.23676323890686 2.0
0 0.582009792327881 3.0
0 0.53883695602417 4.0
0 0.419906377792358 5.0
0 0.284507989883423 6.0
0 0.644753932952881 7.0
38.691218499 37.6206321716309 0.0
0 -0.561127185821533 1.0
0 -0.180784225463867 2.0
0 -0.30973219871521 3.0
0 0.728922367095947 4.0
0 0.273228168487549 5.0
0 0.135430097579956 6.0
0 1.38606548309326 7.0
39.033211971 35.1037139892578 0.0
0 0.0013282299041748 1.0
0 -0.252740144729614 2.0
0 0.291581869125366 3.0
0 0.909387588500977 4.0
0 0.417700290679932 5.0
0 0.218414545059204 6.0
0 2.19197678565979 7.0
37.697547813 35.0412712097168 0.0
0 -0.0570435523986816 1.0
0 -0.221124887466431 2.0
0 -0.113722562789917 3.0
0 0.532733201980591 4.0
0 0.128558158874512 5.0
0 0.14345121383667 6.0
0 0.379098653793335 7.0
35.277541339 38.5072784423828 0.0
0 -0.397107601165771 1.0
0 -0.306294918060303 2.0
0 -0.193712472915649 3.0
0 0.706493139266968 4.0
0 0.363233089447021 5.0
0 0.165807247161865 6.0
0 1.36684584617615 7.0
36.119763966 38.5935516357422 0.0
0 -0.619094848632812 1.0
0 -0.29542088508606 2.0
0 -0.184259653091431 3.0
0 0.39298415184021 4.0
0 0.387165069580078 5.0
0 0.253365516662598 6.0
0 0.186416149139404 7.0
33.14490305 37.2783050537109 0.0
0 -0.462531566619873 1.0
0 -0.361690282821655 2.0
0 0.78197169303894 3.0
0 0.559427261352539 4.0
0 0.0247523784637451 5.0
0 0.151055335998535 6.0
0 0.614703178405762 7.0
34.486800814 34.9423027038574 0.0
0 0.388437986373901 1.0
0 -0.17514181137085 2.0
0 -0.119027376174927 3.0
0 0.969211578369141 4.0
0 0.461718797683716 5.0
0 0.246531248092651 6.0
0 2.88042783737183 7.0
40.933468207 38.3784561157227 0.0
0 -0.903963565826416 1.0
0 -0.342544078826904 2.0
0 0.715445756912231 3.0
0 0.720305919647217 4.0
0 0.252207040786743 5.0
0 0.169690370559692 6.0
0 1.29144525527954 7.0
35.993417928 37.043888092041 0.0
0 -0.286978244781494 1.0
0 -0.379791736602783 2.0
0 1.05108261108398 3.0
0 0.810792207717896 4.0
0 0.118370532989502 5.0
0 0.178534269332886 6.0
0 1.90542936325073 7.0
39.331666923 35.8946533203125 0.0
0 0.0322196483612061 1.0
0 -0.248118162155151 2.0
0 -0.11001181602478 3.0
0 0.732093811035156 4.0
0 0.142559289932251 5.0
0 0.150031566619873 6.0
0 1.36782097816467 7.0
37.559548496 37.613151550293 0.0
0 -0.556365966796875 1.0
0 -0.306352376937866 2.0
0 0.0855538845062256 3.0
0 0.783942222595215 4.0
0 0.238873958587646 5.0
0 0.175072431564331 6.0
0 1.58211600780487 7.0
41.796902482 38.0704231262207 0.0
0 -0.416477680206299 1.0
0 -0.289376497268677 2.0
0 -0.111968517303467 3.0
0 1.02795648574829 4.0
0 0.23125147819519 5.0
0 0.204090595245361 6.0
0 2.68739223480225 7.0
35.679590823 35.2183876037598 0.0
0 -0.0372462272644043 1.0
0 -0.245509386062622 2.0
0 -0.189133882522583 3.0
0 0.568933248519897 4.0
0 0.242826461791992 5.0
0 0.16320013999939 6.0
0 0.645849466323853 7.0
33.227547292 34.8838500976562 0.0
0 -0.38630199432373 1.0
0 -0.225414037704468 2.0
0 -0.202451705932617 3.0
0 0.64586877822876 4.0
0 0.227593183517456 5.0
0 0.194885730743408 6.0
0 0.956827402114868 7.0
28.008071739 36.7180213928223 0.0
0 -0.290129661560059 1.0
0 -0.250248432159424 2.0
0 -0.267576694488525 3.0
0 0.31864333152771 4.0
0 0.327192783355713 5.0
0 0.177103042602539 6.0
0 0.136038303375244 7.0
33.478498841 35.8627510070801 0.0
0 -0.206622123718262 1.0
0 -0.280787706375122 2.0
0 -0.157303333282471 3.0
0 0.703498363494873 4.0
0 0.236218690872192 5.0
0 0.170774221420288 6.0
0 1.21574664115906 7.0
33.12294682 38.1380958557129 0.0
0 -0.188076972961426 1.0
0 -0.329117298126221 2.0
0 -0.277822017669678 3.0
0 0.595375537872314 4.0
0 0.442494630813599 5.0
0 0.190044403076172 6.0
0 0.776278972625732 7.0
34.970228384 35.7795677185059 0.0
0 -0.440441846847534 1.0
0 -0.304561376571655 2.0
0 0.495333194732666 3.0
0 0.619798898696899 4.0
0 0.434397459030151 5.0
0 0.205546855926514 6.0
0 0.947080373764038 7.0
36.637966041 36.4974708557129 0.0
0 -0.150226354598999 1.0
0 -0.334546804428101 2.0
0 0.429143190383911 3.0
0 0.587328910827637 4.0
0 0.243691682815552 5.0
0 0.155328273773193 6.0
0 0.730413198471069 7.0
38.712447295 35.7423439025879 0.0
0 -0.152876853942871 1.0
0 -0.287914752960205 2.0
0 0.0456929206848145 3.0
0 0.977472066879272 4.0
0 0.136602401733398 5.0
0 0.226867198944092 6.0
0 2.67127251625061 7.0
29.08402242 38.7242736816406 0.0
0 -0.728049755096436 1.0
0 -0.341446399688721 2.0
0 0.105962038040161 3.0
0 0.493180274963379 4.0
0 0.337026596069336 5.0
0 0.223398208618164 6.0
0 0.423972606658936 7.0
40.741903011 35.5043487548828 0.0
0 0.0633304119110107 1.0
0 -0.244192123413086 2.0
0 -0.256537914276123 3.0
0 0.696786642074585 4.0
0 0.169236898422241 5.0
0 0.155520677566528 6.0
0 1.06453466415405 7.0
44.423927978 34.9117660522461 0.0
0 0.17990517616272 1.0
0 -0.210843086242676 2.0
0 -0.227201700210571 3.0
0 1.03253149986267 4.0
0 0.448816537857056 5.0
0 0.285309791564941 6.0
0 2.6708927154541 7.0
37.279929011 35.5919036865234 0.0
0 0.0891716480255127 1.0
0 -0.161493062973022 2.0
0 -0.23421573638916 3.0
0 0.695326089859009 4.0
0 0.0598008632659912 5.0
0 0.17897629737854 6.0
0 1.04508233070374 7.0
39.065196566 35.4689292907715 0.0
0 -0.25178861618042 1.0
0 -0.20621919631958 2.0
0 -0.224674701690674 3.0
0 0.7693772315979 4.0
0 0.376047849655151 5.0
0 0.17762017250061 6.0
0 1.61423718929291 7.0
34.346712911 34.9799385070801 0.0
0 0.362586975097656 1.0
0 -0.200630664825439 2.0
0 0.199025630950928 3.0
0 0.494224786758423 4.0
0 0.322439908981323 5.0
0 0.16210412979126 6.0
0 0.338336944580078 7.0
44.832836202 35.5202827453613 0.0
0 0.237345933914185 1.0
0 -0.237946271896362 2.0
0 -0.00507760047912598 3.0
0 0.72044849395752 4.0
0 0.424192428588867 5.0
0 0.21662449836731 6.0
0 1.46492052078247 7.0
37.693005916 36.3368377685547 0.0
0 -0.305291891098022 1.0
0 -0.278827428817749 2.0
0 0.532672166824341 3.0
0 0.39087176322937 4.0
0 0.0712494850158691 5.0
0 0.135739088058472 6.0
0 0.330066680908203 7.0
35.516030206 36.9997940063477 0.0
0 -0.376259803771973 1.0
0 -0.298854351043701 2.0
0 0.00752091407775879 3.0
0 0.562443256378174 4.0
0 0.241495370864868 5.0
0 0.161685943603516 6.0
0 0.609134435653687 7.0
45.449957523 39.3566474914551 0.0
0 -0.661716461181641 1.0
0 -0.350741147994995 2.0
0 0.0310530662536621 3.0
0 1.31658411026001 4.0
0 0.341585874557495 5.0
0 0.161364316940308 6.0
0 3.89645385742188 7.0
36.52210397 36.8252029418945 0.0
0 -0.545600891113281 1.0
0 -0.429390907287598 2.0
0 0.119423389434814 3.0
0 0.481379270553589 4.0
0 0.347846508026123 5.0
0 0.166965961456299 6.0
0 0.329587697982788 7.0
40.809333833 37.0040626525879 0.0
0 -0.254538297653198 1.0
0 -0.325387239456177 2.0
0 -0.139763116836548 3.0
0 0.983713388442993 4.0
0 0.330280065536499 5.0
0 0.144794702529907 6.0
0 2.64484095573425 7.0
43.708815025 34.9810447692871 0.0
0 0.212815046310425 1.0
0 -0.244637250900269 2.0
0 0.121265172958374 3.0
0 0.807683229446411 4.0
0 0.137445688247681 5.0
0 0.137148380279541 6.0
0 1.59451115131378 7.0
35.746307575 35.1154365539551 0.0
0 -0.046907901763916 1.0
0 -0.385356903076172 2.0
0 0.124366998672485 3.0
0 0.823490381240845 4.0
0 0.407038450241089 5.0
0 0.184171915054321 6.0
0 1.92075765132904 7.0
32.565260407 35.1565170288086 0.0
0 -0.196492433547974 1.0
0 -0.238013982772827 2.0
0 -0.274213552474976 3.0
0 0.315068483352661 4.0
0 0.192631006240845 5.0
0 0.156498670578003 6.0
0 0.166394472122192 7.0
37.459437569 34.7231254577637 0.0
0 0.28207540512085 1.0
0 -0.150568008422852 2.0
0 -0.175907611846924 3.0
0 0.944331884384155 4.0
0 0.181029081344604 5.0
0 0.249021053314209 6.0
0 2.29156017303467 7.0
41.600868777 35.0833473205566 0.0
0 -0.0189154148101807 1.0
0 -0.265817403793335 2.0
0 0.526139259338379 3.0
0 0.610723972320557 4.0
0 0.210923910140991 5.0
0 0.122309446334839 6.0
0 0.964538812637329 7.0
43.088648289 34.9966239929199 0.0
0 0.50374698638916 1.0
0 -0.137471914291382 2.0
0 -0.287628173828125 3.0
0 1.05793118476868 4.0
0 0.463600397109985 5.0
0 0.218610763549805 6.0
0 2.88759756088257 7.0
30.268152677 36.7888145446777 0.0
0 -0.58553409576416 1.0
0 -0.317166805267334 2.0
0 0.564502954483032 3.0
0 0.504944801330566 4.0
0 0.123366117477417 5.0
0 0.166923522949219 6.0
0 0.477001428604126 7.0
38.454045888 35.7552261352539 0.0
0 -0.173763275146484 1.0
0 -0.322352886199951 2.0
0 0.404049396514893 3.0
0 0.778441429138184 4.0
0 0.400888442993164 5.0
0 0.166637659072876 6.0
0 1.80211699008942 7.0
36.654056864 35.1896514892578 0.0
0 0.103125810623169 1.0
0 -0.201706171035767 2.0
0 -0.0213396549224854 3.0
0 0.689135313034058 4.0
0 0.436973810195923 5.0
0 0.204730987548828 6.0
0 1.28305697441101 7.0
36.990278128 35.776496887207 0.0
0 0.0712330341339111 1.0
0 -0.126100301742554 2.0
0 -0.19990611076355 3.0
0 0.80268931388855 4.0
0 0.202781677246094 5.0
0 0.177822828292847 6.0
0 1.67760562896729 7.0
39.45539179 37.408935546875 0.0
0 -0.443310737609863 1.0
0 -0.216928720474243 2.0
0 -0.241320133209229 3.0
0 0.768902540206909 4.0
0 0.44538140296936 5.0
0 0.252574682235718 6.0
0 1.45734310150146 7.0
41.49009831 38.2552070617676 0.0
0 -0.505958080291748 1.0
0 -0.400657415390015 2.0
0 0.720285892486572 3.0
0 0.654844760894775 4.0
0 0.310043096542358 5.0
0 0.115434646606445 6.0
0 1.04913353919983 7.0
42.195848764 37.6286315917969 0.0
0 -0.710004806518555 1.0
0 -0.358978509902954 2.0
0 0.214832544326782 3.0
0 0.829174280166626 4.0
0 0.303926229476929 5.0
0 0.192248582839966 6.0
0 1.82904326915741 7.0
36.580423956 36.1610450744629 0.0
0 -0.100921154022217 1.0
0 -0.255610942840576 2.0
0 -0.28169059753418 3.0
0 0.764164686203003 4.0
0 0.445737600326538 5.0
0 0.209580898284912 6.0
0 1.82467806339264 7.0
41.112612846 38.3118209838867 0.0
0 -0.788333892822266 1.0
0 -0.255381107330322 2.0
0 -0.243265867233276 3.0
0 0.813346862792969 4.0
0 0.342572450637817 5.0
0 0.173837900161743 6.0
0 1.60111975669861 7.0
41.165032305 34.912410736084 0.0
0 0.0607092380523682 1.0
0 -0.191718339920044 2.0
0 -0.198416709899902 3.0
0 1.05671906471252 4.0
0 0.322285652160645 5.0
0 0.227124452590942 6.0
0 2.9236466884613 7.0
28.236329769 37.7500953674316 0.0
0 -0.734070777893066 1.0
0 -0.296064615249634 2.0
0 -0.399542093276978 3.0
0 0.51289176940918 4.0
0 0.287590980529785 5.0
0 0.192534923553467 6.0
0 0.410277605056763 7.0
46.563722235 37.4063301086426 0.0
0 -0.0448353290557861 1.0
0 -0.327138185501099 2.0
0 0.460530042648315 3.0
0 1.14788317680359 4.0
0 0.130696535110474 5.0
0 0.121592044830322 6.0
0 3.39174675941467 7.0
38.976039446 36.6878700256348 0.0
0 -0.214313268661499 1.0
0 -0.324925899505615 2.0
0 -0.219121932983398 3.0
0 0.708724021911621 4.0
0 0.240504503250122 5.0
0 0.138987302780151 6.0
0 1.28193616867065 7.0
43.902220165 36.2746086120605 0.0
0 -0.388610601425171 1.0
0 -0.23451828956604 2.0
0 -0.106697559356689 3.0
0 1.27652788162231 4.0
0 0.29651665687561 5.0
0 0.138346433639526 6.0
0 3.85741925239563 7.0
37.104870385 35.2107620239258 0.0
0 -0.173372745513916 1.0
0 -0.278142929077148 2.0
0 0.20162558555603 3.0
0 0.884715795516968 4.0
0 0.239137411117554 5.0
0 0.265449523925781 6.0
0 2.21955728530884 7.0
48.094555135 34.9823608398438 0.0
0 0.143690347671509 1.0
0 -0.357057571411133 2.0
0 -0.0929322242736816 3.0
0 1.08451628684998 4.0
0 0.368434190750122 5.0
0 0.2064049243927 6.0
0 2.90954756736755 7.0
39.863550951 35.8522415161133 0.0
0 0.100322008132935 1.0
0 -0.348156690597534 2.0
0 0.0498251914978027 3.0
0 0.8350830078125 4.0
0 0.138818979263306 5.0
0 0.182126998901367 6.0
0 1.89374268054962 7.0
38.158424706 38.0141983032227 0.0
0 -0.602644920349121 1.0
0 -0.300935029983521 2.0
0 0.250434875488281 3.0
0 1.34978842735291 4.0
0 0.368658065795898 5.0
0 0.194190740585327 6.0
0 4.74382400512695 7.0
37.671552644 36.4828720092773 0.0
0 -0.306879281997681 1.0
0 -0.292096138000488 2.0
0 -0.0983502864837646 3.0
0 0.694288730621338 4.0
0 0.231625318527222 5.0
0 0.152358531951904 6.0
0 1.22536826133728 7.0
38.99000687 35.6059837341309 0.0
0 -0.281753301620483 1.0
0 -0.241165161132812 2.0
0 -0.302188158035278 3.0
0 0.907682418823242 4.0
0 0.175835609436035 5.0
0 0.151539325714111 6.0
0 2.31387829780579 7.0
32.824279197 37.0121459960938 0.0
0 -0.400778293609619 1.0
0 -0.335283041000366 2.0
0 0.065438985824585 3.0
0 0.730565786361694 4.0
0 0.106660604476929 5.0
0 0.253888130187988 6.0
0 1.41993474960327 7.0
39.752392398 38.7344665527344 0.0
0 -0.752787590026855 1.0
0 -0.412492275238037 2.0
0 0.850913524627686 3.0
0 0.656876564025879 4.0
0 0.223501920700073 5.0
0 0.189407587051392 6.0
0 1.01953911781311 7.0
36.014997988 36.4749526977539 0.0
0 -0.238363742828369 1.0
0 -0.297706127166748 2.0
0 -0.035982608795166 3.0
0 0.702810764312744 4.0
0 0.195017099380493 5.0
0 0.230604410171509 6.0
0 1.19725680351257 7.0
42.307205606 37.5948944091797 0.0
0 -0.159608840942383 1.0
0 -0.305545091629028 2.0
0 0.0754504203796387 3.0
0 0.74741530418396 4.0
0 0.41529655456543 5.0
0 0.230007648468018 6.0
0 1.48757004737854 7.0
41.624369934 37.6179237365723 0.0
0 -0.513477802276611 1.0
0 -0.214868068695068 2.0
0 -0.121958255767822 3.0
0 0.785753488540649 4.0
0 0.426601409912109 5.0
0 0.215137720108032 6.0
0 1.76432943344116 7.0
40.927860926 34.8523788452148 0.0
0 0.521666049957275 1.0
0 -0.276309251785278 2.0
0 -0.221619844436646 3.0
0 0.962408781051636 4.0
0 0.311375856399536 5.0
0 0.184108018875122 6.0
0 2.38084173202515 7.0
47.441321812 36.9520072937012 0.0
0 -0.208132982254028 1.0
0 -0.256770133972168 2.0
0 0.238877534866333 3.0
0 0.853635549545288 4.0
0 0.404773712158203 5.0
0 0.191332578659058 6.0
0 2.07670331001282 7.0
39.379505619 35.7211532592773 0.0
0 0.0754573345184326 1.0
0 -0.249717473983765 2.0
0 -0.38543176651001 3.0
0 1.07707524299622 4.0
0 0.345310688018799 5.0
0 0.195640802383423 6.0
0 2.90627312660217 7.0
42.095557299 35.4768409729004 0.0
0 0.051783561706543 1.0
0 -0.286845445632935 2.0
0 -0.0503053665161133 3.0
0 1.16104769706726 4.0
0 0.33863353729248 5.0
0 0.190688371658325 6.0
0 3.07928395271301 7.0
35.151188225 37.2888259887695 0.0
0 -0.475473403930664 1.0
0 -0.244946479797363 2.0
0 -0.408578872680664 3.0
0 0.401235580444336 4.0
0 0.400536775588989 5.0
0 0.220471858978271 6.0
0 0.189782619476318 7.0
37.713935371 37.5988998413086 0.0
0 -0.796328544616699 1.0
0 -0.397212743759155 2.0
0 0.225178718566895 3.0
0 0.924602031707764 4.0
0 0.209428787231445 5.0
0 0.161656379699707 6.0
0 2.27943110466003 7.0
39.66774326 37.0398445129395 0.0
0 -0.222144603729248 1.0
0 -0.302505493164062 2.0
0 -0.330070018768311 3.0
0 0.815786600112915 4.0
0 0.408083915710449 5.0
0 0.182557582855225 6.0
0 1.6671314239502 7.0
41.240556194 36.6697692871094 0.0
0 -0.351570129394531 1.0
0 -0.24855899810791 2.0
0 -0.33103609085083 3.0
0 1.10477876663208 4.0
0 0.278905868530273 5.0
0 0.199224472045898 6.0
0 2.84470868110657 7.0
40.3201946 39.2217102050781 0.0
0 -0.722808361053467 1.0
0 -0.248057842254639 2.0
0 -0.315093994140625 3.0
0 0.96550989151001 4.0
0 -0.0229625701904297 5.0
0 0.144112825393677 6.0
0 2.48224353790283 7.0
39.561151059 38.2839698791504 0.0
0 -0.7145676612854 1.0
0 -0.306675672531128 2.0
0 -0.597090721130371 3.0
0 0.716639518737793 4.0
0 0.329735994338989 5.0
0 0.191384553909302 6.0
0 1.26927495002747 7.0
43.566431947 34.9794578552246 0.0
0 0.118311166763306 1.0
0 -0.283224582672119 2.0
0 -0.222985029220581 3.0
0 0.943839311599731 4.0
0 0.285821437835693 5.0
0 0.160087108612061 6.0
0 2.21693706512451 7.0
37.929366562 36.9765548706055 0.0
0 -0.405695676803589 1.0
0 -0.313417434692383 2.0
0 0.0274081230163574 3.0
0 0.708983659744263 4.0
0 0.158509254455566 5.0
0 0.157469034194946 6.0
0 1.19865441322327 7.0
39.745668641 36.3871574401855 0.0
0 -0.814199924468994 1.0
0 -0.294702529907227 2.0
0 0.307629585266113 3.0
0 0.542653560638428 4.0
0 0.144030094146729 5.0
0 0.119307518005371 6.0
0 0.432701349258423 7.0
37.173794625 36.8767852783203 0.0
0 -0.132876396179199 1.0
0 -0.270767211914062 2.0
0 0.373361349105835 3.0
0 0.675598621368408 4.0
0 0.388581275939941 5.0
0 0.210007667541504 6.0
0 1.11788201332092 7.0
27.124866015 36.8166084289551 0.0
0 -0.252533435821533 1.0
0 -0.251078128814697 2.0
0 0.123565912246704 3.0
0 0.230942964553833 4.0
0 0.361216068267822 5.0
0 0.230597734451294 6.0
0 0.0980198383331299 7.0
27.411746904 35.2586708068848 0.0
0 0.104533433914185 1.0
0 -0.301069021224976 2.0
0 -0.200390338897705 3.0
0 0.200777530670166 4.0
0 0.502002000808716 5.0
0 0.205982208251953 6.0
0 0.133601427078247 7.0
38.866617317 36.5648536682129 0.0
0 -0.0915412902832031 1.0
0 -0.310113191604614 2.0
0 0.229114532470703 3.0
0 0.806035280227661 4.0
0 0.302664041519165 5.0
0 0.194485187530518 6.0
0 1.81620466709137 7.0
42.383057354 37.3433036804199 0.0
0 -0.0384202003479004 1.0
0 -0.281975030899048 2.0
0 -0.276898860931396 3.0
0 0.921064853668213 4.0
0 0.131135702133179 5.0
0 0.157546281814575 6.0
0 2.13922548294067 7.0
47.643751036 35.5366554260254 0.0
0 0.212430000305176 1.0
0 -0.209916114807129 2.0
0 -0.539245128631592 3.0
0 1.163170337677 4.0
0 0.429344415664673 5.0
0 0.22515344619751 6.0
0 3.03847599029541 7.0
38.439399957 36.8292465209961 0.0
0 -0.46328592300415 1.0
0 -0.288815021514893 2.0
0 -0.0641045570373535 3.0
0 0.717837810516357 4.0
0 0.0535271167755127 5.0
0 0.151680469512939 6.0
0 1.2175030708313 7.0
40.263562371 37.0849456787109 0.0
0 -0.403761625289917 1.0
0 -0.295222043991089 2.0
0 -0.200614929199219 3.0
0 0.816816091537476 4.0
0 0.32013988494873 5.0
0 0.177034139633179 6.0
0 1.86587464809418 7.0
41.519528397 37.8470573425293 0.0
0 -0.389542579650879 1.0
0 -0.314706325531006 2.0
0 0.0711114406585693 3.0
0 0.746595621109009 4.0
0 0.239205837249756 5.0
0 0.15038275718689 6.0
0 1.39596176147461 7.0
40.414570474 35.6151313781738 0.0
0 -0.117020130157471 1.0
0 -0.290426731109619 2.0
0 -0.00904703140258789 3.0
0 0.740937232971191 4.0
0 0.384721517562866 5.0
0 0.150301456451416 6.0
0 1.44771862030029 7.0
37.834181221 36.8035469055176 0.0
0 -0.460850238800049 1.0
0 -0.311007022857666 2.0
0 -0.0458815097808838 3.0
0 0.669914484024048 4.0
0 0.265458106994629 5.0
0 0.16695499420166 6.0
0 1.08158111572266 7.0
37.201121556 36.8868522644043 0.0
0 -0.422034502029419 1.0
0 -0.257261037826538 2.0
0 0.00199723243713379 3.0
0 0.587789535522461 4.0
0 0.463330268859863 5.0
0 0.241168737411499 6.0
0 0.701586961746216 7.0
42.94444514 36.9254188537598 0.0
0 -0.545905113220215 1.0
0 -0.303776264190674 2.0
0 0.136491298675537 3.0
0 0.992483377456665 4.0
0 0.325592279434204 5.0
0 0.163835048675537 6.0
0 2.69584798812866 7.0
47.309600034 38.0835418701172 0.0
0 -0.545759201049805 1.0
0 -0.294164657592773 2.0
0 0.088735818862915 3.0
0 1.31948041915894 4.0
0 0.324820041656494 5.0
0 0.135318040847778 6.0
0 4.02141046524048 7.0
34.703660747 37.6689910888672 0.0
0 -0.325380325317383 1.0
0 -0.325445175170898 2.0
0 0.589932441711426 3.0
0 0.41747522354126 4.0
0 0.24850869178772 5.0
0 0.16209888458252 6.0
0 0.176469087600708 7.0
39.355679833 35.1904449462891 0.0
0 0.378657817840576 1.0
0 -0.203660249710083 2.0
0 -0.304840564727783 3.0
0 0.736403703689575 4.0
0 0.338710784912109 5.0
0 0.19580340385437 6.0
0 1.38337755203247 7.0
36.760838135 36.4086456298828 0.0
0 -0.193023920059204 1.0
0 -0.239799737930298 2.0
0 -0.04498291015625 3.0
0 0.646554946899414 4.0
0 0.305450439453125 5.0
0 0.148752212524414 6.0
0 0.92288875579834 7.0
45.817767369 35.1043701171875 0.0
0 -0.200600862503052 1.0
0 -0.288094043731689 2.0
0 0.233930349349976 3.0
0 1.12476301193237 4.0
0 0.402995347976685 5.0
0 0.154536247253418 6.0
0 3.33745384216309 7.0
37.456387944 36.4718322753906 0.0
0 -0.528736591339111 1.0
0 -0.230040073394775 2.0
0 -0.464707851409912 3.0
0 0.894877433776855 4.0
0 0.286540746688843 5.0
0 0.166498899459839 6.0
0 1.79093253612518 7.0
36.979299113 37.454963684082 0.0
0 -0.490121364593506 1.0
0 -0.264573574066162 2.0
0 0.0660631656646729 3.0
0 0.468281984329224 4.0
0 0.284716367721558 5.0
0 0.134176969528198 6.0
0 0.255043506622314 7.0
44.357392157 37.8662147521973 0.0
0 -0.261130332946777 1.0
0 -0.286425828933716 2.0
0 -0.163735389709473 3.0
0 0.844649791717529 4.0
0 0.355852127075195 5.0
0 0.19456934928894 6.0
0 1.9478999376297 7.0
36.812079299 37.6048698425293 0.0
0 -0.341729402542114 1.0
0 -0.160489797592163 2.0
0 -0.349940061569214 3.0
0 0.92185640335083 4.0
0 0.238054513931274 5.0
0 0.197580337524414 6.0
0 2.25563478469849 7.0
43.172254743 36.6764678955078 0.0
0 -0.340490579605103 1.0
0 -0.340424776077271 2.0
0 -0.24942684173584 3.0
0 0.682604789733887 4.0
0 0.444791078567505 5.0
0 0.300320625305176 6.0
0 1.08312892913818 7.0
47.109370567 36.7499923706055 0.0
0 -0.34661602973938 1.0
0 -0.295565843582153 2.0
0 0.425445079803467 3.0
0 1.10153651237488 4.0
0 0.0241773128509521 5.0
0 0.147600889205933 6.0
0 2.95037508010864 7.0
35.871355662 34.9815406799316 0.0
0 0.494839191436768 1.0
0 -0.345178604125977 2.0
0 -0.258123874664307 3.0
0 0.939872980117798 4.0
0 0.450568199157715 5.0
0 0.180020809173584 6.0
0 2.65937280654907 7.0
42.169433911 35.331184387207 0.0
0 -0.183138132095337 1.0
0 -0.30778980255127 2.0
0 0.462181568145752 3.0
0 1.0327045917511 4.0
0 0.237351894378662 5.0
0 0.163395881652832 6.0
0 2.85370659828186 7.0
38.001249401 36.3065605163574 0.0
0 -0.331825971603394 1.0
0 -0.22763729095459 2.0
0 -0.402097225189209 3.0
0 0.858330488204956 4.0
0 0.335723400115967 5.0
0 0.22048020362854 6.0
0 1.81269466876984 7.0
36.026296153 36.1182632446289 0.0
0 0.025681734085083 1.0
0 -0.348989486694336 2.0
0 0.548575639724731 3.0
0 0.865596771240234 4.0
0 0.0814034938812256 5.0
0 0.210360050201416 6.0
0 2.24220395088196 7.0
33.200213159 36.0681076049805 0.0
0 -0.333817958831787 1.0
0 -0.355710029602051 2.0
0 0.266530990600586 3.0
0 0.531875371932983 4.0
0 0.15729284286499 5.0
0 0.180988311767578 6.0
0 0.484956502914429 7.0
36.881947952 38.2229385375977 0.0
0 -0.696336269378662 1.0
0 -0.336338043212891 2.0
0 0.0925045013427734 3.0
0 0.657312631607056 4.0
0 0.403677225112915 5.0
0 0.218650579452515 6.0
0 0.969262361526489 7.0
28.363012012 38.3906898498535 0.0
0 -0.627444744110107 1.0
0 -0.267569780349731 2.0
0 -0.263440847396851 3.0
0 0.565419912338257 4.0
0 0.268229484558105 5.0
0 0.177015781402588 6.0
0 0.691333770751953 7.0
39.574467426 37.640495300293 0.0
0 -0.507296085357666 1.0
0 -0.326681613922119 2.0
0 0.420464992523193 3.0
0 0.992634773254395 4.0
0 0.148752212524414 5.0
0 0.169144868850708 6.0
0 2.65010786056519 7.0
36.22040148 38.0059471130371 0.0
0 -0.546613216400146 1.0
0 -0.315471887588501 2.0
0 -0.129685878753662 3.0
0 0.651448011398315 4.0
0 0.229004859924316 5.0
0 0.171168327331543 6.0
0 1.0278103351593 7.0
31.086432141 37.8759880065918 0.0
0 -0.576459884643555 1.0
0 -0.364636182785034 2.0
0 -0.190219402313232 3.0
0 0.544191360473633 4.0
0 0.0777196884155273 5.0
0 0.180845499038696 6.0
0 0.523284912109375 7.0
36.274353063 34.7889709472656 0.0
0 -0.0258965492248535 1.0
0 -0.270874261856079 2.0
0 -0.0976779460906982 3.0
0 0.466679573059082 4.0
0 0.236052274703979 5.0
0 0.149326086044312 6.0
0 0.227711915969849 7.0
31.65795009 35.8161735534668 0.0
0 -0.194855213165283 1.0
0 -0.232226610183716 2.0
0 -0.232043266296387 3.0
0 0.602954864501953 4.0
0 0.422707080841064 5.0
0 0.18997597694397 6.0
0 0.706630706787109 7.0
37.059171479 36.8880996704102 0.0
0 -0.399651288986206 1.0
0 -0.287773609161377 2.0
0 -0.299047470092773 3.0
0 1.09156084060669 4.0
0 0.461916208267212 5.0
0 0.198938131332397 6.0
0 2.92316722869873 7.0
45.308862357 36.8652992248535 0.0
0 -0.309981107711792 1.0
0 -0.200457572937012 2.0
0 0.0459420680999756 3.0
0 0.986061334609985 4.0
0 0.132490158081055 5.0
0 0.143405914306641 6.0
0 2.56991314888 7.0
32.988279909 36.210765838623 0.0
0 0.17202091217041 1.0
0 -0.272132158279419 2.0
0 0.233931303024292 3.0
0 0.502268552780151 4.0
0 0.314094305038452 5.0
0 0.164807796478271 6.0
0 0.436026334762573 7.0
41.812200593 36.2599906921387 0.0
0 -0.0709729194641113 1.0
0 -0.275076866149902 2.0
0 -0.0572128295898438 3.0
0 0.883049488067627 4.0
0 0.418354034423828 5.0
0 0.218208074569702 6.0
0 1.95886766910553 7.0
34.159578007 35.0773010253906 0.0
0 0.137953758239746 1.0
0 -0.289844036102295 2.0
0 0.36235237121582 3.0
0 0.382478713989258 4.0
0 -0.00413036346435547 5.0
0 0.152661561965942 6.0
0 0.165405988693237 7.0
41.353058204 36.1138916015625 0.0
0 -0.355628252029419 1.0
0 -0.329483032226562 2.0
0 0.933395385742188 3.0
0 1.01461863517761 4.0
0 0.45141863822937 5.0
0 0.201395750045776 6.0
0 2.87705135345459 7.0
35.663236644 35.6828575134277 0.0
0 -0.132169485092163 1.0
0 -0.242761135101318 2.0
0 -0.0993001461029053 3.0
0 0.711432456970215 4.0
0 0.184571027755737 5.0
0 0.235892534255981 6.0
0 1.25230884552002 7.0
40.444916123 35.2093963623047 0.0
0 -0.243934869766235 1.0
0 -0.262676954269409 2.0
0 0.290029525756836 3.0
0 0.979154348373413 4.0
0 0.473401069641113 5.0
0 0.180022001266479 6.0
0 2.3899233341217 7.0
39.900416914 37.5763206481934 0.0
0 -0.368428707122803 1.0
0 -0.126080751419067 2.0
0 -0.389459848403931 3.0
0 0.767287492752075 4.0
0 0.131878137588501 5.0
0 0.150093555450439 6.0
0 1.34332084655762 7.0
34.405324849 38.7121810913086 0.0
0 -0.566517353057861 1.0
0 -0.315571308135986 2.0
0 -0.060286283493042 3.0
0 0.794055938720703 4.0
0 0.357685565948486 5.0
0 0.175518989562988 6.0
0 1.81607222557068 7.0
33.074256004 37.8552093505859 0.0
0 -0.450038433074951 1.0
0 -0.323059320449829 2.0
0 -0.230019330978394 3.0
0 0.575784921646118 4.0
0 0.32622766494751 5.0
0 0.181130886077881 6.0
0 0.620117902755737 7.0
40.036170308 34.7537536621094 0.0
0 0.539894580841064 1.0
0 -0.123880624771118 2.0
0 -0.29494833946228 3.0
0 0.799856424331665 4.0
0 0.303247213363647 5.0
0 0.13506817817688 6.0
0 1.66720008850098 7.0
44.453206241 36.1504745483398 0.0
0 -0.25816798210144 1.0
0 -0.289645433425903 2.0
0 -0.295749664306641 3.0
0 1.2394585609436 4.0
0 0.25753927230835 5.0
0 0.12170934677124 6.0
0 3.38232684135437 7.0
33.85004541 36.2625312805176 0.0
0 0.189345121383667 1.0
0 -0.171904802322388 2.0
0 -0.171475648880005 3.0
0 0.546658992767334 4.0
0 0.390380144119263 5.0
0 0.219710826873779 6.0
0 0.629614353179932 7.0
30.537877308 35.2069931030273 0.0
0 -0.0425846576690674 1.0
0 -0.133653163909912 2.0
0 -0.269747018814087 3.0
0 0.476840496063232 4.0
0 0.361154794692993 5.0
0 0.245875358581543 6.0
0 0.306348562240601 7.0
27.727440942 35.6123542785645 0.0
0 0.0466382503509521 1.0
0 -0.177139520645142 2.0
0 -0.0733387470245361 3.0
0 0.364983797073364 4.0
0 0.00558614730834961 5.0
0 0.241782188415527 6.0
0 0.150818347930908 7.0
35.630182897 37.6505317687988 0.0
0 -0.514773845672607 1.0
0 -0.34829044342041 2.0
0 0.293203353881836 3.0
0 0.617482662200928 4.0
0 0.174357175827026 5.0
0 0.154289484024048 6.0
0 0.741385936737061 7.0
31.038251519 35.0458526611328 0.0
0 0.0954825878143311 1.0
0 -0.272761821746826 2.0
0 -0.126400709152222 3.0
0 0.607957363128662 4.0
0 0.250170707702637 5.0
0 0.200007200241089 6.0
0 0.766695022583008 7.0
31.867011967 35.225830078125 0.0
0 -0.159167528152466 1.0
0 -0.279109239578247 2.0
0 -0.118392467498779 3.0
0 0.63050651550293 4.0
0 0.12956714630127 5.0
0 0.164574146270752 6.0
0 0.985862970352173 7.0
30.754831646 38.0706024169922 0.0
0 -0.586206436157227 1.0
0 -0.306769609451294 2.0
0 0.391008377075195 3.0
0 0.455632925033569 4.0
0 0.0682272911071777 5.0
0 0.154810190200806 6.0
0 0.269750595092773 7.0
38.95156428 39.03662109375 0.0
0 -0.971930980682373 1.0
0 -0.36164402961731 2.0
0 0.0525999069213867 3.0
0 1.00799798965454 4.0
0 0.34689736366272 5.0
0 0.205167055130005 6.0
0 2.69997143745422 7.0
33.384480936 35.5141372680664 0.0
0 0.207480669021606 1.0
0 -0.170225381851196 2.0
0 -0.109003067016602 3.0
0 0.443009853363037 4.0
0 0.354331493377686 5.0
0 0.213359117507935 6.0
0 0.198521137237549 7.0
37.291199282 37.6686515808105 0.0
0 -0.451270580291748 1.0
0 -0.21763014793396 2.0
0 -0.437268018722534 3.0
0 0.608809471130371 4.0
0 0.184311628341675 5.0
0 0.131281614303589 6.0
0 0.686292886734009 7.0
33.571811016 37.7092895507812 0.0
0 -0.528919696807861 1.0
0 -0.315627098083496 2.0
0 -0.262389659881592 3.0
0 0.554710626602173 4.0
0 0.223739862442017 5.0
0 0.163677215576172 6.0
0 0.540818691253662 7.0
44.048761746 37.5322761535645 0.0
0 -0.328079223632812 1.0
0 -0.20240592956543 2.0
0 0.0259914398193359 3.0
0 0.783541679382324 4.0
0 0.196909189224243 5.0
0 0.10933256149292 6.0
0 1.53054738044739 7.0
38.09281546 37.4579734802246 0.0
0 -0.318696975708008 1.0
0 -0.291354894638062 2.0
0 -0.205773591995239 3.0
0 0.748088121414185 4.0
0 0.32144021987915 5.0
0 0.170384645462036 6.0
0 1.44286680221558 7.0
42.616606924 38.4393501281738 0.0
0 -0.756385803222656 1.0
0 -0.257825613021851 2.0
0 -0.0478272438049316 3.0
0 1.10778093338013 4.0
0 0.21465539932251 5.0
0 0.193571805953979 6.0
0 3.33178615570068 7.0
41.080227074 36.5274314880371 0.0
0 -0.384939432144165 1.0
0 -0.310402393341064 2.0
0 0.0411951541900635 3.0
0 0.971628665924072 4.0
0 0.184604167938232 5.0
0 0.164141654968262 6.0
0 2.38429403305054 7.0
36.859087813 36.3649635314941 0.0
0 0.0918116569519043 1.0
0 -0.415070772171021 2.0
0 -0.185579538345337 3.0
0 0.76504111289978 4.0
0 0.140713214874268 5.0
0 0.154449462890625 6.0
0 1.63658976554871 7.0
39.316268876 36.9958000183105 0.0
0 -0.317481517791748 1.0
0 -0.321711778640747 2.0
0 -0.128554821014404 3.0
0 0.777884960174561 4.0
0 0.318092584609985 5.0
0 0.160611867904663 6.0
0 1.52292084693909 7.0
40.023071967 36.770923614502 0.0
0 -0.328018188476562 1.0
0 -0.368217945098877 2.0
0 0.540510892868042 3.0
0 0.818593978881836 4.0
0 0.113914489746094 5.0
0 0.171317100524902 6.0
0 1.85675239562988 7.0
32.691039566 36.2088241577148 0.0
0 -0.440219163894653 1.0
0 -0.246803998947144 2.0
0 -0.113319873809814 3.0
0 0.626848459243774 4.0
0 0.238831043243408 5.0
0 0.198577642440796 6.0
0 0.793100118637085 7.0
39.35017223 36.5497741699219 0.0
0 -0.149323463439941 1.0
0 -0.271410226821899 2.0
0 0.206826448440552 3.0
0 0.818848848342896 4.0
0 0.286503076553345 5.0
0 0.160560607910156 6.0
0 1.89805805683136 7.0
38.62577922 35.3755416870117 0.0
0 -0.133654356002808 1.0
0 -0.248138904571533 2.0
0 -0.190851449966431 3.0
0 0.872822999954224 4.0
0 0.324658155441284 5.0
0 0.199256181716919 6.0
0 1.92160940170288 7.0
43.36321018 35.9823341369629 0.0
0 0.0563368797302246 1.0
0 -0.24370551109314 2.0
0 0.257123231887817 3.0
0 1.02105641365051 4.0
0 0.260989665985107 5.0
0 0.177548408508301 6.0
0 2.79696083068848 7.0
41.711370519 35.5522651672363 0.0
0 -0.395969390869141 1.0
0 -0.278321981430054 2.0
0 0.231859922409058 3.0
0 0.621681690216064 4.0
0 0.295926809310913 5.0
0 0.20905876159668 6.0
0 0.84857702255249 7.0
41.401748585 37.302562713623 0.0
0 -0.414661645889282 1.0
0 -0.280993461608887 2.0
0 0.0715169906616211 3.0
0 0.879554033279419 4.0
0 0.387385845184326 5.0
0 0.1721031665802 6.0
0 2.04483270645142 7.0
45.411303661 34.7420082092285 0.0
0 0.211059808731079 1.0
0 -0.247791528701782 2.0
0 0.474770784378052 3.0
0 0.816185235977173 4.0
0 0.277834177017212 5.0
0 0.144842624664307 6.0
0 1.81610405445099 7.0
46.173622738 35.5511474609375 0.0
0 0.0269923210144043 1.0
0 -0.228210687637329 2.0
0 -0.638853073120117 3.0
0 1.10494565963745 4.0
0 0.361497640609741 5.0
0 0.146261692047119 6.0
0 2.97052049636841 7.0
43.970500892 37.3365020751953 0.0
0 -0.441231250762939 1.0
0 -0.283998489379883 2.0
0 -0.305295467376709 3.0
0 1.28334450721741 4.0
0 0.218288660049438 5.0
0 0.177127122879028 6.0
0 4.22349214553833 7.0
41.561964524 37.4076156616211 0.0
0 -0.224158048629761 1.0
0 -0.280381917953491 2.0
0 -0.023749828338623 3.0
0 0.511963605880737 4.0
0 0.342110872268677 5.0
0 0.131027221679688 6.0
0 0.631897211074829 7.0
42.043570834 34.8548240661621 0.0
0 0.0023043155670166 1.0
0 -0.245981216430664 2.0
0 -0.235840559005737 3.0
0 0.927974462509155 4.0
0 0.31881856918335 5.0
0 0.186933040618896 6.0
0 2.3052659034729 7.0
42.728187908 38.3169288635254 0.0
0 -0.653934478759766 1.0
0 -0.30577278137207 2.0
0 -0.232869386672974 3.0
0 1.0098774433136 4.0
0 0.306349754333496 5.0
0 0.194292306900024 6.0
0 2.47166538238525 7.0
46.647822945 38.3712120056152 0.0
0 -0.822995662689209 1.0
0 -0.344745874404907 2.0
0 0.769909381866455 3.0
0 1.14417910575867 4.0
0 -0.00493645668029785 5.0
0 0.14873194694519 6.0
0 3.3382260799408 7.0
34.157298462 37.4573173522949 0.0
0 -0.492515563964844 1.0
0 -0.342912673950195 2.0
0 0.106889486312866 3.0
0 0.825028657913208 4.0
0 0.215859174728394 5.0
0 0.203790426254272 6.0
0 1.82364273071289 7.0
38.08940786 35.419864654541 0.0
0 -0.393897533416748 1.0
0 -0.222334623336792 2.0
0 -0.415732383728027 3.0
0 0.58973240852356 4.0
0 0.403668403625488 5.0
0 0.309327602386475 6.0
0 0.6341552734375 7.0
48.253456605 35.0853805541992 0.0
0 0.333340883255005 1.0
0 -0.300779581069946 2.0
0 0.44981861114502 3.0
0 0.946572542190552 4.0
0 0.307107925415039 5.0
0 0.168227434158325 6.0
0 2.67914366722107 7.0
41.303512955 39.3003921508789 0.0
0 -0.717178821563721 1.0
0 -0.283269166946411 2.0
0 -0.227101802825928 3.0
0 0.885115146636963 4.0
0 0.0205042362213135 5.0
0 0.201066017150879 6.0
0 1.92747831344604 7.0
34.11247552 34.5782051086426 0.0
0 0.218904733657837 1.0
0 -0.216293096542358 2.0
0 -0.278970003128052 3.0
0 0.432380437850952 4.0
0 0.346678256988525 5.0
0 0.16283130645752 6.0
0 0.207588672637939 7.0
39.346899321 35.5621070861816 0.0
0 0.123582363128662 1.0
0 -0.18756628036499 2.0
0 -0.270231962203979 3.0
0 1.18870663642883 4.0
0 0.222780704498291 5.0
0 0.240122079849243 6.0
0 3.30634570121765 7.0
42.355081321 38.2932510375977 0.0
0 -0.513999938964844 1.0
0 -0.231007814407349 2.0
0 -0.391440629959106 3.0
0 1.14746689796448 4.0
0 0.315509796142578 5.0
0 0.133997440338135 6.0
0 3.2693567276001 7.0
46.025284619 38.1289863586426 0.0
0 -0.387181520462036 1.0
0 -0.244279623031616 2.0
0 -0.0663731098175049 3.0
0 0.85203742980957 4.0
0 0.187544822692871 5.0
0 0.162124872207642 6.0
0 1.88655591011047 7.0
48.790909787 37.5109252929688 0.0
0 -0.38151478767395 1.0
0 -0.336330890655518 2.0
0 0.663587808609009 3.0
0 1.28249430656433 4.0
0 0.218399047851562 5.0
0 0.141432523727417 6.0
0 3.98952007293701 7.0
41.698940502 38.2692832946777 0.0
0 -0.60564661026001 1.0
0 -0.323164224624634 2.0
0 0.18956184387207 3.0
0 1.01653003692627 4.0
0 0.438361167907715 5.0
0 0.190719604492188 6.0
0 2.59361124038696 7.0
47.902972781 36.9591064453125 0.0
0 -0.459914207458496 1.0
0 -0.240079641342163 2.0
0 0.084275484085083 3.0
0 1.13426208496094 4.0
0 0.142693519592285 5.0
0 0.124828577041626 6.0
0 3.35500884056091 7.0
41.37717087 37.5543746948242 0.0
0 -0.543313980102539 1.0
0 -0.287878274917603 2.0
0 0.386805057525635 3.0
0 0.830328941345215 4.0
0 0.339674472808838 5.0
0 0.159838199615479 6.0
0 1.88572716712952 7.0
32.285684902 35.1629447937012 0.0
0 0.0787146091461182 1.0
0 -0.2599036693573 2.0
0 0.196555376052856 3.0
0 0.508222341537476 4.0
0 0.0450396537780762 5.0
0 0.144816160202026 6.0
0 0.358730792999268 7.0
32.344206851 36.8033828735352 0.0
0 -0.0662655830383301 1.0
0 -0.305146455764771 2.0
0 -0.00590777397155762 3.0
0 0.437476396560669 4.0
0 0.320266246795654 5.0
0 0.209136247634888 6.0
0 0.366043090820312 7.0
34.184112331 35.3496627807617 0.0
0 0.142817974090576 1.0
0 -0.246417045593262 2.0
0 -0.210520267486572 3.0
0 0.646750688552856 4.0
0 0.171846151351929 5.0
0 0.230769872665405 6.0
0 0.894271373748779 7.0
34.663221808 39.3657379150391 0.0
0 -0.841586589813232 1.0
0 -0.333966016769409 2.0
0 -0.153372287750244 3.0
0 0.773782014846802 4.0
0 0.175549030303955 5.0
0 0.167372226715088 6.0
0 1.57524919509888 7.0
36.873774543 37.0048332214355 0.0
0 -0.58704662322998 1.0
0 -0.333586692810059 2.0
0 0.629967451095581 3.0
0 0.598467826843262 4.0
0 0.385325193405151 5.0
0 0.226358413696289 6.0
0 0.810904502868652 7.0
40.211352908 35.3646125793457 0.0
0 0.233994722366333 1.0
0 -0.155796766281128 2.0
0 -0.444745063781738 3.0
0 1.21115374565125 4.0
0 0.242395877838135 5.0
0 0.216838121414185 6.0
0 3.71796751022339 7.0
46.160689867 36.7150230407715 0.0
0 -0.069237232208252 1.0
0 -0.271033525466919 2.0
0 -0.0318198204040527 3.0
0 0.981415271759033 4.0
0 0.342847585678101 5.0
0 0.225942850112915 6.0
0 2.60293364524841 7.0
32.727702676 36.5545806884766 0.0
0 -0.49364709854126 1.0
0 -0.244989633560181 2.0
0 -0.328050851821899 3.0
0 0.819921255111694 4.0
0 0.324122667312622 5.0
0 0.18805193901062 6.0
0 1.82244646549225 7.0
40.640985417 36.4715461730957 0.0
0 -0.198003768920898 1.0
0 -0.262907028198242 2.0
0 0.611025810241699 3.0
0 0.831077337265015 4.0
0 0.340401411056519 5.0
0 0.165308475494385 6.0
0 1.8631626367569 7.0
34.611451594 37.8915824890137 0.0
0 -0.474421977996826 1.0
0 -0.339117765426636 2.0
0 -0.0412516593933105 3.0
0 0.454239845275879 4.0
0 0.310339689254761 5.0
0 0.16126537322998 6.0
0 0.292734146118164 7.0
43.311280807 37.878059387207 0.0
0 -0.452921390533447 1.0
0 -0.335891008377075 2.0
0 0.00657010078430176 3.0
0 0.685596704483032 4.0
0 0.412175178527832 5.0
0 0.243127346038818 6.0
0 1.19819235801697 7.0
36.406364057 35.2553672790527 0.0
0 0.247391223907471 1.0
0 -0.202195644378662 2.0
0 -0.345523834228516 3.0
0 0.716246843338013 4.0
0 0.320581197738647 5.0
0 0.186306238174438 6.0
0 1.16245818138123 7.0
31.685630326 37.3143463134766 0.0
0 -0.643002986907959 1.0
0 -0.294859647750854 2.0
0 -0.444401264190674 3.0
0 0.810844421386719 4.0
0 0.23905611038208 5.0
0 0.279279708862305 6.0
0 1.52335023880005 7.0
40.165364432 38.0170021057129 0.0
0 -0.645548820495605 1.0
0 -0.259350538253784 2.0
0 -0.0538983345031738 3.0
0 1.26421332359314 4.0
0 0.107765197753906 5.0
0 0.237394332885742 6.0
0 3.67189884185791 7.0
42.886465647 36.4526443481445 0.0
0 -0.160876035690308 1.0
0 -0.269818544387817 2.0
0 0.053027868270874 3.0
0 1.01626896858215 4.0
0 0.295691013336182 5.0
0 0.183152198791504 6.0
0 2.69845581054688 7.0
46.942592332 35.7676239013672 0.0
0 0.0879402160644531 1.0
0 -0.123027086257935 2.0
0 -0.379568338394165 3.0
0 0.996989727020264 4.0
0 0.167981624603271 5.0
0 0.13997483253479 6.0
0 2.45638537406921 7.0
38.417030568 36.7153244018555 0.0
0 -0.473661422729492 1.0
0 -0.28259801864624 2.0
0 -0.156463861465454 3.0
0 0.911740303039551 4.0
0 0.328292846679688 5.0
0 0.234901666641235 6.0
0 2.07191801071167 7.0
39.82159543 37.3864364624023 0.0
0 -0.717067241668701 1.0
0 -0.243166446685791 2.0
0 -0.214862585067749 3.0
0 0.708191156387329 4.0
0 0.314183950424194 5.0
0 0.138629913330078 6.0
0 1.10707306861877 7.0
42.375846269 37.5878486633301 0.0
0 -0.177824258804321 1.0
0 -0.276440382003784 2.0
0 0.300535202026367 3.0
0 0.609556674957275 4.0
0 0.36335563659668 5.0
0 0.123247623443604 6.0
0 0.81885552406311 7.0
41.909868924 37.6218643188477 0.0
0 -0.196045637130737 1.0
0 -0.149683952331543 2.0
0 -0.336806774139404 3.0
0 1.39528727531433 4.0
0 0.437430620193481 5.0
0 0.183159828186035 6.0
0 4.31301403045654 7.0
38.515403814 35.9683685302734 0.0
0 0.096235990524292 1.0
0 -0.230300664901733 2.0
0 -0.443381309509277 3.0
0 0.79886531829834 4.0
0 0.237633466720581 5.0
0 0.173842668533325 6.0
0 1.56181025505066 7.0
32.187464773 36.3147926330566 0.0
0 -0.0597381591796875 1.0
0 -0.18359375 2.0
0 -0.435871124267578 3.0
0 0.630658626556396 4.0
0 0.153672695159912 5.0
0 0.170638084411621 6.0
0 0.870849370956421 7.0
43.553340498 38.1740760803223 0.0
0 -0.723330497741699 1.0
0 -0.377854585647583 2.0
0 0.0755565166473389 3.0
0 1.05966734886169 4.0
0 0.0489814281463623 5.0
0 0.228707790374756 6.0
0 3.10239672660828 7.0
38.661685118 36.0210304260254 0.0
0 -0.244766235351562 1.0
0 -0.358039140701294 2.0
0 0.355979919433594 3.0
0 1.27649974822998 4.0
0 0.202902793884277 5.0
0 0.162638664245605 6.0
0 3.92802500724792 7.0
47.185976948 36.5498924255371 0.0
0 -0.331822633743286 1.0
0 -0.191561460494995 2.0
0 -0.273258924484253 3.0
0 1.08275699615479 4.0
0 0.374145984649658 5.0
0 0.124014377593994 6.0
0 2.84451627731323 7.0
41.073649825 35.0963935852051 0.0
0 -0.226912498474121 1.0
0 -0.228867530822754 2.0
0 -0.281524419784546 3.0
0 0.709604740142822 4.0
0 0.400318145751953 5.0
0 0.13563346862793 6.0
0 1.16009759902954 7.0
39.086114846 34.9577941894531 0.0
0 0.194481372833252 1.0
0 -0.176359891891479 2.0
0 0.0578479766845703 3.0
0 1.25913381576538 4.0
0 0.113279819488525 5.0
0 0.177304267883301 6.0
0 3.56342101097107 7.0
38.072309151 35.9341850280762 0.0
0 -0.092426061630249 1.0
0 -0.219915151596069 2.0
0 -0.275522470474243 3.0
0 0.713494777679443 4.0
0 0.351775884628296 5.0
0 0.144470453262329 6.0
0 1.21520042419434 7.0
41.585402374 36.5825042724609 0.0
0 -0.291966676712036 1.0
0 -0.211578369140625 2.0
0 -0.209926128387451 3.0
0 1.05741572380066 4.0
0 0.219480991363525 5.0
0 0.169596195220947 6.0
0 2.92778015136719 7.0
42.317317809 35.0919151306152 0.0
0 -0.367944240570068 1.0
0 -0.243978977203369 2.0
0 0.605969190597534 3.0
0 1.19971919059753 4.0
0 0.0259215831756592 5.0
0 0.284422874450684 6.0
0 3.64022278785706 7.0
42.931477585 37.7079734802246 0.0
0 -0.405687093734741 1.0
0 -0.298678636550903 2.0
0 -0.156400442123413 3.0
0 0.714550971984863 4.0
0 0.423005104064941 5.0
0 0.210313558578491 6.0
0 1.26191520690918 7.0
46.096274411 37.4528579711914 0.0
0 -0.523600578308105 1.0
0 -0.326571226119995 2.0
0 -0.124646902084351 3.0
0 1.14818525314331 4.0
0 0.211224794387817 5.0
0 0.161932229995728 6.0
0 3.60044527053833 7.0
39.955261554 38.0551910400391 0.0
0 -0.658121109008789 1.0
0 -0.277116298675537 2.0
0 -0.234277725219727 3.0
0 0.973104238510132 4.0
0 0.0210554599761963 5.0
0 0.182473182678223 6.0
0 2.43658971786499 7.0
45.415961846 36.4129180908203 0.0
0 0.0785212516784668 1.0
0 -0.181880235671997 2.0
0 -0.326329469680786 3.0
0 1.07780504226685 4.0
0 0.418612003326416 5.0
0 0.178118467330933 6.0
0 2.7864453792572 7.0
44.275746657 37.100658416748 0.0
0 -0.253119945526123 1.0
0 -0.291062831878662 2.0
0 0.455915451049805 3.0
0 1.15946340560913 4.0
0 0.334352016448975 5.0
0 0.151990413665771 6.0
0 3.58064889907837 7.0
34.766238966 35.2649154663086 0.0
0 -0.1153244972229 1.0
0 -0.327467441558838 2.0
0 0.363564014434814 3.0
0 0.562569379806519 4.0
0 0.372751951217651 5.0
0 0.223009586334229 6.0
0 0.653786659240723 7.0
39.697519492 36.2470741271973 0.0
0 -0.0125117301940918 1.0
0 -0.202032566070557 2.0
0 -0.27622389793396 3.0
0 0.681288719177246 4.0
0 0.16882061958313 5.0
0 0.146212100982666 6.0
0 0.979946374893188 7.0
36.459952599 36.4965133666992 0.0
0 -0.0541634559631348 1.0
0 -0.158813953399658 2.0
0 -0.373572111129761 3.0
0 0.587464094161987 4.0
0 0.305206060409546 5.0
0 0.159156560897827 6.0
0 0.599611520767212 7.0
44.856515586 36.964771270752 0.0
0 -0.515907764434814 1.0
0 -0.296825408935547 2.0
0 -0.144067049026489 3.0
0 0.862587928771973 4.0
0 0.320629835128784 5.0
0 0.203583478927612 6.0
0 1.88307619094849 7.0
41.157913956 37.4886054992676 0.0
0 -0.480770587921143 1.0
0 -0.182517290115356 2.0
0 -0.380607604980469 3.0
0 0.883991241455078 4.0
0 0.411084175109863 5.0
0 0.248954296112061 6.0
0 2.27777671813965 7.0
32.092085792 35.5234680175781 0.0
0 0.235920906066895 1.0
0 -0.188432931900024 2.0
0 -0.479283809661865 3.0
0 0.682336091995239 4.0
0 0.287475824356079 5.0
0 0.278785467147827 6.0
0 1.05026173591614 7.0
30.327518425 38.1214408874512 0.0
0 -0.486508369445801 1.0
0 -0.32198691368103 2.0
0 -0.0703144073486328 3.0
0 0.426435708999634 4.0
0 0.220352411270142 5.0
0 0.165441513061523 6.0
0 0.179724931716919 7.0
36.482977278 37.2305908203125 0.0
0 -0.378793478012085 1.0
0 -0.310364723205566 2.0
0 -0.166494369506836 3.0
0 0.65861439704895 4.0
0 0.342136859893799 5.0
0 0.179378271102905 6.0
0 0.920423269271851 7.0
40.491170914 36.111888885498 0.0
0 -0.129850149154663 1.0
0 -0.254679918289185 2.0
0 -0.222861528396606 3.0
0 0.926308631896973 4.0
0 0.27197265625 5.0
0 0.188239574432373 6.0
0 2.18350982666016 7.0
41.188898334 37.5333366394043 0.0
0 -0.261491775512695 1.0
0 -0.251400709152222 2.0
0 0.0196659564971924 3.0
0 0.841963768005371 4.0
0 0.457515001296997 5.0
0 0.217264413833618 6.0
0 1.76487219333649 7.0
37.756949904 37.6834678649902 0.0
0 -0.420706510543823 1.0
0 -0.291498899459839 2.0
0 -0.0798277854919434 3.0
0 0.606335401535034 4.0
0 0.19189715385437 5.0
0 0.149266242980957 6.0
0 0.727475643157959 7.0
26.9626458 35.8652610778809 0.0
0 -0.0703082084655762 1.0
0 -0.239067554473877 2.0
0 -0.305839061737061 3.0
0 0.462336301803589 4.0
0 0.197489261627197 5.0
0 0.190312623977661 6.0
0 0.224970817565918 7.0
44.154680866 37.3119964599609 0.0
0 -0.400638103485107 1.0
0 -0.341715335845947 2.0
0 0.527164459228516 3.0
0 1.10128951072693 4.0
0 0.133008718490601 5.0
0 0.111455202102661 6.0
0 3.37572073936462 7.0
36.49920361 35.9495506286621 0.0
0 -0.0202326774597168 1.0
0 -0.132057905197144 2.0
0 -0.381731748580933 3.0
0 0.896907567977905 4.0
0 0.329513072967529 5.0
0 0.184909105300903 6.0
0 2.10375881195068 7.0
36.136187007 36.665657043457 0.0
0 -0.523422241210938 1.0
0 -0.2728271484375 2.0
0 -0.253958463668823 3.0
0 0.981095314025879 4.0
0 0.0715510845184326 5.0
0 0.229984521865845 6.0
0 2.83864068984985 7.0
43.805905709 37.1376762390137 0.0
0 -0.221519708633423 1.0
0 -0.210549116134644 2.0
0 -0.0720810890197754 3.0
0 1.0493175983429 4.0
0 0.141959190368652 5.0
0 0.146390914916992 6.0
0 3.02579736709595 7.0
43.400673478 35.8176651000977 0.0
0 0.0993006229400635 1.0
0 -0.244618654251099 2.0
0 -0.0681335926055908 3.0
0 1.10488605499268 4.0
0 0.207900524139404 5.0
0 0.175083875656128 6.0
0 3.34359622001648 7.0
35.708898932 35.6880760192871 0.0
0 -0.0918292999267578 1.0
0 -0.255877256393433 2.0
0 0.486861228942871 3.0
0 0.757418632507324 4.0
0 0.243467092514038 5.0
0 0.196013927459717 6.0
0 1.49925923347473 7.0
42.195375979 36.3828430175781 0.0
0 -0.36807107925415 1.0
0 -0.297574281692505 2.0
0 0.394485712051392 3.0
0 0.807274103164673 4.0
0 0.320182323455811 5.0
0 0.164222478866577 6.0
0 1.85265791416168 7.0
38.219132252 35.8419380187988 0.0
0 0.135921239852905 1.0
0 -0.227861642837524 2.0
0 0.460942506790161 3.0
0 0.360114097595215 4.0
0 0.330790996551514 5.0
0 0.136455535888672 6.0
0 0.251740694046021 7.0
37.34785904 36.0325393676758 0.0
0 0.0130972862243652 1.0
0 -0.292625427246094 2.0
0 -0.258378982543945 3.0
0 0.58322548866272 4.0
0 0.369256019592285 5.0
0 0.174097299575806 6.0
0 0.741177082061768 7.0
41.413317083 35.1016693115234 0.0
0 0.140158653259277 1.0
0 -0.32844066619873 2.0
0 0.2016921043396 3.0
0 0.694648742675781 4.0
0 0.198110342025757 5.0
0 0.144869089126587 6.0
0 1.19905042648315 7.0
39.408474012 36.153938293457 0.0
0 0.204913854598999 1.0
0 -0.1937096118927 2.0
0 -0.265957593917847 3.0
0 0.772786855697632 4.0
0 0.395538091659546 5.0
0 0.186612367630005 6.0
0 1.67461717128754 7.0
44.621947793 38.0907135009766 0.0
0 -0.552145957946777 1.0
0 -0.263185977935791 2.0
0 -0.265555620193481 3.0
0 1.02800822257996 4.0
0 0.423335075378418 5.0
0 0.190441370010376 6.0
0 2.79073357582092 7.0
35.042314673 34.8445625305176 0.0
0 0.354483604431152 1.0
0 -0.266747951507568 2.0
0 -0.219364643096924 3.0
0 0.714884042739868 4.0
0 0.261660099029541 5.0
0 0.152897596359253 6.0
0 1.25012755393982 7.0
44.668685987 35.3835754394531 0.0
0 0.139268159866333 1.0
0 -0.22425365447998 2.0
0 -0.240451574325562 3.0
0 0.840728282928467 4.0
0 0.232599973678589 5.0
0 0.136221170425415 6.0
0 1.7179434299469 7.0
44.271188235 37.1842956542969 0.0
0 -0.35739278793335 1.0
0 -0.184263944625854 2.0
0 -0.335181951522827 3.0
0 0.762088537216187 4.0
0 0.395375728607178 5.0
0 0.232859373092651 6.0
0 1.2782895565033 7.0
43.329615715 37.907154083252 0.0
0 -0.72040319442749 1.0
0 -0.27954626083374 2.0
0 -0.0433695316314697 3.0
0 1.0179431438446 4.0
0 0.142090797424316 5.0
0 0.180552244186401 6.0
0 3.11606669425964 7.0
37.727034893 35.2072257995605 0.0
0 0.0260553359985352 1.0
0 -0.257445335388184 2.0
0 0.0840744972229004 3.0
0 0.947114229202271 4.0
0 0.366389274597168 5.0
0 0.166133880615234 6.0
0 2.51043796539307 7.0
43.665312113 34.8660659790039 0.0
0 -0.0148446559906006 1.0
0 -0.246460199356079 2.0
0 0.107488632202148 3.0
0 0.808462619781494 4.0
0 0.143927812576294 5.0
0 0.136068344116211 6.0
0 1.58500552177429 7.0
40.656348383 36.5211563110352 0.0
0 -0.0297918319702148 1.0
0 -0.197230815887451 2.0
0 -0.401621341705322 3.0
0 0.905605316162109 4.0
0 0.341344356536865 5.0
0 0.179303169250488 6.0
0 1.91227519512177 7.0
30.759139555 37.5522766113281 0.0
0 -0.358229398727417 1.0
0 -0.233836889266968 2.0
0 -0.272317409515381 3.0
0 0.419462442398071 4.0
0 0.210476875305176 5.0
0 0.17202615737915 6.0
0 0.217479228973389 7.0
36.474229652 35.3104553222656 0.0
0 -0.149678468704224 1.0
0 -0.241619348526001 2.0
0 0.226901531219482 3.0
0 0.504293203353882 4.0
0 0.476634740829468 5.0
0 0.279571533203125 6.0
0 0.44220232963562 7.0
33.340721195 34.8269119262695 0.0
0 -0.0547327995300293 1.0
0 -0.179461717605591 2.0
0 0.00405049324035645 3.0
0 0.664419174194336 4.0
0 0.348035097122192 5.0
0 0.184977054595947 6.0
0 1.13222193717957 7.0
34.143349268 36.7405662536621 0.0
0 -0.797866821289062 1.0
0 -0.380656242370605 2.0
0 0.261074781417847 3.0
0 0.376575469970703 4.0
0 0.189320087432861 5.0
0 0.1573326587677 6.0
0 0.154243469238281 7.0
40.763829757 36.1188583374023 0.0
0 -0.164252042770386 1.0
0 -0.308893442153931 2.0
0 0.311456441879272 3.0
0 1.04964852333069 4.0
0 0.371997833251953 5.0
0 0.19554591178894 6.0
0 3.03148984909058 7.0
29.788224958 38.0368690490723 0.0
0 -0.717395305633545 1.0
0 -0.36043119430542 2.0
0 -0.144011259078979 3.0
0 0.527280569076538 4.0
0 0.450076580047607 5.0
0 0.189592123031616 6.0
0 0.526846647262573 7.0
34.675869554 38.34619140625 0.0
0 -0.427926301956177 1.0
0 -0.333407640457153 2.0
0 0.36329197883606 3.0
0 0.523138284683228 4.0
0 0.278091192245483 5.0
0 0.165680408477783 6.0
0 0.551953315734863 7.0
40.723855828 36.8559646606445 0.0
0 -0.208578109741211 1.0
0 -0.160561800003052 2.0
0 -0.253346920013428 3.0
0 1.21190690994263 4.0
0 0.140474081039429 5.0
0 0.146406412124634 6.0
0 3.30582928657532 7.0
41.414838352 35.3588562011719 0.0
0 0.124618053436279 1.0
0 -0.183770418167114 2.0
0 -0.398579835891724 3.0
0 1.04220294952393 4.0
0 0.403959989547729 5.0
0 0.217929601669312 6.0
0 2.75206279754639 7.0
33.391441214 36.6676559448242 0.0
0 -0.382058143615723 1.0
0 -0.339385986328125 2.0
0 0.265602827072144 3.0
0 0.526551008224487 4.0
0 0.164919853210449 5.0
0 0.184229373931885 6.0
0 0.426371097564697 7.0
36.660645385 35.357852935791 0.0
0 0.171978712081909 1.0
0 -0.302509784698486 2.0
0 0.744490623474121 3.0
0 0.831709861755371 4.0
0 0.272966384887695 5.0
0 0.174340963363647 6.0
0 1.89511489868164 7.0
33.547567166 34.6340866088867 0.0
0 0.123574495315552 1.0
0 -0.211160898208618 2.0
0 0.041048526763916 3.0
0 0.695347785949707 4.0
0 0.218041896820068 5.0
0 0.238435506820679 6.0
0 1.17681646347046 7.0
36.717923495 36.8202857971191 0.0
0 -0.279569387435913 1.0
0 -0.330220222473145 2.0
0 -0.0666990280151367 3.0
0 0.660668134689331 4.0
0 0.14400315284729 5.0
0 0.163755178451538 6.0
0 0.911863565444946 7.0
44.575737494 38.6609230041504 0.0
0 -0.861597537994385 1.0
0 -0.360600709915161 2.0
0 -0.0203323364257812 3.0
0 1.09128427505493 4.0
0 0.28983211517334 5.0
0 0.136961460113525 6.0
0 2.8897807598114 7.0
43.091610321 37.2051811218262 0.0
0 -0.393772840499878 1.0
0 -0.260455369949341 2.0
0 -0.29060959815979 3.0
0 0.968038558959961 4.0
0 0.385106325149536 5.0
0 0.168613433837891 6.0
0 2.53329110145569 7.0
42.647069973 35.4922866821289 0.0
0 0.0420012474060059 1.0
0 -0.228496551513672 2.0
0 -0.205939292907715 3.0
0 1.01033091545105 4.0
0 0.204774141311646 5.0
0 0.146888017654419 6.0
0 2.53515648841858 7.0
36.240225339 36.7855224609375 0.0
0 -0.261372327804565 1.0
0 -0.365136623382568 2.0
0 0.406275510787964 3.0
0 0.669441938400269 4.0
0 0.351948022842407 5.0
0 0.171116590499878 6.0
0 1.11491465568542 7.0
31.876756203 36.7355003356934 0.0
0 -0.286006927490234 1.0
0 -0.316483020782471 2.0
0 -0.148288488388062 3.0
0 0.522504568099976 4.0
0 0.275104761123657 5.0
0 0.180065393447876 6.0
0 0.569944143295288 7.0
38.711049659 35.9443473815918 0.0
0 -0.333484649658203 1.0
0 -0.318233728408813 2.0
0 -0.00156784057617188 3.0
0 0.55603814125061 4.0
0 0.319243431091309 5.0
0 0.202628135681152 6.0
0 0.490298509597778 7.0
33.592635672 37.0877227783203 0.0
0 -0.148147106170654 1.0
0 -0.303240299224854 2.0
0 0.11277961730957 3.0
0 0.515512704849243 4.0
0 0.315187692642212 5.0
0 0.170190334320068 6.0
0 0.494044542312622 7.0
31.939497021 36.1295623779297 0.0
0 -0.0773639678955078 1.0
0 -0.264209985733032 2.0
0 -0.0976216793060303 3.0
0 0.468678712844849 4.0
0 0.212816476821899 5.0
0 0.157766342163086 6.0
0 0.300954103469849 7.0
32.605677369 37.8363914489746 0.0
0 -0.594618797302246 1.0
0 -0.305296182632446 2.0
0 -0.362892627716064 3.0
0 0.418483257293701 4.0
0 0.267769575119019 5.0
0 0.190321683883667 6.0
0 0.17988920211792 7.0
35.02561425 37.8704299926758 0.0
0 -0.196080923080444 1.0
0 -0.237069845199585 2.0
0 0.200029611587524 3.0
0 0.462586641311646 4.0
0 0.410889148712158 5.0
0 0.202885866165161 6.0
0 0.549884557723999 7.0
29.155020371 37.277458190918 0.0
0 -0.250753164291382 1.0
0 -0.286606073379517 2.0
0 0.029465913772583 3.0
0 0.452346324920654 4.0
0 0.41296648979187 5.0
0 0.211092710494995 6.0
0 0.387900114059448 7.0
29.9790999 35.7744331359863 0.0
0 -0.225680828094482 1.0
0 -0.352576971054077 2.0
0 0.264802694320679 3.0
0 0.442096710205078 4.0
0 0.399745464324951 5.0
0 0.214251279830933 6.0
0 0.174049615859985 7.0
40.412374428 37.2005767822266 0.0
0 -0.0908849239349365 1.0
0 -0.28969144821167 2.0
0 -0.297013998031616 3.0
0 0.885177373886108 4.0
0 0.335562229156494 5.0
0 0.209437847137451 6.0
0 2.14613342285156 7.0
44.799669607 38.5814514160156 0.0
0 -0.356328725814819 1.0
0 -0.223425626754761 2.0
0 -0.30827260017395 3.0
0 1.19748902320862 4.0
0 0.31873631477356 5.0
0 0.130985736846924 6.0
0 3.49703240394592 7.0
35.691930085 36.2026062011719 0.0
0 -0.251011848449707 1.0
0 -0.272778987884521 2.0
0 0.43891978263855 3.0
0 0.681761264801025 4.0
0 0.268162965774536 5.0
0 0.155186653137207 6.0
0 1.21305894851685 7.0
37.293002382 37.1493492126465 0.0
0 -0.18047046661377 1.0
0 -0.275598526000977 2.0
0 0.103681564331055 3.0
0 0.870619297027588 4.0
0 0.347799777984619 5.0
0 0.204246997833252 6.0
0 2.15043306350708 7.0
31.749050635 36.7106704711914 0.0
0 -0.00921845436096191 1.0
0 -0.198137044906616 2.0
0 -0.422327756881714 3.0
0 0.582220792770386 4.0
0 0.307007551193237 5.0
0 0.182169198989868 6.0
0 0.637930631637573 7.0
44.438599316 34.5368270874023 0.0
0 0.411152124404907 1.0
0 -0.348851203918457 2.0
0 -0.190930843353271 3.0
0 1.45290541648865 4.0
0 0.285109758377075 5.0
0 0.172658443450928 6.0
0 4.23989343643188 7.0
30.584762345 37.2260665893555 0.0
0 -0.39350438117981 1.0
0 -0.304620981216431 2.0
0 -0.157558679580688 3.0
0 0.677584886550903 4.0
0 0.114891529083252 5.0
0 0.212587356567383 6.0
0 0.998031139373779 7.0
36.167711467 36.7775611877441 0.0
0 0.055823802947998 1.0
0 -0.244131326675415 2.0
0 -0.0723373889923096 3.0
0 0.552104473114014 4.0
0 0.074805736541748 5.0
0 0.123786449432373 6.0
0 0.643245458602905 7.0
45.589023165 34.9028472900391 0.0
0 0.23377537727356 1.0
0 -0.229839324951172 2.0
0 -0.274594068527222 3.0
0 1.30260968208313 4.0
0 0.00836944580078125 5.0
0 0.144048690795898 6.0
0 3.98310947418213 7.0
33.954863975 39.4572257995605 0.0
0 -0.83720588684082 1.0
0 -0.366997003555298 2.0
0 -0.307014226913452 3.0
0 0.751881837844849 4.0
0 0.0414443016052246 5.0
0 0.2225022315979 6.0
0 1.4055061340332 7.0
39.104021217 37.447826385498 0.0
0 -0.380179166793823 1.0
0 -0.309069871902466 2.0
0 0.0324697494506836 3.0
0 0.65343165397644 4.0
0 0.272483825683594 5.0
0 0.149781942367554 6.0
0 0.960371255874634 7.0
41.615139375 35.4986915588379 0.0
0 -0.2718186378479 1.0
0 -0.233350276947021 2.0
0 -0.177411794662476 3.0
0 0.987984895706177 4.0
0 0.424508333206177 5.0
0 0.250570058822632 6.0
0 2.58912372589111 7.0
29.251209169 34.9306907653809 0.0
0 -0.171401262283325 1.0
0 -0.291459798812866 2.0
0 0.23852014541626 3.0
0 0.444803953170776 4.0
0 0.0185027122497559 5.0
0 0.245470285415649 6.0
0 0.255368232727051 7.0
39.145261601 37.5717353820801 0.0
0 -0.661928653717041 1.0
0 -0.229492425918579 2.0
0 -0.252750396728516 3.0
0 0.937916278839111 4.0
0 0.340118408203125 5.0
0 0.16578221321106 6.0
0 2.53820133209229 7.0
43.724223514 34.8684883117676 0.0
0 0.302788496017456 1.0
0 -0.23316478729248 2.0
0 -0.248343944549561 3.0
0 1.3639440536499 4.0
0 0.166401624679565 5.0
0 0.155365943908691 6.0
0 4.00921726226807 7.0
30.694427948 34.8150634765625 0.0
0 0.290907382965088 1.0
0 -0.129974842071533 2.0
0 -0.438190937042236 3.0
0 0.629068851470947 4.0
0 0.281956195831299 5.0
0 0.233678579330444 6.0
0 0.874275207519531 7.0
46.899446079 37.0074920654297 0.0
0 -0.523783206939697 1.0
0 -0.291285037994385 2.0
0 0.128767967224121 3.0
0 1.24919295310974 4.0
0 0.318097591400146 5.0
0 0.232030153274536 6.0
0 4.14059019088745 7.0
30.801741874 36.7322654724121 0.0
0 -0.349495887756348 1.0
0 -0.208088636398315 2.0
0 -0.344642162322998 3.0
0 0.433636665344238 4.0
0 0.137452125549316 5.0
0 0.210865259170532 6.0
0 0.254416465759277 7.0
40.388900757 36.7278251647949 0.0
0 -0.466512203216553 1.0
0 -0.359871864318848 2.0
0 0.736284017562866 3.0
0 0.789740800857544 4.0
0 0.0164930820465088 5.0
0 0.177773952484131 6.0
0 1.75850594043732 7.0
35.801686131 38.2046394348145 0.0
0 -0.518937110900879 1.0
0 -0.18743896484375 2.0
0 -0.347006559371948 3.0
0 0.645337104797363 4.0
0 0.293102025985718 5.0
0 0.119016885757446 6.0
0 0.962273120880127 7.0
43.275430302 35.7680740356445 0.0
0 -0.190526247024536 1.0
0 -0.24742317199707 2.0
0 -0.0042574405670166 3.0
0 0.855082035064697 4.0
0 0.500061511993408 5.0
0 0.240684509277344 6.0
0 2.16242933273315 7.0
36.674533885 36.9891014099121 0.0
0 -0.367173194885254 1.0
0 -0.304259061813354 2.0
0 -0.0175468921661377 3.0
0 0.699902534484863 4.0
0 0.225318431854248 5.0
0 0.162655591964722 6.0
0 1.24039268493652 7.0
42.291929941 35.7292633056641 0.0
0 -0.407721281051636 1.0
0 -0.185325860977173 2.0
0 -0.101832389831543 3.0
0 0.965316772460938 4.0
0 0.205824851989746 5.0
0 0.14600396156311 6.0
0 2.40530014038086 7.0
37.80686855 37.3498001098633 0.0
0 -0.20350170135498 1.0
0 -0.230728149414062 2.0
0 -0.282553672790527 3.0
0 0.458207845687866 4.0
0 0.393935918807983 5.0
0 0.220400810241699 6.0
0 0.236475944519043 7.0
36.296445808 37.5152740478516 0.0
0 -0.0770494937896729 1.0
0 -0.175446033477783 2.0
0 -0.31798243522644 3.0
0 0.525772333145142 4.0
0 0.254282474517822 5.0
0 0.17354416847229 6.0
0 0.303946018218994 7.0
26.786347985 35.118782043457 0.0
0 0.285208702087402 1.0
0 -0.261586904525757 2.0
0 0.778452396392822 3.0
0 0.513319730758667 4.0
0 0.105484008789062 5.0
0 0.178385972976685 6.0
0 0.382270336151123 7.0
38.715518937 36.8473091125488 0.0
0 -0.259464740753174 1.0
0 -0.333176851272583 2.0
0 0.0364620685577393 3.0
0 0.963521003723145 4.0
0 0.261210918426514 5.0
0 0.218620777130127 6.0
0 2.18739748001099 7.0
35.505063002 37.8761787414551 0.0
0 -0.412382364273071 1.0
0 -0.30470609664917 2.0
0 0.0832529067993164 3.0
0 0.468101501464844 4.0
0 0.368189573287964 5.0
0 0.180174112319946 6.0
0 0.253301858901978 7.0
44.227649071 36.8039207458496 0.0
0 -0.224485635757446 1.0
0 -0.301677227020264 2.0
0 0.402795791625977 3.0
0 1.30907511711121 4.0
0 0.367098569869995 5.0
0 0.196034908294678 6.0
0 3.81873822212219 7.0
45.773124255 34.8590812683105 0.0
0 0.00698256492614746 1.0
0 -0.293610095977783 2.0
0 -0.281806468963623 3.0
0 1.03360319137573 4.0
0 0.360316514968872 5.0
0 0.193556308746338 6.0
0 2.71208095550537 7.0
38.615282752 38.5891418457031 0.0
0 -0.74624490737915 1.0
0 -0.382250785827637 2.0
0 0.127957105636597 3.0
0 1.12544298171997 4.0
0 0.264419078826904 5.0
0 0.162266492843628 6.0
0 3.24034523963928 7.0
42.131124836 36.8266410827637 0.0
0 -0.670224189758301 1.0
0 -0.374903440475464 2.0
0 0.553368330001831 3.0
0 0.854133129119873 4.0
0 0.105825185775757 5.0
0 0.143289566040039 6.0
0 1.89838600158691 7.0
38.030369766 36.2832946777344 0.0
0 -0.367695093154907 1.0
0 -0.339453935623169 2.0
0 -0.0340845584869385 3.0
0 0.839179039001465 4.0
0 0.0667197704315186 5.0
0 0.228204011917114 6.0
0 1.80415737628937 7.0
32.694870002 35.0468826293945 0.0
0 -0.120362758636475 1.0
0 -0.201116800308228 2.0
0 -0.19334888458252 3.0
0 0.390397787094116 4.0
0 0.173342943191528 5.0
0 0.15645956993103 6.0
0 0.148246765136719 7.0
41.58883184 36.0160026550293 0.0
0 -0.448805332183838 1.0
0 -0.320191621780396 2.0
0 0.766071796417236 3.0
0 0.755717039108276 4.0
0 0.262525796890259 5.0
0 0.197457790374756 6.0
0 1.49768924713135 7.0
33.622063045 37.1182823181152 0.0
0 -0.0894958972930908 1.0
0 -0.254556894302368 2.0
0 -0.215538263320923 3.0
0 0.668501853942871 4.0
0 0.31981635093689 5.0
0 0.17388391494751 6.0
0 0.972289562225342 7.0
40.101738506 38.1007652282715 0.0
0 -0.557812213897705 1.0
0 -0.295395612716675 2.0
0 -0.133106231689453 3.0
0 1.08997130393982 4.0
0 0.262934446334839 5.0
0 0.210092782974243 6.0
0 3.03352022171021 7.0
39.347750494 36.9291839599609 0.0
0 -0.0950350761413574 1.0
0 -0.267767429351807 2.0
0 0.587551832199097 3.0
0 0.978325843811035 4.0
0 0.36745548248291 5.0
0 0.180153369903564 6.0
0 2.97654557228088 7.0
46.240008463 35.647762298584 0.0
0 0.202317237854004 1.0
0 -0.289443492889404 2.0
0 0.202964305877686 3.0
0 0.933360576629639 4.0
0 0.318665027618408 5.0
0 0.12982702255249 6.0
0 2.40401172637939 7.0
43.885613972 37.5371360778809 0.0
0 -0.456361770629883 1.0
0 -0.331928968429565 2.0
0 0.370083093643188 3.0
0 1.05059385299683 4.0
0 0.125192165374756 5.0
0 0.11235499382019 6.0
0 2.96707248687744 7.0
44.877286055 35.6448135375977 0.0
0 -0.181763648986816 1.0
0 -0.181859016418457 2.0
0 -0.399070501327515 3.0
0 1.33096671104431 4.0
0 0.237063646316528 5.0
0 0.168818712234497 6.0
0 3.68962407112122 7.0
44.281024849 36.1657409667969 0.0
0 -0.132063627243042 1.0
0 -0.331666231155396 2.0
0 -0.125808477401733 3.0
0 0.821219444274902 4.0
0 0.382120132446289 5.0
0 0.199132204055786 6.0
0 1.80734181404114 7.0
39.443195334 38.5993003845215 0.0
0 -0.678755760192871 1.0
0 -0.30858850479126 2.0
0 -0.318851947784424 3.0
0 0.621803998947144 4.0
0 0.181328535079956 5.0
0 0.143912553787231 6.0
0 0.750884294509888 7.0
44.50998012 36.2225952148438 0.0
0 0.0411431789398193 1.0
0 -0.189365386962891 2.0
0 -0.254945993423462 3.0
0 0.796531915664673 4.0
0 0.121331691741943 5.0
0 0.129098653793335 6.0
0 1.79623973369598 7.0
36.079576822 34.6056060791016 0.0
0 0.405097007751465 1.0
0 -0.305070877075195 2.0
0 -0.3541579246521 3.0
0 0.721698999404907 4.0
0 0.401793003082275 5.0
0 0.190984487533569 6.0
0 1.27235102653503 7.0
43.881758876 38.1315078735352 0.0
0 -0.624343395233154 1.0
0 -0.249041557312012 2.0
0 -0.105660915374756 3.0
0 1.16149830818176 4.0
0 0.483824253082275 5.0
0 0.234349012374878 6.0
0 3.6650562286377 7.0
30.108818284 35.1328125 0.0
0 0.278914928436279 1.0
0 -0.138228416442871 2.0
0 -0.320368051528931 3.0
0 0.656855344772339 4.0
0 0.121258735656738 5.0
0 0.226584196090698 6.0
0 0.89711332321167 7.0
34.22051219 34.9653739929199 0.0
0 0.0693609714508057 1.0
0 -0.182999849319458 2.0
0 -0.230875968933105 3.0
0 0.655549049377441 4.0
0 0.407809734344482 5.0
0 0.196042537689209 6.0
0 0.9639573097229 7.0
37.402197971 37.4399147033691 0.0
0 -0.787345886230469 1.0
0 -0.227387666702271 2.0
0 -0.33899712562561 3.0
0 0.691039800643921 4.0
0 0.164315938949585 5.0
0 0.159854173660278 6.0
0 1.12422275543213 7.0
35.453722409 38.7221565246582 0.0
0 -0.65479850769043 1.0
0 -0.275555849075317 2.0
0 -0.0533702373504639 3.0
0 0.689786195755005 4.0
0 0.088902473449707 5.0
0 0.158952713012695 6.0
0 1.12733507156372 7.0
32.872381155 36.7309722900391 0.0
0 -0.455498218536377 1.0
0 -0.359278917312622 2.0
0 0.521559476852417 3.0
0 0.575238466262817 4.0
0 0.159432411193848 5.0
0 0.16349196434021 6.0
0 0.644155502319336 7.0
33.577714316 39.0117568969727 0.0
0 -0.982303619384766 1.0
0 -0.32575535774231 2.0
0 -0.17690896987915 3.0
0 0.855686902999878 4.0
0 0.440007448196411 5.0
0 0.185654401779175 6.0
0 1.84067964553833 7.0
43.986168802 36.2321166992188 0.0
0 -0.209038019180298 1.0
0 -0.277627229690552 2.0
0 0.384305715560913 3.0
0 0.992801904678345 4.0
0 0.241016626358032 5.0
0 0.126047849655151 6.0
0 2.37969517707825 7.0
39.701527205 38.2049179077148 0.0
0 -0.507784366607666 1.0
0 -0.314949035644531 2.0
0 -0.0094602108001709 3.0
0 0.982918500900269 4.0
0 0.0362365245819092 5.0
0 0.174622774124146 6.0
0 2.86395907402039 7.0
36.319422317 39.3521881103516 0.0
0 -0.931595802307129 1.0
0 -0.407305717468262 2.0
0 -0.013624906539917 3.0
0 0.798926591873169 4.0
0 0.319846391677856 5.0
0 0.216780662536621 6.0
0 1.7472470998764 7.0
39.54652972 37.645076751709 0.0
0 -0.391183376312256 1.0
0 -0.241750001907349 2.0
0 -0.290000915527344 3.0
0 1.03361439704895 4.0
0 0.281151294708252 5.0
0 0.174216270446777 6.0
0 2.87820839881897 7.0
46.3288844 35.500186920166 0.0
0 0.0452299118041992 1.0
0 -0.179924249649048 2.0
0 -0.389382600784302 3.0
0 0.893146991729736 4.0
0 0.467753410339355 5.0
0 0.306151628494263 6.0
0 2.19514036178589 7.0
39.359671461 34.898323059082 0.0
0 0.232790946960449 1.0
0 -0.161021709442139 2.0
0 -0.360504627227783 3.0
0 0.879126071929932 4.0
0 0.438492774963379 5.0
0 0.19801664352417 6.0
0 1.93370258808136 7.0
43.813870988 37.8486328125 0.0
0 -0.882665157318115 1.0
0 -0.349601030349731 2.0
0 0.439090490341187 3.0
0 0.910456895828247 4.0
0 0.314844846725464 5.0
0 0.21257758140564 6.0
0 2.13126611709595 7.0
39.146221876 38.850830078125 0.0
0 -0.760386943817139 1.0
0 -0.344521045684814 2.0
0 0.260546684265137 3.0
0 0.956523180007935 4.0
0 0.182870626449585 5.0
0 0.203143358230591 6.0
0 2.4703688621521 7.0
34.753353014 37.1096115112305 0.0
0 -0.117552280426025 1.0
0 -0.249989748001099 2.0
0 -0.234713792800903 3.0
0 0.926862955093384 4.0
0 0.247049570083618 5.0
0 0.198601961135864 6.0
0 2.1302342414856 7.0
36.858915008 36.8985557556152 0.0
0 -0.410741806030273 1.0
0 -0.233355760574341 2.0
0 -0.169047355651855 3.0
0 0.856708765029907 4.0
0 0.24169659614563 5.0
0 0.170842409133911 6.0
0 2.07379865646362 7.0
45.609388725 35.2752151489258 0.0
0 -0.149630784988403 1.0
0 -0.273818731307983 2.0
0 -0.108396291732788 3.0
0 0.968324422836304 4.0
0 0.318283319473267 5.0
0 0.120416879653931 6.0
0 2.46331429481506 7.0
37.221502314 35.4497299194336 0.0
0 -0.22513222694397 1.0
0 -0.344719409942627 2.0
0 0.328586101531982 3.0
0 0.931279897689819 4.0
0 0.235136270523071 5.0
0 0.194885730743408 6.0
0 2.16721868515015 7.0
39.351241848 36.369873046875 0.0
0 -0.650408267974854 1.0
0 -0.274214744567871 2.0
0 -0.176385402679443 3.0
0 1.09853625297546 4.0
0 0.0512738227844238 5.0
0 0.23320460319519 6.0
0 3.57181906700134 7.0
44.354286139 37.940113067627 0.0
0 -0.631168842315674 1.0
0 -0.26283073425293 2.0
0 0.0569820404052734 3.0
0 0.94477367401123 4.0
0 0.295450687408447 5.0
0 0.150166988372803 6.0
0 2.16711378097534 7.0
41.143843969 37.0273284912109 0.0
0 -0.200546026229858 1.0
0 -0.294461011886597 2.0
0 -0.351404666900635 3.0
0 0.851983547210693 4.0
0 0.270292997360229 5.0
0 0.154969215393066 6.0
0 1.71094477176666 7.0
32.436363672 35.4038429260254 0.0
0 0.127327442169189 1.0
0 -0.345468282699585 2.0
0 -0.151887893676758 3.0
0 0.474510192871094 4.0
0 0.125726938247681 5.0
0 0.242553472518921 6.0
0 0.439521551132202 7.0
35.980268826 36.4308204650879 0.0
0 -0.344396352767944 1.0
0 -0.240221738815308 2.0
0 -0.302401781082153 3.0
0 0.735308170318604 4.0
0 0.247837066650391 5.0
0 0.169052600860596 6.0
0 1.19302177429199 7.0
39.361696951 36.293628692627 0.0
0 -0.101364850997925 1.0
0 -0.276361703872681 2.0
0 -0.513451099395752 3.0
0 0.490055561065674 4.0
0 0.398007154464722 5.0
0 0.209672927856445 6.0
0 0.24671483039856 7.0
38.339957501 37.4627571105957 0.0
0 -0.379246711730957 1.0
0 -0.292562484741211 2.0
0 0.437351942062378 3.0
0 0.828794717788696 4.0
0 0.171418190002441 5.0
0 0.171962738037109 6.0
0 1.78594350814819 7.0
38.437078026 37.3630180358887 0.0
0 -0.596734523773193 1.0
0 -0.254617214202881 2.0
0 -0.407095432281494 3.0
0 1.55690288543701 4.0
0 0.291861534118652 5.0
0 0.181710004806519 6.0
0 4.72313261032104 7.0
31.680301161 39.1800193786621 0.0
0 -0.959363460540771 1.0
0 -0.336412906646729 2.0
0 -0.136690616607666 3.0
0 0.694922208786011 4.0
0 0.304638385772705 5.0
0 0.211116552352905 6.0
0 1.23985767364502 7.0
36.698769132 35.5050659179688 0.0
0 -0.154198169708252 1.0
0 -0.346642255783081 2.0
0 -0.266502618789673 3.0
0 0.656166791915894 4.0
0 0.381205320358276 5.0
0 0.247137069702148 6.0
0 0.908086061477661 7.0
40.942334922 36.2492599487305 0.0
0 -0.0153210163116455 1.0
0 -0.279077768325806 2.0
0 0.13150429725647 3.0
0 0.918367147445679 4.0
0 0.215618133544922 5.0
0 0.148518085479736 6.0
0 2.4165632724762 7.0
41.496981537 35.625171661377 0.0
0 -0.124783754348755 1.0
0 -0.254891872406006 2.0
0 -0.208508729934692 3.0
0 1.20965170860291 4.0
0 0.221914291381836 5.0
0 0.14167594909668 6.0
0 3.9689359664917 7.0
33.577466065 35.5959930419922 0.0
0 -0.287380695343018 1.0
0 -0.267948389053345 2.0
0 0.0577962398529053 3.0
0 0.826738357543945 4.0
0 0.184857130050659 5.0
0 0.194498062133789 6.0
0 1.93140530586243 7.0
38.410697595 38.2763824462891 0.0
0 -0.520513534545898 1.0
0 -0.339839458465576 2.0
0 -0.0661749839782715 3.0
0 0.471890926361084 4.0
0 0.149702072143555 5.0
0 0.121147632598877 6.0
0 0.261744022369385 7.0
40.390522112 38.2091064453125 0.0
0 -0.612332344055176 1.0
0 -0.271833181381226 2.0
0 -0.259201765060425 3.0
0 1.25594806671143 4.0
0 0.169102668762207 5.0
0 0.205338954925537 6.0
0 3.94799995422363 7.0
38.02724699 36.063533782959 0.0
0 0.0317010879516602 1.0
0 -0.272143125534058 2.0
0 -0.216784000396729 3.0
0 0.730587005615234 4.0
0 0.124403476715088 5.0
0 0.13898754119873 6.0
0 1.38087725639343 7.0
34.831322295 36.3328666687012 0.0
0 0.0441093444824219 1.0
0 -0.211743831634521 2.0
0 -0.122546911239624 3.0
0 0.738695621490479 4.0
0 0.138259887695312 5.0
0 0.161046266555786 6.0
0 1.44972276687622 7.0
33.152476832 37.790714263916 0.0
0 -0.473422527313232 1.0
0 -0.301004648208618 2.0
0 -0.0581612586975098 3.0
0 0.729478359222412 4.0
0 0.193731069564819 5.0
0 0.178492069244385 6.0
0 1.4444739818573 7.0
37.76181476 35.5246391296387 0.0
0 -0.0927186012268066 1.0
0 -0.216365337371826 2.0
0 -0.251622676849365 3.0
0 0.653107643127441 4.0
0 0.347537517547607 5.0
0 0.184205055236816 6.0
0 0.846905708312988 7.0
46.821650967 36.079029083252 0.0
0 -0.165574312210083 1.0
0 -0.230050563812256 2.0
0 -0.229132652282715 3.0
0 1.2433123588562 4.0
0 0.289715766906738 5.0
0 0.153286695480347 6.0
0 3.77302026748657 7.0
41.183790175 35.139102935791 0.0
0 -0.209216356277466 1.0
0 -0.222369194030762 2.0
0 -0.444353103637695 3.0
0 1.0161075592041 4.0
0 0.116122961044312 5.0
0 0.141939401626587 6.0
0 2.52373600006104 7.0
36.78498864 35.0383758544922 0.0
0 0.378259181976318 1.0
0 -0.335481405258179 2.0
0 -0.20229172706604 3.0
0 0.628661632537842 4.0
0 0.348950624465942 5.0
0 0.193682670593262 6.0
0 0.973303556442261 7.0
37.689494945 36.5761795043945 0.0
0 -0.393375873565674 1.0
0 -0.279083013534546 2.0
0 0.0304028987884521 3.0
0 1.1424548625946 4.0
0 0.320878267288208 5.0
0 0.178584575653076 6.0
0 3.10387277603149 7.0
41.308881097 38.8460121154785 0.0
0 -0.657222747802734 1.0
0 -0.357978343963623 2.0
0 0.0985016822814941 3.0
0 0.962998390197754 4.0
0 0.396562099456787 5.0
0 0.178927898406982 6.0
0 2.53402471542358 7.0
42.253011184 36.4274978637695 0.0
0 -0.201509952545166 1.0
0 -0.252234220504761 2.0
0 -0.317175149917603 3.0
0 1.10002064704895 4.0
0 0.238052368164062 5.0
0 0.164145708084106 6.0
0 2.99787950515747 7.0
40.047660493 37.0765647888184 0.0
0 -0.680072784423828 1.0
0 -0.273121118545532 2.0
0 0.0190396308898926 3.0
0 1.03004670143127 4.0
0 0.26183009147644 5.0
0 0.151882410049438 6.0
0 2.58208966255188 7.0
46.604605733 37.801342010498 0.0
0 -0.370045900344849 1.0
0 -0.213473081588745 2.0
0 -0.216978311538696 3.0
0 1.12363195419312 4.0
0 0.327935695648193 5.0
0 0.13085150718689 6.0
0 3.3141348361969 7.0
45.106295542 37.4737930297852 0.0
0 0.018416166305542 1.0
0 -0.159879446029663 2.0
0 -0.256176471710205 3.0
0 1.46160817146301 4.0
0 0.19580078125 5.0
0 0.161989212036133 6.0
0 4.38176679611206 7.0
38.977055099 35.760368347168 0.0
0 -0.0868351459503174 1.0
0 -0.302794456481934 2.0
0 0.372197866439819 3.0
0 0.933371305465698 4.0
0 0.303013324737549 5.0
0 0.19176173210144 6.0
0 2.36102771759033 7.0
38.935365883 36.5788993835449 0.0
0 -0.0734860897064209 1.0
0 -0.230412483215332 2.0
0 0.0133368968963623 3.0
0 0.730216503143311 4.0
0 0.297123432159424 5.0
0 0.179492235183716 6.0
0 1.34061241149902 7.0
42.741409918 37.6341514587402 0.0
0 -0.516990661621094 1.0
0 -0.303493976593018 2.0
0 0.212670564651489 3.0
0 0.939650297164917 4.0
0 0.371391296386719 5.0
0 0.277822256088257 6.0
0 2.44248819351196 7.0
31.310794576 37.9046859741211 0.0
0 -0.529849052429199 1.0
0 -0.272851705551147 2.0
0 0.0274727344512939 3.0
0 0.561710834503174 4.0
0 0.355588436126709 5.0
0 0.214748382568359 6.0
0 0.641820669174194 7.0
};
\addlegendentry{$R^2$=0.976}
\end{axis}

\end{tikzpicture}
}}
    \subfloat[Actual vs predicted edge flows.] 
    {\label{fig:results_nonlineal_dummy_edge_base_bal_wey}\resizebox{\figurewidth}{\figureheight}{% This file was created with tikzplotlib v0.10.1.
\begin{tikzpicture}

\definecolor{darkgray176}{RGB}{176,176,176}
\definecolor{lightgray204}{RGB}{204,204,204}

\begin{axis}[
colorbar,
colorbar style={ylabel={edge id}},
colormap={mymap}{[1pt]
 rgb(0pt)=(0.12156862745098,0.466666666666667,0.705882352941177);
  rgb(1pt)=(1,0.498039215686275,0.0549019607843137);
  rgb(2pt)=(0.172549019607843,0.627450980392157,0.172549019607843);
  rgb(3pt)=(0.83921568627451,0.152941176470588,0.156862745098039);
  rgb(4pt)=(0.580392156862745,0.403921568627451,0.741176470588235);
  rgb(5pt)=(0.549019607843137,0.337254901960784,0.294117647058824);
  rgb(6pt)=(0.890196078431372,0.466666666666667,0.76078431372549);
  rgb(7pt)=(0.498039215686275,0.498039215686275,0.498039215686275);
  rgb(8pt)=(0.737254901960784,0.741176470588235,0.133333333333333);
  rgb(9pt)=(0.0901960784313725,0.745098039215686,0.811764705882353)
},
legend cell align={left},
legend style={
  fill opacity=0.8,
  draw opacity=1,
  text opacity=1,
  at={(0.03,0.97)},
  anchor=north west,
  draw=lightgray204
},
point meta max=7,
point meta min=0,
tick align=outside,
tick pos=left,
title={},
x grid style={darkgray176},
xlabel={True},
xmajorgrids,
xmin=-15.25776872815, xmax=51.84084685915,
xtick style={color=black},
xtick={-10,0,10,20,30,40,50}, 
xticklabels={-10,0,10,20,30,40,$f_e$},
y grid style={darkgray176},
ylabel={Predicted},
ymajorgrids,
ymin=-9.33694415092468, ymax=42.7169385433197,
ytick={-10,0,10,20,30,40}, 
yticklabels={-10,0,10,20,30,$f_e$},
ytick style={color=black}
]
\addplot [
  colormap={mymap}{[1pt]
 rgb(0pt)=(0.12156862745098,0.466666666666667,0.705882352941177);
  rgb(1pt)=(1,0.498039215686275,0.0549019607843137);
  rgb(2pt)=(0.172549019607843,0.627450980392157,0.172549019607843);
  rgb(3pt)=(0.83921568627451,0.152941176470588,0.156862745098039);
  rgb(4pt)=(0.580392156862745,0.403921568627451,0.741176470588235);
  rgb(5pt)=(0.549019607843137,0.337254901960784,0.294117647058824);
  rgb(6pt)=(0.890196078431372,0.466666666666667,0.76078431372549);
  rgb(7pt)=(0.498039215686275,0.498039215686275,0.498039215686275);
  rgb(8pt)=(0.737254901960784,0.741176470588235,0.133333333333333);
  rgb(9pt)=(0.0901960784313725,0.745098039215686,0.811764705882353)
},
  only marks,
  scatter,
  scatter src=explicit
]
table [x=x, y=y, meta=colordata]{%
x  y  colordata
39.565898635 28.6714134216309 0.0
18.050517226 14.9616947174072 1.0
21.515381409 18.2802124023438 2.0
-8.8987516789 -5.52296113967896 3.0
12.61662972 12.3899621963501 4.0
26.949268915 21.7630672454834 5.0
39.565898645 32.3844261169434 6.0
39.565898645 33.9594993591309 7.0
42.743257171 40.1816635131836 0.0
22.448801828 15.7189064025879 1.0
20.294455342 13.9119443893433 2.0
-4.2455956832 -4.950599193573 3.0
16.048859649 14.2395324707031 4.0
26.694397521 24.0332641601562 5.0
42.743257181 36.9367408752441 6.0
42.743257181 33.8822326660156 7.0
39.367180747 36.6610412597656 0.0
18.970258592 18.3984546661377 1.0
20.39692216 20.2436618804932 2.0
-4.9776642323 -4.86352157592773 3.0
15.419257923 14.9586067199707 4.0
23.947922829 22.4005947113037 5.0
39.367180751 34.8410263061523 6.0
39.367180751 34.0135879516602 7.0
39.605393097 34.1749229431152 0.0
22.082745928 17.2484874725342 1.0
17.522647169 15.4437026977539 2.0
-7.2517903354 -4.85904026031494 3.0
10.270856823 10.6809597015381 4.0
29.334536274 24.9471874237061 5.0
39.605393107 33.6471786499023 6.0
39.605393107 33.9251670837402 7.0
43.937345526 39.5011901855469 0.0
21.865843593 17.8610610961914 1.0
22.071501933 19.6440448760986 2.0
-8.3469765086 -6.65771102905273 3.0
13.724525414 13.3164844512939 4.0
30.212820112 27.1554489135742 5.0
43.937345536 34.7840728759766 6.0
43.937345536 34.2022705078125 7.0
31.061989584 33.1196708679199 0.0
16.203677385 17.1376838684082 1.0
14.858312199 15.9516906738281 2.0
-5.5704045507 -3.49752426147461 3.0
9.2879076384 9.93039131164551 4.0
21.774081945 20.2006034851074 5.0
31.061989594 32.9481315612793 6.0
31.061989594 34.0323638916016 7.0
36.357435265 31.210319519043 0.0
19.38173709 18.2190132141113 1.0
16.975698176 19.56565284729 2.0
-8.3424496148 -4.85638809204102 3.0
8.6332485532 10.9250869750977 4.0
27.724186714 24.035062789917 5.0
36.357435273 32.778564453125 6.0
36.357435273 35.1811866760254 7.0
38.444969607 28.5558280944824 0.0
21.080570389 17.3442096710205 1.0
17.364399218 15.0663089752197 2.0
-5.0980379196 -4.33392143249512 3.0
12.266361289 11.7904319763184 4.0
26.178608318 23.173526763916 5.0
38.444969617 32.1528930664062 6.0
38.444969617 33.7101554870605 7.0
35.498620518 39.6808090209961 0.0
17.900031823 16.4247035980225 1.0
17.598588698 16.0485706329346 2.0
-4.651704216 -4.16256666183472 3.0
12.946884476 11.9936809539795 4.0
22.551736046 20.6595764160156 5.0
35.498620524 36.5796813964844 6.0
35.498620524 33.9381561279297 7.0
36.52099827 36.2504615783691 0.0
18.961240079 16.4903736114502 1.0
17.559758192 15.8271732330322 2.0
-5.3314768445 -5.38143968582153 3.0
12.228281339 11.7870721817017 4.0
24.292716932 23.5897254943848 5.0
36.520998279 34.4774856567383 6.0
36.520998279 33.6644058227539 7.0
36.717272204 35.4489669799805 0.0
19.763035348 18.8932495117188 1.0
16.954236857 18.2436618804932 2.0
-7.0689091682 -4.22119855880737 3.0
9.88532768 10.9079065322876 4.0
26.831944525 21.6159896850586 5.0
36.717272212 33.7317276000977 6.0
36.717272212 33.6340026855469 7.0
32.629628996 30.6418190002441 0.0
17.103127104 18.2054195404053 1.0
15.526501892 19.5009651184082 2.0
-7.1999965109 -5.15085172653198 3.0
8.3265053708 9.94327926635742 4.0
24.303123625 22.9582557678223 5.0
32.629629006 32.1080131530762 6.0
32.629629006 34.6008071899414 7.0
37.75267433 31.2892189025879 0.0
17.533699782 16.1860866546631 1.0
20.218974548 18.7036685943604 2.0
-3.919745576 -5.75310754776001 3.0
16.299228962 15.3567609786987 4.0
21.453445368 21.1630458831787 5.0
37.75267434 32.5889129638672 6.0
37.75267434 33.7459487915039 7.0
38.800291337 33.8247566223145 0.0
21.414385846 17.6930561065674 1.0
17.385905491 15.7327365875244 2.0
-6.3164463541 -5.66953039169312 3.0
11.069459127 11.2251358032227 4.0
27.73083221 25.1215400695801 5.0
38.800291347 33.0986137390137 6.0
38.800291347 33.4470138549805 7.0
38.252729609 33.0313262939453 0.0
20.199036122 16.5485725402832 1.0
18.053693488 16.1784725189209 2.0
-6.9495443437 -4.93247556686401 3.0
11.104149136 11.1711893081665 4.0
27.148580475 23.2659797668457 5.0
38.252729618 33.0936241149902 6.0
38.252729618 34.0962715148926 7.0
43.273596037 40.1034698486328 0.0
20.811072897 15.3373613357544 1.0
22.46252314 18.7854518890381 2.0
-10.185324283 -6.60011720657349 3.0
12.277198847 12.3342218399048 4.0
30.99639719 27.8145084381104 5.0
43.273596047 35.7721900939941 6.0
43.273596047 33.1282806396484 7.0
34.027431477 34.861457824707 0.0
16.463058834 14.6708507537842 1.0
17.564372643 16.3244209289551 2.0
-7.8238749469 -5.6078667640686 3.0
9.7404976863 10.6365346908569 4.0
24.28693379 23.1924362182617 5.0
34.027431486 33.4773254394531 6.0
34.027431486 33.6838340759277 7.0
41.154171382 32.9696846008301 0.0
20.969763 16.3660869598389 1.0
20.184408383 17.3716316223145 2.0
-9.8031418559 -5.98105239868164 3.0
10.381266519 11.0965089797974 4.0
30.772904865 24.0848789215088 5.0
41.154171391 33.1912269592285 6.0
41.154171391 33.9372596740723 7.0
39.9308264078757 34.7917900085449 0.0
21.2554356118761 16.7997760772705 1.0
18.6753908028755 16.4418354034424 2.0
-8.41978314337461 -4.44866561889648 3.0
10.2556076588756 10.3902788162231 4.0
29.6752187558758 22.5502853393555 5.0
39.930826408 33.4311637878418 6.0
39.930826408 34.0292854309082 7.0
45.969703622 39.0343132019043 0.0
25.855804141 18.6205749511719 1.0
20.113899483 14.3028011322021 2.0
-4.729541314 -4.52857732772827 3.0
15.38435816 13.9168071746826 4.0
30.585345464 25.0476341247559 5.0
45.969703631 35.1555099487305 6.0
45.969703631 33.854362487793 7.0
38.398120212 35.7122955322266 0.0
19.546425793 14.4421415328979 1.0
18.851694421 13.7068490982056 2.0
-9.8922695857 -4.79232788085938 3.0
8.9594248259 9.04369831085205 4.0
29.438695387 22.3074417114258 5.0
38.398120221 33.6870079040527 6.0
38.398120221 34.1084938049316 7.0
28.366017296 33.6943626403809 0.0
14.995271371 17.711971282959 1.0
13.370745924 18.1377258300781 2.0
-4.4280592982 -3.50635194778442 3.0
8.9426866162 10.7241020202637 4.0
19.42333068 18.8238830566406 5.0
28.366017306 32.3749046325684 6.0
28.366017306 33.9811019897461 7.0
39.07981897 32.7057304382324 0.0
18.202914308 15.5769052505493 1.0
20.876904662 18.3167839050293 2.0
-6.6281228283 -5.98584270477295 3.0
14.248781824 13.5339803695679 4.0
24.831037146 23.3458042144775 5.0
39.079818979 32.7677307128906 6.0
39.079818979 34.3552284240723 7.0
40.466329374 34.6474609375 0.0
19.842708002 17.3780384063721 1.0
20.623621372 18.9195003509521 2.0
-6.0436252005 -5.56633806228638 3.0
14.579996162 13.9766311645508 4.0
25.886333212 23.3412113189697 5.0
40.466329383 33.7318725585938 6.0
40.466329383 34.3299827575684 7.0
38.293116843 30.7743606567383 0.0
18.225197444 16.9223785400391 1.0
20.067919399 19.1121959686279 2.0
-4.6459628917 -5.563232421875 3.0
15.421956498 14.675347328186 4.0
22.871160346 21.9922332763672 5.0
38.293116853 32.9756622314453 6.0
38.293116853 33.3420562744141 7.0
43.346089487 34.0577201843262 0.0
21.23189324 16.0987129211426 1.0
22.114196248 18.6418685913086 2.0
-9.387268107 -6.62924480438232 3.0
12.726928131 11.7883253097534 4.0
30.619161357 26.7002048492432 5.0
43.346089497 33.1781311035156 6.0
43.346089497 34.3544502258301 7.0
34.547871144 36.2010841369629 0.0
19.900789633 18.1902008056641 1.0
14.647081512 16.2540245056152 2.0
-6.6924528176 -4.76543617248535 3.0
7.9546286844 9.53922843933105 4.0
26.59324246 24.8578338623047 5.0
34.547871154 34.5382957458496 6.0
34.547871154 33.6504592895508 7.0
41.927050143208 29.9999866485596 0.0
21.8340017142739 14.3561973571777 1.0
20.0930484299516 14.6567211151123 2.0
-12.0156472013624 -6.91987419128418 3.0
8.07740122783537 9.29810428619385 4.0
33.8496486756856 28.5373649597168 5.0
41.927050143 32.07080078125 6.0
41.927050143 33.8388175964355 7.0
44.548236367 33.7439002990723 0.0
22.097953544 16.5122013092041 1.0
22.450282823 17.4332752227783 2.0
-6.9116853021 -5.37101984024048 3.0
15.538597511 14.524525642395 4.0
29.009638856 23.7815818786621 5.0
44.548236377 33.4744300842285 6.0
44.548236377 34.2499809265137 7.0
34.958415477 39.8624038696289 0.0
19.046265323 17.5277366638184 1.0
15.912150156 14.9357757568359 2.0
-3.947719982 -3.27914762496948 3.0
11.964430164 11.2095041275024 4.0
22.993985314 20.0630950927734 5.0
34.958415487 36.7505569458008 6.0
34.958415487 33.9701080322266 7.0
41.619773288 37.7965507507324 0.0
18.036088777 14.4248018264771 1.0
23.583684511 18.0355606079102 2.0
-9.1140217471 -6.11025905609131 3.0
14.469662754 14.0260276794434 4.0
27.150110534 22.4116973876953 5.0
41.619773298 34.0704154968262 6.0
41.619773298 33.6299285888672 7.0
35.768623904 27.0737323760986 0.0
18.854929214 16.4241313934326 1.0
16.913694692 16.6487045288086 2.0
-6.8203969536 -5.282395362854 3.0
10.093297731 10.6750040054321 4.0
25.675326176 23.7048034667969 5.0
35.768623912 32.0413131713867 6.0
35.768623912 33.6462287902832 7.0
36.035873071 36.4449501037598 0.0
18.197202866 15.7491416931152 1.0
17.838670207 17.7060317993164 2.0
-9.8802007315 -5.47710132598877 3.0
7.958469467 9.86042976379395 4.0
28.077403606 23.0174045562744 5.0
36.03587308 34.023494720459 6.0
36.03587308 33.7134628295898 7.0
42.671159575 34.6279754638672 0.0
22.542382477 16.0920715332031 1.0
20.128777099 15.4218769073486 2.0
-9.3171954196 -5.54172229766846 3.0
10.811581671 10.7319993972778 4.0
31.859577906 24.2742137908936 5.0
42.671159584 33.5098495483398 6.0
42.671159584 33.8190841674805 7.0
32.270518381 35.8649711608887 0.0
16.341237488 15.9692420959473 1.0
15.929280893 17.1944160461426 2.0
-7.2560391874 -4.57088851928711 3.0
8.6732416955 9.99343013763428 4.0
23.597276685 20.6557102203369 5.0
32.270518391 34.0090751647949 6.0
32.270518391 33.5936470031738 7.0
36.239834932 38.6628913879395 0.0
18.511196273 14.6105813980103 1.0
17.72863866 14.1837491989136 2.0
-5.3875241311 -5.53542184829712 3.0
12.34111452 11.9014883041382 4.0
23.898720413 22.5032482147217 5.0
36.239834941 34.7720108032227 6.0
36.239834941 33.7437934875488 7.0
38.292005172 33.0004615783691 0.0
20.144840979 16.3695659637451 1.0
18.147164195 15.6347694396973 2.0
-6.9715526507 -5.55878210067749 3.0
11.175611535 10.7335519790649 4.0
27.116393638 24.2516593933105 5.0
38.292005181 32.8899574279785 6.0
38.292005181 34.4687881469727 7.0
41.142171116 36.0498466491699 0.0
21.697070846 18.2903289794922 1.0
19.44510027 16.3563594818115 2.0
-5.49061917 -4.04444169998169 3.0
13.95448109 12.5386581420898 4.0
27.187690026 21.5255126953125 5.0
41.142171126 33.6533699035645 6.0
41.142171126 33.5538520812988 7.0
30.660234741 33.4324073791504 0.0
16.571808951 17.8888645172119 1.0
14.08842579 14.9511651992798 2.0
-3.6405774968 -2.9554238319397 3.0
10.447848283 10.4669561386108 4.0
20.212386458 19.7331790924072 5.0
30.660234751 32.7624206542969 6.0
30.660234751 34.313117980957 7.0
42.776716976 34.1352767944336 0.0
18.943122536 14.5784473419189 1.0
23.83359444 17.823371887207 2.0
-9.5992659914 -6.31765127182007 3.0
14.234328438 12.9270210266113 4.0
28.542388537 25.2562255859375 5.0
42.776716986 33.5068435668945 6.0
42.776716986 33.3544120788574 7.0
39.136657948 40.0871772766113 0.0
21.244730642 15.9835863113403 1.0
17.891927309 13.5875654220581 2.0
-5.2069844714 -5.5684552192688 3.0
12.684942831 10.7627792358398 4.0
26.45171512 24.9780826568604 5.0
39.136657955 36.3677597045898 6.0
39.136657955 33.7870903015137 7.0
40.593075656 37.2384567260742 0.0
19.181149666 14.5114803314209 1.0
21.411925992 17.8483943939209 2.0
-11.788434371 -6.65279674530029 3.0
9.6234916129 11.0213146209717 4.0
30.969584045 27.0798034667969 5.0
40.593075664 34.1673126220703 6.0
40.593075664 33.7886390686035 7.0
38.455960047 39.9425239562988 0.0
19.606731652 14.771222114563 1.0
18.849228395 14.5472440719604 2.0
-10.066988594 -5.80388355255127 3.0
8.7822397903 9.74806118011475 4.0
29.673720256 24.9157867431641 5.0
38.455960057 35.7733688354492 6.0
38.455960057 33.4196395874023 7.0
40.029822299 31.3458995819092 0.0
19.696760735 15.1650342941284 1.0
20.333061566 16.1585559844971 2.0
-6.9313004298 -6.32879018783569 3.0
13.401761128 12.6146793365479 4.0
26.628061173 24.3551712036133 5.0
40.029822307 32.387809753418 6.0
40.029822307 34.5062561035156 7.0
39.721089796 31.8825187683105 0.0
21.982867862 17.1625366210938 1.0
17.738221933 14.3313140869141 2.0
-6.2167396454 -4.54512214660645 3.0
11.521482278 11.4670495986938 4.0
28.199607518 22.1923065185547 5.0
39.721089806 32.573902130127 6.0
39.721089806 33.4351997375488 7.0
46.012802771 34.8700218200684 0.0
23.852238991 15.0699338912964 1.0
22.16056378 13.7125539779663 2.0
-9.7843125098 -6.33913946151733 3.0
12.37625126 11.4410648345947 4.0
33.636551511 28.5262031555176 5.0
46.012802781 33.4905891418457 6.0
46.012802781 33.4870414733887 7.0
43.79104115 32.7663421630859 0.0
23.810489927 16.6242256164551 1.0
19.980551224 16.8845596313477 2.0
-8.8972279263 -5.52855825424194 3.0
11.083323289 11.4544820785522 4.0
32.707717862 24.4528713226318 5.0
43.791041158 32.3791198730469 6.0
43.791041158 34.2568244934082 7.0
31.257332414 29.6199493408203 0.0
14.677686479 15.3182153701782 1.0
16.579645934 19.3999881744385 2.0
-7.7653359569 -4.60519552230835 3.0
8.8143099674 10.8893814086914 4.0
22.443022446 20.5452003479004 5.0
31.257332424 32.6313209533691 6.0
31.257332424 33.573429107666 7.0
38.988472902 28.4503135681152 0.0
18.706030748 18.2402591705322 1.0
20.282442155 19.9886455535889 2.0
-4.4931154738 -4.91170501708984 3.0
15.789326673 15.0027542114258 4.0
23.19914623 22.1706581115723 5.0
38.98847291 31.3870258331299 6.0
38.98847291 34.5064888000488 7.0
38.691218489 37.5108032226562 0.0
19.181746937 17.6422061920166 1.0
19.509471553 17.4086608886719 2.0
-4.305791265 -4.12048006057739 3.0
15.203680278 13.692759513855 4.0
23.487538212 20.2782878875732 5.0
38.691218499 35.2638931274414 6.0
38.691218499 33.6208610534668 7.0
39.033211962 32.8992652893066 0.0
21.751284426 18.5744209289551 1.0
17.281927536 16.0493927001953 2.0
-7.1228573229 -4.97467374801636 3.0
10.159070204 10.8500719070435 4.0
28.874141758 25.0282516479492 5.0
39.033211971 32.7217102050781 6.0
39.033211971 33.8525466918945 7.0
37.697547803 30.9217224121094 0.0
18.641388049 16.8474349975586 1.0
19.056159754 19.5434856414795 2.0
-6.2754402376 -6.20063161849976 3.0
12.780719506 12.9490327835083 4.0
24.916828297 23.2609214782715 5.0
37.697547813 32.0891990661621 6.0
37.697547813 33.7926826477051 7.0
35.277541329 39.9620018005371 0.0
16.862419132 16.4343147277832 1.0
18.415122198 18.2758178710938 2.0
-6.2131742864 -4.32426500320435 3.0
12.201947903 12.0800933837891 4.0
23.075593428 19.8888397216797 5.0
35.277541339 36.082389831543 6.0
35.277541339 34.1523857116699 7.0
36.1197639664031 39.0108299255371 0.0
19.1517629624013 17.9280033111572 1.0
16.9680010143999 15.5199060440063 2.0
-0.473187581125011 -3.60655927658081 3.0
16.4948134334014 14.8745765686035 4.0
19.6249505434022 20.0928268432617 5.0
36.119763966 36.0904350280762 6.0
36.119763966 33.9307746887207 7.0
33.14490304 38.7356147766113 0.0
16.831426445 17.7708778381348 1.0
16.313476595 17.028507232666 2.0
-3.637680404 -3.42523193359375 3.0
12.675796181 12.3088283538818 4.0
20.469106859 19.7471542358398 5.0
33.14490305 35.5352172851562 6.0
33.14490305 33.4526100158691 7.0
34.4868008142505 28.3507232666016 0.0
18.3228058422505 17.0216941833496 1.0
16.1639949822505 17.6927928924561 2.0
-8.00476657755046 -4.12974834442139 3.0
8.15922840445046 9.57654857635498 4.0
26.3275723771958 21.4073028564453 5.0
34.486800814 31.9792098999023 6.0
34.486800814 33.944278717041 7.0
40.933468199 36.6454429626465 0.0
20.889274427 18.5637722015381 1.0
20.044193774 19.1471652984619 2.0
-6.533500242 -5.84884262084961 3.0
13.510693524 13.5337324142456 4.0
27.422774677 25.3338356018066 5.0
40.933468207 34.9510498046875 6.0
40.933468207 33.4392395019531 7.0
35.993417919 33.4864730834961 0.0
18.964188614 18.395170211792 1.0
17.029229306 19.3384475708008 2.0
-7.4575389686 -4.76598453521729 3.0
9.5716903284 10.8043489456177 4.0
26.421727592 22.0974216461182 5.0
35.993417928 33.0371170043945 6.0
35.993417928 34.3722610473633 7.0
39.331666914 34.5442504882812 0.0
19.408314017 15.0600910186768 1.0
19.923352898 15.7059650421143 2.0
-7.2428639103 -5.82702159881592 3.0
12.680488979 12.0748987197876 4.0
26.651177937 24.4614677429199 5.0
39.331666923 33.2901268005371 6.0
39.331666923 33.9469680786133 7.0
37.559548486 36.5912666320801 0.0
18.054642609 15.1838779449463 1.0
19.504905877 17.8679122924805 2.0
-8.2806473226 -5.72464323043823 3.0
11.224258545 11.8082818984985 4.0
26.335289942 23.0424404144287 5.0
37.559548496 34.739616394043 6.0
37.559548496 33.379508972168 7.0
41.796902472 38.3743782043457 0.0
20.074580702 14.6620531082153 1.0
21.72232177 17.376579284668 2.0
-11.24258785 -6.97085857391357 3.0
10.479733911 11.5682554244995 4.0
31.317168562 26.5456924438477 5.0
41.796902482 35.6853904724121 6.0
41.796902482 34.0098915100098 7.0
35.679590813 30.8786315917969 0.0
15.932888769 14.8893165588379 1.0
19.746702044 18.8404712677002 2.0
-7.4161334139 -5.7782769203186 3.0
12.330568621 12.5093393325806 4.0
23.349022193 22.068395614624 5.0
35.679590823 32.9843864440918 6.0
35.679590823 34.0761871337891 7.0
33.227547284 28.9948596954346 0.0
19.156810973 17.8572063446045 1.0
14.070736312 14.8906450271606 2.0
-5.2205843651 -4.04523801803589 3.0
8.8501519384 9.96851634979248 4.0
24.377395346 23.0069808959961 5.0
33.227547292 30.6198120117188 6.0
33.227547292 33.9502754211426 7.0
28.00807173 38.253101348877 0.0
12.909757219 14.0942010879517 1.0
15.098314511 15.3083391189575 2.0
-4.3989930962 -4.76540422439575 3.0
10.699321405 11.0566282272339 4.0
17.308750325 18.5778408050537 5.0
28.008071739 34.7386589050293 6.0
28.008071739 33.6183052062988 7.0
33.4784988410636 32.1818313598633 0.0
17.4596045520637 15.0923652648926 1.0
16.0188942990637 14.7895822525024 2.0
-6.65519022176347 -4.76933193206787 3.0
9.36370407706344 9.87860298156738 4.0
24.1147947730637 21.8954296112061 5.0
33.478498841 32.7226715087891 6.0
33.478498841 33.5924987792969 7.0
33.1229468201162 39.224536895752 0.0
16.2738036511163 14.9222793579102 1.0
16.8491431771162 17.8069133758545 2.0
-8.48238354211593 -5.34876871109009 3.0
8.36675963501607 10.0562324523926 4.0
24.7561871931169 22.5764846801758 5.0
33.12294682 36.3618278503418 6.0
33.12294682 34.1277503967285 7.0
34.970228375 31.4226665496826 0.0
17.461716438 17.2542343139648 1.0
17.508511937 16.6972236633301 2.0
-3.5027478521 -3.99958896636963 3.0
14.005764076 12.8190765380859 4.0
20.964464299 19.609203338623 5.0
34.970228384 32.6414489746094 6.0
34.970228384 33.3855361938477 7.0
36.637966033 32.3264083862305 0.0
18.611959282 17.5505294799805 1.0
18.026006753 17.3434429168701 2.0
-5.1789935919 -4.95403337478638 3.0
12.847013153 11.9306678771973 4.0
23.790952882 22.6047439575195 5.0
36.637966041 32.9171028137207 6.0
36.637966041 33.9620323181152 7.0
38.712447285 30.4273014068604 0.0
19.874849154 15.414478302002 1.0
18.837598131 17.0074291229248 2.0
-9.656315418 -5.52200031280518 3.0
9.181282703 9.95587730407715 4.0
29.531164582 24.6537094116211 5.0
38.712447295 31.4977569580078 6.0
38.712447295 34.9917755126953 7.0
29.0840224199388 40.1629829406738 0.0
12.9348822379388 14.5954084396362 1.0
16.1491401919388 18.2351722717285 2.0
-6.2158408037388 -4.20246267318726 3.0
9.93329938853882 10.4630289077759 4.0
19.1507230409388 18.4637298583984 5.0
29.08402242 36.8360786437988 6.0
29.08402242 33.5546112060547 7.0
40.7419030105752 32.6605377197266 0.0
19.6019574718567 14.8246717453003 1.0
21.139945535663 16.8461246490479 2.0
-8.29196876971742 -6.5610818862915 3.0
12.8479767652648 12.6114416122437 4.0
27.8939262422712 25.0109806060791 5.0
40.741903011 33.4219436645508 6.0
40.741903011 33.857048034668 7.0
44.4239279780974 28.5529384613037 0.0
22.6860882760974 16.5444011688232 1.0
21.7378397120974 15.807954788208 2.0
-6.03315836089742 -5.46750497817993 3.0
15.7046813510974 13.9141321182251 4.0
28.7192466370973 23.7347526550293 5.0
44.423927978 31.1482124328613 6.0
44.423927978 34.391960144043 7.0
37.279929002 35.9212379455566 0.0
17.697440733 14.7100296020508 1.0
19.582488269 17.1379089355469 2.0
-8.4860638308 -5.97331953048706 3.0
11.096424428 11.0992374420166 4.0
26.183504574 23.4439754486084 5.0
37.279929011 33.8379325866699 6.0
37.279929011 33.6763000488281 7.0
39.065196556 33.1850624084473 0.0
22.911627527 17.7384967803955 1.0
16.153569029 14.1638126373291 2.0
-7.035333012 -5.62458372116089 3.0
9.1182360067 9.61914157867432 4.0
29.946960549 25.9882946014404 5.0
39.065196566 32.7220268249512 6.0
39.065196566 33.6187057495117 7.0
34.346712901 32.0131988525391 0.0
17.419062718 17.4134731292725 1.0
16.927650183 17.3174476623535 2.0
-3.8964434363 -4.41597080230713 3.0
13.031206737 12.2954778671265 4.0
21.315506165 20.8608951568604 5.0
34.346712911 32.4645690917969 6.0
34.346712911 34.0852699279785 7.0
44.832836192 30.2007999420166 0.0
22.86238329 15.5851564407349 1.0
21.970452902 17.3175163269043 2.0
-8.0950157692 -6.61132574081421 3.0
13.875437123 12.9708089828491 4.0
30.957399069 26.4663581848145 5.0
44.832836202 32.3761444091797 6.0
44.832836202 34.3158683776855 7.0
37.693005907 36.3005485534668 0.0
18.767233493 17.0991134643555 1.0
18.925772415 17.062536239624 2.0
-4.4039460583 -4.94063138961792 3.0
14.521826348 13.3485736846924 4.0
23.17117956 22.691987991333 5.0
37.693005916 34.4279823303223 6.0
37.693005916 33.5827522277832 7.0
35.516030196 34.7728958129883 0.0
17.238388415 15.6741094589233 1.0
18.277641781 17.3056373596191 2.0
-5.9879556024 -5.22831249237061 3.0
12.289686168 12.0210485458374 4.0
23.226344027 22.1016712188721 5.0
35.516030206 34.0054397583008 6.0
35.516030206 34.1200561523438 7.0
45.449957515 40.2533226013184 0.0
24.470260824 17.8115139007568 1.0
20.979696692 15.8422803878784 2.0
-8.6547786371 -5.66350841522217 3.0
12.324918046 11.4776582717896 4.0
33.12503947 26.0830593109131 5.0
45.449957523 36.7085876464844 6.0
45.449957523 33.5790138244629 7.0
36.522103961 31.1787624359131 0.0
17.787809101 14.1589765548706 1.0
18.734294862 13.7739267349243 2.0
-3.7955939001 -5.3304967880249 3.0
14.938700953 13.1843013763428 4.0
21.583403009 20.919994354248 5.0
36.52210397 32.4412727355957 6.0
36.52210397 34.0383949279785 7.0
40.809333824 35.7279815673828 0.0
21.825363738 17.5462684631348 1.0
18.983970087 14.7769727706909 2.0
-4.5096782574 -4.30666446685791 3.0
14.474291821 13.0011587142944 4.0
26.335042004 21.4207363128662 5.0
40.809333833 34.2702865600586 6.0
40.809333833 34.200870513916 7.0
43.708815015 30.2276306152344 0.0
22.06122806 16.0239295959473 1.0
21.647586954 16.5129222869873 2.0
-7.2998869135 -6.46701431274414 3.0
14.347700031 13.1712055206299 4.0
29.361114984 25.8913230895996 5.0
43.708815025 32.9534568786621 6.0
43.708815025 33.4725151062012 7.0
35.746307566 29.1046829223633 0.0
17.418825391 15.3805999755859 1.0
18.327482175 16.642578125 2.0
-8.2684434585 -4.98274278640747 3.0
10.059038708 10.7142553329468 4.0
25.687268859 21.8581256866455 5.0
35.746307575 31.9232406616211 6.0
35.746307575 34.8097610473633 7.0
32.565260397 28.8182735443115 0.0
15.347203209 14.8767175674438 1.0
17.218057188 16.1710433959961 2.0
-4.4968959364 -4.76060247421265 3.0
12.721161242 11.7391157150269 4.0
19.844099155 20.3442401885986 5.0
32.565260407 32.0624771118164 6.0
32.565260407 33.8246612548828 7.0
37.459437559 29.4524707794189 0.0
20.974106733 17.9382209777832 1.0
16.485330826 17.8883972167969 2.0
-7.8025517563 -4.73602628707886 3.0
8.6827790601 9.79353713989258 4.0
28.776658499 24.0733985900879 5.0
37.459437569 32.0814552307129 6.0
37.459437569 33.6031112670898 7.0
41.6008687772188 33.4326972961426 0.0
18.8099504026802 15.2521409988403 1.0
22.7909183777866 18.2038059234619 2.0
-6.50941037372392 -5.90672540664673 3.0
16.2815080036091 14.779956817627 4.0
25.319360776445 23.876256942749 5.0
41.600868777 33.14306640625 6.0
41.600868777 34.9054260253906 7.0
43.08864828 32.5326385498047 0.0
20.75366021 15.2491083145142 1.0
22.334988069 17.8439865112305 2.0
-9.5492158301 -6.69641304016113 3.0
12.785772229 12.5383834838867 4.0
30.30287605 28.1021156311035 5.0
43.088648289 33.1035385131836 6.0
43.088648289 34.0511627197266 7.0
30.268152668 33.559009552002 0.0
15.014233699 16.8845119476318 1.0
15.25391897 18.9466438293457 2.0
-5.5931438946 -3.97121715545654 3.0
9.6607750663 10.9323139190674 4.0
20.607377602 20.2941455841064 5.0
30.268152677 33.2887573242188 6.0
30.268152677 33.7446632385254 7.0
38.454045879 31.5580921173096 0.0
17.159579151 14.8796195983887 1.0
21.29446673 18.0686187744141 2.0
-7.5807931052 -5.12992477416992 3.0
13.713673616 12.9502143859863 4.0
24.740372265 21.3589878082275 5.0
38.454045888 32.4393424987793 6.0
38.454045888 34.9821090698242 7.0
36.654056854 32.3131713867188 0.0
19.439010257 17.0911121368408 1.0
17.215046598 18.0765914916992 2.0
-7.9864590963 -5.99386024475098 3.0
9.2285874923 10.708854675293 4.0
27.425469363 24.8820209503174 5.0
36.654056864 32.7098045349121 6.0
36.654056864 34.3872489929199 7.0
36.990278119 36.7476043701172 0.0
18.644416453 16.268180847168 1.0
18.345861667 18.5864067077637 2.0
-7.9977544937 -5.61820697784424 3.0
10.348107164 11.5107221603394 4.0
26.642170956 23.7674541473389 5.0
36.990278128 34.3084487915039 6.0
36.990278128 33.9188423156738 7.0
39.45539178 35.1009140014648 0.0
17.953554537 15.691291809082 1.0
21.501837243 19.787036895752 2.0
-6.7799927019 -5.63482332229614 3.0
14.721844531 14.1777324676514 4.0
24.733547249 22.2421054840088 5.0
39.45539179 33.8705101013184 6.0
39.45539179 33.9840202331543 7.0
41.490098301 37.9474296569824 0.0
18.418139969 15.2274208068848 1.0
23.071958334 18.5535430908203 2.0
-6.7020850264 -5.80785655975342 3.0
16.369873299 15.2629051208496 4.0
25.120225004 23.3498077392578 5.0
41.49009831 35.2166175842285 6.0
41.49009831 33.9287300109863 7.0
42.195848756 36.4782104492188 0.0
18.431626231 14.4744338989258 1.0
23.764222526 18.1458988189697 2.0
-9.6629820221 -6.45398807525635 3.0
14.101240496 13.5419330596924 4.0
28.094608262 25.5584888458252 5.0
42.195848764 34.5382614135742 6.0
42.195848764 33.205249786377 7.0
36.580423947 34.2167167663574 0.0
18.088597645 14.1087646484375 1.0
18.491826302 15.0872240066528 2.0
-10.331066357 -6.87337350845337 3.0
8.1607599346 9.49189567565918 4.0
28.419664012 27.5381679534912 5.0
36.580423956 33.3945503234863 6.0
36.580423956 34.1883544921875 7.0
41.112612837 39.7242202758789 0.0
21.264748673 14.7340831756592 1.0
19.847864165 13.8649501800537 2.0
-7.8539892402 -6.27590703964233 3.0
11.993874916 11.4767208099365 4.0
29.118737922 25.5770301818848 5.0
41.112612846 36.133659362793 6.0
41.112612846 33.7864608764648 7.0
41.165032295 29.8299751281738 0.0
20.777866565 15.9167327880859 1.0
20.387165729 19.0696430206299 2.0
-11.013080697 -6.79234027862549 3.0
9.3740850227 11.3638954162598 4.0
31.790947272 26.7329597473145 5.0
41.165032305 31.6694927215576 6.0
41.165032305 33.9589576721191 7.0
28.23632976 34.5033302307129 0.0
14.450875753 14.4585390090942 1.0
13.785454008 13.8888349533081 2.0
-5.4067157656 -3.7414436340332 3.0
8.3787382334 9.10614967346191 4.0
19.857591527 18.8974151611328 5.0
28.236329769 33.881519317627 6.0
28.236329769 33.8169174194336 7.0
46.563722225 37.3382453918457 0.0
24.514076121 18.5140075683594 1.0
22.049646104 17.5975379943848 2.0
-6.9458736065 -5.20617198944092 3.0
15.103772488 14.2432203292847 4.0
31.459949737 25.3757553100586 5.0
46.563722235 33.6378364562988 6.0
46.563722235 34.4281349182129 7.0
38.976039438 36.937183380127 0.0
20.47408062 17.7791309356689 1.0
18.50195882 15.8748273849487 2.0
-3.6791774969 -4.44254970550537 3.0
14.822781314 13.5535831451416 4.0
24.153258125 21.3788623809814 5.0
38.976039446 33.1423492431641 6.0
38.976039446 34.5613212585449 7.0
43.902220155 33.3351821899414 0.0
23.085534006 18.2256126403809 1.0
20.816686149 16.3101100921631 2.0
-6.0778252495 -5.64106750488281 3.0
14.738860889 13.3457183837891 4.0
29.163359266 26.2797260284424 5.0
43.902220165 33.1784591674805 6.0
43.902220165 34.0666084289551 7.0
37.104870377 32.314811706543 0.0
21.928667997 18.0384635925293 1.0
15.176202381 13.9891500473022 2.0
-7.1829261037 -4.93001461029053 3.0
7.9932762681 8.18487453460693 4.0
29.11159411 25.0108985900879 5.0
37.104870385 32.7025680541992 6.0
37.104870385 33.8229789733887 7.0
48.094555135036 30.3692626953125 0.0
24.9788607525581 15.576943397522 1.0
23.1156943831877 14.5499401092529 2.0
-7.58647576369851 -6.44797611236572 3.0
15.5292186191899 13.8556184768677 4.0
32.5653365171039 27.3884525299072 5.0
48.094555135 31.5331478118896 6.0
48.094555135 34.6664657592773 7.0
39.863550950908 30.1757316589355 0.0
20.2748443148297 14.6929121017456 1.0
19.5887066378693 15.9700212478638 2.0
-9.95882058391343 -6.65273571014404 3.0
9.62988605424657 10.5360240936279 4.0
30.233664897769 26.7016105651855 5.0
39.863550951 31.6023063659668 6.0
39.863550951 35.0910377502441 7.0
38.158424696 38.5724716186523 0.0
21.093616444 17.7969169616699 1.0
17.064808252 15.4447937011719 2.0
-6.6865614213 -5.02300548553467 3.0
10.378246821 10.7096462249756 4.0
27.780177875 24.0138607025146 5.0
38.158424706 35.6219215393066 6.0
38.158424706 33.7730979919434 7.0
37.671552635 34.4605522155762 0.0
18.060105754 15.5561084747314 1.0
19.611446882 17.3783626556396 2.0
-6.6868904195 -5.42533540725708 3.0
12.924556453 12.5165777206421 4.0
24.746996183 22.4141807556152 5.0
37.671552644 33.2297515869141 6.0
37.671552644 34.1230010986328 7.0
38.9900068701748 33.9290466308594 0.0
19.4317892501748 14.5261182785034 1.0
19.5582176301748 13.936861038208 2.0
-6.7795985865748 -4.8604679107666 3.0
12.7786190431748 11.6314640045166 4.0
26.2113877989032 22.3779602050781 5.0
38.99000687 33.3342933654785 6.0
38.99000687 33.7502670288086 7.0
32.824279187 36.343204498291 0.0
17.524252685 16.4263515472412 1.0
15.300026502 15.3409729003906 2.0
-5.6991008156 -3.21320199966431 3.0
9.6009256767 9.49741649627686 4.0
23.223353511 20.1237773895264 5.0
32.824279197 35.0719566345215 6.0
32.824279197 34.8254890441895 7.0
39.752392388 38.9968643188477 0.0
19.337645943 17.7930164337158 1.0
20.414746445 18.5093593597412 2.0
-4.1151033929 -4.85235118865967 3.0
16.299643042 15.5652389526367 4.0
23.452749346 21.1705684661865 5.0
39.752392398 36.359619140625 6.0
39.752392398 34.3378219604492 7.0
36.0149979881443 36.086353302002 0.0
19.5282530962186 15.7776260375977 1.0
16.4867448931817 13.7107639312744 2.0
-5.97886461861353 -4.3887186050415 3.0
10.5078802751384 9.74533557891846 4.0
25.5071177152005 23.1179180145264 5.0
36.014997988 34.174674987793 6.0
36.014997988 34.1222610473633 7.0
42.307205596 37.2264099121094 0.0
21.688253633 15.6362781524658 1.0
20.618951963 14.7072410583496 2.0
-4.4760898368 -5.53218936920166 3.0
16.142862116 14.0200223922729 4.0
26.164343479 23.4035453796387 5.0
42.307205606 35.0225372314453 6.0
42.307205606 34.0239143371582 7.0
41.624369925 40.249397277832 0.0
21.269456313 15.7261657714844 1.0
20.354913614 17.38498878479 2.0
-9.726585677 -6.77957916259766 3.0
10.628327928 10.6036205291748 4.0
30.996041999 27.4033966064453 5.0
41.624369934 36.5464744567871 6.0
41.624369934 33.8647117614746 7.0
40.927860916 26.5889072418213 0.0
20.558611306 15.6112546920776 1.0
20.36924961 15.7320537567139 2.0
-7.6689328495 -5.67715072631836 3.0
12.700316751 12.2908725738525 4.0
28.227544165 23.5262889862061 5.0
40.927860926 31.5921306610107 6.0
40.927860926 33.9417572021484 7.0
47.441321803 37.3379364013672 0.0
22.43048805 15.566294670105 1.0
25.010833753 17.9787292480469 2.0
-8.9760978683 -6.23886871337891 3.0
16.034735875 14.5223569869995 4.0
31.406585928 27.6603813171387 5.0
47.441321812 33.9703636169434 6.0
47.441321812 33.8244667053223 7.0
39.37950561 30.1233406066895 0.0
19.873167771 15.2402820587158 1.0
19.506337839 17.9895935058594 2.0
-9.8415530061 -6.92810201644897 3.0
9.6647848227 10.9526681900024 4.0
29.714720787 27.8779716491699 5.0
39.379505619 32.3208351135254 6.0
39.379505619 33.7721138000488 7.0
42.095557289 32.8036193847656 0.0
22.277305426 16.0657749176025 1.0
19.818251863 14.3993549346924 2.0
-7.6874055589 -4.80730056762695 3.0
12.130846294 10.752649307251 4.0
29.964710995 24.0853500366211 5.0
42.095557299 33.2439575195312 6.0
42.095557299 33.602970123291 7.0
35.151188218 39.2162818908691 0.0
17.118344744 14.6078157424927 1.0
18.032843477 14.3664665222168 2.0
-2.0060677241 -4.67451763153076 3.0
16.026775747 13.7841510772705 4.0
19.124412475 19.5671234130859 5.0
35.151188225 35.837158203125 6.0
35.151188225 33.3524208068848 7.0
37.713935362 33.484992980957 0.0
20.380527895 17.5011577606201 1.0
17.333407469 15.2515182495117 2.0
-5.6423114193 -3.79121398925781 3.0
11.691096041 11.1427154541016 4.0
26.022839323 22.067533493042 5.0
37.713935371 32.9220352172852 6.0
37.713935371 33.5929069519043 7.0
39.667743251 33.2733764648438 0.0
20.027544136 14.3628301620483 1.0
19.640199117 15.7252006530762 2.0
-10.376032212 -6.8212194442749 3.0
9.2641668955 9.95497894287109 4.0
30.403576357 27.1195335388184 5.0
39.66774326 33.1340751647949 6.0
39.66774326 34.1655616760254 7.0
41.240556184 31.9311065673828 0.0
23.347458936 17.7069454193115 1.0
17.893097248 15.0225629806519 2.0
-7.252142064 -4.81379890441895 3.0
10.640955174 10.8307151794434 4.0
30.59960101 25.4906272888184 5.0
41.240556194 32.499324798584 6.0
41.240556194 33.8688621520996 7.0
40.32019459 40.3508529663086 0.0
19.177093862 14.7271385192871 1.0
21.143100728 15.1514205932617 2.0
-7.2609061044 -5.49231100082397 3.0
13.882194613 12.4325494766235 4.0
26.437999977 23.711275100708 5.0
40.3201946 37.7375717163086 6.0
40.3201946 33.8464660644531 7.0
39.5611510585585 36.0947723388672 0.0
19.6094718095582 14.7123289108276 1.0
19.9516792585585 14.9341125488281 2.0
-8.44726993515885 -6.40233898162842 3.0
11.5044093235587 11.4890851974487 4.0
28.0567416170478 25.6889476776123 5.0
39.561151059 34.5058288574219 6.0
39.561151059 33.5384178161621 7.0
43.5664319470049 32.3591194152832 0.0
22.3670487340048 16.4541358947754 1.0
21.1993832230048 17.70729637146 2.0
-8.61702915780503 -6.58408308029175 3.0
12.582354065005 12.1015901565552 4.0
30.9840778920045 27.4613971710205 5.0
43.566431947 32.9539031982422 6.0
43.566431947 33.8710975646973 7.0
37.929366556 35.8098106384277 0.0
20.105495873 16.4435119628906 1.0
17.823870687 15.1579055786133 2.0
-5.9768423324 -4.84285020828247 3.0
11.847028349 11.2255382537842 4.0
26.082338211 23.8533592224121 5.0
37.929366562 34.0432395935059 6.0
37.929366562 34.4716720581055 7.0
39.7456686413092 33.7749862670898 0.0
20.6016083083092 17.7122020721436 1.0
19.1440603423092 16.2757892608643 2.0
-3.0966467648096 -4.73807001113892 3.0
16.0474135773093 14.3989696502686 4.0
23.6982550733092 22.4012355804443 5.0
39.745668641 32.6007194519043 6.0
39.745668641 33.6088676452637 7.0
37.1737946256367 36.7336540222168 0.0
18.9810950589955 18.1777839660645 1.0
18.1926995622865 18.8153057098389 2.0
-4.83110587561615 -4.47975206375122 3.0
13.3615936860386 13.1506357192993 4.0
23.8122009348139 21.3804302215576 5.0
37.173794625 34.2185478210449 6.0
37.173794625 33.8643074035645 7.0
27.124866007 37.5581474304199 0.0
13.623671484 16.4359798431396 1.0
13.501194525 17.4466743469238 2.0
-4.5998142153 -4.46372032165527 3.0
8.9013803006 10.7723789215088 4.0
18.223485707 19.307897567749 5.0
27.124866015 33.9291648864746 6.0
27.124866015 33.632884979248 7.0
27.411746895 34.284065246582 0.0
13.763683287 16.4949436187744 1.0
13.648063611 16.9026756286621 2.0
-4.2527245718 -4.28990077972412 3.0
9.3953390304 10.6491231918335 4.0
18.016407867 19.4089832305908 5.0
27.411746904 33.4172973632812 6.0
27.411746904 33.9163475036621 7.0
38.866617309 34.8531455993652 0.0
22.319478804 17.6653480529785 1.0
16.547138508 15.5631818771362 2.0
-7.9092899105 -6.17991495132446 3.0
8.6378485894 9.77838706970215 4.0
30.228768722 27.3012237548828 5.0
38.866617317 33.566032409668 6.0
38.866617317 34.162712097168 7.0
42.3830573535328 38.9791488647461 0.0
20.5951530026372 14.7203998565674 1.0
21.7879043479541 16.5209636688232 2.0
-9.3946766066605 -6.15777921676636 3.0
12.3932277407315 11.7748908996582 4.0
29.98982960914 25.8653736114502 5.0
42.383057354 34.337574005127 6.0
42.383057354 34.5035667419434 7.0
47.6437510358734 30.5624160766602 0.0
23.1576165238734 14.846941947937 1.0
24.4861345218735 15.205883026123 2.0
-8.74385685767346 -6.4019627571106 3.0
15.7422776638735 13.8125629425049 4.0
31.9014733818732 26.8260631561279 5.0
47.643751036 32.5336074829102 6.0
47.643751036 33.3897476196289 7.0
38.439399947 38.2271118164062 0.0
19.975660163 15.4457750320435 1.0
18.463739784 14.7129697799683 2.0
-6.4266906816 -5.27031183242798 3.0
12.037049093 11.5834083557129 4.0
26.402350854 23.9629249572754 5.0
38.439399957 35.038272857666 6.0
38.439399957 33.9093589782715 7.0
40.263562361 38.776424407959 0.0
20.42306719 14.4439849853516 1.0
19.840495171 16.7576026916504 2.0
-11.677045299 -6.94737577438354 3.0
8.1634498621 9.81933307647705 4.0
32.100112499 27.2353229522705 5.0
40.263562371 35.639949798584 6.0
40.263562371 34.028507232666 7.0
41.519528387 38.9613342285156 0.0
22.877320367 17.5879135131836 1.0
18.642208021 15.0204696655273 2.0
-5.2280057864 -5.53325891494751 3.0
13.414202225 12.3732271194458 4.0
28.105326163 25.1305046081543 5.0
41.519528397 34.9312210083008 6.0
41.519528397 34.1702117919922 7.0
40.414570464 30.4602699279785 0.0
17.791289494 14.8190383911133 1.0
22.623280971 17.4636821746826 2.0
-6.6442159501 -5.4814248085022 3.0
15.979065012 14.3355541229248 4.0
24.435505453 21.1775875091553 5.0
40.414570474 32.6667175292969 6.0
40.414570474 33.84228515625 7.0
37.834181215 33.745906829834 0.0
18.770560656 15.7620096206665 1.0
19.063620563 17.0179252624512 2.0
-7.0281211763 -5.85780906677246 3.0
12.035499382 11.9053421020508 4.0
25.798681838 23.8311386108398 5.0
37.834181221 33.289794921875 6.0
37.834181221 33.7737922668457 7.0
37.201121546 35.4700164794922 0.0
17.811506649 15.839147567749 1.0
19.389614897 17.7267322540283 2.0
-6.0990519868 -5.34365701675415 3.0
13.290562901 12.8441915512085 4.0
23.910558645 22.2484092712402 5.0
37.201121556 34.182186126709 6.0
37.201121556 33.9789581298828 7.0
42.944445132 34.661319732666 0.0
21.59677927 16.6653251647949 1.0
21.347665865 17.5307197570801 2.0
-6.9735110037 -5.54896450042725 3.0
14.374154853 13.3465070724487 4.0
28.570290281 22.9454460144043 5.0
42.94444514 33.7693481445312 6.0
42.94444514 33.6286087036133 7.0
47.3096000339538 37.7917900085449 0.0
24.7818404359387 18.3308849334717 1.0
22.5277596000527 16.9800891876221 2.0
-6.75460229154402 -4.84400796890259 3.0
15.7731573079881 14.420823097229 4.0
31.5364427279756 23.2513027191162 5.0
47.309600034 35.4986724853516 6.0
47.309600034 34.6284141540527 7.0
34.703660739 38.9877853393555 0.0
16.95243502 17.7534770965576 1.0
17.75122572 19.2276248931885 2.0
-4.712400117 -5.26353693008423 3.0
13.038825595 13.0158004760742 4.0
21.664835145 21.5326251983643 5.0
34.703660747 35.3610954284668 6.0
34.703660747 33.6574630737305 7.0
39.355679823 29.5324287414551 0.0
18.719972615 14.6597881317139 1.0
20.635707207 16.9175357818604 2.0
-8.5340541425 -6.76971864700317 3.0
12.101653055 12.1035509109497 4.0
27.254026768 25.0833549499512 5.0
39.355679833 31.6934795379639 6.0
39.355679833 34.1426162719727 7.0
36.760838125 36.8039512634277 0.0
18.098408908 16.7262477874756 1.0
18.662429217 17.1066246032715 2.0
-4.9002916105 -4.58986282348633 3.0
13.762137597 12.9724521636963 4.0
22.998700528 21.3568668365479 5.0
36.760838135 34.1451950073242 6.0
36.760838135 33.8447113037109 7.0
45.817767369632 30.9034633636475 0.0
23.4066861136316 18.373161315918 1.0
22.4110812656318 18.4657440185547 2.0
-7.99255969573223 -6.48250961303711 3.0
14.4185215696319 13.3270740509033 4.0
31.3992456710555 27.3955078125 5.0
45.817767369 31.6847362518311 6.0
45.817767369 34.2968673706055 7.0
37.456387934 35.2614707946777 0.0
18.714257418 14.8429546356201 1.0
18.742130516 14.9934062957764 2.0
-7.2626807055 -4.64943647384644 3.0
11.4794498 11.2107267379761 4.0
25.976938133 22.6378536224365 5.0
37.456387944 34.0852584838867 6.0
37.456387944 33.6291465759277 7.0
36.979299104 36.6040687561035 0.0
17.001867984 15.9858894348145 1.0
19.977431121 18.04443359375 2.0
-4.4589079828 -5.06448888778687 3.0
15.518523129 14.1072006225586 4.0
21.460775976 21.1046047210693 5.0
36.979299113 34.6591339111328 6.0
36.979299113 33.5069999694824 7.0
44.357392148 39.9372444152832 0.0
23.57148554 16.3552913665771 1.0
20.785906608 15.3686828613281 2.0
-6.1950554121 -5.78831481933594 3.0
14.590851187 13.1817636489868 4.0
29.766540962 25.7648735046387 5.0
44.357392157 36.1244163513184 6.0
44.357392157 34.1508827209473 7.0
36.8120792993302 39.3617172241211 0.0
19.8640110823302 16.4727058410645 1.0
16.9480682263302 17.6121635437012 2.0
-8.51793333553018 -5.98433256149292 3.0
8.43013489113018 10.3896703720093 4.0
28.381944393825 25.5141658782959 5.0
36.812079299 35.6206932067871 6.0
36.812079299 33.4481735229492 7.0
43.172254734 37.357349395752 0.0
21.994700089 15.0198335647583 1.0
21.177554646 14.2503929138184 2.0
-4.8220170211 -6.38405179977417 3.0
16.355537616 13.9477596282959 4.0
26.816717119 24.3528213500977 5.0
43.172254743 34.2628288269043 6.0
43.172254743 33.6219062805176 7.0
47.109370558 37.2593421936035 0.0
25.034746172 17.1536827087402 1.0
22.074624387 17.3862037658691 2.0
-8.7769581235 -6.22754430770874 3.0
13.297666255 12.4234571456909 4.0
33.811704304 28.0095310211182 5.0
47.109370567 34.1859359741211 6.0
47.109370567 33.7683410644531 7.0
35.871355653 33.7590408325195 0.0
18.076771786 14.9152307510376 1.0
17.794583868 16.0058479309082 2.0
-9.6137759762 -5.15521860122681 3.0
8.1808078825 9.68891906738281 4.0
27.690547772 22.2415256500244 5.0
35.871355662 33.251651763916 6.0
35.871355662 34.0987892150879 7.0
42.1694339111434 30.6852474212646 0.0
22.6030080131434 17.481071472168 1.0
19.5664259071434 16.7996406555176 2.0
-7.61628576964338 -5.03039789199829 3.0
11.9501401371434 10.964937210083 4.0
30.2192937831435 24.5740222930908 5.0
42.169433911 32.1740951538086 6.0
42.169433911 34.6803436279297 7.0
38.001249391 29.7541999816895 0.0
19.442391644 15.7402153015137 1.0
18.558857748 16.1113777160645 2.0
-7.4577746389 -4.95927667617798 3.0
11.101083099 10.3120670318604 4.0
26.900166292 23.6478538513184 5.0
38.001249401 31.5364112854004 6.0
38.001249401 33.7190742492676 7.0
36.026296144 30.5778656005859 0.0
18.544058535 18.2755317687988 1.0
17.482237609 19.1367435455322 2.0
-7.0335899837 -4.01062679290771 3.0
10.448647616 10.5138673782349 4.0
25.577648528 21.2138423919678 5.0
36.026296153 32.0607032775879 6.0
36.026296153 34.6325302124023 7.0
33.200213149 31.4931545257568 0.0
14.957525483 14.0463104248047 1.0
18.242687666 17.9665374755859 2.0
-8.613844647 -5.54456567764282 3.0
9.6288430093 11.0273027420044 4.0
23.57137014 22.3408164978027 5.0
33.200213159 33.3044052124023 6.0
33.200213159 33.5393562316895 7.0
36.881947947 36.4368057250977 0.0
19.760267083 17.5321025848389 1.0
17.12168087 16.7903442382812 2.0
-6.5510496361 -5.1974196434021 3.0
10.570631229 11.0930242538452 4.0
26.311316723 24.3980712890625 5.0
36.881947952 34.6589050292969 6.0
36.881947952 33.8932113647461 7.0
28.363012003 39.1016616821289 0.0
14.617095463 16.3095474243164 1.0
13.745916541 16.7786521911621 2.0
-5.3741086073 -3.39616680145264 3.0
8.3718079243 9.73065185546875 4.0
19.991204079 18.7839889526367 5.0
28.363012012 35.8746795654297 6.0
28.363012012 33.9172630310059 7.0
39.574467416 37.2412757873535 0.0
20.649388383 17.2866172790527 1.0
18.925079033 18.6895942687988 2.0
-8.4547238017 -6.70754718780518 3.0
10.470355222 10.9792242050171 4.0
29.104112194 28.2952728271484 5.0
39.574467426 35.2958374023438 6.0
39.574467426 34.854190826416 7.0
36.220401471 36.8761520385742 0.0
17.995668566 15.8578786849976 1.0
18.224732906 16.9647464752197 2.0
-6.9105415737 -5.42833518981934 3.0
11.314191323 11.4988069534302 4.0
24.906210149 23.1337547302246 5.0
36.22040148 35.2265586853027 6.0
36.22040148 34.234489440918 7.0
31.086432133 32.3937492370605 0.0
16.0910287 15.6626081466675 1.0
14.995403435 14.468282699585 2.0
-4.3439743018 -3.48329877853394 3.0
10.651429126 10.4081678390503 4.0
20.43500301 19.5909748077393 5.0
31.086432141 32.5100898742676 6.0
31.086432141 33.696216583252 7.0
36.274353053 29.4332294464111 0.0
18.420887917 16.2647113800049 1.0
17.853465137 15.5676584243774 2.0
-3.6202808126 -4.78589153289795 3.0
14.233184315 12.4327116012573 4.0
22.041168739 21.3143177032471 5.0
36.274353063 31.8765602111816 6.0
36.274353063 34.2067184448242 7.0
31.65795008 34.9971466064453 0.0
17.489254099 16.3881816864014 1.0
14.168695981 14.1239252090454 2.0
-6.0427493715 -3.7094554901123 3.0
8.1259465996 9.17247486114502 4.0
23.53200348 22.1719398498535 5.0
31.65795009 33.5798530578613 6.0
31.65795009 33.8901100158691 7.0
37.059171469 33.8685188293457 0.0
19.339994356 15.3735618591309 1.0
17.719177113 17.2028884887695 2.0
-9.4011683011 -5.4094090461731 3.0
8.318008802 10.0749807357788 4.0
28.741162667 23.0943431854248 5.0
37.059171479 33.0216522216797 6.0
37.059171479 33.9780693054199 7.0
45.308862348 38.6243019104004 0.0
24.95721568 18.0378837585449 1.0
20.351646669 16.8161697387695 2.0
-7.323343681 -6.45991897583008 3.0
13.028302978 12.3750219345093 4.0
32.28055937 27.5056114196777 5.0
45.308862357 34.8849029541016 6.0
45.308862357 33.3600196838379 7.0
32.988279899 35.3693046569824 0.0
17.965429309 17.3810367584229 1.0
15.02285059 14.3315572738647 2.0
-2.6394240327 -3.56156349182129 3.0
12.383426547 11.4781799316406 4.0
20.604853352 20.2009143829346 5.0
32.988279909 33.751293182373 6.0
32.988279909 34.1843948364258 7.0
41.812200584 39.2743721008301 0.0
21.127130287 15.3244962692261 1.0
20.685070296 14.5293712615967 2.0
-5.5457377677 -5.11329889297485 3.0
15.139332519 13.3164596557617 4.0
26.672868065 23.0563545227051 5.0
41.812200593 34.8459434509277 6.0
41.812200593 33.7607154846191 7.0
34.159577997 31.9714622497559 0.0
15.979883142 15.3058452606201 1.0
18.179694855 18.0326156616211 2.0
-6.0058060763 -5.37439012527466 3.0
12.173888768 11.9042339324951 4.0
21.985689228 21.8427829742432 5.0
34.159578007 32.5337142944336 6.0
34.159578007 34.8814697265625 7.0
41.353058194 35.1762161254883 0.0
20.665722502 17.067232131958 1.0
20.687335692 19.2010688781738 2.0
-9.525258586 -6.08874416351318 3.0
11.162077096 11.7814064025879 4.0
30.190981098 24.6347274780273 5.0
41.353058204 33.5925064086914 6.0
41.353058204 33.4457740783691 7.0
35.663236634 31.2813701629639 0.0
18.766917562 16.4772357940674 1.0
16.896319073 17.4322528839111 2.0
-7.670552873 -5.22229242324829 3.0
9.2257661902 9.99485492706299 4.0
26.437470445 23.9601631164551 5.0
35.663236644 32.8155784606934 6.0
35.663236644 33.5602989196777 7.0
40.444916113 31.7372455596924 0.0
18.421748417 15.0231895446777 1.0
22.023167696 18.4233684539795 2.0
-9.937900394 -5.96860790252686 3.0
12.085267292 12.1395988464355 4.0
28.359648821 23.6122989654541 5.0
40.444916123 32.4627952575684 6.0
40.444916123 33.206470489502 7.0
39.900416904 39.6179275512695 0.0
21.000367331 16.2763671875 1.0
18.900049573 15.4606075286865 2.0
-6.1981317567 -6.02829599380493 3.0
12.701917806 11.6071319580078 4.0
27.198499098 26.0955619812012 5.0
39.900416914 35.9917907714844 6.0
39.900416914 33.3842506408691 7.0
34.40532484 39.6817054748535 0.0
18.680133752 18.1321716308594 1.0
15.725191089 17.1687622070312 2.0
-6.5675447079 -4.12961435317993 3.0
9.157646372 10.28053855896 4.0
25.247678469 21.7873611450195 5.0
34.405324849 37.1095314025879 6.0
34.405324849 34.7161445617676 7.0
33.074255995 38.7729988098145 0.0
16.81819884 14.7263879776001 1.0
16.256057155 14.2133550643921 2.0
-5.8048358004 -4.59067869186401 3.0
10.451221345 10.5737333297729 4.0
22.62303465 22.0244979858398 5.0
33.074256004 35.0674438476562 6.0
33.074256004 33.3030090332031 7.0
40.036170298 28.5221576690674 0.0
18.427862329 15.1665544509888 1.0
21.60830797 17.2242946624756 2.0
-6.8170168406 -5.77389335632324 3.0
14.791291119 13.7573318481445 4.0
25.244879179 22.1678943634033 5.0
40.036170308 32.2523956298828 6.0
40.036170308 33.4330902099609 7.0
44.453206231 34.669807434082 0.0
21.51525667 14.6892824172974 1.0
22.937949561 14.1957664489746 2.0
-6.654859187 -4.88871479034424 3.0
16.283090365 14.6305894851685 4.0
28.170115867 22.789270401001 5.0
44.453206241 33.7749786376953 6.0
44.453206241 34.2857055664062 7.0
33.8500454 37.9176597595215 0.0
15.317438725 16.4752311706543 1.0
18.532606675 18.687593460083 2.0
-3.771729854 -3.99189043045044 3.0
14.760876811 14.0328798294067 4.0
19.089168589 18.7161693572998 5.0
33.85004541 34.7932014465332 6.0
33.85004541 33.9476280212402 7.0
30.537877299 31.7074317932129 0.0
15.743236653 17.3660163879395 1.0
14.794640646 18.1912803649902 2.0
-6.2614673178 -4.65231466293335 3.0
8.5331733191 9.90147876739502 4.0
22.00470398 21.7094116210938 5.0
30.537877308 32.1203689575195 6.0
30.537877308 33.3156356811523 7.0
27.727440934 34.6240921020508 0.0
13.298433958 15.7617015838623 1.0
14.429006978 18.1083946228027 2.0
-4.8732361991 -3.99804830551147 3.0
9.5557707702 10.1450023651123 4.0
18.171670165 18.9368534088135 5.0
27.727440942 33.5762596130371 6.0
27.727440942 34.0316429138184 7.0
35.630182891 31.0041732788086 0.0
17.416160369 16.451545715332 1.0
18.214022525 17.6468925476074 2.0
-5.5362398817 -4.87921810150146 3.0
12.677782636 12.5387115478516 4.0
22.952400258 21.0212383270264 5.0
35.630182897 32.2439804077148 6.0
35.630182897 34.206470489502 7.0
31.038251509 29.218017578125 0.0
16.216885208 15.2434921264648 1.0
14.821366301 14.1549654006958 2.0
-5.1281710605 -3.97735404968262 3.0
9.6931952301 10.3868436813354 4.0
21.345056279 19.6275405883789 5.0
31.038251519 32.5528945922852 6.0
31.038251519 33.8567161560059 7.0
31.867011958 28.5846481323242 0.0
14.888086344 15.3431606292725 1.0
16.978925614 17.8219623565674 2.0
-5.7710515718 -3.87612247467041 3.0
11.207874033 11.7614593505859 4.0
20.659137926 19.2782688140869 5.0
31.867011967 32.090461730957 6.0
31.867011967 34.0143508911133 7.0
30.754831636 39.1147041320801 0.0
14.067592922 16.5510501861572 1.0
16.687238713 19.1772785186768 2.0
-4.2836008115 -3.98267221450806 3.0
12.403637892 12.2201881408691 4.0
18.351193744 18.7712287902832 5.0
30.754831646 35.9234199523926 6.0
30.754831646 33.214225769043 7.0
38.951564273 40.1749801635742 0.0
20.789904648 14.9775791168213 1.0
18.161659627 14.4680814743042 2.0
-9.14497609 -5.43001365661621 3.0
9.0166835299 9.97830867767334 4.0
29.934880745 24.4005317687988 5.0
38.95156428 36.9362258911133 6.0
38.95156428 33.4603309631348 7.0
33.384480929 34.4088134765625 0.0
14.280681926 14.3752717971802 1.0
19.103799006 18.9935722351074 2.0
-6.8894608583 -5.09947443008423 3.0
12.214338142 12.8789224624634 4.0
21.170142791 20.9336414337158 5.0
33.384480936 33.3472671508789 6.0
33.384480936 33.2795753479004 7.0
37.291199278 37.6088790893555 0.0
18.976122996 16.7376537322998 1.0
18.315076287 15.8987579345703 2.0
-3.6314782733 -4.17348051071167 3.0
14.68359801 13.6135377883911 4.0
22.607601274 20.8479900360107 5.0
37.291199282 35.126049041748 6.0
37.291199282 33.349552154541 7.0
33.5718110164568 37.4875564575195 0.0
16.2455196066318 14.7040796279907 1.0
17.3262914102844 15.9820585250854 2.0
-6.34124959778268 -4.85224056243896 3.0
10.9850418123699 10.970874786377 4.0
22.5867692042782 21.5450458526611 5.0
33.571811016 34.7284202575684 6.0
33.571811016 33.6931304931641 7.0
44.048761737 38.276554107666 0.0
22.194046458 18.4349708557129 1.0
21.85471528 19.8342952728271 2.0
-6.0219898767 -5.88338899612427 3.0
15.832725394 14.9363975524902 4.0
28.216036344 25.1076068878174 5.0
44.048761746 35.5800666809082 6.0
44.048761746 33.9547348022461 7.0
38.092815451 37.3218040466309 0.0
20.42584869 16.9689102172852 1.0
17.666966762 15.1682605743408 2.0
-5.2714738638 -4.76233243942261 3.0
12.395492889 11.7487945556641 4.0
25.697322563 23.1863460540771 5.0
38.09281546 34.2934455871582 6.0
38.09281546 34.4043388366699 7.0
42.616606914 38.7835311889648 0.0
22.043217204 17.3537006378174 1.0
20.57338971 18.9188842773438 2.0
-9.1142862064 -5.90707445144653 3.0
11.459103494 12.3147773742676 4.0
31.157503421 24.1761302947998 5.0
42.616606924 36.1650161743164 6.0
42.616606924 34.4558410644531 7.0
41.080227066 34.6157913208008 0.0
21.412108977 15.9369049072266 1.0
19.66811809 16.2651691436768 2.0
-8.9308117628 -6.01450967788696 3.0
10.737306318 10.9585237503052 4.0
30.342920748 26.1853370666504 5.0
41.080227074 33.7921485900879 6.0
41.080227074 33.6169967651367 7.0
36.859087804 32.6020812988281 0.0
16.308519924 14.8396053314209 1.0
20.550567882 18.7782211303711 2.0
-8.0027710471 -5.24330234527588 3.0
12.547796826 12.5537204742432 4.0
24.31129098 20.3079967498779 5.0
36.859087813 32.3462753295898 6.0
36.859087813 34.8897819519043 7.0
39.316268866 34.7761917114258 0.0
19.479900065 14.5134811401367 1.0
19.836368801 14.0509357452393 2.0
-7.370707565 -5.56936073303223 3.0
12.465661226 11.5067386627197 4.0
26.85060764 24.2155609130859 5.0
39.316268876 33.7515563964844 6.0
39.316268876 33.9740600585938 7.0
40.023071957 37.4576835632324 0.0
21.374119573 16.1323108673096 1.0
18.648952384 17.0163917541504 2.0
-8.5239396023 -6.41561555862427 3.0
10.125012772 10.9218502044678 4.0
29.898059185 26.6755809783936 5.0
40.023071967 34.5052032470703 6.0
40.023071967 34.4762763977051 7.0
32.691039556 31.5593948364258 0.0
16.99402987 16.3787841796875 1.0
15.697009686 17.3052806854248 2.0
-6.1678662484 -4.4300971031189 3.0
9.5291434282 10.9482498168945 4.0
23.161896128 21.282506942749 5.0
32.691039566 33.0530242919922 6.0
32.691039566 33.3484649658203 7.0
39.350172222 33.374755859375 0.0
19.79571531 17.4793300628662 1.0
19.554456913 18.4271469116211 2.0
-6.9332448751 -5.18442392349243 3.0
12.62121203 12.3308792114258 4.0
26.728960194 23.0038890838623 5.0
39.35017223 33.0682182312012 6.0
39.35017223 34.1196823120117 7.0
38.6257792205169 30.3181056976318 0.0
20.6764320695262 15.5010261535645 1.0
17.9493471475253 16.1070938110352 2.0
-8.74948254057215 -6.13040590286255 3.0
9.19986460693105 10.3546600341797 4.0
29.4259146106884 26.1182613372803 5.0
38.62577922 33.2089195251465 6.0
38.62577922 33.6211700439453 7.0
43.363210173 30.4646492004395 0.0
22.264479357 18.2533798217773 1.0
21.09873082 19.4773998260498 2.0
-7.8566163895 -5.92808485031128 3.0
13.242114424 13.1990051269531 4.0
30.121095753 24.7691307067871 5.0
43.36321018 32.5145721435547 6.0
43.36321018 33.637508392334 7.0
41.7113705187954 28.9406566619873 0.0
19.9560794530664 16.3622398376465 1.0
21.7552910618007 18.5566921234131 2.0
-6.27295647675057 -6.17046880722046 3.0
15.4823345851289 14.7310066223145 4.0
26.2290359296866 24.3822498321533 5.0
41.711370519 31.7861824035645 6.0
41.711370519 33.690013885498 7.0
41.4017485851654 37.938346862793 0.0
21.9191533447582 15.7499752044678 1.0
19.4825952431145 14.4766569137573 2.0
-7.53340700226978 -5.83815431594849 3.0
11.9491882404739 11.5932302474976 4.0
29.4525603463383 25.6702575683594 5.0
41.401748585 34.5024299621582 6.0
41.401748585 33.5937919616699 7.0
45.411303660968 29.21852684021 0.0
22.8793399812672 17.0378742218018 1.0
22.5319636798771 18.2230758666992 2.0
-8.35657341793646 -6.49935293197632 3.0
14.1753902620561 13.2653121948242 4.0
31.2359133986188 27.4910926818848 5.0
45.411303661 31.6624584197998 6.0
45.411303661 33.6430130004883 7.0
46.173622728 32.4397010803223 0.0
22.478207281 14.6974544525146 1.0
23.695415447 13.8666639328003 2.0
-7.2335265538 -6.00758743286133 3.0
16.461888883 14.6171846389771 4.0
29.711733845 24.1408786773682 5.0
46.173622738 33.0164947509766 6.0
46.173622738 33.4625244140625 7.0
43.970500883 36.2109375 0.0
23.319439955 14.799898147583 1.0
20.651060929 13.9527101516724 2.0
-9.6037663307 -5.34968185424805 3.0
11.047294589 11.1500749588013 4.0
32.923206295 24.1462383270264 5.0
43.970500892 34.7617034912109 6.0
43.970500892 34.2060508728027 7.0
41.561964514 37.6726455688477 0.0
22.423587267 17.7991352081299 1.0
19.138377247 14.50621509552 2.0
-2.7143772539 -5.05217504501343 3.0
16.423999984 14.2094211578369 4.0
25.13796453 24.016637802124 5.0
41.561964524 35.0914344787598 6.0
41.561964524 33.5032005310059 7.0
42.043570825 29.1271286010742 0.0
21.184612309 14.8151607513428 1.0
20.858958517 14.6003713607788 2.0
-7.1973175688 -5.73619747161865 3.0
13.66164094 12.5041303634644 4.0
28.381929887 23.7209873199463 5.0
42.043570834 31.6451187133789 6.0
42.043570834 34.398681640625 7.0
42.728187899 38.7241973876953 0.0
22.937429443 15.9577627182007 1.0
19.790758457 14.6688184738159 2.0
-7.3221221307 -5.81673765182495 3.0
12.468636317 12.0788078308105 4.0
30.259551583 25.9030055999756 5.0
42.728187908 36.0931587219238 6.0
42.728187908 34.1450538635254 7.0
46.647822935 38.760440826416 0.0
24.419972425 17.9188175201416 1.0
22.22785051 18.0858745574951 2.0
-9.7005341836 -6.63962554931641 3.0
12.527316316 12.1095924377441 4.0
34.120506619 26.9037666320801 5.0
46.647822945 35.7154273986816 6.0
46.647822945 33.6567649841309 7.0
34.157298452 37.7904434204102 0.0
16.904613725 15.4720134735107 1.0
17.252684728 17.3457355499268 2.0
-7.2828539724 -4.25315713882446 3.0
9.9698307451 11.2801761627197 4.0
24.187467707 20.3797416687012 5.0
34.157298462 35.0883026123047 6.0
34.157298462 34.2551307678223 7.0
38.08940785 28.9458103179932 0.0
20.151317931 17.0137271881104 1.0
17.93808992 15.2080497741699 2.0
-2.5516772312 -4.52888345718384 3.0
15.386412679 14.0055818557739 4.0
22.702995171 21.5809135437012 5.0
38.08940786 32.3483467102051 6.0
38.08940786 33.6013870239258 7.0
48.253456595 30.028736114502 0.0
26.02387816 17.2923316955566 1.0
22.229578435 16.5300369262695 2.0
-8.9125381923 -6.77715539932251 3.0
13.317040233 12.6321811676025 4.0
34.936416362 28.5791931152344 5.0
48.253456605 31.9574966430664 6.0
48.253456605 34.257511138916 7.0
41.303512945 39.7276153564453 0.0
21.842523524 16.5224266052246 1.0
19.460989421 16.9509315490723 2.0
-8.4785475097 -5.99911689758301 3.0
10.982441901 10.7104043960571 4.0
30.321071043 26.8714962005615 5.0
41.303512955 36.6707534790039 6.0
41.303512955 33.9358024597168 7.0
34.11247551 28.9880847930908 0.0
16.407323353 15.2597942352295 1.0
17.705152157 15.7698307037354 2.0
-3.4961419723 -4.69967842102051 3.0
14.209010175 13.1453151702881 4.0
19.903465335 19.9465560913086 5.0
34.11247552 31.499828338623 6.0
34.11247552 33.9257507324219 7.0
39.346899311 30.5435333251953 0.0
20.280952021 15.6774787902832 1.0
19.06594729 16.6551666259766 2.0
-10.212334112 -5.2309422492981 3.0
8.8536131687 9.63478565216064 4.0
30.493286142 24.0889625549316 5.0
39.346899321 32.3076248168945 6.0
39.346899321 34.017463684082 7.0
42.355081312 40.2258415222168 0.0
22.237080125 17.2707195281982 1.0
20.118001187 14.8858594894409 2.0
-4.2250518369 -4.58633852005005 3.0
15.892949341 13.7360935211182 4.0
26.462131971 23.36741065979 5.0
42.355081321 36.6111755371094 6.0
42.355081321 33.5182037353516 7.0
46.02528461 40.065242767334 0.0
20.767999035 14.9154167175293 1.0
25.257285575 17.5548439025879 2.0
-9.1629445671 -6.45261001586914 3.0
16.094340999 15.4160661697388 4.0
29.930943612 26.7094917297363 5.0
46.025284619 36.3703460693359 6.0
46.025284619 33.6425552368164 7.0
48.790909777 35.0825805664062 0.0
25.145056622 18.560230255127 1.0
23.645853155 18.8837699890137 2.0
-9.2179915631 -6.17522192001343 3.0
14.427861582 13.2100954055786 4.0
34.363048195 26.665843963623 5.0
48.790909787 34.0609130859375 6.0
48.790909787 33.7028694152832 7.0
41.698940493 39.6980171203613 0.0
20.507660886 15.2799320220947 1.0
21.191279609 17.25514793396 2.0
-9.1543537316 -5.81030464172363 3.0
12.036925868 11.7911014556885 4.0
29.662014626 24.8424301147461 5.0
41.698940502 36.1993026733398 6.0
41.698940502 33.9681053161621 7.0
47.902972774 37.4615325927734 0.0
23.393159022 16.5062484741211 1.0
24.509813754 17.8821849822998 2.0
-8.5476809737 -6.4543137550354 3.0
15.962132774 14.7798919677734 4.0
31.940840003 25.5084190368652 5.0
47.902972781 35.070915222168 6.0
47.902972781 33.9701118469238 7.0
41.377170862 37.9225044250488 0.0
20.577519401 17.0017433166504 1.0
20.799651462 17.2740020751953 2.0
-6.2821130824 -5.42370557785034 3.0
14.517538372 13.5310544967651 4.0
26.859632492 23.8667945861816 5.0
41.37717087 35.0911521911621 6.0
41.37717087 33.273193359375 7.0
32.285684892 29.4179935455322 0.0
14.316873201 14.8112440109253 1.0
17.968811691 17.5048885345459 2.0
-5.3227711546 -4.24419450759888 3.0
12.646040526 12.5015754699707 4.0
19.639644366 19.1340293884277 5.0
32.285684902 31.8472023010254 6.0
32.285684902 33.3933525085449 7.0
32.344206843 35.086669921875 0.0
16.160894185 14.896258354187 1.0
16.183312659 16.8571929931641 2.0
-7.3310896054 -5.90467357635498 3.0
8.852223045 10.4420623779297 4.0
23.491983799 22.9547882080078 5.0
32.344206851 33.3748168945312 6.0
32.344206851 34.3220367431641 7.0
34.184112321 28.6148509979248 0.0
18.400062549 15.7427444458008 1.0
15.784049772 15.0056219100952 2.0
-7.6583985283 -5.35536813735962 3.0
8.125651234 9.23636627197266 4.0
26.058461087 23.8972244262695 5.0
34.184112331 32.4196243286133 6.0
34.184112331 33.4284553527832 7.0
34.663221798 39.7004432678223 0.0
18.206197344 17.0525341033936 1.0
16.457024454 16.5499324798584 2.0
-6.9582970888 -4.39294052124023 3.0
9.4987273548 10.3617734909058 4.0
25.164494443 22.6108932495117 5.0
34.663221808 36.856616973877 6.0
34.663221808 33.5600547790527 7.0
36.873774533 36.667610168457 0.0
18.579406772 16.8277454376221 1.0
18.294367761 17.6567916870117 2.0
-5.3733020341 -4.85131549835205 3.0
12.921065717 12.5515928268433 4.0
23.952708817 22.4135894775391 5.0
36.873774543 34.3368797302246 6.0
36.873774543 33.5911827087402 7.0
40.211352898 34.8696708679199 0.0
22.061687235 15.9457416534424 1.0
18.149665663 13.8709936141968 2.0
-7.4872031869 -5.14885187149048 3.0
10.662462467 9.93074989318848 4.0
29.548890431 25.2351093292236 5.0
40.211352908 33.6379241943359 6.0
40.211352908 33.3409423828125 7.0
46.160689864 36.3233909606934 0.0
24.176307096 17.74267578125 1.0
21.984382774 17.105354309082 2.0
-5.8472747438 -5.77548980712891 3.0
16.137108027 14.3583221435547 4.0
30.023581843 24.9999332427979 5.0
46.160689867 33.9317588806152 6.0
46.160689867 33.7604827880859 7.0
32.727702668 31.1375980377197 0.0
17.796794822 18.2993392944336 1.0
14.930907847 16.4049549102783 2.0
-5.7416146893 -3.32797193527222 3.0
9.1892931492 10.1494741439819 4.0
23.53840952 20.4116077423096 5.0
32.727702676 32.7083511352539 6.0
32.727702676 33.4696350097656 7.0
40.640985417073 38.5149803161621 0.0
21.2248663380953 18.4435768127441 1.0
19.4161190831271 17.7003650665283 2.0
-6.12795167285201 -5.05939102172852 3.0
13.2881674100867 12.6288928985596 4.0
27.3528180102152 24.2582588195801 5.0
40.640985417 34.2736206054688 6.0
40.640985417 33.617603302002 7.0
34.611451584 39.3276100158691 0.0
16.03184427 14.8479242324829 1.0
18.579607315 17.1337604522705 2.0
-5.6371786801 -5.18347930908203 3.0
12.942428625 12.3646621704102 4.0
21.66902296 21.3845901489258 5.0
34.611451594 35.6823310852051 6.0
34.611451594 33.4854354858398 7.0
43.311280798 30.0225257873535 0.0
19.747206987 14.9484834671021 1.0
23.564073813 18.1164779663086 2.0
-7.5093463576 -6.62976503372192 3.0
16.054727447 14.3621978759766 4.0
27.256553353 24.5000495910645 5.0
43.311280807 31.3708343505859 6.0
43.311280807 33.9453239440918 7.0
36.406364048 29.2056007385254 0.0
19.818637935 16.7136344909668 1.0
16.587726114 14.4322052001953 2.0
-3.8896932395 -4.04791831970215 3.0
12.698032865 12.0124664306641 4.0
23.708331184 21.1738243103027 5.0
36.406364057 31.7959880828857 6.0
36.406364057 33.5538673400879 7.0
31.68563032 36.740406036377 0.0
16.740289757 14.6959133148193 1.0
14.945340567 13.2149133682251 2.0
-6.0451472911 -3.01858806610107 3.0
8.9001932697 8.09790134429932 4.0
22.785437054 19.8266506195068 5.0
31.685630326 34.4793968200684 6.0
31.685630326 33.7775230407715 7.0
40.165364423 40.0859298706055 0.0
21.225216477 16.3107852935791 1.0
18.940147947 18.1470375061035 2.0
-9.5161355118 -5.02768802642822 3.0
9.4240124269 10.0580148696899 4.0
30.741351997 23.2752056121826 5.0
40.165364432 35.7678337097168 6.0
40.165364432 33.7635917663574 7.0
42.886465637 37.1685943603516 0.0
31.633244398 15.0173635482788 1.0
11.253221239 18.4985618591309 2.0
0 -6.75486040115356 3.0
11.253221239 11.9618902206421 4.0
31.633244398 27.2994747161865 5.0
42.886465647 34.1536521911621 6.0
42.886465647 34.0910682678223 7.0
46.942592322 35.6020431518555 0.0
23.694673267 15.3700294494629 1.0
23.247919055 15.6487846374512 2.0
-9.8881851606 -6.57507562637329 3.0
13.359733884 12.7350921630859 4.0
33.582858438 28.7909908294678 5.0
46.942592332 34.0331230163574 6.0
46.942592332 33.3048934936523 7.0
38.417030558 38.0309524536133 0.0
20.711241226 14.7957696914673 1.0
17.705789332 13.716986656189 2.0
-7.7084466149 -5.5734748840332 3.0
9.9973427073 10.4455146789551 4.0
28.419687851 24.8041515350342 5.0
38.417030568 35.036922454834 6.0
38.417030568 33.8112754821777 7.0
39.82159542 33.5161247253418 0.0
19.816428679 17.0362777709961 1.0
20.005166741 17.695951461792 2.0
-5.1739908699 -5.13426208496094 3.0
14.831175861 13.5756492614746 4.0
24.990419559 23.3043518066406 5.0
39.82159543 32.712589263916 6.0
39.82159543 33.7216033935547 7.0
42.375846264 40.0397567749023 0.0
20.538210367 17.9101104736328 1.0
21.837635902 18.974645614624 2.0
-5.3545430676 -5.74054908752441 3.0
16.483092829 15.0282421112061 4.0
25.89275344 23.7619743347168 5.0
42.375846269 35.7970199584961 6.0
42.375846269 34.8389167785645 7.0
41.909868914 36.0860900878906 0.0
21.249762031 16.3106365203857 1.0
20.660106883 18.2888469696045 2.0
-10.971309543 -5.79811239242554 3.0
9.6887973304 10.8924856185913 4.0
32.221071584 23.2725315093994 5.0
41.909868924 34.1776008605957 6.0
41.909868924 33.3393707275391 7.0
38.5154038135067 33.3534355163574 0.0
19.3171244035071 15.2006702423096 1.0
19.1982794195071 16.2316303253174 2.0
-8.28760453680882 -6.07612991333008 3.0
10.9106748825063 11.227502822876 4.0
27.6047289405071 24.533109664917 5.0
38.515403814 32.8358268737793 6.0
38.515403814 33.2316627502441 7.0
32.187464764 38.480052947998 0.0
15.688410189 14.9168310165405 1.0
16.499054576 15.3524322509766 2.0
-5.8055743047 -3.93340396881104 3.0
10.693480262 10.4174089431763 4.0
21.493984503 19.5567626953125 5.0
32.187464773 34.8867073059082 6.0
32.187464773 33.1217384338379 7.0
43.5533404982843 39.3807411193848 0.0
25.257472916292 16.4957847595215 1.0
18.2958675832873 13.5045480728149 2.0
-8.52565235008034 -5.09855127334595 3.0
9.77021523347734 9.62235069274902 4.0
33.7831252662979 26.4927825927734 5.0
43.553340498 35.8480415344238 6.0
43.553340498 33.5822296142578 7.0
38.661685108 29.9397792816162 0.0
21.023698145 16.9823265075684 1.0
17.637986964 14.6624135971069 2.0
-6.4647611773 -5.31769800186157 3.0
11.173225776 10.9911251068115 4.0
27.488459332 25.6451778411865 5.0
38.661685118 32.6013565063477 6.0
38.661685118 33.9438934326172 7.0
47.185976942 34.8277168273926 0.0
20.88210062 14.6918640136719 1.0
26.303876326 17.893196105957 2.0
-9.8059530494 -6.91992521286011 3.0
16.497923272 14.9634027481079 4.0
30.688053675 25.5065383911133 5.0
47.185976948 33.5185890197754 6.0
47.185976948 33.612419128418 7.0
41.0736498246764 31.8872699737549 0.0
17.9299575176762 14.315149307251 1.0
23.1436923166761 16.2618007659912 2.0
-7.36188695227673 -6.16665887832642 3.0
15.7818053646765 14.035719871521 4.0
25.2918444696756 23.2319984436035 5.0
41.073649825 32.3454246520996 6.0
41.073649825 33.7813682556152 7.0
39.086114836 29.4660778045654 0.0
22.054277289 18.1063766479492 1.0
17.031837547 18.437427520752 2.0
-8.5092532605 -4.38440942764282 3.0
8.5225842768 10.3502779006958 4.0
30.563530559 23.5512351989746 5.0
39.086114846 32.5035743713379 6.0
39.086114846 33.7379531860352 7.0
38.072309142 34.1640205383301 0.0
19.059692214 16.7896003723145 1.0
19.012616928 16.3086338043213 2.0
-4.5092621515 -4.4089469909668 3.0
14.503354768 13.2759780883789 4.0
23.568954375 21.0989799499512 5.0
38.072309151 33.6499557495117 6.0
38.072309151 33.868595123291 7.0
41.585402364 36.6245346069336 0.0
21.878346027 17.41455078125 1.0
19.707056337 17.8571014404297 2.0
-7.856520423 -5.5402398109436 3.0
11.850535904 12.0025959014893 4.0
29.73486646 24.212381362915 5.0
41.585402374 34.6069946289062 6.0
41.585402374 33.8843536376953 7.0
42.3173178092696 29.2304973602295 0.0
23.2468941623901 18.8541469573975 1.0
19.0704236443393 18.3956680297852 2.0
-10.4383552532401 -6.01254558563232 3.0
8.63206839120634 9.09577655792236 4.0
33.6852494163203 26.8712425231934 5.0
42.317317809 31.216329574585 6.0
42.317317809 33.3017578125 7.0
42.931477575 39.0751419067383 0.0
22.005453055 15.2786989212036 1.0
20.926024519 14.5615653991699 2.0
-5.8784626661 -6.30641794204712 3.0
15.047561843 13.3314456939697 4.0
27.883915732 25.861515045166 5.0
42.931477585 35.504508972168 6.0
42.931477585 33.1551132202148 7.0
46.096274403 32.2445983886719 0.0
25.32088965 17.9225883483887 1.0
20.775384755 14.9910173416138 2.0
-5.4662065709 -5.24626064300537 3.0
15.309178177 14.1469078063965 4.0
30.787096229 23.2666473388672 5.0
46.096274411 32.6771202087402 6.0
46.096274411 34.016185760498 7.0
39.955261544 39.7577323913574 0.0
19.931762754 15.469690322876 1.0
20.02349879 17.1600570678711 2.0
-8.9110310329 -5.34635972976685 3.0
11.112467747 10.7040987014771 4.0
28.842793797 24.4336776733398 5.0
39.955261554 35.9722557067871 6.0
39.955261554 33.8791542053223 7.0
45.415961836 36.9976997375488 0.0
23.097461473 15.8337993621826 1.0
22.318500363 15.7961835861206 2.0
-8.5889928185 -6.54691648483276 3.0
13.729507535 12.4279298782349 4.0
31.686454301 26.9775142669678 5.0
45.415961846 33.9277305603027 6.0
45.415961846 33.8626518249512 7.0
44.275746647 36.0707931518555 0.0
22.654009807 17.6620273590088 1.0
21.62173684 17.9730968475342 2.0
-7.7547516702 -5.21002674102783 3.0
13.86698516 12.8596687316895 4.0
30.408761488 23.2451820373535 5.0
44.275746657 33.894458770752 6.0
44.275746657 33.5286865234375 7.0
34.766238956 30.046802520752 0.0
17.943458551 17.0423526763916 1.0
16.822780406 16.8462791442871 2.0
-5.6356736314 -4.76292657852173 3.0
11.187106764 11.6229467391968 4.0
23.579132192 22.3595066070557 5.0
34.766238966 31.7195148468018 6.0
34.766238966 33.8686103820801 7.0
39.697519492335 37.3379707336426 0.0
21.2162061883195 16.0039253234863 1.0
18.481313305488 13.9343481063843 2.0
-4.67528909066386 -5.04293441772461 3.0
13.8060242148741 11.9212265014648 4.0
25.8914952793298 23.8365154266357 5.0
39.697519492 34.5218162536621 6.0
39.697519492 33.9563331604004 7.0
36.4599525996002 33.7215690612793 0.0
18.3065166475976 14.9876728057861 1.0
18.1534359526018 14.6183605194092 2.0
-5.15104061002709 -4.95427274703979 3.0
13.0023953426082 11.7957448959351 4.0
23.457557257596 21.8365821838379 5.0
36.459952599 33.1410026550293 6.0
36.459952599 33.6375732421875 7.0
44.856515577 32.6798782348633 0.0
22.473710733 15.8196792602539 1.0
22.382804843 17.5071716308594 2.0
-8.0040806232 -6.83874654769897 3.0
14.37872421 13.4761447906494 4.0
30.477791367 27.2800445556641 5.0
44.856515586 32.6022491455078 6.0
44.856515586 33.7737083435059 7.0
41.1579139562054 39.0831108093262 0.0
18.5564288232053 15.6577415466309 1.0
22.601485143205 19.3069438934326 2.0
-7.31439675300561 -5.645676612854 3.0
15.2870883902055 14.6125259399414 4.0
25.8708255762049 21.1559238433838 5.0
41.157913956 35.7998733520508 6.0
41.157913956 33.2906951904297 7.0
32.092085782 35.0939865112305 0.0
17.272160967 14.8776178359985 1.0
14.819924815 13.3584337234497 2.0
-6.6255124723 -4.02187776565552 3.0
8.1944123328 7.90819120407104 4.0
23.897673449 21.7394695281982 5.0
32.092085792 33.6720542907715 6.0
32.092085792 33.7746543884277 7.0
30.327518415 36.105339050293 0.0
13.541286609 15.1076717376709 1.0
16.786231807 18.7366886138916 2.0
-5.5133467024 -4.55515623092651 3.0
11.272885094 11.7786264419556 4.0
19.054633321 19.0002365112305 5.0
30.327518425 34.3978576660156 6.0
30.327518425 33.6167755126953 7.0
36.482977268 37.3638305664062 0.0
19.553260998 14.9174118041992 1.0
16.929716271 14.4502077102661 2.0
-7.3162847454 -5.91854190826416 3.0
9.613431516 9.8770866394043 4.0
26.869545753 24.420524597168 5.0
36.482977278 34.8076515197754 6.0
36.482977278 33.4292335510254 7.0
40.491170904 36.6239166259766 0.0
20.509684159 14.9066038131714 1.0
19.981486745 15.1091136932373 2.0
-8.9848735316 -5.94620180130005 3.0
10.996613204 10.504828453064 4.0
29.4945577 24.513147354126 5.0
40.491170914 34.3329391479492 6.0
40.491170914 33.6156959533691 7.0
41.188898324 40.2408065795898 0.0
17.482572735 14.6311187744141 1.0
23.70632559 18.1391143798828 2.0
-7.86829602 -6.28754281997681 3.0
15.838029561 14.568434715271 4.0
25.350868764 22.06520652771 5.0
41.188898334 35.8213920593262 6.0
41.188898334 33.2845497131348 7.0
37.756949894 37.992359161377 0.0
19.043754327 15.5602855682373 1.0
18.713195567 15.3088827133179 2.0
-5.5841962186 -5.11476135253906 3.0
13.128999338 11.9322357177734 4.0
24.627950556 23.0299263000488 5.0
37.756949904 35.2485275268555 6.0
37.756949904 33.6566047668457 7.0
26.962645791 29.2616691589355 0.0
14.231795944 15.3409748077393 1.0
12.730849849 14.3523759841919 2.0
-4.3614404899 -3.04569101333618 3.0
8.3694093501 9.65251922607422 4.0
18.593236442 18.488073348999 5.0
26.9626458 30.7942085266113 6.0
26.9626458 34.1633720397949 7.0
44.1546808660895 33.0673141479492 0.0
20.845479755763 17.1774272918701 1.0
23.3092011082446 19.9845867156982 2.0
-6.97221748920505 -6.20532894134521 3.0
16.3369836194072 15.7825584411621 4.0
27.8176971034536 23.2904987335205 5.0
44.154680866 32.7995796203613 6.0
44.154680866 34.0084991455078 7.0
36.4992036 34.3158340454102 0.0
18.499566849 15.8122005462646 1.0
17.999636751 16.2741317749023 2.0
-8.164709678 -5.29982995986938 3.0
9.834927063 10.7561006546021 4.0
26.664276537 22.6304092407227 5.0
36.49920361 33.5780410766602 6.0
36.49920361 33.4358444213867 7.0
36.136186998 36.2581748962402 0.0
19.096538919 14.9956083297729 1.0
17.039648079 14.032172203064 2.0
-8.7939144312 -4.6310601234436 3.0
8.2457336392 9.06781387329102 4.0
27.89045336 21.4476776123047 5.0
36.136187007 34.25927734375 6.0
36.136187007 34.0533638000488 7.0
43.805905699 37.7636833190918 0.0
19.467800315 15.0324907302856 1.0
24.338105385 18.4080238342285 2.0
-10.07714477 -6.22266578674316 3.0
14.260960606 13.4058895111084 4.0
29.544945095 22.841968536377 5.0
43.805905709 34.7382316589355 6.0
43.805905709 33.8235664367676 7.0
43.400673471 33.3035011291504 0.0
21.450145846 15.2315807342529 1.0
21.950527627 17.7010154724121 2.0
-10.718319514 -6.2969913482666 3.0
11.232208106 11.7777719497681 4.0
32.168465368 24.7361907958984 5.0
43.400673478 33.2918891906738 6.0
43.400673478 33.6907386779785 7.0
35.708898922 33.4361305236816 0.0
19.860917727 17.5021419525146 1.0
15.847981195 17.8233222961426 2.0
-6.8010445999 -4.98051834106445 3.0
9.0469365848 10.673680305481 4.0
26.661962337 23.6517276763916 5.0
35.708898932 33.2108688354492 6.0
35.708898932 33.0341606140137 7.0
42.19537597 35.904468536377 0.0
21.611035583 16.6324787139893 1.0
20.584340387 17.5336532592773 2.0
-7.818182446 -6.47146081924438 3.0
12.766157932 12.3620662689209 4.0
29.429218038 25.5243873596191 5.0
42.195375979 34.1463317871094 6.0
42.195375979 33.9001121520996 7.0
38.219132248 33.9203567504883 0.0
18.124133621 17.1711044311523 1.0
20.094998634 18.3188877105713 2.0
-3.9675986496 -5.17582273483276 3.0
16.127399981 14.5839290618896 4.0
22.091732274 22.1484508514404 5.0
38.219132252 33.293586730957 6.0
38.219132252 33.9267311096191 7.0
37.34785903 38.1235046386719 0.0
19.494183125 16.1426258087158 1.0
17.853675906 14.9776849746704 2.0
-4.6743476987 -5.50978088378906 3.0
13.179328197 12.3376789093018 4.0
24.168530833 22.9494457244873 5.0
37.34785904 34.3934669494629 6.0
37.34785904 34.1674995422363 7.0
41.413317073 30.4802875518799 0.0
22.502810393 17.9314231872559 1.0
18.91050668 15.0901193618774 2.0
-4.5479233593 -4.95943880081177 3.0
14.362583311 13.3503246307373 4.0
27.050733762 24.8619651794434 5.0
41.413317083 32.2402572631836 6.0
41.413317083 34.9036674499512 7.0
39.408474003 37.550121307373 0.0
19.787258075 14.5104074478149 1.0
19.621215928 14.8719072341919 2.0
-8.5467597404 -6.25890684127808 3.0
11.074456178 10.8377637863159 4.0
28.334017825 25.1628322601318 5.0
39.408474012 34.6747741699219 6.0
39.408474012 33.7781143188477 7.0
44.6219477930596 39.8539733886719 0.0
23.5295769810594 15.9996166229248 1.0
21.0923708210596 16.5904407501221 2.0
-9.44677516695988 -6.61394834518433 3.0
11.6455956540597 11.598165512085 4.0
32.9763521480588 28.2680053710938 5.0
44.621947793 36.3353576660156 6.0
44.621947793 34.0431175231934 7.0
35.042314664 32.224063873291 0.0
17.377708596 17.3942775726318 1.0
17.66460607 18.006763458252 2.0
-4.7609303256 -3.98704099655151 3.0
12.903675736 12.6022262573242 4.0
22.13863893 19.8569889068604 5.0
35.042314673 33.0745162963867 6.0
35.042314673 33.4315299987793 7.0
44.668685977 35.0515785217285 0.0
21.700098462 14.3920001983643 1.0
22.968587516 14.9752149581909 2.0
-8.4663400465 -6.69385623931885 3.0
14.50224746 13.715313911438 4.0
30.166438518 26.8377780914307 5.0
44.668685987 33.6680374145508 6.0
44.668685987 34.1655426025391 7.0
44.271188228 35.7146072387695 0.0
21.134860564 15.0730419158936 1.0
23.136327666 16.6939334869385 2.0
-6.9953395415 -6.48236465454102 3.0
16.140988117 14.6803073883057 4.0
28.130200113 25.4634132385254 5.0
44.271188235 34.4188117980957 6.0
44.271188235 33.3089599609375 7.0
43.3296157145675 35.6833572387695 0.0
21.8144364805029 15.9654560089111 1.0
21.5151792385451 18.2712593078613 2.0
-10.3071776780671 -6.32030868530273 3.0
11.2080015596363 11.9563941955566 4.0
32.1216141588196 24.5393695831299 5.0
43.329615715 34.4986114501953 6.0
43.329615715 34.0396308898926 7.0
37.727034885 30.7887649536133 0.0
16.740902708 14.5719738006592 1.0
20.986132178 18.6434288024902 2.0
-10.156378742 -5.82095050811768 3.0
10.829753428 11.4353723526001 4.0
26.897281458 21.9201164245605 5.0
37.727034893 32.6993751525879 6.0
37.727034893 33.3203048706055 7.0
43.665312103 29.6928081512451 0.0
21.432332029 14.9519214630127 1.0
22.232980074 15.1571893692017 2.0
-7.2001290055 -6.28315019607544 3.0
15.032851059 13.7988004684448 4.0
28.632461044 25.3748016357422 5.0
43.665312113 31.7108955383301 6.0
43.665312113 33.2372131347656 7.0
40.6563483834553 36.6457748413086 0.0
20.9212538534144 15.8289985656738 1.0
19.7350945303589 15.9143772125244 2.0
-7.59696943049863 -6.84594964981079 3.0
12.1381250996025 11.945294380188 4.0
28.5182231244438 28.1942386627197 5.0
40.656348383 34.4917449951172 6.0
40.656348383 33.5323143005371 7.0
30.759139545 40.1469383239746 0.0
13.433268738 14.3828992843628 1.0
17.325870807 17.8996887207031 2.0
-5.9875785844 -4.9724063873291 3.0
11.338292213 11.6821765899658 4.0
19.420847332 19.6693382263184 5.0
30.759139555 36.4328689575195 6.0
30.759139555 34.1202125549316 7.0
36.474229645 34.4338684082031 0.0
14.989459641 14.5855522155762 1.0
21.484770007 19.4189758300781 2.0
-5.0219280476 -5.4650502204895 3.0
16.462841952 15.4632787704468 4.0
20.011387696 19.4646377563477 5.0
36.474229652 33.4969673156738 6.0
36.474229652 33.3868751525879 7.0
33.340721185 29.7431449890137 0.0
15.596021276 15.4860229492188 1.0
17.744699909 18.7384910583496 2.0
-7.4570294442 -5.03224277496338 3.0
10.287670455 11.4360361099243 4.0
23.05305073 20.6350040435791 5.0
33.340721195 32.153636932373 6.0
33.340721195 33.4690322875977 7.0
34.143349258 33.7906494140625 0.0
15.417033592 14.136266708374 1.0
18.726315666 16.0811519622803 2.0
-5.8334622114 -5.62090826034546 3.0
12.892853445 12.5605411529541 4.0
21.250495814 21.4315166473389 5.0
34.143349268 32.6811828613281 6.0
34.143349268 33.2077713012695 7.0
40.763829748 36.1607971191406 0.0
20.589059268 15.1429901123047 1.0
20.174770481 18.0170974731445 2.0
-11.800756688 -6.72332906723022 3.0
8.374013783 10.147367477417 4.0
32.389815966 26.5056934356689 5.0
40.763829757 34.2499656677246 6.0
40.763829757 33.897144317627 7.0
29.788224949 33.9228134155273 0.0
14.329239328 14.7479057312012 1.0
15.458985621 16.0445861816406 2.0
-6.2969425219 -4.27581834793091 3.0
9.1620430892 10.0990676879883 4.0
20.62618186 19.7975292205811 5.0
29.788224958 33.5008583068848 6.0
29.788224958 33.4190673828125 7.0
34.675869546 38.7698097229004 0.0
17.321679699 18.1680660247803 1.0
17.354189848 18.862096786499 2.0
-5.026706589 -4.59290552139282 3.0
12.32748325 12.4923858642578 4.0
22.348386296 21.7049522399902 5.0
34.675869554 35.5872611999512 6.0
34.675869554 34.2471771240234 7.0
40.7238558279895 38.9007225036621 0.0
19.1061702059935 14.9665660858154 1.0
21.6176856259914 17.5318737030029 2.0
-8.60840523789577 -5.61208724975586 3.0
13.0092803879935 12.4556608200073 4.0
27.7145752997044 23.8104591369629 5.0
40.723855828 34.8043441772461 6.0
40.723855828 33.6087493896484 7.0
41.414838344 29.0117282867432 0.0
21.79665744 15.1443576812744 1.0
19.618180905 16.8554954528809 2.0
-10.700829677 -6.6830792427063 3.0
8.9173512199 9.96065425872803 4.0
32.497487126 27.9513359069824 5.0
41.414838352 32.2420310974121 6.0
41.414838352 33.4429244995117 7.0
33.391441204 32.9511795043945 0.0
16.946662464 16.3146743774414 1.0
16.44477874 17.8827838897705 2.0
-6.5811351715 -5.15197515487671 3.0
9.8636435585 11.2365989685059 4.0
23.527797645 22.6998157501221 5.0
33.391441214 33.2409744262695 6.0
33.391441214 33.6356735229492 7.0
36.660645376 31.1763439178467 0.0
17.87289688 16.7843856811523 1.0
18.787748497 18.6149921417236 2.0
-6.6425323884 -4.40776252746582 3.0
12.1452161 12.188925743103 4.0
24.515429278 20.8597602844238 5.0
36.660645385 32.6221351623535 6.0
36.660645385 33.5233497619629 7.0
33.547567157 29.359489440918 0.0
16.605704713 15.5841617584229 1.0
16.941862445 18.1748428344727 2.0
-8.1253896728 -5.12890625 3.0
8.8164727625 9.97801399230957 4.0
24.731094395 22.958532333374 5.0
33.547567166 31.8932399749756 6.0
33.547567166 33.6604614257812 7.0
36.717923485 33.7363014221191 0.0
19.502688668 16.6746215820312 1.0
17.215234818 15.4522609710693 2.0
-6.0032006006 -5.09930562973022 3.0
11.212034207 10.7326889038086 4.0
25.505889278 23.7376155853271 5.0
36.717923495 33.2668609619141 6.0
36.717923495 33.6364479064941 7.0
44.575737489 39.1472396850586 0.0
24.24689991 15.973804473877 1.0
20.328837584 13.6827735900879 2.0
-6.0388557339 -5.2937126159668 3.0
14.289981845 12.8461647033691 4.0
30.285755649 25.3168468475342 5.0
44.575737494 35.8958740234375 6.0
44.575737494 33.6752662658691 7.0
43.091610313 35.1936454772949 0.0
22.329030269 16.7418785095215 1.0
20.762580045 16.3439998626709 2.0
-7.1251861541 -5.98169326782227 3.0
13.637393882 12.5336647033691 4.0
29.454216432 24.88498878479 5.0
43.091610321 33.8143424987793 6.0
43.091610321 33.8664779663086 7.0
42.647069964 31.2056064605713 0.0
21.774885881 16.4051952362061 1.0
20.872184083 16.6214599609375 2.0
-7.6954983452 -5.75196504592896 3.0
13.176685728 12.6101207733154 4.0
29.470384236 24.585578918457 5.0
42.647069973 32.8492546081543 6.0
42.647069973 33.9870300292969 7.0
36.24022533 35.651123046875 0.0
17.188554164 15.6389808654785 1.0
19.051671166 17.4906463623047 2.0
-6.1891251621 -4.91902208328247 3.0
12.862545995 12.329550743103 4.0
23.377679336 21.2563533782959 5.0
36.240225339 33.8327331542969 6.0
36.240225339 34.1460609436035 7.0
31.876756195 33.5770835876465 0.0
14.446454361 15.1055221557617 1.0
17.430301835 17.7299747467041 2.0
-5.0545683645 -4.42924308776855 3.0
12.375733461 12.4346895217896 4.0
19.501022735 19.2277336120605 5.0
31.876756203 32.9618301391602 6.0
31.876756203 33.9649848937988 7.0
38.711049649 30.5552616119385 0.0
19.068925854 15.7565546035767 1.0
19.642123794 16.7732067108154 2.0
-5.0802825502 -5.54787397384644 3.0
14.561841234 13.5922679901123 4.0
24.149208415 23.0438556671143 5.0
38.711049659 31.6106834411621 6.0
38.711049659 33.6378593444824 7.0
33.592635662 37.0204238891602 0.0
16.042625471 15.2427949905396 1.0
17.550010192 17.0209712982178 2.0
-5.9813560668 -4.81669855117798 3.0
11.568654115 11.5531988143921 4.0
22.023981548 21.4068470001221 5.0
33.592635672 34.8105392456055 6.0
33.592635672 34.2187576293945 7.0
31.9394970213296 32.5104751586914 0.0
15.7221064593445 16.4202671051025 1.0
16.2173905613348 16.5750274658203 2.0
-3.67606552949996 -3.97196102142334 3.0
12.5413250313355 11.8669061660767 4.0
19.3981719893576 19.3383960723877 5.0
31.939497021 32.6041679382324 6.0
31.939497021 34.2675590515137 7.0
32.605677366 36.680965423584 0.0
15.053063258 14.4753217697144 1.0
17.552614114 15.8210859298706 2.0
-5.3475005701 -4.91884803771973 3.0
12.20511354 10.9535608291626 4.0
20.400563832 19.9586524963379 5.0
32.605677369 34.5597076416016 6.0
32.605677369 34.2864570617676 7.0
35.02561424 39.0127983093262 0.0
18.375955543 18.5270900726318 1.0
16.649658698 17.6462936401367 2.0
-5.1216145805 -5.01193046569824 3.0
11.528044108 11.7129669189453 4.0
23.497570133 23.063606262207 5.0
35.02561425 35.7123146057129 6.0
35.02561425 33.8529472351074 7.0
29.155020361 36.2291069030762 0.0
13.928860816 15.9373149871826 1.0
15.226159546 17.6957263946533 2.0
-5.2043069028 -4.16568517684937 3.0
10.021852634 11.129734992981 4.0
19.133167728 19.3790149688721 5.0
29.155020371 33.7279167175293 6.0
29.155020371 34.043384552002 7.0
29.979099891 31.6718444824219 0.0
16.392071584 17.1893138885498 1.0
13.587028308 16.2122001647949 2.0
-4.5603535805 -3.90617609024048 3.0
9.026674718 10.1390771865845 4.0
20.952425173 20.6289672851562 5.0
29.9790999 32.9114418029785 6.0
29.9790999 33.3964080810547 7.0
40.412374419 35.7252349853516 0.0
21.632869342 16.7714138031006 1.0
18.779505078 14.8423585891724 2.0
-4.6354756287 -4.32313108444214 3.0
14.14402944 12.8718223571777 4.0
26.26834498 21.0950241088867 5.0
40.412374428 33.6467781066895 6.0
40.412374428 34.0167236328125 7.0
44.7996696066759 39.7463684082031 0.0
24.3565352806759 17.523754119873 1.0
20.4431343366759 14.4187602996826 2.0
-4.33926525477587 -4.31176376342773 3.0
16.1038690816759 14.4235601425171 4.0
28.6958005009671 23.041259765625 5.0
44.799669607 36.2712478637695 6.0
44.799669607 33.6098403930664 7.0
35.691930078 34.9851531982422 0.0
18.173171081 17.9451007843018 1.0
17.518759 17.0808162689209 2.0
-4.3926482071 -4.21977806091309 3.0
13.126110785 12.5567960739136 4.0
22.565819295 20.4272804260254 5.0
35.691930085 33.659309387207 6.0
35.691930085 33.2203483581543 7.0
37.293002373 39.6377029418945 0.0
17.018201783 14.9905309677124 1.0
20.274800591 17.9555988311768 2.0
-8.9983650782 -5.35573148727417 3.0
11.276435503 11.8130445480347 4.0
26.016566871 21.0475654602051 5.0
37.293002382 35.4098892211914 6.0
37.293002382 33.7149848937988 7.0
31.749050626 39.0323371887207 0.0
16.61083934 15.8318643569946 1.0
15.138211288 14.3148641586304 2.0
-3.9083704158 -3.53192186355591 3.0
11.229840863 10.9260292053223 4.0
20.519209764 19.5066146850586 5.0
31.749050635 34.578685760498 6.0
31.749050635 33.6626853942871 7.0
44.43859931 29.3492259979248 0.0
22.567248059 15.0583972930908 1.0
21.871351255 15.5943050384521 2.0
-10.274790847 -5.54835844039917 3.0
11.596560402 11.5623645782471 4.0
32.842038912 24.2594470977783 5.0
44.438599316 31.4348297119141 6.0
44.438599316 33.8482971191406 7.0
30.584762336 29.4397125244141 0.0
14.448986708 14.6281719207764 1.0
16.13577563 16.3343067169189 2.0
-7.0107275854 -3.81058979034424 3.0
9.1250480359 9.9442663192749 4.0
21.459714302 19.3645973205566 5.0
30.584762345 32.3465003967285 6.0
30.584762345 33.5037879943848 7.0
36.167711461 38.8430442810059 0.0
15.702748451 14.7272577285767 1.0
20.464963013 17.1283283233643 2.0
-4.5943051912 -4.99508714675903 3.0
15.870657815 14.8110198974609 4.0
20.297053649 19.5954990386963 5.0
36.167711467 34.8654556274414 6.0
36.167711467 34.3543281555176 7.0
45.589023155 30.1035232543945 0.0
22.43199889 14.743673324585 1.0
23.157024265 14.4721975326538 2.0
-9.6703339461 -5.42175960540771 3.0
13.486690309 12.1951446533203 4.0
32.102332846 24.5750713348389 5.0
45.589023165 32.2230796813965 6.0
45.589023165 34.1631202697754 7.0
33.954863965 39.9521217346191 0.0
16.569146941 14.7016754150391 1.0
17.385717024 16.5341644287109 2.0
-8.6921868392 -4.99864912033081 3.0
8.6935301746 9.85540580749512 4.0
25.26133379 21.9054775238037 5.0
33.954863975 36.2086601257324 6.0
33.954863975 33.5819854736328 7.0
39.104021209 35.2464828491211 0.0
19.726585997 16.1779861450195 1.0
19.377435214 16.734655380249 2.0
-5.8707549184 -5.7357931137085 3.0
13.506680288 12.6619520187378 4.0
25.597340923 23.8109378814697 5.0
39.104021217 33.9004364013672 6.0
39.104021217 33.759090423584 7.0
41.6151393745108 31.6570243835449 0.0
21.6144474948183 17.4887790679932 1.0
20.0006918800463 16.6872577667236 2.0
-5.68615976863137 -6.73096752166748 3.0
14.3145321112399 13.368631362915 4.0
27.3006071309532 28.4010944366455 5.0
41.615139375 32.3383560180664 6.0
41.615139375 33.5318069458008 7.0
29.251209159 29.6246509552002 0.0
16.305978933 17.2103824615479 1.0
12.945230226 14.5760974884033 2.0
-4.8421763952 -3.58072996139526 3.0
8.1030538205 8.81170749664307 4.0
21.148155339 21.1916961669922 5.0
29.251209169 31.4554576873779 6.0
29.251209169 33.7973937988281 7.0
39.145261593 36.3745613098145 0.0
20.853611949 16.5774402618408 1.0
18.291649647 14.8098068237305 2.0
-6.3281983222 -4.59273433685303 3.0
11.963451316 11.4083986282349 4.0
27.181810279 21.8860816955566 5.0
39.145261601 34.753776550293 6.0
39.145261601 33.8359107971191 7.0
43.724223504 30.3935451507568 0.0
24.240212841 16.6051120758057 1.0
19.484010663 13.4402713775635 2.0
-6.7543223688 -3.75004625320435 3.0
12.729688284 11.0718679428101 4.0
30.99453522 22.8024940490723 5.0
43.724223514 32.8792190551758 6.0
43.724223514 33.9521827697754 7.0
30.694427938 30.0532836914062 0.0
15.962392584 16.2410564422607 1.0
14.732035355 16.4674224853516 2.0
-5.8529078341 -3.52340126037598 3.0
8.8791275119 9.53335952758789 4.0
21.815300427 19.6019058227539 5.0
30.694427948 32.4429779052734 6.0
30.694427948 33.4532737731934 7.0
46.899446071 35.2382926940918 0.0
22.810606072 15.4748058319092 1.0
24.08884 19.269588470459 2.0
-12.207831656 -6.83966302871704 3.0
11.881008335 12.6984272003174 4.0
35.018437738 24.5828971862793 5.0
46.899446079 33.8261375427246 6.0
46.899446079 33.647102355957 7.0
30.801741864 36.0099258422852 0.0
14.581609416 14.6503381729126 1.0
16.220132448 17.0604114532471 2.0
-7.6078809929 -5.38506460189819 3.0
8.6122514454 9.80400848388672 4.0
22.189490419 21.8598651885986 5.0
30.801741874 34.3720321655273 6.0
30.801741874 33.4778213500977 7.0
40.3889007570329 30.6853199005127 0.0
21.8864546730297 17.2987442016602 1.0
18.5024460860314 18.3583602905273 2.0
-8.35977776734346 -5.92380523681641 3.0
10.1426683190367 10.8475532531738 4.0
30.2462322648819 26.8716087341309 5.0
40.388900757 32.5164260864258 6.0
40.388900757 33.9437103271484 7.0
35.801686121 30.9148635864258 0.0
19.152651215 18.0983448028564 1.0
16.649034906 14.8419103622437 2.0
-1.2326560583 -2.99909782409668 3.0
15.416378838 13.7993221282959 4.0
20.385307283 19.4257335662842 5.0
35.801686131 31.3884925842285 6.0
35.801686131 33.3113403320312 7.0
43.275430292 32.328742980957 0.0
21.49768275 17.0301151275635 1.0
21.777747542 18.8957271575928 2.0
-8.5057364073 -6.40105199813843 3.0
13.272011125 13.1933784484863 4.0
30.003419167 25.2573680877686 5.0
43.275430302 33.5512351989746 6.0
43.275430302 33.5783424377441 7.0
36.6745338859035 34.9357223510742 0.0
17.2258453439028 14.9914064407349 1.0
19.4486885479032 17.3068943023682 2.0
-7.45674864010256 -5.52762222290039 3.0
11.9919399079021 12.0305862426758 4.0
24.6825939849038 22.3748912811279 5.0
36.674533885 33.9407005310059 6.0
36.674533885 33.9301605224609 7.0
42.2919299411196 31.747314453125 0.0
20.8701371011196 16.1702861785889 1.0
21.4217928491193 18.2614994049072 2.0
-7.8990978219197 -5.64287328720093 3.0
13.5226950281196 12.8248043060303 4.0
28.769234923119 24.1278553009033 5.0
42.291929941 32.4164352416992 6.0
42.291929941 33.6531028747559 7.0
37.80686854 38.5306434631348 0.0
18.84529386 15.2375946044922 1.0
18.961574682 14.8653011322021 2.0
-3.0668288295 -5.34491634368896 3.0
15.894745843 13.9692411422729 4.0
21.912122698 21.6556816101074 5.0
37.80686855 34.7052230834961 6.0
37.80686855 33.6482887268066 7.0
36.296445798 38.0738296508789 0.0
19.144127702 17.8671798706055 1.0
17.152318097 17.0495109558105 2.0
-4.4799623055 -5.04750394821167 3.0
12.672355782 12.4914073944092 4.0
23.624090017 22.5783462524414 5.0
36.296445808 34.8384170532227 6.0
36.296445808 33.5053634643555 7.0
26.786347975 31.4964599609375 0.0
13.739106858 17.9971408843994 1.0
13.047241117 19.3294258117676 2.0
-5.0192652199 -3.33759880065918 3.0
8.0279758877 10.3339319229126 4.0
18.758372088 18.8393230438232 5.0
26.786347985 32.7799034118652 6.0
26.786347985 33.1244430541992 7.0
38.715518927 28.448600769043 0.0
20.971478582 16.7664470672607 1.0
17.744040345 16.6628322601318 2.0
-7.5775969749 -5.16578578948975 3.0
10.16644336 11.3448362350464 4.0
28.549075567 24.6456413269043 5.0
38.715518937 31.0959453582764 6.0
38.715518937 34.0066070556641 7.0
35.505062995 40.2076950073242 0.0
16.975132375 15.2601003646851 1.0
18.529930624 17.0839996337891 2.0
-6.1934391103 -5.53228902816772 3.0
12.336491507 12.0518293380737 4.0
23.168571492 22.2807483673096 5.0
35.505063002 36.3301200866699 6.0
35.505063002 33.9421081542969 7.0
44.227649061 32.514404296875 0.0
23.455343466 18.3437023162842 1.0
20.772305597 18.1581764221191 2.0
-7.272984789 -4.76472949981689 3.0
13.499320798 13.1022129058838 4.0
30.728328264 22.8649921417236 5.0
44.227649071 33.3443222045898 6.0
44.227649071 33.4186553955078 7.0
45.773124246 28.6814308166504 0.0
24.134680774 15.7350463867188 1.0
21.638443473 14.1221942901611 2.0
-6.2883820036 -6.04140710830688 3.0
15.350061461 13.5572004318237 4.0
30.423062786 25.361270904541 5.0
45.773124255 31.7443237304688 6.0
45.773124255 33.5842514038086 7.0
38.6152827522378 40.0875701904297 0.0
18.6636500942378 14.5042037963867 1.0
19.9516326682378 15.0526485443115 2.0
-8.33253161363777 -5.25962734222412 3.0
11.6191010542378 10.9061679840088 4.0
26.9961816644691 23.8669528961182 5.0
38.615282752 36.1737403869629 6.0
38.615282752 33.3016166687012 7.0
42.131124826 32.2281150817871 0.0
21.88265512 16.8488368988037 1.0
20.248469706 15.887996673584 2.0
-6.067617241 -5.22714805603027 3.0
14.180852455 13.061131477356 4.0
27.950272371 24.1720790863037 5.0
42.131124836 32.5667953491211 6.0
42.131124836 33.5633506774902 7.0
38.030369756 33.9336318969727 0.0
19.698625144 14.4157180786133 1.0
18.331744611 13.6183109283447 2.0
-8.0645153211 -4.93238210678101 3.0
10.26722928 9.70727062225342 4.0
27.763140475 24.5901679992676 5.0
38.030369766 33.3385963439941 6.0
38.030369766 33.8841857910156 7.0
32.694869995 28.2354202270508 0.0
14.429732866 14.6924161911011 1.0
18.265137131 18.7138900756836 2.0
-5.8301144142 -5.21766138076782 3.0
12.435022709 12.7065553665161 4.0
20.259847288 20.5212535858154 5.0
32.694870002 32.2968330383301 6.0
32.694870002 33.3852615356445 7.0
41.58883183 32.5008277893066 0.0
19.184872754 16.1197566986084 1.0
22.403959077 19.0204772949219 2.0
-6.437067104 -5.55561780929565 3.0
15.966891964 14.8530149459839 4.0
25.621939867 22.7263126373291 5.0
41.58883184 32.8825874328613 6.0
41.58883184 33.0800323486328 7.0
33.622063035 38.7408714294434 0.0
16.212159516 14.4588088989258 1.0
17.409903519 17.6352767944336 2.0
-8.2486263793 -5.33488988876343 3.0
9.16127713 10.2345142364502 4.0
24.460785905 22.3743724822998 5.0
33.622063045 35.2449684143066 6.0
33.622063045 33.9338455200195 7.0
40.101738497 37.4102554321289 0.0
22.613544165 17.7214317321777 1.0
17.488194333 15.2913103103638 2.0
-8.294425066 -5.23917245864868 3.0
9.1937692583 10.4266338348389 4.0
30.90796924 25.2358722686768 5.0
40.101738506 35.0578727722168 6.0
40.101738506 34.2817230224609 7.0
39.347750484 36.6980094909668 0.0
19.631600055 17.9547901153564 1.0
19.716150429 19.9317741394043 2.0
-7.2110260491 -5.50476503372192 3.0
12.50512437 12.4341650009155 4.0
26.842626114 23.0375061035156 5.0
39.347750494 34.2221412658691 6.0
39.347750494 35.0439643859863 7.0
46.240008453 33.699089050293 0.0
23.277257726 16.3648014068604 1.0
22.962750727 16.6865730285645 2.0
-6.5077704351 -5.97384166717529 3.0
16.454980282 14.9838809967041 4.0
29.785028171 25.0482807159424 5.0
46.240008463 32.6008644104004 6.0
46.240008463 34.1182060241699 7.0
43.885613963 34.3817520141602 0.0
23.043389811 18.6282272338867 1.0
20.842224153 16.6199913024902 2.0
-5.1243468199 -4.80161046981812 3.0
15.717877324 14.9299221038818 4.0
28.16773664 22.4820156097412 5.0
43.885613972 33.7629013061523 6.0
43.885613972 33.8954887390137 7.0
44.877286045 30.726245880127 0.0
23.85937511 15.2867832183838 1.0
21.017910935 13.9307146072388 2.0
-10.548874854 -6.07728052139282 3.0
10.469036071 10.4736089706421 4.0
34.408249974 26.8969993591309 5.0
44.877286055 32.358772277832 6.0
44.877286055 33.4557342529297 7.0
44.28102484 35.0894622802734 0.0
23.476033214 17.4718990325928 1.0
20.804991626 16.8305759429932 2.0
-7.5751269732 -6.74009895324707 3.0
13.229864644 12.6541776657104 4.0
31.051160197 27.716064453125 5.0
44.281024849 33.946834564209 6.0
44.281024849 34.1309127807617 7.0
39.443195324 38.2572059631348 0.0
17.235953612 13.9859733581543 1.0
22.207241712 18.0403881072998 2.0
-9.075573722 -6.09334087371826 3.0
13.13166798 13.0337705612183 4.0
26.311527344 24.5969085693359 5.0
39.443195334 35.3545684814453 6.0
39.443195334 33.5288772583008 7.0
44.5099801195094 37.2841758728027 0.0
20.7192442195094 14.961820602417 1.0
23.7907359105096 17.3472805023193 2.0
-9.37764966100969 -6.71814203262329 3.0
14.4130862495096 13.5930738449097 4.0
30.0968938805093 26.3724689483643 5.0
44.50998012 34.1292877197266 6.0
44.50998012 33.8107757568359 7.0
36.079576813 28.7741737365723 0.0
18.198560719 14.7033405303955 1.0
17.881016095 16.0131359100342 2.0
-8.6789161085 -6.03257703781128 3.0
9.202099978 10.274754524231 4.0
26.877476836 23.8448333740234 5.0
36.079576822 31.7993812561035 6.0
36.079576822 34.023796081543 7.0
43.881758866 38.6646766662598 0.0
21.090009056 17.8006706237793 1.0
22.79174981 19.2831363677979 2.0
-6.6710954927 -6.50571155548096 3.0
16.120654308 14.9245281219482 4.0
27.761104558 25.8370990753174 5.0
43.881758876 36.1264877319336 6.0
43.881758876 33.9708786010742 7.0
30.108818274 34.3263282775879 0.0
17.112174764 17.4314727783203 1.0
12.996643511 14.2507219314575 2.0
-4.8964528045 -2.79301357269287 3.0
8.1001906976 8.95355701446533 4.0
22.008627577 20.3096199035645 5.0
30.108818284 33.3980674743652 6.0
30.108818284 33.5023384094238 7.0
34.2205121900012 28.9171619415283 0.0
18.8110553650029 17.6510524749756 1.0
15.4094568300067 17.3750953674316 2.0
-7.05168531729669 -5.01790857315063 3.0
8.35777151259514 10.1205625534058 4.0
25.8627406830075 23.6768493652344 5.0
34.22051219 31.9387912750244 6.0
34.22051219 33.4926567077637 7.0
37.402197962 36.4077987670898 0.0
19.960548139 16.2688961029053 1.0
17.441649823 14.7315158843994 2.0
-5.9560955294 -5.1681752204895 3.0
11.485554285 10.9642353057861 4.0
25.916643678 24.3823375701904 5.0
37.402197971 34.4821472167969 6.0
37.402197971 33.338436126709 7.0
35.453722399 39.5826530456543 0.0
16.625051972 15.4293127059937 1.0
18.828670427 18.7067394256592 2.0
-7.5918246106 -5.53767728805542 3.0
11.236845807 11.5582447052002 4.0
24.216876592 22.0829601287842 5.0
35.453722409 36.424919128418 6.0
35.453722409 33.626781463623 7.0
32.872381145 36.6534957885742 0.0
15.679111706 15.872579574585 1.0
17.193269439 18.1069278717041 2.0
-6.3769976884 -4.42827653884888 3.0
10.816271741 11.1519060134888 4.0
22.056109405 20.453483581543 5.0
32.872381155 34.3324317932129 6.0
32.872381155 33.8332939147949 7.0
33.5777143162235 38.7074279785156 0.0
24.3043372519187 14.9099750518799 1.0
9.27337706863705 17.255880355835 2.0
-1.413226005e-07 -4.59070301055908 3.0
9.27337692732672 10.3223133087158 4.0
24.3043373936238 20.5717811584473 5.0
33.577714316 35.8395385742188 6.0
33.577714316 33.4776000976562 7.0
43.986168792 36.6679725646973 0.0
19.73252431 14.9623546600342 1.0
24.253644482 17.6132183074951 2.0
-7.8587048092 -5.36849498748779 3.0
16.394939663 15.5901651382446 4.0
27.591229129 22.4995460510254 5.0
43.986168802 33.9040908813477 6.0
43.986168802 33.6171417236328 7.0
39.701527196 38.7573051452637 0.0
21.433905404 17.6968536376953 1.0
18.267621793 15.9954786300659 2.0
-6.9530592881 -4.11878108978271 3.0
11.314562495 10.8653030395508 4.0
28.386964701 21.813892364502 5.0
39.701527205 36.1121711730957 6.0
39.701527205 34.9335250854492 7.0
36.319422307 39.7003402709961 0.0
18.261170173 14.3986167907715 1.0
18.058252134 14.7858533859253 2.0
-9.3076661868 -5.76422834396362 3.0
8.7505859373 10.081654548645 4.0
27.56883637 25.0770874023438 5.0
36.319422317 37.1622657775879 6.0
36.319422317 33.7271499633789 7.0
39.546529711 37.2238426208496 0.0
21.883828824 17.1145324707031 1.0
17.662700888 14.25901222229 2.0
-6.2516235274 -4.56838274002075 3.0
11.411077352 11.0761566162109 4.0
28.13545236 22.7586669921875 5.0
39.54652972 34.6059532165527 6.0
39.54652972 33.714038848877 7.0
46.328884399 31.9372444152832 0.0
20.182008475 14.4857053756714 1.0
26.146875933 17.9420928955078 2.0
-10.473669833 -5.89632177352905 3.0
15.673206099 14.871187210083 4.0
30.655678309 26.0767498016357 5.0
46.3288844 32.5125198364258 6.0
46.3288844 33.4636497497559 7.0
39.359671453 28.5952739715576 0.0
20.208221774 16.3762645721436 1.0
19.151449681 16.9979953765869 2.0
-7.7565035076 -5.74906444549561 3.0
11.394946165 11.4689083099365 4.0
27.964725289 23.9501571655273 5.0
39.359671461 31.9718132019043 6.0
39.359671461 33.7745666503906 7.0
43.813870978 36.0474510192871 0.0
22.861832963 17.909008026123 1.0
20.952038016 18.2287158966064 2.0
-6.6064627953 -6.06154012680054 3.0
14.345575211 13.6596145629883 4.0
29.468295768 25.7418365478516 5.0
43.813870988 34.5504417419434 6.0
43.813870988 33.4733924865723 7.0
39.146221866 40.0418548583984 0.0
21.274977313 17.4315547943115 1.0
17.871244553 16.6177749633789 2.0
-8.221666714 -5.53723192214966 3.0
9.6495778292 10.7921257019043 4.0
29.496644037 25.0240097045898 5.0
39.146221876 37.237850189209 6.0
39.146221876 34.8084754943848 7.0
34.753353004 38.8173942565918 0.0
19.306587872 15.954761505127 1.0
15.446765133 13.9059505462646 2.0
-7.4077231802 -4.64190530776978 3.0
8.0390419425 9.53512859344482 4.0
26.714311062 23.5445442199707 5.0
34.753353014 34.3567276000977 6.0
34.753353014 33.9307823181152 7.0
36.8589150083515 33.2072715759277 0.0
18.5606086093515 16.8830680847168 1.0
18.2983064083517 18.9252109527588 2.0
-8.423068234052 -5.88819742202759 3.0
9.87523817465195 11.0625047683716 4.0
26.9836767778432 24.9868221282959 5.0
36.858915008 32.8523445129395 6.0
36.858915008 33.9999961853027 7.0
45.609388715 31.2446365356445 0.0
20.431056468 15.0627431869507 1.0
25.178332247 17.774694442749 2.0
-8.8685199312 -6.35565948486328 3.0
16.309812306 15.1470594406128 4.0
29.299576409 25.0435848236084 5.0
45.609388725 32.1097564697266 6.0
45.609388725 34.5055198669434 7.0
37.221502305 30.8088073730469 0.0
19.795691461 15.9629573822021 1.0
17.425810844 15.2280263900757 2.0
-8.0004149655 -4.81371450424194 3.0
9.4253958691 10.4327259063721 4.0
27.796106436 23.8259353637695 5.0
37.221502314 32.1688423156738 6.0
37.221502314 33.8627624511719 7.0
39.3512418478291 31.9741287231445 0.0
21.5003115557496 16.7360877990723 1.0
17.8509302889164 15.5486125946045 2.0
-7.9913457065862 -4.34211540222168 3.0
9.85958458276425 9.88648414611816 4.0
29.4916572628094 22.0274353027344 5.0
39.351241848 32.3752899169922 6.0
39.351241848 33.9843063354492 7.0
44.354286129 36.6340103149414 0.0
23.778468789 17.9034156799316 1.0
20.57581734 16.9055023193359 2.0
-6.7304183908 -5.76015567779541 3.0
13.845398939 12.997673034668 4.0
30.50888719 27.2311401367188 5.0
44.354286139 34.766902923584 6.0
44.354286139 33.8685073852539 7.0
41.1438439687938 38.9571723937988 0.0
20.0722524531884 14.4900884628296 1.0
21.0715915195 15.151801109314 2.0
-7.89338141352269 -5.68438482284546 3.0
13.1782101064826 12.1936016082764 4.0
27.9656338666077 24.5349502563477 5.0
41.143843969 34.3953971862793 6.0
41.143843969 34.2173385620117 7.0
32.436363662 34.8178482055664 0.0
18.29560079 18.4686908721924 1.0
14.140762872 14.9959449768066 2.0
-4.3766657499 -3.82969236373901 3.0
9.7640971121 9.62876605987549 4.0
22.67226655 22.200569152832 5.0
32.436363672 33.5189514160156 6.0
32.436363672 33.9627952575684 7.0
35.980268817 35.3183517456055 0.0
18.371260578 15.1230173110962 1.0
17.609008241 15.5471143722534 2.0
-7.5074447835 -5.5122447013855 3.0
10.101563449 10.6363067626953 4.0
25.87870537 24.0485286712646 5.0
35.980268826 34.1923446655273 6.0
35.980268826 33.7435607910156 7.0
39.361696941 31.1427326202393 0.0
18.798616943 14.2086429595947 1.0
20.563079997 14.1901798248291 2.0
-4.5015726852 -5.84841108322144 3.0
16.061507302 13.9437770843506 4.0
23.300189639 22.0539417266846 5.0
39.361696951 32.4823760986328 6.0
39.361696951 33.7748146057129 7.0
38.339957491 38.7772064208984 0.0
20.325182647 17.8770370483398 1.0
18.014774844 17.3001270294189 2.0
-6.00621 -5.89951324462891 3.0
12.008564835 12.0936346054077 4.0
26.331392656 26.5032272338867 5.0
38.339957501 34.8591842651367 6.0
38.339957501 33.2859535217285 7.0
38.4370780266379 38.0155601501465 0.0
19.1112284596379 15.3477582931519 1.0
19.3258495766379 15.6591644287109 2.0
-8.41787501523793 -5.59217214584351 3.0
10.9079745616379 11.1226491928101 4.0
27.5291034382175 24.1202335357666 5.0
38.437078026 35.3470420837402 6.0
38.437078026 33.4934921264648 7.0
31.680301152 40.0975303649902 0.0
16.922816031 15.7024564743042 1.0
14.757485122 16.1281623840332 2.0
-6.2787127987 -3.761549949646 3.0
8.4787723148 10.1772956848145 4.0
23.201528838 20.0324840545654 5.0
31.680301161 36.7592086791992 6.0
31.680301161 33.6372756958008 7.0
36.698769122 29.2440223693848 0.0
19.594432034 16.256778717041 1.0
17.104337088 14.1620864868164 2.0
-2.472193454 -3.69893503189087 3.0
14.632143624 13.1632289886475 4.0
22.066625498 20.1721630096436 5.0
36.698769132 31.0709438323975 6.0
36.698769132 34.0148124694824 7.0
40.942334912 34.6292152404785 0.0
22.219375241 17.199182510376 1.0
18.722959671 14.8356895446777 2.0
-5.2427967213 -4.72442674636841 3.0
13.48016294 12.4669904708862 4.0
27.462171973 23.0916938781738 5.0
40.942334922 33.6989059448242 6.0
40.942334922 34.1802558898926 7.0
41.496981527 32.1543273925781 0.0
18.951043784 14.8190431594849 1.0
22.545937743 16.9064350128174 2.0
-7.7537930795 -5.4674859046936 3.0
14.792144654 12.9434995651245 4.0
26.704836873 22.7038993835449 5.0
41.496981537 32.6800193786621 6.0
41.496981537 33.8513717651367 7.0
33.577466055 33.7556533813477 0.0
17.755510174 15.8395500183105 1.0
15.821955881 15.1690731048584 2.0
-7.358985858 -4.25211238861084 3.0
8.4629700135 9.94533824920654 4.0
25.114496042 21.8017177581787 5.0
33.577466065 32.7715110778809 6.0
33.577466065 33.54736328125 7.0
38.410697594746 38.6179389953613 0.0
20.0282722997678 15.2040672302246 1.0
18.3824252977607 13.5048837661743 2.0
-2.25462323941058 -5.19568681716919 3.0
16.1278020587646 14.344331741333 4.0
22.2828955387576 21.8681945800781 5.0
38.410697595 35.276611328125 6.0
38.410697595 33.6963043212891 7.0
40.390522102 37.3230476379395 0.0
23.101316011 18.3139419555664 1.0
17.289206093 15.6155853271484 2.0
-8.5131515481 -4.42097854614258 3.0
8.7760545352 9.80537986755371 4.0
31.614467568 23.3512268066406 5.0
40.390522112 35.0877685546875 6.0
40.390522112 34.0990409851074 7.0
38.02724698 31.6664772033691 0.0
17.092159812 14.3190956115723 1.0
20.935087168 16.4159278869629 2.0
-7.0096758313 -5.2364387512207 3.0
13.925411327 13.1337728500366 4.0
24.101835653 21.3973541259766 5.0
38.02724699 32.4522361755371 6.0
38.02724699 34.3166007995605 7.0
34.831322289 39.5478134155273 0.0
17.981428364 17.2845706939697 1.0
16.849893928 17.4903163909912 2.0
-5.7085713019 -3.85347414016724 3.0
11.14132262 11.3087062835693 4.0
23.689999673 20.160457611084 5.0
34.831322295 35.1531410217285 6.0
34.831322295 33.8257484436035 7.0
33.152476822 39.8199653625488 0.0
16.395481675 15.3338613510132 1.0
16.756995148 16.5964641571045 2.0
-6.4008279506 -4.05791234970093 3.0
10.356167188 11.1114530563354 4.0
22.796309635 19.8207931518555 5.0
33.152476832 35.809009552002 6.0
33.152476832 33.5381622314453 7.0
37.7618147597387 33.1659507751465 0.0
20.5714887837879 17.3252868652344 1.0
17.1903259800247 15.0509977340698 2.0
-5.43010176849644 -5.30596494674683 3.0
11.7602242121808 11.2428779602051 4.0
26.0015905517818 24.5269603729248 5.0
37.76181476 32.9468078613281 6.0
37.76181476 33.5687828063965 7.0
46.821650957 31.8533668518066 0.0
22.453957792 15.1412982940674 1.0
24.367693165 17.4647979736328 2.0
-10.797760543 -6.550940990448 3.0
13.569932612 12.4222259521484 4.0
33.251718344 26.5544548034668 5.0
46.821650967 32.7579689025879 6.0
46.821650967 33.7291946411133 7.0
41.1837901754312 29.0738277435303 0.0
19.2503118354314 14.7275848388672 1.0
21.9334783494314 15.0319652557373 2.0
-7.5443023473312 -5.02210998535156 3.0
14.3891760014314 13.0047645568848 4.0
26.7946141824314 21.4453201293945 5.0
41.183790175 31.6111907958984 6.0
41.183790175 33.5149536132812 7.0
36.784988632 32.858757019043 0.0
21.035700431 17.1812133789062 1.0
15.749288201 13.8982257843018 2.0
-6.0648376105 -5.77710151672363 3.0
9.6844505818 9.98845195770264 4.0
27.100538051 25.3605537414551 5.0
36.78498864 32.7570343017578 6.0
36.78498864 34.2344284057617 7.0
37.6894949442102 37.0628356933594 0.0
17.6138134052102 14.7518558502197 1.0
20.0756815502102 18.3886547088623 2.0
-8.78806436921022 -6.7877631187439 3.0
11.2876171802102 11.7476825714111 4.0
26.401877737898 27.227144241333 5.0
37.689494945 34.6603202819824 6.0
37.689494945 33.364875793457 7.0
41.308881088 39.5564956665039 0.0
21.148817904 15.9486713409424 1.0
20.160063185 15.3547315597534 2.0
-6.8065142938 -5.42001295089722 3.0
13.353548882 12.2115602493286 4.0
27.955332207 23.3726654052734 5.0
41.308881097 36.7083778381348 6.0
41.308881097 34.335620880127 7.0
42.253011175 37.8710403442383 0.0
21.557015891 14.8540687561035 1.0
20.695995284 14.6492462158203 2.0
-9.3835381575 -5.5439305305481 3.0
11.312457118 10.9508085250854 4.0
30.940554058 24.9080257415771 5.0
42.253011184 33.8033332824707 6.0
42.253011184 34.1417541503906 7.0
40.047660483 32.6160163879395 0.0
19.392839472 15.6141967773438 1.0
20.654821011 17.5102920532227 2.0
-7.4809136797 -5.49192094802856 3.0
13.173907322 12.4142723083496 4.0
26.873753161 23.7096862792969 5.0
40.047660493 32.7525978088379 6.0
40.047660493 33.6711006164551 7.0
46.604605723 39.7352676391602 0.0
22.954120862 15.5669937133789 1.0
23.650484861 16.4951171875 2.0
-7.9204617962 -6.07011651992798 3.0
15.730023055 14.2584581375122 4.0
30.874582668 24.3416595458984 5.0
46.604605733 35.3487205505371 6.0
46.604605733 33.8227615356445 7.0
45.106295532 39.2217903137207 0.0
23.2126055 16.8969955444336 1.0
21.893690031 18.3119583129883 2.0
-10.542381522 -6.02882862091064 3.0
11.3513085 11.7695627212524 4.0
33.754987032 24.9838523864746 5.0
45.106295542 35.4851875305176 6.0
45.106295542 33.6553268432617 7.0
38.977055089 32.0398826599121 0.0
19.684728957 16.3312511444092 1.0
19.292326132 18.1248512268066 2.0
-8.6831116468 -5.50917291641235 3.0
10.609214475 11.4509048461914 4.0
28.367840614 23.493709564209 5.0
38.977055099 32.8834114074707 6.0
38.977055099 33.6880950927734 7.0
38.935365873 36.0888595581055 0.0
21.088579762 18.4436550140381 1.0
17.846786112 15.6518077850342 2.0
-3.2985252296 -4.15984678268433 3.0
14.548260873 13.1082363128662 4.0
24.387105001 21.6480541229248 5.0
38.935365883 33.2761611938477 6.0
38.935365883 33.9566230773926 7.0
42.74140991 37.1608810424805 0.0
20.9458217 16.2616882324219 1.0
21.795588212 17.211540222168 2.0
-6.3794891462 -5.44038248062134 3.0
15.416099058 14.1448545455933 4.0
27.325310854 23.2354640960693 5.0
42.741409918 34.9537010192871 6.0
42.741409918 33.8908958435059 7.0
31.310794567 37.8871879577637 0.0
16.880318068 17.2633247375488 1.0
14.4304765 16.0284023284912 2.0
-5.6188756586 -3.84900856018066 3.0
8.8116008333 10.2595291137695 4.0
22.499193736 21.4519824981689 5.0
31.310794576 35.7229957580566 6.0
31.310794576 34.0540237426758 7.0
};
\addlegendentry{$R^2$=0.899}
\end{axis}

\end{tikzpicture}
}}
    
    \caption{Model results using only the loss associated with nodal flow predictions in the 8-node network.}
    \label{fig:dummy_base_results}
\end{figure}



In the final stage of the experiment, only the losses associated with nodal flows and the Weymouth equation were considered. The optimal hyperparameters for this configuration were $ N channels=22$, $ N layers=1$, and $ N dense=19$. These parameters yielded a total loss of 2.798, entirely attributed to the node loss, while the Weymouth loss was effectively zero.

The node predictions, as depicted in \cref{fig:results_nonlineal_dummy_node_base_wey}, continued to perform similarly to most of the previous tests, with an $R^2$ of 0.983, indicating consistent and accurate predictions of gas injection patterns at the nodes.

However, the edge predictions, shown in \cref{fig:results_nonlineal_dummy_edge_base_wey}, were significantly off target in this case. The model struggled to generalize edge flows, resulting in a drastically negative $R^2$ of -2.32, signaling a complete failure in predicting gas flows through the network's edges. 

\begin{figure}
    \centering
    \setlength\figurewidth{.53\textwidth}        
    \setlength\figureheight{0.36\textwidth} 
    \subfloat[Actual vs predicted nodal flows.] 
    {\label{fig:results_nonlineal_dummy_node_base_wey}\resizebox{\figurewidth}{\figureheight}{% This file was created with tikzplotlib v0.10.1.
\begin{tikzpicture}

\definecolor{darkgray176}{RGB}{176,176,176}
\definecolor{lightgray204}{RGB}{204,204,204}

\begin{axis}[
colorbar,
colorbar style={ylabel={node id}},
colormap={mymap}{[1pt]
 rgb(0pt)=(0.12156862745098,0.466666666666667,0.705882352941177);
  rgb(1pt)=(1,0.498039215686275,0.0549019607843137);
  rgb(2pt)=(0.172549019607843,0.627450980392157,0.172549019607843);
  rgb(3pt)=(0.83921568627451,0.152941176470588,0.156862745098039);
  rgb(4pt)=(0.580392156862745,0.403921568627451,0.741176470588235);
  rgb(5pt)=(0.549019607843137,0.337254901960784,0.294117647058824);
  rgb(6pt)=(0.890196078431372,0.466666666666667,0.76078431372549);
  rgb(7pt)=(0.498039215686275,0.498039215686275,0.498039215686275);
  rgb(8pt)=(0.737254901960784,0.741176470588235,0.133333333333333);
  rgb(9pt)=(0.0901960784313725,0.745098039215686,0.811764705882353)
},
legend cell align={left},
legend style={
  fill opacity=0.8,
  draw opacity=1,
  text opacity=1,
  at={(0.03,0.97)},
  anchor=north west,
  draw=lightgray204
},
point meta max=7,
point meta min=0,
tick align=outside,
tick pos=left,
title={},
x grid style={darkgray176},
xlabel={True},
xmajorgrids,
xmin=-2.43954548935, xmax=51.23045527635,
xtick style={color=black},
xtick={0,10,20,30,40,50}, 
xticklabels={0,10,20,30,40,$f_n$},
y grid style={darkgray176},
ylabel={Predicted},
ymajorgrids,
ymin=-2.86503676474094, ymax=44.649845340848,
ytick={0,10,20,30,40}, 
yticklabels={0,10,20,30,$f_n$},
ytick style={color=black}
]
\addplot [
  colormap={mymap}{[1pt]
 rgb(0pt)=(0.12156862745098,0.466666666666667,0.705882352941177);
  rgb(1pt)=(1,0.498039215686275,0.0549019607843137);
  rgb(2pt)=(0.172549019607843,0.627450980392157,0.172549019607843);
  rgb(3pt)=(0.83921568627451,0.152941176470588,0.156862745098039);
  rgb(4pt)=(0.580392156862745,0.403921568627451,0.741176470588235);
  rgb(5pt)=(0.549019607843137,0.337254901960784,0.294117647058824);
  rgb(6pt)=(0.890196078431372,0.466666666666667,0.76078431372549);
  rgb(7pt)=(0.498039215686275,0.498039215686275,0.498039215686275);
  rgb(8pt)=(0.737254901960784,0.741176470588235,0.133333333333333);
  rgb(9pt)=(0.0901960784313725,0.745098039215686,0.811764705882353)
},
  only marks,
  scatter,
  scatter src=explicit
]
table [x=x, y=y, meta=colordata]{%
x  y  colordata
39.565898645 40.5426597595215 0.0
0 0.145092219114304 1.0
0 0.0260390639305115 2.0
0 0.110726118087769 3.0
0 0.171965956687927 4.0
0 0.0582078099250793 5.0
0 0.0586434006690979 6.0
0 0.0884757339954376 7.0
42.743257181 38.7022247314453 0.0
0 0.0801073610782623 1.0
0 -0.25225043296814 2.0
0 0.32738333940506 3.0
0 -0.0145607590675354 4.0
0 0.00357294082641602 5.0
0 0.0906016230583191 6.0
0 0.045581579208374 7.0
39.367180751 37.0834579467773 0.0
0 0.128688424825668 1.0
0 -0.56534618139267 2.0
0 0.435968279838562 3.0
0 0.142634123563766 4.0
0 0.328184515237808 5.0
0 0.068950355052948 6.0
0 0.0867062211036682 7.0
39.605393107 38.2855529785156 0.0
0 0.231143891811371 1.0
0 -0.0859021544456482 2.0
0 0.271660625934601 3.0
0 0.183865010738373 4.0
0 0.104519039392471 5.0
0 0.0296648740768433 6.0
0 0.0761119723320007 7.0
43.937345536 39.3159027099609 0.0
0 0.187468826770782 1.0
0 -0.120137631893158 2.0
0 -0.031471848487854 3.0
0 0.226466596126556 4.0
0 0.0802088677883148 5.0
0 0.0898533761501312 6.0
0 0.052675187587738 7.0
31.061989594 37.9811553955078 0.0
0 0.0557106137275696 1.0
0 -0.133005023002625 2.0
0 -0.0267357230186462 3.0
0 -0.0159550905227661 4.0
0 -0.109172761440277 5.0
0 0.0874660909175873 6.0
0 0.0875094532966614 7.0
36.357435273 39.7526588439941 0.0
0 0.475311994552612 1.0
0 -0.00562405586242676 2.0
0 0.182295322418213 3.0
0 0.0519831776618958 4.0
0 0.0228008031845093 5.0
0 0.110404491424561 6.0
0 0.051629364490509 7.0
38.444969617 39.6394348144531 0.0
0 0.148074805736542 1.0
0 -0.191113770008087 2.0
0 0.182699203491211 3.0
0 0.180855989456177 4.0
0 0.0612456798553467 5.0
0 0.0782363414764404 6.0
0 0.0598953366279602 7.0
35.498620524 38.9670143127441 0.0
0 0.108738660812378 1.0
0 -0.158118069171906 2.0
0 0.266199767589569 3.0
0 0.0757691264152527 4.0
0 0.10127454996109 5.0
0 0.0250117778778076 6.0
0 0.0900531411170959 7.0
36.520998279 39.9478378295898 0.0
0 0.134771972894669 1.0
0 -0.159210383892059 2.0
0 0.119450390338898 3.0
0 0.274714291095734 4.0
0 -0.0267781615257263 5.0
0 0.0455420613288879 6.0
0 0.0699519515037537 7.0
36.717272212 39.7521438598633 0.0
0 0.124755322933197 1.0
0 -0.0882979035377502 2.0
0 0.317742794752121 3.0
0 0.195310652256012 4.0
0 0.0247281789779663 5.0
0 0.144979596138 6.0
0 0.105899572372437 7.0
32.629629006 39.3629341125488 0.0
0 0.199994325637817 1.0
0 1.7229220867157 2.0
0 0.503685891628265 3.0
0 0.0664852261543274 4.0
0 -0.0290243625640869 5.0
0 0.052879273891449 6.0
0 0.095192551612854 7.0
37.75267434 38.874626159668 0.0
0 0.0642233490943909 1.0
0 -0.230347633361816 2.0
0 0.131712853908539 3.0
0 0.0963107645511627 4.0
0 0.213985025882721 5.0
0 0.0555095076560974 6.0
0 0.107336759567261 7.0
38.800291347 39.143310546875 0.0
0 0.115271866321564 1.0
0 -0.190921127796173 2.0
0 -0.00386232137680054 3.0
0 -0.175939619541168 4.0
0 -0.0258618593215942 5.0
0 0.0998800992965698 6.0
0 0.105991542339325 7.0
38.252729618 39.1083564758301 0.0
0 0.162024199962616 1.0
0 -0.141071617603302 2.0
0 0.121394395828247 3.0
0 0.162390947341919 4.0
0 0.197208404541016 5.0
0 0.103644371032715 6.0
0 0.0893900394439697 7.0
43.273596047 39.209156036377 0.0
0 0.140776455402374 1.0
0 -0.294976890087128 2.0
0 -0.0821849703788757 3.0
0 -0.0567695498466492 4.0
0 0.0946479439735413 5.0
0 0.0838528871536255 6.0
0 0.0655413269996643 7.0
34.027431486 37.656925201416 0.0
0 0.260916203260422 1.0
0 -0.132371187210083 2.0
0 0.259234815835953 3.0
0 0.200420022010803 4.0
0 0.219733417034149 5.0
0 0.052480936050415 6.0
0 0.0206238031387329 7.0
41.154171391 37.8482360839844 0.0
0 0.381735742092133 1.0
0 -0.16360080242157 2.0
0 -0.00947505235671997 3.0
0 0.185781061649323 4.0
0 -0.0197915434837341 5.0
0 0.0939986705780029 6.0
0 0.0684055685997009 7.0
39.930826408 37.822151184082 0.0
0 0.226688772439957 1.0
0 -0.240205943584442 2.0
0 0.168616354465485 3.0
0 0.106719255447388 4.0
0 0.124590158462524 5.0
0 0.0297362804412842 6.0
0 0.0450549721717834 7.0
45.969703631 37.881534576416 0.0
0 0.0395640134811401 1.0
0 -0.628243267536163 2.0
0 0.178465634584427 3.0
0 0.24739083647728 4.0
0 0.243075609207153 5.0
0 0.0519842505455017 6.0
0 0.0110542774200439 7.0
38.398120221 38.5479698181152 0.0
0 0.0900894105434418 1.0
0 -0.315079033374786 2.0
0 0.00628340244293213 3.0
0 -0.0515677928924561 4.0
0 0.143363237380981 5.0
0 0.105749368667603 6.0
0 0.106077969074249 7.0
28.366017306 38.882984161377 0.0
0 0.0389969348907471 1.0
0 -0.300988256931305 2.0
0 0.501989185810089 3.0
0 0.046191930770874 4.0
0 0.216089099645615 5.0
0 0.102448523044586 6.0
0 0.156714498996735 7.0
39.079818979 38.2352752685547 0.0
0 0.175901800394058 1.0
0 -0.234288930892944 2.0
0 0.393186807632446 3.0
0 0.243352055549622 4.0
0 0.175849616527557 5.0
0 0.0904042422771454 6.0
0 0.046137809753418 7.0
40.466329383 38.0233383178711 0.0
0 0.0561177134513855 1.0
0 -0.0483414530754089 2.0
0 0.161368101835251 3.0
0 0.0623120069503784 4.0
0 -0.077049732208252 5.0
0 0.0734905004501343 6.0
0 0.0597316026687622 7.0
38.293116853 40.0412826538086 0.0
0 0.198104530572891 1.0
0 -0.189802825450897 2.0
0 0.0963339805603027 3.0
0 -0.0326938629150391 4.0
0 0.0450788736343384 5.0
0 0.0501691699028015 6.0
0 0.0522642731666565 7.0
43.346089497 39.5947456359863 0.0
0 0.231369465589523 1.0
0 -0.144290685653687 2.0
0 0.480639934539795 3.0
0 0.247809708118439 4.0
0 -0.0523123145103455 5.0
0 0.0596859455108643 6.0
0 0.0730703473091125 7.0
34.547871154 38.6069984436035 0.0
0 0.114660948514938 1.0
0 -0.244212508201599 2.0
0 0.198428213596344 3.0
0 0.310096561908722 4.0
0 0.175558269023895 5.0
0 0.0772882103919983 6.0
0 -0.0116270184516907 7.0
41.927050143 39.2602195739746 0.0
0 0.0929006338119507 1.0
0 -0.223864853382111 2.0
0 0.139428496360779 3.0
0 0.0303018093109131 4.0
0 0.0757721662521362 5.0
0 0.0932897925376892 6.0
0 0.107891708612442 7.0
44.548236377 38.5465888977051 0.0
0 0.185065597295761 1.0
0 -0.258295178413391 2.0
0 0.356893479824066 3.0
0 0.236538767814636 4.0
0 0.0567924380302429 5.0
0 0.0512214303016663 6.0
0 0.0579875707626343 7.0
34.958415487 38.976734161377 0.0
0 0.0131129622459412 1.0
0 -0.164768636226654 2.0
0 0.0848877429962158 3.0
0 -0.0706037878990173 4.0
0 0.174191027879715 5.0
0 0.0391992330551147 6.0
0 0.0970508456230164 7.0
41.619773298 38.2611312866211 0.0
0 0.345977485179901 1.0
0 -0.433084428310394 2.0
0 0.229236751794815 3.0
0 -0.142131268978119 4.0
0 0.242881387472153 5.0
0 0.00612825155258179 6.0
0 0.0499321818351746 7.0
35.768623912 38.2111434936523 0.0
0 0.0014234185218811 1.0
0 -0.248645663261414 2.0
0 0.130640864372253 3.0
0 0.0764660239219666 4.0
0 0.0553318858146667 5.0
0 0.0777707695960999 6.0
0 0.0900896787643433 7.0
36.03587308 38.918155670166 0.0
0 0.443748593330383 1.0
0 -0.406743288040161 2.0
0 0.0178172588348389 3.0
0 0.066580057144165 4.0
0 -0.0496203303337097 5.0
0 0.0691972374916077 6.0
0 0.0957795977592468 7.0
42.671159584 37.4081802368164 0.0
0 0.234568268060684 1.0
0 -0.451861679553986 2.0
0 0.0119660496711731 3.0
0 0.138284742832184 4.0
0 0.0188685059547424 5.0
0 0.0868125259876251 6.0
0 0.0505709648132324 7.0
32.270518391 38.3097343444824 0.0
0 0.0166030526161194 1.0
0 -0.386179685592651 2.0
0 0.0618256330490112 3.0
0 0.115799576044083 4.0
0 0.122927010059357 5.0
0 0.079294890165329 6.0
0 0.0889498293399811 7.0
36.239834941 40.0504760742188 0.0
0 0.511308908462524 1.0
0 -0.249809265136719 2.0
0 0.362092286348343 3.0
0 0.076030433177948 4.0
0 0.157297521829605 5.0
0 0.0790351033210754 6.0
0 0.0549123287200928 7.0
38.292005181 38.3340721130371 0.0
0 0.111111164093018 1.0
0 -0.0784143209457397 2.0
0 0.147263556718826 3.0
0 0.174648344516754 4.0
0 0.0189358592033386 5.0
0 0.0715079307556152 6.0
0 0.0930928885936737 7.0
41.142171126 38.810489654541 0.0
0 0.10318124294281 1.0
0 -0.192792892456055 2.0
0 0.11028578877449 3.0
0 0.0799543261528015 4.0
0 0.15393278002739 5.0
0 0.081648051738739 6.0
0 0.095683366060257 7.0
30.660234751 38.8001556396484 0.0
0 0.05644291639328 1.0
0 -0.282437026500702 2.0
0 0.469192683696747 3.0
0 0.239657163619995 4.0
0 0.0508624315261841 5.0
0 0.065062940120697 6.0
0 0.0170292258262634 7.0
42.776716986 38.8715553283691 0.0
0 0.400348156690598 1.0
0 -0.36289519071579 2.0
0 0.106494218111038 3.0
0 -0.0354577898979187 4.0
0 -0.101331651210785 5.0
0 0.0263186097145081 6.0
0 0.0486116409301758 7.0
39.136657955 38.8041725158691 0.0
0 0.467760384082794 1.0
0 -0.293076694011688 2.0
0 0.286742657423019 3.0
0 0.0434855818748474 4.0
0 0.027393639087677 5.0
0 0.0405918955802917 6.0
0 0.0852938294410706 7.0
40.593075664 37.3990249633789 0.0
0 0.284495413303375 1.0
0 -0.432918965816498 2.0
0 0.451309144496918 3.0
0 0.0785879492759705 4.0
0 0.261460989713669 5.0
0 0.0685808062553406 6.0
0 0.113257855176926 7.0
38.455960057 37.5076522827148 0.0
0 0.220829725265503 1.0
0 -0.323342740535736 2.0
0 0.156269550323486 3.0
0 0.253829926252365 4.0
0 0.142690777778625 5.0
0 0.0595978498458862 6.0
0 0.0662866234779358 7.0
40.029822307 38.4744873046875 0.0
0 0.234574437141418 1.0
0 -0.594472229480743 2.0
0 3.41831469535828 3.0
0 0.154576897621155 4.0
0 0.0196420550346375 5.0
0 0.0753073692321777 6.0
0 0.0826301574707031 7.0
39.721089806 39.7133293151855 0.0
0 0.0652058720588684 1.0
0 -0.123133301734924 2.0
0 0.260522991418839 3.0
0 0.254095792770386 4.0
0 -0.0960947275161743 5.0
0 0.0911957621574402 6.0
0 0.0460609793663025 7.0
46.012802781 38.6390800476074 0.0
0 0.0827489197254181 1.0
0 -0.257173120975494 2.0
0 -0.12131267786026 3.0
0 -0.048042356967926 4.0
0 0.0769906044006348 5.0
0 0.084507167339325 6.0
0 0.0704671144485474 7.0
43.791041158 38.4805641174316 0.0
0 0.135168790817261 1.0
0 0.561154723167419 2.0
0 0.328770756721497 3.0
0 -0.00984138250350952 4.0
0 0.199253708124161 5.0
0 0.0628642439842224 6.0
0 0.0405715703964233 7.0
31.257332424 39.5592460632324 0.0
0 0.163766115903854 1.0
0 -0.21339076757431 2.0
0 0.1086286008358 3.0
0 -0.00769048929214478 4.0
0 0.197961270809174 5.0
0 0.0384603142738342 6.0
0 0.0262401103973389 7.0
38.98847291 39.3179779052734 0.0
0 0.271989107131958 1.0
0 -0.146739840507507 2.0
0 0.488705396652222 3.0
0 0.284256279468536 4.0
0 0.14201682806015 5.0
0 0.0879067182540894 6.0
0 0.0394768714904785 7.0
38.691218499 37.7004280090332 0.0
0 0.184840142726898 1.0
0 -0.451636970043182 2.0
0 0.349539667367935 3.0
0 -0.129854321479797 4.0
0 0.833345293998718 5.0
0 0.0932630598545074 6.0
0 0.0782561898231506 7.0
39.033211971 37.531867980957 0.0
0 0.264740645885468 1.0
0 -0.354988753795624 2.0
0 0.383143961429596 3.0
0 0.276489436626434 4.0
0 0.00326299667358398 5.0
0 0.148661077022552 6.0
0 0.100697547197342 7.0
37.697547813 37.6144638061523 0.0
0 0.12298709154129 1.0
0 -0.349331557750702 2.0
0 0.39911013841629 3.0
0 0.118018358945847 4.0
0 0.263962239027023 5.0
0 0.0196360945701599 6.0
0 0.159549623727798 7.0
35.277541339 38.0765800476074 0.0
0 0.102826505899429 1.0
0 -0.205773174762726 2.0
0 0.305562555789948 3.0
0 0.100860267877579 4.0
0 0.119604468345642 5.0
0 0.0315089821815491 6.0
0 0.0359089374542236 7.0
36.119763966 38.041576385498 0.0
0 0.130302727222443 1.0
0 -0.29209691286087 2.0
0 0.0934644043445587 3.0
0 0.107922911643982 4.0
0 0.0621289610862732 5.0
0 0.0657391548156738 6.0
0 0.0372042655944824 7.0
33.14490305 39.2572975158691 0.0
0 0.342274844646454 1.0
0 -0.136183798313141 2.0
0 0.206323862075806 3.0
0 -0.0316069722175598 4.0
0 0.189368724822998 5.0
0 0.0738797187805176 6.0
0 0.0883620381355286 7.0
34.486800814 37.606990814209 0.0
0 0.291770040988922 1.0
0 -0.438364803791046 2.0
0 0.199434638023376 3.0
0 0.0902834236621857 4.0
0 0.0472216010093689 5.0
0 0.0208556056022644 6.0
0 0.0676533579826355 7.0
40.933468207 38.6918144226074 0.0
0 0.0648812651634216 1.0
0 -0.111331462860107 2.0
0 0.095224916934967 3.0
0 0.278545022010803 4.0
0 0.283851563930511 5.0
0 0.00394684076309204 6.0
0 0.103064954280853 7.0
35.993417928 39.6131629943848 0.0
0 0.140833497047424 1.0
0 -0.136345982551575 2.0
0 0.393254965543747 3.0
0 -0.0189797878265381 4.0
0 0.260991215705872 5.0
0 0.032534658908844 6.0
0 0.0863047242164612 7.0
39.331666923 38.6834602355957 0.0
0 0.259969025850296 1.0
0 -0.226220369338989 2.0
0 0.0804595649242401 3.0
0 0.128154963254929 4.0
0 0.160684764385223 5.0
0 0.101870059967041 6.0
0 0.0442638993263245 7.0
37.559548496 39.5542335510254 0.0
0 0.055342435836792 1.0
0 -0.108538746833801 2.0
0 0.350729763507843 3.0
0 0.0187118649482727 4.0
0 0.236038595438004 5.0
0 0.101716756820679 6.0
0 0.0553998351097107 7.0
41.796902482 39.0321311950684 0.0
0 0.104136824607849 1.0
0 -0.143547177314758 2.0
0 0.0542685985565186 3.0
0 0.190382450819016 4.0
0 0.231724619865417 5.0
0 0.103609472513199 6.0
0 0.0832533836364746 7.0
35.679590823 38.7148818969727 0.0
0 0.144182085990906 1.0
0 -0.123823225498199 2.0
0 0.161057472229004 3.0
0 0.138453483581543 4.0
0 0.241247266530991 5.0
0 0.0960349440574646 6.0
0 0.0879081189632416 7.0
33.227547292 38.1083450317383 0.0
0 0.0787912011146545 1.0
0 -0.28658139705658 2.0
0 0.263817548751831 3.0
0 0.0491819977760315 4.0
0 0.0704226493835449 5.0
0 0.0352469086647034 6.0
0 0.0221933722496033 7.0
28.008071739 38.378345489502 0.0
0 0.00951153039932251 1.0
0 -0.319582641124725 2.0
0 0.207101821899414 3.0
0 0.0427778959274292 4.0
0 0.133977472782135 5.0
0 0.0658359527587891 6.0
0 0.111936330795288 7.0
33.478498841 39.4603538513184 0.0
0 0.145803987979889 1.0
0 -0.253070831298828 2.0
0 0.0400491952896118 3.0
0 0.0971691906452179 4.0
0 0.213451623916626 5.0
0 0.0828360319137573 6.0
0 0.0726999044418335 7.0
33.12294682 38.6981620788574 0.0
0 0.209075301885605 1.0
0 -0.304560005664825 2.0
0 -0.244968414306641 3.0
0 -0.180980205535889 4.0
0 -0.0546098351478577 5.0
0 0.0613269209861755 6.0
0 0.104271531105042 7.0
34.970228384 39.9566345214844 0.0
0 0.0502583980560303 1.0
0 -0.0424923300743103 2.0
0 0.236930876970291 3.0
0 0.228279680013657 4.0
0 -0.0683795809745789 5.0
0 0.0802847146987915 6.0
0 0.0486304759979248 7.0
36.637966041 38.4300079345703 0.0
0 0.139472037553787 1.0
0 0.0890973806381226 2.0
0 -0.168334364891052 3.0
0 -0.0794855356216431 4.0
0 -0.132347822189331 5.0
0 0.0676541328430176 6.0
0 0.10505011677742 7.0
38.712447295 38.8837051391602 0.0
0 0.114199101924896 1.0
0 0.0497153401374817 2.0
0 0.479654371738434 3.0
0 0.209380626678467 4.0
0 0.15967521071434 5.0
0 0.0906969606876373 6.0
0 0.08391273021698 7.0
29.08402242 39.5174140930176 0.0
0 0.0459280014038086 1.0
0 -0.143795251846313 2.0
0 0.123158574104309 3.0
0 -0.0201910734176636 4.0
0 0.0425354838371277 5.0
0 0.0921835899353027 6.0
0 0.0799164474010468 7.0
40.741903011 39.3620262145996 0.0
0 0.260945558547974 1.0
0 -0.115863800048828 2.0
0 0.418747514486313 3.0
0 -0.020502507686615 4.0
0 0.267893433570862 5.0
0 0.0369779467582703 6.0
0 0.163621187210083 7.0
44.423927978 39.0480270385742 0.0
0 0.119853258132935 1.0
0 -0.18800675868988 2.0
0 0.153214424848557 3.0
0 0.298777341842651 4.0
0 -0.059295117855072 5.0
0 0.0725876688957214 6.0
0 0.0972602963447571 7.0
37.279929011 39.2097320556641 0.0
0 0.387672364711761 1.0
0 -0.325284719467163 2.0
0 0.243467479944229 3.0
0 -0.0215492248535156 4.0
0 0.0471197366714478 5.0
0 0.00916284322738647 6.0
0 0.0689834952354431 7.0
39.065196566 38.1021308898926 0.0
0 0.259630143642426 1.0
0 -0.257915437221527 2.0
0 0.255325436592102 3.0
0 0.103664457798004 4.0
0 0.0229551196098328 5.0
0 0.0505685806274414 6.0
0 0.0578458905220032 7.0
34.346712911 39.2959861755371 0.0
0 0.269457697868347 1.0
0 -0.160470545291901 2.0
0 0.457777678966522 3.0
0 0.0921288430690765 4.0
0 0.101918429136276 5.0
0 0.0499719381332397 6.0
0 0.106923997402191 7.0
44.832836202 39.6014823913574 0.0
0 0.225524067878723 1.0
0 -0.0310853719711304 2.0
0 0.12017360329628 3.0
0 0.0572493076324463 4.0
0 -0.0381684303283691 5.0
0 0.0334181785583496 6.0
0 0.0678684115409851 7.0
37.693005916 38.5363998413086 0.0
0 0.153947561979294 1.0
0 -0.184747397899628 2.0
0 0.0382539033889771 3.0
0 0.102549940347672 4.0
0 0.114327490329742 5.0
0 0.072510302066803 6.0
0 0.0636175274848938 7.0
35.516030206 38.9413642883301 0.0
0 0.0901889204978943 1.0
0 -0.122384309768677 2.0
0 0.184167444705963 3.0
0 0.10418689250946 4.0
0 0.181312203407288 5.0
0 0.100724220275879 6.0
0 0.106427043676376 7.0
45.449957523 38.6004409790039 0.0
0 0.0326815843582153 1.0
0 -0.0303927659988403 2.0
0 0.072080135345459 3.0
0 0.318062126636505 4.0
0 0.0701928734779358 5.0
0 0.0569431185722351 6.0
0 0.0614830851554871 7.0
36.52210397 38.6775436401367 0.0
0 0.0620099306106567 1.0
0 0.00961011648178101 2.0
0 -0.0309826135635376 3.0
0 -0.0550422072410583 4.0
0 -0.00966393947601318 5.0
0 0.101723164319992 6.0
0 0.0517098307609558 7.0
40.809333833 38.6806907653809 0.0
0 0.189730793237686 1.0
0 -0.320043921470642 2.0
0 0.291000187397003 3.0
0 0.0729588866233826 4.0
0 -0.0825626850128174 5.0
0 0.0361968278884888 6.0
0 0.0243459343910217 7.0
43.708815025 39.534236907959 0.0
0 0.270071387290955 1.0
0 -0.157052516937256 2.0
0 0.321382164955139 3.0
0 0.0129179358482361 4.0
0 0.181141406297684 5.0
0 0.0259965658187866 6.0
0 0.129889696836472 7.0
35.746307575 34.0634078979492 0.0
0 0.237255692481995 1.0
0 -0.386012673377991 2.0
0 -0.136266887187958 3.0
0 -0.0747338533401489 4.0
0 -0.0341176390647888 5.0
0 0.0437367558479309 6.0
0 0.0937121212482452 7.0
32.565260407 38.6303405761719 0.0
0 0.134661167860031 1.0
0 -0.32357531785965 2.0
0 0.0996683835983276 3.0
0 0.195170730352402 4.0
0 0.0982731878757477 5.0
0 0.0687288641929626 6.0
0 0.0491070747375488 7.0
37.459437569 39.1743431091309 0.0
0 0.255553901195526 1.0
0 -0.437515676021576 2.0
0 0.272285938262939 3.0
0 -0.0469465255737305 4.0
0 0.0109336376190186 5.0
0 0.0753978490829468 6.0
0 0.100500971078873 7.0
41.600868777 37.7066459655762 0.0
0 0.216877102851868 1.0
0 -0.205387771129608 2.0
0 -0.00417381525039673 3.0
0 0.018146276473999 4.0
0 0.066343367099762 5.0
0 0.0346511602401733 6.0
0 0.0194706320762634 7.0
43.088648289 41.5447845458984 0.0
0 0.660479187965393 1.0
0 -0.117311775684357 2.0
0 0.420657217502594 3.0
0 0.0726568102836609 4.0
0 -0.0337262749671936 5.0
0 0.0654146671295166 6.0
0 0.0572766661643982 7.0
30.268152677 38.5946731567383 0.0
0 0.0424607396125793 1.0
0 -0.138857781887054 2.0
0 0.306084424257278 3.0
0 0.163399815559387 4.0
0 0.137813031673431 5.0
0 0.105857789516449 6.0
0 0.0412822961807251 7.0
38.454045888 38.4499549865723 0.0
0 0.0951032638549805 1.0
0 -0.0465535521507263 2.0
0 0.348876118659973 3.0
0 0.221875458955765 4.0
0 0.0342252850532532 5.0
0 0.0550657510757446 6.0
0 0.00720679759979248 7.0
36.654056864 38.8600578308105 0.0
0 0.0424174666404724 1.0
0 -0.397671163082123 2.0
0 0.338672876358032 3.0
0 0.0919799506664276 4.0
0 0.0717777013778687 5.0
0 0.0383375883102417 6.0
0 0.0733047127723694 7.0
36.990278128 37.524341583252 0.0
0 0.221753299236298 1.0
0 -0.579366981983185 2.0
0 0.270206868648529 3.0
0 -0.0616559982299805 4.0
0 -0.024580180644989 5.0
0 0.0797540247440338 6.0
0 0.0489518642425537 7.0
39.45539179 38.4299812316895 0.0
0 0.0831407308578491 1.0
0 -0.268890500068665 2.0
0 0.424567431211472 3.0
0 -0.385638654232025 4.0
0 -0.04506915807724 5.0
0 0.0903962254524231 6.0
0 0.0487996339797974 7.0
41.49009831 39.3639488220215 0.0
0 0.0762739181518555 1.0
0 -0.130674064159393 2.0
0 0.166725367307663 3.0
0 0.0512491464614868 4.0
0 0.0928978025913239 5.0
0 0.0995952486991882 6.0
0 0.080894261598587 7.0
42.195848764 39.7568054199219 0.0
0 0.0616613030433655 1.0
0 -0.134405374526978 2.0
0 0.00353896617889404 3.0
0 0.264388680458069 4.0
0 0.141353458166122 5.0
0 0.0439846515655518 6.0
0 -0.00565218925476074 7.0
36.580423956 38.7282943725586 0.0
0 0.183682382106781 1.0
0 -0.141757667064667 2.0
0 0.492784112691879 3.0
0 0.22057968378067 4.0
0 0.0484667420387268 5.0
0 0.0267382264137268 6.0
0 0.0157141089439392 7.0
41.112612846 37.6162796020508 0.0
0 0.203520119190216 1.0
0 -0.355780184268951 2.0
0 -0.0584052801132202 3.0
0 0.00520306825637817 4.0
0 0.11335888504982 5.0
0 0.0668054819107056 6.0
0 0.0728349685668945 7.0
41.165032305 38.1451721191406 0.0
0 0.0847059488296509 1.0
0 -0.499329388141632 2.0
0 0.461333751678467 3.0
0 0.166234195232391 4.0
0 0.141565650701523 5.0
0 0.0533431172370911 6.0
0 0.0953706204891205 7.0
28.236329769 38.5068092346191 0.0
0 -0.00911104679107666 1.0
0 -0.162937223911285 2.0
0 0.385250151157379 3.0
0 0.0933529734611511 4.0
0 0.140729486942291 5.0
0 0.044554591178894 6.0
0 0.0878106951713562 7.0
46.563722235 39.5763740539551 0.0
0 0.25886070728302 1.0
0 -0.0917804837226868 2.0
0 -0.12675267457962 3.0
0 0.102075487375259 4.0
0 0.0955046713352203 5.0
0 0.0908852219581604 6.0
0 0.0902951955795288 7.0
38.976039446 35.771369934082 0.0
0 0.328802049160004 1.0
0 -0.500352203845978 2.0
0 0.134711802005768 3.0
0 -0.0495448708534241 4.0
0 -0.020671010017395 5.0
0 0.0586175918579102 6.0
0 0.0768932700157166 7.0
43.902220165 38.1391830444336 0.0
0 0.0146318078041077 1.0
0 -0.0078626275062561 2.0
0 0.00808727741241455 3.0
0 0.158392369747162 4.0
0 0.104336977005005 5.0
0 0.0340125560760498 6.0
0 0.0707731246948242 7.0
37.104870385 38.696475982666 0.0
0 0.210575222969055 1.0
0 -0.178829908370972 2.0
0 -0.0231606364250183 3.0
0 -0.050085723400116 4.0
0 0.146157115697861 5.0
0 0.0806512832641602 6.0
0 0.1118323802948 7.0
48.094555135 38.0212478637695 0.0
0 0.138752818107605 1.0
0 0.627447605133057 2.0
0 0.0248640775680542 3.0
0 -0.12725955247879 4.0
0 0.0249100923538208 5.0
0 0.0378366708755493 6.0
0 0.0956567227840424 7.0
39.863550951 39.0677032470703 0.0
0 0.0834846794605255 1.0
0 -0.342833518981934 2.0
0 -0.0471689105033875 3.0
0 0.0469996333122253 4.0
0 0.163162142038345 5.0
0 0.0675938129425049 6.0
0 0.0520592927932739 7.0
38.158424706 38.4466896057129 0.0
0 0.110916048288345 1.0
0 -0.185331523418427 2.0
0 0.128324925899506 3.0
0 -0.0528095960617065 4.0
0 0.0530819892883301 5.0
0 0.0607088208198547 6.0
0 0.0851479470729828 7.0
37.671552644 39.102352142334 0.0
0 0.0811720490455627 1.0
0 -0.186448633670807 2.0
0 0.176367282867432 3.0
0 0.0559749007225037 4.0
0 0.199252247810364 5.0
0 0.100874602794647 6.0
0 0.0699121356010437 7.0
38.99000687 38.8452186584473 0.0
0 0.239623308181763 1.0
0 -0.205465853214264 2.0
0 0.0615308880805969 3.0
0 -0.0110807418823242 4.0
0 0.224449038505554 5.0
0 0.0642088055610657 6.0
0 0.106650829315186 7.0
32.824279197 38.9305763244629 0.0
0 0.110196232795715 1.0
0 0.0513097643852234 2.0
0 0.116867065429688 3.0
0 0.0836545825004578 4.0
0 0.145430028438568 5.0
0 0.0412125587463379 6.0
0 0.107369273900986 7.0
39.752392398 39.1172828674316 0.0
0 0.0891351103782654 1.0
0 -0.128117263317108 2.0
0 0.159998834133148 3.0
0 0.0194727778434753 4.0
0 0.2054343521595 5.0
0 0.0292778611183167 6.0
0 0.0494336485862732 7.0
36.014997988 39.0655288696289 0.0
0 0.131849199533463 1.0
0 -0.0559805631637573 2.0
0 -0.189809739589691 3.0
0 -0.0796999931335449 4.0
0 -0.0306850671768188 5.0
0 0.100119560956955 6.0
0 0.0829702615737915 7.0
42.307205606 39.3282928466797 0.0
0 0.166833907365799 1.0
0 -0.0959773063659668 2.0
0 0.08231121301651 3.0
0 -0.0119389891624451 4.0
0 0.068534791469574 5.0
0 0.102034687995911 6.0
0 0.106356024742126 7.0
41.624369934 37.9397468566895 0.0
0 0.408578038215637 1.0
0 -0.316567897796631 2.0
0 0.431956708431244 3.0
0 0.19380384683609 4.0
0 0.0192130208015442 5.0
0 0.0811033844947815 6.0
0 0.0549172759056091 7.0
40.927860926 37.1410636901855 0.0
0 0.266502112150192 1.0
0 -0.631340801715851 2.0
0 -0.0431159734725952 3.0
0 0.0782870054244995 4.0
0 0.168088108301163 5.0
0 0.0681278705596924 6.0
0 0.0327015519142151 7.0
47.441321812 39.4807968139648 0.0
0 0.154477477073669 1.0
0 -0.155643403530121 2.0
0 -0.0130937695503235 3.0
0 0.0309373140335083 4.0
0 -0.00672793388366699 5.0
0 0.0314739346504211 6.0
0 0.0964957773685455 7.0
39.379505619 39.3599395751953 0.0
0 0.180346965789795 1.0
0 -0.173007965087891 2.0
0 0.299663126468658 3.0
0 0.067904531955719 4.0
0 0.17448365688324 5.0
0 0.0862320959568024 6.0
0 0.085258424282074 7.0
42.095557299 39.9900054931641 0.0
0 0.246556907892227 1.0
0 -0.158835172653198 2.0
0 0.0141417384147644 3.0
0 0.197205543518066 4.0
0 0.0404314398765564 5.0
0 0.122111618518829 6.0
0 0.0551634430885315 7.0
35.151188225 39.9450454711914 0.0
0 0.252388060092926 1.0
0 -0.170067489147186 2.0
0 0.133020341396332 3.0
0 0.0566971898078918 4.0
0 0.193518400192261 5.0
0 0.050104558467865 6.0
0 0.0717108249664307 7.0
37.713935371 38.9968643188477 0.0
0 0.0916031897068024 1.0
0 -0.161221385002136 2.0
0 0.280525654554367 3.0
0 0.225647449493408 4.0
0 0.0597198605537415 5.0
0 0.0422990322113037 6.0
0 -0.0245756506919861 7.0
39.66774326 38.7548637390137 0.0
0 0.217219531536102 1.0
0 -0.154893457889557 2.0
0 0.306915760040283 3.0
0 0.222349941730499 4.0
0 -0.0475922226905823 5.0
0 0.0813465416431427 6.0
0 0.0447729825973511 7.0
41.240556194 38.2094650268555 0.0
0 0.178702712059021 1.0
0 -0.297106325626373 2.0
0 0.300747036933899 3.0
0 0.0997371971607208 4.0
0 0.132896691560745 5.0
0 0.072957456111908 6.0
0 0.107118368148804 7.0
40.3201946 37.5475425720215 0.0
0 0.0753607749938965 1.0
0 0.763134062290192 2.0
0 0.201288878917694 3.0
0 -0.0294159054756165 4.0
0 0.274525195360184 5.0
0 0.012212872505188 6.0
0 0.0786429941654205 7.0
39.561151059 38.5857200622559 0.0
0 0.0381711721420288 1.0
0 -0.199719727039337 2.0
0 -0.0782226324081421 3.0
0 -0.103427052497864 4.0
0 0.15972238779068 5.0
0 0.0916295349597931 6.0
0 0.0675490498542786 7.0
43.566431947 38.7239227294922 0.0
0 0.26418748497963 1.0
0 -0.62556928396225 2.0
0 0.252990216016769 3.0
0 -0.0120181441307068 4.0
0 -0.0235587954521179 5.0
0 0.103479832410812 6.0
0 0.0314005017280579 7.0
37.929366562 38.4366836547852 0.0
0 0.104140311479568 1.0
0 -0.25110250711441 2.0
0 0.0758557319641113 3.0
0 0.0220474600791931 4.0
0 0.185135126113892 5.0
0 0.0934053659439087 6.0
0 0.0658692717552185 7.0
39.745668641 38.906665802002 0.0
0 0.00956261157989502 1.0
0 -0.413537859916687 2.0
0 0.0785165131092072 3.0
0 -0.0279110074043274 4.0
0 0.0550140738487244 5.0
0 0.0437244176864624 6.0
0 0.0325308442115784 7.0
37.173794625 38.2097358703613 0.0
0 0.216732442378998 1.0
0 -0.392341911792755 2.0
0 0.0515981316566467 3.0
0 0.114793986082077 4.0
0 0.133782774209976 5.0
0 0.0463108420372009 6.0
0 0.0887708365917206 7.0
27.124866015 39.8463973999023 0.0
0 0.161586910486221 1.0
0 -0.28914213180542 2.0
0 0.363075792789459 3.0
0 0.212961465120316 4.0
0 0.103834986686707 5.0
0 0.0697064995765686 6.0
0 0.114366918802261 7.0
27.411746904 37.2249488830566 0.0
0 0.250052213668823 1.0
0 1.25922107696533 2.0
0 0.266794085502625 3.0
0 0.247318059206009 4.0
0 -0.0145117044448853 5.0
0 0.14802560210228 6.0
0 0.0489640831947327 7.0
38.866617317 39.6226921081543 0.0
0 0.148941665887833 1.0
0 -0.145286738872528 2.0
0 0.0668238401412964 3.0
0 0.203303009271622 4.0
0 0.0518251657485962 5.0
0 0.0487178564071655 6.0
0 0.0432092547416687 7.0
42.383057354 39.6079330444336 0.0
0 0.197308748960495 1.0
0 -0.0584260821342468 2.0
0 0.140200614929199 3.0
0 0.150559484958649 4.0
0 0.132845103740692 5.0
0 0.0719248056411743 6.0
0 0.0750908255577087 7.0
47.643751036 39.5297927856445 0.0
0 0.156749576330185 1.0
0 -0.226433455944061 2.0
0 0.0213310122489929 3.0
0 0.0912418961524963 4.0
0 0.0278571248054504 5.0
0 0.0801723301410675 6.0
0 0.0678796172142029 7.0
38.439399957 37.0366592407227 0.0
0 0.1898173391819 1.0
0 -0.40154641866684 2.0
0 0.116838783025742 3.0
0 0.0149243474006653 4.0
0 0.266278386116028 5.0
0 0.0305231809616089 6.0
0 0.0633763670921326 7.0
40.263562371 37.0721817016602 0.0
0 0.120071291923523 1.0
0 -0.374010980129242 2.0
0 0.364515244960785 3.0
0 0.215641111135483 4.0
0 0.205105006694794 5.0
0 0.0901525020599365 6.0
0 0.0414574146270752 7.0
41.519528397 39.1528663635254 0.0
0 0.0889121890068054 1.0
0 -0.253257095813751 2.0
0 -0.0942986011505127 3.0
0 -0.0443814396858215 4.0
0 0.0762296319007874 5.0
0 0.0949130952358246 6.0
0 0.0986947417259216 7.0
40.414570474 39.1307067871094 0.0
0 0.126531958580017 1.0
0 -0.261791408061981 2.0
0 -0.241984128952026 3.0
0 -0.0954576730728149 4.0
0 -0.110146999359131 5.0
0 0.149600654840469 6.0
0 0.105500608682632 7.0
37.834181221 39.4292259216309 0.0
0 0.11461815237999 1.0
0 -0.123417317867279 2.0
0 0.443078249692917 3.0
0 0.113260567188263 4.0
0 0.147823631763458 5.0
0 0.0910327136516571 6.0
0 0.101801633834839 7.0
37.201121556 38.2414054870605 0.0
0 0.077447772026062 1.0
0 -0.158632338047028 2.0
0 -0.0198302865028381 3.0
0 -0.0643317103385925 4.0
0 0.0270264744758606 5.0
0 0.0737972855567932 6.0
0 0.0243866443634033 7.0
42.94444514 39.1467361450195 0.0
0 0.079239159822464 1.0
0 -0.133798956871033 2.0
0 0.329852938652039 3.0
0 0.101798593997955 4.0
0 0.161622673273087 5.0
0 0.0276180505752563 6.0
0 0.0314264297485352 7.0
47.309600034 38.818416595459 0.0
0 0.0903041958808899 1.0
0 -0.230358839035034 2.0
0 0.0878922343254089 3.0
0 0.146759867668152 4.0
0 0.0369924902915955 5.0
0 0.0462936162948608 6.0
0 0.0972353219985962 7.0
34.703660747 39.6769714355469 0.0
0 0.177015542984009 1.0
0 0.012174665927887 2.0
0 -0.0406034588813782 3.0
0 -0.0162671208381653 4.0
0 0.137096464633942 5.0
0 0.0248321890830994 6.0
0 0.0373362898826599 7.0
39.355679833 39.0516014099121 0.0
0 0.176391303539276 1.0
0 -0.419327855110168 2.0
0 0.484982073307037 3.0
0 0.326961874961853 4.0
0 0.220561951398849 5.0
0 0.0693492889404297 6.0
0 0.0123975872993469 7.0
36.760838135 38.8280754089355 0.0
0 0.03059321641922 1.0
0 -0.343878567218781 2.0
0 0.123982071876526 3.0
0 0.286907821893692 4.0
0 0.111400038003922 5.0
0 0.036656379699707 6.0
0 0.0928685665130615 7.0
45.817767369 37.9652404785156 0.0
0 0.036074161529541 1.0
0 1.24860095977783 2.0
0 -0.0908766388893127 3.0
0 0.0278725624084473 4.0
0 -0.0572217106819153 5.0
0 0.102998793125153 6.0
0 0.0729513764381409 7.0
37.456387944 39.1242332458496 0.0
0 0.142101019620895 1.0
0 -0.276444256305695 2.0
0 0.228932797908783 3.0
0 0.241432428359985 4.0
0 -0.107683479785919 5.0
0 0.062935471534729 6.0
0 0.0853358805179596 7.0
36.979299113 38.5721778869629 0.0
0 0.0781960487365723 1.0
0 -0.231070160865784 2.0
0 0.344177663326263 3.0
0 0.149681150913239 4.0
0 0.0559912323951721 5.0
0 0.0283704400062561 6.0
0 0.0685852766036987 7.0
44.357392157 38.9749145507812 0.0
0 0.262557357549667 1.0
0 -0.281647324562073 2.0
0 0.179886698722839 3.0
0 0.168524503707886 4.0
0 0.145610302686691 5.0
0 0.10225385427475 6.0
0 0.0369457602500916 7.0
36.812079299 39.3905410766602 0.0
0 0.0573919415473938 1.0
0 -0.237584590911865 2.0
0 0.169573932886124 3.0
0 0.0058397650718689 4.0
0 0.133465111255646 5.0
0 0.0748407244682312 6.0
0 0.0729802846908569 7.0
43.172254743 37.2553558349609 0.0
0 0.0118964910507202 1.0
0 -0.20399022102356 2.0
0 0.293452680110931 3.0
0 0.0879130065441132 4.0
0 0.0595491528511047 5.0
0 0.0887939035892487 6.0
0 0.0416828989982605 7.0
47.109370567 39.652229309082 0.0
0 0.0989768207073212 1.0
0 -0.210904538631439 2.0
0 0.296420484781265 3.0
0 0.0183889865875244 4.0
0 0.175289452075958 5.0
0 0.0604212284088135 6.0
0 0.0272224545478821 7.0
35.871355662 38.3965454101562 0.0
0 0.434384942054749 1.0
0 -0.257968187332153 2.0
0 0.352834939956665 3.0
0 0.0960039496421814 4.0
0 0.0116159915924072 5.0
0 0.0424315333366394 6.0
0 0.0673096776008606 7.0
42.169433911 38.5645904541016 0.0
0 0.0602736473083496 1.0
0 -0.240524470806122 2.0
0 0.27083632349968 3.0
0 0.104449808597565 4.0
0 0.0637235045433044 5.0
0 0.0376548767089844 6.0
0 0.038861095905304 7.0
38.001249401 38.6129455566406 0.0
0 0.161420524120331 1.0
0 -0.283291757106781 2.0
0 0.249055057764053 3.0
0 0.125750780105591 4.0
0 -0.0688720345497131 5.0
0 0.105373620986938 6.0
0 0.108416140079498 7.0
36.026296153 38.9570083618164 0.0
0 0.148249357938766 1.0
0 -0.324692904949188 2.0
0 -0.0101205706596375 3.0
0 0.114435583353043 4.0
0 0.116407752037048 5.0
0 0.0336210131645203 6.0
0 0.10979837179184 7.0
33.200213159 40.023120880127 0.0
0 0.0748366117477417 1.0
0 -0.206047058105469 2.0
0 0.017072319984436 3.0
0 0.0278149247169495 4.0
0 0.202792108058929 5.0
0 0.056230902671814 6.0
0 0.0774397850036621 7.0
36.881947952 38.589298248291 0.0
0 0.014783501625061 1.0
0 -0.225092470645905 2.0
0 0.0989953875541687 3.0
0 0.240717262029648 4.0
0 0.0358770489692688 5.0
0 0.0567390918731689 6.0
0 0.0921384692192078 7.0
28.363012012 38.8414840698242 0.0
0 0.0854464769363403 1.0
0 -0.159912168979645 2.0
0 0.18282824754715 3.0
0 -0.111528813838959 4.0
0 0.0916304886341095 5.0
0 0.0643031001091003 6.0
0 0.0747440457344055 7.0
39.574467426 38.4574279785156 0.0
0 0.121568530797958 1.0
0 -0.0369134545326233 2.0
0 0.135657519102097 3.0
0 0.0449432134628296 4.0
0 0.224758386611938 5.0
0 0.0636785626411438 6.0
0 0.0602860450744629 7.0
36.22040148 38.8429527282715 0.0
0 0.150512397289276 1.0
0 -0.223124325275421 2.0
0 0.0548608303070068 3.0
0 0.0391703248023987 4.0
0 0.127198755741119 5.0
0 0.088523656129837 6.0
0 0.108347594738007 7.0
31.086432141 38.9085922241211 0.0
0 0.0201451182365417 1.0
0 -0.183176577091217 2.0
0 0.305245012044907 3.0
0 0.0440794229507446 4.0
0 0.0200065970420837 5.0
0 0.00886213779449463 6.0
0 0.100484251976013 7.0
36.274353063 38.7223167419434 0.0
0 0.189898282289505 1.0
0 -0.289578139781952 2.0
0 0.0721847414970398 3.0
0 0.0278837084770203 4.0
0 0.0496424436569214 5.0
0 0.0390727519989014 6.0
0 0.0947475135326385 7.0
31.65795009 38.4950904846191 0.0
0 0.226990938186646 1.0
0 -0.328788697719574 2.0
0 0.236235380172729 3.0
0 0.121083557605743 4.0
0 -0.154299855232239 5.0
0 0.105549097061157 6.0
0 0.0949501395225525 7.0
37.059171479 38.0461044311523 0.0
0 0.033605694770813 1.0
0 1.8029568195343 2.0
0 0.527408123016357 3.0
0 0.275094419717789 4.0
0 -0.0947206616401672 5.0
0 0.0412501096725464 6.0
0 0.0598435401916504 7.0
45.308862357 38.858829498291 0.0
0 0.0303471684455872 1.0
0 -0.275702953338623 2.0
0 0.11937004327774 3.0
0 -0.0525137782096863 4.0
0 0.265968322753906 5.0
0 0.0800740122795105 6.0
0 0.0670572519302368 7.0
32.988279909 39.5008201599121 0.0
0 0.24109011888504 1.0
0 -0.116181790828705 2.0
0 -0.151131570339203 3.0
0 0.00935161113739014 4.0
0 -0.143144965171814 5.0
0 0.0608009099960327 6.0
0 0.0811505913734436 7.0
41.812200593 42.4900779724121 0.0
0 0.564354777336121 1.0
0 -0.0189224481582642 2.0
0 0.203664809465408 3.0
0 0.248653084039688 4.0
0 -0.0324451923370361 5.0
0 0.0745593309402466 6.0
0 0.0483933687210083 7.0
34.159578007 37.1605491638184 0.0
0 0.260037183761597 1.0
0 -0.476249992847443 2.0
0 -0.193120777606964 3.0
0 -0.0794104933738708 4.0
0 0.134109228849411 5.0
0 0.0446624755859375 6.0
0 0.0508489012718201 7.0
41.353058204 39.8873329162598 0.0
0 0.116395115852356 1.0
0 -0.128042459487915 2.0
0 0.112494289875031 3.0
0 0.199858814477921 4.0
0 0.0731068849563599 5.0
0 0.0341318249702454 6.0
0 0.0474046468734741 7.0
35.663236644 39.7619934082031 0.0
0 0.0824553966522217 1.0
0 -0.134775817394257 2.0
0 0.0937197506427765 3.0
0 0.0767030715942383 4.0
0 0.108672499656677 5.0
0 0.0404596328735352 6.0
0 0.056215226650238 7.0
40.444916123 38.567325592041 0.0
0 0.147698998451233 1.0
0 -0.279313385486603 2.0
0 -0.102059721946716 3.0
0 0.21970734000206 4.0
0 -0.050083339214325 5.0
0 0.0290641188621521 6.0
0 0.0952690243721008 7.0
39.900416914 37.1706581115723 0.0
0 0.145297020673752 1.0
0 -0.651456534862518 2.0
0 0.132721096277237 3.0
0 -0.133507311344147 4.0
0 0.0295422673225403 5.0
0 0.0285026431083679 6.0
0 0.107906103134155 7.0
34.405324849 37.6690902709961 0.0
0 0.0867083966732025 1.0
0 -0.0958937406539917 2.0
0 0.13715124130249 3.0
0 0.199980765581131 4.0
0 0.103766769170761 5.0
0 0.0975839495658875 6.0
0 0.0720154643058777 7.0
33.074256004 40.0314445495605 0.0
0 0.084937185049057 1.0
0 -0.129567444324493 2.0
0 0.0830073654651642 3.0
0 0.204702913761139 4.0
0 -0.0151617527008057 5.0
0 0.0279766917228699 6.0
0 0.0326182842254639 7.0
40.036170308 38.6185493469238 0.0
0 0.406276285648346 1.0
0 -0.339377164840698 2.0
0 0.173885643482208 3.0
0 0.215368807315826 4.0
0 -0.0667521953582764 5.0
0 0.103136330842972 6.0
0 0.0660747885704041 7.0
44.453206241 38.3802185058594 0.0
0 0.148606956005096 1.0
0 -0.153367936611176 2.0
0 0.0669742822647095 3.0
0 0.0722439885139465 4.0
0 0.201572835445404 5.0
0 0.070404052734375 6.0
0 0.0786953568458557 7.0
33.85004541 37.702564239502 0.0
0 0.37748521566391 1.0
0 -0.613310277462006 2.0
0 0.161987096071243 3.0
0 0.320268630981445 4.0
0 0.204339534044266 5.0
0 0.0760205388069153 6.0
0 0.0839710831642151 7.0
30.537877308 37.0621490478516 0.0
0 0.212128430604935 1.0
0 -0.426798403263092 2.0
0 0.287069141864777 3.0
0 0.161629766225815 4.0
0 0.0387887358665466 5.0
0 0.0946999490261078 6.0
0 0.0902321338653564 7.0
27.727440942 36.8661155700684 0.0
0 0.259209632873535 1.0
0 -0.586713969707489 2.0
0 0.304106146097183 3.0
0 0.0157517790794373 4.0
0 0.231318175792694 5.0
0 0.0328661203384399 6.0
0 0.0918056070804596 7.0
35.630182897 38.0057830810547 0.0
0 0.0749981999397278 1.0
0 -0.0611766576766968 2.0
0 0.457101345062256 3.0
0 -0.0736782550811768 4.0
0 0.248201042413712 5.0
0 0.0449309349060059 6.0
0 0.078393816947937 7.0
31.038251519 37.1734352111816 0.0
0 0.402861714363098 1.0
0 -0.199102520942688 2.0
0 -0.0862777829170227 3.0
0 -0.109495937824249 4.0
0 -9.27448272705078e-05 5.0
0 0.0769641995429993 6.0
0 0.0615073442459106 7.0
31.867011967 38.8055801391602 0.0
0 0.0131174921989441 1.0
0 -0.335846483707428 2.0
0 0.20711362361908 3.0
0 0.130158722400665 4.0
0 0.203754246234894 5.0
0 0.0449506044387817 6.0
0 0.0284186005592346 7.0
30.754831646 39.0529289245605 0.0
0 0.0271309614181519 1.0
0 -0.228190422058105 2.0
0 0.0312603116035461 3.0
0 0.110029399394989 4.0
0 0.118995249271393 5.0
0 0.0825456082820892 6.0
0 0.107476949691772 7.0
38.95156428 39.1949043273926 0.0
0 0.0667873024940491 1.0
0 -0.103623449802399 2.0
0 0.145884245634079 3.0
0 -0.0421499609947205 4.0
0 0.0453149676322937 5.0
0 0.0208034515380859 6.0
0 0.0598793625831604 7.0
33.384480936 38.3807678222656 0.0
0 0.414274245500565 1.0
0 -0.347193598747253 2.0
0 0.099658340215683 3.0
0 0.184207260608673 4.0
0 0.228469163179398 5.0
0 0.0619305372238159 6.0
0 0.104370176792145 7.0
37.291199282 39.0299606323242 0.0
0 0.0250274538993835 1.0
0 -0.335187494754791 2.0
0 0.185861855745316 3.0
0 -0.0476625561714172 4.0
0 0.270475834608078 5.0
0 0.0317953824996948 6.0
0 0.104500859975815 7.0
33.571811016 38.2288780212402 0.0
0 0.00698250532150269 1.0
0 -0.218649744987488 2.0
0 0.196238696575165 3.0
0 0.174741059541702 4.0
0 0.162351787090302 5.0
0 0.0714041590690613 6.0
0 0.10735285282135 7.0
44.048761746 37.3500823974609 0.0
0 0.135973334312439 1.0
0 -0.621034324169159 2.0
0 0.195103228092194 3.0
0 0.0544013977050781 4.0
0 0.19237431883812 5.0
0 0.0271247625350952 6.0
0 0.0903561115264893 7.0
38.09281546 39.4090423583984 0.0
0 0.138204544782639 1.0
0 -0.163897395133972 2.0
0 0.399324655532837 3.0
0 0.160978436470032 4.0
0 0.190159797668457 5.0
0 0.0922250747680664 6.0
0 0.0774325728416443 7.0
42.616606924 37.7141990661621 0.0
0 0.0782779157161713 1.0
0 -0.477040469646454 2.0
0 0.138512253761292 3.0
0 -0.0893435478210449 4.0
0 0.032568633556366 5.0
0 0.037885844707489 6.0
0 0.0569646954536438 7.0
41.080227074 39.4256401062012 0.0
0 0.054851233959198 1.0
0 -0.261566162109375 2.0
0 0.370655596256256 3.0
0 0.157337039709091 4.0
0 0.184523403644562 5.0
0 0.0843158364295959 6.0
0 0.0698036551475525 7.0
36.859087813 38.8955230712891 0.0
0 0.22929835319519 1.0
0 1.78196382522583 2.0
0 0.2152199447155 3.0
0 0.116740107536316 4.0
0 0.210969179868698 5.0
0 0.040607750415802 6.0
0 0.0581396222114563 7.0
39.316268876 38.714412689209 0.0
0 0.150036334991455 1.0
0 -0.133835196495056 2.0
0 -0.0224224328994751 3.0
0 0.11804386973381 4.0
0 0.00150376558303833 5.0
0 0.102023541927338 6.0
0 0.0294155478477478 7.0
40.023071967 38.5565948486328 0.0
0 0.137606143951416 1.0
0 -0.207938969135284 2.0
0 0.020828902721405 3.0
0 0.0781232118606567 4.0
0 0.197052329778671 5.0
0 0.0319345593452454 6.0
0 0.105673134326935 7.0
32.691039566 39.2269020080566 0.0
0 0.056771457195282 1.0
0 -0.153834104537964 2.0
0 0.237570583820343 3.0
0 0.0418320298194885 4.0
0 0.316655367612839 5.0
0 0.0189256072044373 6.0
0 0.0630873441696167 7.0
39.35017223 39.3302459716797 0.0
0 0.109177112579346 1.0
0 -0.074505627155304 2.0
0 0.228958606719971 3.0
0 0.0828819274902344 4.0
0 0.134199380874634 5.0
0 0.0979034304618835 6.0
0 0.110384374856949 7.0
38.62577922 40.3376846313477 0.0
0 0.107723623514175 1.0
0 -0.185437440872192 2.0
0 0.327141433954239 3.0
0 0.0203369855880737 4.0
0 -0.00463581085205078 5.0
0 0.0650143623352051 6.0
0 0.0184122920036316 7.0
43.36321018 39.6281471252441 0.0
0 0.149484187364578 1.0
0 -0.162706851959229 2.0
0 0.279647022485733 3.0
0 0.0218523740768433 4.0
0 -0.0101012587547302 5.0
0 0.0367602109909058 6.0
0 0.0405479669570923 7.0
41.711370519 38.6783790588379 0.0
0 0.0354200005531311 1.0
0 -0.144565880298615 2.0
0 -0.0485250949859619 3.0
0 -0.0100723505020142 4.0
0 0.100181221961975 5.0
0 0.0419473648071289 6.0
0 0.103758960962296 7.0
41.401748585 39.7063674926758 0.0
0 0.0376008152961731 1.0
0 -0.24824059009552 2.0
0 0.031088650226593 3.0
0 0.174166858196259 4.0
0 0.0339848399162292 5.0
0 0.0339033007621765 6.0
0 0.0839188694953918 7.0
45.411303661 37.8996963500977 0.0
0 0.151774108409882 1.0
0 -0.267806470394135 2.0
0 -0.0815933346748352 3.0
0 -0.166871547698975 4.0
0 0.0159577131271362 5.0
0 0.0681779980659485 6.0
0 0.121238768100739 7.0
46.173622738 39.5688667297363 0.0
0 0.169361144304276 1.0
0 -0.150969207286835 2.0
0 0.110686838626862 3.0
0 0.211682468652725 4.0
0 0.0644299387931824 5.0
0 0.124356240034103 6.0
0 0.0721235871315002 7.0
43.970500892 37.8264770507812 0.0
0 0.152733087539673 1.0
0 -0.268684566020966 2.0
0 0.268917858600616 3.0
0 0.0103269219398499 4.0
0 0.211093783378601 5.0
0 0.0943050384521484 6.0
0 0.0954740047454834 7.0
41.561964524 39.3553123474121 0.0
0 0.162209093570709 1.0
0 -0.318238437175751 2.0
0 0.18193793296814 3.0
0 0.240178048610687 4.0
0 0.0332656502723694 5.0
0 0.0976377725601196 6.0
0 0.0550718307495117 7.0
42.043570834 37.446460723877 0.0
0 0.115569293498993 1.0
0 -0.170804142951965 2.0
0 0.18812283873558 3.0
0 0.166608929634094 4.0
0 0.0624774694442749 5.0
0 0.107832282781601 6.0
0 0.101785838603973 7.0
42.728187908 38.8574638366699 0.0
0 0.0681304931640625 1.0
0 -0.27514386177063 2.0
0 0.435984820127487 3.0
0 0.176028311252594 4.0
0 0.243463844060898 5.0
0 0.0952571630477905 6.0
0 0.104485243558884 7.0
46.647822945 38.9622764587402 0.0
0 0.0185505747795105 1.0
0 -0.181002676486969 2.0
0 0.179768711328506 3.0
0 0.0265342593193054 4.0
0 0.267709940671921 5.0
0 0.0912636816501617 6.0
0 0.0967641174793243 7.0
34.157298462 38.9829216003418 0.0
0 0.107789874076843 1.0
0 -0.201152443885803 2.0
0 0.356271326541901 3.0
0 0.0768932104110718 4.0
0 0.293152451515198 5.0
0 0.0983332395553589 6.0
0 0.104051202535629 7.0
38.08940786 39.877311706543 0.0
0 0.0411693453788757 1.0
0 -0.278272867202759 2.0
0 0.341462016105652 3.0
0 -0.09538334608078 4.0
0 0.00629729032516479 5.0
0 -0.0003623366355896 6.0
0 0.1036356985569 7.0
48.253456605 39.8019409179688 0.0
0 0.303053617477417 1.0
0 -0.18071985244751 2.0
0 0.118003010749817 3.0
0 0.112859696149826 4.0
0 0.0932788848876953 5.0
0 0.104753583669662 6.0
0 0.105512261390686 7.0
41.303512955 37.6852912902832 0.0
0 0.208535581827164 1.0
0 -0.174204409122467 2.0
0 0.506354570388794 3.0
0 0.13662314414978 4.0
0 0.285356998443604 5.0
0 0.000149130821228027 6.0
0 0.0612633228302002 7.0
34.11247552 37.7005462646484 0.0
0 0.20411342382431 1.0
0 -0.347566664218903 2.0
0 0.00163143873214722 3.0
0 0.0357463955879211 4.0
0 0.0413997769355774 5.0
0 0.0217666625976562 6.0
0 0.0280149579048157 7.0
39.346899321 38.9947357177734 0.0
0 0.121137142181396 1.0
0 -0.155180275440216 2.0
0 0.470598578453064 3.0
0 0.26413482427597 4.0
0 0.11567148566246 5.0
0 0.0863151252269745 6.0
0 0.11216539144516 7.0
42.355081321 39.2480583190918 0.0
0 0.184658378362656 1.0
0 -0.293341040611267 2.0
0 0.0757453441619873 3.0
0 0.052903413772583 4.0
0 -0.117736518383026 5.0
0 0.155962198972702 6.0
0 0.0942680239677429 7.0
46.025284619 38.043384552002 0.0
0 0.207319259643555 1.0
0 -0.262724161148071 2.0
0 -0.214931309223175 3.0
0 -0.0535381436347961 4.0
0 0.0829015374183655 5.0
0 0.0505300760269165 6.0
0 0.0327348709106445 7.0
48.790909787 39.0293273925781 0.0
0 0.133867263793945 1.0
0 0.0448413491249084 2.0
0 0.225872159004211 3.0
0 0.176961213350296 4.0
0 0.0626283288002014 5.0
0 0.101919502019882 6.0
0 0.0795972347259521 7.0
41.698940502 38.9550170898438 0.0
0 0.0350402593612671 1.0
0 -0.177680909633636 2.0
0 -0.0743148922920227 3.0
0 -0.0180176496505737 4.0
0 -0.00920963287353516 5.0
0 0.0455068349838257 6.0
0 0.0531827807426453 7.0
47.902972781 37.8991470336914 0.0
0 0.132684141397476 1.0
0 -0.174726188182831 2.0
0 0.426056444644928 3.0
0 0.101516753435135 4.0
0 0.181210726499557 5.0
0 0.0716330409049988 6.0
0 0.0491575598716736 7.0
41.37717087 39.1559295654297 0.0
0 0.13051238656044 1.0
0 -0.177097082138062 2.0
0 0.206390172243118 3.0
0 0.247690796852112 4.0
0 0.14836573600769 5.0
0 0.0657179951667786 6.0
0 0.0257750153541565 7.0
32.285684902 39.1780967712402 0.0
0 0.184206277132034 1.0
0 -0.190393030643463 2.0
0 0.297194123268127 3.0
0 0.103926777839661 4.0
0 0.212099730968475 5.0
0 0.0742629170417786 6.0
0 0.0511477589607239 7.0
32.344206851 39.2967338562012 0.0
0 0.148774534463882 1.0
0 -0.173038244247437 2.0
0 0.0801907181739807 3.0
0 0.230119824409485 4.0
0 0.119050323963165 5.0
0 0.0499336719512939 6.0
0 0.073843240737915 7.0
34.184112331 39.5321731567383 0.0
0 0.118269085884094 1.0
0 -0.211611747741699 2.0
0 0.025023877620697 3.0
0 -0.0896178483963013 4.0
0 0.187612652778625 5.0
0 0.0661206841468811 6.0
0 0.107802271842957 7.0
34.663221808 38.1976699829102 0.0
0 0.0408746004104614 1.0
0 -0.112188637256622 2.0
0 0.398232191801071 3.0
0 0.28102320432663 4.0
0 0.105856329202652 5.0
0 0.095482349395752 6.0
0 0.0347302556037903 7.0
36.873774543 37.4889755249023 0.0
0 0.191829144954681 1.0
0 -0.244913578033447 2.0
0 0.370612889528275 3.0
0 0.12478893995285 4.0
0 0.0444193482398987 5.0
0 0.0239505171775818 6.0
0 0.0792433619499207 7.0
40.211352908 40.6358680725098 0.0
0 0.480438262224197 1.0
0 -0.21019572019577 2.0
0 -0.0725087523460388 3.0
0 -0.0156425833702087 4.0
0 0.0505297780036926 5.0
0 0.0599616169929504 6.0
0 0.104940354824066 7.0
46.160689867 39.8371162414551 0.0
0 0.185544222593307 1.0
0 -0.208363473415375 2.0
0 0.226292133331299 3.0
0 0.0217693448066711 4.0
0 0.181370884180069 5.0
0 0.0858277678489685 6.0
0 0.103877335786819 7.0
32.727702676 38.9891014099121 0.0
0 0.00175970792770386 1.0
0 -0.309086382389069 2.0
0 0.336612224578857 3.0
0 0.0237184762954712 4.0
0 0.108119249343872 5.0
0 0.0459532141685486 6.0
0 0.0519505739212036 7.0
40.640985417 39.3670463562012 0.0
0 0.197829723358154 1.0
0 -0.240063428878784 2.0
0 0.0923107862472534 3.0
0 0.278389036655426 4.0
0 0.0540100336074829 5.0
0 0.131994664669037 6.0
0 0.0385496020317078 7.0
34.611451594 39.5438919067383 0.0
0 0.0699900388717651 1.0
0 -0.106159448623657 2.0
0 0.103709399700165 3.0
0 0.251387268304825 4.0
0 0.0673412084579468 5.0
0 0.0903884172439575 6.0
0 0.0996327996253967 7.0
43.311280807 39.0435104370117 0.0
0 0.060958206653595 1.0
0 -0.214893221855164 2.0
0 -0.00868231058120728 3.0
0 -0.0223328471183777 4.0
0 0.0984504818916321 5.0
0 0.105542302131653 6.0
0 0.0803488790988922 7.0
36.406364057 38.017951965332 0.0
0 0.142668545246124 1.0
0 -0.339687824249268 2.0
0 0.175501704216003 3.0
0 0.107023864984512 4.0
0 0.0722126364707947 5.0
0 0.0281434059143066 6.0
0 0.0320847630500793 7.0
31.685630326 39.0170974731445 0.0
0 0.101808458566666 1.0
0 -0.369668960571289 2.0
0 0.203508615493774 3.0
0 0.292084366083145 4.0
0 0.0371660590171814 5.0
0 0.0919414162635803 6.0
0 0.0745237469673157 7.0
40.165364432 38.6968994140625 0.0
0 0.131273955106735 1.0
0 -0.466958820819855 2.0
0 0.158210456371307 3.0
0 0.22199609875679 4.0
0 0.0529943704605103 5.0
0 0.0734449625015259 6.0
0 0.0605495572090149 7.0
42.886465647 40.1209030151367 0.0
0 0.104512602090836 1.0
0 -0.145015001296997 2.0
0 0.385358452796936 3.0
0 0.174572646617889 4.0
0 0.20073676109314 5.0
0 0.0947732329368591 6.0
0 0.0649006962776184 7.0
46.942592332 39.1396675109863 0.0
0 0.380758196115494 1.0
0 -0.295076906681061 2.0
0 0.375941038131714 3.0
0 0.0216293931007385 4.0
0 0.249813050031662 5.0
0 0.0557432174682617 6.0
0 0.0430898070335388 7.0
38.417030568 38.6289939880371 0.0
0 0.181301057338715 1.0
0 -0.273594439029694 2.0
0 -0.05744469165802 3.0
0 0.16523352265358 4.0
0 0.0766940116882324 5.0
0 0.0932480692863464 6.0
0 0.0734415054321289 7.0
39.82159543 38.8474426269531 0.0
0 0.0548679232597351 1.0
0 -0.284725368022919 2.0
0 0.371740788221359 3.0
0 0.0886658430099487 4.0
0 0.07949298620224 5.0
0 0.107568711042404 6.0
0 0.0667757987976074 7.0
42.375846269 38.6078796386719 0.0
0 0.268896102905273 1.0
0 -0.332190573215485 2.0
0 0.220943063497543 3.0
0 0.0622798204421997 4.0
0 -0.0302289128303528 5.0
0 0.105222851037979 6.0
0 0.0380321145057678 7.0
41.909868924 38.8088722229004 0.0
0 0.187560796737671 1.0
0 -0.30673736333847 2.0
0 0.368500411510468 3.0
0 -0.427521526813507 4.0
0 0.0996734201908112 5.0
0 0.0592648386955261 6.0
0 0.067558228969574 7.0
38.515403814 39.1723937988281 0.0
0 0.128028243780136 1.0
0 -0.608425080776215 2.0
0 0.415702670812607 3.0
0 0.0823712944984436 4.0
0 3.18530702590942 5.0
0 0.0278237462043762 6.0
0 0.10342663526535 7.0
32.187464773 38.3935165405273 0.0
0 0.300832837820053 1.0
0 -0.373364806175232 2.0
0 0.363009542226791 3.0
0 0.042763888835907 4.0
0 0.173055797815323 5.0
0 0.082768976688385 6.0
0 0.0974271595478058 7.0
43.553340498 38.7763977050781 0.0
0 0.116506397724152 1.0
0 -0.21221512556076 2.0
0 0.00590765476226807 3.0
0 -0.052371621131897 4.0
0 0.290745794773102 5.0
0 0.0656821131706238 6.0
0 0.111147195100784 7.0
38.661685118 39.7086563110352 0.0
0 0.140102475881577 1.0
0 -0.134237229824066 2.0
0 0.319185197353363 3.0
0 0.241659730672836 4.0
0 0.0722343325614929 5.0
0 0.104466140270233 6.0
0 0.00614196062088013 7.0
47.185976948 37.2111587524414 0.0
0 0.192602634429932 1.0
0 -0.466671764850616 2.0
0 0.125170111656189 3.0
0 0.0696467757225037 4.0
0 -0.0848284363746643 5.0
0 0.0645761489868164 6.0
0 0.106305807828903 7.0
41.073649825 37.0528984069824 0.0
0 0.074797511100769 1.0
0 -0.524193584918976 2.0
0 0.202597498893738 3.0
0 0.235845297574997 4.0
0 -0.143394351005554 5.0
0 0.12596982717514 6.0
0 0.03476482629776 7.0
39.086114846 38.9791679382324 0.0
0 0.182076245546341 1.0
0 -0.344866871833801 2.0
0 0.138617038726807 3.0
0 0.184357017278671 4.0
0 0.167374223470688 5.0
0 0.0991499423980713 6.0
0 0.0682406425476074 7.0
38.072309151 38.3819618225098 0.0
0 0.265587568283081 1.0
0 -0.24794739484787 2.0
0 0.305415838956833 3.0
0 0.218677669763565 4.0
0 0.0888367891311646 5.0
0 0.0430960059165955 6.0
0 0.1012042760849 7.0
41.585402374 38.1360778808594 0.0
0 0.188892513513565 1.0
0 -0.207720577716827 2.0
0 0.273205041885376 3.0
0 0.136874437332153 4.0
0 0.228333085775375 5.0
0 0.0694810152053833 6.0
0 0.115198612213135 7.0
42.317317809 39.0473022460938 0.0
0 0.0771390199661255 1.0
0 -0.260855674743652 2.0
0 0.0379456877708435 3.0
0 0.0378016233444214 4.0
0 0.13632521033287 5.0
0 0.056754469871521 6.0
0 0.099614143371582 7.0
42.931477585 39.6296730041504 0.0
0 0.239216476678848 1.0
0 -0.195792496204376 2.0
0 0.1410853266716 3.0
0 0.0397462844848633 4.0
0 0.0165173411369324 5.0
0 0.0453606843948364 6.0
0 0.0910906791687012 7.0
46.096274411 31.5154552459717 0.0
0 0.035800039768219 1.0
0 -0.271441280841827 2.0
0 0.409827351570129 3.0
0 0.0484801530838013 4.0
0 0.270801216363907 5.0
0 -0.0134347081184387 6.0
0 0.062346875667572 7.0
39.955261554 38.165340423584 0.0
0 0.0550968050956726 1.0
0 -0.365926384925842 2.0
0 -0.0464365482330322 3.0
0 -0.0072898268699646 4.0
0 0.188099026679993 5.0
0 0.0693721175193787 6.0
0 0.0810456871986389 7.0
45.415961846 39.0248031616211 0.0
0 0.219127953052521 1.0
0 -0.259492874145508 2.0
0 0.166642516851425 3.0
0 0.161540567874908 4.0
0 -0.123895823955536 5.0
0 0.104949772357941 6.0
0 0.105180650949478 7.0
44.275746657 40.0994567871094 0.0
0 0.165111690759659 1.0
0 -0.188293993473053 2.0
0 0.105810612440109 3.0
0 0.0399323105812073 4.0
0 -0.00178307294845581 5.0
0 0.0671154260635376 6.0
0 0.0901864171028137 7.0
34.766238966 39.0199966430664 0.0
0 0.129204541444778 1.0
0 -0.117959678173065 2.0
0 0.25638610124588 3.0
0 0.228958785533905 4.0
0 0.00524568557739258 5.0
0 0.0409550070762634 6.0
0 0.0405973792076111 7.0
39.697519492 37.9191665649414 0.0
0 0.258143156766891 1.0
0 -0.423550546169281 2.0
0 0.494352161884308 3.0
0 0.140525758266449 4.0
0 0.24478280544281 5.0
0 0.0586646199226379 6.0
0 0.103285789489746 7.0
36.459952599 39.6673545837402 0.0
0 0.263325899839401 1.0
0 -0.322393715381622 2.0
0 0.121884375810623 3.0
0 0.0185009241104126 4.0
0 -0.0368326306343079 5.0
0 0.134189933538437 6.0
0 0.048122227191925 7.0
44.856515586 39.0351638793945 0.0
0 0.112403452396393 1.0
0 -0.371538758277893 2.0
0 0.327721834182739 3.0
0 0.237315982580185 4.0
0 0.150531083345413 5.0
0 0.0850524604320526 6.0
0 0.0120969414710999 7.0
41.157913956 38.9983177185059 0.0
0 0.194667369127274 1.0
0 -0.298083245754242 2.0
0 0.221132785081863 3.0
0 0.110576003789902 4.0
0 0.14956259727478 5.0
0 0.0527005195617676 6.0
0 0.0461313724517822 7.0
32.092085792 40.9547920227051 0.0
0 0.592653453350067 1.0
0 -0.243789494037628 2.0
0 0.065924346446991 3.0
0 0.125561088323593 4.0
0 0.0812416672706604 5.0
0 0.0525712966918945 6.0
0 0.080208420753479 7.0
30.327518425 39.3012237548828 0.0
0 0.141818404197693 1.0
0 -0.246282279491425 2.0
0 -0.0552976727485657 3.0
0 -0.11106938123703 4.0
0 0.111958742141724 5.0
0 0.0490158200263977 6.0
0 0.121152520179749 7.0
36.482977278 38.7291831970215 0.0
0 0.248161733150482 1.0
0 -0.245947897434235 2.0
0 0.18922546505928 3.0
0 0.0105641484260559 4.0
0 -0.020421028137207 5.0
0 0.0543847680091858 6.0
0 0.0537300705909729 7.0
40.491170914 39.0370063781738 0.0
0 0.292370975017548 1.0
0 -0.230859577655792 2.0
0 0.0917931497097015 3.0
0 -0.0902102589607239 4.0
0 0.062925398349762 5.0
0 0.0972277522087097 6.0
0 0.154460281133652 7.0
41.188898334 38.0011367797852 0.0
0 0.431780695915222 1.0
0 -0.438966453075409 2.0
0 0.238749742507935 3.0
0 0.168979078531265 4.0
0 0.0611376166343689 5.0
0 0.0461931824684143 6.0
0 0.0673525333404541 7.0
37.756949904 39.2269515991211 0.0
0 0.162413984537125 1.0
0 -0.110921621322632 2.0
0 0.301818639039993 3.0
0 0.132873386144638 4.0
0 0.180952847003937 5.0
0 0.0692065358161926 6.0
0 0.0868747234344482 7.0
26.9626458 38.3779907226562 0.0
0 0.105803370475769 1.0
0 -0.32587718963623 2.0
0 0.216994166374207 3.0
0 -0.0420586466789246 4.0
0 -0.0361351370811462 5.0
0 0.0163033604621887 6.0
0 0.104424744844437 7.0
44.154680866 39.5641326904297 0.0
0 0.0954333245754242 1.0
0 -0.129266917705536 2.0
0 0.202945619821548 3.0
0 -0.0629355311393738 4.0
0 0.0474832057952881 5.0
0 0.0208119750022888 6.0
0 0.0630439519882202 7.0
36.49920361 37.8865585327148 0.0
0 0.209673345088959 1.0
0 -0.504317343235016 2.0
0 0.154029101133347 3.0
0 0.219109982252121 4.0
0 -0.0863462090492249 5.0
0 0.0251454710960388 6.0
0 0.0543121695518494 7.0
36.136187007 38.525806427002 0.0
0 0.0340875387191772 1.0
0 -0.339787840843201 2.0
0 0.133167743682861 3.0
0 0.0256130695343018 4.0
0 0.0963473320007324 5.0
0 0.0775052905082703 6.0
0 0.0714256763458252 7.0
43.805905709 39.1021995544434 0.0
0 0.0993474125862122 1.0
0 -0.219907999038696 2.0
0 0.211238205432892 3.0
0 0.0186963081359863 4.0
0 0.147861003875732 5.0
0 0.035156786441803 6.0
0 0.116459518671036 7.0
43.400673478 40.0527496337891 0.0
0 0.232880890369415 1.0
0 -0.0692093372344971 2.0
0 0.127911180257797 3.0
0 0.0730301737785339 4.0
0 0.101988792419434 5.0
0 0.0564975142478943 6.0
0 0.0432288646697998 7.0
35.708898932 40.0352172851562 0.0
0 0.219367623329163 1.0
0 -0.149587273597717 2.0
0 0.0465837717056274 3.0
0 0.0925531685352325 4.0
0 0.228007614612579 5.0
0 0.0619924664497375 6.0
0 0.0548712611198425 7.0
42.195375979 38.1064109802246 0.0
0 0.116809636354446 1.0
0 -0.335074722766876 2.0
0 -0.0749156475067139 3.0
0 -0.0449526906013489 4.0
0 -0.0377073884010315 5.0
0 0.0605722665786743 6.0
0 0.0948418974876404 7.0
38.219132252 39.0664291381836 0.0
0 0.210696876049042 1.0
0 -0.257489204406738 2.0
0 0.4424147605896 3.0
0 0.239453017711639 4.0
0 -0.00811344385147095 5.0
0 0.0349594950675964 6.0
0 0.00566571950912476 7.0
37.34785904 40.227222442627 0.0
0 0.372851014137268 1.0
0 -0.106050729751587 2.0
0 0.0194820165634155 3.0
0 0.0157269835472107 4.0
0 0.121208369731903 5.0
0 0.0678232312202454 6.0
0 0.0959347784519196 7.0
41.413317083 38.6913108825684 0.0
0 0.108700811862946 1.0
0 -0.382158279418945 2.0
0 0.0394229888916016 3.0
0 -0.00760775804519653 4.0
0 0.0763519406318665 5.0
0 0.0387513637542725 6.0
0 0.104605913162231 7.0
39.408474012 38.9613456726074 0.0
0 0.466259956359863 1.0
0 -0.0983595848083496 2.0
0 -0.11505001783371 3.0
0 0.0223599076271057 4.0
0 0.01643306016922 5.0
0 0.092241108417511 6.0
0 0.051230788230896 7.0
44.621947793 37.7503242492676 0.0
0 0.0953361988067627 1.0
0 -0.364560842514038 2.0
0 0.152010381221771 3.0
0 0.113558948040009 4.0
0 0.00559794902801514 5.0
0 0.0460165739059448 6.0
0 0.0816767513751984 7.0
35.042314673 39.1348876953125 0.0
0 0.413945645093918 1.0
0 -0.558060586452484 2.0
0 0.335229337215424 3.0
0 0.112538754940033 4.0
0 -0.0105783343315125 5.0
0 0.0636071562767029 6.0
0 0.0843658447265625 7.0
44.668685987 37.9752311706543 0.0
0 0.405230939388275 1.0
0 -0.412923336029053 2.0
0 0.257300466299057 3.0
0 0.0947417020797729 4.0
0 0.264541864395142 5.0
0 0.00399124622344971 6.0
0 0.0806060433387756 7.0
44.271188235 38.4106750488281 0.0
0 0.0683602690696716 1.0
0 -0.174694240093231 2.0
0 0.167618572711945 3.0
0 0.217782318592072 4.0
0 0.0258288383483887 5.0
0 0.0183224081993103 6.0
0 0.106726944446564 7.0
43.329615715 38.5289726257324 0.0
0 0.0345512628555298 1.0
0 -0.0461201667785645 2.0
0 0.457494854927063 3.0
0 0.0874112546443939 4.0
0 0.30010136961937 5.0
0 0.0538806915283203 6.0
0 0.0770466923713684 7.0
37.727034893 40.0729866027832 0.0
0 0.219527721405029 1.0
0 -0.184283435344696 2.0
0 0.271896004676819 3.0
0 0.0201199650764465 4.0
0 -0.0104983448982239 5.0
0 0.0638813972473145 6.0
0 0.0057835578918457 7.0
43.665312113 39.2623672485352 0.0
0 0.128541886806488 1.0
0 -0.153792858123779 2.0
0 -0.131169378757477 3.0
0 -0.0215042233467102 4.0
0 0.100671231746674 5.0
0 0.0291687250137329 6.0
0 0.120224922895432 7.0
40.656348383 37.225757598877 0.0
0 0.242831289768219 1.0
0 -0.41424173116684 2.0
0 0.0112413167953491 3.0
0 0.0948969125747681 4.0
0 0.0561203956604004 5.0
0 0.0672270655632019 6.0
0 0.104199856519699 7.0
30.759139555 36.8887100219727 0.0
0 0.175766199827194 1.0
0 -0.412565410137177 2.0
0 0.482173085212708 3.0
0 0.117774099111557 4.0
0 0.22148722410202 5.0
0 0.0520322322845459 6.0
0 0.10602205991745 7.0
36.474229652 40.3473968505859 0.0
0 0.336661756038666 1.0
0 -0.179365634918213 2.0
0 0.211924105882645 3.0
0 0.327255427837372 4.0
0 0.0446092486381531 5.0
0 0.0505663752555847 6.0
0 0.10399317741394 7.0
33.340721195 39.4884223937988 0.0
0 0.158482491970062 1.0
0 -0.102805554866791 2.0
0 0.0833396911621094 3.0
0 0.155240476131439 4.0
0 0.168668746948242 5.0
0 0.10217759013176 6.0
0 0.0574203133583069 7.0
34.143349268 38.9502983093262 0.0
0 0.0428658127784729 1.0
0 -0.254290461540222 2.0
0 -0.219856858253479 3.0
0 -0.112665057182312 4.0
0 0.179022789001465 5.0
0 0.0717840790748596 6.0
0 0.0556066036224365 7.0
40.763829757 39.7238426208496 0.0
0 0.323816895484924 1.0
0 -0.129087269306183 2.0
0 0.38833749294281 3.0
0 0.270066499710083 4.0
0 0.0285493731498718 5.0
0 0.0864822566509247 6.0
0 0.0430599451065063 7.0
29.788224958 38.9521789550781 0.0
0 0.0594376921653748 1.0
0 -0.185148298740387 2.0
0 0.236301064491272 3.0
0 0.105485409498215 4.0
0 -0.0920310020446777 5.0
0 0.0599472522735596 6.0
0 0.101797491312027 7.0
34.675869554 39.263011932373 0.0
0 0.0744889974594116 1.0
0 0.0476109385490417 2.0
0 0.168990612030029 3.0
0 0.0953388810157776 4.0
0 0.167552649974823 5.0
0 0.066012442111969 6.0
0 0.111218154430389 7.0
40.723855828 38.5614967346191 0.0
0 0.126163691282272 1.0
0 -0.463956415653229 2.0
0 0.425771266222 3.0
0 -0.0542635917663574 4.0
0 0.127575784921646 5.0
0 0.0261368155479431 6.0
0 0.0208187103271484 7.0
41.414838352 39.5981712341309 0.0
0 0.154304981231689 1.0
0 -0.188266277313232 2.0
0 0.121966749429703 3.0
0 0.178900063037872 4.0
0 -0.0180523991584778 5.0
0 0.119679629802704 6.0
0 0.0531178116798401 7.0
33.391441214 39.317569732666 0.0
0 0.0748518705368042 1.0
0 -0.0399666428565979 2.0
0 0.123537957668304 3.0
0 0.162208706140518 4.0
0 0.235503911972046 5.0
0 0.0531335473060608 6.0
0 0.042574405670166 7.0
36.660645385 39.8293952941895 0.0
0 0.25019708275795 1.0
0 -0.0199326872825623 2.0
0 0.128274321556091 3.0
0 0.0403843522071838 4.0
0 0.0454885959625244 5.0
0 0.104566782712936 6.0
0 0.0935206115245819 7.0
33.547567166 41.0247764587402 0.0
0 0.392194449901581 1.0
0 -0.159224212169647 2.0
0 0.310957580804825 3.0
0 0.0193325877189636 4.0
0 -0.0394777059555054 5.0
0 0.0606768131256104 6.0
0 -0.000474810600280762 7.0
36.717923495 39.0211868286133 0.0
0 0.0514935851097107 1.0
0 -0.285399079322815 2.0
0 0.237233906984329 3.0
0 0.0391503572463989 4.0
0 0.0133475065231323 5.0
0 0.0853535830974579 6.0
0 0.0328409671783447 7.0
44.575737494 39.4325294494629 0.0
0 -0.011311411857605 1.0
0 -0.0984312891960144 2.0
0 0.220225244760513 3.0
0 0.25248047709465 4.0
0 0.0510119199752808 5.0
0 0.101122677326202 6.0
0 0.0605931282043457 7.0
43.091610321 39.2336082458496 0.0
0 0.0938403010368347 1.0
0 -0.235071182250977 2.0
0 0.448576718568802 3.0
0 0.187233150005341 4.0
0 0.0361866354942322 5.0
0 0.0462285280227661 6.0
0 0.0938141644001007 7.0
42.647069973 38.7813835144043 0.0
0 0.178579658269882 1.0
0 -0.239987373352051 2.0
0 0.074845016002655 3.0
0 0.113256692886353 4.0
0 0.17026475071907 5.0
0 0.102036237716675 6.0
0 0.0562322735786438 7.0
36.240225339 39.8886184692383 0.0
0 0.130421340465546 1.0
0 -0.0108545422554016 2.0
0 0.169640958309174 3.0
0 0.120680630207062 4.0
0 0.0632625222206116 5.0
0 0.0811519622802734 6.0
0 0.103288650512695 7.0
31.876756203 39.1697540283203 0.0
0 0.126743942499161 1.0
0 -0.100460827350616 2.0
0 0.186080634593964 3.0
0 0.305747359991074 4.0
0 0.236165285110474 5.0
0 0.0216341018676758 6.0
0 0.0236929655075073 7.0
38.711049659 39.3955917358398 0.0
0 0.109235614538193 1.0
0 -0.248242557048798 2.0
0 0.278927117586136 3.0
0 -0.00979584455490112 4.0
0 0.00648772716522217 5.0
0 0.0342088937759399 6.0
0 0.0382604002952576 7.0
33.592635672 38.7662162780762 0.0
0 0.234888851642609 1.0
0 -0.185719907283783 2.0
0 0.257292717695236 3.0
0 0.144186675548553 4.0
0 0.14811834692955 5.0
0 0.0967649817466736 6.0
0 0.0896330773830414 7.0
31.939497021 39.0791931152344 0.0
0 0.0538360476493835 1.0
0 -0.293799996376038 2.0
0 0.278938621282578 3.0
0 0.0538580417633057 4.0
0 0.195330709218979 5.0
0 0.0384100675582886 6.0
0 0.0550302863121033 7.0
32.605677369 38.8344955444336 0.0
0 0.0568776726722717 1.0
0 -0.366280257701874 2.0
0 -0.294632852077484 3.0
0 -0.0810278058052063 4.0
0 -0.00966572761535645 5.0
0 0.0225486755371094 6.0
0 0.147163331508636 7.0
35.02561425 38.9368095397949 0.0
0 0.163153469562531 1.0
0 -0.103927135467529 2.0
0 0.198227554559708 3.0
0 0.269168764352798 4.0
0 -0.00122517347335815 5.0
0 0.0427493453025818 6.0
0 0.026974081993103 7.0
29.155020371 39.3215789794922 0.0
0 0.127705097198486 1.0
0 -0.121565222740173 2.0
0 -0.0903001427650452 3.0
0 -0.00636142492294312 4.0
0 -0.0287075638771057 5.0
0 0.0245171785354614 6.0
0 0.0344082117080688 7.0
29.9790999 26.6453838348389 0.0
0 -0.0851418375968933 1.0
0 -0.243894875049591 2.0
0 0.132108777761459 3.0
0 0.0328370928764343 4.0
0 -0.0518620610237122 5.0
0 0.0916978120803833 6.0
0 0.0474416017532349 7.0
40.412374428 39.6186752319336 0.0
0 0.135339289903641 1.0
0 -0.292015969753265 2.0
0 0.255121797323227 3.0
0 -0.134176790714264 4.0
0 0.0454390048980713 5.0
0 0.0724239349365234 6.0
0 0.124707818031311 7.0
44.799669607 38.842399597168 0.0
0 0.0915895700454712 1.0
0 -0.10363757610321 2.0
0 0.281621307134628 3.0
0 0.0344362258911133 4.0
0 -0.00329810380935669 5.0
0 0.103824943304062 6.0
0 0.0174244046211243 7.0
35.691930085 39.6875305175781 0.0
0 0.0594415664672852 1.0
0 -0.18617445230484 2.0
0 0.180970847606659 3.0
0 0.259136170148849 4.0
0 0.0192562341690063 5.0
0 0.0503268241882324 6.0
0 0.0466536283493042 7.0
37.293002382 37.7406158447266 0.0
0 0.244414240121841 1.0
0 -0.244940638542175 2.0
0 -0.0242816805839539 3.0
0 0.0580477714538574 4.0
0 0.0145636200904846 5.0
0 0.0835646390914917 6.0
0 0.0561164021492004 7.0
31.749050635 38.9245262145996 0.0
0 0.200691878795624 1.0
0 -0.28138130903244 2.0
0 0.105671465396881 3.0
0 0.213470906019211 4.0
0 0.136605203151703 5.0
0 0.0262956023216248 6.0
0 0.0869601666927338 7.0
44.438599316 38.3011322021484 0.0
0 0.361769556999207 1.0
0 -0.705269396305084 2.0
0 0.180408895015717 3.0
0 0.132625669240952 4.0
0 0.0657750964164734 5.0
0 0.0081743597984314 6.0
0 0.0436748266220093 7.0
30.584762345 38.2930564880371 0.0
0 0.134768605232239 1.0
0 -0.0457072257995605 2.0
0 0.158049017190933 3.0
0 0.179893761873245 4.0
0 0.203094363212585 5.0
0 0.0325527787208557 6.0
0 0.0563873648643494 7.0
36.167711467 39.0721168518066 0.0
0 0.441249668598175 1.0
0 -0.0911988615989685 2.0
0 -0.00765836238861084 3.0
0 0.223888486623764 4.0
0 0.154467403888702 5.0
0 0.00785666704177856 6.0
0 0.0471174716949463 7.0
45.589023165 38.3483123779297 0.0
0 0.198158442974091 1.0
0 -0.291524708271027 2.0
0 -0.0462918281555176 3.0
0 0.0855742692947388 4.0
0 0.287724405527115 5.0
0 0.0763636827468872 6.0
0 0.0990001559257507 7.0
33.954863975 38.9760093688965 0.0
0 0.116469919681549 1.0
0 -0.192067921161652 2.0
0 0.11505052447319 3.0
0 -0.17372852563858 4.0
0 0.216288059949875 5.0
0 0.0927420258522034 6.0
0 0.12074014544487 7.0
39.104021217 39.436351776123 0.0
0 0.121094346046448 1.0
0 -0.203803658485413 2.0
0 0.41129207611084 3.0
0 0.133088976144791 4.0
0 0.203842431306839 5.0
0 0.102499574422836 6.0
0 0.0822912752628326 7.0
41.615139375 38.4186859130859 0.0
0 0.0983349680900574 1.0
0 -0.413347780704498 2.0
0 0.100610762834549 3.0
0 0.129998296499252 4.0
0 -0.00979059934616089 5.0
0 0.0864686071872711 6.0
0 0.0553542375564575 7.0
29.251209169 38.7161598205566 0.0
0 0.101595014333725 1.0
0 -0.0680388808250427 2.0
0 0.1402188539505 3.0
0 -0.051482081413269 4.0
0 0.199967622756958 5.0
0 0.0315021276473999 6.0
0 0.0386528968811035 7.0
39.145261601 38.6310997009277 0.0
0 0.0101165175437927 1.0
0 -0.322955727577209 2.0
0 0.0877294540405273 3.0
0 -0.0667727589607239 4.0
0 -0.107443451881409 5.0
0 0.0582050681114197 6.0
0 0.101091474294662 7.0
43.724223514 38.0647239685059 0.0
0 0.271201074123383 1.0
0 -0.321541368961334 2.0
0 0.212095767259598 3.0
0 0.283133119344711 4.0
0 0.178793430328369 5.0
0 0.106475293636322 6.0
0 0.0629785060882568 7.0
30.694427948 38.255069732666 0.0
0 0.220324128866196 1.0
0 -0.317168653011322 2.0
0 0.294775009155273 3.0
0 0.0862574577331543 4.0
0 0.116187900304794 5.0
0 0.100367337465286 6.0
0 0.0358895659446716 7.0
46.899446079 39.6550102233887 0.0
0 0.0874642133712769 1.0
0 -0.172606229782104 2.0
0 0.225048065185547 3.0
0 0.258626759052277 4.0
0 0.0723397135734558 5.0
0 0.0199654698371887 6.0
0 0.0415757894515991 7.0
30.801741874 38.3756332397461 0.0
0 0.168315351009369 1.0
0 -0.249357640743256 2.0
0 -0.0641103386878967 3.0
0 0.0626844763755798 4.0
0 0.141472935676575 5.0
0 0.00665950775146484 6.0
0 0.0451257824897766 7.0
40.388900757 38.5836296081543 0.0
0 0.20433509349823 1.0
0 -0.0292037129402161 2.0
0 0.411402106285095 3.0
0 0.110278338193893 4.0
0 0.191338390111923 5.0
0 0.0619016289710999 6.0
0 0.0623325109481812 7.0
35.801686131 36.8812294006348 0.0
0 0.0865603685379028 1.0
0 -0.320180773735046 2.0
0 0.279478222131729 3.0
0 -0.0409498810768127 4.0
0 -0.0884492993354797 5.0
0 0.0629287362098694 6.0
0 0.00447738170623779 7.0
43.275430302 39.5215492248535 0.0
0 0.143214404582977 1.0
0 -0.173079669475555 2.0
0 0.37442073225975 3.0
0 0.216345101594925 4.0
0 0.011504590511322 5.0
0 0.0551148056983948 6.0
0 0.0458220243453979 7.0
36.674533885 39.6237564086914 0.0
0 0.136975258588791 1.0
0 -0.10213565826416 2.0
0 0.124202132225037 3.0
0 0.199204057455063 4.0
0 0.1837297976017 5.0
0 0.0361452698707581 6.0
0 0.102251380681992 7.0
42.291929941 37.8345489501953 0.0
0 0.0397945046424866 1.0
0 -0.283786356449127 2.0
0 0.0274489521980286 3.0
0 0.142290860414505 4.0
0 0.0676493644714355 5.0
0 0.0920018255710602 6.0
0 0.0918528735637665 7.0
37.80686855 39.9807968139648 0.0
0 0.215076595544815 1.0
0 -0.149397134780884 2.0
0 0.178386360406876 3.0
0 0.259475618600845 4.0
0 0.0894412398338318 5.0
0 0.0703402161598206 6.0
0 0.0492225289344788 7.0
36.296445808 39.2528457641602 0.0
0 0.119009852409363 1.0
0 -0.164544403553009 2.0
0 0.185332477092743 3.0
0 0.00344955921173096 4.0
0 0.244509816169739 5.0
0 0.0462156534194946 6.0
0 0.0906082689762115 7.0
26.786347985 39.6782150268555 0.0
0 0.412971884012222 1.0
0 -0.235040366649628 2.0
0 -0.143387794494629 3.0
0 -0.0593023896217346 4.0
0 0.207922041416168 5.0
0 0.0937956571578979 6.0
0 0.106106579303741 7.0
38.715518937 39.0820198059082 0.0
0 0.0776950716972351 1.0
0 -0.0836598873138428 2.0
0 0.282278001308441 3.0
0 0.166859894990921 4.0
0 0.27400153875351 5.0
0 0.100548416376114 6.0
0 0.0777260661125183 7.0
35.505063002 37.7152633666992 0.0
0 0.15953254699707 1.0
0 -0.249560058116913 2.0
0 -0.0497435927391052 3.0
0 -0.0454859733581543 4.0
0 0.0389724373817444 5.0
0 0.0736283659934998 6.0
0 0.0774641036987305 7.0
44.227649071 39.4213829040527 0.0
0 0.0587796568870544 1.0
0 -0.270298898220062 2.0
0 0.135049939155579 3.0
0 0.269066870212555 4.0
0 0.220221281051636 5.0
0 0.100443601608276 6.0
0 0.0373867154121399 7.0
45.773124255 38.3122749328613 0.0
0 0.104377031326294 1.0
0 -0.557223618030548 2.0
0 0.179675191640854 3.0
0 0.185387164354324 4.0
0 0.0538160800933838 5.0
0 0.0961584150791168 6.0
0 0.10276135802269 7.0
38.615282752 39.6666831970215 0.0
0 0.0658456683158875 1.0
0 -0.153467416763306 2.0
0 -0.0603868365287781 3.0
0 0.0600304007530212 4.0
0 0.0507209300994873 5.0
0 0.0638676881790161 6.0
0 0.0661855340003967 7.0
42.131124836 39.284366607666 0.0
0 0.0138945579528809 1.0
0 -0.160726189613342 2.0
0 0.131690561771393 3.0
0 -0.0572517514228821 4.0
0 0.0327674746513367 5.0
0 0.0994507074356079 6.0
0 0.0906044542789459 7.0
38.030369766 38.9131011962891 0.0
0 0.0284256935119629 1.0
0 -0.250559210777283 2.0
0 0.0917838513851166 3.0
0 0.17200368642807 4.0
0 0.237227976322174 5.0
0 0.0941331386566162 6.0
0 0.035011887550354 7.0
32.694870002 39.2197723388672 0.0
0 0.0345250368118286 1.0
0 -0.26122260093689 2.0
0 0.308987259864807 3.0
0 0.0410435199737549 4.0
0 0.272785395383835 5.0
0 0.0019071102142334 6.0
0 0.0694894194602966 7.0
41.58883184 40.1937408447266 0.0
0 0.0676918029785156 1.0
0 -0.0807204842567444 2.0
0 0.00975000858306885 3.0
0 0.0663911700248718 4.0
0 0.233048677444458 5.0
0 0.0666254758834839 6.0
0 0.0920409560203552 7.0
33.622063045 37.9257316589355 0.0
0 0.358190685510635 1.0
0 -0.303199946880341 2.0
0 0.239748179912567 3.0
0 -0.0122246146202087 4.0
0 0.0357761979103088 5.0
0 0.0480456352233887 6.0
0 0.0959286093711853 7.0
40.101738506 38.6501770019531 0.0
0 0.27259624004364 1.0
0 -0.3777756690979 2.0
0 0.0694442987442017 3.0
0 0.0322746634483337 4.0
0 0.201480656862259 5.0
0 0.0250082015991211 6.0
0 0.0560486912727356 7.0
39.347750494 38.8217163085938 0.0
0 0.188764601945877 1.0
0 -0.242400407791138 2.0
0 0.199323385953903 3.0
0 0.0846736133098602 4.0
0 0.107891887426376 5.0
0 0.0603514313697815 6.0
0 0.0682306289672852 7.0
46.240008463 38.0697402954102 0.0
0 0.220066457986832 1.0
0 -0.283336520195007 2.0
0 -0.0602847933769226 3.0
0 -0.151269733905792 4.0
0 0.0578214526176453 5.0
0 0.103074073791504 6.0
0 0.0999358892440796 7.0
43.885613972 38.2498435974121 0.0
0 0.0469804406166077 1.0
0 -0.0113129615783691 2.0
0 0.112399578094482 3.0
0 0.050922691822052 4.0
0 0.265465587377548 5.0
0 0.0450606942176819 6.0
0 0.0663840770721436 7.0
44.877286055 38.6723518371582 0.0
0 0.1880943775177 1.0
0 -0.363060295581818 2.0
0 0.362207323312759 3.0
0 0.113967418670654 4.0
0 -0.0144871473312378 5.0
0 0.0337198972702026 6.0
0 0.0625296235084534 7.0
44.281024849 38.3056526184082 0.0
0 0.100916266441345 1.0
0 -0.26354593038559 2.0
0 0.242095977067947 3.0
0 0.0955604910850525 4.0
0 -0.0503717064857483 5.0
0 0.0475341081619263 6.0
0 0.086369127035141 7.0
39.443195334 38.9969329833984 0.0
0 0.0575727224349976 1.0
0 -0.16666966676712 2.0
0 0.154137432575226 3.0
0 0.187754720449448 4.0
0 0.0264368653297424 5.0
0 0.0890495181083679 6.0
0 0.0456554889678955 7.0
44.50998012 37.593433380127 0.0
0 0.301767587661743 1.0
0 -0.417859613895416 2.0
0 0.36061429977417 3.0
0 0.125781029462814 4.0
0 0.264175802469254 5.0
0 0.0872687995433807 6.0
0 0.0260075926780701 7.0
36.079576822 37.3783569335938 0.0
0 0.249690502882004 1.0
0 -0.536512672901154 2.0
0 0.0708037614822388 3.0
0 0.0872779190540314 4.0
0 0.073954701423645 5.0
0 0.096854567527771 6.0
0 0.0866559147834778 7.0
43.881758876 38.0163497924805 0.0
0 0.0270268321037292 1.0
0 1.29792428016663 2.0
0 4.35810708999634 3.0
0 -0.0804174542427063 4.0
0 -0.0803717374801636 5.0
0 0.0482848286628723 6.0
0 0.0875602960586548 7.0
30.108818284 38.3146514892578 0.0
0 0.406413853168488 1.0
0 -0.403348445892334 2.0
0 0.00151824951171875 3.0
0 0.11205005645752 4.0
0 0.0827207863330841 5.0
0 0.0783774256706238 6.0
0 0.0848283469676971 7.0
34.22051219 38.4804725646973 0.0
0 0.171233355998993 1.0
0 -0.375447690486908 2.0
0 -0.0406938791275024 3.0
0 -0.0347813963890076 4.0
0 -0.059441089630127 5.0
0 0.0309355854988098 6.0
0 0.0663517117500305 7.0
37.402197971 38.7805786132812 0.0
0 0.03448086977005 1.0
0 -0.234183669090271 2.0
0 0.149088859558105 3.0
0 0.30345430970192 4.0
0 0.208263039588928 5.0
0 0.01643306016922 6.0
0 0.0183960795402527 7.0
35.453722409 38.8814964294434 0.0
0 0.0942878723144531 1.0
0 -0.31326562166214 2.0
0 0.339212894439697 3.0
0 0.187250912189484 4.0
0 0.18021160364151 5.0
0 0.0162320137023926 6.0
0 0.0693199038505554 7.0
32.872381155 38.658088684082 0.0
0 0.013266384601593 1.0
0 -0.293338716030121 2.0
0 0.103635370731354 3.0
0 -0.0946682691574097 4.0
0 0.151514977216721 5.0
0 0.0551401376724243 6.0
0 0.0615355968475342 7.0
33.577714316 38.7518882751465 0.0
0 0.0336742401123047 1.0
0 -0.153949558734894 2.0
0 0.251202374696732 3.0
0 0.116009414196014 4.0
0 -0.00855910778045654 5.0
0 0.0881185829639435 6.0
0 0.0696542263031006 7.0
43.986168802 38.9482116699219 0.0
0 0.0693765878677368 1.0
0 -0.188219368457794 2.0
0 0.218058675527573 3.0
0 0.0896818339824677 4.0
0 0.158589243888855 5.0
0 0.066815972328186 6.0
0 0.0503653287887573 7.0
39.701527205 38.0937805175781 0.0
0 0.0862821638584137 1.0
0 -0.254271030426025 2.0
0 0.295866459608078 3.0
0 -0.106491684913635 4.0
0 0.189636945724487 5.0
0 0.0390450954437256 6.0
0 0.034492552280426 7.0
36.319422317 38.4191932678223 0.0
0 0.0492256283760071 1.0
0 0.00347787141799927 2.0
0 0.176424503326416 3.0
0 0.214515924453735 4.0
0 0.100721567869186 5.0
0 0.0749691128730774 6.0
0 0.0204628705978394 7.0
39.54652972 39.1133117675781 0.0
0 0.167510867118835 1.0
0 -0.274939656257629 2.0
0 0.126876771450043 3.0
0 0.271535485982895 4.0
0 0.0801651775836945 5.0
0 0.0770581364631653 6.0
0 0.0402434468269348 7.0
46.3288844 38.3921241760254 0.0
0 0.0874324142932892 1.0
0 -0.412931680679321 2.0
0 0.384413778781891 3.0
0 0.107956856489182 4.0
0 0.229857534170151 5.0
0 0.0586825609207153 6.0
0 0.0560312271118164 7.0
39.359671461 39.0734710693359 0.0
0 0.194706112146378 1.0
0 -0.253210127353668 2.0
0 0.187690079212189 3.0
0 0.239934414625168 4.0
0 0.033532440662384 5.0
0 0.0361685752868652 6.0
0 0.0681063532829285 7.0
43.813870988 39.186653137207 0.0
0 0.0211265683174133 1.0
0 -0.227564871311188 2.0
0 0.136095404624939 3.0
0 0.0978881418704987 4.0
0 0.162379145622253 5.0
0 0.0345104932785034 6.0
0 0.0785377621650696 7.0
39.146221876 37.6334266662598 0.0
0 0.109220832586288 1.0
0 -0.290313899517059 2.0
0 -0.0367190837860107 3.0
0 0.123817771673203 4.0
0 0.141301691532135 5.0
0 0.0816070437431335 6.0
0 0.0900182127952576 7.0
34.753353014 39.9073104858398 0.0
0 0.268780410289764 1.0
0 -0.371871590614319 2.0
0 0.271967202425003 3.0
0 0.255214542150497 4.0
0 0.0102774500846863 5.0
0 0.0965928137302399 6.0
0 0.0295851826667786 7.0
36.858915008 38.5453338623047 0.0
0 0.0444180965423584 1.0
0 -0.542928040027618 2.0
0 0.0802535712718964 3.0
0 0.123565584421158 4.0
0 0.210404872894287 5.0
0 0.0818102955818176 6.0
0 0.00854068994522095 7.0
45.609388725 37.8730926513672 0.0
0 0.104723364114761 1.0
0 -0.431912362575531 2.0
0 0.141285955905914 3.0
0 0.184821665287018 4.0
0 0.0736566185951233 5.0
0 0.0605595707893372 6.0
0 0.0953159332275391 7.0
37.221502314 39.1521453857422 0.0
0 0.113838285207748 1.0
0 -0.0940012335777283 2.0
0 0.0444033145904541 3.0
0 0.250622123479843 4.0
0 0.232276618480682 5.0
0 0.0379044413566589 6.0
0 0.0981320738792419 7.0
39.351241848 38.2533226013184 0.0
0 0.0104843378067017 1.0
0 -0.0972163081169128 2.0
0 0.167330861091614 3.0
0 0.0622261166572571 4.0
0 0.319569796323776 5.0
0 0.0189304947853088 6.0
0 0.104152262210846 7.0
44.354286139 38.3020362854004 0.0
0 0.0911024510860443 1.0
0 -0.327382266521454 2.0
0 0.183288395404816 3.0
0 0.282180309295654 4.0
0 0.0950217545032501 5.0
0 0.0701887011528015 6.0
0 0.0672353506088257 7.0
41.143843969 39.9276161193848 0.0
0 0.129389435052872 1.0
0 -0.0645139217376709 2.0
0 0.159040719270706 3.0
0 0.0249748229980469 4.0
0 0.019354522228241 5.0
0 0.0734050869941711 6.0
0 0.0532505512237549 7.0
32.436363672 38.0251083374023 0.0
0 0.264173120260239 1.0
0 1.85424494743347 2.0
0 0.403092086315155 3.0
0 0.14417177438736 4.0
0 0.24737411737442 5.0
0 0.0458065867424011 6.0
0 0.0517071485519409 7.0
35.980268826 39.2913017272949 0.0
0 0.208267718553543 1.0
0 -0.233141362667084 2.0
0 0.397320091724396 3.0
0 0.057738721370697 4.0
0 0.128570228815079 5.0
0 0.0674723982810974 6.0
0 0.0449497103691101 7.0
39.361696951 38.91845703125 0.0
0 0.0806533992290497 1.0
0 -0.16222071647644 2.0
0 0.199686646461487 3.0
0 -0.0799407362937927 4.0
0 0.0531055927276611 5.0
0 0.0575547814369202 6.0
0 0.0851425528526306 7.0
38.339957501 39.8487701416016 0.0
0 0.168667525053024 1.0
0 -0.065963089466095 2.0
0 0.234850168228149 3.0
0 0.0254517197608948 4.0
0 0.23678070306778 5.0
0 0.0920210778713226 6.0
0 0.0877096354961395 7.0
38.437078026 38.0763969421387 0.0
0 0.155505955219269 1.0
0 -0.29427695274353 2.0
0 0.0224050879478455 3.0
0 0.0383347868919373 4.0
0 0.174937665462494 5.0
0 0.0517436861991882 6.0
0 0.103316485881805 7.0
31.680301161 39.197509765625 0.0
0 0.00982123613357544 1.0
0 -0.235815525054932 2.0
0 0.337950766086578 3.0
0 -0.0311786532402039 4.0
0 0.161551058292389 5.0
0 0.0246113538742065 6.0
0 0.030445396900177 7.0
36.698769132 37.8871879577637 0.0
0 0.01828533411026 1.0
0 -0.289300620555878 2.0
0 0.143347948789597 3.0
0 0.177105188369751 4.0
0 0.0390709042549133 5.0
0 0.0684953927993774 6.0
0 0.0720704793930054 7.0
40.942334922 39.2907485961914 0.0
0 0.162173360586166 1.0
0 -0.0492004752159119 2.0
0 0.236332058906555 3.0
0 0.116420984268188 4.0
0 0.264428704977036 5.0
0 0.0632760524749756 6.0
0 0.0261185765266418 7.0
41.496981537 38.164665222168 0.0
0 0.190178751945496 1.0
0 -0.322232902050018 2.0
0 0.248264163732529 3.0
0 0.101468622684479 4.0
0 0.0491760969161987 5.0
0 0.129750370979309 6.0
0 0.0573945045471191 7.0
33.577466065 37.1262245178223 0.0
0 0.213195323944092 1.0
0 -0.251890063285828 2.0
0 0.0652359127998352 3.0
0 0.0441580414772034 4.0
0 0.212877869606018 5.0
0 0.0567585825920105 6.0
0 0.0103389620780945 7.0
38.410697595 39.7836608886719 0.0
0 0.105471074581146 1.0
0 -0.166187345981598 2.0
0 0.267259806394577 3.0
0 0.0104923248291016 4.0
0 0.277551889419556 5.0
0 0.0621811747550964 6.0
0 0.0928413271903992 7.0
40.390522112 38.9619979858398 0.0
0 0.0747841000556946 1.0
0 -0.211615025997162 2.0
0 0.133994966745377 3.0
0 -0.0528225898742676 4.0
0 0.025395393371582 5.0
0 0.0991334319114685 6.0
0 0.0826863646507263 7.0
38.02724699 38.8506202697754 0.0
0 0.142094850540161 1.0
0 -0.260440170764923 2.0
0 0.151667892932892 3.0
0 0.065653920173645 4.0
0 0.235315710306168 5.0
0 0.0714364051818848 6.0
0 0.0756347179412842 7.0
34.831322295 38.5968894958496 0.0
0 0.361035823822021 1.0
0 -0.502671897411346 2.0
0 0.0508919358253479 3.0
0 -0.0828841328620911 4.0
0 0.235756784677505 5.0
0 0.0940537452697754 6.0
0 0.159471303224564 7.0
33.152476832 39.8421096801758 0.0
0 0.21165806055069 1.0
0 -0.135443925857544 2.0
0 -0.0444071888923645 3.0
0 -0.00606274604797363 4.0
0 0.281281411647797 5.0
0 -0.0303897261619568 6.0
0 0.0778270363807678 7.0
37.76181476 38.0382270812988 0.0
0 0.0460448265075684 1.0
0 -0.447409331798553 2.0
0 0.133678942918777 3.0
0 0.348241060972214 4.0
0 0.0241657495498657 5.0
0 0.0559513568878174 6.0
0 0.102951407432556 7.0
46.821650967 39.5731010437012 0.0
0 0.12138444185257 1.0
0 -0.152928411960602 2.0
0 0.1845842897892 3.0
0 0.150940716266632 4.0
0 0.0679001808166504 5.0
0 0.0957375168800354 6.0
0 0.0817579030990601 7.0
41.183790175 38.2608299255371 0.0
0 0.0641592144966125 1.0
0 -0.141824305057526 2.0
0 0.268264681100845 3.0
0 0.223928213119507 4.0
0 0.199191719293594 5.0
0 0.0150734186172485 6.0
0 0.113120853900909 7.0
36.78498864 38.3884773254395 0.0
0 0.239157378673553 1.0
0 -0.461137592792511 2.0
0 0.165458828210831 3.0
0 0.202468514442444 4.0
0 0.0824925601482391 5.0
0 0.0636657476425171 6.0
0 0.0857754647731781 7.0
37.689494945 39.645622253418 0.0
0 0.282541692256927 1.0
0 -0.168411076068878 2.0
0 0.0453059077262878 3.0
0 0.0911214351654053 4.0
0 0.184910029172897 5.0
0 0.103742480278015 6.0
0 0.0827000141143799 7.0
41.308881097 38.0458145141602 0.0
0 0.0346659421920776 1.0
0 0.00465136766433716 2.0
0 0.440919011831284 3.0
0 0.32344263792038 4.0
0 -0.0704289674758911 5.0
0 0.147878587245941 6.0
0 0.0251806378364563 7.0
42.253011184 37.978443145752 0.0
0 0.347388178110123 1.0
0 -0.298045754432678 2.0
0 0.137027412652969 3.0
0 0.164326339960098 4.0
0 0.141996890306473 5.0
0 0.0860840380191803 6.0
0 0.0869186222553253 7.0
40.047660493 39.0868301391602 0.0
0 0.0183500647544861 1.0
0 -0.179413437843323 2.0
0 0.0820533335208893 3.0
0 0.166805297136307 4.0
0 0.212336629629135 5.0
0 0.0824986398220062 6.0
0 0.0370895862579346 7.0
46.604605733 40.0943489074707 0.0
0 0.177157759666443 1.0
0 -0.264964401721954 2.0
0 0.190988838672638 3.0
0 -0.0170425176620483 4.0
0 0.0070306658744812 5.0
0 0.0923690497875214 6.0
0 0.0974900722503662 7.0
45.106295542 38.7109909057617 0.0
0 0.281895667314529 1.0
0 -0.346625983715057 2.0
0 0.218680948019028 3.0
0 0.172960698604584 4.0
0 0.26248300075531 5.0
0 0.0165182948112488 6.0
0 0.0392112135887146 7.0
38.977055099 39.7192420959473 0.0
0 0.156788170337677 1.0
0 -0.142615795135498 2.0
0 0.20913302898407 3.0
0 0.0905554592609406 4.0
0 0.0578668713569641 5.0
0 0.102139562368393 6.0
0 0.039167582988739 7.0
38.935365883 38.4889030456543 0.0
0 0.0644771456718445 1.0
0 -0.428007423877716 2.0
0 -0.0179930329322815 3.0
0 0.0174145102500916 4.0
0 0.110809832811356 5.0
0 0.0951583385467529 6.0
0 0.107346594333649 7.0
42.741409918 39.1381721496582 0.0
0 0.0943843126296997 1.0
0 -0.139256715774536 2.0
0 0.037326991558075 3.0
0 0.136139392852783 4.0
0 0.13518238067627 5.0
0 0.0282229781150818 6.0
0 0.0101558566093445 7.0
31.310794576 38.8511962890625 0.0
0 0.0104883313179016 1.0
0 -0.177614390850067 2.0
0 -0.054735004901886 3.0
0 0.121023893356323 4.0
0 -0.0384749174118042 5.0
0 0.0464961528778076 6.0
0 0.0561214089393616 7.0
};
\addlegendentry{$R^2$=0.983}
\end{axis}

\end{tikzpicture}
}}
    \subfloat[Actual vs predicted edge flows.] 
    {\label{fig:results_nonlineal_dummy_edge_base_wey}\resizebox{\figurewidth}{\figureheight}{% This file was created with tikzplotlib v0.10.1.
\begin{tikzpicture}

\definecolor{darkgray176}{RGB}{176,176,176}
\definecolor{lightgray204}{RGB}{204,204,204}

\begin{axis}[
colorbar,
colorbar style={ylabel={edge id}},
colormap={mymap}{[1pt]
 rgb(0pt)=(0.12156862745098,0.466666666666667,0.705882352941177);
  rgb(1pt)=(1,0.498039215686275,0.0549019607843137);
  rgb(2pt)=(0.172549019607843,0.627450980392157,0.172549019607843);
  rgb(3pt)=(0.83921568627451,0.152941176470588,0.156862745098039);
  rgb(4pt)=(0.580392156862745,0.403921568627451,0.741176470588235);
  rgb(5pt)=(0.549019607843137,0.337254901960784,0.294117647058824);
  rgb(6pt)=(0.890196078431372,0.466666666666667,0.76078431372549);
  rgb(7pt)=(0.498039215686275,0.498039215686275,0.498039215686275);
  rgb(8pt)=(0.737254901960784,0.741176470588235,0.133333333333333);
  rgb(9pt)=(0.0901960784313725,0.745098039215686,0.811764705882353)
},
legend cell align={left},
legend style={
  fill opacity=0.8,
  draw opacity=1,
  text opacity=1,
  at={(0.03,0.97)},
  anchor=north west,
  draw=lightgray204
},
point meta max=7,
point meta min=0,
tick align=outside,
tick pos=left,
title={ye test-ye pred},
x grid style={darkgray176},
xlabel={ye test},
xmajorgrids,
xmin=-15.25776872815, xmax=51.84084685915,
xtick style={color=black},
y grid style={darkgray176},
ylabel={ye pred},
ymajorgrids,
ymin=-1.40962006486952, ymax=0.158991481736302,
ytick style={color=black}
]
\addplot [
  colormap={mymap}{[1pt]
 rgb(0pt)=(0.12156862745098,0.466666666666667,0.705882352941177);
  rgb(1pt)=(1,0.498039215686275,0.0549019607843137);
  rgb(2pt)=(0.172549019607843,0.627450980392157,0.172549019607843);
  rgb(3pt)=(0.83921568627451,0.152941176470588,0.156862745098039);
  rgb(4pt)=(0.580392156862745,0.403921568627451,0.741176470588235);
  rgb(5pt)=(0.549019607843137,0.337254901960784,0.294117647058824);
  rgb(6pt)=(0.890196078431372,0.466666666666667,0.76078431372549);
  rgb(7pt)=(0.498039215686275,0.498039215686275,0.498039215686275);
  rgb(8pt)=(0.737254901960784,0.741176470588235,0.133333333333333);
  rgb(9pt)=(0.0901960784313725,0.745098039215686,0.811764705882353)
},
  only marks,
  scatter,
  scatter src=explicit
]
table [x=x, y=y, meta=colordata]{%
x  y  colordata
39.565898635 -0.0238991156220436 0.0
18.050517226 0.00332538038492203 1.0
21.515381409 0.0269015673547983 2.0
-8.8987516789 -0.0485587492585182 3.0
12.61662972 -0.0721574351191521 4.0
26.949268915 -0.0397541746497154 5.0
39.565898645 -0.42048591375351 6.0
39.565898645 -0.334282040596008 7.0
42.743257171 0.0340243801474571 0.0
22.448801828 -0.0953749641776085 1.0
20.294455342 -0.0582817047834396 2.0
-4.2455956832 -0.0391556844115257 3.0
16.048859649 -0.0245276317000389 4.0
26.694397521 -0.0186027437448502 5.0
42.743257181 -0.170954197645187 6.0
42.743257181 -0.773417770862579 7.0
39.367180747 0.0360150933265686 0.0
18.970258592 -0.0253578647971153 1.0
20.39692216 0.0203885566443205 2.0
-4.9776642323 -0.0372627377510071 3.0
15.419257923 -0.0234607234597206 4.0
23.947922829 -0.0444648340344429 5.0
39.367180751 -0.241028785705566 6.0
39.367180751 -0.147693634033203 7.0
39.605393097 0.0344657190144062 0.0
22.082745928 -0.0181354433298111 1.0
17.522647169 0.0116253737360239 2.0
-7.2517903354 -0.0411849990487099 3.0
10.270856823 -0.0432266667485237 4.0
29.334536274 -0.0538892447948456 5.0
39.605393107 -0.0306751094758511 6.0
39.605393107 -0.251378536224365 7.0
43.937345526 0.0355739369988441 0.0
21.865843593 0.0368318557739258 1.0
22.071501933 0.0329278409481049 2.0
-8.3469765086 -0.061231441795826 3.0
13.724525414 -0.054615430533886 4.0
30.212820112 -0.0361906513571739 5.0
43.937345536 -0.151454925537109 6.0
43.937345536 -0.0169578157365322 7.0
31.061989584 0.0356883145868778 0.0
16.203677385 0.0229278095066547 1.0
14.858312199 0.0318453684449196 2.0
-5.5704045507 -0.041450634598732 3.0
9.2879076384 -0.0615482553839684 4.0
21.774081945 0.00976173765957355 5.0
31.061989594 -0.270361423492432 6.0
31.061989594 0.00414074957370758 7.0
36.357435265 0.0547002032399178 0.0
19.38173709 -0.00621782243251801 1.0
16.975698176 0.0377776399254799 2.0
-8.3424496148 -0.0546986609697342 3.0
8.6332485532 -0.0194174610078335 4.0
27.724186714 -0.0608667284250259 5.0
36.357435273 -0.621378183364868 6.0
36.357435273 -0.142314225435257 7.0
38.444969607 0.0304392781108618 0.0
21.080570389 0.0141959674656391 1.0
17.364399218 0.00212457031011581 2.0
-5.0980379196 -0.0473319441080093 3.0
12.266361289 -0.0302928574383259 4.0
26.178608318 -0.0221704095602036 5.0
38.444969617 -0.288609355688095 6.0
38.444969617 -0.154539257287979 7.0
35.498620518 0.0215395335108042 0.0
17.900031823 0.0399558059871197 1.0
17.598588698 0.0283940378576517 2.0
-4.651704216 -0.0650109425187111 3.0
12.946884476 -0.0341073349118233 4.0
22.551736046 -0.0676475837826729 5.0
35.498620524 -0.269683480262756 6.0
35.498620524 -0.140443444252014 7.0
36.52099827 0.0419573336839676 0.0
18.961240079 -0.0196424424648285 1.0
17.559758192 0.0126046314835548 2.0
-5.3314768445 -0.0617672130465508 3.0
12.228281339 -0.0544056668877602 4.0
24.292716932 -0.0598031431436539 5.0
36.520998279 -0.244558990001678 6.0
36.520998279 -0.6060431599617 7.0
36.717272204 0.00431920029222965 0.0
19.763035348 0.0409704595804214 1.0
16.954236857 0.034036323428154 2.0
-7.0689091682 -0.0677077025175095 3.0
9.88532768 -0.0272296369075775 4.0
26.831944525 -0.041297435760498 5.0
36.717272212 -0.42214572429657 6.0
36.717272212 -0.04222122579813 7.0
32.629628996 0.0794691741466522 0.0
17.103127104 -0.0410224422812462 1.0
15.526501892 0.012467885389924 2.0
-7.1999965109 -0.0501386746764183 3.0
8.3265053708 -0.033230260014534 4.0
24.303123625 -0.0260629020631313 5.0
32.629629006 -0.0315905064344406 6.0
32.629629006 -0.126410752534866 7.0
37.75267433 0.0264767445623875 0.0
17.533699782 0.0354500859975815 1.0
20.218974548 0.0358465760946274 2.0
-3.919745576 -0.0228904187679291 3.0
16.299228962 -0.0434866920113564 4.0
21.453445368 0.00354587845504284 5.0
37.75267434 -0.358603477478027 6.0
37.75267434 -0.324506253004074 7.0
38.800291337 0.0383501537144184 0.0
21.414385846 0.0287174209952354 1.0
17.385905491 0.00930956564843655 2.0
-6.3164463541 -0.0617266297340393 3.0
11.069459127 -0.0458036437630653 4.0
27.73083221 -0.063577763736248 5.0
38.800291347 -0.351832807064056 6.0
38.800291347 -0.0173816569149494 7.0
38.252729609 -0.00617945939302444 0.0
20.199036122 -0.000817075371742249 1.0
18.053693488 0.0361081771552563 2.0
-6.9495443437 -0.0512995943427086 3.0
11.104149136 -0.052324004471302 4.0
27.148580475 -0.0713173821568489 5.0
38.252729618 -0.576838135719299 6.0
38.252729618 -0.233010172843933 7.0
43.273596037 -0.0200019478797913 0.0
20.811072897 0.0146747399121523 1.0
22.46252314 0.0100301429629326 2.0
-10.185324283 -0.0394825115799904 3.0
12.277198847 -0.0695122256875038 4.0
30.99639719 -0.058222122490406 5.0
43.273596047 -0.728860974311829 6.0
43.273596047 0.0506768487393856 7.0
34.027431477 0.0347824022173882 0.0
16.463058834 0.0347460731863976 1.0
17.564372643 0.0387773662805557 2.0
-7.8238749469 -0.0324900224804878 3.0
9.7404976863 -0.0539126470685005 4.0
24.28693379 -0.066259890794754 5.0
34.027431486 -0.733403086662292 6.0
34.027431486 -0.00634946301579475 7.0
41.154171382 0.0553291961550713 0.0
20.969763 0.0331829562783241 1.0
20.184408383 0.0293899159878492 2.0
-9.8031418559 -0.0594751685857773 3.0
10.381266519 -0.0548145994544029 4.0
30.772904865 -0.0485276430845261 5.0
41.154171391 -0.523542881011963 6.0
41.154171391 -0.0371708646416664 7.0
39.9308264078757 -0.0141486488282681 0.0
21.2554356118761 0.0319434069097042 1.0
18.6753908028755 0.0388351231813431 2.0
-8.41978314337461 -0.0249633602797985 3.0
10.2556076588756 -0.0459069460630417 4.0
29.6752187558758 0.00984556600451469 5.0
39.930826408 -0.53826779127121 6.0
39.930826408 -0.0965717658400536 7.0
45.969703622 -0.0138402283191681 0.0
25.855804141 -0.0153910741209984 1.0
20.113899483 0.00139866769313812 2.0
-4.729541314 -0.00729710608720779 3.0
15.38435816 -0.0672263577580452 4.0
30.585345464 -0.0285279825329781 5.0
45.969703631 -0.245210707187653 6.0
45.969703631 -0.179838359355927 7.0
38.398120212 0.0226706713438034 0.0
19.546425793 0.0403745770454407 1.0
18.851694421 0.0271045435220003 2.0
-9.8922695857 -0.0650681331753731 3.0
8.9594248259 -0.0356181114912033 4.0
29.438695387 -0.0708798244595528 5.0
38.398120221 -0.299365818500519 6.0
38.398120221 -0.0305645689368248 7.0
28.366017296 0.0401455797255039 0.0
14.995271371 -0.101516045629978 1.0
13.370745924 -0.0544300675392151 2.0
-4.4280592982 -0.034656934440136 3.0
8.9426866162 -0.0504290089011192 4.0
19.42333068 -0.0152411460876465 5.0
28.366017306 0.02812460064888 6.0
28.366017306 -0.399636328220367 7.0
39.07981897 0.0464329868555069 0.0
18.202914308 -0.0364222601056099 1.0
20.876904662 -0.00399510562419891 2.0
-6.6281228283 -0.0422910451889038 3.0
14.248781824 -0.0535959973931313 4.0
24.831037146 -0.0670322775840759 5.0
39.079818979 -0.206229597330093 6.0
39.079818979 -0.26226007938385 7.0
40.466329374 0.0461938753724098 0.0
19.842708002 -0.0742344632744789 1.0
20.623621372 -0.0210264325141907 2.0
-6.0436252005 -0.0666363537311554 3.0
14.579996162 -0.04105494171381 4.0
25.886333212 -0.0704445391893387 5.0
40.466329383 0.0507541261613369 6.0
40.466329383 -0.483960628509521 7.0
38.293116843 0.0148117952048779 0.0
18.225197444 -0.0517717376351357 1.0
20.067919399 0.00325938314199448 2.0
-4.6459628917 -0.0620784386992455 3.0
15.421956498 -0.0387416705489159 4.0
22.871160346 -0.0613138824701309 5.0
38.293116853 -0.286478459835052 6.0
38.293116853 -0.554533004760742 7.0
43.346089487 0.0716123059391975 0.0
21.23189324 -0.0148848071694374 1.0
22.114196248 0.0146931149065495 2.0
-9.387268107 -0.0598402172327042 3.0
12.726928131 -0.0621843561530113 4.0
30.619161357 -0.0463641807436943 5.0
43.346089497 -0.0803942307829857 6.0
43.346089497 -0.155504822731018 7.0
34.547871144 0.0287992395460606 0.0
19.900789633 -0.0223984159529209 1.0
14.647081512 0.0157856550067663 2.0
-6.6924528176 -0.0483174920082092 3.0
7.9546286844 -0.0443652868270874 4.0
26.59324246 -0.0872929692268372 5.0
34.547871154 -0.0768236443400383 6.0
34.547871154 -0.287187159061432 7.0
41.927050143208 0.032987579703331 0.0
21.8340017142739 -0.0269362181425095 1.0
20.0930484299516 0.00886699743568897 2.0
-12.0156472013624 -0.0665066912770271 3.0
8.07740122783537 -0.049383781850338 4.0
33.8496486756856 -0.0712066739797592 5.0
41.927050143 -0.15806370973587 6.0
41.927050143 -0.287337601184845 7.0
44.548236367 0.0512427948415279 0.0
22.097953544 -0.00669199973344803 1.0
22.450282823 0.0272305980324745 2.0
-6.9116853021 -0.0537007823586464 3.0
15.538597511 -0.0425683706998825 4.0
29.009638856 -0.0309992730617523 5.0
44.548236377 -0.232228100299835 6.0
44.548236377 -0.139549285173416 7.0
34.958415477 0.0399429239332676 0.0
19.046265323 0.0182391330599785 1.0
15.912150156 0.0383634567260742 2.0
-3.947719982 -0.0633181855082512 3.0
11.964430164 -0.00920833647251129 4.0
22.993985314 -0.0619172304868698 5.0
34.958415487 -0.00924436375498772 6.0
34.958415487 -0.164710015058517 7.0
41.619773288 -0.138545364141464 0.0
18.036088777 0.0360635966062546 1.0
23.583684511 0.0388802662491798 2.0
-9.1140217471 -0.032227136194706 3.0
14.469662754 -0.0778008550405502 4.0
27.150110534 -0.0535897985100746 5.0
41.619773298 -0.936476469039917 6.0
41.619773298 -0.255487501621246 7.0
35.768623904 0.0279814470559359 0.0
18.854929214 0.0323601812124252 1.0
16.913694692 0.0359060615301132 2.0
-6.8203969536 -0.0517507568001747 3.0
10.093297731 -0.0527213886380196 4.0
25.675326176 -0.0694024488329887 5.0
35.768623912 0.00232088938355446 6.0
35.768623912 -0.335041582584381 7.0
36.035873071 -0.232939153909683 0.0
18.197202866 0.0142743159085512 1.0
17.838670207 0.0484629273414612 2.0
-9.8802007315 -0.0629201531410217 3.0
7.958469467 -0.0488350093364716 4.0
28.077403606 -0.0701096653938293 5.0
36.03587308 -0.844761192798615 6.0
36.03587308 -0.0220903716981411 7.0
42.671159575 0.0051116906106472 0.0
22.542382477 0.0326215252280235 1.0
20.128777099 0.0386503525078297 2.0
-9.3171954196 -0.0460382029414177 3.0
10.811581671 -0.0596296116709709 4.0
31.859577906 -0.0472759455442429 5.0
42.671159584 -0.333389639854431 6.0
42.671159584 -0.0171913430094719 7.0
32.270518381 0.0193227548152208 0.0
16.341237488 0.00422832369804382 1.0
15.929280893 0.0223727636039257 2.0
-7.2560391874 -0.0350250527262688 3.0
8.6732416955 -0.0591532811522484 4.0
23.597276685 -0.00988563895225525 5.0
32.270518391 -0.309051752090454 6.0
32.270518391 -0.139461427927017 7.0
36.239834932 0.0338904671370983 0.0
18.511196273 -0.0442107543349266 1.0
17.72863866 -0.00169293582439423 2.0
-5.3875241311 -0.059638075530529 3.0
12.34111452 -0.0290756747126579 4.0
23.898720413 -0.0421582609415054 5.0
36.239834941 -0.763559699058533 6.0
36.239834941 -0.494311571121216 7.0
38.292005172 0.0434024259448051 0.0
20.144840979 -0.0809163600206375 1.0
18.147164195 -0.0238400064408779 2.0
-6.9715526507 -0.0420059189200401 3.0
11.175611535 -0.031975083053112 4.0
27.116393638 -0.0347339883446693 5.0
38.292005181 0.0405075326561928 6.0
38.292005181 -0.388626992702484 7.0
41.142171116 0.0353741459548473 0.0
21.697070846 -0.0290859192609787 1.0
19.44510027 0.0143936220556498 2.0
-5.49061917 -0.0558283254504204 3.0
13.95448109 -0.0330283418297768 4.0
27.187690026 -0.0701957941055298 5.0
41.142171126 -0.0038633719086647 6.0
41.142171126 -0.372062981128693 7.0
30.660234741 0.0223000310361385 0.0
16.571808951 -0.0567008927464485 1.0
14.08842579 -0.0241386294364929 2.0
-3.6405774968 -0.0499443560838699 3.0
10.447848283 -0.0327274724841118 4.0
20.212386458 -0.0666318908333778 5.0
30.660234751 0.0543095506727695 6.0
30.660234751 -0.463568389415741 7.0
42.776716976 0.0175703726708889 0.0
18.943122536 -0.125441581010818 1.0
23.83359444 -0.077523298561573 2.0
-9.5992659914 -0.0577243566513062 3.0
14.234328438 -0.044372133910656 4.0
28.542388537 -0.052537240087986 5.0
42.776716986 -0.570354342460632 6.0
42.776716986 -0.392911314964294 7.0
39.136657948 -0.0202911570668221 0.0
21.244730642 -0.0534873232245445 1.0
17.891927309 -0.0134253427386284 2.0
-5.2069844714 -0.0550957843661308 3.0
12.684942831 -0.0204760208725929 4.0
26.45171512 -0.0477314963936806 5.0
39.136657955 -1.32871127128601 6.0
39.136657955 -0.599502384662628 7.0
40.593075656 -0.0274072922766209 0.0
19.181149666 -0.0237252861261368 1.0
21.411925992 0.00863122567534447 2.0
-11.788434371 -0.0322383716702461 3.0
9.6234916129 -0.0639748200774193 4.0
30.969584045 -0.0394314751029015 5.0
40.593075664 -0.547432661056519 6.0
40.593075664 -0.259924173355103 7.0
38.455960047 0.00858565419912338 0.0
19.606731652 -0.0326585993170738 1.0
18.849228395 0.00999693758785725 2.0
-10.066988594 -0.0514194741845131 3.0
8.7822397903 -0.0245222672820091 4.0
29.673720256 -0.0646765679121017 5.0
38.455960057 -0.495759248733521 6.0
38.455960057 -0.289222002029419 7.0
40.029822299 0.0651206150650978 0.0
19.696760735 -0.0400950536131859 1.0
20.333061566 -0.0222943127155304 2.0
-6.9313004298 -0.0437954366207123 3.0
13.401761128 -0.0619145259261131 4.0
26.628061173 -0.0371271371841431 5.0
40.029822307 -0.103305019438267 6.0
40.029822307 -0.175019383430481 7.0
39.721089796 0.0131113287061453 0.0
21.982867862 -0.165263533592224 1.0
17.738221933 -0.138669610023499 2.0
-6.2167396454 -0.0480737760663033 3.0
11.521482278 -0.0391106382012367 4.0
28.199607518 -0.0365715548396111 5.0
39.721089806 -0.138835966587067 6.0
39.721089806 -1.18525063991547 7.0
46.012802771 0.0329693257808685 0.0
23.852238991 0.0412481799721718 1.0
22.16056378 0.0254914630204439 2.0
-9.7843125098 -0.0656246915459633 3.0
12.37625126 -0.0431941226124763 4.0
33.636551511 -0.0676311478018761 5.0
46.012802781 -0.0221488513052464 6.0
46.012802781 -0.0574330762028694 7.0
43.79104115 0.0639194175601006 0.0
23.810489927 -0.0280871018767357 1.0
19.980551224 0.0316103249788284 2.0
-8.8972279263 -0.0646650418639183 3.0
11.083323289 -0.00351157411932945 4.0
32.707717862 -0.0643198564648628 5.0
43.791041158 -0.0868306756019592 6.0
43.791041158 -0.100779719650745 7.0
31.257332414 0.00402561761438847 0.0
14.677686479 0.00623594783246517 1.0
16.579645934 0.0450124219059944 2.0
-7.7653359569 -0.0553272143006325 3.0
8.8143099674 -0.0197397097945213 4.0
22.443022446 -0.0488361045718193 5.0
31.257332424 -0.417135059833527 6.0
31.257332424 -0.191280484199524 7.0
38.988472902 0.0394459962844849 0.0
18.706030748 0.0329443104565144 1.0
20.282442155 0.0515330843627453 2.0
-4.4931154738 -0.0514202192425728 3.0
15.789326673 -0.0370220839977264 4.0
23.19914623 -0.00783829018473625 5.0
38.98847291 -0.129945516586304 6.0
38.98847291 -0.0596777275204659 7.0
38.691218489 0.0282923709601164 0.0
19.181746937 -0.27314019203186 1.0
19.509471553 -0.163146942853928 2.0
-4.305791265 -0.0366699993610382 3.0
15.203680278 -0.0648800358176231 4.0
23.487538212 -0.0198311433196068 5.0
38.691218499 -0.322670996189117 6.0
38.691218499 -0.784638404846191 7.0
39.033211962 -0.0705403909087181 0.0
21.751284426 0.0229915659874678 1.0
17.281927536 0.0480205826461315 2.0
-7.1228573229 -0.0580105036497116 3.0
10.159070204 -0.0154411159455776 4.0
28.874141758 -0.0149788185954094 5.0
39.033211971 -0.709357261657715 6.0
39.033211971 -0.192971080541611 7.0
37.697547803 0.0364886075258255 0.0
18.641388049 -0.020605206489563 1.0
19.056159754 0.00984616205096245 2.0
-6.2754402376 -0.0441387221217155 3.0
12.780719506 -0.0521333143115044 4.0
24.916828297 -0.0140844583511353 5.0
37.697547813 -0.22128164768219 6.0
37.697547813 -0.242779403924942 7.0
35.277541329 -0.0261854231357574 0.0
16.862419132 -0.0295973867177963 1.0
18.415122198 0.0137681942433119 2.0
-6.2131742864 -0.0603829249739647 3.0
12.201947903 -0.0361359342932701 4.0
23.075593428 -0.0552213415503502 5.0
35.277541339 -0.179789304733276 6.0
35.277541339 -0.157086849212646 7.0
36.1197639664031 0.0279346238821745 0.0
19.1517629624013 0.0335115045309067 1.0
16.9680010143999 0.0154028721153736 2.0
-0.473187581125011 -0.0710391029715538 3.0
16.4948134334014 0.00451654382050037 4.0
19.6249505434022 -0.0302210003137589 5.0
36.119763966 -0.0635288134217262 6.0
36.119763966 0.0154529400169849 7.0
33.14490304 -0.0131397992372513 0.0
16.831426445 0.0318491831421852 1.0
16.313476595 0.00904393568634987 2.0
-3.637680404 -0.0606527253985405 3.0
12.675796181 0.0107801109552383 4.0
20.469106859 -0.0468542352318764 5.0
33.14490305 -1.03884732723236 6.0
33.14490305 -0.00300391763448715 7.0
34.4868008142505 -0.0520719513297081 0.0
18.3228058422505 -0.0463497638702393 1.0
16.1639949822505 -0.00807671248912811 2.0
-8.00476657755046 -0.0431055724620819 3.0
8.15922840445046 -0.0548778101801872 4.0
26.3275723771958 -0.0718900933861732 5.0
34.486800814 -0.626168131828308 6.0
34.486800814 -0.242897599935532 7.0
40.933468199 0.0366358272731304 0.0
20.889274427 0.037084337323904 1.0
20.044193774 0.0387864969670773 2.0
-6.533500242 -0.0304347649216652 3.0
13.510693524 -0.0612362548708916 4.0
27.422774677 -0.0125553049147129 5.0
40.933468207 -0.26828932762146 6.0
40.933468207 -0.332921862602234 7.0
35.993417919 0.0453076995909214 0.0
18.964188614 0.0372690111398697 1.0
17.029229306 0.0361853465437889 2.0
-7.4575389686 -0.0384620353579521 3.0
9.5716903284 -0.0793564394116402 4.0
26.421727592 -0.0355197265744209 5.0
35.993417928 -0.176494687795639 6.0
35.993417928 -0.135837882757187 7.0
39.331666914 0.0111446287482977 0.0
19.408314017 0.0399956963956356 1.0
19.923352898 0.0272528287023306 2.0
-7.2428639103 -0.056524433195591 3.0
12.680488979 -0.0275959223508835 4.0
26.651177937 -0.0646567046642303 5.0
39.331666923 -0.443323910236359 6.0
39.331666923 -0.00965490564703941 7.0
37.559548486 0.0345160402357578 0.0
18.054642609 0.0354160331189632 1.0
19.504905877 0.0347389988601208 2.0
-8.2806473226 -0.0443039163947105 3.0
11.224258545 -0.0408843457698822 4.0
26.335289942 -0.0401677116751671 5.0
37.559548496 -0.215734392404556 6.0
37.559548496 -0.247938811779022 7.0
41.796902472 0.0122576057910919 0.0
20.074580702 0.0351572036743164 1.0
21.72232177 0.0342653468251228 2.0
-11.24258785 -0.0334513112902641 3.0
10.479733911 -0.0222481265664101 4.0
31.317168562 -0.0145808830857277 5.0
41.796902482 -0.155443251132965 6.0
41.796902482 -0.0846418961882591 7.0
35.679590813 0.0640139430761337 0.0
15.932888769 0.0373637899756432 1.0
19.746702044 0.0357978492975235 2.0
-7.4161334139 -0.0456439331173897 3.0
12.330568621 -0.0330755710601807 4.0
23.349022193 -0.0416416749358177 5.0
35.679590823 -0.192131847143173 6.0
35.679590823 -0.0218639150261879 7.0
33.227547284 0.00139320269227028 0.0
19.156810973 -0.0285166800022125 1.0
14.070736312 0.0124198216944933 2.0
-5.2205843651 -0.0498002916574478 3.0
8.8501519384 -0.0553595647215843 4.0
24.377395346 -0.0817312672734261 5.0
33.227547292 -0.368703663349152 6.0
33.227547292 -0.301347255706787 7.0
28.00807173 0.0381058678030968 0.0
12.909757219 0.0410837680101395 1.0
15.098314511 0.0356073081493378 2.0
-4.3989930962 -0.044526219367981 3.0
10.699321405 -0.0625967159867287 4.0
17.308750325 -0.0420708283782005 5.0
28.008071739 -0.25656670331955 6.0
28.008071739 -0.162907779216766 7.0
33.4784988410636 -0.0441521182656288 0.0
17.4596045520637 -0.00754720717668533 1.0
16.0188942990637 0.0333418287336826 2.0
-6.65519022176347 -0.0675163194537163 3.0
9.36370407706344 -0.0206368714570999 4.0
24.1147947730637 -0.0495829060673714 5.0
33.478498841 -0.895894527435303 6.0
33.478498841 -0.471364438533783 7.0
33.1229468201162 0.0710399374365807 0.0
16.2738036511163 0.0327730253338814 1.0
16.8491431771162 0.0217449627816677 2.0
-8.48238354211593 -0.0539432242512703 3.0
8.36675963501607 -0.0575437620282173 4.0
24.7561871931169 -0.0711738988757133 5.0
33.12294682 -0.120937280356884 6.0
33.12294682 0.0114253722131252 7.0
34.970228375 0.018136627972126 0.0
17.461716438 0.0308264326304197 1.0
17.508511937 0.0250676814466715 2.0
-3.5027478521 -0.0558784753084183 3.0
14.005764076 -0.056160993874073 4.0
20.964464299 -0.0539959222078323 5.0
34.970228384 -0.468731880187988 6.0
34.970228384 -0.0211530700325966 7.0
36.637966033 0.0167153552174568 0.0
18.611959282 0.0116537492722273 1.0
18.026006753 -0.00557003915309906 2.0
-5.1789935919 -0.0689981654286385 3.0
12.847013153 -0.0214018300175667 4.0
23.790952882 -0.061345599591732 5.0
36.637966041 -0.0517489984631538 6.0
36.637966041 -0.0110417976975441 7.0
38.712447285 0.0376491472125053 0.0
19.874849154 0.0281737297773361 1.0
18.837598131 0.0504157692193985 2.0
-9.656315418 -0.0468448325991631 3.0
9.181282703 -0.0646747350692749 4.0
29.531164582 -0.0599655583500862 5.0
38.712447295 -0.029995784163475 6.0
38.712447295 -0.0242008157074451 7.0
29.0840224199388 0.0399674996733665 0.0
12.9348822379388 0.0406553186476231 1.0
16.1491401919388 0.034764438867569 2.0
-6.2158408037388 -0.0530892610549927 3.0
9.93329938853882 -0.0592315867543221 4.0
19.1507230409388 -0.0338583663105965 5.0
29.08402242 -0.146131277084351 6.0
29.08402242 -0.376803398132324 7.0
40.7419030105752 0.0442623123526573 0.0
19.6019574718567 0.0262646414339542 1.0
21.139945535663 0.0419323742389679 2.0
-8.29196876971742 -0.0496648699045181 3.0
12.8479767652648 -0.0245125368237495 4.0
27.8939262422712 -0.019054226577282 5.0
40.741903011 -0.381082236766815 6.0
40.741903011 -0.132523119449615 7.0
44.4239279780974 0.0627191960811615 0.0
22.6860882760974 -0.0137287713587284 1.0
21.7378397120974 0.00860906392335892 2.0
-6.03315836089742 -0.0455464199185371 3.0
15.7046813510974 -0.0710050314664841 4.0
28.7192466370973 -0.0329519808292389 5.0
44.423927978 -0.0558008775115013 6.0
44.423927978 -0.18337282538414 7.0
37.279929002 -0.0381205305457115 0.0
17.697440733 -0.0997722968459129 1.0
19.582488269 -0.0387893989682198 2.0
-8.4860638308 -0.0223396494984627 3.0
11.096424428 -0.0882725194096565 4.0
26.183504574 -0.047936387360096 5.0
37.279929011 -1.04672563076019 6.0
37.279929011 -0.619226396083832 7.0
39.065196556 0.00409606471657753 0.0
22.911627527 -0.0717107132077217 1.0
16.153569029 -0.0342000424861908 2.0
-7.035333012 -0.0578815042972565 3.0
9.1182360067 -0.0482169315218925 4.0
29.946960549 -0.0777714028954506 5.0
39.065196566 -0.913275897502899 6.0
39.065196566 -0.519713640213013 7.0
34.346712901 0.0377301126718521 0.0
17.419062718 -0.00908833742141724 1.0
16.927650183 0.0309263579547405 2.0
-3.8964434363 -0.0584635138511658 3.0
13.031206737 -0.0313015952706337 4.0
21.315506165 -0.0692521333694458 5.0
34.346712911 -0.31612890958786 6.0
34.346712911 -0.18205401301384 7.0
44.832836192 0.0348458811640739 0.0
22.86238329 -0.0300396047532558 1.0
21.970452902 0.0144618190824986 2.0
-8.0950157692 -0.070419929921627 3.0
13.875437123 -0.04741320759058 4.0
30.957399069 -0.0788522139191628 5.0
44.832836202 -0.0164255499839783 6.0
44.832836202 -0.309700101613998 7.0
37.693005907 -0.000630728900432587 0.0
18.767233493 0.0276548340916634 1.0
18.925772415 0.0195597782731056 2.0
-4.4039460583 -0.0417528823018074 3.0
14.521826348 -0.0559710711240768 4.0
23.17117956 -0.0616019368171692 5.0
37.693005916 -0.536688327789307 6.0
37.693005916 0.0494093410670757 7.0
35.516030196 0.0387588292360306 0.0
17.238388415 0.0347430966794491 1.0
18.277641781 0.0369351953268051 2.0
-5.9879556024 -0.0502544865012169 3.0
12.289686168 -0.0480404272675514 4.0
23.226344027 -0.0586200430989265 5.0
35.516030206 -0.128453940153122 6.0
35.516030206 -0.0757664665579796 7.0
45.449957515 0.0350357368588448 0.0
24.470260824 0.00532802194356918 1.0
20.979696692 0.0201884545385838 2.0
-8.6547786371 -0.0390110015869141 3.0
12.324918046 -0.0645138397812843 4.0
33.12503947 -0.0409869030117989 5.0
45.449957523 0.0224588811397552 6.0
45.449957523 -0.633076786994934 7.0
36.522103961 0.0323941297829151 0.0
17.787809101 0.00512088648974895 1.0
18.734294862 0.000246655195951462 2.0
-3.7955939001 -0.048587940633297 3.0
14.938700953 -0.05624158680439 4.0
21.583403009 -0.0402565971016884 5.0
36.52210397 -0.262384176254272 6.0
36.52210397 -0.0182113498449326 7.0
40.809333824 0.0277577061206102 0.0
21.825363738 -0.0246926695108414 1.0
18.983970087 0.0125239528715611 2.0
-4.5096782574 -0.0734050869941711 3.0
14.474291821 -0.0362526327371597 4.0
26.335042004 -0.0470868349075317 5.0
40.809333833 -0.0758650228381157 6.0
40.809333833 -0.192981213331223 7.0
43.708815015 0.0164703950285912 0.0
22.06122806 0.038729690015316 1.0
21.647586954 0.0262931641191244 2.0
-7.2998869135 -0.0493279471993446 3.0
14.347700031 -0.025335781276226 4.0
29.361114984 -0.0413842424750328 5.0
43.708815025 -0.502294063568115 6.0
43.708815025 -0.155581265687943 7.0
35.746307566 0.0375050008296967 0.0
17.418825391 0.0152151249349117 1.0
18.327482175 0.0169219262897968 2.0
-8.2684434585 -0.0506240651011467 3.0
10.059038708 -0.0551030561327934 4.0
25.687268859 -0.03457310795784 5.0
35.746307575 -0.0555976554751396 6.0
35.746307575 0.036412950605154 7.0
32.565260397 0.0319007411599159 0.0
15.347203209 0.0250938944518566 1.0
17.218057188 0.0538947954773903 2.0
-4.4968959364 -0.0593015179038048 3.0
12.721161242 -0.0224023461341858 4.0
19.844099155 -0.0757531598210335 5.0
32.565260407 -0.0629311800003052 6.0
32.565260407 -0.0197123140096664 7.0
37.459437559 -0.0062972754240036 0.0
20.974106733 -0.191712468862534 1.0
16.485330826 -0.116372235119343 2.0
-7.8025517563 -0.0325343683362007 3.0
8.6827790601 -0.0412162765860558 4.0
28.776658499 -0.054027758538723 5.0
37.459437569 -0.615337252616882 6.0
37.459437569 -0.940489828586578 7.0
41.6008687772188 -0.0167550928890705 0.0
18.8099504026802 0.0134183894842863 1.0
22.7909183777866 0.00412518903613091 2.0
-6.50941037372392 -0.0577458962798119 3.0
16.2815080036091 -0.0281253531575203 4.0
25.319360776445 -0.0677623599767685 5.0
41.600868777 -0.617533922195435 6.0
41.600868777 0.0351773351430893 7.0
43.08864828 0.0627967566251755 0.0
20.75366021 -0.0429269969463348 1.0
22.334988069 0.00250869989395142 2.0
-9.5492158301 -0.0432991310954094 3.0
12.785772229 -0.0724249556660652 4.0
30.30287605 -0.0585657954216003 5.0
43.088648289 -0.644925951957703 6.0
43.088648289 -0.14562514424324 7.0
30.268152668 -0.0264683961868286 0.0
15.014233699 0.039981372654438 1.0
15.25391897 0.0229527186602354 2.0
-5.5931438946 -0.0530721768736839 3.0
9.6607750663 -0.0370450019836426 4.0
20.607377602 -0.0765727087855339 5.0
30.268152677 -0.417271852493286 6.0
30.268152677 -0.0812019407749176 7.0
38.454045879 -0.00276054441928864 0.0
17.159579151 0.0440130606293678 1.0
21.29446673 0.0315565280616283 2.0
-7.5807931052 -0.0646899566054344 3.0
13.713673616 -0.0485426634550095 4.0
24.740372265 -0.0893486514687538 5.0
38.454045888 -0.463150233030319 6.0
38.454045888 0.0463655069470406 7.0
36.654056854 0.0506092309951782 0.0
19.439010257 -0.0313996896147728 1.0
17.215046598 0.013226205483079 2.0
-7.9864590963 -0.050511822104454 3.0
9.2285874923 -0.0475350469350815 4.0
27.425469363 -0.0699693784117699 5.0
36.654056864 -0.120737485587597 6.0
36.654056864 -0.169730216264725 7.0
36.990278119 0.00115010887384415 0.0
18.644416453 -0.111785314977169 1.0
18.345861667 -0.0550543144345284 2.0
-7.9977544937 -0.0605436712503433 3.0
10.348107164 -0.0279006287455559 4.0
26.642170956 -0.0715448930859566 5.0
36.990278128 -0.276595175266266 6.0
36.990278128 -0.387357354164124 7.0
39.45539178 0.0427047647535801 0.0
17.953554537 -0.217736184597015 1.0
21.501837243 -0.196428596973419 2.0
-6.7799927019 -0.051310732960701 3.0
14.721844531 -0.0502694919705391 4.0
24.733547249 -0.0550230368971825 5.0
39.45539179 -0.0220872312784195 6.0
39.45539179 -0.763410806655884 7.0
41.490098301 0.0348618924617767 0.0
18.418139969 0.00906257890164852 1.0
23.071958334 0.00360444560647011 2.0
-6.7020850264 -0.0487271174788475 3.0
16.369873299 -0.020643338561058 4.0
25.120225004 -0.0532195568084717 5.0
41.49009831 -0.114421494305134 6.0
41.49009831 -0.0179723501205444 7.0
42.195848756 0.038566492497921 0.0
18.431626231 0.0212897937744856 1.0
23.764222526 0.0202868580818176 2.0
-9.6629820221 -0.0197727419435978 3.0
14.101240496 -0.0731519684195518 4.0
28.094608262 -0.045793853700161 5.0
42.195848764 -0.420380890369415 6.0
42.195848764 -0.280926465988159 7.0
36.580423947 -0.00102432072162628 0.0
18.088597645 0.0174756944179535 1.0
18.491826302 0.0497502163052559 2.0
-10.331066357 -0.0507007837295532 3.0
8.1607599346 -0.0593402534723282 4.0
28.419664012 -0.0867006853222847 5.0
36.580423956 -0.348976194858551 6.0
36.580423956 -0.0347517505288124 7.0
41.112612837 -0.0725912675261497 0.0
21.264748673 0.041356198489666 1.0
19.847864165 0.0336714014410973 2.0
-7.8539892402 -0.0500327795743942 3.0
11.993874916 -0.0619980022311211 4.0
29.118737922 -0.0492268949747086 5.0
41.112612846 -1.09833657741547 6.0
41.112612846 -0.367948412895203 7.0
41.165032295 0.0498508289456367 0.0
20.777866565 -0.0979891791939735 1.0
20.387165729 -0.0602460205554962 2.0
-11.013080697 -0.0403402224183083 3.0
9.3740850227 -0.0436243638396263 4.0
31.790947272 -0.0476342216134071 5.0
41.165032305 -0.198389381170273 6.0
41.165032305 -0.268575429916382 7.0
28.23632976 0.0149770248681307 0.0
14.450875753 -0.0225074663758278 1.0
13.785454008 0.013267993927002 2.0
-5.4067157656 -0.0570330023765564 3.0
8.3787382334 -0.0344923213124275 4.0
19.857591527 -0.0398774221539497 5.0
28.236329769 -0.243818014860153 6.0
28.236329769 -0.236115157604218 7.0
46.563722225 0.0338328406214714 0.0
24.514076121 0.0233232136815786 1.0
22.049646104 0.00661056116223335 2.0
-6.9458736065 -0.0447153449058533 3.0
15.103772488 -0.04304338991642 4.0
31.459949737 -0.0567678883671761 5.0
46.563722235 -0.027134969830513 6.0
46.563722235 -0.00824150070548058 7.0
38.976039438 0.0214715376496315 0.0
20.47408062 -0.158564776182175 1.0
18.50195882 -0.11782019585371 2.0
-3.6791774969 -0.0415114089846611 3.0
14.822781314 -0.0465641096234322 4.0
24.153258125 -0.048433355987072 5.0
38.976039446 0.0012460183352232 6.0
38.976039446 -0.430420458316803 7.0
43.902220155 0.0676698908209801 0.0
23.085534006 0.00710297003388405 1.0
20.816686149 0.0376995541155338 2.0
-6.0778252495 -0.0463246926665306 3.0
14.738860889 -0.046246349811554 4.0
29.163359266 -0.0680519044399261 5.0
43.902220165 -0.0848078653216362 6.0
43.902220165 -0.171886891126633 7.0
37.104870377 0.0139776133000851 0.0
21.928667997 0.0327411890029907 1.0
15.176202381 0.0303170476108789 2.0
-7.1829261037 -0.0392002314329147 3.0
7.9932762681 -0.0613669604063034 4.0
29.11159411 -0.0440937206149101 5.0
37.104870385 -0.5134437084198 6.0
37.104870385 0.00546028837561607 7.0
48.094555135036 0.0679186880588531 0.0
24.9788607525581 0.00527384877204895 1.0
23.1156943831877 0.0205556396394968 2.0
-7.58647576369851 -0.0224572457373142 3.0
15.5292186191899 -0.0762899816036224 4.0
32.5653365171039 -0.000253625214099884 5.0
48.094555135 -0.169182598590851 6.0
48.094555135 0.0382001474499702 7.0
39.863550950908 0.0358440242707729 0.0
20.2748443148297 0.0313523411750793 1.0
19.5887066378693 0.018401600420475 2.0
-9.95882058391343 -0.0518960505723953 3.0
9.62988605424657 -0.0585524663329124 4.0
30.233664897769 -0.0559301599860191 5.0
39.863550951 -0.0247589945793152 6.0
39.863550951 0.013158280402422 7.0
38.158424696 -0.00419416651129723 0.0
21.093616444 0.0364547669887543 1.0
17.064808252 0.0319932922720909 2.0
-6.6865614213 -0.0662460327148438 3.0
10.378246821 -0.0445160791277885 4.0
27.780177875 -0.0797174721956253 5.0
38.158424706 -0.385078966617584 6.0
38.158424706 0.0234691724181175 7.0
37.671552635 0.0460416860878468 0.0
18.060105754 0.0170712359249592 1.0
19.611446882 0.0432735458016396 2.0
-6.6868904195 -0.0464995130896568 3.0
12.924556453 -0.0449874624609947 4.0
24.746996183 -0.0696768537163734 5.0
37.671552644 -0.12917023897171 6.0
37.671552644 -0.160054981708527 7.0
38.9900068701748 0.00832345336675644 0.0
19.4317892501748 0.0109282750636339 1.0
19.5582176301748 0.0390925519168377 2.0
-6.7795985865748 -0.0648536011576653 3.0
12.7786190431748 -0.0517859682440758 4.0
26.2113877989032 -0.0798066109418869 5.0
38.99000687 -0.75980806350708 6.0
38.99000687 -0.376951187849045 7.0
32.824279187 0.071341298520565 0.0
17.524252685 0.0371708050370216 1.0
15.300026502 0.0342487543821335 2.0
-5.6991008156 -0.0587230771780014 3.0
9.6009256767 -0.0683742463588715 4.0
23.223353511 -0.0721625164151192 5.0
32.824279197 -0.110756449401379 6.0
32.824279197 -0.0485236272215843 7.0
39.752392388 0.0438123792409897 0.0
19.337645943 0.0244825892150402 1.0
20.414746445 0.0184211749583483 2.0
-4.1151033929 -0.0504275038838387 3.0
16.299643042 -0.065542183816433 4.0
23.452749346 -0.0382724478840828 5.0
39.752392398 -0.150708884000778 6.0
39.752392398 -0.00454882904887199 7.0
36.0149979881443 0.0345787778496742 0.0
19.5282530962186 0.0317067205905914 1.0
16.4867448931817 0.0146452412009239 2.0
-5.97886461861353 -0.0580781176686287 3.0
10.5078802751384 -0.0177062302827835 4.0
25.5071177152005 -0.0602900236845016 5.0
36.014997988 -0.240895986557007 6.0
36.014997988 0.00907246395945549 7.0
42.307205596 0.0130053423345089 0.0
21.688253633 0.028504865244031 1.0
20.618951963 0.0198783855885267 2.0
-4.4760898368 -0.0542707070708275 3.0
16.142862116 -0.0670055374503136 4.0
26.164343479 -0.0677030086517334 5.0
42.307205606 -0.210783630609512 6.0
42.307205606 0.0225387737154961 7.0
41.624369925 -0.163576036691666 0.0
21.269456313 -0.0052882581949234 1.0
20.354913614 0.0246965624392033 2.0
-9.726585677 -0.0475704446434975 3.0
10.628327928 -0.0681593418121338 4.0
30.996041999 -0.0771490707993507 5.0
41.624369934 -0.833120226860046 6.0
41.624369934 -0.0809341371059418 7.0
40.927860916 -0.149912416934967 0.0
20.558611306 0.0284023471176624 1.0
20.36924961 0.0263711251318455 2.0
-7.6689328495 -0.0564164742827415 3.0
12.700316751 -0.0479739755392075 4.0
28.227544165 -0.0149162709712982 5.0
40.927860926 -0.434472680091858 6.0
40.927860926 0.0301152542233467 7.0
47.441321803 0.0448919385671616 0.0
22.43048805 0.026034539565444 1.0
25.010833753 0.0143009889870882 2.0
-8.9760978683 -0.0580389276146889 3.0
16.034735875 -0.0460792407393456 4.0
31.406585928 -0.0519865080714226 5.0
47.441321812 -0.133049786090851 6.0
47.441321812 0.00296715088188648 7.0
39.37950561 0.0426879823207855 0.0
19.873167771 -0.0600776448845863 1.0
19.506337839 -0.0089796669781208 2.0
-9.8415530061 -0.0624041333794594 3.0
9.6647848227 -0.0547167584300041 4.0
29.714720787 -0.0726040229201317 5.0
39.379505619 0.0324022993445396 6.0
39.379505619 -0.357694298028946 7.0
42.095557289 0.0270795095711946 0.0
22.277305426 0.0405007340013981 1.0
19.818251863 0.040886215865612 2.0
-7.6874055589 -0.0371683314442635 3.0
12.130846294 -0.0350765362381935 4.0
29.964710995 -0.0535097196698189 5.0
42.095557299 -0.453540027141571 6.0
42.095557299 -0.537888884544373 7.0
35.151188218 0.0225248783826828 0.0
17.118344744 -0.0352707430720329 1.0
18.032843477 0.000107040628790855 2.0
-2.0060677241 -0.0445692762732506 3.0
16.026775747 -0.0389087349176407 4.0
19.124412475 -0.0736324340105057 5.0
35.151188225 -0.586565375328064 6.0
35.151188225 -0.357975572347641 7.0
37.713935362 0.0464322753250599 0.0
20.380527895 0.0409611351788044 1.0
17.333407469 0.0355185493826866 2.0
-5.6423114193 -0.0401970073580742 3.0
11.691096041 -0.0279815793037415 4.0
26.022839323 -0.0782347172498703 5.0
37.713935371 -0.0992685928940773 6.0
37.713935371 -0.4307981133461 7.0
39.667743251 0.0364027246832848 0.0
20.027544136 -0.0215432941913605 1.0
19.640199117 -0.00389999896287918 2.0
-10.376032212 -0.0598736628890038 3.0
9.2641668955 -0.0616107285022736 4.0
30.403576357 -0.0378010720014572 5.0
39.66774326 -0.042170450091362 6.0
39.66774326 -0.321756839752197 7.0
41.240556184 0.0317694917321205 0.0
23.347458936 0.00601423531770706 1.0
17.893097248 0.0356276035308838 2.0
-7.252142064 -0.0472866147756577 3.0
10.640955174 -0.0550991371273994 4.0
30.59960101 -0.0499329194426537 5.0
41.240556194 -0.0172491818666458 6.0
41.240556194 -0.0984876975417137 7.0
40.32019459 0.0373004376888275 0.0
19.177093862 0.0189635511487722 1.0
21.143100728 0.048512764275074 2.0
-7.2609061044 -0.0400918126106262 3.0
13.882194613 -0.0868244543671608 4.0
26.437999977 -0.0764166861772537 5.0
40.3201946 -0.135506719350815 6.0
40.3201946 -0.0161952748894691 7.0
39.5611510585585 0.0296467430889606 0.0
19.6094718095582 0.0373631455004215 1.0
19.9516792585585 0.0292134881019592 2.0
-8.44726993515885 -0.0601932182908058 3.0
11.5044093235587 -0.0175952836871147 4.0
28.0567416170478 -0.0750376582145691 5.0
39.561151059 0.0264403223991394 6.0
39.561151059 -0.155161052942276 7.0
43.5664319470049 0.0555294305086136 0.0
22.3670487340048 -0.203037619590759 1.0
21.1993832230048 -0.159879118204117 2.0
-8.61702915780503 -0.0577762797474861 3.0
12.582354065005 -0.0160318836569786 4.0
30.9840778920045 -0.055767610669136 5.0
43.566431947 -0.240632563829422 6.0
43.566431947 -0.455777943134308 7.0
37.929366556 0.0287531819194555 0.0
20.105495873 0.0344763174653053 1.0
17.823870687 0.0362092405557632 2.0
-5.9768423324 -0.0566141381859779 3.0
11.847028349 -0.0422639027237892 4.0
26.082338211 -0.0522521361708641 5.0
37.929366562 -0.24813774228096 6.0
37.929366562 -0.068991132080555 7.0
39.7456686413092 0.0348882302641869 0.0
20.6016083083092 -0.0861927047371864 1.0
19.1440603423092 -0.0337637886404991 2.0
-3.0966467648096 -0.064970038831234 3.0
16.0474135773093 -0.0291741713881493 4.0
23.6982550733092 -0.0634662210941315 5.0
39.745668641 -0.660558104515076 6.0
39.745668641 -0.604543566703796 7.0
37.1737946256367 -0.016888402402401 0.0
18.9810950589955 -0.0578820407390594 1.0
18.1926995622865 -0.0146949961781502 2.0
-4.83110587561615 -0.0599600151181221 3.0
13.3615936860386 -0.0285492464900017 4.0
23.8122009348139 -0.0788804516196251 5.0
37.173794625 -0.610031187534332 6.0
37.173794625 -0.561439514160156 7.0
27.124866007 0.0441213361918926 0.0
13.623671484 0.0181560423225164 1.0
13.501194525 0.0235500000417233 2.0
-4.5998142153 -0.0191696360707283 3.0
8.9013803006 -0.0618667900562286 4.0
18.223485707 0.00963034108281136 5.0
27.124866015 -0.366254866123199 6.0
27.124866015 -0.297509372234344 7.0
27.411746895 -0.22494301199913 0.0
13.763683287 0.0173032749444246 1.0
13.648063611 0.0504231974482536 2.0
-4.2527245718 -0.0504076480865479 3.0
9.3953390304 -0.0268751755356789 4.0
18.016407867 -0.0732931718230247 5.0
27.411746904 -0.44386824965477 6.0
27.411746904 0.00905986875295639 7.0
38.866617309 0.0346910506486893 0.0
22.319478804 0.0386263877153397 1.0
16.547138508 0.037807285785675 2.0
-7.9092899105 -0.0386690497398376 3.0
8.6378485894 -0.0651691854000092 4.0
30.228768722 -0.0632599070668221 5.0
38.866617317 -0.152456790208817 6.0
38.866617317 -0.178804367780685 7.0
42.3830573535328 0.0441976189613342 0.0
20.5951530026372 0.00717859901487827 1.0
21.7879043479541 0.0379698351025581 2.0
-9.3946766066605 -0.0421868488192558 3.0
12.3932277407315 -0.0662954598665237 4.0
29.98982960914 -0.0600584670901299 5.0
42.383057354 -0.0371242389082909 6.0
42.383057354 -0.146328836679459 7.0
47.6437510358734 0.0365454778075218 0.0
23.1576165238734 0.0390760600566864 1.0
24.4861345218735 0.0412691831588745 2.0
-8.74385685767346 -0.0413870960474014 3.0
15.7422776638735 -0.070779949426651 4.0
31.9014733818732 -0.0450864359736443 5.0
47.643751036 -0.0227065756917 6.0
47.643751036 -0.237147778272629 7.0
38.439399947 -0.122390456497669 0.0
19.975660163 0.0264585688710213 1.0
18.463739784 0.0405505895614624 2.0
-6.4266906816 -0.0354078412055969 3.0
12.037049093 -0.0373685956001282 4.0
26.402350854 -0.0455854535102844 5.0
38.439399957 -0.763085007667542 6.0
38.439399957 -0.0222226344048977 7.0
40.263562361 0.018993528559804 0.0
20.42306719 -0.0642023831605911 1.0
19.840495171 -0.0249978676438332 2.0
-11.677045299 -0.0361593663692474 3.0
8.1634498621 -0.0374956801533699 4.0
32.100112499 -0.0379363968968391 5.0
40.263562371 -0.270662009716034 6.0
40.263562371 -0.275338351726532 7.0
41.519528387 0.0427182354032993 0.0
22.877320367 0.0376705676317215 1.0
18.642208021 0.0272047556936741 2.0
-5.2280057864 -0.0555838197469711 3.0
13.414202225 -0.031653605401516 4.0
28.105326163 -0.0707293003797531 5.0
41.519528397 -0.121475122869015 6.0
41.519528397 -0.0653637126088142 7.0
40.414570464 0.0200692974030972 0.0
17.791289494 0.034082256257534 1.0
22.623280971 0.0156738813966513 2.0
-6.6442159501 -0.0659820511937141 3.0
15.979065012 -0.00801961869001389 4.0
24.435505453 -0.0650399625301361 5.0
40.414570474 -0.297674477100372 6.0
40.414570474 0.00987672060728073 7.0
37.834181215 0.0415064021945 0.0
18.770560656 -0.00596627593040466 1.0
19.063620563 0.0294556226581335 2.0
-7.0281211763 -0.0587303265929222 3.0
12.035499382 -0.0571348294615746 4.0
25.798681838 -0.0617083981633186 5.0
37.834181221 -0.105256967246532 6.0
37.834181221 -0.322875499725342 7.0
37.201121546 0.0469193086028099 0.0
17.811506649 0.0408440120518208 1.0
19.389614897 0.0283842738717794 2.0
-6.0990519868 -0.0661884173750877 3.0
13.290562901 -0.0385411977767944 4.0
23.910558645 -0.0659202560782433 5.0
37.201121556 -0.144562125205994 6.0
37.201121556 0.0326937101781368 7.0
42.944445132 -0.00222825258970261 0.0
21.59677927 0.00829800218343735 1.0
21.347665865 0.043580424040556 2.0
-6.9735110037 -0.0504516288638115 3.0
14.374154853 -0.0395250618457794 4.0
28.570290281 -0.0495991855859756 5.0
42.94444514 -0.357619106769562 6.0
42.94444514 -0.268938183784485 7.0
47.3096000339538 0.018135879188776 0.0
24.7818404359387 -0.0213067904114723 1.0
22.5277596000527 0.00332845747470856 2.0
-6.75460229154402 -0.0526378527283669 3.0
15.7731573079881 -0.0373589396476746 4.0
31.5364427279756 -0.00864154845476151 5.0
47.309600034 -0.135143369436264 6.0
47.309600034 -0.184653878211975 7.0
34.703660739 0.0374096930027008 0.0
16.95243502 0.0350937247276306 1.0
17.75122572 0.0250813700258732 2.0
-4.712400117 -0.0365715995430946 3.0
13.038825595 -0.031694583594799 4.0
21.664835145 -0.0384746417403221 5.0
34.703660747 -0.0621637627482414 6.0
34.703660747 -0.153873473405838 7.0
39.355679823 0.0659217685461044 0.0
18.719972615 0.00440776161849499 1.0
20.635707207 0.0269798245280981 2.0
-8.5340541425 -0.0331007614731789 3.0
12.101653055 -0.0297985523939133 4.0
27.254026768 -0.0218892768025398 5.0
39.355679833 -0.0534962117671967 6.0
39.355679833 -0.16459983587265 7.0
36.760838125 0.0196651313453913 0.0
18.098408908 -0.0474467426538467 1.0
18.662429217 -0.0226118862628937 2.0
-4.9002916105 -0.0504656061530113 3.0
13.762137597 -0.0336019769310951 4.0
22.998700528 -0.0481990799307823 5.0
36.760838135 -0.359061598777771 6.0
36.760838135 -0.496500849723816 7.0
45.817767369632 0.0753188654780388 0.0
23.4066861136316 0.0350271463394165 1.0
22.4110812656318 0.0306674353778362 2.0
-7.99255969573223 -0.0515327826142311 3.0
14.4185215696319 -0.0613889321684837 4.0
31.3992456710555 -0.0634442344307899 5.0
45.817767369 -0.0993344709277153 6.0
45.817767369 0.0179147310554981 7.0
37.456387934 0.0111894235014915 0.0
18.714257418 -0.16216504573822 1.0
18.742130516 -0.112128995358944 2.0
-7.2626807055 -0.0376872196793556 3.0
11.4794498 -0.0310089960694313 4.0
25.976938133 -0.0309191793203354 5.0
37.456387944 -0.585336148738861 6.0
37.456387944 -1.07507383823395 7.0
36.979299104 -0.0050138533115387 0.0
17.001867984 0.0227993223816156 1.0
19.977431121 0.0459257662296295 2.0
-4.4589079828 -0.0636028349399567 3.0
15.518523129 -0.0403740853071213 4.0
21.460775976 -0.029265321791172 5.0
36.979299113 -0.544855713844299 6.0
36.979299113 -0.283728092908859 7.0
44.357392148 0.0268610995262861 0.0
23.57148554 -0.055705800652504 1.0
20.785906608 -0.0171058773994446 2.0
-6.1950554121 -0.0453581735491753 3.0
14.590851187 -0.0613951534032822 4.0
29.766540962 -0.0754731819033623 5.0
44.357392157 -0.377146303653717 6.0
44.357392157 -0.398692101240158 7.0
36.8120792993302 0.043767835944891 0.0
19.8640110823302 -0.0792993381619453 1.0
16.9480682263302 -0.0177819095551968 2.0
-8.51793333553018 -0.0522296354174614 3.0
8.43013489113018 -0.0356935188174248 4.0
28.381944393825 -0.0789077877998352 5.0
36.812079299 0.00286086648702621 6.0
36.812079299 -0.412260293960571 7.0
43.172254734 0.0171298049390316 0.0
21.994700089 -0.0144734941422939 1.0
21.177554646 0.0152404755353928 2.0
-4.8220170211 -0.0568526610732079 3.0
16.355537616 -0.061071440577507 4.0
26.816717119 -0.0544687882065773 5.0
43.172254743 -0.0547473058104515 6.0
43.172254743 -0.383180618286133 7.0
47.109370558 0.0188897810876369 0.0
25.034746172 0.0338464602828026 1.0
22.074624387 0.0369896218180656 2.0
-8.7769581235 -0.0602837279438972 3.0
13.297666255 -0.0588186830282211 4.0
33.811704304 -0.0591272935271263 5.0
47.109370567 -0.385139286518097 6.0
47.109370567 -0.226035416126251 7.0
35.871355653 -0.261876225471497 0.0
18.076771786 0.0137737840414047 1.0
17.794583868 0.0482020191848278 2.0
-9.6137759762 -0.0592323467135429 3.0
8.1808078825 -0.0434436798095703 4.0
27.690547772 -0.0721292719244957 5.0
35.871355662 -0.549667418003082 6.0
35.871355662 -0.0208089500665665 7.0
42.1694339111434 0.0067223496735096 0.0
22.6030080131434 0.0325951129198074 1.0
19.5664259071434 0.0372525230050087 2.0
-7.61628576964338 -0.0675823912024498 3.0
11.9501401371434 -0.0258526057004929 4.0
30.2192937831435 -0.0611324831843376 5.0
42.169433911 -0.427654802799225 6.0
42.169433911 -0.0293566696345806 7.0
38.001249391 0.0264990273863077 0.0
19.442391644 0.0140718556940556 1.0
18.558857748 0.0276982877403498 2.0
-7.4577746389 -0.0484223738312721 3.0
11.101083099 -0.058791771531105 4.0
26.900166292 -0.0351744145154953 5.0
38.001249401 -0.124504558742046 6.0
38.001249401 -0.115719430148602 7.0
36.026296144 0.0435544885694981 0.0
18.544058535 0.031823456287384 1.0
17.482237609 0.0126717258244753 2.0
-7.0335899837 -0.06505037099123 3.0
10.448647616 -0.00731083750724792 4.0
25.577648528 -0.0453304126858711 5.0
36.026296153 -0.020963903516531 6.0
36.026296153 0.0141795016825199 7.0
33.200213149 0.0423356331884861 0.0
14.957525483 0.0203448385000229 1.0
18.242687666 0.0357859507203102 2.0
-8.613844647 -0.0574529841542244 3.0
9.6288430093 -0.0559827461838722 4.0
23.57137014 -0.0568394884467125 5.0
33.200213159 -0.355901330709457 6.0
33.200213159 -0.491756916046143 7.0
36.881947947 0.0229568611830473 0.0
19.760267083 -0.0499316975474358 1.0
17.12168087 -0.0253972560167313 2.0
-6.5510496361 -0.0631918460130692 3.0
10.570631229 -0.0685790106654167 4.0
26.311316723 -0.0577027723193169 5.0
36.881947952 0.0417246744036674 6.0
36.881947952 -0.588856935501099 7.0
28.363012003 0.0439397096633911 0.0
14.617095463 0.0346934273838997 1.0
13.745916541 0.0371133238077164 2.0
-5.3741086073 -0.0530798360705376 3.0
8.3718079243 -0.053713858127594 4.0
19.991204079 -0.0410141423344612 5.0
28.363012012 -0.0105931423604488 6.0
28.363012012 -0.136456847190857 7.0
39.574467416 0.0350372567772865 0.0
20.649388383 -0.0250292234122753 1.0
18.925079033 0.0201428607106209 2.0
-8.4547238017 -0.0459189191460609 3.0
10.470355222 -0.0477260053157806 4.0
29.104112194 -0.0621249005198479 5.0
39.574467426 -0.453189849853516 6.0
39.574467426 -0.227012515068054 7.0
36.220401471 0.0283256117254496 0.0
17.995668566 0.0364360213279724 1.0
18.224732906 0.0360831394791603 2.0
-6.9105415737 -0.0497512444853783 3.0
11.314191323 -0.0424889326095581 4.0
24.906210149 -0.0448641330003738 5.0
36.22040148 -0.0906462892889977 6.0
36.22040148 -0.0485406666994095 7.0
31.086432133 0.0242864079773426 0.0
16.0910287 -0.0408846735954285 1.0
14.995403435 -0.0136744193732738 2.0
-4.3439743018 -0.0321926549077034 3.0
10.651429126 -0.0802481472492218 4.0
20.43500301 -0.0589613541960716 5.0
31.086432141 0.0125051960349083 6.0
31.086432141 -0.663692772388458 7.0
36.274353053 -0.00117260962724686 0.0
18.420887917 -0.0867306068539619 1.0
17.853465137 -0.0566078200936317 2.0
-3.6202808126 -0.0527646541595459 3.0
14.233184315 -0.0388623848557472 4.0
22.041168739 -0.0678054317831993 5.0
36.274353063 -0.55198210477829 6.0
36.274353063 -0.453119039535522 7.0
31.65795008 -0.0479431375861168 0.0
17.489254099 -0.0950396880507469 1.0
14.168695981 -0.0645492300391197 2.0
-6.0427493715 -0.0569743886590004 3.0
8.1259465996 -0.0373838469386101 4.0
23.53200348 -0.0574532896280289 5.0
31.65795009 -0.981661200523376 6.0
31.65795009 -0.744759440422058 7.0
37.059171469 0.0564282387495041 0.0
19.339994356 -0.132991760969162 1.0
17.719177113 -0.118376843631268 2.0
-9.4011683011 -0.0348194763064384 3.0
8.318008802 -0.0496242046356201 4.0
28.741162667 -0.0327419564127922 5.0
37.059171479 0.00460707768797874 6.0
37.059171479 -0.291975289583206 7.0
45.308862348 0.0165018234401941 0.0
24.95721568 -0.0131680779159069 1.0
20.351646669 0.0373256430029869 2.0
-7.323343681 -0.0720997527241707 3.0
13.028302978 -0.0200275406241417 4.0
32.28055937 -0.0563653409481049 5.0
45.308862357 -0.403715133666992 6.0
45.308862357 -0.255374550819397 7.0
32.988279899 -0.0171960145235062 0.0
17.965429309 0.0281377546489239 1.0
15.02285059 0.00966829992830753 2.0
-2.6394240327 -0.0684048533439636 3.0
12.383426547 -0.0349165499210358 4.0
20.604853352 -0.069566898047924 5.0
32.988279909 -0.544977068901062 6.0
32.988279909 0.0487488619983196 7.0
41.812200584 0.0302371978759766 0.0
21.127130287 0.0367862209677696 1.0
20.685070296 0.0389669500291348 2.0
-5.5457377677 -0.0351228415966034 3.0
15.139332519 -0.0693945959210396 4.0
26.672868065 -0.0578861013054848 5.0
41.812200593 -0.830668568611145 6.0
41.812200593 0.0128047652542591 7.0
34.159577997 -0.0604210942983627 0.0
15.979883142 0.0100911110639572 1.0
18.179694855 0.00574369542300701 2.0
-6.0058060763 -0.0399582609534264 3.0
12.173888768 -0.0577822402119637 4.0
21.985689228 -0.0309098958969116 5.0
34.159578007 -0.527144372463226 6.0
34.159578007 0.0337502732872963 7.0
41.353058194 -0.00614935904741287 0.0
20.665722502 0.0390244498848915 1.0
20.687335692 0.0342376865446568 2.0
-9.525258586 -0.0587189570069313 3.0
11.162077096 -0.035348080098629 4.0
30.190981098 -0.0241890326142311 5.0
41.353058204 -0.702287614345551 6.0
41.353058204 -0.0186640657484531 7.0
35.663236634 0.0303183663636446 0.0
18.766917562 0.00306697003543377 1.0
16.896319073 0.0431029573082924 2.0
-7.670552873 -0.0539950504899025 3.0
9.2257661902 -0.0410744324326515 4.0
26.437470445 -0.0778755396604538 5.0
35.663236644 -0.0443513691425323 6.0
35.663236644 -0.338216006755829 7.0
40.444916113 -0.0190286710858345 0.0
18.421748417 0.0199074875563383 1.0
22.023167696 0.0100390315055847 2.0
-9.937900394 -0.058681957423687 3.0
12.085267292 -0.0375925302505493 4.0
28.359648821 -0.0452221184968948 5.0
40.444916123 -0.786129772663116 6.0
40.444916123 -0.227881133556366 7.0
39.900416904 -0.0325471684336662 0.0
21.000367331 -0.225441575050354 1.0
18.900049573 -0.134783774614334 2.0
-6.1981317567 -0.0702662318944931 3.0
12.701917806 0.010813046246767 4.0
27.198499098 -0.0690015330910683 5.0
39.900416914 -0.252240598201752 6.0
39.900416914 -0.535950124263763 7.0
34.40532484 0.087690956890583 0.0
18.680133752 0.00108133815228939 1.0
15.725191089 0.028207290917635 2.0
-6.5675447079 -0.0489425957202911 3.0
9.157646372 -0.0666032657027245 4.0
25.247678469 -0.0604385733604431 5.0
34.405324849 -0.143321067094803 6.0
34.405324849 -0.12729275226593 7.0
33.074255995 0.0390125624835491 0.0
16.81819884 0.0257429778575897 1.0
16.256057155 0.0213860999792814 2.0
-5.8048358004 -0.0425945073366165 3.0
10.451221345 -0.0463372990489006 4.0
22.62303465 -0.0544388741254807 5.0
33.074256004 -0.343753337860107 6.0
33.074256004 -0.237452775239944 7.0
40.036170298 -0.0437742844223976 0.0
18.427862329 -0.0618081763386726 1.0
21.60830797 -0.0121447890996933 2.0
-6.8170168406 -0.0550939664244652 3.0
14.791291119 -0.0498179197311401 4.0
25.244879179 -0.0502470508217812 5.0
40.036170308 -0.624238133430481 6.0
40.036170308 -0.28652548789978 7.0
44.453206231 0.0690874382853508 0.0
21.51525667 0.035507183521986 1.0
22.937949561 0.0364484116435051 2.0
-6.654859187 -0.0687020421028137 3.0
16.283090365 -0.0341028422117233 4.0
28.170115867 -0.0622693300247192 5.0
44.453206241 -0.119050763547421 6.0
44.453206241 -0.145969688892365 7.0
33.8500454 -0.0675431564450264 0.0
15.317438725 0.00605981983244419 1.0
18.532606675 0.0291151441633701 2.0
-3.771729854 -0.0312505289912224 3.0
14.760876811 -0.0484069958329201 4.0
19.089168589 -0.0351075232028961 5.0
33.85004541 -0.514841794967651 6.0
33.85004541 -0.0811453759670258 7.0
30.537877299 0.0179950892925262 0.0
15.743236653 -0.0376909300684929 1.0
14.794640646 -0.00281290709972382 2.0
-6.2614673178 -0.0671926289796829 3.0
8.5331733191 -0.0253544449806213 4.0
22.00470398 -0.0528230890631676 5.0
30.537877308 -0.406552970409393 6.0
30.537877308 -0.365647107362747 7.0
27.727440934 -0.0750791803002357 0.0
13.298433958 -0.0217454358935356 1.0
14.429006978 0.025978485122323 2.0
-4.8732361991 -0.0381665304303169 3.0
9.5557707702 -0.0581131353974342 4.0
18.171670165 -0.0602271631360054 5.0
27.727440942 -0.513334631919861 6.0
27.727440942 -0.0955781117081642 7.0
35.630182891 0.0478591546416283 0.0
17.416160369 0.0185670796781778 1.0
18.214022525 0.0459841042757034 2.0
-5.5362398817 -0.0644200369715691 3.0
12.677782636 -0.0481854379177094 4.0
22.952400258 -0.056090347468853 5.0
35.630182897 -0.0190120525658131 6.0
35.630182897 -0.142743349075317 7.0
31.038251509 0.0403429046273232 0.0
16.216885208 0.00841322354972363 1.0
14.821366301 -0.0102893933653831 2.0
-5.1281710605 -0.0671710297465324 3.0
9.6931952301 -0.0263101756572723 4.0
21.345056279 -0.0476063936948776 5.0
31.038251519 -0.110031597316265 6.0
31.038251519 0.0455542877316475 7.0
31.867011958 0.0290168803185225 0.0
14.888086344 -0.0344657152891159 1.0
16.978925614 0.0212374627590179 2.0
-5.7710515718 -0.0632317662239075 3.0
11.207874033 0.00433149375021458 4.0
20.659137926 -0.0736644566059113 5.0
31.867011967 -0.0623030886054039 6.0
31.867011967 -0.358501613140106 7.0
30.754831636 0.0239489004015923 0.0
14.067592922 0.0366189442574978 1.0
16.687238713 0.0379426591098309 2.0
-4.2836008115 -0.0432199016213417 3.0
12.403637892 -0.037007562816143 4.0
18.351193744 -0.0349124297499657 5.0
30.754831646 -0.0450859516859055 6.0
30.754831646 -0.196558982133865 7.0
38.951564273 0.0436520799994469 0.0
20.789904648 0.0154745653271675 1.0
18.161659627 0.00150410644710064 2.0
-9.14497609 -0.0637726932764053 3.0
9.0166835299 -0.0441302806138992 4.0
29.934880745 -0.0683315396308899 5.0
38.95156428 -0.246556222438812 6.0
38.95156428 0.0314973704516888 7.0
33.384480929 -0.0932462066411972 0.0
14.280681926 0.0182985048741102 1.0
19.103799006 0.0352427437901497 2.0
-6.8894608583 -0.0592744573950768 3.0
12.214338142 -0.0590131357312202 4.0
21.170142791 -0.0654752254486084 5.0
33.384480936 -1.06752145290375 6.0
33.384480936 -0.311685830354691 7.0
37.291199278 0.0400774367153645 0.0
18.976122996 -0.0499358549714088 1.0
18.315076287 -0.00429799407720566 2.0
-3.6314782733 -0.0494160056114197 3.0
14.68359801 -0.0746961459517479 4.0
22.607601274 -0.0708129331469536 5.0
37.291199282 0.0110992342233658 6.0
37.291199282 -0.51896071434021 7.0
33.5718110164568 0.0205284133553505 0.0
16.2455196066318 -0.0377769395709038 1.0
17.3262914102844 -0.000671122223138809 2.0
-6.34124959778268 -0.053726390004158 3.0
10.9850418123699 -0.0438600406050682 4.0
22.5867692042782 -0.0609134212136269 5.0
33.571811016 0.0492992773652077 6.0
33.571811016 -0.527525663375854 7.0
44.048761737 0.0288491286337376 0.0
22.194046458 0.000301524996757507 1.0
21.85471528 0.0481485910713673 2.0
-6.0219898767 -0.0626436471939087 3.0
15.832725394 -0.0123784318566322 4.0
28.216036344 -0.0456880927085876 5.0
44.048761746 -0.191003620624542 6.0
44.048761746 -0.0755114480853081 7.0
38.092815451 0.0275078900158405 0.0
20.42584869 -0.0338477715849876 1.0
17.666966762 0.00595947355031967 2.0
-5.2714738638 -0.0512436553835869 3.0
12.395492889 -0.0514764338731766 4.0
25.697322563 -0.0632710158824921 5.0
38.09281546 -0.125028550624847 6.0
38.09281546 -0.285274982452393 7.0
42.616606914 0.0312907472252846 0.0
22.043217204 -0.0651348233222961 1.0
20.57338971 0.000939739868044853 2.0
-9.1142862064 -0.0609091147780418 3.0
11.459103494 -0.0513002648949623 4.0
31.157503421 -0.0647010281682014 5.0
42.616606924 -0.211671710014343 6.0
42.616606924 -0.22219780087471 7.0
41.080227066 0.0298075620085001 0.0
21.412108977 0.0264576785266399 1.0
19.66811809 0.0417046509683132 2.0
-8.9308117628 -0.0418629571795464 3.0
10.737306318 -0.0325812548398972 4.0
30.342920748 -0.0264448150992393 5.0
41.080227074 -0.289076209068298 6.0
41.080227074 -0.295007586479187 7.0
36.859087804 0.0294454675167799 0.0
16.308519924 -0.00563553720712662 1.0
20.550567882 0.0366460382938385 2.0
-8.0027710471 -0.049416609108448 3.0
12.547796826 -0.0197320133447647 4.0
24.31129098 -0.0421358048915863 5.0
36.859087813 0.0210134536027908 6.0
36.859087813 -0.104063399136066 7.0
39.316268866 0.0143213532865047 0.0
19.479900065 0.0276161078363657 1.0
19.836368801 0.0269371401518583 2.0
-7.370707565 -0.0584516078233719 3.0
12.465661226 -0.0450881719589233 4.0
26.85060764 -0.0201518647372723 5.0
39.316268876 -0.268287718296051 6.0
39.316268876 -0.0142187029123306 7.0
40.023071957 0.00551008433103561 0.0
21.374119573 0.0329446271061897 1.0
18.648952384 0.0233588516712189 2.0
-8.5239396023 -0.0331303477287292 3.0
10.125012772 -0.0434002950787544 4.0
29.898059185 -0.0325272977352142 5.0
40.023071967 -0.436534941196442 6.0
40.023071967 -0.0280381515622139 7.0
32.691039556 0.0414417609572411 0.0
16.99402987 0.0150389987975359 1.0
15.697009686 0.0336041003465652 2.0
-6.1678662484 -0.0449318438768387 3.0
9.5291434282 -0.0582882761955261 4.0
23.161896128 -0.0657451003789902 5.0
32.691039566 -0.0177557915449142 6.0
32.691039566 -0.381907492876053 7.0
39.350172222 0.0181190688163042 0.0
19.79571531 0.0095364972949028 1.0
19.554456913 0.0461789555847645 2.0
-6.9332448751 -0.0570366680622101 3.0
12.62121203 -0.0541557744145393 4.0
26.728960194 -0.0666891112923622 5.0
39.35017223 -0.131873726844788 6.0
39.35017223 -0.123919941484928 7.0
38.6257792205169 0.0434264913201332 0.0
20.6764320695262 -0.116823725402355 1.0
17.9493471475253 -0.076203428208828 2.0
-8.74948254057215 -0.0488495826721191 3.0
9.19986460693105 -0.0763204246759415 4.0
29.4259146106884 -0.0795109793543816 5.0
38.62577922 -0.173480093479156 6.0
38.62577922 -1.16326355934143 7.0
43.363210173 0.0454277396202087 0.0
22.264479357 -0.0179250985383987 1.0
21.09873082 0.00792696699500084 2.0
-7.8566163895 -0.0315596759319305 3.0
13.242114424 -0.0782894119620323 4.0
30.121095753 -0.0472186282277107 5.0
43.36321018 -0.0394369661808014 6.0
43.36321018 -0.381152480840683 7.0
41.7113705187954 0.0265338905155659 0.0
19.9560794530664 0.0194998774677515 1.0
21.7552910618007 0.0040808767080307 2.0
-6.27295647675057 -0.0499210581183434 3.0
15.4823345851289 -0.0363724753260612 4.0
26.2290359296866 -0.0542789623141289 5.0
41.711370519 -0.239005833864212 6.0
41.711370519 0.00792845152318478 7.0
41.4017485851654 0.0389688685536385 0.0
21.9191533447582 0.0143513642251492 1.0
19.4825952431145 0.0440628528594971 2.0
-7.53340700226978 -0.0528775826096535 3.0
11.9491882404739 -0.0571010038256645 4.0
29.4525603463383 -0.0624911859631538 5.0
41.401748585 -0.378843277692795 6.0
41.401748585 -0.430770337581635 7.0
45.411303660968 -0.0131861343979836 0.0
22.8793399812672 0.0073622353374958 1.0
22.5319636798771 0.00174073874950409 2.0
-8.35657341793646 -0.0485015362501144 3.0
14.1753902620561 -0.0504784658551216 4.0
31.2359133986188 -0.0367661044001579 5.0
45.411303661 -0.38309571146965 6.0
45.411303661 0.0463464185595512 7.0
46.173622728 0.0689187124371529 0.0
22.478207281 0.0410024076700211 1.0
23.695415447 0.0295655149966478 2.0
-7.2335265538 -0.0515206456184387 3.0
16.461888883 -0.00100727379322052 4.0
29.711733845 -0.0508603677153587 5.0
46.173622738 -0.0910835266113281 6.0
46.173622738 -0.314854592084885 7.0
43.970500883 0.0478466153144836 0.0
23.319439955 -0.0784996524453163 1.0
20.651060929 -0.0204342156648636 2.0
-9.6037663307 -0.0472719445824623 3.0
11.047294589 -0.0451957061886787 4.0
32.923206295 -0.0695701390504837 5.0
43.970500892 -0.34167093038559 6.0
43.970500892 -0.300153732299805 7.0
41.561964514 0.0258931126445532 0.0
22.423587267 -0.0170620419085026 1.0
19.138377247 0.0114757474511862 2.0
-2.7143772539 -0.0528124198317528 3.0
16.423999984 -0.0395266115665436 4.0
25.13796453 -0.0645664036273956 5.0
41.561964524 -0.324685841798782 6.0
41.561964524 -0.501964390277863 7.0
42.043570825 0.00462557375431061 0.0
21.184612309 0.00413530692458153 1.0
20.858958517 0.0369522571563721 2.0
-7.1973175688 -0.0463944151997566 3.0
13.66164094 -0.0530467927455902 4.0
28.381929887 -0.0361824780702591 5.0
42.043570834 -0.343560039997101 6.0
42.043570834 -0.0912092328071594 7.0
42.728187899 0.0456720516085625 0.0
22.937429443 -0.0247975662350655 1.0
19.790758457 0.00739779695868492 2.0
-7.3221221307 -0.0438755005598068 3.0
12.468636317 -0.0448030307888985 4.0
30.259551583 -0.0208149217069149 5.0
42.728187908 -0.0665146857500076 6.0
42.728187908 -0.311159521341324 7.0
46.647822935 0.00535415858030319 0.0
24.419972425 0.0312618687748909 1.0
22.22785051 0.0240206439048052 2.0
-9.7005341836 -0.03370051831007 3.0
12.527316316 -0.00766000524163246 4.0
34.120506619 0.00209721550345421 5.0
46.647822945 -0.499615788459778 6.0
46.647822945 -0.0685074701905251 7.0
34.157298452 0.0325709506869316 0.0
16.904613725 0.0224918406456709 1.0
17.252684728 0.0384860411286354 2.0
-7.2828539724 -0.0479861348867416 3.0
9.9698307451 -0.0330589562654495 4.0
24.187467707 -0.0460915863513947 5.0
34.157298462 -0.296641707420349 6.0
34.157298462 -0.217180818319321 7.0
38.08940785 0.0453529208898544 0.0
20.151317931 -0.178822100162506 1.0
17.93808992 -0.0999018773436546 2.0
-2.5516772312 -0.0361093878746033 3.0
15.386412679 -0.0792258977890015 4.0
22.702995171 -0.0507737547159195 5.0
38.08940786 -0.243494212627411 6.0
38.08940786 -1.19506657123566 7.0
48.253456595 -0.00270995497703552 0.0
26.02387816 -0.020561508834362 1.0
22.229578435 0.0157057363539934 2.0
-8.9125381923 -0.0418787524104118 3.0
13.317040233 -0.0543556809425354 4.0
34.936416362 -0.0610704645514488 5.0
48.253456605 -0.549655795097351 6.0
48.253456605 -0.316257119178772 7.0
41.303512945 0.0367762371897697 0.0
21.842523524 0.00420634262263775 1.0
19.460989421 0.0296419654041529 2.0
-8.4785475097 -0.0388789474964142 3.0
10.982441901 -0.0841254144906998 4.0
30.321071043 -0.0464244112372398 5.0
41.303512955 -0.00455212965607643 6.0
41.303512955 -0.165126860141754 7.0
34.11247551 0.0242203325033188 0.0
16.407323353 -0.0128256529569626 1.0
17.705152157 0.0244385860860348 2.0
-3.4961419723 -0.0632969662547112 3.0
14.209010175 -0.0494198799133301 4.0
19.903465335 -0.0612291470170021 5.0
34.11247552 -0.35261857509613 6.0
34.11247552 -0.154126733541489 7.0
39.346899311 0.0142774619162083 0.0
20.280952021 0.00926779024302959 1.0
19.06594729 0.045375756919384 2.0
-10.212334112 -0.0602355971932411 3.0
8.8536131687 -0.0182217247784138 4.0
30.493286142 -0.0435373559594154 5.0
39.346899321 -0.0680725425481796 6.0
39.346899321 -0.105323575437069 7.0
42.355081312 0.041327603161335 0.0
22.237080125 -0.151252001523972 1.0
20.118001187 -0.0853633657097816 2.0
-4.2250518369 -0.0632446780800819 3.0
15.892949341 -0.0147857591509819 4.0
26.462131971 -0.064828060567379 5.0
42.355081321 -0.227037847042084 6.0
42.355081321 -0.935265421867371 7.0
46.02528461 0.0115870703011751 0.0
20.767999035 0.0140018481761217 1.0
25.257285575 0.00501182861626148 2.0
-9.1629445671 -0.0486848875880241 3.0
16.094340999 -0.0416335836052895 4.0
29.930943612 -0.0257430970668793 5.0
46.025284619 -0.388243019580841 6.0
46.025284619 0.0453107394278049 7.0
48.790909777 0.0406514927744865 0.0
25.145056622 0.0311610586941242 1.0
23.645853155 0.0111964475363493 2.0
-9.2179915631 -0.0760889053344727 3.0
14.427861582 -0.00886192917823792 4.0
34.363048195 -0.0592650026082993 5.0
48.790909787 -0.0432110652327538 6.0
48.790909787 -0.0942874774336815 7.0
41.698940493 0.035691998898983 0.0
20.507660886 0.0225581005215645 1.0
21.191279609 0.0270785223692656 2.0
-9.1543537316 -0.0448126345872879 3.0
12.036925868 -0.0509086027741432 4.0
29.662014626 -0.0247186422348022 5.0
41.698940502 -0.233936905860901 6.0
41.698940502 -0.000260546803474426 7.0
47.902972774 0.0232857763767242 0.0
23.393159022 0.0236850064247847 1.0
24.509813754 0.0485463440418243 2.0
-8.5476809737 -0.0634744018316269 3.0
15.962132774 -0.0574222356081009 4.0
31.940840003 -0.0440519079566002 5.0
47.902972781 -0.335597574710846 6.0
47.902972781 -0.0759258791804314 7.0
41.377170862 -0.00195155292749405 0.0
20.577519401 0.0285013020038605 1.0
20.799651462 0.0288595724850893 2.0
-6.2821130824 -0.040125884115696 3.0
14.517538372 -0.0653826594352722 4.0
26.859632492 -0.0574998930096626 5.0
41.37717087 -0.527377367019653 6.0
41.37717087 -0.0440122783184052 7.0
32.285684892 0.00663052499294281 0.0
14.316873201 0.0221371222287416 1.0
17.968811691 0.00568561628460884 2.0
-5.3227711546 -0.0395063459873199 3.0
12.646040526 -0.0341564491391182 4.0
19.639644366 -0.0531905964016914 5.0
32.285684902 -0.382745683193207 6.0
32.285684902 0.000681888312101364 7.0
32.344206843 0.0436337478458881 0.0
16.160894185 0.0336866080760956 1.0
16.183312659 0.0359318479895592 2.0
-7.3310896054 -0.0400741025805473 3.0
8.852223045 -0.0233137309551239 4.0
23.491983799 -0.0406425520777702 5.0
32.344206851 -0.121548973023891 6.0
32.344206851 -0.102648697793484 7.0
34.184112321 0.0411880016326904 0.0
18.400062549 0.0401367694139481 1.0
15.784049772 0.028875945135951 2.0
-7.6583985283 -0.0407380312681198 3.0
8.125651234 -0.0642453581094742 4.0
26.058461087 -0.068032868206501 5.0
34.184112331 -0.020537406206131 6.0
34.184112331 -0.167329221963882 7.0
34.663221798 0.0487473830580711 0.0
18.206197344 -0.00160729140043259 1.0
16.457024454 0.0399510711431503 2.0
-6.9582970888 -0.0648860558867455 3.0
9.4987273548 -0.0368800759315491 4.0
25.164494443 -0.0725341662764549 5.0
34.663221808 0.0166303962469101 6.0
34.663221808 -0.238188236951828 7.0
36.873774533 -0.0518151894211769 0.0
18.579406772 0.0320964083075523 1.0
18.294367761 0.0355938784778118 2.0
-5.3733020341 -0.0638706684112549 3.0
12.921065717 -0.0460816994309425 4.0
23.952708817 -0.0713126882910728 5.0
36.873774543 -0.813101708889008 6.0
36.873774543 -0.00338543951511383 7.0
40.211352898 0.0439534932374954 0.0
22.061687235 -0.0151425339281559 1.0
18.149665663 0.0202755909413099 2.0
-7.4872031869 -0.0507754981517792 3.0
10.662462467 -0.0637961849570274 4.0
29.548890431 -0.0710687339305878 5.0
40.211352908 -0.608526945114136 6.0
40.211352908 -0.232146978378296 7.0
46.160689864 0.0453301146626472 0.0
24.176307096 -0.0240240544080734 1.0
21.984382774 0.0193240493535995 2.0
-5.8472747438 -0.0496827811002731 3.0
16.137108027 -0.0482558086514473 4.0
30.023581843 -0.0689057633280754 5.0
46.160689867 -0.183772534132004 6.0
46.160689867 -0.297767400741577 7.0
32.727702668 0.0280661303550005 0.0
17.796794822 -0.157288044691086 1.0
14.930907847 -0.130205780267715 2.0
-5.7416146893 -0.0318115279078484 3.0
9.1892931492 -0.0783709511160851 4.0
23.53840952 -0.0334509611129761 5.0
32.727702676 0.0168217159807682 6.0
32.727702676 -0.940004110336304 7.0
40.640985417073 -0.0249609872698784 0.0
21.2248663380953 -0.0118576362729073 1.0
19.4161190831271 0.027624512091279 2.0
-6.12795167285201 -0.0593432188034058 3.0
13.2881674100867 -0.0154104679822922 4.0
27.3528180102152 -0.0712790116667747 5.0
40.640985417 -0.836148142814636 6.0
40.640985417 -0.29067450761795 7.0
34.611451584 0.0383331589400768 0.0
16.03184427 -0.000740490853786469 1.0
18.579607315 0.0308886375278234 2.0
-5.6371786801 -0.0651107430458069 3.0
12.942428625 -0.0565173849463463 4.0
21.66902296 -0.0528622344136238 5.0
34.611451594 -0.0309506393969059 6.0
34.611451594 -0.328593224287033 7.0
43.311280798 0.0370707251131535 0.0
19.747206987 0.00650256127119064 1.0
23.564073813 0.0405033566057682 2.0
-7.5093463576 -0.0407328084111214 3.0
16.054727447 -0.0650886669754982 4.0
27.256553353 -0.0704057440161705 5.0
43.311280807 -0.0380830615758896 6.0
43.311280807 -0.163647830486298 7.0
36.406364048 0.0226279180496931 0.0
19.818637935 -0.0181907415390015 1.0
16.587726114 0.0214527230709791 2.0
-3.8896932395 -0.0547431409358978 3.0
12.698032865 -0.0567463934421539 4.0
23.708331184 -0.0512598156929016 5.0
36.406364057 -0.243095755577087 6.0
36.406364057 -0.201257824897766 7.0
31.68563032 -0.000856928527355194 0.0
16.740289757 -0.0779547020792961 1.0
14.945340567 -0.0514571592211723 2.0
-6.0451472911 -0.0531032532453537 3.0
8.9001932697 -0.0362881645560265 4.0
22.785437054 -0.039977639913559 5.0
31.685630326 -0.647223711013794 6.0
31.685630326 -0.899905741214752 7.0
40.165364423 -0.0231708884239197 0.0
21.225216477 -0.121363572776318 1.0
18.940147947 -0.0944210216403008 2.0
-9.5161355118 -0.0449299067258835 3.0
9.4240124269 -0.0459521859884262 4.0
30.741351997 -0.0195789784193039 5.0
40.165364432 -0.816589176654816 6.0
40.165364432 -0.861540019512177 7.0
42.886465637 0.0345371924340725 0.0
31.633244398 -0.0218252390623093 1.0
11.253221239 0.0146696679294109 2.0
0 -0.0478704944252968 3.0
11.253221239 -0.0295307822525501 4.0
31.633244398 -0.0491076111793518 5.0
42.886465647 -0.0803287923336029 6.0
42.886465647 -0.360279500484467 7.0
46.942592322 0.0421572998166084 0.0
23.694673267 -0.0538899600505829 1.0
23.247919055 -7.45188444852829e-05 2.0
-9.8881851606 -0.0560498088598251 3.0
13.359733884 -0.0474737286567688 4.0
33.582858438 -0.0447608232498169 5.0
46.942592332 -0.49423611164093 6.0
46.942592332 -0.263949990272522 7.0
38.417030558 -0.0134609639644623 0.0
20.711241226 0.0398071631789207 1.0
17.705789332 0.0430586561560631 2.0
-7.7084466149 -0.0435935407876968 3.0
9.9973427073 -0.068759098649025 4.0
28.419687851 -0.0470294803380966 5.0
38.417030568 -0.821695566177368 6.0
38.417030568 -0.621136844158173 7.0
39.82159542 0.0365665256977081 0.0
19.816428679 -0.0958569869399071 1.0
20.005166741 -0.0608043447136879 2.0
-5.1739908699 -0.0663977190852165 3.0
14.831175861 -0.0172827318310738 4.0
24.990419559 -0.0469860807061195 5.0
39.82159543 -0.0139769315719604 6.0
39.82159543 -0.691051363945007 7.0
42.375846264 0.0428711511194706 0.0
20.538210367 0.0222928021103144 1.0
21.837635902 0.0538867935538292 2.0
-5.3545430676 -0.056429423391819 3.0
16.483092829 -0.0266161784529686 4.0
25.89275344 -0.0630203187465668 5.0
42.375846269 -0.332432329654694 6.0
42.375846269 0.00767243467271328 7.0
41.909868914 0.0337462425231934 0.0
21.249762031 -0.218082636594772 1.0
20.660106883 -0.245476961135864 2.0
-10.971309543 -0.0636516436934471 3.0
9.6887973304 -0.0338950082659721 4.0
32.221071584 -0.0255532786250114 5.0
41.909868924 0.0051354393362999 6.0
41.909868924 -0.73776650428772 7.0
38.5154038135067 0.0559401959180832 0.0
19.3171244035071 -0.241924792528152 1.0
19.1982794195071 -0.132210999727249 2.0
-8.28760453680882 -0.0474852100014687 3.0
10.9106748825063 -0.0663570761680603 4.0
27.6047289405071 -0.0560895502567291 5.0
38.515403814 -0.000397667288780212 6.0
38.515403814 -0.633684813976288 7.0
32.187464764 -0.0815387144684792 0.0
15.688410189 -0.00973425805568695 1.0
16.499054576 0.0387878492474556 2.0
-5.8055743047 -0.0684396252036095 3.0
10.693480262 -0.0317696332931519 4.0
21.493984503 -0.0736880674958229 5.0
32.187464773 -0.765082359313965 6.0
32.187464773 -0.207578361034393 7.0
43.5533404982843 0.012359730899334 0.0
25.257472916292 0.0374502688646317 1.0
18.2958675832873 0.0349311158061028 2.0
-8.52565235008034 -0.0541724041104317 3.0
9.77021523347734 -0.0379644185304642 4.0
33.7831252662979 -0.0539882183074951 5.0
43.553340498 -0.624590277671814 6.0
43.553340498 -0.525385022163391 7.0
38.661685108 0.0443852804601192 0.0
21.023698145 0.0275101903825998 1.0
17.637986964 0.0379537492990494 2.0
-6.4647611773 -0.0667300149798393 3.0
11.173225776 -0.0327119380235672 4.0
27.488459332 -0.0835198983550072 5.0
38.661685118 -0.187390327453613 6.0
38.661685118 -0.202429950237274 7.0
47.185976942 -0.0180916376411915 0.0
20.88210062 -0.100233428180218 1.0
26.303876326 -0.0971397683024406 2.0
-9.8059530494 -0.0692989900708199 3.0
16.497923272 -0.0362959653139114 4.0
30.688053675 -0.0397776737809181 5.0
47.185976948 -0.538367390632629 6.0
47.185976948 -0.565127313137054 7.0
41.0736498246764 0.0682663843035698 0.0
17.9299575176762 -0.183509528636932 1.0
23.1436923166761 -0.15204593539238 2.0
-7.36188695227673 -0.0515283495187759 3.0
15.7818053646765 -0.0270263366401196 4.0
25.2918444696756 -0.0593035891652107 5.0
41.073649825 -0.226306647062302 6.0
41.073649825 -0.393095672130585 7.0
39.086114836 0.0136268008500338 0.0
22.054277289 -0.0172421187162399 1.0
17.031837547 0.0165599100291729 2.0
-8.5092532605 -0.0460352003574371 3.0
8.5225842768 -0.0371298491954803 4.0
30.563530559 -0.0250404290854931 5.0
39.086114846 -0.317469447851181 6.0
39.086114846 -0.329198896884918 7.0
38.072309142 0.0387933030724525 0.0
19.059692214 -0.044872835278511 1.0
19.012616928 -0.0157881565392017 2.0
-4.5092621515 -0.0493086650967598 3.0
14.503354768 -0.0471374094486237 4.0
23.568954375 -0.0539773926138878 5.0
38.072309151 -0.507426977157593 6.0
38.072309151 -0.372115552425385 7.0
41.585402364 0.035978589206934 0.0
21.878346027 0.00817266292870045 1.0
19.707056337 0.0377081222832203 2.0
-7.856520423 -0.0413884744048119 3.0
11.850535904 -0.0561250075697899 4.0
29.73486646 -0.0407408699393272 5.0
41.585402374 -0.31698477268219 6.0
41.585402374 -0.12131317704916 7.0
42.3173178092696 -0.00304597616195679 0.0
23.2468941623901 0.0382536873221397 1.0
19.0704236443393 0.0212029200047255 2.0
-10.4383552532401 -0.0679594352841377 3.0
8.63206839120634 -0.0178044103085995 4.0
33.6852494163203 -0.0446480736136436 5.0
42.317317809 -0.471923112869263 6.0
42.317317809 0.00699912756681442 7.0
42.931477575 -0.0175524652004242 0.0
22.005453055 0.0289592407643795 1.0
20.926024519 0.0289437044411898 2.0
-5.8784626661 -0.0411980673670769 3.0
15.047561843 -0.0251956805586815 4.0
27.883915732 -0.0256497710943222 5.0
42.931477585 -0.816953241825104 6.0
42.931477585 -0.171273559331894 7.0
46.096274403 0.0133515503257513 0.0
25.32088965 -0.0474396347999573 1.0
20.775384755 -0.0181241109967232 2.0
-5.4662065709 -0.024097453802824 3.0
15.309178177 -0.0826559066772461 4.0
30.787096229 -0.0395466014742851 5.0
46.096274411 0.0437683835625648 6.0
46.096274411 -0.434523642063141 7.0
39.955261544 0.0153319966048002 0.0
19.931762754 0.0283767059445381 1.0
20.02349879 0.0366582944989204 2.0
-8.9110310329 -0.0719461664557457 3.0
11.112467747 0.0110433492809534 4.0
28.842793797 -0.0620010793209076 5.0
39.955261554 -0.337801933288574 6.0
39.955261554 -0.0614997446537018 7.0
45.415961836 0.0573482736945152 0.0
23.097461473 0.00957554765045643 1.0
22.318500363 0.0233232732862234 2.0
-8.5889928185 -0.0532867684960365 3.0
13.729507535 -0.0606549084186554 4.0
31.686454301 -0.0389483198523521 5.0
45.415961846 -0.16955903172493 6.0
45.415961846 -0.175488591194153 7.0
44.275746647 -0.0146600380539894 0.0
22.654009807 0.00417976826429367 1.0
21.62173684 0.0402608104050159 2.0
-7.7547516702 -0.0647008344531059 3.0
13.86698516 -0.0361187979578972 4.0
30.408761488 -0.0646309927105904 5.0
44.275746657 -0.412123024463654 6.0
44.275746657 -0.257621109485626 7.0
34.766238956 0.0129443742334843 0.0
17.943458551 0.0372268855571747 1.0
16.822780406 0.0389610826969147 2.0
-5.6356736314 -0.0387890040874481 3.0
11.187106764 -0.0555916428565979 4.0
23.579132192 -0.0106896683573723 5.0
34.766238966 -0.515183389186859 6.0
34.766238966 -0.187409669160843 7.0
39.697519492335 0.0244477912783623 0.0
21.2162061883195 -0.150190770626068 1.0
18.481313305488 -0.0992643907666206 2.0
-4.67528909066386 -0.0373692587018013 3.0
13.8060242148741 -0.028216190636158 4.0
25.8914952793298 -0.0472244247794151 5.0
39.697519492 -0.380425631999969 6.0
39.697519492 -0.47790539264679 7.0
36.4599525996002 0.040085643529892 0.0
18.3065166475976 -0.105096913874149 1.0
18.1534359526018 -0.0437714233994484 2.0
-5.15104061002709 -0.0488235130906105 3.0
13.0023953426082 -0.0168003961443901 4.0
23.457557257596 -0.0471437647938728 5.0
36.459952599 -0.148527592420578 6.0
36.459952599 -0.492186069488525 7.0
44.856515577 0.0452846959233284 0.0
22.473710733 -0.0317304208874702 1.0
22.382804843 -0.00297778099775314 2.0
-8.0040806232 -0.0396227166056633 3.0
14.37872421 -0.0597831830382347 4.0
30.477791367 -0.0689621344208717 5.0
44.856515586 -0.189900606870651 6.0
44.856515586 -0.396228551864624 7.0
41.1579139562054 0.036126185208559 0.0
18.5564288232053 -0.0646296516060829 1.0
22.601485143205 -0.0172289088368416 2.0
-7.31439675300561 -0.0507326796650887 3.0
15.2870883902055 -0.0228775963187218 4.0
25.8708255762049 -0.0697487518191338 5.0
41.157913956 -0.468858063220978 6.0
41.157913956 -0.355688601732254 7.0
32.092085782 0.0131866820156574 0.0
17.272160967 -0.0663813948631287 1.0
14.819924815 -0.0237710177898407 2.0
-6.6255124723 -0.0372573956847191 3.0
8.1944123328 -0.0451203361153603 4.0
23.897673449 -0.0581251382827759 5.0
32.092085792 -1.07125961780548 6.0
32.092085792 -0.704338908195496 7.0
30.327518415 0.0441619157791138 0.0
13.541286609 -0.003670334815979 1.0
16.786231807 0.0386318415403366 2.0
-5.5133467024 -0.0536324307322502 3.0
11.272885094 -0.0271848738193512 4.0
19.054633321 -0.0415643528103828 5.0
30.327518425 -0.0402799025177956 6.0
30.327518425 -0.304098188877106 7.0
36.482977268 -0.0236172080039978 0.0
19.553260998 0.036774531006813 1.0
16.929716271 0.0381679609417915 2.0
-7.3162847454 -0.0449579358100891 3.0
9.613431516 -0.066955454647541 4.0
26.869545753 -0.059310644865036 5.0
36.482977278 -0.9227055311203 6.0
36.482977278 -0.282854616641998 7.0
40.491170904 0.00147669762372971 0.0
20.509684159 0.0396728701889515 1.0
19.981486745 0.0321230292320251 2.0
-8.9848735316 -0.058633953332901 3.0
10.996613204 -0.0447861105203629 4.0
29.4945577 -0.0289647355675697 5.0
40.491170914 -0.799021363258362 6.0
40.491170914 -0.379526257514954 7.0
41.188898324 -0.114409901201725 0.0
17.482572735 0.0443098992109299 1.0
23.70632559 0.0381044521927834 2.0
-7.86829602 -0.0273598209023476 3.0
15.838029561 -0.0423010215163231 4.0
25.350868764 -0.0143274515867233 5.0
41.188898334 -1.3383195400238 6.0
41.188898334 -0.382336556911469 7.0
37.756949894 0.00394374877214432 0.0
19.043754327 0.0213594734668732 1.0
18.713195567 0.0466979071497917 2.0
-5.5841962186 -0.0589904636144638 3.0
13.128999338 -0.0340212881565094 4.0
24.627950556 -0.060333825647831 5.0
37.756949904 -0.389088749885559 6.0
37.756949904 -0.181131094694138 7.0
26.962645791 0.0453578978776932 0.0
14.231795944 -0.176319479942322 1.0
12.730849849 -0.0826810374855995 2.0
-4.3614404899 -0.0289508178830147 3.0
8.3694093501 -0.042684406042099 4.0
18.593236442 -0.055455394089222 5.0
26.9626458 -0.0540996715426445 6.0
26.9626458 -0.504812180995941 7.0
44.1546808660895 0.0379117056727409 0.0
20.845479755763 -0.0480332300066948 1.0
23.3092011082446 -0.0128673873841763 2.0
-6.97221748920505 -0.035736158490181 3.0
16.3369836194072 -0.0794952884316444 4.0
27.8176971034536 -0.0741001963615417 5.0
44.154680866 -0.0679653659462929 6.0
44.154680866 -0.548070311546326 7.0
36.4992036 0.0300287026911974 0.0
18.499566849 -0.187852919101715 1.0
17.999636751 -0.163050323724747 2.0
-8.164709678 -0.0449751317501068 3.0
9.834927063 -0.0288985148072243 4.0
26.664276537 -0.0280997827649117 5.0
36.49920361 -0.249173581600189 6.0
36.49920361 -0.556239128112793 7.0
36.136186998 0.00251242704689503 0.0
19.096538919 -0.0493427962064743 1.0
17.039648079 0.00183562934398651 2.0
-8.7939144312 -0.0311546623706818 3.0
8.2457336392 -0.0512144640088081 4.0
27.89045336 -0.0304474011063576 5.0
36.136187007 -0.475580811500549 6.0
36.136187007 -0.306123912334442 7.0
43.805905699 0.031564973294735 0.0
19.467800315 0.0340827777981758 1.0
24.338105385 0.0375033318996429 2.0
-10.07714477 -0.060012586414814 3.0
14.260960606 -0.0677684769034386 4.0
29.544945095 -0.0435883775353432 5.0
43.805905709 -0.174519032239914 6.0
43.805905709 -0.0470911338925362 7.0
43.400673471 0.0384170040488243 0.0
21.450145846 -0.0691676884889603 1.0
21.950527627 -0.0290514454245567 2.0
-10.718319514 -0.067328155040741 3.0
11.232208106 -0.0590280741453171 4.0
32.168465368 -0.0813694819808006 5.0
43.400673478 -0.378816545009613 6.0
43.400673478 -0.717455267906189 7.0
35.708898922 -0.0259799808263779 0.0
19.860917727 0.0311456229537725 1.0
15.847981195 0.0350535288453102 2.0
-6.8010445999 -0.0647416412830353 3.0
9.0469365848 -0.0329747498035431 4.0
26.661962337 -0.0854012966156006 5.0
35.708898932 -0.915875792503357 6.0
35.708898932 -0.374882519245148 7.0
42.19537597 -0.00159602239727974 0.0
21.611035583 0.0412402115762234 1.0
20.584340387 0.0292597115039825 2.0
-7.818182446 -0.0614960268139839 3.0
12.766157932 -0.0321671143174171 4.0
29.429218038 -0.0589077770709991 5.0
42.195375979 -0.45931401848793 6.0
42.195375979 -0.0881707742810249 7.0
38.219132248 0.0186933595687151 0.0
18.124133621 0.0283168256282806 1.0
20.094998634 0.0425702258944511 2.0
-3.9675986496 -0.0678679347038269 3.0
16.127399981 -0.059317834675312 4.0
22.091732274 -0.0883785411715508 5.0
38.219132252 -0.341560989618301 6.0
38.219132252 -0.130353599786758 7.0
37.34785903 0.0539383329451084 0.0
19.494183125 0.0319152921438217 1.0
17.853675906 0.0429881177842617 2.0
-4.6743476987 -0.0586477965116501 3.0
13.179328197 -0.0598168894648552 4.0
24.168530833 -0.0474654212594032 5.0
37.34785904 -0.394232034683228 6.0
37.34785904 -0.0688281431794167 7.0
41.413317073 0.0421726703643799 0.0
22.502810393 0.0294196531176567 1.0
18.91050668 0.0308558363467455 2.0
-4.5479233593 -0.0318813771009445 3.0
14.362583311 -0.0720412656664848 4.0
27.050733762 -0.0413965731859207 5.0
41.413317083 -0.14206737279892 6.0
41.413317083 0.0282406695187092 7.0
39.408474003 0.0409170985221863 0.0
19.787258075 0.0373618267476559 1.0
19.621215928 0.0329894088208675 2.0
-8.5467597404 -0.0433010682463646 3.0
11.074456178 -0.0700163170695305 4.0
28.334017825 -0.072429746389389 5.0
39.408474012 -0.53867781162262 6.0
39.408474012 0.0492010489106178 7.0
44.6219477930596 0.0376504808664322 0.0
23.5295769810594 0.0132943876087666 1.0
21.0923708210596 0.0391641519963741 2.0
-9.44677516695988 -0.0509241744875908 3.0
11.6455956540597 -0.0543234273791313 4.0
32.9763521480588 -0.0581609383225441 5.0
44.621947793 -0.207764863967896 6.0
44.621947793 -0.0460867211222649 7.0
35.042314664 0.0449214354157448 0.0
17.377708596 -0.106427021324635 1.0
17.66460607 -0.100858949124813 2.0
-4.7609303256 -0.0510470569133759 3.0
12.903675736 -0.0605852082371712 4.0
22.13863893 -0.0465515926480293 5.0
35.042314673 -0.420044898986816 6.0
35.042314673 -0.384229391813278 7.0
44.668685977 -0.206306427717209 0.0
21.700098462 0.0454387366771698 1.0
22.968587516 0.034297339618206 2.0
-8.4663400465 -0.0268152244389057 3.0
14.50224746 -0.0768599435687065 4.0
30.166438518 -0.0116592459380627 5.0
44.668685987 -0.802750945091248 6.0
44.668685987 -0.062813438475132 7.0
44.271188228 -0.000461943447589874 0.0
21.134860564 -0.061927892267704 1.0
23.136327666 -0.0225412473082542 2.0
-6.9953395415 -0.0282537154853344 3.0
16.140988117 -0.0832410529255867 4.0
28.130200113 -0.0376352444291115 5.0
44.271188235 -0.457073241472244 6.0
44.271188235 -0.347870975732803 7.0
43.3296157145675 0.0458457097411156 0.0
21.8144364805029 0.0184394363313913 1.0
21.5151792385451 0.0388410836458206 2.0
-10.3071776780671 -0.0340952202677727 3.0
11.2080015596363 -0.0761681497097015 4.0
32.1216141588196 -0.0675046443939209 5.0
43.329615715 -0.0084531307220459 6.0
43.329615715 -0.149818688631058 7.0
37.727034885 -0.0320678576827049 0.0
16.740902708 -0.0503512546420097 1.0
20.986132178 -0.0185890570282936 2.0
-10.156378742 -0.0587486028671265 3.0
10.829753428 -0.0659924224019051 4.0
26.897281458 -0.0897417217493057 5.0
37.727034893 -0.864413380622864 6.0
37.727034893 -0.611030101776123 7.0
43.665312103 -0.0493148937821388 0.0
21.432332029 0.00899318605661392 1.0
22.232980074 0.00233099237084389 2.0
-7.2001290055 -0.0539538785815239 3.0
15.032851059 -0.065026082098484 4.0
28.632461044 -0.0320244356989861 5.0
43.665312113 -0.853587865829468 6.0
43.665312113 0.0457298867404461 7.0
40.6563483834553 0.0197696015238762 0.0
20.9212538534144 0.00651375576853752 1.0
19.7350945303589 0.0437163785099983 2.0
-7.59696943049863 -0.0627501532435417 3.0
12.1381250996025 -0.0392291396856308 4.0
28.5182231244438 -0.0592748746275902 5.0
40.656348383 -0.341268599033356 6.0
40.656348383 -0.22330367565155 7.0
30.759139545 -0.00170772895216942 0.0
13.433268738 0.0207201894372702 1.0
17.325870807 0.0456204637885094 2.0
-5.9875785844 -0.0474387481808662 3.0
11.338292213 -0.0773591995239258 4.0
19.420847332 -0.0672819241881371 5.0
30.759139555 -0.286920964717865 6.0
30.759139555 -0.0527705177664757 7.0
36.474229645 -0.0156991556286812 0.0
14.989459641 -0.0277885794639587 1.0
21.484770007 -0.00501667335629463 2.0
-5.0219280476 -0.0217547863721848 3.0
16.462841952 -0.0322116017341614 4.0
20.011387696 -0.00707638263702393 5.0
36.474229652 -1.00103235244751 6.0
36.474229652 -0.591720402240753 7.0
33.340721185 0.0230758748948574 0.0
15.596021276 -0.00905882567167282 1.0
17.744699909 0.0258061345666647 2.0
-7.4570294442 -0.0512359440326691 3.0
10.287670455 -0.0660593062639236 4.0
23.05305073 -0.0814258307218552 5.0
33.340721195 -0.532426297664642 6.0
33.340721195 -0.437651515007019 7.0
34.143349258 0.0278875399380922 0.0
15.417033592 -0.00913995876908302 1.0
18.726315666 0.00316164828836918 2.0
-5.8334622114 -0.00740685313940048 3.0
12.892853445 -0.0614774003624916 4.0
21.250495814 -0.011436752974987 5.0
34.143349268 -0.708887696266174 6.0
34.143349268 0.0211779400706291 7.0
40.763829748 0.0120262950658798 0.0
20.589059268 0.041740745306015 1.0
20.174770481 0.0358161814510822 2.0
-11.800756688 -0.0685833096504211 3.0
8.374013783 -0.0519329681992531 4.0
32.389815966 -0.0447713583707809 5.0
40.763829757 -0.67778491973877 6.0
40.763829757 -0.0631439611315727 7.0
29.788224949 0.0296807549893856 0.0
14.329239328 0.0381401628255844 1.0
15.458985621 0.0309815853834152 2.0
-6.2969425219 -0.0598738715052605 3.0
9.1620430892 -0.0588345751166344 4.0
20.62618186 -0.0610306486487389 5.0
29.788224958 -0.0513161569833755 6.0
29.788224958 -0.290648877620697 7.0
34.675869546 0.0150100961327553 0.0
17.321679699 -0.0396018847823143 1.0
17.354189848 -0.0133070982992649 2.0
-5.026706589 -0.0481890961527824 3.0
12.32748325 -0.0485298186540604 4.0
22.348386296 -0.0371118113398552 5.0
34.675869554 -0.0743059068918228 6.0
34.675869554 -0.456137627363205 7.0
40.7238558279895 -0.000505659729242325 0.0
19.1061702059935 -0.143129408359528 1.0
21.6176856259914 -0.123183570802212 2.0
-8.60840523789577 -0.0335693061351776 3.0
13.0092803879935 -0.0720946118235588 4.0
27.7145752997044 -0.0768553391098976 5.0
40.723855828 -0.37140628695488 6.0
40.723855828 -0.882507801055908 7.0
41.414838344 0.0353512428700924 0.0
21.79665744 -0.049404613673687 1.0
19.618180905 -0.00520555675029755 2.0
-10.700829677 -0.0713887587189674 3.0
8.9173512199 -0.0267147868871689 4.0
32.497487126 -0.0794754847884178 5.0
41.414838352 -0.0416314378380775 6.0
41.414838352 -0.471398413181305 7.0
33.391441204 0.0158781185746193 0.0
16.946662464 0.0193195641040802 1.0
16.44477874 0.0360104292631149 2.0
-6.5811351715 -0.0552391856908798 3.0
9.8636435585 -0.000842045992612839 4.0
23.527797645 -0.0485492274165154 5.0
33.391441214 -0.175154000520706 6.0
33.391441214 -0.452189981937408 7.0
36.660645376 0.0307009499520063 0.0
17.87289688 0.00334222242236137 1.0
18.787748497 -0.00246220082044601 2.0
-6.6425323884 -0.0634918659925461 3.0
12.1452161 -0.0510316491127014 4.0
24.515429278 -0.0371293723583221 5.0
36.660645385 -0.347163379192352 6.0
36.660645385 0.0494976453483105 7.0
33.547567157 0.0146475434303284 0.0
16.605704713 -0.0876878127455711 1.0
16.941862445 -0.0543832778930664 2.0
-8.1253896728 -0.0504325777292252 3.0
8.8164727625 -0.0341066643595695 4.0
24.731094395 -0.0794448480010033 5.0
33.547567166 -0.962101697921753 6.0
33.547567166 -0.724066913127899 7.0
36.717923485 0.0216529350727797 0.0
19.502688668 -0.116446353495121 1.0
17.215234818 -0.0940385088324547 2.0
-6.0032006006 -0.0643620789051056 3.0
11.212034207 -0.0285153724253178 4.0
25.505889278 -0.0118374712765217 5.0
36.717923495 0.0543500669300556 6.0
36.717923495 -0.974186420440674 7.0
44.575737489 0.0430792570114136 0.0
24.24689991 -0.025275643914938 1.0
20.328837584 0.0133531857281923 2.0
-6.0388557339 -0.0619238242506981 3.0
14.289981845 -0.0408931151032448 4.0
30.285755649 -0.054631806910038 5.0
44.575737494 -0.368863880634308 6.0
44.575737494 -0.620966732501984 7.0
43.091610313 0.0454826727509499 0.0
22.329030269 -0.0345123037695885 1.0
20.762580045 -0.00332582741975784 2.0
-7.1251861541 -0.0606040433049202 3.0
13.637393882 -0.0483459159731865 4.0
29.454216432 -0.0597521811723709 5.0
43.091610321 -0.0561566427350044 6.0
43.091610321 -0.304544568061829 7.0
42.647069964 0.0368986167013645 0.0
21.774885881 0.00944393873214722 1.0
20.872184083 0.0368256568908691 2.0
-7.6954983452 -0.0452706143260002 3.0
13.176685728 -0.0402408465743065 4.0
29.470384236 -0.023112379014492 5.0
42.647069973 -0.240408331155777 6.0
42.647069973 -0.202149569988251 7.0
36.24022533 0.0299279056489468 0.0
17.188554164 0.0317372307181358 1.0
19.051671166 0.0298307128250599 2.0
-6.1891251621 -0.0503733679652214 3.0
12.862545995 -0.0630533546209335 4.0
23.377679336 -0.0445699244737625 5.0
36.240225339 -0.215881615877151 6.0
36.240225339 -0.109797455370426 7.0
31.876756195 0.0271597169339657 0.0
14.446454361 0.00227987952530384 1.0
17.430301835 0.0266533475369215 2.0
-5.0545683645 -0.0331083759665489 3.0
12.375733461 -0.0741537287831306 4.0
19.501022735 -0.0547544285655022 5.0
31.876756203 0.0140968821942806 6.0
31.876756203 -0.260087013244629 7.0
38.711049649 0.0423263423144817 0.0
19.068925854 -0.140453666448593 1.0
19.642123794 -0.0843653753399849 2.0
-5.0802825502 -0.0514712333679199 3.0
14.561841234 -0.0724475160241127 4.0
24.149208415 -0.0385587140917778 5.0
38.711049659 -0.103776164352894 6.0
38.711049659 -1.12248015403748 7.0
33.592635662 0.0370645448565483 0.0
16.042625471 0.0318503901362419 1.0
17.550010192 0.0399415232241154 2.0
-5.9813560668 -0.0493297949433327 3.0
11.568654115 -0.0522333905100822 4.0
22.023981548 -0.0565108582377434 5.0
33.592635672 -0.390173077583313 6.0
33.592635672 -0.131703674793243 7.0
31.9394970213296 0.0412381067872047 0.0
15.7221064593445 -0.0461489483714104 1.0
16.2173905613348 -0.00309909880161285 2.0
-3.67606552949996 -0.0423074290156364 3.0
12.5413250313355 -0.0727161541581154 4.0
19.3981719893576 -0.0745234489440918 5.0
31.939497021 -0.0228767283260822 6.0
31.939497021 -0.299278140068054 7.0
32.605677366 0.0419119298458099 0.0
15.053063258 0.0336121506989002 1.0
17.552614114 0.0457029156386852 2.0
-5.3475005701 -0.0501980558037758 3.0
12.20511354 -0.0441466495394707 4.0
20.400563832 -0.0223875008523464 5.0
32.605677369 -0.0228567495942116 6.0
32.605677369 -0.0758798345923424 7.0
35.02561424 0.0449371673166752 0.0
18.375955543 0.031751524657011 1.0
16.649658698 0.0332602672278881 2.0
-5.1216145805 -0.0568042322993279 3.0
11.528044108 -0.0536433383822441 4.0
23.497570133 -0.0312844142317772 5.0
35.02561425 -0.271409928798676 6.0
35.02561425 -0.0503052994608879 7.0
29.155020361 0.0242823585867882 0.0
13.928860816 0.0357009246945381 1.0
15.226159546 0.0270806513726711 2.0
-5.2043069028 -0.0575148090720177 3.0
10.021852634 -0.0391163900494576 4.0
19.133167728 -0.067986823618412 5.0
29.155020371 -0.145381569862366 6.0
29.155020371 -0.0148067809641361 7.0
29.979099891 0.00663774646818638 0.0
16.392071584 0.0322036892175674 1.0
13.587028308 0.0393965542316437 2.0
-4.5603535805 -0.0601035952568054 3.0
9.026674718 -0.0621815696358681 4.0
20.952425173 -0.0629571154713631 5.0
29.9790999 0.0508249215781689 6.0
29.9790999 -0.692238569259644 7.0
40.412374419 0.0436281263828278 0.0
21.632869342 -0.105397962033749 1.0
18.779505078 -0.079868420958519 2.0
-4.6354756287 -0.0382425487041473 3.0
14.14402944 -0.0582858249545097 4.0
26.26834498 -0.0392607972025871 5.0
40.412374428 -0.0292948856949806 6.0
40.412374428 -0.733800292015076 7.0
44.7996696066759 0.0194179639220238 0.0
24.3565352806759 -0.120885454118252 1.0
20.4431343366759 -0.0627522096037865 2.0
-4.33926525477587 -0.0343588069081306 3.0
16.1038690816759 -0.04006577283144 4.0
28.6958005009671 -0.0704167559742928 5.0
44.799669607 -0.0113866962492466 6.0
44.799669607 -0.576400399208069 7.0
35.691930078 0.0292297415435314 0.0
18.173171081 0.000737356022000313 1.0
17.518759 0.0200663469731808 2.0
-4.3926482071 -0.0277392193675041 3.0
13.126110785 -0.0842600911855698 4.0
22.565819295 -0.0538321509957314 5.0
35.691930085 -0.249404579401016 6.0
35.691930085 -0.392779409885406 7.0
37.293002373 -0.0446214750409126 0.0
17.018201783 0.0278812199831009 1.0
20.274800591 0.017226742580533 2.0
-8.9983650782 -0.04693952947855 3.0
11.276435503 -0.0563880056142807 4.0
26.016566871 -0.0454554259777069 5.0
37.293002382 -0.538058280944824 6.0
37.293002382 0.0462939105927944 7.0
31.749050626 0.0671884715557098 0.0
16.61083934 0.0334083437919617 1.0
15.138211288 0.0349057987332344 2.0
-3.9083704158 -0.07121741771698 3.0
11.229840863 -0.0163118802011013 4.0
20.519209764 -0.0275376886129379 5.0
31.749050635 -0.113008372485638 6.0
31.749050635 -0.108994893729687 7.0
44.43859931 -0.129178643226624 0.0
22.567248059 -0.0758409276604652 1.0
21.871351255 -0.0536008849740028 2.0
-10.274790847 -0.038813903927803 3.0
11.596560402 -0.0702027454972267 4.0
32.842038912 -0.0314596593379974 5.0
44.438599316 -0.521636962890625 6.0
44.438599316 -0.298858106136322 7.0
30.584762336 0.0388638526201248 0.0
14.448986708 0.0337464213371277 1.0
16.13577563 0.0333121344447136 2.0
-7.0107275854 -0.0380796864628792 3.0
9.1250480359 -0.0739270523190498 4.0
21.459714302 -0.0356901362538338 5.0
30.584762345 0.0060303695499897 6.0
30.584762345 -0.459355890750885 7.0
36.167711461 0.049235176295042 0.0
15.702748451 0.033476434648037 1.0
20.464963013 0.0428502634167671 2.0
-4.5943051912 -0.0225496217608452 3.0
15.870657815 -0.0741827115416527 4.0
20.297053649 -0.0453729033470154 5.0
36.167711467 -0.496149063110352 6.0
36.167711467 -0.0394425541162491 7.0
45.589023155 0.047964833676815 0.0
22.43199889 0.031599223613739 1.0
23.157024265 0.0281253848224878 2.0
-9.6703339461 -0.0366258099675179 3.0
13.486690309 -0.0355980098247528 4.0
32.102332846 -0.028739221394062 5.0
45.589023165 -0.221017718315125 6.0
45.589023165 0.0145274940878153 7.0
33.954863965 0.045533400028944 0.0
16.569146941 0.035981260240078 1.0
17.385717024 0.0386103242635727 2.0
-8.6921868392 -0.0605257973074913 3.0
8.6935301746 -0.0304661318659782 4.0
25.26133379 -0.0372470691800117 5.0
33.954863975 -0.00936178490519524 6.0
33.954863975 -0.360605984926224 7.0
39.104021209 0.0454395413398743 0.0
19.726585997 0.0116766579449177 1.0
19.377435214 0.0386874042451382 2.0
-5.8707549184 -0.0451090931892395 3.0
13.506680288 -0.0491153374314308 4.0
25.597340923 -0.040258601307869 5.0
39.104021217 -0.0963221564888954 6.0
39.104021217 -0.290283471345901 7.0
41.6151393745108 0.0239913370460272 0.0
21.6144474948183 -0.133388131856918 1.0
20.0006918800463 -0.0707740634679794 2.0
-5.68615976863137 -0.0565975606441498 3.0
14.3145321112399 -0.0397472307085991 4.0
27.3006071309532 -0.0705550238490105 5.0
41.615139375 -0.331534028053284 6.0
41.615139375 -0.810584306716919 7.0
29.251209159 0.00341063737869263 0.0
16.305978933 0.03205781057477 1.0
12.945230226 0.0211697742342949 2.0
-4.8421763952 -0.0600443109869957 3.0
8.1030538205 -0.0457282364368439 4.0
21.148155339 -0.0385666042566299 5.0
29.251209169 -0.630758762359619 6.0
29.251209169 -0.0416149795055389 7.0
39.145261593 0.0190745629370213 0.0
20.853611949 -0.16917696595192 1.0
18.291649647 -0.164510607719421 2.0
-6.3281983222 -0.0695837810635567 3.0
11.963451316 -0.0297097414731979 4.0
27.181810279 -0.0633495599031448 5.0
39.145261601 -0.200102239847183 6.0
39.145261601 -0.655573546886444 7.0
43.724223504 0.0194605253636837 0.0
24.240212841 -0.0398679673671722 1.0
19.484010663 -0.0132566168904305 2.0
-6.7543223688 -0.0334180071949959 3.0
12.729688284 -0.0428884997963905 4.0
30.99453522 -0.0500422492623329 5.0
43.724223514 -0.442178070545197 6.0
43.724223514 -0.453926026821136 7.0
30.694427938 0.0563277825713158 0.0
15.962392584 -0.0449906140565872 1.0
14.732035355 -0.00461433082818985 2.0
-5.8529078341 -0.0664489567279816 3.0
8.8791275119 -0.0307732746005058 4.0
21.815300427 -0.0650650560855865 5.0
30.694427948 -0.202417880296707 6.0
30.694427948 -0.372752010822296 7.0
46.899446071 -0.0594959184527397 0.0
22.810606072 -0.0543573796749115 1.0
24.08884 -0.0237626358866692 2.0
-12.207831656 -0.0462024956941605 3.0
11.881008335 -0.0792135372757912 4.0
35.018437738 -0.0561800226569176 5.0
46.899446079 -0.770863592624664 6.0
46.899446079 -0.860498547554016 7.0
30.801741864 0.0153788886964321 0.0
14.581609416 0.0360695980489254 1.0
16.220132448 0.0413324162364006 2.0
-7.6078809929 -0.0109411589801311 3.0
8.6122514454 -0.0794901102781296 4.0
22.189490419 -0.0153928734362125 5.0
30.801741874 -0.379544824361801 6.0
30.801741874 -0.0213099345564842 7.0
40.3889007570329 0.0374343171715736 0.0
21.8864546730297 0.0319988057017326 1.0
18.5024460860314 0.0348533242940903 2.0
-8.35977776734346 -0.055649496614933 3.0
10.1426683190367 -0.0642454400658607 4.0
30.2462322648819 -0.0772862657904625 5.0
40.388900757 -0.147951513528824 6.0
40.388900757 -0.0625393986701965 7.0
35.801686121 0.0406158789992332 0.0
19.152651215 -0.136203467845917 1.0
16.649034906 -0.0845263376832008 2.0
-1.2326560583 -0.0749367102980614 3.0
15.416378838 -0.0187017694115639 4.0
20.385307283 -0.0841790363192558 5.0
35.801686131 0.0251018069684505 6.0
35.801686131 -0.515821516513824 7.0
43.275430292 0.0332761406898499 0.0
21.49768275 -0.062163382768631 1.0
21.777747542 -0.037589468061924 2.0
-8.5057364073 -0.0353019088506699 3.0
13.272011125 -0.0484748408198357 4.0
30.003419167 -0.0318470224738121 5.0
43.275430302 -0.289263904094696 6.0
43.275430302 -0.580131530761719 7.0
36.6745338859035 0.0391339063644409 0.0
17.2258453439028 0.00314655527472496 1.0
19.4486885479032 0.04186150431633 2.0
-7.45674864010256 -0.0580838695168495 3.0
11.9919399079021 -0.0281209573149681 4.0
24.6825939849038 -0.0507440865039825 5.0
36.674533885 -0.164471477270126 6.0
36.674533885 -0.240570783615112 7.0
42.2919299411196 0.0265255142003298 0.0
20.8701371011196 0.0230023618787527 1.0
21.4217928491193 0.0385902747511864 2.0
-7.8990978219197 -0.0764107778668404 3.0
13.5226950281196 -0.0192489996552467 4.0
28.769234923119 -0.0739177167415619 5.0
42.291929941 -0.308276116847992 6.0
42.291929941 -0.0730810090899467 7.0
37.80686854 -0.00108707696199417 0.0
18.84529386 -0.00602706521749496 1.0
18.961574682 0.0276139192283154 2.0
-3.0668288295 -0.068157896399498 3.0
15.894745843 -0.025709256529808 4.0
21.912122698 -0.0167284607887268 5.0
37.80686855 -0.367038518190384 6.0
37.80686855 -0.468896806240082 7.0
36.296445798 0.0149476379156113 0.0
19.144127702 -0.00757910311222076 1.0
17.152318097 0.0339386649429798 2.0
-4.4799623055 -0.0463962629437447 3.0
12.672355782 -0.0493862852454185 4.0
23.624090017 -0.0481656417250633 5.0
36.296445808 -0.285035967826843 6.0
36.296445808 -0.211645305156708 7.0
26.786347975 -0.0735660195350647 0.0
13.739106858 0.0214489363133907 1.0
13.047241117 0.00260877422988415 2.0
-5.0192652199 -0.0573614239692688 3.0
8.0279758877 -0.0311456024646759 4.0
18.758372088 -0.0592243447899818 5.0
26.786347985 -1.31354546546936 6.0
26.786347985 -0.0501759648323059 7.0
38.715518927 0.0450295209884644 0.0
20.971478582 0.0369489341974258 1.0
17.744040345 0.0426885411143303 2.0
-7.5775969749 -0.0120266601443291 3.0
10.16644336 -0.0507947728037834 4.0
28.549075567 -0.0415381714701653 5.0
38.715518937 0.0275772698223591 6.0
38.715518937 -0.309226393699646 7.0
35.505062995 -0.0016021803021431 0.0
16.975132375 0.033732932060957 1.0
18.529930624 0.0249022673815489 2.0
-6.1934391103 -0.0496753752231598 3.0
12.336491507 -0.0549734607338905 4.0
23.168571492 -0.0479929149150848 5.0
35.505063002 -0.380510836839676 6.0
35.505063002 -0.000774487853050232 7.0
44.227649061 0.0294757541269064 0.0
23.455343466 -0.0351094454526901 1.0
20.772305597 4.33474779129028e-05 2.0
-7.272984789 -0.0647551789879799 3.0
13.499320798 -0.0288832560181618 4.0
30.728328264 -0.0203356444835663 5.0
44.227649071 -0.030811183154583 6.0
44.227649071 -0.910727024078369 7.0
45.773124246 0.0397663414478302 0.0
24.134680774 -0.0107479393482208 1.0
21.638443473 0.0200369376689196 2.0
-6.2883820036 -0.035496786236763 3.0
15.350061461 -0.0674660876393318 4.0
30.423062786 -0.0391784831881523 5.0
45.773124255 -0.110670693218708 6.0
45.773124255 -0.143285006284714 7.0
38.6152827522378 0.048015683889389 0.0
18.6636500942378 0.0190253984183073 1.0
19.9516326682378 0.0091619398444891 2.0
-8.33253161363777 -0.0541369542479515 3.0
11.6191010542378 -0.0672301948070526 4.0
26.9961816644691 -0.0772991329431534 5.0
38.615282752 -0.388342440128326 6.0
38.615282752 -0.0677506774663925 7.0
42.131124826 0.0353062674403191 0.0
21.88265512 0.0196663737297058 1.0
20.248469706 0.000220226123929024 2.0
-6.067617241 -0.0676527246832848 3.0
14.180852455 -0.00379985570907593 4.0
27.950272371 -0.0388520434498787 5.0
42.131124836 -0.229885876178741 6.0
42.131124836 -0.0167781487107277 7.0
38.030369756 0.0476259849965572 0.0
19.698625144 0.0205685291439295 1.0
18.331744611 0.0407779216766357 2.0
-8.0645153211 -0.0455101951956749 3.0
10.26722928 -0.0576887354254723 4.0
27.763140475 -0.074946440756321 5.0
38.030369766 -0.316284656524658 6.0
38.030369766 -0.244773656129837 7.0
32.694869995 0.0379143580794334 0.0
14.429732866 0.0134984087198973 1.0
18.265137131 0.0343946516513824 2.0
-5.8301144142 -0.0523694455623627 3.0
12.435022709 -0.0768027231097221 4.0
20.259847288 -0.0314165987074375 5.0
32.694870002 -0.0812463015317917 6.0
32.694870002 -0.328014492988586 7.0
41.58883183 0.0415440537035465 0.0
19.184872754 0.031100932508707 1.0
22.403959077 0.0262391865253448 2.0
-6.437067104 -0.0470409467816353 3.0
15.966891964 -0.0651161596179008 4.0
25.621939867 -0.058282770216465 5.0
41.58883184 -0.726355075836182 6.0
41.58883184 0.0540831834077835 7.0
33.622063035 -0.0332213640213013 0.0
16.212159516 -0.0324840471148491 1.0
17.409903519 -0.00613262876868248 2.0
-8.2486263793 -0.0471136569976807 3.0
9.16127713 -0.0690039098262787 4.0
24.460785905 -0.0661820024251938 5.0
33.622063045 -0.826356172561646 6.0
33.622063045 -0.350058108568192 7.0
40.101738497 0.0402558334171772 0.0
22.613544165 0.0430605001747608 1.0
17.488194333 0.0366267338395119 2.0
-8.294425066 -0.048090897500515 3.0
9.1937692583 -0.0215891487896442 4.0
30.90796924 -0.0147299207746983 5.0
40.101738506 -0.0126112103462219 6.0
40.101738506 -0.0844502225518227 7.0
39.347750484 0.0137348081916571 0.0
19.631600055 0.0178337506949902 1.0
19.716150429 0.0495990216732025 2.0
-7.2110260491 -0.0474440976977348 3.0
12.50512437 -0.0702125206589699 4.0
26.842626114 -0.0812314972281456 5.0
39.347750494 -0.359071135520935 6.0
39.347750494 -0.00729509443044662 7.0
46.240008453 0.0175799895077944 0.0
23.277257726 0.0397533178329468 1.0
22.962750727 0.0307375565171242 2.0
-6.5077704351 -0.0635449960827827 3.0
16.454980282 -0.0431773513555527 4.0
29.785028171 -0.0688731670379639 5.0
46.240008463 -0.374597012996674 6.0
46.240008463 -0.119456894695759 7.0
43.885613963 -0.0256941094994545 0.0
23.043389811 0.025001497939229 1.0
20.842224153 0.0226894300431013 2.0
-5.1243468199 -0.0423702225089073 3.0
15.717877324 -0.0531255826354027 4.0
28.16773664 -0.0435727387666702 5.0
43.885613972 -0.224503874778748 6.0
43.885613972 -0.0227279774844646 7.0
44.877286045 0.0380213186144829 0.0
23.85937511 -0.109241105616093 1.0
21.017910935 -0.0953859016299248 2.0
-10.548874854 -0.0336078256368637 3.0
10.469036071 -0.035503201186657 4.0
34.408249974 -0.0399357602000237 5.0
44.877286055 -0.0185745283961296 6.0
44.877286055 -0.506370484828949 7.0
44.28102484 0.0584516748785973 0.0
23.476033214 0.0157477799803019 1.0
20.804991626 0.0494015924632549 2.0
-7.5751269732 -0.066639356315136 3.0
13.229864644 -0.0366295948624611 4.0
31.051160197 -0.0580744817852974 5.0
44.281024849 -0.210607200860977 6.0
44.281024849 -0.0769361034035683 7.0
39.443195324 0.0273469351232052 0.0
17.235953612 -0.136932134628296 1.0
22.207241712 -0.0827657580375671 2.0
-9.075573722 -0.023043155670166 3.0
13.13166798 -0.0428328737616539 4.0
26.311527344 -0.0178252756595612 5.0
39.443195334 0.00129193067550659 6.0
39.443195334 -0.947677552700043 7.0
44.5099801195094 -0.186581790447235 0.0
20.7192442195094 0.034106969833374 1.0
23.7907359105096 0.0378242805600166 2.0
-9.37764966100969 -0.0463080033659935 3.0
14.4130862495096 -0.0453982278704643 4.0
30.0968938805093 -0.0541160181164742 5.0
44.50998012 -0.74290931224823 6.0
44.50998012 -0.0458881333470345 7.0
36.079576813 0.0216715298593044 0.0
18.198560719 -0.0189828053116798 1.0
17.881016095 0.0190425701439381 2.0
-8.6789161085 -0.0613208934664726 3.0
9.202099978 -0.0274447612464428 4.0
26.877476836 -0.0397903770208359 5.0
36.079576822 -0.329664707183838 6.0
36.079576822 -0.14914882183075 7.0
43.881758866 0.0531589537858963 0.0
21.090009056 0.0124668926000595 1.0
22.79174981 0.0496449247002602 2.0
-6.6710954927 -0.0659683123230934 3.0
16.120654308 -0.0298447161912918 4.0
27.761104558 -0.0741427913308144 5.0
43.881758876 -0.0512644425034523 6.0
43.881758876 -0.0226159729063511 7.0
30.108818274 -0.139139145612717 0.0
17.112174764 0.040771022439003 1.0
12.996643511 0.0169027913361788 2.0
-4.8964528045 -0.079413115978241 3.0
8.1001906976 0.0129646062850952 4.0
22.008627577 -0.0545546784996986 5.0
30.108818284 -0.773232460021973 6.0
30.108818284 -0.0238148495554924 7.0
34.2205121900012 0.0123198442161083 0.0
18.8110553650029 0.01484165340662 1.0
15.4094568300067 0.0466829389333725 2.0
-7.05168531729669 -0.0588541030883789 3.0
8.35777151259514 -0.0405597239732742 4.0
25.8627406830075 -0.0663212537765503 5.0
34.22051219 -0.37538069486618 6.0
34.22051219 -0.167291224002838 7.0
37.402197962 -0.0160498060286045 0.0
19.960548139 -0.0311234071850777 1.0
17.441649823 -0.00850758701562881 2.0
-5.9560955294 -0.0296746715903282 3.0
11.485554285 -0.0755530148744583 4.0
25.916643678 -0.0523041114211082 5.0
37.402197971 -0.502051830291748 6.0
37.402197971 -0.371624618768692 7.0
35.453722399 0.0245228614658117 0.0
16.625051972 -0.0184239819645882 1.0
18.828670427 0.00572595745325089 2.0
-7.5918246106 -0.0271308571100235 3.0
11.236845807 -0.0786810293793678 4.0
24.216876592 -0.0604655891656876 5.0
35.453722409 -0.139430850744247 6.0
35.453722409 -0.410816490650177 7.0
32.872381145 0.0045931302011013 0.0
15.679111706 0.0365834757685661 1.0
17.193269439 0.0187786966562271 2.0
-6.3769976884 -0.057191364467144 3.0
10.816271741 -0.0164934769272804 4.0
22.056109405 -0.0617453381419182 5.0
32.872381155 -0.49532151222229 6.0
32.872381155 -0.0303388126194477 7.0
33.5777143162235 0.0461070761084557 0.0
24.3043372519187 0.0243756864219904 1.0
9.27337706863705 0.0411293283104897 2.0
-1.413226005e-07 -0.0629880800843239 3.0
9.27337692732672 -0.0494924038648605 4.0
24.3043373936238 -0.0562476068735123 5.0
33.577714316 -0.0354532450437546 6.0
33.577714316 -0.610296785831451 7.0
43.986168792 0.0161185786128044 0.0
19.73252431 -0.000315491110086441 1.0
24.253644482 0.00496954657137394 2.0
-7.8587048092 -0.0195323638617992 3.0
16.394939663 -0.0555857196450233 4.0
27.591229129 0.0144189037382603 5.0
43.986168802 -0.39304792881012 6.0
43.986168802 0.0507133491337299 7.0
39.701527196 0.0518832355737686 0.0
21.433905404 0.0140637289732695 1.0
18.267621793 0.0471869744360447 2.0
-6.9530592881 -0.0602233186364174 3.0
11.314562495 -0.0396337434649467 4.0
28.386964701 -0.0652826726436615 5.0
39.701527205 -0.122670225799084 6.0
39.701527205 -0.0533735677599907 7.0
36.319422307 0.0289317313581705 0.0
18.261170173 0.0381914675235748 1.0
18.058252134 0.0353585667908192 2.0
-9.3076661868 -0.0368011966347694 3.0
8.7505859373 -0.0482951328158379 4.0
27.56883637 -0.0540117546916008 5.0
36.319422317 -0.157881289720535 6.0
36.319422317 -0.114914886653423 7.0
39.546529711 0.0258343126624823 0.0
21.883828824 0.0127948615700006 1.0
17.662700888 0.0354590639472008 2.0
-6.2516235274 -0.038110189139843 3.0
11.411077352 -0.0690771862864494 4.0
28.13545236 -0.0481412336230278 5.0
39.54652972 -0.298800081014633 6.0
39.54652972 -0.118358187377453 7.0
46.328884399 0.0303061883896589 0.0
20.182008475 -0.0493000820279121 1.0
26.146875933 -0.00295168161392212 2.0
-10.473669833 -0.0508612990379333 3.0
15.673206099 -0.0478606522083282 4.0
30.655678309 -0.0587110668420792 5.0
46.3288844 -0.253038555383682 6.0
46.3288844 -0.362861067056656 7.0
39.359671453 0.0561783611774445 0.0
20.208221774 -0.0147736892104149 1.0
19.151449681 -0.00335805863142014 2.0
-7.7565035076 -0.0419585928320885 3.0
11.394946165 -0.0496842935681343 4.0
27.964725289 0.00111869163811207 5.0
39.359671461 -0.185373991727829 6.0
39.359671461 -0.189534217119217 7.0
43.813870978 0.0464143231511116 0.0
22.861832963 0.0347784981131554 1.0
20.952038016 0.0416088737547398 2.0
-6.6064627953 -0.0535899624228477 3.0
14.345575211 -0.063139945268631 4.0
29.468295768 -0.0473154336214066 5.0
43.813870988 -0.25362241268158 6.0
43.813870988 -0.484072327613831 7.0
39.146221866 0.0287702847272158 0.0
21.274977313 0.0309901461005211 1.0
17.871244553 0.0310897678136826 2.0
-8.221666714 -0.0524502694606781 3.0
9.6495778292 -0.0585688352584839 4.0
29.496644037 -0.0147918723523617 5.0
39.146221876 -0.255642652511597 6.0
39.146221876 0.0258503369987011 7.0
34.753353004 -0.0583089664578438 0.0
19.306587872 -0.115402229130268 1.0
15.446765133 -0.0667014047503471 2.0
-7.4077231802 -0.0175822079181671 3.0
8.0390419425 -0.0408850684762001 4.0
26.714311062 -0.059306763112545 5.0
34.753353014 -1.08443379402161 6.0
34.753353014 -0.993342280387878 7.0
36.8589150083515 0.0460368618369102 0.0
18.5606086093515 0.00915817730128765 1.0
18.2983064083517 0.0445416457951069 2.0
-8.423068234052 -0.0458424314856529 3.0
9.87523817465195 -0.0332078337669373 4.0
26.9836767778432 -0.0736252963542938 5.0
36.858915008 -0.0203302353620529 6.0
36.858915008 -0.108746849000454 7.0
45.609388715 0.0371719710528851 0.0
20.431056468 0.0405051037669182 1.0
25.178332247 0.0494216531515121 2.0
-8.8685199312 -0.0396896675229073 3.0
16.309812306 -0.0565603524446487 4.0
29.299576409 -0.0179744437336922 5.0
45.609388725 -0.183114767074585 6.0
45.609388725 -0.00411228090524673 7.0
37.221502305 -0.0219251476228237 0.0
19.795691461 0.0279597099870443 1.0
17.425810844 0.0230675172060728 2.0
-8.0004149655 -0.0404452160000801 3.0
9.4253958691 -0.0501357391476631 4.0
27.796106436 -0.0393300354480743 5.0
37.221502314 -0.737741708755493 6.0
37.221502314 -0.0764943584799767 7.0
39.3512418478291 -0.00406723469495773 0.0
21.5003115557496 0.0211812481284142 1.0
17.8509302889164 0.0426379814743996 2.0
-7.9913457065862 -0.0383897423744202 3.0
9.85958458276425 -0.0737505182623863 4.0
29.4916572628094 -0.0651057437062263 5.0
39.351241848 -0.315747916698456 6.0
39.351241848 -0.0643931403756142 7.0
44.354286129 0.00569574721157551 0.0
23.778468789 0.0266244411468506 1.0
20.57581734 0.0443921536207199 2.0
-6.7304183908 -0.0307711809873581 3.0
13.845398939 -0.0316522195935249 4.0
30.50888719 -0.027744933962822 5.0
44.354286139 -0.269947707653046 6.0
44.354286139 -0.243994146585464 7.0
41.1438439687938 0.0348237603902817 0.0
20.0722524531884 0.0104492250829935 1.0
21.0715915195 0.0399825498461723 2.0
-7.89338141352269 -0.0539875030517578 3.0
13.1782101064826 -0.0600403323769569 4.0
27.9656338666077 -0.0528181865811348 5.0
41.143843969 0.0104867815971375 6.0
41.143843969 -0.210739374160767 7.0
32.436363662 -0.00466299802064896 0.0
18.29560079 0.0274745244532824 1.0
14.140762872 0.0530573837459087 2.0
-4.3766657499 -0.0395492836833 3.0
9.7640971121 -0.0317835658788681 4.0
22.67226655 -0.0664134249091148 5.0
32.436363672 -0.31610894203186 6.0
32.436363672 0.012742143124342 7.0
35.980268817 0.0117108915001154 0.0
18.371260578 -0.192177027463913 1.0
17.609008241 -0.178969204425812 2.0
-7.5074447835 -0.0400704592466354 3.0
10.101563449 -0.0593343749642372 4.0
25.87870537 -0.0346671268343925 5.0
35.980268826 -0.65300714969635 6.0
35.980268826 -1.27952432632446 7.0
39.361696941 0.0463416650891304 0.0
18.798616943 0.00929184630513191 1.0
20.563079997 0.0446576997637749 2.0
-4.5015726852 -0.0520544722676277 3.0
16.061507302 -0.0328521803021431 4.0
23.300189639 -0.0461690053343773 5.0
39.361696951 0.043909315019846 6.0
39.361696951 -0.201144456863403 7.0
38.339957491 -0.00294380635023117 0.0
20.325182647 0.0326989144086838 1.0
18.014774844 0.020635262131691 2.0
-6.00621 -0.0429932922124863 3.0
12.008564835 -0.0514547601342201 4.0
26.331392656 -0.0701330974698067 5.0
38.339957501 -0.545904994010925 6.0
38.339957501 -0.0231463089585304 7.0
38.4370780266379 0.0180586837232113 0.0
19.1112284596379 0.0347723662853241 1.0
19.3258495766379 0.0408600457012653 2.0
-8.41787501523793 -0.0557191893458366 3.0
10.9079745616379 -0.064605638384819 4.0
27.5291034382175 -0.0635936334729195 5.0
38.437078026 -0.504951477050781 6.0
38.437078026 -0.517238795757294 7.0
31.680301152 0.0450367629528046 0.0
16.922816031 -0.065446101129055 1.0
14.757485122 -0.0122943408787251 2.0
-6.2787127987 -0.0556532517075539 3.0
8.4787723148 -0.0173956975340843 4.0
23.201528838 -0.0598513707518578 5.0
31.680301161 -0.158900797367096 6.0
31.680301161 -0.734604835510254 7.0
36.698769122 0.0738632753491402 0.0
19.594432034 -0.0749520286917686 1.0
17.104337088 -0.0259971991181374 2.0
-2.472193454 -0.0246025994420052 3.0
14.632143624 -0.0696694925427437 4.0
22.066625498 -0.0601075813174248 5.0
36.698769132 -0.0336444601416588 6.0
36.698769132 -0.31190150976181 7.0
40.942334912 0.0530637949705124 0.0
22.219375241 0.0337111502885818 1.0
18.722959671 0.0338204875588417 2.0
-5.2427967213 -0.0499864369630814 3.0
13.48016294 -0.0418645590543747 4.0
27.462171973 -0.0757602229714394 5.0
40.942334922 -0.217929810285568 6.0
40.942334922 -0.0222104862332344 7.0
41.496981527 0.0247412361204624 0.0
18.951043784 -0.0306675024330616 1.0
22.545937743 0.0227099657058716 2.0
-7.7537930795 -0.0760406777262688 3.0
14.792144654 -0.00424061715602875 4.0
26.704836873 -0.0867549255490303 5.0
41.496981537 -0.373930692672729 6.0
41.496981537 -0.394791603088379 7.0
33.577466055 -0.0152299031615257 0.0
17.755510174 0.022450415417552 1.0
15.821955881 0.0122185274958611 2.0
-7.358985858 -0.050390861928463 3.0
8.4629700135 -0.0449477061629295 4.0
25.114496042 -0.0695130452513695 5.0
33.577466065 -0.712928414344788 6.0
33.577466065 0.039339255541563 7.0
38.410697594746 0.0269937515258789 0.0
20.0282722997678 0.0387410894036293 1.0
18.3824252977607 0.041601374745369 2.0
-2.25462323941058 -0.0329690203070641 3.0
16.1278020587646 -0.0617784485220909 4.0
22.2828955387576 -0.0560394302010536 5.0
38.410697595 -0.435017943382263 6.0
38.410697595 -0.159393817186356 7.0
40.390522102 0.0422522947192192 0.0
23.101316011 -0.0636272206902504 1.0
17.289206093 -0.0304562002420425 2.0
-8.5131515481 -0.0623565837740898 3.0
8.7760545352 -0.0504121854901314 4.0
31.614467568 -0.0473208129405975 5.0
40.390522112 0.00854658707976341 6.0
40.390522112 -0.389117479324341 7.0
38.02724698 0.034861795604229 0.0
17.092159812 0.0194559898227453 1.0
20.935087168 0.0401287116110325 2.0
-7.0096758313 -0.0780210494995117 3.0
13.925411327 0.0106471348553896 4.0
24.101835653 -0.0678070783615112 5.0
38.02724699 -0.0204877071082592 6.0
38.02724699 -0.119895629584789 7.0
34.831322289 -0.120148710906506 0.0
17.981428364 0.0385380275547504 1.0
16.849893928 0.0340707749128342 2.0
-5.7085713019 -0.0519333556294441 3.0
11.14132262 -0.0211272165179253 4.0
23.689999673 -0.0280231982469559 5.0
34.831322295 -0.553763031959534 6.0
34.831322295 -0.0275328531861305 7.0
33.152476822 0.0257619079202414 0.0
16.395481675 0.0383623614907265 1.0
16.756995148 0.0354451537132263 2.0
-6.4008279506 -0.0253609269857407 3.0
10.356167188 -0.0871413946151733 4.0
22.796309635 -0.0630840584635735 5.0
33.152476832 -0.552077353000641 6.0
33.152476832 -0.629325449466705 7.0
37.7618147597387 0.00843229703605175 0.0
20.5714887837879 -0.0348461344838142 1.0
17.1903259800247 -0.0266869999468327 2.0
-5.43010176849644 -0.0261791683733463 3.0
11.7602242121808 -0.07997677475214 4.0
26.0015905517818 -0.029135949909687 5.0
37.76181476 -0.241385579109192 6.0
37.76181476 -0.347180783748627 7.0
46.821650957 -0.00269975513219833 0.0
22.453957792 -0.00666623190045357 1.0
24.367693165 0.0298597496002913 2.0
-10.797760543 -0.0544396564364433 3.0
13.569932612 -0.0497033149003983 4.0
33.251718344 -0.063508503139019 5.0
46.821650967 -0.386086016893387 6.0
46.821650967 -0.159338563680649 7.0
41.1837901754312 0.0523503646254539 0.0
19.2503118354314 0.0378728285431862 1.0
21.9334783494314 0.0354237630963326 2.0
-7.5443023473312 -0.0552926883101463 3.0
14.3891760014314 -0.0678922012448311 4.0
26.7946141824314 -0.0309577211737633 5.0
41.183790175 -0.149702697992325 6.0
41.183790175 -0.108513198792934 7.0
36.784988632 0.065153032541275 0.0
21.035700431 0.00362491793930531 1.0
15.749288201 0.0196780525147915 2.0
-6.0648376105 -0.0460089817643166 3.0
9.6844505818 -0.0727891847491264 4.0
27.100538051 -0.0343483686447144 5.0
36.78498864 -0.129453301429749 6.0
36.78498864 -0.143798470497131 7.0
37.6894949442102 -0.00506208464503288 0.0
17.6138134052102 0.0200803279876709 1.0
20.0756815502102 0.0353024676442146 2.0
-8.78806436921022 -0.0603085532784462 3.0
11.2876171802102 -0.0571382120251656 4.0
26.401877737898 -0.0769529566168785 5.0
37.689494945 -0.856822848320007 6.0
37.689494945 -0.395802319049835 7.0
41.308881088 -0.0177287049591541 0.0
21.148817904 0.0167484600096941 1.0
20.160063185 0.0520186871290207 2.0
-6.8065142938 -0.0712455734610558 3.0
13.353548882 -0.0167438127100468 4.0
27.955332207 -0.0437614321708679 5.0
41.308881097 -0.188931822776794 6.0
41.308881097 -0.160973757505417 7.0
42.253011175 0.0216627456247807 0.0
21.557015891 0.0339687243103981 1.0
20.695995284 0.0394488349556923 2.0
-9.3835381575 -0.0353001356124878 3.0
11.312457118 -0.0666205808520317 4.0
30.940554058 -0.0157681927084923 5.0
42.253011184 0.00816989690065384 6.0
42.253011184 -0.016807671636343 7.0
40.047660483 0.0240809880197048 0.0
19.392839472 -0.0389183759689331 1.0
20.654821011 0.00219653733074665 2.0
-7.4809136797 -0.0576306581497192 3.0
13.173907322 -0.0391377136111259 4.0
26.873753161 -0.0703042820096016 5.0
40.047660493 -0.483590841293335 6.0
40.047660493 -0.678282916545868 7.0
46.604605723 -0.0526993200182915 0.0
22.954120862 -0.0894709900021553 1.0
23.650484861 -0.0733283683657646 2.0
-7.9204617962 -0.0492031052708626 3.0
15.730023055 -0.055500864982605 4.0
30.874582668 -0.0582521557807922 5.0
46.604605733 -0.725417494773865 6.0
46.604605733 -0.708402276039124 7.0
45.106295532 0.0432944595813751 0.0
23.2126055 -0.0471706315875053 1.0
21.893690031 -0.0194025486707687 2.0
-10.542381522 -0.0272897817194462 3.0
11.3513085 -0.0798490792512894 4.0
33.754987032 -0.0255332291126251 5.0
45.106295542 -0.250940918922424 6.0
45.106295542 -0.361166477203369 7.0
38.977055089 -0.0427911877632141 0.0
19.684728957 0.0141334049403667 1.0
19.292326132 0.0461460426449776 2.0
-8.6831116468 -0.0576963722705841 3.0
10.609214475 -0.0466983392834663 4.0
28.367840614 -0.0587304756045341 5.0
38.977055099 -0.821324348449707 6.0
38.977055099 -0.347184211015701 7.0
38.935365873 0.0371072590351105 0.0
21.088579762 0.0383711419999599 1.0
17.846786112 0.0277849957346916 2.0
-3.2985252296 -0.0591925010085106 3.0
14.548260873 -0.034355990588665 4.0
24.387105001 -0.0668748170137405 5.0
38.935365883 -0.162128567695618 6.0
38.935365883 0.0144474040716887 7.0
42.74140991 0.0301084313541651 0.0
20.9458217 0.029444457963109 1.0
21.795588212 0.0252154767513275 2.0
-6.3794891462 -0.048926904797554 3.0
15.416099058 -0.0674120858311653 4.0
27.325310854 -0.0647847577929497 5.0
42.741409918 -0.169451117515564 6.0
42.741409918 0.016341857612133 7.0
31.310794567 0.0457212254405022 0.0
16.880318068 0.0298823490738869 1.0
14.4304765 0.0343180075287819 2.0
-5.6188756586 -0.066631592810154 3.0
8.8116008333 -0.0393016338348389 4.0
22.499193736 -0.0832588449120522 5.0
31.310794576 -0.0839677304029465 6.0
31.310794576 -0.0567733198404312 7.0
};
\addlegendentry{$R^2$=-2.322}
\end{axis}

\end{tikzpicture}
}}
    
    \caption{Model results using only the loss associated with nodal flow predictions in the 8-node network.}
    \label{fig:dummy_base_results}
\end{figure}



\subsection{Case Study II: 63-node Network (Colombia)}


\section{Discussion and conclusions}


