\chapter{Introduction} \label{cap:introduccion}

\section{Justification}

Natural gas is an energy source that has gained great relevance worldwide, due to two fundamental causes. Firstly, it has been observed that a country's economic growth is closely related to its energy consumption \cite{Alam_M}. Therefore, as nations develop and grow economically, it is expected that they will seek energy security to meet their own demand and continue their progress without interruptions. The second major motivation for the use of natural gas is its lower greenhouse gas emissions compared to other fuels, making it an attractive choice, especially in a context where there is a growing interest in environmental care. Natural gas emits fewer greenhouse gases compared to other fossil fuels, making it a favorable option for climate change mitigation \cite{china_natural_gas}. In this context, the natural gas constitutes an energy source more efficient and less polluting than coal and oil \cite{Yin_Wen_Wu_Han_Mukhtar_Gong_2022}, that supports heating and electricity for the intensive demand of houses and industry \cite{Aydin_2018}.


% In this context, the natural gas system plays a crucial role in providing clean and versatile energy \cite{Yin_Wen_Wu_Han_Mukhtar_Gong_2022}. It is an efficient and less polluting energy source compared to conventional fossil fuels, such as coal and oil \cite{Aydin_2018}. Its use is essential for electricity generation, residential and industrial heating, and also for supplying energy-intensive industrial sectors.

According to the U.S. Energy Information Administration (EIA), global demand for natural gas is projected to increase steadily through 2050, driven by population growth, rising incomes, and industrial expansion in emerging regions. In most modeled scenarios, demand for natural gas rises by between 2\% and 10\% by 2030 and between 11\% and 57\% by 2050, relative to 2022 levels. Despite significant gains in renewables and efficiency, natural gas continues to play a critical role in meeting the world's growing energy needs.\cite{IEA_2024}. 


% In Colombia, natural gas is integral to national energy planning due to its essential role across multiple residential, industrial, and power generation sectors. The \textit{Plan de Abastecimiento de Gas Natural (PAGN)}, adopted in 2020, highlighted the strategic importance of maintaining reliable and sustainable gas supplies. This plan projects the demand growth for natural gas as essential for meeting Colombia's energy needs and underscores the reliance on gas for stabilizing the energy grid and reducing dependency on less sustainable energy sources. Natural gas demand is expected to grow by approximately 17\% from 2021 to 2035, a trend driven primarily by the industrial and residential sectors, representing over 75\% of aggregated consumption. This reliance reflects natural gas relatively low environmental impact compared to other fossil fuels, positioning it as a bridge towards a low-carbon energy transition. Colombia's growing urbanization and industrial needs create a seasonal demand pattern, with increased consumption in the latter months that aligns with general economic activity trends \cite{Promigas_2021}.

In Colombia, natural gas remains a cornerstone of national energy planning, with its importance reaffirmed by the \textit{Plan de Abastecimiento de Gas Natural (PAGN)}. This plan incorporates the technical recommendations made by the Unidad de Planeación Minero-Energética (UPME) in the 2019–2028 study and continues to guide infrastructure priorities. According to the updated projections from UPME (June 2021), natural gas demand is expected to grow by 17\% between April 2021 and December 2035, reflecting an average annual growth rate of approximately 1.18\%. This demand projection includes aggregated consumption from sectors such as residential, industrial, tertiary, transportation, petrochemical, and compression. Notably, more than 75\% of the aggregated demand is concentrated in the industrial and residential sub-sectors, reinforcing the seasonal nature of consumption patterns, with increased usage in the latter months of each year aligned with economic activity trends. While demand from the oil and thermoelectric sectors has been revised downward in medium and low scenarios—due to increased energy efficiency and favorable hydrological forecasts—potential deficits in supply between 2021 and 2030 have been identified. These gaps are expected to be addressed primarily through imported and regasified LNG from facilities like SPEC in Cartagena. Natural gas remains a key energy source for Colombia, balancing reliability, lower environmental impact compared to other fossil fuels, and its strategic role in enabling a gradual transition toward cleaner energy sources \cite{Promigas_2021}.


Although most of the country's electricity demand is commonly met by hydroelectric plants \cite{Arango-Aramburo_Turner_Daenzer_Ríos-Ocampo_Hejazi_Kober_Álvarez-Espinosa_Romero-Otalora_vanderZwaan_2019}, this type of generation presents a significant source of uncertainty in the energy system since its effectiveness and generation capacity are directly linked to the country's climatic and meteorological conditions, especially in extreme cases such as the El Niño phenomenon \cite{Villa-Loaiza_Taype-Huaman_Benavides-Franco_Buenaventura-Vera_Carabalí-Mosquera_2023}. Variations in precipitation, droughts, or floods can have a significant impact on the availability of water for hydroelectric power production, affecting the balance between supply and demand in the electrical system \cite{Ignacio_Fariza_2022}. Additionally, the increase in energy demand and the transition to renewable energy sources pose significant challenges in the efficient and reliable transportation of gas. Optimizing the natural gas transportation system, considering the uncertainty associated with renewable energy generation and demand variability, is essential to ensure a reliable, sustainable, and environmentally friendly energy supply \cite{Shan_Yu_Gong_Huang_Wen_Wang_Ren_Wang_Shi_Liu_2023}.




Therefore, it is necessary for the country not only to have a national gas transportation system, but also to ensure that it is operated as efficiently as possible, in order to make the best use of available natural resources. In the Colombian context, natural gas is a very important energy source as it is used in various sectors such as residential, commercial, industrial, and thermal \cite{Restrepo-Trujillo_Moreno-Chuquen_Jiménez-García_Flores_Chamorro_2022}. It is especially in the latter sector that this fuel becomes more relevant during dry seasons, as it is when reservoir levels drop and thus hydroelectric power generation decreases. This problem is exacerbated in years when the El Niño phenomenon occurs \cite{paper-col}, making it of great interest to have tools that allow for the optimal injection and transportation of natural gas to fully meet demand.

    




\section{Problem statement}

Natural gas transportation is an integral part of the natural gas industry, relying on a pipeline network to transfer natural gas from various sources to consumers, fulfilling their demand. In general, natural gas transmission systems are composed of four fundamental elements: injection fields, responsible for injecting the hydrocarbon from extraction fields or regasification plants into the system; pipelines, which transport the gas from a sending node to a receiving node; compressors, which are responsible for raising the pressure at the outlet node relative to the inlet node; and end user, which are the main consumers of natural gas \cite{review}. Ensuring gas flow to meet end-user demand, minimizing network operating costs, and maintaining system elements within appropriate technical operating limits are critical factors in natural gas transportation. Coordinating these factors requires efficient solutions of optimization problems with a large number of variables and different nature constraints  \cite{Conejo}.
% \cite{ABBASI2011xv}


The optimization problem consists on finding the best operational configurations to meet consumer demand while ensuring the technical and physical constraints of the natural gas transportation system.  It must also be considered that these transportation systems are usually interconnected with electricity systems since the latters usually require natural gas as fuel for the thermal power plants. These power plants are significant natural gas consumers, relying on a steady gas supply to generate electricity \cite{Byeon_Van_Hentenryck_2020}. In Colombia, despite holding 70\% of hydroelectric plants, the remainder consists primarily of thermoelectric plants \cite{Morcillo_Angulo_Franco_2020a}. These thermoelectric plants are key to complementing the hydro plants to meet energy demand, especially during periods of drought, e.g., during El Niño phenomenon when reduced water availability limits hydroelectric generation \cite{droughts_colombia}. As other studies have shown, variations in rainfall, droughts, or floods in countries with significant hydroelectric power plants can significantly affect water availability for hydropower production \cite{Cuartas_Cunha_Alves_Parra_Deusdará_Leal_Costa_Molina_Amore_Broedel_Seluchi_et_al_2022}. 

The above situation necessitates solving the optimization problem multiple times to ensure the system's correct operation across various scenarios. Consequently, this process takes considerable time and is both resource-intensive and time-consuming. Despite the high computational cost of each model execution, the resulting solutions are not utilized in subsequent optimization processes, even in similar operational scenarios. Therefore, there is a pressing need to develop a machine learning strategy that leverages historical solutions to provide faster responses to different operational scenarios by learning from past optimization outcomes. However, the effectiveness of such a strategy also depends on how accurately the underlying physical components of the system are modeled—particularly the transmission pipelines, which pose significant challenges due to their nonlinear behavior.

Although production fields, compressors, and end users of natural gas are well-represented, modeling transmission pipelines remains complex due to the nonlinear relationship between flow and pressures at its ending nodes. This complexity arises from the Weymouth equation, which includes a nonconvex and discontinuous sign function that determines flow direction based on differential pressure. These nonconvexities introduce discontinuities lead to numerical issues and optimization instability~\cite{YANG2020106023, JIANG2021106460}. Various authors have approached the challenge posed by the Weymouth equation differently. One of the most widely accepted methods involves approximating this equation due to its inherent complexity and nonconvex nature. However, since it is an approximation, this solution introduces errors that impact the accuracy of optimization outcomes. Mitigating these errors remains critical for further research and development in natural gas transportation systems \cite{review}.


This research seeks to address these challenges by developing a machine learning-based strategy that leverages solutions from past optimizations to provide rapid, reliable predictions for various operational scenarios. Specifically, this thesis explores using Graph Neural Network (GNN) models and MPCC-based optimization formulations as complementary tools. The GNN model, by learning the structure and patterns from historical optimization outcomes, enables faster scenario evaluations while maintaining acceptable error margins. Furthermore, the Mathematical Programs with Complementarity Constraints (MPCC) approach introduces an accurate modeling of the Weymouth equation, enhancing the fidelity of flow and pressure calculations without compromising computational efficiency.





\subsection{Objectives}

\subsection{General Objective}
To develop an optimization tool that integrates knowledge of the gas transportation network topology, a suitable approximation of the Weymouth equation and stochastic optimization techniques to address the gas transportation task taking into account the uncertainties related to hydroelectric generation and the growth of alternative energy sources.

\subsection{Specific Objectives}
\begin{itemize}

\item Design a Graph Neural Networks-based approach of regression that integrates knowledge of natural gas network topology to reduce computational time for operation estimation.

\item Develop an optimization model for natural gas transportation systems that takes into account the Weymouth equation for reducing that reduces the approximation error in pipeline gas flow calculations.

\item Develop a stochastic gas flow dispatch optimization strategy that quantifies the uncertainty in the objective variables and decision variables associated with the operation of the gas system taking into account the constraints of the transportation problem.

\end{itemize}









% \section{Objetivos}

% \subsection{General}


% \subsection{Específicos}

% \begin{itemize}
	
% 		\item  
	
% 	    \item 
     
% 		\item 
	
% \end{itemize}


% \section{Estado del arte}


% \section{Resultados principales}

% \section{Organización del documento}
  
