\chapter{Introduction} \label{cap:introduccion}

\section{Problem statement}

Natural gas transportation is an integral part of the natural gas industry, relying on a pipeline network to transfer natural gas from various sources to consumers, fulfilling their demand. This network is divided into two main types: transmission and distribution. Transmission networks transport large volumes of gas at high pressure over long distances from gas sources to distribution centers. On the other hand, distribution networks deliver gas to individual consumers \cite{ABBASI2011xv}. In general, natural gas transmission systems are composed of four fundamental elements: injection fields, responsible for injecting the hydrocarbon from extraction fields or regasification plants into the system; pipelines, which transport the gas from a sending node to a receiving node; compressors, which are responsible for raising the pressure at the outlet node relative to the inlet node; and end user \cite{review}. Ensuring gas flow to meet end-user demand, minimizing network operating costs, and maintaining system elements within appropriate technical operating limits are critical factors in natural gas transportation. Coordinating these factors requires formulating optimization problems, which must be solved efficiently, taking into account the numerous variables and the nature of these variables \cite{Conejo}.


The optimization problem is determining the best operational configurations to meet consumer demand while ensuring the technical and physical constraints of the natural gas transportation system.  It must also be considered that these transportation systems are usually interconnected with the electricity systems since the latter usually require natural gas as fuel for the thermal power plants. These power plants are significant natural gas consumers, relying on a steady supply to generate electricity \cite{Byeon_Van_Hentenryck_2020}. The Colombian case is no exception; although the Colombian energy matrix comprises 70\% hydroelectric plants, the remainder consists primarily of thermoelectric plants \cite{Morcillo_Angulo_Franco_2020a}. These thermoelectric plants are crucial for complementing the hydro plants to meet energy demand, especially during periods of drought. They become essential during events like the El Niño phenomenon when reduced water availability limits hydroelectric generation \cite{droughts_colombia}. As other studies have shown, variations in rainfall, droughts, or floods in countries with high hydroelectric power plants can significantly affect water availability for hydropower production \cite{Cuartas_Cunha_Alves_Parra_Deusdará_Leal_Costa_Molina_Amore_Broedel_Seluchi_et_al_2022}. 

The above situation necessitates solving the optimization problem multiple times to ensure the system's correct operation across various scenarios. Consequently, this process takes considerable time and is both resource-intensive and time-consuming. Despite the high computational cost of each model execution, the resulting solutions are not utilized in subsequent optimization processes, even in similar operational scenarios. Therefore, there is a pressing need to develop a machine learning strategy that leverages historical solutions to provide faster responses to different operational scenarios by learning from past optimization outcomes.

Although production fields, compressors, and end users of natural gas are well-represented, modeling transmission pipelines remains complex due to the nonlinear relationship between flow and pressures at its ending nodes. This complexity arises from the Weymouth equation, which includes a nonconvex and discontinuous sign function that determines flow direction based on differential pressure. These nonconvexities introduce discontinuities lead to numerical issues and optimization instability~\cite{YANG2020106023, JIANG2021106460}. Various authors have approached the challenge posed by the Weymouth equation differently. One of the most widely accepted methods involves approximating this equation due to its inherent complexity and nonconvex nature. However, since it is an approximation, this solution introduces errors that impact the accuracy of optimization outcomes. Mitigating these errors remains critical for further research and development in natural gas transportation systems \cite{review}.


This research seeks to address these challenges by developing a machine learning-based strategy that leverages solutions from past optimizations to provide rapid, reliable predictions for various operational scenarios. Specifically, this thesis explores using Graph Neural Network (GNN) models and MPCC-based optimization formulations as complementary tools. The GNN model, by learning the structure and patterns from historical optimization outcomes, enables faster scenario evaluations while maintaining acceptable error margins. Furthermore, the MPCC approach introduces an accurate modeling of the Weymouth equation, enhancing the fidelity of flow and pressure calculations without compromising computational efficiency.


%Due to its inherent limitations related to the Weymouth equation and the significant increase in constraints as the number of nodes in the system grows, the optimization model for natural gas transportation is computationally expensive \cite{Shan_Yu_Gong_Huang_Wen_Wang_Ren_Wang_Shi_Liu_2023}. Additionally, the necessity to run the optimization model multiple times to test various possible scenarios further amplifies the computational burden. 



% The prediction of electric and gas demand is fraught with uncertainty due to the inherent randomness and volatility of consumer behavior and the intermittent nature of renewable energy sources. This unpredictability disrupts the balance between supply and demand.  Therefore, the need arises to approach this optimization problem from a stochastic perspective, considering possible generation scenarios to effectively manage the variability and uncertainties associated with gas transportation, ensuring a reliable and efficient energy supply \cite{KAZI2024110748}.





% In natural gas transportation, it is essential to ensure the optimal flow of gas to meet end-user demand, maintain adequate pressure in the network, and comply with capacity constraints in the pipelines. Coordinating all these aspects requires the formulation of optimization problems that can be efficiently solved, taking into account the large number of variables and the nonlinear nature of the system \cite{Conejo}. 

%The transportation of natural gas presents a challenge in terms of computational complexity due to several factors. Firstly, the gas flow equation in pipelines is highly nonlinear and non-convex, which makes it difficult to directly solve the problem efficiently \cite{review}. The inherent complexity of this equation intensifies in large-scale networks with multiple nodes and links, implying a greater number of variables and constraints that must be considered. Another factor contributing to computational complexity is the need to consider multiple objectives and operational constraints \cite{He_Zhang_Liu_Wu_Shahidehpour_2018}. %  %   Although extensive work has been done in modeling these elements, pipelines in general remain a fairly complex problem to model. Mathematically, the relationship between the pressures at the ends of a pipe and the flow through it is given by the so-called Weymouth equation \cite{mohammad}, an expression quite difficult to work with in the field of optimization due to its non-convexity resulting from a sign function used to determine the flow direction. This difficulty has generated interest in developing approaches that allow for an acceptable approximation of the equation without significantly increasing the computational complexity of the problem \cite{Shan_Yu_Gong_Huang_Wen_Wang_Ren_Wang_Shi_Liu_2023}.

 %   In the context of current energy systems, the inherent coupling between the gas system and the power system must also be highlighted \cite{Byeon_Van_Hentenryck_2020}, which introduces significant uncertainty into the resulting energy system \cite{Bayani_Manshadi_2023}. It is relevant to mention that although hydroelectric plants play a fundamental role in electricity generation by harnessing the energy of water in rivers and reservoirs, their effectiveness and generation capacity are directly linked to the country's climatic and meteorological conditions \cite{Yaseen_Sulaiman_Deo_Chau_2019}. Variations in precipitation, droughts, or floods can significantly impact the availability of water for hydroelectric power production. Additionally, uncertainty in gas demand is also related to the generation of renewable energy, such as solar and wind power \cite{morales2013integrating}. Various authors have addressed this issue in both individual power systems and power systems interconnected with the gas system \cite{Liang_Liao_2007}. The prediction of electric and gas demand presents uncertainty due to the randomness and volatility associated with a large number of consumers, as well as the intermittent nature of renewable energy sources, affecting the balance between supply and demand. Given the inherently uncertain nature of gas demand and availability in a country, it is essential to address the gas transportation problem with a stochastic approach. This approach effectively manages the variability and uncertainties associated with gas transportation, thus ensuring a reliable and efficient supply in the energy system \cite{morales2013integrating}.

  %  The two subsystems, power and gas, have very different dynamic characteristics. Electrical energy travels at the speed of light and, therefore, electricity transportation can be considered an instantaneous process. In contrast, natural gas flow is a much slower process due to the low speed of gas in pipelines, which results in a long response time to disturbances and a significant delay between injection and withdrawal. The delay increases with the distance between injection and extraction points \cite{Fang_Zeng_Ai_Chen_Wen_2018}. Additionally, unlike the electrical system, the natural gas network is characterized by intrinsic flexibility related to the compressibility of gas (i.e., the linepack effect), which allows pipelines to be used as short-term storage \cite{review}. By coordinating both systems, the generation, transportation, and consumption of energy resources can be optimized, mitigating dynamic differences and improving the reliability and flexibility of the overall system \cite{Bai_Li_Cui_Jiang_Sun_Zhu_2016}.

  %  With these difficulties, it is clear that there is still a field of research in this area, leading to the following research question: How can an optimization tool be developed to improve the quality of the solution to the natural gas transportation problem, taking into account the reduction of computational complexity, the increasing energy demand, the rise in renewable energy generation, and the uncertainty associated with both factors?


\section{Justification}

Natural gas is an energy source that has gained great relevance worldwide, and this can be attributed to two fundamental causes. Firstly, it has been observed that a country's economic growth is closely related to its energy consumption \cite{Alam_M}. Therefore, as nations develop and grow economically, it is expected that they will seek energy security to meet their own demand and continue their progress without interruptions. The second major motivation for the use of natural gas is its lower greenhouse gas emissions compared to other fuels, making it a favorable option, especially in a context where there is a growing interest in environmental care. Natural gas emits fewer greenhouse gases compared to other fossil fuels, making it a favorable option for climate change mitigation \cite{china_natural_gas}. In this context, the natural gas system plays a crucial role in providing clean and versatile energy \cite{Yin_Wen_Wu_Han_Mukhtar_Gong_2022}. It is an efficient and less polluting energy source compared to conventional fossil fuels, such as coal and oil \cite{Aydin_2018}. Its use is essential for electricity generation, residential and industrial heating, and also for supplying energy-intensive industrial sectors.

According to the U.S. Energy Information Administration (EIA) Global demand for natural gas will strengthen in 2024, driven by anticipated colder winter conditions and moderating gas prices. Despite a modest 0.5\% increase in global demand in 2023, further demand growth of 2.5\% is projected this year, equivalent to an additional 100 billion cubic meters (bcm). This increase is primarily supported by emerging economies, with China leading consumption recovery by regaining its status as the top LNG importer following a 7\% rise in demand \cite{IEA_2024}. 


% figures from the U.S. Energy Information Administration (EIA), global natural gas consumption in 2015 reached 124.24 trillion cubic feet, with a projected increase of 43\% by 2040, where 75\% is associated with the industrial sector and electricity generation based on thermal power plants. This pronounced increase in consumption contrasts with the total volume of proven reserves worldwide, which reached 154.53 trillion cubic feet in 2015 and projects a growth of about 52\% by 2014 \cite{eia}. 


In Colombia, natural gas is integral to national energy planning due to its essential role across multiple residential, industrial, and power generation sectors. The \textit{Plan de Abastecimiento de Gas Natural (PAGN)}, adopted in 2020, highlights the strategic importance of maintaining reliable and sustainable gas supplies. This plan projects the demand growth for natural gas as essential to meeting Colombia's energy needs and underscores the reliance on gas for stabilizing the energy grid and reducing dependency on less sustainable energy sources. Natural gas demand is expected to grow by approximately 17\% from 2021 to 2035, a trend driven primarily by the industrial and residential sectors, representing over 75\% of aggregated consumption. This reliance reflects natural gas's relatively low environmental impact compared to other fossil fuels, positioning it as a bridge toward a low-carbon energy transition. Colombia's growing urbanization and industrial needs create a seasonal demand pattern, with increased consumption in the latter months that aligns with general economic activity trends \cite{Promigas_2021}.


Therefore, it is necessary for the country not only to have a national gas transportation system but also to ensure that it is operated in the best possible way, so that natural resources are maximized. In the Colombian context, natural gas is a very important energy source as it is used in various sectors such as residential, commercial, industrial, and thermal \cite{Restrepo-Trujillo_Moreno-Chuquen_Jiménez-García_Flores_Chamorro_2022}. It is especially in the latter sector that this fuel becomes more relevant during dry seasons, as it is when reservoir levels drop and thus hydroelectric power generation decreases. This problem is exacerbated in years when the El Niño phenomenon occurs \cite{paper-col}, making it of great interest to have tools that allow for the optimal injection and transportation of natural gas to fully meet demand.

Although most of the country's electricity demand is commonly met by hydroelectric plants \cite{Arango-Aramburo_Turner_Daenzer_Ríos-Ocampo_Hejazi_Kober_Álvarez-Espinosa_Romero-Otalora_vanderZwaan_2019}, this type of generation presents a significant source of uncertainty in the energy system since its effectiveness and generation capacity are directly linked to the country's climatic and meteorological conditions, especially in extreme cases such as the El Niño phenomenon \cite{Villa-Loaiza_Taype-Huaman_Benavides-Franco_Buenaventura-Vera_Carabalí-Mosquera_2023}. Variations in precipitation, droughts, or floods can have a significant impact on the availability of water for hydroelectric power production, affecting the balance between supply and demand in the electrical system \cite{Ignacio_Fariza_2022}. Additionally, the increase in energy demand and the transition to renewable energy sources pose significant challenges in the efficient and reliable transportation of gas. Optimizing the natural gas transportation system, considering the uncertainty associated with renewable energy generation and demand variability, is essential to ensure a reliable, sustainable, and environmentally friendly energy supply \cite{Shan_Yu_Gong_Huang_Wen_Wang_Ren_Wang_Shi_Liu_2023}.
    



\subsection{Objectives}

\subsection{General Objective}
To develop an optimization tool that integrates knowledge of the gas transportation network topology, a suitable approximation of the Weymouth equation and stochastic optimization techniques to address the gas transportation problem taking into account the uncertainties related to hydroelectric generation and the growth of alternative energy sources.

\subsection{Specific Objectives}
\begin{itemize}

\item Design a Graph Neural Networks based approach that integrates knowledge of natural gas network topology to reduce computational time for operation estimation.

\item Develop an optimization model for natural gas transportation systems that takes into account the Weymouth equation that reduces the approximation error in pipeline gas flow calculations.

\item Develop a stochastic optimization strategy that quantifies the uncertainty in the objective variables and decision variables associated with the operation of the gas system from the probability distributions of the constraints of the transportation problem.

\end{itemize}









% \section{Objetivos}

% \subsection{General}


% \subsection{Específicos}

% \begin{itemize}
	
% 		\item  
	
% 	    \item 
     
% 		\item 
	
% \end{itemize}


% \section{Estado del arte}


% \section{Resultados principales}

% \section{Organización del documento}
  
