\chapter{Conclusions and future work}

\section{Conclusions}

This thesis set out to develop an optimization tool for natural gas transportation networks by combining knowledge of the network topology, an appropriate approximation of the Weymouth equation, and stochastic optimization techniques. Each specific objective contributed to achieving this goal, and the progress toward each objective is presented across the thesis chapters.

The first specific objective was to design a Graph Neural Network (GNN)-based approach that integrates natural gas network topology to reduce computational time for operational estimation. This objective was explored in Chapters 2 and 4, where a GNN-based model was applied to predict decision variables of the natural gas system while significantly reducing computation times compared to traditional optimization techniques. Both chapters included comparisons of computation times between optimizer-based methods and the GNN-based approach, with the GNN consistently providing faster results. In Chapter 2, a GNN was trained to approximate a standard linear optimization model for a simplified gas system (without pressure considerations) and then tested on more extensive networks, such as the Colombian gas system. Chapter 4 extended this work by incorporating physical constraints, like gas balance and the Weymouth equation, into the loss function, resulting in a more physically accurate model with consistent gains in computational efficiency. Across all tests, the GNN model provided predictions much faster than optimizers, demonstrating that this approach successfully meets the objective of reducing computational time for operational estimation.

The second specific objective was to develop an optimization model for natural gas transportation systems that incorporates the Weymouth equation to reduce approximation errors in gas flow calculations. Chapter 3 addressed this objective by developing a Mathematical Program with Complementarity Constraints (MPCC)-based optimization model that accurately represents the nonlinear Weymouth equation without needing mixed-integer formulations. The MPCC approach utilized binary-behaving continuous variables to capture the bidirectional nature of gas flows and handle complex, high-demand scenarios in interconnected networks. The model was tested on real-world systems, including the Colombian gas-power system, and consistently achieved a lower error response than traditional methodologies, providing a robust and accurate tool for operational scheduling in energy systems. By effectively modeling the nonlinear constraints of the Weymouth equation, this MPCC-based model met the objective of improving accuracy in pipeline flow calculations.

The third specific objective was to develop a stochastic optimization strategy that quantifies uncertainties in gas system operation by sampling from the probability distributions of the constraints in the transportation problem. Chapters 2 and 4 provided a foundational approach to stochastic optimization by demonstrating that GNN-based models, once trained, can rapidly generate responses to various scenarios through forward propagation alone. This rapid response capability enables the GNN-based model to efficiently test multiple scenarios quickly, effectively managing uncertainties in the system without needing to solve an optimization problem from scratch for each scenario. In Chapter 2, the GNN model was shown to provide fast, low-error responses for new, unseen cases, and Chapter 4 extended this capacity by incorporating additional physical constraints, further enhancing the GNN's ability to model complex and variable network conditions with high accuracy. Although a fully probabilistic optimization framework was not implemented, the capability to quickly test many scenarios aligns with stochastic optimization principles, allowing for rapid evaluation of uncertain conditions within the network.

In summary, this thesis demonstrated the potential of GNN-based and MPCC models as complementary tools for natural gas transportation network optimization and prediction. The GNN model offered substantial computational efficiency and adaptability to network topology, while the MPCC model provided high accuracy in modeling the nonlinear constraints imposed by gas flows. Together, these approaches support real-time applications requiring computational efficiency and accuracy. 

\section{Future Work}

Future research can explore several directions to enhance the application and effectiveness of GNN-based models and MPCC formulations in natural gas and energy systems. One important avenue is extending the GNN-based model to handle transient dynamics and uncertainties, which is particularly valuable in applications that involve renewable energy integration. 

Another potential direction involves the development of stochastic and distributed models. Integrating stochastic modeling into the MPCC formulation could achieve a more robust optimization framework under demand fluctuations and supply variability scenarios. 

Improving the design of loss functions in GNN models is another promising area. As demonstrated in this work, the performance of GNN-based models can be improved by incorporating physical constraints.  In that sense, an area of interest could focus on implementing loss functions associated with the Weymouth equation with a lower computational complexity so that they can be used in complex systems. 

