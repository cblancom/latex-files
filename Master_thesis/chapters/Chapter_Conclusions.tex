\chapter{Conclusions and future work}

\section{Conclusions}

This thesis set out to develop an optimization tool for natural gas transportation networks by combining knowledge of the network topology, an appropriate approximation of the Weymouth equation, and stochastic optimization techniques. Each specific objective contributed to achieving this goal, and the progress toward each objective is presented across the thesis chapters.

The first specific objective was to design a Graph Neural Network (GNN)-based approach that integrates natural gas network topology to reduce computational time for operational estimation. This objective was explored in Chapters 2 and 4, where a GNN-based model was applied to predict decision variables of the natural gas system while significantly reducing computation times compared to traditional optimization techniques. Both chapters included comparisons of computation times between optimizer-based methods and the GNN-based approach, with the GNN consistently providing faster results. In Chapter 2, a GNN was trained to approximate a standard linear optimization model for a simplified gas system (without pressure considerations) and then tested on more extensive networks, such as the Colombian gas system. Chapter 4 extended this work by incorporating physical constraints, like gas balance and the Weymouth equation, into the loss function, resulting in a more physically accurate model with consistent gains in computational efficiency. Across all tests, the GNN model provided predictions much faster than optimizers, demonstrating that this approach successfully meets the objective of reducing computational time for operational estimation.


The second specific objective was to develop an optimization model for natural gas transportation systems that incorporates the Weymouth equation to reduce approximation errors in gas flow calculations. Chapter 3 addressed this objective by developing a Mathematical Program with Complementarity Constraints (MPCC)-based optimization model that accurately represents the nonlinear Weymouth equation without requiring mixed-integer formulations. This approach leveraged binary-behaving continuous variables to capture the bidirectional nature of gas flows, enabling the model to represent operational decisions more precisely under complex and high-demand network conditions. The model was successfully tested on realistic systems, including the Colombian gas-power system, consistently achieving lower flow approximation errors compared to traditional methods. These results underscore the practical applicability of MPCC in gas transport optimization, especially for scenarios that demand reliable and accurate short-term scheduling.



The third specific objective was to develop a stochastic optimization strategy that quantifies uncertainties in gas system operation by sampling from the probability distributions of the constraints in the transportation problem. This objective was addressed in Chapter 4, where the GNN-based model—previously formulated in Chapter 2—was extended to incorporate physical constraints, enhancing its ability to model the complex behavior of natural gas networks. Once trained, the GNN model was capable of rapidly generating responses through forward propagation, enabling efficient evaluation of multiple scenarios without requiring repeated optimization. This rapid-response feature provides a practical mechanism for addressing uncertainty in gas system operation by allowing a wide range of scenarios to be tested quickly and systematically.

To support this strategy, stochastic analyses were conducted in Chapter 4 using kernel density estimates (KDEs) fitted to the training data. These KDEs were used to generate synthetic samples that emulate plausible input conditions. The GNN model’s outputs under these synthetic conditions were then compared to the distribution of training outputs using log-likelihood measures and Kolmogorov–Smirnov (K–S) tests. In both the 8-node and Colombian gas networks, the synthetic outputs were found to be statistically similar to those derived from the training set, with low test statistics and high p-values across multiple K–S test alternatives. These results confirmed that the trained models not only generalize well but also capture the underlying behavior of the system, validating the approach's effectiveness in representing uncertainty.

While a stochastic optimization framework with probabilistic objective functions and constraints was not explicitly implemented, the capacity to efficiently simulate and evaluate numerous uncertain scenarios satisfies the core principles of stochastic optimization. Thus, the proposed methodology achieves the intended objective by enabling a scalable and reliable strategy for uncertainty quantification in gas transportation networks.


\section{Future Work}

Future research can explore several directions to enhance the application and effectiveness of GNN-based models and MPCC formulations in natural gas and broader energy systems. One option is extending the GNN-based model to account for transient dynamics and operational uncertainty. This enhancement is particularly relevant in contexts involving the increasing integration of renewable energy sources, where variability in supply can significantly affect gas system operation.

Another direction is the development of stochastic optimization models that account for the variability introduced by renewable energy sources such as solar and wind. Incorporating stochastic elements into the MPCC formulation would enhance the framework’s ability to maintain reliable operation under uncertain demand and supply conditions. This extension is particularly relevant given the increasing need for robust decision-making in energy systems subject to high levels of variability and uncertainty.


Additionally, improving the design of loss functions in GNN architectures remains an area of development. This work has shown that the inclusion of physical constraints, such as balance conditions and flow equations, enhances predictive accuracy. Future research could focus on implementing loss functions associated with the Weymouth equation that retain physical fidelity while reducing computational complexity, making them suitable for large-scale or real-time applications.
