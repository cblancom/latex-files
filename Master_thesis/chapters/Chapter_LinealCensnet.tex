\chapter{Power System - Censnet} \label{cap:mpcc}

\section{Preliminaries}

\subsection{Graph definition}


A \textbf{graph} $G$ is a mathematical structure that represents a set of interconnected objects. These objects are known as \textbf{vertices} (or \textbf{nodes}), denoted by the set $V(G)$, and the connections between them are called \textbf{edges} (or \textbf{arcs}), denoted by the set $E(G)$. Formally, a graph is defined as an ordered pair $G = (V, E)$, where $V(G)$ is a non-empty set of vertices, and $E(G) \subseteq \{(u, v) \mid u, v \in V(G), u \neq v\}$ is a set of edges, where each edge connects two distinct vertices \cite{Trudeau_2015}.

Graphs can be categorized based on the properties of their edges. An \textbf{undirected graph} has edges that do not have a direction, so the pair $(u, v) = (v, u)$ represents an edge that simply connects vertices $u$ and $v$. In contrast, in a \textbf{directed graph} (or \textbf{digraph}), each edge $(u, v) \in E(G)$ has a direction, meaning it goes from vertex $u$ to vertex $v$. This implies that $(u, v) \neq (v, u)$ unless $u = v$ \cite{Bender_Williamson_2010}.

 %
% \begin{figure}
%     \begin{center}
%         \setlength\figurewidth{.5\textwidth}        
%             \setlength\figureheight{0.4\textwidth}
%         \resizebox{\figurewidth}{\figureheight}{\input{figures/Chapter_LinealCensnet/graph_def}}
%         % \input{figures/Chapter_LinealCensnet/graph_def}
%         % \includegraphics[width=0.95\textwidth]{figures/}
%     \end{center}
%     \caption{Directed graph}\label{fig:graph_definition}
% \end{figure}
%
\begin{figure}
    \centering
        \setlength\figurewidth{.53\textwidth}        
        \setlength\figureheight{0.36\textwidth} 
        \subfloat [Undirected graph] {\label{fig:undirected_graph_def}\resizebox{\figurewidth}{\figureheight}{\begin{tikzpicture}[>=Stealth, node distance=5cm]
    
    % Define styles for the nodes and edges
    \tikzstyle{node_style}=[circle, draw, minimum size=.9cm, inner sep=0pt, font=\normalsize]
    \tikzstyle{edge_style}=[draw, -, thick]

    % Nodes
    \node[node_style] (n1) {1};
    \node[node_style] (n2) [right of=n1] {2};
    \node[node_style] (n3) [right of=n2] {3};
    \node[node_style] (n4) [below of=n2] {4};

    % Edges with labels
    \draw[edge_style] (n1) edge node[above] {A} (n2);
    \draw[edge_style] (n4) edge node[left] {B} (n1);
    \draw[edge_style] (n4) edge node[right] {C} (n2);
    \draw[edge_style] (n3) edge node[right] {D} (n4);
    \draw[edge_style] (n1) to[out=30, in=150] node[above] {E} (n3);
    \draw[edge_style] (n3) edge node[above] {F} (n2);

\end{tikzpicture}
}}
        \subfloat [Directed graph] {\label{fig:direc_graph_def}\resizebox{\figurewidth}{\figureheight}{\begin{tikzpicture}[>=Stealth, node distance=5cm]
    
    % Define styles for the nodes and edges
    \tikzstyle{node_style}=[circle, draw, minimum size=.9cm, inner sep=0pt, font=\normalsize]
    \tikzstyle{edge_style}=[draw, ->, thick]

    % Nodes
    \node[node_style] (n1) {1};
    \node[node_style] (n2) [right of=n1] {2};
    \node[node_style] (n3) [right of=n2] {3};
    \node[node_style] (n4) [below of=n2] {4};

    % Edges with labels
    \draw[edge_style] (n1) edge node[above] {A} (n2);
    \draw[edge_style] (n4) edge node[left] {B} (n1);
    \draw[edge_style] (n4) edge node[right] {C} (n2);
    \draw[edge_style] (n3) edge node[right] {D} (n4);
    % \draw[edge_style] (n1) to[out=30, in=150] node[above] {E} (n3);
    % \draw[edge_style] (n3) edge node[above] {F} (n2);

\end{tikzpicture}
}}
        % \includegraphics[width=0.45\textwidth]{figures/}
    \caption{Types of graphs}\label{fig:graph_definition}
\end{figure}

In \Cref{fig:graph_definition}, two graphs are represented, each composed of four nodes labeled $1$, $2$, $3$, and $4$,, and six edges labeled $A$, $B$, $C$, $D$, $E$, and $F$. The difference between them lies in the type of graph they represent. For example, in \Cref{fig:undirected_graph_def}, the edge c shows a connection between nodes 2 and 4. However, in \Cref{fig:direc_graph_def}, this connection provides additional information: a direction, which, in the context of this study, could represent the direction of a specific element, such as electric power or flow.


A graph can be represented in various ways using matrices, each capturing different aspects of the graph's structure. The two most common matrix representations are the \textbf{adjacency matrix} and the \textbf{incidence matrix}.

The \textbf{adjacency matrix} of a graph is a square matrix used to represent the connections between vertices \cite{wilson_1972}. For a graph $G$ with $n$ vertices, the adjacency matrix $A$ is an $n \times n$ matrix where the entry $a_{ij}$ is defined as follows:

\begin{equation}
 a_{ij} = 
\begin{cases}
1 & \text{if there is an edge from vertex } i \text{ to vertex } j, \\
0 & \text{otherwise}.
\end{cases}
\end{equation}

% \[
% a_{ij} =
% \begin{cases}
% 1 & \text{if there is an edge from vertex } i \text{ to vertex } j, \\
% 0 & \text{otherwise}.
% \end{cases}
% \]

For a directed graph, the adjacency matrix captures the direction of the edges. Below is the adjacency matrix for the directed graph shown earlier:

\[
A = \begin{pmatrix}
0 & 1 & 0 & 1 \\
1 & 0 & 0 & 1 \\
0 & 1 & 0 & 1 \\
1 & 1 & 1 & 0
\end{pmatrix}
\]


The \textbf{incidence matrix} of a graph represents the relationship between vertices and edges \cite{wilson_1972}. For a graph $G$ with $n$ vertices and $m$ edges, the incidence matrix $I$ is an $n \times m$ matrix where the entry $i_{ij}$ is defined as follows:

\begin{equation}
 B_{ij} =
\begin{cases}
1 & \text{if vertex } i \text{ is the starting point of edge } j \text{ in a directed graph}, \\
-1 & \text{if vertex } i \text{ is the endpoint of edge } j \text{ in a directed graph}, \\
0 & \text{if vertex } i \text{ is not connected to edge } j.
\end{cases}  
    \label{eq:}
\end{equation}


For the directed graph previously described, the incidence matrix is given by:

\[
B = \begin{pmatrix}
1 & -1 & 0 & 0 \\
-1 & 0 & -1 & 0 \\
0 & 0 & 0 & 1  \\
0 & 1 & 1 & -1 
\end{pmatrix}
\]

\subsection{Neural networks}

\subsubsection{Multi-Layered Perceptrons}


A Multilayer Perceptron (MLP) is a fundamental type of artificial neural network, often regarded as one of the building blocks of deep learning. At its core, an MLP consists of multiple layers of nodes, or neurons, where each layer is fully connected to the next one. The architecture typically includes an input layer, one or more hidden layers, and an output layer. The neurons in each layer are connected to the neurons in the subsequent layer through weighted connections, which are the key parameters learned during the training process. One of the most significant properties of an MLP is its ability to function as a universal approximator. This means that, given sufficient neurons in the hidden layers, an MLP can approximate any continuous function to an arbitrary degree of accuracy, provided the network is trained properly.

Mathematically, an MLP can be defined as follows. Let $\mathbf{x} \in \mathbb{R}^n$ represent the input vector, where $n$ is the number of features. The output of each neuron in the first hidden layer is calculated as:

\[
\mathbf{z}^{(1)} = \sigma\left(\mathbf{W}^{(1)}\mathbf{x} + \mathbf{b}^{(1)}\right)
\]

where $\mathbf{W}^{(1)} \in \mathbb{R}^{m_1 \times n}$ is the weight matrix for the first hidden layer, with $m_1$ being the number of neurons in this layer, $\mathbf{b}^{(1)} \in \mathbb{R}^{m_1}$ is the bias vector, and $\sigma(\cdot)$ is the activation function, typically a non-linear function such as the ReLU (Rectified Linear Unit) or sigmoid function.

This process is repeated for each subsequent hidden layer $k$, where the output of the $k$-th layer is given by:

\[
\mathbf{z}^{(k)} = \sigma\left(\mathbf{W}^{(k)}\mathbf{z}^{(k-1)} + \mathbf{b}^{(k)}\right)
\]

Here, $\mathbf{W}^{(k)} \in \mathbb{R}^{m_k \times m_{k-1}}$ represents the weight matrix connecting layer $k-1$ to layer $k$, $\mathbf{b}^{(k)} \in \mathbb{R}^{m_k}$ is the bias vector for layer $k$, and $\mathbf{z}^{(k-1)}$ is the output of the previous layer.

Finally, the output layer produces the final prediction $\mathbf{\hat{y}}$:

\[
\mathbf{\hat{y}} = \sigma\left(\mathbf{W}^{(L)}\mathbf{z}^{(L-1)} + \mathbf{b}^{(L)}\right)
\]

where $L$ denotes the number of layers in the network, including the input and output layers. Depending on the nature of the problem (e.g., classification or regression), the activation function $\sigma(\cdot)$ used in the output layer can vary, with softmax being common in multi-class classification problems, and a linear activation for regression tasks.

The entire MLP is trained using a process called backpropagation, combined with an optimization algorithm like gradient descent, to minimize a loss function $J(\mathbf{y}, \mathbf{\hat{y}})$, which measures the difference between the true outputs $\mathbf{y}$ and the predicted outputs $\mathbf{\hat{y}}$.


\begin{figure}
    \centering
    \setlength\figurewidth{1\textwidth}        
    \setlength\figureheight{0.5\textwidth}
    \resizebox{\figurewidth}{\figureheight}{\begin{tikzpicture}[shorten >=1pt, ->, draw=black!50, node distance=1.5cm and 3.5cm, align=center]

    % Styles
    \tikzstyle{input} = [circle, draw, fill=green!50, minimum size=2em]
    \tikzstyle{hidden} = [circle, draw, fill=blue!50, minimum size=2em]
    \tikzstyle{output} = [circle, draw, fill=red!50, minimum size=2em]
    \tikzstyle{connection} = [->, thick]

    % Network Stage Labels
    \node[align=center] at (0,-0.4) {Input \\ Layer};
    \node[align=center] at (6,0.4) {Hidden \\ Layers};
    \node[align=center] at (12,-1.2) {Output \\ Layer};

    % Input Layer
    \foreach \i in {1,2,3}
        \node[input] (I\i) at (0,-\i*1.5) {$x_\i$};

    % Hidden Layer 1
    \foreach \i in {1,2,3,4}
        \node[hidden] (H1\i) at (3,-\i*1.5+0.75) {$z^{(1)}_\i$};

    % Hidden Layer 2
    \foreach \i in {1,2,3,4}
        \node[hidden] (H2\i) at (6,-\i*1.5+0.75) {$z^{(2)}_\i$};

    % Hidden Layer 3
    \foreach \i in {1,2,3,4}
        \node[hidden] (H3\i) at (9,-\i*1.5+0.75) {$z^{(3)}_\i$};

    % Output Layer
    \foreach \i in {1,2}
        \node[output] (O\i) at (12,-\i*1.5-0.75) {$\hat{y}_\i$};

    % Connections from Input to Hidden Layer 1
    \foreach \i in {1,2,3}
        \foreach \j in {1,2,3,4}
            \draw[connection] (I\i) -- (H1\j);

    % Connections from Hidden Layer 1 to Hidden Layer 2
    \foreach \i in {1,2,3,4}
        \foreach \j in {1,2,3,4}
            \draw[connection] (H1\i) -- (H2\j);

    % Connections from Hidden Layer 2 to Hidden Layer 3
    \foreach \i in {1,2,3,4}
        \foreach \j in {1,2,3,4}
            \draw[connection] (H2\i) -- (H3\j);

    % Connections from Hidden Layer 3 to Output Layer
    \foreach \i in {1,2,3,4}
        \foreach \j in {1,2}
            \draw[connection] (H3\i) -- (O\j);

\end{tikzpicture}
}
    % \begin{tikzpicture}[shorten >=1pt, ->, draw=black!50, node distance=1.5cm and 3.5cm, align=center]

    % Styles
    \tikzstyle{input} = [circle, draw, fill=green!50, minimum size=2em]
    \tikzstyle{hidden} = [circle, draw, fill=blue!50, minimum size=2em]
    \tikzstyle{output} = [circle, draw, fill=red!50, minimum size=2em]
    \tikzstyle{connection} = [->, thick]

    % Network Stage Labels
    \node[align=center] at (0,-0.4) {Input \\ Layer};
    \node[align=center] at (6,0.4) {Hidden \\ Layers};
    \node[align=center] at (12,-1.2) {Output \\ Layer};

    % Input Layer
    \foreach \i in {1,2,3}
        \node[input] (I\i) at (0,-\i*1.5) {$x_\i$};

    % Hidden Layer 1
    \foreach \i in {1,2,3,4}
        \node[hidden] (H1\i) at (3,-\i*1.5+0.75) {$z^{(1)}_\i$};

    % Hidden Layer 2
    \foreach \i in {1,2,3,4}
        \node[hidden] (H2\i) at (6,-\i*1.5+0.75) {$z^{(2)}_\i$};

    % Hidden Layer 3
    \foreach \i in {1,2,3,4}
        \node[hidden] (H3\i) at (9,-\i*1.5+0.75) {$z^{(3)}_\i$};

    % Output Layer
    \foreach \i in {1,2}
        \node[output] (O\i) at (12,-\i*1.5-0.75) {$\hat{y}_\i$};

    % Connections from Input to Hidden Layer 1
    \foreach \i in {1,2,3}
        \foreach \j in {1,2,3,4}
            \draw[connection] (I\i) -- (H1\j);

    % Connections from Hidden Layer 1 to Hidden Layer 2
    \foreach \i in {1,2,3,4}
        \foreach \j in {1,2,3,4}
            \draw[connection] (H1\i) -- (H2\j);

    % Connections from Hidden Layer 2 to Hidden Layer 3
    \foreach \i in {1,2,3,4}
        \foreach \j in {1,2,3,4}
            \draw[connection] (H2\i) -- (H3\j);

    % Connections from Hidden Layer 3 to Output Layer
    \foreach \i in {1,2,3,4}
        \foreach \j in {1,2}
            \draw[connection] (H3\i) -- (O\j);

\end{tikzpicture}

    \caption{}\label{fig:}
\end{figure}


\subsubsection{Graph Neural Networks}

In recent years, \textbf{Graph Neural Networks (GNNs)} have emerged as a powerful tool in machine learning, particularly for tasks involving data that can be naturally represented as graphs. Graphs are a universal data structure that can model various systems in numerous fields, including social networks, biological networks, knowledge graphs, and physical systems. Because of their ability to represent relationships and interactions between entities, graphs are used extensively to model complex structures where the data points are not independent but interconnected.

The importance of GNNs lies in their ability to directly operate on graph-structured data, extending the success of neural networks from grid-like data structures, such as images and sequences, to more general and irregular structures. Traditional neural networks, like Convolutional Neural Networks (CNNs) or Recurrent Neural Networks (RNNs), are designed to work with data that has a fixed structure. However, many real-world problems involve data that can be better described by graphs, where nodes represent entities and edges represent relationships between those entities. 


\textbf{Graph Neural Networks} can be broadly defined as a class of neural networks designed to perform inference on data described by graphs. Formally, let $G = (V, E)$ represent a graph, where $V$ is the set of nodes (or vertices) and $E$ is the set of edges. Each node $v \in V$ can be associated with a feature vector $\mathbf{x}_v$, and each edge $(u, v) \in E$ may have an associated weight or feature vector $\mathbf{e}_{uv}$. The goal of a GNN is to learn a representation for each node (or sometimes for the entire graph) by aggregating and transforming the feature information from the node's local neighborhood in the graph.


In the GNN framework, the process of message passing is understood as a series of iterations in which each node updates its representation by exchanging information with its neighbors. To move from the abstract concept to a practical implementation, it is necessary to define the specific functions used for updating and aggregating node features.

The basic message passing operation, which simplifies the original GNN model proposed by [], is expressed by the following equation:
\[
\mathbf{h}_u^{(k)} = \sigma\left( \mathbf{W}_{\text{self}}^{(k)} \mathbf{h}_u^{(k-1)} + \mathbf{W}_{\text{neigh}}^{(k)} \sum_{v \in \mathcal{N}(u)} \mathbf{h}_v^{(k-1)} + \mathbf{b}^{(k)} \right)
\]

In this equation:
\begin{itemize}
    \item \( \mathbf{h}_u^{(k)} \) represents the updated feature vector of node \( u \) at layer \( k \).
    \item The term \( \mathbf{W}_{\text{self}}^{(k)} \mathbf{h}_u^{(k-1)} \) applies a transformation to the node's own feature vector from the previous layer, enabling the node to retain and modify its self-information. 
    \item The term \( \mathbf{W}_{\text{neigh}}^{(k)} \sum_{v \in \mathcal{N}(u)} \mathbf{h}_v^{(k-1)} \) aggregates the feature vectors of the neighboring nodes \( v \) in the set \( \mathcal{N}(u) \), and then applies a transformation via the weight matrix \( \mathbf{W}_{\text{neigh}}^{(k)} \) 
    \item \( \mathbf{b}^{(k)} \) is a bias term that can be added to the weighted sum, though it is sometimes omitted for simplicity. 
    \item  The non-linear function \( \sigma(\cdot) \), such as ReLU or tanh, is applied elementwise to introduce non-linearity into the model, which is essential for capturing complex patterns in the data.

\end{itemize}

% . This aggregation step is crucial as it allows the node to incorporate information from its local neighborhood.


In the context of Graph Neural Networks (GNNs), a Graph Convolutional Network (GCN) is a specialized model that applies the concept of convolution, widely used in image processing, to graphs. First introduced by Thomas Kipf and Max Welling in 2017, GCNs offer a method to perform deep learning on graph-structured data by extending traditional convolution operations to the irregular domain of graphs.

The fundamental idea behind GCNs is to create a spectral filter that operates on graph data. The filter's purpose is to combine features from a node's local neighborhood, taking into account the graph's structure. This process is mathematically formalized in the following way:

\[
\mathbf{H} = \sigma\left( \tilde{\mathbf{D}}^{-\frac{1}{2}} \tilde{\mathbf{A}} \tilde{\mathbf{D}}^{-\frac{1}{2}} \mathbf{X} \mathbf{\Theta} \right)
\]

Where:
\begin{itemize}
    \item \( \mathbf{H} \) represents the matrix of node representations after applying the GCN layer. Each row \( \mathbf{h}_u \) in \( \mathbf{H} \) corresponds to the updated feature vector for node \( u \). 
    \item \( \mathbf{X} \) is the matrix of input node features, where each row \( \mathbf{x}_u \) corresponds to the feature vector for node \( u \) before applying the GCN layer. 
    \item \( \sigma(\cdot) \) denotes a non-linear activation function, such as ReLU, applied elementwise to introduce non-linearity into the model. 
    \item \( \tilde{\mathbf{A}} \) is the adjacency matrix of the graph, with added self-loops to account for the node itself in the aggregation. 
    \item \( \tilde{\mathbf{D}} \) is the degree matrix of the graph, modified to include the self-loops. The degree matrix is diagonal, with each diagonal entry \( \tilde{D}_{ii} \) representing the degree of node \( i \) in the graph. 
    \item \( \mathbf{\Theta} \) is a matrix of trainable parameters, which is learned during the training process to optimize the model's performance.


\end{itemize}

The expression \( \tilde{\mathbf{D}}^{-\frac{1}{2}} \tilde{\mathbf{A}} \tilde{\mathbf{D}}^{-\frac{1}{2}} \) is a normalized version of the adjacency matrix, ensuring that the eigenvalues of the operation are bounded between 0 and 1. This normalization step is crucial as it prevents issues such as exploding or vanishing gradients during the training of deep networks.

Specifically, the adjacency matrix \( \tilde{\mathbf{A}} \) is defined as:

\[
\tilde{\mathbf{A}} = \mathbf{A} + \mathbf{I}
\]

where \( \mathbf{A} \) is the original adjacency matrix, and \( \mathbf{I} \) is the identity matrix. The identity matrix \( \mathbf{I} \) ensures that each node considers its own features when aggregating information from its neighbors.

The degree matrix \( \tilde{\mathbf{D}} \) is defined as:

\[
\tilde{D}_{ii} = \sum_{j \in V} \tilde{A}_{ij}
\]

where \( V \) represents the set of all nodes in the graph. The diagonal entries of \( \tilde{\mathbf{D}} \) correspond to the degree of each node, adjusted to account for the added self-loops.

\subsubsection{Convolution with Edge-Node Switching (CensNet)}

Graph Convolutional Networks (GCNs) have demonstrated considerable success in various graph-based machine learning tasks, particularly in their ability to generalize convolution operations to non-Euclidean data structures like graphs. GCNs operate by aggregating features from a node's neighbors, thereby capturing local neighborhood information and propagating it through the network layers. Despite their effectiveness, GCNs possess certain limitations that hinder their performance in more complex scenarios.

One notable limitation of GCNs is their reliance solely on node features during the convolution process. This approach disregards the information contained within edge features. By neglecting edge features, GCNs fail to fully exploit the underlying structure of the graph, potentially missing out on critical insights that could enhance model performance.

Furthermore, GCNs typically aggregate information from immediate neighbors only, which can limit their ability to capture long-range dependencies in large or densely connected graphs. This restriction can lead to an oversimplified representation of the graph structure, particularly in cases where the graph contains intricate patterns that require deeper and more nuanced analysis.

To overcome the limitations of traditional GCNs, which focus primarily on node features, CensNet introduces a novel approach that integrates both node and edge features into the graph convolution process. The CensNet framework consists of two primary types of layers: the \textit{node layer} and the \textit{edge layer}. These layers work in tandem to update node and edge embeddings alternately, leveraging the information from both nodes and edges in the graph.

The propagation rules in CensNet are designed to incorporate edge features into the convolution process, enabling a more comprehensive feature propagation across the graph. We define the normalized node adjacency matrix with self-loops as follows:

\begin{equation}
\tilde{\mathbf{A}}_v = \mathbf{D}_v^{-\frac{1}{2}} (\mathbf{A}_v + \mathbf{I}_{N_v}) \mathbf{D}_v^{-\frac{1}{2}},
\end{equation}

where $\mathbf{D}_v$ is the diagonal degree matrix of $\mathbf{A}_v + \mathbf{I}_{N_v}$.

\paragraph{Node Layer Propagation:}


In the $(l+1)$-th layer, the node features are updated using the following propagation rule:

\[
H^{(l+1)}_v = \sigma\left(T\Phi\left(H^{(l)}_e P_e\right)T^\top \odot \tilde{A}_v H^{(l)}_v W_v\right)
\]

Where, 

\begin{itemize}
    \item \( T \in \mathbb{R}^{N_v \times N_e} \) is a binary transformation matrix that represents the connections between nodes and edges. Each element \( T_{i,m} \) indicates whether edge \( m \) connects to node \( i \). Specifically, if edge \( m \) is connected to node \( i \), then \( T_{i,m} = 1 \); otherwise, \( T_{i,m} = 0 \). Given that each edge is formed by two nodes, every column of the matrix \( T \) will have exactly two elements equal to 1, corresponding to the two nodes that the edge connects. 
    \item \( H^{(l)}_e \) is the edge feature matrix from the \( l \)-th layer. \( P_e \) is a learnable vector of dimension \( d_e \), which acts as a weight for the edge features. The operation \( \Phi(H^{(l)}_e P_e) \) denotes the diagonalization of the vector \( H^{(l)}_e P_e \), converting it into a diagonal matrix where the elements of the vector are placed on the diagonal. 
    \item The Hadamard product, denoted by \( \odot \), represents element-wise multiplication between matrices. In this context, it combines the transformed edge features with the node adjacency matrix, integrating information from both the original graph and its line graph.
    \item \( \tilde{A}_v = D_v^{-\frac{1}{2}} (A_v + I_{N_v}) D_v^{-\frac{1}{2}} \) is the normalized adjacency matrix for nodes, where \( A_v \) is the original node adjacency matrix and \( I_{N_v} \) is the identity matrix that introduces self-loops. This normalization ensures that the contributions from each node's neighbors are appropriately scaled. 
    \item \( H^{(l)}_v \) represents the node feature matrix from the \( l \)-th layer. \( W_v \) is a learnable weight matrix that is applied to the node features during the propagation process. 
    \item The activation function \( \sigma \) (typically a non-linear function such as ReLU) is applied element-wise to the resulting matrix to introduce non-linearity into the model.

\end{itemize}


This expression can be interpreted as a way of fusing node and edge information. The matrix \( T \) maps edge features into the node domain, and this information is combined with the normalized node adjacency matrix \( \tilde{A}_v \). This fusion creates a new node adjacency matrix that incorporates both node and edge features, which is then used to update the node embeddings.

\paragraph{Edge Layer Propagation:}

Similarly, the normalized (Laplacianized) edge adjacency matrix is defined as:

\begin{equation}
\tilde{A}_e = D_e^{-\frac{1}{2}} \left(A_e + I_{N_e}\right) D_e^{-\frac{1}{2}},
\end{equation}

where \( D_e \) is the degree matrix corresponding to the edge adjacency matrix \( A_e + I_{N_e} \). The matrix \( \tilde{A}_e \) serves as the normalized version of the edge adjacency matrix, similar to how the node adjacency matrix is normalized. This normalization ensures that the influence of each edge is scaled appropriately, which is crucial for the stability of the propagation process.

The propagation rule for edge features is defined as follows:

\begin{equation}
H^{(l+1)}_e = \sigma\left(T^\top \Phi\left(H^{(l)}_v P_v\right) T \odot \tilde{A}_e H^{(l)}_e W_e\right).
\end{equation}

In this expression, the following components are involved:

\begin{itemize}
    \item \textbf{ \( T^\top \)} is the transpose of the binary transformation matrix \( T \) used in the node layer propagation. The matrix \( T^\top \) maps the node features back into the edge domain, allowing the edge features to be updated based on the node information.
    \item \( H^{(l)}_v \) is the node feature matrix from the \( l \)-th layer, and \( P_v \) is a learnable weight matrix for the nodes. The operation \( \Phi(H^{(l)}_v P_v) \) diagonalizes the product of node features and the learnable weights, similar to the transformation applied to edge features in the node layer propagation. 
    \item The matrix \( \tilde{A}_e \) is the normalized edge adjacency matrix, as defined above. This matrix integrates information about the connections between edges, analogous to how \( \tilde{A}_v \) handles connections between nodes. 
    \item \( H^{(l)}_e \) represents the edge feature matrix from the \( l \)-th layer, while \( W_e \) is a learnable weight matrix that is applied to the edge features during the propagation.
    \item The Hadamard product \( \odot \) element-wise multiplies the transformed node features with the edge adjacency matrix, merging the information from both domains.
    \item As in the node layer propagation, the activation function \( \sigma \) is applied element-wise to introduce non-linearity.

\end{itemize}

This propagation rule updates the edge embeddings by integrating information from the node features and the edge structure, thereby enhancing the expressiveness of the edge representations. The alternating updates between node and edge embeddings allow the model to effectively bridge signals across nodes and edges, leading to more robust and informative graph embeddings.




\subsection{Task-Dependent Loss Functions}

The output layer and corresponding loss functions in CensNet are designed to be task-dependent. For regression tasks, the loss function can be formalized as a regularized mean square error (MSE) loss. The MSE loss measures the difference between the predicted outcomes and the actual continuous values, providing a natural fit for regression problems.


We define the loss function for graph regression as follows:

\begin{equation}
\mathcal{L}(\Theta) = \sum_{l \in \mathcal{Y}_L} \sum_{f=1}^{F} \| Y_{lf} - \hat{Y}_{lf} \|^2_2 + \lambda \|\Theta\|_p,
\end{equation}

where:

\begin{itemize}
    \item \( Y_{lf} \) represents the true continuous value for the \( f \)-th feature of the \( l \)-th graph in the training set.
    \item \( \hat{Y}_{lf} \) is the predicted outcome generated from the final node hidden layer of the CensNet model.
    \item \( \| Y_{lf} - \hat{Y}_{lf} \|^2_2 \) is the squared difference between the true and predicted values, summed across all features and all graphs in the training set.
    \item \( \lambda \|\Theta\|_p \) is the regularization term, which helps control the model's complexity and prevents overfitting by penalizing large weights. The parameter \( \lambda \) controls the strength of the regularization, while \( p \) determines the type of regularization norm (e.g., \( p=2 \) for \( L_2 \) regularization).
\end{itemize}



\section{Linear formulation of the natural gas system} \label{sec:LinealCensnet_formulation}


Natural gas is a widely used energy resource, particularly for electricity generation. The natural gas system consists of a network of production centers, pipelines, compressor stations, storage facilities, and distribution points that ensure reliable gas delivery from producers to consumers. Mathematically, this system can be represented as a directed graph defined as $\mathcal{G}_f = \left\{\mathcal{N}_f, \mathcal{E}_f\right\}$ where $\mathcal{N}_f$ is the set of units within the gas system, and $ \mathcal{E}_f$ is the set of different elements linking them. This set of units includes gas supply nodes or wells $\mathcal{W} \subset \mathcal{N}_{f}$, gas demand nodes or users $\mathcal{U} \subset \mathcal{N}_{f}$, and gas storage facilities $\mathcal{S} \subset \mathcal{N}_{f}$. Similarly, the set of directed gas adjacency edges $\mathcal{A} = \left\{(n,m) \mid n,m\in\mathcal{N}_f \right\} \subset \mathcal{E}$ delineates the network structure through two kinds of transmission elements: transport pipelines $\mathcal{P} = \left\{p=(n,m) \mid n,m\in\mathcal{N}_f \right\}$ and compressing stations $\mathcal{C} = \left\{c=(n,m) \mid n,m\in\mathcal{N}_f \right\}$, so that $\mathcal{P}\cup\mathcal{C}=\mathcal{A}$ and $\mathcal{P}\cap\mathcal{C}=\emptyset$.


Natural gas transportation requires coordination to manage the flow through the different elements to maintain safe operating ranges. In optimizing this network, mathematical models minimize overall operating costs associated with the various stages of natural gas transportation, compression, storage, and handling unsupplied demand, ensuring compliance with technical and physical constraints. The function is expressed as:

\begin{equation} \label{eq:obj_func_integrated}
\begin{split}
\min_{\mathcal{P}, \mathcal{F}} \quad  \sum_{w \in \mathcal{W}} C_{w}^t {f_{w}^t} + \sum_{p \in \mathcal{P}} C_{p}^t {f_{p}^t} + \sum_{c \in \mathcal{C}} C_{c}^t {f_{c}^t} + \\ \sum_{u \in \mathcal{U}} C_{u}^{t} {f_{u}^{t}} + \quad \sum_{s \in \mathcal{S}} C_{s+}^{t} {f_{s+}^{t}}  + \sum_{s \in \mathcal{S}} C_{s-}^{t} {f_{s-}^{t}} + \sum_{s \in \mathcal{S}} C_{s}^{t} {V_{s}^{t}}
\end{split}
\end{equation}

The term $\sum_{w \in \mathcal{W}} C_{w}^t {f_{w}^t}$ represents the total cost of gas production at the wells, where $C_{w}^t$ denotes the cost per unit flow of gas at a specific well $w$ during time period $t$, and $f_{w}^t$ corresponds to the flow of gas from well $w$. Similarly, the transportation of gas through pipelines is captured by the term $\sum_{p \in \mathcal{P}} C_{p}^t {f_{p}^t}$, where $C_{p}^t$ is the cost per unit flow through pipeline $p$ during time period $t$, and $f_{p}^t$ represents the flow of gas through pipeline $p$. In addition, the total cost associated with gas compression at compressor stations is accounted for by $\sum_{c \in \mathcal{C}} C_{c}^t {f_{c}^t}$, where $C_{c}^t$ is the cost per unit flow at compressor station $c$ during time period $t$, and $f_{c}^t$ is the flow of gas through compressor station $c$.

Beyond production, transportation, and compression, the model also considers the costs related to unmet gas demand. The term $\sum_{u \in \mathcal{U}} C_{u}^{t} {f_{u}^{t}}$ reflects the penalty cost associated with unsupplied gas demand, where $C_{u}^{t}$ is the penalty cost per unit of unsupplied gas at location $u$ during time period $t$, and $f_{u}^{t}$ represents the volume of unmet demand. Additionally, the model includes costs related to storage operations. The term $\sum_{s \in \mathcal{S}} C_{s+}^{t} {f_{s+}^{t}}$ represents the cost of injecting gas into storage facilities, where $C_{s+}^{t}$ is the cost per unit flow into storage during time period $t$, and $f_{s+}^{t}$ denotes the flow into storage at facility $s$. Conversely, the cost of withdrawing gas from storage is captured by $\sum_{s \in \mathcal{S}} C_{s-}^{t} {f_{s-}^{t}}$, where $C_{s-}^{t}$ is the cost per unit flow out of storage during time period $t$, and $f_{s-}^{t}$ represents the flow out of storage at facility $s$. Finally, the model accounts for the storage cost itself through the term $\sum_{s \in \mathcal{S}} C_{s}^{t} {V_{s}^{t}}$, where $C_{s}^{t}$ is the cost per unit volume of gas stored at facility $s$ during time period $t$, and $V_{s}^{t}$ denotes the volume of gas stored. 


% Lastly, \Cref{eq:weymouth_cons}, known as the Weymouth equation, summarizes the physical behavior of gas flow through pipelines by relating the gas flow through the pipeline $f_{p}^t$ to the pressures at the ends of the pipeline $\pi_{n}^t, \pi_{m}^t \ \forall \ p = (n,m) \in\mathcal{P}$. The Weymouth equation defines a nonlinear, nonconvex, disjunctive flow-pressure relationship that hampers the optimization of the gas transport system.
% \begin{subequations}
\begin{alignat}{4}
    \underline{f_{w}^t} \leq f_{w}^t \leq \overline{f_{w}^t} &\quad \forall \ w \in \mathcal{W} \label{eq:well_limits} \\
    -\overline{f_{p}^t} \leq f_{p}^t \leq \overline{f_{p}^t} &\quad \forall \ p \in \mathcal{P} \label{eq:pipe_limits} \\
    % \underline{\pi_{n}^t} \leq \pi_{n}^t \leq \overline{\pi_{n}^t} &\quad \forall \ n \in \mathcal{N}_f \label{eq:press_limit} \\
    % \pi_{m}^t \leq \beta_{c}^t{\pi_{n}^t} &\quad \forall c=(n,m) \in \mathcal{C} \label{eq:comp_ratio} \\
    0 \leq f_{u}^{t} \leq \overline{f_{u}^{t}} &\quad \forall \ u \in \mathcal{U} \label{eq:dem_limit_gas} \\
    \sum_{m:(m,n)\in\mathcal{A}}{f_{m}^t} = \sum_{m':(n,m')\in\mathcal{A}}{f_{m'}^t} &\quad \forall \ n \in \mathcal{N}_f \label{eq:gas_balance} \\
    0 \leq f_{s+}^t \leq V_{0s} - \underline{V_s} &\quad \forall \ s \in \mathcal{S} \label{eq:sto_limit1} \\ 
    0 \leq f_{s-}^t \leq \overline{V_s} - V_{0s} &\quad \forall \ s \in \mathcal{S} \label{eq:sto_limit2} \\ 
    V_{s}^t = V_{s}^{t-1} + f_{s-}^{t-1} - f_{s+}^{t-1} &\quad \forall \ s \in \mathcal{S} \label{eq:sto_time}\\
    % sgn(f_{p}^t)(f_{p}^t)^2 = K_{nm}((\pi_{n}^t)^2-(\pi_{m}^t)^2) &\quad \forall \ p =(n,m) \in\mathcal{P} \label{eq:weymouth_cons}
\end{alignat}

The constraint set models the gas transportation system: \Cref{eq:well_limits} forces each production well to inject the flow $f_{w}^t$ over the technical minimum $\underline{f_{w}^t}$ and under the maximum capacity $\overline{f_{w}^t}$. \Cref{eq:pipe_limits} upper-bounds the gas flow through pipelines $f_{p}^t$ to the structural capacity $\overline{f_{p}^t}$. \Cref{eq:dem_limit_gas} ensures that the unsupplied demand $f_{u}^{t}$ is lower than the corresponding user demand $\overline{f_{u}^{t}}$. The nodal gas balance in \Cref{eq:gas_balance} guarantees that the gas entering the node $n$ equals the gas leaving it. \Cref{eq:sto_limit1,eq:sto_limit2} limit the gas injection $f_{s+}$  and extraction $f_{s-}$ rates at storage facilities according to the feasible operating range determined by the currently stored volume $V_{s}^t$, respectively. In turn, \Cref{eq:sto_time} balances the gas storage unit such that gas volume at operation period $t$ $V_{s}^t$ equals the volume from period $V_{s}^{t-1}$ plus the difference between injected $f_{s+}^{t-1}$ and extracted $f_{s+}^{t-1}$ gas flow, a fundamental constraint for modeling the dynamics of gas storage over time. 

\section{Experimental Setup}


In this experimental setup, we take the optimization model presented in the previous section and generate samples by introducing noise into the base values of two different gas networks. The noise levels range from 5\% to 25\%, applied to the parameters of the networks to simulate varying operating conditions. The first network is a small-scale test network consisting of 8 nodes, while the second represents the Colombian natural gas transportation system, a larger and more complex network. These networks will be used to evaluate the performance of the proposed model under different scenarios.

The generated samples will serve as the training data for a graph neural network (GNN) model designed to optimize the performance of the natural gas transportation system. This GNN model is built to focus on predicting node and edge-level characteristics, incorporating the structure of the network and its connectivity into the learning process. The model specifically penalizes deviations in node and edge losses, which directly impact the efficiency of gas flow through the system. To achieve this, the architecture is structured as a multi-layer neural network, with customizable depth (number of layers), channels, and dense layers, ensuring flexibility in adapting to both small-scale and large-scale networks, such as the Colombian system. This flexibility allows the network to generalize across different scenarios while retaining the ability to fine-tune for specific operating conditions.

The core components of the model include:

\begin{itemize}
    \item \textbf{Input Channels:} The model receives five types of input data:
        
        
    \begin{itemize}
        \item Node Features: A matrix $\in \mathbb{R}^{N \times 3}$, where $N$ is the number of nodes in the network, containing the features of each node. Each node feature includes the lower and upper limits for injected flow, as well as demanded flow.
        \item Node Laplacian: An adjacency matrix of size $N\times N$ encoding the graph structure of the nodes.
        \item Edge Laplacian: A matrix of size $E\times E$ encoding the connections between edges.
        \item Incidence Matrix: A matrix of size $N\times E$ representing the node-edge incidence relationship, mapping the flow of gas between nodes through edges.
        \item Edge Features: A matrix $\in \mathbb{R}^{N \times 3}$ that includes the features of pipelines and compressors in the gas network. Each edge feature includes the $K$ constant, the maximum compression ratio $\beta$, and the upper and lower flow limits.
    \end{itemize}

    \item \textbf{Normalization and Pre\-dense Layers:} The node and edge inputs undergo feature-wise normalization to standardize the data. Following this, the inputs are passed through two dense layers, each with N\_channels neurons. The purpose of these pre-dense layers is to transform the feature space before applying the convolutional layers 
    \item \textbf{Convolutional Layers:} The main body of the network consists of N\_layers convolutional blocks. Each block applies a CensNet convolution, which operates simultaneously on the node and edge features, taking into account the structural relationships encoded in the node and edge Laplacians and the incidence matrix. Each convolutional block is followed by batch normalization to stabilize the learning process. 
 The convolution layers update both node and edge features, enabling the model to capture the complex interactions between nodes and edges within the network. These layers allow the model to propagate information across the graph structure and learn how local features at one node or edge influence the broader system.   
    \item \textbf{Post-dense Layer:} After passing through the convolutional blocks, the node and edge features are further processed by a series of dense layers. The number of dense layers (N\_dense) is adjustable but typically set to two in this study. These layers further refine the learned features, enabling the model to output node and edge-level predictions. 
    \item \textbf{Losses and Outputs:} The final outputs of the network are the node-level and edge-level predictions. The node predictions correspond to the estimated flow at each node, while the edge predictions represent the flow along the edges. Both outputs are penalized based on their respective losses, which are calculated by comparing the predicted values to ground truth values and evaluating how well the physical constraints are respected.

The loss functions ensure that the model accurately predicts the node and edge flows while satisfying the physical constraints of the system. These constraints are essential for ensuring that the predicted flows are feasible within the operational limitations of the network. 
    \item \textbf{Model Optimization:} The model is trained using backpropagation with the Adam optimizer. The training process involves minimizing the node and edge loss functions, which penalize incorrect flow predictions and deviations from the expected behavior of the network.

\begin{figure}
    \centering
    \setlength\figurewidth{1\textwidth}        
    \setlength\figureheight{0.5\textwidth}
    \resizebox{\figurewidth}{\figureheight}{\begin{tikzpicture}[shorten >=1pt, ->, draw=black!50, node distance=1.5cm and 3.5cm, align=center]

    % Styles
    \tikzstyle{input} = [rectangle, draw, fill=orange!30, minimum width=3cm, minimum height=1cm]
    \tikzstyle{dense} = [rectangle, draw, fill=blue!30, minimum width=3cm, minimum height=1cm]
    \tikzstyle{conv} = [rectangle, draw, fill=green!30, minimum width=3cm, minimum height=1cm]
    \tikzstyle{output} = [rectangle, draw, fill=purple!30, minimum width=3cm, minimum height=1cm]
    \tikzstyle{loss} = [rectangle, draw, fill=red!30, minimum width=3cm, minimum height=1cm]
    \tikzstyle{arrow} = [->, thick]

    % Input Layer
    \node[input] (node_features) at (0,0) {\(\mathbf{X}\)};
    \node[input] (node_laplacian) [below of=node_features] {\(\mathbf{L}_v\)};
    \node[input] (edge_laplacian) [below of=node_laplacian] {\(\mathbf{L}_e\)};
    \node[input] (incidence_matrix) [below of=edge_laplacian] {\(\mathbf{T}\)};
    \node[input] (edge_features) [below of=incidence_matrix] {\(\mathbf{E}\)};

    % Normalization and Pre-dense Layer
    \node[dense] (norm_pre_dense) [right of=edge_laplacian, xshift=3cm] {Normalization \\ \& Pre-dense Layers};

    % Convolutional Layers
    \node[conv] (conv_layers) [right of=norm_pre_dense, xshift=3cm] {CensNet Blocks \\ (Convolutional Layers)};

    % Post-dense Layer
    \node[dense] (post_dense) [right of=conv_layers, xshift=3cm] {Post-dense Layers};

    % Outputs
    \node[output] (node_output) [right of=post_dense, xshift=3cm, yshift=3cm] {\(\hat{\mathbf{X}}_v\)};
    \node[output] (edge_output) [right of=post_dense, xshift=3cm, yshift=1cm] {\(\hat{\mathbf{X}}_e\)};
    \node[output] (balance_output) [right of=post_dense, xshift=3cm, yshift=-1cm] {\(\mathcal{J}_{\text{balance}}\)};
    \node[output] (weymouth_output) [right of=post_dense, xshift=3cm, yshift=-3cm] {\(\mathcal{J}_{\text{Weymouth}}\)};

    % Losses
    \node[loss] (node_loss) [right of=node_output, xshift=3cm] {Node Loss};
    \node[loss] (edge_loss) [right of=edge_output, xshift=3cm] {Edge Loss};
    \node[loss] (balance_loss) [right of=balance_output, xshift=3cm] {Balance Loss};
    \node[loss] (weymouth_loss) [right of=weymouth_output, xshift=3cm] {Weymouth Loss};

    % Arrows
    \draw[arrow] (node_features) -- (norm_pre_dense);
    \draw[arrow] (node_laplacian) -- (norm_pre_dense);
    \draw[arrow] (edge_laplacian) -- (norm_pre_dense);
    \draw[arrow] (incidence_matrix) -- (norm_pre_dense);
    \draw[arrow] (edge_features) -- (norm_pre_dense);

    \draw[arrow] (norm_pre_dense) -- (conv_layers);
    \draw[arrow] (conv_layers) -- (post_dense);

    \draw[arrow] (post_dense) -- (node_output);
    \draw[arrow] (post_dense) -- (edge_output);
    \draw[arrow] (post_dense) -- (balance_output);
    \draw[arrow] (post_dense) -- (weymouth_output);

    \draw[arrow] (node_output) -- (node_loss);
    \draw[arrow] (edge_output) -- (edge_loss);
    \draw[arrow] (balance_output) -- (balance_loss);
    \draw[arrow] (weymouth_output) -- (weymouth_loss);

\end{tikzpicture}

}
    % \begin{tikzpicture}[shorten >=1pt, ->, draw=black!50, node distance=1.5cm and 3.5cm, align=center]

    % Styles
    \tikzstyle{input} = [circle, draw, fill=green!50, minimum size=2em]
    \tikzstyle{hidden} = [circle, draw, fill=blue!50, minimum size=2em]
    \tikzstyle{output} = [circle, draw, fill=red!50, minimum size=2em]
    \tikzstyle{connection} = [->, thick]

    % Network Stage Labels
    \node[align=center] at (0,-0.4) {Input \\ Layer};
    \node[align=center] at (6,0.4) {Hidden \\ Layers};
    \node[align=center] at (12,-1.2) {Output \\ Layer};

    % Input Layer
    \foreach \i in {1,2,3}
        \node[input] (I\i) at (0,-\i*1.5) {$x_\i$};

    % Hidden Layer 1
    \foreach \i in {1,2,3,4}
        \node[hidden] (H1\i) at (3,-\i*1.5+0.75) {$z^{(1)}_\i$};

    % Hidden Layer 2
    \foreach \i in {1,2,3,4}
        \node[hidden] (H2\i) at (6,-\i*1.5+0.75) {$z^{(2)}_\i$};

    % Hidden Layer 3
    \foreach \i in {1,2,3,4}
        \node[hidden] (H3\i) at (9,-\i*1.5+0.75) {$z^{(3)}_\i$};

    % Output Layer
    \foreach \i in {1,2}
        \node[output] (O\i) at (12,-\i*1.5-0.75) {$\hat{y}_\i$};

    % Connections from Input to Hidden Layer 1
    \foreach \i in {1,2,3}
        \foreach \j in {1,2,3,4}
            \draw[connection] (I\i) -- (H1\j);

    % Connections from Hidden Layer 1 to Hidden Layer 2
    \foreach \i in {1,2,3,4}
        \foreach \j in {1,2,3,4}
            \draw[connection] (H1\i) -- (H2\j);

    % Connections from Hidden Layer 2 to Hidden Layer 3
    \foreach \i in {1,2,3,4}
        \foreach \j in {1,2,3,4}
            \draw[connection] (H2\i) -- (H3\j);

    % Connections from Hidden Layer 3 to Output Layer
    \foreach \i in {1,2,3,4}
        \foreach \j in {1,2}
            \draw[connection] (H3\i) -- (O\j);

\end{tikzpicture}

    \caption{}\label{fig:}
\end{figure}




\end{itemize}

\section{Results}


In this chapter, we present the results of the proposed graph neural network (GNN) model, focusing on the relationship between the predicted outputs and the actual observed values in the natural gas transportation networks. The evaluation includes both the 8-node test network and the Colombian natural gas system, with the goal of assessing the model's ability to predict key parameters under varying operational conditions. 


%
% \begin{figure}
%     \centering
%     \begin{tikzpicture}

\definecolor{darkgray176}{RGB}{176,176,176}
\definecolor{lightgray204}{RGB}{204,204,204}

\begin{axis}[
    colorbar,
    colorbar sampled,
    colorbar style={
        samples=8,
        ylabel={node\_id},
        ytick={0,1,...,7},
        yticklabels={0,1,2,3,4,5,6,7},
    },
    colormap={mymap}{[1pt]
        rgb(0pt)=(0.12156862745098,0.466666666666667,0.705882352941177)
        rgb(1pt)=(1,0.498039215686275,0.0549019607843137)
        rgb(2pt)=(0.172549019607843,0.627450980392157,0.172549019607843)
        rgb(3pt)=(0.83921568627451,0.152941176470588,0.156862745098039)
        rgb(4pt)=(0.580392156862745,0.403921568627451,0.741176470588235)
        rgb(5pt)=(0.549019607843137,0.337254901960784,0.294117647058824)
        rgb(6pt)=(0.890196078431372,0.466666666666667,0.76078431372549)
        rgb(7pt)=(0.498039215686275,0.498039215686275,0.498039215686275)
    },
legend cell align={left},
legend style={
  fill opacity=0.8,
  draw opacity=1,
  text opacity=1,
  at={(0.03,0.97)},
  anchor=north west,
  draw=lightgray204
},
point meta max=7,
point meta min=0,
tick align=outside,
tick pos=left,
title={},
x grid style={darkgray176},
xlabel={y true},
xmajorgrids,
xmin=-2.3894448262, xmax=50.1783413502,
% xtick style={color=black},
y grid style={darkgray176},
ylabel={y pred},
ymajorgrids,
ymin=-2.18082528710365, ymax=41.9992832481861,
ytick style={color=black}
]
\addplot [
    colormap={mymap}{[1pt]
        rgb(0pt)=(0.12156862745098,0.466666666666667,0.705882352941177)
        rgb(1pt)=(1,0.498039215686275,0.0549019607843137)
        rgb(2pt)=(0.172549019607843,0.627450980392157,0.172549019607843)
        rgb(3pt)=(0.83921568627451,0.152941176470588,0.156862745098039)
        rgb(4pt)=(0.580392156862745,0.403921568627451,0.741176470588235)
        rgb(5pt)=(0.549019607843137,0.337254901960784,0.294117647058824)
        rgb(6pt)=(0.890196078431372,0.466666666666667,0.76078431372549)
        rgb(7pt)=(0.498039215686275,0.498039215686275,0.498039215686275)
    },
    only marks,
    scatter,
    scatter src=explicit
]
table [x=x, y=y, meta=colordata]{%
x  y  colordata
41.603277614 39.7535934448242 0.0
0 0.0908881425857544 1.0
0 0.139073312282562 2.0
0 0.0316752195358276 3.0
0 -0.0294172167778015 4.0
0 -0.0293693542480469 5.0
0 0.00516742467880249 6.0
0 0.00205957889556885 7.0
37.07638211 38.3591461181641 0.0
0 0.0097118616104126 1.0
0 -0.0219908952713013 2.0
0 -0.0110254883766174 3.0
0 0.00467157363891602 4.0
0 0.00213146209716797 5.0
0 -0.00148683786392212 6.0
0 -0.00642764568328857 7.0
39.245039256 38.5584983825684 0.0
0 0.0252088308334351 1.0
0 0.12731671333313 2.0
0 0.0389952659606934 3.0
0 -0.0199683308601379 4.0
0 -0.0247794985771179 5.0
0 0.00470894575119019 6.0
0 0.00119262933731079 7.0
33.40222464 39.6044502258301 0.0
0 0.0798147916793823 1.0
0 0.0904387831687927 2.0
0 0.0590900778770447 3.0
0 -0.0234290957450867 4.0
0 -0.0263727903366089 5.0
0 0.00418353080749512 6.0
0 0.000778496265411377 7.0
40.177942425 38.2198257446289 0.0
0 0.00198698043823242 1.0
0 0.00119161605834961 2.0
0 -0.0142090320587158 3.0
0 0.00341099500656128 4.0
0 1.15036964416504e-05 5.0
0 0.000508427619934082 6.0
0 -0.00637626647949219 7.0
33.661272263 38.4802513122559 0.0
0 -0.00348472595214844 1.0
0 0.0167836546897888 2.0
0 0.0134863257408142 3.0
0 0.0085369348526001 4.0
0 0.00687175989151001 5.0
0 0.00102633237838745 6.0
0 -0.00393420457839966 7.0
41.918555681 38.3983917236328 0.0
0 -0.00454980134963989 1.0
0 0.0024217963218689 2.0
0 -0.000810742378234863 3.0
0 0.011707603931427 4.0
0 0.00938022136688232 5.0
0 0.00149333477020264 6.0
0 0.000492334365844727 7.0
44.067434787 38.4977531433105 0.0
0 0.00612330436706543 1.0
0 0.00236880779266357 2.0
0 0.00920677185058594 3.0
0 0.0120137929916382 4.0
0 0.0121665596961975 5.0
0 0.000962913036346436 6.0
0 0.000135302543640137 7.0
31.762979365 38.5268859863281 0.0
0 0.00982803106307983 1.0
0 -0.00357627868652344 2.0
0 0.00891649723052979 3.0
0 0.0133818984031677 4.0
0 0.0117686986923218 5.0
0 0.001045823097229 6.0
0 0.000113010406494141 7.0
41.030120159 38.5200500488281 0.0
0 -0.0298738479614258 1.0
0 -0.0113872885704041 2.0
0 0.0223647356033325 3.0
0 -0.0168793201446533 4.0
0 -0.0176181793212891 5.0
0 0.00564652681350708 6.0
0 0.00426644086837769 7.0
40.055714877 38.8306159973145 0.0
0 -7.34925270080566e-05 1.0
0 -0.0110159516334534 2.0
0 0.0156899690628052 3.0
0 0.000327527523040771 4.0
0 -0.00162017345428467 5.0
0 0.00265783071517944 6.0
0 -0.000417232513427734 7.0
40.714244155 38.4069595336914 0.0
0 0.00929087400436401 1.0
0 -0.0149266719818115 2.0
0 0.00734519958496094 3.0
0 0.00849682092666626 4.0
0 0.0102183222770691 5.0
0 0.000343441963195801 6.0
0 -0.00474280118942261 7.0
45.139727698 38.8884696960449 0.0
0 0.000342428684234619 1.0
0 0.00625211000442505 2.0
0 0.00714671611785889 3.0
0 0.00820708274841309 4.0
0 0.00730991363525391 5.0
0 0.00169193744659424 6.0
0 0.000112235546112061 7.0
40.583613059 39.263557434082 0.0
0 0.0559103488922119 1.0
0 0.0864608585834503 2.0
0 0.0712517499923706 3.0
0 -0.0344110727310181 4.0
0 -0.0397318601608276 5.0
0 0.00477957725524902 6.0
0 -0.00180310010910034 7.0
35.441291819 38.5217590332031 0.0
0 -0.00431793928146362 1.0
0 -0.0136868357658386 2.0
0 0.00991904735565186 3.0
0 -0.0215700268745422 4.0
0 -0.0221145749092102 5.0
0 0.00257503986358643 6.0
0 -0.00314760208129883 7.0
36.08841077 38.8402824401855 0.0
0 0.00536257028579712 1.0
0 0.0375674962997437 2.0
0 0.0245164632797241 3.0
0 -0.00596868991851807 4.0
0 -0.00369197130203247 5.0
0 0.00242996215820312 6.0
0 0.00174403190612793 7.0
40.494343306 38.6495323181152 0.0
0 -0.00675559043884277 1.0
0 -0.005912184715271 2.0
0 -0.00215411186218262 3.0
0 0.00915825366973877 4.0
0 0.00896823406219482 5.0
0 0.00157058238983154 6.0
0 0.00130730867385864 7.0
44.400196583 38.5482749938965 0.0
0 0.00762939453125 1.0
0 0.00384992361068726 2.0
0 0.0038907527923584 3.0
0 0.0148396492004395 4.0
0 0.0139151811599731 5.0
0 0.00187838077545166 6.0
0 0.000854194164276123 7.0
40.197028842 38.6482429504395 0.0
0 -0.00523370504379272 1.0
0 0.0368385314941406 2.0
0 0.0156623125076294 3.0
0 0.00713348388671875 4.0
0 0.00676637887954712 5.0
0 0.000627398490905762 6.0
0 -0.000256776809692383 7.0
39.931867517 38.7960472106934 0.0
0 -0.00283390283584595 1.0
0 -0.00708603858947754 2.0
0 -0.00414395332336426 3.0
0 0.00981372594833374 4.0
0 0.00716376304626465 5.0
0 0.0015750527381897 6.0
0 0.000669717788696289 7.0
37.28322745 38.2965774536133 0.0
0 0.00495332479476929 1.0
0 -0.00355356931686401 2.0
0 0.0140035152435303 3.0
0 0.011663019657135 4.0
0 0.00884330272674561 5.0
0 0.00177943706512451 6.0
0 0.00082862377166748 7.0
37.429390014 38.4880485534668 0.0
0 0.0135406255722046 1.0
0 0.00302684307098389 2.0
0 0.0197659134864807 3.0
0 -0.00780308246612549 4.0
0 -0.00855451822280884 5.0
0 0.00318622589111328 6.0
0 0.000750601291656494 7.0
39.499146179 38.6164703369141 0.0
0 0.00107491016387939 1.0
0 -0.00115358829498291 2.0
0 0.0128684639930725 3.0
0 0.0100945830345154 4.0
0 0.00789058208465576 5.0
0 0.000394701957702637 6.0
0 -0.00226247310638428 7.0
35.661352504 38.4398956298828 0.0
0 0.0203315615653992 1.0
0 0.0788954496383667 2.0
0 -0.0757306814193726 3.0
0 -0.0765065550804138 4.0
0 -0.0808522701263428 5.0
0 0.0057339072227478 6.0
0 0.00214129686355591 7.0
44.903668391 38.679759979248 0.0
0 0.0104649066925049 1.0
0 -0.00321555137634277 2.0
0 0.00385379791259766 3.0
0 0.0129628777503967 4.0
0 0.0148612260818481 5.0
0 0.00193989276885986 6.0
0 0.00181066989898682 7.0
34.118473664 39.0513191223145 0.0
0 0.0307442545890808 1.0
0 0.0471727848052979 2.0
0 0.0385069847106934 3.0
0 -0.0240370631217957 4.0
0 -0.0236438512802124 5.0
0 0.00260704755783081 6.0
0 0.011364221572876 7.0
36.466593361 38.987190246582 0.0
0 -0.0103698968887329 1.0
0 0.0800180435180664 2.0
0 0.0414098501205444 3.0
0 -0.0267103314399719 4.0
0 -0.0288975834846497 5.0
0 0.00566142797470093 6.0
0 0.0028914213180542 7.0
34.994247693 38.3411102294922 0.0
0 0.00688672065734863 1.0
0 -0.00696879625320435 2.0
0 0.0116889476776123 3.0
0 0.00811731815338135 4.0
0 0.0102164149284363 5.0
0 0.00122654438018799 6.0
0 0.00214976072311401 7.0
39.553868661 38.6714859008789 0.0
0 -0.00435274839401245 1.0
0 -0.000196635723114014 2.0
0 -0.00985980033874512 3.0
0 0.0103663206100464 4.0
0 0.0117907524108887 5.0
0 0.0014764666557312 6.0
0 0.00132846832275391 7.0
43.158887007 38.4424667358398 0.0
0 0.00920790433883667 1.0
0 -0.00932449102401733 2.0
0 0.0151568651199341 3.0
0 0.0135409832000732 4.0
0 0.00888550281524658 5.0
0 0.00103974342346191 6.0
0 -0.00307869911193848 7.0
36.693891456 39.0564842224121 0.0
0 0.00571656227111816 1.0
0 0.0139431953430176 2.0
0 0.00666135549545288 3.0
0 0.00349867343902588 4.0
0 0.000229835510253906 5.0
0 0.00237119197845459 6.0
0 -0.000682830810546875 7.0
37.419072323 38.5162734985352 0.0
0 0.0100835561752319 1.0
0 -0.00280958414077759 2.0
0 0.011082649230957 3.0
0 0.0125146508216858 4.0
0 0.0106852650642395 5.0
0 0.000358998775482178 6.0
0 -0.00108003616333008 7.0
42.391154887 37.6930847167969 0.0
0 0.00836706161499023 1.0
0 -0.0265249013900757 2.0
0 0.0109959840774536 3.0
0 -0.00843185186386108 4.0
0 -0.00866538286209106 5.0
0 0.00374925136566162 6.0
0 0.00252997875213623 7.0
39.976124969 39.3020973205566 0.0
0 0.0369466543197632 1.0
0 0.0207736492156982 2.0
0 0.00882303714752197 3.0
0 -0.0113847255706787 4.0
0 -0.0140944719314575 5.0
0 0.00480204820632935 6.0
0 0.00208133459091187 7.0
41.193094756 38.7121925354004 0.0
0 0.000246524810791016 1.0
0 0.0112533569335938 2.0
0 -0.00880825519561768 3.0
0 0.0115809440612793 4.0
0 0.00994521379470825 5.0
0 0.00153958797454834 6.0
0 0.000614643096923828 7.0
39.030467364 38.9754753112793 0.0
0 0.0133965015411377 1.0
0 0.0099719762802124 2.0
0 -0.0317625403404236 3.0
0 -0.0213124752044678 4.0
0 -0.0274962782859802 5.0
0 0.0034407377243042 6.0
0 -0.00381076335906982 7.0
39.254434013 38.8635673522949 0.0
0 -0.00899428129196167 1.0
0 0.028767466545105 2.0
0 -0.00187152624130249 3.0
0 0.00816291570663452 4.0
0 0.00780004262924194 5.0
0 0.00216001272201538 6.0
0 0.00177603960037231 7.0
41.474829476 38.490550994873 0.0
0 -0.00636947154998779 1.0
0 0.0157142877578735 2.0
0 -0.00421321392059326 3.0
0 0.0130150318145752 4.0
0 0.0113214254379272 5.0
0 0.00152826309204102 6.0
0 0.000341415405273438 7.0
36.065024548 38.5721321105957 0.0
0 0.00696414709091187 1.0
0 -0.00372463464736938 2.0
0 0.00787001848220825 3.0
0 0.00936907529830933 4.0
0 0.0126588344573975 5.0
0 0.00105565786361694 6.0
0 0.00142580270767212 7.0
33.547657937 39.1791458129883 0.0
0 0.0251626968383789 1.0
0 0.106061309576035 2.0
0 0.0769014954566956 3.0
0 -0.025641918182373 4.0
0 -0.0292165875434875 5.0
0 0.00278604030609131 6.0
0 -0.0029417872428894 7.0
35.860060933 38.5283851623535 0.0
0 -0.00231927633285522 1.0
0 -0.00571024417877197 2.0
0 -0.00389313697814941 3.0
0 0.00245767831802368 4.0
0 0.00144177675247192 5.0
0 0.00015634298324585 6.0
0 -0.00504642724990845 7.0
41.446935888 38.7161064147949 0.0
0 0.0399006009101868 1.0
0 0.118380934000015 2.0
0 -0.0819071531295776 3.0
0 -0.0782929062843323 4.0
0 -0.0900070071220398 5.0
0 0.00521427392959595 6.0
0 0.00152796506881714 7.0
39.09863661 38.6100082397461 0.0
0 -0.0173571109771729 1.0
0 0.0087469220161438 2.0
0 -0.0296517014503479 3.0
0 0.0122166275978088 4.0
0 0.00985217094421387 5.0
0 0.00263488292694092 6.0
0 0.00131088495254517 7.0
37.209660367 38.619556427002 0.0
0 -0.00431281328201294 1.0
0 0.0119017362594604 2.0
0 0.0246157646179199 3.0
0 0.00228214263916016 4.0
0 -0.00246334075927734 5.0
0 0.00261741876602173 6.0
0 -0.00559180974960327 7.0
37.318344805 38.4069023132324 0.0
0 0.00742429494857788 1.0
0 -0.00190436840057373 2.0
0 0.0183868408203125 3.0
0 0.00921815633773804 4.0
0 0.00853586196899414 5.0
0 0.000802695751190186 6.0
0 -0.00107866525650024 7.0
37.542238634 38.8763580322266 0.0
0 -0.00635802745819092 1.0
0 0.0211170315742493 2.0
0 0.00130510330200195 3.0
0 0.0125030279159546 4.0
0 0.00436419248580933 5.0
0 0.00310200452804565 6.0
0 0.00108665227890015 7.0
35.642638424 38.4852333068848 0.0
0 0.00880211591720581 1.0
0 -0.0149785280227661 2.0
0 0.00123864412307739 3.0
0 0.0102723836898804 4.0
0 0.00305092334747314 5.0
0 0.000930190086364746 6.0
0 -0.00708687305450439 7.0
47.4260719 38.6907234191895 0.0
0 -0.00334429740905762 1.0
0 -0.0110288262367249 2.0
0 -0.0155019164085388 3.0
0 -0.0171687602996826 4.0
0 -0.0232598185539246 5.0
0 0.00268292427062988 6.0
0 -0.00473958253860474 7.0
42.664227047 38.3595390319824 0.0
0 0.00736522674560547 1.0
0 0.169191211462021 2.0
0 -0.0193517208099365 3.0
0 -0.00202935934066772 4.0
0 -0.00715571641921997 5.0
0 0.000342905521392822 6.0
0 -0.00674760341644287 7.0
37.881981315 38.4804840087891 0.0
0 0.00734597444534302 1.0
0 0.0022200345993042 2.0
0 0.00960803031921387 3.0
0 0.0106253623962402 4.0
0 0.0120579600334167 5.0
0 0.00171905755996704 6.0
0 0.00124293565750122 7.0
44.031482397 38.7509765625 0.0
0 -0.0097581148147583 1.0
0 -0.0636894702911377 2.0
0 0.0015978217124939 3.0
0 -0.00409984588623047 4.0
0 -0.00204712152481079 5.0
0 0.00405192375183105 6.0
0 0.0043146014213562 7.0
40.215687867 38.6358032226562 0.0
0 0.00362402200698853 1.0
0 0.00420552492141724 2.0
0 0.000331521034240723 3.0
0 0.00297117233276367 4.0
0 0.000524163246154785 5.0
0 0.000945329666137695 6.0
0 -0.00605976581573486 7.0
35.401137883 38.9912109375 0.0
0 -0.00200784206390381 1.0
0 -0.000412046909332275 2.0
0 -0.0133203268051147 3.0
0 0.0117408633232117 4.0
0 0.00856119394302368 5.0
0 0.0021214485168457 6.0
0 -0.000339329242706299 7.0
40.23529515 38.3903541564941 0.0
0 0.0114810466766357 1.0
0 -0.00885409116744995 2.0
0 0.0132594108581543 3.0
0 0.0202721357345581 4.0
0 0.0152925848960876 5.0
0 0.000202775001525879 6.0
0 -0.00713396072387695 7.0
38.451822651 38.8377914428711 0.0
0 0.00414824485778809 1.0
0 -0.00228339433670044 2.0
0 0.00336885452270508 3.0
0 0.0121934413909912 4.0
0 0.00823593139648438 5.0
0 0.00146770477294922 6.0
0 -0.00342243909835815 7.0
41.356859392 38.3796310424805 0.0
0 0.00926601886749268 1.0
0 -0.0124003291130066 2.0
0 0.0076671838760376 3.0
0 0.0100225806236267 4.0
0 0.00797367095947266 5.0
0 0.00054556131362915 6.0
0 -0.00379371643066406 7.0
44.45602451 38.879566192627 0.0
0 0.00672298669815063 1.0
0 0.0123946666717529 2.0
0 -0.00913697481155396 3.0
0 0.0109050869941711 4.0
0 0.00889235734939575 5.0
0 0.00168204307556152 6.0
0 0.000749766826629639 7.0
36.344631329 38.5415077209473 0.0
0 0.00494116544723511 1.0
0 0.00470906496047974 2.0
0 0.00875461101531982 3.0
0 0.00978201627731323 4.0
0 0.00804340839385986 5.0
0 0.00102525949478149 6.0
0 -0.00314891338348389 7.0
35.989587139 39.0207290649414 0.0
0 -0.00257343053817749 1.0
0 -0.00773054361343384 2.0
0 -0.00264960527420044 3.0
0 0.00988805294036865 4.0
0 0.0075453519821167 5.0
0 0.00126409530639648 6.0
0 -0.000481665134429932 7.0
38.453275767 38.5509948730469 0.0
0 -0.00460779666900635 1.0
0 0.0267020463943481 2.0
0 -0.00305271148681641 3.0
0 0.00985264778137207 4.0
0 0.00969010591506958 5.0
0 0.00131440162658691 6.0
0 0.000271499156951904 7.0
36.388812201 38.5625762939453 0.0
0 0.00684624910354614 1.0
0 0.00579524040222168 2.0
0 0.00857532024383545 3.0
0 0.00900334119796753 4.0
0 0.0114079117774963 5.0
0 0.00167042016983032 6.0
0 0.00194859504699707 7.0
45.540732617 38.3559417724609 0.0
0 0.00538235902786255 1.0
0 0.00386589765548706 2.0
0 -0.00125622749328613 3.0
0 0.018405020236969 4.0
0 0.0137466788291931 5.0
0 0.00278580188751221 6.0
0 0.000956237316131592 7.0
34.414627117 38.921630859375 0.0
0 0.0324609279632568 1.0
0 0.0945376753807068 2.0
0 0.0671876668930054 3.0
0 -0.0498137474060059 4.0
0 -0.0511236786842346 5.0
0 0.0085570216178894 6.0
0 0.00632810592651367 7.0
39.317888978 38.3244590759277 0.0
0 -0.00816702842712402 1.0
0 0.0362486243247986 2.0
0 0.0158848762512207 3.0
0 0.00621277093887329 4.0
0 0.00248837471008301 5.0
0 0.00199019908905029 6.0
0 -0.00439673662185669 7.0
36.969505628 39.0306816101074 0.0
0 -0.00373166799545288 1.0
0 0.0273987650871277 2.0
0 -0.0015103816986084 3.0
0 0.00917762517929077 4.0
0 0.00721776485443115 5.0
0 0.00152438879013062 6.0
0 -0.000276803970336914 7.0
44.923617033 39.2973556518555 0.0
0 0.00290322303771973 1.0
0 0.0194564461708069 2.0
0 0.0318148732185364 3.0
0 -0.0022081732749939 4.0
0 -0.00278234481811523 5.0
0 0.00196754932403564 6.0
0 -0.000810801982879639 7.0
37.258159228 38.1323928833008 0.0
0 0.0120348930358887 1.0
0 -0.0394316911697388 2.0
0 -0.050422191619873 3.0
0 -0.0273675322532654 4.0
0 -0.0310018062591553 5.0
0 0.00130391120910645 6.0
0 -0.0026056170463562 7.0
47.488576672 38.4616012573242 0.0
0 0.013430118560791 1.0
0 0.0567920207977295 2.0
0 0.0599616765975952 3.0
0 -0.0214373469352722 4.0
0 -0.0249693393707275 5.0
0 0.00472307205200195 6.0
0 -0.000217974185943604 7.0
42.958399331 38.6427192687988 0.0
0 0.00441265106201172 1.0
0 0.00464749336242676 2.0
0 0.000767230987548828 3.0
0 0.0110027194023132 4.0
0 0.0115892887115479 5.0
0 0.00038456916809082 6.0
0 -0.000238478183746338 7.0
41.161203287 38.1443328857422 0.0
0 0.00848382711410522 1.0
0 -0.0180635452270508 2.0
0 -0.00918519496917725 3.0
0 0.00684559345245361 4.0
0 0.0074118971824646 5.0
0 0.000457406044006348 6.0
0 -0.00672441720962524 7.0
40.466838645 38.8036346435547 0.0
0 -0.00796282291412354 1.0
0 -0.0537744164466858 2.0
0 0.0214377641677856 3.0
0 -0.0128311514854431 4.0
0 -0.0138890147209167 5.0
0 0.00499492883682251 6.0
0 0.00396305322647095 7.0
33.757389454 38.6834144592285 0.0
0 0.028232216835022 1.0
0 0.0404295325279236 2.0
0 0.0279163718223572 3.0
0 -0.0216402411460876 4.0
0 -0.0251529216766357 5.0
0 0.00418293476104736 6.0
0 -0.0031440258026123 7.0
43.437770592 38.343563079834 0.0
0 0.00211387872695923 1.0
0 -0.000490009784698486 2.0
0 -0.00304973125457764 3.0
0 0.00690853595733643 4.0
0 0.00559860467910767 5.0
0 0.000310838222503662 6.0
0 -0.00541353225708008 7.0
39.308931201 38.8775672912598 0.0
0 0.00970524549484253 1.0
0 0.00253796577453613 2.0
0 -0.0164980292320251 3.0
0 0.0100165009498596 4.0
0 0.00814670324325562 5.0
0 0.00208121538162231 6.0
0 0.00108212232589722 7.0
40.600604276 38.547607421875 0.0
0 0.00340026617050171 1.0
0 0.00650149583816528 2.0
0 -0.00628137588500977 3.0
0 0.0181875228881836 4.0
0 0.0199186205863953 5.0
0 0.00151854753494263 6.0
0 0.000111699104309082 7.0
31.520377836 38.581787109375 0.0
0 -0.00357002019882202 1.0
0 0.00153231620788574 2.0
0 0.002410888671875 3.0
0 0.00847429037094116 4.0
0 0.0135117173194885 5.0
0 -0.000303924083709717 6.0
0 0.000370502471923828 7.0
39.969125591 38.6257934570312 0.0
0 -0.00313878059387207 1.0
0 0.0176121592521667 2.0
0 -0.000237464904785156 3.0
0 -0.0116665363311768 4.0
0 -0.0119089484214783 5.0
0 -4.19020652770996e-05 6.0
0 -0.00395911931991577 7.0
30.912106182 38.2456321716309 0.0
0 -0.00355678796768188 1.0
0 0.0208463668823242 2.0
0 0.00883865356445312 3.0
0 0.0121186971664429 4.0
0 0.0138341188430786 5.0
0 0.00130367279052734 6.0
0 0.000901222229003906 7.0
45.180443811 38.3854637145996 0.0
0 -0.00685745477676392 1.0
0 0.0200044512748718 2.0
0 0.0115278959274292 3.0
0 -0.00447863340377808 4.0
0 -0.00777232646942139 5.0
0 0.00198572874069214 6.0
0 -0.00468242168426514 7.0
34.99216559 38.5155220031738 0.0
0 0.00417304039001465 1.0
0 0.0240988731384277 2.0
0 0.0120140314102173 3.0
0 -0.00103104114532471 4.0
0 -0.00382125377655029 5.0
0 0.00195038318634033 6.0
0 -0.0034911036491394 7.0
40.933938659 38.6754302978516 0.0
0 0.00394058227539062 1.0
0 0.0120728611946106 2.0
0 0.00304794311523438 3.0
0 -0.00491464138031006 4.0
0 -0.00522351264953613 5.0
0 0.00198507308959961 6.0
0 -0.00257027149200439 7.0
40.317005952 38.9766120910645 0.0
0 0.00107008218765259 1.0
0 0.00469475984573364 2.0
0 -0.010001540184021 3.0
0 0.00944429636001587 4.0
0 0.00491970777511597 5.0
0 0.00197219848632812 6.0
0 0.000297665596008301 7.0
40.010379617 38.5953788757324 0.0
0 -0.00856190919876099 1.0
0 0.0215948820114136 2.0
0 -0.000319242477416992 3.0
0 0.0154175162315369 4.0
0 0.00876986980438232 5.0
0 0.00185507535934448 6.0
0 -0.00385445356369019 7.0
38.685596488 38.8159713745117 0.0
0 -0.00294101238250732 1.0
0 0.00725901126861572 2.0
0 -0.0348896980285645 3.0
0 0.0134746432304382 4.0
0 0.00706773996353149 5.0
0 0.00252926349639893 6.0
0 -0.00183933973312378 7.0
30.278486108 39.0315551757812 0.0
0 -0.00075840950012207 1.0
0 0.0177011489868164 2.0
0 0.0114507675170898 3.0
0 0.0057494044303894 4.0
0 0.00559014081954956 5.0
0 0.00110125541687012 6.0
0 7.3552131652832e-05 7.0
40.311093813 37.9903717041016 0.0
0 0.000359892845153809 1.0
0 0.00946557521820068 2.0
0 0.00448530912399292 3.0
0 0.010830819606781 4.0
0 0.00182676315307617 5.0
0 0.00240612030029297 6.0
0 -0.00593709945678711 7.0
35.760290532 38.7118949890137 0.0
0 0.00812315940856934 1.0
0 0.0297279953956604 2.0
0 -0.0139551758766174 3.0
0 -0.014398992061615 4.0
0 -0.020226776599884 5.0
0 0.00332540273666382 6.0
0 -0.00403028726577759 7.0
40.236813436 38.5688896179199 0.0
0 -0.00127357244491577 1.0
0 -0.00444197654724121 2.0
0 0.000873267650604248 3.0
0 0.00886917114257812 4.0
0 0.0087096095085144 5.0
0 0.00108206272125244 6.0
0 0.000600993633270264 7.0
43.948095531 38.5622749328613 0.0
0 0.00688648223876953 1.0
0 0.0145112872123718 2.0
0 -0.00303208827972412 3.0
0 0.0120838284492493 4.0
0 0.0151082277297974 5.0
0 0.000646591186523438 6.0
0 9.0181827545166e-05 7.0
40.202612729 38.9503021240234 0.0
0 0.00446075201034546 1.0
0 0.00278854370117188 2.0
0 -0.00881409645080566 3.0
0 0.00978708267211914 4.0
0 0.0118529200553894 5.0
0 0.00184690952301025 6.0
0 0.00139963626861572 7.0
41.441227091 38.7626190185547 0.0
0 0.00998771190643311 1.0
0 0.014593780040741 2.0
0 0.037828803062439 3.0
0 -0.0178765058517456 4.0
0 -0.0181952714920044 5.0
0 0.00372391939163208 6.0
0 -0.000807821750640869 7.0
40.234798657 38.8709907531738 0.0
0 0.0301759243011475 1.0
0 0.0436609983444214 2.0
0 -0.0280125141143799 3.0
0 -0.0313872694969177 4.0
0 -0.0318161249160767 5.0
0 0.00301194190979004 6.0
0 -0.00238752365112305 7.0
40.268402112 38.6101875305176 0.0
0 0.0056917667388916 1.0
0 0.0238692760467529 2.0
0 0.016819953918457 3.0
0 0.00480717420578003 4.0
0 0.00369954109191895 5.0
0 0.00116813182830811 6.0
0 -0.00442075729370117 7.0
34.447402114 38.8466873168945 0.0
0 0.0301073789596558 1.0
0 0.00375080108642578 2.0
0 0.00299829244613647 3.0
0 -0.0253135561943054 4.0
0 -0.0322710871696472 5.0
0 0.0043519139289856 6.0
0 -0.00207918882369995 7.0
40.416533901 38.6745910644531 0.0
0 0.00224298238754272 1.0
0 -0.000726759433746338 2.0
0 0.0128074884414673 3.0
0 0.0108352303504944 4.0
0 0.0103694200515747 5.0
0 0.000999271869659424 6.0
0 -0.00330287218093872 7.0
40.837884543 38.8132362365723 0.0
0 0.00295329093933105 1.0
0 -0.00834047794342041 2.0
0 0.0268865823745728 3.0
0 0.000196278095245361 4.0
0 -0.005332350730896 5.0
0 0.00294673442840576 6.0
0 -0.00294685363769531 7.0
39.823092444 37.8871421813965 0.0
0 0.000456392765045166 1.0
0 -0.00625967979431152 2.0
0 -0.0234071016311646 3.0
0 -0.00285905599594116 4.0
0 -0.00193220376968384 5.0
0 -0.00192928314208984 6.0
0 -0.00620365142822266 7.0
40.41620498 38.7058410644531 0.0
0 -0.00521928071975708 1.0
0 0.000878453254699707 2.0
0 0.0232111811637878 3.0
0 0.0046045184135437 4.0
0 -0.00023186206817627 5.0
0 0.00172221660614014 6.0
0 -0.004547119140625 7.0
33.388504976 38.8013496398926 0.0
0 -0.00490099191665649 1.0
0 0.00648212432861328 2.0
0 -0.0126960277557373 3.0
0 0.015871524810791 4.0
0 0.0092160701751709 5.0
0 0.00187402963638306 6.0
0 -0.00270891189575195 7.0
42.251691394 38.6184730529785 0.0
0 0.0312210917472839 1.0
0 0.00582593679428101 2.0
0 0.0208268165588379 3.0
0 -0.00345629453659058 4.0
0 -0.00656622648239136 5.0
0 0.00353902578353882 6.0
0 -0.00196647644042969 7.0
32.552772789 38.5892791748047 0.0
0 -0.00613367557525635 1.0
0 0.0275079011917114 2.0
0 -0.0156198143959045 3.0
0 -0.0119390487670898 4.0
0 -0.00980442762374878 5.0
0 0.000374138355255127 6.0
0 -0.00487136840820312 7.0
43.396109959 38.253360748291 0.0
0 0.0250413417816162 1.0
0 0.020060658454895 2.0
0 -0.0217271447181702 3.0
0 0.00134474039077759 4.0
0 -0.00472259521484375 5.0
0 0.00346243381500244 6.0
0 -0.00104355812072754 7.0
31.629368094 38.9830894470215 0.0
0 0.00522613525390625 1.0
0 0.000475287437438965 2.0
0 -0.011349081993103 3.0
0 0.01133793592453 4.0
0 0.0121121406555176 5.0
0 0.00200116634368896 6.0
0 0.00154191255569458 7.0
37.779811495 38.5338363647461 0.0
0 -0.0108740329742432 1.0
0 -0.0369836091995239 2.0
0 0.0149946808815002 3.0
0 -0.00636374950408936 4.0
0 -0.00593376159667969 5.0
0 0.00299978256225586 6.0
0 0.00320684909820557 7.0
34.24652714 38.4251098632812 0.0
0 0.00578612089157104 1.0
0 -0.00152945518493652 2.0
0 0.0102679133415222 3.0
0 0.0122678875923157 4.0
0 0.0114915370941162 5.0
0 0.00135761499404907 6.0
0 0.000555694103240967 7.0
40.003780762 39.0511894226074 0.0
0 -0.0189368724822998 1.0
0 0.0200136303901672 2.0
0 -0.0185563564300537 3.0
0 0.0121176242828369 4.0
0 0.0117418766021729 5.0
0 0.00249522924423218 6.0
0 0.00155913829803467 7.0
31.970353396 38.9871673583984 0.0
0 -0.00729846954345703 1.0
0 0.0101170539855957 2.0
0 -0.0196437239646912 3.0
0 0.0132386088371277 4.0
0 0.0108768343925476 5.0
0 0.00228655338287354 6.0
0 0.0010066032409668 7.0
37.935925517 38.9553871154785 0.0
0 -0.00871318578720093 1.0
0 0.0280109643936157 2.0
0 -0.025648832321167 3.0
0 0.0104709267616272 4.0
0 0.0101121068000793 5.0
0 0.00271910429000854 6.0
0 0.00231039524078369 7.0
37.4568261 38.9331321716309 0.0
0 -0.00507557392120361 1.0
0 -0.000133514404296875 2.0
0 0.00147175788879395 3.0
0 0.00708097219467163 4.0
0 0.00657325983047485 5.0
0 0.00235539674758911 6.0
0 0.0013958215713501 7.0
34.938845635 38.2435569763184 0.0
0 0.00165331363677979 1.0
0 -0.00227940082550049 2.0
0 0.00175893306732178 3.0
0 0.0102068185806274 4.0
0 0.00741416215896606 5.0
0 0.00100028514862061 6.0
0 -0.00547868013381958 7.0
33.644682189 39.0608100891113 0.0
0 0.0215523242950439 1.0
0 0.0741968750953674 2.0
0 0.0108304619789124 3.0
0 -0.00966185331344604 4.0
0 -0.0103352665901184 5.0
0 0.00338512659072876 6.0
0 0.000349283218383789 7.0
43.152629724 38.6673278808594 0.0
0 -0.015493631362915 1.0
0 0.0307546257972717 2.0
0 0.00604945421218872 3.0
0 0.00886774063110352 4.0
0 0.010189414024353 5.0
0 0.00204598903656006 6.0
0 0.00179809331893921 7.0
29.29034141 38.7699356079102 0.0
0 0.0308585166931152 1.0
0 0.0640302300453186 2.0
0 0.0491317510604858 3.0
0 -0.0297149419784546 4.0
0 -0.0343635082244873 5.0
0 0.00418633222579956 6.0
0 -0.00212037563323975 7.0
36.529945045 38.9532089233398 0.0
0 0.0035216212272644 1.0
0 -0.0379787087440491 2.0
0 0.0298681259155273 3.0
0 -0.011357307434082 4.0
0 -0.0170662999153137 5.0
0 0.00557559728622437 6.0
0 0.00102359056472778 7.0
39.490028245 38.2533340454102 0.0
0 0.00501155853271484 1.0
0 0.0158194303512573 2.0
0 -0.0231708884239197 3.0
0 -0.00452101230621338 4.0
0 -0.0123972296714783 5.0
0 0.00276535749435425 6.0
0 -0.00519418716430664 7.0
34.709507968 38.3433647155762 0.0
0 0.00644981861114502 1.0
0 0.0225534439086914 2.0
0 0.0125881433486938 3.0
0 0.00445210933685303 4.0
0 0.00744950771331787 5.0
0 0.00118499994277954 6.0
0 -0.00236207246780396 7.0
45.017599836 38.2805442810059 0.0
0 0.00768566131591797 1.0
0 0.00354486703872681 2.0
0 -0.0179717540740967 3.0
0 -0.0011255145072937 4.0
0 -0.00349807739257812 5.0
0 0.000170648097991943 6.0
0 -0.00597703456878662 7.0
32.940669776 38.5253715515137 0.0
0 -0.00512635707855225 1.0
0 0.0313267707824707 2.0
0 -0.00119996070861816 3.0
0 -0.00267255306243896 4.0
0 -0.00358939170837402 5.0
0 -0.00215828418731689 6.0
0 -0.00541460514068604 7.0
44.072190389 38.780647277832 0.0
0 0.0452917814254761 1.0
0 -0.0119504332542419 2.0
0 0.0499585866928101 3.0
0 -0.0182744860649109 4.0
0 -0.0202071666717529 5.0
0 0.0040593147277832 6.0
0 -0.00122308731079102 7.0
35.422022456 38.5963363647461 0.0
0 1.99079513549805e-05 1.0
0 -0.0265647768974304 2.0
0 0.0021822452545166 3.0
0 -0.0189751386642456 4.0
0 -0.01978600025177 5.0
0 0.000371932983398438 6.0
0 -0.00270688533782959 7.0
36.153594117 39.0233688354492 0.0
0 -0.00477045774459839 1.0
0 0.0245848894119263 2.0
0 -0.0127134323120117 3.0
0 0.0103989243507385 4.0
0 0.0108460187911987 5.0
0 0.00221699476242065 6.0
0 0.00128203630447388 7.0
38.951377458 38.6069030761719 0.0
0 0.00339537858963013 1.0
0 -0.00315690040588379 2.0
0 0.0101865530014038 3.0
0 0.00888776779174805 4.0
0 0.00692009925842285 5.0
0 0.00140982866287231 6.0
0 -0.00287401676177979 7.0
40.644172477 38.4279823303223 0.0
0 -0.00347089767456055 1.0
0 -0.00854629278182983 2.0
0 -0.00977569818496704 3.0
0 0.00184166431427002 4.0
0 -0.00165468454360962 5.0
0 -0.00119912624359131 6.0
0 -0.00389707088470459 7.0
35.600531382 38.9765777587891 0.0
0 0.000101447105407715 1.0
0 -0.00169456005096436 2.0
0 -0.0105034112930298 3.0
0 0.0159684419631958 4.0
0 0.0144085288047791 5.0
0 0.00176870822906494 6.0
0 -0.000409126281738281 7.0
41.682033599 38.8813819885254 0.0
0 0.0172576904296875 1.0
0 0.0325183868408203 2.0
0 -0.0265422463417053 3.0
0 0.0114198923110962 4.0
0 0.00883734226226807 5.0
0 0.00202566385269165 6.0
0 -0.00066763162612915 7.0
39.697348938 38.9553298950195 0.0
0 0.00576400756835938 1.0
0 0.00627702474594116 2.0
0 -0.0105857849121094 3.0
0 0.0120393037796021 4.0
0 0.00906950235366821 5.0
0 0.00216853618621826 6.0
0 0.00109446048736572 7.0
40.978668493 37.8965797424316 0.0
0 -0.00433230400085449 1.0
0 0.00928252935409546 2.0
0 -0.0110325813293457 3.0
0 -0.010481059551239 4.0
0 -0.00912737846374512 5.0
0 0.000316202640533447 6.0
0 -0.00422096252441406 7.0
46.524483465 38.4956130981445 0.0
0 0.0072096586227417 1.0
0 -0.00128740072250366 2.0
0 0.00465559959411621 3.0
0 0.0165053009986877 4.0
0 0.00841742753982544 5.0
0 0.00209641456604004 6.0
0 -0.0013575553894043 7.0
39.683969876 38.3914260864258 0.0
0 0.00807690620422363 1.0
0 -0.0101870894432068 2.0
0 0.00545310974121094 3.0
0 0.014818549156189 4.0
0 0.0115963220596313 5.0
0 0.00100439786911011 6.0
0 -0.00206196308135986 7.0
46.294692767 39.0759162902832 0.0
0 0.0209541916847229 1.0
0 -0.00489115715026855 2.0
0 0.0438747406005859 3.0
0 -0.0146594643592834 4.0
0 -0.01612788438797 5.0
0 0.00321435928344727 6.0
0 -0.000924825668334961 7.0
44.785541844 38.8168334960938 0.0
0 0.00351965427398682 1.0
0 0.00326263904571533 2.0
0 0.0207355618476868 3.0
0 0.00808155536651611 4.0
0 0.00271213054656982 5.0
0 0.00201249122619629 6.0
0 -0.0048290491104126 7.0
33.676333572 38.2910499572754 0.0
0 0.00158512592315674 1.0
0 -0.00176751613616943 2.0
0 -0.00117731094360352 3.0
0 0.0104479193687439 4.0
0 0.0066227912902832 5.0
0 0.0014575719833374 6.0
0 -0.00686109066009521 7.0
35.625618922 38.0471572875977 0.0
0 0.00239014625549316 1.0
0 -0.0126388072967529 2.0
0 -0.0120697021484375 3.0
0 0.00128597021102905 4.0
0 0.00199592113494873 5.0
0 -0.000860452651977539 6.0
0 -0.00589412450790405 7.0
40.226157029 38.6796798706055 0.0
0 0.00153261423110962 1.0
0 0.00714719295501709 2.0
0 0.00271427631378174 3.0
0 0.0039246678352356 4.0
0 0.00176608562469482 5.0
0 0.0020751953125 6.0
0 -0.00623393058776855 7.0
44.003195919 38.717472076416 0.0
0 0.00164341926574707 1.0
0 0.00996559858322144 2.0
0 -0.00653207302093506 3.0
0 0.00345414876937866 4.0
0 0.00210237503051758 5.0
0 0.00132018327713013 6.0
0 -0.00519376993179321 7.0
33.88878315 38.4337196350098 0.0
0 0.00824373960494995 1.0
0 -0.000991284847259521 2.0
0 0.0105270147323608 3.0
0 0.0159042477607727 4.0
0 0.0113751888275146 5.0
0 0.00127136707305908 6.0
0 -0.00309556722640991 7.0
42.125521158 38.4778442382812 0.0
0 0.00595742464065552 1.0
0 0.007171630859375 2.0
0 0.0155715346336365 3.0
0 0.00982785224914551 4.0
0 0.00966256856918335 5.0
0 0.0015488862991333 6.0
0 0.00158339738845825 7.0
37.55067892 39.0216331481934 0.0
0 0.000815391540527344 1.0
0 0.0180549621582031 2.0
0 0.0161591768264771 3.0
0 0.00454628467559814 4.0
0 0.00465989112854004 5.0
0 0.00137084722518921 6.0
0 -0.000109493732452393 7.0
37.39948618 38.7834625244141 0.0
0 0.000694096088409424 1.0
0 0.017846941947937 2.0
0 -0.00839388370513916 3.0
0 0.0181208252906799 4.0
0 0.0139331221580505 5.0
0 0.00215917825698853 6.0
0 0.000113487243652344 7.0
34.341868993 38.6611785888672 0.0
0 -0.0150122046470642 1.0
0 -0.0378657579421997 2.0
0 0.0207475423812866 3.0
0 -0.000995099544525146 4.0
0 -0.00474607944488525 5.0
0 0.00366520881652832 6.0
0 0.00226479768753052 7.0
37.425355461 38.7103309631348 0.0
0 0.00776708126068115 1.0
0 0.00792419910430908 2.0
0 -0.00677996873855591 3.0
0 0.0113720297813416 4.0
0 0.00985175371170044 5.0
0 0.000823676586151123 6.0
0 -1.20997428894043e-05 7.0
37.288491912 38.4678955078125 0.0
0 0.00515389442443848 1.0
0 0.00180560350418091 2.0
0 0.0212438106536865 3.0
0 0.0104888081550598 4.0
0 0.00904351472854614 5.0
0 0.000123202800750732 6.0
0 -0.0016171932220459 7.0
39.20347441 38.805908203125 0.0
0 -0.00724637508392334 1.0
0 0.00258266925811768 2.0
0 0.0161644816398621 3.0
0 0.00277906656265259 4.0
0 0.000788569450378418 5.0
0 0.0032578706741333 6.0
0 0.000290453433990479 7.0
36.137884824 38.5297698974609 0.0
0 0.00461000204086304 1.0
0 -0.0104106664657593 2.0
0 -0.002341628074646 3.0
0 0.00298571586608887 4.0
0 0.00196605920791626 5.0
0 0.000843822956085205 6.0
0 -0.00134950876235962 7.0
41.655417288 38.3561172485352 0.0
0 0.00745469331741333 1.0
0 -0.00175750255584717 2.0
0 0.0158591866493225 3.0
0 0.0111903548240662 4.0
0 0.00902730226516724 5.0
0 0.000314235687255859 6.0
0 -0.00557076930999756 7.0
39.894978379 38.5850296020508 0.0
0 0.00680673122406006 1.0
0 0.00699466466903687 2.0
0 -0.0027390718460083 3.0
0 0.0178101062774658 4.0
0 0.00734114646911621 5.0
0 0.0022856593132019 6.0
0 -0.00147330760955811 7.0
39.390677385 38.8203659057617 0.0
0 0.0151079297065735 1.0
0 0.0463476181030273 2.0
0 -0.0157824158668518 3.0
0 -0.0195214152336121 4.0
0 -0.0244238376617432 5.0
0 0.00370627641677856 6.0
0 -0.0034632682800293 7.0
38.248327703 38.6163101196289 0.0
0 0.00582766532897949 1.0
0 -0.00413334369659424 2.0
0 0.00381135940551758 3.0
0 0.0143280029296875 4.0
0 0.0112537741661072 5.0
0 0.0016014575958252 6.0
0 -0.000128448009490967 7.0
38.538636426 38.5361137390137 0.0
0 0.00641006231307983 1.0
0 -0.00027698278427124 2.0
0 0.0049251914024353 3.0
0 0.0107793807983398 4.0
0 0.0084531307220459 5.0
0 0.00101703405380249 6.0
0 -0.000242471694946289 7.0
30.093865158 39.2019309997559 0.0
0 0.0353890657424927 1.0
0 0.00355738401412964 2.0
0 0.04957115650177 3.0
0 -0.0436264872550964 4.0
0 -0.0471203327178955 5.0
0 0.00451117753982544 6.0
0 -0.000642895698547363 7.0
36.059945316 38.318187713623 0.0
0 0.00228762626647949 1.0
0 0.0050504207611084 2.0
0 -0.00272685289382935 3.0
0 0.0147392749786377 4.0
0 0.0124824047088623 5.0
0 0.00258147716522217 6.0
0 0.00151604413986206 7.0
41.382686141 38.4142570495605 0.0
0 -0.00156998634338379 1.0
0 0.0204638242721558 2.0
0 -0.00764346122741699 3.0
0 -0.00123339891433716 4.0
0 -0.0007476806640625 5.0
0 -0.00071108341217041 6.0
0 -0.00559014081954956 7.0
37.374314239 38.6772499084473 0.0
0 -0.00320535898208618 1.0
0 0.00326782464981079 2.0
0 -0.00733256340026855 3.0
0 0.0150027275085449 4.0
0 0.0146105885505676 5.0
0 0.00185835361480713 6.0
0 0.000728189945220947 7.0
37.689487865 38.6625061035156 0.0
0 -0.00688505172729492 1.0
0 0.0156492590904236 2.0
0 0.0139175653457642 3.0
0 0.00787997245788574 4.0
0 0.00172066688537598 5.0
0 0.00320947170257568 6.0
0 0.000371813774108887 7.0
36.541540581 38.4495582580566 0.0
0 0.00823324918746948 1.0
0 -0.00128722190856934 2.0
0 0.0112419724464417 3.0
0 0.0117476582527161 4.0
0 0.0131450295448303 5.0
0 0.000376760959625244 6.0
0 -0.00190281867980957 7.0
36.791141656 38.3827018737793 0.0
0 0.00603717565536499 1.0
0 0.00242966413497925 2.0
0 0.00899404287338257 3.0
0 0.0119273662567139 4.0
0 0.011038064956665 5.0
0 0.00139141082763672 6.0
0 0.000935733318328857 7.0
39.891605825 38.3263244628906 0.0
0 -0.000929117202758789 1.0
0 0.0420316457748413 2.0
0 -0.0310762524604797 3.0
0 -0.00990122556686401 4.0
0 -0.00977462530136108 5.0
0 -0.0024992823600769 6.0
0 -0.00472760200500488 7.0
34.771445328 38.6938209533691 0.0
0 0.00257289409637451 1.0
0 0.0160443186759949 2.0
0 -0.00607508420944214 3.0
0 0.015845775604248 4.0
0 0.0155516862869263 5.0
0 0.00151389837265015 6.0
0 -0.000127732753753662 7.0
44.638557165 38.5900001525879 0.0
0 -0.00749975442886353 1.0
0 0.0173272490501404 2.0
0 0.0101034641265869 3.0
0 0.0125020146369934 4.0
0 0.00756698846817017 5.0
0 0.00205510854721069 6.0
0 -0.00224572420120239 7.0
36.096180936 38.1678276062012 0.0
0 0.00536233186721802 1.0
0 0.00875014066696167 2.0
0 0.0115242004394531 3.0
0 0.0147314667701721 4.0
0 0.00792396068572998 5.0
0 0.00153219699859619 6.0
0 -0.00242865085601807 7.0
45.744029538 38.1977348327637 0.0
0 0.00154680013656616 1.0
0 -0.00525921583175659 2.0
0 0.0133904218673706 3.0
0 0.00973719358444214 4.0
0 0.0101900100708008 5.0
0 0.00129318237304688 6.0
0 0.00124156475067139 7.0
36.287608705 38.4067802429199 0.0
0 -0.00326824188232422 1.0
0 0.00791627168655396 2.0
0 0.00300085544586182 3.0
0 0.0167179703712463 4.0
0 0.012957751750946 5.0
0 0.00116145610809326 6.0
0 -0.00291407108306885 7.0
42.801830865 38.5026664733887 0.0
0 0.00598233938217163 1.0
0 0.00537014007568359 2.0
0 0.00846278667449951 3.0
0 0.00354379415512085 4.0
0 0.00350403785705566 5.0
0 0.00145411491394043 6.0
0 -0.0010840892791748 7.0
47.558612578 38.3811569213867 0.0
0 -0.00544494390487671 1.0
0 0.0251234173774719 2.0
0 0.0158400535583496 3.0
0 0.00700610876083374 4.0
0 0.00496816635131836 5.0
0 0.000510692596435547 6.0
0 -0.00318229198455811 7.0
47.058562462 38.4057235717773 0.0
0 0.00507622957229614 1.0
0 0.0046621561050415 2.0
0 -0.00457650423049927 3.0
0 0.0249253511428833 4.0
0 0.0235775709152222 5.0
0 0.00243175029754639 6.0
0 0.000473499298095703 7.0
43.087048946 38.619270324707 0.0
0 -0.0106337070465088 1.0
0 -0.0265575647354126 2.0
0 0.015188455581665 3.0
0 -0.00168472528457642 4.0
0 -0.00724762678146362 5.0
0 0.00427258014678955 6.0
0 0.0025297999382019 7.0
33.170190568 38.9567565917969 0.0
0 0.00517016649246216 1.0
0 0.00459200143814087 2.0
0 0.0193129181861877 3.0
0 0.00718289613723755 4.0
0 0.00489509105682373 5.0
0 0.00101983547210693 6.0
0 -0.00283050537109375 7.0
34.957431523 38.8416709899902 0.0
0 0.000963389873504639 1.0
0 0.00420975685119629 2.0
0 0.0198730826377869 3.0
0 -0.00562077760696411 4.0
0 -0.00709086656570435 5.0
0 0.00256907939910889 6.0
0 -0.00235521793365479 7.0
36.320328519 38.8440856933594 0.0
0 -0.00225573778152466 1.0
0 -0.00953090190887451 2.0
0 0.00566196441650391 3.0
0 0.00649935007095337 4.0
0 0.00460940599441528 5.0
0 0.0024418830871582 6.0
0 0.000105619430541992 7.0
44.085721677 38.4182586669922 0.0
0 0.000100314617156982 1.0
0 0.0046502947807312 2.0
0 0.000212252140045166 3.0
0 0.00853204727172852 4.0
0 0.00654923915863037 5.0
0 0.000967621803283691 6.0
0 -0.00571078062057495 7.0
45.211814568 38.3988380432129 0.0
0 0.000818192958831787 1.0
0 0.0170605778694153 2.0
0 -0.0164785385131836 3.0
0 0.0199121236801147 4.0
0 0.0166866183280945 5.0
0 0.00196439027786255 6.0
0 -0.000163257122039795 7.0
38.71721827 38.8017578125 0.0
0 -0.00146806240081787 1.0
0 0.00902062654495239 2.0
0 -0.0155308246612549 3.0
0 0.00887376070022583 4.0
0 0.0110338926315308 5.0
0 0.00151538848876953 6.0
0 0.00038301944732666 7.0
42.190454953 38.4542961120605 0.0
0 -0.00158572196960449 1.0
0 -0.00745916366577148 2.0
0 -0.0216897130012512 3.0
0 -0.00159525871276855 4.0
0 -0.00380432605743408 5.0
0 9.28044319152832e-05 6.0
0 -0.00571072101593018 7.0
40.616038376 38.7430953979492 0.0
0 -0.00362867116928101 1.0
0 -0.00363141298294067 2.0
0 -0.0144767761230469 3.0
0 0.0116175413131714 4.0
0 0.0105224847793579 5.0
0 0.00140535831451416 6.0
0 0.000889956951141357 7.0
31.887764532 38.5140266418457 0.0
0 -0.00434798002243042 1.0
0 0.00756442546844482 2.0
0 -0.0390388369560242 3.0
0 -0.0211315751075745 4.0
0 -0.0261675715446472 5.0
0 0.00929063558578491 6.0
0 -0.00467920303344727 7.0
39.764867486 39.1465721130371 0.0
0 0.038346529006958 1.0
0 0.0927492082118988 2.0
0 0.0490598678588867 3.0
0 -0.041797399520874 4.0
0 -0.0453233122825623 5.0
0 0.00138425827026367 6.0
0 -0.000771045684814453 7.0
40.805009287 38.1030654907227 0.0
0 -0.0296259522438049 1.0
0 -0.00586938858032227 2.0
0 0.00417482852935791 3.0
0 -0.00110441446304321 4.0
0 -0.00454360246658325 5.0
0 0.00295650959014893 6.0
0 -0.000849902629852295 7.0
39.795600818 38.255054473877 0.0
0 0.00796306133270264 1.0
0 -0.00954651832580566 2.0
0 0.0155543088912964 3.0
0 0.0102896094322205 4.0
0 0.00733578205108643 5.0
0 0.00113248825073242 6.0
0 -0.00258749723434448 7.0
38.685439721 38.1099815368652 0.0
0 -0.0193047523498535 1.0
0 0.026334285736084 2.0
0 0.014123797416687 3.0
0 0.00438123941421509 4.0
0 0.00159591436386108 5.0
0 0.0030590295791626 6.0
0 0.00219011306762695 7.0
42.152656556 38.6447944641113 0.0
0 0.00958836078643799 1.0
0 0.00431197881698608 2.0
0 0.00279456377029419 3.0
0 0.0114231109619141 4.0
0 0.014231264591217 5.0
0 0.00242865085601807 6.0
0 0.00234651565551758 7.0
37.458343334 39.1002197265625 0.0
0 0.118244767189026 1.0
0 0.0132309198379517 2.0
0 0.179237484931946 3.0
0 -0.0956413745880127 4.0
0 -0.109352290630341 5.0
0 0.0108833312988281 6.0
0 0.0037427544593811 7.0
44.489633946 37.3663024902344 0.0
0 -0.000156283378601074 1.0
0 0.0174140930175781 2.0
0 -0.0660887956619263 3.0
0 -0.0274395942687988 4.0
0 -0.0323089957237244 5.0
0 0.0018419623374939 6.0
0 -0.00307917594909668 7.0
41.506640114 38.712272644043 0.0
0 0.00974112749099731 1.0
0 -0.0041663646697998 2.0
0 0.00747394561767578 3.0
0 0.0130130648612976 4.0
0 0.0114015340805054 5.0
0 0.00225138664245605 6.0
0 0.0011325478553772 7.0
33.262736569 38.4637565612793 0.0
0 -0.00413501262664795 1.0
0 -0.0077093243598938 2.0
0 0.00772899389266968 3.0
0 -0.0170835852622986 4.0
0 -0.0186598896980286 5.0
0 9.43541526794434e-05 6.0
0 -0.00329750776290894 7.0
47.069969764 39.1127662658691 0.0
0 0.0034937858581543 1.0
0 0.0224937796592712 2.0
0 0.0202521085739136 3.0
0 0.00589430332183838 4.0
0 0.00132709741592407 5.0
0 0.00247538089752197 6.0
0 -0.00191307067871094 7.0
43.288539967 38.7025947570801 0.0
0 0.0100835561752319 1.0
0 0.00247675180435181 2.0
0 0.0297904014587402 3.0
0 -0.00783008337020874 4.0
0 -0.00629055500030518 5.0
0 0.00282490253448486 6.0
0 0.00166058540344238 7.0
41.479216264 38.3655395507812 0.0
0 -0.0076107382774353 1.0
0 0.0249927639961243 2.0
0 0.0158962607383728 3.0
0 0.00533664226531982 4.0
0 0.00829720497131348 5.0
0 0.00181961059570312 6.0
0 0.00211477279663086 7.0
36.99162076 38.5888214111328 0.0
0 0.00175654888153076 1.0
0 0.023786187171936 2.0
0 0.0185081958770752 3.0
0 0.010552704334259 4.0
0 0.00637632608413696 5.0
0 0.00184732675552368 6.0
0 -0.0059053897857666 7.0
40.435886771 38.986686706543 0.0
0 -0.00245356559753418 1.0
0 0.0295779705047607 2.0
0 -0.0167503356933594 3.0
0 0.0110731720924377 4.0
0 0.00603735446929932 5.0
0 0.00212985277175903 6.0
0 -0.000238656997680664 7.0
37.140737383 38.4575042724609 0.0
0 0.00847417116165161 1.0
0 0.00707650184631348 2.0
0 0.00589668750762939 3.0
0 0.00967586040496826 4.0
0 0.0113044381141663 5.0
0 0.000883042812347412 6.0
0 0.00042259693145752 7.0
39.233907158 38.9101257324219 0.0
0 0.0550709962844849 1.0
0 0.148562252521515 2.0
0 0.0203211903572083 3.0
0 -0.0188177227973938 4.0
0 -0.0227159857749939 5.0
0 0.00519031286239624 6.0
0 0.000806093215942383 7.0
42.028602534 38.4750175476074 0.0
0 -0.00834649801254272 1.0
0 0.161355286836624 2.0
0 -0.0410969257354736 3.0
0 -0.0144053101539612 4.0
0 -0.0182055234909058 5.0
0 0.000284016132354736 6.0
0 -0.00558304786682129 7.0
39.783103652 39.0378684997559 0.0
0 0.0226157903671265 1.0
0 -0.0173406600952148 2.0
0 0.0640493631362915 3.0
0 -0.022208034992218 4.0
0 -0.0254800319671631 5.0
0 0.00419634580612183 6.0
0 -0.00270748138427734 7.0
39.425516455 38.1142044067383 0.0
0 0.003761887550354 1.0
0 0.00284242630004883 2.0
0 0.00697588920593262 3.0
0 0.00978696346282959 4.0
0 0.00723946094512939 5.0
0 0.00138962268829346 6.0
0 -0.00471216440200806 7.0
40.002559542 39.1183433532715 0.0
0 0.0111833214759827 1.0
0 0.0301645398139954 2.0
0 0.0249743461608887 3.0
0 0.00826179981231689 4.0
0 0.00183463096618652 5.0
0 0.00230365991592407 6.0
0 -0.00201362371444702 7.0
38.322665212 38.248119354248 0.0
0 0.00894296169281006 1.0
0 0.00682508945465088 2.0
0 -0.0240972638130188 3.0
0 -0.00541716814041138 4.0
0 -0.0107141733169556 5.0
0 0.0022168755531311 6.0
0 -0.0030866265296936 7.0
40.568600192 39.0411567687988 0.0
0 0.0145177841186523 1.0
0 0.0150798559188843 2.0
0 0.00254154205322266 3.0
0 -0.01081383228302 4.0
0 -0.014443039894104 5.0
0 0.00246632099151611 6.0
0 -0.00231659412384033 7.0
40.337689123 38.5351600646973 0.0
0 0.00957757234573364 1.0
0 0.0115842819213867 2.0
0 0.0130894184112549 3.0
0 0.00751912593841553 4.0
0 0.00370407104492188 5.0
0 0.00157725811004639 6.0
0 -0.00272804498672485 7.0
45.726007711 39.016544342041 0.0
0 -0.00627726316452026 1.0
0 0.0215297341346741 2.0
0 -0.0119922757148743 3.0
0 0.010688841342926 4.0
0 0.00892925262451172 5.0
0 0.0013124942779541 6.0
0 0.000753998756408691 7.0
36.675787374 38.2430000305176 0.0
0 -0.00158858299255371 1.0
0 0.0261760354042053 2.0
0 -0.00662976503372192 3.0
0 -0.0059894323348999 4.0
0 -0.00984197854995728 5.0
0 0.00290602445602417 6.0
0 -0.00482195615768433 7.0
32.235093382 38.4787483215332 0.0
0 0.00654047727584839 1.0
0 0.00283539295196533 2.0
0 0.0180583000183105 3.0
0 0.0143955945968628 4.0
0 0.00843143463134766 5.0
0 0.00196719169616699 6.0
0 -0.000648856163024902 7.0
39.438537732 38.3393020629883 0.0
0 0.00482273101806641 1.0
0 -0.0100158452987671 2.0
0 0.00550246238708496 3.0
0 0.00398647785186768 4.0
0 0.00508660078048706 5.0
0 0.000476419925689697 6.0
0 -0.000379323959350586 7.0
44.304614406 38.6942520141602 0.0
0 0.00724595785140991 1.0
0 0.00359833240509033 2.0
0 -0.00244873762130737 3.0
0 0.0168184041976929 4.0
0 0.018384575843811 5.0
0 0.00267505645751953 6.0
0 0.00242710113525391 7.0
32.373304325 38.7011947631836 0.0
0 -0.00139033794403076 1.0
0 0.0330175161361694 2.0
0 0.0167226791381836 3.0
0 -0.00249332189559937 4.0
0 -0.00695604085922241 5.0
0 0.00298500061035156 6.0
0 -0.00275629758834839 7.0
35.769379856 38.7898292541504 0.0
0 0.00197267532348633 1.0
0 0.029208242893219 2.0
0 0.0212281942367554 3.0
0 0.00471234321594238 4.0
0 0.00653988122940063 5.0
0 0.000956416130065918 6.0
0 -0.000651717185974121 7.0
38.348316661 38.5863151550293 0.0
0 -0.00654613971710205 1.0
0 0.0357071757316589 2.0
0 0.00473201274871826 3.0
0 0.00128424167633057 4.0
0 -0.00100201368331909 5.0
0 0.00222885608673096 6.0
0 -0.00584298372268677 7.0
30.534523433 39.0130577087402 0.0
0 0.00583195686340332 1.0
0 0.0315658450126648 2.0
0 0.0169404745101929 3.0
0 0.00761932134628296 4.0
0 0.0039747953414917 5.0
0 0.00238054990768433 6.0
0 -0.00270819664001465 7.0
41.665325899 38.7758674621582 0.0
0 0.000107109546661377 1.0
0 0.0110929012298584 2.0
0 0.0183548331260681 3.0
0 0.00898802280426025 4.0
0 0.00583487749099731 5.0
0 0.00194180011749268 6.0
0 -0.00388616323471069 7.0
39.125756769 38.7648773193359 0.0
0 -0.0030866265296936 1.0
0 0.00142174959182739 2.0
0 0.0009346604347229 3.0
0 0.00942152738571167 4.0
0 0.00966179370880127 5.0
0 0.00102788209915161 6.0
0 -0.00109249353408813 7.0
46.922469454 38.955696105957 0.0
0 -0.00348621606826782 1.0
0 0.000359296798706055 2.0
0 0.0120048522949219 3.0
0 0.00776892900466919 4.0
0 0.00494801998138428 5.0
0 0.00185239315032959 6.0
0 0.000420093536376953 7.0
38.082156965 38.3118515014648 0.0
0 0.0151529312133789 1.0
0 -0.0102242231369019 2.0
0 0.0122508406639099 3.0
0 -0.0131518244743347 4.0
0 -0.0139512419700623 5.0
0 0.00426769256591797 6.0
0 0.00295782089233398 7.0
35.281773346 38.6570739746094 0.0
0 -0.00272595882415771 1.0
0 0.0289946794509888 2.0
0 0.00509798526763916 3.0
0 -0.00448018312454224 4.0
0 -0.00222653150558472 5.0
0 0.000601649284362793 6.0
0 -0.00430136919021606 7.0
39.676820236 38.989990234375 0.0
0 0.00472760200500488 1.0
0 0.0294286012649536 2.0
0 0.0172818303108215 3.0
0 0.00415349006652832 4.0
0 0.00443726778030396 5.0
0 0.00188541412353516 6.0
0 -0.000945448875427246 7.0
41.049027324 38.6982574462891 0.0
0 0.00343352556228638 1.0
0 0.0113016366958618 2.0
0 -0.00696408748626709 3.0
0 0.0125917196273804 4.0
0 0.00998294353485107 5.0
0 0.00137799978256226 6.0
0 0.000242471694946289 7.0
41.402616693 39.0029830932617 0.0
0 0.00863611698150635 1.0
0 0.087821900844574 2.0
0 0.0603731870651245 3.0
0 -0.0356480479240417 4.0
0 -0.0399312376976013 5.0
0 0.00590997934341431 6.0
0 -9.55462455749512e-05 7.0
30.3721478 38.6933746337891 0.0
0 -0.00103241205215454 1.0
0 0.00393533706665039 2.0
0 -0.0112425088882446 3.0
0 -0.00764316320419312 4.0
0 -0.0103350281715393 5.0
0 0.000120401382446289 6.0
0 -0.00539594888687134 7.0
41.111385117 38.5219116210938 0.0
0 0.00852882862091064 1.0
0 -0.00877469778060913 2.0
0 0.0205832123756409 3.0
0 0.0103094577789307 4.0
0 0.0121785402297974 5.0
0 0.000267446041107178 6.0
0 -0.000113844871520996 7.0
36.921893091 38.6074638366699 0.0
0 0.00677359104156494 1.0
0 -0.00750589370727539 2.0
0 0.00729429721832275 3.0
0 0.0145912766456604 4.0
0 0.0131081342697144 5.0
0 0.000901758670806885 6.0
0 -0.000841677188873291 7.0
41.531234868 38.1064682006836 0.0
0 0.0153313875198364 1.0
0 0.0151869654655457 2.0
0 -0.00373983383178711 3.0
0 -0.0103966593742371 4.0
0 -0.0106770396232605 5.0
0 0.00317627191543579 6.0
0 0.00247424840927124 7.0
36.809020967 38.4955902099609 0.0
0 0.00135868787765503 1.0
0 0.00796598196029663 2.0
0 -0.00358861684799194 3.0
0 -0.00392121076583862 4.0
0 -0.00673782825469971 5.0
0 0.00253629684448242 6.0
0 -0.00510793924331665 7.0
30.318323192 38.832633972168 0.0
0 -0.00723063945770264 1.0
0 0.0219334959983826 2.0
0 0.010265588760376 3.0
0 0.00689631700515747 4.0
0 0.00381928682327271 5.0
0 0.00178146362304688 6.0
0 -0.00287723541259766 7.0
37.371827278 39.0299491882324 0.0
0 0.0118033885955811 1.0
0 0.0467261672019958 2.0
0 0.0143779516220093 3.0
0 -0.0137546062469482 4.0
0 -0.0207595229148865 5.0
0 0.00627964735031128 6.0
0 0.00312989950180054 7.0
37.183742215 38.1556701660156 0.0
0 0.00127106904983521 1.0
0 0.0066341757774353 2.0
0 -0.0147396326065063 3.0
0 0.00577563047409058 4.0
0 0.0029301643371582 5.0
0 -0.00182098150253296 6.0
0 -0.00620943307876587 7.0
34.629310928 38.365364074707 0.0
0 0.00610154867172241 1.0
0 0.00508695840835571 2.0
0 0.0198239088058472 3.0
0 0.013192892074585 4.0
0 0.00678855180740356 5.0
0 0.00147479772567749 6.0
0 -0.00184899568557739 7.0
42.830967498 38.9176712036133 0.0
0 -0.000261783599853516 1.0
0 0.0146018862724304 2.0
0 -0.011113166809082 3.0
0 0.0185452699661255 4.0
0 0.00816124677658081 5.0
0 0.00284934043884277 6.0
0 -8.62479209899902e-05 7.0
34.694327372 38.3558731079102 0.0
0 0.00509011745452881 1.0
0 0.000261187553405762 2.0
0 0.0105280876159668 3.0
0 0.0111966133117676 4.0
0 0.0079917311668396 5.0
0 0.00138604640960693 6.0
0 -0.00158607959747314 7.0
37.611435874 39.1154098510742 0.0
0 -0.00671660900115967 1.0
0 0.0181795358657837 2.0
0 -0.000290572643280029 3.0
0 0.0106480121612549 4.0
0 0.00586879253387451 5.0
0 0.00247448682785034 6.0
0 0.000689864158630371 7.0
45.529166488 38.6235542297363 0.0
0 -0.00425219535827637 1.0
0 0.0054473876953125 2.0
0 0.0149739980697632 3.0
0 0.00727999210357666 4.0
0 0.00684750080108643 5.0
0 0.00113421678543091 6.0
0 -0.00142967700958252 7.0
40.26689599 38.7462539672852 0.0
0 -0.00593769550323486 1.0
0 -0.00408351421356201 2.0
0 0.00365918874740601 3.0
0 0.00439119338989258 4.0
0 0.00260257720947266 5.0
0 0.00294429063796997 6.0
0 0.00205498933792114 7.0
39.577958661 38.4459991455078 0.0
0 0.00205034017562866 1.0
0 0.00105744600296021 2.0
0 -0.00728344917297363 3.0
0 0.017594575881958 4.0
0 0.0188379287719727 5.0
0 0.00254732370376587 6.0
0 0.00236052274703979 7.0
34.564585703 38.8292350769043 0.0
0 -0.0269166231155396 1.0
0 0.136175036430359 2.0
0 0.053674578666687 3.0
0 -0.047590434551239 4.0
0 -0.0486240386962891 5.0
0 0.00632476806640625 6.0
0 0.00384056568145752 7.0
34.747667205 38.7812385559082 0.0
0 0.0150026679039001 1.0
0 0.00320255756378174 2.0
0 -0.0252484083175659 3.0
0 -0.0210506319999695 4.0
0 -0.0219604969024658 5.0
0 0.00306332111358643 6.0
0 -0.00150471925735474 7.0
44.886437438 38.5786972045898 0.0
0 -0.00230449438095093 1.0
0 0.00125682353973389 2.0
0 0.00380456447601318 3.0
0 0.0123153924942017 4.0
0 0.0131455063819885 5.0
0 0.000817179679870605 6.0
0 -6.07967376708984e-05 7.0
34.458606901 38.426197052002 0.0
0 0.0095527172088623 1.0
0 -0.0173845887184143 2.0
0 0.0205413103103638 3.0
0 0.013698399066925 4.0
0 0.00839120149612427 5.0
0 0.00135630369186401 6.0
0 -0.00317841768264771 7.0
44.985938301 38.635196685791 0.0
0 0.0398619174957275 1.0
0 0.0797373652458191 2.0
0 -0.0432404279708862 3.0
0 -0.0900113582611084 4.0
0 -0.0522611737251282 5.0
0 0.00559622049331665 6.0
0 0.00259339809417725 7.0
36.561528733 38.4419441223145 0.0
0 -0.00396859645843506 1.0
0 -0.0254343748092651 2.0
0 -0.0119385123252869 3.0
0 -0.0315827131271362 4.0
0 -0.0329383611679077 5.0
0 0.00273001194000244 6.0
0 -0.00205951929092407 7.0
35.585636345 38.4601249694824 0.0
0 -0.00341349840164185 1.0
0 0.0154292583465576 2.0
0 0.0149226188659668 3.0
0 0.00507611036300659 4.0
0 0.00384801626205444 5.0
0 0.000981569290161133 6.0
0 -0.0022132396697998 7.0
34.748633645 38.3581428527832 0.0
0 0.0046808123588562 1.0
0 0.00470662117004395 2.0
0 -0.0016711950302124 3.0
0 0.0164695382118225 4.0
0 0.0109076499938965 5.0
0 0.0013577938079834 6.0
0 -0.00368696451187134 7.0
33.997517461 38.4255676269531 0.0
0 -0.00385230779647827 1.0
0 0.0130151510238647 2.0
0 -0.00214368104934692 3.0
0 0.0129605531692505 4.0
0 0.0127876400947571 5.0
0 0.00150924921035767 6.0
0 0.000913619995117188 7.0
31.337837162 38.988941192627 0.0
0 0.00888556241989136 1.0
0 0.0149818062782288 2.0
0 0.0195881128311157 3.0
0 0.00459617376327515 4.0
0 0.00473707914352417 5.0
0 0.00146538019180298 6.0
0 0.000113725662231445 7.0
43.914614211 38.5083885192871 0.0
0 0.0040326714515686 1.0
0 -0.014149010181427 2.0
0 0.0157884955406189 3.0
0 0.0115993022918701 4.0
0 0.00837445259094238 5.0
0 0.00118738412857056 6.0
0 -0.000306010246276855 7.0
45.186574618 38.4571304321289 0.0
0 0.00369459390640259 1.0
0 -0.00190377235412598 2.0
0 0.00506937503814697 3.0
0 0.00619816780090332 4.0
0 0.00701636075973511 5.0
0 0.000674545764923096 6.0
0 -0.00296097993850708 7.0
35.268441876 38.6745338439941 0.0
0 0.0259495973587036 1.0
0 0.00516915321350098 2.0
0 0.0271381139755249 3.0
0 -0.0043373703956604 4.0
0 -0.00707775354385376 5.0
0 0.00333774089813232 6.0
0 -0.000590026378631592 7.0
32.333098982 38.339469909668 0.0
0 0.00736904144287109 1.0
0 -0.012636661529541 2.0
0 -0.00687289237976074 3.0
0 0.0058826208114624 4.0
0 0.000943958759307861 5.0
0 0.000460386276245117 6.0
0 -0.00598776340484619 7.0
38.894207338 38.5355949401855 0.0
0 -0.0060042142868042 1.0
0 0.0341095924377441 2.0
0 0.00643140077590942 3.0
0 -0.00107800960540771 4.0
0 -0.00261771678924561 5.0
0 -0.000578522682189941 6.0
0 -0.00298482179641724 7.0
32.941924742 38.3487243652344 0.0
0 0.01023268699646 1.0
0 0.0263230800628662 2.0
0 -0.056096076965332 3.0
0 -0.0248920917510986 4.0
0 -0.0269808173179626 5.0
0 0.00285071134567261 6.0
0 0.00860178470611572 7.0
40.2164929 37.7063751220703 0.0
0 -0.00206410884857178 1.0
0 -0.0241124033927917 2.0
0 -0.0163887739181519 3.0
0 -0.00840198993682861 4.0
0 -0.0127372145652771 5.0
0 0.00169253349304199 6.0
0 -0.0047113299369812 7.0
40.810463567 39.0738258361816 0.0
0 0.00107890367507935 1.0
0 0.000585675239562988 2.0
0 -0.0132269859313965 3.0
0 0.0105957388877869 4.0
0 0.0103799700737 5.0
0 0.00216841697692871 6.0
0 0.00124680995941162 7.0
33.558410457 38.8956298828125 0.0
0 0.00843101739883423 1.0
0 -0.0107095837593079 2.0
0 -0.0284059047698975 3.0
0 0.000358760356903076 4.0
0 0.00435709953308105 5.0
0 0.00229853391647339 6.0
0 0.00229942798614502 7.0
47.788896524 38.6363067626953 0.0
0 -0.00472438335418701 1.0
0 -0.0201917886734009 2.0
0 0.0364751219749451 3.0
0 -0.0135514140129089 4.0
0 -0.016156792640686 5.0
0 0.00404399633407593 6.0
0 -0.00392872095108032 7.0
33.642782256 38.8176574707031 0.0
0 -0.00302708148956299 1.0
0 0.00466156005859375 2.0
0 0.0140516757965088 3.0
0 0.00737649202346802 4.0
0 0.00594466924667358 5.0
0 0.0017513632774353 6.0
0 -0.00115036964416504 7.0
37.659684029 38.3499221801758 0.0
0 0.0080101490020752 1.0
0 0.000805974006652832 2.0
0 0.0185391902923584 3.0
0 0.0152295827865601 4.0
0 0.00819528102874756 5.0
0 0.00101685523986816 6.0
0 -0.00529128313064575 7.0
40.91969971 38.140266418457 0.0
0 0.00345683097839355 1.0
0 -0.00896137952804565 2.0
0 -0.0119954347610474 3.0
0 0.00301778316497803 4.0
0 0.00167751312255859 5.0
0 0.00145190954208374 6.0
0 -0.00629293918609619 7.0
42.622265927 38.811466217041 0.0
0 0.019258975982666 1.0
0 0.0364774465560913 2.0
0 -0.00904768705368042 3.0
0 -0.0298802852630615 4.0
0 -0.0317170023918152 5.0
0 0.00465452671051025 6.0
0 -0.00226563215255737 7.0
33.946966747 38.4802627563477 0.0
0 -0.00518739223480225 1.0
0 0.032974898815155 2.0
0 0.0197080373764038 3.0
0 0.00946873426437378 4.0
0 0.00464081764221191 5.0
0 0.000941216945648193 6.0
0 -0.00420850515365601 7.0
36.409182194 38.5286026000977 0.0
0 0.00702512264251709 1.0
0 -0.0194560289382935 2.0
0 0.0110805034637451 3.0
0 0.0136269330978394 4.0
0 0.0101454257965088 5.0
0 0.00085604190826416 6.0
0 -0.0050346851348877 7.0
40.593448001 38.5760307312012 0.0
0 0.0027921199798584 1.0
0 0.000395596027374268 2.0
0 -0.00285184383392334 3.0
0 0.0092308521270752 4.0
0 0.010448157787323 5.0
0 0.000471591949462891 6.0
0 0.00133585929870605 7.0
41.715980637 38.8361663818359 0.0
0 0.0497987866401672 1.0
0 0.0365809798240662 2.0
0 0.0144861936569214 3.0
0 -0.0297126173973083 4.0
0 -0.0379680395126343 5.0
0 0.00509655475616455 6.0
0 -0.00242024660110474 7.0
33.77037363 38.5182495117188 0.0
0 0.0106807351112366 1.0
0 0.083455502986908 2.0
0 -0.0791702270507812 3.0
0 -0.052739679813385 4.0
0 -0.0625720620155334 5.0
0 0.00684452056884766 6.0
0 -0.000840663909912109 7.0
46.074402277 38.5328330993652 0.0
0 0.0320846438407898 1.0
0 0.0307254195213318 2.0
0 0.0312415361404419 3.0
0 -0.0108807682991028 4.0
0 -0.0112593173980713 5.0
0 0.00355857610702515 6.0
0 0.00201642513275146 7.0
40.510394826 38.7015953063965 0.0
0 0.00823861360549927 1.0
0 -0.0077546238899231 2.0
0 0.046064555644989 3.0
0 -0.0181393027305603 4.0
0 -0.017975926399231 5.0
0 0.002483069896698 6.0
0 0.00150805711746216 7.0
39.244329381 38.7511940002441 0.0
0 0.0159115791320801 1.0
0 0.0022616982460022 2.0
0 0.0282448530197144 3.0
0 -0.00502586364746094 4.0
0 -0.00408065319061279 5.0
0 0.00202155113220215 6.0
0 0.000927567481994629 7.0
35.7171596 38.6546821594238 0.0
0 0.00409018993377686 1.0
0 -0.00489550828933716 2.0
0 -0.0026392936706543 3.0
0 0.0113359689712524 4.0
0 0.0123302340507507 5.0
0 0.000735223293304443 6.0
0 0.00068289041519165 7.0
39.754714125 38.8880348205566 0.0
0 -0.00539577007293701 1.0
0 -0.00136715173721313 2.0
0 -0.0127389430999756 3.0
0 0.00762850046157837 4.0
0 -0.000317215919494629 5.0
0 0.0031813383102417 6.0
0 -0.00161910057067871 7.0
41.796195011 38.343635559082 0.0
0 0.00867635011672974 1.0
0 -0.008350670337677 2.0
0 0.0206466317176819 3.0
0 0.0107426643371582 4.0
0 0.0104712247848511 5.0
0 0.000457584857940674 6.0
0 -0.000905036926269531 7.0
29.980660557 38.4174346923828 0.0
0 0.00467085838317871 1.0
0 -0.00495713949203491 2.0
0 0.0162566900253296 3.0
0 0.00935602188110352 4.0
0 0.00986051559448242 5.0
0 0.000848650932312012 6.0
0 0.000751495361328125 7.0
36.241260622 38.5575332641602 0.0
0 0.000705897808074951 1.0
0 0.00149893760681152 2.0
0 0.0100005865097046 3.0
0 0.01280277967453 4.0
0 0.0111420154571533 5.0
0 0.000396430492401123 6.0
0 -0.00308972597122192 7.0
38.129703213 38.4214096069336 0.0
0 0.00814342498779297 1.0
0 0.000246703624725342 2.0
0 0.0169895887374878 3.0
0 0.0135315656661987 4.0
0 0.010769248008728 5.0
0 0.00128799676895142 6.0
0 -0.0017850399017334 7.0
36.91522515 38.2449417114258 0.0
0 0.0027318000793457 1.0
0 -0.00151664018630981 2.0
0 0.00988304615020752 3.0
0 0.00847196578979492 4.0
0 0.0108878016471863 5.0
0 0.000810861587524414 6.0
0 0.0011824369430542 7.0
38.102908751 38.1841430664062 0.0
0 0.00115394592285156 1.0
0 0.00325274467468262 2.0
0 0.0126217603683472 3.0
0 0.0149384140968323 4.0
0 0.0158239006996155 5.0
0 0.000880837440490723 6.0
0 -8.7440013885498e-05 7.0
38.34829669 38.6496887207031 0.0
0 -0.00112295150756836 1.0
0 0.0135008692741394 2.0
0 0.00834870338439941 3.0
0 0.0102553367614746 4.0
0 0.00822752714157104 5.0
0 0.00185030698776245 6.0
0 0.00093388557434082 7.0
31.13885609 38.5502471923828 0.0
0 0.000279843807220459 1.0
0 0.00190377235412598 2.0
0 0.00924229621887207 3.0
0 0.0154766440391541 4.0
0 0.0117242932319641 5.0
0 0.00106501579284668 6.0
0 -0.00504231452941895 7.0
35.130346793 39.991096496582 0.0
0 -0.172638535499573 1.0
0 0.176112532615662 2.0
0 0.261671185493469 3.0
0 -0.0960418581962585 4.0
0 -0.047913670539856 5.0
0 0.0104336738586426 6.0
0 0.00360935926437378 7.0
37.970631847 38.0968322753906 0.0
0 -0.00251191854476929 1.0
0 0.0147252678871155 2.0
0 -0.0113922357559204 3.0
0 -0.000210464000701904 4.0
0 -0.00579959154129028 5.0
0 0.00145632028579712 6.0
0 -0.00575977563858032 7.0
37.012414901 38.5231437683105 0.0
0 -0.0027344822883606 1.0
0 0.0352815389633179 2.0
0 0.00825810432434082 3.0
0 -0.00124883651733398 4.0
0 -0.0013393759727478 5.0
0 0.00134366750717163 6.0
0 -0.00226140022277832 7.0
36.779265883 38.894100189209 0.0
0 0.0063093900680542 1.0
0 0.027963399887085 2.0
0 -0.00222969055175781 3.0
0 0.0165801048278809 4.0
0 0.00685769319534302 5.0
0 0.00238025188446045 6.0
0 -0.00534164905548096 7.0
42.665291585 38.5481719970703 0.0
0 0.00762939453125 1.0
0 0.0562233328819275 2.0
0 -0.0685888528823853 3.0
0 -0.0368189811706543 4.0
0 -0.04285728931427 5.0
0 0.00503599643707275 6.0
0 -0.00157469511032104 7.0
35.413046011 38.4058380126953 0.0
0 0.00584030151367188 1.0
0 0.000958383083343506 2.0
0 0.0113482475280762 3.0
0 0.0107216835021973 4.0
0 0.00791442394256592 5.0
0 0.00128114223480225 6.0
0 0.000605762004852295 7.0
35.966339402 38.4775657653809 0.0
0 0.00496381521224976 1.0
0 -0.00544369220733643 2.0
0 0.00988161563873291 3.0
0 0.0164755582809448 4.0
0 0.0110843181610107 5.0
0 0.00106734037399292 6.0
0 -0.00433731079101562 7.0
39.418493945 38.8547630310059 0.0
0 0.00878036022186279 1.0
0 0.000721454620361328 2.0
0 -0.0158603191375732 3.0
0 0.0104245543479919 4.0
0 0.0125052928924561 5.0
0 0.000981569290161133 6.0
0 0.000788688659667969 7.0
32.151549765 38.9702301025391 0.0
0 -0.0014265775680542 1.0
0 0.012227475643158 2.0
0 0.0263000726699829 3.0
0 -0.00517529249191284 4.0
0 -0.00719040632247925 5.0
0 0.00259649753570557 6.0
0 -0.00119811296463013 7.0
31.845645841 38.3061141967773 0.0
0 0.00534671545028687 1.0
0 -0.0176301002502441 2.0
0 -0.0113786458969116 3.0
0 0.00411343574523926 4.0
0 0.00402390956878662 5.0
0 -0.002593994140625 6.0
0 -0.0061572790145874 7.0
40.557128571 38.4783058166504 0.0
0 -0.0107422471046448 1.0
0 0.0388541221618652 2.0
0 0.0169771909713745 3.0
0 0.00476694107055664 4.0
0 0.00180530548095703 5.0
0 0.00234603881835938 6.0
0 0.00116020441055298 7.0
36.793774817 38.6162338256836 0.0
0 0.00310921669006348 1.0
0 0.0043950080871582 2.0
0 0.0118415951728821 3.0
0 -0.0224356055259705 4.0
0 -0.0218193531036377 5.0
0 0.00348061323165894 6.0
0 -0.00138634443283081 7.0
41.333576969 38.9786033630371 0.0
0 0.0102357268333435 1.0
0 -0.00271809101104736 2.0
0 -0.0131592750549316 3.0
0 0.0101926326751709 4.0
0 0.0110886096954346 5.0
0 0.00132131576538086 6.0
0 0.00104010105133057 7.0
43.791455275 38.4271545410156 0.0
0 -0.00638270378112793 1.0
0 0.0200754404067993 2.0
0 0.0120011568069458 3.0
0 0.0064471960067749 4.0
0 0.00578755140304565 5.0
0 0.00162583589553833 6.0
0 0.00128662586212158 7.0
41.481697076 38.2028846740723 0.0
0 0.00590914487838745 1.0
0 0.00448429584503174 2.0
0 -0.0016942024230957 3.0
0 0.00982093811035156 4.0
0 0.0135623216629028 5.0
0 0.000484585762023926 6.0
0 0.000510692596435547 7.0
35.532774621 38.3699798583984 0.0
0 0.00557297468185425 1.0
0 0.00183498859405518 2.0
0 0.00422048568725586 3.0
0 0.0129894614219666 4.0
0 0.0152958631515503 5.0
0 0.00223052501678467 6.0
0 0.00200212001800537 7.0
34.894006808 38.6870956420898 0.0
0 -0.00032496452331543 1.0
0 -0.000553488731384277 2.0
0 -0.0101300477981567 3.0
0 0.0154532194137573 4.0
0 0.0144065618515015 5.0
0 0.00112330913543701 6.0
0 -0.00057142972946167 7.0
38.893810297 38.5068321228027 0.0
0 0.00826197862625122 1.0
0 0.00306463241577148 2.0
0 0.00759679079055786 3.0
0 0.0136953592300415 4.0
0 0.0143409967422485 5.0
0 0.00131523609161377 6.0
0 0.000699937343597412 7.0
39.624663522 38.6048698425293 0.0
0 -0.0116535425186157 1.0
0 0.00954610109329224 2.0
0 -0.00461006164550781 3.0
0 0.017173171043396 4.0
0 0.0123999118804932 5.0
0 0.00213772058486938 6.0
0 3.28421592712402e-05 7.0
42.516231091 39.1451377868652 0.0
0 0.0351818799972534 1.0
0 0.110571384429932 2.0
0 0.0638208389282227 3.0
0 -0.0394923686981201 4.0
0 -0.0376558303833008 5.0
0 0.00706255435943604 6.0
0 0.006949782371521 7.0
38.374768577 38.2217788696289 0.0
0 0.0492962598800659 1.0
0 0.00276666879653931 2.0
0 -0.0941604375839233 3.0
0 -0.0646132230758667 4.0
0 -0.0723511576652527 5.0
0 0.00604933500289917 6.0
0 0.000296711921691895 7.0
33.351459508 39.1657485961914 0.0
0 -0.00859594345092773 1.0
0 0.0212362408638 2.0
0 -0.00135296583175659 3.0
0 0.00774955749511719 4.0
0 0.00817430019378662 5.0
0 0.00212478637695312 6.0
0 0.00165778398513794 7.0
37.08812014 38.7333183288574 0.0
0 -0.000715315341949463 1.0
0 0.00112462043762207 2.0
0 -0.00267255306243896 3.0
0 0.011474072933197 4.0
0 0.011874794960022 5.0
0 0.00210350751876831 6.0
0 0.00163459777832031 7.0
44.445084649 38.5677528381348 0.0
0 0.00638324022293091 1.0
0 -0.00895494222640991 2.0
0 0.00653094053268433 3.0
0 0.0088040828704834 4.0
0 0.0119718909263611 5.0
0 0.000811874866485596 6.0
0 0.00173741579055786 7.0
37.22322104 38.371509552002 0.0
0 0.00562351942062378 1.0
0 0.00389868021011353 2.0
0 0.0112352967262268 3.0
0 0.0115906000137329 4.0
0 0.0116457343101501 5.0
0 0.000585973262786865 6.0
0 -0.00184345245361328 7.0
44.605944609 38.6993217468262 0.0
0 0.0200371146202087 1.0
0 0.0074923038482666 2.0
0 -0.0283963084220886 3.0
0 -0.0215704441070557 4.0
0 -0.0214748382568359 5.0
0 0.00237280130386353 6.0
0 0.00550472736358643 7.0
39.553698498 38.4618606567383 0.0
0 0.00432324409484863 1.0
0 0.0075610876083374 2.0
0 0.00864207744598389 3.0
0 0.015913724899292 4.0
0 0.013094961643219 5.0
0 0.000805139541625977 6.0
0 -0.00503039360046387 7.0
43.287363281 39.1227188110352 0.0
0 0.0155909657478333 1.0
0 0.0208890438079834 2.0
0 0.0387828350067139 3.0
0 -0.00965893268585205 4.0
0 -0.0132709741592407 5.0
0 0.00286310911178589 6.0
0 -0.0010300874710083 7.0
33.992156471 38.1259117126465 0.0
0 -0.000116407871246338 1.0
0 -0.00216031074523926 2.0
0 0.00474166870117188 3.0
0 0.0110854506492615 4.0
0 0.0120851397514343 5.0
0 -0.000650346279144287 6.0
0 -0.00454753637313843 7.0
36.860093248 39.0079536437988 0.0
0 -0.00831305980682373 1.0
0 0.00342237949371338 2.0
0 -0.00300586223602295 3.0
0 0.00467437505722046 4.0
0 0.00290507078170776 5.0
0 0.00205975770950317 6.0
0 0.00154572725296021 7.0
39.672962738 38.0751495361328 0.0
0 0.00984746217727661 1.0
0 -0.0117852091789246 2.0
0 0.0192103981971741 3.0
0 0.00903779268264771 4.0
0 0.00922292470932007 5.0
0 -1.78813934326172e-06 6.0
0 -0.00248432159423828 7.0
32.8535328 38.515998840332 0.0
0 0.00934451818466187 1.0
0 -0.0145254731178284 2.0
0 0.0208010673522949 3.0
0 0.00874799489974976 4.0
0 0.00777202844619751 5.0
0 0.00117224454879761 6.0
0 -0.00193655490875244 7.0
36.422494688 38.2510757446289 0.0
0 0.00413292646408081 1.0
0 0.0195381045341492 2.0
0 -0.0251415371894836 3.0
0 -0.00397628545761108 4.0
0 -0.00545597076416016 5.0
0 0.00688892602920532 6.0
0 -0.00616079568862915 7.0
32.082396771 38.4953002929688 0.0
0 0.0391383171081543 1.0
0 0.0698838829994202 2.0
0 -0.0538221597671509 3.0
0 -0.0767565369606018 4.0
0 -0.080767035484314 5.0
0 0.00638920068740845 6.0
0 0.00216215848922729 7.0
43.742460686 38.6314849853516 0.0
0 0.00743508338928223 1.0
0 0.00358933210372925 2.0
0 0.00233936309814453 3.0
0 0.0148839354515076 4.0
0 0.0103566646575928 5.0
0 0.00171154737472534 6.0
0 -0.00294375419616699 7.0
37.830554934 38.8711471557617 0.0
0 -0.00633442401885986 1.0
0 -0.085806131362915 2.0
0 0.0016249418258667 3.0
0 -0.00366508960723877 4.0
0 -0.00491905212402344 5.0
0 0.00461995601654053 6.0
0 0.00372880697250366 7.0
45.388745763 38.5289192199707 0.0
0 0.00618654489517212 1.0
0 0.00241726636886597 2.0
0 0.00818026065826416 3.0
0 0.0156952738761902 4.0
0 0.0136352181434631 5.0
0 0.00150501728057861 6.0
0 -0.00034487247467041 7.0
40.010376396 38.5327529907227 0.0
0 0.00943154096603394 1.0
0 -0.0237060785293579 2.0
0 0.0177397727966309 3.0
0 0.00880545377731323 4.0
0 0.00820803642272949 5.0
0 0.000179648399353027 6.0
0 -0.00252014398574829 7.0
45.566990187 38.9576721191406 0.0
0 -0.00651991367340088 1.0
0 0.0196529626846313 2.0
0 -0.0233531594276428 3.0
0 0.00508701801300049 4.0
0 0.00925230979919434 5.0
0 0.000830531120300293 6.0
0 0.00188058614730835 7.0
32.300808889 39.3677978515625 0.0
0 -0.139041066169739 1.0
0 0.0485532879829407 2.0
0 0.379734396934509 3.0
0 -0.113313436508179 4.0
0 0.248469144105911 5.0
0 0.0100367069244385 6.0
0 0.00433242321014404 7.0
36.805463378 38.648265838623 0.0
0 -0.0118266344070435 1.0
0 -0.0432865023612976 2.0
0 0.023951530456543 3.0
0 -0.00356596708297729 4.0
0 -0.00542491674423218 5.0
0 0.00420564413070679 6.0
0 0.00177103281021118 7.0
45.039727992 38.5601387023926 0.0
0 0.00589120388031006 1.0
0 -0.0125548243522644 2.0
0 0.00396126508712769 3.0
0 0.0183145403862 4.0
0 0.0117776989936829 5.0
0 0.00193524360656738 6.0
0 -0.00209051370620728 7.0
36.669966877 38.5309829711914 0.0
0 0.00463026762008667 1.0
0 0.0226883888244629 2.0
0 0.00629353523254395 3.0
0 0.0114103555679321 4.0
0 0.00907456874847412 5.0
0 0.00171947479248047 6.0
0 -0.00196748971939087 7.0
40.401972767 38.2327117919922 0.0
0 0.0044560432434082 1.0
0 0.00335395336151123 2.0
0 0.00200176239013672 3.0
0 0.00582796335220337 4.0
0 0.00227099657058716 5.0
0 0.00142323970794678 6.0
0 -0.00324386358261108 7.0
41.491221092 38.6873626708984 0.0
0 0.00546443462371826 1.0
0 -0.00274443626403809 2.0
0 -0.00156420469284058 3.0
0 0.0120398998260498 4.0
0 0.0104166865348816 5.0
0 0.00150889158248901 6.0
0 0.000754177570343018 7.0
32.092723767 38.4019317626953 0.0
0 0.00888198614120483 1.0
0 0.0200417041778564 2.0
0 -0.00451493263244629 3.0
0 0.00675791501998901 4.0
0 0.000231027603149414 5.0
0 0.00174546241760254 6.0
0 -0.00465226173400879 7.0
39.133472273 38.943042755127 0.0
0 0.0176572203636169 1.0
0 -0.010567843914032 2.0
0 0.0199633836746216 3.0
0 -0.0152668356895447 4.0
0 -0.0183753371238708 5.0
0 0.00282925367355347 6.0
0 -0.000978946685791016 7.0
37.053979998 38.9729843139648 0.0
0 -0.000790834426879883 1.0
0 0.026669442653656 2.0
0 0.0143687725067139 3.0
0 0.0024116039276123 4.0
0 -0.00121736526489258 5.0
0 0.00256633758544922 6.0
0 -0.00202929973602295 7.0
39.864494306 38.6545906066895 0.0
0 0.0108628273010254 1.0
0 0.0133299231529236 2.0
0 0.00485479831695557 3.0
0 0.00242054462432861 4.0
0 0.00336581468582153 5.0
0 0.00100594758987427 6.0
0 -0.00150913000106812 7.0
39.054921308 38.6194686889648 0.0
0 -0.00948309898376465 1.0
0 0.0172194242477417 2.0
0 0.00661468505859375 3.0
0 0.00922781229019165 4.0
0 0.010545015335083 5.0
0 0.00140833854675293 6.0
0 0.00142842531204224 7.0
33.466015237 38.9646224975586 0.0
0 9.03606414794922e-05 1.0
0 0.0123940706253052 2.0
0 0.00274002552032471 3.0
0 0.00122815370559692 4.0
0 -0.000679194927215576 5.0
0 0.0030023455619812 6.0
0 0.00157451629638672 7.0
37.653527764 38.5430641174316 0.0
0 -0.0041007399559021 1.0
0 0.0149326324462891 2.0
0 0.00574177503585815 3.0
0 0.00125390291213989 4.0
0 0.00139206647872925 5.0
0 0.000427007675170898 6.0
0 -0.00363874435424805 7.0
37.577039283 38.1301307678223 0.0
0 -0.00789469480514526 1.0
0 0.0148867964744568 2.0
0 0.00783491134643555 3.0
0 0.0112177729606628 4.0
0 0.00835388898849487 5.0
0 0.0013580322265625 6.0
0 -0.00169438123703003 7.0
36.318307627 38.468090057373 0.0
0 0.00593608617782593 1.0
0 0.00778979063034058 2.0
0 0.00651895999908447 3.0
0 0.00518614053726196 4.0
0 0.00192540884017944 5.0
0 0.00126242637634277 6.0
0 -0.00347381830215454 7.0
36.801464194 38.3615798950195 0.0
0 0.00963151454925537 1.0
0 -0.00350320339202881 2.0
0 0.00691354274749756 3.0
0 0.0129764676094055 4.0
0 0.01592618227005 5.0
0 0.00100570917129517 6.0
0 0.000829577445983887 7.0
39.56095567 38.5987434387207 0.0
0 -0.000210285186767578 1.0
0 -0.0386602878570557 2.0
0 0.0109865069389343 3.0
0 -0.0223806500434875 4.0
0 -0.0233038067817688 5.0
0 0.0035366415977478 6.0
0 -0.00128954648971558 7.0
40.443539209 38.8790626525879 0.0
0 0.0022013783454895 1.0
0 0.0163164138793945 2.0
0 0.0207526087760925 3.0
0 0.0038456916809082 4.0
0 0.00528603792190552 5.0
0 0.00127983093261719 6.0
0 -0.00158476829528809 7.0
40.316920108 38.4642219543457 0.0
0 0.00427037477493286 1.0
0 -0.012304425239563 2.0
0 0.0196879506111145 3.0
0 0.0127553343772888 4.0
0 0.00777769088745117 5.0
0 0.00142759084701538 6.0
0 -0.00219595432281494 7.0
33.579405458 39.1697883605957 0.0
0 0.00949239730834961 1.0
0 0.0550859570503235 2.0
0 0.00363188982009888 3.0
0 -0.00875645875930786 4.0
0 -0.00982797145843506 5.0
0 0.00408774614334106 6.0
0 0.00239044427871704 7.0
34.562810956 38.7289924621582 0.0
0 -0.00927829742431641 1.0
0 -0.031139075756073 2.0
0 0.0254233479499817 3.0
0 -0.00332009792327881 4.0
0 -0.00410580635070801 5.0
0 0.00320959091186523 6.0
0 0.00121003389358521 7.0
37.518824119 38.526123046875 0.0
0 -0.00417220592498779 1.0
0 0.0312619805335999 2.0
0 -0.0036507248878479 3.0
0 0.000343263149261475 4.0
0 -0.00207829475402832 5.0
0 0.00180166959762573 6.0
0 -0.00563240051269531 7.0
39.547705636 38.8937301635742 0.0
0 0.00837802886962891 1.0
0 0.0164691209793091 2.0
0 0.0180018544197083 3.0
0 0.00425541400909424 4.0
0 0.00221371650695801 5.0
0 0.00231784582138062 6.0
0 -0.00168567895889282 7.0
42.834945783 38.4985198974609 0.0
0 0.00751316547393799 1.0
0 -0.00238978862762451 2.0
0 0.0189425945281982 3.0
0 0.00994026660919189 4.0
0 0.0116481781005859 5.0
0 -9.0181827545166e-05 6.0
0 -0.00097280740737915 7.0
38.473622953 38.5314598083496 0.0
0 -0.00472825765609741 1.0
0 0.00927400588989258 2.0
0 0.0143502950668335 3.0
0 0.00991624593734741 4.0
0 0.0122590065002441 5.0
0 0.00189471244812012 6.0
0 0.0014805793762207 7.0
45.238523457 38.3235244750977 0.0
0 0.00389409065246582 1.0
0 0.0224522352218628 2.0
0 0.013031542301178 3.0
0 0.00490808486938477 4.0
0 0.00208324193954468 5.0
0 0.00188267230987549 6.0
0 -0.003562331199646 7.0
36.266282562 38.6187591552734 0.0
0 0.00554311275482178 1.0
0 -0.00330221652984619 2.0
0 -3.94582748413086e-05 3.0
0 0.0115361213684082 4.0
0 0.00921422243118286 5.0
0 0.00178325176239014 6.0
0 0.00104552507400513 7.0
37.845233948 38.7518157958984 0.0
0 -0.000949203968048096 1.0
0 0.00222766399383545 2.0
0 0.0181483030319214 3.0
0 0.00984752178192139 4.0
0 0.00675064325332642 5.0
0 0.00161761045455933 6.0
0 -0.00165462493896484 7.0
40.414778929 38.6832542419434 0.0
0 0.0396010279655457 1.0
0 0.0414213538169861 2.0
0 0.034054696559906 3.0
0 -0.0138557553291321 4.0
0 -0.0228312611579895 5.0
0 0.00423502922058105 6.0
0 -0.00209474563598633 7.0
47.104382604 38.6785850524902 0.0
0 -0.000528335571289062 1.0
0 -0.0117012858390808 2.0
0 0.00560390949249268 3.0
0 0.0113921761512756 4.0
0 0.0118933320045471 5.0
0 0.00046849250793457 6.0
0 -0.00160872936248779 7.0
37.557508388 38.6452941894531 0.0
0 0.0107706189155579 1.0
0 0.0139909386634827 2.0
0 0.0265346765518188 3.0
0 -0.00883990526199341 4.0
0 -0.00819224119186401 5.0
0 0.00279784202575684 6.0
0 0.000886917114257812 7.0
40.532317042 38.8532943725586 0.0
0 -0.00338459014892578 1.0
0 0.00208669900894165 2.0
0 -0.00943958759307861 3.0
0 0.0108900666236877 4.0
0 0.0107890367507935 5.0
0 0.00126546621322632 6.0
0 0.00102007389068604 7.0
32.594348946 38.9392776489258 0.0
0 -0.00994551181793213 1.0
0 -0.00474840402603149 2.0
0 0.0156255960464478 3.0
0 0.00870126485824585 4.0
0 -0.000639498233795166 5.0
0 0.00391221046447754 6.0
0 0.00118041038513184 7.0
30.392342502 38.4610557556152 0.0
0 0.00475746393203735 1.0
0 0.00234395265579224 2.0
0 0.00362575054168701 3.0
0 0.0170789361000061 4.0
0 0.0154737830162048 5.0
0 0.00255101919174194 6.0
0 0.00137734413146973 7.0
40.607017947 38.6572570800781 0.0
0 -0.00629860162734985 1.0
0 0.0028877854347229 2.0
0 0.00213658809661865 3.0
0 0.0124313235282898 4.0
0 0.00775974988937378 5.0
0 0.00156289339065552 6.0
0 -0.00161468982696533 7.0
39.11337914 39.3402252197266 0.0
0 0.053605318069458 1.0
0 0.106695532798767 2.0
0 0.0744338035583496 3.0
0 -0.0374942421913147 4.0
0 -0.0441901683807373 5.0
0 0.00579911470413208 6.0
0 -0.00190454721450806 7.0
35.280627531 38.4612503051758 0.0
0 -0.00280666351318359 1.0
0 0.0293827652931213 2.0
0 -0.0139994621276855 3.0
0 -0.0032992959022522 4.0
0 -0.0042424201965332 5.0
0 -0.00224858522415161 6.0
0 -0.0053715705871582 7.0
32.847447182 38.3127861022949 0.0
0 0.00380522012710571 1.0
0 0.00457185506820679 2.0
0 -0.0068669319152832 3.0
0 0.019364058971405 4.0
0 0.0192620754241943 5.0
0 0.00249946117401123 6.0
0 0.00178861618041992 7.0
34.14972673 38.8230133056641 0.0
0 -0.0161435604095459 1.0
0 0.0640211701393127 2.0
0 0.0416672229766846 3.0
0 -0.0326667428016663 4.0
0 -0.0316633582115173 5.0
0 0.00657165050506592 6.0
0 0.0048445463180542 7.0
36.823986173 38.8261451721191 0.0
0 0.000983893871307373 1.0
0 0.0175092816352844 2.0
0 0.0232627391815186 3.0
0 0.00660783052444458 4.0
0 0.00244992971420288 5.0
0 0.00188249349594116 6.0
0 -0.00359439849853516 7.0
43.615049092 38.7269248962402 0.0
0 -0.00817251205444336 1.0
0 -0.016715943813324 2.0
0 0.021268367767334 3.0
0 -0.0094103217124939 4.0
0 -0.00766164064407349 5.0
0 0.00359576940536499 6.0
0 0.00358837842941284 7.0
39.316465342 38.8074150085449 0.0
0 0.00467908382415771 1.0
0 -0.141974627971649 2.0
0 0.00819063186645508 3.0
0 -0.0243714451789856 4.0
0 -0.0265893340110779 5.0
0 0.00668883323669434 6.0
0 0.00609123706817627 7.0
36.676849778 39.086483001709 0.0
0 0.0338906645774841 1.0
0 0.010081946849823 2.0
0 0.0517814159393311 3.0
0 -0.0148599147796631 4.0
0 -0.015914261341095 5.0
0 0.00404250621795654 6.0
0 -0.000586926937103271 7.0
36.163160136 39.0178642272949 0.0
0 0.0213291645050049 1.0
0 0.000769555568695068 2.0
0 0.0280357003211975 3.0
0 -0.0171972513198853 4.0
0 -0.0217657089233398 5.0
0 0.00370281934738159 6.0
0 -0.00175964832305908 7.0
40.305175496 38.8357582092285 0.0
0 0.00539052486419678 1.0
0 0.00232678651809692 2.0
0 -0.00763154029846191 3.0
0 0.00963932275772095 4.0
0 0.00745934247970581 5.0
0 0.00154834985733032 6.0
0 0.000475943088531494 7.0
34.293236222 38.5812950134277 0.0
0 0.0108577609062195 1.0
0 -0.00930273532867432 2.0
0 0.00788706541061401 3.0
0 0.0121802091598511 4.0
0 0.0107622146606445 5.0
0 0.00162261724472046 6.0
0 0.000948786735534668 7.0
38.347299924 38.5173797607422 0.0
0 0.00819861888885498 1.0
0 0.00382977724075317 2.0
0 0.0101303458213806 3.0
0 0.0126463174819946 4.0
0 0.0123133063316345 5.0
0 0.00140637159347534 6.0
0 0.000719904899597168 7.0
39.036910781 38.8415069580078 0.0
0 0.000177502632141113 1.0
0 0.00544929504394531 2.0
0 -0.0120937824249268 3.0
0 0.00871700048446655 4.0
0 0.00971460342407227 5.0
0 0.0013730525970459 6.0
0 0.00144284963607788 7.0
36.237500285 38.6325035095215 0.0
0 0.00555747747421265 1.0
0 -0.00464928150177002 2.0
0 0.00741976499557495 3.0
0 0.0146653056144714 4.0
0 0.0154220461845398 5.0
0 0.00221884250640869 6.0
0 0.00104159116744995 7.0
36.900540753 38.554141998291 0.0
0 -0.00575870275497437 1.0
0 0.0343371629714966 2.0
0 0.00224876403808594 3.0
0 0.013765811920166 4.0
0 0.00792694091796875 5.0
0 0.0021330714225769 6.0
0 -0.00349611043930054 7.0
29.897899633 38.8244743347168 0.0
0 -0.00203788280487061 1.0
0 0.0106021165847778 2.0
0 0.0118087530136108 3.0
0 0.00280171632766724 4.0
0 -0.000840127468109131 5.0
0 0.00222545862197876 6.0
0 -0.00134199857711792 7.0
41.071373967 38.722110748291 0.0
0 -0.000163435935974121 1.0
0 0.00294435024261475 2.0
0 -0.00318533182144165 3.0
0 0.0174000263214111 4.0
0 0.00768005847930908 5.0
0 0.00268244743347168 6.0
0 -0.000645101070404053 7.0
37.123617582 38.6545639038086 0.0
0 0.0106359124183655 1.0
0 -0.00192886590957642 2.0
0 0.0364857912063599 3.0
0 -0.0049210786819458 4.0
0 -0.00993657112121582 5.0
0 0.00295078754425049 6.0
0 -0.00167167186737061 7.0
38.348981744 39.361213684082 0.0
0 0.108397841453552 1.0
0 0.0719989538192749 2.0
0 0.0854716300964355 3.0
0 -0.0621429681777954 4.0
0 -0.0700206160545349 5.0
0 0.00875371694564819 6.0
0 0.00082772970199585 7.0
37.897829023 38.3081512451172 0.0
0 0.0116560459136963 1.0
0 -0.0115576982498169 2.0
0 0.0105234384536743 3.0
0 0.0114278793334961 4.0
0 0.00531607866287231 5.0
0 0.000552892684936523 6.0
0 -0.0051882266998291 7.0
32.839543182 38.3974990844727 0.0
0 0.00181096792221069 1.0
0 0.00699031352996826 2.0
0 -0.00629299879074097 3.0
0 0.0180360674858093 4.0
0 0.0172174572944641 5.0
0 0.00267308950424194 6.0
0 0.00208872556686401 7.0
45.814256844 38.7030944824219 0.0
0 0.00507837533950806 1.0
0 0.00886917114257812 2.0
0 -0.0037347674369812 3.0
0 0.0136943459510803 4.0
0 0.0137937068939209 5.0
0 0.00230222940444946 6.0
0 0.00160950422286987 7.0
38.439958634 38.4825057983398 0.0
0 -0.00018012523651123 1.0
0 0.0141959190368652 2.0
0 -0.0422380566596985 3.0
0 -0.0285893678665161 4.0
0 -0.0323619246482849 5.0
0 0.00240993499755859 6.0
0 -0.00253540277481079 7.0
33.022424116 38.5725517272949 0.0
0 -0.00100618600845337 1.0
0 0.0236772298812866 2.0
0 0.0142524242401123 3.0
0 0.00742864608764648 4.0
0 0.00858741998672485 5.0
0 0.000316262245178223 6.0
0 -0.00097280740737915 7.0
36.232706353 38.6497421264648 0.0
0 0.00666439533233643 1.0
0 0.0102894306182861 2.0
0 -0.00639992952346802 3.0
0 0.0107094049453735 4.0
0 0.00857967138290405 5.0
0 0.00131088495254517 6.0
0 0.000430107116699219 7.0
45.8365445 38.2908325195312 0.0
0 0.00307029485702515 1.0
0 -0.00923150777816772 2.0
0 0.00529932975769043 3.0
0 0.00588589906692505 4.0
0 0.0053897500038147 5.0
0 0.000756263732910156 6.0
0 -0.00225520133972168 7.0
39.344253832 38.8997688293457 0.0
0 0.00560510158538818 1.0
0 0.012593150138855 2.0
0 0.020938515663147 3.0
0 0.00587159395217896 4.0
0 0.00207465887069702 5.0
0 0.00155848264694214 6.0
0 -0.00285935401916504 7.0
43.613379879 38.2364540100098 0.0
0 0.000397384166717529 1.0
0 0.031568169593811 2.0
0 -0.00499022006988525 3.0
0 -0.00342869758605957 4.0
0 -0.00124233961105347 5.0
0 0.00189417600631714 6.0
0 -0.00510674715042114 7.0
41.603825526 38.189998626709 0.0
0 0.00724172592163086 1.0
0 -0.0164687633514404 2.0
0 0.00195282697677612 3.0
0 0.0148271322250366 4.0
0 0.0068591833114624 5.0
0 0.000564455986022949 6.0
0 -0.00725555419921875 7.0
41.290499737 38.4870338439941 0.0
0 0.0102403163909912 1.0
0 -0.00607091188430786 2.0
0 0.0221967697143555 3.0
0 0.0122079849243164 4.0
0 0.00884580612182617 5.0
0 0.000471353530883789 6.0
0 -0.0037955641746521 7.0
38.21592954 38.5617942810059 0.0
0 -0.000384032726287842 1.0
0 0.00387489795684814 2.0
0 -0.00276827812194824 3.0
0 0.011198878288269 4.0
0 0.013520359992981 5.0
0 0.00102299451828003 6.0
0 0.00107133388519287 7.0
34.764993855 38.4070091247559 0.0
0 -0.00358575582504272 1.0
0 0.0124229192733765 2.0
0 -0.00962519645690918 3.0
0 -0.00281190872192383 4.0
0 -0.00192475318908691 5.0
0 0.000529289245605469 6.0
0 -0.00529855489730835 7.0
34.789226058 38.5943145751953 0.0
0 -0.000479459762573242 1.0
0 0.0239938497543335 2.0
0 0.0198885202407837 3.0
0 0.00419110059738159 4.0
0 0.00250375270843506 5.0
0 0.0020715594291687 6.0
0 -0.00529903173446655 7.0
42.488773183 38.6152381896973 0.0
0 0.00836420059204102 1.0
0 -0.00419712066650391 2.0
0 0.00586044788360596 3.0
0 0.0160481929779053 4.0
0 0.0162471532821655 5.0
0 0.00226736068725586 6.0
0 0.000960052013397217 7.0
34.702233392 39.2018165588379 0.0
0 0.00722146034240723 1.0
0 0.0153907537460327 2.0
0 0.0345809459686279 3.0
0 -0.000850021839141846 4.0
0 -0.00449872016906738 5.0
0 0.00279438495635986 6.0
0 -0.00392633676528931 7.0
41.831412446 39.3694381713867 0.0
0 0.0202068090438843 1.0
0 0.00735104084014893 2.0
0 -0.01542729139328 3.0
0 0.00598728656768799 4.0
0 0.000746250152587891 5.0
0 0.00259238481521606 6.0
0 -0.00198984146118164 7.0
45.908850318 38.7893257141113 0.0
0 0.0128718614578247 1.0
0 0.0439763069152832 2.0
0 -0.053998589515686 3.0
0 -0.0368396043777466 4.0
0 -0.0414500832557678 5.0
0 0.00407934188842773 6.0
0 -0.00154948234558105 7.0
33.104401139 39.2357902526855 0.0
0 0.0355506539344788 1.0
0 0.0204304456710815 2.0
0 0.0460935235023499 3.0
0 -0.00619632005691528 4.0
0 -0.0127991437911987 5.0
0 0.00329005718231201 6.0
0 -0.00491511821746826 7.0
35.835170704 38.656005859375 0.0
0 -0.0135376453399658 1.0
0 0.0272335410118103 2.0
0 0.018695592880249 3.0
0 -0.0218937397003174 4.0
0 -0.0219389796257019 5.0
0 0.00600516796112061 6.0
0 0.00515711307525635 7.0
45.844507362 38.3923797607422 0.0
0 0.00346899032592773 1.0
0 -0.0049707293510437 2.0
0 -0.0142909288406372 3.0
0 0.0028308629989624 4.0
0 0.00224250555038452 5.0
0 -0.00219684839248657 6.0
0 -0.00600039958953857 7.0
31.875262674 38.9218482971191 0.0
0 0.00548732280731201 1.0
0 0.03155916929245 2.0
0 0.0234208106994629 3.0
0 0.00218135118484497 4.0
0 0.000923514366149902 5.0
0 0.000942528247833252 6.0
0 -0.00415825843811035 7.0
35.888104099 38.465877532959 0.0
0 -0.00457191467285156 1.0
0 0.0129864811897278 2.0
0 0.0154435038566589 3.0
0 0.00746810436248779 4.0
0 0.00840973854064941 5.0
0 0.000717759132385254 6.0
0 -0.000674307346343994 7.0
35.733201066 38.5028800964355 0.0
0 0.00586575269699097 1.0
0 0.0072481632232666 2.0
0 0.0173937678337097 3.0
0 0.0100708603858948 4.0
0 0.00668007135391235 5.0
0 0.00180494785308838 6.0
0 -0.00377106666564941 7.0
43.710725429 38.3162002563477 0.0
0 -0.00887012481689453 1.0
0 -0.00385814905166626 2.0
0 -0.00318706035614014 3.0
0 -0.0185797214508057 4.0
0 -0.0196681022644043 5.0
0 0.000224053859710693 6.0
0 -0.00327861309051514 7.0
34.934843039 38.8348121643066 0.0
0 -0.00161290168762207 1.0
0 0.0159087181091309 2.0
0 0.0234295129776001 3.0
0 0.0048181414604187 4.0
0 0.00403618812561035 5.0
0 0.00204634666442871 6.0
0 -0.00230425596237183 7.0
38.70218214 38.7474479675293 0.0
0 -0.00108134746551514 1.0
0 0.0091174840927124 2.0
0 -0.00267726182937622 3.0
0 0.00788772106170654 4.0
0 0.00635135173797607 5.0
0 0.00159680843353271 6.0
0 0.0013083815574646 7.0
39.146786058 39.0480194091797 0.0
0 -0.0114285945892334 1.0
0 0.00788998603820801 2.0
0 -0.0154602527618408 3.0
0 0.0109805464744568 4.0
0 0.00477433204650879 5.0
0 0.00266498327255249 6.0
0 0.000880956649780273 7.0
37.956003494 38.3549499511719 0.0
0 0.00789499282836914 1.0
0 0.00629401206970215 2.0
0 -0.0352148413658142 3.0
0 -0.0112420320510864 4.0
0 -0.0142526030540466 5.0
0 0.00200521945953369 6.0
0 -0.00365465879440308 7.0
35.896232558 39.0284271240234 0.0
0 0.00879734754562378 1.0
0 0.0454597473144531 2.0
0 0.00259673595428467 3.0
0 -0.0103989839553833 4.0
0 -0.0144324898719788 5.0
0 0.00407028198242188 6.0
0 -0.000433206558227539 7.0
37.377916966 39.028507232666 0.0
0 0.0277334451675415 1.0
0 0.103057205677032 2.0
0 0.0302638411521912 3.0
0 -0.0119578838348389 4.0
0 -0.0142959356307983 5.0
0 0.00127959251403809 6.0
0 -0.00192588567733765 7.0
36.698956497 38.7372550964355 0.0
0 -0.0109740495681763 1.0
0 0.0191500186920166 2.0
0 -0.0165144205093384 3.0
0 0.0105429291725159 4.0
0 0.00730335712432861 5.0
0 0.00290632247924805 6.0
0 0.00153911113739014 7.0
41.903925455 38.9670372009277 0.0
0 0.0159850120544434 1.0
0 0.0163652896881104 2.0
0 0.0204038023948669 3.0
0 0.00393366813659668 4.0
0 -0.00349020957946777 5.0
0 0.00331735610961914 6.0
0 -0.00523614883422852 7.0
38.934024851 38.2444496154785 0.0
0 0.00585168600082397 1.0
0 0.0038304328918457 2.0
0 -0.00481849908828735 3.0
0 0.0178231000900269 4.0
0 0.0161932110786438 5.0
0 0.00253915786743164 6.0
0 0.00152045488357544 7.0
40.406325889 38.474781036377 0.0
0 -0.00402039289474487 1.0
0 0.0110981464385986 2.0
0 -0.0641270279884338 3.0
0 -0.035135805606842 4.0
0 -0.0373694896697998 5.0
0 0.00301212072372437 6.0
0 -0.00229912996292114 7.0
39.410747699 38.5370864868164 0.0
0 -0.00339770317077637 1.0
0 0.020703911781311 2.0
0 0.00961828231811523 3.0
0 0.00251716375350952 4.0
0 0.00325173139572144 5.0
0 0.00136208534240723 6.0
0 -0.0024869441986084 7.0
};
\addlegendentry{$R^2$=0.988}
\end{axis}
\end{tikzpicture}


%     \caption{Your caption here}\label{fig:your-label-here}
% \end{figure}
%

\subsection{Case Study I: 8-node Network}


In \cref{fig:results_dummy_base}, a scatter plot illustrates the relationship between the actual values of gas generation at the nodes and the corresponding values predicted by the trained neural network, considering only the losses at the nodes. The plot highlights how effectively the network captures the fact that only one of the nodes in the system has significant gas generation. However, while the network successfully identifies the generating node, the predicted values exhibit less dispersion than the actual values, indicating that the model's predictions are more concentrated around certain points.
%
\begin{figure}
    \centering
    \setlength\figurewidth{1\textwidth}        
    \setlength\figureheight{0.5\textwidth}
    \resizebox{\figurewidth}{\figureheight}{\begin{tikzpicture}

\definecolor{darkgray176}{RGB}{176,176,176}
\definecolor{lightgray204}{RGB}{204,204,204}

\begin{axis}[
    colorbar,
    colorbar sampled,
    colorbar style={
        samples=8,
        ylabel={node\_id},
        ytick={0,1,...,7},
        yticklabels={0,1,2,3,4,5,6,7},
    },
    colormap={mymap}{[1pt]
        rgb(0pt)=(0.12156862745098,0.466666666666667,0.705882352941177)
        rgb(1pt)=(1,0.498039215686275,0.0549019607843137)
        rgb(2pt)=(0.172549019607843,0.627450980392157,0.172549019607843)
        rgb(3pt)=(0.83921568627451,0.152941176470588,0.156862745098039)
        rgb(4pt)=(0.580392156862745,0.403921568627451,0.741176470588235)
        rgb(5pt)=(0.549019607843137,0.337254901960784,0.294117647058824)
        rgb(6pt)=(0.890196078431372,0.466666666666667,0.76078431372549)
        rgb(7pt)=(0.498039215686275,0.498039215686275,0.498039215686275)
    },
legend cell align={left},
legend style={
  fill opacity=0.8,
  draw opacity=1,
  text opacity=1,
  at={(0.03,0.97)},
  anchor=north west,
  draw=lightgray204
},
point meta max=7,
point meta min=0,
tick align=outside,
tick pos=left,
title={},
x grid style={darkgray176},
xlabel={y true},
xmajorgrids,
xmin=-2.3894448262, xmax=50.1783413502,
% xtick style={color=black},
y grid style={darkgray176},
ylabel={y pred},
ymajorgrids,
ymin=-2.18082528710365, ymax=41.9992832481861,
ytick style={color=black}
]
\addplot [
    colormap={mymap}{[1pt]
        rgb(0pt)=(0.12156862745098,0.466666666666667,0.705882352941177)
        rgb(1pt)=(1,0.498039215686275,0.0549019607843137)
        rgb(2pt)=(0.172549019607843,0.627450980392157,0.172549019607843)
        rgb(3pt)=(0.83921568627451,0.152941176470588,0.156862745098039)
        rgb(4pt)=(0.580392156862745,0.403921568627451,0.741176470588235)
        rgb(5pt)=(0.549019607843137,0.337254901960784,0.294117647058824)
        rgb(6pt)=(0.890196078431372,0.466666666666667,0.76078431372549)
        rgb(7pt)=(0.498039215686275,0.498039215686275,0.498039215686275)
    },
    only marks,
    scatter,
    scatter src=explicit
]
table [x=x, y=y, meta=colordata]{%
x  y  colordata
41.603277614 39.7535934448242 0.0
0 0.0908881425857544 1.0
0 0.139073312282562 2.0
0 0.0316752195358276 3.0
0 -0.0294172167778015 4.0
0 -0.0293693542480469 5.0
0 0.00516742467880249 6.0
0 0.00205957889556885 7.0
37.07638211 38.3591461181641 0.0
0 0.0097118616104126 1.0
0 -0.0219908952713013 2.0
0 -0.0110254883766174 3.0
0 0.00467157363891602 4.0
0 0.00213146209716797 5.0
0 -0.00148683786392212 6.0
0 -0.00642764568328857 7.0
39.245039256 38.5584983825684 0.0
0 0.0252088308334351 1.0
0 0.12731671333313 2.0
0 0.0389952659606934 3.0
0 -0.0199683308601379 4.0
0 -0.0247794985771179 5.0
0 0.00470894575119019 6.0
0 0.00119262933731079 7.0
33.40222464 39.6044502258301 0.0
0 0.0798147916793823 1.0
0 0.0904387831687927 2.0
0 0.0590900778770447 3.0
0 -0.0234290957450867 4.0
0 -0.0263727903366089 5.0
0 0.00418353080749512 6.0
0 0.000778496265411377 7.0
40.177942425 38.2198257446289 0.0
0 0.00198698043823242 1.0
0 0.00119161605834961 2.0
0 -0.0142090320587158 3.0
0 0.00341099500656128 4.0
0 1.15036964416504e-05 5.0
0 0.000508427619934082 6.0
0 -0.00637626647949219 7.0
33.661272263 38.4802513122559 0.0
0 -0.00348472595214844 1.0
0 0.0167836546897888 2.0
0 0.0134863257408142 3.0
0 0.0085369348526001 4.0
0 0.00687175989151001 5.0
0 0.00102633237838745 6.0
0 -0.00393420457839966 7.0
41.918555681 38.3983917236328 0.0
0 -0.00454980134963989 1.0
0 0.0024217963218689 2.0
0 -0.000810742378234863 3.0
0 0.011707603931427 4.0
0 0.00938022136688232 5.0
0 0.00149333477020264 6.0
0 0.000492334365844727 7.0
44.067434787 38.4977531433105 0.0
0 0.00612330436706543 1.0
0 0.00236880779266357 2.0
0 0.00920677185058594 3.0
0 0.0120137929916382 4.0
0 0.0121665596961975 5.0
0 0.000962913036346436 6.0
0 0.000135302543640137 7.0
31.762979365 38.5268859863281 0.0
0 0.00982803106307983 1.0
0 -0.00357627868652344 2.0
0 0.00891649723052979 3.0
0 0.0133818984031677 4.0
0 0.0117686986923218 5.0
0 0.001045823097229 6.0
0 0.000113010406494141 7.0
41.030120159 38.5200500488281 0.0
0 -0.0298738479614258 1.0
0 -0.0113872885704041 2.0
0 0.0223647356033325 3.0
0 -0.0168793201446533 4.0
0 -0.0176181793212891 5.0
0 0.00564652681350708 6.0
0 0.00426644086837769 7.0
40.055714877 38.8306159973145 0.0
0 -7.34925270080566e-05 1.0
0 -0.0110159516334534 2.0
0 0.0156899690628052 3.0
0 0.000327527523040771 4.0
0 -0.00162017345428467 5.0
0 0.00265783071517944 6.0
0 -0.000417232513427734 7.0
40.714244155 38.4069595336914 0.0
0 0.00929087400436401 1.0
0 -0.0149266719818115 2.0
0 0.00734519958496094 3.0
0 0.00849682092666626 4.0
0 0.0102183222770691 5.0
0 0.000343441963195801 6.0
0 -0.00474280118942261 7.0
45.139727698 38.8884696960449 0.0
0 0.000342428684234619 1.0
0 0.00625211000442505 2.0
0 0.00714671611785889 3.0
0 0.00820708274841309 4.0
0 0.00730991363525391 5.0
0 0.00169193744659424 6.0
0 0.000112235546112061 7.0
40.583613059 39.263557434082 0.0
0 0.0559103488922119 1.0
0 0.0864608585834503 2.0
0 0.0712517499923706 3.0
0 -0.0344110727310181 4.0
0 -0.0397318601608276 5.0
0 0.00477957725524902 6.0
0 -0.00180310010910034 7.0
35.441291819 38.5217590332031 0.0
0 -0.00431793928146362 1.0
0 -0.0136868357658386 2.0
0 0.00991904735565186 3.0
0 -0.0215700268745422 4.0
0 -0.0221145749092102 5.0
0 0.00257503986358643 6.0
0 -0.00314760208129883 7.0
36.08841077 38.8402824401855 0.0
0 0.00536257028579712 1.0
0 0.0375674962997437 2.0
0 0.0245164632797241 3.0
0 -0.00596868991851807 4.0
0 -0.00369197130203247 5.0
0 0.00242996215820312 6.0
0 0.00174403190612793 7.0
40.494343306 38.6495323181152 0.0
0 -0.00675559043884277 1.0
0 -0.005912184715271 2.0
0 -0.00215411186218262 3.0
0 0.00915825366973877 4.0
0 0.00896823406219482 5.0
0 0.00157058238983154 6.0
0 0.00130730867385864 7.0
44.400196583 38.5482749938965 0.0
0 0.00762939453125 1.0
0 0.00384992361068726 2.0
0 0.0038907527923584 3.0
0 0.0148396492004395 4.0
0 0.0139151811599731 5.0
0 0.00187838077545166 6.0
0 0.000854194164276123 7.0
40.197028842 38.6482429504395 0.0
0 -0.00523370504379272 1.0
0 0.0368385314941406 2.0
0 0.0156623125076294 3.0
0 0.00713348388671875 4.0
0 0.00676637887954712 5.0
0 0.000627398490905762 6.0
0 -0.000256776809692383 7.0
39.931867517 38.7960472106934 0.0
0 -0.00283390283584595 1.0
0 -0.00708603858947754 2.0
0 -0.00414395332336426 3.0
0 0.00981372594833374 4.0
0 0.00716376304626465 5.0
0 0.0015750527381897 6.0
0 0.000669717788696289 7.0
37.28322745 38.2965774536133 0.0
0 0.00495332479476929 1.0
0 -0.00355356931686401 2.0
0 0.0140035152435303 3.0
0 0.011663019657135 4.0
0 0.00884330272674561 5.0
0 0.00177943706512451 6.0
0 0.00082862377166748 7.0
37.429390014 38.4880485534668 0.0
0 0.0135406255722046 1.0
0 0.00302684307098389 2.0
0 0.0197659134864807 3.0
0 -0.00780308246612549 4.0
0 -0.00855451822280884 5.0
0 0.00318622589111328 6.0
0 0.000750601291656494 7.0
39.499146179 38.6164703369141 0.0
0 0.00107491016387939 1.0
0 -0.00115358829498291 2.0
0 0.0128684639930725 3.0
0 0.0100945830345154 4.0
0 0.00789058208465576 5.0
0 0.000394701957702637 6.0
0 -0.00226247310638428 7.0
35.661352504 38.4398956298828 0.0
0 0.0203315615653992 1.0
0 0.0788954496383667 2.0
0 -0.0757306814193726 3.0
0 -0.0765065550804138 4.0
0 -0.0808522701263428 5.0
0 0.0057339072227478 6.0
0 0.00214129686355591 7.0
44.903668391 38.679759979248 0.0
0 0.0104649066925049 1.0
0 -0.00321555137634277 2.0
0 0.00385379791259766 3.0
0 0.0129628777503967 4.0
0 0.0148612260818481 5.0
0 0.00193989276885986 6.0
0 0.00181066989898682 7.0
34.118473664 39.0513191223145 0.0
0 0.0307442545890808 1.0
0 0.0471727848052979 2.0
0 0.0385069847106934 3.0
0 -0.0240370631217957 4.0
0 -0.0236438512802124 5.0
0 0.00260704755783081 6.0
0 0.011364221572876 7.0
36.466593361 38.987190246582 0.0
0 -0.0103698968887329 1.0
0 0.0800180435180664 2.0
0 0.0414098501205444 3.0
0 -0.0267103314399719 4.0
0 -0.0288975834846497 5.0
0 0.00566142797470093 6.0
0 0.0028914213180542 7.0
34.994247693 38.3411102294922 0.0
0 0.00688672065734863 1.0
0 -0.00696879625320435 2.0
0 0.0116889476776123 3.0
0 0.00811731815338135 4.0
0 0.0102164149284363 5.0
0 0.00122654438018799 6.0
0 0.00214976072311401 7.0
39.553868661 38.6714859008789 0.0
0 -0.00435274839401245 1.0
0 -0.000196635723114014 2.0
0 -0.00985980033874512 3.0
0 0.0103663206100464 4.0
0 0.0117907524108887 5.0
0 0.0014764666557312 6.0
0 0.00132846832275391 7.0
43.158887007 38.4424667358398 0.0
0 0.00920790433883667 1.0
0 -0.00932449102401733 2.0
0 0.0151568651199341 3.0
0 0.0135409832000732 4.0
0 0.00888550281524658 5.0
0 0.00103974342346191 6.0
0 -0.00307869911193848 7.0
36.693891456 39.0564842224121 0.0
0 0.00571656227111816 1.0
0 0.0139431953430176 2.0
0 0.00666135549545288 3.0
0 0.00349867343902588 4.0
0 0.000229835510253906 5.0
0 0.00237119197845459 6.0
0 -0.000682830810546875 7.0
37.419072323 38.5162734985352 0.0
0 0.0100835561752319 1.0
0 -0.00280958414077759 2.0
0 0.011082649230957 3.0
0 0.0125146508216858 4.0
0 0.0106852650642395 5.0
0 0.000358998775482178 6.0
0 -0.00108003616333008 7.0
42.391154887 37.6930847167969 0.0
0 0.00836706161499023 1.0
0 -0.0265249013900757 2.0
0 0.0109959840774536 3.0
0 -0.00843185186386108 4.0
0 -0.00866538286209106 5.0
0 0.00374925136566162 6.0
0 0.00252997875213623 7.0
39.976124969 39.3020973205566 0.0
0 0.0369466543197632 1.0
0 0.0207736492156982 2.0
0 0.00882303714752197 3.0
0 -0.0113847255706787 4.0
0 -0.0140944719314575 5.0
0 0.00480204820632935 6.0
0 0.00208133459091187 7.0
41.193094756 38.7121925354004 0.0
0 0.000246524810791016 1.0
0 0.0112533569335938 2.0
0 -0.00880825519561768 3.0
0 0.0115809440612793 4.0
0 0.00994521379470825 5.0
0 0.00153958797454834 6.0
0 0.000614643096923828 7.0
39.030467364 38.9754753112793 0.0
0 0.0133965015411377 1.0
0 0.0099719762802124 2.0
0 -0.0317625403404236 3.0
0 -0.0213124752044678 4.0
0 -0.0274962782859802 5.0
0 0.0034407377243042 6.0
0 -0.00381076335906982 7.0
39.254434013 38.8635673522949 0.0
0 -0.00899428129196167 1.0
0 0.028767466545105 2.0
0 -0.00187152624130249 3.0
0 0.00816291570663452 4.0
0 0.00780004262924194 5.0
0 0.00216001272201538 6.0
0 0.00177603960037231 7.0
41.474829476 38.490550994873 0.0
0 -0.00636947154998779 1.0
0 0.0157142877578735 2.0
0 -0.00421321392059326 3.0
0 0.0130150318145752 4.0
0 0.0113214254379272 5.0
0 0.00152826309204102 6.0
0 0.000341415405273438 7.0
36.065024548 38.5721321105957 0.0
0 0.00696414709091187 1.0
0 -0.00372463464736938 2.0
0 0.00787001848220825 3.0
0 0.00936907529830933 4.0
0 0.0126588344573975 5.0
0 0.00105565786361694 6.0
0 0.00142580270767212 7.0
33.547657937 39.1791458129883 0.0
0 0.0251626968383789 1.0
0 0.106061309576035 2.0
0 0.0769014954566956 3.0
0 -0.025641918182373 4.0
0 -0.0292165875434875 5.0
0 0.00278604030609131 6.0
0 -0.0029417872428894 7.0
35.860060933 38.5283851623535 0.0
0 -0.00231927633285522 1.0
0 -0.00571024417877197 2.0
0 -0.00389313697814941 3.0
0 0.00245767831802368 4.0
0 0.00144177675247192 5.0
0 0.00015634298324585 6.0
0 -0.00504642724990845 7.0
41.446935888 38.7161064147949 0.0
0 0.0399006009101868 1.0
0 0.118380934000015 2.0
0 -0.0819071531295776 3.0
0 -0.0782929062843323 4.0
0 -0.0900070071220398 5.0
0 0.00521427392959595 6.0
0 0.00152796506881714 7.0
39.09863661 38.6100082397461 0.0
0 -0.0173571109771729 1.0
0 0.0087469220161438 2.0
0 -0.0296517014503479 3.0
0 0.0122166275978088 4.0
0 0.00985217094421387 5.0
0 0.00263488292694092 6.0
0 0.00131088495254517 7.0
37.209660367 38.619556427002 0.0
0 -0.00431281328201294 1.0
0 0.0119017362594604 2.0
0 0.0246157646179199 3.0
0 0.00228214263916016 4.0
0 -0.00246334075927734 5.0
0 0.00261741876602173 6.0
0 -0.00559180974960327 7.0
37.318344805 38.4069023132324 0.0
0 0.00742429494857788 1.0
0 -0.00190436840057373 2.0
0 0.0183868408203125 3.0
0 0.00921815633773804 4.0
0 0.00853586196899414 5.0
0 0.000802695751190186 6.0
0 -0.00107866525650024 7.0
37.542238634 38.8763580322266 0.0
0 -0.00635802745819092 1.0
0 0.0211170315742493 2.0
0 0.00130510330200195 3.0
0 0.0125030279159546 4.0
0 0.00436419248580933 5.0
0 0.00310200452804565 6.0
0 0.00108665227890015 7.0
35.642638424 38.4852333068848 0.0
0 0.00880211591720581 1.0
0 -0.0149785280227661 2.0
0 0.00123864412307739 3.0
0 0.0102723836898804 4.0
0 0.00305092334747314 5.0
0 0.000930190086364746 6.0
0 -0.00708687305450439 7.0
47.4260719 38.6907234191895 0.0
0 -0.00334429740905762 1.0
0 -0.0110288262367249 2.0
0 -0.0155019164085388 3.0
0 -0.0171687602996826 4.0
0 -0.0232598185539246 5.0
0 0.00268292427062988 6.0
0 -0.00473958253860474 7.0
42.664227047 38.3595390319824 0.0
0 0.00736522674560547 1.0
0 0.169191211462021 2.0
0 -0.0193517208099365 3.0
0 -0.00202935934066772 4.0
0 -0.00715571641921997 5.0
0 0.000342905521392822 6.0
0 -0.00674760341644287 7.0
37.881981315 38.4804840087891 0.0
0 0.00734597444534302 1.0
0 0.0022200345993042 2.0
0 0.00960803031921387 3.0
0 0.0106253623962402 4.0
0 0.0120579600334167 5.0
0 0.00171905755996704 6.0
0 0.00124293565750122 7.0
44.031482397 38.7509765625 0.0
0 -0.0097581148147583 1.0
0 -0.0636894702911377 2.0
0 0.0015978217124939 3.0
0 -0.00409984588623047 4.0
0 -0.00204712152481079 5.0
0 0.00405192375183105 6.0
0 0.0043146014213562 7.0
40.215687867 38.6358032226562 0.0
0 0.00362402200698853 1.0
0 0.00420552492141724 2.0
0 0.000331521034240723 3.0
0 0.00297117233276367 4.0
0 0.000524163246154785 5.0
0 0.000945329666137695 6.0
0 -0.00605976581573486 7.0
35.401137883 38.9912109375 0.0
0 -0.00200784206390381 1.0
0 -0.000412046909332275 2.0
0 -0.0133203268051147 3.0
0 0.0117408633232117 4.0
0 0.00856119394302368 5.0
0 0.0021214485168457 6.0
0 -0.000339329242706299 7.0
40.23529515 38.3903541564941 0.0
0 0.0114810466766357 1.0
0 -0.00885409116744995 2.0
0 0.0132594108581543 3.0
0 0.0202721357345581 4.0
0 0.0152925848960876 5.0
0 0.000202775001525879 6.0
0 -0.00713396072387695 7.0
38.451822651 38.8377914428711 0.0
0 0.00414824485778809 1.0
0 -0.00228339433670044 2.0
0 0.00336885452270508 3.0
0 0.0121934413909912 4.0
0 0.00823593139648438 5.0
0 0.00146770477294922 6.0
0 -0.00342243909835815 7.0
41.356859392 38.3796310424805 0.0
0 0.00926601886749268 1.0
0 -0.0124003291130066 2.0
0 0.0076671838760376 3.0
0 0.0100225806236267 4.0
0 0.00797367095947266 5.0
0 0.00054556131362915 6.0
0 -0.00379371643066406 7.0
44.45602451 38.879566192627 0.0
0 0.00672298669815063 1.0
0 0.0123946666717529 2.0
0 -0.00913697481155396 3.0
0 0.0109050869941711 4.0
0 0.00889235734939575 5.0
0 0.00168204307556152 6.0
0 0.000749766826629639 7.0
36.344631329 38.5415077209473 0.0
0 0.00494116544723511 1.0
0 0.00470906496047974 2.0
0 0.00875461101531982 3.0
0 0.00978201627731323 4.0
0 0.00804340839385986 5.0
0 0.00102525949478149 6.0
0 -0.00314891338348389 7.0
35.989587139 39.0207290649414 0.0
0 -0.00257343053817749 1.0
0 -0.00773054361343384 2.0
0 -0.00264960527420044 3.0
0 0.00988805294036865 4.0
0 0.0075453519821167 5.0
0 0.00126409530639648 6.0
0 -0.000481665134429932 7.0
38.453275767 38.5509948730469 0.0
0 -0.00460779666900635 1.0
0 0.0267020463943481 2.0
0 -0.00305271148681641 3.0
0 0.00985264778137207 4.0
0 0.00969010591506958 5.0
0 0.00131440162658691 6.0
0 0.000271499156951904 7.0
36.388812201 38.5625762939453 0.0
0 0.00684624910354614 1.0
0 0.00579524040222168 2.0
0 0.00857532024383545 3.0
0 0.00900334119796753 4.0
0 0.0114079117774963 5.0
0 0.00167042016983032 6.0
0 0.00194859504699707 7.0
45.540732617 38.3559417724609 0.0
0 0.00538235902786255 1.0
0 0.00386589765548706 2.0
0 -0.00125622749328613 3.0
0 0.018405020236969 4.0
0 0.0137466788291931 5.0
0 0.00278580188751221 6.0
0 0.000956237316131592 7.0
34.414627117 38.921630859375 0.0
0 0.0324609279632568 1.0
0 0.0945376753807068 2.0
0 0.0671876668930054 3.0
0 -0.0498137474060059 4.0
0 -0.0511236786842346 5.0
0 0.0085570216178894 6.0
0 0.00632810592651367 7.0
39.317888978 38.3244590759277 0.0
0 -0.00816702842712402 1.0
0 0.0362486243247986 2.0
0 0.0158848762512207 3.0
0 0.00621277093887329 4.0
0 0.00248837471008301 5.0
0 0.00199019908905029 6.0
0 -0.00439673662185669 7.0
36.969505628 39.0306816101074 0.0
0 -0.00373166799545288 1.0
0 0.0273987650871277 2.0
0 -0.0015103816986084 3.0
0 0.00917762517929077 4.0
0 0.00721776485443115 5.0
0 0.00152438879013062 6.0
0 -0.000276803970336914 7.0
44.923617033 39.2973556518555 0.0
0 0.00290322303771973 1.0
0 0.0194564461708069 2.0
0 0.0318148732185364 3.0
0 -0.0022081732749939 4.0
0 -0.00278234481811523 5.0
0 0.00196754932403564 6.0
0 -0.000810801982879639 7.0
37.258159228 38.1323928833008 0.0
0 0.0120348930358887 1.0
0 -0.0394316911697388 2.0
0 -0.050422191619873 3.0
0 -0.0273675322532654 4.0
0 -0.0310018062591553 5.0
0 0.00130391120910645 6.0
0 -0.0026056170463562 7.0
47.488576672 38.4616012573242 0.0
0 0.013430118560791 1.0
0 0.0567920207977295 2.0
0 0.0599616765975952 3.0
0 -0.0214373469352722 4.0
0 -0.0249693393707275 5.0
0 0.00472307205200195 6.0
0 -0.000217974185943604 7.0
42.958399331 38.6427192687988 0.0
0 0.00441265106201172 1.0
0 0.00464749336242676 2.0
0 0.000767230987548828 3.0
0 0.0110027194023132 4.0
0 0.0115892887115479 5.0
0 0.00038456916809082 6.0
0 -0.000238478183746338 7.0
41.161203287 38.1443328857422 0.0
0 0.00848382711410522 1.0
0 -0.0180635452270508 2.0
0 -0.00918519496917725 3.0
0 0.00684559345245361 4.0
0 0.0074118971824646 5.0
0 0.000457406044006348 6.0
0 -0.00672441720962524 7.0
40.466838645 38.8036346435547 0.0
0 -0.00796282291412354 1.0
0 -0.0537744164466858 2.0
0 0.0214377641677856 3.0
0 -0.0128311514854431 4.0
0 -0.0138890147209167 5.0
0 0.00499492883682251 6.0
0 0.00396305322647095 7.0
33.757389454 38.6834144592285 0.0
0 0.028232216835022 1.0
0 0.0404295325279236 2.0
0 0.0279163718223572 3.0
0 -0.0216402411460876 4.0
0 -0.0251529216766357 5.0
0 0.00418293476104736 6.0
0 -0.0031440258026123 7.0
43.437770592 38.343563079834 0.0
0 0.00211387872695923 1.0
0 -0.000490009784698486 2.0
0 -0.00304973125457764 3.0
0 0.00690853595733643 4.0
0 0.00559860467910767 5.0
0 0.000310838222503662 6.0
0 -0.00541353225708008 7.0
39.308931201 38.8775672912598 0.0
0 0.00970524549484253 1.0
0 0.00253796577453613 2.0
0 -0.0164980292320251 3.0
0 0.0100165009498596 4.0
0 0.00814670324325562 5.0
0 0.00208121538162231 6.0
0 0.00108212232589722 7.0
40.600604276 38.547607421875 0.0
0 0.00340026617050171 1.0
0 0.00650149583816528 2.0
0 -0.00628137588500977 3.0
0 0.0181875228881836 4.0
0 0.0199186205863953 5.0
0 0.00151854753494263 6.0
0 0.000111699104309082 7.0
31.520377836 38.581787109375 0.0
0 -0.00357002019882202 1.0
0 0.00153231620788574 2.0
0 0.002410888671875 3.0
0 0.00847429037094116 4.0
0 0.0135117173194885 5.0
0 -0.000303924083709717 6.0
0 0.000370502471923828 7.0
39.969125591 38.6257934570312 0.0
0 -0.00313878059387207 1.0
0 0.0176121592521667 2.0
0 -0.000237464904785156 3.0
0 -0.0116665363311768 4.0
0 -0.0119089484214783 5.0
0 -4.19020652770996e-05 6.0
0 -0.00395911931991577 7.0
30.912106182 38.2456321716309 0.0
0 -0.00355678796768188 1.0
0 0.0208463668823242 2.0
0 0.00883865356445312 3.0
0 0.0121186971664429 4.0
0 0.0138341188430786 5.0
0 0.00130367279052734 6.0
0 0.000901222229003906 7.0
45.180443811 38.3854637145996 0.0
0 -0.00685745477676392 1.0
0 0.0200044512748718 2.0
0 0.0115278959274292 3.0
0 -0.00447863340377808 4.0
0 -0.00777232646942139 5.0
0 0.00198572874069214 6.0
0 -0.00468242168426514 7.0
34.99216559 38.5155220031738 0.0
0 0.00417304039001465 1.0
0 0.0240988731384277 2.0
0 0.0120140314102173 3.0
0 -0.00103104114532471 4.0
0 -0.00382125377655029 5.0
0 0.00195038318634033 6.0
0 -0.0034911036491394 7.0
40.933938659 38.6754302978516 0.0
0 0.00394058227539062 1.0
0 0.0120728611946106 2.0
0 0.00304794311523438 3.0
0 -0.00491464138031006 4.0
0 -0.00522351264953613 5.0
0 0.00198507308959961 6.0
0 -0.00257027149200439 7.0
40.317005952 38.9766120910645 0.0
0 0.00107008218765259 1.0
0 0.00469475984573364 2.0
0 -0.010001540184021 3.0
0 0.00944429636001587 4.0
0 0.00491970777511597 5.0
0 0.00197219848632812 6.0
0 0.000297665596008301 7.0
40.010379617 38.5953788757324 0.0
0 -0.00856190919876099 1.0
0 0.0215948820114136 2.0
0 -0.000319242477416992 3.0
0 0.0154175162315369 4.0
0 0.00876986980438232 5.0
0 0.00185507535934448 6.0
0 -0.00385445356369019 7.0
38.685596488 38.8159713745117 0.0
0 -0.00294101238250732 1.0
0 0.00725901126861572 2.0
0 -0.0348896980285645 3.0
0 0.0134746432304382 4.0
0 0.00706773996353149 5.0
0 0.00252926349639893 6.0
0 -0.00183933973312378 7.0
30.278486108 39.0315551757812 0.0
0 -0.00075840950012207 1.0
0 0.0177011489868164 2.0
0 0.0114507675170898 3.0
0 0.0057494044303894 4.0
0 0.00559014081954956 5.0
0 0.00110125541687012 6.0
0 7.3552131652832e-05 7.0
40.311093813 37.9903717041016 0.0
0 0.000359892845153809 1.0
0 0.00946557521820068 2.0
0 0.00448530912399292 3.0
0 0.010830819606781 4.0
0 0.00182676315307617 5.0
0 0.00240612030029297 6.0
0 -0.00593709945678711 7.0
35.760290532 38.7118949890137 0.0
0 0.00812315940856934 1.0
0 0.0297279953956604 2.0
0 -0.0139551758766174 3.0
0 -0.014398992061615 4.0
0 -0.020226776599884 5.0
0 0.00332540273666382 6.0
0 -0.00403028726577759 7.0
40.236813436 38.5688896179199 0.0
0 -0.00127357244491577 1.0
0 -0.00444197654724121 2.0
0 0.000873267650604248 3.0
0 0.00886917114257812 4.0
0 0.0087096095085144 5.0
0 0.00108206272125244 6.0
0 0.000600993633270264 7.0
43.948095531 38.5622749328613 0.0
0 0.00688648223876953 1.0
0 0.0145112872123718 2.0
0 -0.00303208827972412 3.0
0 0.0120838284492493 4.0
0 0.0151082277297974 5.0
0 0.000646591186523438 6.0
0 9.0181827545166e-05 7.0
40.202612729 38.9503021240234 0.0
0 0.00446075201034546 1.0
0 0.00278854370117188 2.0
0 -0.00881409645080566 3.0
0 0.00978708267211914 4.0
0 0.0118529200553894 5.0
0 0.00184690952301025 6.0
0 0.00139963626861572 7.0
41.441227091 38.7626190185547 0.0
0 0.00998771190643311 1.0
0 0.014593780040741 2.0
0 0.037828803062439 3.0
0 -0.0178765058517456 4.0
0 -0.0181952714920044 5.0
0 0.00372391939163208 6.0
0 -0.000807821750640869 7.0
40.234798657 38.8709907531738 0.0
0 0.0301759243011475 1.0
0 0.0436609983444214 2.0
0 -0.0280125141143799 3.0
0 -0.0313872694969177 4.0
0 -0.0318161249160767 5.0
0 0.00301194190979004 6.0
0 -0.00238752365112305 7.0
40.268402112 38.6101875305176 0.0
0 0.0056917667388916 1.0
0 0.0238692760467529 2.0
0 0.016819953918457 3.0
0 0.00480717420578003 4.0
0 0.00369954109191895 5.0
0 0.00116813182830811 6.0
0 -0.00442075729370117 7.0
34.447402114 38.8466873168945 0.0
0 0.0301073789596558 1.0
0 0.00375080108642578 2.0
0 0.00299829244613647 3.0
0 -0.0253135561943054 4.0
0 -0.0322710871696472 5.0
0 0.0043519139289856 6.0
0 -0.00207918882369995 7.0
40.416533901 38.6745910644531 0.0
0 0.00224298238754272 1.0
0 -0.000726759433746338 2.0
0 0.0128074884414673 3.0
0 0.0108352303504944 4.0
0 0.0103694200515747 5.0
0 0.000999271869659424 6.0
0 -0.00330287218093872 7.0
40.837884543 38.8132362365723 0.0
0 0.00295329093933105 1.0
0 -0.00834047794342041 2.0
0 0.0268865823745728 3.0
0 0.000196278095245361 4.0
0 -0.005332350730896 5.0
0 0.00294673442840576 6.0
0 -0.00294685363769531 7.0
39.823092444 37.8871421813965 0.0
0 0.000456392765045166 1.0
0 -0.00625967979431152 2.0
0 -0.0234071016311646 3.0
0 -0.00285905599594116 4.0
0 -0.00193220376968384 5.0
0 -0.00192928314208984 6.0
0 -0.00620365142822266 7.0
40.41620498 38.7058410644531 0.0
0 -0.00521928071975708 1.0
0 0.000878453254699707 2.0
0 0.0232111811637878 3.0
0 0.0046045184135437 4.0
0 -0.00023186206817627 5.0
0 0.00172221660614014 6.0
0 -0.004547119140625 7.0
33.388504976 38.8013496398926 0.0
0 -0.00490099191665649 1.0
0 0.00648212432861328 2.0
0 -0.0126960277557373 3.0
0 0.015871524810791 4.0
0 0.0092160701751709 5.0
0 0.00187402963638306 6.0
0 -0.00270891189575195 7.0
42.251691394 38.6184730529785 0.0
0 0.0312210917472839 1.0
0 0.00582593679428101 2.0
0 0.0208268165588379 3.0
0 -0.00345629453659058 4.0
0 -0.00656622648239136 5.0
0 0.00353902578353882 6.0
0 -0.00196647644042969 7.0
32.552772789 38.5892791748047 0.0
0 -0.00613367557525635 1.0
0 0.0275079011917114 2.0
0 -0.0156198143959045 3.0
0 -0.0119390487670898 4.0
0 -0.00980442762374878 5.0
0 0.000374138355255127 6.0
0 -0.00487136840820312 7.0
43.396109959 38.253360748291 0.0
0 0.0250413417816162 1.0
0 0.020060658454895 2.0
0 -0.0217271447181702 3.0
0 0.00134474039077759 4.0
0 -0.00472259521484375 5.0
0 0.00346243381500244 6.0
0 -0.00104355812072754 7.0
31.629368094 38.9830894470215 0.0
0 0.00522613525390625 1.0
0 0.000475287437438965 2.0
0 -0.011349081993103 3.0
0 0.01133793592453 4.0
0 0.0121121406555176 5.0
0 0.00200116634368896 6.0
0 0.00154191255569458 7.0
37.779811495 38.5338363647461 0.0
0 -0.0108740329742432 1.0
0 -0.0369836091995239 2.0
0 0.0149946808815002 3.0
0 -0.00636374950408936 4.0
0 -0.00593376159667969 5.0
0 0.00299978256225586 6.0
0 0.00320684909820557 7.0
34.24652714 38.4251098632812 0.0
0 0.00578612089157104 1.0
0 -0.00152945518493652 2.0
0 0.0102679133415222 3.0
0 0.0122678875923157 4.0
0 0.0114915370941162 5.0
0 0.00135761499404907 6.0
0 0.000555694103240967 7.0
40.003780762 39.0511894226074 0.0
0 -0.0189368724822998 1.0
0 0.0200136303901672 2.0
0 -0.0185563564300537 3.0
0 0.0121176242828369 4.0
0 0.0117418766021729 5.0
0 0.00249522924423218 6.0
0 0.00155913829803467 7.0
31.970353396 38.9871673583984 0.0
0 -0.00729846954345703 1.0
0 0.0101170539855957 2.0
0 -0.0196437239646912 3.0
0 0.0132386088371277 4.0
0 0.0108768343925476 5.0
0 0.00228655338287354 6.0
0 0.0010066032409668 7.0
37.935925517 38.9553871154785 0.0
0 -0.00871318578720093 1.0
0 0.0280109643936157 2.0
0 -0.025648832321167 3.0
0 0.0104709267616272 4.0
0 0.0101121068000793 5.0
0 0.00271910429000854 6.0
0 0.00231039524078369 7.0
37.4568261 38.9331321716309 0.0
0 -0.00507557392120361 1.0
0 -0.000133514404296875 2.0
0 0.00147175788879395 3.0
0 0.00708097219467163 4.0
0 0.00657325983047485 5.0
0 0.00235539674758911 6.0
0 0.0013958215713501 7.0
34.938845635 38.2435569763184 0.0
0 0.00165331363677979 1.0
0 -0.00227940082550049 2.0
0 0.00175893306732178 3.0
0 0.0102068185806274 4.0
0 0.00741416215896606 5.0
0 0.00100028514862061 6.0
0 -0.00547868013381958 7.0
33.644682189 39.0608100891113 0.0
0 0.0215523242950439 1.0
0 0.0741968750953674 2.0
0 0.0108304619789124 3.0
0 -0.00966185331344604 4.0
0 -0.0103352665901184 5.0
0 0.00338512659072876 6.0
0 0.000349283218383789 7.0
43.152629724 38.6673278808594 0.0
0 -0.015493631362915 1.0
0 0.0307546257972717 2.0
0 0.00604945421218872 3.0
0 0.00886774063110352 4.0
0 0.010189414024353 5.0
0 0.00204598903656006 6.0
0 0.00179809331893921 7.0
29.29034141 38.7699356079102 0.0
0 0.0308585166931152 1.0
0 0.0640302300453186 2.0
0 0.0491317510604858 3.0
0 -0.0297149419784546 4.0
0 -0.0343635082244873 5.0
0 0.00418633222579956 6.0
0 -0.00212037563323975 7.0
36.529945045 38.9532089233398 0.0
0 0.0035216212272644 1.0
0 -0.0379787087440491 2.0
0 0.0298681259155273 3.0
0 -0.011357307434082 4.0
0 -0.0170662999153137 5.0
0 0.00557559728622437 6.0
0 0.00102359056472778 7.0
39.490028245 38.2533340454102 0.0
0 0.00501155853271484 1.0
0 0.0158194303512573 2.0
0 -0.0231708884239197 3.0
0 -0.00452101230621338 4.0
0 -0.0123972296714783 5.0
0 0.00276535749435425 6.0
0 -0.00519418716430664 7.0
34.709507968 38.3433647155762 0.0
0 0.00644981861114502 1.0
0 0.0225534439086914 2.0
0 0.0125881433486938 3.0
0 0.00445210933685303 4.0
0 0.00744950771331787 5.0
0 0.00118499994277954 6.0
0 -0.00236207246780396 7.0
45.017599836 38.2805442810059 0.0
0 0.00768566131591797 1.0
0 0.00354486703872681 2.0
0 -0.0179717540740967 3.0
0 -0.0011255145072937 4.0
0 -0.00349807739257812 5.0
0 0.000170648097991943 6.0
0 -0.00597703456878662 7.0
32.940669776 38.5253715515137 0.0
0 -0.00512635707855225 1.0
0 0.0313267707824707 2.0
0 -0.00119996070861816 3.0
0 -0.00267255306243896 4.0
0 -0.00358939170837402 5.0
0 -0.00215828418731689 6.0
0 -0.00541460514068604 7.0
44.072190389 38.780647277832 0.0
0 0.0452917814254761 1.0
0 -0.0119504332542419 2.0
0 0.0499585866928101 3.0
0 -0.0182744860649109 4.0
0 -0.0202071666717529 5.0
0 0.0040593147277832 6.0
0 -0.00122308731079102 7.0
35.422022456 38.5963363647461 0.0
0 1.99079513549805e-05 1.0
0 -0.0265647768974304 2.0
0 0.0021822452545166 3.0
0 -0.0189751386642456 4.0
0 -0.01978600025177 5.0
0 0.000371932983398438 6.0
0 -0.00270688533782959 7.0
36.153594117 39.0233688354492 0.0
0 -0.00477045774459839 1.0
0 0.0245848894119263 2.0
0 -0.0127134323120117 3.0
0 0.0103989243507385 4.0
0 0.0108460187911987 5.0
0 0.00221699476242065 6.0
0 0.00128203630447388 7.0
38.951377458 38.6069030761719 0.0
0 0.00339537858963013 1.0
0 -0.00315690040588379 2.0
0 0.0101865530014038 3.0
0 0.00888776779174805 4.0
0 0.00692009925842285 5.0
0 0.00140982866287231 6.0
0 -0.00287401676177979 7.0
40.644172477 38.4279823303223 0.0
0 -0.00347089767456055 1.0
0 -0.00854629278182983 2.0
0 -0.00977569818496704 3.0
0 0.00184166431427002 4.0
0 -0.00165468454360962 5.0
0 -0.00119912624359131 6.0
0 -0.00389707088470459 7.0
35.600531382 38.9765777587891 0.0
0 0.000101447105407715 1.0
0 -0.00169456005096436 2.0
0 -0.0105034112930298 3.0
0 0.0159684419631958 4.0
0 0.0144085288047791 5.0
0 0.00176870822906494 6.0
0 -0.000409126281738281 7.0
41.682033599 38.8813819885254 0.0
0 0.0172576904296875 1.0
0 0.0325183868408203 2.0
0 -0.0265422463417053 3.0
0 0.0114198923110962 4.0
0 0.00883734226226807 5.0
0 0.00202566385269165 6.0
0 -0.00066763162612915 7.0
39.697348938 38.9553298950195 0.0
0 0.00576400756835938 1.0
0 0.00627702474594116 2.0
0 -0.0105857849121094 3.0
0 0.0120393037796021 4.0
0 0.00906950235366821 5.0
0 0.00216853618621826 6.0
0 0.00109446048736572 7.0
40.978668493 37.8965797424316 0.0
0 -0.00433230400085449 1.0
0 0.00928252935409546 2.0
0 -0.0110325813293457 3.0
0 -0.010481059551239 4.0
0 -0.00912737846374512 5.0
0 0.000316202640533447 6.0
0 -0.00422096252441406 7.0
46.524483465 38.4956130981445 0.0
0 0.0072096586227417 1.0
0 -0.00128740072250366 2.0
0 0.00465559959411621 3.0
0 0.0165053009986877 4.0
0 0.00841742753982544 5.0
0 0.00209641456604004 6.0
0 -0.0013575553894043 7.0
39.683969876 38.3914260864258 0.0
0 0.00807690620422363 1.0
0 -0.0101870894432068 2.0
0 0.00545310974121094 3.0
0 0.014818549156189 4.0
0 0.0115963220596313 5.0
0 0.00100439786911011 6.0
0 -0.00206196308135986 7.0
46.294692767 39.0759162902832 0.0
0 0.0209541916847229 1.0
0 -0.00489115715026855 2.0
0 0.0438747406005859 3.0
0 -0.0146594643592834 4.0
0 -0.01612788438797 5.0
0 0.00321435928344727 6.0
0 -0.000924825668334961 7.0
44.785541844 38.8168334960938 0.0
0 0.00351965427398682 1.0
0 0.00326263904571533 2.0
0 0.0207355618476868 3.0
0 0.00808155536651611 4.0
0 0.00271213054656982 5.0
0 0.00201249122619629 6.0
0 -0.0048290491104126 7.0
33.676333572 38.2910499572754 0.0
0 0.00158512592315674 1.0
0 -0.00176751613616943 2.0
0 -0.00117731094360352 3.0
0 0.0104479193687439 4.0
0 0.0066227912902832 5.0
0 0.0014575719833374 6.0
0 -0.00686109066009521 7.0
35.625618922 38.0471572875977 0.0
0 0.00239014625549316 1.0
0 -0.0126388072967529 2.0
0 -0.0120697021484375 3.0
0 0.00128597021102905 4.0
0 0.00199592113494873 5.0
0 -0.000860452651977539 6.0
0 -0.00589412450790405 7.0
40.226157029 38.6796798706055 0.0
0 0.00153261423110962 1.0
0 0.00714719295501709 2.0
0 0.00271427631378174 3.0
0 0.0039246678352356 4.0
0 0.00176608562469482 5.0
0 0.0020751953125 6.0
0 -0.00623393058776855 7.0
44.003195919 38.717472076416 0.0
0 0.00164341926574707 1.0
0 0.00996559858322144 2.0
0 -0.00653207302093506 3.0
0 0.00345414876937866 4.0
0 0.00210237503051758 5.0
0 0.00132018327713013 6.0
0 -0.00519376993179321 7.0
33.88878315 38.4337196350098 0.0
0 0.00824373960494995 1.0
0 -0.000991284847259521 2.0
0 0.0105270147323608 3.0
0 0.0159042477607727 4.0
0 0.0113751888275146 5.0
0 0.00127136707305908 6.0
0 -0.00309556722640991 7.0
42.125521158 38.4778442382812 0.0
0 0.00595742464065552 1.0
0 0.007171630859375 2.0
0 0.0155715346336365 3.0
0 0.00982785224914551 4.0
0 0.00966256856918335 5.0
0 0.0015488862991333 6.0
0 0.00158339738845825 7.0
37.55067892 39.0216331481934 0.0
0 0.000815391540527344 1.0
0 0.0180549621582031 2.0
0 0.0161591768264771 3.0
0 0.00454628467559814 4.0
0 0.00465989112854004 5.0
0 0.00137084722518921 6.0
0 -0.000109493732452393 7.0
37.39948618 38.7834625244141 0.0
0 0.000694096088409424 1.0
0 0.017846941947937 2.0
0 -0.00839388370513916 3.0
0 0.0181208252906799 4.0
0 0.0139331221580505 5.0
0 0.00215917825698853 6.0
0 0.000113487243652344 7.0
34.341868993 38.6611785888672 0.0
0 -0.0150122046470642 1.0
0 -0.0378657579421997 2.0
0 0.0207475423812866 3.0
0 -0.000995099544525146 4.0
0 -0.00474607944488525 5.0
0 0.00366520881652832 6.0
0 0.00226479768753052 7.0
37.425355461 38.7103309631348 0.0
0 0.00776708126068115 1.0
0 0.00792419910430908 2.0
0 -0.00677996873855591 3.0
0 0.0113720297813416 4.0
0 0.00985175371170044 5.0
0 0.000823676586151123 6.0
0 -1.20997428894043e-05 7.0
37.288491912 38.4678955078125 0.0
0 0.00515389442443848 1.0
0 0.00180560350418091 2.0
0 0.0212438106536865 3.0
0 0.0104888081550598 4.0
0 0.00904351472854614 5.0
0 0.000123202800750732 6.0
0 -0.0016171932220459 7.0
39.20347441 38.805908203125 0.0
0 -0.00724637508392334 1.0
0 0.00258266925811768 2.0
0 0.0161644816398621 3.0
0 0.00277906656265259 4.0
0 0.000788569450378418 5.0
0 0.0032578706741333 6.0
0 0.000290453433990479 7.0
36.137884824 38.5297698974609 0.0
0 0.00461000204086304 1.0
0 -0.0104106664657593 2.0
0 -0.002341628074646 3.0
0 0.00298571586608887 4.0
0 0.00196605920791626 5.0
0 0.000843822956085205 6.0
0 -0.00134950876235962 7.0
41.655417288 38.3561172485352 0.0
0 0.00745469331741333 1.0
0 -0.00175750255584717 2.0
0 0.0158591866493225 3.0
0 0.0111903548240662 4.0
0 0.00902730226516724 5.0
0 0.000314235687255859 6.0
0 -0.00557076930999756 7.0
39.894978379 38.5850296020508 0.0
0 0.00680673122406006 1.0
0 0.00699466466903687 2.0
0 -0.0027390718460083 3.0
0 0.0178101062774658 4.0
0 0.00734114646911621 5.0
0 0.0022856593132019 6.0
0 -0.00147330760955811 7.0
39.390677385 38.8203659057617 0.0
0 0.0151079297065735 1.0
0 0.0463476181030273 2.0
0 -0.0157824158668518 3.0
0 -0.0195214152336121 4.0
0 -0.0244238376617432 5.0
0 0.00370627641677856 6.0
0 -0.0034632682800293 7.0
38.248327703 38.6163101196289 0.0
0 0.00582766532897949 1.0
0 -0.00413334369659424 2.0
0 0.00381135940551758 3.0
0 0.0143280029296875 4.0
0 0.0112537741661072 5.0
0 0.0016014575958252 6.0
0 -0.000128448009490967 7.0
38.538636426 38.5361137390137 0.0
0 0.00641006231307983 1.0
0 -0.00027698278427124 2.0
0 0.0049251914024353 3.0
0 0.0107793807983398 4.0
0 0.0084531307220459 5.0
0 0.00101703405380249 6.0
0 -0.000242471694946289 7.0
30.093865158 39.2019309997559 0.0
0 0.0353890657424927 1.0
0 0.00355738401412964 2.0
0 0.04957115650177 3.0
0 -0.0436264872550964 4.0
0 -0.0471203327178955 5.0
0 0.00451117753982544 6.0
0 -0.000642895698547363 7.0
36.059945316 38.318187713623 0.0
0 0.00228762626647949 1.0
0 0.0050504207611084 2.0
0 -0.00272685289382935 3.0
0 0.0147392749786377 4.0
0 0.0124824047088623 5.0
0 0.00258147716522217 6.0
0 0.00151604413986206 7.0
41.382686141 38.4142570495605 0.0
0 -0.00156998634338379 1.0
0 0.0204638242721558 2.0
0 -0.00764346122741699 3.0
0 -0.00123339891433716 4.0
0 -0.0007476806640625 5.0
0 -0.00071108341217041 6.0
0 -0.00559014081954956 7.0
37.374314239 38.6772499084473 0.0
0 -0.00320535898208618 1.0
0 0.00326782464981079 2.0
0 -0.00733256340026855 3.0
0 0.0150027275085449 4.0
0 0.0146105885505676 5.0
0 0.00185835361480713 6.0
0 0.000728189945220947 7.0
37.689487865 38.6625061035156 0.0
0 -0.00688505172729492 1.0
0 0.0156492590904236 2.0
0 0.0139175653457642 3.0
0 0.00787997245788574 4.0
0 0.00172066688537598 5.0
0 0.00320947170257568 6.0
0 0.000371813774108887 7.0
36.541540581 38.4495582580566 0.0
0 0.00823324918746948 1.0
0 -0.00128722190856934 2.0
0 0.0112419724464417 3.0
0 0.0117476582527161 4.0
0 0.0131450295448303 5.0
0 0.000376760959625244 6.0
0 -0.00190281867980957 7.0
36.791141656 38.3827018737793 0.0
0 0.00603717565536499 1.0
0 0.00242966413497925 2.0
0 0.00899404287338257 3.0
0 0.0119273662567139 4.0
0 0.011038064956665 5.0
0 0.00139141082763672 6.0
0 0.000935733318328857 7.0
39.891605825 38.3263244628906 0.0
0 -0.000929117202758789 1.0
0 0.0420316457748413 2.0
0 -0.0310762524604797 3.0
0 -0.00990122556686401 4.0
0 -0.00977462530136108 5.0
0 -0.0024992823600769 6.0
0 -0.00472760200500488 7.0
34.771445328 38.6938209533691 0.0
0 0.00257289409637451 1.0
0 0.0160443186759949 2.0
0 -0.00607508420944214 3.0
0 0.015845775604248 4.0
0 0.0155516862869263 5.0
0 0.00151389837265015 6.0
0 -0.000127732753753662 7.0
44.638557165 38.5900001525879 0.0
0 -0.00749975442886353 1.0
0 0.0173272490501404 2.0
0 0.0101034641265869 3.0
0 0.0125020146369934 4.0
0 0.00756698846817017 5.0
0 0.00205510854721069 6.0
0 -0.00224572420120239 7.0
36.096180936 38.1678276062012 0.0
0 0.00536233186721802 1.0
0 0.00875014066696167 2.0
0 0.0115242004394531 3.0
0 0.0147314667701721 4.0
0 0.00792396068572998 5.0
0 0.00153219699859619 6.0
0 -0.00242865085601807 7.0
45.744029538 38.1977348327637 0.0
0 0.00154680013656616 1.0
0 -0.00525921583175659 2.0
0 0.0133904218673706 3.0
0 0.00973719358444214 4.0
0 0.0101900100708008 5.0
0 0.00129318237304688 6.0
0 0.00124156475067139 7.0
36.287608705 38.4067802429199 0.0
0 -0.00326824188232422 1.0
0 0.00791627168655396 2.0
0 0.00300085544586182 3.0
0 0.0167179703712463 4.0
0 0.012957751750946 5.0
0 0.00116145610809326 6.0
0 -0.00291407108306885 7.0
42.801830865 38.5026664733887 0.0
0 0.00598233938217163 1.0
0 0.00537014007568359 2.0
0 0.00846278667449951 3.0
0 0.00354379415512085 4.0
0 0.00350403785705566 5.0
0 0.00145411491394043 6.0
0 -0.0010840892791748 7.0
47.558612578 38.3811569213867 0.0
0 -0.00544494390487671 1.0
0 0.0251234173774719 2.0
0 0.0158400535583496 3.0
0 0.00700610876083374 4.0
0 0.00496816635131836 5.0
0 0.000510692596435547 6.0
0 -0.00318229198455811 7.0
47.058562462 38.4057235717773 0.0
0 0.00507622957229614 1.0
0 0.0046621561050415 2.0
0 -0.00457650423049927 3.0
0 0.0249253511428833 4.0
0 0.0235775709152222 5.0
0 0.00243175029754639 6.0
0 0.000473499298095703 7.0
43.087048946 38.619270324707 0.0
0 -0.0106337070465088 1.0
0 -0.0265575647354126 2.0
0 0.015188455581665 3.0
0 -0.00168472528457642 4.0
0 -0.00724762678146362 5.0
0 0.00427258014678955 6.0
0 0.0025297999382019 7.0
33.170190568 38.9567565917969 0.0
0 0.00517016649246216 1.0
0 0.00459200143814087 2.0
0 0.0193129181861877 3.0
0 0.00718289613723755 4.0
0 0.00489509105682373 5.0
0 0.00101983547210693 6.0
0 -0.00283050537109375 7.0
34.957431523 38.8416709899902 0.0
0 0.000963389873504639 1.0
0 0.00420975685119629 2.0
0 0.0198730826377869 3.0
0 -0.00562077760696411 4.0
0 -0.00709086656570435 5.0
0 0.00256907939910889 6.0
0 -0.00235521793365479 7.0
36.320328519 38.8440856933594 0.0
0 -0.00225573778152466 1.0
0 -0.00953090190887451 2.0
0 0.00566196441650391 3.0
0 0.00649935007095337 4.0
0 0.00460940599441528 5.0
0 0.0024418830871582 6.0
0 0.000105619430541992 7.0
44.085721677 38.4182586669922 0.0
0 0.000100314617156982 1.0
0 0.0046502947807312 2.0
0 0.000212252140045166 3.0
0 0.00853204727172852 4.0
0 0.00654923915863037 5.0
0 0.000967621803283691 6.0
0 -0.00571078062057495 7.0
45.211814568 38.3988380432129 0.0
0 0.000818192958831787 1.0
0 0.0170605778694153 2.0
0 -0.0164785385131836 3.0
0 0.0199121236801147 4.0
0 0.0166866183280945 5.0
0 0.00196439027786255 6.0
0 -0.000163257122039795 7.0
38.71721827 38.8017578125 0.0
0 -0.00146806240081787 1.0
0 0.00902062654495239 2.0
0 -0.0155308246612549 3.0
0 0.00887376070022583 4.0
0 0.0110338926315308 5.0
0 0.00151538848876953 6.0
0 0.00038301944732666 7.0
42.190454953 38.4542961120605 0.0
0 -0.00158572196960449 1.0
0 -0.00745916366577148 2.0
0 -0.0216897130012512 3.0
0 -0.00159525871276855 4.0
0 -0.00380432605743408 5.0
0 9.28044319152832e-05 6.0
0 -0.00571072101593018 7.0
40.616038376 38.7430953979492 0.0
0 -0.00362867116928101 1.0
0 -0.00363141298294067 2.0
0 -0.0144767761230469 3.0
0 0.0116175413131714 4.0
0 0.0105224847793579 5.0
0 0.00140535831451416 6.0
0 0.000889956951141357 7.0
31.887764532 38.5140266418457 0.0
0 -0.00434798002243042 1.0
0 0.00756442546844482 2.0
0 -0.0390388369560242 3.0
0 -0.0211315751075745 4.0
0 -0.0261675715446472 5.0
0 0.00929063558578491 6.0
0 -0.00467920303344727 7.0
39.764867486 39.1465721130371 0.0
0 0.038346529006958 1.0
0 0.0927492082118988 2.0
0 0.0490598678588867 3.0
0 -0.041797399520874 4.0
0 -0.0453233122825623 5.0
0 0.00138425827026367 6.0
0 -0.000771045684814453 7.0
40.805009287 38.1030654907227 0.0
0 -0.0296259522438049 1.0
0 -0.00586938858032227 2.0
0 0.00417482852935791 3.0
0 -0.00110441446304321 4.0
0 -0.00454360246658325 5.0
0 0.00295650959014893 6.0
0 -0.000849902629852295 7.0
39.795600818 38.255054473877 0.0
0 0.00796306133270264 1.0
0 -0.00954651832580566 2.0
0 0.0155543088912964 3.0
0 0.0102896094322205 4.0
0 0.00733578205108643 5.0
0 0.00113248825073242 6.0
0 -0.00258749723434448 7.0
38.685439721 38.1099815368652 0.0
0 -0.0193047523498535 1.0
0 0.026334285736084 2.0
0 0.014123797416687 3.0
0 0.00438123941421509 4.0
0 0.00159591436386108 5.0
0 0.0030590295791626 6.0
0 0.00219011306762695 7.0
42.152656556 38.6447944641113 0.0
0 0.00958836078643799 1.0
0 0.00431197881698608 2.0
0 0.00279456377029419 3.0
0 0.0114231109619141 4.0
0 0.014231264591217 5.0
0 0.00242865085601807 6.0
0 0.00234651565551758 7.0
37.458343334 39.1002197265625 0.0
0 0.118244767189026 1.0
0 0.0132309198379517 2.0
0 0.179237484931946 3.0
0 -0.0956413745880127 4.0
0 -0.109352290630341 5.0
0 0.0108833312988281 6.0
0 0.0037427544593811 7.0
44.489633946 37.3663024902344 0.0
0 -0.000156283378601074 1.0
0 0.0174140930175781 2.0
0 -0.0660887956619263 3.0
0 -0.0274395942687988 4.0
0 -0.0323089957237244 5.0
0 0.0018419623374939 6.0
0 -0.00307917594909668 7.0
41.506640114 38.712272644043 0.0
0 0.00974112749099731 1.0
0 -0.0041663646697998 2.0
0 0.00747394561767578 3.0
0 0.0130130648612976 4.0
0 0.0114015340805054 5.0
0 0.00225138664245605 6.0
0 0.0011325478553772 7.0
33.262736569 38.4637565612793 0.0
0 -0.00413501262664795 1.0
0 -0.0077093243598938 2.0
0 0.00772899389266968 3.0
0 -0.0170835852622986 4.0
0 -0.0186598896980286 5.0
0 9.43541526794434e-05 6.0
0 -0.00329750776290894 7.0
47.069969764 39.1127662658691 0.0
0 0.0034937858581543 1.0
0 0.0224937796592712 2.0
0 0.0202521085739136 3.0
0 0.00589430332183838 4.0
0 0.00132709741592407 5.0
0 0.00247538089752197 6.0
0 -0.00191307067871094 7.0
43.288539967 38.7025947570801 0.0
0 0.0100835561752319 1.0
0 0.00247675180435181 2.0
0 0.0297904014587402 3.0
0 -0.00783008337020874 4.0
0 -0.00629055500030518 5.0
0 0.00282490253448486 6.0
0 0.00166058540344238 7.0
41.479216264 38.3655395507812 0.0
0 -0.0076107382774353 1.0
0 0.0249927639961243 2.0
0 0.0158962607383728 3.0
0 0.00533664226531982 4.0
0 0.00829720497131348 5.0
0 0.00181961059570312 6.0
0 0.00211477279663086 7.0
36.99162076 38.5888214111328 0.0
0 0.00175654888153076 1.0
0 0.023786187171936 2.0
0 0.0185081958770752 3.0
0 0.010552704334259 4.0
0 0.00637632608413696 5.0
0 0.00184732675552368 6.0
0 -0.0059053897857666 7.0
40.435886771 38.986686706543 0.0
0 -0.00245356559753418 1.0
0 0.0295779705047607 2.0
0 -0.0167503356933594 3.0
0 0.0110731720924377 4.0
0 0.00603735446929932 5.0
0 0.00212985277175903 6.0
0 -0.000238656997680664 7.0
37.140737383 38.4575042724609 0.0
0 0.00847417116165161 1.0
0 0.00707650184631348 2.0
0 0.00589668750762939 3.0
0 0.00967586040496826 4.0
0 0.0113044381141663 5.0
0 0.000883042812347412 6.0
0 0.00042259693145752 7.0
39.233907158 38.9101257324219 0.0
0 0.0550709962844849 1.0
0 0.148562252521515 2.0
0 0.0203211903572083 3.0
0 -0.0188177227973938 4.0
0 -0.0227159857749939 5.0
0 0.00519031286239624 6.0
0 0.000806093215942383 7.0
42.028602534 38.4750175476074 0.0
0 -0.00834649801254272 1.0
0 0.161355286836624 2.0
0 -0.0410969257354736 3.0
0 -0.0144053101539612 4.0
0 -0.0182055234909058 5.0
0 0.000284016132354736 6.0
0 -0.00558304786682129 7.0
39.783103652 39.0378684997559 0.0
0 0.0226157903671265 1.0
0 -0.0173406600952148 2.0
0 0.0640493631362915 3.0
0 -0.022208034992218 4.0
0 -0.0254800319671631 5.0
0 0.00419634580612183 6.0
0 -0.00270748138427734 7.0
39.425516455 38.1142044067383 0.0
0 0.003761887550354 1.0
0 0.00284242630004883 2.0
0 0.00697588920593262 3.0
0 0.00978696346282959 4.0
0 0.00723946094512939 5.0
0 0.00138962268829346 6.0
0 -0.00471216440200806 7.0
40.002559542 39.1183433532715 0.0
0 0.0111833214759827 1.0
0 0.0301645398139954 2.0
0 0.0249743461608887 3.0
0 0.00826179981231689 4.0
0 0.00183463096618652 5.0
0 0.00230365991592407 6.0
0 -0.00201362371444702 7.0
38.322665212 38.248119354248 0.0
0 0.00894296169281006 1.0
0 0.00682508945465088 2.0
0 -0.0240972638130188 3.0
0 -0.00541716814041138 4.0
0 -0.0107141733169556 5.0
0 0.0022168755531311 6.0
0 -0.0030866265296936 7.0
40.568600192 39.0411567687988 0.0
0 0.0145177841186523 1.0
0 0.0150798559188843 2.0
0 0.00254154205322266 3.0
0 -0.01081383228302 4.0
0 -0.014443039894104 5.0
0 0.00246632099151611 6.0
0 -0.00231659412384033 7.0
40.337689123 38.5351600646973 0.0
0 0.00957757234573364 1.0
0 0.0115842819213867 2.0
0 0.0130894184112549 3.0
0 0.00751912593841553 4.0
0 0.00370407104492188 5.0
0 0.00157725811004639 6.0
0 -0.00272804498672485 7.0
45.726007711 39.016544342041 0.0
0 -0.00627726316452026 1.0
0 0.0215297341346741 2.0
0 -0.0119922757148743 3.0
0 0.010688841342926 4.0
0 0.00892925262451172 5.0
0 0.0013124942779541 6.0
0 0.000753998756408691 7.0
36.675787374 38.2430000305176 0.0
0 -0.00158858299255371 1.0
0 0.0261760354042053 2.0
0 -0.00662976503372192 3.0
0 -0.0059894323348999 4.0
0 -0.00984197854995728 5.0
0 0.00290602445602417 6.0
0 -0.00482195615768433 7.0
32.235093382 38.4787483215332 0.0
0 0.00654047727584839 1.0
0 0.00283539295196533 2.0
0 0.0180583000183105 3.0
0 0.0143955945968628 4.0
0 0.00843143463134766 5.0
0 0.00196719169616699 6.0
0 -0.000648856163024902 7.0
39.438537732 38.3393020629883 0.0
0 0.00482273101806641 1.0
0 -0.0100158452987671 2.0
0 0.00550246238708496 3.0
0 0.00398647785186768 4.0
0 0.00508660078048706 5.0
0 0.000476419925689697 6.0
0 -0.000379323959350586 7.0
44.304614406 38.6942520141602 0.0
0 0.00724595785140991 1.0
0 0.00359833240509033 2.0
0 -0.00244873762130737 3.0
0 0.0168184041976929 4.0
0 0.018384575843811 5.0
0 0.00267505645751953 6.0
0 0.00242710113525391 7.0
32.373304325 38.7011947631836 0.0
0 -0.00139033794403076 1.0
0 0.0330175161361694 2.0
0 0.0167226791381836 3.0
0 -0.00249332189559937 4.0
0 -0.00695604085922241 5.0
0 0.00298500061035156 6.0
0 -0.00275629758834839 7.0
35.769379856 38.7898292541504 0.0
0 0.00197267532348633 1.0
0 0.029208242893219 2.0
0 0.0212281942367554 3.0
0 0.00471234321594238 4.0
0 0.00653988122940063 5.0
0 0.000956416130065918 6.0
0 -0.000651717185974121 7.0
38.348316661 38.5863151550293 0.0
0 -0.00654613971710205 1.0
0 0.0357071757316589 2.0
0 0.00473201274871826 3.0
0 0.00128424167633057 4.0
0 -0.00100201368331909 5.0
0 0.00222885608673096 6.0
0 -0.00584298372268677 7.0
30.534523433 39.0130577087402 0.0
0 0.00583195686340332 1.0
0 0.0315658450126648 2.0
0 0.0169404745101929 3.0
0 0.00761932134628296 4.0
0 0.0039747953414917 5.0
0 0.00238054990768433 6.0
0 -0.00270819664001465 7.0
41.665325899 38.7758674621582 0.0
0 0.000107109546661377 1.0
0 0.0110929012298584 2.0
0 0.0183548331260681 3.0
0 0.00898802280426025 4.0
0 0.00583487749099731 5.0
0 0.00194180011749268 6.0
0 -0.00388616323471069 7.0
39.125756769 38.7648773193359 0.0
0 -0.0030866265296936 1.0
0 0.00142174959182739 2.0
0 0.0009346604347229 3.0
0 0.00942152738571167 4.0
0 0.00966179370880127 5.0
0 0.00102788209915161 6.0
0 -0.00109249353408813 7.0
46.922469454 38.955696105957 0.0
0 -0.00348621606826782 1.0
0 0.000359296798706055 2.0
0 0.0120048522949219 3.0
0 0.00776892900466919 4.0
0 0.00494801998138428 5.0
0 0.00185239315032959 6.0
0 0.000420093536376953 7.0
38.082156965 38.3118515014648 0.0
0 0.0151529312133789 1.0
0 -0.0102242231369019 2.0
0 0.0122508406639099 3.0
0 -0.0131518244743347 4.0
0 -0.0139512419700623 5.0
0 0.00426769256591797 6.0
0 0.00295782089233398 7.0
35.281773346 38.6570739746094 0.0
0 -0.00272595882415771 1.0
0 0.0289946794509888 2.0
0 0.00509798526763916 3.0
0 -0.00448018312454224 4.0
0 -0.00222653150558472 5.0
0 0.000601649284362793 6.0
0 -0.00430136919021606 7.0
39.676820236 38.989990234375 0.0
0 0.00472760200500488 1.0
0 0.0294286012649536 2.0
0 0.0172818303108215 3.0
0 0.00415349006652832 4.0
0 0.00443726778030396 5.0
0 0.00188541412353516 6.0
0 -0.000945448875427246 7.0
41.049027324 38.6982574462891 0.0
0 0.00343352556228638 1.0
0 0.0113016366958618 2.0
0 -0.00696408748626709 3.0
0 0.0125917196273804 4.0
0 0.00998294353485107 5.0
0 0.00137799978256226 6.0
0 0.000242471694946289 7.0
41.402616693 39.0029830932617 0.0
0 0.00863611698150635 1.0
0 0.087821900844574 2.0
0 0.0603731870651245 3.0
0 -0.0356480479240417 4.0
0 -0.0399312376976013 5.0
0 0.00590997934341431 6.0
0 -9.55462455749512e-05 7.0
30.3721478 38.6933746337891 0.0
0 -0.00103241205215454 1.0
0 0.00393533706665039 2.0
0 -0.0112425088882446 3.0
0 -0.00764316320419312 4.0
0 -0.0103350281715393 5.0
0 0.000120401382446289 6.0
0 -0.00539594888687134 7.0
41.111385117 38.5219116210938 0.0
0 0.00852882862091064 1.0
0 -0.00877469778060913 2.0
0 0.0205832123756409 3.0
0 0.0103094577789307 4.0
0 0.0121785402297974 5.0
0 0.000267446041107178 6.0
0 -0.000113844871520996 7.0
36.921893091 38.6074638366699 0.0
0 0.00677359104156494 1.0
0 -0.00750589370727539 2.0
0 0.00729429721832275 3.0
0 0.0145912766456604 4.0
0 0.0131081342697144 5.0
0 0.000901758670806885 6.0
0 -0.000841677188873291 7.0
41.531234868 38.1064682006836 0.0
0 0.0153313875198364 1.0
0 0.0151869654655457 2.0
0 -0.00373983383178711 3.0
0 -0.0103966593742371 4.0
0 -0.0106770396232605 5.0
0 0.00317627191543579 6.0
0 0.00247424840927124 7.0
36.809020967 38.4955902099609 0.0
0 0.00135868787765503 1.0
0 0.00796598196029663 2.0
0 -0.00358861684799194 3.0
0 -0.00392121076583862 4.0
0 -0.00673782825469971 5.0
0 0.00253629684448242 6.0
0 -0.00510793924331665 7.0
30.318323192 38.832633972168 0.0
0 -0.00723063945770264 1.0
0 0.0219334959983826 2.0
0 0.010265588760376 3.0
0 0.00689631700515747 4.0
0 0.00381928682327271 5.0
0 0.00178146362304688 6.0
0 -0.00287723541259766 7.0
37.371827278 39.0299491882324 0.0
0 0.0118033885955811 1.0
0 0.0467261672019958 2.0
0 0.0143779516220093 3.0
0 -0.0137546062469482 4.0
0 -0.0207595229148865 5.0
0 0.00627964735031128 6.0
0 0.00312989950180054 7.0
37.183742215 38.1556701660156 0.0
0 0.00127106904983521 1.0
0 0.0066341757774353 2.0
0 -0.0147396326065063 3.0
0 0.00577563047409058 4.0
0 0.0029301643371582 5.0
0 -0.00182098150253296 6.0
0 -0.00620943307876587 7.0
34.629310928 38.365364074707 0.0
0 0.00610154867172241 1.0
0 0.00508695840835571 2.0
0 0.0198239088058472 3.0
0 0.013192892074585 4.0
0 0.00678855180740356 5.0
0 0.00147479772567749 6.0
0 -0.00184899568557739 7.0
42.830967498 38.9176712036133 0.0
0 -0.000261783599853516 1.0
0 0.0146018862724304 2.0
0 -0.011113166809082 3.0
0 0.0185452699661255 4.0
0 0.00816124677658081 5.0
0 0.00284934043884277 6.0
0 -8.62479209899902e-05 7.0
34.694327372 38.3558731079102 0.0
0 0.00509011745452881 1.0
0 0.000261187553405762 2.0
0 0.0105280876159668 3.0
0 0.0111966133117676 4.0
0 0.0079917311668396 5.0
0 0.00138604640960693 6.0
0 -0.00158607959747314 7.0
37.611435874 39.1154098510742 0.0
0 -0.00671660900115967 1.0
0 0.0181795358657837 2.0
0 -0.000290572643280029 3.0
0 0.0106480121612549 4.0
0 0.00586879253387451 5.0
0 0.00247448682785034 6.0
0 0.000689864158630371 7.0
45.529166488 38.6235542297363 0.0
0 -0.00425219535827637 1.0
0 0.0054473876953125 2.0
0 0.0149739980697632 3.0
0 0.00727999210357666 4.0
0 0.00684750080108643 5.0
0 0.00113421678543091 6.0
0 -0.00142967700958252 7.0
40.26689599 38.7462539672852 0.0
0 -0.00593769550323486 1.0
0 -0.00408351421356201 2.0
0 0.00365918874740601 3.0
0 0.00439119338989258 4.0
0 0.00260257720947266 5.0
0 0.00294429063796997 6.0
0 0.00205498933792114 7.0
39.577958661 38.4459991455078 0.0
0 0.00205034017562866 1.0
0 0.00105744600296021 2.0
0 -0.00728344917297363 3.0
0 0.017594575881958 4.0
0 0.0188379287719727 5.0
0 0.00254732370376587 6.0
0 0.00236052274703979 7.0
34.564585703 38.8292350769043 0.0
0 -0.0269166231155396 1.0
0 0.136175036430359 2.0
0 0.053674578666687 3.0
0 -0.047590434551239 4.0
0 -0.0486240386962891 5.0
0 0.00632476806640625 6.0
0 0.00384056568145752 7.0
34.747667205 38.7812385559082 0.0
0 0.0150026679039001 1.0
0 0.00320255756378174 2.0
0 -0.0252484083175659 3.0
0 -0.0210506319999695 4.0
0 -0.0219604969024658 5.0
0 0.00306332111358643 6.0
0 -0.00150471925735474 7.0
44.886437438 38.5786972045898 0.0
0 -0.00230449438095093 1.0
0 0.00125682353973389 2.0
0 0.00380456447601318 3.0
0 0.0123153924942017 4.0
0 0.0131455063819885 5.0
0 0.000817179679870605 6.0
0 -6.07967376708984e-05 7.0
34.458606901 38.426197052002 0.0
0 0.0095527172088623 1.0
0 -0.0173845887184143 2.0
0 0.0205413103103638 3.0
0 0.013698399066925 4.0
0 0.00839120149612427 5.0
0 0.00135630369186401 6.0
0 -0.00317841768264771 7.0
44.985938301 38.635196685791 0.0
0 0.0398619174957275 1.0
0 0.0797373652458191 2.0
0 -0.0432404279708862 3.0
0 -0.0900113582611084 4.0
0 -0.0522611737251282 5.0
0 0.00559622049331665 6.0
0 0.00259339809417725 7.0
36.561528733 38.4419441223145 0.0
0 -0.00396859645843506 1.0
0 -0.0254343748092651 2.0
0 -0.0119385123252869 3.0
0 -0.0315827131271362 4.0
0 -0.0329383611679077 5.0
0 0.00273001194000244 6.0
0 -0.00205951929092407 7.0
35.585636345 38.4601249694824 0.0
0 -0.00341349840164185 1.0
0 0.0154292583465576 2.0
0 0.0149226188659668 3.0
0 0.00507611036300659 4.0
0 0.00384801626205444 5.0
0 0.000981569290161133 6.0
0 -0.0022132396697998 7.0
34.748633645 38.3581428527832 0.0
0 0.0046808123588562 1.0
0 0.00470662117004395 2.0
0 -0.0016711950302124 3.0
0 0.0164695382118225 4.0
0 0.0109076499938965 5.0
0 0.0013577938079834 6.0
0 -0.00368696451187134 7.0
33.997517461 38.4255676269531 0.0
0 -0.00385230779647827 1.0
0 0.0130151510238647 2.0
0 -0.00214368104934692 3.0
0 0.0129605531692505 4.0
0 0.0127876400947571 5.0
0 0.00150924921035767 6.0
0 0.000913619995117188 7.0
31.337837162 38.988941192627 0.0
0 0.00888556241989136 1.0
0 0.0149818062782288 2.0
0 0.0195881128311157 3.0
0 0.00459617376327515 4.0
0 0.00473707914352417 5.0
0 0.00146538019180298 6.0
0 0.000113725662231445 7.0
43.914614211 38.5083885192871 0.0
0 0.0040326714515686 1.0
0 -0.014149010181427 2.0
0 0.0157884955406189 3.0
0 0.0115993022918701 4.0
0 0.00837445259094238 5.0
0 0.00118738412857056 6.0
0 -0.000306010246276855 7.0
45.186574618 38.4571304321289 0.0
0 0.00369459390640259 1.0
0 -0.00190377235412598 2.0
0 0.00506937503814697 3.0
0 0.00619816780090332 4.0
0 0.00701636075973511 5.0
0 0.000674545764923096 6.0
0 -0.00296097993850708 7.0
35.268441876 38.6745338439941 0.0
0 0.0259495973587036 1.0
0 0.00516915321350098 2.0
0 0.0271381139755249 3.0
0 -0.0043373703956604 4.0
0 -0.00707775354385376 5.0
0 0.00333774089813232 6.0
0 -0.000590026378631592 7.0
32.333098982 38.339469909668 0.0
0 0.00736904144287109 1.0
0 -0.012636661529541 2.0
0 -0.00687289237976074 3.0
0 0.0058826208114624 4.0
0 0.000943958759307861 5.0
0 0.000460386276245117 6.0
0 -0.00598776340484619 7.0
38.894207338 38.5355949401855 0.0
0 -0.0060042142868042 1.0
0 0.0341095924377441 2.0
0 0.00643140077590942 3.0
0 -0.00107800960540771 4.0
0 -0.00261771678924561 5.0
0 -0.000578522682189941 6.0
0 -0.00298482179641724 7.0
32.941924742 38.3487243652344 0.0
0 0.01023268699646 1.0
0 0.0263230800628662 2.0
0 -0.056096076965332 3.0
0 -0.0248920917510986 4.0
0 -0.0269808173179626 5.0
0 0.00285071134567261 6.0
0 0.00860178470611572 7.0
40.2164929 37.7063751220703 0.0
0 -0.00206410884857178 1.0
0 -0.0241124033927917 2.0
0 -0.0163887739181519 3.0
0 -0.00840198993682861 4.0
0 -0.0127372145652771 5.0
0 0.00169253349304199 6.0
0 -0.0047113299369812 7.0
40.810463567 39.0738258361816 0.0
0 0.00107890367507935 1.0
0 0.000585675239562988 2.0
0 -0.0132269859313965 3.0
0 0.0105957388877869 4.0
0 0.0103799700737 5.0
0 0.00216841697692871 6.0
0 0.00124680995941162 7.0
33.558410457 38.8956298828125 0.0
0 0.00843101739883423 1.0
0 -0.0107095837593079 2.0
0 -0.0284059047698975 3.0
0 0.000358760356903076 4.0
0 0.00435709953308105 5.0
0 0.00229853391647339 6.0
0 0.00229942798614502 7.0
47.788896524 38.6363067626953 0.0
0 -0.00472438335418701 1.0
0 -0.0201917886734009 2.0
0 0.0364751219749451 3.0
0 -0.0135514140129089 4.0
0 -0.016156792640686 5.0
0 0.00404399633407593 6.0
0 -0.00392872095108032 7.0
33.642782256 38.8176574707031 0.0
0 -0.00302708148956299 1.0
0 0.00466156005859375 2.0
0 0.0140516757965088 3.0
0 0.00737649202346802 4.0
0 0.00594466924667358 5.0
0 0.0017513632774353 6.0
0 -0.00115036964416504 7.0
37.659684029 38.3499221801758 0.0
0 0.0080101490020752 1.0
0 0.000805974006652832 2.0
0 0.0185391902923584 3.0
0 0.0152295827865601 4.0
0 0.00819528102874756 5.0
0 0.00101685523986816 6.0
0 -0.00529128313064575 7.0
40.91969971 38.140266418457 0.0
0 0.00345683097839355 1.0
0 -0.00896137952804565 2.0
0 -0.0119954347610474 3.0
0 0.00301778316497803 4.0
0 0.00167751312255859 5.0
0 0.00145190954208374 6.0
0 -0.00629293918609619 7.0
42.622265927 38.811466217041 0.0
0 0.019258975982666 1.0
0 0.0364774465560913 2.0
0 -0.00904768705368042 3.0
0 -0.0298802852630615 4.0
0 -0.0317170023918152 5.0
0 0.00465452671051025 6.0
0 -0.00226563215255737 7.0
33.946966747 38.4802627563477 0.0
0 -0.00518739223480225 1.0
0 0.032974898815155 2.0
0 0.0197080373764038 3.0
0 0.00946873426437378 4.0
0 0.00464081764221191 5.0
0 0.000941216945648193 6.0
0 -0.00420850515365601 7.0
36.409182194 38.5286026000977 0.0
0 0.00702512264251709 1.0
0 -0.0194560289382935 2.0
0 0.0110805034637451 3.0
0 0.0136269330978394 4.0
0 0.0101454257965088 5.0
0 0.00085604190826416 6.0
0 -0.0050346851348877 7.0
40.593448001 38.5760307312012 0.0
0 0.0027921199798584 1.0
0 0.000395596027374268 2.0
0 -0.00285184383392334 3.0
0 0.0092308521270752 4.0
0 0.010448157787323 5.0
0 0.000471591949462891 6.0
0 0.00133585929870605 7.0
41.715980637 38.8361663818359 0.0
0 0.0497987866401672 1.0
0 0.0365809798240662 2.0
0 0.0144861936569214 3.0
0 -0.0297126173973083 4.0
0 -0.0379680395126343 5.0
0 0.00509655475616455 6.0
0 -0.00242024660110474 7.0
33.77037363 38.5182495117188 0.0
0 0.0106807351112366 1.0
0 0.083455502986908 2.0
0 -0.0791702270507812 3.0
0 -0.052739679813385 4.0
0 -0.0625720620155334 5.0
0 0.00684452056884766 6.0
0 -0.000840663909912109 7.0
46.074402277 38.5328330993652 0.0
0 0.0320846438407898 1.0
0 0.0307254195213318 2.0
0 0.0312415361404419 3.0
0 -0.0108807682991028 4.0
0 -0.0112593173980713 5.0
0 0.00355857610702515 6.0
0 0.00201642513275146 7.0
40.510394826 38.7015953063965 0.0
0 0.00823861360549927 1.0
0 -0.0077546238899231 2.0
0 0.046064555644989 3.0
0 -0.0181393027305603 4.0
0 -0.017975926399231 5.0
0 0.002483069896698 6.0
0 0.00150805711746216 7.0
39.244329381 38.7511940002441 0.0
0 0.0159115791320801 1.0
0 0.0022616982460022 2.0
0 0.0282448530197144 3.0
0 -0.00502586364746094 4.0
0 -0.00408065319061279 5.0
0 0.00202155113220215 6.0
0 0.000927567481994629 7.0
35.7171596 38.6546821594238 0.0
0 0.00409018993377686 1.0
0 -0.00489550828933716 2.0
0 -0.0026392936706543 3.0
0 0.0113359689712524 4.0
0 0.0123302340507507 5.0
0 0.000735223293304443 6.0
0 0.00068289041519165 7.0
39.754714125 38.8880348205566 0.0
0 -0.00539577007293701 1.0
0 -0.00136715173721313 2.0
0 -0.0127389430999756 3.0
0 0.00762850046157837 4.0
0 -0.000317215919494629 5.0
0 0.0031813383102417 6.0
0 -0.00161910057067871 7.0
41.796195011 38.343635559082 0.0
0 0.00867635011672974 1.0
0 -0.008350670337677 2.0
0 0.0206466317176819 3.0
0 0.0107426643371582 4.0
0 0.0104712247848511 5.0
0 0.000457584857940674 6.0
0 -0.000905036926269531 7.0
29.980660557 38.4174346923828 0.0
0 0.00467085838317871 1.0
0 -0.00495713949203491 2.0
0 0.0162566900253296 3.0
0 0.00935602188110352 4.0
0 0.00986051559448242 5.0
0 0.000848650932312012 6.0
0 0.000751495361328125 7.0
36.241260622 38.5575332641602 0.0
0 0.000705897808074951 1.0
0 0.00149893760681152 2.0
0 0.0100005865097046 3.0
0 0.01280277967453 4.0
0 0.0111420154571533 5.0
0 0.000396430492401123 6.0
0 -0.00308972597122192 7.0
38.129703213 38.4214096069336 0.0
0 0.00814342498779297 1.0
0 0.000246703624725342 2.0
0 0.0169895887374878 3.0
0 0.0135315656661987 4.0
0 0.010769248008728 5.0
0 0.00128799676895142 6.0
0 -0.0017850399017334 7.0
36.91522515 38.2449417114258 0.0
0 0.0027318000793457 1.0
0 -0.00151664018630981 2.0
0 0.00988304615020752 3.0
0 0.00847196578979492 4.0
0 0.0108878016471863 5.0
0 0.000810861587524414 6.0
0 0.0011824369430542 7.0
38.102908751 38.1841430664062 0.0
0 0.00115394592285156 1.0
0 0.00325274467468262 2.0
0 0.0126217603683472 3.0
0 0.0149384140968323 4.0
0 0.0158239006996155 5.0
0 0.000880837440490723 6.0
0 -8.7440013885498e-05 7.0
38.34829669 38.6496887207031 0.0
0 -0.00112295150756836 1.0
0 0.0135008692741394 2.0
0 0.00834870338439941 3.0
0 0.0102553367614746 4.0
0 0.00822752714157104 5.0
0 0.00185030698776245 6.0
0 0.00093388557434082 7.0
31.13885609 38.5502471923828 0.0
0 0.000279843807220459 1.0
0 0.00190377235412598 2.0
0 0.00924229621887207 3.0
0 0.0154766440391541 4.0
0 0.0117242932319641 5.0
0 0.00106501579284668 6.0
0 -0.00504231452941895 7.0
35.130346793 39.991096496582 0.0
0 -0.172638535499573 1.0
0 0.176112532615662 2.0
0 0.261671185493469 3.0
0 -0.0960418581962585 4.0
0 -0.047913670539856 5.0
0 0.0104336738586426 6.0
0 0.00360935926437378 7.0
37.970631847 38.0968322753906 0.0
0 -0.00251191854476929 1.0
0 0.0147252678871155 2.0
0 -0.0113922357559204 3.0
0 -0.000210464000701904 4.0
0 -0.00579959154129028 5.0
0 0.00145632028579712 6.0
0 -0.00575977563858032 7.0
37.012414901 38.5231437683105 0.0
0 -0.0027344822883606 1.0
0 0.0352815389633179 2.0
0 0.00825810432434082 3.0
0 -0.00124883651733398 4.0
0 -0.0013393759727478 5.0
0 0.00134366750717163 6.0
0 -0.00226140022277832 7.0
36.779265883 38.894100189209 0.0
0 0.0063093900680542 1.0
0 0.027963399887085 2.0
0 -0.00222969055175781 3.0
0 0.0165801048278809 4.0
0 0.00685769319534302 5.0
0 0.00238025188446045 6.0
0 -0.00534164905548096 7.0
42.665291585 38.5481719970703 0.0
0 0.00762939453125 1.0
0 0.0562233328819275 2.0
0 -0.0685888528823853 3.0
0 -0.0368189811706543 4.0
0 -0.04285728931427 5.0
0 0.00503599643707275 6.0
0 -0.00157469511032104 7.0
35.413046011 38.4058380126953 0.0
0 0.00584030151367188 1.0
0 0.000958383083343506 2.0
0 0.0113482475280762 3.0
0 0.0107216835021973 4.0
0 0.00791442394256592 5.0
0 0.00128114223480225 6.0
0 0.000605762004852295 7.0
35.966339402 38.4775657653809 0.0
0 0.00496381521224976 1.0
0 -0.00544369220733643 2.0
0 0.00988161563873291 3.0
0 0.0164755582809448 4.0
0 0.0110843181610107 5.0
0 0.00106734037399292 6.0
0 -0.00433731079101562 7.0
39.418493945 38.8547630310059 0.0
0 0.00878036022186279 1.0
0 0.000721454620361328 2.0
0 -0.0158603191375732 3.0
0 0.0104245543479919 4.0
0 0.0125052928924561 5.0
0 0.000981569290161133 6.0
0 0.000788688659667969 7.0
32.151549765 38.9702301025391 0.0
0 -0.0014265775680542 1.0
0 0.012227475643158 2.0
0 0.0263000726699829 3.0
0 -0.00517529249191284 4.0
0 -0.00719040632247925 5.0
0 0.00259649753570557 6.0
0 -0.00119811296463013 7.0
31.845645841 38.3061141967773 0.0
0 0.00534671545028687 1.0
0 -0.0176301002502441 2.0
0 -0.0113786458969116 3.0
0 0.00411343574523926 4.0
0 0.00402390956878662 5.0
0 -0.002593994140625 6.0
0 -0.0061572790145874 7.0
40.557128571 38.4783058166504 0.0
0 -0.0107422471046448 1.0
0 0.0388541221618652 2.0
0 0.0169771909713745 3.0
0 0.00476694107055664 4.0
0 0.00180530548095703 5.0
0 0.00234603881835938 6.0
0 0.00116020441055298 7.0
36.793774817 38.6162338256836 0.0
0 0.00310921669006348 1.0
0 0.0043950080871582 2.0
0 0.0118415951728821 3.0
0 -0.0224356055259705 4.0
0 -0.0218193531036377 5.0
0 0.00348061323165894 6.0
0 -0.00138634443283081 7.0
41.333576969 38.9786033630371 0.0
0 0.0102357268333435 1.0
0 -0.00271809101104736 2.0
0 -0.0131592750549316 3.0
0 0.0101926326751709 4.0
0 0.0110886096954346 5.0
0 0.00132131576538086 6.0
0 0.00104010105133057 7.0
43.791455275 38.4271545410156 0.0
0 -0.00638270378112793 1.0
0 0.0200754404067993 2.0
0 0.0120011568069458 3.0
0 0.0064471960067749 4.0
0 0.00578755140304565 5.0
0 0.00162583589553833 6.0
0 0.00128662586212158 7.0
41.481697076 38.2028846740723 0.0
0 0.00590914487838745 1.0
0 0.00448429584503174 2.0
0 -0.0016942024230957 3.0
0 0.00982093811035156 4.0
0 0.0135623216629028 5.0
0 0.000484585762023926 6.0
0 0.000510692596435547 7.0
35.532774621 38.3699798583984 0.0
0 0.00557297468185425 1.0
0 0.00183498859405518 2.0
0 0.00422048568725586 3.0
0 0.0129894614219666 4.0
0 0.0152958631515503 5.0
0 0.00223052501678467 6.0
0 0.00200212001800537 7.0
34.894006808 38.6870956420898 0.0
0 -0.00032496452331543 1.0
0 -0.000553488731384277 2.0
0 -0.0101300477981567 3.0
0 0.0154532194137573 4.0
0 0.0144065618515015 5.0
0 0.00112330913543701 6.0
0 -0.00057142972946167 7.0
38.893810297 38.5068321228027 0.0
0 0.00826197862625122 1.0
0 0.00306463241577148 2.0
0 0.00759679079055786 3.0
0 0.0136953592300415 4.0
0 0.0143409967422485 5.0
0 0.00131523609161377 6.0
0 0.000699937343597412 7.0
39.624663522 38.6048698425293 0.0
0 -0.0116535425186157 1.0
0 0.00954610109329224 2.0
0 -0.00461006164550781 3.0
0 0.017173171043396 4.0
0 0.0123999118804932 5.0
0 0.00213772058486938 6.0
0 3.28421592712402e-05 7.0
42.516231091 39.1451377868652 0.0
0 0.0351818799972534 1.0
0 0.110571384429932 2.0
0 0.0638208389282227 3.0
0 -0.0394923686981201 4.0
0 -0.0376558303833008 5.0
0 0.00706255435943604 6.0
0 0.006949782371521 7.0
38.374768577 38.2217788696289 0.0
0 0.0492962598800659 1.0
0 0.00276666879653931 2.0
0 -0.0941604375839233 3.0
0 -0.0646132230758667 4.0
0 -0.0723511576652527 5.0
0 0.00604933500289917 6.0
0 0.000296711921691895 7.0
33.351459508 39.1657485961914 0.0
0 -0.00859594345092773 1.0
0 0.0212362408638 2.0
0 -0.00135296583175659 3.0
0 0.00774955749511719 4.0
0 0.00817430019378662 5.0
0 0.00212478637695312 6.0
0 0.00165778398513794 7.0
37.08812014 38.7333183288574 0.0
0 -0.000715315341949463 1.0
0 0.00112462043762207 2.0
0 -0.00267255306243896 3.0
0 0.011474072933197 4.0
0 0.011874794960022 5.0
0 0.00210350751876831 6.0
0 0.00163459777832031 7.0
44.445084649 38.5677528381348 0.0
0 0.00638324022293091 1.0
0 -0.00895494222640991 2.0
0 0.00653094053268433 3.0
0 0.0088040828704834 4.0
0 0.0119718909263611 5.0
0 0.000811874866485596 6.0
0 0.00173741579055786 7.0
37.22322104 38.371509552002 0.0
0 0.00562351942062378 1.0
0 0.00389868021011353 2.0
0 0.0112352967262268 3.0
0 0.0115906000137329 4.0
0 0.0116457343101501 5.0
0 0.000585973262786865 6.0
0 -0.00184345245361328 7.0
44.605944609 38.6993217468262 0.0
0 0.0200371146202087 1.0
0 0.0074923038482666 2.0
0 -0.0283963084220886 3.0
0 -0.0215704441070557 4.0
0 -0.0214748382568359 5.0
0 0.00237280130386353 6.0
0 0.00550472736358643 7.0
39.553698498 38.4618606567383 0.0
0 0.00432324409484863 1.0
0 0.0075610876083374 2.0
0 0.00864207744598389 3.0
0 0.015913724899292 4.0
0 0.013094961643219 5.0
0 0.000805139541625977 6.0
0 -0.00503039360046387 7.0
43.287363281 39.1227188110352 0.0
0 0.0155909657478333 1.0
0 0.0208890438079834 2.0
0 0.0387828350067139 3.0
0 -0.00965893268585205 4.0
0 -0.0132709741592407 5.0
0 0.00286310911178589 6.0
0 -0.0010300874710083 7.0
33.992156471 38.1259117126465 0.0
0 -0.000116407871246338 1.0
0 -0.00216031074523926 2.0
0 0.00474166870117188 3.0
0 0.0110854506492615 4.0
0 0.0120851397514343 5.0
0 -0.000650346279144287 6.0
0 -0.00454753637313843 7.0
36.860093248 39.0079536437988 0.0
0 -0.00831305980682373 1.0
0 0.00342237949371338 2.0
0 -0.00300586223602295 3.0
0 0.00467437505722046 4.0
0 0.00290507078170776 5.0
0 0.00205975770950317 6.0
0 0.00154572725296021 7.0
39.672962738 38.0751495361328 0.0
0 0.00984746217727661 1.0
0 -0.0117852091789246 2.0
0 0.0192103981971741 3.0
0 0.00903779268264771 4.0
0 0.00922292470932007 5.0
0 -1.78813934326172e-06 6.0
0 -0.00248432159423828 7.0
32.8535328 38.515998840332 0.0
0 0.00934451818466187 1.0
0 -0.0145254731178284 2.0
0 0.0208010673522949 3.0
0 0.00874799489974976 4.0
0 0.00777202844619751 5.0
0 0.00117224454879761 6.0
0 -0.00193655490875244 7.0
36.422494688 38.2510757446289 0.0
0 0.00413292646408081 1.0
0 0.0195381045341492 2.0
0 -0.0251415371894836 3.0
0 -0.00397628545761108 4.0
0 -0.00545597076416016 5.0
0 0.00688892602920532 6.0
0 -0.00616079568862915 7.0
32.082396771 38.4953002929688 0.0
0 0.0391383171081543 1.0
0 0.0698838829994202 2.0
0 -0.0538221597671509 3.0
0 -0.0767565369606018 4.0
0 -0.080767035484314 5.0
0 0.00638920068740845 6.0
0 0.00216215848922729 7.0
43.742460686 38.6314849853516 0.0
0 0.00743508338928223 1.0
0 0.00358933210372925 2.0
0 0.00233936309814453 3.0
0 0.0148839354515076 4.0
0 0.0103566646575928 5.0
0 0.00171154737472534 6.0
0 -0.00294375419616699 7.0
37.830554934 38.8711471557617 0.0
0 -0.00633442401885986 1.0
0 -0.085806131362915 2.0
0 0.0016249418258667 3.0
0 -0.00366508960723877 4.0
0 -0.00491905212402344 5.0
0 0.00461995601654053 6.0
0 0.00372880697250366 7.0
45.388745763 38.5289192199707 0.0
0 0.00618654489517212 1.0
0 0.00241726636886597 2.0
0 0.00818026065826416 3.0
0 0.0156952738761902 4.0
0 0.0136352181434631 5.0
0 0.00150501728057861 6.0
0 -0.00034487247467041 7.0
40.010376396 38.5327529907227 0.0
0 0.00943154096603394 1.0
0 -0.0237060785293579 2.0
0 0.0177397727966309 3.0
0 0.00880545377731323 4.0
0 0.00820803642272949 5.0
0 0.000179648399353027 6.0
0 -0.00252014398574829 7.0
45.566990187 38.9576721191406 0.0
0 -0.00651991367340088 1.0
0 0.0196529626846313 2.0
0 -0.0233531594276428 3.0
0 0.00508701801300049 4.0
0 0.00925230979919434 5.0
0 0.000830531120300293 6.0
0 0.00188058614730835 7.0
32.300808889 39.3677978515625 0.0
0 -0.139041066169739 1.0
0 0.0485532879829407 2.0
0 0.379734396934509 3.0
0 -0.113313436508179 4.0
0 0.248469144105911 5.0
0 0.0100367069244385 6.0
0 0.00433242321014404 7.0
36.805463378 38.648265838623 0.0
0 -0.0118266344070435 1.0
0 -0.0432865023612976 2.0
0 0.023951530456543 3.0
0 -0.00356596708297729 4.0
0 -0.00542491674423218 5.0
0 0.00420564413070679 6.0
0 0.00177103281021118 7.0
45.039727992 38.5601387023926 0.0
0 0.00589120388031006 1.0
0 -0.0125548243522644 2.0
0 0.00396126508712769 3.0
0 0.0183145403862 4.0
0 0.0117776989936829 5.0
0 0.00193524360656738 6.0
0 -0.00209051370620728 7.0
36.669966877 38.5309829711914 0.0
0 0.00463026762008667 1.0
0 0.0226883888244629 2.0
0 0.00629353523254395 3.0
0 0.0114103555679321 4.0
0 0.00907456874847412 5.0
0 0.00171947479248047 6.0
0 -0.00196748971939087 7.0
40.401972767 38.2327117919922 0.0
0 0.0044560432434082 1.0
0 0.00335395336151123 2.0
0 0.00200176239013672 3.0
0 0.00582796335220337 4.0
0 0.00227099657058716 5.0
0 0.00142323970794678 6.0
0 -0.00324386358261108 7.0
41.491221092 38.6873626708984 0.0
0 0.00546443462371826 1.0
0 -0.00274443626403809 2.0
0 -0.00156420469284058 3.0
0 0.0120398998260498 4.0
0 0.0104166865348816 5.0
0 0.00150889158248901 6.0
0 0.000754177570343018 7.0
32.092723767 38.4019317626953 0.0
0 0.00888198614120483 1.0
0 0.0200417041778564 2.0
0 -0.00451493263244629 3.0
0 0.00675791501998901 4.0
0 0.000231027603149414 5.0
0 0.00174546241760254 6.0
0 -0.00465226173400879 7.0
39.133472273 38.943042755127 0.0
0 0.0176572203636169 1.0
0 -0.010567843914032 2.0
0 0.0199633836746216 3.0
0 -0.0152668356895447 4.0
0 -0.0183753371238708 5.0
0 0.00282925367355347 6.0
0 -0.000978946685791016 7.0
37.053979998 38.9729843139648 0.0
0 -0.000790834426879883 1.0
0 0.026669442653656 2.0
0 0.0143687725067139 3.0
0 0.0024116039276123 4.0
0 -0.00121736526489258 5.0
0 0.00256633758544922 6.0
0 -0.00202929973602295 7.0
39.864494306 38.6545906066895 0.0
0 0.0108628273010254 1.0
0 0.0133299231529236 2.0
0 0.00485479831695557 3.0
0 0.00242054462432861 4.0
0 0.00336581468582153 5.0
0 0.00100594758987427 6.0
0 -0.00150913000106812 7.0
39.054921308 38.6194686889648 0.0
0 -0.00948309898376465 1.0
0 0.0172194242477417 2.0
0 0.00661468505859375 3.0
0 0.00922781229019165 4.0
0 0.010545015335083 5.0
0 0.00140833854675293 6.0
0 0.00142842531204224 7.0
33.466015237 38.9646224975586 0.0
0 9.03606414794922e-05 1.0
0 0.0123940706253052 2.0
0 0.00274002552032471 3.0
0 0.00122815370559692 4.0
0 -0.000679194927215576 5.0
0 0.0030023455619812 6.0
0 0.00157451629638672 7.0
37.653527764 38.5430641174316 0.0
0 -0.0041007399559021 1.0
0 0.0149326324462891 2.0
0 0.00574177503585815 3.0
0 0.00125390291213989 4.0
0 0.00139206647872925 5.0
0 0.000427007675170898 6.0
0 -0.00363874435424805 7.0
37.577039283 38.1301307678223 0.0
0 -0.00789469480514526 1.0
0 0.0148867964744568 2.0
0 0.00783491134643555 3.0
0 0.0112177729606628 4.0
0 0.00835388898849487 5.0
0 0.0013580322265625 6.0
0 -0.00169438123703003 7.0
36.318307627 38.468090057373 0.0
0 0.00593608617782593 1.0
0 0.00778979063034058 2.0
0 0.00651895999908447 3.0
0 0.00518614053726196 4.0
0 0.00192540884017944 5.0
0 0.00126242637634277 6.0
0 -0.00347381830215454 7.0
36.801464194 38.3615798950195 0.0
0 0.00963151454925537 1.0
0 -0.00350320339202881 2.0
0 0.00691354274749756 3.0
0 0.0129764676094055 4.0
0 0.01592618227005 5.0
0 0.00100570917129517 6.0
0 0.000829577445983887 7.0
39.56095567 38.5987434387207 0.0
0 -0.000210285186767578 1.0
0 -0.0386602878570557 2.0
0 0.0109865069389343 3.0
0 -0.0223806500434875 4.0
0 -0.0233038067817688 5.0
0 0.0035366415977478 6.0
0 -0.00128954648971558 7.0
40.443539209 38.8790626525879 0.0
0 0.0022013783454895 1.0
0 0.0163164138793945 2.0
0 0.0207526087760925 3.0
0 0.0038456916809082 4.0
0 0.00528603792190552 5.0
0 0.00127983093261719 6.0
0 -0.00158476829528809 7.0
40.316920108 38.4642219543457 0.0
0 0.00427037477493286 1.0
0 -0.012304425239563 2.0
0 0.0196879506111145 3.0
0 0.0127553343772888 4.0
0 0.00777769088745117 5.0
0 0.00142759084701538 6.0
0 -0.00219595432281494 7.0
33.579405458 39.1697883605957 0.0
0 0.00949239730834961 1.0
0 0.0550859570503235 2.0
0 0.00363188982009888 3.0
0 -0.00875645875930786 4.0
0 -0.00982797145843506 5.0
0 0.00408774614334106 6.0
0 0.00239044427871704 7.0
34.562810956 38.7289924621582 0.0
0 -0.00927829742431641 1.0
0 -0.031139075756073 2.0
0 0.0254233479499817 3.0
0 -0.00332009792327881 4.0
0 -0.00410580635070801 5.0
0 0.00320959091186523 6.0
0 0.00121003389358521 7.0
37.518824119 38.526123046875 0.0
0 -0.00417220592498779 1.0
0 0.0312619805335999 2.0
0 -0.0036507248878479 3.0
0 0.000343263149261475 4.0
0 -0.00207829475402832 5.0
0 0.00180166959762573 6.0
0 -0.00563240051269531 7.0
39.547705636 38.8937301635742 0.0
0 0.00837802886962891 1.0
0 0.0164691209793091 2.0
0 0.0180018544197083 3.0
0 0.00425541400909424 4.0
0 0.00221371650695801 5.0
0 0.00231784582138062 6.0
0 -0.00168567895889282 7.0
42.834945783 38.4985198974609 0.0
0 0.00751316547393799 1.0
0 -0.00238978862762451 2.0
0 0.0189425945281982 3.0
0 0.00994026660919189 4.0
0 0.0116481781005859 5.0
0 -9.0181827545166e-05 6.0
0 -0.00097280740737915 7.0
38.473622953 38.5314598083496 0.0
0 -0.00472825765609741 1.0
0 0.00927400588989258 2.0
0 0.0143502950668335 3.0
0 0.00991624593734741 4.0
0 0.0122590065002441 5.0
0 0.00189471244812012 6.0
0 0.0014805793762207 7.0
45.238523457 38.3235244750977 0.0
0 0.00389409065246582 1.0
0 0.0224522352218628 2.0
0 0.013031542301178 3.0
0 0.00490808486938477 4.0
0 0.00208324193954468 5.0
0 0.00188267230987549 6.0
0 -0.003562331199646 7.0
36.266282562 38.6187591552734 0.0
0 0.00554311275482178 1.0
0 -0.00330221652984619 2.0
0 -3.94582748413086e-05 3.0
0 0.0115361213684082 4.0
0 0.00921422243118286 5.0
0 0.00178325176239014 6.0
0 0.00104552507400513 7.0
37.845233948 38.7518157958984 0.0
0 -0.000949203968048096 1.0
0 0.00222766399383545 2.0
0 0.0181483030319214 3.0
0 0.00984752178192139 4.0
0 0.00675064325332642 5.0
0 0.00161761045455933 6.0
0 -0.00165462493896484 7.0
40.414778929 38.6832542419434 0.0
0 0.0396010279655457 1.0
0 0.0414213538169861 2.0
0 0.034054696559906 3.0
0 -0.0138557553291321 4.0
0 -0.0228312611579895 5.0
0 0.00423502922058105 6.0
0 -0.00209474563598633 7.0
47.104382604 38.6785850524902 0.0
0 -0.000528335571289062 1.0
0 -0.0117012858390808 2.0
0 0.00560390949249268 3.0
0 0.0113921761512756 4.0
0 0.0118933320045471 5.0
0 0.00046849250793457 6.0
0 -0.00160872936248779 7.0
37.557508388 38.6452941894531 0.0
0 0.0107706189155579 1.0
0 0.0139909386634827 2.0
0 0.0265346765518188 3.0
0 -0.00883990526199341 4.0
0 -0.00819224119186401 5.0
0 0.00279784202575684 6.0
0 0.000886917114257812 7.0
40.532317042 38.8532943725586 0.0
0 -0.00338459014892578 1.0
0 0.00208669900894165 2.0
0 -0.00943958759307861 3.0
0 0.0108900666236877 4.0
0 0.0107890367507935 5.0
0 0.00126546621322632 6.0
0 0.00102007389068604 7.0
32.594348946 38.9392776489258 0.0
0 -0.00994551181793213 1.0
0 -0.00474840402603149 2.0
0 0.0156255960464478 3.0
0 0.00870126485824585 4.0
0 -0.000639498233795166 5.0
0 0.00391221046447754 6.0
0 0.00118041038513184 7.0
30.392342502 38.4610557556152 0.0
0 0.00475746393203735 1.0
0 0.00234395265579224 2.0
0 0.00362575054168701 3.0
0 0.0170789361000061 4.0
0 0.0154737830162048 5.0
0 0.00255101919174194 6.0
0 0.00137734413146973 7.0
40.607017947 38.6572570800781 0.0
0 -0.00629860162734985 1.0
0 0.0028877854347229 2.0
0 0.00213658809661865 3.0
0 0.0124313235282898 4.0
0 0.00775974988937378 5.0
0 0.00156289339065552 6.0
0 -0.00161468982696533 7.0
39.11337914 39.3402252197266 0.0
0 0.053605318069458 1.0
0 0.106695532798767 2.0
0 0.0744338035583496 3.0
0 -0.0374942421913147 4.0
0 -0.0441901683807373 5.0
0 0.00579911470413208 6.0
0 -0.00190454721450806 7.0
35.280627531 38.4612503051758 0.0
0 -0.00280666351318359 1.0
0 0.0293827652931213 2.0
0 -0.0139994621276855 3.0
0 -0.0032992959022522 4.0
0 -0.0042424201965332 5.0
0 -0.00224858522415161 6.0
0 -0.0053715705871582 7.0
32.847447182 38.3127861022949 0.0
0 0.00380522012710571 1.0
0 0.00457185506820679 2.0
0 -0.0068669319152832 3.0
0 0.019364058971405 4.0
0 0.0192620754241943 5.0
0 0.00249946117401123 6.0
0 0.00178861618041992 7.0
34.14972673 38.8230133056641 0.0
0 -0.0161435604095459 1.0
0 0.0640211701393127 2.0
0 0.0416672229766846 3.0
0 -0.0326667428016663 4.0
0 -0.0316633582115173 5.0
0 0.00657165050506592 6.0
0 0.0048445463180542 7.0
36.823986173 38.8261451721191 0.0
0 0.000983893871307373 1.0
0 0.0175092816352844 2.0
0 0.0232627391815186 3.0
0 0.00660783052444458 4.0
0 0.00244992971420288 5.0
0 0.00188249349594116 6.0
0 -0.00359439849853516 7.0
43.615049092 38.7269248962402 0.0
0 -0.00817251205444336 1.0
0 -0.016715943813324 2.0
0 0.021268367767334 3.0
0 -0.0094103217124939 4.0
0 -0.00766164064407349 5.0
0 0.00359576940536499 6.0
0 0.00358837842941284 7.0
39.316465342 38.8074150085449 0.0
0 0.00467908382415771 1.0
0 -0.141974627971649 2.0
0 0.00819063186645508 3.0
0 -0.0243714451789856 4.0
0 -0.0265893340110779 5.0
0 0.00668883323669434 6.0
0 0.00609123706817627 7.0
36.676849778 39.086483001709 0.0
0 0.0338906645774841 1.0
0 0.010081946849823 2.0
0 0.0517814159393311 3.0
0 -0.0148599147796631 4.0
0 -0.015914261341095 5.0
0 0.00404250621795654 6.0
0 -0.000586926937103271 7.0
36.163160136 39.0178642272949 0.0
0 0.0213291645050049 1.0
0 0.000769555568695068 2.0
0 0.0280357003211975 3.0
0 -0.0171972513198853 4.0
0 -0.0217657089233398 5.0
0 0.00370281934738159 6.0
0 -0.00175964832305908 7.0
40.305175496 38.8357582092285 0.0
0 0.00539052486419678 1.0
0 0.00232678651809692 2.0
0 -0.00763154029846191 3.0
0 0.00963932275772095 4.0
0 0.00745934247970581 5.0
0 0.00154834985733032 6.0
0 0.000475943088531494 7.0
34.293236222 38.5812950134277 0.0
0 0.0108577609062195 1.0
0 -0.00930273532867432 2.0
0 0.00788706541061401 3.0
0 0.0121802091598511 4.0
0 0.0107622146606445 5.0
0 0.00162261724472046 6.0
0 0.000948786735534668 7.0
38.347299924 38.5173797607422 0.0
0 0.00819861888885498 1.0
0 0.00382977724075317 2.0
0 0.0101303458213806 3.0
0 0.0126463174819946 4.0
0 0.0123133063316345 5.0
0 0.00140637159347534 6.0
0 0.000719904899597168 7.0
39.036910781 38.8415069580078 0.0
0 0.000177502632141113 1.0
0 0.00544929504394531 2.0
0 -0.0120937824249268 3.0
0 0.00871700048446655 4.0
0 0.00971460342407227 5.0
0 0.0013730525970459 6.0
0 0.00144284963607788 7.0
36.237500285 38.6325035095215 0.0
0 0.00555747747421265 1.0
0 -0.00464928150177002 2.0
0 0.00741976499557495 3.0
0 0.0146653056144714 4.0
0 0.0154220461845398 5.0
0 0.00221884250640869 6.0
0 0.00104159116744995 7.0
36.900540753 38.554141998291 0.0
0 -0.00575870275497437 1.0
0 0.0343371629714966 2.0
0 0.00224876403808594 3.0
0 0.013765811920166 4.0
0 0.00792694091796875 5.0
0 0.0021330714225769 6.0
0 -0.00349611043930054 7.0
29.897899633 38.8244743347168 0.0
0 -0.00203788280487061 1.0
0 0.0106021165847778 2.0
0 0.0118087530136108 3.0
0 0.00280171632766724 4.0
0 -0.000840127468109131 5.0
0 0.00222545862197876 6.0
0 -0.00134199857711792 7.0
41.071373967 38.722110748291 0.0
0 -0.000163435935974121 1.0
0 0.00294435024261475 2.0
0 -0.00318533182144165 3.0
0 0.0174000263214111 4.0
0 0.00768005847930908 5.0
0 0.00268244743347168 6.0
0 -0.000645101070404053 7.0
37.123617582 38.6545639038086 0.0
0 0.0106359124183655 1.0
0 -0.00192886590957642 2.0
0 0.0364857912063599 3.0
0 -0.0049210786819458 4.0
0 -0.00993657112121582 5.0
0 0.00295078754425049 6.0
0 -0.00167167186737061 7.0
38.348981744 39.361213684082 0.0
0 0.108397841453552 1.0
0 0.0719989538192749 2.0
0 0.0854716300964355 3.0
0 -0.0621429681777954 4.0
0 -0.0700206160545349 5.0
0 0.00875371694564819 6.0
0 0.00082772970199585 7.0
37.897829023 38.3081512451172 0.0
0 0.0116560459136963 1.0
0 -0.0115576982498169 2.0
0 0.0105234384536743 3.0
0 0.0114278793334961 4.0
0 0.00531607866287231 5.0
0 0.000552892684936523 6.0
0 -0.0051882266998291 7.0
32.839543182 38.3974990844727 0.0
0 0.00181096792221069 1.0
0 0.00699031352996826 2.0
0 -0.00629299879074097 3.0
0 0.0180360674858093 4.0
0 0.0172174572944641 5.0
0 0.00267308950424194 6.0
0 0.00208872556686401 7.0
45.814256844 38.7030944824219 0.0
0 0.00507837533950806 1.0
0 0.00886917114257812 2.0
0 -0.0037347674369812 3.0
0 0.0136943459510803 4.0
0 0.0137937068939209 5.0
0 0.00230222940444946 6.0
0 0.00160950422286987 7.0
38.439958634 38.4825057983398 0.0
0 -0.00018012523651123 1.0
0 0.0141959190368652 2.0
0 -0.0422380566596985 3.0
0 -0.0285893678665161 4.0
0 -0.0323619246482849 5.0
0 0.00240993499755859 6.0
0 -0.00253540277481079 7.0
33.022424116 38.5725517272949 0.0
0 -0.00100618600845337 1.0
0 0.0236772298812866 2.0
0 0.0142524242401123 3.0
0 0.00742864608764648 4.0
0 0.00858741998672485 5.0
0 0.000316262245178223 6.0
0 -0.00097280740737915 7.0
36.232706353 38.6497421264648 0.0
0 0.00666439533233643 1.0
0 0.0102894306182861 2.0
0 -0.00639992952346802 3.0
0 0.0107094049453735 4.0
0 0.00857967138290405 5.0
0 0.00131088495254517 6.0
0 0.000430107116699219 7.0
45.8365445 38.2908325195312 0.0
0 0.00307029485702515 1.0
0 -0.00923150777816772 2.0
0 0.00529932975769043 3.0
0 0.00588589906692505 4.0
0 0.0053897500038147 5.0
0 0.000756263732910156 6.0
0 -0.00225520133972168 7.0
39.344253832 38.8997688293457 0.0
0 0.00560510158538818 1.0
0 0.012593150138855 2.0
0 0.020938515663147 3.0
0 0.00587159395217896 4.0
0 0.00207465887069702 5.0
0 0.00155848264694214 6.0
0 -0.00285935401916504 7.0
43.613379879 38.2364540100098 0.0
0 0.000397384166717529 1.0
0 0.031568169593811 2.0
0 -0.00499022006988525 3.0
0 -0.00342869758605957 4.0
0 -0.00124233961105347 5.0
0 0.00189417600631714 6.0
0 -0.00510674715042114 7.0
41.603825526 38.189998626709 0.0
0 0.00724172592163086 1.0
0 -0.0164687633514404 2.0
0 0.00195282697677612 3.0
0 0.0148271322250366 4.0
0 0.0068591833114624 5.0
0 0.000564455986022949 6.0
0 -0.00725555419921875 7.0
41.290499737 38.4870338439941 0.0
0 0.0102403163909912 1.0
0 -0.00607091188430786 2.0
0 0.0221967697143555 3.0
0 0.0122079849243164 4.0
0 0.00884580612182617 5.0
0 0.000471353530883789 6.0
0 -0.0037955641746521 7.0
38.21592954 38.5617942810059 0.0
0 -0.000384032726287842 1.0
0 0.00387489795684814 2.0
0 -0.00276827812194824 3.0
0 0.011198878288269 4.0
0 0.013520359992981 5.0
0 0.00102299451828003 6.0
0 0.00107133388519287 7.0
34.764993855 38.4070091247559 0.0
0 -0.00358575582504272 1.0
0 0.0124229192733765 2.0
0 -0.00962519645690918 3.0
0 -0.00281190872192383 4.0
0 -0.00192475318908691 5.0
0 0.000529289245605469 6.0
0 -0.00529855489730835 7.0
34.789226058 38.5943145751953 0.0
0 -0.000479459762573242 1.0
0 0.0239938497543335 2.0
0 0.0198885202407837 3.0
0 0.00419110059738159 4.0
0 0.00250375270843506 5.0
0 0.0020715594291687 6.0
0 -0.00529903173446655 7.0
42.488773183 38.6152381896973 0.0
0 0.00836420059204102 1.0
0 -0.00419712066650391 2.0
0 0.00586044788360596 3.0
0 0.0160481929779053 4.0
0 0.0162471532821655 5.0
0 0.00226736068725586 6.0
0 0.000960052013397217 7.0
34.702233392 39.2018165588379 0.0
0 0.00722146034240723 1.0
0 0.0153907537460327 2.0
0 0.0345809459686279 3.0
0 -0.000850021839141846 4.0
0 -0.00449872016906738 5.0
0 0.00279438495635986 6.0
0 -0.00392633676528931 7.0
41.831412446 39.3694381713867 0.0
0 0.0202068090438843 1.0
0 0.00735104084014893 2.0
0 -0.01542729139328 3.0
0 0.00598728656768799 4.0
0 0.000746250152587891 5.0
0 0.00259238481521606 6.0
0 -0.00198984146118164 7.0
45.908850318 38.7893257141113 0.0
0 0.0128718614578247 1.0
0 0.0439763069152832 2.0
0 -0.053998589515686 3.0
0 -0.0368396043777466 4.0
0 -0.0414500832557678 5.0
0 0.00407934188842773 6.0
0 -0.00154948234558105 7.0
33.104401139 39.2357902526855 0.0
0 0.0355506539344788 1.0
0 0.0204304456710815 2.0
0 0.0460935235023499 3.0
0 -0.00619632005691528 4.0
0 -0.0127991437911987 5.0
0 0.00329005718231201 6.0
0 -0.00491511821746826 7.0
35.835170704 38.656005859375 0.0
0 -0.0135376453399658 1.0
0 0.0272335410118103 2.0
0 0.018695592880249 3.0
0 -0.0218937397003174 4.0
0 -0.0219389796257019 5.0
0 0.00600516796112061 6.0
0 0.00515711307525635 7.0
45.844507362 38.3923797607422 0.0
0 0.00346899032592773 1.0
0 -0.0049707293510437 2.0
0 -0.0142909288406372 3.0
0 0.0028308629989624 4.0
0 0.00224250555038452 5.0
0 -0.00219684839248657 6.0
0 -0.00600039958953857 7.0
31.875262674 38.9218482971191 0.0
0 0.00548732280731201 1.0
0 0.03155916929245 2.0
0 0.0234208106994629 3.0
0 0.00218135118484497 4.0
0 0.000923514366149902 5.0
0 0.000942528247833252 6.0
0 -0.00415825843811035 7.0
35.888104099 38.465877532959 0.0
0 -0.00457191467285156 1.0
0 0.0129864811897278 2.0
0 0.0154435038566589 3.0
0 0.00746810436248779 4.0
0 0.00840973854064941 5.0
0 0.000717759132385254 6.0
0 -0.000674307346343994 7.0
35.733201066 38.5028800964355 0.0
0 0.00586575269699097 1.0
0 0.0072481632232666 2.0
0 0.0173937678337097 3.0
0 0.0100708603858948 4.0
0 0.00668007135391235 5.0
0 0.00180494785308838 6.0
0 -0.00377106666564941 7.0
43.710725429 38.3162002563477 0.0
0 -0.00887012481689453 1.0
0 -0.00385814905166626 2.0
0 -0.00318706035614014 3.0
0 -0.0185797214508057 4.0
0 -0.0196681022644043 5.0
0 0.000224053859710693 6.0
0 -0.00327861309051514 7.0
34.934843039 38.8348121643066 0.0
0 -0.00161290168762207 1.0
0 0.0159087181091309 2.0
0 0.0234295129776001 3.0
0 0.0048181414604187 4.0
0 0.00403618812561035 5.0
0 0.00204634666442871 6.0
0 -0.00230425596237183 7.0
38.70218214 38.7474479675293 0.0
0 -0.00108134746551514 1.0
0 0.0091174840927124 2.0
0 -0.00267726182937622 3.0
0 0.00788772106170654 4.0
0 0.00635135173797607 5.0
0 0.00159680843353271 6.0
0 0.0013083815574646 7.0
39.146786058 39.0480194091797 0.0
0 -0.0114285945892334 1.0
0 0.00788998603820801 2.0
0 -0.0154602527618408 3.0
0 0.0109805464744568 4.0
0 0.00477433204650879 5.0
0 0.00266498327255249 6.0
0 0.000880956649780273 7.0
37.956003494 38.3549499511719 0.0
0 0.00789499282836914 1.0
0 0.00629401206970215 2.0
0 -0.0352148413658142 3.0
0 -0.0112420320510864 4.0
0 -0.0142526030540466 5.0
0 0.00200521945953369 6.0
0 -0.00365465879440308 7.0
35.896232558 39.0284271240234 0.0
0 0.00879734754562378 1.0
0 0.0454597473144531 2.0
0 0.00259673595428467 3.0
0 -0.0103989839553833 4.0
0 -0.0144324898719788 5.0
0 0.00407028198242188 6.0
0 -0.000433206558227539 7.0
37.377916966 39.028507232666 0.0
0 0.0277334451675415 1.0
0 0.103057205677032 2.0
0 0.0302638411521912 3.0
0 -0.0119578838348389 4.0
0 -0.0142959356307983 5.0
0 0.00127959251403809 6.0
0 -0.00192588567733765 7.0
36.698956497 38.7372550964355 0.0
0 -0.0109740495681763 1.0
0 0.0191500186920166 2.0
0 -0.0165144205093384 3.0
0 0.0105429291725159 4.0
0 0.00730335712432861 5.0
0 0.00290632247924805 6.0
0 0.00153911113739014 7.0
41.903925455 38.9670372009277 0.0
0 0.0159850120544434 1.0
0 0.0163652896881104 2.0
0 0.0204038023948669 3.0
0 0.00393366813659668 4.0
0 -0.00349020957946777 5.0
0 0.00331735610961914 6.0
0 -0.00523614883422852 7.0
38.934024851 38.2444496154785 0.0
0 0.00585168600082397 1.0
0 0.0038304328918457 2.0
0 -0.00481849908828735 3.0
0 0.0178231000900269 4.0
0 0.0161932110786438 5.0
0 0.00253915786743164 6.0
0 0.00152045488357544 7.0
40.406325889 38.474781036377 0.0
0 -0.00402039289474487 1.0
0 0.0110981464385986 2.0
0 -0.0641270279884338 3.0
0 -0.035135805606842 4.0
0 -0.0373694896697998 5.0
0 0.00301212072372437 6.0
0 -0.00229912996292114 7.0
39.410747699 38.5370864868164 0.0
0 -0.00339770317077637 1.0
0 0.020703911781311 2.0
0 0.00961828231811523 3.0
0 0.00251716375350952 4.0
0 0.00325173139572144 5.0
0 0.00136208534240723 6.0
0 -0.0024869441986084 7.0
};
\addlegendentry{$R^2$=0.988}
\end{axis}
\end{tikzpicture}

}
    % \begin{tikzpicture}[shorten >=1pt, ->, draw=black!50, node distance=1.5cm and 3.5cm, align=center]

    % Styles
    \tikzstyle{input} = [circle, draw, fill=green!50, minimum size=2em]
    \tikzstyle{hidden} = [circle, draw, fill=blue!50, minimum size=2em]
    \tikzstyle{output} = [circle, draw, fill=red!50, minimum size=2em]
    \tikzstyle{connection} = [->, thick]

    % Network Stage Labels
    \node[align=center] at (0,-0.4) {Input \\ Layer};
    \node[align=center] at (6,0.4) {Hidden \\ Layers};
    \node[align=center] at (12,-1.2) {Output \\ Layer};

    % Input Layer
    \foreach \i in {1,2,3}
        \node[input] (I\i) at (0,-\i*1.5) {$x_\i$};

    % Hidden Layer 1
    \foreach \i in {1,2,3,4}
        \node[hidden] (H1\i) at (3,-\i*1.5+0.75) {$z^{(1)}_\i$};

    % Hidden Layer 2
    \foreach \i in {1,2,3,4}
        \node[hidden] (H2\i) at (6,-\i*1.5+0.75) {$z^{(2)}_\i$};

    % Hidden Layer 3
    \foreach \i in {1,2,3,4}
        \node[hidden] (H3\i) at (9,-\i*1.5+0.75) {$z^{(3)}_\i$};

    % Output Layer
    \foreach \i in {1,2}
        \node[output] (O\i) at (12,-\i*1.5-0.75) {$\hat{y}_\i$};

    % Connections from Input to Hidden Layer 1
    \foreach \i in {1,2,3}
        \foreach \j in {1,2,3,4}
            \draw[connection] (I\i) -- (H1\j);

    % Connections from Hidden Layer 1 to Hidden Layer 2
    \foreach \i in {1,2,3,4}
        \foreach \j in {1,2,3,4}
            \draw[connection] (H1\i) -- (H2\j);

    % Connections from Hidden Layer 2 to Hidden Layer 3
    \foreach \i in {1,2,3,4}
        \foreach \j in {1,2,3,4}
            \draw[connection] (H2\i) -- (H3\j);

    % Connections from Hidden Layer 3 to Output Layer
    \foreach \i in {1,2,3,4}
        \foreach \j in {1,2}
            \draw[connection] (H3\i) -- (O\j);

\end{tikzpicture}

    \caption{}\label{fig:results_dummy_base}
\end{figure}

\begin{figure}
    \centering
    \setlength\figurewidth{1\textwidth}        
    \setlength\figureheight{0.5\textwidth}
    \resizebox{\figurewidth}{\figureheight}{% This file was created with tikzplotlib v0.10.1.
\begin{tikzpicture}

\definecolor{darkgray176}{RGB}{176,176,176}
\definecolor{lightgray204}{RGB}{204,204,204}

\begin{axis}[
colorbar,
colorbar style={ylabel={edge id}},
colormap={mymap}{[1pt]
 rgb(0pt)=(0.12156862745098,0.466666666666667,0.705882352941177);
  rgb(1pt)=(1,0.498039215686275,0.0549019607843137);
  rgb(2pt)=(0.172549019607843,0.627450980392157,0.172549019607843);
  rgb(3pt)=(0.83921568627451,0.152941176470588,0.156862745098039);
  rgb(4pt)=(0.580392156862745,0.403921568627451,0.741176470588235);
  rgb(5pt)=(0.549019607843137,0.337254901960784,0.294117647058824);
  rgb(6pt)=(0.890196078431372,0.466666666666667,0.76078431372549);
  rgb(7pt)=(0.498039215686275,0.498039215686275,0.498039215686275);
  rgb(8pt)=(0.737254901960784,0.741176470588235,0.133333333333333);
  rgb(9pt)=(0.0901960784313725,0.745098039215686,0.811764705882353)
},
legend cell align={left},
legend style={
  fill opacity=0.8,
  draw opacity=1,
  text opacity=1,
  at={(0.03,0.97)},
  anchor=north west,
  draw=lightgray204
},
point meta max=7,
point meta min=0,
tick align=outside,
tick pos=left,
title={ye true - ye pred},
x grid style={darkgray176},
xlabel={ye true},
xmajorgrids,
xmin=-6.176293119925, xmax=50.358667459425,
xtick style={color=black},
y grid style={darkgray176},
ylabel={ye pred},
ymajorgrids,
ymin=-1.33763232827187, ymax=23.8523469269276,
ytick style={color=black}
]
\addplot [
  colormap={mymap}{[1pt]
 rgb(0pt)=(0.12156862745098,0.466666666666667,0.705882352941177);
  rgb(1pt)=(1,0.498039215686275,0.0549019607843137);
  rgb(2pt)=(0.172549019607843,0.627450980392157,0.172549019607843);
  rgb(3pt)=(0.83921568627451,0.152941176470588,0.156862745098039);
  rgb(4pt)=(0.580392156862745,0.403921568627451,0.741176470588235);
  rgb(5pt)=(0.549019607843137,0.337254901960784,0.294117647058824);
  rgb(6pt)=(0.890196078431372,0.466666666666667,0.76078431372549);
  rgb(7pt)=(0.498039215686275,0.498039215686275,0.498039215686275);
  rgb(8pt)=(0.737254901960784,0.741176470588235,0.133333333333333);
  rgb(9pt)=(0.0901960784313725,0.745098039215686,0.811764705882353)
},
  only marks,
  scatter,
  scatter src=explicit
]
table [x=x, y=y, meta=colordata]{%
x  y  colordata
41.603277614 0.18362183868885 0.0
28.537755408 1.66803848743439 1.0
13.065522207 1.67883598804474 2.0
0 0.169846832752228 3.0
13.065522207 0.412138372659683 4.0
28.537755408 0.453680574893951 5.0
41.603277614 0.780018985271454 6.0
41.603277614 17.8501033782959 7.0
37.07638211 0.798396944999695 0.0
22.707924148 0.537246406078339 1.0
14.368457962 0.584073185920715 2.0
0 0.244699701666832 3.0
14.368457962 0.434437125921249 4.0
22.707924148 0.473161071538925 5.0
37.07638211 0.503022789955139 6.0
37.07638211 0.500305235385895 7.0
39.245039256 0.144021779298782 0.0
23.531643333 1.72768700122833 1.0
15.713395923 1.74414300918579 2.0
0 0.184231892228127 3.0
15.713395923 0.44206565618515 4.0
23.531643333 0.463650107383728 5.0
39.245039256 1.11973834037781 6.0
39.245039256 22.707347869873 7.0
33.40222464 0.154635012149811 0.0
21.391931512 1.16647207736969 1.0
12.010293129 1.18021678924561 2.0
0 0.213556349277496 3.0
12.010293129 0.440131068229675 4.0
21.391931512 0.483441412448883 5.0
33.40222464 0.526355564594269 6.0
33.40222464 13.7839498519897 7.0
40.177942425 0.844456911087036 0.0
28.641810021 0.538357973098755 1.0
11.536132404 0.575416743755341 2.0
0 0.234241291880608 3.0
11.536132404 0.430620700120926 4.0
28.641810021 0.464340895414352 5.0
40.177942425 0.520999729633331 6.0
40.177942425 0.431796610355377 7.0
33.661272263 0.764284193515778 0.0
23.801018924 0.511304080486298 1.0
9.8602533385 0.543007612228394 2.0
0 0.215575829148293 3.0
9.8602533385 0.435438483953476 4.0
23.801018924 0.468720197677612 5.0
33.661272263 0.259720057249069 6.0
33.661272263 0.395147949457169 7.0
41.918555681 0.77016544342041 0.0
27.391600177 0.492178738117218 1.0
14.526955504 0.537823259830475 2.0
0 0.21575616300106 3.0
14.526955504 0.450648248195648 4.0
27.391600177 0.48611044883728 5.0
41.918555681 0.300421893596649 6.0
41.918555681 0.500450730323792 7.0
44.067434787 0.818282723426819 0.0
29.373915823 0.474608778953552 1.0
14.693518964 0.531799495220184 2.0
0 0.226863473653793 3.0
14.693518964 0.452735751867294 4.0
29.373915823 0.491840779781342 5.0
44.067434787 0.558213770389557 6.0
44.067434787 0.527023255825043 7.0
31.762979365 0.800955593585968 0.0
21.846398167 0.460764288902283 1.0
9.9165811975 0.511068344116211 2.0
0 0.220168948173523 3.0
9.9165811975 0.442803055047989 4.0
21.846398167 0.489266276359558 5.0
31.762979365 0.530630528926849 6.0
31.762979365 0.514157295227051 7.0
41.030120159 0.229777932167053 0.0
27.86612118 0.59070760011673 1.0
13.163998979 0.619265496730804 2.0
0 0.147530913352966 3.0
13.163998979 0.434190273284912 4.0
27.86612118 0.449956715106964 5.0
41.030120159 -0.0576869435608387 6.0
41.030120159 -0.094196654856205 7.0
40.055714877 0.605728924274445 0.0
26.116660402 0.49225977063179 1.0
13.939054474 0.524616122245789 2.0
0 0.205983638763428 3.0
13.939054474 0.436747550964355 4.0
26.116660402 0.475488305091858 5.0
40.055714877 0.312271147966385 6.0
40.055714877 0.161313235759735 7.0
40.714244155 0.830942988395691 0.0
29.532761783 0.52602881193161 1.0
11.181482371 0.560135900974274 2.0
0 0.23785687983036 3.0
11.181482371 0.436676442623138 4.0
29.532761783 0.484379798173904 5.0
40.714244155 0.546932458877563 6.0
40.714244155 0.523251533508301 7.0
45.139727698 0.794304311275482 0.0
31.047422979 0.495071738958359 1.0
14.092304718 0.527405142784119 2.0
0 0.224304437637329 3.0
14.092304718 0.455165594816208 4.0
31.047422979 0.492412984371185 5.0
45.139727698 0.279202252626419 6.0
45.139727698 0.409322202205658 7.0
40.583613059 0.175717517733574 0.0
28.007694947 0.327290862798691 1.0
12.575918112 0.347913056612015 2.0
0 0.204129546880722 3.0
12.575918112 0.411338806152344 4.0
28.007694947 0.453929126262665 5.0
40.583613059 0.333184242248535 6.0
40.583613059 0.589721024036407 7.0
35.441291819 0.683855414390564 0.0
26.125304789 0.51181834936142 1.0
9.3159870299 0.53029328584671 2.0
0 0.197433292865753 3.0
9.3159870299 0.402535080909729 4.0
26.125304789 0.447339415550232 5.0
35.441291819 0.247122347354889 6.0
35.441291819 0.311066895723343 7.0
36.08841077 0.306239098310471 0.0
22.250971448 0.4757981300354 1.0
13.837439322 0.507725119590759 2.0
0 0.21941702067852 3.0
13.837439322 0.441788822412491 4.0
22.250971448 0.50071507692337 5.0
36.08841077 0.0430356860160828 6.0
36.08841077 0.0698571056127548 7.0
40.494343306 0.791692972183228 0.0
29.628817779 0.487594813108444 1.0
10.865525528 0.532764315605164 2.0
0 0.229744613170624 3.0
10.865525528 0.465688437223434 4.0
29.628817779 0.505403399467468 5.0
40.494343306 0.337573498487473 6.0
40.494343306 0.533054888248444 7.0
44.400196583 0.805976390838623 0.0
30.425150814 0.395661115646362 1.0
13.975045769 0.430603176355362 2.0
-1.1363941574 0.214308008551598 3.0
12.838651611 0.450592577457428 4.0
31.561544972 0.48526269197464 5.0
44.400196583 0.561578810214996 6.0
44.400196583 0.482012271881104 7.0
40.197028842 0.716627299785614 0.0
28.568150674 0.486696600914001 1.0
11.628878168 0.520955622196198 2.0
0 0.21579971909523 3.0
11.628878168 0.431205600500107 4.0
28.568150674 0.48511803150177 5.0
40.197028842 0.216360852122307 6.0
40.197028842 0.455694228410721 7.0
39.931867517 0.820916175842285 0.0
24.727856907 0.496078073978424 1.0
15.20401061 0.538481950759888 2.0
0 0.231600284576416 3.0
15.20401061 0.459666788578033 4.0
24.727856907 0.500480115413666 5.0
39.931867517 0.292605042457581 6.0
39.931867517 0.477511912584305 7.0
37.28322745 0.797858595848083 0.0
27.425754281 0.483238518238068 1.0
9.8574731699 0.543225765228271 2.0
0 0.230617880821228 3.0
9.8574731699 0.464555144309998 4.0
27.425754281 0.497691094875336 5.0
37.28322745 0.524982571601868 6.0
37.28322745 0.459729492664337 7.0
37.429390014 0.400299191474915 0.0
22.132902814 0.515194952487946 1.0
15.2964872 0.550267934799194 2.0
0 0.201122522354126 3.0
15.2964872 0.43369522690773 4.0
22.132902814 0.474303901195526 5.0
37.429390014 0.018711406737566 6.0
37.429390014 0.0179443359375 7.0
39.499146179 0.812548875808716 0.0
30.151299195 0.509443402290344 1.0
9.3478469839 0.549915850162506 2.0
0 0.240803509950638 3.0
9.3478469839 0.452278822660446 4.0
30.151299195 0.490095019340515 5.0
39.499146179 0.392056614160538 6.0
39.499146179 0.505513727664948 7.0
35.661352504 0.603023767471313 0.0
26.617773772 0.125488445162773 1.0
9.0435787317 0.139381006360054 2.0
0 0.240266904234886 3.0
9.0435787317 0.396227955818176 4.0
26.617773772 0.465910911560059 5.0
35.661352504 4.54981899261475 6.0
35.661352504 0.126817345619202 7.0
44.903668391 0.810700595378876 0.0
28.406134318 0.384008765220642 1.0
16.497534073 0.41478756070137 2.0
-2.2866598043 0.21231085062027 3.0
14.210874268 0.445897072553635 4.0
30.692794122 0.496664673089981 5.0
44.903668391 0.515478253364563 6.0
44.903668391 0.495566487312317 7.0
34.118473664 0.267246276140213 0.0
24.847237913 0.362864077091217 1.0
9.271235751 0.400662004947662 2.0
0 0.287841975688934 3.0
9.271235751 0.467766225337982 4.0
24.847237913 0.537852227687836 5.0
34.118473664 0.0151599179953337 6.0
34.118473664 0.234499678015709 7.0
36.466593361 0.0886330008506775 0.0
24.924999426 1.26627230644226 1.0
11.541593934 1.2690771818161 2.0
0 0.180928602814674 3.0
11.541593934 0.431752681732178 4.0
24.924999426 0.466692566871643 5.0
36.466593361 0.895050227642059 6.0
36.466593361 13.2778911590576 7.0
34.994247693 0.732260346412659 0.0
26.099734292 0.488581299781799 1.0
8.8945134012 0.544753968715668 2.0
0 0.228192523121834 3.0
8.8945134012 0.450730562210083 4.0
26.099734292 0.516707539558411 5.0
34.994247693 0.48125809431076 6.0
34.994247693 0.607306361198425 7.0
39.553868661 0.807461321353912 0.0
27.422096073 0.466728627681732 1.0
12.131772588 0.52154940366745 2.0
0 0.216750770807266 3.0
12.131772588 0.447430670261383 4.0
27.422096073 0.493136942386627 5.0
39.553868661 0.389254927635193 6.0
39.553868661 0.561555624008179 7.0
43.158887007 0.848639488220215 0.0
28.546889082 0.479268044233322 1.0
14.611997925 0.538083851337433 2.0
0 0.231367483735085 3.0
14.611997925 0.454253047704697 4.0
28.546889082 0.480862885713577 5.0
43.158887007 0.545655906200409 6.0
43.158887007 0.518004715442657 7.0
36.693891456 0.553474247455597 0.0
24.873430755 0.481561362743378 1.0
11.820460701 0.51918613910675 2.0
0 0.183027699589729 3.0
11.820460701 0.427139818668365 4.0
24.873430755 0.459326833486557 5.0
36.693891456 0.213233962655067 6.0
36.693891456 0.152907565236092 7.0
37.419072323 0.809681236743927 0.0
26.218281513 0.483605414628983 1.0
11.20079081 0.542487561702728 2.0
0 0.224215656518936 3.0
11.20079081 0.440650761127472 4.0
26.218281513 0.484191060066223 5.0
37.419072323 0.518257737159729 6.0
37.419072323 0.508003413677216 7.0
42.391154887 0.257783174514771 0.0
28.17202345 0.767748475074768 1.0
14.219131437 0.80058491230011 2.0
0 0.16021528840065 3.0
14.219131437 0.426328301429749 4.0
28.17202345 0.454959571361542 5.0
42.391154887 0.177222549915314 6.0
42.391154887 -0.171856135129929 7.0
39.976124969 0.155863389372826 0.0
30.003187585 1.05664110183716 1.0
9.9729373839 1.05924808979034 2.0
0 0.146673485636711 3.0
9.9729373839 0.428589582443237 4.0
30.003187585 0.44323518872261 5.0
39.976124969 -0.0813200622797012 6.0
39.976124969 1.89679765701294 7.0
41.193094756 0.8075270652771 0.0
26.04533013 0.47657573223114 1.0
15.147764626 0.531694948673248 2.0
0 0.221025288105011 3.0
15.147764626 0.457021355628967 4.0
26.04533013 0.4905945956707 5.0
41.193094756 0.390842884778976 6.0
41.193094756 0.500106513500214 7.0
39.030467364 0.736577272415161 0.0
28.666995108 0.409848719835281 1.0
10.363472256 0.444745510816574 2.0
0 0.254743278026581 3.0
10.363472256 0.442532122135162 4.0
28.666995108 0.479927569627762 5.0
39.030467364 0.158288449048996 6.0
39.030467364 0.366771429777145 7.0
39.254434013 0.679574370384216 0.0
24.077904618 0.456481516361237 1.0
15.176529395 0.497190952301025 2.0
0 0.216706901788712 3.0
15.176529395 0.461384415626526 4.0
24.077904618 0.502110123634338 5.0
39.254434013 0.305881977081299 6.0
39.254434013 0.565269470214844 7.0
41.474829476 0.725732564926147 0.0
25.835349121 0.468871086835861 1.0
15.639480356 0.519743978977203 2.0
0 0.191539064049721 3.0
15.639480356 0.435394406318665 4.0
25.835349121 0.472873091697693 5.0
41.474829476 0.289753705263138 6.0
41.474829476 0.559043526649475 7.0
36.065024548 0.793076515197754 0.0
27.353858358 0.454936861991882 1.0
8.7111661904 0.501248955726624 2.0
0 0.226842939853668 3.0
8.7111661904 0.451495110988617 4.0
27.353858358 0.501835227012634 5.0
36.065024548 0.528780281543732 6.0
36.065024548 0.53072053194046 7.0
33.547657937 0.185216248035431 0.0
23.678365531 0.427676320075989 1.0
9.8692924059 0.444449096918106 2.0
0 0.166136011481285 3.0
9.8692924059 0.392721325159073 4.0
23.678365531 0.426126092672348 5.0
33.547657937 0.17101676762104 6.0
33.547657937 0.667055249214172 7.0
35.860060933 0.845887005329132 0.0
22.637636259 0.522960662841797 1.0
13.222424674 0.565746366977692 2.0
0 0.260444730520248 3.0
13.222424674 0.444882959127426 4.0
22.637636259 0.496082723140717 5.0
35.860060933 0.389720112085342 6.0
35.860060933 0.427040904760361 7.0
41.446935888 0.34843835234642 0.0
32.532039789 0.122403137385845 1.0
8.9148960991 0.138558104634285 2.0
0 0.226195082068443 3.0
8.9148960991 0.394131362438202 4.0
32.532039789 0.450389623641968 5.0
41.446935888 0.0849817842245102 6.0
41.446935888 0.253898501396179 7.0
39.09863661 0.574728846549988 0.0
24.812113349 0.430166661739349 1.0
14.286523261 0.471245467662811 2.0
0 0.164977610111237 3.0
14.286523261 0.434725403785706 4.0
24.812113349 0.459047257900238 5.0
39.09863661 0.166856959462166 6.0
39.09863661 0.333010584115982 7.0
37.209660367 0.686252474784851 0.0
25.112424491 0.494599163532257 1.0
12.097235877 0.529250800609589 2.0
0 0.205128222703934 3.0
12.097235877 0.429956793785095 4.0
25.112424491 0.454052209854126 5.0
37.209660367 0.307500422000885 6.0
37.209660367 0.314246386289597 7.0
37.318344805 0.804287254810333 0.0
25.306300138 0.485686361789703 1.0
12.012044667 0.541722774505615 2.0
0 0.251310735940933 3.0
12.012044667 0.460202783346176 4.0
25.306300138 0.505881547927856 5.0
37.318344805 0.542162656784058 6.0
37.318344805 0.517050623893738 7.0
37.542238634 0.633315086364746 0.0
26.460617918 0.447912096977234 1.0
11.081620716 0.485553950071335 2.0
0 0.178163453936577 3.0
11.081620716 0.457998186349869 4.0
26.460617918 0.462548673152924 5.0
37.542238634 0.275898844003677 6.0
37.542238634 0.523411750793457 7.0
35.642638424 0.810593724250793 0.0
20.426904297 0.526028275489807 1.0
15.215734127 0.5652214884758 2.0
0 0.247393652796745 3.0
15.215734127 0.451821058988571 4.0
20.426904297 0.476411789655685 5.0
35.642638424 0.524780631065369 6.0
35.642638424 0.530322670936584 7.0
47.4260719 0.679734826087952 0.0
31.140200753 0.560370206832886 1.0
16.285871146 0.594865441322327 2.0
-1.8634699235 0.251972764730453 3.0
14.422401223 0.435448974370956 4.0
33.003670677 0.475022196769714 5.0
47.4260719 0.325695693492889 6.0
47.4260719 0.29996195435524 7.0
42.664227047 0.860981822013855 0.0
29.252186505 0.559815347194672 1.0
13.412040542 0.602366387844086 2.0
0 0.266680747270584 3.0
13.412040542 0.44575434923172 4.0
29.252186505 0.485280245542526 5.0
42.664227047 0.508468151092529 6.0
42.664227047 0.44195294380188 7.0
37.881981315 0.833380520343781 0.0
24.58462335 0.448726058006287 1.0
13.297357965 0.493956297636032 2.0
0 0.223773553967476 3.0
13.297357965 0.458041459321976 4.0
24.58462335 0.496246784925461 5.0
37.881981315 0.561169564723969 6.0
37.881981315 0.511483311653137 7.0
44.031482397 0.162285029888153 0.0
28.706496003 0.44884005188942 1.0
15.324986394 0.474067270755768 2.0
0 0.158654078841209 3.0
15.324986394 0.439268618822098 4.0
28.706496003 0.477900832891464 5.0
44.031482397 0.255633234977722 6.0
44.031482397 0.0328879728913307 7.0
40.215687867 0.828767418861389 0.0
29.785471623 0.530020475387573 1.0
10.430216245 0.557447910308838 2.0
0 0.222647652029991 3.0
10.430216245 0.425323098897934 4.0
29.785471623 0.461571216583252 5.0
40.215687867 0.340684413909912 6.0
40.215687867 0.291825175285339 7.0
35.401137883 0.79507964849472 0.0
25.079356252 0.463181585073471 1.0
10.32178163 0.51850962638855 2.0
0 0.213682353496552 3.0
10.32178163 0.4574314057827 4.0
25.079356252 0.47861659526825 5.0
35.401137883 0.275768011808395 6.0
35.401137883 0.565602242946625 7.0
40.23529515 0.789519309997559 0.0
24.942469333 0.501329481601715 1.0
15.292825817 0.557558357715607 2.0
0 0.200252652168274 3.0
15.292825817 0.427802085876465 4.0
24.942469333 0.449140250682831 5.0
40.23529515 0.530221343040466 6.0
40.23529515 0.508595824241638 7.0
38.451822651 0.855121433734894 0.0
26.67825551 0.503783524036407 1.0
11.773567141 0.543715238571167 2.0
0 0.218413278460503 3.0
11.773567141 0.4441197514534 4.0
26.67825551 0.473533779382706 5.0
38.451822651 0.351095050573349 6.0
38.451822651 0.445495098829269 7.0
41.356859392 0.779768109321594 0.0
27.293812006 0.520446956157684 1.0
14.063047386 0.558641016483307 2.0
0 0.239295333623886 3.0
14.063047386 0.450157880783081 4.0
27.293812006 0.482391387224197 5.0
41.356859392 0.503314435482025 6.0
41.356859392 0.519955694675446 7.0
44.45602451 0.798856616020203 0.0
28.653738823 0.469792872667313 1.0
15.802285687 0.524374723434448 2.0
0 0.219654411077499 3.0
15.802285687 0.456302762031555 4.0
28.653738823 0.49110534787178 5.0
44.45602451 0.36150535941124 6.0
44.45602451 0.473621129989624 7.0
36.344631329 0.802997171878815 0.0
24.004198046 0.504283487796783 1.0
12.340433282 0.52661919593811 2.0
0 0.211866229772568 3.0
12.340433282 0.431315869092941 4.0
24.004198046 0.475714683532715 5.0
36.344631329 0.370603084564209 6.0
36.344631329 0.29797887802124 7.0
35.989587139 0.811484813690186 0.0
25.842707382 0.490815788507462 1.0
10.146879757 0.532808542251587 2.0
0 0.228138387203217 3.0
10.146879757 0.45358419418335 4.0
25.842707382 0.488609611988068 5.0
35.989587139 0.262153208255768 6.0
35.989587139 0.486478298902512 7.0
38.453275767 0.708113312721252 0.0
22.650197879 0.476131916046143 1.0
15.803077888 0.51755279302597 2.0
0 0.207930564880371 3.0
15.803077888 0.441082447767258 4.0
22.650197879 0.480470240116119 5.0
38.453275767 0.277470767498016 6.0
38.453275767 0.501384317874908 7.0
36.388812201 0.785589814186096 0.0
20.539942681 0.433730512857437 1.0
15.84886952 0.474904119968414 2.0
0 0.230642005801201 3.0
15.84886952 0.455040693283081 4.0
20.539942681 0.512764811515808 5.0
36.388812201 0.547524690628052 6.0
36.388812201 0.513620257377625 7.0
45.540732617 0.795330166816711 0.0
31.547017021 0.342924922704697 1.0
13.993715596 0.381518423557281 2.0
0 0.215660035610199 3.0
13.993715596 0.458406984806061 4.0
31.547017021 0.487756162881851 5.0
45.540732617 0.579283654689789 6.0
45.540732617 0.45035058259964 7.0
34.414627117 0.144705131649971 0.0
20.463252495 1.84070909023285 1.0
13.951374622 1.875736951828 2.0
0 0.108744889497757 3.0
13.951374622 0.411313712596893 4.0
20.463252495 0.421262860298157 5.0
34.414627117 -0.192633271217346 6.0
34.414627117 17.2122135162354 7.0
39.317888978 0.72857391834259 0.0
28.187840428 0.488027274608612 1.0
11.13004855 0.521728932857513 2.0
0 0.237912744283676 3.0
11.13004855 0.457086831331253 4.0
28.187840428 0.481893122196198 5.0
39.317888978 0.251074910163879 6.0
39.317888978 0.467821776866913 7.0
36.969505628 0.68063485622406 0.0
22.432585624 0.452622324228287 1.0
14.536920003 0.490714818239212 2.0
0 0.203278660774231 3.0
14.536920003 0.436907947063446 4.0
22.432585624 0.475482374429703 5.0
36.969505628 0.233877420425415 6.0
36.969505628 0.485995829105377 7.0
44.923617033 0.557306230068207 0.0
29.042831601 0.43771356344223 1.0
15.880785432 0.465666651725769 2.0
0 0.222870618104935 3.0
15.880785432 0.434020638465881 4.0
29.042831601 0.492365062236786 5.0
44.923617033 0.232635527849197 6.0
44.923617033 0.259026378393173 7.0
37.258159228 0.768329739570618 0.0
27.142741137 0.365099221467972 1.0
10.115418091 0.387372672557831 2.0
0 0.228445708751678 3.0
10.115418091 0.404116153717041 4.0
27.142741137 0.460410088300705 5.0
37.258159228 -0.000946193933486938 6.0
37.258159228 0.52266651391983 7.0
47.488576672 0.152595415711403 0.0
33.063951806 0.442923456430435 1.0
14.424624867 0.478711307048798 2.0
0 0.210755586624146 3.0
14.424624867 0.437796413898468 4.0
33.063951806 0.470580518245697 5.0
47.488576672 0.30705001950264 6.0
47.488576672 0.732701599597931 7.0
42.958399331 0.859491169452667 0.0
31.220988276 0.484890818595886 1.0
11.737411055 0.540959537029266 2.0
0 0.22070375084877 3.0
11.737411055 0.43827611207962 4.0
31.220988276 0.48798406124115 5.0
42.958399331 0.43698126077652 6.0
42.958399331 0.475166112184525 7.0
41.161203287 0.782101571559906 0.0
28.736175493 0.533224105834961 1.0
12.425027794 0.562227368354797 2.0
0 0.223516747355461 3.0
12.425027794 0.424655824899673 4.0
28.736175493 0.46190994977951 5.0
41.161203287 0.614599823951721 6.0
41.161203287 0.479119539260864 7.0
40.466838645 0.233402594923973 0.0
27.990157761 0.48283714056015 1.0
12.476680884 0.517958521842957 2.0
0 0.163876846432686 3.0
12.476680884 0.437112897634506 4.0
27.990157761 0.463956981897354 5.0
40.466838645 0.191814824938774 6.0
40.466838645 -0.0138135775923729 7.0
33.757389454 0.615502893924713 0.0
21.260951942 0.372696757316589 1.0
12.496437512 0.377712935209274 2.0
0 0.201444372534752 3.0
12.496437512 0.414563626050949 4.0
21.260951942 0.447394013404846 5.0
33.757389454 -0.0158464070409536 6.0
33.757389454 0.335472166538239 7.0
43.437770592 0.789210200309753 0.0
28.513785933 0.527660489082336 1.0
14.923984658 0.559856295585632 2.0
-3.6065221845 0.23783078789711 3.0
11.317462474 0.434644103050232 4.0
32.120308118 0.479543030261993 5.0
43.437770592 0.429423332214355 6.0
43.437770592 0.448895871639252 7.0
39.308931201 0.781916677951813 0.0
29.281572846 0.466488122940063 1.0
10.027358355 0.515016257762909 2.0
0 0.209983885288239 3.0
10.027358355 0.457143664360046 4.0
29.281572846 0.490869373083115 5.0
39.308931201 0.261780679225922 6.0
39.308931201 0.467330455780029 7.0
40.600604276 0.770367622375488 0.0
30.253095292 0.468821167945862 1.0
10.347508984 0.506372272968292 2.0
0 0.181537300348282 3.0
10.347508984 0.434204757213593 4.0
30.253095292 0.467293858528137 5.0
40.600604276 0.441387176513672 6.0
40.600604276 0.543189108371735 7.0
31.520377836 0.806576192378998 0.0
22.250038738 0.500841498374939 1.0
9.270339098 0.546994805335999 2.0
0 0.22181124985218 3.0
9.270339098 0.4323570728302 4.0
22.250038738 0.491334110498428 5.0
31.520377836 0.388052403926849 6.0
31.520377836 0.481838166713715 7.0
39.969125591 0.802157282829285 0.0
26.836211143 0.49263808131218 1.0
13.132914448 0.51548957824707 2.0
0 0.209799587726593 3.0
13.132914448 0.404154777526855 4.0
26.836211143 0.452528893947601 5.0
39.969125591 0.246517613530159 6.0
39.969125591 0.355340123176575 7.0
30.912106182 0.721765518188477 0.0
20.724206455 0.492092698812485 1.0
10.187899727 0.543611109256744 2.0
0 0.196833938360214 3.0
10.187899727 0.434330224990845 4.0
20.724206455 0.477423161268234 5.0
30.912106182 0.364529460668564 6.0
30.912106182 0.614051938056946 7.0
45.180443811 0.795559942722321 0.0
30.231656324 0.537768244743347 1.0
14.948787487 0.57769101858139 2.0
0 0.219425022602081 3.0
14.948787487 0.424855619668961 4.0
30.231656324 0.462855398654938 5.0
45.180443811 0.270683288574219 6.0
45.180443811 0.286943197250366 7.0
34.99216559 0.83259391784668 0.0
21.1924894 0.51744669675827 1.0
13.79967619 0.551594913005829 2.0
0 0.229277148842812 3.0
13.79967619 0.432374060153961 4.0
21.1924894 0.476321309804916 5.0
34.99216559 0.360452473163605 6.0
34.99216559 0.292109996080399 7.0
40.933938659 0.759024143218994 0.0
31.89275394 0.456924259662628 1.0
9.0411847187 0.489495277404785 2.0
0 0.234382063150406 3.0
9.0411847187 0.436167657375336 4.0
31.89275394 0.487885892391205 5.0
40.933938659 0.306376665830612 6.0
40.933938659 0.338122367858887 7.0
40.317005952 0.796028435230255 0.0
30.547820945 0.486862927675247 1.0
9.7691850077 0.525702953338623 2.0
0 0.22910363972187 3.0
9.7691850077 0.461818605661392 4.0
30.547820945 0.490650057792664 5.0
40.317005952 0.259598672389984 6.0
40.317005952 0.449278295040131 7.0
40.010379617 0.725167810916901 0.0
24.888658184 0.475520610809326 1.0
15.121721433 0.520340502262115 2.0
0 0.203900307416916 3.0
15.121721433 0.452974617481232 4.0
24.888658184 0.459793925285339 5.0
40.010379617 0.301140874624252 6.0
40.010379617 0.509556353092194 7.0
38.685596488 0.545413076877594 0.0
28.65763434 0.484680414199829 1.0
10.027962148 0.512455344200134 2.0
0 0.15990948677063 3.0
10.027962148 0.430635631084442 4.0
28.65763434 0.44330245256424 5.0
38.685596488 0.166761636734009 6.0
38.685596488 0.0974187925457954 7.0
30.278486108 0.680630147457123 0.0
19.759275031 0.4747314453125 1.0
10.519211077 0.510950446128845 2.0
0 0.208611071109772 3.0
10.519211077 0.434152841567993 4.0
19.759275031 0.47933030128479 5.0
30.278486108 0.277019590139389 6.0
30.278486108 0.337114572525024 7.0
40.311093813 0.838933169841766 0.0
25.154058321 0.50713711977005 1.0
15.157035492 0.53829550743103 2.0
0 0.222964823246002 3.0
15.157035492 0.449454784393311 4.0
25.154058321 0.461470901966095 5.0
40.311093813 0.534602999687195 6.0
40.311093813 0.306107550859451 7.0
35.760290532 0.795566439628601 0.0
22.277274346 0.437686055898666 1.0
13.483016186 0.456868290901184 2.0
0 0.225329905748367 3.0
13.483016186 0.423708826303482 4.0
22.277274346 0.459177553653717 5.0
35.760290532 0.14213390648365 6.0
35.760290532 0.441093921661377 7.0
40.236813436 0.84432327747345 0.0
29.89918024 0.472468703985214 1.0
10.337633196 0.528862416744232 2.0
0 0.237433448433876 3.0
10.337633196 0.456857800483704 4.0
29.89918024 0.503780543804169 5.0
40.236813436 0.434519618749619 6.0
40.236813436 0.531514585018158 7.0
43.948095531 0.789321303367615 0.0
28.711232243 0.504738867282867 1.0
15.236863288 0.547475576400757 2.0
0 0.208419188857079 3.0
15.236863288 0.436082422733307 4.0
28.711232243 0.488769561052322 5.0
43.948095531 0.435896664857864 6.0
43.948095531 0.452348262071609 7.0
40.202612729 0.828105270862579 0.0
28.531548022 0.465203404426575 1.0
11.671064707 0.518478989601135 2.0
0 0.207637935876846 3.0
11.671064707 0.45008185505867 4.0
28.531548022 0.496331423521042 5.0
40.202612729 0.345711708068848 6.0
40.202612729 0.477390229701996 7.0
41.441227091 0.621005356311798 0.0
26.925574624 0.443998485803604 1.0
14.515652467 0.474034249782562 2.0
0 0.226450517773628 3.0
14.515652467 0.437115073204041 4.0
26.925574624 0.479772955179214 5.0
41.441227091 0.046924889087677 6.0
41.441227091 0.271219819784164 7.0
40.234798657 0.650882482528687 0.0
27.003828746 0.327398955821991 1.0
13.230969911 0.355285704135895 2.0
0 0.266498386859894 3.0
13.230969911 0.433842986822128 4.0
27.003828746 0.497165530920029 5.0
40.234798657 0.102080248296261 6.0
40.234798657 0.20906837284565 7.0
40.268402112 0.813858330249786 0.0
28.231212758 0.513734221458435 1.0
12.037189354 0.536538064479828 2.0
0 0.218942224979401 3.0
12.037189354 0.423304378986359 4.0
28.231212758 0.471041440963745 5.0
40.268402112 0.3616583943367 6.0
40.268402112 0.33258181810379 7.0
34.447402114 0.633421778678894 0.0
23.031509698 0.365182667970657 1.0
11.415892415 0.401213705539703 2.0
0 0.254134804010391 3.0
11.415892415 0.45295324921608 4.0
23.031509698 0.48541647195816 5.0
34.447402114 0.0253872461616993 6.0
34.447402114 0.25474601984024 7.0
40.416533901 0.863305509090424 0.0
29.448368969 0.516085147857666 1.0
10.968164932 0.543672204017639 2.0
0 0.209050863981247 3.0
10.968164932 0.429725050926208 4.0
29.448368969 0.470479547977448 5.0
40.416533901 0.374122381210327 6.0
40.416533901 0.391345858573914 7.0
40.837884543 0.663856685161591 0.0
27.69685337 0.480268776416779 1.0
13.141031172 0.517706096172333 2.0
0 0.211321860551834 3.0
13.141031172 0.443167448043823 4.0
27.69685337 0.46275007724762 5.0
40.837884543 0.317364066839218 6.0
40.837884543 0.268826901912689 7.0
39.823092444 0.824232161045074 0.0
29.374613061 0.557817697525024 1.0
10.448479383 0.587029635906219 2.0
0 0.244592979550362 3.0
10.448479383 0.412196606397629 4.0
29.374613061 0.480683118104935 5.0
39.823092444 0.595793902873993 6.0
39.823092444 0.427825629711151 7.0
40.41620498 0.614220798015594 0.0
30.830981709 0.488616585731506 1.0
9.585223271 0.523065090179443 2.0
0 0.201513364911079 3.0
9.585223271 0.432989567518234 4.0
30.830981709 0.452209651470184 5.0
40.41620498 0.302327334880829 6.0
40.41620498 0.125452041625977 7.0
33.388504976 0.811709880828857 0.0
23.687097805 0.476081937551498 1.0
9.7014071712 0.529811501502991 2.0
0 0.201718866825104 3.0
9.7014071712 0.448134273290634 4.0
23.687097805 0.462773442268372 5.0
33.388504976 0.33895868062973 6.0
33.388504976 0.460971146821976 7.0
42.251691394 0.554833352565765 0.0
30.360448061 0.453369140625 1.0
11.891243333 0.477365523576736 2.0
-1.4316411835 0.154209062457085 3.0
10.459602149 0.413379430770874 4.0
31.792089245 0.43540221452713 5.0
42.251691394 -0.00697393622249365 6.0
42.251691394 0.202454119920731 7.0
32.552772789 0.77423107624054 0.0
23.119967808 0.57982861995697 1.0
9.4328049818 0.610740184783936 2.0
0 0.22412434220314 3.0
9.4328049818 0.410817474126816 4.0
23.119967808 0.468638122081757 5.0
32.552772789 0.255223393440247 6.0
32.552772789 0.282040506601334 7.0
43.396109959 0.33837902545929 0.0
28.433148915 0.648148953914642 1.0
14.962961044 0.687121391296387 2.0
0 0.171254634857178 3.0
14.962961044 0.43771168589592 4.0
28.433148915 0.447233825922012 5.0
43.396109959 0.097889207303524 6.0
43.396109959 -0.100659817457199 7.0
31.629368094 0.817347586154938 0.0
21.596105546 0.438172817230225 1.0
10.033262548 0.47769957780838 2.0
0 0.20130243897438 3.0
10.033262548 0.448046833276749 4.0
21.596105546 0.491194278001785 5.0
31.629368094 0.29368868470192 6.0
31.629368094 0.517411112785339 7.0
37.779811495 0.360480278730392 0.0
25.908076324 0.499527901411057 1.0
11.87173517 0.533571779727936 2.0
0 0.201054617762566 3.0
11.87173517 0.447454243898392 4.0
25.908076324 0.490897119045258 5.0
37.779811495 0.00640070717781782 6.0
37.779811495 -0.01367262378335 7.0
34.24652714 0.864746689796448 0.0
24.424699608 0.462296605110168 1.0
9.8218275322 0.513426959514618 2.0
0 0.221890494227409 3.0
9.8218275322 0.450292438268661 4.0
24.424699608 0.488625884056091 5.0
34.24652714 0.548952043056488 6.0
34.24652714 0.519272208213806 7.0
40.003780762 0.579022347927094 0.0
30.808398459 0.439267694950104 1.0
9.1953823028 0.484471887350082 2.0
0 0.174132138490677 3.0
9.1953823028 0.443940937519073 4.0
30.808398459 0.469607055187225 5.0
40.003780762 0.233612209558487 6.0
40.003780762 0.551254510879517 7.0
31.970353396 0.745029509067535 0.0
22.618886495 0.454454064369202 1.0
9.3514669012 0.507682800292969 2.0
0 0.195197150111198 3.0
9.3514669012 0.453574597835541 4.0
22.618886495 0.47710308432579 5.0
31.970353396 0.287955611944199 6.0
31.970353396 0.571686148643494 7.0
37.935925517 0.695942640304565 0.0
23.719907188 0.427724689245224 1.0
14.21601833 0.480260848999023 2.0
0 0.19111455976963 3.0
14.21601833 0.450302481651306 4.0
23.719907188 0.491835057735443 5.0
37.935925517 0.244796127080917 6.0
37.935925517 0.536572694778442 7.0
37.4568261 0.622248947620392 0.0
24.360290248 0.445997267961502 1.0
13.096535852 0.474628478288651 2.0
0 0.181538850069046 3.0
13.096535852 0.437245428562164 4.0
24.360290248 0.470304071903229 5.0
37.4568261 0.255560845136642 6.0
37.4568261 0.32320311665535 7.0
34.938845635 0.839130938053131 0.0
24.698541768 0.524740874767303 1.0
10.240303867 0.55354917049408 2.0
0 0.226498663425446 3.0
10.240303867 0.437295198440552 4.0
24.698541768 0.469970852136612 5.0
34.938845635 0.5274458527565 6.0
34.938845635 0.377839386463165 7.0
33.644682189 0.137220337986946 0.0
23.878275232 0.721189916133881 1.0
9.7664069574 0.760638177394867 2.0
0 0.203536480665207 3.0
9.7664069574 0.442537099123001 4.0
23.878275232 0.474965572357178 5.0
33.644682189 8.11368227005005e-05 6.0
33.644682189 0.378574907779694 7.0
43.152629724 0.674295663833618 0.0
29.600563314 0.459979474544525 1.0
13.55206641 0.499686449766159 2.0
0 0.189341709017754 3.0
13.55206641 0.43625795841217 4.0
29.600563314 0.484194368124008 5.0
43.152629724 0.293773740530014 6.0
43.152629724 0.588226914405823 7.0
29.29034141 0.163494497537613 0.0
20.287635655 0.308476328849792 1.0
9.0027057545 0.341448158025742 2.0
0 0.250177025794983 3.0
9.0027057545 0.442974776029587 4.0
20.287635655 0.487383157014847 5.0
29.29034141 0.346538364887238 6.0
29.29034141 0.662787675857544 7.0
36.529945045 0.170982643961906 0.0
25.861121308 0.66765946149826 1.0
10.668823737 0.702960729598999 2.0
0 0.15674951672554 3.0
10.668823737 0.43869349360466 4.0
25.861121308 0.442617177963257 5.0
36.529945045 -0.0639661401510239 6.0
36.529945045 -0.0794261395931244 7.0
39.490028245 0.823695361614227 0.0
28.809011403 0.429608643054962 1.0
10.681016842 0.467456668615341 2.0
0 0.251990288496017 3.0
10.681016842 0.451483488082886 4.0
28.809011403 0.480885475873947 5.0
39.490028245 0.276479691267014 6.0
39.490028245 0.369283437728882 7.0
34.709507968 0.780969500541687 0.0
23.450878064 0.509256184101105 1.0
11.258629904 0.525171518325806 2.0
0 0.210477709770203 3.0
11.258629904 0.422484666109085 4.0
23.450878064 0.478887259960175 5.0
34.709507968 0.387906849384308 6.0
34.709507968 0.302719742059708 7.0
45.017599836 0.815544247627258 0.0
31.949025859 0.556151449680328 1.0
13.068573978 0.600121974945068 2.0
0 0.249904006719589 3.0
13.068573978 0.440150320529938 4.0
31.949025859 0.477317094802856 5.0
45.017599836 0.453605622053146 6.0
45.017599836 0.436097681522369 7.0
32.940669776 0.741158068180084 0.0
21.374565291 0.520636737346649 1.0
11.566104485 0.562661528587341 2.0
0 0.231569096446037 3.0
11.566104485 0.423524588346481 4.0
21.374565291 0.463613629341125 5.0
32.940669776 0.265047669410706 6.0
32.940669776 0.336740583181381 7.0
44.072190389 0.433577060699463 0.0
29.032860365 0.425351172685623 1.0
15.039330024 0.4416264295578 2.0
0 0.158616498112679 3.0
15.039330024 0.401291519403458 4.0
29.032860365 0.43481433391571 5.0
44.072190389 0.219403371214867 6.0
44.072190389 0.5691938996315 7.0
35.422022456 0.726730763912201 0.0
20.530264722 0.518710136413574 1.0
14.891757734 0.56212055683136 2.0
0 0.24020990729332 3.0
14.891757734 0.425267457962036 4.0
20.530264722 0.472503215074539 5.0
35.422022456 0.28703436255455 6.0
35.422022456 0.310002148151398 7.0
36.153594117 0.659077346324921 0.0
27.513199792 0.448379248380661 1.0
8.6403943249 0.48860490322113 2.0
0 0.178680658340454 3.0
8.6403943249 0.436225116252899 4.0
27.513199792 0.469920605421066 5.0
36.153594117 0.259342521429062 6.0
36.153594117 0.490491926670074 7.0
38.951377458 0.842035710811615 0.0
27.10341817 0.523775339126587 1.0
11.847959289 0.552973449230194 2.0
0 0.230052292346954 3.0
11.847959289 0.444388121366501 4.0
27.10341817 0.485168009996414 5.0
38.951377458 0.409061193466187 6.0
38.951377458 0.407182365655899 7.0
40.644172477 0.845954060554504 0.0
28.871927364 0.548332691192627 1.0
11.772245113 0.596139550209045 2.0
0 0.233447670936584 3.0
11.772245113 0.429199695587158 4.0
28.871927364 0.472139537334442 5.0
40.644172477 0.3404261469841 6.0
40.644172477 0.350879848003387 7.0
35.600531382 0.828681766986847 0.0
24.235795061 0.448939561843872 1.0
11.364736321 0.488180965185165 2.0
0 0.193133816123009 3.0
11.364736321 0.438678920269012 4.0
24.235795061 0.467833697795868 5.0
35.600531382 0.299473255872726 6.0
35.600531382 0.531875610351562 7.0
41.682033599 0.724063754081726 0.0
29.765882281 0.486572623252869 1.0
11.916151318 0.506341814994812 2.0
-0.12012993231 0.152846574783325 3.0
11.796021386 0.417903214693069 4.0
29.886012214 0.448732495307922 5.0
41.682033599 0.291404068470001 6.0
41.682033599 0.356805473566055 7.0
39.697348938 0.811676681041718 0.0
25.334104361 0.444702982902527 1.0
14.363244578 0.491698712110519 2.0
0 0.210645765066147 3.0
14.363244578 0.450691521167755 4.0
25.334104361 0.488625377416611 5.0
39.697348938 0.310435563325882 6.0
39.697348938 0.52720707654953 7.0
40.978668493 0.80098682641983 0.0
27.414183867 0.53301215171814 1.0
13.564484626 0.570398330688477 2.0
0 0.23868964612484 3.0
13.564484626 0.418851256370544 4.0
27.414183867 0.477587670087814 5.0
40.978668493 0.51816600561142 6.0
40.978668493 0.300620943307877 7.0
46.524483465 0.838250756263733 0.0
30.674387827 0.470500409603119 1.0
15.850095638 0.529142618179321 2.0
0 0.21652752161026 3.0
15.850095638 0.457629770040512 4.0
30.674387827 0.477099806070328 5.0
46.524483465 0.525684118270874 6.0
46.524483465 0.50344055891037 7.0
39.683969876 0.862537205219269 0.0
28.712254285 0.483742594718933 1.0
10.971715591 0.542938649654388 2.0
0 0.21887469291687 3.0
10.971715591 0.449684053659439 4.0
28.712254285 0.476190030574799 5.0
39.683969876 0.554523289203644 6.0
39.683969876 0.506298899650574 7.0
46.294692767 0.477879345417023 0.0
31.830997773 0.452083319425583 1.0
14.463694994 0.479299604892731 2.0
0 0.196213334798813 3.0
14.463694994 0.416516005992889 4.0
31.830997773 0.463115870952606 5.0
46.294692767 -0.00854126177728176 6.0
46.294692767 0.157146468758583 7.0
44.785541844 0.655936896800995 0.0
30.101539434 0.483851373195648 1.0
14.684002411 0.521312713623047 2.0
-1.0807278173 0.189014405012131 3.0
13.603274593 0.422543615102768 4.0
31.182267251 0.448022991418839 5.0
44.785541844 0.331685781478882 6.0
44.785541844 0.288005709648132 7.0
33.676333572 0.841765701770782 0.0
24.221820866 0.524360656738281 1.0
9.4545127063 0.549581289291382 2.0
0 0.222773998975754 3.0
9.4545127063 0.436137497425079 4.0
24.221820866 0.463169515132904 5.0
33.676333572 0.501551389694214 6.0
33.676333572 0.367675125598907 7.0
35.625618922 0.829233169555664 0.0
20.256511762 0.548307716846466 1.0
15.369107161 0.579243063926697 2.0
0 0.231296673417091 3.0
15.369107161 0.420522570610046 4.0
20.256511762 0.469740390777588 5.0
35.625618922 0.541749894618988 6.0
35.625618922 0.391774535179138 7.0
40.226157029 0.842509925365448 0.0
28.220684127 0.518953800201416 1.0
12.005472902 0.546702146530151 2.0
0 0.22388231754303 3.0
12.005472902 0.438953995704651 4.0
28.220684127 0.463092058897018 5.0
40.226157029 0.339834183454514 6.0
40.226157029 0.320722728967667 7.0
44.003195919 0.739822208881378 0.0
30.390659394 0.545289099216461 1.0
13.612536526 0.580045402050018 2.0
0 0.229024648666382 3.0
13.612536526 0.432973176240921 4.0
30.390659394 0.474433839321136 5.0
44.003195919 0.378883242607117 6.0
44.003195919 0.387751132249832 7.0
33.88878315 0.829002737998962 0.0
22.572006823 0.482835292816162 1.0
11.316776327 0.541327178478241 2.0
0 0.220925360918045 3.0
11.316776327 0.453711211681366 4.0
22.572006823 0.476474702358246 5.0
33.88878315 0.561817407608032 6.0
33.88878315 0.52483594417572 7.0
42.125521158 0.719116151332855 0.0
27.124375068 0.523897469043732 1.0
15.00114609 0.570456087589264 2.0
0 0.227881103754044 3.0
15.00114609 0.455991506576538 4.0
27.124375068 0.508558690547943 5.0
42.125521158 0.509978175163269 6.0
42.125521158 0.520654022693634 7.0
37.55067892 0.768102884292603 0.0
28.417766606 0.490339487791061 1.0
9.1329123142 0.522340536117554 2.0
0 0.217231333255768 3.0
9.1329123142 0.435216307640076 4.0
28.417766606 0.486202239990234 5.0
37.55067892 0.236067235469818 6.0
37.55067892 0.362213283777237 7.0
37.39948618 0.800835609436035 0.0
25.284755164 0.428945183753967 1.0
12.114731017 0.464847445487976 2.0
0 0.193325266242027 3.0
12.114731017 0.441429674625397 4.0
25.284755164 0.47260519862175 5.0
37.39948618 0.374361455440521 6.0
37.39948618 0.510845422744751 7.0
34.341868993 0.356555223464966 0.0
25.736031722 0.444835960865021 1.0
8.6058372707 0.476170599460602 2.0
0 0.18976503610611 3.0
8.6058372707 0.444457173347473 4.0
25.736031722 0.472164005041122 5.0
34.341868993 0.231031194329262 6.0
34.341868993 0.170882254838943 7.0
37.425355461 0.832730412483215 0.0
23.108404202 0.497640788555145 1.0
14.316951259 0.543325126171112 2.0
0 0.217540055513382 3.0
14.316951259 0.440960109233856 4.0
23.108404202 0.487115442752838 5.0
37.425355461 0.40717938542366 6.0
37.425355461 0.43628466129303 7.0
37.288491912 0.803253650665283 0.0
28.215703892 0.492514699697495 1.0
9.0727880197 0.547711491584778 2.0
0 0.239780783653259 3.0
9.0727880197 0.449190139770508 4.0
28.215703892 0.492241114377975 5.0
37.288491912 0.533772230148315 6.0
37.288491912 0.543611705303192 7.0
39.20347441 0.576323688030243 0.0
25.942862958 0.443642258644104 1.0
13.260611453 0.478121519088745 2.0
0 0.176715776324272 3.0
13.260611453 0.440610766410828 4.0
25.942862958 0.456402003765106 5.0
39.20347441 0.363000363111496 6.0
39.20347441 0.353269219398499 7.0
36.137884824 0.855295062065125 0.0
22.759896907 0.482566803693771 1.0
13.377987918 0.525928616523743 2.0
0 0.262983947992325 3.0
13.377987918 0.456713706254959 4.0
22.759896907 0.515828669071198 5.0
36.137884824 0.412505596876144 6.0
36.137884824 0.288487166166306 7.0
41.655417288 0.788184702396393 0.0
31.067976794 0.520488858222961 1.0
10.587440495 0.559696137905121 2.0
0 0.240109115839005 3.0
10.587440495 0.446697115898132 4.0
31.067976794 0.481211751699448 5.0
41.655417288 0.519771575927734 6.0
41.655417288 0.527117908000946 7.0
39.894978379 0.84921258687973 0.0
27.87905175 0.470551490783691 1.0
12.01592663 0.530196487903595 2.0
0 0.20897364616394 3.0
12.01592663 0.456653952598572 4.0
27.87905175 0.472027778625488 5.0
39.894978379 0.469559520483017 6.0
39.894978379 0.468638926744461 7.0
39.390677385 0.764048337936401 0.0
23.775179739 0.408933073282242 1.0
15.615497647 0.414964109659195 2.0
0 0.218922987580299 3.0
15.615497647 0.417196154594421 4.0
23.775179739 0.454847812652588 5.0
39.390677385 0.0920688360929489 6.0
39.390677385 0.478188902139664 7.0
38.248327703 0.831787049770355 0.0
28.805837831 0.466651886701584 1.0
9.4424898716 0.52033531665802 2.0
0 0.214888229966164 3.0
9.4424898716 0.455206483602524 4.0
28.805837831 0.481190145015717 5.0
38.248327703 0.488779932260513 6.0
38.248327703 0.524509251117706 7.0
38.538636426 0.829860508441925 0.0
26.334072091 0.474645912647247 1.0
12.204564335 0.533638954162598 2.0
0 0.239574670791626 3.0
12.204564335 0.459662795066833 4.0
26.334072091 0.49693375825882 5.0
38.538636426 0.510874629020691 6.0
38.538636426 0.518048405647278 7.0
30.093865158 0.4898422062397 0.0
19.25470281 0.392176419496536 1.0
10.839162348 0.428547084331512 2.0
0 0.195507362484932 3.0
10.839162348 0.405080378055573 4.0
19.25470281 0.428700983524323 5.0
30.093865158 0.0379255786538124 6.0
30.093865158 0.345939457416534 7.0
36.059945316 0.829988896846771 0.0
27.426776021 0.343459218740463 1.0
8.6331692955 0.382553219795227 2.0
0 0.230791017413139 3.0
8.6331692955 0.469299912452698 4.0
27.426776021 0.501267552375793 5.0
36.059945316 0.579757034778595 6.0
36.059945316 0.443004876375198 7.0
41.382686141 0.758825421333313 0.0
30.919992406 0.533676385879517 1.0
10.462693735 0.576283514499664 2.0
0 0.234919294714928 3.0
10.462693735 0.426487922668457 4.0
30.919992406 0.469380289316177 5.0
41.382686141 0.330253809690475 6.0
41.382686141 0.335124254226685 7.0
37.374314239 0.77263605594635 0.0
24.809000191 0.479119598865509 1.0
12.565314048 0.53385066986084 2.0
0 0.195990845561028 3.0
12.565314048 0.444365739822388 4.0
24.809000191 0.478054314851761 5.0
37.374314239 0.394645750522614 6.0
37.374314239 0.581712543964386 7.0
37.689487865 0.581778585910797 0.0
23.778452679 0.442288726568222 1.0
13.911035186 0.482822239398956 2.0
0 0.194729581475258 3.0
13.911035186 0.456823498010635 4.0
23.778452679 0.467709898948669 5.0
37.689487865 0.304965138435364 6.0
37.689487865 0.447632879018784 7.0
36.541540581 0.829984843730927 0.0
25.292349782 0.486431211233139 1.0
11.249190798 0.542470693588257 2.0
0 0.222336813807487 3.0
11.249190798 0.440548926591873 4.0
25.292349782 0.483181059360504 5.0
36.541540581 0.533578872680664 6.0
36.541540581 0.502507150173187 7.0
36.791141656 0.733271241188049 0.0
23.632233062 0.476416707038879 1.0
13.158908594 0.533956289291382 2.0
0 0.224450215697289 3.0
13.158908594 0.452402710914612 4.0
23.632233062 0.497908294200897 5.0
36.791141656 0.554515063762665 6.0
36.791141656 0.512175440788269 7.0
39.891605825 0.859999179840088 0.0
27.59587785 0.588183343410492 1.0
12.295727975 0.640135526657104 2.0
0 0.250788122415543 3.0
12.295727975 0.419509828090668 4.0
27.59587785 0.472173094749451 5.0
39.891605825 0.485537171363831 6.0
39.891605825 0.217408895492554 7.0
34.771445328 0.839740812778473 0.0
22.721923981 0.462479084730148 1.0
12.049521348 0.505071043968201 2.0
0 0.188699215650558 3.0
12.049521348 0.433919608592987 4.0
22.721923981 0.470237225294113 5.0
34.771445328 0.412918895483017 6.0
34.771445328 0.460507780313492 7.0
44.638557165 0.69169282913208 0.0
30.610302295 0.458334386348724 1.0
14.02825487 0.494569957256317 2.0
0 0.178731292486191 3.0
14.02825487 0.433125674724579 4.0
30.610302295 0.452012360095978 5.0
44.638557165 0.222351759672165 6.0
44.638557165 0.445273250341415 7.0
36.096180936 0.740875005722046 0.0
20.898821406 0.516526639461517 1.0
15.19735953 0.549449145793915 2.0
0 0.223183423280716 3.0
15.19735953 0.454674065113068 4.0
20.898821406 0.481818407773972 5.0
36.096180936 0.542501509189606 6.0
36.096180936 0.401081770658493 7.0
45.744029538 0.769133448600769 0.0
31.432297526 0.504853248596191 1.0
14.311732013 0.553110837936401 2.0
0 0.220660224556923 3.0
14.311732013 0.453269302845001 4.0
31.432297526 0.495472222566605 5.0
45.744029538 0.4182408452034 6.0
45.744029538 0.624072909355164 7.0
36.287608705 0.731373846530914 0.0
25.16521997 0.488534957170486 1.0
11.122388735 0.540647983551025 2.0
0 0.195405185222626 3.0
11.122388735 0.438650548458099 4.0
25.16521997 0.460792422294617 5.0
36.287608705 0.327145844697952 6.0
36.287608705 0.552481532096863 7.0
42.801830865 0.825211226940155 0.0
27.040244124 0.501466155052185 1.0
15.761586741 0.530231118202209 2.0
0 0.234962090849876 3.0
15.761586741 0.440328687429428 4.0
27.040244124 0.49865847826004 5.0
42.801830865 0.401632100343704 6.0
42.801830865 0.287659615278244 7.0
47.558612578 0.765089511871338 0.0
32.997784418 0.509720623493195 1.0
14.56082816 0.541511714458466 2.0
0 0.213970094919205 3.0
14.56082816 0.425100088119507 4.0
32.997784418 0.469201624393463 5.0
47.558612578 0.294362962245941 6.0
47.558612578 0.361552685499191 7.0
47.058562462 0.836169064044952 0.0
29.793950686 0.324345916509628 1.0
17.264611775 0.354446798563004 2.0
-1.5381975965 0.179364904761314 3.0
15.726414179 0.443008989095688 4.0
31.332148283 0.46812230348587 5.0
47.058562462 0.581381976604462 6.0
47.058562462 0.453312873840332 7.0
43.087048946 0.251832842826843 0.0
31.678082071 0.458430022001266 1.0
11.408966876 0.492297351360321 2.0
0 0.187294393777847 3.0
11.408966876 0.460924446582794 4.0
31.678082071 0.474490106105804 5.0
43.087048946 0.216200292110443 6.0
43.087048946 0.048327349126339 7.0
33.170190568 0.804521858692169 0.0
20.693720573 0.493608355522156 1.0
12.476469995 0.529906570911407 2.0
0 0.238997519016266 3.0
12.476469995 0.451689124107361 4.0
20.693720573 0.488741934299469 5.0
33.170190568 0.294514268636703 6.0
33.170190568 0.416710525751114 7.0
34.957431523 0.767542898654938 0.0
21.148722929 0.486349880695343 1.0
13.808708594 0.521443724632263 2.0
0 0.223169252276421 3.0
13.808708594 0.434784531593323 4.0
21.148722929 0.476090759038925 5.0
34.957431523 0.284330487251282 6.0
34.957431523 0.336738646030426 7.0
36.320328519 0.453794866800308 0.0
23.458317243 0.458259791135788 1.0
12.862011276 0.485505253076553 2.0
0 0.18622088432312 3.0
12.862011276 0.437724530696869 4.0
23.458317243 0.464717984199524 5.0
36.320328519 0.309058040380478 6.0
36.320328519 0.242196589708328 7.0
44.085721677 0.853151321411133 0.0
30.493758234 0.525163412094116 1.0
13.591963443 0.554679691791534 2.0
0 0.235083416104317 3.0
13.591963443 0.438571840524673 4.0
30.493758234 0.477535903453827 5.0
44.085721677 0.476452648639679 6.0
44.085721677 0.426071226596832 7.0
45.211814568 0.730757117271423 0.0
29.654020724 0.472769200801849 1.0
15.557793845 0.516591668128967 2.0
-1.8265292485 0.182553306221962 3.0
13.731264596 0.442037343978882 4.0
31.480549972 0.464348018169403 5.0
45.211814568 0.405426651239395 6.0
45.211814568 0.562685012817383 7.0
38.71721827 0.55953848361969 0.0
25.13774692 0.46828681230545 1.0
13.57947135 0.498782217502594 2.0
0 0.182582199573517 3.0
13.57947135 0.432104974985123 4.0
25.13774692 0.466955572366714 5.0
38.71721827 0.325168043375015 6.0
38.71721827 0.310213476419449 7.0
42.190454953 0.83099353313446 0.0
27.535288565 0.548747360706329 1.0
14.655166388 0.59564208984375 2.0
0 0.247689604759216 3.0
14.655166388 0.423866987228394 4.0
27.535288565 0.472783446311951 5.0
42.190454953 0.412186861038208 6.0
42.190454953 0.371216416358948 7.0
40.616038376 0.790348708629608 0.0
28.55006814 0.475070029497147 1.0
12.065970236 0.52785450220108 2.0
0 0.210344105958939 3.0
12.065970236 0.450180858373642 4.0
28.55006814 0.491857826709747 5.0
40.616038376 0.297338336706161 6.0
40.616038376 0.525890052318573 7.0
31.887764532 0.765896677970886 0.0
20.470687746 0.639325737953186 1.0
11.417076785 0.683603048324585 2.0
0 0.274116992950439 3.0
11.417076785 0.428325235843658 4.0
20.470687746 0.469673693180084 5.0
31.887764532 0.28444117307663 6.0
31.887764532 0.245365917682648 7.0
39.764867486 0.269996792078018 0.0
29.907448957 0.322924047708511 1.0
9.8574185294 0.334641575813293 2.0
0 0.179844990372658 3.0
9.8574185294 0.381209373474121 4.0
29.907448957 0.434848785400391 5.0
39.764867486 0.141658395528793 6.0
39.764867486 0.0861054286360741 7.0
40.805009287 0.175317004323006 0.0
29.452805352 0.605387985706329 1.0
11.352203935 0.646177470684052 2.0
0 0.201307207345963 3.0
11.352203935 0.446570247411728 4.0
29.452805352 0.469405233860016 5.0
40.805009287 -0.00762186199426651 6.0
40.805009287 -0.102342367172241 7.0
39.795600818 0.788265228271484 0.0
26.601347902 0.51937198638916 1.0
13.194252916 0.558219134807587 2.0
0 0.245271265506744 3.0
13.194252916 0.457745015621185 4.0
26.601347902 0.497253656387329 5.0
39.795600818 0.492575109004974 6.0
39.795600818 0.542112946510315 7.0
38.685439721 0.663237988948822 0.0
26.326388497 0.443660914897919 1.0
12.359051224 0.491417169570923 2.0
0 0.18854609131813 3.0
12.359051224 0.442904442548752 4.0
26.326388497 0.48028028011322 5.0
38.685439721 0.249089851975441 6.0
38.685439721 0.580181360244751 7.0
42.152656556 0.816113829612732 0.0
27.567345332 0.361722230911255 1.0
14.585311224 0.397283583879471 2.0
0 0.221218183636665 3.0
14.585311224 0.460176408290863 4.0
27.567345332 0.507707715034485 5.0
42.152656556 0.539868772029877 6.0
42.152656556 0.485894799232483 7.0
37.458343334 0.363158285617828 0.0
22.42476406 0.0884044766426086 1.0
15.033579274 0.114171668887138 2.0
0 0.251310020685196 3.0
15.033579274 0.427126348018646 4.0
22.42476406 0.478590905666351 5.0
37.458343334 1.43568873405457 6.0
37.458343334 0.334411859512329 7.0
44.489633946 0.753030359745026 0.0
28.961471546 0.320512682199478 1.0
15.5281624 0.361837208271027 2.0
0 0.235205307602882 3.0
15.5281624 0.411390990018845 4.0
28.961471546 0.456252753734589 5.0
44.489633946 0.123183891177177 6.0
44.489633946 0.0495978444814682 7.0
41.506640114 0.827824175357819 0.0
26.860159768 0.41014689207077 1.0
14.646480346 0.447557389736176 2.0
0 0.219842374324799 3.0
14.646480346 0.46123218536377 4.0
26.860159768 0.493260473012924 5.0
41.506640114 0.500347316265106 6.0
41.506640114 0.509965062141418 7.0
33.262736569 0.727373361587524 0.0
19.270575871 0.513414621353149 1.0
13.992160697 0.551025807857513 2.0
0 0.216522976756096 3.0
13.992160697 0.41122567653656 4.0
19.270575871 0.453184992074966 5.0
33.262736569 0.278282254934311 6.0
33.262736569 0.329334437847137 7.0
47.069969764 0.492064982652664 0.0
31.242343365 0.462925910949707 1.0
15.827626399 0.497679144144058 2.0
0 0.196746796369553 3.0
15.827626399 0.446607738733292 4.0
31.242343365 0.465272665023804 5.0
47.069969764 0.239511042833328 6.0
47.069969764 0.167534202337265 7.0
43.288539967 0.579037070274353 0.0
27.687399618 0.464620441198349 1.0
15.601140349 0.495984852313995 2.0
0 0.218285277485847 3.0
15.601140349 0.440158605575562 4.0
27.687399618 0.493175268173218 5.0
43.288539967 0.0332961976528168 6.0
43.288539967 0.158869370818138 7.0
41.479216264 0.710120022296906 0.0
26.515367489 0.471049815416336 1.0
14.963848776 0.507046699523926 2.0
0 0.211191147565842 3.0
14.963848776 0.45024573802948 4.0
26.515367489 0.49560621380806 5.0
41.479216264 0.274210393428802 6.0
41.479216264 0.583715617656708 7.0
36.99162076 0.795879006385803 0.0
21.889008767 0.507975518703461 1.0
15.102611993 0.531408727169037 2.0
0 0.20392695069313 3.0
15.102611993 0.431393563747406 4.0
21.889008767 0.453881442546844 5.0
36.99162076 0.326482117176056 6.0
36.99162076 0.340961486101151 7.0
40.435886771 0.727305114269257 0.0
28.088815114 0.468338489532471 1.0
12.347071657 0.509993076324463 2.0
-2.1989446774 0.215879648923874 3.0
10.148126979 0.460224539041519 4.0
30.287759792 0.481431186199188 5.0
40.435886771 0.2606540620327 6.0
40.435886771 0.443439096212387 7.0
37.140737383 0.791135430335999 0.0
25.451742904 0.492880642414093 1.0
11.68899448 0.54355651140213 2.0
0 0.230941355228424 3.0
11.68899448 0.452860891819 4.0
25.451742904 0.500938773155212 5.0
37.140737383 0.50535774230957 6.0
37.140737383 0.485817134380341 7.0
39.233907158 0.0924374684691429 0.0
24.739135527 1.78525233268738 1.0
14.494771631 1.80091655254364 2.0
0 0.170271530747414 3.0
14.494771631 0.436626821756363 4.0
24.739135527 0.452435255050659 5.0
39.233907158 0.788700044155121 6.0
39.233907158 20.8937492370605 7.0
42.028602534 0.867590487003326 0.0
26.365954124 0.622472107410431 1.0
15.66264841 0.657801985740662 2.0
0 0.262709647417068 3.0
15.66264841 0.425092339515686 4.0
26.365954124 0.485545814037323 5.0
42.028602534 0.360188394784927 6.0
42.028602534 0.197759345173836 7.0
39.783103652 0.502975761890411 0.0
24.444813382 0.433340847492218 1.0
15.33829027 0.458600401878357 2.0
0 0.198222145438194 3.0
15.33829027 0.418039619922638 4.0
24.444813382 0.450436592102051 5.0
39.783103652 -0.00185212679207325 6.0
39.783103652 0.302625715732574 7.0
39.425516455 0.780720233917236 0.0
28.190796848 0.525082349777222 1.0
11.234719607 0.551677942276001 2.0
0 0.226160451769829 3.0
11.234719607 0.439467042684555 4.0
28.190796848 0.474152326583862 5.0
39.425516455 0.535301506519318 6.0
39.425516455 0.364696711301804 7.0
40.002559542 0.735043048858643 0.0
25.191107177 0.462506115436554 1.0
14.811452365 0.503107845783234 2.0
0 0.213103771209717 3.0
14.811452365 0.455327749252319 4.0
25.191107177 0.473927319049835 5.0
40.002559542 0.311489075422287 6.0
40.002559542 0.35161566734314 7.0
38.322665212 0.812806725502014 0.0
24.775138059 0.38771778345108 1.0
13.547527153 0.425627171993256 2.0
0 0.271121084690094 3.0
13.547527153 0.45910370349884 4.0
24.775138059 0.507686078548431 5.0
38.322665212 0.277381956577301 6.0
38.322665212 0.336555778980255 7.0
40.568600192 0.685778021812439 0.0
25.445690154 0.413862586021423 1.0
15.122910038 0.446553111076355 2.0
0 0.245540499687195 3.0
15.122910038 0.443612605333328 4.0
25.445690154 0.490930318832397 5.0
40.568600192 0.224337220191956 6.0
40.568600192 0.26363268494606 7.0
40.337689123 0.785712778568268 0.0
28.556160654 0.491209149360657 1.0
11.781528468 0.525064468383789 2.0
0 0.227939784526825 3.0
11.781528468 0.448422849178314 4.0
28.556160654 0.484233528375626 5.0
40.337689123 0.368429094552994 6.0
40.337689123 0.322571516036987 7.0
45.726007711 0.732672512531281 0.0
32.201501409 0.456068962812424 1.0
13.524506302 0.504116475582123 2.0
0 0.210868641734123 3.0
13.524506302 0.445475161075592 4.0
32.201501409 0.487887620925903 5.0
45.726007711 0.258503466844559 6.0
45.726007711 0.556243777275085 7.0
36.675787374 0.843296051025391 0.0
27.948349606 0.50584465265274 1.0
8.7274377682 0.546075284481049 2.0
0 0.231807455420494 3.0
8.7274377682 0.434578567743301 4.0
27.948349606 0.467675775289536 5.0
36.675787374 0.389026641845703 6.0
36.675787374 0.321653783321381 7.0
32.235093382 0.718962132930756 0.0
21.205932177 0.504782438278198 1.0
11.029161206 0.561880946159363 2.0
0 0.223219633102417 3.0
11.029161206 0.464746475219727 4.0
21.205932177 0.483413815498352 5.0
32.235093382 0.528319954872131 6.0
32.235093382 0.506022155284882 7.0
39.438537732 0.832511425018311 0.0
26.045647237 0.495471060276031 1.0
13.392890495 0.53849720954895 2.0
0 0.26960626244545 3.0
13.392890495 0.459973007440567 4.0
26.045647237 0.529717326164246 5.0
39.438537732 0.504038572311401 6.0
39.438537732 0.35876390337944 7.0
44.304614406 0.798607170581818 0.0
30.417756083 0.333976238965988 1.0
13.886858323 0.364942610263824 2.0
0 0.206213891506195 3.0
13.886858323 0.453167736530304 4.0
30.417756083 0.498596489429474 5.0
44.304614406 0.522411227226257 6.0
44.304614406 0.492323905229568 7.0
32.373304325 0.282308220863342 0.0
22.786241426 0.562253654003143 1.0
9.5870628991 0.601056575775146 2.0
0 0.201459258794785 3.0
9.5870628991 0.43321305513382 4.0
22.786241426 0.461896419525146 5.0
32.373304325 -0.0194011591374874 6.0
32.373304325 -0.0373598337173462 7.0
35.769379856 0.75627338886261 0.0
25.430130877 0.478938102722168 1.0
10.339248979 0.513798475265503 2.0
0 0.227053493261337 3.0
10.339248979 0.440272837877274 4.0
25.430130877 0.496189117431641 5.0
35.769379856 0.287195980548859 6.0
35.769379856 0.357178866863251 7.0
38.348316661 0.803281724452972 0.0
28.599053532 0.523897588253021 1.0
9.7492631289 0.55842912197113 2.0
0 0.214372098445892 3.0
9.7492631289 0.431454211473465 4.0
28.599053532 0.455771505832672 5.0
38.348316661 0.273515313863754 6.0
38.348316661 0.323279052972794 7.0
30.534523433 0.783188581466675 0.0
21.176430979 0.491856038570404 1.0
9.3580924534 0.522153377532959 2.0
0 0.205045312643051 3.0
9.3580924534 0.440807461738586 4.0
21.176430979 0.467590093612671 5.0
30.534523433 0.284853339195251 6.0
30.534523433 0.362046033143997 7.0
41.665325899 0.831326961517334 0.0
29.240172477 0.505439758300781 1.0
12.425153422 0.531885266304016 2.0
0 0.214288294315338 3.0
12.425153422 0.445488333702087 4.0
29.240172477 0.467988133430481 5.0
41.665325899 0.317501962184906 6.0
41.665325899 0.360829174518585 7.0
39.125756769 0.847474098205566 0.0
23.306411016 0.501657128334045 1.0
15.819345753 0.540571093559265 2.0
0 0.228399977087975 3.0
15.819345753 0.452569514513016 4.0
23.306411016 0.487775087356567 5.0
39.125756769 0.351486504077911 6.0
39.125756769 0.460732877254486 7.0
46.922469454 0.793150365352631 0.0
31.861747681 0.48756730556488 1.0
15.060721773 0.526739358901978 2.0
0 0.2343660145998 3.0
15.060721773 0.463797241449356 4.0
31.861747681 0.50634104013443 5.0
46.922469454 0.278255045413971 6.0
46.922469454 0.398054420948029 7.0
38.082156965 0.115104548633099 0.0
26.037781012 0.99053829908371 1.0
12.044375953 0.990469932556152 2.0
0 0.161712527275085 3.0
12.044375953 0.432081460952759 4.0
26.037781012 0.45561084151268 5.0
38.082156965 0.17919896543026 6.0
38.082156965 0.871791899204254 7.0
35.281773346 0.846243500709534 0.0
22.096480246 0.53650027513504 1.0
13.185293101 0.570440709590912 2.0
0 0.228667870163918 3.0
13.185293101 0.423119157552719 4.0
22.096480246 0.473610013723373 5.0
35.281773346 0.297596633434296 6.0
35.281773346 0.286280870437622 7.0
39.676820236 0.759587705135345 0.0
30.920209498 0.486791640520096 1.0
8.7566107375 0.519767224788666 2.0
0 0.210380792617798 3.0
8.7566107375 0.436247229576111 4.0
30.920209498 0.477970629930496 5.0
39.676820236 0.27116459608078 6.0
39.676820236 0.359000980854034 7.0
41.049027324 0.8468257188797 0.0
27.686936742 0.477710127830505 1.0
13.362090582 0.533372282981873 2.0
0 0.216967672109604 3.0
13.362090582 0.453312605619431 4.0
27.686936742 0.488722890615463 5.0
41.049027324 0.420675337314606 6.0
41.049027324 0.46754252910614 7.0
41.402616693 0.116684474050999 0.0
31.630523314 0.596835970878601 1.0
9.7720933796 0.61355072259903 2.0
0 0.187870591878891 3.0
9.7720933796 0.416359454393387 4.0
31.630523314 0.449009299278259 5.0
41.402616693 0.911306500434875 6.0
41.402616693 4.10548400878906 7.0
30.3721478 0.798549175262451 0.0
20.852771908 0.511887729167938 1.0
9.5193758911 0.544537425041199 2.0
0 0.204956233501434 3.0
9.5193758911 0.406032979488373 4.0
20.852771908 0.444952815771103 5.0
30.3721478 0.277315467596054 6.0
30.3721478 0.344078153371811 7.0
41.111385117 0.815402925014496 0.0
26.925919629 0.49236187338829 1.0
14.185465488 0.548638582229614 2.0
0 0.228421822190285 3.0
14.185465488 0.438965141773224 4.0
26.925919629 0.494895398616791 5.0
41.111385117 0.542184472084045 6.0
41.111385117 0.526611328125 7.0
36.921893091 0.837696254253387 0.0
26.261414813 0.458658546209335 1.0
10.660478278 0.5060133934021 2.0
0 0.212253212928772 3.0
10.660478278 0.438376665115356 4.0
26.261414813 0.477657079696655 5.0
36.921893091 0.492435544729233 6.0
36.921893091 0.525696575641632 7.0
41.531234868 0.48067119717598 0.0
29.844730664 1.18256807327271 1.0
11.686504204 1.18841540813446 2.0
0 0.169182822108269 3.0
11.686504204 0.425159096717834 4.0
29.844730664 0.462562799453735 5.0
41.531234868 0.574762463569641 6.0
41.531234868 3.10449957847595 7.0
36.809020967 0.838907361030579 0.0
22.744117042 0.498662650585175 1.0
14.064903924 0.521833300590515 2.0
0 0.202075615525246 3.0
14.064903924 0.414070338010788 4.0
22.744117042 0.448627650737762 5.0
36.809020967 0.297870963811874 6.0
36.809020967 0.361833810806274 7.0
30.318323192 0.568775296211243 0.0
20.066163087 0.470534354448318 1.0
10.252160105 0.506835281848907 2.0
0 0.206245064735413 3.0
10.252160105 0.440004825592041 4.0
20.066163087 0.46474677324295 5.0
30.318323192 0.319919466972351 6.0
30.318323192 0.316286742687225 7.0
37.371827278 0.155422329902649 0.0
28.713218734 0.850831985473633 1.0
8.6586085434 0.860785007476807 2.0
0 0.109290763735771 3.0
8.6586085434 0.42778417468071 4.0
28.713218734 0.418621897697449 5.0
37.371827278 -0.111837916076183 6.0
37.371827278 0.915547609329224 7.0
37.183742215 0.812906444072723 0.0
23.390545172 0.541105449199677 1.0
13.793197043 0.571206510066986 2.0
0 0.219559550285339 3.0
13.793197043 0.415613323450089 4.0
23.390545172 0.455676674842834 5.0
37.183742215 0.51192831993103 6.0
37.183742215 0.372735321521759 7.0
34.629310928 0.761132299900055 0.0
23.313997557 0.49059322476387 1.0
11.315313371 0.547568917274475 2.0
0 0.235555797815323 3.0
11.315313371 0.46245801448822 4.0
23.313997557 0.486496567726135 5.0
34.629310928 0.537822544574738 6.0
34.629310928 0.546171724796295 7.0
42.830967498 0.801610887050629 0.0
30.470948882 0.445679157972336 1.0
12.360018616 0.494025975465775 2.0
0 0.200350180268288 3.0
12.360018616 0.460591524839401 4.0
30.470948882 0.471467971801758 5.0
42.830967498 0.345294207334518 6.0
42.830967498 0.52705854177475 7.0
34.694327372 0.838520586490631 0.0
21.852205513 0.459666609764099 1.0
12.842121859 0.518162369728088 2.0
0 0.24217700958252 3.0
12.842121859 0.459462255239487 4.0
21.852205513 0.494565367698669 5.0
34.694327372 0.549441397190094 6.0
34.694327372 0.489366173744202 7.0
37.611435874 0.680758833885193 0.0
23.212866181 0.44694721698761 1.0
14.398569692 0.485555827617645 2.0
0 0.195902183651924 3.0
14.398569692 0.451625823974609 4.0
23.212866181 0.473365128040314 5.0
37.611435874 0.23055824637413 6.0
37.611435874 0.501015543937683 7.0
45.529166488 0.80208283662796 0.0
32.298489175 0.504412710666656 1.0
13.230677313 0.537142038345337 2.0
0 0.236793756484985 3.0
13.230677313 0.449020445346832 4.0
32.298489175 0.495468020439148 5.0
45.529166488 0.353417158126831 6.0
45.529166488 0.42054945230484 7.0
40.26689599 0.635780274868011 0.0
29.798490664 0.438106626272202 1.0
10.468405325 0.472557872533798 2.0
0 0.195166438817978 3.0
10.468405325 0.453054785728455 4.0
29.798490664 0.483116209506989 5.0
40.26689599 0.23454350233078 6.0
40.26689599 0.351504117250443 7.0
39.577958661 0.705699801445007 0.0
25.13822578 0.38491952419281 1.0
14.439732881 0.413358777761459 2.0
0 0.188097313046455 3.0
14.439732881 0.450108766555786 4.0
25.13822578 0.48660272359848 5.0
39.577958661 0.456398516893387 6.0
39.577958661 0.559784948825836 7.0
34.564585703 0.1031673848629 0.0
21.369259648 0.970134854316711 1.0
13.195326055 0.973898589611053 2.0
0 0.206713736057281 3.0
13.195326055 0.416798621416092 4.0
21.369259648 0.480736911296844 5.0
34.564585703 1.61226439476013 6.0
34.564585703 13.1785802841187 7.0
34.747667205 0.780750632286072 0.0
26.117279278 0.417109549045563 1.0
8.6303879268 0.447264492511749 2.0
0 0.263707190752029 3.0
8.6303879268 0.446226865053177 4.0
26.117279278 0.501128673553467 5.0
34.747667205 0.173817381262779 6.0
34.747667205 0.352001130580902 7.0
44.886437438 0.777854263782501 0.0
29.455366783 0.487889856100082 1.0
15.431070655 0.541507244110107 2.0
0 0.209404528141022 3.0
15.431070655 0.441912233829498 4.0
29.455366783 0.480743050575256 5.0
44.886437438 0.400206416845322 6.0
44.886437438 0.571026206016541 7.0
34.458606901 0.797810792922974 0.0
21.914634893 0.499802261590958 1.0
12.543972008 0.551523983478546 2.0
0 0.229595735669136 3.0
12.543972008 0.452793300151825 4.0
21.914634893 0.482966959476471 5.0
34.458606901 0.507129073143005 6.0
34.458606901 0.53798633813858 7.0
44.985938301 0.41882136464119 0.0
31.318666065 0.129012525081635 1.0
13.667272236 0.132424503564835 2.0
0 0.203358441591263 3.0
13.667272236 0.383377522230148 4.0
31.318666065 0.422077089548111 5.0
44.985938301 1.73021841049194 6.0
44.985938301 0.297722816467285 7.0
36.561528733 0.714788317680359 0.0
26.365627821 0.514517903327942 1.0
10.195900911 0.544709384441376 2.0
0 0.207633256912231 3.0
10.195900911 0.406996697187424 4.0
26.365627821 0.444898247718811 5.0
36.561528733 0.218347415328026 6.0
36.561528733 0.396773308515549 7.0
35.585636345 0.764884173870087 0.0
21.938078684 0.508736312389374 1.0
13.647557661 0.54513418674469 2.0
0 0.232083410024643 3.0
13.647557661 0.439616352319717 4.0
21.938078684 0.488339155912399 5.0
35.585636345 0.28092622756958 6.0
35.585636345 0.383685231208801 7.0
34.748633645 0.838452875614166 0.0
24.774050599 0.504496097564697 1.0
9.9745830452 0.553761720657349 2.0
0 0.205702409148216 3.0
9.9745830452 0.441912770271301 4.0
24.774050599 0.465254724025726 5.0
34.748633645 0.495783120393753 6.0
34.748633645 0.417722940444946 7.0
33.997517461 0.739884197711945 0.0
23.648741818 0.482376635074615 1.0
10.348775643 0.536668360233307 2.0
0 0.201690495014191 3.0
10.348775643 0.442936450242996 4.0
23.648741818 0.480537950992584 5.0
33.997517461 0.382371306419373 6.0
33.997517461 0.593596696853638 7.0
31.337837162 0.784162521362305 0.0
18.725674883 0.481356918811798 1.0
12.612162279 0.522033929824829 2.0
0 0.242307260632515 3.0
12.612162279 0.461631298065186 4.0
18.725674883 0.509175002574921 5.0
31.337837162 0.298844397068024 6.0
31.337837162 0.37770688533783 7.0
43.914614211 0.797639966011047 0.0
31.443326284 0.486601561307907 1.0
12.471287926 0.543023109436035 2.0
0 0.231027245521545 3.0
12.471287926 0.456566870212555 4.0
31.443326284 0.491628110408783 5.0
43.914614211 0.488014161586761 6.0
43.914614211 0.5650235414505 7.0
45.186574618 0.805290222167969 0.0
30.950820377 0.524407565593719 1.0
14.23575424 0.556483626365662 2.0
0 0.243118852376938 3.0
14.23575424 0.447125226259232 4.0
30.950820377 0.491475224494934 5.0
45.186574618 0.42483589053154 6.0
45.186574618 0.451940327882767 7.0
35.268441876 0.6221644282341 0.0
23.396711515 0.456338375806808 1.0
11.871730361 0.479887366294861 2.0
0 0.174734458327293 3.0
11.871730361 0.417648822069168 4.0
23.396711515 0.453825712203979 5.0
35.268441876 0.0411452725529671 6.0
35.268441876 0.253070086240768 7.0
32.333098982 0.796977818012238 0.0
21.927214576 0.539125263690948 1.0
10.405884406 0.579756557941437 2.0
0 0.235182166099548 3.0
10.405884406 0.438152939081192 4.0
21.927214576 0.466354012489319 5.0
32.333098982 0.460986167192459 6.0
32.333098982 0.420589715242386 7.0
38.894207338 0.766211807727814 0.0
28.008989525 0.514622807502747 1.0
10.885217813 0.551746726036072 2.0
0 0.239541932940483 3.0
10.885217813 0.4331955909729 4.0
28.008989525 0.484436839818954 5.0
38.894207338 0.196327805519104 6.0
38.894207338 0.370959758758545 7.0
32.941924742 0.7943394780159 0.0
24.375613859 0.348554015159607 1.0
8.5663108832 0.382530987262726 2.0
0 0.285973608493805 3.0
8.5663108832 0.450339496135712 4.0
24.375613859 0.506501317024231 5.0
32.941924742 0.108090221881866 6.0
32.941924742 0.376112759113312 7.0
40.2164929 0.659590840339661 0.0
31.023294947 0.457521498203278 1.0
9.193197953 0.486428260803223 2.0
0 0.234955370426178 3.0
9.193197953 0.423990905284882 4.0
31.023294947 0.466635525226593 5.0
40.2164929 0.333583414554596 6.0
40.2164929 0.386746108531952 7.0
40.810463567 0.80683034658432 0.0
27.134577858 0.45493471622467 1.0
13.675885708 0.507707893848419 2.0
0 0.208212405443192 3.0
13.675885708 0.45153871178627 4.0
27.134577858 0.486934125423431 5.0
40.810463567 0.293470174074173 6.0
40.810463567 0.528074204921722 7.0
33.558410457 0.361560076475143 0.0
21.95745024 0.563504457473755 1.0
11.600960217 0.599802851676941 2.0
0 0.173850744962692 3.0
11.600960217 0.430446624755859 4.0
21.95745024 0.471129536628723 5.0
33.558410457 -0.0199100375175476 6.0
33.558410457 -0.0376835241913795 7.0
47.788896524 0.631999969482422 0.0
33.360420205 0.489188492298126 1.0
14.428476318 0.517815411090851 2.0
0 0.210557922720909 3.0
14.428476318 0.432338953018188 4.0
33.360420205 0.457118064165115 5.0
47.788896524 0.261306464672089 6.0
47.788896524 0.261991620063782 7.0
33.642782256 0.825193166732788 0.0
20.78231535 0.502078771591187 1.0
12.860466905 0.534224152565002 2.0
0 0.225138127803802 3.0
12.860466905 0.45270162820816 4.0
20.78231535 0.484966337680817 5.0
33.642782256 0.299549788236618 6.0
33.642782256 0.383639514446259 7.0
37.659684029 0.8466836810112 0.0
23.272895976 0.487178862094879 1.0
14.386788054 0.546540915966034 2.0
0 0.232217460870743 3.0
14.386788054 0.449869334697723 4.0
23.272895976 0.474632441997528 5.0
37.659684029 0.564521431922913 6.0
37.659684029 0.531721770763397 7.0
40.91969971 0.790131390094757 0.0
31.549891178 0.545541703701019 1.0
9.3698085329 0.56895512342453 2.0
-0.3917019807 0.229334518313408 3.0
8.9781065522 0.426971971988678 4.0
31.941593158 0.467986762523651 5.0
40.91969971 0.557514429092407 6.0
40.91969971 0.342424154281616 7.0
42.622265927 0.658822357654572 0.0
29.180561659 0.39220130443573 1.0
13.441704268 0.405523419380188 2.0
0 0.213094890117645 3.0
13.441704268 0.414789766073227 4.0
29.180561659 0.452495396137238 5.0
42.622265927 0.0905643105506897 6.0
42.622265927 0.413849860429764 7.0
33.946966747 0.716403245925903 0.0
24.555540255 0.483281433582306 1.0
9.3914264914 0.518898248672485 2.0
0 0.219684183597565 3.0
9.3914264914 0.437317371368408 4.0
24.555540255 0.468373000621796 5.0
33.946966747 0.278890997171402 6.0
33.946966747 0.415055125951767 7.0
36.409182194 0.817650854587555 0.0
20.90503326 0.523270726203918 1.0
15.504148935 0.561609387397766 2.0
0 0.221443548798561 3.0
15.504148935 0.442168325185776 4.0
20.90503326 0.468184262514114 5.0
36.409182194 0.46590194106102 6.0
36.409182194 0.494686514139175 7.0
40.593448001 0.849292278289795 0.0
30.912037848 0.469382286071777 1.0
9.6814101538 0.527931392192841 2.0
0 0.232673138380051 3.0
9.6814101538 0.443548291921616 4.0
30.912037848 0.510373532772064 5.0
40.593448001 0.463393747806549 6.0
40.593448001 0.503175795078278 7.0
41.715980637 0.556692898273468 0.0
31.196210855 0.32874721288681 1.0
10.519769782 0.347456783056259 2.0
0 0.223968252539635 3.0
10.519769782 0.425417810678482 4.0
31.196210855 0.463868021965027 5.0
41.715980637 -0.00443122163414955 6.0
41.715980637 0.193049341440201 7.0
33.77037363 0.676774740219116 0.0
22.237227098 0.224589169025421 1.0
11.533146533 0.254376500844955 2.0
0 0.229098111391068 3.0
11.533146533 0.415278255939484 4.0
22.237227098 0.445237457752228 5.0
33.77037363 1.43201887607574 6.0
33.77037363 -5.8397650718689e-05 7.0
46.074402277 0.402931928634644 0.0
32.676175596 0.481015115976334 1.0
13.398226681 0.517407119274139 2.0
0 0.216364696621895 3.0
13.398226681 0.445328414440155 4.0
32.676175596 0.490692436695099 5.0
46.074402277 0.0589327253401279 6.0
46.074402277 0.0768218487501144 7.0
40.510394826 0.359742701053619 0.0
26.352618036 0.40872597694397 1.0
14.15777679 0.446791589260101 2.0
0 0.274457454681396 3.0
14.15777679 0.464470744132996 4.0
26.352618036 0.534984707832336 5.0
40.510394826 -0.0082878265529871 6.0
40.510394826 0.0884052440524101 7.0
39.244329381 0.767924249172211 0.0
28.414205494 0.47465044260025 1.0
10.830123887 0.509310901165009 2.0
0 0.235845848917961 3.0
10.830123887 0.44011464715004 4.0
28.414205494 0.503638207912445 5.0
39.244329381 0.271658182144165 6.0
39.244329381 0.360188186168671 7.0
35.7171596 0.836733818054199 0.0
22.103003179 0.477286338806152 1.0
13.614156422 0.534376621246338 2.0
0 0.224237650632858 3.0
13.614156422 0.451998442411423 4.0
22.103003179 0.497269511222839 5.0
35.7171596 0.463051348924637 6.0
35.7171596 0.530500590801239 7.0
39.754714125 0.510714113712311 0.0
24.390436102 0.493158787488937 1.0
15.364278023 0.535041272640228 2.0
0 0.17920146882534 3.0
15.364278023 0.443864673376083 4.0
24.390436102 0.451549470424652 5.0
39.754714125 0.0535186901688576 6.0
39.754714125 0.07455213367939 7.0
41.796195011 0.77821934223175 0.0
28.322556682 0.502561986446381 1.0
13.473638329 0.552352130413055 2.0
0 0.2341118901968 3.0
13.473638329 0.449372172355652 4.0
28.322556682 0.493766993284225 5.0
41.796195011 0.511218547821045 6.0
41.796195011 0.533612251281738 7.0
29.980660557 0.791936993598938 0.0
20.698164237 0.484211564064026 1.0
9.2824963201 0.540518879890442 2.0
0 0.234925523400307 3.0
9.2824963201 0.453288316726685 4.0
20.698164237 0.503138661384583 5.0
29.980660557 0.504876792430878 6.0
29.980660557 0.564027547836304 7.0
36.241260622 0.855786204338074 0.0
22.638938794 0.514446020126343 1.0
13.602321828 0.554309487342834 2.0
0 0.220699414610863 3.0
13.602321828 0.439164817333221 4.0
22.638938794 0.475201845169067 5.0
36.241260622 0.430722028017044 6.0
36.241260622 0.466972321271896 7.0
38.129703213 0.784092128276825 0.0
27.128465029 0.50480318069458 1.0
11.001238184 0.552473545074463 2.0
0 0.22157022356987 3.0
11.001238184 0.447287768125534 4.0
27.128465029 0.483276903629303 5.0
38.129703213 0.497900545597076 6.0
38.129703213 0.502704739570618 7.0
36.91522515 0.747430086135864 0.0
22.73211488 0.51236355304718 1.0
14.18311027 0.553923308849335 2.0
0 0.224657788872719 3.0
14.18311027 0.444507122039795 4.0
22.73211488 0.502201914787292 5.0
36.91522515 0.406154006719589 6.0
36.91522515 0.591113328933716 7.0
38.102908751 0.786026120185852 0.0
24.659837316 0.495492219924927 1.0
13.443071435 0.551822364330292 2.0
0 0.19953291118145 3.0
13.443071435 0.433183819055557 4.0
24.659837316 0.47648149728775 5.0
38.102908751 0.422329992055893 6.0
38.102908751 0.57693886756897 7.0
38.34829669 0.689752519130707 0.0
28.277225983 0.479798257350922 1.0
10.071070707 0.523808717727661 2.0
0 0.217210829257965 3.0
10.071070707 0.456964492797852 4.0
28.277225983 0.491562336683273 5.0
38.34829669 0.303100556135178 6.0
38.34829669 0.583636879920959 7.0
31.13885609 0.828653693199158 0.0
20.686107563 0.52132785320282 1.0
10.452748527 0.551999926567078 2.0
0 0.202164590358734 3.0
10.452748527 0.429225265979767 4.0
20.686107563 0.458648681640625 5.0
31.13885609 0.407663404941559 6.0
31.13885609 0.400772571563721 7.0
35.130346793 0.231792241334915 0.0
24.419018025 0.135346367955208 1.0
10.711328768 0.145855277776718 2.0
0 0.212332427501678 3.0
10.711328768 0.406260430812836 4.0
24.419018025 0.451832205057144 5.0
35.130346793 2.14815354347229 6.0
35.130346793 2.69133973121643 7.0
37.970631847 0.882675647735596 0.0
26.545844038 0.541614472866058 1.0
11.424787809 0.580301702022552 2.0
0 0.232728838920593 3.0
11.424787809 0.43044438958168 4.0
26.545844038 0.461890876293182 5.0
37.970631847 0.524353444576263 6.0
37.970631847 0.275272876024246 7.0
37.012414901 0.785380125045776 0.0
22.13356734 0.532437562942505 1.0
14.87884756 0.571928918361664 2.0
0 0.234910443425179 3.0
14.87884756 0.436798989772797 4.0
22.13356734 0.486879914999008 5.0
37.012414901 0.29100438952446 6.0
37.012414901 0.315387010574341 7.0
36.779265883 0.786756575107574 0.0
23.404145317 0.498701304197311 1.0
13.375120566 0.527487993240356 2.0
0 0.179477170109749 3.0
13.375120566 0.435167253017426 4.0
23.404145317 0.441477417945862 5.0
36.779265883 0.292698740959167 6.0
36.779265883 0.371560633182526 7.0
42.665291585 0.835052907466888 0.0
29.517542924 0.328700840473175 1.0
13.147748661 0.360368579626083 2.0
0 0.23216313123703 3.0
13.147748661 0.412986993789673 4.0
29.517542924 0.456535547971725 5.0
42.665291585 0.307083904743195 6.0
42.665291585 0.169484212994576 7.0
35.413046011 0.778972208499908 0.0
20.7862186 0.471845209598541 1.0
14.626827411 0.532719016075134 2.0
0 0.239923730492592 3.0
14.626827411 0.460447013378143 4.0
20.7862186 0.504892945289612 5.0
35.413046011 0.556623458862305 6.0
35.413046011 0.500726282596588 7.0
35.966339402 0.838905990123749 0.0
24.209055046 0.515986859798431 1.0
11.757284356 0.558837413787842 2.0
0 0.212583720684052 3.0
11.757284356 0.438374102115631 4.0
24.209055046 0.467411279678345 5.0
35.966339402 0.489383667707443 6.0
35.966339402 0.471481651067734 7.0
39.418493945 0.773100435733795 0.0
25.880638068 0.476518869400024 1.0
13.537855877 0.525939702987671 2.0
0 0.208407908678055 3.0
13.537855877 0.447026193141937 4.0
25.880638068 0.491723120212555 5.0
39.418493945 0.307085871696472 6.0
39.418493945 0.47972360253334 7.0
32.151549765 0.532033622264862 0.0
22.152388929 0.470560908317566 1.0
9.9991608362 0.502604782581329 2.0
0 0.218607828021049 3.0
9.9991608362 0.435739487409592 4.0
22.152388929 0.477415293455124 5.0
32.151549765 0.224642693996429 6.0
32.151549765 0.224260151386261 7.0
31.845645841 0.79373037815094 0.0
21.472593563 0.538468182086945 1.0
10.373052277 0.583850026130676 2.0
0 0.243438094854355 3.0
10.373052277 0.428942620754242 4.0
21.472593563 0.478391081094742 5.0
31.845645841 0.435470104217529 6.0
31.845645841 0.456660658121109 7.0
40.557128571 0.642606794834137 0.0
29.931774079 0.458705574274063 1.0
10.625354492 0.49876606464386 2.0
0 0.220759630203247 3.0
10.625354492 0.453893929719925 4.0
29.931774079 0.491826981306076 5.0
40.557128571 0.281459778547287 6.0
40.557128571 0.550151467323303 7.0
36.793774817 0.678914785385132 0.0
26.241358064 0.467928230762482 1.0
10.552416753 0.495481967926025 2.0
0 0.225812911987305 3.0
10.552416753 0.422163724899292 4.0
26.241358064 0.475648134946823 5.0
36.793774817 0.224385097622871 6.0
36.793774817 0.320366889238358 7.0
41.333576969 0.809330403804779 0.0
27.007286594 0.474648237228394 1.0
14.326290374 0.52199250459671 2.0
0 0.199088141322136 3.0
14.326290374 0.441621571779251 4.0
27.007286594 0.486652612686157 5.0
41.333576969 0.286911129951477 6.0
41.333576969 0.434594005346298 7.0
43.791455275 0.708551585674286 0.0
29.094231349 0.477465391159058 1.0
14.697223926 0.515256583690643 2.0
0 0.217513531446457 3.0
14.697223926 0.452193707227707 4.0
29.094231349 0.492011368274689 5.0
43.791455275 0.282183140516281 6.0
43.791455275 0.591366350650787 7.0
41.481697076 0.797793328762054 0.0
26.568629648 0.513566553592682 1.0
14.913067428 0.549324214458466 2.0
0 0.221061259508133 3.0
14.913067428 0.437928825616837 4.0
26.568629648 0.503415107727051 5.0
41.481697076 0.524230659008026 6.0
41.481697076 0.417155981063843 7.0
35.532774621 0.769791781902313 0.0
25.55101033 0.40546989440918 1.0
9.9817642912 0.43705627322197 2.0
0 0.212361067533493 3.0
9.9817642912 0.45470780134201 4.0
25.55101033 0.495052993297577 5.0
35.532774621 0.551842510700226 6.0
35.532774621 0.456487029790878 7.0
34.894006808 0.818658947944641 0.0
24.580727451 0.47858202457428 1.0
10.313279357 0.533339202404022 2.0
0 0.200773850083351 3.0
10.313279357 0.44056248664856 4.0
24.580727451 0.473890781402588 5.0
34.894006808 0.386016488075256 6.0
34.894006808 0.51338791847229 7.0
38.893810297 0.80176442861557 0.0
25.470645748 0.437883168458939 1.0
13.423164549 0.476764410734177 2.0
0 0.216817662119865 3.0
13.423164549 0.445521414279938 4.0
25.470645748 0.492699980735779 5.0
38.893810297 0.562205970287323 6.0
38.893810297 0.501301646232605 7.0
39.624663522 0.720480620861053 0.0
30.244927686 0.469310879707336 1.0
9.3797358363 0.52405446767807 2.0
-0.51314313752 0.182924702763557 3.0
8.8665926988 0.440292418003082 4.0
30.758070824 0.465507686138153 5.0
39.624663522 0.325776219367981 6.0
39.624663522 0.602838337421417 7.0
42.516231091 0.107762150466442 0.0
30.477530163 1.74265503883362 1.0
12.038700928 1.77203917503357 2.0
0 0.115819543600082 3.0
12.038700928 0.405315697193146 4.0
30.477530163 0.439746290445328 5.0
42.516231091 0.00878572463989258 6.0
42.516231091 15.7744998931885 7.0
38.374768577 0.935347497463226 0.0
25.100410303 0.146507918834686 1.0
13.274358274 0.185639381408691 2.0
0 0.273235440254211 3.0
13.274358274 0.425538003444672 4.0
25.100410303 0.481759160757065 5.0
38.374768577 3.80069637298584 6.0
38.374768577 -0.0799577087163925 7.0
33.351459508 0.653099775314331 0.0
24.24183423 0.448150217533112 1.0
9.1096252774 0.485326647758484 2.0
0 0.202369466423988 3.0
9.1096252774 0.45039826631546 4.0
24.24183423 0.491240680217743 5.0
33.351459508 0.24879164993763 6.0
33.351459508 0.526399731636047 7.0
37.08812014 0.823539137840271 0.0
24.469772372 0.445278763771057 1.0
12.618347768 0.488136410713196 2.0
0 0.21197646856308 3.0
12.618347768 0.457527458667755 4.0
24.469772372 0.498036444187164 5.0
37.08812014 0.42143976688385 6.0
37.08812014 0.536540627479553 7.0
44.445084649 0.801821172237396 0.0
30.066790167 0.468158304691315 1.0
14.378294481 0.523413181304932 2.0
0 0.229920834302902 3.0
14.378294481 0.446336925029755 4.0
30.066790167 0.512181401252747 5.0
44.445084649 0.521897971630096 6.0
44.445084649 0.548552751541138 7.0
37.22322104 0.783655762672424 0.0
25.951146113 0.524165987968445 1.0
11.272074927 0.561542153358459 2.0
0 0.215939819812775 3.0
11.272074927 0.433879733085632 4.0
25.951146113 0.479905605316162 5.0
37.22322104 0.433674424886703 6.0
37.22322104 0.449913144111633 7.0
44.605944609 0.746692478656769 0.0
29.214608515 0.386610478162766 1.0
15.391336093 0.421648353338242 2.0
0 0.283472031354904 3.0
15.391336093 0.455490589141846 4.0
29.214608515 0.516548991203308 5.0
44.605944609 0.201327949762344 6.0
44.605944609 0.315508514642715 7.0
39.553698498 0.793002188205719 0.0
28.17711576 0.513224542140961 1.0
11.376582739 0.552668929100037 2.0
-1.2942488147 0.194055020809174 3.0
10.082333924 0.423885554075241 4.0
29.471364574 0.453202277421951 5.0
39.553698498 0.411467522382736 6.0
39.553698498 0.408542811870575 7.0
43.287363281 0.298907071352005 0.0
32.742330647 0.457957804203033 1.0
10.545032634 0.49205270409584 2.0
0 0.216464668512344 3.0
10.545032634 0.432183533906937 4.0
32.742330647 0.479440659284592 5.0
43.287363281 0.063571110367775 6.0
43.287363281 0.0830489322543144 7.0
33.992156471 0.853398740291595 0.0
23.696427762 0.528397977352142 1.0
10.29572871 0.5522580742836 2.0
0 0.208763971924782 3.0
10.29572871 0.416666448116302 4.0
23.696427762 0.464191973209381 5.0
33.992156471 0.544486820697784 6.0
33.992156471 0.329224973917007 7.0
36.860093248 0.491945624351501 0.0
25.387893295 0.463874369859695 1.0
11.472199952 0.497593820095062 2.0
0 0.212247997522354 3.0
11.472199952 0.450315713882446 4.0
25.387893295 0.492523431777954 5.0
36.860093248 0.253744572401047 6.0
36.860093248 0.230671539902687 7.0
39.672962738 0.802188098430634 0.0
26.143344656 0.519133031368256 1.0
13.529618083 0.558072149753571 2.0
0 0.240838795900345 3.0
13.529618083 0.438046336174011 4.0
26.143344656 0.493671596050262 5.0
39.672962738 0.619565665721893 6.0
39.672962738 0.52202433347702 7.0
32.8535328 0.818777441978455 0.0
22.510390494 0.510645925998688 1.0
10.343142306 0.551082968711853 2.0
0 0.244649052619934 3.0
10.343142306 0.45677325129509 4.0
22.510390494 0.496970593929291 5.0
32.8535328 0.522443294525146 6.0
32.8535328 0.529251992702484 7.0
36.422494688 0.814515650272369 0.0
22.927536178 0.565119743347168 1.0
13.49495851 0.611795663833618 2.0
0 0.252452313899994 3.0
13.49495851 0.424361705780029 4.0
22.927536178 0.471203505992889 5.0
36.422494688 0.433085888624191 6.0
36.422494688 0.419890999794006 7.0
32.082396771 0.523463785648346 0.0
23.483778533 0.118190005421638 1.0
8.5986182376 0.142070055007935 2.0
0 0.250994294881821 3.0
8.5986182376 0.407981216907501 4.0
23.483778533 0.475044399499893 5.0
32.082396771 1.75894832611084 6.0
32.082396771 0.195892497897148 7.0
43.742460686 0.807655513286591 0.0
30.22029377 0.514916896820068 1.0
13.522166916 0.543078601360321 2.0
0 0.196560308337212 3.0
13.522166916 0.444112032651901 4.0
30.22029377 0.464447468519211 5.0
43.742460686 0.40048959851265 6.0
43.742460686 0.377436846494675 7.0
37.830554934 0.153893768787384 0.0
26.515637118 0.490710586309433 1.0
11.314917816 0.50986236333847 2.0
0 0.13872617483139 3.0
11.314917816 0.429201632738113 4.0
26.515637118 0.455825746059418 5.0
37.830554934 0.049899835139513 6.0
37.830554934 -0.0239836275577545 7.0
45.388745763 0.819045722484589 0.0
30.82392363 0.44107449054718 1.0
14.564822134 0.481811046600342 2.0
0 0.211774438619614 3.0
14.564822134 0.445371866226196 4.0
30.82392363 0.478679835796356 5.0
45.388745763 0.54824036359787 6.0
45.388745763 0.512443602085114 7.0
40.010376396 0.833685517311096 0.0
29.307244164 0.516844630241394 1.0
10.703132231 0.555249810218811 2.0
0 0.245881825685501 3.0
10.703132231 0.448737919330597 4.0
29.307244164 0.495409727096558 5.0
40.010376396 0.505243003368378 6.0
40.010376396 0.527343928813934 7.0
45.566990187 0.667127907276154 0.0
30.534595735 0.471810549497604 1.0
15.032394453 0.502224266529083 2.0
0 0.205851137638092 3.0
15.032394453 0.433914601802826 4.0
30.534595735 0.49374932050705 5.0
45.566990187 0.222603261470795 6.0
45.566990187 0.349613875150681 7.0
32.300808889 0.192198380827904 0.0
23.352129325 0.0638346076011658 1.0
8.9486795641 0.0579142421483994 2.0
0 0.185486271977425 3.0
8.9486795641 0.378392189741135 4.0
23.352129325 0.433267891407013 5.0
32.300808889 2.042076587677 6.0
32.300808889 0.207809910178185 7.0
36.805463378 0.349518835544586 0.0
21.194272929 0.448767900466919 1.0
15.611190449 0.477525472640991 2.0
0 0.15731754899025 3.0
15.611190449 0.430160015821457 4.0
21.194272929 0.448507726192474 5.0
36.805463378 0.271630823612213 6.0
36.805463378 0.108321413397789 7.0
45.039727992 0.823137104511261 0.0
30.80326593 0.47361347079277 1.0
14.236462062 0.52825129032135 2.0
-0.8432667817 0.206390202045441 3.0
13.39319528 0.448150426149368 4.0
31.646532712 0.469447821378708 5.0
45.039727992 0.500889718532562 6.0
45.039727992 0.534724056720734 7.0
36.669966877 0.756607711315155 0.0
26.160739166 0.504479646682739 1.0
10.509227711 0.531142711639404 2.0
0 0.192024558782578 3.0
10.509227711 0.43343597650528 4.0
26.160739166 0.463618725538254 5.0
36.669966877 0.280248939990997 6.0
36.669966877 0.369796633720398 7.0
40.401972767 0.77946412563324 0.0
30.207092791 0.505314826965332 1.0
10.194879977 0.547717750072479 2.0
0 0.25090354681015 3.0
10.194879977 0.459393560886383 4.0
30.207092791 0.494312226772308 5.0
40.401972767 0.488574683666229 6.0
40.401972767 0.340999901294708 7.0
41.491221092 0.861725389957428 0.0
30.356832429 0.465277820825577 1.0
11.134388663 0.522004663944244 2.0
0 0.217917054891586 3.0
11.134388663 0.45466148853302 4.0
30.356832429 0.494959682226181 5.0
41.491221092 0.436310738325119 6.0
41.491221092 0.498245388269424 7.0
32.092723767 0.775565624237061 0.0
23.053601154 0.461987733840942 1.0
9.0391226128 0.501230835914612 2.0
0 0.23835700750351 3.0
9.0391226128 0.452335029840469 4.0
23.053601154 0.482768714427948 5.0
32.092723767 0.370695114135742 6.0
32.092723767 0.322657376527786 7.0
39.133472273 0.65519118309021 0.0
30.010976118 0.427304804325104 1.0
9.1224961547 0.463709592819214 2.0
0 0.246744215488434 3.0
9.1224961547 0.444575130939484 4.0
30.010976118 0.495782971382141 5.0
39.133472273 0.176375344395638 6.0
39.133472273 0.240880891680717 7.0
37.053979998 0.496413469314575 0.0
23.198859944 0.476689249277115 1.0
13.855120054 0.512836933135986 2.0
0 0.19786012172699 3.0
13.855120054 0.434638619422913 4.0
23.198859944 0.460710287094116 5.0
37.053979998 0.239212170243263 6.0
37.053979998 0.172073602676392 7.0
39.864494306 0.798298060894012 0.0
30.858392243 0.480160474777222 1.0
9.0061020631 0.514791011810303 2.0
0 0.244815409183502 3.0
9.0061020631 0.442190319299698 4.0
30.858392243 0.505454778671265 5.0
39.864494306 0.360647082328796 6.0
39.864494306 0.32193511724472 7.0
39.054921308 0.707032859325409 0.0
29.468453199 0.467125564813614 1.0
9.5864681089 0.508595764636993 2.0
0 0.19974459707737 3.0
9.5864681089 0.435658097267151 4.0
29.468453199 0.489574193954468 5.0
39.054921308 0.290908753871918 6.0
39.054921308 0.570812821388245 7.0
33.466015237 0.352787017822266 0.0
23.587793621 0.483494371175766 1.0
9.8782216161 0.520083904266357 2.0
0 0.204697132110596 3.0
9.8782216161 0.450554132461548 4.0
23.587793621 0.481909811496735 5.0
33.466015237 0.244381189346313 6.0
33.466015237 0.106617256999016 7.0
37.653527764 0.844609260559082 0.0
25.326112836 0.535805642604828 1.0
12.327414928 0.566404759883881 2.0
0 0.221628084778786 3.0
12.327414928 0.422803848981857 4.0
25.326112836 0.471214354038239 5.0
37.653527764 0.344829797744751 6.0
37.653527764 0.294674694538116 7.0
37.577039283 0.77074807882309 0.0
25.072263055 0.497348815202713 1.0
12.504776228 0.54113233089447 2.0
0 0.217451497912407 3.0
12.504776228 0.444974899291992 4.0
25.072263055 0.478440463542938 5.0
37.577039283 0.309034466743469 6.0
37.577039283 0.577344417572021 7.0
36.318307627 0.850686490535736 0.0
24.535944301 0.491100549697876 1.0
11.782363326 0.525781989097595 2.0
0 0.238496959209442 3.0
11.782363326 0.442957311868668 4.0
24.535944301 0.487862557172775 5.0
36.318307627 0.414986789226532 6.0
36.318307627 0.309123456478119 7.0
36.801464194 0.822178304195404 0.0
24.290044341 0.459740847349167 1.0
12.511419853 0.500440657138824 2.0
0 0.202016532421112 3.0
12.511419853 0.436940908432007 4.0
24.290044341 0.488248407840729 5.0
36.801464194 0.556570410728455 6.0
36.801464194 0.489524275064468 7.0
39.56095567 0.728096544742584 0.0
28.15773783 0.526086270809174 1.0
11.40321784 0.560514688491821 2.0
0 0.224995955824852 3.0
11.40321784 0.424825847148895 4.0
28.15773783 0.47204601764679 5.0
39.56095567 0.277188926935196 6.0
39.56095567 0.308047026395798 7.0
40.443539209 0.725812375545502 0.0
25.360759604 0.482755750417709 1.0
15.082779605 0.515860557556152 2.0
0 0.197520405054092 3.0
15.082779605 0.420302093029022 4.0
25.360759604 0.468140482902527 5.0
40.443539209 0.303331285715103 6.0
40.443539209 0.337807387113571 7.0
40.316920108 0.806762635707855 0.0
27.366582023 0.488151788711548 1.0
12.950338085 0.544325768947601 2.0
0 0.231440186500549 3.0
12.950338085 0.460821926593781 4.0
27.366582023 0.482191771268845 5.0
40.316920108 0.507617950439453 6.0
40.316920108 0.563139796257019 7.0
33.579405458 0.164192318916321 0.0
23.019033323 0.987584590911865 1.0
10.560372135 0.987498879432678 2.0
0 0.140216618776321 3.0
10.560372135 0.423770993947983 4.0
23.019033323 0.443666219711304 5.0
33.579405458 -0.0555683746933937 6.0
33.579405458 1.49115216732025 7.0
34.562810956 0.47293958067894 0.0
23.168439176 0.451626062393188 1.0
11.394371781 0.482025057077408 2.0
0 0.172515615820885 3.0
11.394371781 0.427609920501709 4.0
23.168439176 0.457060217857361 5.0
34.562810956 0.317228317260742 6.0
34.562810956 0.159909710288048 7.0
37.518824119 0.84033340215683 0.0
25.014720915 0.506138563156128 1.0
12.504103204 0.542189478874207 2.0
0 0.22418786585331 3.0
12.504103204 0.42998868227005 4.0
25.014720915 0.463923782110214 5.0
37.518824119 0.356952130794525 6.0
37.518824119 0.30832314491272 7.0
39.547705636 0.743907511234283 0.0
26.235912464 0.479886144399643 1.0
13.311793172 0.5180703997612 2.0
0 0.204040840268135 3.0
13.311793172 0.438969701528549 4.0
26.235912464 0.467820525169373 5.0
39.547705636 0.273032248020172 6.0
39.547705636 0.356776595115662 7.0
42.834945783 0.797136664390564 0.0
27.69423425 0.497721791267395 1.0
15.140711533 0.546959459781647 2.0
0 0.231731474399567 3.0
15.140711533 0.438295811414719 4.0
27.69423425 0.492678582668304 5.0
42.834945783 0.49882447719574 6.0
42.834945783 0.521791040897369 7.0
38.473622953 0.71720689535141 0.0
25.667479717 0.488111138343811 1.0
12.806143237 0.531570792198181 2.0
0 0.188634425401688 3.0
12.806143237 0.435577273368835 4.0
25.667479717 0.480162739753723 5.0
38.473622953 0.293171912431717 6.0
38.473622953 0.606328129768372 7.0
45.238523457 0.783157229423523 0.0
31.33795643 0.525926113128662 1.0
13.900567026 0.545917570590973 2.0
0 0.201349630951881 3.0
13.900567026 0.421518057584763 4.0
31.33795643 0.461016595363617 5.0
45.238523457 0.362048387527466 6.0
45.238523457 0.277319222688675 7.0
36.266282562 0.797958791255951 0.0
23.153527457 0.475134909152985 1.0
13.112755105 0.533487558364868 2.0
0 0.230640649795532 3.0
13.112755105 0.465743839740753 4.0
23.153527457 0.505384564399719 5.0
36.266282562 0.473486363887787 6.0
36.266282562 0.556395411491394 7.0
37.845233948 0.833093523979187 0.0
24.957293945 0.499535381793976 1.0
12.887940003 0.530216336250305 2.0
0 0.220720082521439 3.0
12.887940003 0.448181062936783 4.0
24.957293945 0.483818739652634 5.0
37.845233948 0.326466232538223 6.0
37.845233948 0.37402155995369 7.0
40.414778929 0.556234896183014 0.0
24.767056635 0.333821088075638 1.0
15.647722294 0.345989435911179 2.0
0 0.22102415561676 3.0
15.647722294 0.438133656978607 4.0
24.767056635 0.472770690917969 5.0
40.414778929 -0.0216581858694553 6.0
40.414778929 0.23974397778511 7.0
47.104382604 0.857323527336121 0.0
31.371428063 0.493219316005707 1.0
15.732954541 0.546459376811981 2.0
0 0.218034461140633 3.0
15.732954541 0.43857741355896 4.0
31.371428063 0.480269581079483 5.0
47.104382604 0.37899586558342 6.0
47.104382604 0.503019392490387 7.0
37.557508388 0.470868796110153 0.0
22.881195893 0.475994110107422 1.0
14.676312494 0.506914794445038 2.0
0 0.192366659641266 3.0
14.676312494 0.42032253742218 4.0
22.881195893 0.47070175409317 5.0
37.557508388 0.00154859572649002 6.0
37.557508388 0.107718147337437 7.0
40.532317042 0.835288286209106 0.0
25.290512841 0.470301926136017 1.0
15.241804202 0.52490109205246 2.0
0 0.212425827980042 3.0
15.241804202 0.440819591283798 4.0
25.290512841 0.492089807987213 5.0
40.532317042 0.34355902671814 6.0
40.532317042 0.469875246286392 7.0
32.594348946 0.478582203388214 0.0
19.62547534 0.436318188905716 1.0
12.968873606 0.472652792930603 2.0
0 0.165714010596275 3.0
12.968873606 0.451954782009125 4.0
19.62547534 0.451986908912659 5.0
32.594348946 0.284762620925903 6.0
32.594348946 0.483106672763824 7.0
30.392342502 0.785122990608215 0.0
20.487642954 0.394538402557373 1.0
9.9046995479 0.429537504911423 2.0
0 0.205640405416489 3.0
9.9046995479 0.452246218919754 4.0
20.487642954 0.488789618015289 5.0
30.392342502 0.559096574783325 6.0
30.392342502 0.464942425489426 7.0
40.607017947 0.744522929191589 0.0
29.93470242 0.472891181707382 1.0
10.672315527 0.525819063186646 2.0
0 0.216996148228645 3.0
10.672315527 0.449378609657288 4.0
29.93470242 0.476381599903107 5.0
40.607017947 0.319554626941681 6.0
40.607017947 0.592917799949646 7.0
39.11337914 0.153473138809204 0.0
28.397508882 0.29890251159668 1.0
10.715870258 0.324839651584625 2.0
0 0.220989137887955 3.0
10.715870258 0.428418070077896 4.0
28.397508882 0.460609644651413 5.0
39.11337914 0.545633673667908 6.0
39.11337914 1.18321931362152 7.0
35.280627531 0.816200137138367 0.0
25.522950121 0.536251604557037 1.0
9.7576774108 0.581959307193756 2.0
0 0.233102843165398 3.0
9.7576774108 0.421558827161789 4.0
25.522950121 0.462329506874084 5.0
35.280627531 0.346254706382751 6.0
35.280627531 0.30106794834137 7.0
32.847447182 0.789288640022278 0.0
19.105381186 0.324578583240509 1.0
13.742065995 0.358002245426178 2.0
0 0.206191539764404 3.0
13.742065995 0.457682609558105 4.0
19.105381186 0.494605958461761 5.0
32.847447182 0.585290789604187 6.0
32.847447182 0.435584306716919 7.0
34.14972673 0.141642183065414 0.0
19.081196776 0.942439913749695 1.0
15.068529955 0.958122313022614 2.0
0 0.13691671192646 3.0
15.068529955 0.423149526119232 4.0
19.081196776 0.437961280345917 5.0
34.14972673 -0.07948337495327 6.0
34.14972673 1.28819298744202 7.0
36.823986173 0.73734587430954 0.0
27.983721771 0.485915660858154 1.0
8.8402644018 0.518760800361633 2.0
0 0.190528646111488 3.0
8.8402644018 0.420883774757385 4.0
27.983721771 0.452564656734467 5.0
36.823986173 0.306255519390106 6.0
36.823986173 0.336701631546021 7.0
43.615049092 0.193106591701508 0.0
31.355429376 0.503526151180267 1.0
12.259619715 0.533508241176605 2.0
0 0.180047810077667 3.0
12.259619715 0.431888163089752 4.0
31.355429376 0.473054319620132 5.0
43.615049092 0.0751299634575844 6.0
43.615049092 -0.0203849785029888 7.0
39.316465342 0.167208656668663 0.0
25.189593371 0.825720131397247 1.0
14.126871971 0.849033176898956 2.0
0 0.122839465737343 3.0
14.126871971 0.438793987035751 4.0
25.189593371 0.451445549726486 5.0
39.316465342 -0.191388815641403 6.0
39.316465342 2.2212233543396 7.0
36.676849778 0.556848466396332 0.0
26.033044381 0.429980218410492 1.0
10.643805397 0.447683960199356 2.0
0 0.180013179779053 3.0
10.643805397 0.415343642234802 4.0
26.033044381 0.453181892633438 5.0
36.676849778 -0.0115641998127103 6.0
36.676849778 0.265274077653885 7.0
36.163160136 0.662860095500946 0.0
20.77667961 0.445120632648468 1.0
15.386480527 0.476068615913391 2.0
0 0.225154057145119 3.0
15.386480527 0.432472884654999 4.0
20.77667961 0.473351925611496 5.0
36.163160136 0.17772513628006 6.0
36.163160136 0.286793619394302 7.0
40.305175496 0.860043466091156 0.0
26.598094238 0.473178565502167 1.0
13.707081258 0.528907299041748 2.0
0 0.227568432688713 3.0
13.707081258 0.45763772726059 4.0
26.598094238 0.491118878126144 5.0
40.305175496 0.370219707489014 6.0
40.305175496 0.457871496677399 7.0
34.293236222 0.786695063114166 0.0
19.681914121 0.453191220760345 1.0
14.611322101 0.500743627548218 2.0
0 0.222016662359238 3.0
14.611322101 0.45496666431427 4.0
19.681914121 0.495029300451279 5.0
34.293236222 0.512635171413422 6.0
34.293236222 0.527366757392883 7.0
38.347299924 0.810702979564667 0.0
24.696601669 0.446979641914368 1.0
13.650698255 0.489473074674606 2.0
0 0.21877521276474 3.0
13.650698255 0.448098063468933 4.0
24.696601669 0.488399475812912 5.0
38.347299924 0.553648471832275 6.0
38.347299924 0.511225283145905 7.0
39.036910781 0.833088397979736 0.0
30.172693912 0.486397504806519 1.0
8.8642168691 0.528972625732422 2.0
0 0.222864761948586 3.0
8.8642168691 0.451327860355377 4.0
30.172693912 0.504492104053497 5.0
39.036910781 0.34694230556488 6.0
39.036910781 0.446685642004013 7.0
36.237500285 0.834708869457245 0.0
23.386338638 0.42783111333847 1.0
12.851161647 0.461830347776413 2.0
0 0.202673763036728 3.0
12.851161647 0.450617849826813 4.0
23.386338638 0.487475603818893 5.0
36.237500285 0.495664745569229 6.0
36.237500285 0.517594397068024 7.0
36.900540753 0.689771294593811 0.0
25.827912025 0.469310343265533 1.0
11.072628728 0.512154281139374 2.0
0 0.199667632579803 3.0
11.072628728 0.450947254896164 4.0
25.827912025 0.457955569028854 5.0
36.900540753 0.282003670930862 6.0
36.900540753 0.515583992004395 7.0
29.897899633 0.371837884187698 0.0
21.013378654 0.478151887655258 1.0
8.8845209788 0.514681279659271 2.0
0 0.210011467337608 3.0
8.8845209788 0.439906865358353 4.0
21.013378654 0.47275573015213 5.0
29.897899633 0.29294615983963 6.0
29.897899633 0.132280513644218 7.0
41.071373967 0.85857355594635 0.0
26.148963023 0.437184631824493 1.0
14.922410944 0.482749044895172 2.0
0 0.204023033380508 3.0
14.922410944 0.459974974393845 4.0
26.148963023 0.47003909945488 5.0
41.071373967 0.38837605714798 6.0
41.071373967 0.49370139837265 7.0
37.123617582 0.623152792453766 0.0
24.027785174 0.452051132917404 1.0
13.095832408 0.486309319734573 2.0
0 0.214345276355743 3.0
13.095832408 0.438201785087585 4.0
24.027785174 0.471274733543396 5.0
37.123617582 0.0859622210264206 6.0
37.123617582 0.258602350950241 7.0
38.348981744 0.212327629327774 0.0
22.510464493 0.232768326997757 1.0
15.838517252 0.250914663076401 2.0
0 0.22936649620533 3.0
15.838517252 0.439298152923584 4.0
22.510464493 0.45872363448143 5.0
38.348981744 0.677861034870148 6.0
38.348981744 0.0388529226183891 7.0
37.897829023 0.784526169300079 0.0
28.975694369 0.517520487308502 1.0
8.9221346538 0.556415379047394 2.0
0 0.248773604631424 3.0
8.9221346538 0.454890072345734 4.0
28.975694369 0.483397006988525 5.0
37.897829023 0.544650018215179 6.0
37.897829023 0.515705525875092 7.0
32.839543182 0.841200828552246 0.0
23.33116807 0.315212339162827 1.0
9.508375112 0.350408136844635 2.0
0 0.21237625181675 3.0
9.508375112 0.460321485996246 4.0
23.33116807 0.498431771993637 5.0
32.839543182 0.58209240436554 6.0
32.839543182 0.451024889945984 7.0
45.814256844 0.682292997837067 0.0
33.908219787 0.488246649503708 1.0
11.906037057 0.54346776008606 2.0
0 0.196275174617767 3.0
11.906037057 0.451681643724442 4.0
33.908219787 0.487065851688385 5.0
45.814256844 0.41088992357254 6.0
45.814256844 0.642491698265076 7.0
38.439958634 0.799407422542572 0.0
26.743506222 0.429951220750809 1.0
11.696452411 0.454189002513885 2.0
0 0.214524537324905 3.0
11.696452411 0.405736982822418 4.0
26.743506222 0.446028083562851 5.0
38.439958634 0.102512009441853 6.0
38.439958634 0.476474016904831 7.0
33.022424116 0.708929657936096 0.0
22.336407878 0.502338528633118 1.0
10.686016238 0.534897327423096 2.0
0 0.223665297031403 3.0
10.686016238 0.435609847307205 4.0
22.336407878 0.485464096069336 5.0
33.022424116 0.316222548484802 6.0
33.022424116 0.426861703395844 7.0
36.232706353 0.846128702163696 0.0
25.757367318 0.476988852024078 1.0
10.475339035 0.532946646213531 2.0
0 0.224471241235733 3.0
10.475339035 0.455255150794983 4.0
25.757367318 0.491031527519226 5.0
36.232706353 0.43617057800293 6.0
36.232706353 0.457829803228378 7.0
45.8365445 0.845935583114624 0.0
33.58061703 0.509175658226013 1.0
12.255927471 0.548246741294861 2.0
0 0.256445556879044 3.0
12.255927471 0.456503212451935 4.0
33.58061703 0.504544973373413 5.0
45.8365445 0.515251278877258 6.0
45.8365445 0.370245933532715 7.0
39.344253832 0.727225303649902 0.0
27.946117456 0.481205821037292 1.0
11.398136376 0.522868096828461 2.0
0 0.211924061179161 3.0
11.398136376 0.436674982309341 4.0
27.946117456 0.469503313302994 5.0
39.344253832 0.257318168878555 6.0
39.344253832 0.338038444519043 7.0
43.613379879 0.782723128795624 0.0
29.144363875 0.550306916236877 1.0
14.469016005 0.575423538684845 2.0
0 0.219284847378731 3.0
14.469016005 0.420279264450073 4.0
29.144363875 0.467034876346588 5.0
43.613379879 0.380591779947281 6.0
43.613379879 0.268673866987228 7.0
41.603825526 0.8511843085289 0.0
32.060099244 0.524749636650085 1.0
9.5437262816 0.563086628913879 2.0
0 0.22827285528183 3.0
9.5437262816 0.437069892883301 4.0
32.060099244 0.464132964611053 5.0
41.603825526 0.580938935279846 6.0
41.603825526 0.523666620254517 7.0
41.290499737 0.801497101783752 0.0
29.938150836 0.495566040277481 1.0
11.352348902 0.546329975128174 2.0
0 0.237442418932915 3.0
11.352348902 0.446959614753723 4.0
29.938150836 0.486640512943268 5.0
41.290499737 0.538205087184906 6.0
41.290499737 0.539330363273621 7.0
38.21592954 0.863522171974182 0.0
24.146046072 0.468281179666519 1.0
14.069883468 0.522523581981659 2.0
0 0.21638697385788 3.0
14.069883468 0.448403805494308 4.0
24.146046072 0.49951446056366 5.0
38.21592954 0.432720094919205 6.0
38.21592954 0.498788744211197 7.0
34.764993855 0.804285168647766 0.0
21.168684565 0.545716643333435 1.0
13.59630929 0.584675252437592 2.0
0 0.244631439447403 3.0
13.59630929 0.437400698661804 4.0
21.168684565 0.480846703052521 5.0
34.764993855 0.276511549949646 6.0
34.764993855 0.370005458593369 7.0
34.789226058 0.747975766658783 0.0
25.072029366 0.503214359283447 1.0
9.7171966918 0.53385454416275 2.0
0 0.213837444782257 3.0
9.7171966918 0.433814704418182 4.0
25.072029366 0.462974339723587 5.0
34.789226058 0.248471558094025 6.0
34.789226058 0.339322805404663 7.0
42.488773183 0.784261345863342 0.0
27.1767235 0.416187703609467 1.0
15.312049683 0.450192630290985 2.0
0 0.202218979597092 3.0
15.312049683 0.44810026884079 4.0
27.1767235 0.483612358570099 5.0
42.488773183 0.507346987724304 6.0
42.488773183 0.518072664737701 7.0
34.702233392 0.673397839069366 0.0
22.091827715 0.469591915607452 1.0
12.610405677 0.494048565626144 2.0
0 0.188118740916252 3.0
12.610405677 0.417220771312714 4.0
22.091827715 0.448714405298233 5.0
34.702233392 0.22102016210556 6.0
34.702233392 0.314397394657135 7.0
41.831412446 0.413453876972198 0.0
26.30240797 0.538361668586731 1.0
15.529004476 0.579675316810608 2.0
0 0.184669941663742 3.0
15.529004476 0.43661430478096 4.0
26.30240797 0.456446141004562 5.0
41.831412446 0.190557420253754 6.0
41.831412446 -0.0193130299448967 7.0
45.908850318 0.768204987049103 0.0
30.587589099 0.353834599256516 1.0
15.321261219 0.368061184883118 2.0
0 0.239825814962387 3.0
15.321261219 0.413129061460495 4.0
30.587589099 0.46953558921814 5.0
45.908850318 0.255926847457886 6.0
45.908850318 0.303692132234573 7.0
33.104401139 0.569926261901855 0.0
23.063487553 0.456467121839523 1.0
10.040913586 0.484063982963562 2.0
0 0.169265776872635 3.0
10.040913586 0.409132540225983 4.0
23.063487553 0.4262815117836 5.0
33.104401139 0.0120941177010536 6.0
33.104401139 0.186056971549988 7.0
35.835170704 0.120568253099918 0.0
22.566106428 0.797751188278198 1.0
13.269064276 0.813399910926819 2.0
0 0.123183041810989 3.0
13.269064276 0.427151262760162 4.0
22.566106428 0.442026913166046 5.0
35.835170704 -0.0666267424821854 6.0
35.835170704 1.1040997505188 7.0
45.844507362 0.837592363357544 0.0
30.026497858 0.536162436008453 1.0
15.818009504 0.575047135353088 2.0
0 0.239195078611374 3.0
15.818009504 0.424634486436844 4.0
30.026497858 0.472514361143112 5.0
45.844507362 0.484661400318146 6.0
45.844507362 0.421688139438629 7.0
31.875262674 0.738079071044922 0.0
22.567942874 0.484143435955048 1.0
9.3073198007 0.523465573787689 2.0
0 0.221648275852203 3.0
9.3073198007 0.430354863405228 4.0
22.567942874 0.468249648809433 5.0
31.875262674 0.290105938911438 6.0
31.875262674 0.359275758266449 7.0
35.888104099 0.744237005710602 0.0
22.991591125 0.501128077507019 1.0
12.896512974 0.535313427448273 2.0
0 0.222997575998306 3.0
12.896512974 0.441955626010895 4.0
22.991591125 0.485698252916336 5.0
35.888104099 0.277660727500916 6.0
35.888104099 0.435194790363312 7.0
35.733201066 0.789843797683716 0.0
21.809644226 0.517688751220703 1.0
13.923556841 0.542868673801422 2.0
0 0.215907409787178 3.0
13.923556841 0.443316251039505 4.0
21.809644226 0.46927747130394 5.0
35.733201066 0.399901121854782 6.0
35.733201066 0.359512358903885 7.0
43.710725429 0.770245790481567 0.0
28.286031516 0.539265990257263 1.0
15.424693913 0.584526181221008 2.0
0 0.224556282162666 3.0
15.424693913 0.415617108345032 4.0
28.286031516 0.45531302690506 5.0
43.710725429 0.279928267002106 6.0
43.710725429 0.350989788770676 7.0
34.934843039 0.790276706218719 0.0
24.591035665 0.505245923995972 1.0
10.343807374 0.53099250793457 2.0
0 0.198081210255623 3.0
10.343807374 0.427411258220673 4.0
24.591035665 0.465415775775909 5.0
34.934843039 0.263571232557297 6.0
34.934843039 0.317151784896851 7.0
38.70218214 0.822845339775085 0.0
26.81631331 0.469875037670135 1.0
11.88586883 0.524389386177063 2.0
0 0.248640030622482 3.0
11.88586883 0.470990300178528 4.0
26.81631331 0.518216550350189 5.0
38.70218214 0.385273903608322 6.0
38.70218214 0.532739162445068 7.0
39.146786058 0.745629549026489 0.0
24.796133157 0.440602242946625 1.0
14.350652901 0.497564852237701 2.0
0 0.223359242081642 3.0
14.350652901 0.470826983451843 4.0
24.796133157 0.492195636034012 5.0
39.146786058 0.262956321239471 6.0
39.146786058 0.544199824333191 7.0
37.956003494 0.806739985942841 0.0
28.41977132 0.401349157094955 1.0
9.5362321739 0.428436696529388 2.0
0 0.255447804927826 3.0
9.5362321739 0.433036088943481 4.0
28.41977132 0.494789183139801 5.0
37.956003494 0.215267330408096 6.0
37.956003494 0.357328742742538 7.0
35.896232558 0.183311372995377 0.0
24.51348528 1.01320362091064 1.0
11.382747278 1.01181352138519 2.0
0 0.170761704444885 3.0
11.382747278 0.42978909611702 4.0
24.51348528 0.449950039386749 5.0
35.896232558 -0.081779845058918 6.0
35.896232558 2.02452826499939 7.0
37.377916966 0.154534935951233 0.0
24.033978696 0.631878972053528 1.0
13.343938271 0.668708801269531 2.0
0 0.187174826860428 3.0
13.343938271 0.413269490003586 4.0
24.033978696 0.44527080655098 5.0
37.377916966 0.000567857176065445 6.0
37.377916966 0.335900008678436 7.0
36.698956497 0.650843143463135 0.0
26.266791255 0.433510094881058 1.0
10.432165242 0.477599740028381 2.0
0 0.181102186441422 3.0
10.432165242 0.451521784067154 4.0
26.266791255 0.470768570899963 5.0
36.698956497 0.21667343378067 6.0
36.698956497 0.521405875682831 7.0
41.903925455 0.614661335945129 0.0
28.026899882 0.462207466363907 1.0
13.877025573 0.493976980447769 2.0
0 0.162575006484985 3.0
13.877025573 0.417298078536987 4.0
28.026899882 0.427234709262848 5.0
41.903925455 0.158513262867928 6.0
41.903925455 0.219808265566826 7.0
38.934024851 0.792782068252563 0.0
24.896228133 0.334390997886658 1.0
14.037796718 0.37085434794426 2.0
0 0.215390622615814 3.0
14.037796718 0.461321830749512 4.0
24.896228133 0.493830919265747 5.0
38.934024851 0.571407437324524 6.0
38.934024851 0.435825437307358 7.0
40.406325889 0.82914125919342 0.0
26.468139853 0.416990011930466 1.0
13.938186036 0.450109362602234 2.0
0 0.237914592027664 3.0
13.938186036 0.418347954750061 4.0
26.468139853 0.462546586990356 5.0
40.406325889 0.0872844234108925 6.0
40.406325889 0.464638262987137 7.0
39.410747699 0.851421535015106 0.0
27.084541691 0.527815699577332 1.0
12.326206008 0.551845252513885 2.0
0 0.215160667896271 3.0
12.326206008 0.425378054380417 4.0
27.084541691 0.475477933883667 5.0
39.410747699 0.358349710702896 6.0
39.410747699 0.283366084098816 7.0
};
\addlegendentry{$R^2$=-2.818}
\end{axis}

\end{tikzpicture}
}
    % \begin{tikzpicture}[shorten >=1pt, ->, draw=black!50, node distance=1.5cm and 3.5cm, align=center]

    % Styles
    \tikzstyle{input} = [circle, draw, fill=green!50, minimum size=2em]
    \tikzstyle{hidden} = [circle, draw, fill=blue!50, minimum size=2em]
    \tikzstyle{output} = [circle, draw, fill=red!50, minimum size=2em]
    \tikzstyle{connection} = [->, thick]

    % Network Stage Labels
    \node[align=center] at (0,-0.4) {Input \\ Layer};
    \node[align=center] at (6,0.4) {Hidden \\ Layers};
    \node[align=center] at (12,-1.2) {Output \\ Layer};

    % Input Layer
    \foreach \i in {1,2,3}
        \node[input] (I\i) at (0,-\i*1.5) {$x_\i$};

    % Hidden Layer 1
    \foreach \i in {1,2,3,4}
        \node[hidden] (H1\i) at (3,-\i*1.5+0.75) {$z^{(1)}_\i$};

    % Hidden Layer 2
    \foreach \i in {1,2,3,4}
        \node[hidden] (H2\i) at (6,-\i*1.5+0.75) {$z^{(2)}_\i$};

    % Hidden Layer 3
    \foreach \i in {1,2,3,4}
        \node[hidden] (H3\i) at (9,-\i*1.5+0.75) {$z^{(3)}_\i$};

    % Output Layer
    \foreach \i in {1,2}
        \node[output] (O\i) at (12,-\i*1.5-0.75) {$\hat{y}_\i$};

    % Connections from Input to Hidden Layer 1
    \foreach \i in {1,2,3}
        \foreach \j in {1,2,3,4}
            \draw[connection] (I\i) -- (H1\j);

    % Connections from Hidden Layer 1 to Hidden Layer 2
    \foreach \i in {1,2,3,4}
        \foreach \j in {1,2,3,4}
            \draw[connection] (H1\i) -- (H2\j);

    % Connections from Hidden Layer 2 to Hidden Layer 3
    \foreach \i in {1,2,3,4}
        \foreach \j in {1,2,3,4}
            \draw[connection] (H2\i) -- (H3\j);

    % Connections from Hidden Layer 3 to Output Layer
    \foreach \i in {1,2,3,4}
        \foreach \j in {1,2}
            \draw[connection] (H3\i) -- (O\j);

\end{tikzpicture}

    \caption{}\label{fig:results_dummy_base}
\end{figure}







\section{Discussion and conclusions}
