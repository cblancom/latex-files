\chapter{Natural Gas System Prediction Using Graph Neural Networks} \label{cap:lienal-censnet}

\section{Preliminaries}

\subsection{Graph definition}


A graph $G$ is a mathematical structure that represents a set of interconnected objects. These objects are known as vertices (or nodes), denoted by the set $V(G)$, and the connections between them are called edges (or arcs), denoted by the set $E(G)$. Formally, a graph is defined as an ordered pair $G = (V, E)$, where $V(G)$ is a non-empty set of vertices, and $E(G) \subseteq \{(u, v) \mid u, v \in V(G), u \neq v\}$ is a set of edges, where each edge connects two distinct vertices \cite{Trudeau_2015}.

Graphs can be categorized based on the properties of their edges. An undirected graph has edges that do not have a direction, so the pair $(u, v) = (v, u)$ represents an edge that simply connects vertices $u$ and $v$. In contrast, in a directed graph (or digraph), each edge $(u, v) \in E(G)$ has a direction, meaning it goes from vertex $u$ to vertex $v$. This implies that $(u, v) \neq (v, u)$ unless $u = v$ \cite{Bender_Williamson_2010}.

 %
% \begin{figure}
%     \begin{center}
%         \setlength\figurewidth{.5\textwidth}        
%             \setlength\figureheight{0.4\textwidth}
%         \resizebox{\figurewidth}{\figureheight}{\input{figures/Chapter_LinealCensnet/graph_def}}
%         % \input{figures/Chapter_LinealCensnet/graph_def}
%         % \includegraphics[width=0.95\textwidth]{figures/}
%     \end{center}
%     \caption{Directed graph}\label{fig:graph_definition}
% \end{figure}
%
\begin{figure}
    \centering
        \setlength\figurewidth{.53\textwidth}        
        \setlength\figureheight{0.36\textwidth} 
        \subfloat [Undirected graph] {\label{fig:undirected_graph_def}\resizebox{\figurewidth}{\figureheight}{\begin{tikzpicture}[>=Stealth, node distance=5cm]
    
    % Define styles for the nodes and edges
    \tikzstyle{node_style}=[circle, draw, minimum size=.9cm, inner sep=0pt, font=\normalsize]
    \tikzstyle{edge_style}=[draw, -, thick]

    % Nodes
    \node[node_style] (n1) {1};
    \node[node_style] (n2) [right of=n1] {2};
    \node[node_style] (n3) [right of=n2] {3};
    \node[node_style] (n4) [below of=n2] {4};

    % Edges with labels
    \draw[edge_style] (n1) edge node[above] {A} (n2);
    \draw[edge_style] (n4) edge node[left] {B} (n1);
    \draw[edge_style] (n4) edge node[right] {C} (n2);
    \draw[edge_style] (n3) edge node[right] {D} (n4);
    \draw[edge_style] (n1) to[out=30, in=150] node[above] {E} (n3);
    \draw[edge_style] (n3) edge node[above] {F} (n2);

\end{tikzpicture}
}}
        \subfloat [Directed graph] {\label{fig:direc_graph_def}\resizebox{\figurewidth}{\figureheight}{\begin{tikzpicture}[>=Stealth, node distance=5cm]
    
    % Define styles for the nodes and edges
    \tikzstyle{node_style}=[circle, draw, minimum size=.9cm, inner sep=0pt, font=\normalsize]
    \tikzstyle{edge_style}=[draw, ->, thick]

    % Nodes
    \node[node_style] (n1) {1};
    \node[node_style] (n2) [right of=n1] {2};
    \node[node_style] (n3) [right of=n2] {3};
    \node[node_style] (n4) [below of=n2] {4};

    % Edges with labels
    \draw[edge_style] (n1) edge node[above] {A} (n2);
    \draw[edge_style] (n4) edge node[left] {B} (n1);
    \draw[edge_style] (n4) edge node[right] {C} (n2);
    \draw[edge_style] (n3) edge node[right] {D} (n4);
    % \draw[edge_style] (n1) to[out=30, in=150] node[above] {E} (n3);
    % \draw[edge_style] (n3) edge node[above] {F} (n2);

\end{tikzpicture}
}}
        % \includegraphics[width=0.45\textwidth]{figures/}
    \caption{Types of graphs}\label{fig:graph_definition}
\end{figure}

In \cref{fig:graph_definition}, two graphs are represented, each composed of four nodes labeled $1$, $2$, $3$, and $4$, and six edges labeled $A$, $B$, $C$, $D$, $E$, and $F$. The difference between them lies in the type of graph they represent. For example, in \cref{fig:undirected_graph_def}, the edge c shows a connection between nodes 2 and 4. However, in \cref{fig:direc_graph_def}, this connection provides additional information: a direction, which, in the context of this study, could represent the direction of a specific element, such as electric power or gas flow.


A graph can be represented in various ways using matrices, each capturing different aspects of the graph's structure. The two most common matrix representations are the adjacency matrix and the incidence matrix.

The adjacency matrix of a graph is a square matrix used to represent the connections between vertices \cite{wilson_1972}. For a graph $G$ with $n$ vertices, the adjacency matrix $A$ is an $n \times n$ matrix where the entry $a_{ij}$ is defined as follows:

\begin{equation}
 a_{ij} = 
\begin{cases}
1 & \text{if there is an edge from vertex } i \text{ to vertex } j, \\
0 & \text{otherwise}.
\end{cases}
    \label{eq:adjacency_matrix_definition}
\end{equation}

% \[
% a_{ij} =
% \begin{cases}
% 1 & \text{if there is an edge from vertex } i \text{ to vertex } j, \\
% 0 & \text{otherwise}.
% \end{cases}
% \]

For a directed graph, the adjacency matrix captures the direction of the edges. Below is the adjacency matrix for the directed graph shown earlier:

\[
A = \begin{pmatrix}
0 & 1 & 0 & 1 \\
1 & 0 & 0 & 1 \\
0 & 1 & 0 & 1 \\
1 & 1 & 1 & 0
\end{pmatrix}
\]


The incidence matrix of a graph represents the relationship between vertices and edges \cite{wilson_1972}. For a graph $G$ with $n$ vertices and $m$ edges, the incidence matrix $I$ is an $n \times m$ matrix where the entry $i_{ij}$ is defined as follows:

\begin{equation}
 B_{ij} =
\begin{cases}
1 & \text{if vertex } i \text{ is the starting point of edge } j \text{ in a directed graph}, \\
-1 & \text{if vertex } i \text{ is the endpoint of edge } j \text{ in a directed graph}, \\
0 & \text{if vertex } i \text{ is not connected to edge } j.
\end{cases}  
    \label{eq:incidence_matrix_definition}
\end{equation}


For the directed graph previously described, the incidence matrix is given by:

\[
B = \begin{pmatrix}
1 & -1 & 0 & 0 \\
-1 & 0 & -1 & 0 \\
0 & 0 & 0 & 1  \\
0 & 1 & 1 & -1 
\end{pmatrix}
\]

\subsection{Neural networks}

\subsubsection{Multi-Layered Perceptrons}

A Multilayer Perceptron (MLP) is a fundamental type of artificial neural network, often regarded as one of the building blocks of deep learning. At its core, an MLP consists of multiple layers of nodes, or neurons, where each layer is fully connected to the next one. The architecture typically includes an input layer, one or more hidden layers, and an output layer, as can be seen in the \cref{fig:FCNN_representation}.


\begin{figure}
    \centering
    \setlength\figurewidth{1\textwidth}        
    \setlength\figureheight{0.5\textwidth}
    \resizebox{\figurewidth}{\figureheight}{\begin{tikzpicture}[shorten >=1pt, ->, draw=black!50, node distance=1.5cm and 3.5cm, align=center]

    % Styles
    \tikzstyle{input} = [circle, draw, fill=green!50, minimum size=2em]
    \tikzstyle{hidden} = [circle, draw, fill=blue!50, minimum size=2em]
    \tikzstyle{output} = [circle, draw, fill=red!50, minimum size=2em]
    \tikzstyle{connection} = [->, thick]

    % Network Stage Labels
    \node[align=center] at (0,-0.4) {Input \\ Layer};
    \node[align=center] at (6,0.4) {Hidden \\ Layers};
    \node[align=center] at (12,-1.2) {Output \\ Layer};

    % Input Layer
    \foreach \i in {1,2,3}
        \node[input] (I\i) at (0,-\i*1.5) {$x_\i$};

    % Hidden Layer 1
    \foreach \i in {1,2,3,4}
        \node[hidden] (H1\i) at (3,-\i*1.5+0.75) {$z^{(1)}_\i$};

    % Hidden Layer 2
    \foreach \i in {1,2,3,4}
        \node[hidden] (H2\i) at (6,-\i*1.5+0.75) {$z^{(2)}_\i$};

    % Hidden Layer 3
    \foreach \i in {1,2,3,4}
        \node[hidden] (H3\i) at (9,-\i*1.5+0.75) {$z^{(3)}_\i$};

    % Output Layer
    \foreach \i in {1,2}
        \node[output] (O\i) at (12,-\i*1.5-0.75) {$\hat{y}_\i$};

    % Connections from Input to Hidden Layer 1
    \foreach \i in {1,2,3}
        \foreach \j in {1,2,3,4}
            \draw[connection] (I\i) -- (H1\j);

    % Connections from Hidden Layer 1 to Hidden Layer 2
    \foreach \i in {1,2,3,4}
        \foreach \j in {1,2,3,4}
            \draw[connection] (H1\i) -- (H2\j);

    % Connections from Hidden Layer 2 to Hidden Layer 3
    \foreach \i in {1,2,3,4}
        \foreach \j in {1,2,3,4}
            \draw[connection] (H2\i) -- (H3\j);

    % Connections from Hidden Layer 3 to Output Layer
    \foreach \i in {1,2,3,4}
        \foreach \j in {1,2}
            \draw[connection] (H3\i) -- (O\j);

\end{tikzpicture}
}
    % \begin{tikzpicture}[shorten >=1pt, ->, draw=black!50, node distance=1.5cm and 3.5cm, align=center]

    % Styles
    \tikzstyle{input} = [circle, draw, fill=green!50, minimum size=2em]
    \tikzstyle{hidden} = [circle, draw, fill=blue!50, minimum size=2em]
    \tikzstyle{output} = [circle, draw, fill=red!50, minimum size=2em]
    \tikzstyle{connection} = [->, thick]

    % Network Stage Labels
    \node[align=center] at (0,-0.4) {Input \\ Layer};
    \node[align=center] at (6,0.4) {Hidden \\ Layers};
    \node[align=center] at (12,-1.2) {Output \\ Layer};

    % Input Layer
    \foreach \i in {1,2,3}
        \node[input] (I\i) at (0,-\i*1.5) {$x_\i$};

    % Hidden Layer 1
    \foreach \i in {1,2,3,4}
        \node[hidden] (H1\i) at (3,-\i*1.5+0.75) {$z^{(1)}_\i$};

    % Hidden Layer 2
    \foreach \i in {1,2,3,4}
        \node[hidden] (H2\i) at (6,-\i*1.5+0.75) {$z^{(2)}_\i$};

    % Hidden Layer 3
    \foreach \i in {1,2,3,4}
        \node[hidden] (H3\i) at (9,-\i*1.5+0.75) {$z^{(3)}_\i$};

    % Output Layer
    \foreach \i in {1,2}
        \node[output] (O\i) at (12,-\i*1.5-0.75) {$\hat{y}_\i$};

    % Connections from Input to Hidden Layer 1
    \foreach \i in {1,2,3}
        \foreach \j in {1,2,3,4}
            \draw[connection] (I\i) -- (H1\j);

    % Connections from Hidden Layer 1 to Hidden Layer 2
    \foreach \i in {1,2,3,4}
        \foreach \j in {1,2,3,4}
            \draw[connection] (H1\i) -- (H2\j);

    % Connections from Hidden Layer 2 to Hidden Layer 3
    \foreach \i in {1,2,3,4}
        \foreach \j in {1,2,3,4}
            \draw[connection] (H2\i) -- (H3\j);

    % Connections from Hidden Layer 3 to Output Layer
    \foreach \i in {1,2,3,4}
        \foreach \j in {1,2}
            \draw[connection] (H3\i) -- (O\j);

\end{tikzpicture}

    \caption{General diagram of a multilayer perceptron, showing the input layers in green, the hidden layers in blue and the outputs in red.}\label{fig:FCNN_representation}
\end{figure}



The neurons in each layer are connected to the neurons in the subsequent layer through weighted connections, the key parameters learned during the training process \cite{Hippert_Pedreira_Souza_2001}. One of the most significant properties of an MLP is its ability to function as a universal approximator. Given sufficient neurons in the hidden layers, an MLP can approximate any continuous function to an arbitrary degree of accuracy, provided the network is trained correctly \cite{Zhang_Eddy_Patuwo_Y_Hu_1998}.

Mathematically, an MLP can be defined as follows. Let $\mathbf{x} \in \mathbb{R}^n$ represent the input vector, where $n$ is the number of features. The output of each neuron in the first hidden layer is calculated as:

\begin{equation}
 \mathbf{z}^{(1)} = \sigma\left(\mathbf{W}^{(1)}\mathbf{x} + \mathbf{b}^{(1)}\right)   
    \label{eq:output_neuron}
\end{equation}



where $\mathbf{W}^{(1)} \in \mathbb{R}^{m_1 \times n}$ is the weight matrix for the first hidden layer, with $m_1$ being the number of neurons in this layer, $\mathbf{b}^{(1)} \in \mathbb{R}^{m_1}$ is the bias vector, and $\sigma(\cdot)$ is the activation function, typically a non-linear function such as the ReLU (Rectified Linear Unit) or sigmoid function.

This process is repeated for each subsequent hidden layer $k$, where the output of the $k$-th layer is given by:

\begin{equation}
 \mathbf{z}^{(k)} = \sigma\left(\mathbf{W}^{(k)}\mathbf{z}^{(k-1)} + \mathbf{b}^{(k)}\right) 
    \label{eq:hidden_layers_output}
\end{equation}


Here, $\mathbf{W}^{(k)} \in \mathbb{R}^{m_k \times m_{k-1}}$ represents the weight matrix connecting layer $k-1$ to layer $k$, $\mathbf{b}^{(k)} \in \mathbb{R}^{m_k}$ is the bias vector for layer $k$, and $\mathbf{z}^{(k-1)}$ is the output of the previous layer.

Finally, the output layer produces the final prediction $\mathbf{\hat{y}}$:

\begin{equation}
 \mathbf{\hat{y}} = \sigma\left(\mathbf{W}^{(L)}\mathbf{z}^{(L-1)} + \mathbf{b}^{(L)}\right)   
    \label{eq:output_layer}
\end{equation}

where $L$ denotes the number of layers in the network, including the input and output layers. Depending on the nature of the problem (e.g., classification or regression), the activation function $\sigma(\cdot)$ used in the output layer can vary, with softmax being common in multi-class classification problems, and a linear activation for regression tasks. The entire MLP is trained using a process called backpropagation, combined with an optimization algorithm like gradient descent, to minimize a loss function $J(\mathbf{y}, \mathbf{\hat{y}})$, which measures the difference between the true outputs $\mathbf{y}$ and the predicted outputs $\mathbf{\hat{y}}$. 


\subsubsection{Graph Neural Networks}

In recent years, Graph Neural Networks (GNNs) have emerged as a powerful tool in machine learning, particularly for tasks involving data that can be naturally represented as graphs. Graphs are a universal data structure that can model various systems in numerous fields, including social networks, biological networks, knowledge graphs, and physical systems. Because of their ability to represent relationships and interactions between entities, graphs are used extensively to model complex structures where the data points are not independent but interconnected \cite{Jia_Wang_Shou_Hosseini_Bai_2023}.

GNNs are important because they can directly operate on graph-structured data, extending neural networks' success from grid-like data structures, such as images and sequences, to more general and irregular structures \cite{Jia_Wang_Shou_Hosseini_Bai_2023a} . Traditional neural networks, like Convolutional Neural Networks (CNNs) or Recurrent Neural Networks (RNNs), are designed to work with data that has a fixed structure. However, many real-world problems involve data that can be better described by graphs, where nodes represent entities and edges represent relationships between those entities \cite{Gupta_Matta_Pant_2021}. 

Graph Neural Networks can be broadly defined as a class of neural networks designed to perform inference on data described by graphs. Formally, let $G = (V, E)$ represent a graph, where $V$ is the set of nodes (or vertices) and $E$ is the set of edges. Each node $v \in V$ can be associated with a feature vector $\mathbf{x}_v$, and each edge $(u, v) \in E$ may have an associated weight or feature vector $\mathbf{e}_{uv}$. The goal of a GNN is to learn a representation for each node (or sometimes for the entire graph) by aggregating and transforming the feature information from the node's local neighborhood in the graph.


In the GNN framework, message passing is understood as a series of iterations in which each node updates its representation by exchanging information with its neighbors. To move from the abstract concept to a practical implementation, the specific functions used for updating and aggregating node features must be defined \cite{Liu_Wu_Liu_Hu_2021}.


The basic message passing operation, which simplifies the original GNN model proposed by \cite{GRLB_Hamilton}, is expressed by the following equation:
\begin{equation}
 \mathbf{h}_u^{(k)} = \sigma\left( \mathbf{W}_{\text{self}}^{(k)} \mathbf{h}_u^{(k-1)} + \mathbf{W}_{\text{neigh}}^{(k)} \sum_{v \in \mathcal{N}(u)} \mathbf{h}_v^{(k-1)} + \mathbf{b}^{(k)} \right)  
    \label{eq:message_passing}
\end{equation}

In this equation:
\begin{itemize}
    \item \( \mathbf{h}_u^{(k)} \) represents the updated feature vector of node \( u \) at layer \( k \).
    \item The term \( \mathbf{W}_{\text{self}}^{(k)} \mathbf{h}_u^{(k-1)} \) applies a transformation to the node's own feature vector from the previous layer, enabling the node to retain and modify its self-information. 
    \item The term \( \mathbf{W}_{\text{neigh}}^{(k)} \sum_{v \in \mathcal{N}(u)} \mathbf{h}_v^{(k-1)} \) aggregates the feature vectors of the neighboring nodes \( v \) in the set \( \mathcal{N}(u) \), and then applies a transformation via the weight matrix \( \mathbf{W}_{\text{neigh}}^{(k)} \) 
    \item \( \mathbf{b}^{(k)} \) is a bias term that can be added to the weighted sum, though it is sometimes omitted for simplicity. 
    \item  The non-linear function \( \sigma(\cdot) \), such as ReLU or tanh, is applied elementwise to introduce non-linearity into the model, which is essential for capturing complex patterns in the data.

\end{itemize}

% . This aggregation step is crucial as it allows the node to incorporate information from its local neighborhood.


In the context of Graph Neural Networks (GNNs), a Graph Convolutional Network (GCN) is a specialized model that applies the concept of convolution, widely used in image processing, to graphs. First introduced by \cite{Kipf:2016tc}, GCNs offer a method to perform deep learning on graph-structured data by extending traditional convolution operations to the irregular domain of graphs.

The fundamental idea behind GCNs is to create a spectral filter that operates on graph data. The filter's purpose is to combine features from a node's local neighborhood, taking into account the graph's structure. This process is mathematically formalized in the following way:

\begin{equation}
 \mathbf{H} = \sigma\left( \tilde{\mathbf{D}}^{-\frac{1}{2}} \tilde{\mathbf{A}} \tilde{\mathbf{D}}^{-\frac{1}{2}} \mathbf{X} \mathbf{\Theta} \right)   
    \label{eq:GCN_filter}
\end{equation}

Where:
\begin{itemize}
    \item \( \mathbf{H} \) represents the matrix of node representations after applying the GCN layer. Each row \( \mathbf{h}_u \) in \( \mathbf{H} \) corresponds to the updated feature vector for node \( u \). 
    \item \( \mathbf{X} \) is the matrix of input node features, where each row \( \mathbf{x}_u \) corresponds to the feature vector for node \( u \) before applying the GCN layer. 
    \item \( \sigma(\cdot) \) denotes a non-linear activation function, such as ReLU, applied elementwise to introduce non-linearity into the model. 
    \item \( \tilde{\mathbf{A}} \) is the adjacency matrix of the graph, with added self-loops to account for the node itself in the aggregation. 
    \item \( \tilde{\mathbf{D}} \) is the degree matrix of the graph, modified to include the self-loops. The degree matrix is diagonal, with each diagonal entry \( \tilde{D}_{ii} \) representing the degree of node \( i \) in the graph. 
    \item \( \mathbf{\Theta} \) is a matrix of trainable parameters, which is learned during the training process to optimize the model's performance.


\end{itemize}

The expression \( \tilde{\mathbf{D}}^{-\frac{1}{2}} \tilde{\mathbf{A}} \tilde{\mathbf{D}}^{-\frac{1}{2}} \) is a normalized version of the adjacency matrix, ensuring that the eigenvalues of the operation are bounded between 0 and 1. This normalization step is crucial as it prevents issues such as exploding or vanishing gradients during the training of deep networks.

Specifically, the adjacency matrix \( \tilde{\mathbf{A}} \) is defined as:

\begin{equation}
 \tilde{\mathbf{A}} = \mathbf{A} + \mathbf{I}   
    \label{eq:modified_adjacency_matrix}
\end{equation}

where \( \mathbf{A} \) is the original adjacency matrix, and \( \mathbf{I} \) is the identity matrix. The identity matrix \( \mathbf{I} \) ensures that each node considers its own features when aggregating information from its neighbors.

The degree matrix \( \tilde{\mathbf{D}} \) is defined as:

\begin{equation}
 \tilde{D}_{ii} = \sum_{j \in V} \tilde{A}_{ij}   
    \label{eq:degree_matrix}
\end{equation}
%
where \( V \) represents the set of all nodes in the graph. The diagonal entries of \( \tilde{\mathbf{D}} \) correspond to the degree of each node, adjusted to account for the added self-loops.

\subsubsection{Convolution with Edge-Node Switching (CensNet)}

Graph Convolutional Networks (GCNs) have demonstrated considerable success in various graph-based machine learning tasks, particularly in their ability to generalize convolution operations to non-Euclidean data structures like graphs \cite{Fu_Wang_Liu_Liu_Zhou_You_Peng_Jing_2022} . GCNs operate by aggregating features from a node's neighbors, thereby capturing local neighborhood information and propagating it through the network layers. Despite their effectiveness, GCNs possess certain limitations that hinder their performance in more complex scenarios.

One notable limitation of GCNs is their reliance solely on node features during the convolution process. This approach disregards the information contained within edge features. By neglecting edge features, GCNs fail to fully exploit the underlying structure of the graph, potentially missing out on critical insights that could enhance model performance \cite{8954414}.

Furthermore, GCNs typically aggregate information from immediate neighbors only, which can limit their ability to capture long-range dependencies in large or densely connected graphs. This restriction can lead to an oversimplified representation of the graph structure, particularly in cases where the graph contains intricate patterns that require deeper and more nuanced analysis \cite{Guo_Zhang_Teng_Lu_2019}.

To overcome the limitations of traditional GCNs, which focus primarily on node features, CensNet introduces a novel approach that integrates both node and edge features into the graph convolution process. The CensNet framework consists of two primary types of layers: the \textit{node layer} and the \textit{edge layer}. These layers work in concert to update node and edge embeddings alternately, leveraging the information from both nodes and edges in the graph \cite{9224195}.

The propagation rules in CensNet are designed to incorporate edge features into the convolution process, enabling a more comprehensive feature propagation across the graph. We define the normalized node adjacency matrix with self-loops as follows:

\begin{equation}
\tilde{\mathbf{A}}_v = \mathbf{D}_v^{-\frac{1}{2}} (\mathbf{A}_v + \mathbf{I}_{N_v}) \mathbf{D}_v^{-\frac{1}{2}},
    \label{eq:normalized_node_adjacency}
\end{equation}

where $\mathbf{D}_v$ is the diagonal degree matrix of $\mathbf{A}_v + \mathbf{I}_{N_v}$.

\paragraph{Node Layer Propagation:}


In the $(l+1)$-th layer, the node features are updated using the following propagation rule:


\begin{equation}
    \mathbf{H}^{(l+1)}_v = \sigma\left(\mathbf{T} \Phi\left(\mathbf{H}^{(l)}_e \mathbf{P}_e\right)\mathbf{T}^\top \odot \tilde{\mathbf{A}}_v \mathbf{H}^{(l)}_v \mathbf{W}_v\right) 
    \label{eq:CensNet_propagation}
\end{equation}

Where, 

\begin{itemize}
    \item \( \mathbf{T} \in \mathbb{R}^{N_v \times N_e} \) is a binary transformation matrix that represents the connections between nodes and edges. Each element \( T_{i,m} \) indicates whether edge \( m \) connects to node \( i \). Specifically, if edge \( m \) is connected to node \( i \), then \( T_{i,m} = 1 \); otherwise, \( T_{i,m} = 0 \). Given that each edge is formed by two nodes, every column of the matrix \( \mathbf{T} \) will have exactly two elements equal to 1, corresponding to the two nodes that the edge connects. 
    \item \( \mathbf{H}^{(l)}_e \) is the edge feature matrix from the \( l \)-th layer. \( \mathbf{P}_e \) is a learnable vector of dimension \( d_e \), which acts as a weight for the edge features. The operation \( \Phi(\mathbf{H}^{(l)}_e \mathbf{P}_e) \) denotes the diagonalization of the vector \( \mathbf{H}^{(l)}_e \mathbf{P}_e \), converting it into a diagonal matrix where the elements of the vector are placed on the diagonal. 
    \item The Hadamard product, denoted by \( \odot \), represents element-wise multiplication between matrices. In this context, it combines the transformed edge features with the node adjacency matrix, integrating information from both the original graph and its line graph.
    \item \( \tilde{\mathbf{A}}_v\) is the normalized adjacency matrix for nodes, as shown in \cref{eq:normalized_node_adjacency}, where \( \mathbf{A}_v \) is the original node adjacency matrix and \( \mathbf{I}_{N_v} \) is the identity matrix that introduces self-loops. This normalization ensures that the contributions from each node's neighbors are appropriately scaled. 
    \item \( \mathbf{H}^{(l)}_v \) represents the node feature matrix from the \( l \)-th layer. \( \mathbf{W}_v \) is a learnable weight matrix that is applied to the node features during the propagation process. 
    \item The activation function \( \sigma \) (typically a non-linear function such as ReLU) is applied element-wise to the resulting matrix to introduce non-linearity into the model.
\end{itemize}


This expression can be understood as a mechanism for integrating node and edge information. The matrix \( \mathbf{T} \) is responsible for transferring edge features into the node domain, allowing these edge-derived features to be merged with the normalized node adjacency matrix \( \tilde{\mathbf{A}}_v \). 



\paragraph{Edge Layer Propagation:}

Similarly, the normalized (Laplacianized) edge adjacency matrix is defined as:

\begin{equation}
    \tilde{\mathbf{A}}_e = \mathbf{D}_e^{-\frac{1}{2}} \left(\mathbf{A}_e + \mathbf{I}_{N_e}\right) \mathbf{D}_e^{-\frac{1}{2}},
    \label{eq:normalized_edge_adjacency}
\end{equation}


where \( D_e \) is the degree matrix corresponding to the edge adjacency matrix \( A_e + I_{N_e} \). The matrix \( \tilde{A}_e \) serves as the normalized version of the edge adjacency matrix, similar to how the node adjacency matrix is normalized. This normalization ensures that the influence of each edge is scaled appropriately, which is crucial for the stability of the propagation process.

The propagation rule for edge features is defined as follows:


\begin{equation}
    \mathbf{H}^{(l+1)}_e = \sigma\left(\mathbf{T}^\top \Phi\left(\mathbf{H}^{(l)}_v \mathbf{P}_v\right) \mathbf{T} \odot \tilde{\mathbf{A}}_e \mathbf{H}^{(l)}_e \mathbf{W}_e\right),
\end{equation}

In this expression, the following components are involved:

\begin{itemize}
    \item \textbf{\( \mathbf{T}^\top \)} is the transpose of the binary transformation matrix \( \mathbf{T} \) used in the node layer propagation. The matrix \( \mathbf{T}^\top \) maps the node features back into the edge domain, allowing the edge features to be updated based on the node information.
    \item \( \mathbf{H}^{(l)}_v \) is the node feature matrix from the \( l \)-th layer, and \( \mathbf{P}_v \) is a learnable weight matrix for the nodes. The operation \( \Phi(\mathbf{H}^{(l)}_v \mathbf{P}_v) \) diagonalizes the product of node features and the learnable weights, similar to the transformation applied to edge features in the node layer propagation.
    \item The matrix \( \tilde{\mathbf{A}}_e \) is the normalized edge adjacency matrix, as defined in \cref{eq:normalized_edge_adjacency}. This matrix integrates information about the connections between edges, analogous to how \( \tilde{\mathbf{A}}_v \) handles connections between nodes.
    \item \( \mathbf{H}^{(l)}_e \) represents the edge feature matrix from the \( l \)-th layer, while \( \mathbf{W}_e \) is a learnable weight matrix that is applied to the edge features during the propagation.
    \item The Hadamard product \( \odot \) element-wise multiplies the transformed node features with the edge adjacency matrix, merging the information from both domains.
    \item As in the node layer propagation, the activation function \( \sigma \) is applied element-wise to introduce non-linearity.
\end{itemize}


This propagation rule updates the edge embeddings by integrating information from the node features and the edge structure, thereby enhancing the expressiveness of the edge representations. The alternating updates between node and edge embeddings allow the model to effectively bridge signals across nodes and edges, leading to more robust and informative graph embeddings.




\subsection{Task-Dependent Loss Functions}

The output layer and corresponding loss functions in CensNet are designed to be task-dependent. For regression tasks, the loss function can be formalized as a regularized mean square error (MSE) loss. The MSE loss measures the difference between the predicted outcomes and the actual continuous values, providing a natural fit for regression problems.


We define the loss function for graph regression as follows:

\begin{equation}
    \mathcal{L}(\Theta) = \sum_{l \in \mathcal{Y}_L} \sum_{f=1}^{F} \| Y_{lf} - \hat{Y}_{lf} \|^2_2 + \lambda \|\Theta\|_p,
\end{equation}

where:

\begin{itemize}
    \item \( Y_{lf} \) represents the true continuous value for the \( f \)-th feature of the \( l \)-th graph in the training set.
    \item \( \hat{Y}_{lf} \) is the predicted outcome generated from the final node hidden layer of the CensNet model.
    \item \( \| Y_{lf} - \hat{Y}_{lf} \|^2_2 \) is the squared difference between the true and predicted values, summed across all features and all graphs in the training set.
    \item \( \lambda \|\Theta\|_p \) is the regularization term, which helps control the model's complexity and prevents overfitting by penalizing large weights. The parameter \( \lambda \) controls the strength of the regularization, while \( p \) determines the type of regularization norm (e.g., \( p=2 \) for \( L_2 \) regularization).
\end{itemize}



\section{Linear formulation of the natural gas system} \label{sec:LinealCensnet_formulation}


Natural gas is a widely used energy resource, particularly for electricity generation. The natural gas system consists of a network of production centers, pipelines, compressor stations, storage facilities, and distribution points that ensure reliable gas delivery from producers to consumers. Mathematically, this system can be represented as a directed graph defined as $\mathcal{G}_f = \left\{\mathcal{N}_f, \mathcal{E}_f\right\}$ where $\mathcal{N}_f$ is the set of units within the gas system, and $ \mathcal{E}_f$ is the set of different elements linking them. This set of units includes gas supply nodes or wells $\mathcal{W} \subset \mathcal{N}_{f}$, gas demand nodes or users $\mathcal{U} \subset \mathcal{N}_{f}$, and gas storage facilities $\mathcal{S} \subset \mathcal{N}_{f}$. Similarly, the set of directed gas adjacency edges $\mathcal{A} = \left\{(n,m) \mid n,m\in\mathcal{N}_f \right\} \subset \mathcal{E}$ delineates the network structure through two kinds of transmission elements: transport pipelines $\mathcal{P} = \left\{p=(n,m) \mid n,m\in\mathcal{N}_f \right\}$ and compressing stations $\mathcal{C} = \left\{c=(n,m) \mid n,m\in\mathcal{N}_f \right\}$, so that $\mathcal{P}\cup\mathcal{C}=\mathcal{A}$ and $\mathcal{P}\cap\mathcal{C}=\emptyset$.


Natural gas transportation requires coordination to manage the flow through the different elements to maintain safe operating ranges. In optimizing this network, mathematical models minimize overall operating costs associated with the various stages of natural gas transportation, compression, storage, and handling unsupplied demand, ensuring compliance with technical and physical constraints. The function is expressed as:

% \begin{equation} \label{eq:obj_func_integrated}
% \begin{split}
% \min_{\mathcal{P}, \mathcal{F}} \quad  \sum_{w \in \mathcal{W}} C_{w}^t {f_{w}^t} + \sum_{p \in \mathcal{P}} C_{p}^t {f_{p}^t} + \sum_{c \in \mathcal{C}} C_{c}^t {f_{c}^t} + \\ \sum_{u \in \mathcal{U}} C_{u}^{t} {f_{u}^{t}} + \quad \sum_{s \in \mathcal{S}} C_{s+}^{t} {f_{s+}^{t}}  + \sum_{s \in \mathcal{S}} C_{s-}^{t} {f_{s-}^{t}} + \sum_{s \in \mathcal{S}} C_{s}^{t} {V_{s}^{t}}
% \end{split}
% \end{equation}

\begin{equation} \label{eq:obj_func_linear_gas}
    \begin{split}
    \min_{\mathcal{P}, \mathcal{F}} \quad  \sum_{w \in \mathcal{W}} C_{w}^t {f_{w}^t} + \sum_{p \in \mathcal{P}} C_{p}^t {f_{p}^t} + \sum_{c \in \mathcal{C}} C_{c}^t {f_{c}^t} + \sum_{u \in \mathcal{U}} C_{u}^{t} {f_{u}^{t}} 
        % + \quad \sum_{s \in \mathcal{S}} C_{s+}^{t} {f_{s+}^{t}}  + \sum_{s \in \mathcal{S}} C_{s-}^{t} {f_{s-}^{t}} + \sum_{s \in \mathcal{S}} C_{s}^{t} {V_{s}^{t}}
    \end{split}
\end{equation}


The term $\sum_{w \in \mathcal{W}} C_{w}^t {f_{w}^t}$ represents the total cost of gas production at the wells, where $C_{w}^t$ denotes the cost per unit flow of gas at a specific well $w$ during time period $t$, and $f_{w}^t$ corresponds to the flow of gas from well $w$. Similarly, the transportation of gas through pipelines is captured by the term $\sum_{p \in \mathcal{P}} C_{p}^t {f_{p}^t}$, where $C_{p}^t$ is the cost per unit flow through pipeline $p$ during time period $t$, and $f_{p}^t$ represents the flow of gas through pipeline $p$. In addition, the total cost associated with gas compression at compressor stations is accounted for by $\sum_{c \in \mathcal{C}} C_{c}^t {f_{c}^t}$, where $C_{c}^t$ is the cost per unit flow at compressor station $c$ during time period $t$, and $f_{c}^t$ is the flow of gas through compressor station $c$.

Beyond production, transportation, and compression, the model also considers the costs related to unmet gas demand. The term $\sum_{u \in \mathcal{U}} C_{u}^{t} {f_{u}^{t}}$ reflects the penalty cost associated with unsupplied gas demand, where $C_{u}^{t}$ is the penalty cost per unit of unsupplied gas at location $u$ during time period $t$, and $f_{u}^{t}$ represents the volume of unmet demand. 



% Additionally, the model includes costs related to storage operations. The term $\sum_{s \in \mathcal{S}} C_{s+}^{t} {f_{s+}^{t}}$ represents the cost of injecting gas into storage facilities, where $C_{s+}^{t}$ is the cost per unit flow into storage during time period $t$, and $f_{s+}^{t}$ denotes the flow into storage at facility $s$. Conversely, the cost of withdrawing gas from storage is captured by $\sum_{s \in \mathcal{S}} C_{s-}^{t} {f_{s-}^{t}}$, where $C_{s-}^{t}$ is the cost per unit flow out of storage during time period $t$, and $f_{s-}^{t}$ represents the flow out of storage at facility $s$. Finally, the model accounts for the storage cost itself through the term $\sum_{s \in \mathcal{S}} C_{s}^{t} {V_{s}^{t}}$, where $C_{s}^{t}$ is the cost per unit volume of gas stored at facility $s$ during time period $t$, and $V_{s}^{t}$ denotes the volume of gas stored. 


% Lastly, \Cref{eq:weymouth_cons}, known as the Weymouth equation, summarizes the physical behavior of gas flow through pipelines by relating the gas flow through the pipeline $f_{p}^t$ to the pressures at the ends of the pipeline $\pi_{n}^t, \pi_{m}^t \ \forall \ p = (n,m) \in\mathcal{P}$. The Weymouth equation defines a nonlinear, nonconvex, disjunctive flow-pressure relationship that hampers the optimization of the gas transport system.
% \begin{subequations}
\begin{alignat}{4}
    \underline{f_{w}^t} \leq f_{w}^t \leq \overline{f_{w}^t} &\quad \forall \ w \in \mathcal{W} \label{eq:well_limits} \\
    -\overline{f_{p}^t} \leq f_{p}^t \leq \overline{f_{p}^t} &\quad \forall \ p \in \mathcal{P} \label{eq:pipe_limits} \\
    % \underline{\pi_{n}^t} \leq \pi_{n}^t \leq \overline{\pi_{n}^t} &\quad \forall \ n \in \mathcal{N}_f \label{eq:press_limit} \\
    % \pi_{m}^t \leq \beta_{c}^t{\pi_{n}^t} &\quad \forall c=(n,m) \in \mathcal{C} \label{eq:comp_ratio} \\
    0 \leq f_{u}^{t} \leq \overline{f_{u}^{t}} &\quad \forall \ u \in \mathcal{U} \label{eq:dem_limit_gas} \\
    \sum_{m:(m,n)\in\mathcal{A}}{f_{m}^t} = \sum_{m':(n,m')\in\mathcal{A}}{f_{m'}^t} &\quad \forall \ n \in \mathcal{N}_f \label{eq:gas_balance}
    % 0 \leq f_{s+}^t \leq V_{0s} - \underline{V_s} &\quad \forall \ s \in \mathcal{S} \label{eq:sto_limit1} \\ 
    % 0 \leq f_{s-}^t \leq \overline{V_s} - V_{0s} &\quad \forall \ s \in \mathcal{S} \label{eq:sto_limit2} \\ 
    % V_{s}^t = V_{s}^{t-1} + f_{s-}^{t-1} - f_{s+}^{t-1} &\quad \forall \ s \in \mathcal{S} \label{eq:sto_time}\\
    % sgn(f_{p}^t)(f_{p}^t)^2 = K_{nm}((\pi_{n}^t)^2-(\pi_{m}^t)^2) &\quad \forall \ p =(n,m) \in\mathcal{P} \label{eq:weymouth_cons}
\end{alignat}

The constraint set models the gas transportation system: \Cref{eq:well_limits} forces each production well to inject the flow $f_{w}^t$ over the technical minimum $\underline{f_{w}^t}$ and under the maximum capacity $\overline{f_{w}^t}$. \Cref{eq:pipe_limits} upper-bounds the gas flow through pipelines $f_{p}^t$ to the structural capacity $\overline{f_{p}^t}$. \Cref{eq:dem_limit_gas} ensures that the unsupplied demand $f_{u}^{t}$ is lower than the corresponding user demand $\overline{f_{u}^{t}}$. The nodal gas balance in \Cref{eq:gas_balance} guarantees that the gas entering the node $n$ equals the gas leaving it. 

% \Cref{eq:sto_limit1,eq:sto_limit2} limit the gas injection $f_{s+}$  and extraction $f_{s-}$ rates at storage facilities according to the feasible operating range determined by the currently stored volume $V_{s}^t$, respectively. In turn, \Cref{eq:sto_time} balances the gas storage unit such that gas volume at operation period $t$ $V_{s}^t$ equals the volume from period $V_{s}^{t-1}$ plus the difference between injected $f_{s+}^{t-1}$ and extracted $f_{s+}^{t-1}$ gas flow, a fundamental constraint for modeling the dynamics of gas storage over time. 

\section{Experimental Setup} \label{sec:LinealCensnet_ExperimentalSetup}



In this experimental setup, we take the optimization model presented in the previous section and generate samples by introducing noise into the base values of two gas networks. The noise levels range from 5\% to 25\%, applied to the parameters of the networks to simulate varying operating conditions. The first network is a small-scale test network consisting of 8 nodes, while the second represents the Colombian natural gas transportation system, a more extensive and complex network. These networks were used to evaluate the performance of the proposed model in different scenarios.

The generated samples served as training data for a GNN-based model to solve the natural gas transportation problem, which was designed as a faster alternative to the optimization-based model. This GNN model is built to focus on predicting node and edge-level characteristics, incorporating the structure of the network and its connectivity into the learning process. The model explicitly penalizes deviations in node and edge losses, which directly impact gas flow efficiency through the system. To achieve this, the architecture is structured as a multi-layer neural network, with customizable depth (number of layers), channels, and dense layers, ensuring flexibility in adapting to small-scale and large-scale networks, such as the Colombian system.



A general outline of the model can be seen in \cref{fig:model_description} and the basic components of the model are explained below:

\begin{itemize}
    \item \textbf{Input Channels:} The model receives five types of input data:
        
        
    \begin{itemize}
        \item Node Features: A matrix $\in \mathbb{R}^{N \times 3}$, where $N$ is the number of nodes in the network, containing the features of each node. Each node feature includes the lower and upper limits for injected flow, as well as demanded flow.
        \item Node Laplacian: An adjacency matrix of size $N\times N$ encoding the graph structure of the nodes.
        \item Edge Laplacian: A matrix of size $E\times E$ encoding the connections between edges.
        \item Incidence Matrix: A matrix of size $N\times E$ representing the node-edge incidence relationship, mapping the flow of gas between nodes through edges.
        \item Edge Features: A matrix $\in \mathbb{R}^{N \times 3}$ that includes the features of pipelines and compressors in the gas network. Each edge feature includes the $K$ constant, the maximum compression ratio $\beta$, and the upper and lower flow limits.
    \end{itemize}

    \item \textbf{Normalization and Pre\-dense Layers:} The node and edge inputs receive feature-wise normalization to standardize the data. Following this, the inputs are passed through two dense layers, each with $N \ channels$ neurons. The purpose of these pre-dense layers is to transform the feature space before applying the convolutional layers 

    \item \textbf{Convolutional Layers:} The main body of the network consists of \(N\) convolutional blocks. Each block applies a CensNet convolution, which updates both node and edge features by considering the structural relationships encoded in the node and edge Laplacians, as well as the incidence matrix. Batch normalization follows each convolution to stabilize learning. This structure allows the model to capture complex interactions between nodes and edges and propagate information across the graph, learning how local features influence the broader system.
   
    \item \textbf{Post-dense Layer:} After passing through the convolutional blocks, the node and edge features are further processed by a series of dense layers. The number of dense layers $N \ dense$ is adjustable. These layers further refine the learned features, enabling the model to output node and edge-level predictions. 
   
    \item \textbf{Losses and Outputs:} The final outputs of the network are the node-level and edge-level predictions. The node predictions correspond to the estimated flow at each node, while the edge predictions represent the flow along the edges. Both outputs are penalized based on their respective losses, which are calculated by comparing the predicted values to ground truth values and evaluating how well the physical constraints are respected.

The loss functions ensure that the model accurately predicts the node and edge flows while satisfying the physical constraints of the system. These constraints are essential for ensuring that the predicted flows are feasible within the operational limitations of the network. 
    \item \textbf{Model Optimization:} The model is trained using backpropagation with the Adam optimizer. The training process involves minimizing the node and edge loss functions, which penalize incorrect flow predictions and deviations from the expected behavior of the network.

\begin{figure}
    \centering
    \setlength\figurewidth{1\textwidth}        
    \setlength\figureheight{0.5\textwidth}
    \resizebox{\figurewidth}{\figureheight}{\begin{tikzpicture}[shorten >=1pt, ->, draw=black!50, node distance=1.5cm and 3.5cm, align=center]

    % Styles
    \tikzstyle{input} = [rectangle, draw, fill=orange!30, minimum width=3cm, minimum height=1cm]
    \tikzstyle{dense} = [rectangle, draw, fill=blue!30, minimum width=3cm, minimum height=1cm]
    \tikzstyle{conv} = [rectangle, draw, fill=green!30, minimum width=3cm, minimum height=1cm]
    \tikzstyle{output} = [rectangle, draw, fill=purple!30, minimum width=3cm, minimum height=1cm]
    \tikzstyle{loss} = [rectangle, draw, fill=red!30, minimum width=3cm, minimum height=1cm]
    \tikzstyle{arrow} = [->, thick]

    % Input Layer
    \node[input] (node_features) at (0,0) {\(\mathbf{X}\)};
    \node[input] (node_laplacian) [below of=node_features] {\(\mathbf{L}_v\)};
    \node[input] (edge_laplacian) [below of=node_laplacian] {\(\mathbf{L}_e\)};
    \node[input] (incidence_matrix) [below of=edge_laplacian] {\(\mathbf{T}\)};
    \node[input] (edge_features) [below of=incidence_matrix] {\(\mathbf{E}\)};

    % Normalization and Pre-dense Layer
    \node[dense] (norm_pre_dense) [right of=edge_laplacian, xshift=3cm] {Normalization \\ \& Pre-dense Layers};

    % Convolutional Layers
    \node[conv] (conv_layers) [right of=norm_pre_dense, xshift=3cm] {CensNet Blocks \\ (Convolutional Layers)};

    % Post-dense Layer
    \node[dense] (post_dense) [right of=conv_layers, xshift=3cm] {Post-dense Layers};

    % Outputs
    \node[output] (node_output) [right of=post_dense, xshift=3cm, yshift=3cm] {\(\hat{\mathbf{X}}_v\)};
    \node[output] (edge_output) [right of=post_dense, xshift=3cm, yshift=1cm] {\(\hat{\mathbf{X}}_e\)};
    \node[output] (balance_output) [right of=post_dense, xshift=3cm, yshift=-1cm] {\(\mathcal{J}_{\text{balance}}\)};
    \node[output] (weymouth_output) [right of=post_dense, xshift=3cm, yshift=-3cm] {\(\mathcal{J}_{\text{Weymouth}}\)};

    % Losses
    \node[loss] (node_loss) [right of=node_output, xshift=3cm] {Node Loss};
    \node[loss] (edge_loss) [right of=edge_output, xshift=3cm] {Edge Loss};
    \node[loss] (balance_loss) [right of=balance_output, xshift=3cm] {Balance Loss};
    \node[loss] (weymouth_loss) [right of=weymouth_output, xshift=3cm] {Weymouth Loss};

    % Arrows
    \draw[arrow] (node_features) -- (norm_pre_dense);
    \draw[arrow] (node_laplacian) -- (norm_pre_dense);
    \draw[arrow] (edge_laplacian) -- (norm_pre_dense);
    \draw[arrow] (incidence_matrix) -- (norm_pre_dense);
    \draw[arrow] (edge_features) -- (norm_pre_dense);

    \draw[arrow] (norm_pre_dense) -- (conv_layers);
    \draw[arrow] (conv_layers) -- (post_dense);

    \draw[arrow] (post_dense) -- (node_output);
    \draw[arrow] (post_dense) -- (edge_output);
    \draw[arrow] (post_dense) -- (balance_output);
    \draw[arrow] (post_dense) -- (weymouth_output);

    \draw[arrow] (node_output) -- (node_loss);
    \draw[arrow] (edge_output) -- (edge_loss);
    \draw[arrow] (balance_output) -- (balance_loss);
    \draw[arrow] (weymouth_output) -- (weymouth_loss);

\end{tikzpicture}

}
    % \begin{tikzpicture}[shorten >=1pt, ->, draw=black!50, node distance=1.5cm and 3.5cm, align=center]

    % Styles
    \tikzstyle{input} = [circle, draw, fill=green!50, minimum size=2em]
    \tikzstyle{hidden} = [circle, draw, fill=blue!50, minimum size=2em]
    \tikzstyle{output} = [circle, draw, fill=red!50, minimum size=2em]
    \tikzstyle{connection} = [->, thick]

    % Network Stage Labels
    \node[align=center] at (0,-0.4) {Input \\ Layer};
    \node[align=center] at (6,0.4) {Hidden \\ Layers};
    \node[align=center] at (12,-1.2) {Output \\ Layer};

    % Input Layer
    \foreach \i in {1,2,3}
        \node[input] (I\i) at (0,-\i*1.5) {$x_\i$};

    % Hidden Layer 1
    \foreach \i in {1,2,3,4}
        \node[hidden] (H1\i) at (3,-\i*1.5+0.75) {$z^{(1)}_\i$};

    % Hidden Layer 2
    \foreach \i in {1,2,3,4}
        \node[hidden] (H2\i) at (6,-\i*1.5+0.75) {$z^{(2)}_\i$};

    % Hidden Layer 3
    \foreach \i in {1,2,3,4}
        \node[hidden] (H3\i) at (9,-\i*1.5+0.75) {$z^{(3)}_\i$};

    % Output Layer
    \foreach \i in {1,2}
        \node[output] (O\i) at (12,-\i*1.5-0.75) {$\hat{y}_\i$};

    % Connections from Input to Hidden Layer 1
    \foreach \i in {1,2,3}
        \foreach \j in {1,2,3,4}
            \draw[connection] (I\i) -- (H1\j);

    % Connections from Hidden Layer 1 to Hidden Layer 2
    \foreach \i in {1,2,3,4}
        \foreach \j in {1,2,3,4}
            \draw[connection] (H1\i) -- (H2\j);

    % Connections from Hidden Layer 2 to Hidden Layer 3
    \foreach \i in {1,2,3,4}
        \foreach \j in {1,2,3,4}
            \draw[connection] (H2\i) -- (H3\j);

    % Connections from Hidden Layer 3 to Output Layer
    \foreach \i in {1,2,3,4}
        \foreach \j in {1,2}
            \draw[connection] (H3\i) -- (O\j);

\end{tikzpicture}

    \caption{General outline of the CensNet-based model used.}
    \label{fig:model_description}
\end{figure}
\end{itemize}


The network training was done under the following conditions: data was partitioned into training, validation, and test sets, with percentages of 60\%, 20\%, and 20\%. The learning rate schedule followed an Exponential Decay approach, with an initial learning rate of $1 \times 10^{-2}$ decay steps of 1000, and a decay rate of 0.9. The model was trained over 1500 epochs, ensuring the parameters had sufficient time to converge for both node- and edge-level predictions using a Leaky Relu activation function with an alpha parameter of 0.2.


The model utilized several key hyperparameters: $N \ channels$, which corresponds to the units in the pre-dense and convolutional layers (CensNet), N Dense, representing the number of post-dense layers, and $N \ layers$, denoting the number of convolutional layers. Additionally, the model incorporated weights to penalize deviations in the network's response, specifically targeting MSE between the actual flow and the predicted flow at both the nodes and pipelines. Two tests were conducted to determine the optimal set of hyperparameters. The first test considered only the weight associated with node flow preconditions, while the second test included weights for both node and pipeline flows. In each test, the hyperparameters were optimized using the open-source Optuna framework \cite{optuna_2019}, with the following search space: $N \ channels$ ranged from 16 to 64, $N \ layers$ from 1 to 5, and $N \ dense$ from 2 to 32.




\section{Results}

In this section, we present the results of the proposed GNN model, focusing on the relationship between the predicted outputs and the actual observed values in the natural gas transportation networks. The evaluation includes both the 8-node test network and the Colombian natural gas system, with the goal of assessing the model's ability to predict key parameters under varying operational conditions. 


%
% \begin{figure}
%     \centering
%     \begin{tikzpicture}

\definecolor{darkgray176}{RGB}{176,176,176}
\definecolor{lightgray204}{RGB}{204,204,204}

\begin{axis}[
    colorbar,
    colorbar sampled,
    colorbar style={
        samples=8,
        ylabel={node\_id},
        ytick={0,1,...,7},
        yticklabels={0,1,2,3,4,5,6,7},
    },
    colormap={mymap}{[1pt]
        rgb(0pt)=(0.12156862745098,0.466666666666667,0.705882352941177)
        rgb(1pt)=(1,0.498039215686275,0.0549019607843137)
        rgb(2pt)=(0.172549019607843,0.627450980392157,0.172549019607843)
        rgb(3pt)=(0.83921568627451,0.152941176470588,0.156862745098039)
        rgb(4pt)=(0.580392156862745,0.403921568627451,0.741176470588235)
        rgb(5pt)=(0.549019607843137,0.337254901960784,0.294117647058824)
        rgb(6pt)=(0.890196078431372,0.466666666666667,0.76078431372549)
        rgb(7pt)=(0.498039215686275,0.498039215686275,0.498039215686275)
    },
legend cell align={left},
legend style={
  fill opacity=0.8,
  draw opacity=1,
  text opacity=1,
  at={(0.03,0.97)},
  anchor=north west,
  draw=lightgray204
},
point meta max=7,
point meta min=0,
tick align=outside,
tick pos=left,
title={},
x grid style={darkgray176},
xlabel={y true},
xmajorgrids,
xmin=-2.3894448262, xmax=50.1783413502,
% xtick style={color=black},
y grid style={darkgray176},
ylabel={y pred},
ymajorgrids,
ymin=-2.18082528710365, ymax=41.9992832481861,
ytick style={color=black}
]
\addplot [
    colormap={mymap}{[1pt]
        rgb(0pt)=(0.12156862745098,0.466666666666667,0.705882352941177)
        rgb(1pt)=(1,0.498039215686275,0.0549019607843137)
        rgb(2pt)=(0.172549019607843,0.627450980392157,0.172549019607843)
        rgb(3pt)=(0.83921568627451,0.152941176470588,0.156862745098039)
        rgb(4pt)=(0.580392156862745,0.403921568627451,0.741176470588235)
        rgb(5pt)=(0.549019607843137,0.337254901960784,0.294117647058824)
        rgb(6pt)=(0.890196078431372,0.466666666666667,0.76078431372549)
        rgb(7pt)=(0.498039215686275,0.498039215686275,0.498039215686275)
    },
    only marks,
    scatter,
    scatter src=explicit
]
table [x=x, y=y, meta=colordata]{%
x  y  colordata
41.603277614 39.7535934448242 0.0
0 0.0908881425857544 1.0
0 0.139073312282562 2.0
0 0.0316752195358276 3.0
0 -0.0294172167778015 4.0
0 -0.0293693542480469 5.0
0 0.00516742467880249 6.0
0 0.00205957889556885 7.0
37.07638211 38.3591461181641 0.0
0 0.0097118616104126 1.0
0 -0.0219908952713013 2.0
0 -0.0110254883766174 3.0
0 0.00467157363891602 4.0
0 0.00213146209716797 5.0
0 -0.00148683786392212 6.0
0 -0.00642764568328857 7.0
39.245039256 38.5584983825684 0.0
0 0.0252088308334351 1.0
0 0.12731671333313 2.0
0 0.0389952659606934 3.0
0 -0.0199683308601379 4.0
0 -0.0247794985771179 5.0
0 0.00470894575119019 6.0
0 0.00119262933731079 7.0
33.40222464 39.6044502258301 0.0
0 0.0798147916793823 1.0
0 0.0904387831687927 2.0
0 0.0590900778770447 3.0
0 -0.0234290957450867 4.0
0 -0.0263727903366089 5.0
0 0.00418353080749512 6.0
0 0.000778496265411377 7.0
40.177942425 38.2198257446289 0.0
0 0.00198698043823242 1.0
0 0.00119161605834961 2.0
0 -0.0142090320587158 3.0
0 0.00341099500656128 4.0
0 1.15036964416504e-05 5.0
0 0.000508427619934082 6.0
0 -0.00637626647949219 7.0
33.661272263 38.4802513122559 0.0
0 -0.00348472595214844 1.0
0 0.0167836546897888 2.0
0 0.0134863257408142 3.0
0 0.0085369348526001 4.0
0 0.00687175989151001 5.0
0 0.00102633237838745 6.0
0 -0.00393420457839966 7.0
41.918555681 38.3983917236328 0.0
0 -0.00454980134963989 1.0
0 0.0024217963218689 2.0
0 -0.000810742378234863 3.0
0 0.011707603931427 4.0
0 0.00938022136688232 5.0
0 0.00149333477020264 6.0
0 0.000492334365844727 7.0
44.067434787 38.4977531433105 0.0
0 0.00612330436706543 1.0
0 0.00236880779266357 2.0
0 0.00920677185058594 3.0
0 0.0120137929916382 4.0
0 0.0121665596961975 5.0
0 0.000962913036346436 6.0
0 0.000135302543640137 7.0
31.762979365 38.5268859863281 0.0
0 0.00982803106307983 1.0
0 -0.00357627868652344 2.0
0 0.00891649723052979 3.0
0 0.0133818984031677 4.0
0 0.0117686986923218 5.0
0 0.001045823097229 6.0
0 0.000113010406494141 7.0
41.030120159 38.5200500488281 0.0
0 -0.0298738479614258 1.0
0 -0.0113872885704041 2.0
0 0.0223647356033325 3.0
0 -0.0168793201446533 4.0
0 -0.0176181793212891 5.0
0 0.00564652681350708 6.0
0 0.00426644086837769 7.0
40.055714877 38.8306159973145 0.0
0 -7.34925270080566e-05 1.0
0 -0.0110159516334534 2.0
0 0.0156899690628052 3.0
0 0.000327527523040771 4.0
0 -0.00162017345428467 5.0
0 0.00265783071517944 6.0
0 -0.000417232513427734 7.0
40.714244155 38.4069595336914 0.0
0 0.00929087400436401 1.0
0 -0.0149266719818115 2.0
0 0.00734519958496094 3.0
0 0.00849682092666626 4.0
0 0.0102183222770691 5.0
0 0.000343441963195801 6.0
0 -0.00474280118942261 7.0
45.139727698 38.8884696960449 0.0
0 0.000342428684234619 1.0
0 0.00625211000442505 2.0
0 0.00714671611785889 3.0
0 0.00820708274841309 4.0
0 0.00730991363525391 5.0
0 0.00169193744659424 6.0
0 0.000112235546112061 7.0
40.583613059 39.263557434082 0.0
0 0.0559103488922119 1.0
0 0.0864608585834503 2.0
0 0.0712517499923706 3.0
0 -0.0344110727310181 4.0
0 -0.0397318601608276 5.0
0 0.00477957725524902 6.0
0 -0.00180310010910034 7.0
35.441291819 38.5217590332031 0.0
0 -0.00431793928146362 1.0
0 -0.0136868357658386 2.0
0 0.00991904735565186 3.0
0 -0.0215700268745422 4.0
0 -0.0221145749092102 5.0
0 0.00257503986358643 6.0
0 -0.00314760208129883 7.0
36.08841077 38.8402824401855 0.0
0 0.00536257028579712 1.0
0 0.0375674962997437 2.0
0 0.0245164632797241 3.0
0 -0.00596868991851807 4.0
0 -0.00369197130203247 5.0
0 0.00242996215820312 6.0
0 0.00174403190612793 7.0
40.494343306 38.6495323181152 0.0
0 -0.00675559043884277 1.0
0 -0.005912184715271 2.0
0 -0.00215411186218262 3.0
0 0.00915825366973877 4.0
0 0.00896823406219482 5.0
0 0.00157058238983154 6.0
0 0.00130730867385864 7.0
44.400196583 38.5482749938965 0.0
0 0.00762939453125 1.0
0 0.00384992361068726 2.0
0 0.0038907527923584 3.0
0 0.0148396492004395 4.0
0 0.0139151811599731 5.0
0 0.00187838077545166 6.0
0 0.000854194164276123 7.0
40.197028842 38.6482429504395 0.0
0 -0.00523370504379272 1.0
0 0.0368385314941406 2.0
0 0.0156623125076294 3.0
0 0.00713348388671875 4.0
0 0.00676637887954712 5.0
0 0.000627398490905762 6.0
0 -0.000256776809692383 7.0
39.931867517 38.7960472106934 0.0
0 -0.00283390283584595 1.0
0 -0.00708603858947754 2.0
0 -0.00414395332336426 3.0
0 0.00981372594833374 4.0
0 0.00716376304626465 5.0
0 0.0015750527381897 6.0
0 0.000669717788696289 7.0
37.28322745 38.2965774536133 0.0
0 0.00495332479476929 1.0
0 -0.00355356931686401 2.0
0 0.0140035152435303 3.0
0 0.011663019657135 4.0
0 0.00884330272674561 5.0
0 0.00177943706512451 6.0
0 0.00082862377166748 7.0
37.429390014 38.4880485534668 0.0
0 0.0135406255722046 1.0
0 0.00302684307098389 2.0
0 0.0197659134864807 3.0
0 -0.00780308246612549 4.0
0 -0.00855451822280884 5.0
0 0.00318622589111328 6.0
0 0.000750601291656494 7.0
39.499146179 38.6164703369141 0.0
0 0.00107491016387939 1.0
0 -0.00115358829498291 2.0
0 0.0128684639930725 3.0
0 0.0100945830345154 4.0
0 0.00789058208465576 5.0
0 0.000394701957702637 6.0
0 -0.00226247310638428 7.0
35.661352504 38.4398956298828 0.0
0 0.0203315615653992 1.0
0 0.0788954496383667 2.0
0 -0.0757306814193726 3.0
0 -0.0765065550804138 4.0
0 -0.0808522701263428 5.0
0 0.0057339072227478 6.0
0 0.00214129686355591 7.0
44.903668391 38.679759979248 0.0
0 0.0104649066925049 1.0
0 -0.00321555137634277 2.0
0 0.00385379791259766 3.0
0 0.0129628777503967 4.0
0 0.0148612260818481 5.0
0 0.00193989276885986 6.0
0 0.00181066989898682 7.0
34.118473664 39.0513191223145 0.0
0 0.0307442545890808 1.0
0 0.0471727848052979 2.0
0 0.0385069847106934 3.0
0 -0.0240370631217957 4.0
0 -0.0236438512802124 5.0
0 0.00260704755783081 6.0
0 0.011364221572876 7.0
36.466593361 38.987190246582 0.0
0 -0.0103698968887329 1.0
0 0.0800180435180664 2.0
0 0.0414098501205444 3.0
0 -0.0267103314399719 4.0
0 -0.0288975834846497 5.0
0 0.00566142797470093 6.0
0 0.0028914213180542 7.0
34.994247693 38.3411102294922 0.0
0 0.00688672065734863 1.0
0 -0.00696879625320435 2.0
0 0.0116889476776123 3.0
0 0.00811731815338135 4.0
0 0.0102164149284363 5.0
0 0.00122654438018799 6.0
0 0.00214976072311401 7.0
39.553868661 38.6714859008789 0.0
0 -0.00435274839401245 1.0
0 -0.000196635723114014 2.0
0 -0.00985980033874512 3.0
0 0.0103663206100464 4.0
0 0.0117907524108887 5.0
0 0.0014764666557312 6.0
0 0.00132846832275391 7.0
43.158887007 38.4424667358398 0.0
0 0.00920790433883667 1.0
0 -0.00932449102401733 2.0
0 0.0151568651199341 3.0
0 0.0135409832000732 4.0
0 0.00888550281524658 5.0
0 0.00103974342346191 6.0
0 -0.00307869911193848 7.0
36.693891456 39.0564842224121 0.0
0 0.00571656227111816 1.0
0 0.0139431953430176 2.0
0 0.00666135549545288 3.0
0 0.00349867343902588 4.0
0 0.000229835510253906 5.0
0 0.00237119197845459 6.0
0 -0.000682830810546875 7.0
37.419072323 38.5162734985352 0.0
0 0.0100835561752319 1.0
0 -0.00280958414077759 2.0
0 0.011082649230957 3.0
0 0.0125146508216858 4.0
0 0.0106852650642395 5.0
0 0.000358998775482178 6.0
0 -0.00108003616333008 7.0
42.391154887 37.6930847167969 0.0
0 0.00836706161499023 1.0
0 -0.0265249013900757 2.0
0 0.0109959840774536 3.0
0 -0.00843185186386108 4.0
0 -0.00866538286209106 5.0
0 0.00374925136566162 6.0
0 0.00252997875213623 7.0
39.976124969 39.3020973205566 0.0
0 0.0369466543197632 1.0
0 0.0207736492156982 2.0
0 0.00882303714752197 3.0
0 -0.0113847255706787 4.0
0 -0.0140944719314575 5.0
0 0.00480204820632935 6.0
0 0.00208133459091187 7.0
41.193094756 38.7121925354004 0.0
0 0.000246524810791016 1.0
0 0.0112533569335938 2.0
0 -0.00880825519561768 3.0
0 0.0115809440612793 4.0
0 0.00994521379470825 5.0
0 0.00153958797454834 6.0
0 0.000614643096923828 7.0
39.030467364 38.9754753112793 0.0
0 0.0133965015411377 1.0
0 0.0099719762802124 2.0
0 -0.0317625403404236 3.0
0 -0.0213124752044678 4.0
0 -0.0274962782859802 5.0
0 0.0034407377243042 6.0
0 -0.00381076335906982 7.0
39.254434013 38.8635673522949 0.0
0 -0.00899428129196167 1.0
0 0.028767466545105 2.0
0 -0.00187152624130249 3.0
0 0.00816291570663452 4.0
0 0.00780004262924194 5.0
0 0.00216001272201538 6.0
0 0.00177603960037231 7.0
41.474829476 38.490550994873 0.0
0 -0.00636947154998779 1.0
0 0.0157142877578735 2.0
0 -0.00421321392059326 3.0
0 0.0130150318145752 4.0
0 0.0113214254379272 5.0
0 0.00152826309204102 6.0
0 0.000341415405273438 7.0
36.065024548 38.5721321105957 0.0
0 0.00696414709091187 1.0
0 -0.00372463464736938 2.0
0 0.00787001848220825 3.0
0 0.00936907529830933 4.0
0 0.0126588344573975 5.0
0 0.00105565786361694 6.0
0 0.00142580270767212 7.0
33.547657937 39.1791458129883 0.0
0 0.0251626968383789 1.0
0 0.106061309576035 2.0
0 0.0769014954566956 3.0
0 -0.025641918182373 4.0
0 -0.0292165875434875 5.0
0 0.00278604030609131 6.0
0 -0.0029417872428894 7.0
35.860060933 38.5283851623535 0.0
0 -0.00231927633285522 1.0
0 -0.00571024417877197 2.0
0 -0.00389313697814941 3.0
0 0.00245767831802368 4.0
0 0.00144177675247192 5.0
0 0.00015634298324585 6.0
0 -0.00504642724990845 7.0
41.446935888 38.7161064147949 0.0
0 0.0399006009101868 1.0
0 0.118380934000015 2.0
0 -0.0819071531295776 3.0
0 -0.0782929062843323 4.0
0 -0.0900070071220398 5.0
0 0.00521427392959595 6.0
0 0.00152796506881714 7.0
39.09863661 38.6100082397461 0.0
0 -0.0173571109771729 1.0
0 0.0087469220161438 2.0
0 -0.0296517014503479 3.0
0 0.0122166275978088 4.0
0 0.00985217094421387 5.0
0 0.00263488292694092 6.0
0 0.00131088495254517 7.0
37.209660367 38.619556427002 0.0
0 -0.00431281328201294 1.0
0 0.0119017362594604 2.0
0 0.0246157646179199 3.0
0 0.00228214263916016 4.0
0 -0.00246334075927734 5.0
0 0.00261741876602173 6.0
0 -0.00559180974960327 7.0
37.318344805 38.4069023132324 0.0
0 0.00742429494857788 1.0
0 -0.00190436840057373 2.0
0 0.0183868408203125 3.0
0 0.00921815633773804 4.0
0 0.00853586196899414 5.0
0 0.000802695751190186 6.0
0 -0.00107866525650024 7.0
37.542238634 38.8763580322266 0.0
0 -0.00635802745819092 1.0
0 0.0211170315742493 2.0
0 0.00130510330200195 3.0
0 0.0125030279159546 4.0
0 0.00436419248580933 5.0
0 0.00310200452804565 6.0
0 0.00108665227890015 7.0
35.642638424 38.4852333068848 0.0
0 0.00880211591720581 1.0
0 -0.0149785280227661 2.0
0 0.00123864412307739 3.0
0 0.0102723836898804 4.0
0 0.00305092334747314 5.0
0 0.000930190086364746 6.0
0 -0.00708687305450439 7.0
47.4260719 38.6907234191895 0.0
0 -0.00334429740905762 1.0
0 -0.0110288262367249 2.0
0 -0.0155019164085388 3.0
0 -0.0171687602996826 4.0
0 -0.0232598185539246 5.0
0 0.00268292427062988 6.0
0 -0.00473958253860474 7.0
42.664227047 38.3595390319824 0.0
0 0.00736522674560547 1.0
0 0.169191211462021 2.0
0 -0.0193517208099365 3.0
0 -0.00202935934066772 4.0
0 -0.00715571641921997 5.0
0 0.000342905521392822 6.0
0 -0.00674760341644287 7.0
37.881981315 38.4804840087891 0.0
0 0.00734597444534302 1.0
0 0.0022200345993042 2.0
0 0.00960803031921387 3.0
0 0.0106253623962402 4.0
0 0.0120579600334167 5.0
0 0.00171905755996704 6.0
0 0.00124293565750122 7.0
44.031482397 38.7509765625 0.0
0 -0.0097581148147583 1.0
0 -0.0636894702911377 2.0
0 0.0015978217124939 3.0
0 -0.00409984588623047 4.0
0 -0.00204712152481079 5.0
0 0.00405192375183105 6.0
0 0.0043146014213562 7.0
40.215687867 38.6358032226562 0.0
0 0.00362402200698853 1.0
0 0.00420552492141724 2.0
0 0.000331521034240723 3.0
0 0.00297117233276367 4.0
0 0.000524163246154785 5.0
0 0.000945329666137695 6.0
0 -0.00605976581573486 7.0
35.401137883 38.9912109375 0.0
0 -0.00200784206390381 1.0
0 -0.000412046909332275 2.0
0 -0.0133203268051147 3.0
0 0.0117408633232117 4.0
0 0.00856119394302368 5.0
0 0.0021214485168457 6.0
0 -0.000339329242706299 7.0
40.23529515 38.3903541564941 0.0
0 0.0114810466766357 1.0
0 -0.00885409116744995 2.0
0 0.0132594108581543 3.0
0 0.0202721357345581 4.0
0 0.0152925848960876 5.0
0 0.000202775001525879 6.0
0 -0.00713396072387695 7.0
38.451822651 38.8377914428711 0.0
0 0.00414824485778809 1.0
0 -0.00228339433670044 2.0
0 0.00336885452270508 3.0
0 0.0121934413909912 4.0
0 0.00823593139648438 5.0
0 0.00146770477294922 6.0
0 -0.00342243909835815 7.0
41.356859392 38.3796310424805 0.0
0 0.00926601886749268 1.0
0 -0.0124003291130066 2.0
0 0.0076671838760376 3.0
0 0.0100225806236267 4.0
0 0.00797367095947266 5.0
0 0.00054556131362915 6.0
0 -0.00379371643066406 7.0
44.45602451 38.879566192627 0.0
0 0.00672298669815063 1.0
0 0.0123946666717529 2.0
0 -0.00913697481155396 3.0
0 0.0109050869941711 4.0
0 0.00889235734939575 5.0
0 0.00168204307556152 6.0
0 0.000749766826629639 7.0
36.344631329 38.5415077209473 0.0
0 0.00494116544723511 1.0
0 0.00470906496047974 2.0
0 0.00875461101531982 3.0
0 0.00978201627731323 4.0
0 0.00804340839385986 5.0
0 0.00102525949478149 6.0
0 -0.00314891338348389 7.0
35.989587139 39.0207290649414 0.0
0 -0.00257343053817749 1.0
0 -0.00773054361343384 2.0
0 -0.00264960527420044 3.0
0 0.00988805294036865 4.0
0 0.0075453519821167 5.0
0 0.00126409530639648 6.0
0 -0.000481665134429932 7.0
38.453275767 38.5509948730469 0.0
0 -0.00460779666900635 1.0
0 0.0267020463943481 2.0
0 -0.00305271148681641 3.0
0 0.00985264778137207 4.0
0 0.00969010591506958 5.0
0 0.00131440162658691 6.0
0 0.000271499156951904 7.0
36.388812201 38.5625762939453 0.0
0 0.00684624910354614 1.0
0 0.00579524040222168 2.0
0 0.00857532024383545 3.0
0 0.00900334119796753 4.0
0 0.0114079117774963 5.0
0 0.00167042016983032 6.0
0 0.00194859504699707 7.0
45.540732617 38.3559417724609 0.0
0 0.00538235902786255 1.0
0 0.00386589765548706 2.0
0 -0.00125622749328613 3.0
0 0.018405020236969 4.0
0 0.0137466788291931 5.0
0 0.00278580188751221 6.0
0 0.000956237316131592 7.0
34.414627117 38.921630859375 0.0
0 0.0324609279632568 1.0
0 0.0945376753807068 2.0
0 0.0671876668930054 3.0
0 -0.0498137474060059 4.0
0 -0.0511236786842346 5.0
0 0.0085570216178894 6.0
0 0.00632810592651367 7.0
39.317888978 38.3244590759277 0.0
0 -0.00816702842712402 1.0
0 0.0362486243247986 2.0
0 0.0158848762512207 3.0
0 0.00621277093887329 4.0
0 0.00248837471008301 5.0
0 0.00199019908905029 6.0
0 -0.00439673662185669 7.0
36.969505628 39.0306816101074 0.0
0 -0.00373166799545288 1.0
0 0.0273987650871277 2.0
0 -0.0015103816986084 3.0
0 0.00917762517929077 4.0
0 0.00721776485443115 5.0
0 0.00152438879013062 6.0
0 -0.000276803970336914 7.0
44.923617033 39.2973556518555 0.0
0 0.00290322303771973 1.0
0 0.0194564461708069 2.0
0 0.0318148732185364 3.0
0 -0.0022081732749939 4.0
0 -0.00278234481811523 5.0
0 0.00196754932403564 6.0
0 -0.000810801982879639 7.0
37.258159228 38.1323928833008 0.0
0 0.0120348930358887 1.0
0 -0.0394316911697388 2.0
0 -0.050422191619873 3.0
0 -0.0273675322532654 4.0
0 -0.0310018062591553 5.0
0 0.00130391120910645 6.0
0 -0.0026056170463562 7.0
47.488576672 38.4616012573242 0.0
0 0.013430118560791 1.0
0 0.0567920207977295 2.0
0 0.0599616765975952 3.0
0 -0.0214373469352722 4.0
0 -0.0249693393707275 5.0
0 0.00472307205200195 6.0
0 -0.000217974185943604 7.0
42.958399331 38.6427192687988 0.0
0 0.00441265106201172 1.0
0 0.00464749336242676 2.0
0 0.000767230987548828 3.0
0 0.0110027194023132 4.0
0 0.0115892887115479 5.0
0 0.00038456916809082 6.0
0 -0.000238478183746338 7.0
41.161203287 38.1443328857422 0.0
0 0.00848382711410522 1.0
0 -0.0180635452270508 2.0
0 -0.00918519496917725 3.0
0 0.00684559345245361 4.0
0 0.0074118971824646 5.0
0 0.000457406044006348 6.0
0 -0.00672441720962524 7.0
40.466838645 38.8036346435547 0.0
0 -0.00796282291412354 1.0
0 -0.0537744164466858 2.0
0 0.0214377641677856 3.0
0 -0.0128311514854431 4.0
0 -0.0138890147209167 5.0
0 0.00499492883682251 6.0
0 0.00396305322647095 7.0
33.757389454 38.6834144592285 0.0
0 0.028232216835022 1.0
0 0.0404295325279236 2.0
0 0.0279163718223572 3.0
0 -0.0216402411460876 4.0
0 -0.0251529216766357 5.0
0 0.00418293476104736 6.0
0 -0.0031440258026123 7.0
43.437770592 38.343563079834 0.0
0 0.00211387872695923 1.0
0 -0.000490009784698486 2.0
0 -0.00304973125457764 3.0
0 0.00690853595733643 4.0
0 0.00559860467910767 5.0
0 0.000310838222503662 6.0
0 -0.00541353225708008 7.0
39.308931201 38.8775672912598 0.0
0 0.00970524549484253 1.0
0 0.00253796577453613 2.0
0 -0.0164980292320251 3.0
0 0.0100165009498596 4.0
0 0.00814670324325562 5.0
0 0.00208121538162231 6.0
0 0.00108212232589722 7.0
40.600604276 38.547607421875 0.0
0 0.00340026617050171 1.0
0 0.00650149583816528 2.0
0 -0.00628137588500977 3.0
0 0.0181875228881836 4.0
0 0.0199186205863953 5.0
0 0.00151854753494263 6.0
0 0.000111699104309082 7.0
31.520377836 38.581787109375 0.0
0 -0.00357002019882202 1.0
0 0.00153231620788574 2.0
0 0.002410888671875 3.0
0 0.00847429037094116 4.0
0 0.0135117173194885 5.0
0 -0.000303924083709717 6.0
0 0.000370502471923828 7.0
39.969125591 38.6257934570312 0.0
0 -0.00313878059387207 1.0
0 0.0176121592521667 2.0
0 -0.000237464904785156 3.0
0 -0.0116665363311768 4.0
0 -0.0119089484214783 5.0
0 -4.19020652770996e-05 6.0
0 -0.00395911931991577 7.0
30.912106182 38.2456321716309 0.0
0 -0.00355678796768188 1.0
0 0.0208463668823242 2.0
0 0.00883865356445312 3.0
0 0.0121186971664429 4.0
0 0.0138341188430786 5.0
0 0.00130367279052734 6.0
0 0.000901222229003906 7.0
45.180443811 38.3854637145996 0.0
0 -0.00685745477676392 1.0
0 0.0200044512748718 2.0
0 0.0115278959274292 3.0
0 -0.00447863340377808 4.0
0 -0.00777232646942139 5.0
0 0.00198572874069214 6.0
0 -0.00468242168426514 7.0
34.99216559 38.5155220031738 0.0
0 0.00417304039001465 1.0
0 0.0240988731384277 2.0
0 0.0120140314102173 3.0
0 -0.00103104114532471 4.0
0 -0.00382125377655029 5.0
0 0.00195038318634033 6.0
0 -0.0034911036491394 7.0
40.933938659 38.6754302978516 0.0
0 0.00394058227539062 1.0
0 0.0120728611946106 2.0
0 0.00304794311523438 3.0
0 -0.00491464138031006 4.0
0 -0.00522351264953613 5.0
0 0.00198507308959961 6.0
0 -0.00257027149200439 7.0
40.317005952 38.9766120910645 0.0
0 0.00107008218765259 1.0
0 0.00469475984573364 2.0
0 -0.010001540184021 3.0
0 0.00944429636001587 4.0
0 0.00491970777511597 5.0
0 0.00197219848632812 6.0
0 0.000297665596008301 7.0
40.010379617 38.5953788757324 0.0
0 -0.00856190919876099 1.0
0 0.0215948820114136 2.0
0 -0.000319242477416992 3.0
0 0.0154175162315369 4.0
0 0.00876986980438232 5.0
0 0.00185507535934448 6.0
0 -0.00385445356369019 7.0
38.685596488 38.8159713745117 0.0
0 -0.00294101238250732 1.0
0 0.00725901126861572 2.0
0 -0.0348896980285645 3.0
0 0.0134746432304382 4.0
0 0.00706773996353149 5.0
0 0.00252926349639893 6.0
0 -0.00183933973312378 7.0
30.278486108 39.0315551757812 0.0
0 -0.00075840950012207 1.0
0 0.0177011489868164 2.0
0 0.0114507675170898 3.0
0 0.0057494044303894 4.0
0 0.00559014081954956 5.0
0 0.00110125541687012 6.0
0 7.3552131652832e-05 7.0
40.311093813 37.9903717041016 0.0
0 0.000359892845153809 1.0
0 0.00946557521820068 2.0
0 0.00448530912399292 3.0
0 0.010830819606781 4.0
0 0.00182676315307617 5.0
0 0.00240612030029297 6.0
0 -0.00593709945678711 7.0
35.760290532 38.7118949890137 0.0
0 0.00812315940856934 1.0
0 0.0297279953956604 2.0
0 -0.0139551758766174 3.0
0 -0.014398992061615 4.0
0 -0.020226776599884 5.0
0 0.00332540273666382 6.0
0 -0.00403028726577759 7.0
40.236813436 38.5688896179199 0.0
0 -0.00127357244491577 1.0
0 -0.00444197654724121 2.0
0 0.000873267650604248 3.0
0 0.00886917114257812 4.0
0 0.0087096095085144 5.0
0 0.00108206272125244 6.0
0 0.000600993633270264 7.0
43.948095531 38.5622749328613 0.0
0 0.00688648223876953 1.0
0 0.0145112872123718 2.0
0 -0.00303208827972412 3.0
0 0.0120838284492493 4.0
0 0.0151082277297974 5.0
0 0.000646591186523438 6.0
0 9.0181827545166e-05 7.0
40.202612729 38.9503021240234 0.0
0 0.00446075201034546 1.0
0 0.00278854370117188 2.0
0 -0.00881409645080566 3.0
0 0.00978708267211914 4.0
0 0.0118529200553894 5.0
0 0.00184690952301025 6.0
0 0.00139963626861572 7.0
41.441227091 38.7626190185547 0.0
0 0.00998771190643311 1.0
0 0.014593780040741 2.0
0 0.037828803062439 3.0
0 -0.0178765058517456 4.0
0 -0.0181952714920044 5.0
0 0.00372391939163208 6.0
0 -0.000807821750640869 7.0
40.234798657 38.8709907531738 0.0
0 0.0301759243011475 1.0
0 0.0436609983444214 2.0
0 -0.0280125141143799 3.0
0 -0.0313872694969177 4.0
0 -0.0318161249160767 5.0
0 0.00301194190979004 6.0
0 -0.00238752365112305 7.0
40.268402112 38.6101875305176 0.0
0 0.0056917667388916 1.0
0 0.0238692760467529 2.0
0 0.016819953918457 3.0
0 0.00480717420578003 4.0
0 0.00369954109191895 5.0
0 0.00116813182830811 6.0
0 -0.00442075729370117 7.0
34.447402114 38.8466873168945 0.0
0 0.0301073789596558 1.0
0 0.00375080108642578 2.0
0 0.00299829244613647 3.0
0 -0.0253135561943054 4.0
0 -0.0322710871696472 5.0
0 0.0043519139289856 6.0
0 -0.00207918882369995 7.0
40.416533901 38.6745910644531 0.0
0 0.00224298238754272 1.0
0 -0.000726759433746338 2.0
0 0.0128074884414673 3.0
0 0.0108352303504944 4.0
0 0.0103694200515747 5.0
0 0.000999271869659424 6.0
0 -0.00330287218093872 7.0
40.837884543 38.8132362365723 0.0
0 0.00295329093933105 1.0
0 -0.00834047794342041 2.0
0 0.0268865823745728 3.0
0 0.000196278095245361 4.0
0 -0.005332350730896 5.0
0 0.00294673442840576 6.0
0 -0.00294685363769531 7.0
39.823092444 37.8871421813965 0.0
0 0.000456392765045166 1.0
0 -0.00625967979431152 2.0
0 -0.0234071016311646 3.0
0 -0.00285905599594116 4.0
0 -0.00193220376968384 5.0
0 -0.00192928314208984 6.0
0 -0.00620365142822266 7.0
40.41620498 38.7058410644531 0.0
0 -0.00521928071975708 1.0
0 0.000878453254699707 2.0
0 0.0232111811637878 3.0
0 0.0046045184135437 4.0
0 -0.00023186206817627 5.0
0 0.00172221660614014 6.0
0 -0.004547119140625 7.0
33.388504976 38.8013496398926 0.0
0 -0.00490099191665649 1.0
0 0.00648212432861328 2.0
0 -0.0126960277557373 3.0
0 0.015871524810791 4.0
0 0.0092160701751709 5.0
0 0.00187402963638306 6.0
0 -0.00270891189575195 7.0
42.251691394 38.6184730529785 0.0
0 0.0312210917472839 1.0
0 0.00582593679428101 2.0
0 0.0208268165588379 3.0
0 -0.00345629453659058 4.0
0 -0.00656622648239136 5.0
0 0.00353902578353882 6.0
0 -0.00196647644042969 7.0
32.552772789 38.5892791748047 0.0
0 -0.00613367557525635 1.0
0 0.0275079011917114 2.0
0 -0.0156198143959045 3.0
0 -0.0119390487670898 4.0
0 -0.00980442762374878 5.0
0 0.000374138355255127 6.0
0 -0.00487136840820312 7.0
43.396109959 38.253360748291 0.0
0 0.0250413417816162 1.0
0 0.020060658454895 2.0
0 -0.0217271447181702 3.0
0 0.00134474039077759 4.0
0 -0.00472259521484375 5.0
0 0.00346243381500244 6.0
0 -0.00104355812072754 7.0
31.629368094 38.9830894470215 0.0
0 0.00522613525390625 1.0
0 0.000475287437438965 2.0
0 -0.011349081993103 3.0
0 0.01133793592453 4.0
0 0.0121121406555176 5.0
0 0.00200116634368896 6.0
0 0.00154191255569458 7.0
37.779811495 38.5338363647461 0.0
0 -0.0108740329742432 1.0
0 -0.0369836091995239 2.0
0 0.0149946808815002 3.0
0 -0.00636374950408936 4.0
0 -0.00593376159667969 5.0
0 0.00299978256225586 6.0
0 0.00320684909820557 7.0
34.24652714 38.4251098632812 0.0
0 0.00578612089157104 1.0
0 -0.00152945518493652 2.0
0 0.0102679133415222 3.0
0 0.0122678875923157 4.0
0 0.0114915370941162 5.0
0 0.00135761499404907 6.0
0 0.000555694103240967 7.0
40.003780762 39.0511894226074 0.0
0 -0.0189368724822998 1.0
0 0.0200136303901672 2.0
0 -0.0185563564300537 3.0
0 0.0121176242828369 4.0
0 0.0117418766021729 5.0
0 0.00249522924423218 6.0
0 0.00155913829803467 7.0
31.970353396 38.9871673583984 0.0
0 -0.00729846954345703 1.0
0 0.0101170539855957 2.0
0 -0.0196437239646912 3.0
0 0.0132386088371277 4.0
0 0.0108768343925476 5.0
0 0.00228655338287354 6.0
0 0.0010066032409668 7.0
37.935925517 38.9553871154785 0.0
0 -0.00871318578720093 1.0
0 0.0280109643936157 2.0
0 -0.025648832321167 3.0
0 0.0104709267616272 4.0
0 0.0101121068000793 5.0
0 0.00271910429000854 6.0
0 0.00231039524078369 7.0
37.4568261 38.9331321716309 0.0
0 -0.00507557392120361 1.0
0 -0.000133514404296875 2.0
0 0.00147175788879395 3.0
0 0.00708097219467163 4.0
0 0.00657325983047485 5.0
0 0.00235539674758911 6.0
0 0.0013958215713501 7.0
34.938845635 38.2435569763184 0.0
0 0.00165331363677979 1.0
0 -0.00227940082550049 2.0
0 0.00175893306732178 3.0
0 0.0102068185806274 4.0
0 0.00741416215896606 5.0
0 0.00100028514862061 6.0
0 -0.00547868013381958 7.0
33.644682189 39.0608100891113 0.0
0 0.0215523242950439 1.0
0 0.0741968750953674 2.0
0 0.0108304619789124 3.0
0 -0.00966185331344604 4.0
0 -0.0103352665901184 5.0
0 0.00338512659072876 6.0
0 0.000349283218383789 7.0
43.152629724 38.6673278808594 0.0
0 -0.015493631362915 1.0
0 0.0307546257972717 2.0
0 0.00604945421218872 3.0
0 0.00886774063110352 4.0
0 0.010189414024353 5.0
0 0.00204598903656006 6.0
0 0.00179809331893921 7.0
29.29034141 38.7699356079102 0.0
0 0.0308585166931152 1.0
0 0.0640302300453186 2.0
0 0.0491317510604858 3.0
0 -0.0297149419784546 4.0
0 -0.0343635082244873 5.0
0 0.00418633222579956 6.0
0 -0.00212037563323975 7.0
36.529945045 38.9532089233398 0.0
0 0.0035216212272644 1.0
0 -0.0379787087440491 2.0
0 0.0298681259155273 3.0
0 -0.011357307434082 4.0
0 -0.0170662999153137 5.0
0 0.00557559728622437 6.0
0 0.00102359056472778 7.0
39.490028245 38.2533340454102 0.0
0 0.00501155853271484 1.0
0 0.0158194303512573 2.0
0 -0.0231708884239197 3.0
0 -0.00452101230621338 4.0
0 -0.0123972296714783 5.0
0 0.00276535749435425 6.0
0 -0.00519418716430664 7.0
34.709507968 38.3433647155762 0.0
0 0.00644981861114502 1.0
0 0.0225534439086914 2.0
0 0.0125881433486938 3.0
0 0.00445210933685303 4.0
0 0.00744950771331787 5.0
0 0.00118499994277954 6.0
0 -0.00236207246780396 7.0
45.017599836 38.2805442810059 0.0
0 0.00768566131591797 1.0
0 0.00354486703872681 2.0
0 -0.0179717540740967 3.0
0 -0.0011255145072937 4.0
0 -0.00349807739257812 5.0
0 0.000170648097991943 6.0
0 -0.00597703456878662 7.0
32.940669776 38.5253715515137 0.0
0 -0.00512635707855225 1.0
0 0.0313267707824707 2.0
0 -0.00119996070861816 3.0
0 -0.00267255306243896 4.0
0 -0.00358939170837402 5.0
0 -0.00215828418731689 6.0
0 -0.00541460514068604 7.0
44.072190389 38.780647277832 0.0
0 0.0452917814254761 1.0
0 -0.0119504332542419 2.0
0 0.0499585866928101 3.0
0 -0.0182744860649109 4.0
0 -0.0202071666717529 5.0
0 0.0040593147277832 6.0
0 -0.00122308731079102 7.0
35.422022456 38.5963363647461 0.0
0 1.99079513549805e-05 1.0
0 -0.0265647768974304 2.0
0 0.0021822452545166 3.0
0 -0.0189751386642456 4.0
0 -0.01978600025177 5.0
0 0.000371932983398438 6.0
0 -0.00270688533782959 7.0
36.153594117 39.0233688354492 0.0
0 -0.00477045774459839 1.0
0 0.0245848894119263 2.0
0 -0.0127134323120117 3.0
0 0.0103989243507385 4.0
0 0.0108460187911987 5.0
0 0.00221699476242065 6.0
0 0.00128203630447388 7.0
38.951377458 38.6069030761719 0.0
0 0.00339537858963013 1.0
0 -0.00315690040588379 2.0
0 0.0101865530014038 3.0
0 0.00888776779174805 4.0
0 0.00692009925842285 5.0
0 0.00140982866287231 6.0
0 -0.00287401676177979 7.0
40.644172477 38.4279823303223 0.0
0 -0.00347089767456055 1.0
0 -0.00854629278182983 2.0
0 -0.00977569818496704 3.0
0 0.00184166431427002 4.0
0 -0.00165468454360962 5.0
0 -0.00119912624359131 6.0
0 -0.00389707088470459 7.0
35.600531382 38.9765777587891 0.0
0 0.000101447105407715 1.0
0 -0.00169456005096436 2.0
0 -0.0105034112930298 3.0
0 0.0159684419631958 4.0
0 0.0144085288047791 5.0
0 0.00176870822906494 6.0
0 -0.000409126281738281 7.0
41.682033599 38.8813819885254 0.0
0 0.0172576904296875 1.0
0 0.0325183868408203 2.0
0 -0.0265422463417053 3.0
0 0.0114198923110962 4.0
0 0.00883734226226807 5.0
0 0.00202566385269165 6.0
0 -0.00066763162612915 7.0
39.697348938 38.9553298950195 0.0
0 0.00576400756835938 1.0
0 0.00627702474594116 2.0
0 -0.0105857849121094 3.0
0 0.0120393037796021 4.0
0 0.00906950235366821 5.0
0 0.00216853618621826 6.0
0 0.00109446048736572 7.0
40.978668493 37.8965797424316 0.0
0 -0.00433230400085449 1.0
0 0.00928252935409546 2.0
0 -0.0110325813293457 3.0
0 -0.010481059551239 4.0
0 -0.00912737846374512 5.0
0 0.000316202640533447 6.0
0 -0.00422096252441406 7.0
46.524483465 38.4956130981445 0.0
0 0.0072096586227417 1.0
0 -0.00128740072250366 2.0
0 0.00465559959411621 3.0
0 0.0165053009986877 4.0
0 0.00841742753982544 5.0
0 0.00209641456604004 6.0
0 -0.0013575553894043 7.0
39.683969876 38.3914260864258 0.0
0 0.00807690620422363 1.0
0 -0.0101870894432068 2.0
0 0.00545310974121094 3.0
0 0.014818549156189 4.0
0 0.0115963220596313 5.0
0 0.00100439786911011 6.0
0 -0.00206196308135986 7.0
46.294692767 39.0759162902832 0.0
0 0.0209541916847229 1.0
0 -0.00489115715026855 2.0
0 0.0438747406005859 3.0
0 -0.0146594643592834 4.0
0 -0.01612788438797 5.0
0 0.00321435928344727 6.0
0 -0.000924825668334961 7.0
44.785541844 38.8168334960938 0.0
0 0.00351965427398682 1.0
0 0.00326263904571533 2.0
0 0.0207355618476868 3.0
0 0.00808155536651611 4.0
0 0.00271213054656982 5.0
0 0.00201249122619629 6.0
0 -0.0048290491104126 7.0
33.676333572 38.2910499572754 0.0
0 0.00158512592315674 1.0
0 -0.00176751613616943 2.0
0 -0.00117731094360352 3.0
0 0.0104479193687439 4.0
0 0.0066227912902832 5.0
0 0.0014575719833374 6.0
0 -0.00686109066009521 7.0
35.625618922 38.0471572875977 0.0
0 0.00239014625549316 1.0
0 -0.0126388072967529 2.0
0 -0.0120697021484375 3.0
0 0.00128597021102905 4.0
0 0.00199592113494873 5.0
0 -0.000860452651977539 6.0
0 -0.00589412450790405 7.0
40.226157029 38.6796798706055 0.0
0 0.00153261423110962 1.0
0 0.00714719295501709 2.0
0 0.00271427631378174 3.0
0 0.0039246678352356 4.0
0 0.00176608562469482 5.0
0 0.0020751953125 6.0
0 -0.00623393058776855 7.0
44.003195919 38.717472076416 0.0
0 0.00164341926574707 1.0
0 0.00996559858322144 2.0
0 -0.00653207302093506 3.0
0 0.00345414876937866 4.0
0 0.00210237503051758 5.0
0 0.00132018327713013 6.0
0 -0.00519376993179321 7.0
33.88878315 38.4337196350098 0.0
0 0.00824373960494995 1.0
0 -0.000991284847259521 2.0
0 0.0105270147323608 3.0
0 0.0159042477607727 4.0
0 0.0113751888275146 5.0
0 0.00127136707305908 6.0
0 -0.00309556722640991 7.0
42.125521158 38.4778442382812 0.0
0 0.00595742464065552 1.0
0 0.007171630859375 2.0
0 0.0155715346336365 3.0
0 0.00982785224914551 4.0
0 0.00966256856918335 5.0
0 0.0015488862991333 6.0
0 0.00158339738845825 7.0
37.55067892 39.0216331481934 0.0
0 0.000815391540527344 1.0
0 0.0180549621582031 2.0
0 0.0161591768264771 3.0
0 0.00454628467559814 4.0
0 0.00465989112854004 5.0
0 0.00137084722518921 6.0
0 -0.000109493732452393 7.0
37.39948618 38.7834625244141 0.0
0 0.000694096088409424 1.0
0 0.017846941947937 2.0
0 -0.00839388370513916 3.0
0 0.0181208252906799 4.0
0 0.0139331221580505 5.0
0 0.00215917825698853 6.0
0 0.000113487243652344 7.0
34.341868993 38.6611785888672 0.0
0 -0.0150122046470642 1.0
0 -0.0378657579421997 2.0
0 0.0207475423812866 3.0
0 -0.000995099544525146 4.0
0 -0.00474607944488525 5.0
0 0.00366520881652832 6.0
0 0.00226479768753052 7.0
37.425355461 38.7103309631348 0.0
0 0.00776708126068115 1.0
0 0.00792419910430908 2.0
0 -0.00677996873855591 3.0
0 0.0113720297813416 4.0
0 0.00985175371170044 5.0
0 0.000823676586151123 6.0
0 -1.20997428894043e-05 7.0
37.288491912 38.4678955078125 0.0
0 0.00515389442443848 1.0
0 0.00180560350418091 2.0
0 0.0212438106536865 3.0
0 0.0104888081550598 4.0
0 0.00904351472854614 5.0
0 0.000123202800750732 6.0
0 -0.0016171932220459 7.0
39.20347441 38.805908203125 0.0
0 -0.00724637508392334 1.0
0 0.00258266925811768 2.0
0 0.0161644816398621 3.0
0 0.00277906656265259 4.0
0 0.000788569450378418 5.0
0 0.0032578706741333 6.0
0 0.000290453433990479 7.0
36.137884824 38.5297698974609 0.0
0 0.00461000204086304 1.0
0 -0.0104106664657593 2.0
0 -0.002341628074646 3.0
0 0.00298571586608887 4.0
0 0.00196605920791626 5.0
0 0.000843822956085205 6.0
0 -0.00134950876235962 7.0
41.655417288 38.3561172485352 0.0
0 0.00745469331741333 1.0
0 -0.00175750255584717 2.0
0 0.0158591866493225 3.0
0 0.0111903548240662 4.0
0 0.00902730226516724 5.0
0 0.000314235687255859 6.0
0 -0.00557076930999756 7.0
39.894978379 38.5850296020508 0.0
0 0.00680673122406006 1.0
0 0.00699466466903687 2.0
0 -0.0027390718460083 3.0
0 0.0178101062774658 4.0
0 0.00734114646911621 5.0
0 0.0022856593132019 6.0
0 -0.00147330760955811 7.0
39.390677385 38.8203659057617 0.0
0 0.0151079297065735 1.0
0 0.0463476181030273 2.0
0 -0.0157824158668518 3.0
0 -0.0195214152336121 4.0
0 -0.0244238376617432 5.0
0 0.00370627641677856 6.0
0 -0.0034632682800293 7.0
38.248327703 38.6163101196289 0.0
0 0.00582766532897949 1.0
0 -0.00413334369659424 2.0
0 0.00381135940551758 3.0
0 0.0143280029296875 4.0
0 0.0112537741661072 5.0
0 0.0016014575958252 6.0
0 -0.000128448009490967 7.0
38.538636426 38.5361137390137 0.0
0 0.00641006231307983 1.0
0 -0.00027698278427124 2.0
0 0.0049251914024353 3.0
0 0.0107793807983398 4.0
0 0.0084531307220459 5.0
0 0.00101703405380249 6.0
0 -0.000242471694946289 7.0
30.093865158 39.2019309997559 0.0
0 0.0353890657424927 1.0
0 0.00355738401412964 2.0
0 0.04957115650177 3.0
0 -0.0436264872550964 4.0
0 -0.0471203327178955 5.0
0 0.00451117753982544 6.0
0 -0.000642895698547363 7.0
36.059945316 38.318187713623 0.0
0 0.00228762626647949 1.0
0 0.0050504207611084 2.0
0 -0.00272685289382935 3.0
0 0.0147392749786377 4.0
0 0.0124824047088623 5.0
0 0.00258147716522217 6.0
0 0.00151604413986206 7.0
41.382686141 38.4142570495605 0.0
0 -0.00156998634338379 1.0
0 0.0204638242721558 2.0
0 -0.00764346122741699 3.0
0 -0.00123339891433716 4.0
0 -0.0007476806640625 5.0
0 -0.00071108341217041 6.0
0 -0.00559014081954956 7.0
37.374314239 38.6772499084473 0.0
0 -0.00320535898208618 1.0
0 0.00326782464981079 2.0
0 -0.00733256340026855 3.0
0 0.0150027275085449 4.0
0 0.0146105885505676 5.0
0 0.00185835361480713 6.0
0 0.000728189945220947 7.0
37.689487865 38.6625061035156 0.0
0 -0.00688505172729492 1.0
0 0.0156492590904236 2.0
0 0.0139175653457642 3.0
0 0.00787997245788574 4.0
0 0.00172066688537598 5.0
0 0.00320947170257568 6.0
0 0.000371813774108887 7.0
36.541540581 38.4495582580566 0.0
0 0.00823324918746948 1.0
0 -0.00128722190856934 2.0
0 0.0112419724464417 3.0
0 0.0117476582527161 4.0
0 0.0131450295448303 5.0
0 0.000376760959625244 6.0
0 -0.00190281867980957 7.0
36.791141656 38.3827018737793 0.0
0 0.00603717565536499 1.0
0 0.00242966413497925 2.0
0 0.00899404287338257 3.0
0 0.0119273662567139 4.0
0 0.011038064956665 5.0
0 0.00139141082763672 6.0
0 0.000935733318328857 7.0
39.891605825 38.3263244628906 0.0
0 -0.000929117202758789 1.0
0 0.0420316457748413 2.0
0 -0.0310762524604797 3.0
0 -0.00990122556686401 4.0
0 -0.00977462530136108 5.0
0 -0.0024992823600769 6.0
0 -0.00472760200500488 7.0
34.771445328 38.6938209533691 0.0
0 0.00257289409637451 1.0
0 0.0160443186759949 2.0
0 -0.00607508420944214 3.0
0 0.015845775604248 4.0
0 0.0155516862869263 5.0
0 0.00151389837265015 6.0
0 -0.000127732753753662 7.0
44.638557165 38.5900001525879 0.0
0 -0.00749975442886353 1.0
0 0.0173272490501404 2.0
0 0.0101034641265869 3.0
0 0.0125020146369934 4.0
0 0.00756698846817017 5.0
0 0.00205510854721069 6.0
0 -0.00224572420120239 7.0
36.096180936 38.1678276062012 0.0
0 0.00536233186721802 1.0
0 0.00875014066696167 2.0
0 0.0115242004394531 3.0
0 0.0147314667701721 4.0
0 0.00792396068572998 5.0
0 0.00153219699859619 6.0
0 -0.00242865085601807 7.0
45.744029538 38.1977348327637 0.0
0 0.00154680013656616 1.0
0 -0.00525921583175659 2.0
0 0.0133904218673706 3.0
0 0.00973719358444214 4.0
0 0.0101900100708008 5.0
0 0.00129318237304688 6.0
0 0.00124156475067139 7.0
36.287608705 38.4067802429199 0.0
0 -0.00326824188232422 1.0
0 0.00791627168655396 2.0
0 0.00300085544586182 3.0
0 0.0167179703712463 4.0
0 0.012957751750946 5.0
0 0.00116145610809326 6.0
0 -0.00291407108306885 7.0
42.801830865 38.5026664733887 0.0
0 0.00598233938217163 1.0
0 0.00537014007568359 2.0
0 0.00846278667449951 3.0
0 0.00354379415512085 4.0
0 0.00350403785705566 5.0
0 0.00145411491394043 6.0
0 -0.0010840892791748 7.0
47.558612578 38.3811569213867 0.0
0 -0.00544494390487671 1.0
0 0.0251234173774719 2.0
0 0.0158400535583496 3.0
0 0.00700610876083374 4.0
0 0.00496816635131836 5.0
0 0.000510692596435547 6.0
0 -0.00318229198455811 7.0
47.058562462 38.4057235717773 0.0
0 0.00507622957229614 1.0
0 0.0046621561050415 2.0
0 -0.00457650423049927 3.0
0 0.0249253511428833 4.0
0 0.0235775709152222 5.0
0 0.00243175029754639 6.0
0 0.000473499298095703 7.0
43.087048946 38.619270324707 0.0
0 -0.0106337070465088 1.0
0 -0.0265575647354126 2.0
0 0.015188455581665 3.0
0 -0.00168472528457642 4.0
0 -0.00724762678146362 5.0
0 0.00427258014678955 6.0
0 0.0025297999382019 7.0
33.170190568 38.9567565917969 0.0
0 0.00517016649246216 1.0
0 0.00459200143814087 2.0
0 0.0193129181861877 3.0
0 0.00718289613723755 4.0
0 0.00489509105682373 5.0
0 0.00101983547210693 6.0
0 -0.00283050537109375 7.0
34.957431523 38.8416709899902 0.0
0 0.000963389873504639 1.0
0 0.00420975685119629 2.0
0 0.0198730826377869 3.0
0 -0.00562077760696411 4.0
0 -0.00709086656570435 5.0
0 0.00256907939910889 6.0
0 -0.00235521793365479 7.0
36.320328519 38.8440856933594 0.0
0 -0.00225573778152466 1.0
0 -0.00953090190887451 2.0
0 0.00566196441650391 3.0
0 0.00649935007095337 4.0
0 0.00460940599441528 5.0
0 0.0024418830871582 6.0
0 0.000105619430541992 7.0
44.085721677 38.4182586669922 0.0
0 0.000100314617156982 1.0
0 0.0046502947807312 2.0
0 0.000212252140045166 3.0
0 0.00853204727172852 4.0
0 0.00654923915863037 5.0
0 0.000967621803283691 6.0
0 -0.00571078062057495 7.0
45.211814568 38.3988380432129 0.0
0 0.000818192958831787 1.0
0 0.0170605778694153 2.0
0 -0.0164785385131836 3.0
0 0.0199121236801147 4.0
0 0.0166866183280945 5.0
0 0.00196439027786255 6.0
0 -0.000163257122039795 7.0
38.71721827 38.8017578125 0.0
0 -0.00146806240081787 1.0
0 0.00902062654495239 2.0
0 -0.0155308246612549 3.0
0 0.00887376070022583 4.0
0 0.0110338926315308 5.0
0 0.00151538848876953 6.0
0 0.00038301944732666 7.0
42.190454953 38.4542961120605 0.0
0 -0.00158572196960449 1.0
0 -0.00745916366577148 2.0
0 -0.0216897130012512 3.0
0 -0.00159525871276855 4.0
0 -0.00380432605743408 5.0
0 9.28044319152832e-05 6.0
0 -0.00571072101593018 7.0
40.616038376 38.7430953979492 0.0
0 -0.00362867116928101 1.0
0 -0.00363141298294067 2.0
0 -0.0144767761230469 3.0
0 0.0116175413131714 4.0
0 0.0105224847793579 5.0
0 0.00140535831451416 6.0
0 0.000889956951141357 7.0
31.887764532 38.5140266418457 0.0
0 -0.00434798002243042 1.0
0 0.00756442546844482 2.0
0 -0.0390388369560242 3.0
0 -0.0211315751075745 4.0
0 -0.0261675715446472 5.0
0 0.00929063558578491 6.0
0 -0.00467920303344727 7.0
39.764867486 39.1465721130371 0.0
0 0.038346529006958 1.0
0 0.0927492082118988 2.0
0 0.0490598678588867 3.0
0 -0.041797399520874 4.0
0 -0.0453233122825623 5.0
0 0.00138425827026367 6.0
0 -0.000771045684814453 7.0
40.805009287 38.1030654907227 0.0
0 -0.0296259522438049 1.0
0 -0.00586938858032227 2.0
0 0.00417482852935791 3.0
0 -0.00110441446304321 4.0
0 -0.00454360246658325 5.0
0 0.00295650959014893 6.0
0 -0.000849902629852295 7.0
39.795600818 38.255054473877 0.0
0 0.00796306133270264 1.0
0 -0.00954651832580566 2.0
0 0.0155543088912964 3.0
0 0.0102896094322205 4.0
0 0.00733578205108643 5.0
0 0.00113248825073242 6.0
0 -0.00258749723434448 7.0
38.685439721 38.1099815368652 0.0
0 -0.0193047523498535 1.0
0 0.026334285736084 2.0
0 0.014123797416687 3.0
0 0.00438123941421509 4.0
0 0.00159591436386108 5.0
0 0.0030590295791626 6.0
0 0.00219011306762695 7.0
42.152656556 38.6447944641113 0.0
0 0.00958836078643799 1.0
0 0.00431197881698608 2.0
0 0.00279456377029419 3.0
0 0.0114231109619141 4.0
0 0.014231264591217 5.0
0 0.00242865085601807 6.0
0 0.00234651565551758 7.0
37.458343334 39.1002197265625 0.0
0 0.118244767189026 1.0
0 0.0132309198379517 2.0
0 0.179237484931946 3.0
0 -0.0956413745880127 4.0
0 -0.109352290630341 5.0
0 0.0108833312988281 6.0
0 0.0037427544593811 7.0
44.489633946 37.3663024902344 0.0
0 -0.000156283378601074 1.0
0 0.0174140930175781 2.0
0 -0.0660887956619263 3.0
0 -0.0274395942687988 4.0
0 -0.0323089957237244 5.0
0 0.0018419623374939 6.0
0 -0.00307917594909668 7.0
41.506640114 38.712272644043 0.0
0 0.00974112749099731 1.0
0 -0.0041663646697998 2.0
0 0.00747394561767578 3.0
0 0.0130130648612976 4.0
0 0.0114015340805054 5.0
0 0.00225138664245605 6.0
0 0.0011325478553772 7.0
33.262736569 38.4637565612793 0.0
0 -0.00413501262664795 1.0
0 -0.0077093243598938 2.0
0 0.00772899389266968 3.0
0 -0.0170835852622986 4.0
0 -0.0186598896980286 5.0
0 9.43541526794434e-05 6.0
0 -0.00329750776290894 7.0
47.069969764 39.1127662658691 0.0
0 0.0034937858581543 1.0
0 0.0224937796592712 2.0
0 0.0202521085739136 3.0
0 0.00589430332183838 4.0
0 0.00132709741592407 5.0
0 0.00247538089752197 6.0
0 -0.00191307067871094 7.0
43.288539967 38.7025947570801 0.0
0 0.0100835561752319 1.0
0 0.00247675180435181 2.0
0 0.0297904014587402 3.0
0 -0.00783008337020874 4.0
0 -0.00629055500030518 5.0
0 0.00282490253448486 6.0
0 0.00166058540344238 7.0
41.479216264 38.3655395507812 0.0
0 -0.0076107382774353 1.0
0 0.0249927639961243 2.0
0 0.0158962607383728 3.0
0 0.00533664226531982 4.0
0 0.00829720497131348 5.0
0 0.00181961059570312 6.0
0 0.00211477279663086 7.0
36.99162076 38.5888214111328 0.0
0 0.00175654888153076 1.0
0 0.023786187171936 2.0
0 0.0185081958770752 3.0
0 0.010552704334259 4.0
0 0.00637632608413696 5.0
0 0.00184732675552368 6.0
0 -0.0059053897857666 7.0
40.435886771 38.986686706543 0.0
0 -0.00245356559753418 1.0
0 0.0295779705047607 2.0
0 -0.0167503356933594 3.0
0 0.0110731720924377 4.0
0 0.00603735446929932 5.0
0 0.00212985277175903 6.0
0 -0.000238656997680664 7.0
37.140737383 38.4575042724609 0.0
0 0.00847417116165161 1.0
0 0.00707650184631348 2.0
0 0.00589668750762939 3.0
0 0.00967586040496826 4.0
0 0.0113044381141663 5.0
0 0.000883042812347412 6.0
0 0.00042259693145752 7.0
39.233907158 38.9101257324219 0.0
0 0.0550709962844849 1.0
0 0.148562252521515 2.0
0 0.0203211903572083 3.0
0 -0.0188177227973938 4.0
0 -0.0227159857749939 5.0
0 0.00519031286239624 6.0
0 0.000806093215942383 7.0
42.028602534 38.4750175476074 0.0
0 -0.00834649801254272 1.0
0 0.161355286836624 2.0
0 -0.0410969257354736 3.0
0 -0.0144053101539612 4.0
0 -0.0182055234909058 5.0
0 0.000284016132354736 6.0
0 -0.00558304786682129 7.0
39.783103652 39.0378684997559 0.0
0 0.0226157903671265 1.0
0 -0.0173406600952148 2.0
0 0.0640493631362915 3.0
0 -0.022208034992218 4.0
0 -0.0254800319671631 5.0
0 0.00419634580612183 6.0
0 -0.00270748138427734 7.0
39.425516455 38.1142044067383 0.0
0 0.003761887550354 1.0
0 0.00284242630004883 2.0
0 0.00697588920593262 3.0
0 0.00978696346282959 4.0
0 0.00723946094512939 5.0
0 0.00138962268829346 6.0
0 -0.00471216440200806 7.0
40.002559542 39.1183433532715 0.0
0 0.0111833214759827 1.0
0 0.0301645398139954 2.0
0 0.0249743461608887 3.0
0 0.00826179981231689 4.0
0 0.00183463096618652 5.0
0 0.00230365991592407 6.0
0 -0.00201362371444702 7.0
38.322665212 38.248119354248 0.0
0 0.00894296169281006 1.0
0 0.00682508945465088 2.0
0 -0.0240972638130188 3.0
0 -0.00541716814041138 4.0
0 -0.0107141733169556 5.0
0 0.0022168755531311 6.0
0 -0.0030866265296936 7.0
40.568600192 39.0411567687988 0.0
0 0.0145177841186523 1.0
0 0.0150798559188843 2.0
0 0.00254154205322266 3.0
0 -0.01081383228302 4.0
0 -0.014443039894104 5.0
0 0.00246632099151611 6.0
0 -0.00231659412384033 7.0
40.337689123 38.5351600646973 0.0
0 0.00957757234573364 1.0
0 0.0115842819213867 2.0
0 0.0130894184112549 3.0
0 0.00751912593841553 4.0
0 0.00370407104492188 5.0
0 0.00157725811004639 6.0
0 -0.00272804498672485 7.0
45.726007711 39.016544342041 0.0
0 -0.00627726316452026 1.0
0 0.0215297341346741 2.0
0 -0.0119922757148743 3.0
0 0.010688841342926 4.0
0 0.00892925262451172 5.0
0 0.0013124942779541 6.0
0 0.000753998756408691 7.0
36.675787374 38.2430000305176 0.0
0 -0.00158858299255371 1.0
0 0.0261760354042053 2.0
0 -0.00662976503372192 3.0
0 -0.0059894323348999 4.0
0 -0.00984197854995728 5.0
0 0.00290602445602417 6.0
0 -0.00482195615768433 7.0
32.235093382 38.4787483215332 0.0
0 0.00654047727584839 1.0
0 0.00283539295196533 2.0
0 0.0180583000183105 3.0
0 0.0143955945968628 4.0
0 0.00843143463134766 5.0
0 0.00196719169616699 6.0
0 -0.000648856163024902 7.0
39.438537732 38.3393020629883 0.0
0 0.00482273101806641 1.0
0 -0.0100158452987671 2.0
0 0.00550246238708496 3.0
0 0.00398647785186768 4.0
0 0.00508660078048706 5.0
0 0.000476419925689697 6.0
0 -0.000379323959350586 7.0
44.304614406 38.6942520141602 0.0
0 0.00724595785140991 1.0
0 0.00359833240509033 2.0
0 -0.00244873762130737 3.0
0 0.0168184041976929 4.0
0 0.018384575843811 5.0
0 0.00267505645751953 6.0
0 0.00242710113525391 7.0
32.373304325 38.7011947631836 0.0
0 -0.00139033794403076 1.0
0 0.0330175161361694 2.0
0 0.0167226791381836 3.0
0 -0.00249332189559937 4.0
0 -0.00695604085922241 5.0
0 0.00298500061035156 6.0
0 -0.00275629758834839 7.0
35.769379856 38.7898292541504 0.0
0 0.00197267532348633 1.0
0 0.029208242893219 2.0
0 0.0212281942367554 3.0
0 0.00471234321594238 4.0
0 0.00653988122940063 5.0
0 0.000956416130065918 6.0
0 -0.000651717185974121 7.0
38.348316661 38.5863151550293 0.0
0 -0.00654613971710205 1.0
0 0.0357071757316589 2.0
0 0.00473201274871826 3.0
0 0.00128424167633057 4.0
0 -0.00100201368331909 5.0
0 0.00222885608673096 6.0
0 -0.00584298372268677 7.0
30.534523433 39.0130577087402 0.0
0 0.00583195686340332 1.0
0 0.0315658450126648 2.0
0 0.0169404745101929 3.0
0 0.00761932134628296 4.0
0 0.0039747953414917 5.0
0 0.00238054990768433 6.0
0 -0.00270819664001465 7.0
41.665325899 38.7758674621582 0.0
0 0.000107109546661377 1.0
0 0.0110929012298584 2.0
0 0.0183548331260681 3.0
0 0.00898802280426025 4.0
0 0.00583487749099731 5.0
0 0.00194180011749268 6.0
0 -0.00388616323471069 7.0
39.125756769 38.7648773193359 0.0
0 -0.0030866265296936 1.0
0 0.00142174959182739 2.0
0 0.0009346604347229 3.0
0 0.00942152738571167 4.0
0 0.00966179370880127 5.0
0 0.00102788209915161 6.0
0 -0.00109249353408813 7.0
46.922469454 38.955696105957 0.0
0 -0.00348621606826782 1.0
0 0.000359296798706055 2.0
0 0.0120048522949219 3.0
0 0.00776892900466919 4.0
0 0.00494801998138428 5.0
0 0.00185239315032959 6.0
0 0.000420093536376953 7.0
38.082156965 38.3118515014648 0.0
0 0.0151529312133789 1.0
0 -0.0102242231369019 2.0
0 0.0122508406639099 3.0
0 -0.0131518244743347 4.0
0 -0.0139512419700623 5.0
0 0.00426769256591797 6.0
0 0.00295782089233398 7.0
35.281773346 38.6570739746094 0.0
0 -0.00272595882415771 1.0
0 0.0289946794509888 2.0
0 0.00509798526763916 3.0
0 -0.00448018312454224 4.0
0 -0.00222653150558472 5.0
0 0.000601649284362793 6.0
0 -0.00430136919021606 7.0
39.676820236 38.989990234375 0.0
0 0.00472760200500488 1.0
0 0.0294286012649536 2.0
0 0.0172818303108215 3.0
0 0.00415349006652832 4.0
0 0.00443726778030396 5.0
0 0.00188541412353516 6.0
0 -0.000945448875427246 7.0
41.049027324 38.6982574462891 0.0
0 0.00343352556228638 1.0
0 0.0113016366958618 2.0
0 -0.00696408748626709 3.0
0 0.0125917196273804 4.0
0 0.00998294353485107 5.0
0 0.00137799978256226 6.0
0 0.000242471694946289 7.0
41.402616693 39.0029830932617 0.0
0 0.00863611698150635 1.0
0 0.087821900844574 2.0
0 0.0603731870651245 3.0
0 -0.0356480479240417 4.0
0 -0.0399312376976013 5.0
0 0.00590997934341431 6.0
0 -9.55462455749512e-05 7.0
30.3721478 38.6933746337891 0.0
0 -0.00103241205215454 1.0
0 0.00393533706665039 2.0
0 -0.0112425088882446 3.0
0 -0.00764316320419312 4.0
0 -0.0103350281715393 5.0
0 0.000120401382446289 6.0
0 -0.00539594888687134 7.0
41.111385117 38.5219116210938 0.0
0 0.00852882862091064 1.0
0 -0.00877469778060913 2.0
0 0.0205832123756409 3.0
0 0.0103094577789307 4.0
0 0.0121785402297974 5.0
0 0.000267446041107178 6.0
0 -0.000113844871520996 7.0
36.921893091 38.6074638366699 0.0
0 0.00677359104156494 1.0
0 -0.00750589370727539 2.0
0 0.00729429721832275 3.0
0 0.0145912766456604 4.0
0 0.0131081342697144 5.0
0 0.000901758670806885 6.0
0 -0.000841677188873291 7.0
41.531234868 38.1064682006836 0.0
0 0.0153313875198364 1.0
0 0.0151869654655457 2.0
0 -0.00373983383178711 3.0
0 -0.0103966593742371 4.0
0 -0.0106770396232605 5.0
0 0.00317627191543579 6.0
0 0.00247424840927124 7.0
36.809020967 38.4955902099609 0.0
0 0.00135868787765503 1.0
0 0.00796598196029663 2.0
0 -0.00358861684799194 3.0
0 -0.00392121076583862 4.0
0 -0.00673782825469971 5.0
0 0.00253629684448242 6.0
0 -0.00510793924331665 7.0
30.318323192 38.832633972168 0.0
0 -0.00723063945770264 1.0
0 0.0219334959983826 2.0
0 0.010265588760376 3.0
0 0.00689631700515747 4.0
0 0.00381928682327271 5.0
0 0.00178146362304688 6.0
0 -0.00287723541259766 7.0
37.371827278 39.0299491882324 0.0
0 0.0118033885955811 1.0
0 0.0467261672019958 2.0
0 0.0143779516220093 3.0
0 -0.0137546062469482 4.0
0 -0.0207595229148865 5.0
0 0.00627964735031128 6.0
0 0.00312989950180054 7.0
37.183742215 38.1556701660156 0.0
0 0.00127106904983521 1.0
0 0.0066341757774353 2.0
0 -0.0147396326065063 3.0
0 0.00577563047409058 4.0
0 0.0029301643371582 5.0
0 -0.00182098150253296 6.0
0 -0.00620943307876587 7.0
34.629310928 38.365364074707 0.0
0 0.00610154867172241 1.0
0 0.00508695840835571 2.0
0 0.0198239088058472 3.0
0 0.013192892074585 4.0
0 0.00678855180740356 5.0
0 0.00147479772567749 6.0
0 -0.00184899568557739 7.0
42.830967498 38.9176712036133 0.0
0 -0.000261783599853516 1.0
0 0.0146018862724304 2.0
0 -0.011113166809082 3.0
0 0.0185452699661255 4.0
0 0.00816124677658081 5.0
0 0.00284934043884277 6.0
0 -8.62479209899902e-05 7.0
34.694327372 38.3558731079102 0.0
0 0.00509011745452881 1.0
0 0.000261187553405762 2.0
0 0.0105280876159668 3.0
0 0.0111966133117676 4.0
0 0.0079917311668396 5.0
0 0.00138604640960693 6.0
0 -0.00158607959747314 7.0
37.611435874 39.1154098510742 0.0
0 -0.00671660900115967 1.0
0 0.0181795358657837 2.0
0 -0.000290572643280029 3.0
0 0.0106480121612549 4.0
0 0.00586879253387451 5.0
0 0.00247448682785034 6.0
0 0.000689864158630371 7.0
45.529166488 38.6235542297363 0.0
0 -0.00425219535827637 1.0
0 0.0054473876953125 2.0
0 0.0149739980697632 3.0
0 0.00727999210357666 4.0
0 0.00684750080108643 5.0
0 0.00113421678543091 6.0
0 -0.00142967700958252 7.0
40.26689599 38.7462539672852 0.0
0 -0.00593769550323486 1.0
0 -0.00408351421356201 2.0
0 0.00365918874740601 3.0
0 0.00439119338989258 4.0
0 0.00260257720947266 5.0
0 0.00294429063796997 6.0
0 0.00205498933792114 7.0
39.577958661 38.4459991455078 0.0
0 0.00205034017562866 1.0
0 0.00105744600296021 2.0
0 -0.00728344917297363 3.0
0 0.017594575881958 4.0
0 0.0188379287719727 5.0
0 0.00254732370376587 6.0
0 0.00236052274703979 7.0
34.564585703 38.8292350769043 0.0
0 -0.0269166231155396 1.0
0 0.136175036430359 2.0
0 0.053674578666687 3.0
0 -0.047590434551239 4.0
0 -0.0486240386962891 5.0
0 0.00632476806640625 6.0
0 0.00384056568145752 7.0
34.747667205 38.7812385559082 0.0
0 0.0150026679039001 1.0
0 0.00320255756378174 2.0
0 -0.0252484083175659 3.0
0 -0.0210506319999695 4.0
0 -0.0219604969024658 5.0
0 0.00306332111358643 6.0
0 -0.00150471925735474 7.0
44.886437438 38.5786972045898 0.0
0 -0.00230449438095093 1.0
0 0.00125682353973389 2.0
0 0.00380456447601318 3.0
0 0.0123153924942017 4.0
0 0.0131455063819885 5.0
0 0.000817179679870605 6.0
0 -6.07967376708984e-05 7.0
34.458606901 38.426197052002 0.0
0 0.0095527172088623 1.0
0 -0.0173845887184143 2.0
0 0.0205413103103638 3.0
0 0.013698399066925 4.0
0 0.00839120149612427 5.0
0 0.00135630369186401 6.0
0 -0.00317841768264771 7.0
44.985938301 38.635196685791 0.0
0 0.0398619174957275 1.0
0 0.0797373652458191 2.0
0 -0.0432404279708862 3.0
0 -0.0900113582611084 4.0
0 -0.0522611737251282 5.0
0 0.00559622049331665 6.0
0 0.00259339809417725 7.0
36.561528733 38.4419441223145 0.0
0 -0.00396859645843506 1.0
0 -0.0254343748092651 2.0
0 -0.0119385123252869 3.0
0 -0.0315827131271362 4.0
0 -0.0329383611679077 5.0
0 0.00273001194000244 6.0
0 -0.00205951929092407 7.0
35.585636345 38.4601249694824 0.0
0 -0.00341349840164185 1.0
0 0.0154292583465576 2.0
0 0.0149226188659668 3.0
0 0.00507611036300659 4.0
0 0.00384801626205444 5.0
0 0.000981569290161133 6.0
0 -0.0022132396697998 7.0
34.748633645 38.3581428527832 0.0
0 0.0046808123588562 1.0
0 0.00470662117004395 2.0
0 -0.0016711950302124 3.0
0 0.0164695382118225 4.0
0 0.0109076499938965 5.0
0 0.0013577938079834 6.0
0 -0.00368696451187134 7.0
33.997517461 38.4255676269531 0.0
0 -0.00385230779647827 1.0
0 0.0130151510238647 2.0
0 -0.00214368104934692 3.0
0 0.0129605531692505 4.0
0 0.0127876400947571 5.0
0 0.00150924921035767 6.0
0 0.000913619995117188 7.0
31.337837162 38.988941192627 0.0
0 0.00888556241989136 1.0
0 0.0149818062782288 2.0
0 0.0195881128311157 3.0
0 0.00459617376327515 4.0
0 0.00473707914352417 5.0
0 0.00146538019180298 6.0
0 0.000113725662231445 7.0
43.914614211 38.5083885192871 0.0
0 0.0040326714515686 1.0
0 -0.014149010181427 2.0
0 0.0157884955406189 3.0
0 0.0115993022918701 4.0
0 0.00837445259094238 5.0
0 0.00118738412857056 6.0
0 -0.000306010246276855 7.0
45.186574618 38.4571304321289 0.0
0 0.00369459390640259 1.0
0 -0.00190377235412598 2.0
0 0.00506937503814697 3.0
0 0.00619816780090332 4.0
0 0.00701636075973511 5.0
0 0.000674545764923096 6.0
0 -0.00296097993850708 7.0
35.268441876 38.6745338439941 0.0
0 0.0259495973587036 1.0
0 0.00516915321350098 2.0
0 0.0271381139755249 3.0
0 -0.0043373703956604 4.0
0 -0.00707775354385376 5.0
0 0.00333774089813232 6.0
0 -0.000590026378631592 7.0
32.333098982 38.339469909668 0.0
0 0.00736904144287109 1.0
0 -0.012636661529541 2.0
0 -0.00687289237976074 3.0
0 0.0058826208114624 4.0
0 0.000943958759307861 5.0
0 0.000460386276245117 6.0
0 -0.00598776340484619 7.0
38.894207338 38.5355949401855 0.0
0 -0.0060042142868042 1.0
0 0.0341095924377441 2.0
0 0.00643140077590942 3.0
0 -0.00107800960540771 4.0
0 -0.00261771678924561 5.0
0 -0.000578522682189941 6.0
0 -0.00298482179641724 7.0
32.941924742 38.3487243652344 0.0
0 0.01023268699646 1.0
0 0.0263230800628662 2.0
0 -0.056096076965332 3.0
0 -0.0248920917510986 4.0
0 -0.0269808173179626 5.0
0 0.00285071134567261 6.0
0 0.00860178470611572 7.0
40.2164929 37.7063751220703 0.0
0 -0.00206410884857178 1.0
0 -0.0241124033927917 2.0
0 -0.0163887739181519 3.0
0 -0.00840198993682861 4.0
0 -0.0127372145652771 5.0
0 0.00169253349304199 6.0
0 -0.0047113299369812 7.0
40.810463567 39.0738258361816 0.0
0 0.00107890367507935 1.0
0 0.000585675239562988 2.0
0 -0.0132269859313965 3.0
0 0.0105957388877869 4.0
0 0.0103799700737 5.0
0 0.00216841697692871 6.0
0 0.00124680995941162 7.0
33.558410457 38.8956298828125 0.0
0 0.00843101739883423 1.0
0 -0.0107095837593079 2.0
0 -0.0284059047698975 3.0
0 0.000358760356903076 4.0
0 0.00435709953308105 5.0
0 0.00229853391647339 6.0
0 0.00229942798614502 7.0
47.788896524 38.6363067626953 0.0
0 -0.00472438335418701 1.0
0 -0.0201917886734009 2.0
0 0.0364751219749451 3.0
0 -0.0135514140129089 4.0
0 -0.016156792640686 5.0
0 0.00404399633407593 6.0
0 -0.00392872095108032 7.0
33.642782256 38.8176574707031 0.0
0 -0.00302708148956299 1.0
0 0.00466156005859375 2.0
0 0.0140516757965088 3.0
0 0.00737649202346802 4.0
0 0.00594466924667358 5.0
0 0.0017513632774353 6.0
0 -0.00115036964416504 7.0
37.659684029 38.3499221801758 0.0
0 0.0080101490020752 1.0
0 0.000805974006652832 2.0
0 0.0185391902923584 3.0
0 0.0152295827865601 4.0
0 0.00819528102874756 5.0
0 0.00101685523986816 6.0
0 -0.00529128313064575 7.0
40.91969971 38.140266418457 0.0
0 0.00345683097839355 1.0
0 -0.00896137952804565 2.0
0 -0.0119954347610474 3.0
0 0.00301778316497803 4.0
0 0.00167751312255859 5.0
0 0.00145190954208374 6.0
0 -0.00629293918609619 7.0
42.622265927 38.811466217041 0.0
0 0.019258975982666 1.0
0 0.0364774465560913 2.0
0 -0.00904768705368042 3.0
0 -0.0298802852630615 4.0
0 -0.0317170023918152 5.0
0 0.00465452671051025 6.0
0 -0.00226563215255737 7.0
33.946966747 38.4802627563477 0.0
0 -0.00518739223480225 1.0
0 0.032974898815155 2.0
0 0.0197080373764038 3.0
0 0.00946873426437378 4.0
0 0.00464081764221191 5.0
0 0.000941216945648193 6.0
0 -0.00420850515365601 7.0
36.409182194 38.5286026000977 0.0
0 0.00702512264251709 1.0
0 -0.0194560289382935 2.0
0 0.0110805034637451 3.0
0 0.0136269330978394 4.0
0 0.0101454257965088 5.0
0 0.00085604190826416 6.0
0 -0.0050346851348877 7.0
40.593448001 38.5760307312012 0.0
0 0.0027921199798584 1.0
0 0.000395596027374268 2.0
0 -0.00285184383392334 3.0
0 0.0092308521270752 4.0
0 0.010448157787323 5.0
0 0.000471591949462891 6.0
0 0.00133585929870605 7.0
41.715980637 38.8361663818359 0.0
0 0.0497987866401672 1.0
0 0.0365809798240662 2.0
0 0.0144861936569214 3.0
0 -0.0297126173973083 4.0
0 -0.0379680395126343 5.0
0 0.00509655475616455 6.0
0 -0.00242024660110474 7.0
33.77037363 38.5182495117188 0.0
0 0.0106807351112366 1.0
0 0.083455502986908 2.0
0 -0.0791702270507812 3.0
0 -0.052739679813385 4.0
0 -0.0625720620155334 5.0
0 0.00684452056884766 6.0
0 -0.000840663909912109 7.0
46.074402277 38.5328330993652 0.0
0 0.0320846438407898 1.0
0 0.0307254195213318 2.0
0 0.0312415361404419 3.0
0 -0.0108807682991028 4.0
0 -0.0112593173980713 5.0
0 0.00355857610702515 6.0
0 0.00201642513275146 7.0
40.510394826 38.7015953063965 0.0
0 0.00823861360549927 1.0
0 -0.0077546238899231 2.0
0 0.046064555644989 3.0
0 -0.0181393027305603 4.0
0 -0.017975926399231 5.0
0 0.002483069896698 6.0
0 0.00150805711746216 7.0
39.244329381 38.7511940002441 0.0
0 0.0159115791320801 1.0
0 0.0022616982460022 2.0
0 0.0282448530197144 3.0
0 -0.00502586364746094 4.0
0 -0.00408065319061279 5.0
0 0.00202155113220215 6.0
0 0.000927567481994629 7.0
35.7171596 38.6546821594238 0.0
0 0.00409018993377686 1.0
0 -0.00489550828933716 2.0
0 -0.0026392936706543 3.0
0 0.0113359689712524 4.0
0 0.0123302340507507 5.0
0 0.000735223293304443 6.0
0 0.00068289041519165 7.0
39.754714125 38.8880348205566 0.0
0 -0.00539577007293701 1.0
0 -0.00136715173721313 2.0
0 -0.0127389430999756 3.0
0 0.00762850046157837 4.0
0 -0.000317215919494629 5.0
0 0.0031813383102417 6.0
0 -0.00161910057067871 7.0
41.796195011 38.343635559082 0.0
0 0.00867635011672974 1.0
0 -0.008350670337677 2.0
0 0.0206466317176819 3.0
0 0.0107426643371582 4.0
0 0.0104712247848511 5.0
0 0.000457584857940674 6.0
0 -0.000905036926269531 7.0
29.980660557 38.4174346923828 0.0
0 0.00467085838317871 1.0
0 -0.00495713949203491 2.0
0 0.0162566900253296 3.0
0 0.00935602188110352 4.0
0 0.00986051559448242 5.0
0 0.000848650932312012 6.0
0 0.000751495361328125 7.0
36.241260622 38.5575332641602 0.0
0 0.000705897808074951 1.0
0 0.00149893760681152 2.0
0 0.0100005865097046 3.0
0 0.01280277967453 4.0
0 0.0111420154571533 5.0
0 0.000396430492401123 6.0
0 -0.00308972597122192 7.0
38.129703213 38.4214096069336 0.0
0 0.00814342498779297 1.0
0 0.000246703624725342 2.0
0 0.0169895887374878 3.0
0 0.0135315656661987 4.0
0 0.010769248008728 5.0
0 0.00128799676895142 6.0
0 -0.0017850399017334 7.0
36.91522515 38.2449417114258 0.0
0 0.0027318000793457 1.0
0 -0.00151664018630981 2.0
0 0.00988304615020752 3.0
0 0.00847196578979492 4.0
0 0.0108878016471863 5.0
0 0.000810861587524414 6.0
0 0.0011824369430542 7.0
38.102908751 38.1841430664062 0.0
0 0.00115394592285156 1.0
0 0.00325274467468262 2.0
0 0.0126217603683472 3.0
0 0.0149384140968323 4.0
0 0.0158239006996155 5.0
0 0.000880837440490723 6.0
0 -8.7440013885498e-05 7.0
38.34829669 38.6496887207031 0.0
0 -0.00112295150756836 1.0
0 0.0135008692741394 2.0
0 0.00834870338439941 3.0
0 0.0102553367614746 4.0
0 0.00822752714157104 5.0
0 0.00185030698776245 6.0
0 0.00093388557434082 7.0
31.13885609 38.5502471923828 0.0
0 0.000279843807220459 1.0
0 0.00190377235412598 2.0
0 0.00924229621887207 3.0
0 0.0154766440391541 4.0
0 0.0117242932319641 5.0
0 0.00106501579284668 6.0
0 -0.00504231452941895 7.0
35.130346793 39.991096496582 0.0
0 -0.172638535499573 1.0
0 0.176112532615662 2.0
0 0.261671185493469 3.0
0 -0.0960418581962585 4.0
0 -0.047913670539856 5.0
0 0.0104336738586426 6.0
0 0.00360935926437378 7.0
37.970631847 38.0968322753906 0.0
0 -0.00251191854476929 1.0
0 0.0147252678871155 2.0
0 -0.0113922357559204 3.0
0 -0.000210464000701904 4.0
0 -0.00579959154129028 5.0
0 0.00145632028579712 6.0
0 -0.00575977563858032 7.0
37.012414901 38.5231437683105 0.0
0 -0.0027344822883606 1.0
0 0.0352815389633179 2.0
0 0.00825810432434082 3.0
0 -0.00124883651733398 4.0
0 -0.0013393759727478 5.0
0 0.00134366750717163 6.0
0 -0.00226140022277832 7.0
36.779265883 38.894100189209 0.0
0 0.0063093900680542 1.0
0 0.027963399887085 2.0
0 -0.00222969055175781 3.0
0 0.0165801048278809 4.0
0 0.00685769319534302 5.0
0 0.00238025188446045 6.0
0 -0.00534164905548096 7.0
42.665291585 38.5481719970703 0.0
0 0.00762939453125 1.0
0 0.0562233328819275 2.0
0 -0.0685888528823853 3.0
0 -0.0368189811706543 4.0
0 -0.04285728931427 5.0
0 0.00503599643707275 6.0
0 -0.00157469511032104 7.0
35.413046011 38.4058380126953 0.0
0 0.00584030151367188 1.0
0 0.000958383083343506 2.0
0 0.0113482475280762 3.0
0 0.0107216835021973 4.0
0 0.00791442394256592 5.0
0 0.00128114223480225 6.0
0 0.000605762004852295 7.0
35.966339402 38.4775657653809 0.0
0 0.00496381521224976 1.0
0 -0.00544369220733643 2.0
0 0.00988161563873291 3.0
0 0.0164755582809448 4.0
0 0.0110843181610107 5.0
0 0.00106734037399292 6.0
0 -0.00433731079101562 7.0
39.418493945 38.8547630310059 0.0
0 0.00878036022186279 1.0
0 0.000721454620361328 2.0
0 -0.0158603191375732 3.0
0 0.0104245543479919 4.0
0 0.0125052928924561 5.0
0 0.000981569290161133 6.0
0 0.000788688659667969 7.0
32.151549765 38.9702301025391 0.0
0 -0.0014265775680542 1.0
0 0.012227475643158 2.0
0 0.0263000726699829 3.0
0 -0.00517529249191284 4.0
0 -0.00719040632247925 5.0
0 0.00259649753570557 6.0
0 -0.00119811296463013 7.0
31.845645841 38.3061141967773 0.0
0 0.00534671545028687 1.0
0 -0.0176301002502441 2.0
0 -0.0113786458969116 3.0
0 0.00411343574523926 4.0
0 0.00402390956878662 5.0
0 -0.002593994140625 6.0
0 -0.0061572790145874 7.0
40.557128571 38.4783058166504 0.0
0 -0.0107422471046448 1.0
0 0.0388541221618652 2.0
0 0.0169771909713745 3.0
0 0.00476694107055664 4.0
0 0.00180530548095703 5.0
0 0.00234603881835938 6.0
0 0.00116020441055298 7.0
36.793774817 38.6162338256836 0.0
0 0.00310921669006348 1.0
0 0.0043950080871582 2.0
0 0.0118415951728821 3.0
0 -0.0224356055259705 4.0
0 -0.0218193531036377 5.0
0 0.00348061323165894 6.0
0 -0.00138634443283081 7.0
41.333576969 38.9786033630371 0.0
0 0.0102357268333435 1.0
0 -0.00271809101104736 2.0
0 -0.0131592750549316 3.0
0 0.0101926326751709 4.0
0 0.0110886096954346 5.0
0 0.00132131576538086 6.0
0 0.00104010105133057 7.0
43.791455275 38.4271545410156 0.0
0 -0.00638270378112793 1.0
0 0.0200754404067993 2.0
0 0.0120011568069458 3.0
0 0.0064471960067749 4.0
0 0.00578755140304565 5.0
0 0.00162583589553833 6.0
0 0.00128662586212158 7.0
41.481697076 38.2028846740723 0.0
0 0.00590914487838745 1.0
0 0.00448429584503174 2.0
0 -0.0016942024230957 3.0
0 0.00982093811035156 4.0
0 0.0135623216629028 5.0
0 0.000484585762023926 6.0
0 0.000510692596435547 7.0
35.532774621 38.3699798583984 0.0
0 0.00557297468185425 1.0
0 0.00183498859405518 2.0
0 0.00422048568725586 3.0
0 0.0129894614219666 4.0
0 0.0152958631515503 5.0
0 0.00223052501678467 6.0
0 0.00200212001800537 7.0
34.894006808 38.6870956420898 0.0
0 -0.00032496452331543 1.0
0 -0.000553488731384277 2.0
0 -0.0101300477981567 3.0
0 0.0154532194137573 4.0
0 0.0144065618515015 5.0
0 0.00112330913543701 6.0
0 -0.00057142972946167 7.0
38.893810297 38.5068321228027 0.0
0 0.00826197862625122 1.0
0 0.00306463241577148 2.0
0 0.00759679079055786 3.0
0 0.0136953592300415 4.0
0 0.0143409967422485 5.0
0 0.00131523609161377 6.0
0 0.000699937343597412 7.0
39.624663522 38.6048698425293 0.0
0 -0.0116535425186157 1.0
0 0.00954610109329224 2.0
0 -0.00461006164550781 3.0
0 0.017173171043396 4.0
0 0.0123999118804932 5.0
0 0.00213772058486938 6.0
0 3.28421592712402e-05 7.0
42.516231091 39.1451377868652 0.0
0 0.0351818799972534 1.0
0 0.110571384429932 2.0
0 0.0638208389282227 3.0
0 -0.0394923686981201 4.0
0 -0.0376558303833008 5.0
0 0.00706255435943604 6.0
0 0.006949782371521 7.0
38.374768577 38.2217788696289 0.0
0 0.0492962598800659 1.0
0 0.00276666879653931 2.0
0 -0.0941604375839233 3.0
0 -0.0646132230758667 4.0
0 -0.0723511576652527 5.0
0 0.00604933500289917 6.0
0 0.000296711921691895 7.0
33.351459508 39.1657485961914 0.0
0 -0.00859594345092773 1.0
0 0.0212362408638 2.0
0 -0.00135296583175659 3.0
0 0.00774955749511719 4.0
0 0.00817430019378662 5.0
0 0.00212478637695312 6.0
0 0.00165778398513794 7.0
37.08812014 38.7333183288574 0.0
0 -0.000715315341949463 1.0
0 0.00112462043762207 2.0
0 -0.00267255306243896 3.0
0 0.011474072933197 4.0
0 0.011874794960022 5.0
0 0.00210350751876831 6.0
0 0.00163459777832031 7.0
44.445084649 38.5677528381348 0.0
0 0.00638324022293091 1.0
0 -0.00895494222640991 2.0
0 0.00653094053268433 3.0
0 0.0088040828704834 4.0
0 0.0119718909263611 5.0
0 0.000811874866485596 6.0
0 0.00173741579055786 7.0
37.22322104 38.371509552002 0.0
0 0.00562351942062378 1.0
0 0.00389868021011353 2.0
0 0.0112352967262268 3.0
0 0.0115906000137329 4.0
0 0.0116457343101501 5.0
0 0.000585973262786865 6.0
0 -0.00184345245361328 7.0
44.605944609 38.6993217468262 0.0
0 0.0200371146202087 1.0
0 0.0074923038482666 2.0
0 -0.0283963084220886 3.0
0 -0.0215704441070557 4.0
0 -0.0214748382568359 5.0
0 0.00237280130386353 6.0
0 0.00550472736358643 7.0
39.553698498 38.4618606567383 0.0
0 0.00432324409484863 1.0
0 0.0075610876083374 2.0
0 0.00864207744598389 3.0
0 0.015913724899292 4.0
0 0.013094961643219 5.0
0 0.000805139541625977 6.0
0 -0.00503039360046387 7.0
43.287363281 39.1227188110352 0.0
0 0.0155909657478333 1.0
0 0.0208890438079834 2.0
0 0.0387828350067139 3.0
0 -0.00965893268585205 4.0
0 -0.0132709741592407 5.0
0 0.00286310911178589 6.0
0 -0.0010300874710083 7.0
33.992156471 38.1259117126465 0.0
0 -0.000116407871246338 1.0
0 -0.00216031074523926 2.0
0 0.00474166870117188 3.0
0 0.0110854506492615 4.0
0 0.0120851397514343 5.0
0 -0.000650346279144287 6.0
0 -0.00454753637313843 7.0
36.860093248 39.0079536437988 0.0
0 -0.00831305980682373 1.0
0 0.00342237949371338 2.0
0 -0.00300586223602295 3.0
0 0.00467437505722046 4.0
0 0.00290507078170776 5.0
0 0.00205975770950317 6.0
0 0.00154572725296021 7.0
39.672962738 38.0751495361328 0.0
0 0.00984746217727661 1.0
0 -0.0117852091789246 2.0
0 0.0192103981971741 3.0
0 0.00903779268264771 4.0
0 0.00922292470932007 5.0
0 -1.78813934326172e-06 6.0
0 -0.00248432159423828 7.0
32.8535328 38.515998840332 0.0
0 0.00934451818466187 1.0
0 -0.0145254731178284 2.0
0 0.0208010673522949 3.0
0 0.00874799489974976 4.0
0 0.00777202844619751 5.0
0 0.00117224454879761 6.0
0 -0.00193655490875244 7.0
36.422494688 38.2510757446289 0.0
0 0.00413292646408081 1.0
0 0.0195381045341492 2.0
0 -0.0251415371894836 3.0
0 -0.00397628545761108 4.0
0 -0.00545597076416016 5.0
0 0.00688892602920532 6.0
0 -0.00616079568862915 7.0
32.082396771 38.4953002929688 0.0
0 0.0391383171081543 1.0
0 0.0698838829994202 2.0
0 -0.0538221597671509 3.0
0 -0.0767565369606018 4.0
0 -0.080767035484314 5.0
0 0.00638920068740845 6.0
0 0.00216215848922729 7.0
43.742460686 38.6314849853516 0.0
0 0.00743508338928223 1.0
0 0.00358933210372925 2.0
0 0.00233936309814453 3.0
0 0.0148839354515076 4.0
0 0.0103566646575928 5.0
0 0.00171154737472534 6.0
0 -0.00294375419616699 7.0
37.830554934 38.8711471557617 0.0
0 -0.00633442401885986 1.0
0 -0.085806131362915 2.0
0 0.0016249418258667 3.0
0 -0.00366508960723877 4.0
0 -0.00491905212402344 5.0
0 0.00461995601654053 6.0
0 0.00372880697250366 7.0
45.388745763 38.5289192199707 0.0
0 0.00618654489517212 1.0
0 0.00241726636886597 2.0
0 0.00818026065826416 3.0
0 0.0156952738761902 4.0
0 0.0136352181434631 5.0
0 0.00150501728057861 6.0
0 -0.00034487247467041 7.0
40.010376396 38.5327529907227 0.0
0 0.00943154096603394 1.0
0 -0.0237060785293579 2.0
0 0.0177397727966309 3.0
0 0.00880545377731323 4.0
0 0.00820803642272949 5.0
0 0.000179648399353027 6.0
0 -0.00252014398574829 7.0
45.566990187 38.9576721191406 0.0
0 -0.00651991367340088 1.0
0 0.0196529626846313 2.0
0 -0.0233531594276428 3.0
0 0.00508701801300049 4.0
0 0.00925230979919434 5.0
0 0.000830531120300293 6.0
0 0.00188058614730835 7.0
32.300808889 39.3677978515625 0.0
0 -0.139041066169739 1.0
0 0.0485532879829407 2.0
0 0.379734396934509 3.0
0 -0.113313436508179 4.0
0 0.248469144105911 5.0
0 0.0100367069244385 6.0
0 0.00433242321014404 7.0
36.805463378 38.648265838623 0.0
0 -0.0118266344070435 1.0
0 -0.0432865023612976 2.0
0 0.023951530456543 3.0
0 -0.00356596708297729 4.0
0 -0.00542491674423218 5.0
0 0.00420564413070679 6.0
0 0.00177103281021118 7.0
45.039727992 38.5601387023926 0.0
0 0.00589120388031006 1.0
0 -0.0125548243522644 2.0
0 0.00396126508712769 3.0
0 0.0183145403862 4.0
0 0.0117776989936829 5.0
0 0.00193524360656738 6.0
0 -0.00209051370620728 7.0
36.669966877 38.5309829711914 0.0
0 0.00463026762008667 1.0
0 0.0226883888244629 2.0
0 0.00629353523254395 3.0
0 0.0114103555679321 4.0
0 0.00907456874847412 5.0
0 0.00171947479248047 6.0
0 -0.00196748971939087 7.0
40.401972767 38.2327117919922 0.0
0 0.0044560432434082 1.0
0 0.00335395336151123 2.0
0 0.00200176239013672 3.0
0 0.00582796335220337 4.0
0 0.00227099657058716 5.0
0 0.00142323970794678 6.0
0 -0.00324386358261108 7.0
41.491221092 38.6873626708984 0.0
0 0.00546443462371826 1.0
0 -0.00274443626403809 2.0
0 -0.00156420469284058 3.0
0 0.0120398998260498 4.0
0 0.0104166865348816 5.0
0 0.00150889158248901 6.0
0 0.000754177570343018 7.0
32.092723767 38.4019317626953 0.0
0 0.00888198614120483 1.0
0 0.0200417041778564 2.0
0 -0.00451493263244629 3.0
0 0.00675791501998901 4.0
0 0.000231027603149414 5.0
0 0.00174546241760254 6.0
0 -0.00465226173400879 7.0
39.133472273 38.943042755127 0.0
0 0.0176572203636169 1.0
0 -0.010567843914032 2.0
0 0.0199633836746216 3.0
0 -0.0152668356895447 4.0
0 -0.0183753371238708 5.0
0 0.00282925367355347 6.0
0 -0.000978946685791016 7.0
37.053979998 38.9729843139648 0.0
0 -0.000790834426879883 1.0
0 0.026669442653656 2.0
0 0.0143687725067139 3.0
0 0.0024116039276123 4.0
0 -0.00121736526489258 5.0
0 0.00256633758544922 6.0
0 -0.00202929973602295 7.0
39.864494306 38.6545906066895 0.0
0 0.0108628273010254 1.0
0 0.0133299231529236 2.0
0 0.00485479831695557 3.0
0 0.00242054462432861 4.0
0 0.00336581468582153 5.0
0 0.00100594758987427 6.0
0 -0.00150913000106812 7.0
39.054921308 38.6194686889648 0.0
0 -0.00948309898376465 1.0
0 0.0172194242477417 2.0
0 0.00661468505859375 3.0
0 0.00922781229019165 4.0
0 0.010545015335083 5.0
0 0.00140833854675293 6.0
0 0.00142842531204224 7.0
33.466015237 38.9646224975586 0.0
0 9.03606414794922e-05 1.0
0 0.0123940706253052 2.0
0 0.00274002552032471 3.0
0 0.00122815370559692 4.0
0 -0.000679194927215576 5.0
0 0.0030023455619812 6.0
0 0.00157451629638672 7.0
37.653527764 38.5430641174316 0.0
0 -0.0041007399559021 1.0
0 0.0149326324462891 2.0
0 0.00574177503585815 3.0
0 0.00125390291213989 4.0
0 0.00139206647872925 5.0
0 0.000427007675170898 6.0
0 -0.00363874435424805 7.0
37.577039283 38.1301307678223 0.0
0 -0.00789469480514526 1.0
0 0.0148867964744568 2.0
0 0.00783491134643555 3.0
0 0.0112177729606628 4.0
0 0.00835388898849487 5.0
0 0.0013580322265625 6.0
0 -0.00169438123703003 7.0
36.318307627 38.468090057373 0.0
0 0.00593608617782593 1.0
0 0.00778979063034058 2.0
0 0.00651895999908447 3.0
0 0.00518614053726196 4.0
0 0.00192540884017944 5.0
0 0.00126242637634277 6.0
0 -0.00347381830215454 7.0
36.801464194 38.3615798950195 0.0
0 0.00963151454925537 1.0
0 -0.00350320339202881 2.0
0 0.00691354274749756 3.0
0 0.0129764676094055 4.0
0 0.01592618227005 5.0
0 0.00100570917129517 6.0
0 0.000829577445983887 7.0
39.56095567 38.5987434387207 0.0
0 -0.000210285186767578 1.0
0 -0.0386602878570557 2.0
0 0.0109865069389343 3.0
0 -0.0223806500434875 4.0
0 -0.0233038067817688 5.0
0 0.0035366415977478 6.0
0 -0.00128954648971558 7.0
40.443539209 38.8790626525879 0.0
0 0.0022013783454895 1.0
0 0.0163164138793945 2.0
0 0.0207526087760925 3.0
0 0.0038456916809082 4.0
0 0.00528603792190552 5.0
0 0.00127983093261719 6.0
0 -0.00158476829528809 7.0
40.316920108 38.4642219543457 0.0
0 0.00427037477493286 1.0
0 -0.012304425239563 2.0
0 0.0196879506111145 3.0
0 0.0127553343772888 4.0
0 0.00777769088745117 5.0
0 0.00142759084701538 6.0
0 -0.00219595432281494 7.0
33.579405458 39.1697883605957 0.0
0 0.00949239730834961 1.0
0 0.0550859570503235 2.0
0 0.00363188982009888 3.0
0 -0.00875645875930786 4.0
0 -0.00982797145843506 5.0
0 0.00408774614334106 6.0
0 0.00239044427871704 7.0
34.562810956 38.7289924621582 0.0
0 -0.00927829742431641 1.0
0 -0.031139075756073 2.0
0 0.0254233479499817 3.0
0 -0.00332009792327881 4.0
0 -0.00410580635070801 5.0
0 0.00320959091186523 6.0
0 0.00121003389358521 7.0
37.518824119 38.526123046875 0.0
0 -0.00417220592498779 1.0
0 0.0312619805335999 2.0
0 -0.0036507248878479 3.0
0 0.000343263149261475 4.0
0 -0.00207829475402832 5.0
0 0.00180166959762573 6.0
0 -0.00563240051269531 7.0
39.547705636 38.8937301635742 0.0
0 0.00837802886962891 1.0
0 0.0164691209793091 2.0
0 0.0180018544197083 3.0
0 0.00425541400909424 4.0
0 0.00221371650695801 5.0
0 0.00231784582138062 6.0
0 -0.00168567895889282 7.0
42.834945783 38.4985198974609 0.0
0 0.00751316547393799 1.0
0 -0.00238978862762451 2.0
0 0.0189425945281982 3.0
0 0.00994026660919189 4.0
0 0.0116481781005859 5.0
0 -9.0181827545166e-05 6.0
0 -0.00097280740737915 7.0
38.473622953 38.5314598083496 0.0
0 -0.00472825765609741 1.0
0 0.00927400588989258 2.0
0 0.0143502950668335 3.0
0 0.00991624593734741 4.0
0 0.0122590065002441 5.0
0 0.00189471244812012 6.0
0 0.0014805793762207 7.0
45.238523457 38.3235244750977 0.0
0 0.00389409065246582 1.0
0 0.0224522352218628 2.0
0 0.013031542301178 3.0
0 0.00490808486938477 4.0
0 0.00208324193954468 5.0
0 0.00188267230987549 6.0
0 -0.003562331199646 7.0
36.266282562 38.6187591552734 0.0
0 0.00554311275482178 1.0
0 -0.00330221652984619 2.0
0 -3.94582748413086e-05 3.0
0 0.0115361213684082 4.0
0 0.00921422243118286 5.0
0 0.00178325176239014 6.0
0 0.00104552507400513 7.0
37.845233948 38.7518157958984 0.0
0 -0.000949203968048096 1.0
0 0.00222766399383545 2.0
0 0.0181483030319214 3.0
0 0.00984752178192139 4.0
0 0.00675064325332642 5.0
0 0.00161761045455933 6.0
0 -0.00165462493896484 7.0
40.414778929 38.6832542419434 0.0
0 0.0396010279655457 1.0
0 0.0414213538169861 2.0
0 0.034054696559906 3.0
0 -0.0138557553291321 4.0
0 -0.0228312611579895 5.0
0 0.00423502922058105 6.0
0 -0.00209474563598633 7.0
47.104382604 38.6785850524902 0.0
0 -0.000528335571289062 1.0
0 -0.0117012858390808 2.0
0 0.00560390949249268 3.0
0 0.0113921761512756 4.0
0 0.0118933320045471 5.0
0 0.00046849250793457 6.0
0 -0.00160872936248779 7.0
37.557508388 38.6452941894531 0.0
0 0.0107706189155579 1.0
0 0.0139909386634827 2.0
0 0.0265346765518188 3.0
0 -0.00883990526199341 4.0
0 -0.00819224119186401 5.0
0 0.00279784202575684 6.0
0 0.000886917114257812 7.0
40.532317042 38.8532943725586 0.0
0 -0.00338459014892578 1.0
0 0.00208669900894165 2.0
0 -0.00943958759307861 3.0
0 0.0108900666236877 4.0
0 0.0107890367507935 5.0
0 0.00126546621322632 6.0
0 0.00102007389068604 7.0
32.594348946 38.9392776489258 0.0
0 -0.00994551181793213 1.0
0 -0.00474840402603149 2.0
0 0.0156255960464478 3.0
0 0.00870126485824585 4.0
0 -0.000639498233795166 5.0
0 0.00391221046447754 6.0
0 0.00118041038513184 7.0
30.392342502 38.4610557556152 0.0
0 0.00475746393203735 1.0
0 0.00234395265579224 2.0
0 0.00362575054168701 3.0
0 0.0170789361000061 4.0
0 0.0154737830162048 5.0
0 0.00255101919174194 6.0
0 0.00137734413146973 7.0
40.607017947 38.6572570800781 0.0
0 -0.00629860162734985 1.0
0 0.0028877854347229 2.0
0 0.00213658809661865 3.0
0 0.0124313235282898 4.0
0 0.00775974988937378 5.0
0 0.00156289339065552 6.0
0 -0.00161468982696533 7.0
39.11337914 39.3402252197266 0.0
0 0.053605318069458 1.0
0 0.106695532798767 2.0
0 0.0744338035583496 3.0
0 -0.0374942421913147 4.0
0 -0.0441901683807373 5.0
0 0.00579911470413208 6.0
0 -0.00190454721450806 7.0
35.280627531 38.4612503051758 0.0
0 -0.00280666351318359 1.0
0 0.0293827652931213 2.0
0 -0.0139994621276855 3.0
0 -0.0032992959022522 4.0
0 -0.0042424201965332 5.0
0 -0.00224858522415161 6.0
0 -0.0053715705871582 7.0
32.847447182 38.3127861022949 0.0
0 0.00380522012710571 1.0
0 0.00457185506820679 2.0
0 -0.0068669319152832 3.0
0 0.019364058971405 4.0
0 0.0192620754241943 5.0
0 0.00249946117401123 6.0
0 0.00178861618041992 7.0
34.14972673 38.8230133056641 0.0
0 -0.0161435604095459 1.0
0 0.0640211701393127 2.0
0 0.0416672229766846 3.0
0 -0.0326667428016663 4.0
0 -0.0316633582115173 5.0
0 0.00657165050506592 6.0
0 0.0048445463180542 7.0
36.823986173 38.8261451721191 0.0
0 0.000983893871307373 1.0
0 0.0175092816352844 2.0
0 0.0232627391815186 3.0
0 0.00660783052444458 4.0
0 0.00244992971420288 5.0
0 0.00188249349594116 6.0
0 -0.00359439849853516 7.0
43.615049092 38.7269248962402 0.0
0 -0.00817251205444336 1.0
0 -0.016715943813324 2.0
0 0.021268367767334 3.0
0 -0.0094103217124939 4.0
0 -0.00766164064407349 5.0
0 0.00359576940536499 6.0
0 0.00358837842941284 7.0
39.316465342 38.8074150085449 0.0
0 0.00467908382415771 1.0
0 -0.141974627971649 2.0
0 0.00819063186645508 3.0
0 -0.0243714451789856 4.0
0 -0.0265893340110779 5.0
0 0.00668883323669434 6.0
0 0.00609123706817627 7.0
36.676849778 39.086483001709 0.0
0 0.0338906645774841 1.0
0 0.010081946849823 2.0
0 0.0517814159393311 3.0
0 -0.0148599147796631 4.0
0 -0.015914261341095 5.0
0 0.00404250621795654 6.0
0 -0.000586926937103271 7.0
36.163160136 39.0178642272949 0.0
0 0.0213291645050049 1.0
0 0.000769555568695068 2.0
0 0.0280357003211975 3.0
0 -0.0171972513198853 4.0
0 -0.0217657089233398 5.0
0 0.00370281934738159 6.0
0 -0.00175964832305908 7.0
40.305175496 38.8357582092285 0.0
0 0.00539052486419678 1.0
0 0.00232678651809692 2.0
0 -0.00763154029846191 3.0
0 0.00963932275772095 4.0
0 0.00745934247970581 5.0
0 0.00154834985733032 6.0
0 0.000475943088531494 7.0
34.293236222 38.5812950134277 0.0
0 0.0108577609062195 1.0
0 -0.00930273532867432 2.0
0 0.00788706541061401 3.0
0 0.0121802091598511 4.0
0 0.0107622146606445 5.0
0 0.00162261724472046 6.0
0 0.000948786735534668 7.0
38.347299924 38.5173797607422 0.0
0 0.00819861888885498 1.0
0 0.00382977724075317 2.0
0 0.0101303458213806 3.0
0 0.0126463174819946 4.0
0 0.0123133063316345 5.0
0 0.00140637159347534 6.0
0 0.000719904899597168 7.0
39.036910781 38.8415069580078 0.0
0 0.000177502632141113 1.0
0 0.00544929504394531 2.0
0 -0.0120937824249268 3.0
0 0.00871700048446655 4.0
0 0.00971460342407227 5.0
0 0.0013730525970459 6.0
0 0.00144284963607788 7.0
36.237500285 38.6325035095215 0.0
0 0.00555747747421265 1.0
0 -0.00464928150177002 2.0
0 0.00741976499557495 3.0
0 0.0146653056144714 4.0
0 0.0154220461845398 5.0
0 0.00221884250640869 6.0
0 0.00104159116744995 7.0
36.900540753 38.554141998291 0.0
0 -0.00575870275497437 1.0
0 0.0343371629714966 2.0
0 0.00224876403808594 3.0
0 0.013765811920166 4.0
0 0.00792694091796875 5.0
0 0.0021330714225769 6.0
0 -0.00349611043930054 7.0
29.897899633 38.8244743347168 0.0
0 -0.00203788280487061 1.0
0 0.0106021165847778 2.0
0 0.0118087530136108 3.0
0 0.00280171632766724 4.0
0 -0.000840127468109131 5.0
0 0.00222545862197876 6.0
0 -0.00134199857711792 7.0
41.071373967 38.722110748291 0.0
0 -0.000163435935974121 1.0
0 0.00294435024261475 2.0
0 -0.00318533182144165 3.0
0 0.0174000263214111 4.0
0 0.00768005847930908 5.0
0 0.00268244743347168 6.0
0 -0.000645101070404053 7.0
37.123617582 38.6545639038086 0.0
0 0.0106359124183655 1.0
0 -0.00192886590957642 2.0
0 0.0364857912063599 3.0
0 -0.0049210786819458 4.0
0 -0.00993657112121582 5.0
0 0.00295078754425049 6.0
0 -0.00167167186737061 7.0
38.348981744 39.361213684082 0.0
0 0.108397841453552 1.0
0 0.0719989538192749 2.0
0 0.0854716300964355 3.0
0 -0.0621429681777954 4.0
0 -0.0700206160545349 5.0
0 0.00875371694564819 6.0
0 0.00082772970199585 7.0
37.897829023 38.3081512451172 0.0
0 0.0116560459136963 1.0
0 -0.0115576982498169 2.0
0 0.0105234384536743 3.0
0 0.0114278793334961 4.0
0 0.00531607866287231 5.0
0 0.000552892684936523 6.0
0 -0.0051882266998291 7.0
32.839543182 38.3974990844727 0.0
0 0.00181096792221069 1.0
0 0.00699031352996826 2.0
0 -0.00629299879074097 3.0
0 0.0180360674858093 4.0
0 0.0172174572944641 5.0
0 0.00267308950424194 6.0
0 0.00208872556686401 7.0
45.814256844 38.7030944824219 0.0
0 0.00507837533950806 1.0
0 0.00886917114257812 2.0
0 -0.0037347674369812 3.0
0 0.0136943459510803 4.0
0 0.0137937068939209 5.0
0 0.00230222940444946 6.0
0 0.00160950422286987 7.0
38.439958634 38.4825057983398 0.0
0 -0.00018012523651123 1.0
0 0.0141959190368652 2.0
0 -0.0422380566596985 3.0
0 -0.0285893678665161 4.0
0 -0.0323619246482849 5.0
0 0.00240993499755859 6.0
0 -0.00253540277481079 7.0
33.022424116 38.5725517272949 0.0
0 -0.00100618600845337 1.0
0 0.0236772298812866 2.0
0 0.0142524242401123 3.0
0 0.00742864608764648 4.0
0 0.00858741998672485 5.0
0 0.000316262245178223 6.0
0 -0.00097280740737915 7.0
36.232706353 38.6497421264648 0.0
0 0.00666439533233643 1.0
0 0.0102894306182861 2.0
0 -0.00639992952346802 3.0
0 0.0107094049453735 4.0
0 0.00857967138290405 5.0
0 0.00131088495254517 6.0
0 0.000430107116699219 7.0
45.8365445 38.2908325195312 0.0
0 0.00307029485702515 1.0
0 -0.00923150777816772 2.0
0 0.00529932975769043 3.0
0 0.00588589906692505 4.0
0 0.0053897500038147 5.0
0 0.000756263732910156 6.0
0 -0.00225520133972168 7.0
39.344253832 38.8997688293457 0.0
0 0.00560510158538818 1.0
0 0.012593150138855 2.0
0 0.020938515663147 3.0
0 0.00587159395217896 4.0
0 0.00207465887069702 5.0
0 0.00155848264694214 6.0
0 -0.00285935401916504 7.0
43.613379879 38.2364540100098 0.0
0 0.000397384166717529 1.0
0 0.031568169593811 2.0
0 -0.00499022006988525 3.0
0 -0.00342869758605957 4.0
0 -0.00124233961105347 5.0
0 0.00189417600631714 6.0
0 -0.00510674715042114 7.0
41.603825526 38.189998626709 0.0
0 0.00724172592163086 1.0
0 -0.0164687633514404 2.0
0 0.00195282697677612 3.0
0 0.0148271322250366 4.0
0 0.0068591833114624 5.0
0 0.000564455986022949 6.0
0 -0.00725555419921875 7.0
41.290499737 38.4870338439941 0.0
0 0.0102403163909912 1.0
0 -0.00607091188430786 2.0
0 0.0221967697143555 3.0
0 0.0122079849243164 4.0
0 0.00884580612182617 5.0
0 0.000471353530883789 6.0
0 -0.0037955641746521 7.0
38.21592954 38.5617942810059 0.0
0 -0.000384032726287842 1.0
0 0.00387489795684814 2.0
0 -0.00276827812194824 3.0
0 0.011198878288269 4.0
0 0.013520359992981 5.0
0 0.00102299451828003 6.0
0 0.00107133388519287 7.0
34.764993855 38.4070091247559 0.0
0 -0.00358575582504272 1.0
0 0.0124229192733765 2.0
0 -0.00962519645690918 3.0
0 -0.00281190872192383 4.0
0 -0.00192475318908691 5.0
0 0.000529289245605469 6.0
0 -0.00529855489730835 7.0
34.789226058 38.5943145751953 0.0
0 -0.000479459762573242 1.0
0 0.0239938497543335 2.0
0 0.0198885202407837 3.0
0 0.00419110059738159 4.0
0 0.00250375270843506 5.0
0 0.0020715594291687 6.0
0 -0.00529903173446655 7.0
42.488773183 38.6152381896973 0.0
0 0.00836420059204102 1.0
0 -0.00419712066650391 2.0
0 0.00586044788360596 3.0
0 0.0160481929779053 4.0
0 0.0162471532821655 5.0
0 0.00226736068725586 6.0
0 0.000960052013397217 7.0
34.702233392 39.2018165588379 0.0
0 0.00722146034240723 1.0
0 0.0153907537460327 2.0
0 0.0345809459686279 3.0
0 -0.000850021839141846 4.0
0 -0.00449872016906738 5.0
0 0.00279438495635986 6.0
0 -0.00392633676528931 7.0
41.831412446 39.3694381713867 0.0
0 0.0202068090438843 1.0
0 0.00735104084014893 2.0
0 -0.01542729139328 3.0
0 0.00598728656768799 4.0
0 0.000746250152587891 5.0
0 0.00259238481521606 6.0
0 -0.00198984146118164 7.0
45.908850318 38.7893257141113 0.0
0 0.0128718614578247 1.0
0 0.0439763069152832 2.0
0 -0.053998589515686 3.0
0 -0.0368396043777466 4.0
0 -0.0414500832557678 5.0
0 0.00407934188842773 6.0
0 -0.00154948234558105 7.0
33.104401139 39.2357902526855 0.0
0 0.0355506539344788 1.0
0 0.0204304456710815 2.0
0 0.0460935235023499 3.0
0 -0.00619632005691528 4.0
0 -0.0127991437911987 5.0
0 0.00329005718231201 6.0
0 -0.00491511821746826 7.0
35.835170704 38.656005859375 0.0
0 -0.0135376453399658 1.0
0 0.0272335410118103 2.0
0 0.018695592880249 3.0
0 -0.0218937397003174 4.0
0 -0.0219389796257019 5.0
0 0.00600516796112061 6.0
0 0.00515711307525635 7.0
45.844507362 38.3923797607422 0.0
0 0.00346899032592773 1.0
0 -0.0049707293510437 2.0
0 -0.0142909288406372 3.0
0 0.0028308629989624 4.0
0 0.00224250555038452 5.0
0 -0.00219684839248657 6.0
0 -0.00600039958953857 7.0
31.875262674 38.9218482971191 0.0
0 0.00548732280731201 1.0
0 0.03155916929245 2.0
0 0.0234208106994629 3.0
0 0.00218135118484497 4.0
0 0.000923514366149902 5.0
0 0.000942528247833252 6.0
0 -0.00415825843811035 7.0
35.888104099 38.465877532959 0.0
0 -0.00457191467285156 1.0
0 0.0129864811897278 2.0
0 0.0154435038566589 3.0
0 0.00746810436248779 4.0
0 0.00840973854064941 5.0
0 0.000717759132385254 6.0
0 -0.000674307346343994 7.0
35.733201066 38.5028800964355 0.0
0 0.00586575269699097 1.0
0 0.0072481632232666 2.0
0 0.0173937678337097 3.0
0 0.0100708603858948 4.0
0 0.00668007135391235 5.0
0 0.00180494785308838 6.0
0 -0.00377106666564941 7.0
43.710725429 38.3162002563477 0.0
0 -0.00887012481689453 1.0
0 -0.00385814905166626 2.0
0 -0.00318706035614014 3.0
0 -0.0185797214508057 4.0
0 -0.0196681022644043 5.0
0 0.000224053859710693 6.0
0 -0.00327861309051514 7.0
34.934843039 38.8348121643066 0.0
0 -0.00161290168762207 1.0
0 0.0159087181091309 2.0
0 0.0234295129776001 3.0
0 0.0048181414604187 4.0
0 0.00403618812561035 5.0
0 0.00204634666442871 6.0
0 -0.00230425596237183 7.0
38.70218214 38.7474479675293 0.0
0 -0.00108134746551514 1.0
0 0.0091174840927124 2.0
0 -0.00267726182937622 3.0
0 0.00788772106170654 4.0
0 0.00635135173797607 5.0
0 0.00159680843353271 6.0
0 0.0013083815574646 7.0
39.146786058 39.0480194091797 0.0
0 -0.0114285945892334 1.0
0 0.00788998603820801 2.0
0 -0.0154602527618408 3.0
0 0.0109805464744568 4.0
0 0.00477433204650879 5.0
0 0.00266498327255249 6.0
0 0.000880956649780273 7.0
37.956003494 38.3549499511719 0.0
0 0.00789499282836914 1.0
0 0.00629401206970215 2.0
0 -0.0352148413658142 3.0
0 -0.0112420320510864 4.0
0 -0.0142526030540466 5.0
0 0.00200521945953369 6.0
0 -0.00365465879440308 7.0
35.896232558 39.0284271240234 0.0
0 0.00879734754562378 1.0
0 0.0454597473144531 2.0
0 0.00259673595428467 3.0
0 -0.0103989839553833 4.0
0 -0.0144324898719788 5.0
0 0.00407028198242188 6.0
0 -0.000433206558227539 7.0
37.377916966 39.028507232666 0.0
0 0.0277334451675415 1.0
0 0.103057205677032 2.0
0 0.0302638411521912 3.0
0 -0.0119578838348389 4.0
0 -0.0142959356307983 5.0
0 0.00127959251403809 6.0
0 -0.00192588567733765 7.0
36.698956497 38.7372550964355 0.0
0 -0.0109740495681763 1.0
0 0.0191500186920166 2.0
0 -0.0165144205093384 3.0
0 0.0105429291725159 4.0
0 0.00730335712432861 5.0
0 0.00290632247924805 6.0
0 0.00153911113739014 7.0
41.903925455 38.9670372009277 0.0
0 0.0159850120544434 1.0
0 0.0163652896881104 2.0
0 0.0204038023948669 3.0
0 0.00393366813659668 4.0
0 -0.00349020957946777 5.0
0 0.00331735610961914 6.0
0 -0.00523614883422852 7.0
38.934024851 38.2444496154785 0.0
0 0.00585168600082397 1.0
0 0.0038304328918457 2.0
0 -0.00481849908828735 3.0
0 0.0178231000900269 4.0
0 0.0161932110786438 5.0
0 0.00253915786743164 6.0
0 0.00152045488357544 7.0
40.406325889 38.474781036377 0.0
0 -0.00402039289474487 1.0
0 0.0110981464385986 2.0
0 -0.0641270279884338 3.0
0 -0.035135805606842 4.0
0 -0.0373694896697998 5.0
0 0.00301212072372437 6.0
0 -0.00229912996292114 7.0
39.410747699 38.5370864868164 0.0
0 -0.00339770317077637 1.0
0 0.020703911781311 2.0
0 0.00961828231811523 3.0
0 0.00251716375350952 4.0
0 0.00325173139572144 5.0
0 0.00136208534240723 6.0
0 -0.0024869441986084 7.0
};
\addlegendentry{$R^2$=0.988}
\end{axis}
\end{tikzpicture}


%     \caption{Your caption here}\label{fig:your-label-here}
% \end{figure}
%

\subsection{Case Study I: 8-node Network}


As a result of the optimization process for this first test, the following hyperparameters were obtained: $N \ channels=21$, $N \ layers=5$, and $N \ dense=4$. With these optimized values, the model achieved an MSE of 1.98. In \cref{fig:results_dummy_base_node}, a scatter plot illustrates the relationship between the actual values of gas generation at the nodes and the corresponding values predicted by the trained neural network, considering only the losses at the nodes. The plot highlights how effectively the network captures the fact that only one of the nodes in the system has gas generation. However, while the network successfully identifies the generating node, the predicted values exhibit less dispersion than the actual values, indicating that the model's predictions are more concentrated around a certain point.


\Cref{fig:results_dummy_base_f} displays the relationship between the actual gas flows through the edges and the predictions made by the neural network. In this case, the model struggled to predict the flow values accurately, demonstrating a significant deviation from the actual data. This result is not unexpected, as the model used in this experiment focused solely on the losses related to gas flows at the injection nodes without accounting for the gas transported through the edges. As such, the absence of edge-related loss consideration likely contributed to the poor prediction performance for edge flows.


\begin{figure}
    \centering
    \setlength\figurewidth{.53\textwidth}        
    \setlength\figureheight{0.36\textwidth} 
    \subfloat[Actual vs predicted nodal flows.] 
    {\label{fig:results_dummy_base_node}\resizebox{\figurewidth}{\figureheight}{\begin{tikzpicture}

\definecolor{darkgray176}{RGB}{176,176,176}
\definecolor{lightgray204}{RGB}{204,204,204}

\begin{axis}[
    colorbar,
    colorbar sampled,
    colorbar style={
        samples=8,
        ylabel={node\_id},
        ytick={0,1,...,7},
        yticklabels={0,1,2,3,4,5,6,7},
    },
    colormap={mymap}{[1pt]
        rgb(0pt)=(0.12156862745098,0.466666666666667,0.705882352941177)
        rgb(1pt)=(1,0.498039215686275,0.0549019607843137)
        rgb(2pt)=(0.172549019607843,0.627450980392157,0.172549019607843)
        rgb(3pt)=(0.83921568627451,0.152941176470588,0.156862745098039)
        rgb(4pt)=(0.580392156862745,0.403921568627451,0.741176470588235)
        rgb(5pt)=(0.549019607843137,0.337254901960784,0.294117647058824)
        rgb(6pt)=(0.890196078431372,0.466666666666667,0.76078431372549)
        rgb(7pt)=(0.498039215686275,0.498039215686275,0.498039215686275)
    },
legend cell align={left},
legend style={
  fill opacity=0.8,
  draw opacity=1,
  text opacity=1,
  at={(0.03,0.97)},
  anchor=north west,
  draw=lightgray204
},
point meta max=7,
point meta min=0,
tick align=outside,
tick pos=left,
title={},
x grid style={darkgray176},
xlabel={y true},
xmajorgrids,
xmin=-2.3894448262, xmax=50.1783413502,
% xtick style={color=black},
y grid style={darkgray176},
ylabel={y pred},
ymajorgrids,
ymin=-2.18082528710365, ymax=41.9992832481861,
ytick style={color=black}
]
\addplot [
    colormap={mymap}{[1pt]
        rgb(0pt)=(0.12156862745098,0.466666666666667,0.705882352941177)
        rgb(1pt)=(1,0.498039215686275,0.0549019607843137)
        rgb(2pt)=(0.172549019607843,0.627450980392157,0.172549019607843)
        rgb(3pt)=(0.83921568627451,0.152941176470588,0.156862745098039)
        rgb(4pt)=(0.580392156862745,0.403921568627451,0.741176470588235)
        rgb(5pt)=(0.549019607843137,0.337254901960784,0.294117647058824)
        rgb(6pt)=(0.890196078431372,0.466666666666667,0.76078431372549)
        rgb(7pt)=(0.498039215686275,0.498039215686275,0.498039215686275)
    },
    only marks,
    scatter,
    scatter src=explicit
]
table [x=x, y=y, meta=colordata]{%
x  y  colordata
41.603277614 39.7535934448242 0.0
0 0.0908881425857544 1.0
0 0.139073312282562 2.0
0 0.0316752195358276 3.0
0 -0.0294172167778015 4.0
0 -0.0293693542480469 5.0
0 0.00516742467880249 6.0
0 0.00205957889556885 7.0
37.07638211 38.3591461181641 0.0
0 0.0097118616104126 1.0
0 -0.0219908952713013 2.0
0 -0.0110254883766174 3.0
0 0.00467157363891602 4.0
0 0.00213146209716797 5.0
0 -0.00148683786392212 6.0
0 -0.00642764568328857 7.0
39.245039256 38.5584983825684 0.0
0 0.0252088308334351 1.0
0 0.12731671333313 2.0
0 0.0389952659606934 3.0
0 -0.0199683308601379 4.0
0 -0.0247794985771179 5.0
0 0.00470894575119019 6.0
0 0.00119262933731079 7.0
33.40222464 39.6044502258301 0.0
0 0.0798147916793823 1.0
0 0.0904387831687927 2.0
0 0.0590900778770447 3.0
0 -0.0234290957450867 4.0
0 -0.0263727903366089 5.0
0 0.00418353080749512 6.0
0 0.000778496265411377 7.0
40.177942425 38.2198257446289 0.0
0 0.00198698043823242 1.0
0 0.00119161605834961 2.0
0 -0.0142090320587158 3.0
0 0.00341099500656128 4.0
0 1.15036964416504e-05 5.0
0 0.000508427619934082 6.0
0 -0.00637626647949219 7.0
33.661272263 38.4802513122559 0.0
0 -0.00348472595214844 1.0
0 0.0167836546897888 2.0
0 0.0134863257408142 3.0
0 0.0085369348526001 4.0
0 0.00687175989151001 5.0
0 0.00102633237838745 6.0
0 -0.00393420457839966 7.0
41.918555681 38.3983917236328 0.0
0 -0.00454980134963989 1.0
0 0.0024217963218689 2.0
0 -0.000810742378234863 3.0
0 0.011707603931427 4.0
0 0.00938022136688232 5.0
0 0.00149333477020264 6.0
0 0.000492334365844727 7.0
44.067434787 38.4977531433105 0.0
0 0.00612330436706543 1.0
0 0.00236880779266357 2.0
0 0.00920677185058594 3.0
0 0.0120137929916382 4.0
0 0.0121665596961975 5.0
0 0.000962913036346436 6.0
0 0.000135302543640137 7.0
31.762979365 38.5268859863281 0.0
0 0.00982803106307983 1.0
0 -0.00357627868652344 2.0
0 0.00891649723052979 3.0
0 0.0133818984031677 4.0
0 0.0117686986923218 5.0
0 0.001045823097229 6.0
0 0.000113010406494141 7.0
41.030120159 38.5200500488281 0.0
0 -0.0298738479614258 1.0
0 -0.0113872885704041 2.0
0 0.0223647356033325 3.0
0 -0.0168793201446533 4.0
0 -0.0176181793212891 5.0
0 0.00564652681350708 6.0
0 0.00426644086837769 7.0
40.055714877 38.8306159973145 0.0
0 -7.34925270080566e-05 1.0
0 -0.0110159516334534 2.0
0 0.0156899690628052 3.0
0 0.000327527523040771 4.0
0 -0.00162017345428467 5.0
0 0.00265783071517944 6.0
0 -0.000417232513427734 7.0
40.714244155 38.4069595336914 0.0
0 0.00929087400436401 1.0
0 -0.0149266719818115 2.0
0 0.00734519958496094 3.0
0 0.00849682092666626 4.0
0 0.0102183222770691 5.0
0 0.000343441963195801 6.0
0 -0.00474280118942261 7.0
45.139727698 38.8884696960449 0.0
0 0.000342428684234619 1.0
0 0.00625211000442505 2.0
0 0.00714671611785889 3.0
0 0.00820708274841309 4.0
0 0.00730991363525391 5.0
0 0.00169193744659424 6.0
0 0.000112235546112061 7.0
40.583613059 39.263557434082 0.0
0 0.0559103488922119 1.0
0 0.0864608585834503 2.0
0 0.0712517499923706 3.0
0 -0.0344110727310181 4.0
0 -0.0397318601608276 5.0
0 0.00477957725524902 6.0
0 -0.00180310010910034 7.0
35.441291819 38.5217590332031 0.0
0 -0.00431793928146362 1.0
0 -0.0136868357658386 2.0
0 0.00991904735565186 3.0
0 -0.0215700268745422 4.0
0 -0.0221145749092102 5.0
0 0.00257503986358643 6.0
0 -0.00314760208129883 7.0
36.08841077 38.8402824401855 0.0
0 0.00536257028579712 1.0
0 0.0375674962997437 2.0
0 0.0245164632797241 3.0
0 -0.00596868991851807 4.0
0 -0.00369197130203247 5.0
0 0.00242996215820312 6.0
0 0.00174403190612793 7.0
40.494343306 38.6495323181152 0.0
0 -0.00675559043884277 1.0
0 -0.005912184715271 2.0
0 -0.00215411186218262 3.0
0 0.00915825366973877 4.0
0 0.00896823406219482 5.0
0 0.00157058238983154 6.0
0 0.00130730867385864 7.0
44.400196583 38.5482749938965 0.0
0 0.00762939453125 1.0
0 0.00384992361068726 2.0
0 0.0038907527923584 3.0
0 0.0148396492004395 4.0
0 0.0139151811599731 5.0
0 0.00187838077545166 6.0
0 0.000854194164276123 7.0
40.197028842 38.6482429504395 0.0
0 -0.00523370504379272 1.0
0 0.0368385314941406 2.0
0 0.0156623125076294 3.0
0 0.00713348388671875 4.0
0 0.00676637887954712 5.0
0 0.000627398490905762 6.0
0 -0.000256776809692383 7.0
39.931867517 38.7960472106934 0.0
0 -0.00283390283584595 1.0
0 -0.00708603858947754 2.0
0 -0.00414395332336426 3.0
0 0.00981372594833374 4.0
0 0.00716376304626465 5.0
0 0.0015750527381897 6.0
0 0.000669717788696289 7.0
37.28322745 38.2965774536133 0.0
0 0.00495332479476929 1.0
0 -0.00355356931686401 2.0
0 0.0140035152435303 3.0
0 0.011663019657135 4.0
0 0.00884330272674561 5.0
0 0.00177943706512451 6.0
0 0.00082862377166748 7.0
37.429390014 38.4880485534668 0.0
0 0.0135406255722046 1.0
0 0.00302684307098389 2.0
0 0.0197659134864807 3.0
0 -0.00780308246612549 4.0
0 -0.00855451822280884 5.0
0 0.00318622589111328 6.0
0 0.000750601291656494 7.0
39.499146179 38.6164703369141 0.0
0 0.00107491016387939 1.0
0 -0.00115358829498291 2.0
0 0.0128684639930725 3.0
0 0.0100945830345154 4.0
0 0.00789058208465576 5.0
0 0.000394701957702637 6.0
0 -0.00226247310638428 7.0
35.661352504 38.4398956298828 0.0
0 0.0203315615653992 1.0
0 0.0788954496383667 2.0
0 -0.0757306814193726 3.0
0 -0.0765065550804138 4.0
0 -0.0808522701263428 5.0
0 0.0057339072227478 6.0
0 0.00214129686355591 7.0
44.903668391 38.679759979248 0.0
0 0.0104649066925049 1.0
0 -0.00321555137634277 2.0
0 0.00385379791259766 3.0
0 0.0129628777503967 4.0
0 0.0148612260818481 5.0
0 0.00193989276885986 6.0
0 0.00181066989898682 7.0
34.118473664 39.0513191223145 0.0
0 0.0307442545890808 1.0
0 0.0471727848052979 2.0
0 0.0385069847106934 3.0
0 -0.0240370631217957 4.0
0 -0.0236438512802124 5.0
0 0.00260704755783081 6.0
0 0.011364221572876 7.0
36.466593361 38.987190246582 0.0
0 -0.0103698968887329 1.0
0 0.0800180435180664 2.0
0 0.0414098501205444 3.0
0 -0.0267103314399719 4.0
0 -0.0288975834846497 5.0
0 0.00566142797470093 6.0
0 0.0028914213180542 7.0
34.994247693 38.3411102294922 0.0
0 0.00688672065734863 1.0
0 -0.00696879625320435 2.0
0 0.0116889476776123 3.0
0 0.00811731815338135 4.0
0 0.0102164149284363 5.0
0 0.00122654438018799 6.0
0 0.00214976072311401 7.0
39.553868661 38.6714859008789 0.0
0 -0.00435274839401245 1.0
0 -0.000196635723114014 2.0
0 -0.00985980033874512 3.0
0 0.0103663206100464 4.0
0 0.0117907524108887 5.0
0 0.0014764666557312 6.0
0 0.00132846832275391 7.0
43.158887007 38.4424667358398 0.0
0 0.00920790433883667 1.0
0 -0.00932449102401733 2.0
0 0.0151568651199341 3.0
0 0.0135409832000732 4.0
0 0.00888550281524658 5.0
0 0.00103974342346191 6.0
0 -0.00307869911193848 7.0
36.693891456 39.0564842224121 0.0
0 0.00571656227111816 1.0
0 0.0139431953430176 2.0
0 0.00666135549545288 3.0
0 0.00349867343902588 4.0
0 0.000229835510253906 5.0
0 0.00237119197845459 6.0
0 -0.000682830810546875 7.0
37.419072323 38.5162734985352 0.0
0 0.0100835561752319 1.0
0 -0.00280958414077759 2.0
0 0.011082649230957 3.0
0 0.0125146508216858 4.0
0 0.0106852650642395 5.0
0 0.000358998775482178 6.0
0 -0.00108003616333008 7.0
42.391154887 37.6930847167969 0.0
0 0.00836706161499023 1.0
0 -0.0265249013900757 2.0
0 0.0109959840774536 3.0
0 -0.00843185186386108 4.0
0 -0.00866538286209106 5.0
0 0.00374925136566162 6.0
0 0.00252997875213623 7.0
39.976124969 39.3020973205566 0.0
0 0.0369466543197632 1.0
0 0.0207736492156982 2.0
0 0.00882303714752197 3.0
0 -0.0113847255706787 4.0
0 -0.0140944719314575 5.0
0 0.00480204820632935 6.0
0 0.00208133459091187 7.0
41.193094756 38.7121925354004 0.0
0 0.000246524810791016 1.0
0 0.0112533569335938 2.0
0 -0.00880825519561768 3.0
0 0.0115809440612793 4.0
0 0.00994521379470825 5.0
0 0.00153958797454834 6.0
0 0.000614643096923828 7.0
39.030467364 38.9754753112793 0.0
0 0.0133965015411377 1.0
0 0.0099719762802124 2.0
0 -0.0317625403404236 3.0
0 -0.0213124752044678 4.0
0 -0.0274962782859802 5.0
0 0.0034407377243042 6.0
0 -0.00381076335906982 7.0
39.254434013 38.8635673522949 0.0
0 -0.00899428129196167 1.0
0 0.028767466545105 2.0
0 -0.00187152624130249 3.0
0 0.00816291570663452 4.0
0 0.00780004262924194 5.0
0 0.00216001272201538 6.0
0 0.00177603960037231 7.0
41.474829476 38.490550994873 0.0
0 -0.00636947154998779 1.0
0 0.0157142877578735 2.0
0 -0.00421321392059326 3.0
0 0.0130150318145752 4.0
0 0.0113214254379272 5.0
0 0.00152826309204102 6.0
0 0.000341415405273438 7.0
36.065024548 38.5721321105957 0.0
0 0.00696414709091187 1.0
0 -0.00372463464736938 2.0
0 0.00787001848220825 3.0
0 0.00936907529830933 4.0
0 0.0126588344573975 5.0
0 0.00105565786361694 6.0
0 0.00142580270767212 7.0
33.547657937 39.1791458129883 0.0
0 0.0251626968383789 1.0
0 0.106061309576035 2.0
0 0.0769014954566956 3.0
0 -0.025641918182373 4.0
0 -0.0292165875434875 5.0
0 0.00278604030609131 6.0
0 -0.0029417872428894 7.0
35.860060933 38.5283851623535 0.0
0 -0.00231927633285522 1.0
0 -0.00571024417877197 2.0
0 -0.00389313697814941 3.0
0 0.00245767831802368 4.0
0 0.00144177675247192 5.0
0 0.00015634298324585 6.0
0 -0.00504642724990845 7.0
41.446935888 38.7161064147949 0.0
0 0.0399006009101868 1.0
0 0.118380934000015 2.0
0 -0.0819071531295776 3.0
0 -0.0782929062843323 4.0
0 -0.0900070071220398 5.0
0 0.00521427392959595 6.0
0 0.00152796506881714 7.0
39.09863661 38.6100082397461 0.0
0 -0.0173571109771729 1.0
0 0.0087469220161438 2.0
0 -0.0296517014503479 3.0
0 0.0122166275978088 4.0
0 0.00985217094421387 5.0
0 0.00263488292694092 6.0
0 0.00131088495254517 7.0
37.209660367 38.619556427002 0.0
0 -0.00431281328201294 1.0
0 0.0119017362594604 2.0
0 0.0246157646179199 3.0
0 0.00228214263916016 4.0
0 -0.00246334075927734 5.0
0 0.00261741876602173 6.0
0 -0.00559180974960327 7.0
37.318344805 38.4069023132324 0.0
0 0.00742429494857788 1.0
0 -0.00190436840057373 2.0
0 0.0183868408203125 3.0
0 0.00921815633773804 4.0
0 0.00853586196899414 5.0
0 0.000802695751190186 6.0
0 -0.00107866525650024 7.0
37.542238634 38.8763580322266 0.0
0 -0.00635802745819092 1.0
0 0.0211170315742493 2.0
0 0.00130510330200195 3.0
0 0.0125030279159546 4.0
0 0.00436419248580933 5.0
0 0.00310200452804565 6.0
0 0.00108665227890015 7.0
35.642638424 38.4852333068848 0.0
0 0.00880211591720581 1.0
0 -0.0149785280227661 2.0
0 0.00123864412307739 3.0
0 0.0102723836898804 4.0
0 0.00305092334747314 5.0
0 0.000930190086364746 6.0
0 -0.00708687305450439 7.0
47.4260719 38.6907234191895 0.0
0 -0.00334429740905762 1.0
0 -0.0110288262367249 2.0
0 -0.0155019164085388 3.0
0 -0.0171687602996826 4.0
0 -0.0232598185539246 5.0
0 0.00268292427062988 6.0
0 -0.00473958253860474 7.0
42.664227047 38.3595390319824 0.0
0 0.00736522674560547 1.0
0 0.169191211462021 2.0
0 -0.0193517208099365 3.0
0 -0.00202935934066772 4.0
0 -0.00715571641921997 5.0
0 0.000342905521392822 6.0
0 -0.00674760341644287 7.0
37.881981315 38.4804840087891 0.0
0 0.00734597444534302 1.0
0 0.0022200345993042 2.0
0 0.00960803031921387 3.0
0 0.0106253623962402 4.0
0 0.0120579600334167 5.0
0 0.00171905755996704 6.0
0 0.00124293565750122 7.0
44.031482397 38.7509765625 0.0
0 -0.0097581148147583 1.0
0 -0.0636894702911377 2.0
0 0.0015978217124939 3.0
0 -0.00409984588623047 4.0
0 -0.00204712152481079 5.0
0 0.00405192375183105 6.0
0 0.0043146014213562 7.0
40.215687867 38.6358032226562 0.0
0 0.00362402200698853 1.0
0 0.00420552492141724 2.0
0 0.000331521034240723 3.0
0 0.00297117233276367 4.0
0 0.000524163246154785 5.0
0 0.000945329666137695 6.0
0 -0.00605976581573486 7.0
35.401137883 38.9912109375 0.0
0 -0.00200784206390381 1.0
0 -0.000412046909332275 2.0
0 -0.0133203268051147 3.0
0 0.0117408633232117 4.0
0 0.00856119394302368 5.0
0 0.0021214485168457 6.0
0 -0.000339329242706299 7.0
40.23529515 38.3903541564941 0.0
0 0.0114810466766357 1.0
0 -0.00885409116744995 2.0
0 0.0132594108581543 3.0
0 0.0202721357345581 4.0
0 0.0152925848960876 5.0
0 0.000202775001525879 6.0
0 -0.00713396072387695 7.0
38.451822651 38.8377914428711 0.0
0 0.00414824485778809 1.0
0 -0.00228339433670044 2.0
0 0.00336885452270508 3.0
0 0.0121934413909912 4.0
0 0.00823593139648438 5.0
0 0.00146770477294922 6.0
0 -0.00342243909835815 7.0
41.356859392 38.3796310424805 0.0
0 0.00926601886749268 1.0
0 -0.0124003291130066 2.0
0 0.0076671838760376 3.0
0 0.0100225806236267 4.0
0 0.00797367095947266 5.0
0 0.00054556131362915 6.0
0 -0.00379371643066406 7.0
44.45602451 38.879566192627 0.0
0 0.00672298669815063 1.0
0 0.0123946666717529 2.0
0 -0.00913697481155396 3.0
0 0.0109050869941711 4.0
0 0.00889235734939575 5.0
0 0.00168204307556152 6.0
0 0.000749766826629639 7.0
36.344631329 38.5415077209473 0.0
0 0.00494116544723511 1.0
0 0.00470906496047974 2.0
0 0.00875461101531982 3.0
0 0.00978201627731323 4.0
0 0.00804340839385986 5.0
0 0.00102525949478149 6.0
0 -0.00314891338348389 7.0
35.989587139 39.0207290649414 0.0
0 -0.00257343053817749 1.0
0 -0.00773054361343384 2.0
0 -0.00264960527420044 3.0
0 0.00988805294036865 4.0
0 0.0075453519821167 5.0
0 0.00126409530639648 6.0
0 -0.000481665134429932 7.0
38.453275767 38.5509948730469 0.0
0 -0.00460779666900635 1.0
0 0.0267020463943481 2.0
0 -0.00305271148681641 3.0
0 0.00985264778137207 4.0
0 0.00969010591506958 5.0
0 0.00131440162658691 6.0
0 0.000271499156951904 7.0
36.388812201 38.5625762939453 0.0
0 0.00684624910354614 1.0
0 0.00579524040222168 2.0
0 0.00857532024383545 3.0
0 0.00900334119796753 4.0
0 0.0114079117774963 5.0
0 0.00167042016983032 6.0
0 0.00194859504699707 7.0
45.540732617 38.3559417724609 0.0
0 0.00538235902786255 1.0
0 0.00386589765548706 2.0
0 -0.00125622749328613 3.0
0 0.018405020236969 4.0
0 0.0137466788291931 5.0
0 0.00278580188751221 6.0
0 0.000956237316131592 7.0
34.414627117 38.921630859375 0.0
0 0.0324609279632568 1.0
0 0.0945376753807068 2.0
0 0.0671876668930054 3.0
0 -0.0498137474060059 4.0
0 -0.0511236786842346 5.0
0 0.0085570216178894 6.0
0 0.00632810592651367 7.0
39.317888978 38.3244590759277 0.0
0 -0.00816702842712402 1.0
0 0.0362486243247986 2.0
0 0.0158848762512207 3.0
0 0.00621277093887329 4.0
0 0.00248837471008301 5.0
0 0.00199019908905029 6.0
0 -0.00439673662185669 7.0
36.969505628 39.0306816101074 0.0
0 -0.00373166799545288 1.0
0 0.0273987650871277 2.0
0 -0.0015103816986084 3.0
0 0.00917762517929077 4.0
0 0.00721776485443115 5.0
0 0.00152438879013062 6.0
0 -0.000276803970336914 7.0
44.923617033 39.2973556518555 0.0
0 0.00290322303771973 1.0
0 0.0194564461708069 2.0
0 0.0318148732185364 3.0
0 -0.0022081732749939 4.0
0 -0.00278234481811523 5.0
0 0.00196754932403564 6.0
0 -0.000810801982879639 7.0
37.258159228 38.1323928833008 0.0
0 0.0120348930358887 1.0
0 -0.0394316911697388 2.0
0 -0.050422191619873 3.0
0 -0.0273675322532654 4.0
0 -0.0310018062591553 5.0
0 0.00130391120910645 6.0
0 -0.0026056170463562 7.0
47.488576672 38.4616012573242 0.0
0 0.013430118560791 1.0
0 0.0567920207977295 2.0
0 0.0599616765975952 3.0
0 -0.0214373469352722 4.0
0 -0.0249693393707275 5.0
0 0.00472307205200195 6.0
0 -0.000217974185943604 7.0
42.958399331 38.6427192687988 0.0
0 0.00441265106201172 1.0
0 0.00464749336242676 2.0
0 0.000767230987548828 3.0
0 0.0110027194023132 4.0
0 0.0115892887115479 5.0
0 0.00038456916809082 6.0
0 -0.000238478183746338 7.0
41.161203287 38.1443328857422 0.0
0 0.00848382711410522 1.0
0 -0.0180635452270508 2.0
0 -0.00918519496917725 3.0
0 0.00684559345245361 4.0
0 0.0074118971824646 5.0
0 0.000457406044006348 6.0
0 -0.00672441720962524 7.0
40.466838645 38.8036346435547 0.0
0 -0.00796282291412354 1.0
0 -0.0537744164466858 2.0
0 0.0214377641677856 3.0
0 -0.0128311514854431 4.0
0 -0.0138890147209167 5.0
0 0.00499492883682251 6.0
0 0.00396305322647095 7.0
33.757389454 38.6834144592285 0.0
0 0.028232216835022 1.0
0 0.0404295325279236 2.0
0 0.0279163718223572 3.0
0 -0.0216402411460876 4.0
0 -0.0251529216766357 5.0
0 0.00418293476104736 6.0
0 -0.0031440258026123 7.0
43.437770592 38.343563079834 0.0
0 0.00211387872695923 1.0
0 -0.000490009784698486 2.0
0 -0.00304973125457764 3.0
0 0.00690853595733643 4.0
0 0.00559860467910767 5.0
0 0.000310838222503662 6.0
0 -0.00541353225708008 7.0
39.308931201 38.8775672912598 0.0
0 0.00970524549484253 1.0
0 0.00253796577453613 2.0
0 -0.0164980292320251 3.0
0 0.0100165009498596 4.0
0 0.00814670324325562 5.0
0 0.00208121538162231 6.0
0 0.00108212232589722 7.0
40.600604276 38.547607421875 0.0
0 0.00340026617050171 1.0
0 0.00650149583816528 2.0
0 -0.00628137588500977 3.0
0 0.0181875228881836 4.0
0 0.0199186205863953 5.0
0 0.00151854753494263 6.0
0 0.000111699104309082 7.0
31.520377836 38.581787109375 0.0
0 -0.00357002019882202 1.0
0 0.00153231620788574 2.0
0 0.002410888671875 3.0
0 0.00847429037094116 4.0
0 0.0135117173194885 5.0
0 -0.000303924083709717 6.0
0 0.000370502471923828 7.0
39.969125591 38.6257934570312 0.0
0 -0.00313878059387207 1.0
0 0.0176121592521667 2.0
0 -0.000237464904785156 3.0
0 -0.0116665363311768 4.0
0 -0.0119089484214783 5.0
0 -4.19020652770996e-05 6.0
0 -0.00395911931991577 7.0
30.912106182 38.2456321716309 0.0
0 -0.00355678796768188 1.0
0 0.0208463668823242 2.0
0 0.00883865356445312 3.0
0 0.0121186971664429 4.0
0 0.0138341188430786 5.0
0 0.00130367279052734 6.0
0 0.000901222229003906 7.0
45.180443811 38.3854637145996 0.0
0 -0.00685745477676392 1.0
0 0.0200044512748718 2.0
0 0.0115278959274292 3.0
0 -0.00447863340377808 4.0
0 -0.00777232646942139 5.0
0 0.00198572874069214 6.0
0 -0.00468242168426514 7.0
34.99216559 38.5155220031738 0.0
0 0.00417304039001465 1.0
0 0.0240988731384277 2.0
0 0.0120140314102173 3.0
0 -0.00103104114532471 4.0
0 -0.00382125377655029 5.0
0 0.00195038318634033 6.0
0 -0.0034911036491394 7.0
40.933938659 38.6754302978516 0.0
0 0.00394058227539062 1.0
0 0.0120728611946106 2.0
0 0.00304794311523438 3.0
0 -0.00491464138031006 4.0
0 -0.00522351264953613 5.0
0 0.00198507308959961 6.0
0 -0.00257027149200439 7.0
40.317005952 38.9766120910645 0.0
0 0.00107008218765259 1.0
0 0.00469475984573364 2.0
0 -0.010001540184021 3.0
0 0.00944429636001587 4.0
0 0.00491970777511597 5.0
0 0.00197219848632812 6.0
0 0.000297665596008301 7.0
40.010379617 38.5953788757324 0.0
0 -0.00856190919876099 1.0
0 0.0215948820114136 2.0
0 -0.000319242477416992 3.0
0 0.0154175162315369 4.0
0 0.00876986980438232 5.0
0 0.00185507535934448 6.0
0 -0.00385445356369019 7.0
38.685596488 38.8159713745117 0.0
0 -0.00294101238250732 1.0
0 0.00725901126861572 2.0
0 -0.0348896980285645 3.0
0 0.0134746432304382 4.0
0 0.00706773996353149 5.0
0 0.00252926349639893 6.0
0 -0.00183933973312378 7.0
30.278486108 39.0315551757812 0.0
0 -0.00075840950012207 1.0
0 0.0177011489868164 2.0
0 0.0114507675170898 3.0
0 0.0057494044303894 4.0
0 0.00559014081954956 5.0
0 0.00110125541687012 6.0
0 7.3552131652832e-05 7.0
40.311093813 37.9903717041016 0.0
0 0.000359892845153809 1.0
0 0.00946557521820068 2.0
0 0.00448530912399292 3.0
0 0.010830819606781 4.0
0 0.00182676315307617 5.0
0 0.00240612030029297 6.0
0 -0.00593709945678711 7.0
35.760290532 38.7118949890137 0.0
0 0.00812315940856934 1.0
0 0.0297279953956604 2.0
0 -0.0139551758766174 3.0
0 -0.014398992061615 4.0
0 -0.020226776599884 5.0
0 0.00332540273666382 6.0
0 -0.00403028726577759 7.0
40.236813436 38.5688896179199 0.0
0 -0.00127357244491577 1.0
0 -0.00444197654724121 2.0
0 0.000873267650604248 3.0
0 0.00886917114257812 4.0
0 0.0087096095085144 5.0
0 0.00108206272125244 6.0
0 0.000600993633270264 7.0
43.948095531 38.5622749328613 0.0
0 0.00688648223876953 1.0
0 0.0145112872123718 2.0
0 -0.00303208827972412 3.0
0 0.0120838284492493 4.0
0 0.0151082277297974 5.0
0 0.000646591186523438 6.0
0 9.0181827545166e-05 7.0
40.202612729 38.9503021240234 0.0
0 0.00446075201034546 1.0
0 0.00278854370117188 2.0
0 -0.00881409645080566 3.0
0 0.00978708267211914 4.0
0 0.0118529200553894 5.0
0 0.00184690952301025 6.0
0 0.00139963626861572 7.0
41.441227091 38.7626190185547 0.0
0 0.00998771190643311 1.0
0 0.014593780040741 2.0
0 0.037828803062439 3.0
0 -0.0178765058517456 4.0
0 -0.0181952714920044 5.0
0 0.00372391939163208 6.0
0 -0.000807821750640869 7.0
40.234798657 38.8709907531738 0.0
0 0.0301759243011475 1.0
0 0.0436609983444214 2.0
0 -0.0280125141143799 3.0
0 -0.0313872694969177 4.0
0 -0.0318161249160767 5.0
0 0.00301194190979004 6.0
0 -0.00238752365112305 7.0
40.268402112 38.6101875305176 0.0
0 0.0056917667388916 1.0
0 0.0238692760467529 2.0
0 0.016819953918457 3.0
0 0.00480717420578003 4.0
0 0.00369954109191895 5.0
0 0.00116813182830811 6.0
0 -0.00442075729370117 7.0
34.447402114 38.8466873168945 0.0
0 0.0301073789596558 1.0
0 0.00375080108642578 2.0
0 0.00299829244613647 3.0
0 -0.0253135561943054 4.0
0 -0.0322710871696472 5.0
0 0.0043519139289856 6.0
0 -0.00207918882369995 7.0
40.416533901 38.6745910644531 0.0
0 0.00224298238754272 1.0
0 -0.000726759433746338 2.0
0 0.0128074884414673 3.0
0 0.0108352303504944 4.0
0 0.0103694200515747 5.0
0 0.000999271869659424 6.0
0 -0.00330287218093872 7.0
40.837884543 38.8132362365723 0.0
0 0.00295329093933105 1.0
0 -0.00834047794342041 2.0
0 0.0268865823745728 3.0
0 0.000196278095245361 4.0
0 -0.005332350730896 5.0
0 0.00294673442840576 6.0
0 -0.00294685363769531 7.0
39.823092444 37.8871421813965 0.0
0 0.000456392765045166 1.0
0 -0.00625967979431152 2.0
0 -0.0234071016311646 3.0
0 -0.00285905599594116 4.0
0 -0.00193220376968384 5.0
0 -0.00192928314208984 6.0
0 -0.00620365142822266 7.0
40.41620498 38.7058410644531 0.0
0 -0.00521928071975708 1.0
0 0.000878453254699707 2.0
0 0.0232111811637878 3.0
0 0.0046045184135437 4.0
0 -0.00023186206817627 5.0
0 0.00172221660614014 6.0
0 -0.004547119140625 7.0
33.388504976 38.8013496398926 0.0
0 -0.00490099191665649 1.0
0 0.00648212432861328 2.0
0 -0.0126960277557373 3.0
0 0.015871524810791 4.0
0 0.0092160701751709 5.0
0 0.00187402963638306 6.0
0 -0.00270891189575195 7.0
42.251691394 38.6184730529785 0.0
0 0.0312210917472839 1.0
0 0.00582593679428101 2.0
0 0.0208268165588379 3.0
0 -0.00345629453659058 4.0
0 -0.00656622648239136 5.0
0 0.00353902578353882 6.0
0 -0.00196647644042969 7.0
32.552772789 38.5892791748047 0.0
0 -0.00613367557525635 1.0
0 0.0275079011917114 2.0
0 -0.0156198143959045 3.0
0 -0.0119390487670898 4.0
0 -0.00980442762374878 5.0
0 0.000374138355255127 6.0
0 -0.00487136840820312 7.0
43.396109959 38.253360748291 0.0
0 0.0250413417816162 1.0
0 0.020060658454895 2.0
0 -0.0217271447181702 3.0
0 0.00134474039077759 4.0
0 -0.00472259521484375 5.0
0 0.00346243381500244 6.0
0 -0.00104355812072754 7.0
31.629368094 38.9830894470215 0.0
0 0.00522613525390625 1.0
0 0.000475287437438965 2.0
0 -0.011349081993103 3.0
0 0.01133793592453 4.0
0 0.0121121406555176 5.0
0 0.00200116634368896 6.0
0 0.00154191255569458 7.0
37.779811495 38.5338363647461 0.0
0 -0.0108740329742432 1.0
0 -0.0369836091995239 2.0
0 0.0149946808815002 3.0
0 -0.00636374950408936 4.0
0 -0.00593376159667969 5.0
0 0.00299978256225586 6.0
0 0.00320684909820557 7.0
34.24652714 38.4251098632812 0.0
0 0.00578612089157104 1.0
0 -0.00152945518493652 2.0
0 0.0102679133415222 3.0
0 0.0122678875923157 4.0
0 0.0114915370941162 5.0
0 0.00135761499404907 6.0
0 0.000555694103240967 7.0
40.003780762 39.0511894226074 0.0
0 -0.0189368724822998 1.0
0 0.0200136303901672 2.0
0 -0.0185563564300537 3.0
0 0.0121176242828369 4.0
0 0.0117418766021729 5.0
0 0.00249522924423218 6.0
0 0.00155913829803467 7.0
31.970353396 38.9871673583984 0.0
0 -0.00729846954345703 1.0
0 0.0101170539855957 2.0
0 -0.0196437239646912 3.0
0 0.0132386088371277 4.0
0 0.0108768343925476 5.0
0 0.00228655338287354 6.0
0 0.0010066032409668 7.0
37.935925517 38.9553871154785 0.0
0 -0.00871318578720093 1.0
0 0.0280109643936157 2.0
0 -0.025648832321167 3.0
0 0.0104709267616272 4.0
0 0.0101121068000793 5.0
0 0.00271910429000854 6.0
0 0.00231039524078369 7.0
37.4568261 38.9331321716309 0.0
0 -0.00507557392120361 1.0
0 -0.000133514404296875 2.0
0 0.00147175788879395 3.0
0 0.00708097219467163 4.0
0 0.00657325983047485 5.0
0 0.00235539674758911 6.0
0 0.0013958215713501 7.0
34.938845635 38.2435569763184 0.0
0 0.00165331363677979 1.0
0 -0.00227940082550049 2.0
0 0.00175893306732178 3.0
0 0.0102068185806274 4.0
0 0.00741416215896606 5.0
0 0.00100028514862061 6.0
0 -0.00547868013381958 7.0
33.644682189 39.0608100891113 0.0
0 0.0215523242950439 1.0
0 0.0741968750953674 2.0
0 0.0108304619789124 3.0
0 -0.00966185331344604 4.0
0 -0.0103352665901184 5.0
0 0.00338512659072876 6.0
0 0.000349283218383789 7.0
43.152629724 38.6673278808594 0.0
0 -0.015493631362915 1.0
0 0.0307546257972717 2.0
0 0.00604945421218872 3.0
0 0.00886774063110352 4.0
0 0.010189414024353 5.0
0 0.00204598903656006 6.0
0 0.00179809331893921 7.0
29.29034141 38.7699356079102 0.0
0 0.0308585166931152 1.0
0 0.0640302300453186 2.0
0 0.0491317510604858 3.0
0 -0.0297149419784546 4.0
0 -0.0343635082244873 5.0
0 0.00418633222579956 6.0
0 -0.00212037563323975 7.0
36.529945045 38.9532089233398 0.0
0 0.0035216212272644 1.0
0 -0.0379787087440491 2.0
0 0.0298681259155273 3.0
0 -0.011357307434082 4.0
0 -0.0170662999153137 5.0
0 0.00557559728622437 6.0
0 0.00102359056472778 7.0
39.490028245 38.2533340454102 0.0
0 0.00501155853271484 1.0
0 0.0158194303512573 2.0
0 -0.0231708884239197 3.0
0 -0.00452101230621338 4.0
0 -0.0123972296714783 5.0
0 0.00276535749435425 6.0
0 -0.00519418716430664 7.0
34.709507968 38.3433647155762 0.0
0 0.00644981861114502 1.0
0 0.0225534439086914 2.0
0 0.0125881433486938 3.0
0 0.00445210933685303 4.0
0 0.00744950771331787 5.0
0 0.00118499994277954 6.0
0 -0.00236207246780396 7.0
45.017599836 38.2805442810059 0.0
0 0.00768566131591797 1.0
0 0.00354486703872681 2.0
0 -0.0179717540740967 3.0
0 -0.0011255145072937 4.0
0 -0.00349807739257812 5.0
0 0.000170648097991943 6.0
0 -0.00597703456878662 7.0
32.940669776 38.5253715515137 0.0
0 -0.00512635707855225 1.0
0 0.0313267707824707 2.0
0 -0.00119996070861816 3.0
0 -0.00267255306243896 4.0
0 -0.00358939170837402 5.0
0 -0.00215828418731689 6.0
0 -0.00541460514068604 7.0
44.072190389 38.780647277832 0.0
0 0.0452917814254761 1.0
0 -0.0119504332542419 2.0
0 0.0499585866928101 3.0
0 -0.0182744860649109 4.0
0 -0.0202071666717529 5.0
0 0.0040593147277832 6.0
0 -0.00122308731079102 7.0
35.422022456 38.5963363647461 0.0
0 1.99079513549805e-05 1.0
0 -0.0265647768974304 2.0
0 0.0021822452545166 3.0
0 -0.0189751386642456 4.0
0 -0.01978600025177 5.0
0 0.000371932983398438 6.0
0 -0.00270688533782959 7.0
36.153594117 39.0233688354492 0.0
0 -0.00477045774459839 1.0
0 0.0245848894119263 2.0
0 -0.0127134323120117 3.0
0 0.0103989243507385 4.0
0 0.0108460187911987 5.0
0 0.00221699476242065 6.0
0 0.00128203630447388 7.0
38.951377458 38.6069030761719 0.0
0 0.00339537858963013 1.0
0 -0.00315690040588379 2.0
0 0.0101865530014038 3.0
0 0.00888776779174805 4.0
0 0.00692009925842285 5.0
0 0.00140982866287231 6.0
0 -0.00287401676177979 7.0
40.644172477 38.4279823303223 0.0
0 -0.00347089767456055 1.0
0 -0.00854629278182983 2.0
0 -0.00977569818496704 3.0
0 0.00184166431427002 4.0
0 -0.00165468454360962 5.0
0 -0.00119912624359131 6.0
0 -0.00389707088470459 7.0
35.600531382 38.9765777587891 0.0
0 0.000101447105407715 1.0
0 -0.00169456005096436 2.0
0 -0.0105034112930298 3.0
0 0.0159684419631958 4.0
0 0.0144085288047791 5.0
0 0.00176870822906494 6.0
0 -0.000409126281738281 7.0
41.682033599 38.8813819885254 0.0
0 0.0172576904296875 1.0
0 0.0325183868408203 2.0
0 -0.0265422463417053 3.0
0 0.0114198923110962 4.0
0 0.00883734226226807 5.0
0 0.00202566385269165 6.0
0 -0.00066763162612915 7.0
39.697348938 38.9553298950195 0.0
0 0.00576400756835938 1.0
0 0.00627702474594116 2.0
0 -0.0105857849121094 3.0
0 0.0120393037796021 4.0
0 0.00906950235366821 5.0
0 0.00216853618621826 6.0
0 0.00109446048736572 7.0
40.978668493 37.8965797424316 0.0
0 -0.00433230400085449 1.0
0 0.00928252935409546 2.0
0 -0.0110325813293457 3.0
0 -0.010481059551239 4.0
0 -0.00912737846374512 5.0
0 0.000316202640533447 6.0
0 -0.00422096252441406 7.0
46.524483465 38.4956130981445 0.0
0 0.0072096586227417 1.0
0 -0.00128740072250366 2.0
0 0.00465559959411621 3.0
0 0.0165053009986877 4.0
0 0.00841742753982544 5.0
0 0.00209641456604004 6.0
0 -0.0013575553894043 7.0
39.683969876 38.3914260864258 0.0
0 0.00807690620422363 1.0
0 -0.0101870894432068 2.0
0 0.00545310974121094 3.0
0 0.014818549156189 4.0
0 0.0115963220596313 5.0
0 0.00100439786911011 6.0
0 -0.00206196308135986 7.0
46.294692767 39.0759162902832 0.0
0 0.0209541916847229 1.0
0 -0.00489115715026855 2.0
0 0.0438747406005859 3.0
0 -0.0146594643592834 4.0
0 -0.01612788438797 5.0
0 0.00321435928344727 6.0
0 -0.000924825668334961 7.0
44.785541844 38.8168334960938 0.0
0 0.00351965427398682 1.0
0 0.00326263904571533 2.0
0 0.0207355618476868 3.0
0 0.00808155536651611 4.0
0 0.00271213054656982 5.0
0 0.00201249122619629 6.0
0 -0.0048290491104126 7.0
33.676333572 38.2910499572754 0.0
0 0.00158512592315674 1.0
0 -0.00176751613616943 2.0
0 -0.00117731094360352 3.0
0 0.0104479193687439 4.0
0 0.0066227912902832 5.0
0 0.0014575719833374 6.0
0 -0.00686109066009521 7.0
35.625618922 38.0471572875977 0.0
0 0.00239014625549316 1.0
0 -0.0126388072967529 2.0
0 -0.0120697021484375 3.0
0 0.00128597021102905 4.0
0 0.00199592113494873 5.0
0 -0.000860452651977539 6.0
0 -0.00589412450790405 7.0
40.226157029 38.6796798706055 0.0
0 0.00153261423110962 1.0
0 0.00714719295501709 2.0
0 0.00271427631378174 3.0
0 0.0039246678352356 4.0
0 0.00176608562469482 5.0
0 0.0020751953125 6.0
0 -0.00623393058776855 7.0
44.003195919 38.717472076416 0.0
0 0.00164341926574707 1.0
0 0.00996559858322144 2.0
0 -0.00653207302093506 3.0
0 0.00345414876937866 4.0
0 0.00210237503051758 5.0
0 0.00132018327713013 6.0
0 -0.00519376993179321 7.0
33.88878315 38.4337196350098 0.0
0 0.00824373960494995 1.0
0 -0.000991284847259521 2.0
0 0.0105270147323608 3.0
0 0.0159042477607727 4.0
0 0.0113751888275146 5.0
0 0.00127136707305908 6.0
0 -0.00309556722640991 7.0
42.125521158 38.4778442382812 0.0
0 0.00595742464065552 1.0
0 0.007171630859375 2.0
0 0.0155715346336365 3.0
0 0.00982785224914551 4.0
0 0.00966256856918335 5.0
0 0.0015488862991333 6.0
0 0.00158339738845825 7.0
37.55067892 39.0216331481934 0.0
0 0.000815391540527344 1.0
0 0.0180549621582031 2.0
0 0.0161591768264771 3.0
0 0.00454628467559814 4.0
0 0.00465989112854004 5.0
0 0.00137084722518921 6.0
0 -0.000109493732452393 7.0
37.39948618 38.7834625244141 0.0
0 0.000694096088409424 1.0
0 0.017846941947937 2.0
0 -0.00839388370513916 3.0
0 0.0181208252906799 4.0
0 0.0139331221580505 5.0
0 0.00215917825698853 6.0
0 0.000113487243652344 7.0
34.341868993 38.6611785888672 0.0
0 -0.0150122046470642 1.0
0 -0.0378657579421997 2.0
0 0.0207475423812866 3.0
0 -0.000995099544525146 4.0
0 -0.00474607944488525 5.0
0 0.00366520881652832 6.0
0 0.00226479768753052 7.0
37.425355461 38.7103309631348 0.0
0 0.00776708126068115 1.0
0 0.00792419910430908 2.0
0 -0.00677996873855591 3.0
0 0.0113720297813416 4.0
0 0.00985175371170044 5.0
0 0.000823676586151123 6.0
0 -1.20997428894043e-05 7.0
37.288491912 38.4678955078125 0.0
0 0.00515389442443848 1.0
0 0.00180560350418091 2.0
0 0.0212438106536865 3.0
0 0.0104888081550598 4.0
0 0.00904351472854614 5.0
0 0.000123202800750732 6.0
0 -0.0016171932220459 7.0
39.20347441 38.805908203125 0.0
0 -0.00724637508392334 1.0
0 0.00258266925811768 2.0
0 0.0161644816398621 3.0
0 0.00277906656265259 4.0
0 0.000788569450378418 5.0
0 0.0032578706741333 6.0
0 0.000290453433990479 7.0
36.137884824 38.5297698974609 0.0
0 0.00461000204086304 1.0
0 -0.0104106664657593 2.0
0 -0.002341628074646 3.0
0 0.00298571586608887 4.0
0 0.00196605920791626 5.0
0 0.000843822956085205 6.0
0 -0.00134950876235962 7.0
41.655417288 38.3561172485352 0.0
0 0.00745469331741333 1.0
0 -0.00175750255584717 2.0
0 0.0158591866493225 3.0
0 0.0111903548240662 4.0
0 0.00902730226516724 5.0
0 0.000314235687255859 6.0
0 -0.00557076930999756 7.0
39.894978379 38.5850296020508 0.0
0 0.00680673122406006 1.0
0 0.00699466466903687 2.0
0 -0.0027390718460083 3.0
0 0.0178101062774658 4.0
0 0.00734114646911621 5.0
0 0.0022856593132019 6.0
0 -0.00147330760955811 7.0
39.390677385 38.8203659057617 0.0
0 0.0151079297065735 1.0
0 0.0463476181030273 2.0
0 -0.0157824158668518 3.0
0 -0.0195214152336121 4.0
0 -0.0244238376617432 5.0
0 0.00370627641677856 6.0
0 -0.0034632682800293 7.0
38.248327703 38.6163101196289 0.0
0 0.00582766532897949 1.0
0 -0.00413334369659424 2.0
0 0.00381135940551758 3.0
0 0.0143280029296875 4.0
0 0.0112537741661072 5.0
0 0.0016014575958252 6.0
0 -0.000128448009490967 7.0
38.538636426 38.5361137390137 0.0
0 0.00641006231307983 1.0
0 -0.00027698278427124 2.0
0 0.0049251914024353 3.0
0 0.0107793807983398 4.0
0 0.0084531307220459 5.0
0 0.00101703405380249 6.0
0 -0.000242471694946289 7.0
30.093865158 39.2019309997559 0.0
0 0.0353890657424927 1.0
0 0.00355738401412964 2.0
0 0.04957115650177 3.0
0 -0.0436264872550964 4.0
0 -0.0471203327178955 5.0
0 0.00451117753982544 6.0
0 -0.000642895698547363 7.0
36.059945316 38.318187713623 0.0
0 0.00228762626647949 1.0
0 0.0050504207611084 2.0
0 -0.00272685289382935 3.0
0 0.0147392749786377 4.0
0 0.0124824047088623 5.0
0 0.00258147716522217 6.0
0 0.00151604413986206 7.0
41.382686141 38.4142570495605 0.0
0 -0.00156998634338379 1.0
0 0.0204638242721558 2.0
0 -0.00764346122741699 3.0
0 -0.00123339891433716 4.0
0 -0.0007476806640625 5.0
0 -0.00071108341217041 6.0
0 -0.00559014081954956 7.0
37.374314239 38.6772499084473 0.0
0 -0.00320535898208618 1.0
0 0.00326782464981079 2.0
0 -0.00733256340026855 3.0
0 0.0150027275085449 4.0
0 0.0146105885505676 5.0
0 0.00185835361480713 6.0
0 0.000728189945220947 7.0
37.689487865 38.6625061035156 0.0
0 -0.00688505172729492 1.0
0 0.0156492590904236 2.0
0 0.0139175653457642 3.0
0 0.00787997245788574 4.0
0 0.00172066688537598 5.0
0 0.00320947170257568 6.0
0 0.000371813774108887 7.0
36.541540581 38.4495582580566 0.0
0 0.00823324918746948 1.0
0 -0.00128722190856934 2.0
0 0.0112419724464417 3.0
0 0.0117476582527161 4.0
0 0.0131450295448303 5.0
0 0.000376760959625244 6.0
0 -0.00190281867980957 7.0
36.791141656 38.3827018737793 0.0
0 0.00603717565536499 1.0
0 0.00242966413497925 2.0
0 0.00899404287338257 3.0
0 0.0119273662567139 4.0
0 0.011038064956665 5.0
0 0.00139141082763672 6.0
0 0.000935733318328857 7.0
39.891605825 38.3263244628906 0.0
0 -0.000929117202758789 1.0
0 0.0420316457748413 2.0
0 -0.0310762524604797 3.0
0 -0.00990122556686401 4.0
0 -0.00977462530136108 5.0
0 -0.0024992823600769 6.0
0 -0.00472760200500488 7.0
34.771445328 38.6938209533691 0.0
0 0.00257289409637451 1.0
0 0.0160443186759949 2.0
0 -0.00607508420944214 3.0
0 0.015845775604248 4.0
0 0.0155516862869263 5.0
0 0.00151389837265015 6.0
0 -0.000127732753753662 7.0
44.638557165 38.5900001525879 0.0
0 -0.00749975442886353 1.0
0 0.0173272490501404 2.0
0 0.0101034641265869 3.0
0 0.0125020146369934 4.0
0 0.00756698846817017 5.0
0 0.00205510854721069 6.0
0 -0.00224572420120239 7.0
36.096180936 38.1678276062012 0.0
0 0.00536233186721802 1.0
0 0.00875014066696167 2.0
0 0.0115242004394531 3.0
0 0.0147314667701721 4.0
0 0.00792396068572998 5.0
0 0.00153219699859619 6.0
0 -0.00242865085601807 7.0
45.744029538 38.1977348327637 0.0
0 0.00154680013656616 1.0
0 -0.00525921583175659 2.0
0 0.0133904218673706 3.0
0 0.00973719358444214 4.0
0 0.0101900100708008 5.0
0 0.00129318237304688 6.0
0 0.00124156475067139 7.0
36.287608705 38.4067802429199 0.0
0 -0.00326824188232422 1.0
0 0.00791627168655396 2.0
0 0.00300085544586182 3.0
0 0.0167179703712463 4.0
0 0.012957751750946 5.0
0 0.00116145610809326 6.0
0 -0.00291407108306885 7.0
42.801830865 38.5026664733887 0.0
0 0.00598233938217163 1.0
0 0.00537014007568359 2.0
0 0.00846278667449951 3.0
0 0.00354379415512085 4.0
0 0.00350403785705566 5.0
0 0.00145411491394043 6.0
0 -0.0010840892791748 7.0
47.558612578 38.3811569213867 0.0
0 -0.00544494390487671 1.0
0 0.0251234173774719 2.0
0 0.0158400535583496 3.0
0 0.00700610876083374 4.0
0 0.00496816635131836 5.0
0 0.000510692596435547 6.0
0 -0.00318229198455811 7.0
47.058562462 38.4057235717773 0.0
0 0.00507622957229614 1.0
0 0.0046621561050415 2.0
0 -0.00457650423049927 3.0
0 0.0249253511428833 4.0
0 0.0235775709152222 5.0
0 0.00243175029754639 6.0
0 0.000473499298095703 7.0
43.087048946 38.619270324707 0.0
0 -0.0106337070465088 1.0
0 -0.0265575647354126 2.0
0 0.015188455581665 3.0
0 -0.00168472528457642 4.0
0 -0.00724762678146362 5.0
0 0.00427258014678955 6.0
0 0.0025297999382019 7.0
33.170190568 38.9567565917969 0.0
0 0.00517016649246216 1.0
0 0.00459200143814087 2.0
0 0.0193129181861877 3.0
0 0.00718289613723755 4.0
0 0.00489509105682373 5.0
0 0.00101983547210693 6.0
0 -0.00283050537109375 7.0
34.957431523 38.8416709899902 0.0
0 0.000963389873504639 1.0
0 0.00420975685119629 2.0
0 0.0198730826377869 3.0
0 -0.00562077760696411 4.0
0 -0.00709086656570435 5.0
0 0.00256907939910889 6.0
0 -0.00235521793365479 7.0
36.320328519 38.8440856933594 0.0
0 -0.00225573778152466 1.0
0 -0.00953090190887451 2.0
0 0.00566196441650391 3.0
0 0.00649935007095337 4.0
0 0.00460940599441528 5.0
0 0.0024418830871582 6.0
0 0.000105619430541992 7.0
44.085721677 38.4182586669922 0.0
0 0.000100314617156982 1.0
0 0.0046502947807312 2.0
0 0.000212252140045166 3.0
0 0.00853204727172852 4.0
0 0.00654923915863037 5.0
0 0.000967621803283691 6.0
0 -0.00571078062057495 7.0
45.211814568 38.3988380432129 0.0
0 0.000818192958831787 1.0
0 0.0170605778694153 2.0
0 -0.0164785385131836 3.0
0 0.0199121236801147 4.0
0 0.0166866183280945 5.0
0 0.00196439027786255 6.0
0 -0.000163257122039795 7.0
38.71721827 38.8017578125 0.0
0 -0.00146806240081787 1.0
0 0.00902062654495239 2.0
0 -0.0155308246612549 3.0
0 0.00887376070022583 4.0
0 0.0110338926315308 5.0
0 0.00151538848876953 6.0
0 0.00038301944732666 7.0
42.190454953 38.4542961120605 0.0
0 -0.00158572196960449 1.0
0 -0.00745916366577148 2.0
0 -0.0216897130012512 3.0
0 -0.00159525871276855 4.0
0 -0.00380432605743408 5.0
0 9.28044319152832e-05 6.0
0 -0.00571072101593018 7.0
40.616038376 38.7430953979492 0.0
0 -0.00362867116928101 1.0
0 -0.00363141298294067 2.0
0 -0.0144767761230469 3.0
0 0.0116175413131714 4.0
0 0.0105224847793579 5.0
0 0.00140535831451416 6.0
0 0.000889956951141357 7.0
31.887764532 38.5140266418457 0.0
0 -0.00434798002243042 1.0
0 0.00756442546844482 2.0
0 -0.0390388369560242 3.0
0 -0.0211315751075745 4.0
0 -0.0261675715446472 5.0
0 0.00929063558578491 6.0
0 -0.00467920303344727 7.0
39.764867486 39.1465721130371 0.0
0 0.038346529006958 1.0
0 0.0927492082118988 2.0
0 0.0490598678588867 3.0
0 -0.041797399520874 4.0
0 -0.0453233122825623 5.0
0 0.00138425827026367 6.0
0 -0.000771045684814453 7.0
40.805009287 38.1030654907227 0.0
0 -0.0296259522438049 1.0
0 -0.00586938858032227 2.0
0 0.00417482852935791 3.0
0 -0.00110441446304321 4.0
0 -0.00454360246658325 5.0
0 0.00295650959014893 6.0
0 -0.000849902629852295 7.0
39.795600818 38.255054473877 0.0
0 0.00796306133270264 1.0
0 -0.00954651832580566 2.0
0 0.0155543088912964 3.0
0 0.0102896094322205 4.0
0 0.00733578205108643 5.0
0 0.00113248825073242 6.0
0 -0.00258749723434448 7.0
38.685439721 38.1099815368652 0.0
0 -0.0193047523498535 1.0
0 0.026334285736084 2.0
0 0.014123797416687 3.0
0 0.00438123941421509 4.0
0 0.00159591436386108 5.0
0 0.0030590295791626 6.0
0 0.00219011306762695 7.0
42.152656556 38.6447944641113 0.0
0 0.00958836078643799 1.0
0 0.00431197881698608 2.0
0 0.00279456377029419 3.0
0 0.0114231109619141 4.0
0 0.014231264591217 5.0
0 0.00242865085601807 6.0
0 0.00234651565551758 7.0
37.458343334 39.1002197265625 0.0
0 0.118244767189026 1.0
0 0.0132309198379517 2.0
0 0.179237484931946 3.0
0 -0.0956413745880127 4.0
0 -0.109352290630341 5.0
0 0.0108833312988281 6.0
0 0.0037427544593811 7.0
44.489633946 37.3663024902344 0.0
0 -0.000156283378601074 1.0
0 0.0174140930175781 2.0
0 -0.0660887956619263 3.0
0 -0.0274395942687988 4.0
0 -0.0323089957237244 5.0
0 0.0018419623374939 6.0
0 -0.00307917594909668 7.0
41.506640114 38.712272644043 0.0
0 0.00974112749099731 1.0
0 -0.0041663646697998 2.0
0 0.00747394561767578 3.0
0 0.0130130648612976 4.0
0 0.0114015340805054 5.0
0 0.00225138664245605 6.0
0 0.0011325478553772 7.0
33.262736569 38.4637565612793 0.0
0 -0.00413501262664795 1.0
0 -0.0077093243598938 2.0
0 0.00772899389266968 3.0
0 -0.0170835852622986 4.0
0 -0.0186598896980286 5.0
0 9.43541526794434e-05 6.0
0 -0.00329750776290894 7.0
47.069969764 39.1127662658691 0.0
0 0.0034937858581543 1.0
0 0.0224937796592712 2.0
0 0.0202521085739136 3.0
0 0.00589430332183838 4.0
0 0.00132709741592407 5.0
0 0.00247538089752197 6.0
0 -0.00191307067871094 7.0
43.288539967 38.7025947570801 0.0
0 0.0100835561752319 1.0
0 0.00247675180435181 2.0
0 0.0297904014587402 3.0
0 -0.00783008337020874 4.0
0 -0.00629055500030518 5.0
0 0.00282490253448486 6.0
0 0.00166058540344238 7.0
41.479216264 38.3655395507812 0.0
0 -0.0076107382774353 1.0
0 0.0249927639961243 2.0
0 0.0158962607383728 3.0
0 0.00533664226531982 4.0
0 0.00829720497131348 5.0
0 0.00181961059570312 6.0
0 0.00211477279663086 7.0
36.99162076 38.5888214111328 0.0
0 0.00175654888153076 1.0
0 0.023786187171936 2.0
0 0.0185081958770752 3.0
0 0.010552704334259 4.0
0 0.00637632608413696 5.0
0 0.00184732675552368 6.0
0 -0.0059053897857666 7.0
40.435886771 38.986686706543 0.0
0 -0.00245356559753418 1.0
0 0.0295779705047607 2.0
0 -0.0167503356933594 3.0
0 0.0110731720924377 4.0
0 0.00603735446929932 5.0
0 0.00212985277175903 6.0
0 -0.000238656997680664 7.0
37.140737383 38.4575042724609 0.0
0 0.00847417116165161 1.0
0 0.00707650184631348 2.0
0 0.00589668750762939 3.0
0 0.00967586040496826 4.0
0 0.0113044381141663 5.0
0 0.000883042812347412 6.0
0 0.00042259693145752 7.0
39.233907158 38.9101257324219 0.0
0 0.0550709962844849 1.0
0 0.148562252521515 2.0
0 0.0203211903572083 3.0
0 -0.0188177227973938 4.0
0 -0.0227159857749939 5.0
0 0.00519031286239624 6.0
0 0.000806093215942383 7.0
42.028602534 38.4750175476074 0.0
0 -0.00834649801254272 1.0
0 0.161355286836624 2.0
0 -0.0410969257354736 3.0
0 -0.0144053101539612 4.0
0 -0.0182055234909058 5.0
0 0.000284016132354736 6.0
0 -0.00558304786682129 7.0
39.783103652 39.0378684997559 0.0
0 0.0226157903671265 1.0
0 -0.0173406600952148 2.0
0 0.0640493631362915 3.0
0 -0.022208034992218 4.0
0 -0.0254800319671631 5.0
0 0.00419634580612183 6.0
0 -0.00270748138427734 7.0
39.425516455 38.1142044067383 0.0
0 0.003761887550354 1.0
0 0.00284242630004883 2.0
0 0.00697588920593262 3.0
0 0.00978696346282959 4.0
0 0.00723946094512939 5.0
0 0.00138962268829346 6.0
0 -0.00471216440200806 7.0
40.002559542 39.1183433532715 0.0
0 0.0111833214759827 1.0
0 0.0301645398139954 2.0
0 0.0249743461608887 3.0
0 0.00826179981231689 4.0
0 0.00183463096618652 5.0
0 0.00230365991592407 6.0
0 -0.00201362371444702 7.0
38.322665212 38.248119354248 0.0
0 0.00894296169281006 1.0
0 0.00682508945465088 2.0
0 -0.0240972638130188 3.0
0 -0.00541716814041138 4.0
0 -0.0107141733169556 5.0
0 0.0022168755531311 6.0
0 -0.0030866265296936 7.0
40.568600192 39.0411567687988 0.0
0 0.0145177841186523 1.0
0 0.0150798559188843 2.0
0 0.00254154205322266 3.0
0 -0.01081383228302 4.0
0 -0.014443039894104 5.0
0 0.00246632099151611 6.0
0 -0.00231659412384033 7.0
40.337689123 38.5351600646973 0.0
0 0.00957757234573364 1.0
0 0.0115842819213867 2.0
0 0.0130894184112549 3.0
0 0.00751912593841553 4.0
0 0.00370407104492188 5.0
0 0.00157725811004639 6.0
0 -0.00272804498672485 7.0
45.726007711 39.016544342041 0.0
0 -0.00627726316452026 1.0
0 0.0215297341346741 2.0
0 -0.0119922757148743 3.0
0 0.010688841342926 4.0
0 0.00892925262451172 5.0
0 0.0013124942779541 6.0
0 0.000753998756408691 7.0
36.675787374 38.2430000305176 0.0
0 -0.00158858299255371 1.0
0 0.0261760354042053 2.0
0 -0.00662976503372192 3.0
0 -0.0059894323348999 4.0
0 -0.00984197854995728 5.0
0 0.00290602445602417 6.0
0 -0.00482195615768433 7.0
32.235093382 38.4787483215332 0.0
0 0.00654047727584839 1.0
0 0.00283539295196533 2.0
0 0.0180583000183105 3.0
0 0.0143955945968628 4.0
0 0.00843143463134766 5.0
0 0.00196719169616699 6.0
0 -0.000648856163024902 7.0
39.438537732 38.3393020629883 0.0
0 0.00482273101806641 1.0
0 -0.0100158452987671 2.0
0 0.00550246238708496 3.0
0 0.00398647785186768 4.0
0 0.00508660078048706 5.0
0 0.000476419925689697 6.0
0 -0.000379323959350586 7.0
44.304614406 38.6942520141602 0.0
0 0.00724595785140991 1.0
0 0.00359833240509033 2.0
0 -0.00244873762130737 3.0
0 0.0168184041976929 4.0
0 0.018384575843811 5.0
0 0.00267505645751953 6.0
0 0.00242710113525391 7.0
32.373304325 38.7011947631836 0.0
0 -0.00139033794403076 1.0
0 0.0330175161361694 2.0
0 0.0167226791381836 3.0
0 -0.00249332189559937 4.0
0 -0.00695604085922241 5.0
0 0.00298500061035156 6.0
0 -0.00275629758834839 7.0
35.769379856 38.7898292541504 0.0
0 0.00197267532348633 1.0
0 0.029208242893219 2.0
0 0.0212281942367554 3.0
0 0.00471234321594238 4.0
0 0.00653988122940063 5.0
0 0.000956416130065918 6.0
0 -0.000651717185974121 7.0
38.348316661 38.5863151550293 0.0
0 -0.00654613971710205 1.0
0 0.0357071757316589 2.0
0 0.00473201274871826 3.0
0 0.00128424167633057 4.0
0 -0.00100201368331909 5.0
0 0.00222885608673096 6.0
0 -0.00584298372268677 7.0
30.534523433 39.0130577087402 0.0
0 0.00583195686340332 1.0
0 0.0315658450126648 2.0
0 0.0169404745101929 3.0
0 0.00761932134628296 4.0
0 0.0039747953414917 5.0
0 0.00238054990768433 6.0
0 -0.00270819664001465 7.0
41.665325899 38.7758674621582 0.0
0 0.000107109546661377 1.0
0 0.0110929012298584 2.0
0 0.0183548331260681 3.0
0 0.00898802280426025 4.0
0 0.00583487749099731 5.0
0 0.00194180011749268 6.0
0 -0.00388616323471069 7.0
39.125756769 38.7648773193359 0.0
0 -0.0030866265296936 1.0
0 0.00142174959182739 2.0
0 0.0009346604347229 3.0
0 0.00942152738571167 4.0
0 0.00966179370880127 5.0
0 0.00102788209915161 6.0
0 -0.00109249353408813 7.0
46.922469454 38.955696105957 0.0
0 -0.00348621606826782 1.0
0 0.000359296798706055 2.0
0 0.0120048522949219 3.0
0 0.00776892900466919 4.0
0 0.00494801998138428 5.0
0 0.00185239315032959 6.0
0 0.000420093536376953 7.0
38.082156965 38.3118515014648 0.0
0 0.0151529312133789 1.0
0 -0.0102242231369019 2.0
0 0.0122508406639099 3.0
0 -0.0131518244743347 4.0
0 -0.0139512419700623 5.0
0 0.00426769256591797 6.0
0 0.00295782089233398 7.0
35.281773346 38.6570739746094 0.0
0 -0.00272595882415771 1.0
0 0.0289946794509888 2.0
0 0.00509798526763916 3.0
0 -0.00448018312454224 4.0
0 -0.00222653150558472 5.0
0 0.000601649284362793 6.0
0 -0.00430136919021606 7.0
39.676820236 38.989990234375 0.0
0 0.00472760200500488 1.0
0 0.0294286012649536 2.0
0 0.0172818303108215 3.0
0 0.00415349006652832 4.0
0 0.00443726778030396 5.0
0 0.00188541412353516 6.0
0 -0.000945448875427246 7.0
41.049027324 38.6982574462891 0.0
0 0.00343352556228638 1.0
0 0.0113016366958618 2.0
0 -0.00696408748626709 3.0
0 0.0125917196273804 4.0
0 0.00998294353485107 5.0
0 0.00137799978256226 6.0
0 0.000242471694946289 7.0
41.402616693 39.0029830932617 0.0
0 0.00863611698150635 1.0
0 0.087821900844574 2.0
0 0.0603731870651245 3.0
0 -0.0356480479240417 4.0
0 -0.0399312376976013 5.0
0 0.00590997934341431 6.0
0 -9.55462455749512e-05 7.0
30.3721478 38.6933746337891 0.0
0 -0.00103241205215454 1.0
0 0.00393533706665039 2.0
0 -0.0112425088882446 3.0
0 -0.00764316320419312 4.0
0 -0.0103350281715393 5.0
0 0.000120401382446289 6.0
0 -0.00539594888687134 7.0
41.111385117 38.5219116210938 0.0
0 0.00852882862091064 1.0
0 -0.00877469778060913 2.0
0 0.0205832123756409 3.0
0 0.0103094577789307 4.0
0 0.0121785402297974 5.0
0 0.000267446041107178 6.0
0 -0.000113844871520996 7.0
36.921893091 38.6074638366699 0.0
0 0.00677359104156494 1.0
0 -0.00750589370727539 2.0
0 0.00729429721832275 3.0
0 0.0145912766456604 4.0
0 0.0131081342697144 5.0
0 0.000901758670806885 6.0
0 -0.000841677188873291 7.0
41.531234868 38.1064682006836 0.0
0 0.0153313875198364 1.0
0 0.0151869654655457 2.0
0 -0.00373983383178711 3.0
0 -0.0103966593742371 4.0
0 -0.0106770396232605 5.0
0 0.00317627191543579 6.0
0 0.00247424840927124 7.0
36.809020967 38.4955902099609 0.0
0 0.00135868787765503 1.0
0 0.00796598196029663 2.0
0 -0.00358861684799194 3.0
0 -0.00392121076583862 4.0
0 -0.00673782825469971 5.0
0 0.00253629684448242 6.0
0 -0.00510793924331665 7.0
30.318323192 38.832633972168 0.0
0 -0.00723063945770264 1.0
0 0.0219334959983826 2.0
0 0.010265588760376 3.0
0 0.00689631700515747 4.0
0 0.00381928682327271 5.0
0 0.00178146362304688 6.0
0 -0.00287723541259766 7.0
37.371827278 39.0299491882324 0.0
0 0.0118033885955811 1.0
0 0.0467261672019958 2.0
0 0.0143779516220093 3.0
0 -0.0137546062469482 4.0
0 -0.0207595229148865 5.0
0 0.00627964735031128 6.0
0 0.00312989950180054 7.0
37.183742215 38.1556701660156 0.0
0 0.00127106904983521 1.0
0 0.0066341757774353 2.0
0 -0.0147396326065063 3.0
0 0.00577563047409058 4.0
0 0.0029301643371582 5.0
0 -0.00182098150253296 6.0
0 -0.00620943307876587 7.0
34.629310928 38.365364074707 0.0
0 0.00610154867172241 1.0
0 0.00508695840835571 2.0
0 0.0198239088058472 3.0
0 0.013192892074585 4.0
0 0.00678855180740356 5.0
0 0.00147479772567749 6.0
0 -0.00184899568557739 7.0
42.830967498 38.9176712036133 0.0
0 -0.000261783599853516 1.0
0 0.0146018862724304 2.0
0 -0.011113166809082 3.0
0 0.0185452699661255 4.0
0 0.00816124677658081 5.0
0 0.00284934043884277 6.0
0 -8.62479209899902e-05 7.0
34.694327372 38.3558731079102 0.0
0 0.00509011745452881 1.0
0 0.000261187553405762 2.0
0 0.0105280876159668 3.0
0 0.0111966133117676 4.0
0 0.0079917311668396 5.0
0 0.00138604640960693 6.0
0 -0.00158607959747314 7.0
37.611435874 39.1154098510742 0.0
0 -0.00671660900115967 1.0
0 0.0181795358657837 2.0
0 -0.000290572643280029 3.0
0 0.0106480121612549 4.0
0 0.00586879253387451 5.0
0 0.00247448682785034 6.0
0 0.000689864158630371 7.0
45.529166488 38.6235542297363 0.0
0 -0.00425219535827637 1.0
0 0.0054473876953125 2.0
0 0.0149739980697632 3.0
0 0.00727999210357666 4.0
0 0.00684750080108643 5.0
0 0.00113421678543091 6.0
0 -0.00142967700958252 7.0
40.26689599 38.7462539672852 0.0
0 -0.00593769550323486 1.0
0 -0.00408351421356201 2.0
0 0.00365918874740601 3.0
0 0.00439119338989258 4.0
0 0.00260257720947266 5.0
0 0.00294429063796997 6.0
0 0.00205498933792114 7.0
39.577958661 38.4459991455078 0.0
0 0.00205034017562866 1.0
0 0.00105744600296021 2.0
0 -0.00728344917297363 3.0
0 0.017594575881958 4.0
0 0.0188379287719727 5.0
0 0.00254732370376587 6.0
0 0.00236052274703979 7.0
34.564585703 38.8292350769043 0.0
0 -0.0269166231155396 1.0
0 0.136175036430359 2.0
0 0.053674578666687 3.0
0 -0.047590434551239 4.0
0 -0.0486240386962891 5.0
0 0.00632476806640625 6.0
0 0.00384056568145752 7.0
34.747667205 38.7812385559082 0.0
0 0.0150026679039001 1.0
0 0.00320255756378174 2.0
0 -0.0252484083175659 3.0
0 -0.0210506319999695 4.0
0 -0.0219604969024658 5.0
0 0.00306332111358643 6.0
0 -0.00150471925735474 7.0
44.886437438 38.5786972045898 0.0
0 -0.00230449438095093 1.0
0 0.00125682353973389 2.0
0 0.00380456447601318 3.0
0 0.0123153924942017 4.0
0 0.0131455063819885 5.0
0 0.000817179679870605 6.0
0 -6.07967376708984e-05 7.0
34.458606901 38.426197052002 0.0
0 0.0095527172088623 1.0
0 -0.0173845887184143 2.0
0 0.0205413103103638 3.0
0 0.013698399066925 4.0
0 0.00839120149612427 5.0
0 0.00135630369186401 6.0
0 -0.00317841768264771 7.0
44.985938301 38.635196685791 0.0
0 0.0398619174957275 1.0
0 0.0797373652458191 2.0
0 -0.0432404279708862 3.0
0 -0.0900113582611084 4.0
0 -0.0522611737251282 5.0
0 0.00559622049331665 6.0
0 0.00259339809417725 7.0
36.561528733 38.4419441223145 0.0
0 -0.00396859645843506 1.0
0 -0.0254343748092651 2.0
0 -0.0119385123252869 3.0
0 -0.0315827131271362 4.0
0 -0.0329383611679077 5.0
0 0.00273001194000244 6.0
0 -0.00205951929092407 7.0
35.585636345 38.4601249694824 0.0
0 -0.00341349840164185 1.0
0 0.0154292583465576 2.0
0 0.0149226188659668 3.0
0 0.00507611036300659 4.0
0 0.00384801626205444 5.0
0 0.000981569290161133 6.0
0 -0.0022132396697998 7.0
34.748633645 38.3581428527832 0.0
0 0.0046808123588562 1.0
0 0.00470662117004395 2.0
0 -0.0016711950302124 3.0
0 0.0164695382118225 4.0
0 0.0109076499938965 5.0
0 0.0013577938079834 6.0
0 -0.00368696451187134 7.0
33.997517461 38.4255676269531 0.0
0 -0.00385230779647827 1.0
0 0.0130151510238647 2.0
0 -0.00214368104934692 3.0
0 0.0129605531692505 4.0
0 0.0127876400947571 5.0
0 0.00150924921035767 6.0
0 0.000913619995117188 7.0
31.337837162 38.988941192627 0.0
0 0.00888556241989136 1.0
0 0.0149818062782288 2.0
0 0.0195881128311157 3.0
0 0.00459617376327515 4.0
0 0.00473707914352417 5.0
0 0.00146538019180298 6.0
0 0.000113725662231445 7.0
43.914614211 38.5083885192871 0.0
0 0.0040326714515686 1.0
0 -0.014149010181427 2.0
0 0.0157884955406189 3.0
0 0.0115993022918701 4.0
0 0.00837445259094238 5.0
0 0.00118738412857056 6.0
0 -0.000306010246276855 7.0
45.186574618 38.4571304321289 0.0
0 0.00369459390640259 1.0
0 -0.00190377235412598 2.0
0 0.00506937503814697 3.0
0 0.00619816780090332 4.0
0 0.00701636075973511 5.0
0 0.000674545764923096 6.0
0 -0.00296097993850708 7.0
35.268441876 38.6745338439941 0.0
0 0.0259495973587036 1.0
0 0.00516915321350098 2.0
0 0.0271381139755249 3.0
0 -0.0043373703956604 4.0
0 -0.00707775354385376 5.0
0 0.00333774089813232 6.0
0 -0.000590026378631592 7.0
32.333098982 38.339469909668 0.0
0 0.00736904144287109 1.0
0 -0.012636661529541 2.0
0 -0.00687289237976074 3.0
0 0.0058826208114624 4.0
0 0.000943958759307861 5.0
0 0.000460386276245117 6.0
0 -0.00598776340484619 7.0
38.894207338 38.5355949401855 0.0
0 -0.0060042142868042 1.0
0 0.0341095924377441 2.0
0 0.00643140077590942 3.0
0 -0.00107800960540771 4.0
0 -0.00261771678924561 5.0
0 -0.000578522682189941 6.0
0 -0.00298482179641724 7.0
32.941924742 38.3487243652344 0.0
0 0.01023268699646 1.0
0 0.0263230800628662 2.0
0 -0.056096076965332 3.0
0 -0.0248920917510986 4.0
0 -0.0269808173179626 5.0
0 0.00285071134567261 6.0
0 0.00860178470611572 7.0
40.2164929 37.7063751220703 0.0
0 -0.00206410884857178 1.0
0 -0.0241124033927917 2.0
0 -0.0163887739181519 3.0
0 -0.00840198993682861 4.0
0 -0.0127372145652771 5.0
0 0.00169253349304199 6.0
0 -0.0047113299369812 7.0
40.810463567 39.0738258361816 0.0
0 0.00107890367507935 1.0
0 0.000585675239562988 2.0
0 -0.0132269859313965 3.0
0 0.0105957388877869 4.0
0 0.0103799700737 5.0
0 0.00216841697692871 6.0
0 0.00124680995941162 7.0
33.558410457 38.8956298828125 0.0
0 0.00843101739883423 1.0
0 -0.0107095837593079 2.0
0 -0.0284059047698975 3.0
0 0.000358760356903076 4.0
0 0.00435709953308105 5.0
0 0.00229853391647339 6.0
0 0.00229942798614502 7.0
47.788896524 38.6363067626953 0.0
0 -0.00472438335418701 1.0
0 -0.0201917886734009 2.0
0 0.0364751219749451 3.0
0 -0.0135514140129089 4.0
0 -0.016156792640686 5.0
0 0.00404399633407593 6.0
0 -0.00392872095108032 7.0
33.642782256 38.8176574707031 0.0
0 -0.00302708148956299 1.0
0 0.00466156005859375 2.0
0 0.0140516757965088 3.0
0 0.00737649202346802 4.0
0 0.00594466924667358 5.0
0 0.0017513632774353 6.0
0 -0.00115036964416504 7.0
37.659684029 38.3499221801758 0.0
0 0.0080101490020752 1.0
0 0.000805974006652832 2.0
0 0.0185391902923584 3.0
0 0.0152295827865601 4.0
0 0.00819528102874756 5.0
0 0.00101685523986816 6.0
0 -0.00529128313064575 7.0
40.91969971 38.140266418457 0.0
0 0.00345683097839355 1.0
0 -0.00896137952804565 2.0
0 -0.0119954347610474 3.0
0 0.00301778316497803 4.0
0 0.00167751312255859 5.0
0 0.00145190954208374 6.0
0 -0.00629293918609619 7.0
42.622265927 38.811466217041 0.0
0 0.019258975982666 1.0
0 0.0364774465560913 2.0
0 -0.00904768705368042 3.0
0 -0.0298802852630615 4.0
0 -0.0317170023918152 5.0
0 0.00465452671051025 6.0
0 -0.00226563215255737 7.0
33.946966747 38.4802627563477 0.0
0 -0.00518739223480225 1.0
0 0.032974898815155 2.0
0 0.0197080373764038 3.0
0 0.00946873426437378 4.0
0 0.00464081764221191 5.0
0 0.000941216945648193 6.0
0 -0.00420850515365601 7.0
36.409182194 38.5286026000977 0.0
0 0.00702512264251709 1.0
0 -0.0194560289382935 2.0
0 0.0110805034637451 3.0
0 0.0136269330978394 4.0
0 0.0101454257965088 5.0
0 0.00085604190826416 6.0
0 -0.0050346851348877 7.0
40.593448001 38.5760307312012 0.0
0 0.0027921199798584 1.0
0 0.000395596027374268 2.0
0 -0.00285184383392334 3.0
0 0.0092308521270752 4.0
0 0.010448157787323 5.0
0 0.000471591949462891 6.0
0 0.00133585929870605 7.0
41.715980637 38.8361663818359 0.0
0 0.0497987866401672 1.0
0 0.0365809798240662 2.0
0 0.0144861936569214 3.0
0 -0.0297126173973083 4.0
0 -0.0379680395126343 5.0
0 0.00509655475616455 6.0
0 -0.00242024660110474 7.0
33.77037363 38.5182495117188 0.0
0 0.0106807351112366 1.0
0 0.083455502986908 2.0
0 -0.0791702270507812 3.0
0 -0.052739679813385 4.0
0 -0.0625720620155334 5.0
0 0.00684452056884766 6.0
0 -0.000840663909912109 7.0
46.074402277 38.5328330993652 0.0
0 0.0320846438407898 1.0
0 0.0307254195213318 2.0
0 0.0312415361404419 3.0
0 -0.0108807682991028 4.0
0 -0.0112593173980713 5.0
0 0.00355857610702515 6.0
0 0.00201642513275146 7.0
40.510394826 38.7015953063965 0.0
0 0.00823861360549927 1.0
0 -0.0077546238899231 2.0
0 0.046064555644989 3.0
0 -0.0181393027305603 4.0
0 -0.017975926399231 5.0
0 0.002483069896698 6.0
0 0.00150805711746216 7.0
39.244329381 38.7511940002441 0.0
0 0.0159115791320801 1.0
0 0.0022616982460022 2.0
0 0.0282448530197144 3.0
0 -0.00502586364746094 4.0
0 -0.00408065319061279 5.0
0 0.00202155113220215 6.0
0 0.000927567481994629 7.0
35.7171596 38.6546821594238 0.0
0 0.00409018993377686 1.0
0 -0.00489550828933716 2.0
0 -0.0026392936706543 3.0
0 0.0113359689712524 4.0
0 0.0123302340507507 5.0
0 0.000735223293304443 6.0
0 0.00068289041519165 7.0
39.754714125 38.8880348205566 0.0
0 -0.00539577007293701 1.0
0 -0.00136715173721313 2.0
0 -0.0127389430999756 3.0
0 0.00762850046157837 4.0
0 -0.000317215919494629 5.0
0 0.0031813383102417 6.0
0 -0.00161910057067871 7.0
41.796195011 38.343635559082 0.0
0 0.00867635011672974 1.0
0 -0.008350670337677 2.0
0 0.0206466317176819 3.0
0 0.0107426643371582 4.0
0 0.0104712247848511 5.0
0 0.000457584857940674 6.0
0 -0.000905036926269531 7.0
29.980660557 38.4174346923828 0.0
0 0.00467085838317871 1.0
0 -0.00495713949203491 2.0
0 0.0162566900253296 3.0
0 0.00935602188110352 4.0
0 0.00986051559448242 5.0
0 0.000848650932312012 6.0
0 0.000751495361328125 7.0
36.241260622 38.5575332641602 0.0
0 0.000705897808074951 1.0
0 0.00149893760681152 2.0
0 0.0100005865097046 3.0
0 0.01280277967453 4.0
0 0.0111420154571533 5.0
0 0.000396430492401123 6.0
0 -0.00308972597122192 7.0
38.129703213 38.4214096069336 0.0
0 0.00814342498779297 1.0
0 0.000246703624725342 2.0
0 0.0169895887374878 3.0
0 0.0135315656661987 4.0
0 0.010769248008728 5.0
0 0.00128799676895142 6.0
0 -0.0017850399017334 7.0
36.91522515 38.2449417114258 0.0
0 0.0027318000793457 1.0
0 -0.00151664018630981 2.0
0 0.00988304615020752 3.0
0 0.00847196578979492 4.0
0 0.0108878016471863 5.0
0 0.000810861587524414 6.0
0 0.0011824369430542 7.0
38.102908751 38.1841430664062 0.0
0 0.00115394592285156 1.0
0 0.00325274467468262 2.0
0 0.0126217603683472 3.0
0 0.0149384140968323 4.0
0 0.0158239006996155 5.0
0 0.000880837440490723 6.0
0 -8.7440013885498e-05 7.0
38.34829669 38.6496887207031 0.0
0 -0.00112295150756836 1.0
0 0.0135008692741394 2.0
0 0.00834870338439941 3.0
0 0.0102553367614746 4.0
0 0.00822752714157104 5.0
0 0.00185030698776245 6.0
0 0.00093388557434082 7.0
31.13885609 38.5502471923828 0.0
0 0.000279843807220459 1.0
0 0.00190377235412598 2.0
0 0.00924229621887207 3.0
0 0.0154766440391541 4.0
0 0.0117242932319641 5.0
0 0.00106501579284668 6.0
0 -0.00504231452941895 7.0
35.130346793 39.991096496582 0.0
0 -0.172638535499573 1.0
0 0.176112532615662 2.0
0 0.261671185493469 3.0
0 -0.0960418581962585 4.0
0 -0.047913670539856 5.0
0 0.0104336738586426 6.0
0 0.00360935926437378 7.0
37.970631847 38.0968322753906 0.0
0 -0.00251191854476929 1.0
0 0.0147252678871155 2.0
0 -0.0113922357559204 3.0
0 -0.000210464000701904 4.0
0 -0.00579959154129028 5.0
0 0.00145632028579712 6.0
0 -0.00575977563858032 7.0
37.012414901 38.5231437683105 0.0
0 -0.0027344822883606 1.0
0 0.0352815389633179 2.0
0 0.00825810432434082 3.0
0 -0.00124883651733398 4.0
0 -0.0013393759727478 5.0
0 0.00134366750717163 6.0
0 -0.00226140022277832 7.0
36.779265883 38.894100189209 0.0
0 0.0063093900680542 1.0
0 0.027963399887085 2.0
0 -0.00222969055175781 3.0
0 0.0165801048278809 4.0
0 0.00685769319534302 5.0
0 0.00238025188446045 6.0
0 -0.00534164905548096 7.0
42.665291585 38.5481719970703 0.0
0 0.00762939453125 1.0
0 0.0562233328819275 2.0
0 -0.0685888528823853 3.0
0 -0.0368189811706543 4.0
0 -0.04285728931427 5.0
0 0.00503599643707275 6.0
0 -0.00157469511032104 7.0
35.413046011 38.4058380126953 0.0
0 0.00584030151367188 1.0
0 0.000958383083343506 2.0
0 0.0113482475280762 3.0
0 0.0107216835021973 4.0
0 0.00791442394256592 5.0
0 0.00128114223480225 6.0
0 0.000605762004852295 7.0
35.966339402 38.4775657653809 0.0
0 0.00496381521224976 1.0
0 -0.00544369220733643 2.0
0 0.00988161563873291 3.0
0 0.0164755582809448 4.0
0 0.0110843181610107 5.0
0 0.00106734037399292 6.0
0 -0.00433731079101562 7.0
39.418493945 38.8547630310059 0.0
0 0.00878036022186279 1.0
0 0.000721454620361328 2.0
0 -0.0158603191375732 3.0
0 0.0104245543479919 4.0
0 0.0125052928924561 5.0
0 0.000981569290161133 6.0
0 0.000788688659667969 7.0
32.151549765 38.9702301025391 0.0
0 -0.0014265775680542 1.0
0 0.012227475643158 2.0
0 0.0263000726699829 3.0
0 -0.00517529249191284 4.0
0 -0.00719040632247925 5.0
0 0.00259649753570557 6.0
0 -0.00119811296463013 7.0
31.845645841 38.3061141967773 0.0
0 0.00534671545028687 1.0
0 -0.0176301002502441 2.0
0 -0.0113786458969116 3.0
0 0.00411343574523926 4.0
0 0.00402390956878662 5.0
0 -0.002593994140625 6.0
0 -0.0061572790145874 7.0
40.557128571 38.4783058166504 0.0
0 -0.0107422471046448 1.0
0 0.0388541221618652 2.0
0 0.0169771909713745 3.0
0 0.00476694107055664 4.0
0 0.00180530548095703 5.0
0 0.00234603881835938 6.0
0 0.00116020441055298 7.0
36.793774817 38.6162338256836 0.0
0 0.00310921669006348 1.0
0 0.0043950080871582 2.0
0 0.0118415951728821 3.0
0 -0.0224356055259705 4.0
0 -0.0218193531036377 5.0
0 0.00348061323165894 6.0
0 -0.00138634443283081 7.0
41.333576969 38.9786033630371 0.0
0 0.0102357268333435 1.0
0 -0.00271809101104736 2.0
0 -0.0131592750549316 3.0
0 0.0101926326751709 4.0
0 0.0110886096954346 5.0
0 0.00132131576538086 6.0
0 0.00104010105133057 7.0
43.791455275 38.4271545410156 0.0
0 -0.00638270378112793 1.0
0 0.0200754404067993 2.0
0 0.0120011568069458 3.0
0 0.0064471960067749 4.0
0 0.00578755140304565 5.0
0 0.00162583589553833 6.0
0 0.00128662586212158 7.0
41.481697076 38.2028846740723 0.0
0 0.00590914487838745 1.0
0 0.00448429584503174 2.0
0 -0.0016942024230957 3.0
0 0.00982093811035156 4.0
0 0.0135623216629028 5.0
0 0.000484585762023926 6.0
0 0.000510692596435547 7.0
35.532774621 38.3699798583984 0.0
0 0.00557297468185425 1.0
0 0.00183498859405518 2.0
0 0.00422048568725586 3.0
0 0.0129894614219666 4.0
0 0.0152958631515503 5.0
0 0.00223052501678467 6.0
0 0.00200212001800537 7.0
34.894006808 38.6870956420898 0.0
0 -0.00032496452331543 1.0
0 -0.000553488731384277 2.0
0 -0.0101300477981567 3.0
0 0.0154532194137573 4.0
0 0.0144065618515015 5.0
0 0.00112330913543701 6.0
0 -0.00057142972946167 7.0
38.893810297 38.5068321228027 0.0
0 0.00826197862625122 1.0
0 0.00306463241577148 2.0
0 0.00759679079055786 3.0
0 0.0136953592300415 4.0
0 0.0143409967422485 5.0
0 0.00131523609161377 6.0
0 0.000699937343597412 7.0
39.624663522 38.6048698425293 0.0
0 -0.0116535425186157 1.0
0 0.00954610109329224 2.0
0 -0.00461006164550781 3.0
0 0.017173171043396 4.0
0 0.0123999118804932 5.0
0 0.00213772058486938 6.0
0 3.28421592712402e-05 7.0
42.516231091 39.1451377868652 0.0
0 0.0351818799972534 1.0
0 0.110571384429932 2.0
0 0.0638208389282227 3.0
0 -0.0394923686981201 4.0
0 -0.0376558303833008 5.0
0 0.00706255435943604 6.0
0 0.006949782371521 7.0
38.374768577 38.2217788696289 0.0
0 0.0492962598800659 1.0
0 0.00276666879653931 2.0
0 -0.0941604375839233 3.0
0 -0.0646132230758667 4.0
0 -0.0723511576652527 5.0
0 0.00604933500289917 6.0
0 0.000296711921691895 7.0
33.351459508 39.1657485961914 0.0
0 -0.00859594345092773 1.0
0 0.0212362408638 2.0
0 -0.00135296583175659 3.0
0 0.00774955749511719 4.0
0 0.00817430019378662 5.0
0 0.00212478637695312 6.0
0 0.00165778398513794 7.0
37.08812014 38.7333183288574 0.0
0 -0.000715315341949463 1.0
0 0.00112462043762207 2.0
0 -0.00267255306243896 3.0
0 0.011474072933197 4.0
0 0.011874794960022 5.0
0 0.00210350751876831 6.0
0 0.00163459777832031 7.0
44.445084649 38.5677528381348 0.0
0 0.00638324022293091 1.0
0 -0.00895494222640991 2.0
0 0.00653094053268433 3.0
0 0.0088040828704834 4.0
0 0.0119718909263611 5.0
0 0.000811874866485596 6.0
0 0.00173741579055786 7.0
37.22322104 38.371509552002 0.0
0 0.00562351942062378 1.0
0 0.00389868021011353 2.0
0 0.0112352967262268 3.0
0 0.0115906000137329 4.0
0 0.0116457343101501 5.0
0 0.000585973262786865 6.0
0 -0.00184345245361328 7.0
44.605944609 38.6993217468262 0.0
0 0.0200371146202087 1.0
0 0.0074923038482666 2.0
0 -0.0283963084220886 3.0
0 -0.0215704441070557 4.0
0 -0.0214748382568359 5.0
0 0.00237280130386353 6.0
0 0.00550472736358643 7.0
39.553698498 38.4618606567383 0.0
0 0.00432324409484863 1.0
0 0.0075610876083374 2.0
0 0.00864207744598389 3.0
0 0.015913724899292 4.0
0 0.013094961643219 5.0
0 0.000805139541625977 6.0
0 -0.00503039360046387 7.0
43.287363281 39.1227188110352 0.0
0 0.0155909657478333 1.0
0 0.0208890438079834 2.0
0 0.0387828350067139 3.0
0 -0.00965893268585205 4.0
0 -0.0132709741592407 5.0
0 0.00286310911178589 6.0
0 -0.0010300874710083 7.0
33.992156471 38.1259117126465 0.0
0 -0.000116407871246338 1.0
0 -0.00216031074523926 2.0
0 0.00474166870117188 3.0
0 0.0110854506492615 4.0
0 0.0120851397514343 5.0
0 -0.000650346279144287 6.0
0 -0.00454753637313843 7.0
36.860093248 39.0079536437988 0.0
0 -0.00831305980682373 1.0
0 0.00342237949371338 2.0
0 -0.00300586223602295 3.0
0 0.00467437505722046 4.0
0 0.00290507078170776 5.0
0 0.00205975770950317 6.0
0 0.00154572725296021 7.0
39.672962738 38.0751495361328 0.0
0 0.00984746217727661 1.0
0 -0.0117852091789246 2.0
0 0.0192103981971741 3.0
0 0.00903779268264771 4.0
0 0.00922292470932007 5.0
0 -1.78813934326172e-06 6.0
0 -0.00248432159423828 7.0
32.8535328 38.515998840332 0.0
0 0.00934451818466187 1.0
0 -0.0145254731178284 2.0
0 0.0208010673522949 3.0
0 0.00874799489974976 4.0
0 0.00777202844619751 5.0
0 0.00117224454879761 6.0
0 -0.00193655490875244 7.0
36.422494688 38.2510757446289 0.0
0 0.00413292646408081 1.0
0 0.0195381045341492 2.0
0 -0.0251415371894836 3.0
0 -0.00397628545761108 4.0
0 -0.00545597076416016 5.0
0 0.00688892602920532 6.0
0 -0.00616079568862915 7.0
32.082396771 38.4953002929688 0.0
0 0.0391383171081543 1.0
0 0.0698838829994202 2.0
0 -0.0538221597671509 3.0
0 -0.0767565369606018 4.0
0 -0.080767035484314 5.0
0 0.00638920068740845 6.0
0 0.00216215848922729 7.0
43.742460686 38.6314849853516 0.0
0 0.00743508338928223 1.0
0 0.00358933210372925 2.0
0 0.00233936309814453 3.0
0 0.0148839354515076 4.0
0 0.0103566646575928 5.0
0 0.00171154737472534 6.0
0 -0.00294375419616699 7.0
37.830554934 38.8711471557617 0.0
0 -0.00633442401885986 1.0
0 -0.085806131362915 2.0
0 0.0016249418258667 3.0
0 -0.00366508960723877 4.0
0 -0.00491905212402344 5.0
0 0.00461995601654053 6.0
0 0.00372880697250366 7.0
45.388745763 38.5289192199707 0.0
0 0.00618654489517212 1.0
0 0.00241726636886597 2.0
0 0.00818026065826416 3.0
0 0.0156952738761902 4.0
0 0.0136352181434631 5.0
0 0.00150501728057861 6.0
0 -0.00034487247467041 7.0
40.010376396 38.5327529907227 0.0
0 0.00943154096603394 1.0
0 -0.0237060785293579 2.0
0 0.0177397727966309 3.0
0 0.00880545377731323 4.0
0 0.00820803642272949 5.0
0 0.000179648399353027 6.0
0 -0.00252014398574829 7.0
45.566990187 38.9576721191406 0.0
0 -0.00651991367340088 1.0
0 0.0196529626846313 2.0
0 -0.0233531594276428 3.0
0 0.00508701801300049 4.0
0 0.00925230979919434 5.0
0 0.000830531120300293 6.0
0 0.00188058614730835 7.0
32.300808889 39.3677978515625 0.0
0 -0.139041066169739 1.0
0 0.0485532879829407 2.0
0 0.379734396934509 3.0
0 -0.113313436508179 4.0
0 0.248469144105911 5.0
0 0.0100367069244385 6.0
0 0.00433242321014404 7.0
36.805463378 38.648265838623 0.0
0 -0.0118266344070435 1.0
0 -0.0432865023612976 2.0
0 0.023951530456543 3.0
0 -0.00356596708297729 4.0
0 -0.00542491674423218 5.0
0 0.00420564413070679 6.0
0 0.00177103281021118 7.0
45.039727992 38.5601387023926 0.0
0 0.00589120388031006 1.0
0 -0.0125548243522644 2.0
0 0.00396126508712769 3.0
0 0.0183145403862 4.0
0 0.0117776989936829 5.0
0 0.00193524360656738 6.0
0 -0.00209051370620728 7.0
36.669966877 38.5309829711914 0.0
0 0.00463026762008667 1.0
0 0.0226883888244629 2.0
0 0.00629353523254395 3.0
0 0.0114103555679321 4.0
0 0.00907456874847412 5.0
0 0.00171947479248047 6.0
0 -0.00196748971939087 7.0
40.401972767 38.2327117919922 0.0
0 0.0044560432434082 1.0
0 0.00335395336151123 2.0
0 0.00200176239013672 3.0
0 0.00582796335220337 4.0
0 0.00227099657058716 5.0
0 0.00142323970794678 6.0
0 -0.00324386358261108 7.0
41.491221092 38.6873626708984 0.0
0 0.00546443462371826 1.0
0 -0.00274443626403809 2.0
0 -0.00156420469284058 3.0
0 0.0120398998260498 4.0
0 0.0104166865348816 5.0
0 0.00150889158248901 6.0
0 0.000754177570343018 7.0
32.092723767 38.4019317626953 0.0
0 0.00888198614120483 1.0
0 0.0200417041778564 2.0
0 -0.00451493263244629 3.0
0 0.00675791501998901 4.0
0 0.000231027603149414 5.0
0 0.00174546241760254 6.0
0 -0.00465226173400879 7.0
39.133472273 38.943042755127 0.0
0 0.0176572203636169 1.0
0 -0.010567843914032 2.0
0 0.0199633836746216 3.0
0 -0.0152668356895447 4.0
0 -0.0183753371238708 5.0
0 0.00282925367355347 6.0
0 -0.000978946685791016 7.0
37.053979998 38.9729843139648 0.0
0 -0.000790834426879883 1.0
0 0.026669442653656 2.0
0 0.0143687725067139 3.0
0 0.0024116039276123 4.0
0 -0.00121736526489258 5.0
0 0.00256633758544922 6.0
0 -0.00202929973602295 7.0
39.864494306 38.6545906066895 0.0
0 0.0108628273010254 1.0
0 0.0133299231529236 2.0
0 0.00485479831695557 3.0
0 0.00242054462432861 4.0
0 0.00336581468582153 5.0
0 0.00100594758987427 6.0
0 -0.00150913000106812 7.0
39.054921308 38.6194686889648 0.0
0 -0.00948309898376465 1.0
0 0.0172194242477417 2.0
0 0.00661468505859375 3.0
0 0.00922781229019165 4.0
0 0.010545015335083 5.0
0 0.00140833854675293 6.0
0 0.00142842531204224 7.0
33.466015237 38.9646224975586 0.0
0 9.03606414794922e-05 1.0
0 0.0123940706253052 2.0
0 0.00274002552032471 3.0
0 0.00122815370559692 4.0
0 -0.000679194927215576 5.0
0 0.0030023455619812 6.0
0 0.00157451629638672 7.0
37.653527764 38.5430641174316 0.0
0 -0.0041007399559021 1.0
0 0.0149326324462891 2.0
0 0.00574177503585815 3.0
0 0.00125390291213989 4.0
0 0.00139206647872925 5.0
0 0.000427007675170898 6.0
0 -0.00363874435424805 7.0
37.577039283 38.1301307678223 0.0
0 -0.00789469480514526 1.0
0 0.0148867964744568 2.0
0 0.00783491134643555 3.0
0 0.0112177729606628 4.0
0 0.00835388898849487 5.0
0 0.0013580322265625 6.0
0 -0.00169438123703003 7.0
36.318307627 38.468090057373 0.0
0 0.00593608617782593 1.0
0 0.00778979063034058 2.0
0 0.00651895999908447 3.0
0 0.00518614053726196 4.0
0 0.00192540884017944 5.0
0 0.00126242637634277 6.0
0 -0.00347381830215454 7.0
36.801464194 38.3615798950195 0.0
0 0.00963151454925537 1.0
0 -0.00350320339202881 2.0
0 0.00691354274749756 3.0
0 0.0129764676094055 4.0
0 0.01592618227005 5.0
0 0.00100570917129517 6.0
0 0.000829577445983887 7.0
39.56095567 38.5987434387207 0.0
0 -0.000210285186767578 1.0
0 -0.0386602878570557 2.0
0 0.0109865069389343 3.0
0 -0.0223806500434875 4.0
0 -0.0233038067817688 5.0
0 0.0035366415977478 6.0
0 -0.00128954648971558 7.0
40.443539209 38.8790626525879 0.0
0 0.0022013783454895 1.0
0 0.0163164138793945 2.0
0 0.0207526087760925 3.0
0 0.0038456916809082 4.0
0 0.00528603792190552 5.0
0 0.00127983093261719 6.0
0 -0.00158476829528809 7.0
40.316920108 38.4642219543457 0.0
0 0.00427037477493286 1.0
0 -0.012304425239563 2.0
0 0.0196879506111145 3.0
0 0.0127553343772888 4.0
0 0.00777769088745117 5.0
0 0.00142759084701538 6.0
0 -0.00219595432281494 7.0
33.579405458 39.1697883605957 0.0
0 0.00949239730834961 1.0
0 0.0550859570503235 2.0
0 0.00363188982009888 3.0
0 -0.00875645875930786 4.0
0 -0.00982797145843506 5.0
0 0.00408774614334106 6.0
0 0.00239044427871704 7.0
34.562810956 38.7289924621582 0.0
0 -0.00927829742431641 1.0
0 -0.031139075756073 2.0
0 0.0254233479499817 3.0
0 -0.00332009792327881 4.0
0 -0.00410580635070801 5.0
0 0.00320959091186523 6.0
0 0.00121003389358521 7.0
37.518824119 38.526123046875 0.0
0 -0.00417220592498779 1.0
0 0.0312619805335999 2.0
0 -0.0036507248878479 3.0
0 0.000343263149261475 4.0
0 -0.00207829475402832 5.0
0 0.00180166959762573 6.0
0 -0.00563240051269531 7.0
39.547705636 38.8937301635742 0.0
0 0.00837802886962891 1.0
0 0.0164691209793091 2.0
0 0.0180018544197083 3.0
0 0.00425541400909424 4.0
0 0.00221371650695801 5.0
0 0.00231784582138062 6.0
0 -0.00168567895889282 7.0
42.834945783 38.4985198974609 0.0
0 0.00751316547393799 1.0
0 -0.00238978862762451 2.0
0 0.0189425945281982 3.0
0 0.00994026660919189 4.0
0 0.0116481781005859 5.0
0 -9.0181827545166e-05 6.0
0 -0.00097280740737915 7.0
38.473622953 38.5314598083496 0.0
0 -0.00472825765609741 1.0
0 0.00927400588989258 2.0
0 0.0143502950668335 3.0
0 0.00991624593734741 4.0
0 0.0122590065002441 5.0
0 0.00189471244812012 6.0
0 0.0014805793762207 7.0
45.238523457 38.3235244750977 0.0
0 0.00389409065246582 1.0
0 0.0224522352218628 2.0
0 0.013031542301178 3.0
0 0.00490808486938477 4.0
0 0.00208324193954468 5.0
0 0.00188267230987549 6.0
0 -0.003562331199646 7.0
36.266282562 38.6187591552734 0.0
0 0.00554311275482178 1.0
0 -0.00330221652984619 2.0
0 -3.94582748413086e-05 3.0
0 0.0115361213684082 4.0
0 0.00921422243118286 5.0
0 0.00178325176239014 6.0
0 0.00104552507400513 7.0
37.845233948 38.7518157958984 0.0
0 -0.000949203968048096 1.0
0 0.00222766399383545 2.0
0 0.0181483030319214 3.0
0 0.00984752178192139 4.0
0 0.00675064325332642 5.0
0 0.00161761045455933 6.0
0 -0.00165462493896484 7.0
40.414778929 38.6832542419434 0.0
0 0.0396010279655457 1.0
0 0.0414213538169861 2.0
0 0.034054696559906 3.0
0 -0.0138557553291321 4.0
0 -0.0228312611579895 5.0
0 0.00423502922058105 6.0
0 -0.00209474563598633 7.0
47.104382604 38.6785850524902 0.0
0 -0.000528335571289062 1.0
0 -0.0117012858390808 2.0
0 0.00560390949249268 3.0
0 0.0113921761512756 4.0
0 0.0118933320045471 5.0
0 0.00046849250793457 6.0
0 -0.00160872936248779 7.0
37.557508388 38.6452941894531 0.0
0 0.0107706189155579 1.0
0 0.0139909386634827 2.0
0 0.0265346765518188 3.0
0 -0.00883990526199341 4.0
0 -0.00819224119186401 5.0
0 0.00279784202575684 6.0
0 0.000886917114257812 7.0
40.532317042 38.8532943725586 0.0
0 -0.00338459014892578 1.0
0 0.00208669900894165 2.0
0 -0.00943958759307861 3.0
0 0.0108900666236877 4.0
0 0.0107890367507935 5.0
0 0.00126546621322632 6.0
0 0.00102007389068604 7.0
32.594348946 38.9392776489258 0.0
0 -0.00994551181793213 1.0
0 -0.00474840402603149 2.0
0 0.0156255960464478 3.0
0 0.00870126485824585 4.0
0 -0.000639498233795166 5.0
0 0.00391221046447754 6.0
0 0.00118041038513184 7.0
30.392342502 38.4610557556152 0.0
0 0.00475746393203735 1.0
0 0.00234395265579224 2.0
0 0.00362575054168701 3.0
0 0.0170789361000061 4.0
0 0.0154737830162048 5.0
0 0.00255101919174194 6.0
0 0.00137734413146973 7.0
40.607017947 38.6572570800781 0.0
0 -0.00629860162734985 1.0
0 0.0028877854347229 2.0
0 0.00213658809661865 3.0
0 0.0124313235282898 4.0
0 0.00775974988937378 5.0
0 0.00156289339065552 6.0
0 -0.00161468982696533 7.0
39.11337914 39.3402252197266 0.0
0 0.053605318069458 1.0
0 0.106695532798767 2.0
0 0.0744338035583496 3.0
0 -0.0374942421913147 4.0
0 -0.0441901683807373 5.0
0 0.00579911470413208 6.0
0 -0.00190454721450806 7.0
35.280627531 38.4612503051758 0.0
0 -0.00280666351318359 1.0
0 0.0293827652931213 2.0
0 -0.0139994621276855 3.0
0 -0.0032992959022522 4.0
0 -0.0042424201965332 5.0
0 -0.00224858522415161 6.0
0 -0.0053715705871582 7.0
32.847447182 38.3127861022949 0.0
0 0.00380522012710571 1.0
0 0.00457185506820679 2.0
0 -0.0068669319152832 3.0
0 0.019364058971405 4.0
0 0.0192620754241943 5.0
0 0.00249946117401123 6.0
0 0.00178861618041992 7.0
34.14972673 38.8230133056641 0.0
0 -0.0161435604095459 1.0
0 0.0640211701393127 2.0
0 0.0416672229766846 3.0
0 -0.0326667428016663 4.0
0 -0.0316633582115173 5.0
0 0.00657165050506592 6.0
0 0.0048445463180542 7.0
36.823986173 38.8261451721191 0.0
0 0.000983893871307373 1.0
0 0.0175092816352844 2.0
0 0.0232627391815186 3.0
0 0.00660783052444458 4.0
0 0.00244992971420288 5.0
0 0.00188249349594116 6.0
0 -0.00359439849853516 7.0
43.615049092 38.7269248962402 0.0
0 -0.00817251205444336 1.0
0 -0.016715943813324 2.0
0 0.021268367767334 3.0
0 -0.0094103217124939 4.0
0 -0.00766164064407349 5.0
0 0.00359576940536499 6.0
0 0.00358837842941284 7.0
39.316465342 38.8074150085449 0.0
0 0.00467908382415771 1.0
0 -0.141974627971649 2.0
0 0.00819063186645508 3.0
0 -0.0243714451789856 4.0
0 -0.0265893340110779 5.0
0 0.00668883323669434 6.0
0 0.00609123706817627 7.0
36.676849778 39.086483001709 0.0
0 0.0338906645774841 1.0
0 0.010081946849823 2.0
0 0.0517814159393311 3.0
0 -0.0148599147796631 4.0
0 -0.015914261341095 5.0
0 0.00404250621795654 6.0
0 -0.000586926937103271 7.0
36.163160136 39.0178642272949 0.0
0 0.0213291645050049 1.0
0 0.000769555568695068 2.0
0 0.0280357003211975 3.0
0 -0.0171972513198853 4.0
0 -0.0217657089233398 5.0
0 0.00370281934738159 6.0
0 -0.00175964832305908 7.0
40.305175496 38.8357582092285 0.0
0 0.00539052486419678 1.0
0 0.00232678651809692 2.0
0 -0.00763154029846191 3.0
0 0.00963932275772095 4.0
0 0.00745934247970581 5.0
0 0.00154834985733032 6.0
0 0.000475943088531494 7.0
34.293236222 38.5812950134277 0.0
0 0.0108577609062195 1.0
0 -0.00930273532867432 2.0
0 0.00788706541061401 3.0
0 0.0121802091598511 4.0
0 0.0107622146606445 5.0
0 0.00162261724472046 6.0
0 0.000948786735534668 7.0
38.347299924 38.5173797607422 0.0
0 0.00819861888885498 1.0
0 0.00382977724075317 2.0
0 0.0101303458213806 3.0
0 0.0126463174819946 4.0
0 0.0123133063316345 5.0
0 0.00140637159347534 6.0
0 0.000719904899597168 7.0
39.036910781 38.8415069580078 0.0
0 0.000177502632141113 1.0
0 0.00544929504394531 2.0
0 -0.0120937824249268 3.0
0 0.00871700048446655 4.0
0 0.00971460342407227 5.0
0 0.0013730525970459 6.0
0 0.00144284963607788 7.0
36.237500285 38.6325035095215 0.0
0 0.00555747747421265 1.0
0 -0.00464928150177002 2.0
0 0.00741976499557495 3.0
0 0.0146653056144714 4.0
0 0.0154220461845398 5.0
0 0.00221884250640869 6.0
0 0.00104159116744995 7.0
36.900540753 38.554141998291 0.0
0 -0.00575870275497437 1.0
0 0.0343371629714966 2.0
0 0.00224876403808594 3.0
0 0.013765811920166 4.0
0 0.00792694091796875 5.0
0 0.0021330714225769 6.0
0 -0.00349611043930054 7.0
29.897899633 38.8244743347168 0.0
0 -0.00203788280487061 1.0
0 0.0106021165847778 2.0
0 0.0118087530136108 3.0
0 0.00280171632766724 4.0
0 -0.000840127468109131 5.0
0 0.00222545862197876 6.0
0 -0.00134199857711792 7.0
41.071373967 38.722110748291 0.0
0 -0.000163435935974121 1.0
0 0.00294435024261475 2.0
0 -0.00318533182144165 3.0
0 0.0174000263214111 4.0
0 0.00768005847930908 5.0
0 0.00268244743347168 6.0
0 -0.000645101070404053 7.0
37.123617582 38.6545639038086 0.0
0 0.0106359124183655 1.0
0 -0.00192886590957642 2.0
0 0.0364857912063599 3.0
0 -0.0049210786819458 4.0
0 -0.00993657112121582 5.0
0 0.00295078754425049 6.0
0 -0.00167167186737061 7.0
38.348981744 39.361213684082 0.0
0 0.108397841453552 1.0
0 0.0719989538192749 2.0
0 0.0854716300964355 3.0
0 -0.0621429681777954 4.0
0 -0.0700206160545349 5.0
0 0.00875371694564819 6.0
0 0.00082772970199585 7.0
37.897829023 38.3081512451172 0.0
0 0.0116560459136963 1.0
0 -0.0115576982498169 2.0
0 0.0105234384536743 3.0
0 0.0114278793334961 4.0
0 0.00531607866287231 5.0
0 0.000552892684936523 6.0
0 -0.0051882266998291 7.0
32.839543182 38.3974990844727 0.0
0 0.00181096792221069 1.0
0 0.00699031352996826 2.0
0 -0.00629299879074097 3.0
0 0.0180360674858093 4.0
0 0.0172174572944641 5.0
0 0.00267308950424194 6.0
0 0.00208872556686401 7.0
45.814256844 38.7030944824219 0.0
0 0.00507837533950806 1.0
0 0.00886917114257812 2.0
0 -0.0037347674369812 3.0
0 0.0136943459510803 4.0
0 0.0137937068939209 5.0
0 0.00230222940444946 6.0
0 0.00160950422286987 7.0
38.439958634 38.4825057983398 0.0
0 -0.00018012523651123 1.0
0 0.0141959190368652 2.0
0 -0.0422380566596985 3.0
0 -0.0285893678665161 4.0
0 -0.0323619246482849 5.0
0 0.00240993499755859 6.0
0 -0.00253540277481079 7.0
33.022424116 38.5725517272949 0.0
0 -0.00100618600845337 1.0
0 0.0236772298812866 2.0
0 0.0142524242401123 3.0
0 0.00742864608764648 4.0
0 0.00858741998672485 5.0
0 0.000316262245178223 6.0
0 -0.00097280740737915 7.0
36.232706353 38.6497421264648 0.0
0 0.00666439533233643 1.0
0 0.0102894306182861 2.0
0 -0.00639992952346802 3.0
0 0.0107094049453735 4.0
0 0.00857967138290405 5.0
0 0.00131088495254517 6.0
0 0.000430107116699219 7.0
45.8365445 38.2908325195312 0.0
0 0.00307029485702515 1.0
0 -0.00923150777816772 2.0
0 0.00529932975769043 3.0
0 0.00588589906692505 4.0
0 0.0053897500038147 5.0
0 0.000756263732910156 6.0
0 -0.00225520133972168 7.0
39.344253832 38.8997688293457 0.0
0 0.00560510158538818 1.0
0 0.012593150138855 2.0
0 0.020938515663147 3.0
0 0.00587159395217896 4.0
0 0.00207465887069702 5.0
0 0.00155848264694214 6.0
0 -0.00285935401916504 7.0
43.613379879 38.2364540100098 0.0
0 0.000397384166717529 1.0
0 0.031568169593811 2.0
0 -0.00499022006988525 3.0
0 -0.00342869758605957 4.0
0 -0.00124233961105347 5.0
0 0.00189417600631714 6.0
0 -0.00510674715042114 7.0
41.603825526 38.189998626709 0.0
0 0.00724172592163086 1.0
0 -0.0164687633514404 2.0
0 0.00195282697677612 3.0
0 0.0148271322250366 4.0
0 0.0068591833114624 5.0
0 0.000564455986022949 6.0
0 -0.00725555419921875 7.0
41.290499737 38.4870338439941 0.0
0 0.0102403163909912 1.0
0 -0.00607091188430786 2.0
0 0.0221967697143555 3.0
0 0.0122079849243164 4.0
0 0.00884580612182617 5.0
0 0.000471353530883789 6.0
0 -0.0037955641746521 7.0
38.21592954 38.5617942810059 0.0
0 -0.000384032726287842 1.0
0 0.00387489795684814 2.0
0 -0.00276827812194824 3.0
0 0.011198878288269 4.0
0 0.013520359992981 5.0
0 0.00102299451828003 6.0
0 0.00107133388519287 7.0
34.764993855 38.4070091247559 0.0
0 -0.00358575582504272 1.0
0 0.0124229192733765 2.0
0 -0.00962519645690918 3.0
0 -0.00281190872192383 4.0
0 -0.00192475318908691 5.0
0 0.000529289245605469 6.0
0 -0.00529855489730835 7.0
34.789226058 38.5943145751953 0.0
0 -0.000479459762573242 1.0
0 0.0239938497543335 2.0
0 0.0198885202407837 3.0
0 0.00419110059738159 4.0
0 0.00250375270843506 5.0
0 0.0020715594291687 6.0
0 -0.00529903173446655 7.0
42.488773183 38.6152381896973 0.0
0 0.00836420059204102 1.0
0 -0.00419712066650391 2.0
0 0.00586044788360596 3.0
0 0.0160481929779053 4.0
0 0.0162471532821655 5.0
0 0.00226736068725586 6.0
0 0.000960052013397217 7.0
34.702233392 39.2018165588379 0.0
0 0.00722146034240723 1.0
0 0.0153907537460327 2.0
0 0.0345809459686279 3.0
0 -0.000850021839141846 4.0
0 -0.00449872016906738 5.0
0 0.00279438495635986 6.0
0 -0.00392633676528931 7.0
41.831412446 39.3694381713867 0.0
0 0.0202068090438843 1.0
0 0.00735104084014893 2.0
0 -0.01542729139328 3.0
0 0.00598728656768799 4.0
0 0.000746250152587891 5.0
0 0.00259238481521606 6.0
0 -0.00198984146118164 7.0
45.908850318 38.7893257141113 0.0
0 0.0128718614578247 1.0
0 0.0439763069152832 2.0
0 -0.053998589515686 3.0
0 -0.0368396043777466 4.0
0 -0.0414500832557678 5.0
0 0.00407934188842773 6.0
0 -0.00154948234558105 7.0
33.104401139 39.2357902526855 0.0
0 0.0355506539344788 1.0
0 0.0204304456710815 2.0
0 0.0460935235023499 3.0
0 -0.00619632005691528 4.0
0 -0.0127991437911987 5.0
0 0.00329005718231201 6.0
0 -0.00491511821746826 7.0
35.835170704 38.656005859375 0.0
0 -0.0135376453399658 1.0
0 0.0272335410118103 2.0
0 0.018695592880249 3.0
0 -0.0218937397003174 4.0
0 -0.0219389796257019 5.0
0 0.00600516796112061 6.0
0 0.00515711307525635 7.0
45.844507362 38.3923797607422 0.0
0 0.00346899032592773 1.0
0 -0.0049707293510437 2.0
0 -0.0142909288406372 3.0
0 0.0028308629989624 4.0
0 0.00224250555038452 5.0
0 -0.00219684839248657 6.0
0 -0.00600039958953857 7.0
31.875262674 38.9218482971191 0.0
0 0.00548732280731201 1.0
0 0.03155916929245 2.0
0 0.0234208106994629 3.0
0 0.00218135118484497 4.0
0 0.000923514366149902 5.0
0 0.000942528247833252 6.0
0 -0.00415825843811035 7.0
35.888104099 38.465877532959 0.0
0 -0.00457191467285156 1.0
0 0.0129864811897278 2.0
0 0.0154435038566589 3.0
0 0.00746810436248779 4.0
0 0.00840973854064941 5.0
0 0.000717759132385254 6.0
0 -0.000674307346343994 7.0
35.733201066 38.5028800964355 0.0
0 0.00586575269699097 1.0
0 0.0072481632232666 2.0
0 0.0173937678337097 3.0
0 0.0100708603858948 4.0
0 0.00668007135391235 5.0
0 0.00180494785308838 6.0
0 -0.00377106666564941 7.0
43.710725429 38.3162002563477 0.0
0 -0.00887012481689453 1.0
0 -0.00385814905166626 2.0
0 -0.00318706035614014 3.0
0 -0.0185797214508057 4.0
0 -0.0196681022644043 5.0
0 0.000224053859710693 6.0
0 -0.00327861309051514 7.0
34.934843039 38.8348121643066 0.0
0 -0.00161290168762207 1.0
0 0.0159087181091309 2.0
0 0.0234295129776001 3.0
0 0.0048181414604187 4.0
0 0.00403618812561035 5.0
0 0.00204634666442871 6.0
0 -0.00230425596237183 7.0
38.70218214 38.7474479675293 0.0
0 -0.00108134746551514 1.0
0 0.0091174840927124 2.0
0 -0.00267726182937622 3.0
0 0.00788772106170654 4.0
0 0.00635135173797607 5.0
0 0.00159680843353271 6.0
0 0.0013083815574646 7.0
39.146786058 39.0480194091797 0.0
0 -0.0114285945892334 1.0
0 0.00788998603820801 2.0
0 -0.0154602527618408 3.0
0 0.0109805464744568 4.0
0 0.00477433204650879 5.0
0 0.00266498327255249 6.0
0 0.000880956649780273 7.0
37.956003494 38.3549499511719 0.0
0 0.00789499282836914 1.0
0 0.00629401206970215 2.0
0 -0.0352148413658142 3.0
0 -0.0112420320510864 4.0
0 -0.0142526030540466 5.0
0 0.00200521945953369 6.0
0 -0.00365465879440308 7.0
35.896232558 39.0284271240234 0.0
0 0.00879734754562378 1.0
0 0.0454597473144531 2.0
0 0.00259673595428467 3.0
0 -0.0103989839553833 4.0
0 -0.0144324898719788 5.0
0 0.00407028198242188 6.0
0 -0.000433206558227539 7.0
37.377916966 39.028507232666 0.0
0 0.0277334451675415 1.0
0 0.103057205677032 2.0
0 0.0302638411521912 3.0
0 -0.0119578838348389 4.0
0 -0.0142959356307983 5.0
0 0.00127959251403809 6.0
0 -0.00192588567733765 7.0
36.698956497 38.7372550964355 0.0
0 -0.0109740495681763 1.0
0 0.0191500186920166 2.0
0 -0.0165144205093384 3.0
0 0.0105429291725159 4.0
0 0.00730335712432861 5.0
0 0.00290632247924805 6.0
0 0.00153911113739014 7.0
41.903925455 38.9670372009277 0.0
0 0.0159850120544434 1.0
0 0.0163652896881104 2.0
0 0.0204038023948669 3.0
0 0.00393366813659668 4.0
0 -0.00349020957946777 5.0
0 0.00331735610961914 6.0
0 -0.00523614883422852 7.0
38.934024851 38.2444496154785 0.0
0 0.00585168600082397 1.0
0 0.0038304328918457 2.0
0 -0.00481849908828735 3.0
0 0.0178231000900269 4.0
0 0.0161932110786438 5.0
0 0.00253915786743164 6.0
0 0.00152045488357544 7.0
40.406325889 38.474781036377 0.0
0 -0.00402039289474487 1.0
0 0.0110981464385986 2.0
0 -0.0641270279884338 3.0
0 -0.035135805606842 4.0
0 -0.0373694896697998 5.0
0 0.00301212072372437 6.0
0 -0.00229912996292114 7.0
39.410747699 38.5370864868164 0.0
0 -0.00339770317077637 1.0
0 0.020703911781311 2.0
0 0.00961828231811523 3.0
0 0.00251716375350952 4.0
0 0.00325173139572144 5.0
0 0.00136208534240723 6.0
0 -0.0024869441986084 7.0
};
\addlegendentry{$R^2$=0.988}
\end{axis}
\end{tikzpicture}

}}
    \subfloat[Actual vs predicted edge flows.] 
    {\label{fig:results_dummy_base_f}\resizebox{\figurewidth}{\figureheight}{% This file was created with tikzplotlib v0.10.1.
\begin{tikzpicture}

\definecolor{darkgray176}{RGB}{176,176,176}
\definecolor{lightgray204}{RGB}{204,204,204}

\begin{axis}[
colorbar,
colorbar style={ylabel={edge id}},
colormap={mymap}{[1pt]
 rgb(0pt)=(0.12156862745098,0.466666666666667,0.705882352941177);
  rgb(1pt)=(1,0.498039215686275,0.0549019607843137);
  rgb(2pt)=(0.172549019607843,0.627450980392157,0.172549019607843);
  rgb(3pt)=(0.83921568627451,0.152941176470588,0.156862745098039);
  rgb(4pt)=(0.580392156862745,0.403921568627451,0.741176470588235);
  rgb(5pt)=(0.549019607843137,0.337254901960784,0.294117647058824);
  rgb(6pt)=(0.890196078431372,0.466666666666667,0.76078431372549);
  rgb(7pt)=(0.498039215686275,0.498039215686275,0.498039215686275);
  rgb(8pt)=(0.737254901960784,0.741176470588235,0.133333333333333);
  rgb(9pt)=(0.0901960784313725,0.745098039215686,0.811764705882353)
},
legend cell align={left},
legend style={
  fill opacity=0.8,
  draw opacity=1,
  text opacity=1,
  at={(0.03,0.97)},
  anchor=north west,
  draw=lightgray204
},
point meta max=7,
point meta min=0,
tick align=outside,
tick pos=left,
title={ye true - ye pred},
x grid style={darkgray176},
xlabel={ye true},
xmajorgrids,
xmin=-6.176293119925, xmax=50.358667459425,
xtick style={color=black},
y grid style={darkgray176},
ylabel={ye pred},
ymajorgrids,
ymin=-1.33763232827187, ymax=23.8523469269276,
ytick style={color=black}
]
\addplot [
  colormap={mymap}{[1pt]
 rgb(0pt)=(0.12156862745098,0.466666666666667,0.705882352941177);
  rgb(1pt)=(1,0.498039215686275,0.0549019607843137);
  rgb(2pt)=(0.172549019607843,0.627450980392157,0.172549019607843);
  rgb(3pt)=(0.83921568627451,0.152941176470588,0.156862745098039);
  rgb(4pt)=(0.580392156862745,0.403921568627451,0.741176470588235);
  rgb(5pt)=(0.549019607843137,0.337254901960784,0.294117647058824);
  rgb(6pt)=(0.890196078431372,0.466666666666667,0.76078431372549);
  rgb(7pt)=(0.498039215686275,0.498039215686275,0.498039215686275);
  rgb(8pt)=(0.737254901960784,0.741176470588235,0.133333333333333);
  rgb(9pt)=(0.0901960784313725,0.745098039215686,0.811764705882353)
},
  only marks,
  scatter,
  scatter src=explicit
]
table [x=x, y=y, meta=colordata]{%
x  y  colordata
41.603277614 0.18362183868885 0.0
28.537755408 1.66803848743439 1.0
13.065522207 1.67883598804474 2.0
0 0.169846832752228 3.0
13.065522207 0.412138372659683 4.0
28.537755408 0.453680574893951 5.0
41.603277614 0.780018985271454 6.0
41.603277614 17.8501033782959 7.0
37.07638211 0.798396944999695 0.0
22.707924148 0.537246406078339 1.0
14.368457962 0.584073185920715 2.0
0 0.244699701666832 3.0
14.368457962 0.434437125921249 4.0
22.707924148 0.473161071538925 5.0
37.07638211 0.503022789955139 6.0
37.07638211 0.500305235385895 7.0
39.245039256 0.144021779298782 0.0
23.531643333 1.72768700122833 1.0
15.713395923 1.74414300918579 2.0
0 0.184231892228127 3.0
15.713395923 0.44206565618515 4.0
23.531643333 0.463650107383728 5.0
39.245039256 1.11973834037781 6.0
39.245039256 22.707347869873 7.0
33.40222464 0.154635012149811 0.0
21.391931512 1.16647207736969 1.0
12.010293129 1.18021678924561 2.0
0 0.213556349277496 3.0
12.010293129 0.440131068229675 4.0
21.391931512 0.483441412448883 5.0
33.40222464 0.526355564594269 6.0
33.40222464 13.7839498519897 7.0
40.177942425 0.844456911087036 0.0
28.641810021 0.538357973098755 1.0
11.536132404 0.575416743755341 2.0
0 0.234241291880608 3.0
11.536132404 0.430620700120926 4.0
28.641810021 0.464340895414352 5.0
40.177942425 0.520999729633331 6.0
40.177942425 0.431796610355377 7.0
33.661272263 0.764284193515778 0.0
23.801018924 0.511304080486298 1.0
9.8602533385 0.543007612228394 2.0
0 0.215575829148293 3.0
9.8602533385 0.435438483953476 4.0
23.801018924 0.468720197677612 5.0
33.661272263 0.259720057249069 6.0
33.661272263 0.395147949457169 7.0
41.918555681 0.77016544342041 0.0
27.391600177 0.492178738117218 1.0
14.526955504 0.537823259830475 2.0
0 0.21575616300106 3.0
14.526955504 0.450648248195648 4.0
27.391600177 0.48611044883728 5.0
41.918555681 0.300421893596649 6.0
41.918555681 0.500450730323792 7.0
44.067434787 0.818282723426819 0.0
29.373915823 0.474608778953552 1.0
14.693518964 0.531799495220184 2.0
0 0.226863473653793 3.0
14.693518964 0.452735751867294 4.0
29.373915823 0.491840779781342 5.0
44.067434787 0.558213770389557 6.0
44.067434787 0.527023255825043 7.0
31.762979365 0.800955593585968 0.0
21.846398167 0.460764288902283 1.0
9.9165811975 0.511068344116211 2.0
0 0.220168948173523 3.0
9.9165811975 0.442803055047989 4.0
21.846398167 0.489266276359558 5.0
31.762979365 0.530630528926849 6.0
31.762979365 0.514157295227051 7.0
41.030120159 0.229777932167053 0.0
27.86612118 0.59070760011673 1.0
13.163998979 0.619265496730804 2.0
0 0.147530913352966 3.0
13.163998979 0.434190273284912 4.0
27.86612118 0.449956715106964 5.0
41.030120159 -0.0576869435608387 6.0
41.030120159 -0.094196654856205 7.0
40.055714877 0.605728924274445 0.0
26.116660402 0.49225977063179 1.0
13.939054474 0.524616122245789 2.0
0 0.205983638763428 3.0
13.939054474 0.436747550964355 4.0
26.116660402 0.475488305091858 5.0
40.055714877 0.312271147966385 6.0
40.055714877 0.161313235759735 7.0
40.714244155 0.830942988395691 0.0
29.532761783 0.52602881193161 1.0
11.181482371 0.560135900974274 2.0
0 0.23785687983036 3.0
11.181482371 0.436676442623138 4.0
29.532761783 0.484379798173904 5.0
40.714244155 0.546932458877563 6.0
40.714244155 0.523251533508301 7.0
45.139727698 0.794304311275482 0.0
31.047422979 0.495071738958359 1.0
14.092304718 0.527405142784119 2.0
0 0.224304437637329 3.0
14.092304718 0.455165594816208 4.0
31.047422979 0.492412984371185 5.0
45.139727698 0.279202252626419 6.0
45.139727698 0.409322202205658 7.0
40.583613059 0.175717517733574 0.0
28.007694947 0.327290862798691 1.0
12.575918112 0.347913056612015 2.0
0 0.204129546880722 3.0
12.575918112 0.411338806152344 4.0
28.007694947 0.453929126262665 5.0
40.583613059 0.333184242248535 6.0
40.583613059 0.589721024036407 7.0
35.441291819 0.683855414390564 0.0
26.125304789 0.51181834936142 1.0
9.3159870299 0.53029328584671 2.0
0 0.197433292865753 3.0
9.3159870299 0.402535080909729 4.0
26.125304789 0.447339415550232 5.0
35.441291819 0.247122347354889 6.0
35.441291819 0.311066895723343 7.0
36.08841077 0.306239098310471 0.0
22.250971448 0.4757981300354 1.0
13.837439322 0.507725119590759 2.0
0 0.21941702067852 3.0
13.837439322 0.441788822412491 4.0
22.250971448 0.50071507692337 5.0
36.08841077 0.0430356860160828 6.0
36.08841077 0.0698571056127548 7.0
40.494343306 0.791692972183228 0.0
29.628817779 0.487594813108444 1.0
10.865525528 0.532764315605164 2.0
0 0.229744613170624 3.0
10.865525528 0.465688437223434 4.0
29.628817779 0.505403399467468 5.0
40.494343306 0.337573498487473 6.0
40.494343306 0.533054888248444 7.0
44.400196583 0.805976390838623 0.0
30.425150814 0.395661115646362 1.0
13.975045769 0.430603176355362 2.0
-1.1363941574 0.214308008551598 3.0
12.838651611 0.450592577457428 4.0
31.561544972 0.48526269197464 5.0
44.400196583 0.561578810214996 6.0
44.400196583 0.482012271881104 7.0
40.197028842 0.716627299785614 0.0
28.568150674 0.486696600914001 1.0
11.628878168 0.520955622196198 2.0
0 0.21579971909523 3.0
11.628878168 0.431205600500107 4.0
28.568150674 0.48511803150177 5.0
40.197028842 0.216360852122307 6.0
40.197028842 0.455694228410721 7.0
39.931867517 0.820916175842285 0.0
24.727856907 0.496078073978424 1.0
15.20401061 0.538481950759888 2.0
0 0.231600284576416 3.0
15.20401061 0.459666788578033 4.0
24.727856907 0.500480115413666 5.0
39.931867517 0.292605042457581 6.0
39.931867517 0.477511912584305 7.0
37.28322745 0.797858595848083 0.0
27.425754281 0.483238518238068 1.0
9.8574731699 0.543225765228271 2.0
0 0.230617880821228 3.0
9.8574731699 0.464555144309998 4.0
27.425754281 0.497691094875336 5.0
37.28322745 0.524982571601868 6.0
37.28322745 0.459729492664337 7.0
37.429390014 0.400299191474915 0.0
22.132902814 0.515194952487946 1.0
15.2964872 0.550267934799194 2.0
0 0.201122522354126 3.0
15.2964872 0.43369522690773 4.0
22.132902814 0.474303901195526 5.0
37.429390014 0.018711406737566 6.0
37.429390014 0.0179443359375 7.0
39.499146179 0.812548875808716 0.0
30.151299195 0.509443402290344 1.0
9.3478469839 0.549915850162506 2.0
0 0.240803509950638 3.0
9.3478469839 0.452278822660446 4.0
30.151299195 0.490095019340515 5.0
39.499146179 0.392056614160538 6.0
39.499146179 0.505513727664948 7.0
35.661352504 0.603023767471313 0.0
26.617773772 0.125488445162773 1.0
9.0435787317 0.139381006360054 2.0
0 0.240266904234886 3.0
9.0435787317 0.396227955818176 4.0
26.617773772 0.465910911560059 5.0
35.661352504 4.54981899261475 6.0
35.661352504 0.126817345619202 7.0
44.903668391 0.810700595378876 0.0
28.406134318 0.384008765220642 1.0
16.497534073 0.41478756070137 2.0
-2.2866598043 0.21231085062027 3.0
14.210874268 0.445897072553635 4.0
30.692794122 0.496664673089981 5.0
44.903668391 0.515478253364563 6.0
44.903668391 0.495566487312317 7.0
34.118473664 0.267246276140213 0.0
24.847237913 0.362864077091217 1.0
9.271235751 0.400662004947662 2.0
0 0.287841975688934 3.0
9.271235751 0.467766225337982 4.0
24.847237913 0.537852227687836 5.0
34.118473664 0.0151599179953337 6.0
34.118473664 0.234499678015709 7.0
36.466593361 0.0886330008506775 0.0
24.924999426 1.26627230644226 1.0
11.541593934 1.2690771818161 2.0
0 0.180928602814674 3.0
11.541593934 0.431752681732178 4.0
24.924999426 0.466692566871643 5.0
36.466593361 0.895050227642059 6.0
36.466593361 13.2778911590576 7.0
34.994247693 0.732260346412659 0.0
26.099734292 0.488581299781799 1.0
8.8945134012 0.544753968715668 2.0
0 0.228192523121834 3.0
8.8945134012 0.450730562210083 4.0
26.099734292 0.516707539558411 5.0
34.994247693 0.48125809431076 6.0
34.994247693 0.607306361198425 7.0
39.553868661 0.807461321353912 0.0
27.422096073 0.466728627681732 1.0
12.131772588 0.52154940366745 2.0
0 0.216750770807266 3.0
12.131772588 0.447430670261383 4.0
27.422096073 0.493136942386627 5.0
39.553868661 0.389254927635193 6.0
39.553868661 0.561555624008179 7.0
43.158887007 0.848639488220215 0.0
28.546889082 0.479268044233322 1.0
14.611997925 0.538083851337433 2.0
0 0.231367483735085 3.0
14.611997925 0.454253047704697 4.0
28.546889082 0.480862885713577 5.0
43.158887007 0.545655906200409 6.0
43.158887007 0.518004715442657 7.0
36.693891456 0.553474247455597 0.0
24.873430755 0.481561362743378 1.0
11.820460701 0.51918613910675 2.0
0 0.183027699589729 3.0
11.820460701 0.427139818668365 4.0
24.873430755 0.459326833486557 5.0
36.693891456 0.213233962655067 6.0
36.693891456 0.152907565236092 7.0
37.419072323 0.809681236743927 0.0
26.218281513 0.483605414628983 1.0
11.20079081 0.542487561702728 2.0
0 0.224215656518936 3.0
11.20079081 0.440650761127472 4.0
26.218281513 0.484191060066223 5.0
37.419072323 0.518257737159729 6.0
37.419072323 0.508003413677216 7.0
42.391154887 0.257783174514771 0.0
28.17202345 0.767748475074768 1.0
14.219131437 0.80058491230011 2.0
0 0.16021528840065 3.0
14.219131437 0.426328301429749 4.0
28.17202345 0.454959571361542 5.0
42.391154887 0.177222549915314 6.0
42.391154887 -0.171856135129929 7.0
39.976124969 0.155863389372826 0.0
30.003187585 1.05664110183716 1.0
9.9729373839 1.05924808979034 2.0
0 0.146673485636711 3.0
9.9729373839 0.428589582443237 4.0
30.003187585 0.44323518872261 5.0
39.976124969 -0.0813200622797012 6.0
39.976124969 1.89679765701294 7.0
41.193094756 0.8075270652771 0.0
26.04533013 0.47657573223114 1.0
15.147764626 0.531694948673248 2.0
0 0.221025288105011 3.0
15.147764626 0.457021355628967 4.0
26.04533013 0.4905945956707 5.0
41.193094756 0.390842884778976 6.0
41.193094756 0.500106513500214 7.0
39.030467364 0.736577272415161 0.0
28.666995108 0.409848719835281 1.0
10.363472256 0.444745510816574 2.0
0 0.254743278026581 3.0
10.363472256 0.442532122135162 4.0
28.666995108 0.479927569627762 5.0
39.030467364 0.158288449048996 6.0
39.030467364 0.366771429777145 7.0
39.254434013 0.679574370384216 0.0
24.077904618 0.456481516361237 1.0
15.176529395 0.497190952301025 2.0
0 0.216706901788712 3.0
15.176529395 0.461384415626526 4.0
24.077904618 0.502110123634338 5.0
39.254434013 0.305881977081299 6.0
39.254434013 0.565269470214844 7.0
41.474829476 0.725732564926147 0.0
25.835349121 0.468871086835861 1.0
15.639480356 0.519743978977203 2.0
0 0.191539064049721 3.0
15.639480356 0.435394406318665 4.0
25.835349121 0.472873091697693 5.0
41.474829476 0.289753705263138 6.0
41.474829476 0.559043526649475 7.0
36.065024548 0.793076515197754 0.0
27.353858358 0.454936861991882 1.0
8.7111661904 0.501248955726624 2.0
0 0.226842939853668 3.0
8.7111661904 0.451495110988617 4.0
27.353858358 0.501835227012634 5.0
36.065024548 0.528780281543732 6.0
36.065024548 0.53072053194046 7.0
33.547657937 0.185216248035431 0.0
23.678365531 0.427676320075989 1.0
9.8692924059 0.444449096918106 2.0
0 0.166136011481285 3.0
9.8692924059 0.392721325159073 4.0
23.678365531 0.426126092672348 5.0
33.547657937 0.17101676762104 6.0
33.547657937 0.667055249214172 7.0
35.860060933 0.845887005329132 0.0
22.637636259 0.522960662841797 1.0
13.222424674 0.565746366977692 2.0
0 0.260444730520248 3.0
13.222424674 0.444882959127426 4.0
22.637636259 0.496082723140717 5.0
35.860060933 0.389720112085342 6.0
35.860060933 0.427040904760361 7.0
41.446935888 0.34843835234642 0.0
32.532039789 0.122403137385845 1.0
8.9148960991 0.138558104634285 2.0
0 0.226195082068443 3.0
8.9148960991 0.394131362438202 4.0
32.532039789 0.450389623641968 5.0
41.446935888 0.0849817842245102 6.0
41.446935888 0.253898501396179 7.0
39.09863661 0.574728846549988 0.0
24.812113349 0.430166661739349 1.0
14.286523261 0.471245467662811 2.0
0 0.164977610111237 3.0
14.286523261 0.434725403785706 4.0
24.812113349 0.459047257900238 5.0
39.09863661 0.166856959462166 6.0
39.09863661 0.333010584115982 7.0
37.209660367 0.686252474784851 0.0
25.112424491 0.494599163532257 1.0
12.097235877 0.529250800609589 2.0
0 0.205128222703934 3.0
12.097235877 0.429956793785095 4.0
25.112424491 0.454052209854126 5.0
37.209660367 0.307500422000885 6.0
37.209660367 0.314246386289597 7.0
37.318344805 0.804287254810333 0.0
25.306300138 0.485686361789703 1.0
12.012044667 0.541722774505615 2.0
0 0.251310735940933 3.0
12.012044667 0.460202783346176 4.0
25.306300138 0.505881547927856 5.0
37.318344805 0.542162656784058 6.0
37.318344805 0.517050623893738 7.0
37.542238634 0.633315086364746 0.0
26.460617918 0.447912096977234 1.0
11.081620716 0.485553950071335 2.0
0 0.178163453936577 3.0
11.081620716 0.457998186349869 4.0
26.460617918 0.462548673152924 5.0
37.542238634 0.275898844003677 6.0
37.542238634 0.523411750793457 7.0
35.642638424 0.810593724250793 0.0
20.426904297 0.526028275489807 1.0
15.215734127 0.5652214884758 2.0
0 0.247393652796745 3.0
15.215734127 0.451821058988571 4.0
20.426904297 0.476411789655685 5.0
35.642638424 0.524780631065369 6.0
35.642638424 0.530322670936584 7.0
47.4260719 0.679734826087952 0.0
31.140200753 0.560370206832886 1.0
16.285871146 0.594865441322327 2.0
-1.8634699235 0.251972764730453 3.0
14.422401223 0.435448974370956 4.0
33.003670677 0.475022196769714 5.0
47.4260719 0.325695693492889 6.0
47.4260719 0.29996195435524 7.0
42.664227047 0.860981822013855 0.0
29.252186505 0.559815347194672 1.0
13.412040542 0.602366387844086 2.0
0 0.266680747270584 3.0
13.412040542 0.44575434923172 4.0
29.252186505 0.485280245542526 5.0
42.664227047 0.508468151092529 6.0
42.664227047 0.44195294380188 7.0
37.881981315 0.833380520343781 0.0
24.58462335 0.448726058006287 1.0
13.297357965 0.493956297636032 2.0
0 0.223773553967476 3.0
13.297357965 0.458041459321976 4.0
24.58462335 0.496246784925461 5.0
37.881981315 0.561169564723969 6.0
37.881981315 0.511483311653137 7.0
44.031482397 0.162285029888153 0.0
28.706496003 0.44884005188942 1.0
15.324986394 0.474067270755768 2.0
0 0.158654078841209 3.0
15.324986394 0.439268618822098 4.0
28.706496003 0.477900832891464 5.0
44.031482397 0.255633234977722 6.0
44.031482397 0.0328879728913307 7.0
40.215687867 0.828767418861389 0.0
29.785471623 0.530020475387573 1.0
10.430216245 0.557447910308838 2.0
0 0.222647652029991 3.0
10.430216245 0.425323098897934 4.0
29.785471623 0.461571216583252 5.0
40.215687867 0.340684413909912 6.0
40.215687867 0.291825175285339 7.0
35.401137883 0.79507964849472 0.0
25.079356252 0.463181585073471 1.0
10.32178163 0.51850962638855 2.0
0 0.213682353496552 3.0
10.32178163 0.4574314057827 4.0
25.079356252 0.47861659526825 5.0
35.401137883 0.275768011808395 6.0
35.401137883 0.565602242946625 7.0
40.23529515 0.789519309997559 0.0
24.942469333 0.501329481601715 1.0
15.292825817 0.557558357715607 2.0
0 0.200252652168274 3.0
15.292825817 0.427802085876465 4.0
24.942469333 0.449140250682831 5.0
40.23529515 0.530221343040466 6.0
40.23529515 0.508595824241638 7.0
38.451822651 0.855121433734894 0.0
26.67825551 0.503783524036407 1.0
11.773567141 0.543715238571167 2.0
0 0.218413278460503 3.0
11.773567141 0.4441197514534 4.0
26.67825551 0.473533779382706 5.0
38.451822651 0.351095050573349 6.0
38.451822651 0.445495098829269 7.0
41.356859392 0.779768109321594 0.0
27.293812006 0.520446956157684 1.0
14.063047386 0.558641016483307 2.0
0 0.239295333623886 3.0
14.063047386 0.450157880783081 4.0
27.293812006 0.482391387224197 5.0
41.356859392 0.503314435482025 6.0
41.356859392 0.519955694675446 7.0
44.45602451 0.798856616020203 0.0
28.653738823 0.469792872667313 1.0
15.802285687 0.524374723434448 2.0
0 0.219654411077499 3.0
15.802285687 0.456302762031555 4.0
28.653738823 0.49110534787178 5.0
44.45602451 0.36150535941124 6.0
44.45602451 0.473621129989624 7.0
36.344631329 0.802997171878815 0.0
24.004198046 0.504283487796783 1.0
12.340433282 0.52661919593811 2.0
0 0.211866229772568 3.0
12.340433282 0.431315869092941 4.0
24.004198046 0.475714683532715 5.0
36.344631329 0.370603084564209 6.0
36.344631329 0.29797887802124 7.0
35.989587139 0.811484813690186 0.0
25.842707382 0.490815788507462 1.0
10.146879757 0.532808542251587 2.0
0 0.228138387203217 3.0
10.146879757 0.45358419418335 4.0
25.842707382 0.488609611988068 5.0
35.989587139 0.262153208255768 6.0
35.989587139 0.486478298902512 7.0
38.453275767 0.708113312721252 0.0
22.650197879 0.476131916046143 1.0
15.803077888 0.51755279302597 2.0
0 0.207930564880371 3.0
15.803077888 0.441082447767258 4.0
22.650197879 0.480470240116119 5.0
38.453275767 0.277470767498016 6.0
38.453275767 0.501384317874908 7.0
36.388812201 0.785589814186096 0.0
20.539942681 0.433730512857437 1.0
15.84886952 0.474904119968414 2.0
0 0.230642005801201 3.0
15.84886952 0.455040693283081 4.0
20.539942681 0.512764811515808 5.0
36.388812201 0.547524690628052 6.0
36.388812201 0.513620257377625 7.0
45.540732617 0.795330166816711 0.0
31.547017021 0.342924922704697 1.0
13.993715596 0.381518423557281 2.0
0 0.215660035610199 3.0
13.993715596 0.458406984806061 4.0
31.547017021 0.487756162881851 5.0
45.540732617 0.579283654689789 6.0
45.540732617 0.45035058259964 7.0
34.414627117 0.144705131649971 0.0
20.463252495 1.84070909023285 1.0
13.951374622 1.875736951828 2.0
0 0.108744889497757 3.0
13.951374622 0.411313712596893 4.0
20.463252495 0.421262860298157 5.0
34.414627117 -0.192633271217346 6.0
34.414627117 17.2122135162354 7.0
39.317888978 0.72857391834259 0.0
28.187840428 0.488027274608612 1.0
11.13004855 0.521728932857513 2.0
0 0.237912744283676 3.0
11.13004855 0.457086831331253 4.0
28.187840428 0.481893122196198 5.0
39.317888978 0.251074910163879 6.0
39.317888978 0.467821776866913 7.0
36.969505628 0.68063485622406 0.0
22.432585624 0.452622324228287 1.0
14.536920003 0.490714818239212 2.0
0 0.203278660774231 3.0
14.536920003 0.436907947063446 4.0
22.432585624 0.475482374429703 5.0
36.969505628 0.233877420425415 6.0
36.969505628 0.485995829105377 7.0
44.923617033 0.557306230068207 0.0
29.042831601 0.43771356344223 1.0
15.880785432 0.465666651725769 2.0
0 0.222870618104935 3.0
15.880785432 0.434020638465881 4.0
29.042831601 0.492365062236786 5.0
44.923617033 0.232635527849197 6.0
44.923617033 0.259026378393173 7.0
37.258159228 0.768329739570618 0.0
27.142741137 0.365099221467972 1.0
10.115418091 0.387372672557831 2.0
0 0.228445708751678 3.0
10.115418091 0.404116153717041 4.0
27.142741137 0.460410088300705 5.0
37.258159228 -0.000946193933486938 6.0
37.258159228 0.52266651391983 7.0
47.488576672 0.152595415711403 0.0
33.063951806 0.442923456430435 1.0
14.424624867 0.478711307048798 2.0
0 0.210755586624146 3.0
14.424624867 0.437796413898468 4.0
33.063951806 0.470580518245697 5.0
47.488576672 0.30705001950264 6.0
47.488576672 0.732701599597931 7.0
42.958399331 0.859491169452667 0.0
31.220988276 0.484890818595886 1.0
11.737411055 0.540959537029266 2.0
0 0.22070375084877 3.0
11.737411055 0.43827611207962 4.0
31.220988276 0.48798406124115 5.0
42.958399331 0.43698126077652 6.0
42.958399331 0.475166112184525 7.0
41.161203287 0.782101571559906 0.0
28.736175493 0.533224105834961 1.0
12.425027794 0.562227368354797 2.0
0 0.223516747355461 3.0
12.425027794 0.424655824899673 4.0
28.736175493 0.46190994977951 5.0
41.161203287 0.614599823951721 6.0
41.161203287 0.479119539260864 7.0
40.466838645 0.233402594923973 0.0
27.990157761 0.48283714056015 1.0
12.476680884 0.517958521842957 2.0
0 0.163876846432686 3.0
12.476680884 0.437112897634506 4.0
27.990157761 0.463956981897354 5.0
40.466838645 0.191814824938774 6.0
40.466838645 -0.0138135775923729 7.0
33.757389454 0.615502893924713 0.0
21.260951942 0.372696757316589 1.0
12.496437512 0.377712935209274 2.0
0 0.201444372534752 3.0
12.496437512 0.414563626050949 4.0
21.260951942 0.447394013404846 5.0
33.757389454 -0.0158464070409536 6.0
33.757389454 0.335472166538239 7.0
43.437770592 0.789210200309753 0.0
28.513785933 0.527660489082336 1.0
14.923984658 0.559856295585632 2.0
-3.6065221845 0.23783078789711 3.0
11.317462474 0.434644103050232 4.0
32.120308118 0.479543030261993 5.0
43.437770592 0.429423332214355 6.0
43.437770592 0.448895871639252 7.0
39.308931201 0.781916677951813 0.0
29.281572846 0.466488122940063 1.0
10.027358355 0.515016257762909 2.0
0 0.209983885288239 3.0
10.027358355 0.457143664360046 4.0
29.281572846 0.490869373083115 5.0
39.308931201 0.261780679225922 6.0
39.308931201 0.467330455780029 7.0
40.600604276 0.770367622375488 0.0
30.253095292 0.468821167945862 1.0
10.347508984 0.506372272968292 2.0
0 0.181537300348282 3.0
10.347508984 0.434204757213593 4.0
30.253095292 0.467293858528137 5.0
40.600604276 0.441387176513672 6.0
40.600604276 0.543189108371735 7.0
31.520377836 0.806576192378998 0.0
22.250038738 0.500841498374939 1.0
9.270339098 0.546994805335999 2.0
0 0.22181124985218 3.0
9.270339098 0.4323570728302 4.0
22.250038738 0.491334110498428 5.0
31.520377836 0.388052403926849 6.0
31.520377836 0.481838166713715 7.0
39.969125591 0.802157282829285 0.0
26.836211143 0.49263808131218 1.0
13.132914448 0.51548957824707 2.0
0 0.209799587726593 3.0
13.132914448 0.404154777526855 4.0
26.836211143 0.452528893947601 5.0
39.969125591 0.246517613530159 6.0
39.969125591 0.355340123176575 7.0
30.912106182 0.721765518188477 0.0
20.724206455 0.492092698812485 1.0
10.187899727 0.543611109256744 2.0
0 0.196833938360214 3.0
10.187899727 0.434330224990845 4.0
20.724206455 0.477423161268234 5.0
30.912106182 0.364529460668564 6.0
30.912106182 0.614051938056946 7.0
45.180443811 0.795559942722321 0.0
30.231656324 0.537768244743347 1.0
14.948787487 0.57769101858139 2.0
0 0.219425022602081 3.0
14.948787487 0.424855619668961 4.0
30.231656324 0.462855398654938 5.0
45.180443811 0.270683288574219 6.0
45.180443811 0.286943197250366 7.0
34.99216559 0.83259391784668 0.0
21.1924894 0.51744669675827 1.0
13.79967619 0.551594913005829 2.0
0 0.229277148842812 3.0
13.79967619 0.432374060153961 4.0
21.1924894 0.476321309804916 5.0
34.99216559 0.360452473163605 6.0
34.99216559 0.292109996080399 7.0
40.933938659 0.759024143218994 0.0
31.89275394 0.456924259662628 1.0
9.0411847187 0.489495277404785 2.0
0 0.234382063150406 3.0
9.0411847187 0.436167657375336 4.0
31.89275394 0.487885892391205 5.0
40.933938659 0.306376665830612 6.0
40.933938659 0.338122367858887 7.0
40.317005952 0.796028435230255 0.0
30.547820945 0.486862927675247 1.0
9.7691850077 0.525702953338623 2.0
0 0.22910363972187 3.0
9.7691850077 0.461818605661392 4.0
30.547820945 0.490650057792664 5.0
40.317005952 0.259598672389984 6.0
40.317005952 0.449278295040131 7.0
40.010379617 0.725167810916901 0.0
24.888658184 0.475520610809326 1.0
15.121721433 0.520340502262115 2.0
0 0.203900307416916 3.0
15.121721433 0.452974617481232 4.0
24.888658184 0.459793925285339 5.0
40.010379617 0.301140874624252 6.0
40.010379617 0.509556353092194 7.0
38.685596488 0.545413076877594 0.0
28.65763434 0.484680414199829 1.0
10.027962148 0.512455344200134 2.0
0 0.15990948677063 3.0
10.027962148 0.430635631084442 4.0
28.65763434 0.44330245256424 5.0
38.685596488 0.166761636734009 6.0
38.685596488 0.0974187925457954 7.0
30.278486108 0.680630147457123 0.0
19.759275031 0.4747314453125 1.0
10.519211077 0.510950446128845 2.0
0 0.208611071109772 3.0
10.519211077 0.434152841567993 4.0
19.759275031 0.47933030128479 5.0
30.278486108 0.277019590139389 6.0
30.278486108 0.337114572525024 7.0
40.311093813 0.838933169841766 0.0
25.154058321 0.50713711977005 1.0
15.157035492 0.53829550743103 2.0
0 0.222964823246002 3.0
15.157035492 0.449454784393311 4.0
25.154058321 0.461470901966095 5.0
40.311093813 0.534602999687195 6.0
40.311093813 0.306107550859451 7.0
35.760290532 0.795566439628601 0.0
22.277274346 0.437686055898666 1.0
13.483016186 0.456868290901184 2.0
0 0.225329905748367 3.0
13.483016186 0.423708826303482 4.0
22.277274346 0.459177553653717 5.0
35.760290532 0.14213390648365 6.0
35.760290532 0.441093921661377 7.0
40.236813436 0.84432327747345 0.0
29.89918024 0.472468703985214 1.0
10.337633196 0.528862416744232 2.0
0 0.237433448433876 3.0
10.337633196 0.456857800483704 4.0
29.89918024 0.503780543804169 5.0
40.236813436 0.434519618749619 6.0
40.236813436 0.531514585018158 7.0
43.948095531 0.789321303367615 0.0
28.711232243 0.504738867282867 1.0
15.236863288 0.547475576400757 2.0
0 0.208419188857079 3.0
15.236863288 0.436082422733307 4.0
28.711232243 0.488769561052322 5.0
43.948095531 0.435896664857864 6.0
43.948095531 0.452348262071609 7.0
40.202612729 0.828105270862579 0.0
28.531548022 0.465203404426575 1.0
11.671064707 0.518478989601135 2.0
0 0.207637935876846 3.0
11.671064707 0.45008185505867 4.0
28.531548022 0.496331423521042 5.0
40.202612729 0.345711708068848 6.0
40.202612729 0.477390229701996 7.0
41.441227091 0.621005356311798 0.0
26.925574624 0.443998485803604 1.0
14.515652467 0.474034249782562 2.0
0 0.226450517773628 3.0
14.515652467 0.437115073204041 4.0
26.925574624 0.479772955179214 5.0
41.441227091 0.046924889087677 6.0
41.441227091 0.271219819784164 7.0
40.234798657 0.650882482528687 0.0
27.003828746 0.327398955821991 1.0
13.230969911 0.355285704135895 2.0
0 0.266498386859894 3.0
13.230969911 0.433842986822128 4.0
27.003828746 0.497165530920029 5.0
40.234798657 0.102080248296261 6.0
40.234798657 0.20906837284565 7.0
40.268402112 0.813858330249786 0.0
28.231212758 0.513734221458435 1.0
12.037189354 0.536538064479828 2.0
0 0.218942224979401 3.0
12.037189354 0.423304378986359 4.0
28.231212758 0.471041440963745 5.0
40.268402112 0.3616583943367 6.0
40.268402112 0.33258181810379 7.0
34.447402114 0.633421778678894 0.0
23.031509698 0.365182667970657 1.0
11.415892415 0.401213705539703 2.0
0 0.254134804010391 3.0
11.415892415 0.45295324921608 4.0
23.031509698 0.48541647195816 5.0
34.447402114 0.0253872461616993 6.0
34.447402114 0.25474601984024 7.0
40.416533901 0.863305509090424 0.0
29.448368969 0.516085147857666 1.0
10.968164932 0.543672204017639 2.0
0 0.209050863981247 3.0
10.968164932 0.429725050926208 4.0
29.448368969 0.470479547977448 5.0
40.416533901 0.374122381210327 6.0
40.416533901 0.391345858573914 7.0
40.837884543 0.663856685161591 0.0
27.69685337 0.480268776416779 1.0
13.141031172 0.517706096172333 2.0
0 0.211321860551834 3.0
13.141031172 0.443167448043823 4.0
27.69685337 0.46275007724762 5.0
40.837884543 0.317364066839218 6.0
40.837884543 0.268826901912689 7.0
39.823092444 0.824232161045074 0.0
29.374613061 0.557817697525024 1.0
10.448479383 0.587029635906219 2.0
0 0.244592979550362 3.0
10.448479383 0.412196606397629 4.0
29.374613061 0.480683118104935 5.0
39.823092444 0.595793902873993 6.0
39.823092444 0.427825629711151 7.0
40.41620498 0.614220798015594 0.0
30.830981709 0.488616585731506 1.0
9.585223271 0.523065090179443 2.0
0 0.201513364911079 3.0
9.585223271 0.432989567518234 4.0
30.830981709 0.452209651470184 5.0
40.41620498 0.302327334880829 6.0
40.41620498 0.125452041625977 7.0
33.388504976 0.811709880828857 0.0
23.687097805 0.476081937551498 1.0
9.7014071712 0.529811501502991 2.0
0 0.201718866825104 3.0
9.7014071712 0.448134273290634 4.0
23.687097805 0.462773442268372 5.0
33.388504976 0.33895868062973 6.0
33.388504976 0.460971146821976 7.0
42.251691394 0.554833352565765 0.0
30.360448061 0.453369140625 1.0
11.891243333 0.477365523576736 2.0
-1.4316411835 0.154209062457085 3.0
10.459602149 0.413379430770874 4.0
31.792089245 0.43540221452713 5.0
42.251691394 -0.00697393622249365 6.0
42.251691394 0.202454119920731 7.0
32.552772789 0.77423107624054 0.0
23.119967808 0.57982861995697 1.0
9.4328049818 0.610740184783936 2.0
0 0.22412434220314 3.0
9.4328049818 0.410817474126816 4.0
23.119967808 0.468638122081757 5.0
32.552772789 0.255223393440247 6.0
32.552772789 0.282040506601334 7.0
43.396109959 0.33837902545929 0.0
28.433148915 0.648148953914642 1.0
14.962961044 0.687121391296387 2.0
0 0.171254634857178 3.0
14.962961044 0.43771168589592 4.0
28.433148915 0.447233825922012 5.0
43.396109959 0.097889207303524 6.0
43.396109959 -0.100659817457199 7.0
31.629368094 0.817347586154938 0.0
21.596105546 0.438172817230225 1.0
10.033262548 0.47769957780838 2.0
0 0.20130243897438 3.0
10.033262548 0.448046833276749 4.0
21.596105546 0.491194278001785 5.0
31.629368094 0.29368868470192 6.0
31.629368094 0.517411112785339 7.0
37.779811495 0.360480278730392 0.0
25.908076324 0.499527901411057 1.0
11.87173517 0.533571779727936 2.0
0 0.201054617762566 3.0
11.87173517 0.447454243898392 4.0
25.908076324 0.490897119045258 5.0
37.779811495 0.00640070717781782 6.0
37.779811495 -0.01367262378335 7.0
34.24652714 0.864746689796448 0.0
24.424699608 0.462296605110168 1.0
9.8218275322 0.513426959514618 2.0
0 0.221890494227409 3.0
9.8218275322 0.450292438268661 4.0
24.424699608 0.488625884056091 5.0
34.24652714 0.548952043056488 6.0
34.24652714 0.519272208213806 7.0
40.003780762 0.579022347927094 0.0
30.808398459 0.439267694950104 1.0
9.1953823028 0.484471887350082 2.0
0 0.174132138490677 3.0
9.1953823028 0.443940937519073 4.0
30.808398459 0.469607055187225 5.0
40.003780762 0.233612209558487 6.0
40.003780762 0.551254510879517 7.0
31.970353396 0.745029509067535 0.0
22.618886495 0.454454064369202 1.0
9.3514669012 0.507682800292969 2.0
0 0.195197150111198 3.0
9.3514669012 0.453574597835541 4.0
22.618886495 0.47710308432579 5.0
31.970353396 0.287955611944199 6.0
31.970353396 0.571686148643494 7.0
37.935925517 0.695942640304565 0.0
23.719907188 0.427724689245224 1.0
14.21601833 0.480260848999023 2.0
0 0.19111455976963 3.0
14.21601833 0.450302481651306 4.0
23.719907188 0.491835057735443 5.0
37.935925517 0.244796127080917 6.0
37.935925517 0.536572694778442 7.0
37.4568261 0.622248947620392 0.0
24.360290248 0.445997267961502 1.0
13.096535852 0.474628478288651 2.0
0 0.181538850069046 3.0
13.096535852 0.437245428562164 4.0
24.360290248 0.470304071903229 5.0
37.4568261 0.255560845136642 6.0
37.4568261 0.32320311665535 7.0
34.938845635 0.839130938053131 0.0
24.698541768 0.524740874767303 1.0
10.240303867 0.55354917049408 2.0
0 0.226498663425446 3.0
10.240303867 0.437295198440552 4.0
24.698541768 0.469970852136612 5.0
34.938845635 0.5274458527565 6.0
34.938845635 0.377839386463165 7.0
33.644682189 0.137220337986946 0.0
23.878275232 0.721189916133881 1.0
9.7664069574 0.760638177394867 2.0
0 0.203536480665207 3.0
9.7664069574 0.442537099123001 4.0
23.878275232 0.474965572357178 5.0
33.644682189 8.11368227005005e-05 6.0
33.644682189 0.378574907779694 7.0
43.152629724 0.674295663833618 0.0
29.600563314 0.459979474544525 1.0
13.55206641 0.499686449766159 2.0
0 0.189341709017754 3.0
13.55206641 0.43625795841217 4.0
29.600563314 0.484194368124008 5.0
43.152629724 0.293773740530014 6.0
43.152629724 0.588226914405823 7.0
29.29034141 0.163494497537613 0.0
20.287635655 0.308476328849792 1.0
9.0027057545 0.341448158025742 2.0
0 0.250177025794983 3.0
9.0027057545 0.442974776029587 4.0
20.287635655 0.487383157014847 5.0
29.29034141 0.346538364887238 6.0
29.29034141 0.662787675857544 7.0
36.529945045 0.170982643961906 0.0
25.861121308 0.66765946149826 1.0
10.668823737 0.702960729598999 2.0
0 0.15674951672554 3.0
10.668823737 0.43869349360466 4.0
25.861121308 0.442617177963257 5.0
36.529945045 -0.0639661401510239 6.0
36.529945045 -0.0794261395931244 7.0
39.490028245 0.823695361614227 0.0
28.809011403 0.429608643054962 1.0
10.681016842 0.467456668615341 2.0
0 0.251990288496017 3.0
10.681016842 0.451483488082886 4.0
28.809011403 0.480885475873947 5.0
39.490028245 0.276479691267014 6.0
39.490028245 0.369283437728882 7.0
34.709507968 0.780969500541687 0.0
23.450878064 0.509256184101105 1.0
11.258629904 0.525171518325806 2.0
0 0.210477709770203 3.0
11.258629904 0.422484666109085 4.0
23.450878064 0.478887259960175 5.0
34.709507968 0.387906849384308 6.0
34.709507968 0.302719742059708 7.0
45.017599836 0.815544247627258 0.0
31.949025859 0.556151449680328 1.0
13.068573978 0.600121974945068 2.0
0 0.249904006719589 3.0
13.068573978 0.440150320529938 4.0
31.949025859 0.477317094802856 5.0
45.017599836 0.453605622053146 6.0
45.017599836 0.436097681522369 7.0
32.940669776 0.741158068180084 0.0
21.374565291 0.520636737346649 1.0
11.566104485 0.562661528587341 2.0
0 0.231569096446037 3.0
11.566104485 0.423524588346481 4.0
21.374565291 0.463613629341125 5.0
32.940669776 0.265047669410706 6.0
32.940669776 0.336740583181381 7.0
44.072190389 0.433577060699463 0.0
29.032860365 0.425351172685623 1.0
15.039330024 0.4416264295578 2.0
0 0.158616498112679 3.0
15.039330024 0.401291519403458 4.0
29.032860365 0.43481433391571 5.0
44.072190389 0.219403371214867 6.0
44.072190389 0.5691938996315 7.0
35.422022456 0.726730763912201 0.0
20.530264722 0.518710136413574 1.0
14.891757734 0.56212055683136 2.0
0 0.24020990729332 3.0
14.891757734 0.425267457962036 4.0
20.530264722 0.472503215074539 5.0
35.422022456 0.28703436255455 6.0
35.422022456 0.310002148151398 7.0
36.153594117 0.659077346324921 0.0
27.513199792 0.448379248380661 1.0
8.6403943249 0.48860490322113 2.0
0 0.178680658340454 3.0
8.6403943249 0.436225116252899 4.0
27.513199792 0.469920605421066 5.0
36.153594117 0.259342521429062 6.0
36.153594117 0.490491926670074 7.0
38.951377458 0.842035710811615 0.0
27.10341817 0.523775339126587 1.0
11.847959289 0.552973449230194 2.0
0 0.230052292346954 3.0
11.847959289 0.444388121366501 4.0
27.10341817 0.485168009996414 5.0
38.951377458 0.409061193466187 6.0
38.951377458 0.407182365655899 7.0
40.644172477 0.845954060554504 0.0
28.871927364 0.548332691192627 1.0
11.772245113 0.596139550209045 2.0
0 0.233447670936584 3.0
11.772245113 0.429199695587158 4.0
28.871927364 0.472139537334442 5.0
40.644172477 0.3404261469841 6.0
40.644172477 0.350879848003387 7.0
35.600531382 0.828681766986847 0.0
24.235795061 0.448939561843872 1.0
11.364736321 0.488180965185165 2.0
0 0.193133816123009 3.0
11.364736321 0.438678920269012 4.0
24.235795061 0.467833697795868 5.0
35.600531382 0.299473255872726 6.0
35.600531382 0.531875610351562 7.0
41.682033599 0.724063754081726 0.0
29.765882281 0.486572623252869 1.0
11.916151318 0.506341814994812 2.0
-0.12012993231 0.152846574783325 3.0
11.796021386 0.417903214693069 4.0
29.886012214 0.448732495307922 5.0
41.682033599 0.291404068470001 6.0
41.682033599 0.356805473566055 7.0
39.697348938 0.811676681041718 0.0
25.334104361 0.444702982902527 1.0
14.363244578 0.491698712110519 2.0
0 0.210645765066147 3.0
14.363244578 0.450691521167755 4.0
25.334104361 0.488625377416611 5.0
39.697348938 0.310435563325882 6.0
39.697348938 0.52720707654953 7.0
40.978668493 0.80098682641983 0.0
27.414183867 0.53301215171814 1.0
13.564484626 0.570398330688477 2.0
0 0.23868964612484 3.0
13.564484626 0.418851256370544 4.0
27.414183867 0.477587670087814 5.0
40.978668493 0.51816600561142 6.0
40.978668493 0.300620943307877 7.0
46.524483465 0.838250756263733 0.0
30.674387827 0.470500409603119 1.0
15.850095638 0.529142618179321 2.0
0 0.21652752161026 3.0
15.850095638 0.457629770040512 4.0
30.674387827 0.477099806070328 5.0
46.524483465 0.525684118270874 6.0
46.524483465 0.50344055891037 7.0
39.683969876 0.862537205219269 0.0
28.712254285 0.483742594718933 1.0
10.971715591 0.542938649654388 2.0
0 0.21887469291687 3.0
10.971715591 0.449684053659439 4.0
28.712254285 0.476190030574799 5.0
39.683969876 0.554523289203644 6.0
39.683969876 0.506298899650574 7.0
46.294692767 0.477879345417023 0.0
31.830997773 0.452083319425583 1.0
14.463694994 0.479299604892731 2.0
0 0.196213334798813 3.0
14.463694994 0.416516005992889 4.0
31.830997773 0.463115870952606 5.0
46.294692767 -0.00854126177728176 6.0
46.294692767 0.157146468758583 7.0
44.785541844 0.655936896800995 0.0
30.101539434 0.483851373195648 1.0
14.684002411 0.521312713623047 2.0
-1.0807278173 0.189014405012131 3.0
13.603274593 0.422543615102768 4.0
31.182267251 0.448022991418839 5.0
44.785541844 0.331685781478882 6.0
44.785541844 0.288005709648132 7.0
33.676333572 0.841765701770782 0.0
24.221820866 0.524360656738281 1.0
9.4545127063 0.549581289291382 2.0
0 0.222773998975754 3.0
9.4545127063 0.436137497425079 4.0
24.221820866 0.463169515132904 5.0
33.676333572 0.501551389694214 6.0
33.676333572 0.367675125598907 7.0
35.625618922 0.829233169555664 0.0
20.256511762 0.548307716846466 1.0
15.369107161 0.579243063926697 2.0
0 0.231296673417091 3.0
15.369107161 0.420522570610046 4.0
20.256511762 0.469740390777588 5.0
35.625618922 0.541749894618988 6.0
35.625618922 0.391774535179138 7.0
40.226157029 0.842509925365448 0.0
28.220684127 0.518953800201416 1.0
12.005472902 0.546702146530151 2.0
0 0.22388231754303 3.0
12.005472902 0.438953995704651 4.0
28.220684127 0.463092058897018 5.0
40.226157029 0.339834183454514 6.0
40.226157029 0.320722728967667 7.0
44.003195919 0.739822208881378 0.0
30.390659394 0.545289099216461 1.0
13.612536526 0.580045402050018 2.0
0 0.229024648666382 3.0
13.612536526 0.432973176240921 4.0
30.390659394 0.474433839321136 5.0
44.003195919 0.378883242607117 6.0
44.003195919 0.387751132249832 7.0
33.88878315 0.829002737998962 0.0
22.572006823 0.482835292816162 1.0
11.316776327 0.541327178478241 2.0
0 0.220925360918045 3.0
11.316776327 0.453711211681366 4.0
22.572006823 0.476474702358246 5.0
33.88878315 0.561817407608032 6.0
33.88878315 0.52483594417572 7.0
42.125521158 0.719116151332855 0.0
27.124375068 0.523897469043732 1.0
15.00114609 0.570456087589264 2.0
0 0.227881103754044 3.0
15.00114609 0.455991506576538 4.0
27.124375068 0.508558690547943 5.0
42.125521158 0.509978175163269 6.0
42.125521158 0.520654022693634 7.0
37.55067892 0.768102884292603 0.0
28.417766606 0.490339487791061 1.0
9.1329123142 0.522340536117554 2.0
0 0.217231333255768 3.0
9.1329123142 0.435216307640076 4.0
28.417766606 0.486202239990234 5.0
37.55067892 0.236067235469818 6.0
37.55067892 0.362213283777237 7.0
37.39948618 0.800835609436035 0.0
25.284755164 0.428945183753967 1.0
12.114731017 0.464847445487976 2.0
0 0.193325266242027 3.0
12.114731017 0.441429674625397 4.0
25.284755164 0.47260519862175 5.0
37.39948618 0.374361455440521 6.0
37.39948618 0.510845422744751 7.0
34.341868993 0.356555223464966 0.0
25.736031722 0.444835960865021 1.0
8.6058372707 0.476170599460602 2.0
0 0.18976503610611 3.0
8.6058372707 0.444457173347473 4.0
25.736031722 0.472164005041122 5.0
34.341868993 0.231031194329262 6.0
34.341868993 0.170882254838943 7.0
37.425355461 0.832730412483215 0.0
23.108404202 0.497640788555145 1.0
14.316951259 0.543325126171112 2.0
0 0.217540055513382 3.0
14.316951259 0.440960109233856 4.0
23.108404202 0.487115442752838 5.0
37.425355461 0.40717938542366 6.0
37.425355461 0.43628466129303 7.0
37.288491912 0.803253650665283 0.0
28.215703892 0.492514699697495 1.0
9.0727880197 0.547711491584778 2.0
0 0.239780783653259 3.0
9.0727880197 0.449190139770508 4.0
28.215703892 0.492241114377975 5.0
37.288491912 0.533772230148315 6.0
37.288491912 0.543611705303192 7.0
39.20347441 0.576323688030243 0.0
25.942862958 0.443642258644104 1.0
13.260611453 0.478121519088745 2.0
0 0.176715776324272 3.0
13.260611453 0.440610766410828 4.0
25.942862958 0.456402003765106 5.0
39.20347441 0.363000363111496 6.0
39.20347441 0.353269219398499 7.0
36.137884824 0.855295062065125 0.0
22.759896907 0.482566803693771 1.0
13.377987918 0.525928616523743 2.0
0 0.262983947992325 3.0
13.377987918 0.456713706254959 4.0
22.759896907 0.515828669071198 5.0
36.137884824 0.412505596876144 6.0
36.137884824 0.288487166166306 7.0
41.655417288 0.788184702396393 0.0
31.067976794 0.520488858222961 1.0
10.587440495 0.559696137905121 2.0
0 0.240109115839005 3.0
10.587440495 0.446697115898132 4.0
31.067976794 0.481211751699448 5.0
41.655417288 0.519771575927734 6.0
41.655417288 0.527117908000946 7.0
39.894978379 0.84921258687973 0.0
27.87905175 0.470551490783691 1.0
12.01592663 0.530196487903595 2.0
0 0.20897364616394 3.0
12.01592663 0.456653952598572 4.0
27.87905175 0.472027778625488 5.0
39.894978379 0.469559520483017 6.0
39.894978379 0.468638926744461 7.0
39.390677385 0.764048337936401 0.0
23.775179739 0.408933073282242 1.0
15.615497647 0.414964109659195 2.0
0 0.218922987580299 3.0
15.615497647 0.417196154594421 4.0
23.775179739 0.454847812652588 5.0
39.390677385 0.0920688360929489 6.0
39.390677385 0.478188902139664 7.0
38.248327703 0.831787049770355 0.0
28.805837831 0.466651886701584 1.0
9.4424898716 0.52033531665802 2.0
0 0.214888229966164 3.0
9.4424898716 0.455206483602524 4.0
28.805837831 0.481190145015717 5.0
38.248327703 0.488779932260513 6.0
38.248327703 0.524509251117706 7.0
38.538636426 0.829860508441925 0.0
26.334072091 0.474645912647247 1.0
12.204564335 0.533638954162598 2.0
0 0.239574670791626 3.0
12.204564335 0.459662795066833 4.0
26.334072091 0.49693375825882 5.0
38.538636426 0.510874629020691 6.0
38.538636426 0.518048405647278 7.0
30.093865158 0.4898422062397 0.0
19.25470281 0.392176419496536 1.0
10.839162348 0.428547084331512 2.0
0 0.195507362484932 3.0
10.839162348 0.405080378055573 4.0
19.25470281 0.428700983524323 5.0
30.093865158 0.0379255786538124 6.0
30.093865158 0.345939457416534 7.0
36.059945316 0.829988896846771 0.0
27.426776021 0.343459218740463 1.0
8.6331692955 0.382553219795227 2.0
0 0.230791017413139 3.0
8.6331692955 0.469299912452698 4.0
27.426776021 0.501267552375793 5.0
36.059945316 0.579757034778595 6.0
36.059945316 0.443004876375198 7.0
41.382686141 0.758825421333313 0.0
30.919992406 0.533676385879517 1.0
10.462693735 0.576283514499664 2.0
0 0.234919294714928 3.0
10.462693735 0.426487922668457 4.0
30.919992406 0.469380289316177 5.0
41.382686141 0.330253809690475 6.0
41.382686141 0.335124254226685 7.0
37.374314239 0.77263605594635 0.0
24.809000191 0.479119598865509 1.0
12.565314048 0.53385066986084 2.0
0 0.195990845561028 3.0
12.565314048 0.444365739822388 4.0
24.809000191 0.478054314851761 5.0
37.374314239 0.394645750522614 6.0
37.374314239 0.581712543964386 7.0
37.689487865 0.581778585910797 0.0
23.778452679 0.442288726568222 1.0
13.911035186 0.482822239398956 2.0
0 0.194729581475258 3.0
13.911035186 0.456823498010635 4.0
23.778452679 0.467709898948669 5.0
37.689487865 0.304965138435364 6.0
37.689487865 0.447632879018784 7.0
36.541540581 0.829984843730927 0.0
25.292349782 0.486431211233139 1.0
11.249190798 0.542470693588257 2.0
0 0.222336813807487 3.0
11.249190798 0.440548926591873 4.0
25.292349782 0.483181059360504 5.0
36.541540581 0.533578872680664 6.0
36.541540581 0.502507150173187 7.0
36.791141656 0.733271241188049 0.0
23.632233062 0.476416707038879 1.0
13.158908594 0.533956289291382 2.0
0 0.224450215697289 3.0
13.158908594 0.452402710914612 4.0
23.632233062 0.497908294200897 5.0
36.791141656 0.554515063762665 6.0
36.791141656 0.512175440788269 7.0
39.891605825 0.859999179840088 0.0
27.59587785 0.588183343410492 1.0
12.295727975 0.640135526657104 2.0
0 0.250788122415543 3.0
12.295727975 0.419509828090668 4.0
27.59587785 0.472173094749451 5.0
39.891605825 0.485537171363831 6.0
39.891605825 0.217408895492554 7.0
34.771445328 0.839740812778473 0.0
22.721923981 0.462479084730148 1.0
12.049521348 0.505071043968201 2.0
0 0.188699215650558 3.0
12.049521348 0.433919608592987 4.0
22.721923981 0.470237225294113 5.0
34.771445328 0.412918895483017 6.0
34.771445328 0.460507780313492 7.0
44.638557165 0.69169282913208 0.0
30.610302295 0.458334386348724 1.0
14.02825487 0.494569957256317 2.0
0 0.178731292486191 3.0
14.02825487 0.433125674724579 4.0
30.610302295 0.452012360095978 5.0
44.638557165 0.222351759672165 6.0
44.638557165 0.445273250341415 7.0
36.096180936 0.740875005722046 0.0
20.898821406 0.516526639461517 1.0
15.19735953 0.549449145793915 2.0
0 0.223183423280716 3.0
15.19735953 0.454674065113068 4.0
20.898821406 0.481818407773972 5.0
36.096180936 0.542501509189606 6.0
36.096180936 0.401081770658493 7.0
45.744029538 0.769133448600769 0.0
31.432297526 0.504853248596191 1.0
14.311732013 0.553110837936401 2.0
0 0.220660224556923 3.0
14.311732013 0.453269302845001 4.0
31.432297526 0.495472222566605 5.0
45.744029538 0.4182408452034 6.0
45.744029538 0.624072909355164 7.0
36.287608705 0.731373846530914 0.0
25.16521997 0.488534957170486 1.0
11.122388735 0.540647983551025 2.0
0 0.195405185222626 3.0
11.122388735 0.438650548458099 4.0
25.16521997 0.460792422294617 5.0
36.287608705 0.327145844697952 6.0
36.287608705 0.552481532096863 7.0
42.801830865 0.825211226940155 0.0
27.040244124 0.501466155052185 1.0
15.761586741 0.530231118202209 2.0
0 0.234962090849876 3.0
15.761586741 0.440328687429428 4.0
27.040244124 0.49865847826004 5.0
42.801830865 0.401632100343704 6.0
42.801830865 0.287659615278244 7.0
47.558612578 0.765089511871338 0.0
32.997784418 0.509720623493195 1.0
14.56082816 0.541511714458466 2.0
0 0.213970094919205 3.0
14.56082816 0.425100088119507 4.0
32.997784418 0.469201624393463 5.0
47.558612578 0.294362962245941 6.0
47.558612578 0.361552685499191 7.0
47.058562462 0.836169064044952 0.0
29.793950686 0.324345916509628 1.0
17.264611775 0.354446798563004 2.0
-1.5381975965 0.179364904761314 3.0
15.726414179 0.443008989095688 4.0
31.332148283 0.46812230348587 5.0
47.058562462 0.581381976604462 6.0
47.058562462 0.453312873840332 7.0
43.087048946 0.251832842826843 0.0
31.678082071 0.458430022001266 1.0
11.408966876 0.492297351360321 2.0
0 0.187294393777847 3.0
11.408966876 0.460924446582794 4.0
31.678082071 0.474490106105804 5.0
43.087048946 0.216200292110443 6.0
43.087048946 0.048327349126339 7.0
33.170190568 0.804521858692169 0.0
20.693720573 0.493608355522156 1.0
12.476469995 0.529906570911407 2.0
0 0.238997519016266 3.0
12.476469995 0.451689124107361 4.0
20.693720573 0.488741934299469 5.0
33.170190568 0.294514268636703 6.0
33.170190568 0.416710525751114 7.0
34.957431523 0.767542898654938 0.0
21.148722929 0.486349880695343 1.0
13.808708594 0.521443724632263 2.0
0 0.223169252276421 3.0
13.808708594 0.434784531593323 4.0
21.148722929 0.476090759038925 5.0
34.957431523 0.284330487251282 6.0
34.957431523 0.336738646030426 7.0
36.320328519 0.453794866800308 0.0
23.458317243 0.458259791135788 1.0
12.862011276 0.485505253076553 2.0
0 0.18622088432312 3.0
12.862011276 0.437724530696869 4.0
23.458317243 0.464717984199524 5.0
36.320328519 0.309058040380478 6.0
36.320328519 0.242196589708328 7.0
44.085721677 0.853151321411133 0.0
30.493758234 0.525163412094116 1.0
13.591963443 0.554679691791534 2.0
0 0.235083416104317 3.0
13.591963443 0.438571840524673 4.0
30.493758234 0.477535903453827 5.0
44.085721677 0.476452648639679 6.0
44.085721677 0.426071226596832 7.0
45.211814568 0.730757117271423 0.0
29.654020724 0.472769200801849 1.0
15.557793845 0.516591668128967 2.0
-1.8265292485 0.182553306221962 3.0
13.731264596 0.442037343978882 4.0
31.480549972 0.464348018169403 5.0
45.211814568 0.405426651239395 6.0
45.211814568 0.562685012817383 7.0
38.71721827 0.55953848361969 0.0
25.13774692 0.46828681230545 1.0
13.57947135 0.498782217502594 2.0
0 0.182582199573517 3.0
13.57947135 0.432104974985123 4.0
25.13774692 0.466955572366714 5.0
38.71721827 0.325168043375015 6.0
38.71721827 0.310213476419449 7.0
42.190454953 0.83099353313446 0.0
27.535288565 0.548747360706329 1.0
14.655166388 0.59564208984375 2.0
0 0.247689604759216 3.0
14.655166388 0.423866987228394 4.0
27.535288565 0.472783446311951 5.0
42.190454953 0.412186861038208 6.0
42.190454953 0.371216416358948 7.0
40.616038376 0.790348708629608 0.0
28.55006814 0.475070029497147 1.0
12.065970236 0.52785450220108 2.0
0 0.210344105958939 3.0
12.065970236 0.450180858373642 4.0
28.55006814 0.491857826709747 5.0
40.616038376 0.297338336706161 6.0
40.616038376 0.525890052318573 7.0
31.887764532 0.765896677970886 0.0
20.470687746 0.639325737953186 1.0
11.417076785 0.683603048324585 2.0
0 0.274116992950439 3.0
11.417076785 0.428325235843658 4.0
20.470687746 0.469673693180084 5.0
31.887764532 0.28444117307663 6.0
31.887764532 0.245365917682648 7.0
39.764867486 0.269996792078018 0.0
29.907448957 0.322924047708511 1.0
9.8574185294 0.334641575813293 2.0
0 0.179844990372658 3.0
9.8574185294 0.381209373474121 4.0
29.907448957 0.434848785400391 5.0
39.764867486 0.141658395528793 6.0
39.764867486 0.0861054286360741 7.0
40.805009287 0.175317004323006 0.0
29.452805352 0.605387985706329 1.0
11.352203935 0.646177470684052 2.0
0 0.201307207345963 3.0
11.352203935 0.446570247411728 4.0
29.452805352 0.469405233860016 5.0
40.805009287 -0.00762186199426651 6.0
40.805009287 -0.102342367172241 7.0
39.795600818 0.788265228271484 0.0
26.601347902 0.51937198638916 1.0
13.194252916 0.558219134807587 2.0
0 0.245271265506744 3.0
13.194252916 0.457745015621185 4.0
26.601347902 0.497253656387329 5.0
39.795600818 0.492575109004974 6.0
39.795600818 0.542112946510315 7.0
38.685439721 0.663237988948822 0.0
26.326388497 0.443660914897919 1.0
12.359051224 0.491417169570923 2.0
0 0.18854609131813 3.0
12.359051224 0.442904442548752 4.0
26.326388497 0.48028028011322 5.0
38.685439721 0.249089851975441 6.0
38.685439721 0.580181360244751 7.0
42.152656556 0.816113829612732 0.0
27.567345332 0.361722230911255 1.0
14.585311224 0.397283583879471 2.0
0 0.221218183636665 3.0
14.585311224 0.460176408290863 4.0
27.567345332 0.507707715034485 5.0
42.152656556 0.539868772029877 6.0
42.152656556 0.485894799232483 7.0
37.458343334 0.363158285617828 0.0
22.42476406 0.0884044766426086 1.0
15.033579274 0.114171668887138 2.0
0 0.251310020685196 3.0
15.033579274 0.427126348018646 4.0
22.42476406 0.478590905666351 5.0
37.458343334 1.43568873405457 6.0
37.458343334 0.334411859512329 7.0
44.489633946 0.753030359745026 0.0
28.961471546 0.320512682199478 1.0
15.5281624 0.361837208271027 2.0
0 0.235205307602882 3.0
15.5281624 0.411390990018845 4.0
28.961471546 0.456252753734589 5.0
44.489633946 0.123183891177177 6.0
44.489633946 0.0495978444814682 7.0
41.506640114 0.827824175357819 0.0
26.860159768 0.41014689207077 1.0
14.646480346 0.447557389736176 2.0
0 0.219842374324799 3.0
14.646480346 0.46123218536377 4.0
26.860159768 0.493260473012924 5.0
41.506640114 0.500347316265106 6.0
41.506640114 0.509965062141418 7.0
33.262736569 0.727373361587524 0.0
19.270575871 0.513414621353149 1.0
13.992160697 0.551025807857513 2.0
0 0.216522976756096 3.0
13.992160697 0.41122567653656 4.0
19.270575871 0.453184992074966 5.0
33.262736569 0.278282254934311 6.0
33.262736569 0.329334437847137 7.0
47.069969764 0.492064982652664 0.0
31.242343365 0.462925910949707 1.0
15.827626399 0.497679144144058 2.0
0 0.196746796369553 3.0
15.827626399 0.446607738733292 4.0
31.242343365 0.465272665023804 5.0
47.069969764 0.239511042833328 6.0
47.069969764 0.167534202337265 7.0
43.288539967 0.579037070274353 0.0
27.687399618 0.464620441198349 1.0
15.601140349 0.495984852313995 2.0
0 0.218285277485847 3.0
15.601140349 0.440158605575562 4.0
27.687399618 0.493175268173218 5.0
43.288539967 0.0332961976528168 6.0
43.288539967 0.158869370818138 7.0
41.479216264 0.710120022296906 0.0
26.515367489 0.471049815416336 1.0
14.963848776 0.507046699523926 2.0
0 0.211191147565842 3.0
14.963848776 0.45024573802948 4.0
26.515367489 0.49560621380806 5.0
41.479216264 0.274210393428802 6.0
41.479216264 0.583715617656708 7.0
36.99162076 0.795879006385803 0.0
21.889008767 0.507975518703461 1.0
15.102611993 0.531408727169037 2.0
0 0.20392695069313 3.0
15.102611993 0.431393563747406 4.0
21.889008767 0.453881442546844 5.0
36.99162076 0.326482117176056 6.0
36.99162076 0.340961486101151 7.0
40.435886771 0.727305114269257 0.0
28.088815114 0.468338489532471 1.0
12.347071657 0.509993076324463 2.0
-2.1989446774 0.215879648923874 3.0
10.148126979 0.460224539041519 4.0
30.287759792 0.481431186199188 5.0
40.435886771 0.2606540620327 6.0
40.435886771 0.443439096212387 7.0
37.140737383 0.791135430335999 0.0
25.451742904 0.492880642414093 1.0
11.68899448 0.54355651140213 2.0
0 0.230941355228424 3.0
11.68899448 0.452860891819 4.0
25.451742904 0.500938773155212 5.0
37.140737383 0.50535774230957 6.0
37.140737383 0.485817134380341 7.0
39.233907158 0.0924374684691429 0.0
24.739135527 1.78525233268738 1.0
14.494771631 1.80091655254364 2.0
0 0.170271530747414 3.0
14.494771631 0.436626821756363 4.0
24.739135527 0.452435255050659 5.0
39.233907158 0.788700044155121 6.0
39.233907158 20.8937492370605 7.0
42.028602534 0.867590487003326 0.0
26.365954124 0.622472107410431 1.0
15.66264841 0.657801985740662 2.0
0 0.262709647417068 3.0
15.66264841 0.425092339515686 4.0
26.365954124 0.485545814037323 5.0
42.028602534 0.360188394784927 6.0
42.028602534 0.197759345173836 7.0
39.783103652 0.502975761890411 0.0
24.444813382 0.433340847492218 1.0
15.33829027 0.458600401878357 2.0
0 0.198222145438194 3.0
15.33829027 0.418039619922638 4.0
24.444813382 0.450436592102051 5.0
39.783103652 -0.00185212679207325 6.0
39.783103652 0.302625715732574 7.0
39.425516455 0.780720233917236 0.0
28.190796848 0.525082349777222 1.0
11.234719607 0.551677942276001 2.0
0 0.226160451769829 3.0
11.234719607 0.439467042684555 4.0
28.190796848 0.474152326583862 5.0
39.425516455 0.535301506519318 6.0
39.425516455 0.364696711301804 7.0
40.002559542 0.735043048858643 0.0
25.191107177 0.462506115436554 1.0
14.811452365 0.503107845783234 2.0
0 0.213103771209717 3.0
14.811452365 0.455327749252319 4.0
25.191107177 0.473927319049835 5.0
40.002559542 0.311489075422287 6.0
40.002559542 0.35161566734314 7.0
38.322665212 0.812806725502014 0.0
24.775138059 0.38771778345108 1.0
13.547527153 0.425627171993256 2.0
0 0.271121084690094 3.0
13.547527153 0.45910370349884 4.0
24.775138059 0.507686078548431 5.0
38.322665212 0.277381956577301 6.0
38.322665212 0.336555778980255 7.0
40.568600192 0.685778021812439 0.0
25.445690154 0.413862586021423 1.0
15.122910038 0.446553111076355 2.0
0 0.245540499687195 3.0
15.122910038 0.443612605333328 4.0
25.445690154 0.490930318832397 5.0
40.568600192 0.224337220191956 6.0
40.568600192 0.26363268494606 7.0
40.337689123 0.785712778568268 0.0
28.556160654 0.491209149360657 1.0
11.781528468 0.525064468383789 2.0
0 0.227939784526825 3.0
11.781528468 0.448422849178314 4.0
28.556160654 0.484233528375626 5.0
40.337689123 0.368429094552994 6.0
40.337689123 0.322571516036987 7.0
45.726007711 0.732672512531281 0.0
32.201501409 0.456068962812424 1.0
13.524506302 0.504116475582123 2.0
0 0.210868641734123 3.0
13.524506302 0.445475161075592 4.0
32.201501409 0.487887620925903 5.0
45.726007711 0.258503466844559 6.0
45.726007711 0.556243777275085 7.0
36.675787374 0.843296051025391 0.0
27.948349606 0.50584465265274 1.0
8.7274377682 0.546075284481049 2.0
0 0.231807455420494 3.0
8.7274377682 0.434578567743301 4.0
27.948349606 0.467675775289536 5.0
36.675787374 0.389026641845703 6.0
36.675787374 0.321653783321381 7.0
32.235093382 0.718962132930756 0.0
21.205932177 0.504782438278198 1.0
11.029161206 0.561880946159363 2.0
0 0.223219633102417 3.0
11.029161206 0.464746475219727 4.0
21.205932177 0.483413815498352 5.0
32.235093382 0.528319954872131 6.0
32.235093382 0.506022155284882 7.0
39.438537732 0.832511425018311 0.0
26.045647237 0.495471060276031 1.0
13.392890495 0.53849720954895 2.0
0 0.26960626244545 3.0
13.392890495 0.459973007440567 4.0
26.045647237 0.529717326164246 5.0
39.438537732 0.504038572311401 6.0
39.438537732 0.35876390337944 7.0
44.304614406 0.798607170581818 0.0
30.417756083 0.333976238965988 1.0
13.886858323 0.364942610263824 2.0
0 0.206213891506195 3.0
13.886858323 0.453167736530304 4.0
30.417756083 0.498596489429474 5.0
44.304614406 0.522411227226257 6.0
44.304614406 0.492323905229568 7.0
32.373304325 0.282308220863342 0.0
22.786241426 0.562253654003143 1.0
9.5870628991 0.601056575775146 2.0
0 0.201459258794785 3.0
9.5870628991 0.43321305513382 4.0
22.786241426 0.461896419525146 5.0
32.373304325 -0.0194011591374874 6.0
32.373304325 -0.0373598337173462 7.0
35.769379856 0.75627338886261 0.0
25.430130877 0.478938102722168 1.0
10.339248979 0.513798475265503 2.0
0 0.227053493261337 3.0
10.339248979 0.440272837877274 4.0
25.430130877 0.496189117431641 5.0
35.769379856 0.287195980548859 6.0
35.769379856 0.357178866863251 7.0
38.348316661 0.803281724452972 0.0
28.599053532 0.523897588253021 1.0
9.7492631289 0.55842912197113 2.0
0 0.214372098445892 3.0
9.7492631289 0.431454211473465 4.0
28.599053532 0.455771505832672 5.0
38.348316661 0.273515313863754 6.0
38.348316661 0.323279052972794 7.0
30.534523433 0.783188581466675 0.0
21.176430979 0.491856038570404 1.0
9.3580924534 0.522153377532959 2.0
0 0.205045312643051 3.0
9.3580924534 0.440807461738586 4.0
21.176430979 0.467590093612671 5.0
30.534523433 0.284853339195251 6.0
30.534523433 0.362046033143997 7.0
41.665325899 0.831326961517334 0.0
29.240172477 0.505439758300781 1.0
12.425153422 0.531885266304016 2.0
0 0.214288294315338 3.0
12.425153422 0.445488333702087 4.0
29.240172477 0.467988133430481 5.0
41.665325899 0.317501962184906 6.0
41.665325899 0.360829174518585 7.0
39.125756769 0.847474098205566 0.0
23.306411016 0.501657128334045 1.0
15.819345753 0.540571093559265 2.0
0 0.228399977087975 3.0
15.819345753 0.452569514513016 4.0
23.306411016 0.487775087356567 5.0
39.125756769 0.351486504077911 6.0
39.125756769 0.460732877254486 7.0
46.922469454 0.793150365352631 0.0
31.861747681 0.48756730556488 1.0
15.060721773 0.526739358901978 2.0
0 0.2343660145998 3.0
15.060721773 0.463797241449356 4.0
31.861747681 0.50634104013443 5.0
46.922469454 0.278255045413971 6.0
46.922469454 0.398054420948029 7.0
38.082156965 0.115104548633099 0.0
26.037781012 0.99053829908371 1.0
12.044375953 0.990469932556152 2.0
0 0.161712527275085 3.0
12.044375953 0.432081460952759 4.0
26.037781012 0.45561084151268 5.0
38.082156965 0.17919896543026 6.0
38.082156965 0.871791899204254 7.0
35.281773346 0.846243500709534 0.0
22.096480246 0.53650027513504 1.0
13.185293101 0.570440709590912 2.0
0 0.228667870163918 3.0
13.185293101 0.423119157552719 4.0
22.096480246 0.473610013723373 5.0
35.281773346 0.297596633434296 6.0
35.281773346 0.286280870437622 7.0
39.676820236 0.759587705135345 0.0
30.920209498 0.486791640520096 1.0
8.7566107375 0.519767224788666 2.0
0 0.210380792617798 3.0
8.7566107375 0.436247229576111 4.0
30.920209498 0.477970629930496 5.0
39.676820236 0.27116459608078 6.0
39.676820236 0.359000980854034 7.0
41.049027324 0.8468257188797 0.0
27.686936742 0.477710127830505 1.0
13.362090582 0.533372282981873 2.0
0 0.216967672109604 3.0
13.362090582 0.453312605619431 4.0
27.686936742 0.488722890615463 5.0
41.049027324 0.420675337314606 6.0
41.049027324 0.46754252910614 7.0
41.402616693 0.116684474050999 0.0
31.630523314 0.596835970878601 1.0
9.7720933796 0.61355072259903 2.0
0 0.187870591878891 3.0
9.7720933796 0.416359454393387 4.0
31.630523314 0.449009299278259 5.0
41.402616693 0.911306500434875 6.0
41.402616693 4.10548400878906 7.0
30.3721478 0.798549175262451 0.0
20.852771908 0.511887729167938 1.0
9.5193758911 0.544537425041199 2.0
0 0.204956233501434 3.0
9.5193758911 0.406032979488373 4.0
20.852771908 0.444952815771103 5.0
30.3721478 0.277315467596054 6.0
30.3721478 0.344078153371811 7.0
41.111385117 0.815402925014496 0.0
26.925919629 0.49236187338829 1.0
14.185465488 0.548638582229614 2.0
0 0.228421822190285 3.0
14.185465488 0.438965141773224 4.0
26.925919629 0.494895398616791 5.0
41.111385117 0.542184472084045 6.0
41.111385117 0.526611328125 7.0
36.921893091 0.837696254253387 0.0
26.261414813 0.458658546209335 1.0
10.660478278 0.5060133934021 2.0
0 0.212253212928772 3.0
10.660478278 0.438376665115356 4.0
26.261414813 0.477657079696655 5.0
36.921893091 0.492435544729233 6.0
36.921893091 0.525696575641632 7.0
41.531234868 0.48067119717598 0.0
29.844730664 1.18256807327271 1.0
11.686504204 1.18841540813446 2.0
0 0.169182822108269 3.0
11.686504204 0.425159096717834 4.0
29.844730664 0.462562799453735 5.0
41.531234868 0.574762463569641 6.0
41.531234868 3.10449957847595 7.0
36.809020967 0.838907361030579 0.0
22.744117042 0.498662650585175 1.0
14.064903924 0.521833300590515 2.0
0 0.202075615525246 3.0
14.064903924 0.414070338010788 4.0
22.744117042 0.448627650737762 5.0
36.809020967 0.297870963811874 6.0
36.809020967 0.361833810806274 7.0
30.318323192 0.568775296211243 0.0
20.066163087 0.470534354448318 1.0
10.252160105 0.506835281848907 2.0
0 0.206245064735413 3.0
10.252160105 0.440004825592041 4.0
20.066163087 0.46474677324295 5.0
30.318323192 0.319919466972351 6.0
30.318323192 0.316286742687225 7.0
37.371827278 0.155422329902649 0.0
28.713218734 0.850831985473633 1.0
8.6586085434 0.860785007476807 2.0
0 0.109290763735771 3.0
8.6586085434 0.42778417468071 4.0
28.713218734 0.418621897697449 5.0
37.371827278 -0.111837916076183 6.0
37.371827278 0.915547609329224 7.0
37.183742215 0.812906444072723 0.0
23.390545172 0.541105449199677 1.0
13.793197043 0.571206510066986 2.0
0 0.219559550285339 3.0
13.793197043 0.415613323450089 4.0
23.390545172 0.455676674842834 5.0
37.183742215 0.51192831993103 6.0
37.183742215 0.372735321521759 7.0
34.629310928 0.761132299900055 0.0
23.313997557 0.49059322476387 1.0
11.315313371 0.547568917274475 2.0
0 0.235555797815323 3.0
11.315313371 0.46245801448822 4.0
23.313997557 0.486496567726135 5.0
34.629310928 0.537822544574738 6.0
34.629310928 0.546171724796295 7.0
42.830967498 0.801610887050629 0.0
30.470948882 0.445679157972336 1.0
12.360018616 0.494025975465775 2.0
0 0.200350180268288 3.0
12.360018616 0.460591524839401 4.0
30.470948882 0.471467971801758 5.0
42.830967498 0.345294207334518 6.0
42.830967498 0.52705854177475 7.0
34.694327372 0.838520586490631 0.0
21.852205513 0.459666609764099 1.0
12.842121859 0.518162369728088 2.0
0 0.24217700958252 3.0
12.842121859 0.459462255239487 4.0
21.852205513 0.494565367698669 5.0
34.694327372 0.549441397190094 6.0
34.694327372 0.489366173744202 7.0
37.611435874 0.680758833885193 0.0
23.212866181 0.44694721698761 1.0
14.398569692 0.485555827617645 2.0
0 0.195902183651924 3.0
14.398569692 0.451625823974609 4.0
23.212866181 0.473365128040314 5.0
37.611435874 0.23055824637413 6.0
37.611435874 0.501015543937683 7.0
45.529166488 0.80208283662796 0.0
32.298489175 0.504412710666656 1.0
13.230677313 0.537142038345337 2.0
0 0.236793756484985 3.0
13.230677313 0.449020445346832 4.0
32.298489175 0.495468020439148 5.0
45.529166488 0.353417158126831 6.0
45.529166488 0.42054945230484 7.0
40.26689599 0.635780274868011 0.0
29.798490664 0.438106626272202 1.0
10.468405325 0.472557872533798 2.0
0 0.195166438817978 3.0
10.468405325 0.453054785728455 4.0
29.798490664 0.483116209506989 5.0
40.26689599 0.23454350233078 6.0
40.26689599 0.351504117250443 7.0
39.577958661 0.705699801445007 0.0
25.13822578 0.38491952419281 1.0
14.439732881 0.413358777761459 2.0
0 0.188097313046455 3.0
14.439732881 0.450108766555786 4.0
25.13822578 0.48660272359848 5.0
39.577958661 0.456398516893387 6.0
39.577958661 0.559784948825836 7.0
34.564585703 0.1031673848629 0.0
21.369259648 0.970134854316711 1.0
13.195326055 0.973898589611053 2.0
0 0.206713736057281 3.0
13.195326055 0.416798621416092 4.0
21.369259648 0.480736911296844 5.0
34.564585703 1.61226439476013 6.0
34.564585703 13.1785802841187 7.0
34.747667205 0.780750632286072 0.0
26.117279278 0.417109549045563 1.0
8.6303879268 0.447264492511749 2.0
0 0.263707190752029 3.0
8.6303879268 0.446226865053177 4.0
26.117279278 0.501128673553467 5.0
34.747667205 0.173817381262779 6.0
34.747667205 0.352001130580902 7.0
44.886437438 0.777854263782501 0.0
29.455366783 0.487889856100082 1.0
15.431070655 0.541507244110107 2.0
0 0.209404528141022 3.0
15.431070655 0.441912233829498 4.0
29.455366783 0.480743050575256 5.0
44.886437438 0.400206416845322 6.0
44.886437438 0.571026206016541 7.0
34.458606901 0.797810792922974 0.0
21.914634893 0.499802261590958 1.0
12.543972008 0.551523983478546 2.0
0 0.229595735669136 3.0
12.543972008 0.452793300151825 4.0
21.914634893 0.482966959476471 5.0
34.458606901 0.507129073143005 6.0
34.458606901 0.53798633813858 7.0
44.985938301 0.41882136464119 0.0
31.318666065 0.129012525081635 1.0
13.667272236 0.132424503564835 2.0
0 0.203358441591263 3.0
13.667272236 0.383377522230148 4.0
31.318666065 0.422077089548111 5.0
44.985938301 1.73021841049194 6.0
44.985938301 0.297722816467285 7.0
36.561528733 0.714788317680359 0.0
26.365627821 0.514517903327942 1.0
10.195900911 0.544709384441376 2.0
0 0.207633256912231 3.0
10.195900911 0.406996697187424 4.0
26.365627821 0.444898247718811 5.0
36.561528733 0.218347415328026 6.0
36.561528733 0.396773308515549 7.0
35.585636345 0.764884173870087 0.0
21.938078684 0.508736312389374 1.0
13.647557661 0.54513418674469 2.0
0 0.232083410024643 3.0
13.647557661 0.439616352319717 4.0
21.938078684 0.488339155912399 5.0
35.585636345 0.28092622756958 6.0
35.585636345 0.383685231208801 7.0
34.748633645 0.838452875614166 0.0
24.774050599 0.504496097564697 1.0
9.9745830452 0.553761720657349 2.0
0 0.205702409148216 3.0
9.9745830452 0.441912770271301 4.0
24.774050599 0.465254724025726 5.0
34.748633645 0.495783120393753 6.0
34.748633645 0.417722940444946 7.0
33.997517461 0.739884197711945 0.0
23.648741818 0.482376635074615 1.0
10.348775643 0.536668360233307 2.0
0 0.201690495014191 3.0
10.348775643 0.442936450242996 4.0
23.648741818 0.480537950992584 5.0
33.997517461 0.382371306419373 6.0
33.997517461 0.593596696853638 7.0
31.337837162 0.784162521362305 0.0
18.725674883 0.481356918811798 1.0
12.612162279 0.522033929824829 2.0
0 0.242307260632515 3.0
12.612162279 0.461631298065186 4.0
18.725674883 0.509175002574921 5.0
31.337837162 0.298844397068024 6.0
31.337837162 0.37770688533783 7.0
43.914614211 0.797639966011047 0.0
31.443326284 0.486601561307907 1.0
12.471287926 0.543023109436035 2.0
0 0.231027245521545 3.0
12.471287926 0.456566870212555 4.0
31.443326284 0.491628110408783 5.0
43.914614211 0.488014161586761 6.0
43.914614211 0.5650235414505 7.0
45.186574618 0.805290222167969 0.0
30.950820377 0.524407565593719 1.0
14.23575424 0.556483626365662 2.0
0 0.243118852376938 3.0
14.23575424 0.447125226259232 4.0
30.950820377 0.491475224494934 5.0
45.186574618 0.42483589053154 6.0
45.186574618 0.451940327882767 7.0
35.268441876 0.6221644282341 0.0
23.396711515 0.456338375806808 1.0
11.871730361 0.479887366294861 2.0
0 0.174734458327293 3.0
11.871730361 0.417648822069168 4.0
23.396711515 0.453825712203979 5.0
35.268441876 0.0411452725529671 6.0
35.268441876 0.253070086240768 7.0
32.333098982 0.796977818012238 0.0
21.927214576 0.539125263690948 1.0
10.405884406 0.579756557941437 2.0
0 0.235182166099548 3.0
10.405884406 0.438152939081192 4.0
21.927214576 0.466354012489319 5.0
32.333098982 0.460986167192459 6.0
32.333098982 0.420589715242386 7.0
38.894207338 0.766211807727814 0.0
28.008989525 0.514622807502747 1.0
10.885217813 0.551746726036072 2.0
0 0.239541932940483 3.0
10.885217813 0.4331955909729 4.0
28.008989525 0.484436839818954 5.0
38.894207338 0.196327805519104 6.0
38.894207338 0.370959758758545 7.0
32.941924742 0.7943394780159 0.0
24.375613859 0.348554015159607 1.0
8.5663108832 0.382530987262726 2.0
0 0.285973608493805 3.0
8.5663108832 0.450339496135712 4.0
24.375613859 0.506501317024231 5.0
32.941924742 0.108090221881866 6.0
32.941924742 0.376112759113312 7.0
40.2164929 0.659590840339661 0.0
31.023294947 0.457521498203278 1.0
9.193197953 0.486428260803223 2.0
0 0.234955370426178 3.0
9.193197953 0.423990905284882 4.0
31.023294947 0.466635525226593 5.0
40.2164929 0.333583414554596 6.0
40.2164929 0.386746108531952 7.0
40.810463567 0.80683034658432 0.0
27.134577858 0.45493471622467 1.0
13.675885708 0.507707893848419 2.0
0 0.208212405443192 3.0
13.675885708 0.45153871178627 4.0
27.134577858 0.486934125423431 5.0
40.810463567 0.293470174074173 6.0
40.810463567 0.528074204921722 7.0
33.558410457 0.361560076475143 0.0
21.95745024 0.563504457473755 1.0
11.600960217 0.599802851676941 2.0
0 0.173850744962692 3.0
11.600960217 0.430446624755859 4.0
21.95745024 0.471129536628723 5.0
33.558410457 -0.0199100375175476 6.0
33.558410457 -0.0376835241913795 7.0
47.788896524 0.631999969482422 0.0
33.360420205 0.489188492298126 1.0
14.428476318 0.517815411090851 2.0
0 0.210557922720909 3.0
14.428476318 0.432338953018188 4.0
33.360420205 0.457118064165115 5.0
47.788896524 0.261306464672089 6.0
47.788896524 0.261991620063782 7.0
33.642782256 0.825193166732788 0.0
20.78231535 0.502078771591187 1.0
12.860466905 0.534224152565002 2.0
0 0.225138127803802 3.0
12.860466905 0.45270162820816 4.0
20.78231535 0.484966337680817 5.0
33.642782256 0.299549788236618 6.0
33.642782256 0.383639514446259 7.0
37.659684029 0.8466836810112 0.0
23.272895976 0.487178862094879 1.0
14.386788054 0.546540915966034 2.0
0 0.232217460870743 3.0
14.386788054 0.449869334697723 4.0
23.272895976 0.474632441997528 5.0
37.659684029 0.564521431922913 6.0
37.659684029 0.531721770763397 7.0
40.91969971 0.790131390094757 0.0
31.549891178 0.545541703701019 1.0
9.3698085329 0.56895512342453 2.0
-0.3917019807 0.229334518313408 3.0
8.9781065522 0.426971971988678 4.0
31.941593158 0.467986762523651 5.0
40.91969971 0.557514429092407 6.0
40.91969971 0.342424154281616 7.0
42.622265927 0.658822357654572 0.0
29.180561659 0.39220130443573 1.0
13.441704268 0.405523419380188 2.0
0 0.213094890117645 3.0
13.441704268 0.414789766073227 4.0
29.180561659 0.452495396137238 5.0
42.622265927 0.0905643105506897 6.0
42.622265927 0.413849860429764 7.0
33.946966747 0.716403245925903 0.0
24.555540255 0.483281433582306 1.0
9.3914264914 0.518898248672485 2.0
0 0.219684183597565 3.0
9.3914264914 0.437317371368408 4.0
24.555540255 0.468373000621796 5.0
33.946966747 0.278890997171402 6.0
33.946966747 0.415055125951767 7.0
36.409182194 0.817650854587555 0.0
20.90503326 0.523270726203918 1.0
15.504148935 0.561609387397766 2.0
0 0.221443548798561 3.0
15.504148935 0.442168325185776 4.0
20.90503326 0.468184262514114 5.0
36.409182194 0.46590194106102 6.0
36.409182194 0.494686514139175 7.0
40.593448001 0.849292278289795 0.0
30.912037848 0.469382286071777 1.0
9.6814101538 0.527931392192841 2.0
0 0.232673138380051 3.0
9.6814101538 0.443548291921616 4.0
30.912037848 0.510373532772064 5.0
40.593448001 0.463393747806549 6.0
40.593448001 0.503175795078278 7.0
41.715980637 0.556692898273468 0.0
31.196210855 0.32874721288681 1.0
10.519769782 0.347456783056259 2.0
0 0.223968252539635 3.0
10.519769782 0.425417810678482 4.0
31.196210855 0.463868021965027 5.0
41.715980637 -0.00443122163414955 6.0
41.715980637 0.193049341440201 7.0
33.77037363 0.676774740219116 0.0
22.237227098 0.224589169025421 1.0
11.533146533 0.254376500844955 2.0
0 0.229098111391068 3.0
11.533146533 0.415278255939484 4.0
22.237227098 0.445237457752228 5.0
33.77037363 1.43201887607574 6.0
33.77037363 -5.8397650718689e-05 7.0
46.074402277 0.402931928634644 0.0
32.676175596 0.481015115976334 1.0
13.398226681 0.517407119274139 2.0
0 0.216364696621895 3.0
13.398226681 0.445328414440155 4.0
32.676175596 0.490692436695099 5.0
46.074402277 0.0589327253401279 6.0
46.074402277 0.0768218487501144 7.0
40.510394826 0.359742701053619 0.0
26.352618036 0.40872597694397 1.0
14.15777679 0.446791589260101 2.0
0 0.274457454681396 3.0
14.15777679 0.464470744132996 4.0
26.352618036 0.534984707832336 5.0
40.510394826 -0.0082878265529871 6.0
40.510394826 0.0884052440524101 7.0
39.244329381 0.767924249172211 0.0
28.414205494 0.47465044260025 1.0
10.830123887 0.509310901165009 2.0
0 0.235845848917961 3.0
10.830123887 0.44011464715004 4.0
28.414205494 0.503638207912445 5.0
39.244329381 0.271658182144165 6.0
39.244329381 0.360188186168671 7.0
35.7171596 0.836733818054199 0.0
22.103003179 0.477286338806152 1.0
13.614156422 0.534376621246338 2.0
0 0.224237650632858 3.0
13.614156422 0.451998442411423 4.0
22.103003179 0.497269511222839 5.0
35.7171596 0.463051348924637 6.0
35.7171596 0.530500590801239 7.0
39.754714125 0.510714113712311 0.0
24.390436102 0.493158787488937 1.0
15.364278023 0.535041272640228 2.0
0 0.17920146882534 3.0
15.364278023 0.443864673376083 4.0
24.390436102 0.451549470424652 5.0
39.754714125 0.0535186901688576 6.0
39.754714125 0.07455213367939 7.0
41.796195011 0.77821934223175 0.0
28.322556682 0.502561986446381 1.0
13.473638329 0.552352130413055 2.0
0 0.2341118901968 3.0
13.473638329 0.449372172355652 4.0
28.322556682 0.493766993284225 5.0
41.796195011 0.511218547821045 6.0
41.796195011 0.533612251281738 7.0
29.980660557 0.791936993598938 0.0
20.698164237 0.484211564064026 1.0
9.2824963201 0.540518879890442 2.0
0 0.234925523400307 3.0
9.2824963201 0.453288316726685 4.0
20.698164237 0.503138661384583 5.0
29.980660557 0.504876792430878 6.0
29.980660557 0.564027547836304 7.0
36.241260622 0.855786204338074 0.0
22.638938794 0.514446020126343 1.0
13.602321828 0.554309487342834 2.0
0 0.220699414610863 3.0
13.602321828 0.439164817333221 4.0
22.638938794 0.475201845169067 5.0
36.241260622 0.430722028017044 6.0
36.241260622 0.466972321271896 7.0
38.129703213 0.784092128276825 0.0
27.128465029 0.50480318069458 1.0
11.001238184 0.552473545074463 2.0
0 0.22157022356987 3.0
11.001238184 0.447287768125534 4.0
27.128465029 0.483276903629303 5.0
38.129703213 0.497900545597076 6.0
38.129703213 0.502704739570618 7.0
36.91522515 0.747430086135864 0.0
22.73211488 0.51236355304718 1.0
14.18311027 0.553923308849335 2.0
0 0.224657788872719 3.0
14.18311027 0.444507122039795 4.0
22.73211488 0.502201914787292 5.0
36.91522515 0.406154006719589 6.0
36.91522515 0.591113328933716 7.0
38.102908751 0.786026120185852 0.0
24.659837316 0.495492219924927 1.0
13.443071435 0.551822364330292 2.0
0 0.19953291118145 3.0
13.443071435 0.433183819055557 4.0
24.659837316 0.47648149728775 5.0
38.102908751 0.422329992055893 6.0
38.102908751 0.57693886756897 7.0
38.34829669 0.689752519130707 0.0
28.277225983 0.479798257350922 1.0
10.071070707 0.523808717727661 2.0
0 0.217210829257965 3.0
10.071070707 0.456964492797852 4.0
28.277225983 0.491562336683273 5.0
38.34829669 0.303100556135178 6.0
38.34829669 0.583636879920959 7.0
31.13885609 0.828653693199158 0.0
20.686107563 0.52132785320282 1.0
10.452748527 0.551999926567078 2.0
0 0.202164590358734 3.0
10.452748527 0.429225265979767 4.0
20.686107563 0.458648681640625 5.0
31.13885609 0.407663404941559 6.0
31.13885609 0.400772571563721 7.0
35.130346793 0.231792241334915 0.0
24.419018025 0.135346367955208 1.0
10.711328768 0.145855277776718 2.0
0 0.212332427501678 3.0
10.711328768 0.406260430812836 4.0
24.419018025 0.451832205057144 5.0
35.130346793 2.14815354347229 6.0
35.130346793 2.69133973121643 7.0
37.970631847 0.882675647735596 0.0
26.545844038 0.541614472866058 1.0
11.424787809 0.580301702022552 2.0
0 0.232728838920593 3.0
11.424787809 0.43044438958168 4.0
26.545844038 0.461890876293182 5.0
37.970631847 0.524353444576263 6.0
37.970631847 0.275272876024246 7.0
37.012414901 0.785380125045776 0.0
22.13356734 0.532437562942505 1.0
14.87884756 0.571928918361664 2.0
0 0.234910443425179 3.0
14.87884756 0.436798989772797 4.0
22.13356734 0.486879914999008 5.0
37.012414901 0.29100438952446 6.0
37.012414901 0.315387010574341 7.0
36.779265883 0.786756575107574 0.0
23.404145317 0.498701304197311 1.0
13.375120566 0.527487993240356 2.0
0 0.179477170109749 3.0
13.375120566 0.435167253017426 4.0
23.404145317 0.441477417945862 5.0
36.779265883 0.292698740959167 6.0
36.779265883 0.371560633182526 7.0
42.665291585 0.835052907466888 0.0
29.517542924 0.328700840473175 1.0
13.147748661 0.360368579626083 2.0
0 0.23216313123703 3.0
13.147748661 0.412986993789673 4.0
29.517542924 0.456535547971725 5.0
42.665291585 0.307083904743195 6.0
42.665291585 0.169484212994576 7.0
35.413046011 0.778972208499908 0.0
20.7862186 0.471845209598541 1.0
14.626827411 0.532719016075134 2.0
0 0.239923730492592 3.0
14.626827411 0.460447013378143 4.0
20.7862186 0.504892945289612 5.0
35.413046011 0.556623458862305 6.0
35.413046011 0.500726282596588 7.0
35.966339402 0.838905990123749 0.0
24.209055046 0.515986859798431 1.0
11.757284356 0.558837413787842 2.0
0 0.212583720684052 3.0
11.757284356 0.438374102115631 4.0
24.209055046 0.467411279678345 5.0
35.966339402 0.489383667707443 6.0
35.966339402 0.471481651067734 7.0
39.418493945 0.773100435733795 0.0
25.880638068 0.476518869400024 1.0
13.537855877 0.525939702987671 2.0
0 0.208407908678055 3.0
13.537855877 0.447026193141937 4.0
25.880638068 0.491723120212555 5.0
39.418493945 0.307085871696472 6.0
39.418493945 0.47972360253334 7.0
32.151549765 0.532033622264862 0.0
22.152388929 0.470560908317566 1.0
9.9991608362 0.502604782581329 2.0
0 0.218607828021049 3.0
9.9991608362 0.435739487409592 4.0
22.152388929 0.477415293455124 5.0
32.151549765 0.224642693996429 6.0
32.151549765 0.224260151386261 7.0
31.845645841 0.79373037815094 0.0
21.472593563 0.538468182086945 1.0
10.373052277 0.583850026130676 2.0
0 0.243438094854355 3.0
10.373052277 0.428942620754242 4.0
21.472593563 0.478391081094742 5.0
31.845645841 0.435470104217529 6.0
31.845645841 0.456660658121109 7.0
40.557128571 0.642606794834137 0.0
29.931774079 0.458705574274063 1.0
10.625354492 0.49876606464386 2.0
0 0.220759630203247 3.0
10.625354492 0.453893929719925 4.0
29.931774079 0.491826981306076 5.0
40.557128571 0.281459778547287 6.0
40.557128571 0.550151467323303 7.0
36.793774817 0.678914785385132 0.0
26.241358064 0.467928230762482 1.0
10.552416753 0.495481967926025 2.0
0 0.225812911987305 3.0
10.552416753 0.422163724899292 4.0
26.241358064 0.475648134946823 5.0
36.793774817 0.224385097622871 6.0
36.793774817 0.320366889238358 7.0
41.333576969 0.809330403804779 0.0
27.007286594 0.474648237228394 1.0
14.326290374 0.52199250459671 2.0
0 0.199088141322136 3.0
14.326290374 0.441621571779251 4.0
27.007286594 0.486652612686157 5.0
41.333576969 0.286911129951477 6.0
41.333576969 0.434594005346298 7.0
43.791455275 0.708551585674286 0.0
29.094231349 0.477465391159058 1.0
14.697223926 0.515256583690643 2.0
0 0.217513531446457 3.0
14.697223926 0.452193707227707 4.0
29.094231349 0.492011368274689 5.0
43.791455275 0.282183140516281 6.0
43.791455275 0.591366350650787 7.0
41.481697076 0.797793328762054 0.0
26.568629648 0.513566553592682 1.0
14.913067428 0.549324214458466 2.0
0 0.221061259508133 3.0
14.913067428 0.437928825616837 4.0
26.568629648 0.503415107727051 5.0
41.481697076 0.524230659008026 6.0
41.481697076 0.417155981063843 7.0
35.532774621 0.769791781902313 0.0
25.55101033 0.40546989440918 1.0
9.9817642912 0.43705627322197 2.0
0 0.212361067533493 3.0
9.9817642912 0.45470780134201 4.0
25.55101033 0.495052993297577 5.0
35.532774621 0.551842510700226 6.0
35.532774621 0.456487029790878 7.0
34.894006808 0.818658947944641 0.0
24.580727451 0.47858202457428 1.0
10.313279357 0.533339202404022 2.0
0 0.200773850083351 3.0
10.313279357 0.44056248664856 4.0
24.580727451 0.473890781402588 5.0
34.894006808 0.386016488075256 6.0
34.894006808 0.51338791847229 7.0
38.893810297 0.80176442861557 0.0
25.470645748 0.437883168458939 1.0
13.423164549 0.476764410734177 2.0
0 0.216817662119865 3.0
13.423164549 0.445521414279938 4.0
25.470645748 0.492699980735779 5.0
38.893810297 0.562205970287323 6.0
38.893810297 0.501301646232605 7.0
39.624663522 0.720480620861053 0.0
30.244927686 0.469310879707336 1.0
9.3797358363 0.52405446767807 2.0
-0.51314313752 0.182924702763557 3.0
8.8665926988 0.440292418003082 4.0
30.758070824 0.465507686138153 5.0
39.624663522 0.325776219367981 6.0
39.624663522 0.602838337421417 7.0
42.516231091 0.107762150466442 0.0
30.477530163 1.74265503883362 1.0
12.038700928 1.77203917503357 2.0
0 0.115819543600082 3.0
12.038700928 0.405315697193146 4.0
30.477530163 0.439746290445328 5.0
42.516231091 0.00878572463989258 6.0
42.516231091 15.7744998931885 7.0
38.374768577 0.935347497463226 0.0
25.100410303 0.146507918834686 1.0
13.274358274 0.185639381408691 2.0
0 0.273235440254211 3.0
13.274358274 0.425538003444672 4.0
25.100410303 0.481759160757065 5.0
38.374768577 3.80069637298584 6.0
38.374768577 -0.0799577087163925 7.0
33.351459508 0.653099775314331 0.0
24.24183423 0.448150217533112 1.0
9.1096252774 0.485326647758484 2.0
0 0.202369466423988 3.0
9.1096252774 0.45039826631546 4.0
24.24183423 0.491240680217743 5.0
33.351459508 0.24879164993763 6.0
33.351459508 0.526399731636047 7.0
37.08812014 0.823539137840271 0.0
24.469772372 0.445278763771057 1.0
12.618347768 0.488136410713196 2.0
0 0.21197646856308 3.0
12.618347768 0.457527458667755 4.0
24.469772372 0.498036444187164 5.0
37.08812014 0.42143976688385 6.0
37.08812014 0.536540627479553 7.0
44.445084649 0.801821172237396 0.0
30.066790167 0.468158304691315 1.0
14.378294481 0.523413181304932 2.0
0 0.229920834302902 3.0
14.378294481 0.446336925029755 4.0
30.066790167 0.512181401252747 5.0
44.445084649 0.521897971630096 6.0
44.445084649 0.548552751541138 7.0
37.22322104 0.783655762672424 0.0
25.951146113 0.524165987968445 1.0
11.272074927 0.561542153358459 2.0
0 0.215939819812775 3.0
11.272074927 0.433879733085632 4.0
25.951146113 0.479905605316162 5.0
37.22322104 0.433674424886703 6.0
37.22322104 0.449913144111633 7.0
44.605944609 0.746692478656769 0.0
29.214608515 0.386610478162766 1.0
15.391336093 0.421648353338242 2.0
0 0.283472031354904 3.0
15.391336093 0.455490589141846 4.0
29.214608515 0.516548991203308 5.0
44.605944609 0.201327949762344 6.0
44.605944609 0.315508514642715 7.0
39.553698498 0.793002188205719 0.0
28.17711576 0.513224542140961 1.0
11.376582739 0.552668929100037 2.0
-1.2942488147 0.194055020809174 3.0
10.082333924 0.423885554075241 4.0
29.471364574 0.453202277421951 5.0
39.553698498 0.411467522382736 6.0
39.553698498 0.408542811870575 7.0
43.287363281 0.298907071352005 0.0
32.742330647 0.457957804203033 1.0
10.545032634 0.49205270409584 2.0
0 0.216464668512344 3.0
10.545032634 0.432183533906937 4.0
32.742330647 0.479440659284592 5.0
43.287363281 0.063571110367775 6.0
43.287363281 0.0830489322543144 7.0
33.992156471 0.853398740291595 0.0
23.696427762 0.528397977352142 1.0
10.29572871 0.5522580742836 2.0
0 0.208763971924782 3.0
10.29572871 0.416666448116302 4.0
23.696427762 0.464191973209381 5.0
33.992156471 0.544486820697784 6.0
33.992156471 0.329224973917007 7.0
36.860093248 0.491945624351501 0.0
25.387893295 0.463874369859695 1.0
11.472199952 0.497593820095062 2.0
0 0.212247997522354 3.0
11.472199952 0.450315713882446 4.0
25.387893295 0.492523431777954 5.0
36.860093248 0.253744572401047 6.0
36.860093248 0.230671539902687 7.0
39.672962738 0.802188098430634 0.0
26.143344656 0.519133031368256 1.0
13.529618083 0.558072149753571 2.0
0 0.240838795900345 3.0
13.529618083 0.438046336174011 4.0
26.143344656 0.493671596050262 5.0
39.672962738 0.619565665721893 6.0
39.672962738 0.52202433347702 7.0
32.8535328 0.818777441978455 0.0
22.510390494 0.510645925998688 1.0
10.343142306 0.551082968711853 2.0
0 0.244649052619934 3.0
10.343142306 0.45677325129509 4.0
22.510390494 0.496970593929291 5.0
32.8535328 0.522443294525146 6.0
32.8535328 0.529251992702484 7.0
36.422494688 0.814515650272369 0.0
22.927536178 0.565119743347168 1.0
13.49495851 0.611795663833618 2.0
0 0.252452313899994 3.0
13.49495851 0.424361705780029 4.0
22.927536178 0.471203505992889 5.0
36.422494688 0.433085888624191 6.0
36.422494688 0.419890999794006 7.0
32.082396771 0.523463785648346 0.0
23.483778533 0.118190005421638 1.0
8.5986182376 0.142070055007935 2.0
0 0.250994294881821 3.0
8.5986182376 0.407981216907501 4.0
23.483778533 0.475044399499893 5.0
32.082396771 1.75894832611084 6.0
32.082396771 0.195892497897148 7.0
43.742460686 0.807655513286591 0.0
30.22029377 0.514916896820068 1.0
13.522166916 0.543078601360321 2.0
0 0.196560308337212 3.0
13.522166916 0.444112032651901 4.0
30.22029377 0.464447468519211 5.0
43.742460686 0.40048959851265 6.0
43.742460686 0.377436846494675 7.0
37.830554934 0.153893768787384 0.0
26.515637118 0.490710586309433 1.0
11.314917816 0.50986236333847 2.0
0 0.13872617483139 3.0
11.314917816 0.429201632738113 4.0
26.515637118 0.455825746059418 5.0
37.830554934 0.049899835139513 6.0
37.830554934 -0.0239836275577545 7.0
45.388745763 0.819045722484589 0.0
30.82392363 0.44107449054718 1.0
14.564822134 0.481811046600342 2.0
0 0.211774438619614 3.0
14.564822134 0.445371866226196 4.0
30.82392363 0.478679835796356 5.0
45.388745763 0.54824036359787 6.0
45.388745763 0.512443602085114 7.0
40.010376396 0.833685517311096 0.0
29.307244164 0.516844630241394 1.0
10.703132231 0.555249810218811 2.0
0 0.245881825685501 3.0
10.703132231 0.448737919330597 4.0
29.307244164 0.495409727096558 5.0
40.010376396 0.505243003368378 6.0
40.010376396 0.527343928813934 7.0
45.566990187 0.667127907276154 0.0
30.534595735 0.471810549497604 1.0
15.032394453 0.502224266529083 2.0
0 0.205851137638092 3.0
15.032394453 0.433914601802826 4.0
30.534595735 0.49374932050705 5.0
45.566990187 0.222603261470795 6.0
45.566990187 0.349613875150681 7.0
32.300808889 0.192198380827904 0.0
23.352129325 0.0638346076011658 1.0
8.9486795641 0.0579142421483994 2.0
0 0.185486271977425 3.0
8.9486795641 0.378392189741135 4.0
23.352129325 0.433267891407013 5.0
32.300808889 2.042076587677 6.0
32.300808889 0.207809910178185 7.0
36.805463378 0.349518835544586 0.0
21.194272929 0.448767900466919 1.0
15.611190449 0.477525472640991 2.0
0 0.15731754899025 3.0
15.611190449 0.430160015821457 4.0
21.194272929 0.448507726192474 5.0
36.805463378 0.271630823612213 6.0
36.805463378 0.108321413397789 7.0
45.039727992 0.823137104511261 0.0
30.80326593 0.47361347079277 1.0
14.236462062 0.52825129032135 2.0
-0.8432667817 0.206390202045441 3.0
13.39319528 0.448150426149368 4.0
31.646532712 0.469447821378708 5.0
45.039727992 0.500889718532562 6.0
45.039727992 0.534724056720734 7.0
36.669966877 0.756607711315155 0.0
26.160739166 0.504479646682739 1.0
10.509227711 0.531142711639404 2.0
0 0.192024558782578 3.0
10.509227711 0.43343597650528 4.0
26.160739166 0.463618725538254 5.0
36.669966877 0.280248939990997 6.0
36.669966877 0.369796633720398 7.0
40.401972767 0.77946412563324 0.0
30.207092791 0.505314826965332 1.0
10.194879977 0.547717750072479 2.0
0 0.25090354681015 3.0
10.194879977 0.459393560886383 4.0
30.207092791 0.494312226772308 5.0
40.401972767 0.488574683666229 6.0
40.401972767 0.340999901294708 7.0
41.491221092 0.861725389957428 0.0
30.356832429 0.465277820825577 1.0
11.134388663 0.522004663944244 2.0
0 0.217917054891586 3.0
11.134388663 0.45466148853302 4.0
30.356832429 0.494959682226181 5.0
41.491221092 0.436310738325119 6.0
41.491221092 0.498245388269424 7.0
32.092723767 0.775565624237061 0.0
23.053601154 0.461987733840942 1.0
9.0391226128 0.501230835914612 2.0
0 0.23835700750351 3.0
9.0391226128 0.452335029840469 4.0
23.053601154 0.482768714427948 5.0
32.092723767 0.370695114135742 6.0
32.092723767 0.322657376527786 7.0
39.133472273 0.65519118309021 0.0
30.010976118 0.427304804325104 1.0
9.1224961547 0.463709592819214 2.0
0 0.246744215488434 3.0
9.1224961547 0.444575130939484 4.0
30.010976118 0.495782971382141 5.0
39.133472273 0.176375344395638 6.0
39.133472273 0.240880891680717 7.0
37.053979998 0.496413469314575 0.0
23.198859944 0.476689249277115 1.0
13.855120054 0.512836933135986 2.0
0 0.19786012172699 3.0
13.855120054 0.434638619422913 4.0
23.198859944 0.460710287094116 5.0
37.053979998 0.239212170243263 6.0
37.053979998 0.172073602676392 7.0
39.864494306 0.798298060894012 0.0
30.858392243 0.480160474777222 1.0
9.0061020631 0.514791011810303 2.0
0 0.244815409183502 3.0
9.0061020631 0.442190319299698 4.0
30.858392243 0.505454778671265 5.0
39.864494306 0.360647082328796 6.0
39.864494306 0.32193511724472 7.0
39.054921308 0.707032859325409 0.0
29.468453199 0.467125564813614 1.0
9.5864681089 0.508595764636993 2.0
0 0.19974459707737 3.0
9.5864681089 0.435658097267151 4.0
29.468453199 0.489574193954468 5.0
39.054921308 0.290908753871918 6.0
39.054921308 0.570812821388245 7.0
33.466015237 0.352787017822266 0.0
23.587793621 0.483494371175766 1.0
9.8782216161 0.520083904266357 2.0
0 0.204697132110596 3.0
9.8782216161 0.450554132461548 4.0
23.587793621 0.481909811496735 5.0
33.466015237 0.244381189346313 6.0
33.466015237 0.106617256999016 7.0
37.653527764 0.844609260559082 0.0
25.326112836 0.535805642604828 1.0
12.327414928 0.566404759883881 2.0
0 0.221628084778786 3.0
12.327414928 0.422803848981857 4.0
25.326112836 0.471214354038239 5.0
37.653527764 0.344829797744751 6.0
37.653527764 0.294674694538116 7.0
37.577039283 0.77074807882309 0.0
25.072263055 0.497348815202713 1.0
12.504776228 0.54113233089447 2.0
0 0.217451497912407 3.0
12.504776228 0.444974899291992 4.0
25.072263055 0.478440463542938 5.0
37.577039283 0.309034466743469 6.0
37.577039283 0.577344417572021 7.0
36.318307627 0.850686490535736 0.0
24.535944301 0.491100549697876 1.0
11.782363326 0.525781989097595 2.0
0 0.238496959209442 3.0
11.782363326 0.442957311868668 4.0
24.535944301 0.487862557172775 5.0
36.318307627 0.414986789226532 6.0
36.318307627 0.309123456478119 7.0
36.801464194 0.822178304195404 0.0
24.290044341 0.459740847349167 1.0
12.511419853 0.500440657138824 2.0
0 0.202016532421112 3.0
12.511419853 0.436940908432007 4.0
24.290044341 0.488248407840729 5.0
36.801464194 0.556570410728455 6.0
36.801464194 0.489524275064468 7.0
39.56095567 0.728096544742584 0.0
28.15773783 0.526086270809174 1.0
11.40321784 0.560514688491821 2.0
0 0.224995955824852 3.0
11.40321784 0.424825847148895 4.0
28.15773783 0.47204601764679 5.0
39.56095567 0.277188926935196 6.0
39.56095567 0.308047026395798 7.0
40.443539209 0.725812375545502 0.0
25.360759604 0.482755750417709 1.0
15.082779605 0.515860557556152 2.0
0 0.197520405054092 3.0
15.082779605 0.420302093029022 4.0
25.360759604 0.468140482902527 5.0
40.443539209 0.303331285715103 6.0
40.443539209 0.337807387113571 7.0
40.316920108 0.806762635707855 0.0
27.366582023 0.488151788711548 1.0
12.950338085 0.544325768947601 2.0
0 0.231440186500549 3.0
12.950338085 0.460821926593781 4.0
27.366582023 0.482191771268845 5.0
40.316920108 0.507617950439453 6.0
40.316920108 0.563139796257019 7.0
33.579405458 0.164192318916321 0.0
23.019033323 0.987584590911865 1.0
10.560372135 0.987498879432678 2.0
0 0.140216618776321 3.0
10.560372135 0.423770993947983 4.0
23.019033323 0.443666219711304 5.0
33.579405458 -0.0555683746933937 6.0
33.579405458 1.49115216732025 7.0
34.562810956 0.47293958067894 0.0
23.168439176 0.451626062393188 1.0
11.394371781 0.482025057077408 2.0
0 0.172515615820885 3.0
11.394371781 0.427609920501709 4.0
23.168439176 0.457060217857361 5.0
34.562810956 0.317228317260742 6.0
34.562810956 0.159909710288048 7.0
37.518824119 0.84033340215683 0.0
25.014720915 0.506138563156128 1.0
12.504103204 0.542189478874207 2.0
0 0.22418786585331 3.0
12.504103204 0.42998868227005 4.0
25.014720915 0.463923782110214 5.0
37.518824119 0.356952130794525 6.0
37.518824119 0.30832314491272 7.0
39.547705636 0.743907511234283 0.0
26.235912464 0.479886144399643 1.0
13.311793172 0.5180703997612 2.0
0 0.204040840268135 3.0
13.311793172 0.438969701528549 4.0
26.235912464 0.467820525169373 5.0
39.547705636 0.273032248020172 6.0
39.547705636 0.356776595115662 7.0
42.834945783 0.797136664390564 0.0
27.69423425 0.497721791267395 1.0
15.140711533 0.546959459781647 2.0
0 0.231731474399567 3.0
15.140711533 0.438295811414719 4.0
27.69423425 0.492678582668304 5.0
42.834945783 0.49882447719574 6.0
42.834945783 0.521791040897369 7.0
38.473622953 0.71720689535141 0.0
25.667479717 0.488111138343811 1.0
12.806143237 0.531570792198181 2.0
0 0.188634425401688 3.0
12.806143237 0.435577273368835 4.0
25.667479717 0.480162739753723 5.0
38.473622953 0.293171912431717 6.0
38.473622953 0.606328129768372 7.0
45.238523457 0.783157229423523 0.0
31.33795643 0.525926113128662 1.0
13.900567026 0.545917570590973 2.0
0 0.201349630951881 3.0
13.900567026 0.421518057584763 4.0
31.33795643 0.461016595363617 5.0
45.238523457 0.362048387527466 6.0
45.238523457 0.277319222688675 7.0
36.266282562 0.797958791255951 0.0
23.153527457 0.475134909152985 1.0
13.112755105 0.533487558364868 2.0
0 0.230640649795532 3.0
13.112755105 0.465743839740753 4.0
23.153527457 0.505384564399719 5.0
36.266282562 0.473486363887787 6.0
36.266282562 0.556395411491394 7.0
37.845233948 0.833093523979187 0.0
24.957293945 0.499535381793976 1.0
12.887940003 0.530216336250305 2.0
0 0.220720082521439 3.0
12.887940003 0.448181062936783 4.0
24.957293945 0.483818739652634 5.0
37.845233948 0.326466232538223 6.0
37.845233948 0.37402155995369 7.0
40.414778929 0.556234896183014 0.0
24.767056635 0.333821088075638 1.0
15.647722294 0.345989435911179 2.0
0 0.22102415561676 3.0
15.647722294 0.438133656978607 4.0
24.767056635 0.472770690917969 5.0
40.414778929 -0.0216581858694553 6.0
40.414778929 0.23974397778511 7.0
47.104382604 0.857323527336121 0.0
31.371428063 0.493219316005707 1.0
15.732954541 0.546459376811981 2.0
0 0.218034461140633 3.0
15.732954541 0.43857741355896 4.0
31.371428063 0.480269581079483 5.0
47.104382604 0.37899586558342 6.0
47.104382604 0.503019392490387 7.0
37.557508388 0.470868796110153 0.0
22.881195893 0.475994110107422 1.0
14.676312494 0.506914794445038 2.0
0 0.192366659641266 3.0
14.676312494 0.42032253742218 4.0
22.881195893 0.47070175409317 5.0
37.557508388 0.00154859572649002 6.0
37.557508388 0.107718147337437 7.0
40.532317042 0.835288286209106 0.0
25.290512841 0.470301926136017 1.0
15.241804202 0.52490109205246 2.0
0 0.212425827980042 3.0
15.241804202 0.440819591283798 4.0
25.290512841 0.492089807987213 5.0
40.532317042 0.34355902671814 6.0
40.532317042 0.469875246286392 7.0
32.594348946 0.478582203388214 0.0
19.62547534 0.436318188905716 1.0
12.968873606 0.472652792930603 2.0
0 0.165714010596275 3.0
12.968873606 0.451954782009125 4.0
19.62547534 0.451986908912659 5.0
32.594348946 0.284762620925903 6.0
32.594348946 0.483106672763824 7.0
30.392342502 0.785122990608215 0.0
20.487642954 0.394538402557373 1.0
9.9046995479 0.429537504911423 2.0
0 0.205640405416489 3.0
9.9046995479 0.452246218919754 4.0
20.487642954 0.488789618015289 5.0
30.392342502 0.559096574783325 6.0
30.392342502 0.464942425489426 7.0
40.607017947 0.744522929191589 0.0
29.93470242 0.472891181707382 1.0
10.672315527 0.525819063186646 2.0
0 0.216996148228645 3.0
10.672315527 0.449378609657288 4.0
29.93470242 0.476381599903107 5.0
40.607017947 0.319554626941681 6.0
40.607017947 0.592917799949646 7.0
39.11337914 0.153473138809204 0.0
28.397508882 0.29890251159668 1.0
10.715870258 0.324839651584625 2.0
0 0.220989137887955 3.0
10.715870258 0.428418070077896 4.0
28.397508882 0.460609644651413 5.0
39.11337914 0.545633673667908 6.0
39.11337914 1.18321931362152 7.0
35.280627531 0.816200137138367 0.0
25.522950121 0.536251604557037 1.0
9.7576774108 0.581959307193756 2.0
0 0.233102843165398 3.0
9.7576774108 0.421558827161789 4.0
25.522950121 0.462329506874084 5.0
35.280627531 0.346254706382751 6.0
35.280627531 0.30106794834137 7.0
32.847447182 0.789288640022278 0.0
19.105381186 0.324578583240509 1.0
13.742065995 0.358002245426178 2.0
0 0.206191539764404 3.0
13.742065995 0.457682609558105 4.0
19.105381186 0.494605958461761 5.0
32.847447182 0.585290789604187 6.0
32.847447182 0.435584306716919 7.0
34.14972673 0.141642183065414 0.0
19.081196776 0.942439913749695 1.0
15.068529955 0.958122313022614 2.0
0 0.13691671192646 3.0
15.068529955 0.423149526119232 4.0
19.081196776 0.437961280345917 5.0
34.14972673 -0.07948337495327 6.0
34.14972673 1.28819298744202 7.0
36.823986173 0.73734587430954 0.0
27.983721771 0.485915660858154 1.0
8.8402644018 0.518760800361633 2.0
0 0.190528646111488 3.0
8.8402644018 0.420883774757385 4.0
27.983721771 0.452564656734467 5.0
36.823986173 0.306255519390106 6.0
36.823986173 0.336701631546021 7.0
43.615049092 0.193106591701508 0.0
31.355429376 0.503526151180267 1.0
12.259619715 0.533508241176605 2.0
0 0.180047810077667 3.0
12.259619715 0.431888163089752 4.0
31.355429376 0.473054319620132 5.0
43.615049092 0.0751299634575844 6.0
43.615049092 -0.0203849785029888 7.0
39.316465342 0.167208656668663 0.0
25.189593371 0.825720131397247 1.0
14.126871971 0.849033176898956 2.0
0 0.122839465737343 3.0
14.126871971 0.438793987035751 4.0
25.189593371 0.451445549726486 5.0
39.316465342 -0.191388815641403 6.0
39.316465342 2.2212233543396 7.0
36.676849778 0.556848466396332 0.0
26.033044381 0.429980218410492 1.0
10.643805397 0.447683960199356 2.0
0 0.180013179779053 3.0
10.643805397 0.415343642234802 4.0
26.033044381 0.453181892633438 5.0
36.676849778 -0.0115641998127103 6.0
36.676849778 0.265274077653885 7.0
36.163160136 0.662860095500946 0.0
20.77667961 0.445120632648468 1.0
15.386480527 0.476068615913391 2.0
0 0.225154057145119 3.0
15.386480527 0.432472884654999 4.0
20.77667961 0.473351925611496 5.0
36.163160136 0.17772513628006 6.0
36.163160136 0.286793619394302 7.0
40.305175496 0.860043466091156 0.0
26.598094238 0.473178565502167 1.0
13.707081258 0.528907299041748 2.0
0 0.227568432688713 3.0
13.707081258 0.45763772726059 4.0
26.598094238 0.491118878126144 5.0
40.305175496 0.370219707489014 6.0
40.305175496 0.457871496677399 7.0
34.293236222 0.786695063114166 0.0
19.681914121 0.453191220760345 1.0
14.611322101 0.500743627548218 2.0
0 0.222016662359238 3.0
14.611322101 0.45496666431427 4.0
19.681914121 0.495029300451279 5.0
34.293236222 0.512635171413422 6.0
34.293236222 0.527366757392883 7.0
38.347299924 0.810702979564667 0.0
24.696601669 0.446979641914368 1.0
13.650698255 0.489473074674606 2.0
0 0.21877521276474 3.0
13.650698255 0.448098063468933 4.0
24.696601669 0.488399475812912 5.0
38.347299924 0.553648471832275 6.0
38.347299924 0.511225283145905 7.0
39.036910781 0.833088397979736 0.0
30.172693912 0.486397504806519 1.0
8.8642168691 0.528972625732422 2.0
0 0.222864761948586 3.0
8.8642168691 0.451327860355377 4.0
30.172693912 0.504492104053497 5.0
39.036910781 0.34694230556488 6.0
39.036910781 0.446685642004013 7.0
36.237500285 0.834708869457245 0.0
23.386338638 0.42783111333847 1.0
12.851161647 0.461830347776413 2.0
0 0.202673763036728 3.0
12.851161647 0.450617849826813 4.0
23.386338638 0.487475603818893 5.0
36.237500285 0.495664745569229 6.0
36.237500285 0.517594397068024 7.0
36.900540753 0.689771294593811 0.0
25.827912025 0.469310343265533 1.0
11.072628728 0.512154281139374 2.0
0 0.199667632579803 3.0
11.072628728 0.450947254896164 4.0
25.827912025 0.457955569028854 5.0
36.900540753 0.282003670930862 6.0
36.900540753 0.515583992004395 7.0
29.897899633 0.371837884187698 0.0
21.013378654 0.478151887655258 1.0
8.8845209788 0.514681279659271 2.0
0 0.210011467337608 3.0
8.8845209788 0.439906865358353 4.0
21.013378654 0.47275573015213 5.0
29.897899633 0.29294615983963 6.0
29.897899633 0.132280513644218 7.0
41.071373967 0.85857355594635 0.0
26.148963023 0.437184631824493 1.0
14.922410944 0.482749044895172 2.0
0 0.204023033380508 3.0
14.922410944 0.459974974393845 4.0
26.148963023 0.47003909945488 5.0
41.071373967 0.38837605714798 6.0
41.071373967 0.49370139837265 7.0
37.123617582 0.623152792453766 0.0
24.027785174 0.452051132917404 1.0
13.095832408 0.486309319734573 2.0
0 0.214345276355743 3.0
13.095832408 0.438201785087585 4.0
24.027785174 0.471274733543396 5.0
37.123617582 0.0859622210264206 6.0
37.123617582 0.258602350950241 7.0
38.348981744 0.212327629327774 0.0
22.510464493 0.232768326997757 1.0
15.838517252 0.250914663076401 2.0
0 0.22936649620533 3.0
15.838517252 0.439298152923584 4.0
22.510464493 0.45872363448143 5.0
38.348981744 0.677861034870148 6.0
38.348981744 0.0388529226183891 7.0
37.897829023 0.784526169300079 0.0
28.975694369 0.517520487308502 1.0
8.9221346538 0.556415379047394 2.0
0 0.248773604631424 3.0
8.9221346538 0.454890072345734 4.0
28.975694369 0.483397006988525 5.0
37.897829023 0.544650018215179 6.0
37.897829023 0.515705525875092 7.0
32.839543182 0.841200828552246 0.0
23.33116807 0.315212339162827 1.0
9.508375112 0.350408136844635 2.0
0 0.21237625181675 3.0
9.508375112 0.460321485996246 4.0
23.33116807 0.498431771993637 5.0
32.839543182 0.58209240436554 6.0
32.839543182 0.451024889945984 7.0
45.814256844 0.682292997837067 0.0
33.908219787 0.488246649503708 1.0
11.906037057 0.54346776008606 2.0
0 0.196275174617767 3.0
11.906037057 0.451681643724442 4.0
33.908219787 0.487065851688385 5.0
45.814256844 0.41088992357254 6.0
45.814256844 0.642491698265076 7.0
38.439958634 0.799407422542572 0.0
26.743506222 0.429951220750809 1.0
11.696452411 0.454189002513885 2.0
0 0.214524537324905 3.0
11.696452411 0.405736982822418 4.0
26.743506222 0.446028083562851 5.0
38.439958634 0.102512009441853 6.0
38.439958634 0.476474016904831 7.0
33.022424116 0.708929657936096 0.0
22.336407878 0.502338528633118 1.0
10.686016238 0.534897327423096 2.0
0 0.223665297031403 3.0
10.686016238 0.435609847307205 4.0
22.336407878 0.485464096069336 5.0
33.022424116 0.316222548484802 6.0
33.022424116 0.426861703395844 7.0
36.232706353 0.846128702163696 0.0
25.757367318 0.476988852024078 1.0
10.475339035 0.532946646213531 2.0
0 0.224471241235733 3.0
10.475339035 0.455255150794983 4.0
25.757367318 0.491031527519226 5.0
36.232706353 0.43617057800293 6.0
36.232706353 0.457829803228378 7.0
45.8365445 0.845935583114624 0.0
33.58061703 0.509175658226013 1.0
12.255927471 0.548246741294861 2.0
0 0.256445556879044 3.0
12.255927471 0.456503212451935 4.0
33.58061703 0.504544973373413 5.0
45.8365445 0.515251278877258 6.0
45.8365445 0.370245933532715 7.0
39.344253832 0.727225303649902 0.0
27.946117456 0.481205821037292 1.0
11.398136376 0.522868096828461 2.0
0 0.211924061179161 3.0
11.398136376 0.436674982309341 4.0
27.946117456 0.469503313302994 5.0
39.344253832 0.257318168878555 6.0
39.344253832 0.338038444519043 7.0
43.613379879 0.782723128795624 0.0
29.144363875 0.550306916236877 1.0
14.469016005 0.575423538684845 2.0
0 0.219284847378731 3.0
14.469016005 0.420279264450073 4.0
29.144363875 0.467034876346588 5.0
43.613379879 0.380591779947281 6.0
43.613379879 0.268673866987228 7.0
41.603825526 0.8511843085289 0.0
32.060099244 0.524749636650085 1.0
9.5437262816 0.563086628913879 2.0
0 0.22827285528183 3.0
9.5437262816 0.437069892883301 4.0
32.060099244 0.464132964611053 5.0
41.603825526 0.580938935279846 6.0
41.603825526 0.523666620254517 7.0
41.290499737 0.801497101783752 0.0
29.938150836 0.495566040277481 1.0
11.352348902 0.546329975128174 2.0
0 0.237442418932915 3.0
11.352348902 0.446959614753723 4.0
29.938150836 0.486640512943268 5.0
41.290499737 0.538205087184906 6.0
41.290499737 0.539330363273621 7.0
38.21592954 0.863522171974182 0.0
24.146046072 0.468281179666519 1.0
14.069883468 0.522523581981659 2.0
0 0.21638697385788 3.0
14.069883468 0.448403805494308 4.0
24.146046072 0.49951446056366 5.0
38.21592954 0.432720094919205 6.0
38.21592954 0.498788744211197 7.0
34.764993855 0.804285168647766 0.0
21.168684565 0.545716643333435 1.0
13.59630929 0.584675252437592 2.0
0 0.244631439447403 3.0
13.59630929 0.437400698661804 4.0
21.168684565 0.480846703052521 5.0
34.764993855 0.276511549949646 6.0
34.764993855 0.370005458593369 7.0
34.789226058 0.747975766658783 0.0
25.072029366 0.503214359283447 1.0
9.7171966918 0.53385454416275 2.0
0 0.213837444782257 3.0
9.7171966918 0.433814704418182 4.0
25.072029366 0.462974339723587 5.0
34.789226058 0.248471558094025 6.0
34.789226058 0.339322805404663 7.0
42.488773183 0.784261345863342 0.0
27.1767235 0.416187703609467 1.0
15.312049683 0.450192630290985 2.0
0 0.202218979597092 3.0
15.312049683 0.44810026884079 4.0
27.1767235 0.483612358570099 5.0
42.488773183 0.507346987724304 6.0
42.488773183 0.518072664737701 7.0
34.702233392 0.673397839069366 0.0
22.091827715 0.469591915607452 1.0
12.610405677 0.494048565626144 2.0
0 0.188118740916252 3.0
12.610405677 0.417220771312714 4.0
22.091827715 0.448714405298233 5.0
34.702233392 0.22102016210556 6.0
34.702233392 0.314397394657135 7.0
41.831412446 0.413453876972198 0.0
26.30240797 0.538361668586731 1.0
15.529004476 0.579675316810608 2.0
0 0.184669941663742 3.0
15.529004476 0.43661430478096 4.0
26.30240797 0.456446141004562 5.0
41.831412446 0.190557420253754 6.0
41.831412446 -0.0193130299448967 7.0
45.908850318 0.768204987049103 0.0
30.587589099 0.353834599256516 1.0
15.321261219 0.368061184883118 2.0
0 0.239825814962387 3.0
15.321261219 0.413129061460495 4.0
30.587589099 0.46953558921814 5.0
45.908850318 0.255926847457886 6.0
45.908850318 0.303692132234573 7.0
33.104401139 0.569926261901855 0.0
23.063487553 0.456467121839523 1.0
10.040913586 0.484063982963562 2.0
0 0.169265776872635 3.0
10.040913586 0.409132540225983 4.0
23.063487553 0.4262815117836 5.0
33.104401139 0.0120941177010536 6.0
33.104401139 0.186056971549988 7.0
35.835170704 0.120568253099918 0.0
22.566106428 0.797751188278198 1.0
13.269064276 0.813399910926819 2.0
0 0.123183041810989 3.0
13.269064276 0.427151262760162 4.0
22.566106428 0.442026913166046 5.0
35.835170704 -0.0666267424821854 6.0
35.835170704 1.1040997505188 7.0
45.844507362 0.837592363357544 0.0
30.026497858 0.536162436008453 1.0
15.818009504 0.575047135353088 2.0
0 0.239195078611374 3.0
15.818009504 0.424634486436844 4.0
30.026497858 0.472514361143112 5.0
45.844507362 0.484661400318146 6.0
45.844507362 0.421688139438629 7.0
31.875262674 0.738079071044922 0.0
22.567942874 0.484143435955048 1.0
9.3073198007 0.523465573787689 2.0
0 0.221648275852203 3.0
9.3073198007 0.430354863405228 4.0
22.567942874 0.468249648809433 5.0
31.875262674 0.290105938911438 6.0
31.875262674 0.359275758266449 7.0
35.888104099 0.744237005710602 0.0
22.991591125 0.501128077507019 1.0
12.896512974 0.535313427448273 2.0
0 0.222997575998306 3.0
12.896512974 0.441955626010895 4.0
22.991591125 0.485698252916336 5.0
35.888104099 0.277660727500916 6.0
35.888104099 0.435194790363312 7.0
35.733201066 0.789843797683716 0.0
21.809644226 0.517688751220703 1.0
13.923556841 0.542868673801422 2.0
0 0.215907409787178 3.0
13.923556841 0.443316251039505 4.0
21.809644226 0.46927747130394 5.0
35.733201066 0.399901121854782 6.0
35.733201066 0.359512358903885 7.0
43.710725429 0.770245790481567 0.0
28.286031516 0.539265990257263 1.0
15.424693913 0.584526181221008 2.0
0 0.224556282162666 3.0
15.424693913 0.415617108345032 4.0
28.286031516 0.45531302690506 5.0
43.710725429 0.279928267002106 6.0
43.710725429 0.350989788770676 7.0
34.934843039 0.790276706218719 0.0
24.591035665 0.505245923995972 1.0
10.343807374 0.53099250793457 2.0
0 0.198081210255623 3.0
10.343807374 0.427411258220673 4.0
24.591035665 0.465415775775909 5.0
34.934843039 0.263571232557297 6.0
34.934843039 0.317151784896851 7.0
38.70218214 0.822845339775085 0.0
26.81631331 0.469875037670135 1.0
11.88586883 0.524389386177063 2.0
0 0.248640030622482 3.0
11.88586883 0.470990300178528 4.0
26.81631331 0.518216550350189 5.0
38.70218214 0.385273903608322 6.0
38.70218214 0.532739162445068 7.0
39.146786058 0.745629549026489 0.0
24.796133157 0.440602242946625 1.0
14.350652901 0.497564852237701 2.0
0 0.223359242081642 3.0
14.350652901 0.470826983451843 4.0
24.796133157 0.492195636034012 5.0
39.146786058 0.262956321239471 6.0
39.146786058 0.544199824333191 7.0
37.956003494 0.806739985942841 0.0
28.41977132 0.401349157094955 1.0
9.5362321739 0.428436696529388 2.0
0 0.255447804927826 3.0
9.5362321739 0.433036088943481 4.0
28.41977132 0.494789183139801 5.0
37.956003494 0.215267330408096 6.0
37.956003494 0.357328742742538 7.0
35.896232558 0.183311372995377 0.0
24.51348528 1.01320362091064 1.0
11.382747278 1.01181352138519 2.0
0 0.170761704444885 3.0
11.382747278 0.42978909611702 4.0
24.51348528 0.449950039386749 5.0
35.896232558 -0.081779845058918 6.0
35.896232558 2.02452826499939 7.0
37.377916966 0.154534935951233 0.0
24.033978696 0.631878972053528 1.0
13.343938271 0.668708801269531 2.0
0 0.187174826860428 3.0
13.343938271 0.413269490003586 4.0
24.033978696 0.44527080655098 5.0
37.377916966 0.000567857176065445 6.0
37.377916966 0.335900008678436 7.0
36.698956497 0.650843143463135 0.0
26.266791255 0.433510094881058 1.0
10.432165242 0.477599740028381 2.0
0 0.181102186441422 3.0
10.432165242 0.451521784067154 4.0
26.266791255 0.470768570899963 5.0
36.698956497 0.21667343378067 6.0
36.698956497 0.521405875682831 7.0
41.903925455 0.614661335945129 0.0
28.026899882 0.462207466363907 1.0
13.877025573 0.493976980447769 2.0
0 0.162575006484985 3.0
13.877025573 0.417298078536987 4.0
28.026899882 0.427234709262848 5.0
41.903925455 0.158513262867928 6.0
41.903925455 0.219808265566826 7.0
38.934024851 0.792782068252563 0.0
24.896228133 0.334390997886658 1.0
14.037796718 0.37085434794426 2.0
0 0.215390622615814 3.0
14.037796718 0.461321830749512 4.0
24.896228133 0.493830919265747 5.0
38.934024851 0.571407437324524 6.0
38.934024851 0.435825437307358 7.0
40.406325889 0.82914125919342 0.0
26.468139853 0.416990011930466 1.0
13.938186036 0.450109362602234 2.0
0 0.237914592027664 3.0
13.938186036 0.418347954750061 4.0
26.468139853 0.462546586990356 5.0
40.406325889 0.0872844234108925 6.0
40.406325889 0.464638262987137 7.0
39.410747699 0.851421535015106 0.0
27.084541691 0.527815699577332 1.0
12.326206008 0.551845252513885 2.0
0 0.215160667896271 3.0
12.326206008 0.425378054380417 4.0
27.084541691 0.475477933883667 5.0
39.410747699 0.358349710702896 6.0
39.410747699 0.283366084098816 7.0
};
\addlegendentry{$R^2$=-2.818}
\end{axis}

\end{tikzpicture}
}}
    
    \caption{Model results using only the loss associated with nodal flow predictions in the 8-node network.}
    \label{fig:dummy_base_results}
\end{figure}



To evaluate the performance of the optimization model and the GNN-based model, a t-test was performed, comparing the time each took to generate 100 predictions for new, unseen cases. The results showed a T-statistic of $14.94$, 198 degrees of freedom and a p-value of 1.32e-34, indicating that the optimization model took significantly longer than the GNN-based model, confirming its higher computational cost.

% The table was here !!

In the next stage of the experiment, the model was updated to account for the loss associated with edge flow predictions. This allowed the network to better capture the dynamics of gas transportation across the entire network, not just at the nodes. In this case, the hyperparameter tuning revealed that the best combination was $N channels=44$, $N Layers=4$, and $N Dense=4$, with a test data loss of 2.02, where 1.996 corresponds to the loss associated with the prediction of flows injected at the nodes, and 0.207 corresponds to the loss related to the prediction of flows through the edges. Additionally, the MSE between the gas balance based on actual flows and predicted flows was calculated, resulting in a loss of 1.719. It is essential to highlight that, in the previous experiment, this same loss was 284.764. This loss was computed outside the training stage, so it did not influence the optimization of parameters or hyperparameters.


\Cref{fig:results_dummy_node_base_f} shows the relationship between the actual injection values at the nodes and the corresponding predictions. As in the previous experiment, the model successfully recognizes that gas injection occurs only at a single node, although the predicted values differ slightly from the actual data. However, a significant improvement is observed when analyzing the flows through the edges. \Cref{fig:results_dummy_edge_base_f} highlights this, where the network's predictions for edge flows are almost perfectly aligned with the actual test data, as evidenced by an $R^2$ of 0.999. The near-perfect scatter plot in \cref{fig:results_dummy_edge_base_f} demonstrates the network's ability to make highly accurate predictions for gas flows through the edges in this updated experiment.



\begin{figure}
    \centering
        \setlength\figurewidth{.53\textwidth}        
        \setlength\figureheight{0.36\textwidth} 
        \subfloat [Actual vs predicted nodal flows.] {\label{fig:results_dummy_node_base_f}\resizebox{\figurewidth}{\figureheight}{% This file was created with tikzplotlib v0.10.1.
\begin{tikzpicture}

\definecolor{darkgray176}{RGB}{176,176,176}
\definecolor{lightgray204}{RGB}{204,204,204}

\begin{axis}[
colorbar,
colorbar style={ylabel={node_id}},
colormap={mymap}{[1pt]
 rgb(0pt)=(0.12156862745098,0.466666666666667,0.705882352941177);
  rgb(1pt)=(1,0.498039215686275,0.0549019607843137);
  rgb(2pt)=(0.172549019607843,0.627450980392157,0.172549019607843);
  rgb(3pt)=(0.83921568627451,0.152941176470588,0.156862745098039);
  rgb(4pt)=(0.580392156862745,0.403921568627451,0.741176470588235);
  rgb(5pt)=(0.549019607843137,0.337254901960784,0.294117647058824);
  rgb(6pt)=(0.890196078431372,0.466666666666667,0.76078431372549);
  rgb(7pt)=(0.498039215686275,0.498039215686275,0.498039215686275);
  rgb(8pt)=(0.737254901960784,0.741176470588235,0.133333333333333);
  rgb(9pt)=(0.0901960784313725,0.745098039215686,0.811764705882353)
},
legend cell align={left},
legend style={
  fill opacity=0.8,
  draw opacity=1,
  text opacity=1,
  at={(0.03,0.97)},
  anchor=north west,
  draw=lightgray204
},
point meta max=7,
point meta min=0,
tick align=outside,
tick pos=left,
title={yn_test-y_pred},
x grid style={darkgray176},
xlabel={yn_test},
xmajorgrids,
xmin=-2.43954548935, xmax=51.23045527635,
xtick style={color=black},
y grid style={darkgray176},
ylabel={y_pred},
ymajorgrids,
ymin=-2.19038361012936, ymax=45.0462195843458,
ytick style={color=black}
]
\addplot [
  colormap={mymap}{[1pt]
 rgb(0pt)=(0.12156862745098,0.466666666666667,0.705882352941177);
  rgb(1pt)=(1,0.498039215686275,0.0549019607843137);
  rgb(2pt)=(0.172549019607843,0.627450980392157,0.172549019607843);
  rgb(3pt)=(0.83921568627451,0.152941176470588,0.156862745098039);
  rgb(4pt)=(0.580392156862745,0.403921568627451,0.741176470588235);
  rgb(5pt)=(0.549019607843137,0.337254901960784,0.294117647058824);
  rgb(6pt)=(0.890196078431372,0.466666666666667,0.76078431372549);
  rgb(7pt)=(0.498039215686275,0.498039215686275,0.498039215686275);
  rgb(8pt)=(0.737254901960784,0.741176470588235,0.133333333333333);
  rgb(9pt)=(0.0901960784313725,0.745098039215686,0.811764705882353)
},
  only marks,
  scatter,
  scatter src=explicit
]
table [x=x, y=y, meta=colordata]{%
x  y  colordata
39.565898645 38.7153625488281 0.0
0 0.0123127698898315 1.0
0 0.071319580078125 2.0
0 -0.0128939747810364 3.0
0 0.0256100296974182 4.0
0 -0.0238697528839111 5.0
0 0.00411289930343628 6.0
0 -0.0127748847007751 7.0
42.743257181 39.4932670593262 0.0
0 0.00441712141036987 1.0
0 0.0637285709381104 2.0
0 -0.0179623365402222 3.0
0 0.0215394496917725 4.0
0 -0.0140223503112793 5.0
0 1.97887420654297e-05 6.0
0 -0.0135558843612671 7.0
39.367180751 38.7526664733887 0.0
0 0.0138788819313049 1.0
0 0.0361418128013611 2.0
0 -0.0201342701911926 3.0
0 0.00914174318313599 4.0
0 -0.0191978812217712 5.0
0 0.000798225402832031 6.0
0 -0.00776267051696777 7.0
39.605393107 39.5196304321289 0.0
0 0.00788706541061401 1.0
0 0.0466488599777222 2.0
0 -0.0212488174438477 3.0
0 0.0412870049476624 4.0
0 -0.0301215648651123 5.0
0 0.00819826126098633 6.0
0 -0.0149205923080444 7.0
43.937345536 38.9456558227539 0.0
0 0.0131928324699402 1.0
0 0.0427597761154175 2.0
0 -0.0308229327201843 3.0
0 0.0441254377365112 4.0
0 -0.0282912254333496 5.0
0 0.00409531593322754 6.0
0 -0.0128695964813232 7.0
31.061989594 39.310115814209 0.0
0 0.0139701366424561 1.0
0 0.0382223725318909 2.0
0 -0.0262283682823181 3.0
0 0.0097421407699585 4.0
0 -0.0335473418235779 5.0
0 0.00768345594406128 6.0
0 -0.00622838735580444 7.0
36.357435273 42.0804176330566 0.0
0 0.00529545545578003 1.0
0 0.0581262111663818 2.0
0 -0.0150731205940247 3.0
0 0.0298828482627869 4.0
0 -0.028703510761261 5.0
0 0.00751966238021851 6.0
0 -0.0148077011108398 7.0
38.444969617 39.1996421813965 0.0
0 0.00592869520187378 1.0
0 0.0615658760070801 2.0
0 -0.0213605761528015 3.0
0 0.0231988430023193 4.0
0 -0.026212215423584 5.0
0 0.00326085090637207 6.0
0 -0.0112636089324951 7.0
35.498620524 39.3897895812988 0.0
0 0.00891655683517456 1.0
0 0.0605899095535278 2.0
0 -0.018476665019989 3.0
0 0.00775080919265747 4.0
0 -0.024048924446106 5.0
0 0.0055384635925293 6.0
0 -0.0057908296585083 7.0
36.520998279 39.447883605957 0.0
0 0.00435954332351685 1.0
0 0.0819315910339355 2.0
0 -0.0160262584686279 3.0
0 0.0174445509910583 4.0
0 -0.0244802236557007 5.0
0 0.00764340162277222 6.0
0 -0.00894832611083984 7.0
36.717272212 39.199634552002 0.0
0 0.00995087623596191 1.0
0 0.066163182258606 2.0
0 -0.0166041851043701 3.0
0 0.0310661196708679 4.0
0 -0.0306993126869202 5.0
0 0.0094258189201355 6.0
0 -0.0108351707458496 7.0
32.629629006 39.185905456543 0.0
0 0.0115001201629639 1.0
0 0.0270580649375916 2.0
0 -0.0159772038459778 3.0
0 0.00784498453140259 4.0
0 -0.0299507975578308 5.0
0 0.00833970308303833 6.0
0 -0.00938189029693604 7.0
37.75267434 39.1989135742188 0.0
0 0.0082511305809021 1.0
0 0.06497722864151 2.0
0 -0.0128266215324402 3.0
0 0.00258684158325195 4.0
0 -0.018075704574585 5.0
0 -0.000989556312561035 6.0
0 -0.0049208402633667 7.0
38.800291347 40.1666564941406 0.0
0 0.0118288397789001 1.0
0 0.0622610449790955 2.0
0 -0.0261341333389282 3.0
0 0.0260022878646851 4.0
0 -0.0306769013404846 5.0
0 0.0043455958366394 6.0
0 -0.0105564594268799 7.0
38.252729618 39.3290748596191 0.0
0 0.00866585969924927 1.0
0 0.0685842037200928 2.0
0 -0.0130652189254761 3.0
0 0.0246871113777161 4.0
0 -0.0263183116912842 5.0
0 0.00541287660598755 6.0
0 -0.0122238397598267 7.0
43.273596047 39.3154678344727 0.0
0 0.00989145040512085 1.0
0 0.073027491569519 2.0
0 -0.0273558497428894 3.0
0 0.0494243502616882 4.0
0 -0.0284897089004517 5.0
0 0.00563955307006836 6.0
0 -0.0137518644332886 7.0
34.027431486 38.857494354248 0.0
0 0.00671476125717163 1.0
0 0.0619230270385742 2.0
0 -0.0194013118743896 3.0
0 0.0136286020278931 4.0
0 -0.029705822467804 5.0
0 0.003440260887146 6.0
0 -0.00797498226165771 7.0
41.154171391 39.8500900268555 0.0
0 0.00906610488891602 1.0
0 0.0560967326164246 2.0
0 -0.0276532769203186 3.0
0 0.0497760772705078 4.0
0 -0.0301504731178284 5.0
0 0.00726139545440674 6.0
0 -0.0125929117202759 7.0
39.930826408 39.8617324829102 0.0
0 0.00576150417327881 1.0
0 0.0628125667572021 2.0
0 -0.0210210680961609 3.0
0 0.0447211265563965 4.0
0 -0.0306063890457153 5.0
0 0.00760847330093384 6.0
0 -0.0124554634094238 7.0
45.969703631 39.1456565856934 0.0
0 0.0107769966125488 1.0
0 0.0357969403266907 2.0
0 -0.0225849151611328 3.0
0 0.043922483921051 4.0
0 -0.0197867751121521 5.0
0 0.00242942571640015 6.0
0 -0.0128521323204041 7.0
38.398120221 39.2407455444336 0.0
0 0.00990653038024902 1.0
0 0.0493940114974976 2.0
0 -0.0257954001426697 3.0
0 0.035546600818634 4.0
0 -0.0293555855751038 5.0
0 0.00442242622375488 6.0
0 -0.0125865340232849 7.0
28.366017306 39.314582824707 0.0
0 1.03116035461426e-05 1.0
0 0.0335801839828491 2.0
0 -0.00797528028488159 3.0
0 -0.0116368532180786 4.0
0 -0.0308080911636353 5.0
0 0.00525516271591187 6.0
0 -0.00420612096786499 7.0
39.079818979 39.1686325073242 0.0
0 0.011313796043396 1.0
0 0.0414641499519348 2.0
0 -0.0202111601829529 3.0
0 0.0133368968963623 4.0
0 -0.0244540572166443 5.0
0 0.00453060865402222 6.0
0 -0.0085756778717041 7.0
40.466329383 39.3136444091797 0.0
0 -0.00223791599273682 1.0
0 0.0511056780815125 2.0
0 -0.0211246013641357 3.0
0 0.0184855461120605 4.0
0 -0.0249439477920532 5.0
0 0.00523895025253296 6.0
0 -0.00943392515182495 7.0
38.293116853 38.996208190918 0.0
0 0.00625330209732056 1.0
0 0.069944441318512 2.0
0 -0.0136637687683105 3.0
0 0.000947535037994385 4.0
0 -0.0120516419410706 5.0
0 0.000261187553405762 6.0
0 -0.00625407695770264 7.0
43.346089497 38.9964256286621 0.0
0 0.0137380957603455 1.0
0 0.0395742058753967 2.0
0 -0.0222710371017456 3.0
0 0.0477991700172424 4.0
0 -0.030639111995697 5.0
0 0.00966173410415649 6.0
0 -0.0138188004493713 7.0
34.547871154 39.3231048583984 0.0
0 0.00445806980133057 1.0
0 0.0469914674758911 2.0
0 -0.0166859030723572 3.0
0 0.0288587212562561 4.0
0 -0.0295593738555908 5.0
0 0.00691384077072144 6.0
0 -0.0114007592201233 7.0
41.927050143 39.338737487793 0.0
0 0.00693070888519287 1.0
0 0.0592769980430603 2.0
0 -0.0221498608589172 3.0
0 0.0644987821578979 4.0
0 -0.029447615146637 5.0
0 0.0115132927894592 6.0
0 -0.0177930593490601 7.0
44.548236377 39.3082656860352 0.0
0 0.0133536458015442 1.0
0 0.0384448766708374 2.0
0 -0.0213861465454102 3.0
0 0.0355319976806641 4.0
0 -0.0190299153327942 5.0
0 0.00214254856109619 6.0
0 -0.0123872756958008 7.0
34.958415487 39.313117980957 0.0
0 0.00151073932647705 1.0
0 0.0384351015090942 2.0
0 -0.0213229656219482 3.0
0 0.00255352258682251 4.0
0 -0.025343120098114 5.0
0 0.00335896015167236 6.0
0 -0.00606870651245117 7.0
41.619773298 39.9483871459961 0.0
0 -0.00310182571411133 1.0
0 0.0696684718132019 2.0
0 -0.0155016183853149 3.0
0 0.0176678895950317 4.0
0 -0.0168115496635437 5.0
0 0.000762403011322021 6.0
0 -0.0129686594009399 7.0
35.768623912 39.1677360534668 0.0
0 -0.00235003232955933 1.0
0 0.0565180778503418 2.0
0 -0.0165175199508667 3.0
0 0.0183342695236206 4.0
0 -0.0309199094772339 5.0
0 0.00655144453048706 6.0
0 -0.0103568434715271 7.0
36.03587308 37.5861854553223 0.0
0 -0.00157850980758667 1.0
0 0.0492555499076843 2.0
0 -0.0198881030082703 3.0
0 0.0342521667480469 4.0
0 -0.0291762351989746 5.0
0 0.0110065340995789 6.0
0 -0.011877179145813 7.0
42.671159584 39.3272399902344 0.0
0 0.0113665461540222 1.0
0 0.0392718911170959 2.0
0 -0.0274774432182312 3.0
0 0.052215039730072 4.0
0 -0.0293958783149719 5.0
0 0.00858646631240845 6.0
0 -0.0139873623847961 7.0
32.270518391 39.3328552246094 0.0
0 0.0110946297645569 1.0
0 0.05193692445755 2.0
0 -0.0178480744361877 3.0
0 0.00724279880523682 4.0
0 -0.0301480889320374 5.0
0 0.00648993253707886 6.0
0 -0.00789618492126465 7.0
36.239834941 41.1309661865234 0.0
0 0.00345695018768311 1.0
0 0.0747806429862976 2.0
0 -0.0145701169967651 3.0
0 0.0049513578414917 4.0
0 -0.0192160606384277 5.0
0 0.00256305932998657 6.0
0 -0.00866025686264038 7.0
38.292005181 39.2797698974609 0.0
0 0.00284379720687866 1.0
0 0.0393624305725098 2.0
0 -0.0206927061080933 3.0
0 0.0245354771614075 4.0
0 -0.028574526309967 5.0
0 0.00577658414840698 6.0
0 -0.0118468999862671 7.0
41.142171126 39.369083404541 0.0
0 0.000563442707061768 1.0
0 0.0569069981575012 2.0
0 -0.0239884257316589 3.0
0 0.0219588279724121 4.0
0 -0.0221072435379028 5.0
0 0.0049242377281189 6.0
0 -0.0127769112586975 7.0
30.660234751 38.9642333984375 0.0
0 -0.00240546464920044 1.0
0 0.0487861633300781 2.0
0 -0.00998640060424805 3.0
0 0.00200992822647095 4.0
0 -0.0323919057846069 5.0
0 0.00852972269058228 6.0
0 -0.00369340181350708 7.0
42.776716986 41.5339431762695 0.0
0 0.00600254535675049 1.0
0 0.0594097375869751 2.0
0 -0.0193828344345093 3.0
0 0.0313163995742798 4.0
0 -0.0243908762931824 5.0
0 0.0065605640411377 6.0
0 -0.016025185585022 7.0
39.136657955 39.7308387756348 0.0
0 0.00334018468856812 1.0
0 0.099915623664856 2.0
0 -0.0114416480064392 3.0
0 0.0245475769042969 4.0
0 -0.0192947387695312 5.0
0 0.00287395715713501 6.0
0 -0.0121320486068726 7.0
40.593075664 39.7711791992188 0.0
0 0.00522077083587646 1.0
0 0.0596650242805481 2.0
0 -0.0162090659141541 3.0
0 0.045604944229126 4.0
0 -0.0273600816726685 5.0
0 0.00408452749252319 6.0
0 -0.0147544145584106 7.0
38.455960057 40.2578964233398 0.0
0 0.00629007816314697 1.0
0 0.0590381622314453 2.0
0 -0.0165366530418396 3.0
0 0.043002724647522 4.0
0 -0.0271528959274292 5.0
0 0.00735443830490112 6.0
0 -0.0150219798088074 7.0
40.029822307 39.2045021057129 0.0
0 0.0109648704528809 1.0
0 0.0333818793296814 2.0
0 -0.0219238996505737 3.0
0 0.0246446132659912 4.0
0 -0.0267536044120789 5.0
0 0.00693017244338989 6.0
0 -0.0108383893966675 7.0
39.721089806 39.6830825805664 0.0
0 0.00558066368103027 1.0
0 0.0897490382194519 2.0
0 -0.0145083665847778 3.0
0 0.0387337803840637 4.0
0 -0.0278629660606384 5.0
0 0.00985139608383179 6.0
0 -0.0144490599632263 7.0
46.012802781 39.0627212524414 0.0
0 0.00166887044906616 1.0
0 0.0386090874671936 2.0
0 -0.0324397087097168 3.0
0 0.0550388693809509 4.0
0 -0.028700590133667 5.0
0 0.00663286447525024 6.0
0 -0.0132797360420227 7.0
43.791041158 39.1844825744629 0.0
0 0.0124698877334595 1.0
0 0.0309100747108459 2.0
0 -0.0285602211952209 3.0
0 0.046930730342865 4.0
0 -0.0275624394416809 5.0
0 0.00615853071212769 6.0
0 -0.0148841142654419 7.0
31.257332424 39.1901626586914 0.0
0 0.0106835961341858 1.0
0 0.0625916123390198 2.0
0 -0.0122222304344177 3.0
0 0.000936150550842285 4.0
0 -0.0310866236686707 5.0
0 0.00312197208404541 6.0
0 -0.00706321001052856 7.0
38.98847291 39.2609672546387 0.0
0 0.0101978778839111 1.0
0 0.0407465100288391 2.0
0 -0.0224083065986633 3.0
0 0.0104560256004333 4.0
0 -0.0216193795204163 5.0
0 0.00195193290710449 6.0
0 -0.00701618194580078 7.0
38.691218499 38.9683609008789 0.0
0 0.00895541906356812 1.0
0 0.0564865469932556 2.0
0 -0.0135195851325989 3.0
0 0.0024409294128418 4.0
0 -0.0234909653663635 5.0
0 0.0044097900390625 6.0
0 -0.00805675983428955 7.0
39.033211971 38.7177581787109 0.0
0 0.00353097915649414 1.0
0 0.0651161074638367 2.0
0 -0.0148383378982544 3.0
0 0.0438032746315002 4.0
0 -0.0298718214035034 5.0
0 0.00918209552764893 6.0
0 -0.0137563943862915 7.0
37.697547813 39.1868095397949 0.0
0 0.00985407829284668 1.0
0 0.0413744449615479 2.0
0 -0.0205171704292297 3.0
0 0.00754988193511963 4.0
0 -0.0230441689491272 5.0
0 0.00145184993743896 6.0
0 -0.0105559825897217 7.0
35.277541339 39.2073364257812 0.0
0 0.0143079161643982 1.0
0 0.0378283262252808 2.0
0 -0.0201423168182373 3.0
0 0.00532984733581543 4.0
0 -0.0268043279647827 5.0
0 0.00716966390609741 6.0
0 -0.0073121190071106 7.0
36.119763966 39.4090423583984 0.0
0 0.0113481879234314 1.0
0 0.0379583835601807 2.0
0 -0.0301769375801086 3.0
0 0.0023917555809021 4.0
0 -0.0140483379364014 5.0
0 -0.00150942802429199 6.0
0 -0.00346070528030396 7.0
33.14490305 39.51171875 0.0
0 0.00569027662277222 1.0
0 0.0744785666465759 2.0
0 -0.0195279121398926 3.0
0 0.00108557939529419 4.0
0 -0.0206176042556763 5.0
0 0.00269472599029541 6.0
0 -0.00410282611846924 7.0
34.486800814 40.0056648254395 0.0
0 0.0040428638458252 1.0
0 0.0613293647766113 2.0
0 -0.0154016017913818 3.0
0 0.0307614207267761 4.0
0 -0.0287876129150391 5.0
0 0.0124437212944031 6.0
0 -0.0129595398902893 7.0
40.933468207 40.1773338317871 0.0
0 0.0079658031463623 1.0
0 0.0732653141021729 2.0
0 -0.0176567435264587 3.0
0 0.0293158888816833 4.0
0 -0.0187020897865295 5.0
0 0.00174516439437866 6.0
0 -0.0126280188560486 7.0
35.993417928 39.3662071228027 0.0
0 0.00715410709381104 1.0
0 0.0595747232437134 2.0
0 -0.0184565782546997 3.0
0 0.0182121396064758 4.0
0 -0.0294560790061951 5.0
0 0.00316703319549561 6.0
0 -0.0110476613044739 7.0
39.331666923 39.7296485900879 0.0
0 0.00837588310241699 1.0
0 0.0545561909675598 2.0
0 -0.0260232090950012 3.0
0 0.0253786444664001 4.0
0 -0.0261372923851013 5.0
0 0.0041089653968811 6.0
0 -0.0101762413978577 7.0
37.559548496 39.3728485107422 0.0
0 0.00726288557052612 1.0
0 0.0635012984275818 2.0
0 -0.0159628987312317 3.0
0 0.0196371674537659 4.0
0 -0.0268263816833496 5.0
0 0.00435435771942139 6.0
0 -0.0115820169448853 7.0
41.796902482 39.2759208679199 0.0
0 0.0100352764129639 1.0
0 0.0439282655715942 2.0
0 -0.0254364013671875 3.0
0 0.0484634637832642 4.0
0 -0.0301735997200012 5.0
0 0.00479131937026978 6.0
0 -0.0128893852233887 7.0
35.679590823 39.2148361206055 0.0
0 0.0133700370788574 1.0
0 0.0399370193481445 2.0
0 -0.0224254131317139 3.0
0 0.0077054500579834 4.0
0 -0.0263437032699585 5.0
0 0.00416111946105957 6.0
0 -0.00710749626159668 7.0
33.227547292 38.8616104125977 0.0
0 0.0111078023910522 1.0
0 0.061786949634552 2.0
0 -0.0128821134567261 3.0
0 0.00850886106491089 4.0
0 -0.0278047919273376 5.0
0 0.00242877006530762 6.0
0 -0.00745308399200439 7.0
28.008071739 39.3282775878906 0.0
0 0.00727421045303345 1.0
0 0.0578927397727966 2.0
0 -0.0101944804191589 3.0
0 -0.00735962390899658 4.0
0 -0.0292975306510925 5.0
0 0.0059630274772644 6.0
0 0.00163012742996216 7.0
33.478498841 39.2475204467773 0.0
0 0.0053410530090332 1.0
0 0.0993016362190247 2.0
0 -0.00882583856582642 3.0
0 0.0095442533493042 4.0
0 -0.0248811841011047 5.0
0 0.00372958183288574 6.0
0 -0.00759077072143555 7.0
33.12294682 39.1935424804688 0.0
0 0.0134350061416626 1.0
0 0.0337791442871094 2.0
0 -0.0293073058128357 3.0
0 0.0119555592536926 4.0
0 -0.03260737657547 5.0
0 0.00758576393127441 6.0
0 -0.00859177112579346 7.0
34.970228384 39.3222885131836 0.0
0 0.0112384557723999 1.0
0 0.0673366785049438 2.0
0 -0.0168597102165222 3.0
0 0.00697022676467896 4.0
0 -0.0265811085700989 5.0
0 0.00594782829284668 6.0
0 0.0834335684776306 7.0
36.637966041 39.6626014709473 0.0
0 0.0114080905914307 1.0
0 0.039689302444458 2.0
0 -0.0293136239051819 3.0
0 0.0113623738288879 4.0
0 -0.028203547000885 5.0
0 0.00562357902526855 6.0
0 -0.00684285163879395 7.0
38.712447295 39.2186279296875 0.0
0 0.00919914245605469 1.0
0 0.0359517931938171 2.0
0 -0.0268067121505737 3.0
0 0.0361806750297546 4.0
0 -0.0292361378669739 5.0
0 0.00580942630767822 6.0
0 -0.0120308995246887 7.0
29.08402242 39.358585357666 0.0
0 0.00405758619308472 1.0
0 0.0620464682579041 2.0
0 -0.00910604000091553 3.0
0 -0.00566118955612183 4.0
0 -0.0290875434875488 5.0
0 0.00550538301467896 6.0
0 -0.00326275825500488 7.0
40.741903011 39.7179641723633 0.0
0 0.0086064338684082 1.0
0 0.0604866147041321 2.0
0 -0.0193439722061157 3.0
0 0.0264756679534912 4.0
0 -0.0243978500366211 5.0
0 0.0040895938873291 6.0
0 -0.0132392048835754 7.0
44.423927978 39.2226028442383 0.0
0 0.0112552642822266 1.0
0 0.0406379103660583 2.0
0 -0.0212950110435486 3.0
0 0.0372845530509949 4.0
0 -0.0296989679336548 5.0
0 0.00628405809402466 6.0
0 -0.0137932300567627 7.0
37.279929011 42.0506362915039 0.0
0 0.000679612159729004 1.0
0 0.0922307968139648 2.0
0 -0.0117670297622681 3.0
0 0.00707334280014038 4.0
0 -0.0226204395294189 5.0
0 0.001944899559021 6.0
0 -0.0140491127967834 7.0
39.065196566 39.0746231079102 0.0
0 0.00692451000213623 1.0
0 0.0894067883491516 2.0
0 -0.0125413537025452 3.0
0 0.0433874726295471 4.0
0 -0.0262654423713684 5.0
0 0.00937050580978394 6.0
0 -0.0140787959098816 7.0
34.346712911 39.6062927246094 0.0
0 0.00730884075164795 1.0
0 0.0575341582298279 2.0
0 -0.0125131011009216 3.0
0 0.00180590152740479 4.0
0 -0.0254272222518921 5.0
0 0.00626826286315918 6.0
0 -0.00440073013305664 7.0
44.832836202 39.1740646362305 0.0
0 0.00540620088577271 1.0
0 0.048359215259552 2.0
0 -0.0217028856277466 3.0
0 0.0456007122993469 4.0
0 -0.0261398553848267 5.0
0 0.00735974311828613 6.0
0 -0.0143005847930908 7.0
37.693005916 39.4430923461914 0.0
0 0.00866585969924927 1.0
0 0.0588068962097168 2.0
0 -0.0298804044723511 3.0
0 0.012173056602478 4.0
0 -0.0230597257614136 5.0
0 0.00236654281616211 6.0
0 -0.00678610801696777 7.0
35.516030206 39.2848510742188 0.0
0 0.00840789079666138 1.0
0 0.0401602387428284 2.0
0 -0.0224798321723938 3.0
0 0.00727975368499756 4.0
0 -0.0262488722801208 5.0
0 0.00442099571228027 6.0
0 -0.00669002532958984 7.0
45.449957523 39.3815727233887 0.0
0 0.00223982334136963 1.0
0 0.0652434825897217 2.0
0 -0.0192306637763977 3.0
0 0.0608529448509216 4.0
0 -0.027153491973877 5.0
0 0.00786775350570679 6.0
0 -0.0163425207138062 7.0
36.52210397 40.1063613891602 0.0
0 0.0123141407966614 1.0
0 0.0607444047927856 2.0
0 -0.0281732082366943 3.0
0 0.00769400596618652 4.0
0 -0.0218368172645569 5.0
0 0.00279814004898071 6.0
0 0.214010000228882 7.0
40.809333833 39.2384757995605 0.0
0 0.0117188096046448 1.0
0 0.039944052696228 2.0
0 -0.0229901075363159 3.0
0 0.0258876085281372 4.0
0 -0.0261783003807068 5.0
0 0.0060725212097168 6.0
0 -0.00992476940155029 7.0
43.708815025 39.8203887939453 0.0
0 0.00554096698760986 1.0
0 0.0680207014083862 2.0
0 -0.0179526805877686 3.0
0 0.0337607264518738 4.0
0 -0.0186030864715576 5.0
0 0.00325441360473633 6.0
0 -0.0130414366722107 7.0
35.746307575 39.0528678894043 0.0
0 0.00616163015365601 1.0
0 0.031990647315979 2.0
0 -0.0340034961700439 3.0
0 0.0250157713890076 4.0
0 -0.0317110419273376 5.0
0 0.00777506828308105 6.0
0 -0.00998783111572266 7.0
32.565260407 39.287166595459 0.0
0 0.0112178325653076 1.0
0 0.0362849831581116 2.0
0 -0.0211132764816284 3.0
0 0.000355720520019531 4.0
0 -0.0266507863998413 5.0
0 0.00528323650360107 6.0
0 -0.00296187400817871 7.0
37.459437569 39.7067375183105 0.0
0 0.00287795066833496 1.0
0 0.0870211720466614 2.0
0 -0.0136895775794983 3.0
0 0.0324925780296326 4.0
0 -0.0261927247047424 5.0
0 0.00662326812744141 6.0
0 -0.0170584321022034 7.0
41.600868777 39.6331176757812 0.0
0 0.00617289543151855 1.0
0 0.0472753047943115 2.0
0 -0.0370604395866394 3.0
0 0.0207778215408325 4.0
0 -0.0186801552772522 5.0
0 0.000774919986724854 6.0
0 -0.00782781839370728 7.0
43.088648289 42.4566268920898 0.0
0 0.00171780586242676 1.0
0 0.0445231199264526 2.0
0 -0.0263301730155945 3.0
0 0.0577532052993774 4.0
0 -0.0282729268074036 5.0
0 0.00841760635375977 6.0
0 -0.0169897675514221 7.0
30.268152677 39.3095664978027 0.0
0 0.0115619301795959 1.0
0 0.0627694725990295 2.0
0 -0.0172976851463318 3.0
0 0.000941872596740723 4.0
0 -0.0301180481910706 5.0
0 0.00371986627578735 6.0
0 -0.00358611345291138 7.0
38.454045888 38.8899307250977 0.0
0 0.0116772055625916 1.0
0 0.0517902970314026 2.0
0 -0.0236786603927612 3.0
0 0.0169258117675781 4.0
0 -0.0281082987785339 5.0
0 0.0070832371711731 6.0
0 -0.00702857971191406 7.0
36.654056864 39.1972312927246 0.0
0 0.00971025228500366 1.0
0 0.0378752946853638 2.0
0 -0.021247386932373 3.0
0 0.0270963311195374 4.0
0 -0.029135525226593 5.0
0 0.00854533910751343 6.0
0 -0.0112012028694153 7.0
36.990278128 39.4844970703125 0.0
0 0.010595440864563 1.0
0 0.0305523872375488 2.0
0 -0.0160865783691406 3.0
0 0.0211124420166016 4.0
0 -0.0285171270370483 5.0
0 0.00439399480819702 6.0
0 -0.0110663771629333 7.0
39.45539179 39.3758239746094 0.0
0 0.00812649726867676 1.0
0 0.0442739725112915 2.0
0 -0.0175419449806213 3.0
0 0.0141112208366394 4.0
0 -0.027271032333374 5.0
0 0.00580936670303345 6.0
0 -0.00898617506027222 7.0
41.49009831 39.2659378051758 0.0
0 0.00471115112304688 1.0
0 0.0437565445899963 2.0
0 -0.0283921957015991 3.0
0 0.0177510380744934 4.0
0 -0.018986701965332 5.0
0 0.00044095516204834 6.0
0 -0.00875365734100342 7.0
42.195848764 39.7440223693848 0.0
0 0.00705784559249878 1.0
0 0.0813563466072083 2.0
0 -0.0185672044754028 3.0
0 0.0338060855865479 4.0
0 -0.0192969441413879 5.0
0 0.00195407867431641 6.0
0 -0.0113794207572937 7.0
36.580423956 39.3618698120117 0.0
0 0.0103769898414612 1.0
0 0.0483078956604004 2.0
0 -0.0236889123916626 3.0
0 0.0567827224731445 4.0
0 -0.028109610080719 5.0
0 0.0141017436981201 6.0
0 -0.0142652988433838 7.0
41.112612846 38.2719535827637 0.0
0 0.00170391798019409 1.0
0 0.105811357498169 2.0
0 -0.0118167400360107 3.0
0 0.036638617515564 4.0
0 -0.0218501687049866 5.0
0 0.00543737411499023 6.0
0 -0.0120397210121155 7.0
41.165032305 39.1872482299805 0.0
0 0.0157047510147095 1.0
0 0.0349633693695068 2.0
0 -0.0216887593269348 3.0
0 0.0505847930908203 4.0
0 -0.0271188616752625 5.0
0 0.00774049758911133 6.0
0 -0.0159221291542053 7.0
28.236329769 39.7965545654297 0.0
0 0.013361930847168 1.0
0 0.0601737499237061 2.0
0 -0.00644588470458984 3.0
0 -0.00728082656860352 4.0
0 -0.0303657054901123 5.0
0 0.00617170333862305 6.0
0 -0.00426137447357178 7.0
46.563722235 39.247730255127 0.0
0 0.0073997974395752 1.0
0 0.0409015417098999 2.0
0 -0.0347752571105957 3.0
0 0.0438136458396912 4.0
0 -0.0267916321754456 5.0
0 0.00308680534362793 6.0
0 -0.01231449842453 7.0
38.976039446 38.2965316772461 0.0
0 0.000528395175933838 1.0
0 0.025409460067749 2.0
0 -0.0173540711402893 3.0
0 0.00302207469940186 4.0
0 -0.0253949165344238 5.0
0 0.00395089387893677 6.0
0 -0.00843304395675659 7.0
43.902220165 39.8411254882812 0.0
0 0.0118625164031982 1.0
0 0.0401428937911987 2.0
0 -0.0226224660873413 3.0
0 0.0575634241104126 4.0
0 -0.0248637795448303 5.0
0 0.00626331567764282 6.0
0 -0.016412615776062 7.0
37.104870385 38.8935470581055 0.0
0 0.00900518894195557 1.0
0 0.0604310631752014 2.0
0 -0.0258265733718872 3.0
0 0.0378197431564331 4.0
0 -0.0289685726165771 5.0
0 0.00780671834945679 6.0
0 -0.0103233456611633 7.0
48.094555135 39.2413864135742 0.0
0 0.0141285061836243 1.0
0 0.0317863821983337 2.0
0 -0.0394392609596252 3.0
0 0.0513074994087219 4.0
0 -0.0285180807113647 5.0
0 0.0042346715927124 6.0
0 -0.0118893384933472 7.0
39.863550951 38.9883651733398 0.0
0 0.00595420598983765 1.0
0 0.0344380140304565 2.0
0 -0.0326076745986938 3.0
0 0.0386518836021423 4.0
0 -0.0331050753593445 5.0
0 0.00429987907409668 6.0
0 -0.0127902626991272 7.0
38.158424706 38.903678894043 0.0
0 0.0116057395935059 1.0
0 0.0537588000297546 2.0
0 -0.0285190343856812 3.0
0 0.0557814836502075 4.0
0 -0.029060959815979 5.0
0 0.00855320692062378 6.0
0 -0.0132871866226196 7.0
37.671552644 39.2374725341797 0.0
0 0.006736159324646 1.0
0 0.0430566668510437 2.0
0 -0.0213061571121216 3.0
0 0.0102139711380005 4.0
0 -0.0230097770690918 5.0
0 0.0043870210647583 6.0
0 -0.00778210163116455 7.0
38.99000687 39.3680267333984 0.0
0 0.00496220588684082 1.0
0 0.0836588740348816 2.0
0 -0.0154605507850647 3.0
0 0.0217403173446655 4.0
0 -0.0192769169807434 5.0
0 0.00258815288543701 6.0
0 -0.0127060413360596 7.0
32.824279197 39.2818450927734 0.0
0 0.0120186805725098 1.0
0 0.0409272313117981 2.0
0 -0.0234582424163818 3.0
0 0.00576364994049072 4.0
0 -0.0319589376449585 5.0
0 0.00346595048904419 6.0
0 -0.00725382566452026 7.0
39.752392398 39.3454055786133 0.0
0 0.00749439001083374 1.0
0 0.0421807169914246 2.0
0 -0.0276132225990295 3.0
0 0.00937461853027344 4.0
0 -0.0163375735282898 5.0
0 -0.000748932361602783 6.0
0 -0.00666379928588867 7.0
36.014997988 39.1476974487305 0.0
0 0.0123811960220337 1.0
0 0.0417296886444092 2.0
0 -0.0282233953475952 3.0
0 0.0180580019950867 4.0
0 -0.032294750213623 5.0
0 0.00644737482070923 6.0
0 -0.00982779264450073 7.0
42.307205606 39.2100219726562 0.0
0 0.0106115341186523 1.0
0 0.0420219302177429 2.0
0 -0.030609130859375 3.0
0 0.0188378095626831 4.0
0 -0.0219831466674805 5.0
0 0.00154978036880493 6.0
0 -0.00879871845245361 7.0
41.624369934 37.9989166259766 0.0
0 0.00612103939056396 1.0
0 0.0464754104614258 2.0
0 -0.0220385789871216 3.0
0 0.0484929084777832 4.0
0 -0.0285543203353882 5.0
0 0.00955373048782349 6.0
0 -0.0139495730400085 7.0
40.927860926 38.1056976318359 0.0
0 0.00858557224273682 1.0
0 0.0353962779045105 2.0
0 -0.0306072235107422 3.0
0 0.0340653657913208 4.0
0 -0.0286934375762939 5.0
0 0.00470423698425293 6.0
0 -0.0121994614601135 7.0
47.441321812 39.2602005004883 0.0
0 0.00840264558792114 1.0
0 0.0400999188423157 2.0
0 -0.0347042083740234 3.0
0 0.0447063446044922 4.0
0 -0.0227860808372498 5.0
0 0.00289112329483032 6.0
0 -0.0108776092529297 7.0
39.379505619 39.218936920166 0.0
0 0.00225824117660522 1.0
0 0.0381470918655396 2.0
0 -0.0227357745170593 3.0
0 0.0566052198410034 4.0
0 -0.0273053050041199 5.0
0 0.00671166181564331 6.0
0 -0.0176942348480225 7.0
42.095557299 39.302864074707 0.0
0 0.000669658184051514 1.0
0 0.0900809168815613 2.0
0 -0.0144271850585938 3.0
0 0.046349048614502 4.0
0 -0.0227258205413818 5.0
0 0.00791877508163452 6.0
0 -0.0147258043289185 7.0
35.151188225 39.5698585510254 0.0
0 0.00752902030944824 1.0
0 0.0650426149368286 2.0
0 -0.0100555419921875 3.0
0 -0.00416684150695801 4.0
0 -0.0206260681152344 5.0
0 -0.000355720520019531 6.0
0 -0.00151586532592773 7.0
37.713935371 39.5196762084961 0.0
0 0.000767946243286133 1.0
0 0.0639498233795166 2.0
0 -0.0160928964614868 3.0
0 0.0237512588500977 4.0
0 -0.0243517756462097 5.0
0 0.0049559473991394 6.0
0 -0.00916051864624023 7.0
39.66774326 39.3000450134277 0.0
0 0.00557941198348999 1.0
0 0.0528358221054077 2.0
0 -0.0230751037597656 3.0
0 0.0496906638145447 4.0
0 -0.0281819701194763 5.0
0 0.00943386554718018 6.0
0 -0.0162165760993958 7.0
41.240556194 39.485954284668 0.0
0 0.0104520320892334 1.0
0 0.0391672253608704 2.0
0 -0.0241261720657349 3.0
0 0.0442981123924255 4.0
0 -0.029579222202301 5.0
0 0.00497233867645264 6.0
0 -0.01584392786026 7.0
40.3201946 39.4521560668945 0.0
0 0.0161011219024658 1.0
0 0.0323320627212524 2.0
0 -0.026906430721283 3.0
0 0.021517276763916 4.0
0 -0.0247675180435181 5.0
0 0.00135600566864014 6.0
0 -0.0140462517738342 7.0
39.561151059 39.2782707214355 0.0
0 -0.00219696760177612 1.0
0 0.0362330079078674 2.0
0 -0.0246546864509583 3.0
0 0.0312585830688477 4.0
0 -0.0289401412010193 5.0
0 0.00624489784240723 6.0
0 -0.012248694896698 7.0
43.566431947 39.6282119750977 0.0
0 0.0138700008392334 1.0
0 0.025522768497467 2.0
0 -0.0197473764419556 3.0
0 0.0512849688529968 4.0
0 -0.0301445722579956 5.0
0 0.00931829214096069 6.0
0 -0.0174809098243713 7.0
37.929366562 39.231330871582 0.0
0 0.010800302028656 1.0
0 0.043013334274292 2.0
0 -0.0228856801986694 3.0
0 0.0199460387229919 4.0
0 -0.0271242260932922 5.0
0 0.00468665361404419 6.0
0 -0.011076033115387 7.0
39.745668641 41.7682800292969 0.0
0 0.00947737693786621 1.0
0 0.0982784032821655 2.0
0 -0.0173757672309875 3.0
0 0.00384366512298584 4.0
0 -0.0101626515388489 5.0
0 -0.000672698020935059 6.0
0 -0.00756984949111938 7.0
37.173794625 39.3144912719727 0.0
0 0.00357526540756226 1.0
0 0.0787873268127441 2.0
0 -0.0133687853813171 3.0
0 0.00534951686859131 4.0
0 -0.0198338031768799 5.0
0 0.00434726476669312 6.0
0 -0.00697380304336548 7.0
27.124866015 39.1525268554688 0.0
0 0.00913196802139282 1.0
0 0.0650781393051147 2.0
0 -0.00544387102127075 3.0
0 -0.00907778739929199 4.0
0 -0.0324002504348755 5.0
0 0.00577831268310547 6.0
0 -0.00207573175430298 7.0
27.411746904 36.7663421630859 0.0
0 0.0113141536712646 1.0
0 0.0197427868843079 2.0
0 -0.0141464471817017 3.0
0 -0.00583177804946899 4.0
0 -0.0341864824295044 5.0
0 0.00935220718383789 6.0
0 0.000265181064605713 7.0
38.866617317 39.1861763000488 0.0
0 0.00661951303482056 1.0
0 0.056182324886322 2.0
0 -0.021723210811615 3.0
0 0.0445036292076111 4.0
0 -0.029047966003418 5.0
0 0.00600475072860718 6.0
0 -0.0126184225082397 7.0
42.383057354 39.06201171875 0.0
0 0.0104770064353943 1.0
0 0.0411927103996277 2.0
0 -0.0221678614616394 3.0
0 0.0380172729492188 4.0
0 -0.0284179449081421 5.0
0 0.00465226173400879 6.0
0 -0.0151697993278503 7.0
47.643751036 39.1715316772461 0.0
0 0.000791609287261963 1.0
0 0.0480496883392334 2.0
0 -0.0215412378311157 3.0
0 0.053810715675354 4.0
0 -0.0269570350646973 5.0
0 0.00569742918014526 6.0
0 -0.0143246054649353 7.0
38.439399957 37.768684387207 0.0
0 0.00674319267272949 1.0
0 0.064213752746582 2.0
0 -0.0178337693214417 3.0
0 0.0179169774055481 4.0
0 -0.0244011878967285 5.0
0 0.0028071403503418 6.0
0 -0.0114209651947021 7.0
40.263562371 39.1669425964355 0.0
0 0.0111555457115173 1.0
0 0.0356557369232178 2.0
0 -0.0216387510299683 3.0
0 0.0576434135437012 4.0
0 -0.0272286534309387 5.0
0 0.0108609199523926 6.0
0 -0.0150966048240662 7.0
41.519528397 39.2469940185547 0.0
0 0.00689941644668579 1.0
0 0.0410262942314148 2.0
0 -0.0277458429336548 3.0
0 0.0269702672958374 4.0
0 -0.0244253873825073 5.0
0 0.0042043924331665 6.0
0 -0.0120133757591248 7.0
40.414570474 39.2424240112305 0.0
0 0.0107041597366333 1.0
0 0.0481968522071838 2.0
0 -0.0277416110038757 3.0
0 0.0130128264427185 4.0
0 -0.02088862657547 5.0
0 0.0020906925201416 6.0
0 -0.00745880603790283 7.0
37.834181221 39.3338584899902 0.0
0 0.00449800491333008 1.0
0 0.0605843067169189 2.0
0 -0.0180467963218689 3.0
0 0.0176889896392822 4.0
0 -0.0260162949562073 5.0
0 0.0046238899230957 6.0
0 -0.0100391507148743 7.0
37.201121556 39.261905670166 0.0
0 0.0136870741844177 1.0
0 0.0373289585113525 2.0
0 -0.0291894674301147 3.0
0 0.0102047920227051 4.0
0 -0.027827262878418 5.0
0 0.0071987509727478 6.0
0 -0.00730001926422119 7.0
42.94444514 39.2730560302734 0.0
0 0.0106481313705444 1.0
0 0.0675126910209656 2.0
0 -0.0177286267280579 3.0
0 0.0313223600387573 4.0
0 -0.0179081559181213 5.0
0 0.00385010242462158 6.0
0 -0.0127323865890503 7.0
47.309600034 39.3127517700195 0.0
0 0.0105430483818054 1.0
0 0.0398989319801331 2.0
0 -0.0270081758499146 3.0
0 0.0480376482009888 4.0
0 -0.0255812406539917 5.0
0 0.00514119863510132 6.0
0 -0.0150682926177979 7.0
34.703660747 39.2433204650879 0.0
0 0.0071418285369873 1.0
0 0.0537604689598083 2.0
0 -0.020726203918457 3.0
0 0.00321263074874878 4.0
0 -0.021537184715271 5.0
0 0.00369763374328613 6.0
0 -0.00550210475921631 7.0
39.355679833 39.1759719848633 0.0
0 0.00848031044006348 1.0
0 0.0391460657119751 2.0
0 -0.0200943946838379 3.0
0 0.0308890342712402 4.0
0 -0.0281250476837158 5.0
0 0.00596320629119873 6.0
0 -0.0112216472625732 7.0
36.760838135 39.1233863830566 0.0
0 0.00670671463012695 1.0
0 0.0653762817382812 2.0
0 -0.0138376355171204 3.0
0 0.0073549747467041 4.0
0 -0.0216763615608215 5.0
0 0.00479042530059814 6.0
0 -0.00692027807235718 7.0
45.817767369 39.8167724609375 0.0
0 0.0150848031044006 1.0
0 0.0327014923095703 2.0
0 -0.0391030311584473 3.0
0 0.056266725063324 4.0
0 -0.0295649170875549 5.0
0 0.00774598121643066 6.0
0 -0.0133306980133057 7.0
37.456387944 38.9908180236816 0.0
0 0.00394445657730103 1.0
0 0.0999663472175598 2.0
0 -0.00932472944259644 3.0
0 0.0233494639396667 4.0
0 -0.026712954044342 5.0
0 0.00659197568893433 6.0
0 -0.0119200348854065 7.0
36.979299113 39.0708389282227 0.0
0 0.00956428050994873 1.0
0 0.069067120552063 2.0
0 -0.0124732851982117 3.0
0 0.00505238771438599 4.0
0 -0.0217453241348267 5.0
0 0.00202125310897827 6.0
0 -0.00531578063964844 7.0
44.357392157 39.4652366638184 0.0
0 0.00713968276977539 1.0
0 0.0625491738319397 2.0
0 -0.0172478556632996 3.0
0 0.037219226360321 4.0
0 -0.0215816497802734 5.0
0 0.00404489040374756 6.0
0 -0.0137313008308411 7.0
36.812079299 39.2094459533691 0.0
0 0.00176447629928589 1.0
0 0.0411084890365601 2.0
0 -0.0210328102111816 3.0
0 0.0358702540397644 4.0
0 -0.026917576789856 5.0
0 0.00472712516784668 6.0
0 -0.0162766575813293 7.0
43.172254743 39.3419456481934 0.0
0 0.000443518161773682 1.0
0 0.0612084865570068 2.0
0 -0.0245235562324524 3.0
0 0.0217550992965698 4.0
0 -0.0219128727912903 5.0
0 0.00296598672866821 6.0
0 -0.0115439891815186 7.0
47.109370567 39.1553535461426 0.0
0 0.00755888223648071 1.0
0 0.0679612159729004 2.0
0 -0.0227341055870056 3.0
0 0.0571240186691284 4.0
0 -0.0264320373535156 5.0
0 0.0023043155670166 6.0
0 -0.0164520740509033 7.0
35.871355662 35.5522994995117 0.0
0 0.00543594360351562 1.0
0 0.0244265198707581 2.0
0 -0.0247161984443665 3.0
0 0.0317372679710388 4.0
0 -0.0295183062553406 5.0
0 0.0111661553382874 6.0
0 -0.0115025639533997 7.0
42.169433911 39.2534637451172 0.0
0 0.0117353200912476 1.0
0 0.062811553478241 2.0
0 -0.0239560604095459 3.0
0 0.0422621369361877 4.0
0 -0.0282317399978638 5.0
0 0.00753867626190186 6.0
0 -0.0137905478477478 7.0
38.001249401 39.3316230773926 0.0
0 0.00929278135299683 1.0
0 0.0384349226951599 2.0
0 -0.0226041078567505 3.0
0 0.0314591526985168 4.0
0 -0.0318945050239563 5.0
0 0.00869399309158325 6.0
0 -0.0126127600669861 7.0
36.026296153 39.0085372924805 0.0
0 0.00563478469848633 1.0
0 0.0343274474143982 2.0
0 -0.0304900407791138 3.0
0 0.0181832313537598 4.0
0 -0.0314008593559265 5.0
0 0.00474011898040771 6.0
0 -0.00925129652023315 7.0
33.200213159 39.6004867553711 0.0
0 0.00608885288238525 1.0
0 0.0842117667198181 2.0
0 -0.0121983289718628 3.0
0 0.00233906507492065 4.0
0 -0.0259609818458557 5.0
0 0.00225377082824707 6.0
0 -0.0079498291015625 7.0
36.881947952 39.2454299926758 0.0
0 -0.00684928894042969 1.0
0 0.0569502115249634 2.0
0 -0.0175040364265442 3.0
0 0.0261569619178772 4.0
0 -0.0287220478057861 5.0
0 0.00709050893783569 6.0
0 -0.0138422846794128 7.0
28.363012012 39.311351776123 0.0
0 0.00370430946350098 1.0
0 0.0378327369689941 2.0
0 -0.0167774558067322 3.0
0 -0.00418192148208618 4.0
0 -0.0319845080375671 5.0
0 0.00403189659118652 6.0
0 -0.00433927774429321 7.0
39.574467426 39.1792831420898 0.0
0 0.0124419927597046 1.0
0 0.0643044710159302 2.0
0 -0.0187584757804871 3.0
0 0.0552770495414734 4.0
0 -0.0264163017272949 5.0
0 0.00504046678543091 6.0
0 -0.0174887776374817 7.0
36.22040148 39.3068237304688 0.0
0 0.00884103775024414 1.0
0 0.0396783351898193 2.0
0 -0.0248955488204956 3.0
0 0.0143897533416748 4.0
0 -0.02907395362854 5.0
0 0.00538623332977295 6.0
0 -0.00855058431625366 7.0
31.086432141 39.3125267028809 0.0
0 -0.00659275054931641 1.0
0 0.0630043745040894 2.0
0 -0.0123117566108704 3.0
0 -0.004280686378479 4.0
0 -0.0248154997825623 5.0
0 0.00149953365325928 6.0
0 -0.00380051136016846 7.0
36.274353063 39.3318557739258 0.0
0 0.00835269689559937 1.0
0 0.0660974383354187 2.0
0 -0.0127261281013489 3.0
0 0.000513792037963867 4.0
0 -0.0210362672805786 5.0
0 0.00450301170349121 6.0
0 -0.00523066520690918 7.0
31.65795009 40.0534744262695 0.0
0 0.00283342599868774 1.0
0 0.0983990430831909 2.0
0 -0.0014115571975708 3.0
0 0.0108137130737305 4.0
0 -0.0277411937713623 5.0
0 0.0125419497489929 6.0
0 -0.00817984342575073 7.0
37.059171479 39.6106262207031 0.0
0 0.0132450461387634 1.0
0 0.0257991552352905 2.0
0 -0.0181992650032043 3.0
0 0.0418131947517395 4.0
0 -0.0292995572090149 5.0
0 0.0161434412002563 6.0
0 -0.014696478843689 7.0
45.308862357 39.1841506958008 0.0
0 0.00910359621047974 1.0
0 0.0663491487503052 2.0
0 -0.0206829309463501 3.0
0 0.0444293618202209 4.0
0 -0.0230956673622131 5.0
0 0.0029442310333252 6.0
0 -0.0143704414367676 7.0
32.988279909 39.4551620483398 0.0
0 0.00968480110168457 1.0
0 0.0575180649757385 2.0
0 -0.0297290086746216 3.0
0 0.00657224655151367 4.0
0 -0.0277795195579529 5.0
0 0.0065266489982605 6.0
0 -0.00360023975372314 7.0
41.812200593 42.8991012573242 0.0
0 -0.000670015811920166 1.0
0 0.0709576010704041 2.0
0 -0.022020161151886 3.0
0 0.0297814011573792 4.0
0 -0.0287713408470154 5.0
0 0.006084144115448 6.0
0 -0.0115088224411011 7.0
34.159578007 38.4319915771484 0.0
0 0.0111063122749329 1.0
0 0.034248411655426 2.0
0 -0.0333222150802612 3.0
0 0.00719785690307617 4.0
0 -0.0321474075317383 5.0
0 0.00268566608428955 6.0
0 -0.00621837377548218 7.0
41.353058204 39.204532623291 0.0
0 0.00952857732772827 1.0
0 0.081007719039917 2.0
0 -0.0175600051879883 3.0
0 0.046159565448761 4.0
0 -0.0286591649055481 5.0
0 0.00810974836349487 6.0
0 -0.0123917460441589 7.0
35.663236644 39.2133636474609 0.0
0 0.00283634662628174 1.0
0 0.0574272274971008 2.0
0 -0.0158212780952454 3.0
0 0.0198498964309692 4.0
0 -0.0298638343811035 5.0
0 0.00515240430831909 6.0
0 -0.0104875564575195 7.0
40.444916123 39.1653671264648 0.0
0 0.00414353609085083 1.0
0 0.0910453796386719 2.0
0 -0.0167463421821594 3.0
0 0.038258969783783 4.0
0 -0.0234010815620422 5.0
0 0.00739949941635132 6.0
0 -0.0134220123291016 7.0
39.900416914 39.0888595581055 0.0
0 0.00893539190292358 1.0
0 0.0359787344932556 2.0
0 -0.0211640000343323 3.0
0 0.0231063961982727 4.0
0 -0.0259357690811157 5.0
0 0.00329744815826416 6.0
0 -0.0172143578529358 7.0
34.405324849 39.97607421875 0.0
0 0.0172434449195862 1.0
0 0.0376632213592529 2.0
0 -0.0179058313369751 3.0
0 0.0187467336654663 4.0
0 -0.0309320092201233 5.0
0 0.00770348310470581 6.0
0 -0.00930380821228027 7.0
33.074256004 39.249397277832 0.0
0 0.00738769769668579 1.0
0 0.0685418844223022 2.0
0 -0.0165396332740784 3.0
0 0.00878924131393433 4.0
0 -0.028192400932312 5.0
0 0.00620543956756592 6.0
0 -0.0068475604057312 7.0
40.036170308 41.1524620056152 0.0
0 0.00465995073318481 1.0
0 0.0542076826095581 2.0
0 -0.0169073343276978 3.0
0 0.0155050754547119 4.0
0 -0.0213350653648376 5.0
0 0.00354379415512085 6.0
0 -0.00880926847457886 7.0
44.453206241 39.2687759399414 0.0
0 0.0084221363067627 1.0
0 0.0385976433753967 2.0
0 -0.0236493945121765 3.0
0 0.0401104688644409 4.0
0 -0.0222529768943787 5.0
0 0.0013420581817627 6.0
0 -0.0154792070388794 7.0
33.85004541 39.3401641845703 0.0
0 0.0100825428962708 1.0
0 0.0304266810417175 2.0
0 -0.0158665776252747 3.0
0 -0.00110459327697754 4.0
0 -0.0240499973297119 5.0
0 0.00319308042526245 6.0
0 -0.00285923480987549 7.0
30.537877308 38.9493637084961 0.0
0 0.00709092617034912 1.0
0 0.0584277510643005 2.0
0 -0.010155200958252 3.0
0 0.00216293334960938 4.0
0 -0.0305030345916748 5.0
0 0.00729340314865112 6.0
0 -0.00616723299026489 7.0
27.727440942 38.7583274841309 0.0
0 0.0101714730262756 1.0
0 0.0332983732223511 2.0
0 -0.00794166326522827 3.0
0 -0.0133890509605408 4.0
0 -0.0270724296569824 5.0
0 0.00212264060974121 6.0
0 0.000737905502319336 7.0
35.630182897 39.3510971069336 0.0
0 0.00258451700210571 1.0
0 0.0429811477661133 2.0
0 -0.0208121538162231 3.0
0 0.00156229734420776 4.0
0 -0.0212336182594299 5.0
0 0.00289946794509888 6.0
0 -0.00715601444244385 7.0
31.038251519 39.2236862182617 0.0
0 0.00821179151535034 1.0
0 0.0344059467315674 2.0
0 -0.0368871092796326 3.0
0 0.00533008575439453 4.0
0 -0.0332455635070801 5.0
0 0.00354135036468506 6.0
0 0.527037620544434 7.0
31.867011967 39.251651763916 0.0
0 0.00107324123382568 1.0
0 0.054533839225769 2.0
0 -0.0123904943466187 3.0
0 0.000221490859985352 4.0
0 -0.0262243151664734 5.0
0 0.00323355197906494 6.0
0 -0.00364047288894653 7.0
30.754831646 39.2847442626953 0.0
0 0.000662744045257568 1.0
0 0.0500643253326416 2.0
0 -0.0136899352073669 3.0
0 -0.00537914037704468 4.0
0 -0.0222152471542358 5.0
0 0.00284117460250854 6.0
0 -0.00049436092376709 7.0
38.95156428 39.6339340209961 0.0
0 0.0100151300430298 1.0
0 0.0536243319511414 2.0
0 -0.0311551094055176 3.0
0 0.0400863289833069 4.0
0 -0.0307068228721619 5.0
0 0.00652027130126953 6.0
0 -0.0129345059394836 7.0
33.384480936 40.8775634765625 0.0
0 -0.00124102830886841 1.0
0 0.082556426525116 2.0
0 -0.0175243020057678 3.0
0 0.00131016969680786 4.0
0 -0.0186545848846436 5.0
0 0.0023646354675293 6.0
0 -0.00465917587280273 7.0
37.291199282 39.228099822998 0.0
0 -0.00408190488815308 1.0
0 0.0565981864929199 2.0
0 -0.0146593451499939 3.0
0 -0.00024259090423584 4.0
0 -0.0163029432296753 5.0
0 -0.000132083892822266 6.0
0 -0.00570875406265259 7.0
33.571811016 39.1781425476074 0.0
0 -0.00461328029632568 1.0
0 0.0580087304115295 2.0
0 -0.0127177834510803 3.0
0 0.00326383113861084 4.0
0 -0.0277678370475769 5.0
0 0.00513482093811035 6.0
0 -0.00657427310943604 7.0
44.048761746 39.1108169555664 0.0
0 0.0133167505264282 1.0
0 0.0345013737678528 2.0
0 -0.0266736149787903 3.0
0 0.0249786972999573 4.0
0 -0.0194141864776611 5.0
0 0.0016477108001709 6.0
0 -0.0132050514221191 7.0
38.09281546 39.2019882202148 0.0
0 0.0076175332069397 1.0
0 0.0517597794532776 2.0
0 -0.0209031105041504 3.0
0 0.0179016590118408 4.0
0 -0.026101291179657 5.0
0 0.00557214021682739 6.0
0 -0.00983411073684692 7.0
42.616606924 39.1387519836426 0.0
0 0.0139750242233276 1.0
0 0.0358852744102478 2.0
0 -0.0247586965560913 3.0
0 0.0312950015068054 4.0
0 -0.0261248350143433 5.0
0 0.00321567058563232 6.0
0 -0.0164498090744019 7.0
41.080227074 39.3054122924805 0.0
0 0.00673854351043701 1.0
0 0.0645447969436646 2.0
0 -0.0166183114051819 3.0
0 0.0444710850715637 4.0
0 -0.0268169045448303 5.0
0 0.00533783435821533 6.0
0 -0.01416015625 7.0
36.859087813 39.3295364379883 0.0
0 0.011201024055481 1.0
0 0.0334835648536682 2.0
0 -0.0205079913139343 3.0
0 0.00591433048248291 4.0
0 -0.0269621610641479 5.0
0 0.00424182415008545 6.0
0 -0.00879001617431641 7.0
39.316268876 39.2730102539062 0.0
0 0.0123557448387146 1.0
0 0.0479531288146973 2.0
0 -0.026253879070282 3.0
0 0.0314406752586365 4.0
0 -0.0275577306747437 5.0
0 0.00751090049743652 6.0
0 -0.0127378702163696 7.0
40.023071967 39.1557769775391 0.0
0 0.00954633951187134 1.0
0 0.0609726905822754 2.0
0 -0.0256583690643311 3.0
0 0.0391594171524048 4.0
0 -0.0301165580749512 5.0
0 0.00344556570053101 6.0
0 -0.0116539001464844 7.0
32.691039566 39.2794570922852 0.0
0 -0.0015331506729126 1.0
0 0.0595820546150208 2.0
0 -0.0134975910186768 3.0
0 0.000495076179504395 4.0
0 -0.02903151512146 5.0
0 0.00216841697692871 6.0
0 -0.00742316246032715 7.0
39.35017223 39.2273406982422 0.0
0 0.0103467106819153 1.0
0 0.0433016419410706 2.0
0 -0.0213442444801331 3.0
0 0.0215471982955933 4.0
0 -0.0266906023025513 5.0
0 0.00573867559432983 6.0
0 -0.0110342502593994 7.0
38.62577922 39.0582504272461 0.0
0 0.00206905603408813 1.0
0 0.0952557921409607 2.0
0 -0.0130048990249634 3.0
0 0.0412141084671021 4.0
0 -0.0248604416847229 5.0
0 0.00641036033630371 6.0
0 -0.0157877802848816 7.0
43.36321018 39.2144470214844 0.0
0 0.000869214534759521 1.0
0 0.0585317015647888 2.0
0 -0.0207402110099792 3.0
0 0.0382145643234253 4.0
0 -0.0245130062103271 5.0
0 0.00293725728988647 6.0
0 -0.0152141451835632 7.0
41.711370519 39.4875259399414 0.0
0 0.0113811492919922 1.0
0 0.0472018122673035 2.0
0 -0.0314878225326538 3.0
0 0.0203774571418762 4.0
0 -0.0202274322509766 5.0
0 0.00135642290115356 6.0
0 -0.0107417106628418 7.0
41.401748585 39.4137382507324 0.0
0 0.00584369897842407 1.0
0 0.0841087698936462 2.0
0 -0.0202708840370178 3.0
0 0.0407012104988098 4.0
0 -0.026205837726593 5.0
0 0.00837677717208862 6.0
0 -0.0136880874633789 7.0
45.411303661 39.0654220581055 0.0
0 0.0100109577178955 1.0
0 0.0425238609313965 2.0
0 -0.0348400473594666 3.0
0 0.0446252226829529 4.0
0 -0.0273600816726685 5.0
0 0.00585973262786865 6.0
0 -0.011528491973877 7.0
46.173622738 39.1124382019043 0.0
0 0.00331169366836548 1.0
0 0.0585666298866272 2.0
0 -0.0199772119522095 3.0
0 0.0386195778846741 4.0
0 -0.0185463428497314 5.0
0 0.000947475433349609 6.0
0 -0.0127515196800232 7.0
43.970500892 39.0283279418945 0.0
0 0.0122954845428467 1.0
0 0.0489515066146851 2.0
0 -0.0201903581619263 3.0
0 0.047145664691925 4.0
0 -0.0270442962646484 5.0
0 0.00498276948928833 6.0
0 -0.0150191783905029 7.0
41.561964524 39.1679534912109 0.0
0 0.00369572639465332 1.0
0 0.0706163048744202 2.0
0 -0.0154618620872498 3.0
0 0.0192911028862 4.0
0 -0.0254248380661011 5.0
0 0.00374150276184082 6.0
0 -0.00939935445785522 7.0
42.043570834 39.0953025817871 0.0
0 0.0122132301330566 1.0
0 0.0394858121871948 2.0
0 -0.0237278342247009 3.0
0 0.0304210186004639 4.0
0 -0.0229068994522095 5.0
0 0.00413888692855835 6.0
0 -0.0118696093559265 7.0
42.728187908 39.2809371948242 0.0
0 0.00305062532424927 1.0
0 0.0476612448692322 2.0
0 -0.0216407179832458 3.0
0 0.0432797074317932 4.0
0 -0.0265682935714722 5.0
0 0.0038069486618042 6.0
0 -0.0142127871513367 7.0
46.647822945 39.149055480957 0.0
0 0.0102046728134155 1.0
0 0.0686874389648438 2.0
0 -0.0248903036117554 3.0
0 0.0577712059020996 4.0
0 -0.0270435810089111 5.0
0 0.00424551963806152 6.0
0 -0.0131389498710632 7.0
34.157298462 39.2264099121094 0.0
0 0.00863933563232422 1.0
0 0.0604138970375061 2.0
0 -0.0150264501571655 3.0
0 0.00885719060897827 4.0
0 -0.0287423133850098 5.0
0 0.00353121757507324 6.0
0 -0.00855332612991333 7.0
38.08940786 39.6085052490234 0.0
0 0.00380009412765503 1.0
0 0.0974803566932678 2.0
0 -0.0130190849304199 3.0
0 -0.00274521112442017 4.0
0 -0.0061413049697876 5.0
0 -0.00102388858795166 6.0
0 -0.00624740123748779 7.0
48.253456605 39.9512825012207 0.0
0 0.0059741735458374 1.0
0 0.0713126659393311 2.0
0 -0.0189449787139893 3.0
0 0.059030294418335 4.0
0 -0.0232948064804077 5.0
0 0.00563764572143555 6.0
0 -0.0141817331314087 7.0
41.303512955 39.4999008178711 0.0
0 0.00800210237503052 1.0
0 0.0413928627967834 2.0
0 -0.0225116014480591 3.0
0 0.0409272313117981 4.0
0 -0.0284388065338135 5.0
0 0.00299149751663208 6.0
0 -0.0151891708374023 7.0
34.11247552 39.3124771118164 0.0
0 0.0101791024208069 1.0
0 0.0430975556373596 2.0
0 -0.0161041021347046 3.0
0 -0.00142985582351685 4.0
0 -0.0245183706283569 5.0
0 0.00401866436004639 6.0
0 -0.00263869762420654 7.0
39.346899321 39.1841049194336 0.0
0 0.0121107697486877 1.0
0 0.0421473383903503 2.0
0 -0.0220503211021423 3.0
0 0.0479787588119507 4.0
0 -0.0291696786880493 5.0
0 0.00787287950515747 6.0
0 -0.015704333782196 7.0
42.355081321 39.1616287231445 0.0
0 0.00376284122467041 1.0
0 0.0752883553504944 2.0
0 -0.0166050791740417 3.0
0 0.0418010354042053 4.0
0 -0.0204254984855652 5.0
0 0.00545191764831543 6.0
0 -0.0175421833992004 7.0
46.025284619 39.0997200012207 0.0
0 0.0107316970825195 1.0
0 0.0411190390586853 2.0
0 -0.0379164218902588 3.0
0 0.0383503437042236 4.0
0 -0.0208945274353027 5.0
0 0.000841259956359863 6.0
0 -0.012586236000061 7.0
48.790909787 39.3467254638672 0.0
0 0.00544178485870361 1.0
0 0.0473642945289612 2.0
0 -0.0263350009918213 3.0
0 0.0592371225357056 4.0
0 -0.0242483615875244 5.0
0 0.0057494044303894 6.0
0 -0.0136021971702576 7.0
41.698940502 39.3436889648438 0.0
0 0.0103285908699036 1.0
0 0.0436311960220337 2.0
0 -0.0286429524421692 3.0
0 0.0448897480964661 4.0
0 -0.0313480496406555 5.0
0 0.00959122180938721 6.0
0 -0.0129580497741699 7.0
47.902972781 39.1910247802734 0.0
0 0.0104819536209106 1.0
0 0.0497010946273804 2.0
0 -0.0263906717300415 3.0
0 0.0467828512191772 4.0
0 -0.0187551379203796 5.0
0 0.00208765268325806 6.0
0 -0.0136069655418396 7.0
41.37717087 39.0758895874023 0.0
0 0.00883805751800537 1.0
0 0.0678055286407471 2.0
0 -0.0207234621047974 3.0
0 0.029201865196228 4.0
0 -0.0210285186767578 5.0
0 0.00388187170028687 6.0
0 -0.0106710195541382 7.0
32.285684902 39.1620063781738 0.0
0 0.00987410545349121 1.0
0 0.0596850514411926 2.0
0 -0.0258843302726746 3.0
0 -0.00052642822265625 4.0
0 -0.0234203934669495 5.0
0 0.00188684463500977 6.0
0 -0.00331372022628784 7.0
32.344206851 39.1839065551758 0.0
0 0.00693005323410034 1.0
0 0.0404759049415588 2.0
0 -0.0222252607345581 3.0
0 0.0102599263191223 4.0
0 -0.0320175290107727 5.0
0 0.00594037771224976 6.0
0 -0.00761276483535767 7.0
34.184112331 39.1595001220703 0.0
0 0.000822484493255615 1.0
0 0.0435900092124939 2.0
0 -0.0234014391899109 3.0
0 0.016313374042511 4.0
0 -0.0279474854469299 5.0
0 0.00414592027664185 6.0
0 -0.0104918479919434 7.0
34.663221808 39.3465347290039 0.0
0 -0.000472068786621094 1.0
0 0.0439326167106628 2.0
0 -0.0197044610977173 3.0
0 0.0183951854705811 4.0
0 -0.0307208895683289 5.0
0 0.00468224287033081 6.0
0 -0.00934088230133057 7.0
36.873774543 37.5921669006348 0.0
0 0.0069698691368103 1.0
0 0.0686353445053101 2.0
0 -0.0163471102714539 3.0
0 0.0109089612960815 4.0
0 -0.0253394842147827 5.0
0 0.00617462396621704 6.0
0 -0.0072709321975708 7.0
40.211352908 41.326587677002 0.0
0 0.00541871786117554 1.0
0 0.0645110607147217 2.0
0 -0.01763916015625 3.0
0 0.0509659647941589 4.0
0 -0.027746319770813 5.0
0 0.0062638521194458 6.0
0 -0.0150855183601379 7.0
46.160689867 39.1345748901367 0.0
0 0.00654143095016479 1.0
0 0.0590987801551819 2.0
0 -0.0207692384719849 3.0
0 0.0336949825286865 4.0
0 -0.0170185565948486 5.0
0 0.000889062881469727 6.0
0 -0.0137842893600464 7.0
32.727702676 39.3031883239746 0.0
0 -0.00463449954986572 1.0
0 0.0640050768852234 2.0
0 -0.0136813521385193 3.0
0 0.00972533226013184 4.0
0 -0.0276746153831482 5.0
0 0.00688004493713379 6.0
0 -0.00769346952438354 7.0
40.640985417 39.6990623474121 0.0
0 0.00356471538543701 1.0
0 0.0938190817832947 2.0
0 -0.0121647715568542 3.0
0 0.0314699411392212 4.0
0 -0.0217582583427429 5.0
0 0.00673258304595947 6.0
0 -0.0126134753227234 7.0
34.611451594 39.2935600280762 0.0
0 0.000468611717224121 1.0
0 0.0598671436309814 2.0
0 -0.0131622552871704 3.0
0 0.00562536716461182 4.0
0 -0.0259057879447937 5.0
0 0.00558769702911377 6.0
0 -0.00558269023895264 7.0
43.311280807 39.3564987182617 0.0
0 0.00210869312286377 1.0
0 0.0423729419708252 2.0
0 -0.0237454771995544 3.0
0 0.0202456712722778 4.0
0 -0.0209356546401978 5.0
0 0.00234854221343994 6.0
0 -0.0102128982543945 7.0
36.406364057 39.175666809082 0.0
0 0.0102632641792297 1.0
0 0.0409222841262817 2.0
0 -0.0188425183296204 3.0
0 0.00562071800231934 4.0
0 -0.0216001868247986 5.0
0 0.00233513116836548 6.0
0 -0.00800949335098267 7.0
31.685630326 40.3169288635254 0.0
0 0.00647974014282227 1.0
0 0.101521492004395 2.0
0 -0.00225663185119629 3.0
0 0.00412189960479736 4.0
0 -0.0269104242324829 5.0
0 0.00584179162979126 6.0
0 -0.00681865215301514 7.0
40.165364432 39.4257736206055 0.0
0 0.0061718225479126 1.0
0 0.105965197086334 2.0
0 -0.0121121406555176 3.0
0 0.0475353002548218 4.0
0 -0.0240756273269653 5.0
0 0.00451123714447021 6.0
0 -0.0161829590797424 7.0
42.886465647 39.1565208435059 0.0
0 0.00445181131362915 1.0
0 0.058378279209137 2.0
0 -0.0202310085296631 3.0
0 0.0512354969978333 4.0
0 -0.0279179811477661 5.0
0 0.00581037998199463 6.0
0 -0.0149148106575012 7.0
46.942592332 40.7634811401367 0.0
0 0.00733822584152222 1.0
0 0.0581620931625366 2.0
0 -0.0207223296165466 3.0
0 0.0556744337081909 4.0
0 -0.0250055193901062 5.0
0 0.00304579734802246 6.0
0 -0.0171255469322205 7.0
38.417030568 39.1845207214355 0.0
0 -0.000441789627075195 1.0
0 0.105224907398224 2.0
0 -0.00935876369476318 3.0
0 0.0345804691314697 4.0
0 -0.0238878726959229 5.0
0 0.00515294075012207 6.0
0 -0.0114102363586426 7.0
39.82159543 39.3340148925781 0.0
0 -0.00290924310684204 1.0
0 0.0601381659507751 2.0
0 -0.0178883075714111 3.0
0 0.0185866355895996 4.0
0 -0.0235638618469238 5.0
0 0.00506407022476196 6.0
0 -0.00952208042144775 7.0
42.375846269 40.3157577514648 0.0
0 0.00914698839187622 1.0
0 0.0349929332733154 2.0
0 -0.0344812273979187 3.0
0 0.013907790184021 4.0
0 -0.0226353406906128 5.0
0 0.00246036052703857 6.0
0 -0.010317862033844 7.0
41.909868924 39.2151527404785 0.0
0 0.00799280405044556 1.0
0 0.0414422750473022 2.0
0 -0.022936224937439 3.0
0 0.0615046620368958 4.0
0 -0.0280951261520386 5.0
0 0.0132050514221191 6.0
0 -0.0195711851119995 7.0
38.515403814 39.1933288574219 0.0
0 0.00757837295532227 1.0
0 0.0305657386779785 2.0
0 -0.0187130570411682 3.0
0 0.00982528924942017 4.0
0 -0.0262295603752136 5.0
0 0.00381177663803101 6.0
0 -0.0142884850502014 7.0
32.187464773 40.187915802002 0.0
0 0.00200361013412476 1.0
0 0.064132034778595 2.0
0 -0.0111357569694519 3.0
0 -0.000443577766418457 4.0
0 -0.025220513343811 5.0
0 0.0035330057144165 6.0
0 -0.00435435771942139 7.0
43.553340498 39.1683349609375 0.0
0 0.00320440530776978 1.0
0 0.0983927845954895 2.0
0 -0.0131569504737854 3.0
0 0.0549652576446533 4.0
0 -0.0220618844032288 5.0
0 0.00318139791488647 6.0
0 -0.0133134126663208 7.0
38.661685118 39.4650688171387 0.0
0 0.00699847936630249 1.0
0 0.0637048482894897 2.0
0 -0.0227053165435791 3.0
0 0.053022027015686 4.0
0 -0.0268428325653076 5.0
0 0.00573450326919556 6.0
0 -0.0151920914649963 7.0
47.185976948 38.8878593444824 0.0
0 0.00848960876464844 1.0
0 0.0649888515472412 2.0
0 -0.0193367600440979 3.0
0 0.0486640930175781 4.0
0 -0.0238392949104309 5.0
0 0.00460058450698853 6.0
0 -0.0163248777389526 7.0
41.073649825 39.4418640136719 0.0
0 0.01736980676651 1.0
0 0.0300519466400146 2.0
0 -0.0189222693443298 3.0
0 0.0178244709968567 4.0
0 -0.0293866395950317 5.0
0 0.00656688213348389 6.0
0 -0.00890588760375977 7.0
39.086114846 39.2041931152344 0.0
0 0.00746828317642212 1.0
0 0.0617871284484863 2.0
0 -0.0160078406333923 3.0
0 0.0467209815979004 4.0
0 -0.0276315808296204 5.0
0 0.00437235832214355 6.0
0 -0.0158487558364868 7.0
38.072309151 39.6437606811523 0.0
0 0.00959813594818115 1.0
0 0.0610776543617249 2.0
0 -0.0135322213172913 3.0
0 0.00896525382995605 4.0
0 -0.023659884929657 5.0
0 0.00476402044296265 6.0
0 -0.00752836465835571 7.0
41.585402374 39.4675216674805 0.0
0 0.0107553005218506 1.0
0 0.0416585206985474 2.0
0 -0.0218178033828735 3.0
0 0.0383657217025757 4.0
0 -0.0271846652030945 5.0
0 0.00441157817840576 6.0
0 -0.0129627585411072 7.0
42.317317809 39.0025901794434 0.0
0 0.0110228657722473 1.0
0 0.0656244158744812 2.0
0 -0.0249553322792053 3.0
0 0.0582363605499268 4.0
0 -0.0284330248832703 5.0
0 0.00378590822219849 6.0
0 -0.0140643119812012 7.0
42.931477585 39.7166442871094 0.0
0 0.00569617748260498 1.0
0 0.0841277241706848 2.0
0 -0.0175623297691345 3.0
0 0.0348663926124573 4.0
0 -0.0192890167236328 5.0
0 0.00405728816986084 6.0
0 -0.0120517611503601 7.0
46.096274411 39.246166229248 0.0
0 -0.00434160232543945 1.0
0 0.0482379794120789 2.0
0 -0.0227041244506836 3.0
0 0.0343828797340393 4.0
0 -0.0184457898139954 5.0
0 0.000456690788269043 6.0
0 -0.0140078067779541 7.0
39.955261554 39.2918701171875 0.0
0 0.0116240978240967 1.0
0 0.0501741766929626 2.0
0 -0.0237830281257629 3.0
0 0.0337701439857483 4.0
0 -0.0290184617042542 5.0
0 0.00481361150741577 6.0
0 -0.0128209590911865 7.0
45.415961846 39.1588554382324 0.0
0 0.00861090421676636 1.0
0 0.0404647588729858 2.0
0 -0.0227407813072205 3.0
0 0.0548158288002014 4.0
0 -0.0274008512496948 5.0
0 0.00865215063095093 6.0
0 -0.0156762003898621 7.0
44.275746657 39.1352005004883 0.0
0 0.00998026132583618 1.0
0 0.0705584287643433 2.0
0 -0.0192747116088867 3.0
0 0.0429856181144714 4.0
0 -0.0221502780914307 5.0
0 0.00686490535736084 6.0
0 -0.0132014751434326 7.0
34.766238966 39.1581649780273 0.0
0 0.0108010172843933 1.0
0 0.066469132900238 2.0
0 -0.0145248174667358 3.0
0 0.0109792947769165 4.0
0 -0.0286319255828857 5.0
0 0.00657302141189575 6.0
0 -0.00772237777709961 7.0
39.697519492 40.0110397338867 0.0
0 0.00621956586837769 1.0
0 0.0532541871070862 2.0
0 -0.0158854722976685 3.0
0 0.0149997472763062 4.0
0 -0.0200769305229187 5.0
0 0.00282829999923706 6.0
0 -0.0118199586868286 7.0
36.459952599 39.2017936706543 0.0
0 0.00732451677322388 1.0
0 0.0604976415634155 2.0
0 -0.016093373298645 3.0
0 0.00341081619262695 4.0
0 -0.0244671106338501 5.0
0 0.00652408599853516 6.0
0 -0.00836789608001709 7.0
44.856515586 39.419303894043 0.0
0 0.00477176904678345 1.0
0 0.0616564154624939 2.0
0 -0.0213950872421265 3.0
0 0.0443920493125916 4.0
0 -0.0205050706863403 5.0
0 0.00288867950439453 6.0
0 -0.0148175954818726 7.0
41.157913956 39.2678260803223 0.0
0 0.00864797830581665 1.0
0 0.0603436231613159 2.0
0 -0.0151415467262268 3.0
0 0.0148828029632568 4.0
0 -0.0174505710601807 5.0
0 0.00234842300415039 6.0
0 -0.00891876220703125 7.0
32.092085792 42.0078506469727 0.0
0 -0.00147992372512817 1.0
0 0.0986078977584839 2.0
0 -0.00447070598602295 3.0
0 0.00738292932510376 4.0
0 -0.027155876159668 5.0
0 0.00754690170288086 6.0
0 -0.00817745923995972 7.0
30.327518425 39.2771072387695 0.0
0 0.000197649002075195 1.0
0 0.05263751745224 2.0
0 -0.00881850719451904 3.0
0 -0.00964021682739258 4.0
0 -0.0268546342849731 5.0
0 0.00452160835266113 6.0
0 -0.00201511383056641 7.0
36.482977278 39.6762313842773 0.0
0 0.00489264726638794 1.0
0 0.0885087847709656 2.0
0 -0.0142347812652588 3.0
0 0.0247588753700256 4.0
0 -0.028696596622467 5.0
0 0.00561767816543579 6.0
0 -0.0122042894363403 7.0
40.491170914 40.3549003601074 0.0
0 0.000296533107757568 1.0
0 0.0906490683555603 2.0
0 -0.0146321058273315 3.0
0 0.0381223559379578 4.0
0 -0.0233554840087891 5.0
0 0.00532442331314087 6.0
0 -0.012855052947998 7.0
41.188898334 40.0176544189453 0.0
0 -0.00143259763717651 1.0
0 0.0951639413833618 2.0
0 -0.0130124092102051 3.0
0 0.0215404033660889 4.0
0 -0.0166791677474976 5.0
0 0.00263166427612305 6.0
0 -0.00915664434432983 7.0
37.756949904 39.1947593688965 0.0
0 0.0107933282852173 1.0
0 0.0656872391700745 2.0
0 -0.0142897963523865 3.0
0 0.0106445550918579 4.0
0 -0.0203924179077148 5.0
0 0.00396895408630371 6.0
0 -0.00912606716156006 7.0
26.9626458 39.3925285339355 0.0
0 0.0110253691673279 1.0
0 0.0370810627937317 2.0
0 -0.00381720066070557 3.0
0 -0.016218900680542 4.0
0 -0.0292441248893738 5.0
0 0.00169575214385986 6.0
0 -0.00216329097747803 7.0
44.154680866 39.3581161499023 0.0
0 0.00145381689071655 1.0
0 0.0635524392127991 2.0
0 -0.0199567079544067 3.0
0 0.029373824596405 4.0
0 -0.0141153931617737 5.0
0 0.000345468521118164 6.0
0 -0.0163211822509766 7.0
36.49920361 39.3040390014648 0.0
0 0.00980162620544434 1.0
0 0.046657919883728 2.0
0 -0.0151821374893188 3.0
0 0.0272837281227112 4.0
0 -0.0288670659065247 5.0
0 0.00800848007202148 6.0
0 -0.0117116570472717 7.0
36.136187007 39.2891960144043 0.0
0 0.0114787220954895 1.0
0 0.0657549500465393 2.0
0 -0.0165747404098511 3.0
0 0.0228102207183838 4.0
0 -0.0275917053222656 5.0
0 0.00385874509811401 6.0
0 -0.0123043656349182 7.0
43.805905709 39.2161407470703 0.0
0 0.00916695594787598 1.0
0 0.0414623022079468 2.0
0 -0.0270732641220093 3.0
0 0.0342727899551392 4.0
0 -0.0226308107376099 5.0
0 0.00324547290802002 6.0
0 -0.012790858745575 7.0
43.400673478 38.5996551513672 0.0
0 0.00959324836730957 1.0
0 0.07652348279953 2.0
0 -0.0166631937026978 3.0
0 0.0510177612304688 4.0
0 -0.0256300568580627 5.0
0 0.00306117534637451 6.0
0 -0.0166022777557373 7.0
35.708898932 39.7909126281738 0.0
0 0.00258690118789673 1.0
0 0.101891338825226 2.0
0 -0.0113672614097595 3.0
0 0.0187270045280457 4.0
0 -0.026531994342804 5.0
0 0.00352787971496582 6.0
0 -0.010202944278717 7.0
42.195375979 39.2065582275391 0.0
0 0.00936859846115112 1.0
0 0.0624610185623169 2.0
0 -0.0231862664222717 3.0
0 0.0374898910522461 4.0
0 -0.0261919498443604 5.0
0 0.00701963901519775 6.0
0 -0.0106062889099121 7.0
38.219132252 39.2476501464844 0.0
0 0.00796747207641602 1.0
0 0.060630202293396 2.0
0 -0.0189728736877441 3.0
0 0.00981682538986206 4.0
0 -0.0244853496551514 5.0
0 0.00241279602050781 6.0
0 -0.00450032949447632 7.0
37.34785904 40.7463417053223 0.0
0 0.00804460048675537 1.0
0 0.0386269092559814 2.0
0 -0.0224984884262085 3.0
0 0.0120346546173096 4.0
0 -0.0248059034347534 5.0
0 0.0053831934928894 6.0
0 -0.00742852687835693 7.0
41.413317083 39.1686630249023 0.0
0 0.0103911161422729 1.0
0 0.0369403958320618 2.0
0 -0.0325966477394104 3.0
0 0.0260374546051025 4.0
0 -0.0254538059234619 5.0
0 0.00346553325653076 6.0
0 -0.0124757289886475 7.0
39.408474012 41.2074737548828 0.0
0 0.00463259220123291 1.0
0 0.0393736362457275 2.0
0 -0.0281912088394165 3.0
0 0.0336098074913025 4.0
0 -0.0318562984466553 5.0
0 0.00816363096237183 6.0
0 -0.0106446146965027 7.0
44.621947793 39.146858215332 0.0
0 0.0129572153091431 1.0
0 0.0404406189918518 2.0
0 -0.0323220491409302 3.0
0 0.0574005246162415 4.0
0 -0.0294423699378967 5.0
0 0.0104334950447083 6.0
0 -0.0138011574745178 7.0
35.042314673 41.5502014160156 0.0
0 0.0104658603668213 1.0
0 0.0302814841270447 2.0
0 -0.0129294991493225 3.0
0 0.00461137294769287 4.0
0 -0.0273417234420776 5.0
0 0.00530403852462769 6.0
0 -0.00574469566345215 7.0
44.668685987 37.9078559875488 0.0
0 0.00188130140304565 1.0
0 0.0569884777069092 2.0
0 -0.0238006114959717 3.0
0 0.0405648350715637 4.0
0 -0.0205771923065186 5.0
0 0.00241458415985107 6.0
0 -0.0139407515525818 7.0
44.271188235 39.1885566711426 0.0
0 0.0127209424972534 1.0
0 0.0644296407699585 2.0
0 -0.0185632109642029 3.0
0 0.0303701162338257 4.0
0 -0.0184800624847412 5.0
0 0.00196337699890137 6.0
0 -0.0148364305496216 7.0
43.329615715 39.419849395752 0.0
0 0.00306707620620728 1.0
0 0.0421066284179688 2.0
0 -0.0237050652503967 3.0
0 0.0451827645301819 4.0
0 -0.0286803841590881 5.0
0 0.00414633750915527 6.0
0 -0.0128758549690247 7.0
37.727034893 39.9785270690918 0.0
0 0.00373846292495728 1.0
0 0.103487849235535 2.0
0 -0.00995522737503052 3.0
0 0.0277867317199707 4.0
0 -0.0272249579429626 5.0
0 0.00663357973098755 6.0
0 -0.0104453563690186 7.0
43.665312113 39.5566329956055 0.0
0 0.00762540102005005 1.0
0 0.0750016570091248 2.0
0 -0.0351837277412415 3.0
0 0.0328378081321716 4.0
0 -0.0215832591056824 5.0
0 0.00253617763519287 6.0
0 -0.0119035840034485 7.0
40.656348383 39.3315620422363 0.0
0 0.00649392604827881 1.0
0 0.0533838868141174 2.0
0 -0.0215229988098145 3.0
0 0.0549891591072083 4.0
0 -0.0267018675804138 5.0
0 0.0048946738243103 6.0
0 -0.0185717940330505 7.0
30.759139555 39.0844345092773 0.0
0 0.0121983289718628 1.0
0 0.0359728336334229 2.0
0 -0.0149562954902649 3.0
0 -0.00471550226211548 4.0
0 -0.028002142906189 5.0
0 0.0035325288772583 6.0
0 -0.00117355585098267 7.0
36.474229652 40.8910751342773 0.0
0 0.00264835357666016 1.0
0 0.0991296172142029 2.0
0 -0.00933718681335449 3.0
0 0.00034099817276001 4.0
0 -0.0150880813598633 5.0
0 -0.000272393226623535 6.0
0 -0.00282025337219238 7.0
33.340721195 39.4016265869141 0.0
0 0.00756460428237915 1.0
0 0.0755277276039124 2.0
0 -0.0127190351486206 3.0
0 0.00348019599914551 4.0
0 -0.0280179381370544 5.0
0 0.00535142421722412 6.0
0 -0.00605171918869019 7.0
34.143349268 41.1993026733398 0.0
0 0.0105020999908447 1.0
0 0.0738517045974731 2.0
0 -0.0288435220718384 3.0
0 0.00792866945266724 4.0
0 -0.022797167301178 5.0
0 0.00235277414321899 6.0
0 -0.00532907247543335 7.0
40.763829757 40.4009170532227 0.0
0 0.00627636909484863 1.0
0 0.0678209066390991 2.0
0 -0.0207937955856323 3.0
0 0.0571699142456055 4.0
0 -0.0280265212059021 5.0
0 0.00891619920730591 6.0
0 -0.0133203268051147 7.0
29.788224958 39.3779296875 0.0
0 -0.00283634662628174 1.0
0 0.0589134097099304 2.0
0 -0.011816680431366 3.0
0 0.000213921070098877 4.0
0 -0.03299880027771 5.0
0 0.00723862648010254 6.0
0 -0.00425004959106445 7.0
34.675869554 39.2007484436035 0.0
0 0.00924175977706909 1.0
0 0.062283456325531 2.0
0 -0.0111774206161499 3.0
0 0.00481492280960083 4.0
0 -0.0254631042480469 5.0
0 0.00442290306091309 6.0
0 -0.00621861219406128 7.0
40.723855828 39.139232635498 0.0
0 0.004505455493927 1.0
0 0.0670126080513 2.0
0 -0.0169151425361633 3.0
0 0.0427116751670837 4.0
0 -0.0239702463150024 5.0
0 0.00283890962600708 6.0
0 -0.0185911655426025 7.0
41.414838352 39.2057685852051 0.0
0 0.002247154712677 1.0
0 0.059136688709259 2.0
0 -0.0206869840621948 3.0
0 0.0594696998596191 4.0
0 -0.0269739031791687 5.0
0 0.0113312602043152 6.0
0 -0.0169715285301208 7.0
33.391441214 39.5404624938965 0.0
0 0.00687164068222046 1.0
0 0.068155825138092 2.0
0 -0.0151040554046631 3.0
0 0.00336939096450806 4.0
0 -0.0272725820541382 5.0
0 0.00364166498184204 6.0
0 -0.0084540843963623 7.0
36.660645385 39.1693992614746 0.0
0 0.0112992525100708 1.0
0 0.0539188981056213 2.0
0 -0.0432652831077576 3.0
0 0.0196741223335266 4.0
0 -0.0308671593666077 5.0
0 0.00485384464263916 6.0
0 -0.00823885202407837 7.0
33.547567166 41.1571044921875 0.0
0 0.00413417816162109 1.0
0 0.0913617014884949 2.0
0 -0.0106722712516785 3.0
0 0.0163525938987732 4.0
0 -0.0269976854324341 5.0
0 0.00591397285461426 6.0
0 -0.00831866264343262 7.0
36.717923495 39.3228797912598 0.0
0 -0.00490498542785645 1.0
0 0.0684824585914612 2.0
0 -0.0163276791572571 3.0
0 0.0187729001045227 4.0
0 -0.0261236429214478 5.0
0 0.00353103876113892 6.0
0 -0.0108597874641418 7.0
44.575737494 40.2921981811523 0.0
0 0.00726407766342163 1.0
0 0.0921012759208679 2.0
0 -0.0169275999069214 3.0
0 0.0439640879631042 4.0
0 -0.0148934721946716 5.0
0 0.00246167182922363 6.0
0 -0.0144742131233215 7.0
43.091610321 39.251880645752 0.0
0 0.00294780731201172 1.0
0 0.0471649765968323 2.0
0 -0.0214514136314392 3.0
0 0.0389942526817322 4.0
0 -0.0256434082984924 5.0
0 0.00726276636123657 6.0
0 -0.0136053562164307 7.0
42.647069973 39.2061424255371 0.0
0 0.00906306505203247 1.0
0 0.0546945333480835 2.0
0 -0.0212883949279785 3.0
0 0.0380060076713562 4.0
0 -0.0249283909797668 5.0
0 0.00426143407821655 6.0
0 -0.0141628384590149 7.0
36.240225339 39.1404800415039 0.0
0 0.00918090343475342 1.0
0 0.0593395233154297 2.0
0 -0.0215871930122375 3.0
0 0.0104026794433594 4.0
0 -0.0252101421356201 5.0
0 0.00523948669433594 6.0
0 -0.00701224803924561 7.0
31.876756203 39.139762878418 0.0
0 -0.000991940498352051 1.0
0 0.0431054830551147 2.0
0 -0.0127851963043213 3.0
0 -0.00101745128631592 4.0
0 -0.0230093598365784 5.0
0 0.00262880325317383 6.0
0 -0.00305753946304321 7.0
38.711049659 39.3581237792969 0.0
0 0.00102627277374268 1.0
0 0.0828373432159424 2.0
0 -0.0169112682342529 3.0
0 0.00627750158309937 4.0
0 -0.0123506188392639 5.0
0 -0.000407099723815918 6.0
0 -0.00948333740234375 7.0
33.592635672 39.4242515563965 0.0
0 0.00957810878753662 1.0
0 0.0583613514900208 2.0
0 -0.016112208366394 3.0
0 0.00602555274963379 4.0
0 -0.027643084526062 5.0
0 0.00605815649032593 6.0
0 -0.00576615333557129 7.0
31.939497021 39.1622772216797 0.0
0 0.00103104114532471 1.0
0 0.0396385192871094 2.0
0 -0.0136800408363342 3.0
0 -0.00527811050415039 4.0
0 -0.0252284407615662 5.0
0 0.00408285856246948 6.0
0 -0.00103867053985596 7.0
32.605677369 39.2626190185547 0.0
0 0.00559169054031372 1.0
0 0.0369923114776611 2.0
0 -0.0197780132293701 3.0
0 0.000530064105987549 4.0
0 -0.0273765325546265 5.0
0 0.00718575716018677 6.0
0 -0.00460433959960938 7.0
35.02561425 39.1090469360352 0.0
0 0.0119553208351135 1.0
0 0.0453833937644958 2.0
0 -0.0258418917655945 3.0
0 0.0193158388137817 4.0
0 -0.0308746695518494 5.0
0 0.0087929368019104 6.0
0 -0.007557213306427 7.0
29.155020371 39.2253952026367 0.0
0 0.0101317167282104 1.0
0 0.0378549098968506 2.0
0 -0.0250255465507507 3.0
0 0.00145792961120605 4.0
0 -0.0328030586242676 5.0
0 0.00670236349105835 6.0
0 0.182495445013046 7.0
29.9790999 38.9883193969727 0.0
0 -0.00512790679931641 1.0
0 0.0625416040420532 2.0
0 -0.00465649366378784 3.0
0 -0.00537872314453125 4.0
0 -0.0296611785888672 5.0
0 0.00600504875183105 6.0
0 -0.00503420829772949 7.0
40.412374428 39.2049255371094 0.0
0 -0.000606119632720947 1.0
0 0.0611275434494019 2.0
0 -0.0168014168739319 3.0
0 0.00644654035568237 4.0
0 -0.0177046060562134 5.0
0 0.00162214040756226 6.0
0 -0.0136350393295288 7.0
44.799669607 39.0750770568848 0.0
0 0.00750100612640381 1.0
0 0.0563011765480042 2.0
0 -0.0221543312072754 3.0
0 0.0368587374687195 4.0
0 -0.0180134177207947 5.0
0 0.000889480113983154 6.0
0 -0.0161266922950745 7.0
35.691930085 39.3859329223633 0.0
0 0.00633901357650757 1.0
0 0.0712013244628906 2.0
0 -0.0153719782829285 3.0
0 0.00453132390975952 4.0
0 -0.0222364664077759 5.0
0 0.00377142429351807 6.0
0 -0.00545322895050049 7.0
37.293002382 39.293083190918 0.0
0 0.00778055191040039 1.0
0 0.047900378704071 2.0
0 -0.0283905863761902 3.0
0 0.0200028419494629 4.0
0 -0.0298619866371155 5.0
0 0.00535517930984497 6.0
0 -0.011212944984436 7.0
31.749050635 39.1932945251465 0.0
0 0.00823521614074707 1.0
0 0.0382753014564514 2.0
0 -0.0209130644798279 3.0
0 0.00449585914611816 4.0
0 -0.028892993927002 5.0
0 0.00504481792449951 6.0
0 -0.00455832481384277 7.0
44.438599316 38.0644073486328 0.0
0 0.00910055637359619 1.0
0 0.0278371572494507 2.0
0 -0.0197588205337524 3.0
0 0.0566070675849915 4.0
0 -0.0264150500297546 5.0
0 0.00440853834152222 6.0
0 -0.0190314054489136 7.0
30.584762345 39.4195861816406 0.0
0 0.00186467170715332 1.0
0 0.0638080835342407 2.0
0 -0.0128000378608704 3.0
0 -0.00119149684906006 4.0
0 -0.0299015045166016 5.0
0 0.00222301483154297 6.0
0 -0.00576400756835938 7.0
36.167711467 40.7079963684082 0.0
0 0.00583952665328979 1.0
0 0.0441656112670898 2.0
0 -0.0209394693374634 3.0
0 0.00305551290512085 4.0
0 -0.019661009311676 5.0
0 -0.000128626823425293 6.0
0 -0.0033729076385498 7.0
45.589023165 39.2716369628906 0.0
0 0.0118179321289062 1.0
0 0.0400128960609436 2.0
0 -0.0317465662956238 3.0
0 0.0473290085792542 4.0
0 -0.0295619964599609 5.0
0 0.00329500436782837 6.0
0 -0.0119420289993286 7.0
33.954863975 39.3449859619141 0.0
0 -0.00191313028335571 1.0
0 0.057437539100647 2.0
0 -0.0151700377464294 3.0
0 0.00851249694824219 4.0
0 -0.0251034498214722 5.0
0 0.00245237350463867 6.0
0 -0.0112330913543701 7.0
39.104021217 39.2743453979492 0.0
0 0.00290310382843018 1.0
0 0.0589911937713623 2.0
0 -0.018439769744873 3.0
0 0.0172427892684937 4.0
0 -0.020279586315155 5.0
0 0.00379747152328491 6.0
0 -0.0103532075881958 7.0
41.615139375 39.283073425293 0.0
0 0.00299423933029175 1.0
0 0.0724921226501465 2.0
0 -0.0180994868278503 3.0
0 0.058605432510376 4.0
0 -0.0211931467056274 5.0
0 0.00603270530700684 6.0
0 -0.0181660652160645 7.0
29.251209169 39.1626129150391 0.0
0 0.0102241635322571 1.0
0 0.0657804012298584 2.0
0 -0.0218889713287354 3.0
0 0.000598728656768799 4.0
0 -0.0287654995918274 5.0
0 0.00124531984329224 6.0
0 -0.00544106960296631 7.0
39.145261601 39.434383392334 0.0
0 0.0106642246246338 1.0
0 0.0615565180778503 2.0
0 -0.0195338726043701 3.0
0 0.0287595987319946 4.0
0 -0.0294601917266846 5.0
0 0.0090785026550293 6.0
0 -0.0129284858703613 7.0
43.724223514 39.9443969726562 0.0
0 0.00348567962646484 1.0
0 0.0611519813537598 2.0
0 -0.0157661437988281 3.0
0 0.0497835278511047 4.0
0 -0.0242002606391907 5.0
0 0.00416934490203857 6.0
0 -0.016603946685791 7.0
30.694427948 39.1826820373535 0.0
0 0.00585633516311646 1.0
0 0.0551376342773438 2.0
0 -0.00917786359786987 3.0
0 -0.00169706344604492 4.0
0 -0.0302058458328247 5.0
0 0.00556594133377075 6.0
0 -0.00630462169647217 7.0
46.899446079 39.3772010803223 0.0
0 0.00522828102111816 1.0
0 0.119101643562317 2.0
0 -0.0105419754981995 3.0
0 0.105043053627014 4.0
0 -0.0214530229568481 5.0
0 0.00253283977508545 6.0
0 -0.0160475373268127 7.0
30.801741874 39.1385078430176 0.0
0 0.0105280876159668 1.0
0 0.0485381484031677 2.0
0 -0.0218062996864319 3.0
0 0.00451433658599854 4.0
0 -0.0322909355163574 5.0
0 0.00279700756072998 6.0
0 -0.00666153430938721 7.0
40.388900757 39.6161880493164 0.0
0 0.0105310678482056 1.0
0 0.0530675649642944 2.0
0 -0.0235248208045959 3.0
0 0.0383188128471375 4.0
0 -0.0301554203033447 5.0
0 0.00359994173049927 6.0
0 -0.0129978656768799 7.0
35.801686131 39.3320503234863 0.0
0 0.00162678956985474 1.0
0 0.0328916311264038 2.0
0 -0.0161664485931396 3.0
0 0.0023643970489502 4.0
0 -0.0239839553833008 5.0
0 0.00194603204727173 6.0
0 -0.00305426120758057 7.0
43.275430302 39.1461601257324 0.0
0 0.00677180290222168 1.0
0 0.066714346408844 2.0
0 -0.0179618000984192 3.0
0 0.0459111332893372 4.0
0 -0.0245304107666016 5.0
0 0.00858467817306519 6.0
0 -0.013133704662323 7.0
36.674533885 39.2840042114258 0.0
0 0.00746917724609375 1.0
0 0.0606139898300171 2.0
0 -0.0169159770011902 3.0
0 0.0125620365142822 4.0
0 -0.0260381698608398 5.0
0 0.00515270233154297 6.0
0 -0.00862163305282593 7.0
42.291929941 39.0463104248047 0.0
0 0.011316180229187 1.0
0 0.0493571162223816 2.0
0 -0.0247151851654053 3.0
0 0.0330401062965393 4.0
0 -0.0230516195297241 5.0
0 0.00425034761428833 6.0
0 -0.0135107636451721 7.0
37.80686855 38.719539642334 0.0
0 0.00790625810623169 1.0
0 0.0712704658508301 2.0
0 -0.0141570568084717 3.0
0 0.00576889514923096 4.0
0 -0.0159686803817749 5.0
0 -0.000587940216064453 6.0
0 -0.00549298524856567 7.0
36.296445808 38.5794677734375 0.0
0 0.0126692056655884 1.0
0 0.0595385432243347 2.0
0 -0.0157817602157593 3.0
0 0.00248444080352783 4.0
0 -0.0212079882621765 5.0
0 0.00206708908081055 6.0
0 -0.00864028930664062 7.0
26.786347985 41.6884956359863 0.0
0 1.05500221252441e-05 1.0
0 0.0941086411476135 2.0
0 -0.0172778964042664 3.0
0 -0.00798177719116211 4.0
0 -0.0263301730155945 5.0
0 0.00166332721710205 6.0
0 0.325894445180893 7.0
38.715518937 39.3310127258301 0.0
0 -0.00107306241989136 1.0
0 0.0533255338668823 2.0
0 -0.0208864808082581 3.0
0 0.0355126857757568 4.0
0 -0.0311377644538879 5.0
0 0.00435280799865723 6.0
0 -0.0146806836128235 7.0
35.505063002 39.0710105895996 0.0
0 0.0100252032279968 1.0
0 0.0440443158149719 2.0
0 -0.0252014994621277 3.0
0 0.00799816846847534 4.0
0 -0.0276076197624207 5.0
0 0.00622522830963135 6.0
0 -0.00744861364364624 7.0
44.227649071 39.2503204345703 0.0
0 -0.00350522994995117 1.0
0 0.0830144882202148 2.0
0 -0.0194167494773865 3.0
0 0.0467356443405151 4.0
0 -0.0179108381271362 5.0
0 0.00280278921127319 6.0
0 -0.0161828398704529 7.0
45.773124255 39.1983108520508 0.0
0 0.0102026462554932 1.0
0 0.0390667915344238 2.0
0 -0.0257700681686401 3.0
0 0.0395534634590149 4.0
0 -0.0186536312103271 5.0
0 0.00253766775131226 6.0
0 -0.0136967897415161 7.0
38.615282752 39.6247634887695 0.0
0 0.00875389575958252 1.0
0 0.0653161406517029 2.0
0 -0.0225114226341248 3.0
0 0.0422231554985046 4.0
0 -0.0281192064285278 5.0
0 0.00611317157745361 6.0
0 -0.0130001306533813 7.0
42.131124836 39.6604461669922 0.0
0 0.00814712047576904 1.0
0 0.0577985048294067 2.0
0 -0.0264377593994141 3.0
0 0.0277695655822754 4.0
0 -0.0232623219490051 5.0
0 0.00406908988952637 6.0
0 -0.0118699073791504 7.0
38.030369766 39.4241714477539 0.0
0 0.00730973482131958 1.0
0 0.0649347305297852 2.0
0 -0.0156338810920715 3.0
0 0.028175950050354 4.0
0 -0.0283916592597961 5.0
0 0.00388896465301514 6.0
0 -0.0123919248580933 7.0
32.694870002 39.2483978271484 0.0
0 0.00167417526245117 1.0
0 0.0581066608428955 2.0
0 -0.0132492780685425 3.0
0 -0.00167709589004517 4.0
0 -0.0220803618431091 5.0
0 0.000322103500366211 6.0
0 -0.00424700975418091 7.0
41.58883184 39.7600479125977 0.0
0 0.0111787915229797 1.0
0 0.0836202502250671 2.0
0 -0.0201756954193115 3.0
0 0.014514684677124 4.0
0 -0.0171653032302856 5.0
0 -0.000364601612091064 6.0
0 -0.00911998748779297 7.0
33.622063045 40.3607978820801 0.0
0 0.00406837463378906 1.0
0 0.0649199485778809 2.0
0 -0.0115718841552734 3.0
0 0.0118268728256226 4.0
0 -0.0288293957710266 5.0
0 0.00606423616409302 6.0
0 -0.00943201780319214 7.0
40.101738506 39.3203430175781 0.0
0 0.00397443771362305 1.0
0 0.0380821228027344 2.0
0 -0.0269546508789062 3.0
0 0.0460603833198547 4.0
0 -0.0289177894592285 5.0
0 0.0054628849029541 6.0
0 -0.0136521458625793 7.0
39.347750494 39.5650939941406 0.0
0 0.0116128921508789 1.0
0 0.0399914383888245 2.0
0 -0.0279043912887573 3.0
0 0.0306953191757202 4.0
0 -0.0280945897102356 5.0
0 0.00678908824920654 6.0
0 -0.0120279788970947 7.0
46.240008463 39.3097534179688 0.0
0 0.00658661127090454 1.0
0 0.0587683320045471 2.0
0 -0.0242289304733276 3.0
0 0.0355812907218933 4.0
0 -0.0214058756828308 5.0
0 0.00193798542022705 6.0
0 -0.0121954083442688 7.0
43.885613972 39.8051948547363 0.0
0 0.0136743783950806 1.0
0 0.0521467924118042 2.0
0 -0.0285190343856812 3.0
0 0.0310320258140564 4.0
0 -0.0195971727371216 5.0
0 0.000680863857269287 6.0
0 -0.0110752582550049 7.0
44.877286055 39.2165946960449 0.0
0 0.00419837236404419 1.0
0 0.0352147817611694 2.0
0 -0.0227287411689758 3.0
0 0.062357485294342 4.0
0 -0.0253933668136597 5.0
0 0.00503700971603394 6.0
0 -0.0205078125 7.0
44.281024849 39.100700378418 0.0
0 0.0138094425201416 1.0
0 0.0409066081047058 2.0
0 -0.0262832045555115 3.0
0 0.0474932789802551 4.0
0 -0.0281514525413513 5.0
0 0.00751256942749023 6.0
0 -0.0144731402397156 7.0
39.443195334 39.5614204406738 0.0
0 0.00306022167205811 1.0
0 0.0668423175811768 2.0
0 -0.0195117592811584 3.0
0 0.0196396708488464 4.0
0 -0.0204795002937317 5.0
0 0.00201261043548584 6.0
0 -0.013425886631012 7.0
44.50998012 37.5527420043945 0.0
0 0.00241619348526001 1.0
0 0.0602812767028809 2.0
0 -0.0234832167625427 3.0
0 0.0416800379753113 4.0
0 -0.0199583768844604 5.0
0 0.00301408767700195 6.0
0 -0.0121491551399231 7.0
36.079576822 39.6479034423828 0.0
0 0.0111091732978821 1.0
0 0.0342797040939331 2.0
0 -0.0194944739341736 3.0
0 0.0268609523773193 4.0
0 -0.0297252535820007 5.0
0 0.00870823860168457 6.0
0 -0.0126831531524658 7.0
43.881758876 39.3958206176758 0.0
0 0.0129520893096924 1.0
0 0.0298763513565063 2.0
0 -0.0417283773422241 3.0
0 0.0476492047309875 4.0
0 -0.0290360450744629 5.0
0 0.00537043809890747 6.0
0 -0.013556182384491 7.0
30.108818284 39.1312637329102 0.0
0 0.00127166509628296 1.0
0 0.0570345520973206 2.0
0 -0.0248159170150757 3.0
0 0.00391620397567749 4.0
0 -0.0310516953468323 5.0
0 0.00319862365722656 6.0
0 -0.0063481330871582 7.0
34.22051219 39.2516632080078 0.0
0 0.00860375165939331 1.0
0 0.059270441532135 2.0
0 -0.0170743465423584 3.0
0 0.0202519297599792 4.0
0 -0.030121386051178 5.0
0 0.00781601667404175 6.0
0 -0.0109772086143494 7.0
37.402197971 38.8370780944824 0.0
0 0.0108410120010376 1.0
0 0.0724775195121765 2.0
0 -0.0135229825973511 3.0
0 0.0213173627853394 4.0
0 -0.0238214731216431 5.0
0 0.00379276275634766 6.0
0 -0.00984638929367065 7.0
35.453722409 39.3546600341797 0.0
0 0.00394952297210693 1.0
0 0.0594062805175781 2.0
0 -0.0146932005882263 3.0
0 0.00734615325927734 4.0
0 -0.0265988707542419 5.0
0 0.0027625560760498 6.0
0 -0.00827115774154663 7.0
32.872381155 39.2892608642578 0.0
0 0.010620653629303 1.0
0 0.0609287023544312 2.0
0 -0.0179244875907898 3.0
0 0.00361537933349609 4.0
0 -0.0287152528762817 5.0
0 0.00480866432189941 6.0
0 -0.0058131217956543 7.0
33.577714316 39.4748306274414 0.0
0 -0.00118172168731689 1.0
0 0.0713554620742798 2.0
0 -0.0126799345016479 3.0
0 0.016187310218811 4.0
0 -0.029799222946167 5.0
0 0.00732976198196411 6.0
0 -0.00867438316345215 7.0
43.986168802 39.2046051025391 0.0
0 0.0113417506217957 1.0
0 0.0563039183616638 2.0
0 -0.0383703708648682 3.0
0 0.0311552882194519 4.0
0 -0.0199617147445679 5.0
0 0.000735759735107422 6.0
0 -0.0111079216003418 7.0
39.701527205 39.3187294006348 0.0
0 0.0148863196372986 1.0
0 0.0385289788246155 2.0
0 -0.0257417559623718 3.0
0 0.0237706899642944 4.0
0 -0.0290372967720032 5.0
0 0.00385552644729614 6.0
0 -0.0131402611732483 7.0
36.319422317 40.3933143615723 0.0
0 0.0118604898452759 1.0
0 0.0607532262802124 2.0
0 -0.0213486552238464 3.0
0 0.0343860387802124 4.0
0 -0.0309789180755615 5.0
0 0.00633883476257324 6.0
0 -0.0120580196380615 7.0
39.54652972 39.2991943359375 0.0
0 0.0112186670303345 1.0
0 0.0607815980911255 2.0
0 -0.0222302675247192 3.0
0 0.0338104367256165 4.0
0 -0.0278091430664062 5.0
0 0.00617092847824097 6.0
0 -0.0118278861045837 7.0
46.3288844 39.1624336242676 0.0
0 0.00678884983062744 1.0
0 0.0581302642822266 2.0
0 -0.0223013758659363 3.0
0 0.0426577925682068 4.0
0 -0.0176820755004883 5.0
0 0.00250375270843506 6.0
0 -0.0138975381851196 7.0
39.359671461 39.2167892456055 0.0
0 0.0114791989326477 1.0
0 0.0373216867446899 2.0
0 -0.0210093259811401 3.0
0 0.0348582863807678 4.0
0 -0.0306043028831482 5.0
0 0.00945502519607544 6.0
0 -0.0132561922073364 7.0
43.813870988 39.6971778869629 0.0
0 0.00339007377624512 1.0
0 0.0766655206680298 2.0
0 -0.0190767049789429 3.0
0 0.0360779166221619 4.0
0 -0.0164307951927185 5.0
0 0.00113844871520996 6.0
0 -0.0129740834236145 7.0
39.146221876 38.7799415588379 0.0
0 0.0131780505180359 1.0
0 0.0397027730941772 2.0
0 -0.0297419428825378 3.0
0 0.0398967862129211 4.0
0 -0.0307811498641968 5.0
0 0.00434565544128418 6.0
0 -0.0123134255409241 7.0
34.753353014 39.8708610534668 0.0
0 -0.000608384609222412 1.0
0 0.115673184394836 2.0
0 -0.00372225046157837 3.0
0 0.0298271775245667 4.0
0 -0.0289943218231201 5.0
0 0.00512754917144775 6.0
0 -0.0145651698112488 7.0
36.858915008 39.2941970825195 0.0
0 0.00608658790588379 1.0
0 0.0364677906036377 2.0
0 -0.0228100419044495 3.0
0 0.0330633521080017 4.0
0 -0.0306199789047241 5.0
0 0.0058516263961792 6.0
0 -0.0124207735061646 7.0
45.609388725 39.0649108886719 0.0
0 0.0138698220252991 1.0
0 0.0337219834327698 2.0
0 -0.0362816452980042 3.0
0 0.0378729701042175 4.0
0 -0.0258591175079346 5.0
0 0.00363445281982422 6.0
0 -0.0116531848907471 7.0
37.221502314 39.0854949951172 0.0
0 0.00831472873687744 1.0
0 0.0745607614517212 2.0
0 -0.0210608243942261 3.0
0 0.0341051816940308 4.0
0 -0.0302374362945557 5.0
0 0.00336241722106934 6.0
0 -0.0136440992355347 7.0
39.351241848 39.079216003418 0.0
0 0.0129474401473999 1.0
0 0.0575575828552246 2.0
0 -0.0210944414138794 3.0
0 0.0328056812286377 4.0
0 -0.0277805328369141 5.0
0 0.00349217653274536 6.0
0 -0.0115208029747009 7.0
44.354286139 39.4327774047852 0.0
0 0.00980257987976074 1.0
0 0.0603073239326477 2.0
0 -0.0212836861610413 3.0
0 0.0488134622573853 4.0
0 -0.0223110914230347 5.0
0 0.00612592697143555 6.0
0 -0.0144089460372925 7.0
41.143843969 39.2214050292969 0.0
0 0.00443631410598755 1.0
0 0.0381275415420532 2.0
0 -0.0224068760871887 3.0
0 0.0291399955749512 4.0
0 -0.027215301990509 5.0
0 0.00449514389038086 6.0
0 -0.0142806768417358 7.0
32.436363672 40.0223999023438 0.0
0 0.00905328989028931 1.0
0 0.0224776268005371 2.0
0 -0.0241499543190002 3.0
0 0.00383108854293823 4.0
0 -0.031097412109375 5.0
0 0.00539493560791016 6.0
0 -0.00590020418167114 7.0
35.980268826 39.2067642211914 0.0
0 0.00584536790847778 1.0
0 0.101477026939392 2.0
0 -0.0109339952468872 3.0
0 0.0248546004295349 4.0
0 -0.0253920555114746 5.0
0 0.0057300329208374 6.0
0 -0.0112724304199219 7.0
39.361696951 39.1074066162109 0.0
0 8.82148742675781e-06 1.0
0 0.0373160839080811 2.0
0 -0.0219159722328186 3.0
0 0.00553721189498901 4.0
0 -0.0202788710594177 5.0
0 0.0008392333984375 6.0
0 -0.00792026519775391 7.0
38.339957501 39.3969459533691 0.0
0 0.00817453861236572 1.0
0 0.0689851045608521 2.0
0 -0.0219828486442566 3.0
0 0.0383086204528809 4.0
0 -0.0291087031364441 5.0
0 0.00392496585845947 6.0
0 -0.0140419602394104 7.0
38.437078026 39.1383895874023 0.0
0 0.00185298919677734 1.0
0 0.086402416229248 2.0
0 -0.015372633934021 3.0
0 0.0599687695503235 4.0
0 -0.0258504748344421 5.0
0 0.00392729043960571 6.0
0 -0.0180753469467163 7.0
31.680301161 39.6670150756836 0.0
0 0.00319278240203857 1.0
0 0.0746790766716003 2.0
0 -0.00584864616394043 3.0
0 -0.00392329692840576 4.0
0 -0.0279016494750977 5.0
0 0.00421690940856934 6.0
0 -0.00840109586715698 7.0
36.698769132 39.5708198547363 0.0
0 0.0115896463394165 1.0
0 0.0365105271339417 2.0
0 -0.0163916349411011 3.0
0 0.00393229722976685 4.0
0 -0.0225992798805237 5.0
0 0.002402663230896 6.0
0 -0.00545710325241089 7.0
40.942334922 39.1316337585449 0.0
0 0.0144314765930176 1.0
0 0.0406362414360046 2.0
0 -0.0247728824615479 3.0
0 0.0245896577835083 4.0
0 -0.0239621996879578 5.0
0 0.00334888696670532 6.0
0 -0.0101450085639954 7.0
41.496981537 39.2279319763184 0.0
0 0.00682604312896729 1.0
0 0.0635049939155579 2.0
0 -0.0187178254127502 3.0
0 0.0420576930046082 4.0
0 -0.0189405679702759 5.0
0 0.00453871488571167 6.0
0 -0.015053927898407 7.0
33.577466065 37.6259269714355 0.0
0 0.00761991739273071 1.0
0 0.0568365454673767 2.0
0 -0.0344957709312439 3.0
0 0.0208529233932495 4.0
0 -0.0321889519691467 5.0
0 0.00291848182678223 6.0
0 -0.00838345289230347 7.0
38.410697595 39.4420318603516 0.0
0 0.0113469362258911 1.0
0 0.0686453580856323 2.0
0 -0.0142496228218079 3.0
0 0.00417876243591309 4.0
0 -0.0155113935470581 5.0
0 -0.000855326652526855 6.0
0 -0.00508135557174683 7.0
40.390522112 39.2720718383789 0.0
0 0.000581443309783936 1.0
0 0.0393170714378357 2.0
0 -0.0221692323684692 3.0
0 0.0482469201087952 4.0
0 -0.0264850854873657 5.0
0 0.00661689043045044 6.0
0 -0.0178845524787903 7.0
38.02724699 38.9787788391113 0.0
0 0.00272881984710693 1.0
0 0.0378914475440979 2.0
0 -0.0236902236938477 3.0
0 0.00943964719772339 4.0
0 -0.0208128094673157 5.0
0 0.00278943777084351 6.0
0 -0.00711840391159058 7.0
34.831322295 38.5845375061035 0.0
0 0.00436198711395264 1.0
0 0.0335705280303955 2.0
0 -0.0219966173171997 3.0
0 0.00454270839691162 4.0
0 -0.0286591649055481 5.0
0 0.00389528274536133 6.0
0 -0.008575439453125 7.0
33.152476832 39.2792053222656 0.0
0 0.000803530216217041 1.0
0 0.0964491367340088 2.0
0 -0.00595211982727051 3.0
0 -0.00248456001281738 4.0
0 -0.0183451175689697 5.0
0 0.000454962253570557 6.0
0 -0.00583839416503906 7.0
37.76181476 39.1988182067871 0.0
0 0.00701266527175903 1.0
0 0.0517593026161194 2.0
0 -0.0180467367172241 3.0
0 0.0260825753211975 4.0
0 -0.0297691822052002 5.0
0 0.00736707448959351 6.0
0 -0.0125783681869507 7.0
46.821650967 39.1685638427734 0.0
0 0.0124068260192871 1.0
0 0.0685530304908752 2.0
0 -0.0276700854301453 3.0
0 0.0559806823730469 4.0
0 -0.0227029323577881 5.0
0 0.00737607479095459 6.0
0 -0.0143417716026306 7.0
41.183790175 39.3380546569824 0.0
0 0.00915086269378662 1.0
0 0.043701171875 2.0
0 -0.0248683094978333 3.0
0 0.0265670418739319 4.0
0 -0.0222071409225464 5.0
0 0.00195217132568359 6.0
0 -0.0123401284217834 7.0
36.78498864 39.2007217407227 0.0
0 0.0107810497283936 1.0
0 0.0372036695480347 2.0
0 -0.0210047960281372 3.0
0 0.0306706428527832 4.0
0 -0.0301961898803711 5.0
0 0.00700825452804565 6.0
0 -0.0121517777442932 7.0
37.689494945 40.0001411437988 0.0
0 0.00271826982498169 1.0
0 0.0959305763244629 2.0
0 -0.0141395926475525 3.0
0 0.0551460385322571 4.0
0 -0.0244983434677124 5.0
0 0.00532084703445435 6.0
0 -0.0170699954032898 7.0
41.308881097 39.8917083740234 0.0
0 0.013401985168457 1.0
0 0.0613721013069153 2.0
0 -0.0210734605789185 3.0
0 0.0379018187522888 4.0
0 -0.0249822735786438 5.0
0 0.00856763124465942 6.0
0 -0.0117775201797485 7.0
42.253011184 38.8202743530273 0.0
0 0.00312644243240356 1.0
0 0.0345909595489502 2.0
0 -0.0317898988723755 3.0
0 0.0441475510597229 4.0
0 -0.0303978323936462 5.0
0 0.00695013999938965 6.0
0 -0.0131359100341797 7.0
40.047660493 41.552059173584 0.0
0 0.00905096530914307 1.0
0 0.101017355918884 2.0
0 -0.0131558179855347 3.0
0 0.0354183316230774 4.0
0 -0.018403947353363 5.0
0 0.00396114587783813 6.0
0 -0.0143160223960876 7.0
46.604605733 39.235408782959 0.0
0 0.00629884004592896 1.0
0 0.105640709400177 2.0
0 -0.015018105506897 3.0
0 0.0429240465164185 4.0
0 -0.0164434313774109 5.0
0 0.00407266616821289 6.0
0 -0.0142799615859985 7.0
45.106295542 39.4154434204102 0.0
0 0.00659686326980591 1.0
0 0.0552960634231567 2.0
0 -0.0203244686126709 3.0
0 0.0602788925170898 4.0
0 -0.0267068147659302 5.0
0 0.00430470705032349 6.0
0 -0.018653392791748 7.0
38.977055099 39.2278289794922 0.0
0 0.00663143396377563 1.0
0 0.0948184132575989 2.0
0 -0.0122199654579163 3.0
0 0.0315920114517212 4.0
0 -0.0249831080436707 5.0
0 0.00514775514602661 6.0
0 -0.0134918689727783 7.0
38.935365883 39.1499710083008 0.0
0 0.00999718904495239 1.0
0 0.0386626124382019 2.0
0 -0.0293843746185303 3.0
0 0.0103581547737122 4.0
0 -0.0220971703529358 5.0
0 0.00317615270614624 6.0
0 -0.00762468576431274 7.0
42.741409918 39.2675628662109 0.0
0 0.00993281602859497 1.0
0 0.0399733185768127 2.0
0 -0.031740128993988 3.0
0 0.0284399390220642 4.0
0 -0.0234741568565369 5.0
0 0.00261366367340088 6.0
0 -0.00984954833984375 7.0
31.310794576 39.3400192260742 0.0
0 0.0062834620475769 1.0
0 0.038298487663269 2.0
0 -0.0212872624397278 3.0
0 0.00529992580413818 4.0
0 -0.0331861972808838 5.0
0 0.00680768489837646 6.0
0 -0.00564706325531006 7.0
};
\addlegendentry{$R^2$=0.983}
\end{axis}

\end{tikzpicture}
}}
        \subfloat [Actual vs predicted edge flows.] {\label{fig:results_dummy_edge_base_f}\resizebox{\figurewidth}{\figureheight}{% This file was created with tikzplotlib v0.10.1.
\begin{tikzpicture}

\definecolor{darkgray176}{RGB}{176,176,176}
\definecolor{lightgray204}{RGB}{204,204,204}

\begin{axis}[
colorbar,
colorbar style={ylabel={edge_id}},
colormap={mymap}{[1pt]
 rgb(0pt)=(0.12156862745098,0.466666666666667,0.705882352941177);
  rgb(1pt)=(1,0.498039215686275,0.0549019607843137);
  rgb(2pt)=(0.172549019607843,0.627450980392157,0.172549019607843);
  rgb(3pt)=(0.83921568627451,0.152941176470588,0.156862745098039);
  rgb(4pt)=(0.580392156862745,0.403921568627451,0.741176470588235);
  rgb(5pt)=(0.549019607843137,0.337254901960784,0.294117647058824);
  rgb(6pt)=(0.890196078431372,0.466666666666667,0.76078431372549);
  rgb(7pt)=(0.498039215686275,0.498039215686275,0.498039215686275);
  rgb(8pt)=(0.737254901960784,0.741176470588235,0.133333333333333);
  rgb(9pt)=(0.0901960784313725,0.745098039215686,0.811764705882353)
},
legend cell align={left},
legend style={
  fill opacity=0.8,
  draw opacity=1,
  text opacity=1,
  at={(0.03,0.97)},
  anchor=north west,
  draw=lightgray204
},
point meta max=7,
point meta min=0,
tick align=outside,
tick pos=left,
title={ye_test-ye_pred},
x grid style={darkgray176},
xlabel={ye_test},
xmajorgrids,
xmin=-15.25776872815, xmax=51.84084685915,
xtick style={color=black},
y grid style={darkgray176},
ylabel={ye_pred},
ymajorgrids,
ymin=-14.0268879413605, ymax=52.5269442081451,
ytick style={color=black}
]
\addplot [
  colormap={mymap}{[1pt]
 rgb(0pt)=(0.12156862745098,0.466666666666667,0.705882352941177);
  rgb(1pt)=(1,0.498039215686275,0.0549019607843137);
  rgb(2pt)=(0.172549019607843,0.627450980392157,0.172549019607843);
  rgb(3pt)=(0.83921568627451,0.152941176470588,0.156862745098039);
  rgb(4pt)=(0.580392156862745,0.403921568627451,0.741176470588235);
  rgb(5pt)=(0.549019607843137,0.337254901960784,0.294117647058824);
  rgb(6pt)=(0.890196078431372,0.466666666666667,0.76078431372549);
  rgb(7pt)=(0.498039215686275,0.498039215686275,0.498039215686275);
  rgb(8pt)=(0.737254901960784,0.741176470588235,0.133333333333333);
  rgb(9pt)=(0.0901960784313725,0.745098039215686,0.811764705882353)
},
  only marks,
  scatter,
  scatter src=explicit
]
table [x=x, y=y, meta=colordata]{%
x  y  colordata
39.565898635 38.596305847168 0.0
18.050517226 19.9765090942383 1.0
21.515381409 19.5053291320801 2.0
-8.8987516789 -7.5859842300415 3.0
12.61662972 11.9246826171875 4.0
26.949268915 26.8885326385498 5.0
39.565898645 40.6012878417969 6.0
39.565898645 39.244514465332 7.0
42.743257171 38.6418075561523 0.0
22.448801828 20.5235271453857 1.0
20.294455342 20.9817905426025 2.0
-4.2455956832 -6.40148591995239 3.0
16.048859649 15.7577791213989 4.0
26.694397521 26.0581035614014 5.0
42.743257181 41.7566375732422 6.0
42.743257181 41.8745651245117 7.0
39.367180747 38.091423034668 0.0
18.970258592 19.4065704345703 1.0
20.39692216 19.8818187713623 2.0
-4.9776642323 -5.26164960861206 3.0
15.419257923 14.7249002456665 4.0
23.947922829 23.1789817810059 5.0
39.367180751 38.5191268920898 6.0
39.367180751 39.1420822143555 7.0
39.605393097 37.8412399291992 0.0
22.082745928 20.4279651641846 1.0
17.522647169 18.7483291625977 2.0
-7.2517903354 -9.21843147277832 3.0
10.270856823 9.68563938140869 4.0
29.334536274 28.5466556549072 5.0
39.605393107 39.5127182006836 6.0
39.605393107 40.0177230834961 7.0
43.937345526 38.6058807373047 0.0
21.865843593 21.8540668487549 1.0
22.071501933 20.891975402832 2.0
-8.3469765086 -7.86768054962158 3.0
13.724525414 13.045955657959 4.0
30.212820112 28.7595252990723 5.0
43.937345536 43.1537170410156 6.0
43.937345536 43.2702102661133 7.0
31.061989584 37.880500793457 0.0
16.203677385 15.8516836166382 1.0
14.858312199 15.0089178085327 2.0
-5.5704045507 -6.62269258499146 3.0
9.2879076384 9.24989891052246 4.0
21.774081945 20.9252452850342 5.0
31.061989594 31.7036800384521 6.0
31.061989594 31.6152458190918 7.0
36.357435265 40.6078720092773 0.0
19.38173709 18.5710964202881 1.0
16.975698176 16.8981056213379 2.0
-8.3424496148 -9.10635566711426 3.0
8.6332485532 8.03556060791016 4.0
27.724186714 27.9642848968506 5.0
36.357435273 35.7285537719727 6.0
36.357435273 35.4694442749023 7.0
38.444969607 38.5469589233398 0.0
21.080570389 19.3154773712158 1.0
17.364399218 18.5943260192871 2.0
-5.0980379196 -7.51685047149658 3.0
12.266361289 11.6779813766479 4.0
26.178608318 25.8334617614746 5.0
38.444969617 38.1791000366211 6.0
38.444969617 38.3308563232422 7.0
35.498620518 38.3295974731445 0.0
17.900031823 17.3579730987549 1.0
17.598588698 18.0279216766357 2.0
-4.651704216 -6.18915939331055 3.0
12.946884476 12.3227348327637 4.0
22.551736046 22.3548603057861 5.0
35.498620524 35.0162658691406 6.0
35.498620524 35.7486877441406 7.0
36.52099827 39.1421966552734 0.0
18.961240079 18.6394557952881 1.0
17.559758192 18.268310546875 2.0
-5.3314768445 -6.87822914123535 3.0
12.228281339 11.798867225647 4.0
24.292716932 24.457181930542 5.0
36.520998279 36.9719772338867 6.0
36.520998279 36.2696228027344 7.0
36.717272204 38.9959335327148 0.0
19.763035348 19.2362728118896 1.0
16.954236857 18.1576652526855 2.0
-7.0689091682 -8.69227981567383 3.0
9.88532768 9.23490810394287 4.0
26.831944525 26.86110496521 5.0
36.717272212 37.3186569213867 6.0
36.717272212 37.7003936767578 7.0
32.629628996 38.8705673217773 0.0
17.103127104 16.7790718078613 1.0
15.526501892 15.2252159118652 2.0
-7.1999965109 -8.01188087463379 3.0
8.3265053708 8.09377098083496 4.0
24.303123625 24.2568454742432 5.0
32.629629006 33.6453323364258 6.0
32.629629006 33.813346862793 7.0
37.75267433 38.2975540161133 0.0
17.533699782 18.3296546936035 1.0
20.218974548 19.5139389038086 2.0
-3.919745576 -4.89840984344482 3.0
16.299228962 15.4050731658936 4.0
21.453445368 20.1247844696045 5.0
37.75267434 37.2333374023438 6.0
37.75267434 37.7951812744141 7.0
38.800291337 38.8440856933594 0.0
21.414385846 19.40407371521 1.0
17.385905491 18.7268810272217 2.0
-6.3164463541 -8.4114408493042 3.0
11.069459127 10.6674652099609 4.0
27.73083221 27.2714900970459 5.0
38.800291347 37.6164703369141 6.0
38.800291347 38.5971755981445 7.0
38.252729609 38.4748001098633 0.0
20.199036122 19.4157733917236 1.0
18.053693488 18.6936721801758 2.0
-6.9495443437 -8.02486419677734 3.0
11.104149136 10.5527362823486 4.0
27.148580475 27.2325859069824 5.0
38.252729618 38.1705474853516 6.0
38.252729618 38.0124740600586 7.0
43.273596037 40.0211029052734 0.0
20.811072897 22.2805271148682 1.0
22.46252314 20.9060554504395 2.0
-10.185324283 -9.22338104248047 3.0
12.277198847 11.4580049514771 4.0
30.99639719 30.8943367004395 5.0
43.273596047 43.1003189086914 6.0
43.273596047 43.7081604003906 7.0
34.027431477 37.5528106689453 0.0
16.463058834 17.5791893005371 1.0
17.564372643 16.3833618164062 2.0
-7.8238749469 -7.02876377105713 3.0
9.7404976863 9.56681632995605 4.0
24.28693379 23.8013954162598 5.0
34.027431486 34.5615615844727 6.0
34.027431486 33.9703750610352 7.0
41.154171382 38.8431701660156 0.0
20.969763 20.984806060791 1.0
20.184408383 19.1587142944336 2.0
-9.8031418559 -9.94899749755859 3.0
10.381266519 9.61151504516602 4.0
30.772904865 29.7687873840332 5.0
41.154171391 41.1792373657227 6.0
41.154171391 40.7767715454102 7.0
39.9308264078757 38.3818206787109 0.0
21.2554356118761 20.7513122558594 1.0
18.6753908028755 19.0751094818115 2.0
-8.41978314337461 -9.43831157684326 3.0
10.2556076588756 9.64951229095459 4.0
29.6752187558758 28.8787422180176 5.0
39.930826408 39.7014236450195 6.0
39.930826408 40.0568618774414 7.0
45.969703622 38.6865692138672 0.0
25.855804141 22.8934497833252 1.0
20.113899483 22.2165870666504 2.0
-4.729541314 -7.56897640228271 3.0
15.38435816 14.5867805480957 4.0
30.585345464 29.0078372955322 5.0
45.969703631 44.1232299804688 6.0
45.969703631 45.2887420654297 7.0
38.398120212 38.6374282836914 0.0
19.546425793 19.6057834625244 1.0
18.851694421 18.1635780334473 2.0
-9.8922695857 -9.75459480285645 3.0
8.9594248259 8.33925437927246 4.0
29.438695387 28.486421585083 5.0
38.398120221 38.1621627807617 6.0
38.398120221 38.5771560668945 7.0
28.366017296 39.6057739257812 0.0
14.995271371 14.1404008865356 1.0
13.370745924 13.0689744949341 2.0
-4.4280592982 -6.7935733795166 3.0
8.9426866162 8.9682502746582 4.0
19.42333068 19.1228618621826 5.0
28.366017306 29.9161052703857 6.0
28.366017306 28.6994247436523 7.0
39.07981897 37.8760223388672 0.0
18.202914308 19.484001159668 1.0
20.876904662 19.5376033782959 2.0
-6.6281228283 -6.22390174865723 3.0
14.248781824 13.701379776001 4.0
24.831037146 24.1412754058838 5.0
39.079818979 38.8075942993164 6.0
39.079818979 39.393684387207 7.0
40.466329374 38.3118438720703 0.0
19.842708002 19.766731262207 1.0
20.623621372 19.9568691253662 2.0
-6.0436252005 -6.6066198348999 3.0
14.579996162 13.9918174743652 4.0
25.886333212 24.9481201171875 5.0
40.466329383 39.8389892578125 6.0
40.466329383 40.9878768920898 7.0
38.293116843 37.9650497436523 0.0
18.225197444 18.1354637145996 1.0
20.067919399 19.4418029785156 2.0
-4.6459628917 -5.61400127410889 3.0
15.421956498 15.0479040145874 4.0
22.871160346 22.6656131744385 5.0
38.293116853 37.6979522705078 6.0
38.293116853 37.6028518676758 7.0
43.346089487 38.575080871582 0.0
21.23189324 21.8341217041016 1.0
22.114196248 20.7341327667236 2.0
-9.387268107 -8.84220218658447 3.0
12.726928131 11.9266443252563 4.0
30.619161357 29.7657775878906 5.0
43.346089497 42.8919219970703 6.0
43.346089497 43.2442321777344 7.0
34.547871144 38.1383056640625 0.0
19.900789633 18.2591228485107 1.0
14.647081512 16.0518455505371 2.0
-6.6924528176 -8.45735645294189 3.0
7.9546286844 7.69030857086182 4.0
26.59324246 26.038480758667 5.0
34.547871154 34.2125091552734 6.0
34.547871154 34.9699249267578 7.0
41.927050143208 38.3602600097656 0.0
21.8340017142739 21.933032989502 1.0
20.0930484299516 19.568078994751 2.0
-12.0156472013624 -10.9687099456787 3.0
8.07740122783537 8.01268196105957 4.0
33.8496486756856 32.4716110229492 5.0
41.927050143 41.8528671264648 6.0
41.927050143 42.3923263549805 7.0
44.548236367 38.2497406005859 0.0
22.097953544 21.9802989959717 1.0
22.450282823 21.7488422393799 2.0
-6.9116853021 -6.82119369506836 3.0
15.538597511 15.3108968734741 4.0
29.009638856 27.8976154327393 5.0
44.548236377 43.4468307495117 6.0
44.548236377 43.7278213500977 7.0
34.958415477 39.1437835693359 0.0
19.046265323 17.0440578460693 1.0
15.912150156 17.5630054473877 2.0
-3.947719982 -6.61912107467651 3.0
11.964430164 11.5081748962402 4.0
22.993985314 22.4576072692871 5.0
34.958415487 34.7578430175781 6.0
34.958415487 35.6662673950195 7.0
41.619773288 38.0942611694336 0.0
18.036088777 20.262638092041 1.0
23.583684511 20.2230663299561 2.0
-9.1140217471 -7.23081398010254 3.0
14.469662754 13.9196548461914 4.0
27.150110534 27.2458477020264 5.0
41.619773298 39.2246780395508 6.0
41.619773298 40.2623062133789 7.0
35.768623904 38.4565124511719 0.0
18.854929214 18.1294422149658 1.0
16.913694692 17.2556228637695 2.0
-6.8203969536 -8.47326946258545 3.0
10.093297731 9.59994029998779 4.0
25.675326176 25.5502910614014 5.0
35.768623912 36.0705490112305 6.0
35.768623912 36.3270874023438 7.0
36.035873071 36.6593170166016 0.0
18.197202866 18.8770313262939 1.0
17.838670207 17.1606254577637 2.0
-9.8802007315 -9.52300262451172 3.0
7.958469467 7.84091186523438 4.0
28.077403606 28.0217590332031 5.0
36.03587308 35.13916015625 6.0
36.03587308 36.5958099365234 7.0
42.671159575 38.0124359130859 0.0
22.542382477 21.8425483703613 1.0
20.128777099 20.0145320892334 2.0
-9.3171954196 -10.1334800720215 3.0
10.811581671 10.0967769622803 4.0
31.859577906 31.1639347076416 5.0
42.671159584 41.8028335571289 6.0
42.671159584 42.3421096801758 7.0
32.270518381 38.76953125 0.0
16.341237488 17.2153034210205 1.0
15.929280893 15.8293142318726 2.0
-7.2560391874 -7.5936803817749 3.0
8.6732416955 8.44317626953125 4.0
23.597276685 23.5255069732666 5.0
32.270518391 32.6548080444336 6.0
32.270518391 33.550910949707 7.0
36.239834932 41.2490692138672 0.0
18.511196273 18.1738815307617 1.0
17.72863866 18.0583553314209 2.0
-5.3875241311 -7.16932010650635 3.0
12.34111452 11.7859210968018 4.0
23.898720413 23.8287391662598 5.0
36.239834941 35.6087036132812 6.0
36.239834941 35.4636993408203 7.0
38.292005172 37.9891510009766 0.0
20.144840979 19.4964828491211 1.0
18.147164195 18.4614276885986 2.0
-6.9715526507 -7.96281623840332 3.0
11.175611535 10.6878089904785 4.0
27.116393638 26.6137714385986 5.0
38.292005181 38.5519485473633 6.0
38.292005181 39.5151824951172 7.0
41.142171116 39.3999252319336 0.0
21.697070846 20.3426818847656 1.0
19.44510027 20.4383220672607 2.0
-5.49061917 -7.43276405334473 3.0
13.95448109 13.2350759506226 4.0
27.187690026 26.9425754547119 5.0
41.142171126 40.9181289672852 6.0
41.142171126 41.5554046630859 7.0
30.660234741 37.4738845825195 0.0
16.571808951 15.7378721237183 1.0
14.08842579 14.9877004623413 2.0
-3.6405774968 -5.70759725570679 3.0
10.447848283 10.1889791488647 4.0
20.212386458 19.7445640563965 5.0
30.660234751 32.1174278259277 6.0
30.660234751 32.041820526123 7.0
42.776716976 39.9827499389648 0.0
18.943122536 21.1849422454834 1.0
23.83359444 20.9559173583984 2.0
-9.5992659914 -7.79702854156494 3.0
14.234328438 13.7019624710083 4.0
28.542388537 28.1028633117676 5.0
42.776716986 41.7659683227539 6.0
42.776716986 42.5332946777344 7.0
39.136657948 38.8526840209961 0.0
21.244730642 19.9374752044678 1.0
17.891927309 19.648380279541 2.0
-5.2069844714 -7.2206449508667 3.0
12.684942831 11.9592514038086 4.0
26.45171512 26.3674869537354 5.0
39.136657955 37.942253112793 6.0
39.136657955 38.3528442382812 7.0
40.593075656 38.2235794067383 0.0
19.181149666 21.0331974029541 1.0
21.411925992 19.1207962036133 2.0
-11.788434371 -9.97361087799072 3.0
9.6234916129 9.05599689483643 4.0
30.969584045 30.2422485351562 5.0
40.593075664 39.523551940918 6.0
40.593075664 40.1297454833984 7.0
38.455960047 38.7767868041992 0.0
19.606731652 20.1121311187744 1.0
18.849228395 18.2991676330566 2.0
-10.066988594 -9.78773784637451 3.0
8.7822397903 8.13940715789795 4.0
29.673720256 29.5408210754395 5.0
38.455960057 38.0539016723633 6.0
38.455960057 38.3448028564453 7.0
40.029822299 38.7315826416016 0.0
19.696760735 20.0480403900146 1.0
20.333061566 19.6599903106689 2.0
-6.9313004298 -7.38136005401611 3.0
13.401761128 12.8033199310303 4.0
26.628061173 25.8108005523682 5.0
40.029822307 39.3719940185547 6.0
40.029822307 40.4371185302734 7.0
39.721089796 38.3653564453125 0.0
21.982867862 20.3135223388672 1.0
17.738221933 19.1263008117676 2.0
-6.2167396454 -8.72383594512939 3.0
11.521482278 11.0404500961304 4.0
28.199607518 27.3751811981201 5.0
39.721089806 40.900260925293 6.0
39.721089806 39.7707290649414 7.0
46.012802771 38.5219421386719 0.0
23.852238991 22.91357421875 1.0
22.16056378 21.7913818359375 2.0
-9.7843125098 -10.1084747314453 3.0
12.37625126 11.3134632110596 4.0
33.636551511 32.5210723876953 5.0
46.012802781 45.08203125 6.0
46.012802781 46.1660232543945 7.0
43.79104115 38.6123657226562 0.0
23.810489927 22.027738571167 1.0
19.980551224 20.673770904541 2.0
-8.8972279263 -10.03688621521 3.0
11.083323289 9.92902183532715 4.0
32.707717862 31.3464870452881 5.0
43.791041158 42.2916946411133 6.0
43.791041158 43.6183166503906 7.0
31.257332414 38.2450332641602 0.0
14.677686479 15.6956052780151 1.0
16.579645934 14.9001684188843 2.0
-7.7653359569 -7.08528423309326 3.0
8.8143099674 8.48609733581543 4.0
22.443022446 22.779369354248 5.0
31.257332424 30.916130065918 6.0
31.257332424 30.941858291626 7.0
38.988472902 38.1238403320312 0.0
18.706030748 19.4086837768555 1.0
20.282442155 19.8146343231201 2.0
-4.4931154738 -4.96176481246948 3.0
15.789326673 14.8181753158569 4.0
23.19914623 22.3707904815674 5.0
38.98847291 39.0290298461914 6.0
38.98847291 39.5676727294922 7.0
38.691218489 38.2635803222656 0.0
19.181746937 18.3563499450684 1.0
19.509471553 18.8885822296143 2.0
-4.305791265 -6.0208797454834 3.0
15.203680278 14.929160118103 4.0
23.487538212 22.4304141998291 5.0
38.691218499 37.3509140014648 6.0
38.691218499 38.6561508178711 7.0
39.033211962 37.2093505859375 0.0
21.751284426 20.4594287872314 1.0
17.281927536 18.6953716278076 2.0
-7.1228573229 -9.43244361877441 3.0
10.159070204 9.23776054382324 4.0
28.874141758 28.742280960083 5.0
39.033211971 38.8525695800781 6.0
39.033211971 39.3592529296875 7.0
37.697547803 38.6268463134766 0.0
18.641388049 18.8006534576416 1.0
19.056159754 18.699592590332 2.0
-6.2754402376 -6.98374080657959 3.0
12.780719506 12.4110231399536 4.0
24.916828297 24.5393657684326 5.0
37.697547813 36.8487167358398 6.0
37.697547813 37.9475479125977 7.0
35.277541329 38.1812057495117 0.0
16.862419132 17.8773555755615 1.0
18.415122198 17.9661350250244 2.0
-6.2131742864 -6.36807012557983 3.0
12.201947903 11.7862014770508 4.0
23.075593428 23.2379150390625 5.0
35.277541339 34.7342910766602 6.0
35.277541339 36.3307418823242 7.0
36.1197639664031 38.3135223388672 0.0
19.1517629624013 17.7718601226807 1.0
16.9680010143999 18.865629196167 2.0
-0.473187581125011 -3.97485685348511 3.0
16.4948134334014 14.9833126068115 4.0
19.6249505434022 19.2428321838379 5.0
36.119763966 36.810417175293 6.0
36.119763966 37.0653457641602 7.0
33.14490304 38.2044525146484 0.0
16.831426445 16.1671714782715 1.0
16.313476595 16.9022331237793 2.0
-3.637680404 -4.73959493637085 3.0
12.675796181 12.6736574172974 4.0
20.469106859 20.0596942901611 5.0
33.14490305 33.2030410766602 6.0
33.14490305 32.5096092224121 7.0
34.4868008142505 38.2473602294922 0.0
18.3228058422505 18.7227478027344 1.0
16.1639949822505 16.799503326416 2.0
-8.00476657755046 -8.97555541992188 3.0
8.15922840445046 7.85224437713623 4.0
26.3275723771958 27.5033168792725 5.0
34.486800814 34.5710830688477 6.0
34.486800814 35.8238296508789 7.0
40.933468199 38.9690628051758 0.0
20.889274427 20.8757381439209 1.0
20.044193774 20.2090339660645 2.0
-6.533500242 -7.47461318969727 3.0
13.510693524 12.9050893783569 4.0
27.422774677 27.0413875579834 5.0
40.933468207 41.4710693359375 6.0
40.933468207 40.4933700561523 7.0
35.993417919 38.3794860839844 0.0
18.964188614 18.5946922302246 1.0
17.029229306 17.2728900909424 2.0
-7.4575389686 -8.26856327056885 3.0
9.5716903284 9.30193614959717 4.0
26.421727592 26.5541229248047 5.0
35.993417928 36.0943832397461 6.0
35.993417928 36.9887390136719 7.0
39.331666914 38.3532333374023 0.0
19.408314017 19.5726890563965 1.0
19.923352898 19.2500877380371 2.0
-7.2428639103 -7.33345890045166 3.0
12.680488979 12.0278520584106 4.0
26.651177937 25.8608589172363 5.0
39.331666923 39.1381759643555 6.0
39.331666923 39.3521347045898 7.0
37.559548486 38.3558502197266 0.0
18.054642609 19.0616474151611 1.0
19.504905877 18.4792575836182 2.0
-8.2806473226 -8.43041896820068 3.0
11.224258545 10.6311960220337 4.0
26.335289942 26.4604682922363 5.0
37.559548496 37.2250823974609 6.0
37.559548496 37.6933441162109 7.0
41.796902472 38.2650909423828 0.0
20.074580702 21.5512237548828 1.0
21.72232177 19.6002368927002 2.0
-11.24258785 -9.91202449798584 3.0
10.479733911 9.6037425994873 4.0
31.317168562 30.1995830535889 5.0
41.796902482 41.5065841674805 6.0
41.796902482 41.99755859375 7.0
35.679590813 38.4417953491211 0.0
15.932888769 17.7633152008057 1.0
19.746702044 17.7697124481201 2.0
-7.4161334139 -6.28019094467163 3.0
12.330568621 12.0021047592163 4.0
23.349022193 23.0090293884277 5.0
35.679590823 35.9085006713867 6.0
35.679590823 36.6312637329102 7.0
33.227547284 37.9404754638672 0.0
19.156810973 17.3763046264648 1.0
14.070736312 15.8583564758301 2.0
-5.2205843651 -6.91761207580566 3.0
8.8501519384 8.57953262329102 4.0
24.377395346 24.0266704559326 5.0
33.227547292 32.8059196472168 6.0
33.227547292 33.2664070129395 7.0
28.00807173 38.808479309082 0.0
12.909757219 14.271240234375 1.0
15.098314511 14.2345685958862 2.0
-4.3989930962 -3.41055345535278 3.0
10.699321405 10.7542247772217 4.0
17.308750325 17.332633972168 5.0
28.008071739 29.8767280578613 6.0
28.008071739 29.2562599182129 7.0
33.4784988410636 39.5744171142578 0.0
17.4596045520637 17.5532684326172 1.0
16.0188942990637 16.7638511657715 2.0
-6.65519022176347 -7.79201221466064 3.0
9.36370407706344 8.58208847045898 4.0
24.1147947730637 23.8467273712158 5.0
33.478498841 34.1995620727539 6.0
33.478498841 32.6516494750977 7.0
33.1229468201162 38.7150039672852 0.0
16.2738036511163 16.6416358947754 1.0
16.8491431771162 15.7469615936279 2.0
-8.48238354211593 -8.02114486694336 3.0
8.36675963501607 8.30586051940918 4.0
24.7561871931169 24.5005741119385 5.0
33.12294682 33.6349487304688 6.0
33.12294682 33.6500625610352 7.0
34.970228375 38.9385833740234 0.0
17.461716438 17.4915294647217 1.0
17.508511937 18.2084255218506 2.0
-3.5027478521 -4.3347749710083 3.0
14.005764076 13.2965602874756 4.0
20.964464299 20.1009483337402 5.0
34.970228384 35.6356582641602 6.0
34.970228384 35.9989166259766 7.0
36.637966033 38.3001022338867 0.0
18.611959282 17.4169902801514 1.0
18.026006753 17.929515838623 2.0
-5.1789935919 -6.43232107162476 3.0
12.847013153 12.1779184341431 4.0
23.790952882 23.1638679504395 5.0
36.637966041 36.7235488891602 6.0
36.637966041 36.484733581543 7.0
38.712447285 38.7288208007812 0.0
19.874849154 20.1867618560791 1.0
18.837598131 18.1591987609863 2.0
-9.656315418 -9.04708290100098 3.0
9.181282703 8.90895366668701 4.0
29.531164582 29.003870010376 5.0
38.712447295 38.2031784057617 6.0
38.712447295 39.4380950927734 7.0
29.0840224199388 38.1819839477539 0.0
12.9348822379388 14.5154609680176 1.0
16.1491401919388 14.3640298843384 2.0
-6.2158408037388 -5.854407787323 3.0
9.93329938853882 9.7581787109375 4.0
19.1507230409388 19.3183727264404 5.0
29.08402242 29.7144298553467 6.0
29.08402242 29.3537063598633 7.0
40.7419030105752 38.7526321411133 0.0
19.6019574718567 20.2419586181641 1.0
21.139945535663 20.0191478729248 2.0
-8.29196876971742 -8.19618988037109 3.0
12.8479767652648 12.1341876983643 4.0
27.8939262422712 28.183198928833 5.0
40.741903011 40.4545211791992 6.0
40.741903011 40.4759216308594 7.0
44.4239279780974 38.7700424194336 0.0
22.6860882760974 21.8470287322998 1.0
21.7378397120974 21.6257266998291 2.0
-6.03315836089742 -7.33268356323242 3.0
15.7046813510974 14.6819858551025 4.0
28.7192466370973 27.9537601470947 5.0
44.423927978 43.4490966796875 6.0
44.423927978 44.0868301391602 7.0
37.279929002 40.722297668457 0.0
17.697440733 18.6612033843994 1.0
19.582488269 17.9926910400391 2.0
-8.4860638308 -8.0453052520752 3.0
11.096424428 10.6737976074219 4.0
26.183504574 26.7601871490479 5.0
37.279929011 35.4391937255859 6.0
37.279929011 35.7633285522461 7.0
39.065196556 38.2746047973633 0.0
22.911627527 20.3264713287354 1.0
16.153569029 18.2511138916016 2.0
-7.035333012 -9.65429401397705 3.0
9.1182360067 8.52085018157959 4.0
29.946960549 29.2601470947266 5.0
39.065196566 38.0067367553711 6.0
39.065196566 38.2126693725586 7.0
34.346712901 38.617561340332 0.0
17.419062718 16.6762275695801 1.0
16.927650183 17.3127841949463 2.0
-3.8964434363 -5.28463315963745 3.0
13.031206737 12.3719930648804 4.0
21.315506165 21.3649024963379 5.0
34.346712911 34.0277557373047 6.0
34.346712911 34.495849609375 7.0
44.832836192 37.8687896728516 0.0
22.86238329 21.5256099700928 1.0
21.970452902 21.2041835784912 2.0
-8.0950157692 -8.58045959472656 3.0
13.875437123 13.1890459060669 4.0
30.957399069 29.8986225128174 5.0
44.832836202 43.7193450927734 6.0
44.832836202 43.844123840332 7.0
37.693005907 37.8074417114258 0.0
18.767233493 18.4735507965088 1.0
18.925772415 18.8588104248047 2.0
-4.4039460583 -4.7475790977478 3.0
14.521826348 14.3328046798706 4.0
23.17117956 22.4735546112061 5.0
37.693005916 37.4263381958008 6.0
37.693005916 37.8930740356445 7.0
35.516030196 38.2577285766602 0.0
17.238388415 17.4624156951904 1.0
18.277641781 17.627254486084 2.0
-5.9879556024 -6.29086351394653 3.0
12.289686168 11.9834842681885 4.0
23.226344027 22.9670333862305 5.0
35.516030206 35.4134750366211 6.0
35.516030206 36.0930252075195 7.0
45.449957515 36.9237747192383 0.0
24.470260824 22.9747180938721 1.0
20.979696692 21.4246616363525 2.0
-8.6547786371 -10.2861642837524 3.0
12.324918046 11.495792388916 4.0
33.12503947 32.4793243408203 5.0
45.449957523 45.0430755615234 6.0
45.449957523 44.0608062744141 7.0
36.522103961 38.1218719482422 0.0
17.787809101 17.765209197998 1.0
18.734294862 18.7959289550781 2.0
-3.7955939001 -4.28736639022827 3.0
14.938700953 14.4545946121216 4.0
21.583403009 20.6723518371582 5.0
36.52210397 35.9171676635742 6.0
36.52210397 36.7773666381836 7.0
40.809333824 38.1619262695312 0.0
21.825363738 20.3745574951172 1.0
18.983970087 20.2993831634521 2.0
-4.5096782574 -6.41374826431274 3.0
14.474291821 13.794828414917 4.0
26.335042004 25.4069137573242 5.0
40.809333833 40.2103576660156 6.0
40.809333833 40.7029113769531 7.0
43.708815015 39.1823577880859 0.0
22.06122806 21.4146289825439 1.0
21.647586954 21.0050201416016 2.0
-7.2998869135 -8.47407913208008 3.0
14.347700031 13.7081642150879 4.0
29.361114984 29.4641246795654 5.0
43.708815025 43.1477203369141 6.0
43.708815025 43.0359878540039 7.0
35.746307566 38.1672134399414 0.0
17.418825391 18.5653247833252 1.0
18.327482175 17.4033260345459 2.0
-8.2684434585 -7.56141662597656 3.0
10.059038708 9.9279899597168 4.0
25.687268859 25.0303840637207 5.0
35.746307575 37.0441131591797 6.0
35.746307575 36.9562149047852 7.0
32.565260397 38.3837738037109 0.0
15.347203209 16.2376251220703 1.0
17.218057188 16.839334487915 2.0
-4.4968959364 -4.29552936553955 3.0
12.721161242 12.2814855575562 4.0
19.844099155 19.8660144805908 5.0
32.565260407 33.5784378051758 6.0
32.565260407 34.2695922851562 7.0
37.459437559 39.1984710693359 0.0
20.974106733 19.2715492248535 1.0
16.485330826 17.0758037567139 2.0
-7.8025517563 -9.46932220458984 3.0
8.6827790601 8.27703666687012 4.0
28.776658499 28.5797557830811 5.0
37.459437569 36.4917373657227 6.0
37.459437569 37.0200958251953 7.0
41.6008687772188 37.8575820922852 0.0
18.8099504026802 19.9241485595703 1.0
22.7909183777866 20.1553192138672 2.0
-6.50941037372392 -5.38870763778687 3.0
16.2815080036091 15.0416536331177 4.0
25.319360776445 23.9512557983398 5.0
41.600868777 39.8447265625 6.0
41.600868777 40.6090469360352 7.0
43.08864828 42.9987716674805 0.0
20.75366021 23.2980003356934 1.0
22.334988069 22.0826263427734 2.0
-9.5492158301 -9.45747566223145 3.0
12.785772229 11.5652866363525 4.0
30.30287605 32.2585906982422 5.0
43.088648289 45.6551971435547 6.0
43.088648289 46.1906280517578 7.0
30.268152668 37.1409225463867 0.0
15.014233699 15.0411100387573 1.0
15.25391897 14.9551963806152 2.0
-5.5931438946 -5.8237099647522 3.0
9.6607750663 9.57966232299805 4.0
20.607377602 20.4172859191895 5.0
30.268152677 30.5712718963623 6.0
30.268152677 30.7838001251221 7.0
38.454045879 37.2824249267578 0.0
17.159579151 18.6700572967529 1.0
21.29446673 18.9856376647949 2.0
-7.5807931052 -5.8023157119751 3.0
13.713673616 13.2957019805908 4.0
24.740372265 23.9335842132568 5.0
38.454045888 38.1710433959961 6.0
38.454045888 38.3021545410156 7.0
36.654056854 38.5471496582031 0.0
19.439010257 18.9937343597412 1.0
17.215046598 17.5202980041504 2.0
-7.9864590963 -8.61619663238525 3.0
9.2285874923 8.71558570861816 4.0
27.425469363 27.1988697052002 5.0
36.654056864 36.5646362304688 6.0
36.654056864 37.8646697998047 7.0
36.990278119 38.5366821289062 0.0
18.644416453 18.7880783081055 1.0
18.345861667 17.3624744415283 2.0
-7.9977544937 -7.63241291046143 3.0
10.348107164 9.8026123046875 4.0
26.642170956 25.8676433563232 5.0
36.990278128 36.491813659668 6.0
36.990278128 37.627555847168 7.0
39.45539178 38.8359832763672 0.0
17.953554537 19.0125274658203 1.0
21.501837243 18.6644802093506 2.0
-6.7799927019 -5.81520938873291 3.0
14.721844531 14.5281772613525 4.0
24.733547249 23.6711483001709 5.0
39.45539179 38.500602722168 6.0
39.45539179 39.6684875488281 7.0
41.490098301 38.646110534668 0.0
18.418139969 20.1088981628418 1.0
23.071958334 20.5167922973633 2.0
-6.7020850264 -5.84434700012207 3.0
16.369873299 15.1859064102173 4.0
25.120225004 24.2188415527344 5.0
41.49009831 40.7906951904297 6.0
41.49009831 40.9147186279297 7.0
42.195848756 40.1018142700195 0.0
18.431626231 21.4116287231445 1.0
23.764222526 20.5440311431885 2.0
-9.6629820221 -7.10689353942871 3.0
14.101240496 13.4462242126465 4.0
28.094608262 27.11061668396 5.0
42.195848764 42.3811187744141 6.0
42.195848764 41.2984848022461 7.0
36.580423947 37.9035873413086 0.0
18.088597645 21.2339611053467 1.0
18.491826302 19.0799865722656 2.0
-10.331066357 -9.98916530609131 3.0
8.1607599346 7.98917293548584 4.0
28.419664012 30.2720394134521 5.0
36.580423956 39.9783248901367 6.0
36.580423956 40.3662643432617 7.0
41.112612837 38.6265029907227 0.0
21.264748673 21.0978622436523 1.0
19.847864165 19.9049644470215 2.0
-7.8539892402 -8.99751663208008 3.0
11.993874916 11.0752277374268 4.0
29.118737922 29.3703079223633 5.0
41.112612846 39.4206314086914 6.0
41.112612846 39.5524520874023 7.0
41.165032295 38.2608489990234 0.0
20.777866565 21.3963184356689 1.0
20.387165729 19.1337871551514 2.0
-11.013080697 -10.121452331543 3.0
9.3740850227 8.92656707763672 4.0
31.790947272 30.5620822906494 5.0
41.165032305 40.1152267456055 6.0
41.165032305 41.0985946655273 7.0
28.23632976 38.1864700317383 0.0
14.450875753 14.5610046386719 1.0
13.785454008 13.6108846664429 2.0
-5.4067157656 -6.53684329986572 3.0
8.3787382334 8.26820278167725 4.0
19.857591527 20.2184524536133 5.0
28.236329769 29.2296028137207 6.0
28.236329769 29.4686431884766 7.0
46.563722225 38.1880493164062 0.0
24.514076121 22.7633590698242 1.0
22.049646104 21.9152851104736 2.0
-6.9458736065 -7.58726024627686 3.0
15.103772488 14.7064065933228 4.0
31.459949737 29.7712516784668 5.0
46.563722235 44.9567184448242 6.0
46.563722235 45.9777069091797 7.0
38.976039438 34.1136169433594 0.0
20.47408062 18.8379173278809 1.0
18.50195882 19.1683197021484 2.0
-3.6791774969 -6.09064865112305 3.0
14.822781314 14.9086608886719 4.0
24.153258125 23.3799839019775 5.0
38.976039446 39.2413177490234 6.0
38.976039446 39.4658050537109 7.0
43.902220155 39.1786575317383 0.0
23.085534006 23.4488563537598 1.0
20.816686149 22.4787120819092 2.0
-6.0778252495 -9.12560844421387 3.0
14.738860889 14.0046243667603 4.0
29.163359266 31.5139141082764 5.0
43.902220165 46.4647827148438 6.0
43.902220165 47.53662109375 7.0
37.104870377 37.707893371582 0.0
21.928667997 19.5249156951904 1.0
15.176202381 17.5614700317383 2.0
-7.1829261037 -9.80313682556152 3.0
7.9932762681 7.98862075805664 4.0
29.11159411 28.3439083099365 5.0
37.104870385 37.2066040039062 6.0
37.104870385 37.9261703491211 7.0
48.094555135036 38.6682662963867 0.0
24.9788607525581 25.1454029083252 1.0
23.1156943831877 22.4715099334717 2.0
-7.58647576369851 -7.68739700317383 3.0
15.5292186191899 14.3461151123047 4.0
32.5653365171039 30.5723094940186 5.0
48.094555135 44.4077377319336 6.0
48.094555135 46.8743133544922 7.0
39.863550950908 38.1398620605469 0.0
20.2748443148297 20.5085353851318 1.0
19.5887066378693 18.5262336730957 2.0
-9.95882058391343 -9.65867233276367 3.0
9.62988605424657 9.29629039764404 4.0
30.233664897769 29.5470676422119 5.0
39.863550951 39.6430892944336 6.0
39.863550951 40.0736465454102 7.0
38.158424696 37.6681747436523 0.0
21.093616444 22.0732917785645 1.0
17.064808252 20.5325775146484 2.0
-6.6865614213 -10.2466697692871 3.0
10.378246821 9.59789752960205 4.0
27.780177875 31.4843349456787 5.0
38.158424706 42.7158203125 6.0
38.158424706 43.3326950073242 7.0
37.671552635 38.2944183349609 0.0
18.060105754 18.5829219818115 1.0
19.611446882 18.7748870849609 2.0
-6.6868904195 -6.5537691116333 3.0
12.924556453 12.4624843597412 4.0
24.746996183 24.1268634796143 5.0
37.671552644 36.9315948486328 6.0
37.671552644 38.0746917724609 7.0
38.9900068701748 38.3024597167969 0.0
19.4317892501748 19.9019870758057 1.0
19.5582176301748 19.5858058929443 2.0
-6.7795985865748 -7.9151029586792 3.0
12.7786190431748 12.0273065567017 4.0
26.2113877989032 26.9956111907959 5.0
38.99000687 39.3139495849609 6.0
38.99000687 38.7906112670898 7.0
32.824279187 38.8112869262695 0.0
17.524252685 16.0635318756104 1.0
15.300026502 15.6275568008423 2.0
-5.6991008156 -6.74575519561768 3.0
9.6009256767 9.46850776672363 4.0
23.223353511 23.2582511901855 5.0
32.824279197 32.8469848632812 6.0
32.824279197 32.7363967895508 7.0
39.752392388 38.4769897460938 0.0
19.337645943 19.1063632965088 1.0
20.414746445 19.9582710266113 2.0
-4.1151033929 -5.25276899337769 3.0
16.299643042 15.4891633987427 4.0
23.452749346 22.6432285308838 5.0
39.752392398 39.1198425292969 6.0
39.752392398 39.4828796386719 7.0
36.0149979881443 38.0255355834961 0.0
19.5282530962186 17.835147857666 1.0
16.4867448931817 17.4042663574219 2.0
-5.97886461861353 -7.98457145690918 3.0
10.5078802751384 9.96859836578369 4.0
25.5071177152005 25.1389141082764 5.0
36.014997988 35.8599395751953 6.0
36.014997988 36.1396865844727 7.0
42.307205596 38.211555480957 0.0
21.688253633 20.0025787353516 1.0
20.618951963 20.7712383270264 2.0
-4.4760898368 -5.28428745269775 3.0
16.142862116 15.6773233413696 4.0
26.164343479 25.2863941192627 5.0
42.307205606 40.986946105957 6.0
42.307205606 41.2777633666992 7.0
41.624369925 38.2369766235352 0.0
21.269456313 21.4407863616943 1.0
20.354913614 20.0920639038086 2.0
-9.726585677 -9.24065017700195 3.0
10.628327928 9.91324234008789 4.0
30.996041999 29.5769729614258 5.0
41.624369934 39.8592681884766 6.0
41.624369934 41.3274841308594 7.0
40.927860916 37.5762405395508 0.0
20.558611306 20.8498229980469 1.0
20.36924961 19.8419551849365 2.0
-7.6689328495 -7.97664070129395 3.0
12.700316751 12.0926990509033 4.0
28.227544165 27.9054050445557 5.0
40.927860926 39.9788818359375 6.0
40.927860926 40.8771896362305 7.0
47.441321803 38.3906173706055 0.0
22.43048805 23.1683750152588 1.0
25.010833753 22.8570461273193 2.0
-8.9760978683 -7.52433204650879 3.0
16.034735875 15.56809425354 4.0
31.406585928 30.2101974487305 5.0
47.441321812 45.6768646240234 6.0
47.441321812 46.8501434326172 7.0
39.37950561 38.0905303955078 0.0
19.873167771 21.6447582244873 1.0
19.506337839 19.4344139099121 2.0
-9.8415530061 -10.8011245727539 3.0
9.6647848227 8.9887752532959 4.0
29.714720787 31.1449851989746 5.0
39.379505619 41.7300796508789 6.0
39.379505619 42.3433303833008 7.0
42.095557289 38.8531799316406 0.0
22.277305426 21.7924900054932 1.0
19.818251863 19.8177433013916 2.0
-7.6874055589 -9.42489433288574 3.0
12.130846294 11.3354415893555 4.0
29.964710995 29.3569068908691 5.0
42.095557299 41.9169235229492 6.0
42.095557299 40.2805328369141 7.0
35.151188218 38.4096450805664 0.0
17.118344744 16.9611892700195 1.0
18.032843477 18.1718235015869 2.0
-2.0060677241 -3.1261191368103 3.0
16.026775747 15.6545400619507 4.0
19.124412475 19.1798362731934 5.0
35.151188225 34.8697280883789 6.0
35.151188225 34.7039413452148 7.0
37.713935362 38.1939315795898 0.0
20.380527895 19.2701034545898 1.0
17.333407469 18.2382144927979 2.0
-5.6423114193 -7.66351127624512 3.0
11.691096041 11.0271034240723 4.0
26.022839323 25.0261936187744 5.0
37.713935371 37.0099563598633 6.0
37.713935371 37.2810211181641 7.0
39.667743251 37.5596237182617 0.0
20.027544136 20.8216609954834 1.0
19.640199117 18.7193450927734 2.0
-10.376032212 -10.334132194519 3.0
9.2641668955 8.83496761322021 4.0
30.403576357 30.1484069824219 5.0
39.66774326 40.1139984130859 6.0
39.66774326 40.4533233642578 7.0
41.240556184 38.3862609863281 0.0
23.347458936 21.1435432434082 1.0
17.893097248 19.3683052062988 2.0
-7.252142064 -9.32804203033447 3.0
10.640955174 10.0189075469971 4.0
30.59960101 29.6779499053955 5.0
41.240556194 40.7343292236328 6.0
41.240556194 41.4971542358398 7.0
40.32019459 38.6126174926758 0.0
19.177093862 20.7322006225586 1.0
21.143100728 20.89084815979 2.0
-7.2609061044 -6.85930156707764 3.0
13.882194613 13.1603937149048 4.0
26.437999977 28.0188465118408 5.0
40.3201946 39.28173828125 6.0
40.3201946 40.6734466552734 7.0
39.5611510585585 39.3859558105469 0.0
19.6094718095582 19.8197803497314 1.0
19.9516792585585 19.0167236328125 2.0
-8.44726993515885 -8.88727855682373 3.0
11.5044093235587 10.6241550445557 4.0
28.0567416170478 28.0198516845703 5.0
39.561151059 38.9889526367188 6.0
39.561151059 39.7982940673828 7.0
43.5664319470049 39.2536926269531 0.0
22.3670487340048 21.7380771636963 1.0
21.1993832230048 20.2213344573975 2.0
-8.61702915780503 -8.69567966461182 3.0
12.582354065005 12.0542526245117 4.0
30.9840778920045 29.7219352722168 5.0
43.566431947 41.4726181030273 6.0
43.566431947 42.7189102172852 7.0
37.929366556 38.362907409668 0.0
20.105495873 19.1329498291016 1.0
17.823870687 18.7470550537109 2.0
-5.9768423324 -7.61201190948486 3.0
11.847028349 11.3475866317749 4.0
26.082338211 25.8750991821289 5.0
37.929366562 37.3192825317383 6.0
37.929366562 38.3035202026367 7.0
39.7456686413092 42.3975677490234 0.0
20.6016083083092 19.5039939880371 1.0
19.1440603423092 20.3956241607666 2.0
-3.0966467648096 -5.32644462585449 3.0
16.0474135773093 15.4909830093384 4.0
23.6982550733092 23.5807266235352 5.0
39.745668641 38.9904098510742 6.0
39.745668641 38.6185531616211 7.0
37.1737946256367 38.6771087646484 0.0
18.9810950589955 18.28173828125 1.0
18.1926995622865 18.5749950408936 2.0
-4.83110587561615 -6.28658008575439 3.0
13.3615936860386 12.8778429031372 4.0
23.8122009348139 23.6073837280273 5.0
37.173794625 36.0051956176758 6.0
37.173794625 36.3588485717773 7.0
27.124866007 40.3398284912109 0.0
13.623671484 13.8152093887329 1.0
13.501194525 12.9474258422852 2.0
-4.5998142153 -5.25400972366333 3.0
8.9013803006 9.02766990661621 4.0
18.223485707 18.1296672821045 5.0
27.124866015 28.8868579864502 6.0
27.124866015 28.4412879943848 7.0
27.411746895 36.7463607788086 0.0
13.763683287 13.9854421615601 1.0
13.648063611 13.749698638916 2.0
-4.2527245718 -4.60786962509155 3.0
9.3953390304 9.39307498931885 4.0
18.016407867 18.6928081512451 5.0
27.411746904 29.2872047424316 6.0
27.411746904 28.4803886413574 7.0
38.866617309 38.1587142944336 0.0
22.319478804 20.197925567627 1.0
16.547138508 17.9812850952148 2.0
-7.9092899105 -10.3619441986084 3.0
8.6378485894 8.25391960144043 4.0
30.228768722 29.4518909454346 5.0
38.866617317 38.6341781616211 6.0
38.866617317 39.0581817626953 7.0
42.3830573535328 38.3150253295898 0.0
20.5951530026372 20.8092937469482 1.0
21.7879043479541 19.9050922393799 2.0
-9.3946766066605 -8.83049869537354 3.0
12.3932277407315 11.7097043991089 4.0
29.98982960914 29.5242958068848 5.0
42.383057354 41.7898635864258 6.0
42.383057354 41.9013290405273 7.0
47.6437510358734 38.1060256958008 0.0
23.1576165238734 23.1068058013916 1.0
24.4861345218735 22.5069370269775 2.0
-8.74385685767346 -8.68577289581299 3.0
15.7422776638735 14.8399543762207 4.0
31.9014733818732 30.6543979644775 5.0
47.643751036 46.4351348876953 6.0
47.643751036 46.9786758422852 7.0
38.439399947 36.4208679199219 0.0
19.975660163 19.6007099151611 1.0
18.463739784 18.9128589630127 2.0
-6.4266906816 -7.36500644683838 3.0
12.037049093 11.4773578643799 4.0
26.402350854 26.1536636352539 5.0
38.439399957 37.2386703491211 6.0
38.439399957 38.7358169555664 7.0
40.263562361 38.7521286010742 0.0
20.42306719 21.6324996948242 1.0
19.840495171 19.0169887542725 2.0
-11.677045299 -10.7362909317017 3.0
8.1634498621 7.76362562179565 4.0
32.100112499 31.1445732116699 5.0
40.263562371 39.7273941040039 6.0
40.263562371 40.8870162963867 7.0
41.519528387 38.4955291748047 0.0
22.877320367 20.5484390258789 1.0
18.642208021 20.2042045593262 2.0
-5.2280057864 -7.99882888793945 3.0
13.414202225 12.5762281417847 4.0
28.105326163 27.7521781921387 5.0
41.519528397 41.0124359130859 6.0
41.519528397 41.6467742919922 7.0
40.414570464 38.545280456543 0.0
17.791289494 19.2980270385742 1.0
22.623280971 20.0405139923096 2.0
-6.6442159501 -5.92895364761353 3.0
15.979065012 14.6843795776367 4.0
24.435505453 23.8397598266602 5.0
40.414570474 39.4348983764648 6.0
40.414570474 40.1382675170898 7.0
37.834181215 38.0458221435547 0.0
18.770560656 19.026725769043 1.0
19.063620563 18.743106842041 2.0
-7.0281211763 -7.5380687713623 3.0
12.035499382 11.4961061477661 4.0
25.798681838 25.4843254089355 5.0
37.834181221 37.5501403808594 6.0
37.834181221 38.2561798095703 7.0
37.201121546 38.5927581787109 0.0
17.811506649 18.0195503234863 1.0
19.389614897 18.8371429443359 2.0
-6.0990519868 -6.29935646057129 3.0
13.290562901 12.8052835464478 4.0
23.910558645 23.9273529052734 5.0
37.201121556 36.8999633789062 6.0
37.201121556 38.0799560546875 7.0
42.944445132 37.6338272094727 0.0
21.59677927 21.4080848693848 1.0
21.347665865 21.0880851745605 2.0
-6.9735110037 -7.91884708404541 3.0
14.374154853 13.9020223617554 4.0
28.570290281 28.1455478668213 5.0
42.94444514 42.7225570678711 6.0
42.94444514 42.663444519043 7.0
47.3096000339538 38.3098220825195 0.0
24.7818404359387 23.2113876342773 1.0
22.5277596000527 22.52197265625 2.0
-6.75460229154402 -7.96077537536621 3.0
15.7731573079881 15.2921838760376 4.0
31.5364427279756 30.5247344970703 5.0
47.309600034 45.4152984619141 6.0
47.309600034 46.799430847168 7.0
34.703660739 37.7339782714844 0.0
16.95243502 16.7246932983398 1.0
17.75122572 17.4133987426758 2.0
-4.712400117 -5.60176753997803 3.0
13.038825595 12.5838718414307 4.0
21.664835145 21.2091007232666 5.0
34.703660747 34.5719451904297 6.0
34.703660747 35.0426940917969 7.0
39.355679823 38.3111801147461 0.0
18.719972615 20.0054531097412 1.0
20.635707207 19.1246032714844 2.0
-8.5340541425 -7.71829414367676 3.0
12.101653055 11.4348659515381 4.0
27.254026768 26.2859840393066 5.0
39.355679833 39.5544891357422 6.0
39.355679833 40.2200546264648 7.0
36.760838125 39.3952713012695 0.0
18.098408908 18.4492740631104 1.0
18.662429217 18.7476558685303 2.0
-4.9002916105 -5.8651704788208 3.0
13.762137597 13.3496942520142 4.0
22.998700528 22.7466430664062 5.0
36.760838135 36.7012939453125 6.0
36.760838135 37.1171569824219 7.0
45.817767369632 39.4295349121094 0.0
23.4066861136316 24.7768516540527 1.0
22.4110812656318 22.5875473022461 2.0
-7.99255969573223 -8.99110126495361 3.0
14.4185215696319 13.5078659057617 4.0
31.3992456710555 31.421703338623 5.0
45.817767369 46.1188812255859 6.0
45.817767369 47.965446472168 7.0
37.456387934 38.718879699707 0.0
18.714257418 19.2436027526855 1.0
18.742130516 18.3056735992432 2.0
-7.2626807055 -7.89689064025879 3.0
11.4794498 10.9147844314575 4.0
25.976938133 25.7266731262207 5.0
37.456387944 38.0156631469727 6.0
37.456387944 37.4311981201172 7.0
36.979299104 38.0109329223633 0.0
17.001867984 18.2078247070312 1.0
19.977431121 19.2218017578125 2.0
-4.4589079828 -4.98952913284302 3.0
15.518523129 14.8784132003784 4.0
21.460775976 21.1240882873535 5.0
36.979299113 37.1302108764648 6.0
36.979299113 37.1038970947266 7.0
44.357392148 38.3760147094727 0.0
23.57148554 21.6354026794434 1.0
20.785906608 21.3629627227783 2.0
-6.1950554121 -7.60999011993408 3.0
14.590851187 14.170747756958 4.0
29.766540962 28.0877113342285 5.0
44.357392157 43.3780136108398 6.0
44.357392157 43.6659927368164 7.0
36.8120792993302 38.4635009765625 0.0
19.8640110823302 19.7038173675537 1.0
16.9480682263302 17.6865768432617 2.0
-8.51793333553018 -9.63223648071289 3.0
8.43013489113018 7.97719144821167 4.0
28.381944393825 28.8917350769043 5.0
36.812079299 37.8954010009766 6.0
36.812079299 38.8740615844727 7.0
43.172254734 39.1506042480469 0.0
21.994700089 21.287561416626 1.0
21.177554646 21.6107559204102 2.0
-4.8220170211 -6.68520212173462 3.0
16.355537616 15.6073360443115 4.0
26.816717119 26.7941513061523 5.0
43.172254743 43.32275390625 6.0
43.172254743 43.4117965698242 7.0
47.109370558 39.7863998413086 0.0
25.034746172 23.357873916626 1.0
22.074624387 22.3467330932617 2.0
-8.7769581235 -9.06822299957275 3.0
13.297666255 12.5269060134888 4.0
33.811704304 32.8016815185547 5.0
47.109370567 46.5079803466797 6.0
47.109370567 46.2982406616211 7.0
35.871355653 37.1250152587891 0.0
18.076771786 18.8049793243408 1.0
17.794583868 17.2032375335693 2.0
-9.6137759762 -9.29005146026611 3.0
8.1808078825 7.89497995376587 4.0
27.690547772 27.5203189849854 5.0
35.871355662 35.1652069091797 6.0
35.871355662 36.8341217041016 7.0
42.1694339111434 38.8314437866211 0.0
22.6030080131434 21.3081550598145 1.0
19.5664259071434 20.3241405487061 2.0
-7.61628576964338 -9.15699577331543 3.0
11.9501401371434 11.008526802063 4.0
30.2192937831435 29.7380466461182 5.0
42.169433911 41.5776596069336 6.0
42.169433911 42.0681686401367 7.0
38.001249391 38.1650390625 0.0
19.442391644 19.7656803131104 1.0
18.558857748 18.629207611084 2.0
-7.4577746389 -8.1876916885376 3.0
11.101083099 10.5302896499634 4.0
26.900166292 26.8108882904053 5.0
38.001249401 37.9244537353516 6.0
38.001249401 38.9150009155273 7.0
36.026296144 38.3478240966797 0.0
18.544058535 18.6628570556641 1.0
17.482237609 17.9380531311035 2.0
-7.0335899837 -7.57428932189941 3.0
10.448647616 9.87599658966064 4.0
25.577648528 24.9330654144287 5.0
36.026296153 37.2376403808594 6.0
36.026296153 37.703010559082 7.0
33.200213149 40.287467956543 0.0
14.957525483 17.025318145752 1.0
18.242687666 16.2773590087891 2.0
-8.613844647 -7.50124549865723 3.0
9.6288430093 9.21061706542969 4.0
23.57137014 23.7718601226807 5.0
33.200213159 33.4814682006836 6.0
33.200213159 32.6489105224609 7.0
36.881947947 39.2395782470703 0.0
19.760267083 18.9527988433838 1.0
17.12168087 17.8010845184326 2.0
-6.5510496361 -8.39307975769043 3.0
10.570631229 9.94493865966797 4.0
26.311316723 26.6371574401855 5.0
36.881947952 36.7554473876953 6.0
36.881947952 37.4116897583008 7.0
28.363012003 38.7270050048828 0.0
14.617095463 14.1622161865234 1.0
13.745916541 13.3465538024902 2.0
-5.3741086073 -6.76098442077637 3.0
8.3718079243 8.29381370544434 4.0
19.991204079 19.9495868682861 5.0
28.363012012 29.9768486022949 6.0
28.363012012 29.0645999908447 7.0
39.574467416 37.6061553955078 0.0
20.649388383 22.4414443969727 1.0
18.925079033 20.6889381408691 2.0
-8.4547238017 -10.4059810638428 3.0
10.470355222 9.51583576202393 4.0
29.104112194 32.0760040283203 5.0
39.574467426 43.910514831543 6.0
39.574467426 43.7261199951172 7.0
36.220401471 38.3949584960938 0.0
17.995668566 18.2893886566162 1.0
18.224732906 17.837739944458 2.0
-6.9105415737 -7.24199485778809 3.0
11.314191323 11.0085277557373 4.0
24.906210149 24.359188079834 5.0
36.22040148 36.4723129272461 6.0
36.22040148 37.1749725341797 7.0
31.086432133 39.1634368896484 0.0
16.0910287 15.4654598236084 1.0
14.995403435 14.9265394210815 2.0
-4.3439743018 -5.34590101242065 3.0
10.651429126 10.6104555130005 4.0
20.43500301 20.3031311035156 5.0
31.086432141 31.1367130279541 6.0
31.086432141 31.2280387878418 7.0
36.274353053 38.4463577270508 0.0
18.420887917 17.6843662261963 1.0
17.853465137 18.4291648864746 2.0
-3.6202808126 -5.21385526657104 3.0
14.233184315 13.9288187026978 4.0
22.041168739 22.1156845092773 5.0
36.274353063 35.6696166992188 6.0
36.274353063 36.3318481445312 7.0
31.65795008 39.2199935913086 0.0
17.489254099 16.4708976745605 1.0
14.168695981 14.7580490112305 2.0
-6.0427493715 -7.45183753967285 3.0
8.1259465996 7.81711769104004 4.0
23.53200348 23.9427928924561 5.0
31.65795009 32.1667098999023 6.0
31.65795009 30.7517910003662 7.0
37.059171469 39.3203582763672 0.0
19.339994356 19.4019470214844 1.0
17.719177113 16.8399124145508 2.0
-9.4011683011 -9.28989315032959 3.0
8.318008802 8.35234451293945 4.0
28.741162667 28.0324687957764 5.0
37.059171479 37.2253952026367 6.0
37.059171479 37.5436477661133 7.0
45.308862348 39.220817565918 0.0
24.95721568 22.3412780761719 1.0
20.351646669 21.7738075256348 2.0
-7.323343681 -9.45879745483398 3.0
13.028302978 11.908106803894 4.0
32.28055937 31.4420166015625 5.0
45.308862357 44.7452087402344 6.0
45.308862357 44.6438064575195 7.0
32.988279899 39.2047119140625 0.0
17.965429309 16.0819435119629 1.0
15.02285059 16.926342010498 2.0
-2.6394240327 -4.85493850708008 3.0
12.383426547 11.7717523574829 4.0
20.604853352 20.5145568847656 5.0
32.988279909 33.3665161132812 6.0
32.988279909 33.1205062866211 7.0
41.812200584 40.716796875 0.0
21.127130287 20.7889308929443 1.0
20.685070296 20.4194183349609 2.0
-5.5457377677 -6.51394557952881 3.0
15.139332519 14.2325305938721 4.0
26.672868065 26.1188449859619 5.0
41.812200593 41.8736572265625 6.0
41.812200593 41.9870223999023 7.0
34.159577997 37.265266418457 0.0
15.979883142 16.5733184814453 1.0
18.179694855 16.5619583129883 2.0
-6.0058060763 -4.70657825469971 3.0
12.173888768 12.3701992034912 4.0
21.985689228 21.2810325622559 5.0
34.159578007 33.7586975097656 6.0
34.159578007 33.9638290405273 7.0
41.353058194 40.8923721313477 0.0
20.665722502 21.6208305358887 1.0
20.687335692 20.076301574707 2.0
-9.525258586 -9.4038724899292 3.0
11.162077096 10.2773103713989 4.0
30.190981098 29.7053661346436 5.0
41.353058204 42.2395629882812 6.0
41.353058204 41.7193908691406 7.0
35.663236634 38.0364151000977 0.0
18.766917562 18.0305480957031 1.0
16.896319073 16.7553672790527 2.0
-7.670552873 -8.74667644500732 3.0
9.2257661902 8.80641746520996 4.0
26.437470445 25.8856792449951 5.0
35.663236644 35.025276184082 6.0
35.663236644 35.8400115966797 7.0
40.444916113 39.8431777954102 0.0
18.421748417 20.8138008117676 1.0
22.023167696 19.4187850952148 2.0
-9.937900394 -9.13444900512695 3.0
12.085267292 11.1179285049438 4.0
28.359648821 28.2522449493408 5.0
40.444916123 40.682243347168 6.0
40.444916123 39.7233963012695 7.0
39.900416904 39.0326843261719 0.0
21.000367331 20.732442855835 1.0
18.900049573 19.8056240081787 2.0
-6.1981317567 -8.53449058532715 3.0
12.701917806 12.1823568344116 4.0
27.198499098 27.9057292938232 5.0
39.900416914 40.4196701049805 6.0
39.900416914 41.707633972168 7.0
34.40532484 38.8125534057617 0.0
18.680133752 17.6248817443848 1.0
15.725191089 16.2882823944092 2.0
-6.5675447079 -8.02764225006104 3.0
9.157646372 8.86200714111328 4.0
25.247678469 24.9007549285889 5.0
34.405324849 33.9266052246094 6.0
34.405324849 33.9994049072266 7.0
33.074255995 39.7259750366211 0.0
16.81819884 17.1022033691406 1.0
16.256057155 16.4729804992676 2.0
-5.8048358004 -6.80570220947266 3.0
10.451221345 10.1145238876343 4.0
22.62303465 22.4940090179443 5.0
33.074256004 33.1487350463867 6.0
33.074256004 33.1579895019531 7.0
40.036170298 40.0705032348633 0.0
18.427862329 19.8924922943115 1.0
21.60830797 20.1940879821777 2.0
-6.8170168406 -6.05037450790405 3.0
14.791291119 14.3158731460571 4.0
25.244879179 24.3987693786621 5.0
40.036170308 39.1054306030273 6.0
40.036170308 40.0101089477539 7.0
44.453206231 38.9618377685547 0.0
21.51525667 22.0500297546387 1.0
22.937949561 21.89235496521 2.0
-6.654859187 -7.4339656829834 3.0
16.283090365 15.3504457473755 4.0
28.170115867 28.1608390808105 5.0
44.453206241 44.2733459472656 6.0
44.453206241 44.8532028198242 7.0
33.8500454 39.1324462890625 0.0
15.317438725 16.899019241333 1.0
18.532606675 17.4372062683105 2.0
-3.771729854 -3.66815042495728 3.0
14.760876811 14.0392112731934 4.0
19.089168589 18.587818145752 5.0
33.85004541 33.9061126708984 6.0
33.85004541 34.3162841796875 7.0
30.537877299 37.9304428100586 0.0
15.743236653 15.7335357666016 1.0
14.794640646 14.6241807937622 2.0
-6.2614673178 -7.16267967224121 3.0
8.5331733191 8.3601188659668 4.0
22.00470398 22.2800445556641 5.0
30.537877308 30.127908706665 6.0
30.537877308 30.27783203125 7.0
27.727440934 36.8406219482422 0.0
13.298433958 13.9568004608154 1.0
14.429006978 13.708288192749 2.0
-4.8732361991 -4.95635652542114 3.0
9.5557707702 9.69101238250732 4.0
18.171670165 18.6766796112061 5.0
27.727440942 29.0367889404297 6.0
27.727440942 28.8692474365234 7.0
35.630182891 38.8659591674805 0.0
17.416160369 17.4986476898193 1.0
18.214022525 17.9752025604248 2.0
-5.5362398817 -6.49297046661377 3.0
12.677782636 12.2979021072388 4.0
22.952400258 23.0043239593506 5.0
35.630182897 35.8207244873047 6.0
35.630182897 36.9947814941406 7.0
31.038251509 38.4938354492188 0.0
16.216885208 15.719181060791 1.0
14.821366301 15.8819465637207 2.0
-5.1281710605 -5.38061714172363 3.0
9.6931952301 9.57784557342529 4.0
21.345056279 21.1744575500488 5.0
31.038251519 35.3713836669922 6.0
31.038251519 33.4895095825195 7.0
31.867011958 38.5478668212891 0.0
14.888086344 16.0857772827148 1.0
16.978925614 16.0145263671875 2.0
-5.7710515718 -5.4265308380127 3.0
11.207874033 10.7744665145874 4.0
20.659137926 20.0649967193604 5.0
31.867011967 32.6286582946777 6.0
31.867011967 32.7651405334473 7.0
30.754831636 38.8118896484375 0.0
14.067592922 15.3162145614624 1.0
16.687238713 15.7319660186768 2.0
-4.2836008115 -4.33065414428711 3.0
12.403637892 12.184775352478 4.0
18.351193744 18.0475749969482 5.0
30.754831646 31.9293422698975 6.0
30.754831646 31.8863906860352 7.0
38.951564273 38.2272415161133 0.0
20.789904648 19.7062282562256 1.0
18.161659627 17.7537155151367 2.0
-9.14497609 -9.92282485961914 3.0
9.0166835299 8.43372058868408 4.0
29.934880745 29.3792514801025 5.0
38.95156428 38.0840377807617 6.0
38.95156428 38.6076202392578 7.0
33.384480929 39.1025085449219 0.0
14.280681926 16.4918975830078 1.0
19.103799006 16.9669761657715 2.0
-6.8894608583 -5.55620813369751 3.0
12.214338142 11.8145179748535 4.0
21.170142791 21.3958950042725 5.0
33.384480936 32.6654891967773 6.0
33.384480936 31.9091911315918 7.0
37.291199278 38.8030319213867 0.0
18.976122996 17.4957332611084 1.0
18.315076287 18.4499969482422 2.0
-3.6314782733 -5.38515138626099 3.0
14.68359801 14.2914695739746 4.0
22.607601274 22.0485286712646 5.0
37.291199282 36.0691604614258 6.0
37.291199282 36.6016616821289 7.0
33.5718110164568 38.6800003051758 0.0
16.2455196066318 16.9113616943359 1.0
17.3262914102844 16.5193481445312 2.0
-6.34124959778268 -6.53616285324097 3.0
10.9850418123699 10.714958190918 4.0
22.5867692042782 22.6549491882324 5.0
33.571811016 34.1067886352539 6.0
33.571811016 34.3091583251953 7.0
44.048761737 38.4862060546875 0.0
22.194046458 21.6261215209961 1.0
21.85471528 21.7322368621826 2.0
-6.0219898767 -7.19724655151367 3.0
15.832725394 14.8925008773804 4.0
28.216036344 28.0335731506348 5.0
44.048761746 42.1424026489258 6.0
44.048761746 43.2741317749023 7.0
38.092815451 38.2420806884766 0.0
20.42584869 19.3928108215332 1.0
17.666966762 19.0491485595703 2.0
-5.2714738638 -7.20650196075439 3.0
12.395492889 11.8210115432739 4.0
25.697322563 25.2565593719482 5.0
38.09281546 37.8123321533203 6.0
38.09281546 38.8998031616211 7.0
42.616606914 38.5621795654297 0.0
22.043217204 21.0991725921631 1.0
20.57338971 20.2528762817383 2.0
-9.1142862064 -9.46041584014893 3.0
11.459103494 10.7132511138916 4.0
31.157503421 30.0427093505859 5.0
42.616606924 40.8032150268555 6.0
42.616606924 41.7413864135742 7.0
41.080227066 38.7166213989258 0.0
21.412108977 21.3951625823975 1.0
19.66811809 19.7237911224365 2.0
-8.9308117628 -9.66688823699951 3.0
10.737306318 9.92698860168457 4.0
30.342920748 29.7334499359131 5.0
41.080227074 41.443359375 6.0
41.080227074 41.4077987670898 7.0
36.859087804 38.433349609375 0.0
16.308519924 18.2599468231201 1.0
20.550567882 18.3344230651855 2.0
-8.0027710471 -6.93605136871338 3.0
12.547796826 12.0907773971558 4.0
24.31129098 23.9222564697266 5.0
36.859087813 37.3687133789062 6.0
36.859087813 37.9873733520508 7.0
39.316268866 38.0415191650391 0.0
19.479900065 19.8143157958984 1.0
19.836368801 19.0760173797607 2.0
-7.370707565 -7.87371635437012 3.0
12.465661226 11.8045310974121 4.0
26.85060764 26.5419731140137 5.0
39.316268876 39.4646453857422 6.0
39.316268876 39.6341781616211 7.0
40.023071957 38.8050231933594 0.0
21.374119573 20.7730007171631 1.0
18.648952384 18.9011993408203 2.0
-8.5239396023 -9.63755702972412 3.0
10.125012772 9.36878681182861 4.0
29.898059185 29.2027492523193 5.0
40.023071967 40.0522842407227 6.0
40.023071967 40.5714340209961 7.0
32.691039556 38.3575057983398 0.0
16.99402987 16.4254302978516 1.0
15.697009686 15.8058452606201 2.0
-6.1678662484 -7.45424652099609 3.0
9.5291434282 9.1234302520752 4.0
23.161896128 23.3277568817139 5.0
32.691039566 32.5299263000488 6.0
32.691039566 32.8125190734863 7.0
39.350172222 38.2304534912109 0.0
19.79571531 19.6176052093506 1.0
19.554456913 19.3827495574951 2.0
-6.9332448751 -7.61796092987061 3.0
12.62121203 12.0058012008667 4.0
26.728960194 26.4710826873779 5.0
39.35017223 38.9909057617188 6.0
39.35017223 39.605842590332 7.0
38.6257792205169 39.7686233520508 0.0
20.6764320695262 20.0450859069824 1.0
17.9493471475253 18.4269847869873 2.0
-8.74948254057215 -9.06752014160156 3.0
9.19986460693105 8.63833236694336 4.0
29.4259146106884 28.8499183654785 5.0
38.62577922 40.032356262207 6.0
38.62577922 38.3957214355469 7.0
43.363210173 38.3566284179688 0.0
22.264479357 21.3504753112793 1.0
21.09873082 20.8455333709717 2.0
-7.8566163895 -8.34503364562988 3.0
13.242114424 12.4946718215942 4.0
30.121095753 29.5680236816406 5.0
43.36321018 42.360481262207 6.0
43.36321018 42.765510559082 7.0
41.7113705187954 38.410530090332 0.0
19.9560794530664 20.5620975494385 1.0
21.7552910618007 20.6483879089355 2.0
-6.27295647675057 -6.27053594589233 3.0
15.4823345851289 14.6441297531128 4.0
26.2290359296866 25.2980918884277 5.0
41.711370519 40.6120376586914 6.0
41.711370519 41.5839080810547 7.0
41.4017485851654 40.012321472168 0.0
21.9191533447582 21.2350120544434 1.0
19.4825952431145 20.0624504089355 2.0
-7.53340700226978 -9.23404026031494 3.0
11.9491882404739 11.171706199646 4.0
29.4525603463383 29.4800243377686 5.0
41.401748585 41.8496017456055 6.0
41.401748585 40.5664215087891 7.0
45.411303660968 38.6011657714844 0.0
22.8793399812672 22.183198928833 1.0
22.5319636798771 21.0902271270752 2.0
-8.35657341793646 -8.23290729522705 3.0
14.1753902620561 13.4885587692261 4.0
31.2359133986188 30.6988735198975 5.0
45.411303661 42.7552947998047 6.0
45.411303661 44.0513076782227 7.0
46.173622728 38.2765502929688 0.0
22.478207281 22.1792011260986 1.0
23.695415447 21.4904689788818 2.0
-7.2335265538 -8.16776943206787 3.0
16.461888883 16.5144691467285 4.0
29.711733845 28.9169616699219 5.0
46.173622738 44.499885559082 6.0
46.173622738 45.0760192871094 7.0
43.970500883 38.4109878540039 0.0
23.319439955 22.0423831939697 1.0
20.651060929 20.3925151824951 2.0
-9.6037663307 -10.1343755722046 3.0
11.047294589 10.3460187911987 4.0
32.923206295 31.1921272277832 5.0
43.970500892 42.2292098999023 6.0
43.970500892 43.0829010009766 7.0
41.561964514 38.8353042602539 0.0
22.423587267 20.4294242858887 1.0
19.138377247 20.6321563720703 2.0
-2.7143772539 -6.24344968795776 3.0
16.423999984 16.1530227661133 4.0
25.13796453 24.6569118499756 5.0
41.561964524 41.7213363647461 6.0
41.561964524 41.050407409668 7.0
42.043570825 38.1755065917969 0.0
21.184612309 21.3031368255615 1.0
20.858958517 20.6786994934082 2.0
-7.1973175688 -7.39851951599121 3.0
13.66164094 13.0213384628296 4.0
28.381929887 27.8728427886963 5.0
42.043570834 41.2108306884766 6.0
42.043570834 41.688117980957 7.0
42.728187899 38.2817077636719 0.0
22.937429443 21.7080574035645 1.0
19.790758457 20.6395950317383 2.0
-7.3221221307 -8.65526103973389 3.0
12.468636317 11.8406934738159 4.0
30.259551583 29.384672164917 5.0
42.728187908 41.8752899169922 6.0
42.728187908 42.7050247192383 7.0
46.647822935 38.5361404418945 0.0
24.419972425 23.8600158691406 1.0
22.22785051 22.0308856964111 2.0
-9.7005341836 -9.42664051055908 3.0
12.527316316 11.4878358840942 4.0
34.120506619 33.8525314331055 5.0
46.647822945 46.0634155273438 6.0
46.647822945 47.0327758789062 7.0
34.157298452 38.6374435424805 0.0
16.904613725 17.4004230499268 1.0
17.252684728 16.6335182189941 2.0
-7.2828539724 -7.40594291687012 3.0
9.9698307451 9.54412651062012 4.0
24.187467707 23.9261646270752 5.0
34.157298462 33.9298858642578 6.0
34.157298462 34.4733428955078 7.0
38.08940785 40.2585372924805 0.0
20.151317931 18.6608867645264 1.0
17.93808992 19.6084251403809 2.0
-2.5516772312 -5.72512340545654 3.0
15.386412679 14.8432426452637 4.0
22.702995171 22.112154006958 5.0
38.08940786 39.3014678955078 6.0
38.08940786 38.1493377685547 7.0
48.253456595 39.4123382568359 0.0
26.02387816 23.8605499267578 1.0
22.229578435 22.6421127319336 2.0
-8.9125381923 -9.442946434021 3.0
13.317040233 12.2097778320312 4.0
34.936416362 33.5578842163086 5.0
48.253456605 47.5990447998047 6.0
48.253456605 47.9421081542969 7.0
41.303512945 38.0241088867188 0.0
21.842523524 20.9907550811768 1.0
19.460989421 19.5410175323486 2.0
-8.4785475097 -9.25720405578613 3.0
10.982441901 10.3447942733765 4.0
30.321071043 29.6855010986328 5.0
41.303512955 40.9467391967773 6.0
41.303512955 41.3833389282227 7.0
34.11247551 38.0281600952148 0.0
16.407323353 16.5559768676758 1.0
17.705152157 17.6839389801025 2.0
-3.4961419723 -4.1192774772644 3.0
14.209010175 13.9205007553101 4.0
19.903465335 19.9902420043945 5.0
34.11247552 33.7182388305664 6.0
34.11247552 34.1492080688477 7.0
39.346899311 38.4550399780273 0.0
20.280952021 20.4826831817627 1.0
19.06594729 18.3528308868408 2.0
-10.212334112 -9.99576377868652 3.0
8.8536131687 8.21247959136963 4.0
30.493286142 29.7933006286621 5.0
39.346899321 39.6096343994141 6.0
39.346899321 39.724723815918 7.0
42.355081312 37.3127059936523 0.0
22.237080125 22.240650177002 1.0
20.118001187 21.6637630462646 2.0
-4.2250518369 -8.08274555206299 3.0
15.892949341 15.4623432159424 4.0
26.462131971 29.0546703338623 5.0
42.355081321 45.5446014404297 6.0
42.355081321 45.1520080566406 7.0
46.02528461 37.5210113525391 0.0
20.767999035 22.6121349334717 1.0
25.257285575 21.6847133636475 2.0
-9.1629445671 -6.89923095703125 3.0
16.094340999 15.181357383728 4.0
29.930943612 28.2912178039551 5.0
46.025284619 43.1628112792969 6.0
46.025284619 44.9817733764648 7.0
48.790909777 37.9999694824219 0.0
25.145056622 23.4483261108398 1.0
23.645853155 22.4899921417236 2.0
-9.2179915631 -9.22602367401123 3.0
14.427861582 13.5308208465576 4.0
34.363048195 32.7184677124023 5.0
48.790909787 47.4653244018555 6.0
48.790909787 47.6438446044922 7.0
41.698940493 38.5718154907227 0.0
20.507660886 21.1916675567627 1.0
21.191279609 19.8242645263672 2.0
-9.1543537316 -8.90018558502197 3.0
12.036925868 11.3282794952393 4.0
29.662014626 28.7836875915527 5.0
41.698940502 40.8714141845703 6.0
41.698940502 41.4614944458008 7.0
47.902972774 37.9236907958984 0.0
23.393159022 23.3439140319824 1.0
24.509813754 23.0763282775879 2.0
-8.5476809737 -7.97155380249023 3.0
15.962132774 15.1462869644165 4.0
31.940840003 31.0478076934814 5.0
47.902972781 46.0086822509766 6.0
47.902972781 46.6090774536133 7.0
41.377170862 38.4424133300781 0.0
20.577519401 20.8493041992188 1.0
20.799651462 20.52956199646 2.0
-6.2821130824 -6.64414548873901 3.0
14.517538372 13.8285627365112 4.0
26.859632492 26.135570526123 5.0
41.37717087 41.5638656616211 6.0
41.37717087 41.2836227416992 7.0
32.285684892 38.3121643066406 0.0
14.316873201 15.6851854324341 1.0
17.968811691 16.3081245422363 2.0
-5.3227711546 -4.38966703414917 3.0
12.646040526 12.6627941131592 4.0
19.639644366 19.7672290802002 5.0
32.285684902 32.7703018188477 6.0
32.285684902 32.6642761230469 7.0
32.344206843 37.9958114624023 0.0
16.160894185 16.8645553588867 1.0
16.183312659 15.6117248535156 2.0
-7.3310896054 -7.66491508483887 3.0
8.852223045 8.37063026428223 4.0
23.491983799 23.2982540130615 5.0
32.344206851 33.25244140625 6.0
32.344206851 33.7069854736328 7.0
34.184112321 38.4180679321289 0.0
18.400062549 17.376672744751 1.0
15.784049772 16.0339813232422 2.0
-7.6583985283 -8.91202545166016 3.0
8.125651234 7.94446277618408 4.0
26.058461087 25.9154930114746 5.0
34.184112331 34.4313507080078 6.0
34.184112331 34.9890899658203 7.0
34.663221798 39.060661315918 0.0
18.206197344 17.9848823547363 1.0
16.457024454 16.6225280761719 2.0
-6.9582970888 -7.89465618133545 3.0
9.4987273548 9.08846759796143 4.0
25.164494443 24.9151744842529 5.0
34.663221808 34.8955383300781 6.0
34.663221808 36.0787048339844 7.0
36.873774533 36.4842071533203 0.0
18.579406772 18.2184753417969 1.0
18.294367761 18.4318828582764 2.0
-5.3733020341 -6.18012809753418 3.0
12.921065717 12.6412115097046 4.0
23.952708817 23.7828521728516 5.0
36.873774543 36.1619033813477 6.0
36.873774543 36.9672317504883 7.0
40.211352898 40.8574752807617 0.0
22.061687235 21.8373050689697 1.0
18.149665663 20.0331058502197 2.0
-7.4872031869 -10.1407079696655 3.0
10.662462467 9.82650661468506 4.0
29.548890431 31.1030902862549 5.0
40.211352908 42.4312973022461 6.0
40.211352908 42.6530380249023 7.0
46.160689864 38.6199569702148 0.0
24.176307096 22.1570148468018 1.0
21.984382774 22.1123676300049 2.0
-5.8472747438 -7.55188083648682 3.0
16.137108027 15.5524559020996 4.0
30.023581843 29.1270599365234 5.0
46.160689867 44.5380935668945 6.0
46.160689867 45.6408157348633 7.0
32.727702668 39.0109100341797 0.0
17.796794822 16.9581928253174 1.0
14.930907847 15.1722841262817 2.0
-5.7416146893 -6.93614101409912 3.0
9.1892931492 8.79934120178223 4.0
23.53840952 22.9070301055908 5.0
32.727702676 32.644401550293 6.0
32.727702676 33.5182876586914 7.0
40.640985417073 40.9353637695312 0.0
21.2248663380953 20.5563144683838 1.0
19.4161190831271 19.8506031036377 2.0
-6.12795167285201 -7.54850482940674 3.0
13.2881674100867 12.5912218093872 4.0
27.3528180102152 27.0946311950684 5.0
40.640985417 40.7124404907227 6.0
40.640985417 39.9781875610352 7.0
34.611451584 38.7155685424805 0.0
16.03184427 17.1442260742188 1.0
18.579607315 17.3146991729736 2.0
-5.6371786801 -5.41832494735718 3.0
12.942428625 12.5042181015015 4.0
21.66902296 21.6537895202637 5.0
34.611451594 35.0280303955078 6.0
34.611451594 35.4404067993164 7.0
43.311280798 39.3773040771484 0.0
19.747206987 20.760814666748 1.0
23.564073813 21.1719360351562 2.0
-7.5093463576 -5.80857801437378 3.0
16.054727447 15.5904731750488 4.0
27.256553353 26.2792015075684 5.0
43.311280807 41.630485534668 6.0
43.311280807 42.7580184936523 7.0
36.406364048 38.5013809204102 0.0
19.818637935 18.2263565063477 1.0
16.587726114 18.2509441375732 2.0
-3.8896932395 -6.24281692504883 3.0
12.698032865 12.240761756897 4.0
23.708331184 23.5217761993408 5.0
36.406364057 35.6973571777344 6.0
36.406364057 37.0864334106445 7.0
31.68563032 42.1182861328125 0.0
16.740289757 16.6609230041504 1.0
14.945340567 15.3443574905396 2.0
-6.0451472911 -7.26226711273193 3.0
8.9001932697 8.35330390930176 4.0
22.785437054 22.8738708496094 5.0
31.685630326 32.1779098510742 6.0
31.685630326 31.5273990631104 7.0
40.165364423 41.7417984008789 0.0
21.225216477 21.3776092529297 1.0
18.940147947 19.0668811798096 2.0
-9.5161355118 -9.72500038146973 3.0
9.4240124269 8.83594512939453 4.0
30.741351997 30.0516490936279 5.0
40.165364432 40.1582717895508 6.0
40.165364432 39.4980697631836 7.0
42.886465637 38.1925201416016 0.0
31.633244398 22.099588394165 1.0
11.253221239 20.3598155975342 2.0
0 -10.0018396377563 3.0
11.253221239 10.3876953125 4.0
31.633244398 30.4753379821777 5.0
42.886465647 42.8859329223633 6.0
42.886465647 43.4424133300781 7.0
46.942592322 39.7968139648438 0.0
23.694673267 23.3517818450928 1.0
23.247919055 22.2436904907227 2.0
-9.8881851606 -9.55183219909668 3.0
13.359733884 12.4381628036499 4.0
33.582858438 32.7497406005859 5.0
46.942592332 45.6073760986328 6.0
46.942592332 46.5753326416016 7.0
38.417030558 40.2535781860352 0.0
20.711241226 20.4441623687744 1.0
17.705789332 18.2222671508789 2.0
-7.7084466149 -9.50608062744141 3.0
9.9973427073 9.2020206451416 4.0
28.419687851 28.4029159545898 5.0
38.417030568 38.0373611450195 6.0
38.417030568 37.2130508422852 7.0
39.82159542 38.9206085205078 0.0
19.816428679 19.6259670257568 1.0
20.005166741 19.7445392608643 2.0
-5.1739908699 -6.13262510299683 3.0
14.831175861 14.4361400604248 4.0
24.990419559 24.3615665435791 5.0
39.82159543 38.943603515625 6.0
39.82159543 40.2914505004883 7.0
42.375846264 39.1178283691406 0.0
20.538210367 20.5903358459473 1.0
21.837635902 21.1823558807373 2.0
-5.3545430676 -6.22136545181274 3.0
16.483092829 15.3301210403442 4.0
25.89275344 25.6335678100586 5.0
42.375846269 40.8890533447266 6.0
42.375846269 42.2571411132812 7.0
41.909868914 38.1749801635742 0.0
21.249762031 21.4168834686279 1.0
20.660106883 18.9072437286377 2.0
-10.971309543 -10.2181768417358 3.0
9.6887973304 9.57851219177246 4.0
32.221071584 30.601146697998 5.0
41.909868924 41.6845245361328 6.0
41.909868924 42.7259216308594 7.0
38.5154038135067 38.4745330810547 0.0
19.3171244035071 18.9394989013672 1.0
19.1982794195071 17.5154647827148 2.0
-8.28760453680882 -8.4829626083374 3.0
10.9106748825063 10.6974229812622 4.0
27.6047289405071 26.8294239044189 5.0
38.515403814 37.6025390625 6.0
38.515403814 39.1212463378906 7.0
32.187464764 37.9287490844727 0.0
15.688410189 16.0790748596191 1.0
16.499054576 16.0943984985352 2.0
-5.8055743047 -5.66058874130249 3.0
10.693480262 10.3485889434814 4.0
21.493984503 21.5010547637939 5.0
32.187464773 31.9919776916504 6.0
32.187464773 31.9350719451904 7.0
43.5533404982843 39.243034362793 0.0
25.257472916292 22.6826000213623 1.0
18.2958675832873 20.0144195556641 2.0
-8.52565235008034 -10.6790180206299 3.0
9.77021523347734 8.71351432800293 4.0
33.7831252662979 33.5288772583008 5.0
43.553340498 42.7242431640625 6.0
43.553340498 41.6242828369141 7.0
38.661685108 38.3978805541992 0.0
21.023698145 22.1767997741699 1.0
17.637986964 20.4815006256104 2.0
-6.4647611773 -9.44525146484375 3.0
11.173225776 10.1704559326172 4.0
27.488459332 30.011926651001 5.0
38.661685118 43.3254928588867 6.0
38.661685118 43.201042175293 7.0
47.185976942 37.8694458007812 0.0
20.88210062 22.9890632629395 1.0
26.303876326 22.8522891998291 2.0
-9.8059530494 -7.71758937835693 3.0
16.497923272 16.7459602355957 4.0
30.688053675 29.5526866912842 5.0
47.185976948 46.2387313842773 6.0
47.185976948 46.8695526123047 7.0
41.0736498246764 38.6558151245117 0.0
17.9299575176762 20.1273136138916 1.0
23.1436923166761 20.1229667663574 2.0
-7.36188695227673 -5.97925853729248 3.0
15.7818053646765 15.4023761749268 4.0
25.2918444696756 24.1534004211426 5.0
41.073649825 39.4811935424805 6.0
41.073649825 40.7490234375 7.0
39.086114836 38.8619689941406 0.0
22.054277289 20.55735206604 1.0
17.031837547 18.5319519042969 2.0
-8.5092532605 -9.80507850646973 3.0
8.5225842768 8.27036571502686 4.0
30.563530559 29.8800678253174 5.0
39.086114846 38.731803894043 6.0
39.086114846 38.9209671020508 7.0
38.072309142 38.6814727783203 0.0
19.059692214 18.7046546936035 1.0
19.012616928 19.1962356567383 2.0
-4.5092621515 -6.01113700866699 3.0
14.503354768 14.0734481811523 4.0
23.568954375 23.2052192687988 5.0
38.072309151 37.7062454223633 6.0
38.072309151 37.8989028930664 7.0
41.585402364 38.2527008056641 0.0
21.878346027 21.2242889404297 1.0
19.707056337 20.1088809967041 2.0
-7.856520423 -8.65490341186523 3.0
11.850535904 11.2097215652466 4.0
29.73486646 29.1271724700928 5.0
41.585402374 41.1975479125977 6.0
41.585402374 41.5237426757812 7.0
42.3173178092696 38.4215469360352 0.0
23.2468941623901 22.1233310699463 1.0
19.0704236443393 18.9379272460938 2.0
-10.4383552532401 -11.0017137527466 3.0
8.63206839120634 8.19525623321533 4.0
33.6852494163203 32.4750366210938 5.0
42.317317809 42.1171417236328 6.0
42.317317809 42.6608352661133 7.0
42.931477575 39.3991851806641 0.0
22.005453055 21.6870460510254 1.0
20.926024519 21.0156536102295 2.0
-5.8784626661 -7.5215368270874 3.0
15.047561843 14.1241235733032 4.0
27.883915732 27.3795833587646 5.0
42.931477585 43.399299621582 6.0
42.931477585 43.0428009033203 7.0
46.096274403 39.2615356445312 0.0
25.32088965 22.5405292510986 1.0
20.775384755 22.2419261932373 2.0
-5.4662065709 -8.19584560394287 3.0
15.309178177 14.5654554367065 4.0
30.787096229 30.0127086639404 5.0
46.096274411 44.4457626342773 6.0
46.096274411 45.9870376586914 7.0
39.955261544 38.6042404174805 0.0
19.931762754 20.3806228637695 1.0
20.02349879 19.182035446167 2.0
-8.9110310329 -8.76078701019287 3.0
11.112467747 10.7508745193481 4.0
28.842793797 28.0816440582275 5.0
39.955261554 39.1785736083984 6.0
39.955261554 39.9708480834961 7.0
45.415961836 38.7581024169922 0.0
23.097461473 22.6445693969727 1.0
22.318500363 21.4108543395996 2.0
-8.5889928185 -9.13318157196045 3.0
13.729507535 12.8146114349365 4.0
31.686454301 30.696325302124 5.0
45.415961846 43.9062652587891 6.0
45.415961846 45.0336608886719 7.0
44.275746647 38.3189926147461 0.0
22.654009807 22.217716217041 1.0
21.62173684 21.3062992095947 2.0
-7.7547516702 -8.74852466583252 3.0
13.86698516 12.9053621292114 4.0
30.408761488 29.4190692901611 5.0
44.275746657 43.9858322143555 6.0
44.275746657 44.1960830688477 7.0
34.766238956 37.287353515625 0.0
17.943458551 17.6131191253662 1.0
16.822780406 17.1192417144775 2.0
-5.6356736314 -7.00773429870605 3.0
11.187106764 10.7948007583618 4.0
23.579132192 23.2619380950928 5.0
34.766238966 34.7213668823242 6.0
34.766238966 34.5344619750977 7.0
39.697519492335 38.7826766967773 0.0
21.2162061883195 19.4346141815186 1.0
18.481313305488 19.3157329559326 2.0
-4.67528909066386 -7.0928783416748 3.0
13.8060242148741 13.4178695678711 4.0
25.8914952793298 25.2283058166504 5.0
39.697519492 38.5190505981445 6.0
39.697519492 39.5111618041992 7.0
36.4599525996002 38.5531311035156 0.0
18.3065166475976 18.3537063598633 1.0
18.1534359526018 18.5036849975586 2.0
-5.15104061002709 -6.58333730697632 3.0
13.0023953426082 12.6136074066162 4.0
23.457557257596 23.4695262908936 5.0
36.459952599 36.2533340454102 6.0
36.459952599 37.4287109375 7.0
44.856515577 38.2222518920898 0.0
22.473710733 22.5195598602295 1.0
22.382804843 21.8416500091553 2.0
-8.0040806232 -7.90747356414795 3.0
14.37872421 13.7032289505005 4.0
30.477791367 29.1743850708008 5.0
44.856515586 44.1223602294922 6.0
44.856515586 44.4332046508789 7.0
41.1579139562054 38.4082717895508 0.0
18.5564288232053 19.9055194854736 1.0
22.601485143205 20.1454486846924 2.0
-7.31439675300561 -6.26758575439453 3.0
15.2870883902055 14.8041162490845 4.0
25.8708255762049 24.740291595459 5.0
41.157913956 39.4463882446289 6.0
41.157913956 40.2281723022461 7.0
32.092085782 42.7318878173828 0.0
17.272160967 16.5270881652832 1.0
14.819924815 15.0818977355957 2.0
-6.6255124723 -7.97371292114258 3.0
8.1944123328 7.85368251800537 4.0
23.897673449 24.0389785766602 5.0
32.092085792 31.7450618743896 6.0
32.092085792 30.3968200683594 7.0
30.327518415 38.4048843383789 0.0
13.541286609 14.6094779968262 1.0
16.786231807 15.1072158813477 2.0
-5.5133467024 -5.7421088218689 3.0
11.272885094 11.1871385574341 4.0
19.054633321 19.3803768157959 5.0
30.327518425 30.3478355407715 6.0
30.327518425 30.4830265045166 7.0
36.482977268 39.0758666992188 0.0
19.553260998 19.031644821167 1.0
16.929716271 17.6167831420898 2.0
-7.3162847454 -9.2794132232666 3.0
9.613431516 8.96224021911621 4.0
26.869545753 27.4428234100342 5.0
36.482977278 36.0645523071289 6.0
36.482977278 36.1489791870117 7.0
40.491170904 39.51025390625 0.0
20.509684159 20.8911056518555 1.0
19.981486745 19.167610168457 2.0
-8.9848735316 -10.0219697952271 3.0
10.996613204 10.0886106491089 4.0
29.4945577 30.1088333129883 5.0
40.491170914 39.7505264282227 6.0
40.491170914 39.4850921630859 7.0
41.188898324 38.781135559082 0.0
17.482572735 20.9087905883789 1.0
23.70632559 20.5050888061523 2.0
-7.86829602 -6.30735492706299 3.0
15.838029561 15.0808620452881 4.0
25.350868764 24.5638866424561 5.0
41.188898334 39.6158218383789 6.0
41.188898334 40.3856430053711 7.0
37.756949894 37.3375091552734 0.0
19.043754327 18.7420654296875 1.0
18.713195567 19.0237312316895 2.0
-5.5841962186 -6.77661228179932 3.0
13.128999338 12.5939464569092 4.0
24.627950556 24.4968547821045 5.0
37.756949904 38.0179138183594 6.0
37.756949904 37.8417816162109 7.0
26.962645791 38.6975555419922 0.0
14.231795944 13.5606527328491 1.0
12.730849849 12.0478553771973 2.0
-4.3614404899 -5.80632019042969 3.0
8.3694093501 8.35910892486572 4.0
18.593236442 18.5433235168457 5.0
26.9626458 29.0926151275635 6.0
26.9626458 27.8304920196533 7.0
44.1546808660895 38.0645523071289 0.0
20.845479755763 21.6654720306396 1.0
23.3092011082446 22.5813045501709 2.0
-6.97221748920505 -7.73952484130859 3.0
16.3369836194072 15.5416536331177 4.0
27.8176971034536 29.6074733734131 5.0
44.154680866 44.8336486816406 6.0
44.154680866 45.3158950805664 7.0
36.4992036 38.1218414306641 0.0
18.499566849 18.9169368743896 1.0
17.999636751 17.2083702087402 2.0
-8.164709678 -8.1932954788208 3.0
9.834927063 9.57297229766846 4.0
26.664276537 25.952974319458 5.0
36.49920361 36.0940704345703 6.0
36.49920361 37.3394546508789 7.0
36.136186998 39.2956619262695 0.0
19.096538919 19.0549125671387 1.0
17.039648079 17.2332229614258 2.0
-8.7939144312 -8.95334053039551 3.0
8.2457336392 8.1334285736084 4.0
27.89045336 27.9088649749756 5.0
36.136187007 35.604606628418 6.0
36.136187007 36.6808471679688 7.0
43.805905699 38.4082794189453 0.0
19.467800315 21.7324333190918 1.0
24.338105385 21.3228511810303 2.0
-10.07714477 -7.99375820159912 3.0
14.260960606 13.4028282165527 4.0
29.544945095 29.3950824737549 5.0
43.805905709 42.9095916748047 6.0
43.805905709 43.1864471435547 7.0
43.400673471 37.3553619384766 0.0
21.450145846 21.3319797515869 1.0
21.950527627 20.0062026977539 2.0
-10.718319514 -9.52839851379395 3.0
11.232208106 10.443018913269 4.0
32.168465368 30.1349048614502 5.0
43.400673478 44.3151473999023 6.0
43.400673478 42.2234802246094 7.0
35.708898922 41.3468627929688 0.0
19.860917727 18.4082355499268 1.0
15.847981195 17.1911163330078 2.0
-6.8010445999 -8.64494705200195 3.0
9.0469365848 8.26922416687012 4.0
26.661962337 26.3812522888184 5.0
35.708898932 35.3210830688477 6.0
35.708898932 34.2125473022461 7.0
42.19537597 38.6839065551758 0.0
21.611035583 21.2544460296631 1.0
20.584340387 20.4304485321045 2.0
-7.818182446 -8.66779899597168 3.0
12.766157932 11.9916820526123 4.0
29.429218038 28.7422046661377 5.0
42.195375979 41.3631820678711 6.0
42.195375979 42.2733154296875 7.0
38.219132248 38.8931350708008 0.0
18.124133621 18.5880088806152 1.0
20.094998634 19.4918537139893 2.0
-3.9675986496 -4.18937349319458 3.0
16.127399981 15.6467847824097 4.0
22.091732274 21.4282341003418 5.0
38.219132252 37.8382568359375 6.0
38.219132252 38.0583267211914 7.0
37.34785903 39.6710510253906 0.0
19.494183125 18.5862197875977 1.0
17.853675906 18.7472915649414 2.0
-4.6743476987 -6.32650375366211 3.0
13.179328197 12.6402902603149 4.0
24.168530833 23.53444480896 5.0
37.34785904 37.4297256469727 6.0
37.34785904 37.857795715332 7.0
41.413317073 38.4210968017578 0.0
22.502810393 20.7934989929199 1.0
18.91050668 20.2211971282959 2.0
-4.5479233593 -6.43856811523438 3.0
14.362583311 13.6872663497925 4.0
27.050733762 26.306791305542 5.0
41.413317083 40.7909088134766 6.0
41.413317083 41.0473098754883 7.0
39.408474003 40.3025207519531 0.0
19.787258075 19.7200088500977 1.0
19.621215928 18.9290428161621 2.0
-8.5467597404 -8.02848339080811 3.0
11.074456178 10.4783992767334 4.0
28.334017825 27.7507095336914 5.0
39.408474012 39.2541885375977 6.0
39.408474012 39.7915420532227 7.0
44.6219477930596 37.9476928710938 0.0
23.5295769810594 22.9357490539551 1.0
21.0923708210596 21.2818946838379 2.0
-9.44677516695988 -9.62642574310303 3.0
11.6455956540597 10.8971891403198 4.0
32.9763521480588 31.588306427002 5.0
44.621947793 43.3761825561523 6.0
44.621947793 44.4126815795898 7.0
35.042314664 40.4611358642578 0.0
17.377708596 17.5871887207031 1.0
17.66460607 17.1801624298096 2.0
-4.7609303256 -5.11108732223511 3.0
12.903675736 12.5969161987305 4.0
22.13863893 21.5433940887451 5.0
35.042314673 34.7478256225586 6.0
35.042314673 34.9639511108398 7.0
44.668685977 36.4160079956055 0.0
21.700098462 22.3417587280273 1.0
22.968587516 21.7795524597168 2.0
-8.4663400465 -8.06090354919434 3.0
14.50224746 13.6951875686646 4.0
30.166438518 29.3170509338379 5.0
44.668685987 43.0581436157227 6.0
44.668685987 44.2628860473633 7.0
44.271188228 37.6531753540039 0.0
21.134860564 21.8332481384277 1.0
23.136327666 22.3055229187012 2.0
-6.9953395415 -6.5513596534729 3.0
16.140988117 15.0220232009888 4.0
28.130200113 27.2446060180664 5.0
44.271188235 44.1387329101562 6.0
44.271188235 43.6082611083984 7.0
43.3296157145675 39.0664367675781 0.0
21.8144364805029 21.656759262085 1.0
21.5151792385451 20.2940578460693 2.0
-10.3071776780671 -9.62773132324219 3.0
11.2080015596363 10.501633644104 4.0
32.1216141588196 30.5324611663818 5.0
43.329615715 41.9489364624023 6.0
43.329615715 42.7121810913086 7.0
37.727034885 40.9678497314453 0.0
16.740902708 19.1518745422363 1.0
20.986132178 18.1882343292236 2.0
-10.156378742 -7.44830799102783 3.0
10.829753428 10.1057720184326 4.0
26.897281458 26.3376235961914 5.0
37.727034893 37.7822113037109 6.0
37.727034893 36.8564376831055 7.0
43.665312103 38.6473083496094 0.0
21.432332029 21.7133026123047 1.0
22.232980074 21.0747184753418 2.0
-7.2001290055 -6.87165451049805 3.0
15.032851059 14.5441732406616 4.0
28.632461044 28.4710445404053 5.0
43.665312113 42.2714920043945 6.0
43.665312113 43.1181564331055 7.0
40.6563483834553 38.2322006225586 0.0
20.9212538534144 22.2874374389648 1.0
19.7350945303589 20.575080871582 2.0
-7.59696943049863 -9.7576961517334 3.0
12.1381250996025 11.3042478561401 4.0
28.5182231244438 30.6564598083496 5.0
40.656348383 42.8358840942383 6.0
40.656348383 43.5187072753906 7.0
30.759139545 38.3696823120117 0.0
13.433268738 15.0255012512207 1.0
17.325870807 15.4236707687378 2.0
-5.9875785844 -4.64086389541626 3.0
11.338292213 11.294243812561 4.0
19.420847332 19.5131206512451 5.0
30.759139555 31.2909145355225 6.0
30.759139555 31.389928817749 7.0
36.474229645 40.226318359375 0.0
14.989459641 17.8264293670654 1.0
21.484770007 19.1009178161621 2.0
-5.0219280476 -3.41641426086426 3.0
16.462841952 16.4279747009277 4.0
20.011387696 18.6752967834473 5.0
36.474229652 35.9620742797852 6.0
36.474229652 35.4279861450195 7.0
33.340721185 38.9816360473633 0.0
15.596021276 16.8473777770996 1.0
17.744699909 16.3874359130859 2.0
-7.4570294442 -6.50424766540527 3.0
10.287670455 9.99333381652832 4.0
23.05305073 23.0990505218506 5.0
33.340721195 34.0192718505859 6.0
33.340721195 32.3213653564453 7.0
34.143349258 40.9910354614258 0.0
15.417033592 17.2636260986328 1.0
18.726315666 17.2722606658936 2.0
-5.8334622114 -4.53218936920166 3.0
12.892853445 13.1548357009888 4.0
21.250495814 20.2220420837402 5.0
34.143349268 33.7969512939453 6.0
34.143349268 34.1047210693359 7.0
40.763829748 38.9162445068359 0.0
20.589059268 21.5375442504883 1.0
20.174770481 18.90452003479 2.0
-11.800756688 -10.6801910400391 3.0
8.374013783 8.09120941162109 4.0
32.389815966 31.1681251525879 5.0
40.763829757 41.1318054199219 6.0
40.763829757 41.0230712890625 7.0
29.788224949 39.1653823852539 0.0
14.329239328 15.0021514892578 1.0
15.458985621 14.4682674407959 2.0
-6.2969425219 -6.21877908706665 3.0
9.1620430892 8.91473579406738 4.0
20.62618186 20.7115249633789 5.0
29.788224958 30.6734504699707 6.0
29.788224958 29.9834861755371 7.0
34.675869546 37.6262817382812 0.0
17.321679699 17.2443523406982 1.0
17.354189848 16.9586791992188 2.0
-5.026706589 -5.94963693618774 3.0
12.32748325 11.9344501495361 4.0
22.348386296 21.9416027069092 5.0
34.675869554 35.6226425170898 6.0
34.675869554 35.7264633178711 7.0
40.7238558279895 39.4912109375 0.0
19.1061702059935 21.142370223999 1.0
21.6176856259914 20.7369403839111 2.0
-8.60840523789577 -7.55541706085205 3.0
13.0092803879935 12.2754936218262 4.0
27.7145752997044 28.849515914917 5.0
40.723855828 42.8596496582031 6.0
40.723855828 42.3968048095703 7.0
41.414838344 37.8283233642578 0.0
21.79665744 21.0733413696289 1.0
19.618180905 18.9981441497803 2.0
-10.700829677 -10.7724571228027 3.0
8.9173512199 8.26488208770752 4.0
32.497487126 30.8184375762939 5.0
41.414838352 41.5673294067383 6.0
41.414838352 41.5150299072266 7.0
33.391441204 37.6373596191406 0.0
16.946662464 16.966480255127 1.0
16.44477874 16.3719825744629 2.0
-6.5811351715 -7.87235927581787 3.0
9.8636435585 9.08365631103516 4.0
23.527797645 23.7306652069092 5.0
33.391441214 33.2765502929688 6.0
33.391441214 32.7680435180664 7.0
36.660645376 37.8180084228516 0.0
17.87289688 18.109561920166 1.0
18.787748497 17.614595413208 2.0
-6.6425323884 -6.07526779174805 3.0
12.1452161 11.9890327453613 4.0
24.515429278 23.5778293609619 5.0
36.660645385 36.2939300537109 6.0
36.660645385 36.242317199707 7.0
33.547567157 39.7165679931641 0.0
16.605704713 17.7154903411865 1.0
16.941862445 15.9527902603149 2.0
-8.1253896728 -7.4156379699707 3.0
8.8164727625 8.34013557434082 4.0
24.731094395 24.3980903625488 5.0
33.547567166 34.0128707885742 6.0
33.547567166 32.6779708862305 7.0
36.717923485 39.2869567871094 0.0
19.502688668 18.7698841094971 1.0
17.215234818 17.8815155029297 2.0
-6.0032006006 -7.53859424591064 3.0
11.212034207 10.7419080734253 4.0
25.505889278 25.0586738586426 5.0
36.717923495 37.1936569213867 6.0
36.717923495 37.5066909790039 7.0
44.575737489 40.5854110717773 0.0
24.24689991 22.124979019165 1.0
20.328837584 22.1266841888428 2.0
-6.0388557339 -8.2476863861084 3.0
14.289981845 13.3454523086548 4.0
30.285755649 29.336555480957 5.0
44.575737494 45.5420227050781 6.0
44.575737494 43.1909103393555 7.0
43.091610313 38.3678588867188 0.0
22.329030269 21.560863494873 1.0
20.762580045 20.8383655548096 2.0
-7.1251861541 -8.47113418579102 3.0
13.637393882 12.9377851486206 4.0
29.454216432 28.8676280975342 5.0
43.091610321 42.3122711181641 6.0
43.091610321 43.2684097290039 7.0
42.647069964 37.9135208129883 0.0
21.774885881 21.5169792175293 1.0
20.872184083 20.7233734130859 2.0
-7.6954983452 -8.57610130310059 3.0
13.176685728 12.4285440444946 4.0
29.470384236 29.0644855499268 5.0
42.647069973 41.9329528808594 6.0
42.647069973 42.6456298828125 7.0
36.24022533 38.1860504150391 0.0
17.188554164 17.9934139251709 1.0
19.051671166 18.1627101898193 2.0
-6.1891251621 -6.36701965332031 3.0
12.862545995 12.3815546035767 4.0
23.377679336 22.9789581298828 5.0
36.240225339 36.4092025756836 6.0
36.240225339 36.9738311767578 7.0
31.876756195 38.5105590820312 0.0
14.446454361 16.0429267883301 1.0
17.430301835 15.9513492584229 2.0
-5.0545683645 -4.60816955566406 3.0
12.375733461 12.1305341720581 4.0
19.501022735 18.7645626068115 5.0
31.876756203 32.8788146972656 6.0
31.876756203 33.262825012207 7.0
38.711049649 38.8478851318359 0.0
19.068925854 18.7625312805176 1.0
19.642123794 19.1813526153564 2.0
-5.0802825502 -6.32479190826416 3.0
14.561841234 13.9412775039673 4.0
24.149208415 23.7705402374268 5.0
38.711049659 37.8623046875 6.0
38.711049659 38.0536041259766 7.0
33.592635662 38.2101058959961 0.0
16.042625471 16.7317085266113 1.0
17.550010192 16.7213745117188 2.0
-5.9813560668 -6.13964128494263 3.0
11.568654115 11.3106861114502 4.0
22.023981548 21.8307762145996 5.0
33.592635672 34.0051498413086 6.0
33.592635672 33.7525329589844 7.0
31.9394970213296 38.4451141357422 0.0
15.7221064593445 15.8866930007935 1.0
16.2173905613348 16.2083702087402 2.0
-3.67606552949996 -4.31400108337402 3.0
12.5413250313355 12.2592344284058 4.0
19.3981719893576 19.1507167816162 5.0
31.939497021 32.6907539367676 6.0
31.939497021 33.176887512207 7.0
32.605677366 39.1546630859375 0.0
15.053063258 16.0942420959473 1.0
17.552614114 16.665340423584 2.0
-5.3475005701 -5.25166940689087 3.0
12.20511354 11.8578281402588 4.0
20.400563832 20.0710620880127 5.0
32.605677369 33.5073165893555 6.0
32.605677369 33.7071304321289 7.0
35.02561424 37.8257446289062 0.0
18.375955543 17.9033374786377 1.0
16.649658698 17.3728160858154 2.0
-5.1216145805 -6.55123424530029 3.0
11.528044108 11.2901048660278 4.0
23.497570133 22.6965236663818 5.0
35.02561425 35.544075012207 6.0
35.02561425 35.8606414794922 7.0
29.155020361 38.2448348999023 0.0
13.928860816 14.4688167572021 1.0
15.226159546 14.524450302124 2.0
-5.2043069028 -4.99999237060547 3.0
10.021852634 10.0219116210938 4.0
19.133167728 19.0475940704346 5.0
29.155020371 30.7879123687744 6.0
29.155020371 30.5458068847656 7.0
29.979099891 36.8306045532227 0.0
16.392071584 14.8867559432983 1.0
13.587028308 14.4228944778442 2.0
-4.5603535805 -7.07263088226318 3.0
9.026674718 8.6599006652832 4.0
20.952425173 21.9011154174805 5.0
29.9790999 31.8999938964844 6.0
29.9790999 29.1347827911377 7.0
40.412374419 38.1481323242188 0.0
21.632869342 19.2404441833496 1.0
18.779505078 19.3067245483398 2.0
-4.6354756287 -7.43406772613525 3.0
14.14402944 13.6475553512573 4.0
26.26834498 26.1334705352783 5.0
40.412374428 39.5415802001953 6.0
40.412374428 39.9898986816406 7.0
44.7996696066759 37.9980392456055 0.0
24.3565352806759 21.7528915405273 1.0
20.4431343366759 21.9079704284668 2.0
-4.33926525477587 -6.89923667907715 3.0
16.1038690816759 15.8342761993408 4.0
28.6958005009671 27.6428508758545 5.0
44.799669607 44.8251876831055 6.0
44.799669607 44.7814636230469 7.0
35.691930078 39.358154296875 0.0
18.173171081 17.9495601654053 1.0
17.518759 18.1219100952148 2.0
-4.3926482071 -5.22031116485596 3.0
13.126110785 12.507830619812 4.0
22.565819295 22.0303707122803 5.0
35.691930085 36.1713790893555 6.0
35.691930085 36.1316299438477 7.0
37.293002373 37.5319519042969 0.0
17.018201783 18.5514183044434 1.0
20.274800591 17.8870372772217 2.0
-8.9983650782 -7.6791410446167 3.0
11.276435503 10.9807929992676 4.0
26.016566871 25.8867511749268 5.0
37.293002382 36.7497329711914 6.0
37.293002382 37.4043655395508 7.0
31.749050626 38.4727096557617 0.0
16.61083934 15.8307504653931 1.0
15.138211288 15.9538488388062 2.0
-3.9083704158 -5.61944961547852 3.0
11.229840863 10.8547801971436 4.0
20.519209764 19.7830390930176 5.0
31.749050635 32.5288543701172 6.0
31.749050635 32.7002868652344 7.0
44.43859931 38.7235336303711 0.0
22.567248059 22.5149822235107 1.0
21.871351255 20.7959957122803 2.0
-10.274790847 -9.63494777679443 3.0
11.596560402 10.814507484436 4.0
32.842038912 31.5965003967285 5.0
44.438599316 41.4154891967773 6.0
44.438599316 43.2081680297852 7.0
30.584762336 36.7992782592773 0.0
14.448986708 15.5870504379272 1.0
16.13577563 14.6688270568848 2.0
-7.0107275854 -6.67455816268921 3.0
9.1250480359 9.00351905822754 4.0
21.459714302 21.4524440765381 5.0
30.584762345 32.0151710510254 6.0
30.584762345 30.9605445861816 7.0
36.167711461 40.1857681274414 0.0
15.702748451 17.8237361907959 1.0
20.464963013 18.4717121124268 2.0
-4.5943051912 -3.68921661376953 3.0
15.870657815 15.2465696334839 4.0
20.297053649 19.1244277954102 5.0
36.167711467 36.386474609375 6.0
36.167711467 36.5115127563477 7.0
45.589023155 38.3743133544922 0.0
22.43199889 22.9356861114502 1.0
23.157024265 21.4268608093262 2.0
-9.6703339461 -8.2329273223877 3.0
13.486690309 12.9158134460449 4.0
32.102332846 30.6000347137451 5.0
45.589023165 44.1303176879883 6.0
45.589023165 45.0658645629883 7.0
33.954863965 38.4884490966797 0.0
16.569146941 17.1004333496094 1.0
17.385717024 15.977596282959 2.0
-8.6921868392 -9.10738182067871 3.0
8.6935301746 8.16158390045166 4.0
25.26133379 25.3978309631348 5.0
33.954863975 32.5750885009766 6.0
33.954863975 33.2103843688965 7.0
39.104021209 38.1976928710938 0.0
19.726585997 19.6325626373291 1.0
19.377435214 19.6350040435791 2.0
-5.8707549184 -7.07136917114258 3.0
13.506680288 12.9698495864868 4.0
25.597340923 25.1903820037842 5.0
39.104021217 39.1459655761719 6.0
39.104021217 39.5250015258789 7.0
41.6151393745108 39.4013900756836 0.0
21.6144474948183 23.5700511932373 1.0
20.0006918800463 22.4477863311768 2.0
-5.68615976863137 -9.3006649017334 3.0
14.3145321112399 13.2300615310669 4.0
27.3006071309532 31.0678310394287 5.0
41.615139375 47.0245132446289 6.0
41.615139375 46.963737487793 7.0
29.251209159 37.5830535888672 0.0
16.305978933 14.7973585128784 1.0
12.945230226 13.9417028427124 2.0
-4.8421763952 -6.6864709854126 3.0
8.1030538205 7.87594175338745 4.0
21.148155339 21.0361042022705 5.0
29.251209169 30.5941467285156 6.0
29.251209169 29.7554321289062 7.0
39.145261593 38.5635375976562 0.0
20.853611949 19.8906574249268 1.0
18.291649647 18.9076118469238 2.0
-6.3281983222 -7.84693908691406 3.0
11.963451316 11.5609912872314 4.0
27.181810279 26.8756942749023 5.0
39.145261601 38.62890625 6.0
39.145261601 39.4422607421875 7.0
43.724223504 38.7131423950195 0.0
24.240212841 21.9978733062744 1.0
19.484010663 20.526388168335 2.0
-6.7543223688 -9.00374126434326 3.0
12.729688284 12.0810670852661 4.0
30.99453522 29.9828624725342 5.0
43.724223514 42.9493255615234 6.0
43.724223514 43.0419464111328 7.0
30.694427938 38.4703216552734 0.0
15.962392584 15.455174446106 1.0
14.732035355 14.5811805725098 2.0
-5.8529078341 -7.19793224334717 3.0
8.8791275119 8.61276054382324 4.0
21.815300427 22.3543128967285 5.0
30.694427948 30.4355545043945 6.0
30.694427948 30.8030529022217 7.0
46.899446071 41.2384796142578 0.0
22.810606072 23.9734020233154 1.0
24.08884 22.4933032989502 2.0
-12.207831656 -9.55136775970459 3.0
11.881008335 10.8042459487915 4.0
35.018437738 33.4036178588867 5.0
46.899446079 48.6947555541992 6.0
46.899446079 46.1362533569336 7.0
30.801741864 37.6285858154297 0.0
14.581609416 15.9768629074097 1.0
16.220132448 14.6345262527466 2.0
-7.6078809929 -6.72684097290039 3.0
8.6122514454 8.90434455871582 4.0
22.189490419 21.7099018096924 5.0
30.801741874 31.9071769714355 6.0
30.801741874 31.7369689941406 7.0
40.3889007570329 38.0543670654297 0.0
21.8864546730297 20.3379878997803 1.0
18.5024460860314 18.9957790374756 2.0
-8.35977776734346 -9.02691745758057 3.0
10.1426683190367 9.57519721984863 4.0
30.2462322648819 29.0913619995117 5.0
40.388900757 39.9377670288086 6.0
40.388900757 40.1006011962891 7.0
35.801686121 37.8537368774414 0.0
19.152651215 17.5728244781494 1.0
16.649034906 17.9854488372803 2.0
-1.2326560583 -3.47315120697021 3.0
15.416378838 14.7052917480469 4.0
20.385307283 19.0002346038818 5.0
35.801686131 35.7912902832031 6.0
35.801686131 36.2294082641602 7.0
43.275430292 37.515022277832 0.0
21.49768275 21.9793605804443 1.0
21.777747542 20.6276302337646 2.0
-8.5057364073 -8.83036231994629 3.0
13.272011125 12.5774688720703 4.0
30.003419167 29.1209125518799 5.0
43.275430302 43.2871246337891 6.0
43.275430302 43.1338653564453 7.0
36.6745338859035 38.2283172607422 0.0
17.2258453439028 18.5150699615479 1.0
19.4486885479032 18.3444747924805 2.0
-7.45674864010256 -7.02922916412354 3.0
11.9919399079021 11.4609842300415 4.0
24.6825939849038 24.3843479156494 5.0
36.674533885 36.7487487792969 6.0
36.674533885 37.4470138549805 7.0
42.2919299411196 38.5026702880859 0.0
20.8701371011196 20.9765014648438 1.0
21.4217928491193 20.4122142791748 2.0
-7.8990978219197 -8.14122104644775 3.0
13.5226950281196 12.5381097793579 4.0
28.769234923119 28.1013698577881 5.0
42.291929941 40.9769821166992 6.0
42.291929941 41.6912384033203 7.0
37.80686854 37.7174224853516 0.0
18.84529386 18.549186706543 1.0
18.961574682 19.310604095459 2.0
-3.0668288295 -4.97530174255371 3.0
15.894745843 15.2099599838257 4.0
21.912122698 21.1462707519531 5.0
37.80686855 38.2931594848633 6.0
37.80686855 37.5266723632812 7.0
36.296445798 38.1344528198242 0.0
19.144127702 17.7966156005859 1.0
17.152318097 18.110237121582 2.0
-4.4799623055 -6.88712787628174 3.0
12.672355782 12.2650785446167 4.0
23.624090017 23.5981693267822 5.0
36.296445808 35.3136367797852 6.0
36.296445808 36.5476531982422 7.0
26.786347975 42.6801834106445 0.0
13.739106858 13.5130414962769 1.0
13.047241117 13.2145490646362 2.0
-5.0192652199 -5.87567567825317 3.0
8.0279758877 7.60976409912109 4.0
18.758372088 18.9660835266113 5.0
26.786347985 28.9723720550537 6.0
26.786347985 28.2391414642334 7.0
38.715518927 38.507682800293 0.0
20.971478582 19.7143726348877 1.0
17.744040345 18.1030349731445 2.0
-7.5775969749 -9.30812072753906 3.0
10.16644336 9.681884765625 4.0
28.549075567 27.826286315918 5.0
38.715518937 37.4333877563477 6.0
38.715518937 38.1795654296875 7.0
35.505062995 38.513427734375 0.0
16.975132375 17.2819538116455 1.0
18.529930624 17.7054233551025 2.0
-6.1934391103 -6.34982109069824 3.0
12.336491507 12.0390348434448 4.0
23.168571492 23.276496887207 5.0
35.505063002 35.3272552490234 6.0
35.505063002 35.8840560913086 7.0
44.227649061 38.7107391357422 0.0
23.455343466 22.3802471160889 1.0
20.772305597 21.5785312652588 2.0
-7.272984789 -9.04497814178467 3.0
13.499320798 12.4648637771606 4.0
30.728328264 30.2675457000732 5.0
44.227649071 44.5139617919922 6.0
44.227649071 43.0474395751953 7.0
45.773124246 38.4521026611328 0.0
24.134680774 22.6842613220215 1.0
21.638443473 22.3891849517822 2.0
-6.2883820036 -7.2651834487915 3.0
15.350061461 14.9112253189087 4.0
30.423062786 29.5425624847412 5.0
45.773124255 43.8929138183594 6.0
45.773124255 44.469352722168 7.0
38.6152827522378 39.1381759643555 0.0
18.6636500942378 21.1879348754883 1.0
19.9516326682378 20.1070251464844 2.0
-8.33253161363777 -9.0809211730957 3.0
11.6191010542378 10.7593660354614 4.0
26.9961816644691 29.2537994384766 5.0
38.615282752 41.3080062866211 6.0
38.615282752 41.2782516479492 7.0
42.131124826 38.4854736328125 0.0
21.88265512 20.588752746582 1.0
20.248469706 20.2322254180908 2.0
-6.067617241 -7.61580181121826 3.0
14.180852455 13.8058958053589 4.0
27.950272371 27.9883079528809 5.0
42.131124836 40.829704284668 6.0
42.131124836 41.1903457641602 7.0
38.030369756 38.8973083496094 0.0
19.698625144 19.6794357299805 1.0
18.331744611 18.5008449554443 2.0
-8.0645153211 -8.2222261428833 3.0
10.26722928 9.75477409362793 4.0
27.763140475 27.2713928222656 5.0
38.030369766 38.3273696899414 6.0
38.030369766 38.1484527587891 7.0
32.694869995 38.2904281616211 0.0
14.429732866 16.1712837219238 1.0
18.265137131 16.580057144165 2.0
-5.8301144142 -4.95665502548218 3.0
12.435022709 12.3130111694336 4.0
20.259847288 19.9836120605469 5.0
32.694870002 33.1489868164062 6.0
32.694870002 33.3169479370117 7.0
41.58883183 41.6641159057617 0.0
19.184872754 20.3355979919434 1.0
22.403959077 20.9348182678223 2.0
-6.437067104 -5.56141662597656 3.0
15.966891964 15.7411727905273 4.0
25.621939867 24.977367401123 5.0
41.58883184 41.1224365234375 6.0
41.58883184 41.019416809082 7.0
33.622063035 38.832878112793 0.0
16.212159516 17.3043212890625 1.0
17.409903519 15.6869668960571 2.0
-8.2486263793 -7.31462860107422 3.0
9.16127713 8.79054832458496 4.0
24.460785905 24.6647968292236 5.0
33.622063045 33.4089508056641 6.0
33.622063045 32.7049369812012 7.0
40.101738497 39.5207138061523 0.0
22.613544165 20.7906303405762 1.0
17.488194333 18.8444232940674 2.0
-8.294425066 -10.4181060791016 3.0
9.1937692583 8.37910270690918 4.0
30.90796924 30.3288879394531 5.0
40.101738506 39.3720703125 6.0
40.101738506 40.6680603027344 7.0
39.347750484 38.0299377441406 0.0
19.631600055 20.4716777801514 1.0
19.716150429 19.8641872406006 2.0
-7.2110260491 -7.61504364013672 3.0
12.50512437 11.6787710189819 4.0
26.842626114 27.9090576171875 5.0
39.347750494 40.1459808349609 6.0
39.347750494 41.011962890625 7.0
46.240008453 38.8345947265625 0.0
23.277257726 22.0742149353027 1.0
22.962750727 22.1802196502686 2.0
-6.5077704351 -7.46328449249268 3.0
16.454980282 15.5035314559937 4.0
29.785028171 28.4560871124268 5.0
46.240008463 44.5308380126953 6.0
46.240008463 45.5188598632812 7.0
43.885613963 38.0525588989258 0.0
23.043389811 21.3101081848145 1.0
20.842224153 21.1093502044678 2.0
-5.1243468199 -6.55911016464233 3.0
15.717877324 15.2683639526367 4.0
28.16773664 27.1201992034912 5.0
43.885613972 43.3039321899414 6.0
43.885613972 43.0225067138672 7.0
44.877286045 38.8000869750977 0.0
23.85937511 22.7260208129883 1.0
21.017910935 20.5420475006104 2.0
-10.548874854 -10.5124816894531 3.0
10.469036071 9.67226791381836 4.0
34.408249974 32.6380271911621 5.0
44.877286055 43.9720916748047 6.0
44.877286055 44.9793243408203 7.0
44.28102484 38.0077743530273 0.0
23.476033214 22.1387367248535 1.0
20.804991626 21.0905380249023 2.0
-7.5751269732 -9.15844440460205 3.0
13.229864644 12.7179937362671 4.0
31.051160197 30.1925029754639 5.0
44.281024849 43.0857620239258 6.0
44.281024849 43.9228057861328 7.0
39.443195324 37.3810882568359 0.0
17.235953612 19.3434944152832 1.0
22.207241712 18.8450679779053 2.0
-9.075573722 -7.60132884979248 3.0
13.13166798 12.7153692245483 4.0
26.311527344 25.5991306304932 5.0
39.443195334 38.8532257080078 6.0
39.443195334 38.8245544433594 7.0
44.5099801195094 36.8591995239258 0.0
20.7192442195094 22.3555526733398 1.0
23.7907359105096 21.7317199707031 2.0
-9.37764966100969 -7.63202953338623 3.0
14.4130862495096 13.609694480896 4.0
30.0968938805093 28.47580909729 5.0
44.50998012 42.7778625488281 6.0
44.50998012 43.8332748413086 7.0
36.079576813 38.4415283203125 0.0
18.198560719 18.7957248687744 1.0
17.881016095 17.4483032226562 2.0
-8.6789161085 -9.28349304199219 3.0
9.202099978 8.59134674072266 4.0
26.877476836 27.5283660888672 5.0
36.079576822 36.163818359375 6.0
36.079576822 36.8237457275391 7.0
43.881758866 39.027214050293 0.0
21.090009056 25.6206130981445 1.0
22.79174981 23.5201053619385 2.0
-6.6710954927 -8.04027843475342 3.0
16.120654308 14.9540529251099 4.0
27.761104558 30.4392051696777 5.0
43.881758876 46.893798828125 6.0
43.881758876 49.5017700195312 7.0
30.108818274 38.0872573852539 0.0
17.112174764 14.8953723907471 1.0
12.996643511 14.1075763702393 2.0
-4.8964528045 -7.41787242889404 3.0
8.1001906976 8.07676124572754 4.0
22.008627577 22.1093921661377 5.0
30.108818284 30.7107715606689 6.0
30.108818284 29.7873954772949 7.0
34.2205121900012 38.3739700317383 0.0
18.8110553650029 17.6508178710938 1.0
15.4094568300067 16.343204498291 2.0
-7.05168531729669 -9.02465915679932 3.0
8.35777151259514 8.10700225830078 4.0
25.8627406830075 26.1589450836182 5.0
34.22051219 33.3663787841797 6.0
34.22051219 34.2508392333984 7.0
37.402197962 37.5958633422852 0.0
19.960548139 19.5640735626221 1.0
17.441649823 18.4205322265625 2.0
-5.9560955294 -7.01253890991211 3.0
11.485554285 10.9263305664062 4.0
25.916643678 25.4087047576904 5.0
37.402197971 38.1326751708984 6.0
37.402197971 37.9016265869141 7.0
35.453722399 38.0886306762695 0.0
16.625051972 17.825475692749 1.0
18.828670427 17.2883014678955 2.0
-7.5918246106 -7.05069732666016 3.0
11.236845807 10.8335590362549 4.0
24.216876592 23.7833080291748 5.0
35.453722409 34.7095642089844 6.0
35.453722409 35.6290512084961 7.0
32.872381145 38.9959030151367 0.0
15.679111706 16.1414489746094 1.0
17.193269439 16.6131019592285 2.0
-6.3769976884 -6.32965803146362 3.0
10.816271741 10.4307107925415 4.0
22.056109405 22.1622524261475 5.0
32.872381155 32.7730331420898 6.0
32.872381155 33.1625518798828 7.0
33.5777143162235 38.2724685668945 0.0
24.3043372519187 17.6195983886719 1.0
9.27337706863705 16.3525238037109 2.0
-1.413226005e-07 -8.37569046020508 3.0
9.27337692732672 8.6875524520874 4.0
24.3043373936238 24.3537464141846 5.0
33.577714316 33.2163162231445 6.0
33.577714316 33.0355224609375 7.0
43.986168792 39.041877746582 0.0
19.73252431 21.5856075286865 1.0
24.253644482 20.8663463592529 2.0
-7.8587048092 -5.91611337661743 3.0
16.394939663 15.5781421661377 4.0
27.591229129 26.2266521453857 5.0
43.986168802 41.7158050537109 6.0
43.986168802 42.3371353149414 7.0
39.701527196 38.9045562744141 0.0
21.433905404 19.840934753418 1.0
18.267621793 19.1758861541748 2.0
-6.9530592881 -8.71702003479004 3.0
11.314562495 10.6607789993286 4.0
28.386964701 28.2417583465576 5.0
39.701527205 39.0857620239258 6.0
39.701527205 39.897834777832 7.0
36.319422307 37.8450088500977 0.0
18.261170173 19.0830535888672 1.0
18.058252134 17.1586532592773 2.0
-9.3076661868 -8.92712688446045 3.0
8.7505859373 8.37872791290283 4.0
27.56883637 27.2055644989014 5.0
36.319422317 36.7306365966797 6.0
36.319422317 37.2687530517578 7.0
39.546529711 38.6017227172852 0.0
21.883828824 20.6165561676025 1.0
17.662700888 19.2193470001221 2.0
-6.2516235274 -7.78132343292236 3.0
11.411077352 10.8421440124512 4.0
28.13545236 27.5925903320312 5.0
39.54652972 39.2355422973633 6.0
39.54652972 39.8413543701172 7.0
46.328884399 38.8549118041992 0.0
20.182008475 22.8751964569092 1.0
26.146875933 22.5814533233643 2.0
-10.473669833 -8.06719207763672 3.0
15.673206099 14.8391761779785 4.0
30.655678309 29.4269580841064 5.0
46.3288844 45.0779724121094 6.0
46.3288844 46.2499313354492 7.0
39.359671453 38.194091796875 0.0
20.208221774 20.3423404693604 1.0
19.151449681 19.1097717285156 2.0
-7.7565035076 -8.42878341674805 3.0
11.394946165 10.7688608169556 4.0
27.964725289 27.7421894073486 5.0
39.359671461 39.5275573730469 6.0
39.359671461 40.1919708251953 7.0
43.813870978 38.8582763671875 0.0
22.861832963 22.0358791351318 1.0
20.952038016 21.2789440155029 2.0
-6.6064627953 -8.3498363494873 3.0
14.345575211 13.5406513214111 4.0
29.468295768 29.1588001251221 5.0
43.813870988 43.1231842041016 6.0
43.813870988 42.5593948364258 7.0
39.146221866 37.9765167236328 0.0
21.274977313 20.3604354858398 1.0
17.871244553 18.3191699981689 2.0
-8.221666714 -9.06628322601318 3.0
9.6495778292 9.37703990936279 4.0
29.496644037 28.4865341186523 5.0
39.146221876 38.9574432373047 6.0
39.146221876 39.3885269165039 7.0
34.753353004 43.2433700561523 0.0
19.306587872 19.1949615478516 1.0
15.446765133 17.0121841430664 2.0
-7.4077231802 -8.44002819061279 3.0
8.0390419425 7.87218189239502 4.0
26.714311062 26.8391723632812 5.0
34.753353014 35.7559432983398 6.0
34.753353014 35.3633346557617 7.0
36.8589150083515 39.3040237426758 0.0
18.5606086093515 19.5789794921875 1.0
18.2983064083517 18.169942855835 2.0
-8.423068234052 -8.68423557281494 3.0
9.87523817465195 9.27243614196777 4.0
26.9836767778432 27.2485866546631 5.0
36.858915008 37.5935211181641 6.0
36.858915008 38.6410980224609 7.0
45.609388715 38.4172973632812 0.0
20.431056468 22.9536399841309 1.0
25.178332247 22.4702548980713 2.0
-8.8685199312 -6.5566611289978 3.0
16.309812306 15.7383069992065 4.0
29.299576409 28.2491645812988 5.0
45.609388725 43.5644912719727 6.0
45.609388725 45.0015563964844 7.0
37.221502305 38.3516006469727 0.0
19.795691461 19.6236991882324 1.0
17.425810844 17.979154586792 2.0
-8.0004149655 -8.99427700042725 3.0
9.4253958691 8.80390930175781 4.0
27.796106436 27.9256324768066 5.0
37.221502314 37.9484024047852 6.0
37.221502314 37.553581237793 7.0
39.3512418478291 37.6393203735352 0.0
21.5003115557496 20.3177757263184 1.0
17.8509302889164 18.7557277679443 2.0
-7.9913457065862 -8.93670558929443 3.0
9.85958458276425 9.30246639251709 4.0
29.4916572628094 28.8016185760498 5.0
39.351241848 38.7210159301758 6.0
39.351241848 39.3560943603516 7.0
44.354286129 38.2987594604492 0.0
23.778468789 22.5706043243408 1.0
20.57581734 21.4556121826172 2.0
-6.7304183908 -8.56882190704346 3.0
13.845398939 13.29296875 4.0
30.50888719 29.5489082336426 5.0
44.354286139 43.3493347167969 6.0
44.354286139 44.251838684082 7.0
41.1438439687938 38.1038208007812 0.0
20.0722524531884 20.0741729736328 1.0
21.0715915195 19.5480842590332 2.0
-7.89338141352269 -8.26870822906494 3.0
13.1782101064826 12.4874887466431 4.0
27.9656338666077 27.7842121124268 5.0
41.143843969 40.4757919311523 6.0
41.143843969 40.8191757202148 7.0
32.436363662 39.0113754272461 0.0
18.29560079 16.6161632537842 1.0
14.140762872 16.010425567627 2.0
-4.3766657499 -6.25706720352173 3.0
9.7640971121 9.62257289886475 4.0
22.67226655 22.4643726348877 5.0
32.436363672 33.1113319396973 6.0
32.436363672 33.7142486572266 7.0
35.980268817 38.7608642578125 0.0
18.371260578 19.0526332855225 1.0
17.609008241 17.4489593505859 2.0
-7.5074447835 -7.72255992889404 3.0
10.101563449 9.86926555633545 4.0
25.87870537 25.642749786377 5.0
35.980268826 38.5067367553711 6.0
35.980268826 37.0414428710938 7.0
39.361696941 38.5839385986328 0.0
18.798616943 18.5381984710693 1.0
20.563079997 19.60231590271 2.0
-4.5015726852 -5.77445220947266 3.0
16.061507302 15.1822214126587 4.0
23.300189639 22.9885711669922 5.0
39.361696951 39.0705871582031 6.0
39.361696951 39.5996551513672 7.0
38.339957491 38.8900909423828 0.0
20.325182647 21.0611763000488 1.0
18.014774844 20.0291538238525 2.0
-6.00621 -8.46483707427979 3.0
12.008564835 11.5466804504395 4.0
26.331392656 29.179708480835 5.0
38.339957501 41.6321411132812 6.0
38.339957501 41.3705215454102 7.0
38.4370780266379 38.1817092895508 0.0
19.1112284596379 22.9684066772461 1.0
19.3258495766379 20.8143291473389 2.0
-8.41787501523793 -10.4658393859863 3.0
10.9079745616379 9.77639198303223 4.0
27.5291034382175 32.4787940979004 5.0
38.437078026 43.5633163452148 6.0
38.437078026 42.662971496582 7.0
31.680301152 38.6968383789062 0.0
16.922816031 15.7942514419556 1.0
14.757485122 14.8312072753906 2.0
-6.2787127987 -8.25482273101807 3.0
8.4787723148 7.88749504089355 4.0
23.201528838 23.5726051330566 5.0
31.680301161 30.23606300354 6.0
31.680301161 30.2849559783936 7.0
36.698769122 39.2011032104492 0.0
19.594432034 17.6511535644531 1.0
17.104337088 18.1099872589111 2.0
-2.472193454 -4.85581254959106 3.0
14.632143624 14.4233636856079 4.0
22.066625498 21.2976608276367 5.0
36.698769132 35.2066955566406 6.0
36.698769132 36.1286392211914 7.0
40.942334912 38.3624801635742 0.0
22.219375241 19.7867298126221 1.0
18.722959671 19.6671524047852 2.0
-5.2427967213 -7.02317523956299 3.0
13.48016294 12.8304309844971 4.0
27.462171973 26.1630973815918 5.0
40.942334922 40.4672698974609 6.0
40.942334922 40.4371871948242 7.0
41.496981527 37.8840942382812 0.0
18.951043784 22.03586769104 1.0
22.545937743 21.4198608398438 2.0
-7.7537930795 -8.08576393127441 3.0
14.792144654 14.3052701950073 4.0
26.704836873 29.4385719299316 5.0
41.496981537 44.5245513916016 6.0
41.496981537 44.6415786743164 7.0
33.577466055 36.7898712158203 0.0
17.755510174 17.1780185699463 1.0
15.821955881 15.8019666671753 2.0
-7.358985858 -7.47995376586914 3.0
8.4629700135 8.25488758087158 4.0
25.114496042 24.0903701782227 5.0
33.577466065 34.1602630615234 6.0
33.577466065 33.8193740844727 7.0
38.410697594746 39.6828231811523 0.0
20.0282722997678 18.9050846099854 1.0
18.3824252977607 20.1255741119385 2.0
-2.25462323941058 -4.23089838027954 3.0
16.1278020587646 15.6055974960327 4.0
22.2828955387576 21.443187713623 5.0
38.410697595 38.4636077880859 6.0
38.410697595 38.7058334350586 7.0
40.390522102 39.0937118530273 0.0
23.101316011 20.7243633270264 1.0
17.289206093 18.6004695892334 2.0
-8.5131515481 -10.4230661392212 3.0
8.7760545352 8.47414588928223 4.0
31.614467568 30.8545036315918 5.0
40.390522112 39.4214401245117 6.0
40.390522112 40.6173629760742 7.0
38.02724698 38.0636672973633 0.0
17.092159812 18.586088180542 1.0
20.935087168 19.0112609863281 2.0
-7.0096758313 -6.10312986373901 3.0
13.925411327 13.6782245635986 4.0
24.101835653 23.475118637085 5.0
38.02724699 38.3910369873047 6.0
38.02724699 38.6760025024414 7.0
34.831322289 38.1769027709961 0.0
17.981428364 17.1401195526123 1.0
16.849893928 17.1651268005371 2.0
-5.7085713019 -7.30678272247314 3.0
11.14132262 10.6464405059814 4.0
23.689999673 23.5980854034424 5.0
34.831322295 34.5875854492188 6.0
34.831322295 35.0507965087891 7.0
33.152476822 39.5353088378906 0.0
16.395481675 16.7693042755127 1.0
16.756995148 16.0176315307617 2.0
-6.4008279506 -6.76018333435059 3.0
10.356167188 9.77749252319336 4.0
22.796309635 22.7096328735352 5.0
33.152476832 33.6477966308594 6.0
33.152476832 31.664514541626 7.0
37.7618147597387 38.8365478515625 0.0
20.5714887837879 19.5433959960938 1.0
17.1903259800247 18.5501022338867 2.0
-5.43010176849644 -7.48969841003418 3.0
11.7602242121808 11.0746173858643 4.0
26.0015905517818 25.8585243225098 5.0
37.76181476 37.0658569335938 6.0
37.76181476 38.1935043334961 7.0
46.821650957 38.3199157714844 0.0
22.453957792 23.5474395751953 1.0
24.367693165 22.5898189544678 2.0
-10.797760543 -9.05830097198486 3.0
13.569932612 12.4644193649292 4.0
33.251718344 31.6772708892822 5.0
46.821650967 46.1562271118164 6.0
46.821650967 46.6109848022461 7.0
41.1837901754312 38.6112060546875 0.0
19.2503118354314 20.2322578430176 1.0
21.9334783494314 20.0590953826904 2.0
-7.5443023473312 -6.67366218566895 3.0
14.3891760014314 13.9186019897461 4.0
26.7946141824314 26.1711616516113 5.0
41.183790175 40.1880493164062 6.0
41.183790175 40.5314788818359 7.0
36.784988632 38.8896408081055 0.0
21.035700431 19.03271484375 1.0
15.749288201 17.5402450561523 2.0
-6.0648376105 -8.63184070587158 3.0
9.6844505818 9.24567317962646 4.0
27.100538051 26.8921699523926 5.0
36.78498864 36.8820114135742 6.0
36.78498864 37.6822967529297 7.0
37.6894949442102 38.8402252197266 0.0
17.6138134052102 22.3506031036377 1.0
20.0756815502102 20.9995899200439 2.0
-8.78806436921022 -9.89967250823975 3.0
11.2876171802102 10.2714033126831 4.0
26.401877737898 31.1973533630371 5.0
37.689494945 43.9602508544922 6.0
37.689494945 42.8615112304688 7.0
41.308881088 37.9395294189453 0.0
21.148817904 21.0546417236328 1.0
20.160063185 20.1850166320801 2.0
-6.8065142938 -8.00381851196289 3.0
13.353548882 12.6052093505859 4.0
27.955332207 27.4679794311523 5.0
41.308881097 41.778434753418 6.0
41.308881097 41.7641677856445 7.0
42.253011175 37.220344543457 0.0
21.557015891 21.945873260498 1.0
20.695995284 20.3488464355469 2.0
-9.3835381575 -9.3173999786377 3.0
11.312457118 10.7231931686401 4.0
30.940554058 30.3450355529785 5.0
42.253011184 41.5468368530273 6.0
42.253011184 42.3424377441406 7.0
40.047660483 39.439697265625 0.0
19.392839472 21.3383140563965 1.0
20.654821011 20.470890045166 2.0
-7.4809136797 -7.925124168396 3.0
13.173907322 12.2221212387085 4.0
26.873753161 27.2404041290283 5.0
40.047660493 43.3388900756836 6.0
40.047660493 40.7363052368164 7.0
46.604605723 40.7205352783203 0.0
22.954120862 22.8051605224609 1.0
23.650484861 23.0985088348389 2.0
-7.9204617962 -7.7554407119751 3.0
15.730023055 15.3602828979492 4.0
30.874582668 29.926815032959 5.0
46.604605733 47.6629104614258 6.0
46.604605733 46.2474212646484 7.0
45.106295532 38.264518737793 0.0
23.2126055 22.8173294067383 1.0
21.893690031 21.1630477905273 2.0
-10.542381522 -10.2076473236084 3.0
11.3513085 10.5679950714111 4.0
33.754987032 32.8176498413086 5.0
45.106295542 43.587158203125 6.0
45.106295542 44.4005889892578 7.0
38.977055089 39.6535949707031 0.0
19.684728957 19.9886054992676 1.0
19.292326132 18.5582733154297 2.0
-8.6831116468 -9.32623386383057 3.0
10.609214475 9.9409351348877 4.0
28.367840614 28.3870124816895 5.0
38.977055099 39.1203765869141 6.0
38.977055099 38.0659027099609 7.0
38.935365873 38.247673034668 0.0
21.088579762 18.8313083648682 1.0
17.846786112 19.5821418762207 2.0
-3.2985252296 -5.88558101654053 3.0
14.548260873 13.8563652038574 4.0
24.387105001 23.9787139892578 5.0
38.935365883 38.69384765625 6.0
38.935365883 39.1697158813477 7.0
42.74140991 38.2397308349609 0.0
20.9458217 20.9357070922852 1.0
21.795588212 20.7361640930176 2.0
-6.3794891462 -5.80559158325195 3.0
15.416099058 14.8113632202148 4.0
27.325310854 25.7390193939209 5.0
42.741409918 42.1773986816406 6.0
42.741409918 42.1405639648438 7.0
31.310794567 38.8892364501953 0.0
16.880318068 15.570930480957 1.0
14.4304765 14.8110074996948 2.0
-5.6188756586 -6.82168197631836 3.0
8.8116008333 8.73148727416992 4.0
22.499193736 22.5182342529297 5.0
31.310794576 31.7474250793457 6.0
31.310794576 31.7730731964111 7.0
};
\addlegendentry{$R^2$=0.983}
\end{axis}

\end{tikzpicture}
}}
        \caption{Model results using the losses associated with the flows in nodes and edges predictions in the 8-node network.}
        \label{fig:dummy_base_f_results}
\end{figure}


To evaluate the computational performance when considering the loss associated with pipelines and compressor flows, a t-test was performed comparing the computation times of the optimization model and the GNN-based model. The results showed a T-statistic of $14.81$, $198$ degrees of freedom and a p-value of $3.47e-34$, indicating that the optimization model took significantly longer to compute than the GNN-based model. This confirms the consistently higher computational cost of the optimization model, even when additional complexities, such as pipeline flows, are included in the network.



% \begin{table}
%     \centering
%     \begin{tabular}{|c|c|c|c|c|c|}
%         \hline
%         Method & Source & Node Value & Edge Value & Balance Value & Time \\ \hline
%         Opt & Base   & 4.844 ± 12.874 & 24.221 ± 14.041 & -0.001 ± 0.038 & 0.792 ± 0.436 \\ \hline
%         GNN & Base   & 4.840 ± 12.794 & 0.455 ± 0.262   & -5.711 ± 16.854 & 0.144 ± 0.044 \\ \hline
%         Opt & Base-f & 4.844 ± 12.874 & 24.221 ± 14.041 & -0.001 ± 0.038 & 0.792 ± 0.436 \\ \hline
%         GNN & Base-f & 4.767 ± 12.701 & 24.142 ± 14.083 & -0.079 ± 1.174  & 0.145 ± 0.060 \\ \hline
%     \end{tabular}
%     \caption{Comparison of mean and standard deviation values for nodal flows, edge flows, nodal balance, and prediction time between the optimization model and the GNN-based model, across 100 samples for Base and Base-f.}
%     \label{tab:lineal_dummy_results}
% \end{table}

\begin{table}
    \centering
    \begin{tabular}{|c|p{0.8cm}|p{0.55cm}|c|c|c|c|}
        \hline
        Method & \multicolumn{2}{|c|}{Configuration} & Node Value & Edge Value & Balance Value & Time \\ \hline
               & \centering N & \centering E &  &  &  &  \\ \hline
        APOPT & \makebox[0.8cm]{\centering \checkmark} &   & 4.844 ± 12.874 & 24.221 ± 14.041 & -0.001 ± 0.038 & 0.792 ± 0.436 \\ \hline
        GNN & \makebox[0.8cm]{\centering \checkmark} &   & 4.840 ± 12.794 & 0.455 ± 0.262   & -5.711 ± 16.854 & 0.144 ± 0.044 \\ \hline
        % APOPT & \makebox[0.8cm]{\centering \checkmark} & \makebox[0.55cm]{\centering \checkmark} & 4.844 ± 12.874 & 24.221 ± 14.041 & -0.001 ± 0.038 & 0.792 ± 0.436 \\ \hline
        GNN & \makebox[0.8cm]{\centering \checkmark} & \makebox[0.55cm]{\centering \checkmark} & 4.767 ± 12.701 & 24.142 ± 14.083 & -0.079 ± 1.174  & 0.145 ± 0.060 \\ \hline
    \end{tabular}
    \caption{Comparison of mean and standard deviation values for nodal flows, edge flows, nodal balance, and prediction time between the optimization model and the GNN-based model, across 100 samples. The 'N' column refers to experiments considering the nodal loss, while the 'E' column refers to experiments considering the edge loss.}
    \label{tab:lineal_dummy_results}
\end{table}



The \cref{tab:lineal_dummy_results} presents a combined comparison of the optimization model (APOPT) and the GNN-based model across four experiments. Each experiment has three degrees of freedom: the model method (APOPT or GNN), whether nodal loss (N) was included, and whether edge loss (E) was included. The table focuses on four key metrics: nodal flows, edge flows, nodal balance, and prediction time. Mean and standard deviation values are reported based on 100 samples.


Both models display similar results for nodal flows in the experiments, considering only the nodal loss (N). For instance, the APOPT model consistently records a mean nodal flow of \(4.844 \pm 12.874\), while the GNN shows a slightly lower mean of \(4.767 \pm 12.701\) in experiments where both N and E were considered. 


On the other hand, edge flow (E) values vary more significantly between the models. In experiments with only N, the GNN-based model achieves a much lower mean edge flow of \(0.455 \pm 0.262\) compared to the APOPT model’s \(24.221 \pm 14.041\). However, when both N and E were included, both models showed similar edge flow values (\(24.142 \pm 14.083\) for GNN and \(24.221 \pm 14.041\) for APOPT), demonstrating that under these conditions, the GNN model can match the accuracy of the optimization model in predicting edge flows.


Regarding nodal balance, the APOPT model remains consistent across all experiments, showing a near-zero mean balance. The GNN-based model, however, fluctuates more, especially in the N-only experiment, with a mean balance of \(-5.711 \pm 16.854\), though this variance diminishes to \(-0.079 \pm 1.174\) when both N and E are included.

Finally, the GNN model consistently outperforms APOPT in terms of prediction time. Across all experiments, the GNN-based model achieves much faster predictions (\(0.144\) to \(0.145\) seconds) compared to the APOPT model’s \(0.792\) seconds. This superior computational efficiency makes the GNN model particularly appealing for real-time applications.




\subsection{Case Study II: 63-node Network (Colombia)}


The second case studied corresponds to a gas transportation network consisting of 63 nodes representing the Colombian natural gas system. In the first test, where only the loss associated with the flows injected into the nodes was considered the hyperparameter tuning resulted in the following values: $N \ channels = 43$, $N \ layers = 2$, and $N \ dense = 2$. The total loss obtained was 10.598, with the loss associated with the nodal flows. The gas balance loss was 3426.659, but it is important to note that only the nodal flows were considered during the network optimization, and the balance loss was computed after the training process.

The \cref{fig:yn_col_base} shows the scatter plot relating the actual injected flows at the nodes to the predicted flows. This figure demonstrates the network's ability to accurately identify which nodes had gas injections and which did not. Furthermore, with an $R^2$ value of 0.996, the network effectively captured the correct proportions of these injected flows, indicating a robust predictive performance. 
On the other hand, \cref{fig:ye_col_base} presents the relationship between the actual and predicted flows through the edges, encompassing both gas pipelines and compressors. In this case, the network struggled to predict the flow values accurately for these elements, which is unsurprising given that the loss associated with the edge flows was not included in the network's optimization process for this test.

\begin{figure}
    \centering
        \setlength\figurewidth{.53\textwidth}        
        \setlength\figureheight{0.36\textwidth} 
        \subfloat [Actual vs predicted nodal flows.] {\label{fig:yn_col_base}\resizebox{\figurewidth}{\figureheight}{% This file was created with tikzplotlib v0.10.1.
\begin{tikzpicture}

\definecolor{darkgray176}{RGB}{176,176,176}
\definecolor{lightgray204}{RGB}{204,204,204}

\begin{axis}[
colorbar,
colorbar style={ylabel={node\_id}},
colormap={mymap}{[1pt]
 rgb(0pt)=(0.12156862745098,0.466666666666667,0.705882352941177);
  rgb(1pt)=(1,0.498039215686275,0.0549019607843137);
  rgb(2pt)=(0.172549019607843,0.627450980392157,0.172549019607843);
  rgb(3pt)=(0.83921568627451,0.152941176470588,0.156862745098039);
  rgb(4pt)=(0.580392156862745,0.403921568627451,0.741176470588235);
  rgb(5pt)=(0.549019607843137,0.337254901960784,0.294117647058824);
  rgb(6pt)=(0.890196078431372,0.466666666666667,0.76078431372549);
  rgb(7pt)=(0.498039215686275,0.498039215686275,0.498039215686275);
  rgb(8pt)=(0.737254901960784,0.741176470588235,0.133333333333333);
  rgb(9pt)=(0.0901960784313725,0.745098039215686,0.811764705882353)
},
legend cell align={left},
legend style={
  fill opacity=0.8,
  draw opacity=1,
  text opacity=1,
  at={(0.03,0.97)},
  anchor=north west,
  draw=lightgray204
},
point meta max=62,
point meta min=0,
tick align=outside,
tick pos=left,
title={},
x grid style={darkgray176},
xlabel={},
xmajorgrids,
xmin=-16.2727338975, xmax=341.7274118475,
xtick style={color=black},
y grid style={darkgray176},
ylabel={},
ymajorgrids,
ymin=-16.0407207727432, ymax=314.927173924446,
ytick style={color=black}
]
\addplot [
  colormap={mymap}{[1pt]
 rgb(0pt)=(0.12156862745098,0.466666666666667,0.705882352941177);
  rgb(1pt)=(1,0.498039215686275,0.0549019607843137);
  rgb(2pt)=(0.172549019607843,0.627450980392157,0.172549019607843);
  rgb(3pt)=(0.83921568627451,0.152941176470588,0.156862745098039);
  rgb(4pt)=(0.580392156862745,0.403921568627451,0.741176470588235);
  rgb(5pt)=(0.549019607843137,0.337254901960784,0.294117647058824);
  rgb(6pt)=(0.890196078431372,0.466666666666667,0.76078431372549);
  rgb(7pt)=(0.498039215686275,0.498039215686275,0.498039215686275);
  rgb(8pt)=(0.737254901960784,0.741176470588235,0.133333333333333);
  rgb(9pt)=(0.0901960784313725,0.745098039215686,0.811764705882353)
},
  only marks,
  scatter,
  scatter src=explicit
]
table [x=x, y=y, meta=colordata]{%
x  y  colordata
266.47324716 254.830139160156 0.0
0 -0.0604578256607056 1.0
0 -0.0564011335372925 2.0
0 0.011543869972229 3.0
0 -0.0575782060623169 4.0
0 -0.000517010688781738 5.0
0 -0.0282131433486938 6.0
0 -0.0530000925064087 7.0
0 -0.0176455974578857 8.0
0 -0.0414701700210571 9.0
0 0.00264441967010498 10.0
0 0.00634610652923584 11.0
0 0.0464004278182983 12.0
54.420842408 53.9971389770508 13.0
31.76838707 44.0138092041016 14.0
0 -0.00194418430328369 15.0
0 -0.00984871387481689 16.0
31.767708115 32.1540260314941 17.0
0 0.0399836301803589 18.0
0 0.0387800931930542 19.0
0 0.00225865840911865 20.0
0 -0.0348664522171021 21.0
0 0.0404399633407593 22.0
32.042735172 31.1821155548096 23.0
0 -0.0413995981216431 24.0
0 0.697662711143494 25.0
0 -0.0103107690811157 26.0
0 -0.0617239475250244 27.0
0 0.104034304618835 28.0
0 -0.0359839200973511 29.0
0 -0.059648871421814 30.0
0 0.00303661823272705 31.0
0 0.00103461742401123 32.0
0 0.0140689611434937 33.0
0 -0.0409117937088013 34.0
0 0.0552313327789307 35.0
0 -0.112777590751648 36.0
0 -0.0831533670425415 37.0
0 -0.025821328163147 38.0
0 0.0310994386672974 39.0
0 -0.0350772142410278 40.0
0 -0.0475615262985229 41.0
0 -0.0262857675552368 42.0
293.56654935 294.624450683594 43.0
0 0.0263007879257202 44.0
0 0.0474588871002197 45.0
0 -0.0114878416061401 46.0
0 -0.0369507074356079 47.0
0 -0.0432416200637817 48.0
0 0.275271773338318 49.0
0.088502709347 0.219268441200256 50.0
2.2103942585 0.912931799888611 51.0
0 -0.0932635068893433 52.0
0 -0.0801998376846313 53.0
0 0.032162070274353 54.0
0 -0.136098980903625 55.0
0 -0.0201824903488159 56.0
0 -0.0243595838546753 57.0
0 -0.0765374898910522 58.0
0 -0.0227288007736206 59.0
0 0.00363075733184814 60.0
0 -0.00510776042938232 61.0
0 0.875074028968811 62.0
248.37393055 257.598419189453 0.0
0 -0.0721079111099243 1.0
0 0.0416766405105591 2.0
0 -0.0360950231552124 3.0
0 -0.0523117780685425 4.0
0 0.00735342502593994 5.0
0 -0.0395909547805786 6.0
0 -0.0556138753890991 7.0
0 -0.0478034019470215 8.0
0 -0.0532671213150024 9.0
0 0.022004246711731 10.0
0 -0.0489047765731812 11.0
0 0.0305131673812866 12.0
64.885396715 63.5141296386719 13.0
37.251651647 34.5427284240723 14.0
0 -0.0454279184341431 15.0
0 -0.0131489038467407 16.0
30.798970514 30.2305164337158 17.0
0 0.0227028131484985 18.0
0 0.0641278028488159 19.0
0 -0.00254666805267334 20.0
0 -0.0313137769699097 21.0
0 -0.00500023365020752 22.0
29.963403657 28.9648532867432 23.0
0 0.0850428342819214 24.0
0 0.529940962791443 25.0
0 0.0976837873458862 26.0
0 -0.0161570310592651 27.0
0 -0.0136400461196899 28.0
0 -0.0375488996505737 29.0
0 0.0363656282424927 30.0
0 -0.0532363653182983 31.0
0 0.0142644643783569 32.0
0 -0.0145641565322876 33.0
0 -0.0379899740219116 34.0
0 -0.0283051729202271 35.0
0 0.0636593103408813 36.0
0 -0.0155619382858276 37.0
0 -0.0161994695663452 38.0
0 -0.0943590402603149 39.0
0 0.0731042623519897 40.0
0 -0.0522137880325317 41.0
0 0.0434020757675171 42.0
304.35351591 296.832550048828 43.0
0 0.00853192806243896 44.0
0 0.00534301996231079 45.0
0 -0.0553430318832397 46.0
0 -0.0241192579269409 47.0
0 0.00501811504364014 48.0
0 0.302496075630188 49.0
0.084699044305 0.0760651826858521 50.0
1.9745309823 0.927548289299011 51.0
0 -0.0553728342056274 52.0
0 -0.0342055559158325 53.0
0 0.00629675388336182 54.0
0 -0.0878752470016479 55.0
0 -0.0542269945144653 56.0
0 -0.0185602903366089 57.0
0 -0.0572308301925659 58.0
0 -0.0982891321182251 59.0
0 -0.0766142606735229 60.0
0 0.0329269170761108 61.0
0 0.880565285682678 62.0
302.37464816 251.471176147461 0.0
0 -0.0543814897537231 1.0
0 -0.0119353532791138 2.0
0 0.0210083723068237 3.0
0 -0.0655399560928345 4.0
0 -0.177741646766663 5.0
0 0.0876401662826538 6.0
0 -0.0566378831863403 7.0
0 -0.0264221429824829 8.0
0 0.0273374319076538 9.0
0 0.00444519519805908 10.0
0 -0.0224345922470093 11.0
0 -0.207379937171936 12.0
45.482115614 46.0681953430176 13.0
37.007772682 37.6948318481445 14.0
0 0.0214051008224487 15.0
0 0.011623740196228 16.0
29.828999935 29.2057914733887 17.0
0 0.0762842893600464 18.0
0 0.0608888864517212 19.0
0 -0.02826988697052 20.0
0 -0.0207257270812988 21.0
0 0.00192821025848389 22.0
44.134750702 43.9306602478027 23.0
0 -0.158129334449768 24.0
0 0.778813898563385 25.0
0 -0.00452530384063721 26.0
0 0.0955862998962402 27.0
0 -0.0380111932754517 28.0
0 -0.0291081666946411 29.0
0 -0.0316846370697021 30.0
0 -0.0281528234481812 31.0
0 -0.0478488206863403 32.0
0 0.117920756340027 33.0
0 -0.0109885931015015 34.0
0 -0.0110570192337036 35.0
0 0.149499297142029 36.0
0 0.0483723878860474 37.0
0 -0.0421692132949829 38.0
0 0.0639818906784058 39.0
0 -0.0964938402175903 40.0
0 -0.0364047288894653 41.0
0 0.0927070379257202 42.0
291.82175176 294.288818359375 43.0
0 0.0282634496688843 44.0
0 0.0237641334533691 45.0
0 -0.0223654508590698 46.0
0 -0.0800613164901733 47.0
0 -0.0788542032241821 48.0
0 0.313975185155869 49.0
0.075606181489 0.0520530939102173 50.0
1.8743350179 0.998045384883881 51.0
0 -0.0904077291488647 52.0
0 6.63995742797852e-05 53.0
0 -0.0359970331192017 54.0
0 -0.112070918083191 55.0
0 -0.077785849571228 56.0
0 -0.0162554979324341 57.0
0 -0.0353637933731079 58.0
0 -0.0729893445968628 59.0
0 0.00585854053497314 60.0
0 0.0445305109024048 61.0
0 0.823134064674377 62.0
265.89131818 248.667190551758 0.0
0 0.0129836797714233 1.0
0 0.063355565071106 2.0
0 0.0453544855117798 3.0
0 -0.0662285089492798 4.0
0 -0.0242558717727661 5.0
0 0.538580775260925 6.0
0 -0.0490812063217163 7.0
0 -0.0613925457000732 8.0
0 0.0148471593856812 9.0
0 0.0138939619064331 10.0
0 -0.0378570556640625 11.0
0 0.0431627035140991 12.0
60.373084879 59.1540946960449 13.0
38.622708843 36.7255935668945 14.0
0 -0.0713421106338501 15.0
0 0.0170062780380249 16.0
25.188702693 25.1341743469238 17.0
0 0.0068744421005249 18.0
0 -0.0156279802322388 19.0
0 -0.0388609170913696 20.0
0 -0.0237487554550171 21.0
0 0.0486286878585815 22.0
26.672164285 26.4318714141846 23.0
0 -0.0728026628494263 24.0
31.524665793 0.693723917007446 25.0
0 0.0131300687789917 26.0
0 -0.064660906791687 27.0
0 0.107573390007019 28.0
0 -0.050571084022522 29.0
0 -0.0594033002853394 30.0
0 -0.0511931180953979 31.0
0 0.0194457769393921 32.0
0 -0.0560365915298462 33.0
0 -0.118831038475037 34.0
0 -0.0413826704025269 35.0
0 0.0231200456619263 36.0
0 -0.0320342779159546 37.0
0 -0.0384520292282104 38.0
0 -0.00460445880889893 39.0
0 0.00957095623016357 40.0
0 -0.0631929636001587 41.0
0 -0.0240007638931274 42.0
299.2801212 296.138702392578 43.0
0 -0.0392147302627563 44.0
0 0.0391197204589844 45.0
0 -0.0326570272445679 46.0
0 -0.0204793214797974 47.0
0 0.0870717763900757 48.0
0 0.343021869659424 49.0
0.073176300482 0.0760399103164673 50.0
1.5103348501 0.918804883956909 51.0
0 -0.105297923088074 52.0
0 0.0111135244369507 53.0
0 -0.0175446271896362 54.0
0 -0.09076988697052 55.0
0 0.00798428058624268 56.0
0 -0.0365558862686157 57.0
0 -0.0504246950149536 58.0
0 -0.0732270479202271 59.0
0 0.0467602014541626 60.0
0 0.0553885698318481 61.0
0 0.856082201004028 62.0
252.51018499 257.167572021484 0.0
0 -0.13785707950592 1.0
0 -0.0892423391342163 2.0
0 0.0309196710586548 3.0
0 -0.0386437177658081 4.0
0 0.0210925340652466 5.0
0 -0.0131489038467407 6.0
0 -0.07184898853302 7.0
0 -0.0410555601119995 8.0
0 -0.0126191377639771 9.0
0 -0.0391038656234741 10.0
0 -0.0180346965789795 11.0
0 0.0616830587387085 12.0
70.621115574 70.6860885620117 13.0
16.43797061 25.7340984344482 14.0
0 -0.10361635684967 15.0
0 -0.0230857133865356 16.0
40.005265497 40.0512466430664 17.0
0 -0.00592076778411865 18.0
0 -0.00645959377288818 19.0
0 0.0478428602218628 20.0
0 -0.0324605703353882 21.0
0 0.00612199306488037 22.0
26.836412684 26.2132873535156 23.0
0 -0.0848947763442993 24.0
0 0.573138654232025 25.0
0 -0.0159164667129517 26.0
0 0.116510272026062 27.0
0 0.0210217237472534 28.0
0 -0.063233494758606 29.0
0 -0.0630667209625244 30.0
0 -0.00879013538360596 31.0
0 -0.0227299928665161 32.0
0 0.087319016456604 33.0
0 -0.034387469291687 34.0
0 -0.0508567094802856 35.0
0 0.193228125572205 36.0
0 -0.103325247764587 37.0
0 -0.108135581016541 38.0
0 0.0297850370407104 39.0
0 -0.182294964790344 40.0
0 -0.0718101263046265 41.0
0 -0.136250138282776 42.0
300.13171616 296.14013671875 43.0
0 0.000605940818786621 44.0
0 0.0517353415489197 45.0
0 0.00688493251800537 46.0
0 0.0137771368026733 47.0
0 -0.0215188264846802 48.0
1.420043763 0.292773127555847 49.0
0.077893563853 0.157982468605042 50.0
1.5864886701 0.797338008880615 51.0
0 -0.164029002189636 52.0
0 -0.0671540498733521 53.0
0 0.0834416151046753 54.0
0 -0.19973623752594 55.0
0 0.00814235210418701 56.0
0 -0.0321050882339478 57.0
0 -0.114420056343079 58.0
0 -0.029197096824646 59.0
0 -0.00533044338226318 60.0
0 -0.0589092969894409 61.0
0 0.818038702011108 62.0
263.65710187 260.523132324219 0.0
0 -0.0818649530410767 1.0
0 -0.0172361135482788 2.0
0 0.00720512866973877 3.0
0 -0.0479785203933716 4.0
0 -0.0107628107070923 5.0
0 -0.061318039894104 6.0
0 -0.0691457986831665 7.0
0 0.0927548408508301 8.0
0 -0.0261446237564087 9.0
0 -0.0230468511581421 10.0
0 -0.0406216382980347 11.0
0 0.0213683843612671 12.0
48.304542497 48.0496292114258 13.0
50.754517728 47.9450378417969 14.0
0 -0.090004563331604 15.0
0 -0.018616795539856 16.0
37.314581088 36.9991607666016 17.0
0 0.0115777254104614 18.0
0 0.000820279121398926 19.0
0 0.00465261936187744 20.0
0 -0.046362042427063 21.0
0 0.0651739835739136 22.0
25.997095401 25.7217769622803 23.0
0 -0.0402823686599731 24.0
0 0.591066598892212 25.0
0 -0.0176926851272583 26.0
0 0.0204865336418152 27.0
0 0.0185669660568237 28.0
0 -0.0299035310745239 29.0
0 -0.0101017951965332 30.0
0 -0.0314308404922485 31.0
0 0.0636461973190308 32.0
0 0.0718780755996704 33.0
0 -0.0291756391525269 34.0
0 -0.0724366903305054 35.0
0 0.00179970264434814 36.0
0 -0.0602725744247437 37.0
0 0.0330067873001099 38.0
0 0.00128781795501709 39.0
0 -0.0245181322097778 40.0
0 -0.0032273530960083 41.0
0 -0.049686074256897 42.0
297.40720397 296.152282714844 43.0
0 -0.00766432285308838 44.0
0 0.00987374782562256 45.0
0 -0.0172156095504761 46.0
0 -0.0597242116928101 47.0
0 -0.00412189960479736 48.0
0.68317674862 0.302956879138947 49.0
0.072270753595 0.0829511880874634 50.0
2.1325840066 0.875571191310883 51.0
0 -0.0496684312820435 52.0
0 -0.0176538228988647 53.0
0 0.0697439908981323 54.0
0 -0.0518127679824829 55.0
0 0.0366247892379761 56.0
0 -0.0432921648025513 57.0
0 -0.0690230131149292 58.0
0 -0.079137921333313 59.0
0 -0.0299619436264038 60.0
0 0.0480331182479858 61.0
0 0.776832818984985 62.0
248.68242343 255.063873291016 0.0
0 -0.0353361368179321 1.0
0 -0.0835291147232056 2.0
0 -0.0159815549850464 3.0
0 -0.0225032567977905 4.0
0 0.101075053215027 5.0
0 0.036119818687439 6.0
0 -0.0523413419723511 7.0
0 0.00886237621307373 8.0
0 -0.113604187965393 9.0
0 -0.0254195928573608 10.0
0 -0.0299229621887207 11.0
0 0.0578254461288452 12.0
62.217350368 60.9726982116699 13.0
20.055296633 35.2752265930176 14.0
0 -0.145155072212219 15.0
0 0.041502833366394 16.0
31.204645306 30.7023487091064 17.0
0 -0.0581706762313843 18.0
0 -0.0355054140090942 19.0
0 0.000272393226623535 20.0
0 -0.0383659601211548 21.0
0 0.0343636274337769 22.0
33.461561051 31.8261737823486 23.0
0 -0.0856138467788696 24.0
0 0.62156468629837 25.0
0 0.0365279912948608 26.0
0 0.0359748005867004 27.0
0 0.0811113119125366 28.0
0 -0.0986124277114868 29.0
0 -0.0294220447540283 30.0
0 -0.0494424104690552 31.0
0 -0.00867760181427002 32.0
0 -0.0216888189315796 33.0
0 -0.0414758920669556 34.0
0 -0.0925992727279663 35.0
0 0.13070547580719 36.0
0 -0.0369881391525269 37.0
0 -0.0155981779098511 38.0
0 0.00401437282562256 39.0
0 -0.0079118013381958 40.0
0 0.00727736949920654 41.0
0 -0.008644700050354 42.0
288.11219534 292.427093505859 43.0
0 0.0642017126083374 44.0
0 -0.061133861541748 45.0
0 0.0103329420089722 46.0
0 -0.0568369626998901 47.0
0 -0.024666428565979 48.0
0 0.344553530216217 49.0
0.10009390756 0.0706421136856079 50.0
1.4612472871 1.01592373847961 51.0
0 -0.106598734855652 52.0
0 -0.0537196397781372 53.0
0 0.00599801540374756 54.0
0 -0.109769225120544 55.0
0 -0.0719410181045532 56.0
0 -0.0248690843582153 57.0
0 -0.0496848821640015 58.0
0 -0.0570298433303833 59.0
0 -0.0121093988418579 60.0
0 -0.0138090848922729 61.0
0 0.786695122718811 62.0
281.58607857 252.308639526367 0.0
0 -0.0676456689834595 1.0
0 -0.0762413740158081 2.0
0 -0.000368952751159668 3.0
0 -0.0406805276870728 4.0
0 -0.0823394060134888 5.0
0 0.02385413646698 6.0
0 -0.0743004083633423 7.0
0 -0.0660365223884583 8.0
0 0.0645459890365601 9.0
0 -0.115730166435242 10.0
0 0.0207117795944214 11.0
0 -0.161004662513733 12.0
59.957018581 59.7449340820312 13.0
37.03469917 32.6224327087402 14.0
0 -0.0171562433242798 15.0
0 0.0259674787521362 16.0
35.944780243 35.4755172729492 17.0
0 -0.0228210687637329 18.0
0 -0.00541698932647705 19.0
0 -0.00341618061065674 20.0
0 -0.0313147306442261 21.0
0 -0.0147610902786255 22.0
30.155357818 30.259693145752 23.0
0 -0.0444608926773071 24.0
0 2.16888236999512 25.0
0 0.117000937461853 26.0
0 0.00496232509613037 27.0
0 -0.00119316577911377 28.0
0 -0.0723520517349243 29.0
0 -0.0500982999801636 30.0
0 -0.0598090887069702 31.0
0 -0.0220452547073364 32.0
0 -0.0572503805160522 33.0
0 -0.0574253797531128 34.0
0 0.0828113555908203 35.0
0 -0.0514694452285767 36.0
0 -0.0118609666824341 37.0
0 -0.0263990163803101 38.0
0 0.0789033174514771 39.0
0 0.200515151023865 40.0
0 -0.0577315092086792 41.0
0 -0.108537793159485 42.0
283.85723649 296.250030517578 43.0
0 0.0628150701522827 44.0
0 -0.0281444787979126 45.0
0 0.00165259838104248 46.0
0 -0.0940843820571899 47.0
0 0.0180257558822632 48.0
0.59758696653 0.277588248252869 49.0
0.09403947758 0.0524958372116089 50.0
1.8894940797 0.924862205982208 51.0
0 -0.0775810480117798 52.0
0 -0.0395733118057251 53.0
0 0.028751015663147 54.0
0 -0.0928162336349487 55.0
0 -0.00274002552032471 56.0
0 -0.0295137166976929 57.0
0 -0.0997601747512817 58.0
0 -0.0345391035079956 59.0
0 0.0139752626419067 60.0
0 0.00659453868865967 61.0
0 0.855497598648071 62.0
240.45036669 254.085479736328 0.0
0 -0.0295921564102173 1.0
0 -0.0635694265365601 2.0
0 -0.00052940845489502 3.0
0 -0.057867169380188 4.0
0 -0.0448137521743774 5.0
0 -0.077947735786438 6.0
0 -0.152593970298767 7.0
0 -0.0462085008621216 8.0
0 -0.0115782022476196 9.0
0 0.0039832592010498 10.0
0 -0.00988078117370605 11.0
0 0.0580581426620483 12.0
61.509187996 60.929859161377 13.0
37.846701323 38.0774688720703 14.0
0 -0.0353742837905884 15.0
0 -0.00609171390533447 16.0
37.498431824 37.292610168457 17.0
0 0.0428639650344849 18.0
0 0.0516895055770874 19.0
0 0.00200212001800537 20.0
0 -0.0167006254196167 21.0
0 -0.0322586297988892 22.0
38.5066457 38.3373756408691 23.0
0 -0.0946727991104126 24.0
0 0.682299256324768 25.0
0 0.0270842313766479 26.0
0 0.0657035112380981 27.0
0 0.0350230932235718 28.0
0 -0.0403376817703247 29.0
0 -0.0965877771377563 30.0
0 -0.0351775884628296 31.0
0 0.0501669645309448 32.0
0 0.0670746564865112 33.0
0 -0.0247594118118286 34.0
0 0.0361443758010864 35.0
0 -0.0207821130752563 36.0
0 -0.0912610292434692 37.0
0 -0.00733697414398193 38.0
0 0.030779242515564 39.0
0 0.0426682233810425 40.0
0 0.0196146965026855 41.0
0 0.0201457738876343 42.0
285.37898489 298.132751464844 43.0
0 -0.0113450288772583 44.0
0 0.0709236860275269 45.0
0 0.0299209356307983 46.0
0 -0.0616863965988159 47.0
0 -0.0126563310623169 48.0
0 0.304810672998428 49.0
0.095311076704 0.0704606771469116 50.0
2.2306514896 0.922522604465485 51.0
0 -0.0301815271377563 52.0
0 -0.0321129560470581 53.0
0 0.0702842473983765 54.0
0 -0.12122642993927 55.0
0 -0.0333389043807983 56.0
0 -0.0113102197647095 57.0
0 -0.0375670194625854 58.0
0 -0.0448583364486694 59.0
0 -0.105716347694397 60.0
0 -0.0292671918869019 61.0
0 0.82089364528656 62.0
257.41831347 257.840972900391 0.0
0 -0.0710388422012329 1.0
0 -0.00369632244110107 2.0
0 -0.0390983819961548 3.0
0 -0.0382527112960815 4.0
0 -0.100690484046936 5.0
0 -0.0215555429458618 6.0
0 -0.0945934057235718 7.0
0 -0.00331902503967285 8.0
0 0.00686872005462646 9.0
0 0.00956821441650391 10.0
0 -0.0200666189193726 11.0
0 -0.000786423683166504 12.0
57.116207597 56.8888626098633 13.0
43.608587743 38.6412773132324 14.0
0 -0.0644317865371704 15.0
0 0.0245648622512817 16.0
28.671625187 27.8269538879395 17.0
0 -0.0354894399642944 18.0
0 -0.016832709312439 19.0
0 0.0271962881088257 20.0
0 -0.002707839012146 21.0
0 0.0143955945968628 22.0
31.518957077 31.8812999725342 23.0
0 -0.0925337076187134 24.0
0 0.964307546615601 25.0
0 0.0252126455307007 26.0
0 0.1097172498703 27.0
0 -0.054996132850647 28.0
0 -0.0800615549087524 29.0
0 -0.0526853799819946 30.0
0 -0.0128113031387329 31.0
0 -0.0171319246292114 32.0
0 -0.0182119607925415 33.0
0 -0.045007586479187 34.0
0 0.0601397752761841 35.0
0 -0.0250052213668823 36.0
0 -0.0378822088241577 37.0
0 -0.0162574052810669 38.0
0 0.0286229848861694 39.0
0 -0.0904170274734497 40.0
0 -0.0604907274246216 41.0
0 -0.014468789100647 42.0
295.98166452 297.586608886719 43.0
0 0.0194724798202515 44.0
0 -0.037293553352356 45.0
0 0.011243462562561 46.0
0 -0.0881601572036743 47.0
0 0.0242623090744019 48.0
0 0.338068157434464 49.0
0.067485059706 0.0503660440444946 50.0
1.8775220283 1.01453423500061 51.0
0 -0.0807939767837524 52.0
0 -0.0243417024612427 53.0
0 0.00217282772064209 54.0
0 -0.0153361558914185 55.0
0 -0.0467633008956909 56.0
0 -0.00495374202728271 57.0
0 -0.0229398012161255 58.0
0 -0.0255528688430786 59.0
0 -0.0833555459976196 60.0
0 0.0305813550949097 61.0
0 0.899666368961334 62.0
277.18428757 264.750244140625 0.0
0 -0.102965950965881 1.0
0 -0.0455504655838013 2.0
0 -0.0306137800216675 3.0
0 -0.060033917427063 4.0
0 -0.0696054697036743 5.0
0 -0.00116884708404541 6.0
0 -0.122283101081848 7.0
0 -0.020748496055603 8.0
0 0.0533841848373413 9.0
0 -0.0227353572845459 10.0
0 -0.00933718681335449 11.0
0 -0.142857193946838 12.0
59.991681026 59.7598648071289 13.0
30.636467525 36.6489410400391 14.0
0 0.000833868980407715 15.0
0 -0.0291868448257446 16.0
34.714832981 34.0120048522949 17.0
0 -0.00678980350494385 18.0
0 0.0303462743759155 19.0
0 0.0328088998794556 20.0
0 0.0384312868118286 21.0
0 -0.0421653985977173 22.0
39.054364376 38.7474937438965 23.0
0 0.0127824544906616 24.0
0 0.490291774272919 25.0
0 0.0693682432174683 26.0
0 0.0706015825271606 27.0
0 -0.00286591053009033 28.0
0 -0.0589278936386108 29.0
0 -0.0369526147842407 30.0
0 -0.0095897912979126 31.0
0 -0.0175720453262329 32.0
0 0.0963658094406128 33.0
0 -0.0507439374923706 34.0
0 0.0665174722671509 35.0
0 -0.0071340799331665 36.0
0 0.0251535177230835 37.0
0 0.00662004947662354 38.0
0 0.0304614305496216 39.0
0 0.0422729253768921 40.0
0 -0.013713002204895 41.0
0 -0.0284792184829712 42.0
288.53734099 296.903686523438 43.0
0 -0.0607563257217407 44.0
0 0.00446510314941406 45.0
0 0.00880253314971924 46.0
0 -0.0519851446151733 47.0
0 0.0446447134017944 48.0
0 0.342181324958801 49.0
0.097389770807 0.0655349493026733 50.0
1.9364159224 0.988350987434387 51.0
0 -0.0505982637405396 52.0
0 -0.0118128061294556 53.0
0 0.0206960439682007 54.0
0 -0.106101870536804 55.0
0 0.00170266628265381 56.0
0 -0.0150421857833862 57.0
0 -0.0590840578079224 58.0
0 -0.0496217012405396 59.0
0 -0.0433069467544556 60.0
0 -0.00193464756011963 61.0
0 0.821757316589355 62.0
219.63762772 256.810516357422 0.0
0 -0.202701210975647 1.0
0 0.0396636724472046 2.0
0 -0.0427032709121704 3.0
0 -0.0896152257919312 4.0
0 0.0199626684188843 5.0
0 0.000989317893981934 6.0
0 -0.0687474012374878 7.0
0 -0.023725152015686 8.0
0 0.00310170650482178 9.0
0 0.0252751111984253 10.0
0 0.0242254734039307 11.0
0 -0.155414223670959 12.0
52.549372331 52.4594535827637 13.0
37.200062017 42.5815582275391 14.0
0 -0.0165010690689087 15.0
0 -0.0581761598587036 16.0
34.268062137 33.762767791748 17.0
0 -0.0187212228775024 18.0
0 -0.0240210294723511 19.0
0 -0.016602635383606 20.0
0 -0.0250369310379028 21.0
0 -0.0612431764602661 22.0
37.198402042 37.0516777038574 23.0
0 -0.110767960548401 24.0
0 0.703439950942993 25.0
0 0.0445159673690796 26.0
0 0.0342283844947815 27.0
0 0.0432685613632202 28.0
0 -0.0336846113204956 29.0
0 -0.0885823965072632 30.0
0 -0.0028005838394165 31.0
0 0.00294673442840576 32.0
0 0.137920260429382 33.0
0 -0.0540200471878052 34.0
0 -0.0342305898666382 35.0
0 0.000149130821228027 36.0
0 -0.106435179710388 37.0
0 -0.00839579105377197 38.0
0 0.0201023817062378 39.0
0 0.0823858976364136 40.0
0 -0.00114965438842773 41.0
0 -0.0149542093276978 42.0
303.61881332 296.621856689453 43.0
0 -0.0483995676040649 44.0
0 -0.0164095163345337 45.0
0 -0.0480290651321411 46.0
0 -0.0422629117965698 47.0
0 0.0197361707687378 48.0
0 0.263083517551422 49.0
0.081955663656 0.0819371938705444 50.0
1.5765686668 0.936159133911133 51.0
0 -0.059264063835144 52.0
0 -0.0410798788070679 53.0
0 0.0442005395889282 54.0
0 -0.134649157524109 55.0
0 -0.0131431818008423 56.0
0 -0.0321823358535767 57.0
0 -0.0295363664627075 58.0
0 -0.0719846487045288 59.0
0 -0.0787540674209595 60.0
0 0.00457561016082764 61.0
0 0.87531852722168 62.0
277.02563844 254.205200195312 0.0
0 -0.0749508142471313 1.0
0 -0.0189207792282104 2.0
0 0.00561892986297607 3.0
0 -0.0563973188400269 4.0
0 -0.040103554725647 5.0
0 0.0349792242050171 6.0
0 -0.0521732568740845 7.0
0 -0.0633919835090637 8.0
0 -0.0119491815567017 9.0
0 0.0267636775970459 10.0
0 -0.0462577342987061 11.0
0 -0.0433021783828735 12.0
64.273208711 63.521614074707 13.0
35.459761569 31.6702709197998 14.0
0 -0.112190365791321 15.0
0 -0.0186597108840942 16.0
28.903957412 28.7480983734131 17.0
0 0.0259004831314087 18.0
0 -0.0213092565536499 19.0
0 0.054669976234436 20.0
0 -0.00876486301422119 21.0
0 -0.020936131477356 22.0
32.432418734 31.5994930267334 23.0
0 -0.0394693613052368 24.0
26.791545875 0.616921782493591 25.0
0 0.0132838487625122 26.0
0 -0.0512622594833374 27.0
0 0.00441586971282959 28.0
0 -0.0706208944320679 29.0
0 -0.130900263786316 30.0
0 -0.0604668855667114 31.0
0 0.0271347761154175 32.0
0 -0.122304081916809 33.0
0 -0.0514971017837524 34.0
0 0.0518821477890015 35.0
0 -0.123602509498596 36.0
0 -0.00669682025909424 37.0
0 0.04212486743927 38.0
0 0.0450817346572876 39.0
0 0.15072762966156 40.0
0 -0.0202051401138306 41.0
0 0.0267080068588257 42.0
287.50648982 297.138916015625 43.0
0 -0.010719895362854 44.0
0 0.035027801990509 45.0
0 -0.0165921449661255 46.0
0 -0.0217779874801636 47.0
0 0.0360397100448608 48.0
0 0.230863809585571 49.0
0.074090112329 0.0949217081069946 50.0
2.0212578915 1.00499844551086 51.0
0 -0.0598193407058716 52.0
0 -0.0443984270095825 53.0
0 0.0403224229812622 54.0
0 -0.118971467018127 55.0
0 -0.015189528465271 56.0
0 0.0517579317092896 57.0
0 0.0202065706253052 58.0
0 -0.0329278707504272 59.0
0 0.0199240446090698 60.0
0 0.0288673639297485 61.0
0 0.77167284488678 62.0
270.73877896 254.213806152344 0.0
0 -0.0276361703872681 1.0
0 0.00584685802459717 2.0
0 0.0181251764297485 3.0
0 -0.0456637144088745 4.0
0 0.0465406179428101 5.0
0 -0.0618211030960083 6.0
0 0.0556162595748901 7.0
0 -0.0439850091934204 8.0
0 0.0226858854293823 9.0
0 -0.044162392616272 10.0
0 -0.0439473390579224 11.0
0 -0.0661436319351196 12.0
68.199669813 67.8721160888672 13.0
30.510971411 28.8250522613525 14.0
0 -0.0150352716445923 15.0
0 -0.0695797204971313 16.0
37.989399197 37.6838798522949 17.0
0 -0.0159715414047241 18.0
0 0.000444531440734863 19.0
0 0.0270506143569946 20.0
0 0.00675308704376221 21.0
0 -0.0264703035354614 22.0
33.222988626 32.2240028381348 23.0
0 -0.0213605165481567 24.0
0 0.552788078784943 25.0
0 -0.0273333787918091 26.0
0 0.00305688381195068 27.0
0 0.00181853771209717 28.0
0 -0.0215657949447632 29.0
0 -0.0436713695526123 30.0
0 -0.0200752019882202 31.0
0 0.0502570867538452 32.0
0 -0.0736523866653442 33.0
0 -0.0325571298599243 34.0
0 0.011685848236084 35.0
0 0.114262938499451 36.0
0 0.106234431266785 37.0
0 0.0100821256637573 38.0
0 0.014588475227356 39.0
0 0.102983832359314 40.0
0 -0.03925621509552 41.0
0 0.0565274953842163 42.0
287.29912801 296.47021484375 43.0
0 -0.0371690988540649 44.0
0 0.0148251056671143 45.0
0 -0.0429569482803345 46.0
0 -0.0403352975845337 47.0
0 0.000595927238464355 48.0
0 0.293145358562469 49.0
0.087115678208 0.0753270387649536 50.0
2.1212313726 0.932723164558411 51.0
0 -0.0556079149246216 52.0
0 -0.0111767053604126 53.0
0 0.00374305248260498 54.0
0 -0.00321090221405029 55.0
0 -0.0523109436035156 56.0
0 -0.0424371957778931 57.0
0 -0.0214148759841919 58.0
0 -0.0385502576828003 59.0
0 0.0089040994644165 60.0
0 -0.0444437265396118 61.0
0 0.849875330924988 62.0
231.19885795 256.573638916016 0.0
0 -0.048230767250061 1.0
0 -0.0196226835250854 2.0
0 0.0034099817276001 3.0
0 -0.0763381719589233 4.0
0 -0.0420130491256714 5.0
0 -0.000568985939025879 6.0
0 -0.077820897102356 7.0
0 0.0879127979278564 8.0
0 -0.0470176935195923 9.0
0 0.0122075080871582 10.0
0 -0.0369445085525513 11.0
0 0.000594496726989746 12.0
65.020040517 64.7997131347656 13.0
40.687835721 32.953971862793 14.0
0 -0.000112175941467285 15.0
0 -0.0406595468521118 16.0
30.094040615 29.4756526947021 17.0
0 0.00796353816986084 18.0
0 0.00959813594818115 19.0
0 0.0142937898635864 20.0
0 -0.00105726718902588 21.0
0 -0.0105220079421997 22.0
27.402017585 26.6193561553955 23.0
0 -0.0330396890640259 24.0
0 0.560851633548737 25.0
0 0.0571295022964478 26.0
0 0.0479334592819214 27.0
0 -0.0576106309890747 28.0
0 -0.0786365270614624 29.0
0 -0.0139943361282349 30.0
0 -0.0142616033554077 31.0
0 0.0106171369552612 32.0
0 0.13047468662262 33.0
0 0.00700771808624268 34.0
0 0.0855950117111206 35.0
0 0.14079487323761 36.0
0 -0.0703452825546265 37.0
0 -0.0325759649276733 38.0
0 0.0092235803604126 39.0
0 0.0211027860641479 40.0
0 -0.0444644689559937 41.0
0 0.00157105922698975 42.0
305.4629829 295.246002197266 43.0
0 -0.0172325372695923 44.0
0 -0.0287264585494995 45.0
0 0.00148975849151611 46.0
0 -0.0188814401626587 47.0
0 0.00956904888153076 48.0
0 0.392800271511078 49.0
0.092406918628 0.0751668214797974 50.0
1.5232393136 1.01661765575409 51.0
0 -0.115747570991516 52.0
0 -0.0715442895889282 53.0
0 0.00603783130645752 54.0
0 -0.000961661338806152 55.0
0 -0.0421403646469116 56.0
0 -0.043470025062561 57.0
0 -0.0691202878952026 58.0
0 -0.0855895280838013 59.0
0 0.00665938854217529 60.0
0 -0.0137490034103394 61.0
0 0.837965369224548 62.0
274.49396386 258.076751708984 0.0
0 -0.0734487771987915 1.0
0 -0.0990632772445679 2.0
0 0.016889214515686 3.0
0 -0.0138367414474487 4.0
0 -0.0407563447952271 5.0
0 -0.136975884437561 6.0
0 -0.0261560678482056 7.0
0 -0.0233784914016724 8.0
0 -0.0287700891494751 9.0
0 0.0440772771835327 10.0
0 0.0207798480987549 11.0
0 -0.0854336023330688 12.0
59.602986736 59.3425254821777 13.0
29.124264368 37.357967376709 14.0
0 -0.0860887765884399 15.0
0 0.0297766923904419 16.0
33.254527792 33.5297660827637 17.0
0 0.0183573961257935 18.0
0 -0.011802077293396 19.0
0 0.0280123949050903 20.0
0 -0.0110546350479126 21.0
0 0.0233629941940308 22.0
34.466574702 33.6218605041504 23.0
0 0.0571306943893433 24.0
0 0.605771124362946 25.0
0 0.0865253210067749 26.0
0 -0.0125640630722046 27.0
0 0.021689772605896 28.0
0 -0.107144951820374 29.0
0 -0.0371471643447876 30.0
0 -0.0574115514755249 31.0
0 -0.0928117036819458 32.0
0 -0.0108717679977417 33.0
0 -0.0561813116073608 34.0
0 0.0115700960159302 35.0
0 0.173348307609558 36.0
0 -0.0326277017593384 37.0
0 -0.0687943696975708 38.0
0 0.0395432710647583 39.0
0 0.114375472068787 40.0
0 -0.0173435211181641 41.0
0 0.00263631343841553 42.0
269.77000823 297.207183837891 43.0
0 -0.0151330232620239 44.0
0 0.0414849519729614 45.0
0 0.00352728366851807 46.0
0 -0.0198177099227905 47.0
0 -0.0484417676925659 48.0
0.74326893869 0.282846033573151 49.0
0.080048601195 0.133586645126343 50.0
1.7339306734 0.857535362243652 51.0
0 -0.0743600130081177 52.0
0 -0.0365687608718872 53.0
0 -0.0117257833480835 54.0
0 -0.095849871635437 55.0
0 -0.0290764570236206 56.0
0 -0.0236474275588989 57.0
0 -0.0644470453262329 58.0
0 0.0046921968460083 59.0
0 -0.0188788175582886 60.0
0 0.0439506769180298 61.0
0 0.814223766326904 62.0
262.18985663 237.40950012207 0.0
0 -0.0816880464553833 1.0
0 -0.0875998735427856 2.0
0 -0.0199490785598755 3.0
0 -0.0607410669326782 4.0
0 -0.181594491004944 5.0
0 0.0244706869125366 6.0
0 -0.020084023475647 7.0
0 0.0172183513641357 8.0
0 0.0893710851669312 9.0
0 -0.0182641744613647 10.0
0 -0.00422108173370361 11.0
0 -0.0431085824966431 12.0
56.365983751 55.1877098083496 13.0
43.80118332 40.9006004333496 14.0
0 -0.0835338830947876 15.0
0 0.0103710889816284 16.0
33.066608519 32.6851348876953 17.0
0 -0.0319219827651978 18.0
0 0.0274208784103394 19.0
0 -0.0410641431808472 20.0
0 -0.0418335199356079 21.0
0 0.0290673971176147 22.0
26.521451708 26.4500331878662 23.0
0 -0.0932141542434692 24.0
29.879480622 0.745060563087463 25.0
0 0.0764435529708862 26.0
0 0.0557099580764771 27.0
0 0.0362681150436401 28.0
0 -0.0707961320877075 29.0
0 -0.0586673021316528 30.0
0 -0.0694242715835571 31.0
0 0.00565469264984131 32.0
0 0.0362703800201416 33.0
0 0.00263392925262451 34.0
0 -0.0211334228515625 35.0
0 0.0376378297805786 36.0
0 0.0475698709487915 37.0
0 -0.0350044965744019 38.0
0 0.0055161714553833 39.0
0 0.0544852018356323 40.0
0 -0.0462228059768677 41.0
0 0.0200761556625366 42.0
297.33797651 294.96923828125 43.0
0 -0.0551410913467407 44.0
0 -0.0519270896911621 45.0
0 -0.019330620765686 46.0
0 -0.0676122903823853 47.0
0 -0.0338960886001587 48.0
0 0.261680126190186 49.0
0.079504869356 0.0683704614639282 50.0
1.8250729833 0.943182706832886 51.0
0 -0.066043496131897 52.0
0 -0.0547209978103638 53.0
0 0.01121985912323 54.0
0 -0.0781775712966919 55.0
0 0.0512324571609497 56.0
0 -0.0251661539077759 57.0
0 -0.0150972604751587 58.0
0 -0.00988614559173584 59.0
0 -0.00679886341094971 60.0
0 0.0121148824691772 61.0
0 0.862267255783081 62.0
258.46001951 255.599685668945 0.0
0 -0.115374684333801 1.0
0 -0.0426181554794312 2.0
0 0.0285335779190063 3.0
0 -0.0463854074478149 4.0
0 0.00982701778411865 5.0
0 -0.0269733667373657 6.0
0 0.046106219291687 7.0
0 -0.0324357748031616 8.0
0 0.0179580450057983 9.0
0 0.0492148399353027 10.0
0 -0.00652420520782471 11.0
0 -0.133175492286682 12.0
58.696723069 57.8626594543457 13.0
46.16623135 39.8468818664551 14.0
0 -0.0300885438919067 15.0
0 -0.0375999212265015 16.0
34.130535 33.8634300231934 17.0
0 0.00799500942230225 18.0
0 -0.0165106058120728 19.0
0 -0.0290452241897583 20.0
0 -0.0436280965805054 21.0
0 -0.048951268196106 22.0
40.509589291 39.8111534118652 23.0
0 0.0223697423934937 24.0
0 0.439448952674866 25.0
0 0.0141195058822632 26.0
0 0.114681839942932 27.0
0 -0.0478874444961548 28.0
0 -0.0727149248123169 29.0
0 -0.0376166105270386 30.0
0 -0.0279632806777954 31.0
0 0.0225452184677124 32.0
0 0.044425368309021 33.0
0 -0.010193943977356 34.0
0 -0.0155314207077026 35.0
0 0.0365046262741089 36.0
0 -0.0270324945449829 37.0
0 0.021929144859314 38.0
0 0.0134149789810181 39.0
0 0.0807512998580933 40.0
0 0.0124279260635376 41.0
0 -0.00537741184234619 42.0
295.39941846 296.440338134766 43.0
0 -0.0406080484390259 44.0
0 0.00944912433624268 45.0
0 -0.0532814264297485 46.0
0 -0.0273929834365845 47.0
0 -0.0284949541091919 48.0
0 0.293433248996735 49.0
0.096890267417 0.052294135093689 50.0
2.0204951746 0.884654462337494 51.0
0 -0.0756016969680786 52.0
0 -0.0378793478012085 53.0
0 0.0117329359054565 54.0
0 -0.0713359117507935 55.0
0 0.00418150424957275 56.0
0 -0.0445538759231567 57.0
0 -0.0772796869277954 58.0
0 -0.0737975835800171 59.0
0 -0.0138665437698364 60.0
0 0.0566040277481079 61.0
0 0.926084995269775 62.0
260.84088372 253.680938720703 0.0
0 -0.0188528299331665 1.0
0 0.0475279092788696 2.0
0 -0.0398060083389282 3.0
0 0.0467771291732788 4.0
0 -0.135257840156555 5.0
0 -0.0317951440811157 6.0
0 -0.0980323553085327 7.0
0 0.0234320163726807 8.0
0 0.019283652305603 9.0
0 -0.0408177375793457 10.0
0 -0.00347626209259033 11.0
0 -0.0956207513809204 12.0
77.226145413 76.0974044799805 13.0
30.144033821 19.7732257843018 14.0
0 -0.00508320331573486 15.0
0 0.0506676435470581 16.0
27.472449448 27.4778919219971 17.0
0 -0.0217005014419556 18.0
0 0.0227376222610474 19.0
0 -0.0161916017532349 20.0
0 -0.0278326272964478 21.0
0 0.00250089168548584 22.0
41.496493639 40.0816612243652 23.0
0 -0.0494943857192993 24.0
0 0.628345489501953 25.0
0 -0.014618992805481 26.0
0 -0.018561840057373 27.0
0 -0.0765296220779419 28.0
0 -0.0911768674850464 29.0
0 0.186081528663635 30.0
0 -0.0379194021224976 31.0
0 0.0245603322982788 32.0
0 -0.110544800758362 33.0
0 -0.0615140199661255 34.0
0 -0.0762677192687988 35.0
0 -0.141628861427307 36.0
0 -0.0488361120223999 37.0
0 0.00537478923797607 38.0
0 0.0400460958480835 39.0
0 0.131765246391296 40.0
0 -0.0730069875717163 41.0
0 -0.0371900796890259 42.0
298.19111964 296.785675048828 43.0
0 -0.0170189142227173 44.0
0 -0.0413494110107422 45.0
0 -0.0240894556045532 46.0
0 -0.0406080484390259 47.0
0 -0.120667099952698 48.0
0.47237531641 0.285261392593384 49.0
0.095352842646 0.0475698709487915 50.0
1.5235113519 0.928472936153412 51.0
0 -0.0152403116226196 52.0
0 -0.0220030546188354 53.0
0 -0.0960270166397095 54.0
0 -0.0972024202346802 55.0
0 -0.0441144704818726 56.0
0 -0.0141540765762329 57.0
0 -0.0505274534225464 58.0
0 -0.0796207189559937 59.0
0 0.0465444326400757 60.0
0 -0.00418221950531006 61.0
0 0.788939595222473 62.0
230.50385669 259.928649902344 0.0
0 -0.195397973060608 1.0
0 -0.0795780420303345 2.0
0 -0.029380202293396 3.0
0 0.0464774370193481 4.0
0 -0.0290344953536987 5.0
0 -0.0998374223709106 6.0
0 0.0120834112167358 7.0
0 -0.0613003969192505 8.0
0 0.0782006978988647 9.0
0 -0.124851107597351 10.0
0 0.0791617631912231 11.0
0 -0.0312954187393188 12.0
65.045428007 64.9669494628906 13.0
29.011970421 31.5978050231934 14.0
0 -0.133297562599182 15.0
0 -0.0364648103713989 16.0
40.299083308 41.0629081726074 17.0
0 -0.000119566917419434 18.0
0 -0.0245975255966187 19.0
0 -0.0136197805404663 20.0
0 0.00379455089569092 21.0
0 0.0440844297409058 22.0
25.746741789 24.737606048584 23.0
0 0.0406464338302612 24.0
0 0.59772926568985 25.0
0 -0.0112954378128052 26.0
0 0.0130082368850708 27.0
0 0.147057890892029 28.0
0 -0.103741526603699 29.0
0 0.000611186027526855 30.0
0 0.0405622720718384 31.0
0 -0.140693306922913 32.0
0 -0.0800095796585083 33.0
0 -0.0237315893173218 34.0
0 0.0455344915390015 35.0
0 -0.0830701589584351 36.0
0 -0.0835248231887817 37.0
0 0.0267118215560913 38.0
0 0.0133661031723022 39.0
0 -0.110083222389221 40.0
0 -0.0513640642166138 41.0
0 0.0298494100570679 42.0
287.98597517 298.565277099609 43.0
0 0.0172737836837769 44.0
0 -0.026442289352417 45.0
0 0.0146857500076294 46.0
0 0.012778639793396 47.0
0 0.0157016515731812 48.0
0 0.211380183696747 49.0
0.098778867648 0.0609735250473022 50.0
1.6607063282 0.893165588378906 51.0
0 -0.0610569715499878 52.0
0 -0.0485144853591919 53.0
0 0.0482012033462524 54.0
0 -0.0865007638931274 55.0
0 -0.0294114351272583 56.0
0 -0.00762331485748291 57.0
0 -0.0906234979629517 58.0
0 0.0112694501876831 59.0
0 0.248081088066101 60.0
0 -0.0243850946426392 61.0
0 0.861635327339172 62.0
226.20306875 259.2568359375 0.0
0 -0.11137068271637 1.0
0 0.0592683553695679 2.0
0 -0.0712774991989136 3.0
0 0.0806599855422974 4.0
0 -0.106027483940125 5.0
0 0.0277990102767944 6.0
0 -0.0845502614974976 7.0
0 -0.00692617893218994 8.0
0 0.0295032262802124 9.0
0 -0.00925970077514648 10.0
0 0.0379195809364319 11.0
0 -0.0781141519546509 12.0
76.751917862 76.0064697265625 13.0
25.357078416 23.5065212249756 14.0
0 -0.0661779642105103 15.0
0 0.0182312726974487 16.0
38.559164148 37.8808250427246 17.0
0 0.0173319578170776 18.0
0 -0.032378077507019 19.0
0 0.00684463977813721 20.0
0 -0.0518275499343872 21.0
0 0.0140494108200073 22.0
32.584985522 31.7761478424072 23.0
0 -0.0351182222366333 24.0
0 1.83571708202362 25.0
0 -0.0341302156448364 26.0
0 0.0819698572158813 27.0
0 -0.047918438911438 28.0
0 -0.0511578321456909 29.0
0 -0.028766393661499 30.0
0 0.0288194417953491 31.0
0 0.0161951780319214 32.0
0 0.0493921041488647 33.0
0 -0.0345910787582397 34.0
0 0.00192224979400635 35.0
0 0.13767421245575 36.0
0 -0.0107558965682983 37.0
0 0.00487959384918213 38.0
0 0.0628639459609985 39.0
0 0.142804980278015 40.0
0 -0.0208579301834106 41.0
0 0.0112813711166382 42.0
285.37745411 292.152984619141 43.0
0 0.011949896812439 44.0
0 0.129995465278625 45.0
0 -0.0135620832443237 46.0
0 -0.0377293825149536 47.0
0 -0.0299907922744751 48.0
0 0.313579380512238 49.0
0.09433340491 0.169052958488464 50.0
2.0155198805 0.972519159317017 51.0
0 -0.0403801202774048 52.0
0 -0.0180927515029907 53.0
0 -0.0257517099380493 54.0
0 -0.104358077049255 55.0
0 0.00970757007598877 56.0
0 -0.0469356775283813 57.0
0 -0.0513442754745483 58.0
0 -0.0564361810684204 59.0
0 -0.110150694847107 60.0
0 0.00603020191192627 61.0
0 0.890087604522705 62.0
273.37405008 254.008117675781 0.0
0 -0.0575748682022095 1.0
0 0.00667726993560791 2.0
0 -0.0607565641403198 3.0
0 -0.038596510887146 4.0
0 -0.0954991579055786 5.0
0 0.0340312719345093 6.0
0 -0.0581449270248413 7.0
0 -0.0487676858901978 8.0
0 -0.00357639789581299 9.0
0 0.0126675367355347 10.0
0 0.00253713130950928 11.0
0 -0.0108572244644165 12.0
49.60035969 49.2877655029297 13.0
45.538523217 43.1324348449707 14.0
0 -0.106914639472961 15.0
0 0.00843775272369385 16.0
26.826627673 25.811918258667 17.0
0 -0.0219002962112427 18.0
0 -0.00513160228729248 19.0
0 0.0528954267501831 20.0
0 0.0357075929641724 21.0
0 -0.096868634223938 22.0
35.561575837 34.8934326171875 23.0
0 0.00205552577972412 24.0
0 0.558002412319183 25.0
0 0.0432413816452026 26.0
0 0.0232237577438354 27.0
0 0.0309175252914429 28.0
0 -0.0587548017501831 29.0
0 -0.0697551965713501 30.0
0 -0.0253127813339233 31.0
0 0.0428494215011597 32.0
0 0.00175797939300537 33.0
0 -0.0176218748092651 34.0
0 -0.00426793098449707 35.0
0 0.0045095682144165 36.0
0 0.0022050142288208 37.0
0 0.00889766216278076 38.0
0 0.0374623537063599 39.0
0 0.194888949394226 40.0
0 -0.00498855113983154 41.0
0 -0.0224529504776001 42.0
308.07830011 296.619232177734 43.0
0 -0.0138784646987915 44.0
0 0.0557207465171814 45.0
0 0.00750124454498291 46.0
0 -0.0122038125991821 47.0
0 0.037737250328064 48.0
0 0.327320754528046 49.0
0.089665729146 0.0272139310836792 50.0
1.6969396624 0.909597456455231 51.0
0 -0.0690709352493286 52.0
0 -0.00446999073028564 53.0
0 0.0687073469161987 54.0
0 -0.112648367881775 55.0
0 -0.0516961812973022 56.0
0 -0.0195742845535278 57.0
0 -0.0550457239151001 58.0
0 0.0106614828109741 59.0
0 -0.0698870420455933 60.0
0 0.028164267539978 61.0
0 0.792020440101624 62.0
218.1536978 258.396728515625 0.0
0 -0.0198589563369751 1.0
0 0.0152066946029663 2.0
0 0.0188668966293335 3.0
0 -0.0413674116134644 4.0
0 -0.0366164445877075 5.0
0 -0.0279995203018188 6.0
0 -0.0556305646896362 7.0
0 0.00965619087219238 8.0
0 -0.0488468408584595 9.0
0 0.0232092142105103 10.0
0 -0.00703644752502441 11.0
0 -0.0856176614761353 12.0
60.343555122 60.2880249023438 13.0
21.654113421 36.1322135925293 14.0
0 -0.0831266641616821 15.0
0 -0.050916314125061 16.0
34.781138185 35.2653503417969 17.0
0 0.0322145223617554 18.0
0 -0.0369802713394165 19.0
0 0.0367563962936401 20.0
0 -0.0435465574264526 21.0
0 0.0304259061813354 22.0
33.77347685 33.2287940979004 23.0
0 0.0425294637680054 24.0
0 0.522016942501068 25.0
0 0.00721681118011475 26.0
0 -0.0723663568496704 27.0
0 -0.101208806037903 28.0
0 -0.0670205354690552 29.0
0 -0.0997421741485596 30.0
0 -0.0203636884689331 31.0
0 0.000945448875427246 32.0
0 0.0987387895584106 33.0
0 -0.0405541658401489 34.0
0 -0.0339559316635132 35.0
0 -0.12542188167572 36.0
0 -0.0303949117660522 37.0
0 -0.0139175653457642 38.0
0 0.0379089117050171 39.0
0 -0.0583373308181763 40.0
0 -0.0733853578567505 41.0
0 -0.0376192331314087 42.0
293.30127679 295.758575439453 43.0
0 -0.0035024881362915 44.0
0 0.046305239200592 45.0
0 0.00420558452606201 46.0
0 -0.0634113550186157 47.0
0 -0.0450345277786255 48.0
0 0.329753667116165 49.0
0.077405766366 0.106029152870178 50.0
1.9521658064 0.886920213699341 51.0
0 -0.0661343336105347 52.0
0 -0.00540697574615479 53.0
0 0.0613909959793091 54.0
0 -0.061301589012146 55.0
0 0.00892651081085205 56.0
0 0.0256258249282837 57.0
0 -0.0332885980606079 58.0
0 -0.0130475759506226 59.0
0 -0.0341774225234985 60.0
0 -0.0531617403030396 61.0
0 0.799977779388428 62.0
302.03272967 254.948699951172 0.0
0 -0.0243879556655884 1.0
0 0.0107547044754028 2.0
0 -0.00157845020294189 3.0
0 -0.0401374101638794 4.0
0 0.0909022092819214 5.0
0 -0.138896584510803 6.0
0 -0.237808346748352 7.0
0 -0.0461345911026001 8.0
0 0.0904697179794312 9.0
0 -0.0232199430465698 10.0
0 -0.0161771774291992 11.0
0 -0.0877958536148071 12.0
53.511361669 52.4902763366699 13.0
36.923514082 37.008129119873 14.0
0 -0.102022290229797 15.0
0 -0.0378986597061157 16.0
33.455848132 34.7260322570801 17.0
0 0.002632737159729 18.0
0 0.00976336002349854 19.0
0 0.0207062959671021 20.0
0 -0.049284815788269 21.0
0 0.0347784757614136 22.0
35.140796652 33.6424942016602 23.0
0 -0.0166803598403931 24.0
0 0.490261852741241 25.0
0 -0.0133744478225708 26.0
0 0.0702067613601685 27.0
0 -0.0220681428909302 28.0
0 -0.0330268144607544 29.0
0 -0.0523053407669067 30.0
0 -0.0526002645492554 31.0
0 -0.0201669931411743 32.0
0 -0.0345901250839233 33.0
0 -0.059144139289856 34.0
0 0.013484001159668 35.0
0 -0.103927731513977 36.0
0 -0.0491122007369995 37.0
0 0.0140136480331421 38.0
0 0.0802110433578491 39.0
0 0.0931400060653687 40.0
0 -0.055145263671875 41.0
0 0.0412725210189819 42.0
285.35254219 298.382934570312 43.0
0 -0.00583422183990479 44.0
0 0.0599714517593384 45.0
0 -0.0490120649337769 46.0
0 -0.0191718339920044 47.0
0 0.00858438014984131 48.0
0 0.275680780410767 49.0
0.080450465431 0.126032948493958 50.0
2.0296904768 0.832187175750732 51.0
0 -0.0430713891983032 52.0
0 -0.0117791891098022 53.0
0 0.00463926792144775 54.0
0 -0.0582536458969116 55.0
0 -0.0646251440048218 56.0
0 -0.0566390752792358 57.0
0 -0.0731264352798462 58.0
0 -0.0726231336593628 59.0
0 0.0687721967697144 60.0
0 -0.0333713293075562 61.0
0 0.793750405311584 62.0
281.65383391 258.783538818359 0.0
0 0.00141942501068115 1.0
0 -0.0782753229141235 2.0
0 -0.00694191455841064 3.0
0 -0.028767466545105 4.0
0 -0.0753074884414673 5.0
0 -0.0326482057571411 6.0
0 -0.00486481189727783 7.0
0 -0.0249351263046265 8.0
0 0.00984132289886475 9.0
0 -0.0108414888381958 10.0
0 0.0124706625938416 11.0
0 -0.0199819803237915 12.0
69.937782717 68.8640670776367 13.0
22.520216183 28.6866779327393 14.0
0 -0.0228005647659302 15.0
0 -0.0362082719802856 16.0
37.859321768 36.8619270324707 17.0
0 0.0448669195175171 18.0
0 0.0090707540512085 19.0
0 0.014562726020813 20.0
0 -0.0285698175430298 21.0
0 0.0165623426437378 22.0
30.101321779 28.9958419799805 23.0
0 0.113897681236267 24.0
0 0.50898003578186 25.0
0 0.0273298025131226 26.0
0 0.0284454822540283 27.0
0 -0.0361682176589966 28.0
0 -0.0712035894393921 29.0
0 -0.0396093130111694 30.0
0 -0.0640996694564819 31.0
0 0.0160549879074097 32.0
0 0.00473034381866455 33.0
0 -0.0550011396408081 34.0
0 -0.0740464925765991 35.0
0 0.0305548906326294 36.0
0 -0.0103565454483032 37.0
0 0.0114299058914185 38.0
0 0.0891379117965698 39.0
0 0.0608562231063843 40.0
0 -0.0544549226760864 41.0
0 -0.0500637292861938 42.0
287.77091394 296.065368652344 43.0
0 0.101570010185242 44.0
0 -0.0393255949020386 45.0
0 -0.00657856464385986 46.0
0 -0.00528299808502197 47.0
0 -0.0211535692214966 48.0
0.9774023286 0.279220461845398 49.0
0.078716506487 0.0664762258529663 50.0
1.8785712839 0.910422444343567 51.0
0 -0.0749398469924927 52.0
0 -0.048841118812561 53.0
0 0.0354546308517456 54.0
0 -0.0333431959152222 55.0
0 -0.072602391242981 56.0
0 0.0248512029647827 57.0
0 0.0317536592483521 58.0
0 -0.0127855539321899 59.0
0 -0.040798544883728 60.0
0 -0.0317472219467163 61.0
0 0.830275416374207 62.0
240.52189135 257.079223632812 0.0
0 -0.0656653642654419 1.0
0 0.00909626483917236 2.0
0 -0.00501048564910889 3.0
0 -0.0720394849777222 4.0
0 -0.0685094594955444 5.0
0 0.00711572170257568 6.0
0 -0.0841871500015259 7.0
0 -0.028652548789978 8.0
0 0.0887922048568726 9.0
0 0.00205588340759277 10.0
0 -0.0525200366973877 11.0
0 0.049669623374939 12.0
50.730950994 50.5748176574707 13.0
36.387536856 35.5160026550293 14.0
0 -0.0281721353530884 15.0
0 -0.0142003297805786 16.0
36.900484103 35.9821357727051 17.0
0 0.0738636255264282 18.0
0 0.0843280553817749 19.0
0 -0.0373090505599976 20.0
0 -0.0313377380371094 21.0
0 0.0292023420333862 22.0
34.1855562 33.0617713928223 23.0
0 -0.000608086585998535 24.0
0 0.587084233760834 25.0
0 0.0730322599411011 26.0
0 -0.00465047359466553 27.0
0 0.222626805305481 28.0
0 -0.0460950136184692 29.0
0 -0.0621956586837769 30.0
0 -0.0152527093887329 31.0
0 -0.0554052591323853 32.0
0 0.0476862192153931 33.0
0 -0.00562608242034912 34.0
0 0.026536226272583 35.0
0 0.0384234189987183 36.0
0 -0.068684458732605 37.0
0 -0.0109668970108032 38.0
0 0.0288020372390747 39.0
0 0.063592791557312 40.0
0 -0.0376217365264893 41.0
0 0.0416082143783569 42.0
285.05879149 293.642120361328 43.0
0 -0.109662652015686 44.0
0 0.0489763021469116 45.0
0 -0.0101310014724731 46.0
0 -0.0203484296798706 47.0
0 0.0313705205917358 48.0
0 0.30613249540329 49.0
0.095180279051 0.0523420572280884 50.0
1.8512660172 0.924652874469757 51.0
0 -0.0979729890823364 52.0
0 -0.0691276788711548 53.0
0 -0.000132203102111816 54.0
0 -0.0545169115066528 55.0
0 0.0231624841690063 56.0
0 -0.0419341325759888 57.0
0 -0.0567854642868042 58.0
0 -0.0763100385665894 59.0
0 0.0278631448745728 60.0
0 -0.00556099414825439 61.0
0 0.807333827018738 62.0
236.6339844 254.119079589844 0.0
0 0.0644038915634155 1.0
0 0.0836313962936401 2.0
0 0.18232262134552 3.0
0 -0.0334440469741821 4.0
0 -0.0652767419815063 5.0
0 -0.000116944313049316 6.0
0 -0.0455752611160278 7.0
0 0.0206235647201538 8.0
0 -0.113423943519592 9.0
0 -0.0735336542129517 10.0
0 0.0244003534317017 11.0
0 -0.132132172584534 12.0
59.449372573 58.5284233093262 13.0
15.745915508 31.6430015563965 14.0
0 -0.0494905710220337 15.0
0 0.0326741933822632 16.0
32.026563891 31.2637977600098 17.0
0 0.052334189414978 18.0
0 -0.0416771173477173 19.0
0 0.0377160310745239 20.0
0 -0.0665892362594604 21.0
0 -0.0101915597915649 22.0
34.846677492 34.2076187133789 23.0
0 0.0839771032333374 24.0
0 0.473676919937134 25.0
0 -0.0370155572891235 26.0
0 -0.037366509437561 27.0
0 -0.000888466835021973 28.0
0 -0.112356781959534 29.0
0 -0.0598568916320801 30.0
0 -0.0367304086685181 31.0
0 0.0036085844039917 32.0
0 -0.29701817035675 33.0
0 -0.0829588174819946 34.0
0 -0.0904120206832886 35.0
0 -0.136583924293518 36.0
0 -0.00390183925628662 37.0
0 -0.0216625928878784 38.0
0 0.106490015983582 39.0
0 0.0703810453414917 40.0
0 -0.0398622751235962 41.0
0 -0.000885605812072754 42.0
287.0933292 296.217468261719 43.0
0 -0.0143634080886841 44.0
0 0.0989395380020142 45.0
0 -0.010053277015686 46.0
0 -0.0384210348129272 47.0
0 -0.0212675333023071 48.0
3.5451557676 0.247553527355194 49.0
0.082024197858 0.0671933889389038 50.0
2.2337714666 0.986272513866425 51.0
0 -0.0718241930007935 52.0
0 -0.0719419717788696 53.0
0 0.0653654336929321 54.0
0 -0.0513855218887329 55.0
0 -0.0577470064163208 56.0
0 0.0779827833175659 57.0
0 -0.0675338506698608 58.0
0 -0.0481821298599243 59.0
0 0.0294979810714722 60.0
0 -0.0546222925186157 61.0
0 0.803779363632202 62.0
215.19573634 253.841079711914 0.0
0 -0.0167509317398071 1.0
0 0.048064112663269 2.0
0 0.0424710512161255 3.0
0 0.00760281085968018 4.0
0 0.0224624872207642 5.0
0 -0.187706589698792 6.0
0 -0.0683156251907349 7.0
0 -0.0404341816902161 8.0
0 -0.0217300653457642 9.0
0 0.0386706590652466 10.0
0 0.0180696249008179 11.0
0 -0.0351024866104126 12.0
67.540711229 67.3607711791992 13.0
40.6001798 30.5306510925293 14.0
0 -0.165357232093811 15.0
0 0.092272162437439 16.0
39.016725313 38.6279602050781 17.0
0 -0.0718230009078979 18.0
0 -0.0272699594497681 19.0
0 0.0752938985824585 20.0
0 -0.00612461566925049 21.0
0 0.0261114835739136 22.0
35.933291588 35.4706993103027 23.0
0 -0.037629246711731 24.0
0 1.04738855361938 25.0
0 -0.00493013858795166 26.0
0 0.118184924125671 27.0
0 -0.0573073625564575 28.0
0 -0.0397409200668335 29.0
0 -0.0874911546707153 30.0
0 -0.0756412744522095 31.0
0 -0.135587334632874 32.0
0 0.0221213102340698 33.0
0 -0.0410202741622925 34.0
0 -0.0298470258712769 35.0
0 0.0679134130477905 36.0
0 -0.0553795099258423 37.0
0 0.0416573286056519 38.0
0 0.0387715101242065 39.0
0 -0.137790322303772 40.0
0 -0.0242232084274292 41.0
0 0.0364516973495483 42.0
287.72554061 298.603820800781 43.0
0 -0.0307399034500122 44.0
0 0.0132673978805542 45.0
0 -0.0187221765518188 46.0
0 -0.0437699556350708 47.0
0 0.0860942602157593 48.0
0 0.271111994981766 49.0
0.08161973328 0.0505384206771851 50.0
1.5830603668 0.858700454235077 51.0
0 -0.0503174066543579 52.0
0 -0.0138777494430542 53.0
0 -0.00980722904205322 54.0
0 -0.0767964124679565 55.0
0 -0.0809320211410522 56.0
0 -0.0141340494155884 57.0
0 -0.0710748434066772 58.0
0 0.00358664989471436 59.0
0 -0.10329282283783 60.0
0 0.000813841819763184 61.0
0 0.882827520370483 62.0
251.69285645 259.933471679688 0.0
0 -0.0443645715713501 1.0
0 0.00287330150604248 2.0
0 -0.0159093141555786 3.0
0 0.0168296098709106 4.0
0 -0.0180526971817017 5.0
0 -0.0441619157791138 6.0
0 -0.0822209119796753 7.0
0 0.0154393315315247 8.0
0 0.029659628868103 9.0
0 -0.022217869758606 10.0
0 0.0277049541473389 11.0
0 0.0417922735214233 12.0
71.91577744 71.1606140136719 13.0
28.223806798 28.3504543304443 14.0
0 -0.148474335670471 15.0
0 -0.0356401205062866 16.0
33.813247194 33.977424621582 17.0
0 0.0066753625869751 18.0
0 0.0336772203445435 19.0
0 -0.0184916257858276 20.0
0 -0.0157498121261597 21.0
0 -0.0180269479751587 22.0
28.382888812 27.5539722442627 23.0
0 -0.0345808267593384 24.0
0 0.530280828475952 25.0
0 0.0518301725387573 26.0
0 0.0361520648002625 27.0
0 0.113260626792908 28.0
0 -0.0290411710739136 29.0
0 -0.0779834985733032 30.0
0 0.020598292350769 31.0
0 -0.0435434579849243 32.0
0 0.00685012340545654 33.0
0 -0.015079140663147 34.0
0 -0.0341747999191284 35.0
0 0.0265547037124634 36.0
0 0.0410264730453491 37.0
0 0.0310674905776978 38.0
0 0.0338314771652222 39.0
0 0.0494779348373413 40.0
0 -0.0354498624801636 41.0
0 -0.0680824518203735 42.0
308.54984051 296.887298583984 43.0
0 -0.0766845941543579 44.0
0 0.0279195308685303 45.0
0 0.0145927667617798 46.0
0 -0.0868493318557739 47.0
0 0.0126708745956421 48.0
2.0499608765 0.261411786079407 49.0
0.074958321275 0.207199692726135 50.0
1.8423908066 0.882958769798279 51.0
0 -0.0626050233840942 52.0
0 -0.0697265863418579 53.0
0 0.0826288461685181 54.0
0 -0.119532942771912 55.0
0 -0.0325146913528442 56.0
0 0.00592601299285889 57.0
0 -0.0887085199356079 58.0
0 -0.038288950920105 59.0
0 0.00749599933624268 60.0
0 -0.0254002809524536 61.0
0 0.870384216308594 62.0
244.98744882 257.871032714844 0.0
0 -0.0441275835037231 1.0
0 0.0356823205947876 2.0
0 -0.0257211923599243 3.0
0 0.0401014089584351 4.0
0 -0.00427782535552979 5.0
0 -0.201090455055237 6.0
0 -0.127573370933533 7.0
0 -0.0605722665786743 8.0
0 -0.00252211093902588 9.0
0 -0.0174022912979126 10.0
0 -0.0378793478012085 11.0
0 0.104175925254822 12.0
49.221472722 48.3797492980957 13.0
37.243934768 36.4374885559082 14.0
0 0.00244176387786865 15.0
0 -0.0321747064590454 16.0
32.447358494 31.6799030303955 17.0
0 0.0272842645645142 18.0
0 -0.00231707096099854 19.0
0 -0.0275970697402954 20.0
0 -0.000843405723571777 21.0
0 -0.0661417245864868 22.0
34.500268397 33.951602935791 23.0
0 -0.0800882577896118 24.0
0 0.554633796215057 25.0
0 0.0576952695846558 26.0
0 -0.0448974370956421 27.0
0 -0.0677772760391235 28.0
0 -0.0451842546463013 29.0
0 -0.0949842929840088 30.0
0 -0.0247219800949097 31.0
0 -0.00930130481719971 32.0
0 0.0420893430709839 33.0
0 -0.00764811038970947 34.0
0 -0.0795439481735229 35.0
0 0.126952528953552 36.0
0 -0.0990697145462036 37.0
0 -0.00160706043243408 38.0
0 -0.00585639476776123 39.0
0 0.142553210258484 40.0
0 -0.0763369798660278 41.0
0 -0.103460907936096 42.0
287.56160535 294.712249755859 43.0
0 0.0952447652816772 44.0
0 0.0161840915679932 45.0
0 -0.000429272651672363 46.0
0 -0.0360215902328491 47.0
0 -0.0124045610427856 48.0
0 0.300973206758499 49.0
0.097107580134 0.064820408821106 50.0
1.5984762539 0.987081944942474 51.0
0 -0.0870448350906372 52.0
0 -0.0438119173049927 53.0
0 0.0447646379470825 54.0
0 -0.0948354005813599 55.0
0 -0.0193389654159546 56.0
0 0.000741839408874512 57.0
0 -0.000187277793884277 58.0
0 -0.064534068107605 59.0
0 -0.0101407766342163 60.0
0 -0.0144187211990356 61.0
0 0.831101059913635 62.0
247.09320596 258.331359863281 0.0
0 -0.0412212610244751 1.0
0 -0.134866118431091 2.0
0 -0.0611976385116577 3.0
0 -0.0616401433944702 4.0
0 -0.00804102420806885 5.0
0 -0.00332319736480713 6.0
0 -0.0995732545852661 7.0
0 -0.0180923342704773 8.0
0 0.0892308950424194 9.0
0 -0.0197021961212158 10.0
0 -0.0215643644332886 11.0
0 -0.127657532691956 12.0
54.450613912 54.5135154724121 13.0
34.343487461 33.9948616027832 14.0
0 -0.13235342502594 15.0
0 0.0478142499923706 16.0
24.392296269 21.9385509490967 17.0
0 -0.0396958589553833 18.0
0 0.0194798707962036 19.0
0 0.0168718099594116 20.0
0 -0.043883204460144 21.0
0 0.00163590908050537 22.0
43.532837748 43.4380531311035 23.0
0 0.0821841955184937 24.0
0 0.397442638874054 25.0
0 0.0149129629135132 26.0
0 0.0402959585189819 27.0
0 -0.0702749490737915 28.0
0 -0.0784696340560913 29.0
0 -0.0580277442932129 30.0
0 -0.0409961938858032 31.0
0 0.0530003309249878 32.0
0 -0.000864624977111816 33.0
0 -0.1069415807724 34.0
0 0.0202080011367798 35.0
0 0.121342778205872 36.0
0 0.00873744487762451 37.0
0 -0.0196911096572876 38.0
0 0.0154365301132202 39.0
0 -0.0240136384963989 40.0
0 0.0695890188217163 41.0
0 -0.114635109901428 42.0
297.73046784 295.155120849609 43.0
0 -0.00687658786773682 44.0
0 0.0946489572525024 45.0
0 -0.00283420085906982 46.0
0 -0.0241378545761108 47.0
0 0.0246866941452026 48.0
0 0.281061410903931 49.0
0.065017611978 0.0578221082687378 50.0
2.1002984395 0.940765142440796 51.0
0 -0.0256563425064087 52.0
0 0.00407350063323975 53.0
0 0.044944167137146 54.0
0 -0.128679156303406 55.0
0 -0.0185879468917847 56.0
0 -0.0524231195449829 57.0
0 -0.076703667640686 58.0
0 -0.0305346250534058 59.0
0 0.169840455055237 60.0
0 0.0808364152908325 61.0
0 0.899409294128418 62.0
238.56829679 253.343811035156 0.0
0 -0.0161877870559692 1.0
0 0.081236720085144 2.0
0 -0.0401569604873657 3.0
0 -0.0725637674331665 4.0
0 -0.0826798677444458 5.0
0 0.0764592885971069 6.0
0 -0.0786882638931274 7.0
0 -0.0379375815391541 8.0
0 0.0610183477401733 9.0
0 0.0587921142578125 10.0
0 -0.0395801067352295 11.0
0 0.0242952108383179 12.0
61.203433403 60.2759780883789 13.0
40.373790485 35.9856033325195 14.0
0 0.0177830457687378 15.0
0 -0.0439225435256958 16.0
27.610804155 26.9791965484619 17.0
0 0.0308631658554077 18.0
0 0.0325318574905396 19.0
0 0.0179671049118042 20.0
0 -0.0544594526290894 21.0
0 0.0449317693710327 22.0
37.964466863 37.3728256225586 23.0
0 -0.041451096534729 24.0
0 0.544910132884979 25.0
0 0.0361508131027222 26.0
0 0.00811922550201416 27.0
0 0.041674017906189 28.0
0 -0.0392805337905884 29.0
0 -0.109840989112854 30.0
0 -0.0235103368759155 31.0
0 -0.00619089603424072 32.0
0 0.00420582294464111 33.0
0 -0.0326722860336304 34.0
0 -0.0504986047744751 35.0
0 0.0847619771957397 36.0
0 -0.0447236299514771 37.0
0 -0.0229328870773315 38.0
0 0.0712260007858276 39.0
0 0.0275310277938843 40.0
0 -0.0553890466690063 41.0
0 -0.0899454355239868 42.0
289.81499856 296.233123779297 43.0
0 0.0705298185348511 44.0
0 -0.0231996774673462 45.0
0 -0.00597679615020752 46.0
0 -0.0261799097061157 47.0
0 -0.0395699739456177 48.0
0 0.296436369419098 49.0
0.080239982549 0.0531455278396606 50.0
2.0920683675 0.976310014724731 51.0
0 -0.0908418893814087 52.0
0 -0.0972081422805786 53.0
0 0.044441819190979 54.0
0 -0.0796457529067993 55.0
0 -0.0492335557937622 56.0
0 -0.0129677057266235 57.0
0 -0.0366424322128296 58.0
0 -0.0320218801498413 59.0
0 0.0155044794082642 60.0
0 -0.0323971509933472 61.0
0 0.869967460632324 62.0
231.4158619 259.840545654297 0.0
0 0.0464633703231812 1.0
0 0.0382386445999146 2.0
0 0.0142785310745239 3.0
0 -0.0663667917251587 4.0
0 0.0944007635116577 5.0
0 0.0839446783065796 6.0
0 -0.141045451164246 7.0
0 -0.0720556974411011 8.0
0 -0.0306812524795532 9.0
0 -0.0381380319595337 10.0
0 0.016635537147522 11.0
0 -0.0355097055435181 12.0
58.892119169 57.5822792053223 13.0
43.022170451 39.6679725646973 14.0
0 -0.023685097694397 15.0
0 -0.0483707189559937 16.0
26.080215447 26.1929721832275 17.0
0 -0.00232803821563721 18.0
0 0.0410805940628052 19.0
0 -0.044770359992981 20.0
0 -0.045923113822937 21.0
0 -0.0137516260147095 22.0
37.167939797 36.2448692321777 23.0
0 -0.0330978631973267 24.0
0 0.518569111824036 25.0
0 -0.0146561861038208 26.0
0 -0.00748717784881592 27.0
0 -0.0722872018814087 28.0
0 -0.0709496736526489 29.0
0 -0.0454787015914917 30.0
0 -0.0617636442184448 31.0
0 -0.0327011346817017 32.0
0 -0.00268256664276123 33.0
0 -0.0526372194290161 34.0
0 -0.0366429090499878 35.0
0 0.0688329935073853 36.0
0 -0.0383056402206421 37.0
0 0.0244168043136597 38.0
0 0.0487555265426636 39.0
0 0.0974429845809937 40.0
0 -0.0586022138595581 41.0
0 0.0186752080917358 42.0
288.44292383 297.012054443359 43.0
0 -0.0577093362808228 44.0
0 0.0653573274612427 45.0
0 -0.00551807880401611 46.0
0 -0.0652283430099487 47.0
0 -0.0195461511611938 48.0
0.94196107254 0.312752842903137 49.0
0.093090072867 0.0558191537857056 50.0
1.6760415212 0.989922165870667 51.0
0 -0.0367811918258667 52.0
0 -0.0184684991836548 53.0
0 -0.0209506750106812 54.0
0 -0.0621837377548218 55.0
0 -0.0738104581832886 56.0
0 -0.0232816934585571 57.0
0 -0.0379232168197632 58.0
0 -0.0799998044967651 59.0
0 -0.0802818536758423 60.0
0 0.0324321985244751 61.0
0 0.905873596668243 62.0
259.83994203 257.395935058594 0.0
0 -0.0113469362258911 1.0
0 -0.0564900636672974 2.0
0 -0.0258957147598267 3.0
0 -0.0409978628158569 4.0
0 -0.0168572664260864 5.0
0 -0.170237183570862 6.0
0 0.0331815481185913 7.0
0 -0.0545753836631775 8.0
0 -0.0245033502578735 9.0
0 0.047308087348938 10.0
0 -0.000423312187194824 11.0
0 0.0332168340682983 12.0
67.357114068 67.3463592529297 13.0
42.735096612 29.5465679168701 14.0
0 -0.0814462900161743 15.0
0 -0.0424786806106567 16.0
37.638693255 37.1003189086914 17.0
0 0.0315645933151245 18.0
0 0.0519586801528931 19.0
0 -0.00318610668182373 20.0
0 -0.0247267484664917 21.0
0 0.0148452520370483 22.0
29.28688007 28.9941329956055 23.0
0 -0.0486818552017212 24.0
0 0.654044151306152 25.0
0 0.040244460105896 26.0
0 -0.0526415109634399 27.0
0 0.0416048765182495 28.0
0 -0.0420624017715454 29.0
0 -0.114294409751892 30.0
0 0.0118564367294312 31.0
0 -0.0528107881546021 32.0
0 0.0357774496078491 33.0
0 -0.0467313528060913 34.0
0 -0.052773118019104 35.0
0 0.0787786245346069 36.0
0 0.00775706768035889 37.0
0 -0.0194498300552368 38.0
0 0.00764596462249756 39.0
0 -0.0510612726211548 40.0
0 -0.0606306791305542 41.0
0 -0.0782629251480103 42.0
291.19377683 296.489440917969 43.0
0 -0.0195375680923462 44.0
0 -0.0770955085754395 45.0
0 -0.00830912590026855 46.0
0 -0.018677830696106 47.0
0 0.0479139089584351 48.0
0 0.24960720539093 49.0
0.092492395225 0.0623403787612915 50.0
2.1822047799 0.95843893289566 51.0
0 -0.108577847480774 52.0
0 -0.0856887102127075 53.0
0 0.00944387912750244 54.0
0 -0.0721973180770874 55.0
0 0.00867569446563721 56.0
0 -0.0464657545089722 57.0
0 -0.100919842720032 58.0
0 0.0106934309005737 59.0
0 0.0602000951766968 60.0
0 -0.0505753755569458 61.0
0 0.817159414291382 62.0
249.7342568 240.005981445312 0.0
0 0.018696665763855 1.0
0 -0.0109728574752808 2.0
0 0.0283800363540649 3.0
0 -0.0859788656234741 4.0
0 -0.0651808977127075 5.0
0 0.0646308660507202 6.0
0 0.00453221797943115 7.0
0 -0.0465837717056274 8.0
0 -0.0239816904067993 9.0
0 -0.0471609830856323 10.0
0 -0.0157104134559631 11.0
0 0.0140146017074585 12.0
55.983310316 54.9752578735352 13.0
39.349298436 44.2345771789551 14.0
0 -0.0899287462234497 15.0
0 -0.00440418720245361 16.0
28.769802159 29.762300491333 17.0
0 -0.000123858451843262 18.0
0 0.0746463537216187 19.0
0 0.0825134515762329 20.0
0 0.0650404691696167 21.0
0 -0.0418425798416138 22.0
28.886249846 28.9810657501221 23.0
0 -0.134537816047668 24.0
20.064828518 0.602717995643616 25.0
0 0.0565623044967651 26.0
0 0.0914837121963501 27.0
0 0.124143481254578 28.0
0 -0.155672430992126 29.0
0 0.141492247581482 30.0
0 -0.0287643671035767 31.0
0 -0.0103055238723755 32.0
0 0.106612086296082 33.0
0 -0.0558100938796997 34.0
0 0.0265930891036987 35.0
0 0.0679165124893188 36.0
0 0.0201958417892456 37.0
0 -0.0038071870803833 38.0
0 0.00014960765838623 39.0
0 -0.0690799951553345 40.0
0 0.0151501893997192 41.0
0 0.00885140895843506 42.0
319.42421708 297.324157714844 43.0
0 -0.0567761659622192 44.0
0 0.201028943061829 45.0
0 0.0059429407119751 46.0
0 0.0438088178634644 47.0
0 -0.00797092914581299 48.0
0 0.280594289302826 49.0
0.10591559171 0.0360242128372192 50.0
2.0195445708 0.984821557998657 51.0
0 -0.0969232320785522 52.0
0 -0.0262342691421509 53.0
0 -0.0214263200759888 54.0
0 -0.040246844291687 55.0
0 -0.0554789304733276 56.0
0 0.0150154829025269 57.0
0 -0.0651694536209106 58.0
0 -0.0409671068191528 59.0
0 -0.0380498170852661 60.0
0 -0.0632966756820679 61.0
0 0.880013823509216 62.0
262.34667389 256.249694824219 0.0
0 -0.0881506204605103 1.0
0 -0.112135052680969 2.0
0 -0.0156553983688354 3.0
0 -0.00916159152984619 4.0
0 -0.00555050373077393 5.0
0 -0.0151220560073853 6.0
0 -0.0411607027053833 7.0
0 0.0569864511489868 8.0
0 0.0160554647445679 9.0
0 -0.0132776498794556 10.0
0 0.0325974225997925 11.0
0 -0.144972443580627 12.0
59.6971132 59.2729568481445 13.0
38.203402497 35.5514106750488 14.0
0 -0.014325737953186 15.0
0 -0.0252877473831177 16.0
28.68551223 28.474630355835 17.0
0 0.0419250726699829 18.0
0 0.0100950002670288 19.0
0 -0.00370252132415771 20.0
0 -0.00344407558441162 21.0
0 0.00801694393157959 22.0
27.957151674 27.4533157348633 23.0
0 0.0412629842758179 24.0
0 0.550023257732391 25.0
0 0.0863670110702515 26.0
0 0.0262076854705811 27.0
0 -0.0488249063491821 28.0
0 -0.0531433820724487 29.0
0 -0.0881035327911377 30.0
0 -0.0421756505966187 31.0
0 0.0493296384811401 32.0
0 0.0608953237533569 33.0
0 -0.0297592878341675 34.0
0 0.00227558612823486 35.0
0 0.0218313932418823 36.0
0 -0.0743323564529419 37.0
0 -0.000949501991271973 38.0
0 0.0117465257644653 39.0
0 -0.00525486469268799 40.0
0 0.0224454402923584 41.0
0 0.0157254934310913 42.0
311.19147889 296.875396728516 43.0
0 0.0119184255599976 44.0
0 0.0116782188415527 45.0
0 -0.0556281805038452 46.0
0 -0.0190123319625854 47.0
0 -0.00596630573272705 48.0
0 0.317243933677673 49.0
0.070464162568 0.0384217500686646 50.0
1.9549566721 0.863732397556305 51.0
0 -0.0900636911392212 52.0
0 -0.0357886552810669 53.0
0 0.06566321849823 54.0
0 -0.109397053718567 55.0
0 0.0290728807449341 56.0
0 -0.0332475900650024 57.0
0 -0.0387483835220337 58.0
0 0.0214036703109741 59.0
0 -0.0567220449447632 60.0
0 0.0214062929153442 61.0
0 0.797612190246582 62.0
223.06373393 254.985275268555 0.0
0 -0.0535074472427368 1.0
0 -0.0949920415878296 2.0
0 -0.0424016714096069 3.0
0 -0.0154331922531128 4.0
0 -0.00609791278839111 5.0
0 -0.0187994241714478 6.0
0 -0.0501688718795776 7.0
0 0.00766825675964355 8.0
0 0.105002760887146 9.0
0 -0.00483167171478271 10.0
0 -0.0167385339736938 11.0
0 0.00551068782806396 12.0
65.9390245 64.8536605834961 13.0
36.628116556 31.7431373596191 14.0
0 -0.0578194856643677 15.0
0 0.0356782674789429 16.0
36.307275383 35.1997032165527 17.0
0 0.0406683683395386 18.0
0 0.0173481702804565 19.0
0 -0.0415085554122925 20.0
0 -0.00678145885467529 21.0
0 -0.00735914707183838 22.0
40.359030908 40.2240715026855 23.0
0 0.111457228660583 24.0
0 0.931888103485107 25.0
0 0.0273207426071167 26.0
0 -0.0728746652603149 27.0
0 0.0124291181564331 28.0
0 -0.0969911813735962 29.0
0 0.119258761405945 30.0
0 0.00139343738555908 31.0
0 -0.0204750299453735 32.0
0 -0.0307774543762207 33.0
0 -0.0297452211380005 34.0
0 -0.00680553913116455 35.0
0 0.0219243764877319 36.0
0 -0.0289567708969116 37.0
0 0.0368539094924927 38.0
0 0.11548388004303 39.0
0 -0.0116854906082153 40.0
0 -0.081544041633606 41.0
0 -0.0639563798904419 42.0
313.13159493 294.063385009766 43.0
0 -0.0255342721939087 44.0
0 0.0441718101501465 45.0
0 -0.0584079027175903 46.0
0 -0.0858651399612427 47.0
0 -0.0771523714065552 48.0
0 0.261748790740967 49.0
0.074620194132 0.0614989995956421 50.0
1.9976541855 0.912985324859619 51.0
0 -0.098065972328186 52.0
0 -0.0220047235488892 53.0
0 -0.0127390623092651 54.0
0 -0.106822609901428 55.0
0 0.0326474905014038 56.0
0 -0.0379296541213989 57.0
0 -0.0454765558242798 58.0
0 -0.0505770444869995 59.0
0 -0.00287806987762451 60.0
0 -0.0168443918228149 61.0
0 0.8658607006073 62.0
258.4712988 259.772827148438 0.0
0 -0.207200646400452 1.0
0 -0.111701130867004 2.0
0 -0.0407968759536743 3.0
0 -0.0406640768051147 4.0
0 -0.0199791193008423 5.0
0 -0.172605156898499 6.0
0 0.00278890132904053 7.0
0 0.00359761714935303 8.0
0 0.0889676809310913 9.0
0 -0.0845412015914917 10.0
0 0.003581702709198 11.0
0 0.111361861228943 12.0
71.157359765 72.4315490722656 13.0
37.723159568 32.2368927001953 14.0
0 0.0473154783248901 15.0
0 0.00290954113006592 16.0
39.499359333 39.709228515625 17.0
0 0.0344427824020386 18.0
0 -0.0231353044509888 19.0
0 0.0199786424636841 20.0
0 -0.0486273765563965 21.0
0 0.0224446058273315 22.0
36.874433179 35.8205604553223 23.0
0 -0.0305606126785278 24.0
0 0.528499841690063 25.0
0 0.0303949117660522 26.0
0 0.0763536691665649 27.0
0 -0.0343343019485474 28.0
0 -0.0555802583694458 29.0
0 -0.0413786172866821 30.0
0 -0.0139435529708862 31.0
0 -0.0284968614578247 32.0
0 0.0358442068099976 33.0
0 -0.0487262010574341 34.0
0 -0.0936203002929688 35.0
0 -0.0639930963516235 36.0
0 0.0196853876113892 37.0
0 -0.0199168920516968 38.0
0 -0.0118488073348999 39.0
0 -0.0087662935256958 40.0
0 -0.0519485473632812 41.0
0 -0.0903412103652954 42.0
311.56635111 297.474975585938 43.0
0 0.0269292593002319 44.0
0 -0.00607013702392578 45.0
0 0.00281155109405518 46.0
0 -0.114247679710388 47.0
0 -0.022891640663147 48.0
0 0.282208859920502 49.0
0.099727854963 0.0844053030014038 50.0
1.4307932821 0.895402491092682 51.0
0 -0.0522416830062866 52.0
0 -0.0304888486862183 53.0
0 0.0519088506698608 54.0
0 -0.123562693595886 55.0
0 -0.0867880582809448 56.0
0 0.0147465467453003 57.0
0 -0.0537775754928589 58.0
0 -0.0869690179824829 59.0
0 0.00498449802398682 60.0
0 0.0538789033889771 61.0
0 0.826453447341919 62.0
302.95985872 262.737243652344 0.0
0 -0.0922237634658813 1.0
0 -0.0308612585067749 2.0
0 0.0175296068191528 3.0
0 -0.0317872762680054 4.0
0 0.0247477293014526 5.0
0 0.121189475059509 6.0
0 -0.185807347297668 7.0
0 0.0555891990661621 8.0
0 -0.0644956827163696 9.0
0 0.0160800218582153 10.0
0 -0.0187159776687622 11.0
0 -0.129806160926819 12.0
50.958177298 50.2662048339844 13.0
35.606354115 38.0859756469727 14.0
0 -0.110582947731018 15.0
0 0.0485764741897583 16.0
39.441759774 38.9452896118164 17.0
0 0.062914252281189 18.0
0 0.112151741981506 19.0
0 -0.00162827968597412 20.0
0 -0.0136662721633911 21.0
0 -0.0137816667556763 22.0
28.253024995 26.4614276885986 23.0
0 0.141393065452576 24.0
0 0.447182536125183 25.0
0 -0.00528967380523682 26.0
0 0.090711236000061 27.0
0 0.142937541007996 28.0
0 -0.0325230360031128 29.0
0 -0.0555266141891479 30.0
0 0.0075613260269165 31.0
0 0.00396573543548584 32.0
0 -0.058623194694519 33.0
0 0.0111845731735229 34.0
0 -5.20944595336914e-05 35.0
0 0.0404115915298462 36.0
0 -0.0449401140213013 37.0
0 -0.0489820241928101 38.0
0 -0.138656258583069 39.0
0 0.219640612602234 40.0
0 -0.107213854789734 41.0
0 0.0814059972763062 42.0
290.8022228 290.064025878906 43.0
0 -0.0193353891372681 44.0
0 0.0274664163589478 45.0
0 -0.000825762748718262 46.0
0 -0.079170823097229 47.0
0 -0.0625172853469849 48.0
0 0.297043979167938 49.0
0.093729941839 0.0257047414779663 50.0
1.7306271921 0.837983727455139 51.0
0 0.0152522325515747 52.0
0 -0.018513560295105 53.0
0 0.0760222673416138 54.0
0 -0.0893388986587524 55.0
0 0.0111004114151001 56.0
0 -0.0153049230575562 57.0
0 -0.0321451425552368 58.0
0 -0.040129542350769 59.0
0 0.00593698024749756 60.0
0 -0.0229476690292358 61.0
0 0.899093627929688 62.0
246.75535333 253.572265625 0.0
0 -0.0230056047439575 1.0
0 0.00910866260528564 2.0
0 -0.00983822345733643 3.0
0 -0.0259689092636108 4.0
0 -0.129011511802673 5.0
0 -0.0697790384292603 6.0
0 -0.0672191381454468 7.0
0 -0.0332655906677246 8.0
0 -0.0296417474746704 9.0
0 -0.113216757774353 10.0
0 -0.0209286212921143 11.0
0 -0.0899587869644165 12.0
67.443635552 66.3090515136719 13.0
37.974647887 30.1697406768799 14.0
0 0.0338472127914429 15.0
0 -0.0407072305679321 16.0
31.439478866 31.6992473602295 17.0
0 0.0125108957290649 18.0
0 -0.0142282247543335 19.0
0 -0.00117957592010498 20.0
0 -0.0423070192337036 21.0
0 -0.0565072298049927 22.0
40.423062733 39.9624824523926 23.0
0 0.00356519222259521 24.0
0 0.486453592777252 25.0
0 0.0799978971481323 26.0
0 0.0627671480178833 27.0
0 0.0755504369735718 28.0
0 -0.0551351308822632 29.0
0 -0.0569223165512085 30.0
0 -0.0040515661239624 31.0
0 -0.0377455949783325 32.0
0 0.0565069913864136 33.0
0 -0.0128144025802612 34.0
0 -0.0267770290374756 35.0
0 0.029723048210144 36.0
0 -0.050439715385437 37.0
0 0.0476173162460327 38.0
0 -0.00279152393341064 39.0
0 0.0414942502975464 40.0
0 -0.049934983253479 41.0
0 0.0560344457626343 42.0
301.40284389 296.386749267578 43.0
0 -0.0270754098892212 44.0
0 0.00877392292022705 45.0
0 -0.00906097888946533 46.0
0 -0.0520621538162231 47.0
0 0.0386799573898315 48.0
0 0.287111282348633 49.0
0.07882727628 0.0383952856063843 50.0
1.7281058789 0.907239139080048 51.0
0 -0.0454665422439575 52.0
0 -0.0461560487747192 53.0
0 0.0616985559463501 54.0
0 -0.0553323030471802 55.0
0 0.0310636758804321 56.0
0 0.00620424747467041 57.0
0 -0.0242317914962769 58.0
0 -0.0673147439956665 59.0
0 -0.0161970853805542 60.0
0 -0.0166231393814087 61.0
0 0.813776731491089 62.0
239.4102587 256.269317626953 0.0
0 -0.0843111276626587 1.0
0 -0.00136291980743408 2.0
0 -0.052822470664978 3.0
0 -0.0749524831771851 4.0
0 -0.0374387502670288 5.0
0 -0.114884972572327 6.0
0 -0.0547736883163452 7.0
0 0.021070659160614 8.0
0 0.0709861516952515 9.0
0 -0.00432980060577393 10.0
0 -0.00353705883026123 11.0
0 0.0728801488876343 12.0
51.630477482 51.6240577697754 13.0
50.409177176 45.65966796875 14.0
0 -0.0181256532669067 15.0
0 -0.0775226354598999 16.0
28.600080496 28.5621757507324 17.0
0 0.0192300081253052 18.0
0 -0.0159004926681519 19.0
0 0.0193432569503784 20.0
0 -0.0165277719497681 21.0
0 -0.0151301622390747 22.0
31.901942293 31.1652927398682 23.0
0 -0.05073082447052 24.0
0 0.600889325141907 25.0
0 0.0701664686203003 26.0
0 0.0785847902297974 27.0
0 -0.0446368455886841 28.0
0 -0.0346320867538452 29.0
0 -0.0713505744934082 30.0
0 -0.00575864315032959 31.0
0 0.0358413457870483 32.0
0 0.0620220899581909 33.0
0 -0.11323869228363 34.0
0 -0.0430413484573364 35.0
0 0.0586642026901245 36.0
0 0.0938397645950317 37.0
0 0.0389889478683472 38.0
0 0.00627195835113525 39.0
0 -0.0275150537490845 40.0
0 -0.03084397315979 41.0
0 -0.0372158288955688 42.0
292.28426679 294.372253417969 43.0
0 0.0396503210067749 44.0
0 -0.0688381195068359 45.0
0 -0.0065377950668335 46.0
0 0.0153402090072632 47.0
0 -0.0754848718643188 48.0
0.66646253983 0.275604665279388 49.0
0.097045015093 0.04808509349823 50.0
2.0688533976 1.00464701652527 51.0
0 -0.0807021856307983 52.0
0 -0.0566192865371704 53.0
0 0.0173519849777222 54.0
0 -0.0546921491622925 55.0
0 -0.0809851884841919 56.0
0 0.0480602979660034 57.0
0 -0.0828715562820435 58.0
0 -0.0732008218765259 59.0
0 -0.0252677202224731 60.0
0 0.0221368074417114 61.0
0 0.781675457954407 62.0
279.58684323 253.816055297852 0.0
0 -0.0568462610244751 1.0
0 0.0232170820236206 2.0
0 -0.0222936868667603 3.0
0 -0.00322234630584717 4.0
0 0.0146352052688599 5.0
0 0.0312331914901733 6.0
0 -0.104946732521057 7.0
0 0.0186264514923096 8.0
0 0.0293391942977905 9.0
0 -0.0276116132736206 10.0
0 -0.0048598051071167 11.0
0 -0.0877825021743774 12.0
60.811241434 59.9907112121582 13.0
31.046755268 35.9992790222168 14.0
0 -0.020447850227356 15.0
0 -0.0034860372543335 16.0
29.520966241 30.3606567382812 17.0
0 -0.0676733255386353 18.0
0 0.00469100475311279 19.0
0 -0.0278538465499878 20.0
0 -0.00356972217559814 21.0
0 0.0114706754684448 22.0
34.103073711 33.8550453186035 23.0
0 -0.0611463785171509 24.0
0 0.574187397956848 25.0
0 0.0113052129745483 26.0
0 0.0225485563278198 27.0
0 -0.0604983568191528 28.0
0 -0.0534471273422241 29.0
0 -0.0567111968994141 30.0
0 -0.0464237928390503 31.0
0 -0.00873768329620361 32.0
0 -0.0131291151046753 33.0
0 -0.012848973274231 34.0
0 -0.032530665397644 35.0
0 0.112283825874329 36.0
0 -0.068787693977356 37.0
0 0.0234941244125366 38.0
0 0.0537022352218628 39.0
0 0.0463284254074097 40.0
0 -0.0212923288345337 41.0
0 0.00815427303314209 42.0
299.16791168 293.904602050781 43.0
0 -0.0346857309341431 44.0
0 0.0342214107513428 45.0
0 0.0170296430587769 46.0
0 -0.0265903472900391 47.0
0 0.0609811544418335 48.0
0 0.37254124879837 49.0
0.080006698255 0.055193305015564 50.0
2.0212136382 0.882812738418579 51.0
0 -0.0473157167434692 52.0
0 -0.0259619951248169 53.0
0 0.0600701570510864 54.0
0 -0.0834945440292358 55.0
0 -0.0916327238082886 56.0
0 0.0148485898971558 57.0
0 -0.0588506460189819 58.0
0 -0.0547951459884644 59.0
0 0.0368825197219849 60.0
0 -0.00224053859710693 61.0
0 0.80039393901825 62.0
225.41471717 253.864196777344 0.0
0 -0.075750470161438 1.0
0 -0.0966631174087524 2.0
0 0.0170062780380249 3.0
0 -0.0100270509719849 4.0
0 0.0741461515426636 5.0
0 0.0368512868881226 6.0
0 -0.0704275369644165 7.0
0 -0.0675175189971924 8.0
0 -0.0343042612075806 9.0
0 -0.0215226411819458 10.0
0 0.0270884335041046 11.0
0 -0.106342911720276 12.0
75.458597559 73.9071731567383 13.0
28.733769808 24.5824146270752 14.0
0 -0.0253521203994751 15.0
0 0.0714524984359741 16.0
41.290846831 40.1963958740234 17.0
0 -0.0296846628189087 18.0
0 -0.0275171995162964 19.0
0 0.00713098049163818 20.0
0 -0.0519758462905884 21.0
0 0.0232118368148804 22.0
31.922558824 30.9220809936523 23.0
0 -0.0610224008560181 24.0
0 0.554198801517487 25.0
0 0.0619534254074097 26.0
0 0.0163123607635498 27.0
0 0.0300887823104858 28.0
0 0.0107780694961548 29.0
0 -0.0858649015426636 30.0
0 -0.0579828023910522 31.0
0 -0.00881302356719971 32.0
0 -0.0251160860061646 33.0
0 0.015825629234314 34.0
0 -0.0523644685745239 35.0
0 -0.084492564201355 36.0
0 -0.0495554208755493 37.0
0 0.00953781604766846 38.0
0 0.0190284252166748 39.0
0 0.0604943037033081 40.0
0 0.00661623477935791 41.0
0 0.0119680166244507 42.0
316.91099372 294.787750244141 43.0
0 0.00525414943695068 44.0
0 0.0434471368789673 45.0
0 -0.0272527933120728 46.0
0 0.0198186635971069 47.0
0 0.0252283811569214 48.0
0 0.316264301538467 49.0
0.076421123912 0.0553702116012573 50.0
1.8259803841 0.999873042106628 51.0
0 -0.038965106010437 52.0
0 -0.0292221307754517 53.0
0 0.0113564729690552 54.0
0 -0.035209059715271 55.0
0 -0.0688227415084839 56.0
0 0.00202357769012451 57.0
0 -0.0640889406204224 58.0
0 -0.0151351690292358 59.0
0 -0.0459855794906616 60.0
0 -0.0122247934341431 61.0
0 0.820055484771729 62.0
221.80881799 254.430068969727 0.0
0 0.0500224828720093 1.0
0 0.000610232353210449 2.0
0 -0.0625609159469604 3.0
0 -0.0630289316177368 4.0
0 0.00340712070465088 5.0
0 0.154429793357849 6.0
0 -0.0846120119094849 7.0
0 0.0360318422317505 8.0
0 -0.0362783670425415 9.0
0 0.0461596250534058 10.0
0 0.010820746421814 11.0
0 -0.0229202508926392 12.0
55.080488394 54.6320037841797 13.0
23.806101469 42.6477890014648 14.0
0 -0.0583416223526001 15.0
0 -0.0330926179885864 16.0
29.705788544 29.6274471282959 17.0
0 0.0356932878494263 18.0
0 -0.0454257726669312 19.0
0 0.0863765478134155 20.0
0 0.00822651386260986 21.0
0 -0.00941288471221924 22.0
29.371852987 28.25852394104 23.0
0 0.0551437139511108 24.0
0 0.471997857093811 25.0
0 -0.0209308862686157 26.0
0 0.0744863748550415 27.0
0 -0.0158244371414185 28.0
0 -0.0881108045578003 29.0
0 -0.0780348777770996 30.0
0 -0.0370885133743286 31.0
0 -0.010600209236145 32.0
0 -0.0724743604660034 33.0
0 -0.127240300178528 34.0
0 -0.0256928205490112 35.0
0 0.0954173803329468 36.0
0 0.0124870538711548 37.0
0 -0.00509631633758545 38.0
0 0.116014122962952 39.0
0 0.165175795555115 40.0
0 -0.0300852060317993 41.0
0 0.0296033620834351 42.0
284.83336163 296.052398681641 43.0
0 0.0818847417831421 44.0
0 -0.0344744920730591 45.0
0 -0.0300256013870239 46.0
0 -0.0631109476089478 47.0
0 -0.0640469789505005 48.0
0 0.270979881286621 49.0
0.082940674443 0.0890506505966187 50.0
2.2128975286 1.01739025115967 51.0
0 -0.0533894300460815 52.0
0 0.0178776979446411 53.0
0 8.76188278198242e-05 54.0
0 -0.0522893667221069 55.0
0 -0.0244904756546021 56.0
0 0.00829684734344482 57.0
0 0.0721098184585571 58.0
0 -0.0462683439254761 59.0
0 0.0130442380905151 60.0
0 -0.0841313600540161 61.0
0 0.814498662948608 62.0
285.37011784 254.064422607422 0.0
0 -0.00621998310089111 1.0
0 0.0259894132614136 2.0
0 0.0197170972824097 3.0
0 -0.0531145334243774 4.0
0 -0.103846192359924 5.0
0 0.0671600103378296 6.0
0 0.0321635007858276 7.0
0 -0.0804657936096191 8.0
0 0.00955665111541748 9.0
0 0.00882327556610107 10.0
0 -0.0509797334671021 11.0
0 0.0872005224227905 12.0
53.916539587 53.3480949401855 13.0
35.992029034 34.7096519470215 14.0
0 -0.0115724802017212 15.0
0 -0.0135296583175659 16.0
32.49615338 32.5874099731445 17.0
0 0.0782898664474487 18.0
0 0.04993736743927 19.0
0 0.0423043966293335 20.0
0 0.00347387790679932 21.0
0 -0.0141164064407349 22.0
26.70583649 26.660680770874 23.0
0 0.125214457511902 24.0
1.2126831654 0.434941947460175 25.0
0 0.106924414634705 26.0
0 0.11127769947052 27.0
0 0.0726927518844604 28.0
0 -0.0546051263809204 29.0
0 -0.021077036857605 30.0
0 -0.0554169416427612 31.0
0 -0.0325814485549927 32.0
0 -0.0209811925888062 33.0
0 -0.0810290575027466 34.0
0 -0.00552403926849365 35.0
0 0.170972228050232 36.0
0 -0.0560425519943237 37.0
0 -0.00903642177581787 38.0
0 -0.0279546976089478 39.0
0 0.119471430778503 40.0
0 0.0603643655776978 41.0
0 -0.037850022315979 42.0
287.26699285 294.7890625 43.0
0 0.031328558921814 44.0
0 0.10879909992218 45.0
0 0.0185884237289429 46.0
0 -0.0577151775360107 47.0
0 -0.0285135507583618 48.0
1.8381315509 0.26027637720108 49.0
0.066300080933 0.187836050987244 50.0
1.6463063873 0.86328113079071 51.0
0 -0.0425933599472046 52.0
0 -0.0350056886672974 53.0
0 0.0323730707168579 54.0
0 -0.0818966627120972 55.0
0 -0.0147641897201538 56.0
0 -0.0459531545639038 57.0
0 -0.047308087348938 58.0
0 -0.085934042930603 59.0
0 -0.0155817270278931 60.0
0 0.0764967203140259 61.0
0 0.924370288848877 62.0
261.52040326 258.876953125 0.0
0 -0.0735155344009399 1.0
0 -0.0191093683242798 2.0
0 -0.00735080242156982 3.0
0 -0.0889698266983032 4.0
0 -0.0725256204605103 5.0
0 0.0214818716049194 6.0
0 -0.0266505479812622 7.0
0 0.038830041885376 8.0
0 0.017168402671814 9.0
0 -0.0166505575180054 10.0
0 0.0345274209976196 11.0
0 -0.00187551975250244 12.0
70.818061754 70.7741775512695 13.0
34.797736183 27.8467597961426 14.0
0 -0.0719667673110962 15.0
0 -0.0190008878707886 16.0
29.615922232 29.1607608795166 17.0
0 0.0146647691726685 18.0
0 -0.00387680530548096 19.0
0 0.0278748273849487 20.0
0 -0.00210821628570557 21.0
0 -0.016823410987854 22.0
27.770254747 27.27707862854 23.0
0 0.0799587965011597 24.0
0 0.449578046798706 25.0
0 0.0247970819473267 26.0
0 0.0104528069496155 27.0
0 0.0497235059738159 28.0
0 -0.0546718835830688 29.0
0 -0.0603643655776978 30.0
0 -0.026218056678772 31.0
0 -0.0356861352920532 32.0
0 0.0515114068984985 33.0
0 -0.0359946489334106 34.0
0 -0.0244172811508179 35.0
0 0.121090769767761 36.0
0 -0.0150266885757446 37.0
0 0.0336266756057739 38.0
0 0.0138314962387085 39.0
0 0.118306040763855 40.0
0 -0.086567759513855 41.0
0 -0.0707288980484009 42.0
289.84860216 296.191802978516 43.0
0 0.0878084897994995 44.0
0 -0.0551302433013916 45.0
0 0.00104939937591553 46.0
0 -0.0437932014465332 47.0
0 0.0210155248641968 48.0
0 0.270795106887817 49.0
0.070241881099 0.0523041486740112 50.0
2.2117969638 0.999995648860931 51.0
0 -0.0543771982192993 52.0
0 -0.096558690071106 53.0
0 0.0225664377212524 54.0
0 -0.0864506959915161 55.0
0 0.0395685434341431 56.0
0 -0.0428692102432251 57.0
0 -0.0659893751144409 58.0
0 -0.0575429201126099 59.0
0 -0.139204859733582 60.0
0 0.0435224771499634 61.0
0 0.789160966873169 62.0
267.52910161 244.967742919922 0.0
0 -0.113075852394104 1.0
0 -0.0613030195236206 2.0
0 -0.0746315717697144 3.0
0 -0.0391911268234253 4.0
0 -0.0183221101760864 5.0
0 0.0466092824935913 6.0
0 0.0524104833602905 7.0
0 -0.126513600349426 8.0
0 0.107492804527283 9.0
0 0.0118166208267212 10.0
0 0.0118308067321777 11.0
0 -0.0456339120864868 12.0
57.948408858 56.6223106384277 13.0
48.69157276 42.442813873291 14.0
0 -0.0631753206253052 15.0
0 -0.0544661283493042 16.0
26.179323087 26.2555675506592 17.0
0 0.00871312618255615 18.0
0 -0.0319238901138306 19.0
0 0.015142560005188 20.0
0 -0.0452574491500854 21.0
0 0.00956571102142334 22.0
35.083281152 35.3099899291992 23.0
0 -0.0855342149734497 24.0
0 0.620515048503876 25.0
0 0.105216860771179 26.0
0 0.0620971918106079 27.0
0 0.33365523815155 28.0
0 -0.121111512184143 29.0
0 0.0672425031661987 30.0
0 -0.0713073015213013 31.0
0 -0.00154697895050049 32.0
0 -0.0325983762741089 33.0
0 -0.068869948387146 34.0
0 0.0361553430557251 35.0
0 0.114861845970154 36.0
0 -0.0313824415206909 37.0
0 -0.0263968706130981 38.0
0 0.160626530647278 39.0
0 0.17761504650116 40.0
0 -0.103845477104187 41.0
0 0.010393500328064 42.0
302.64790712 296.988952636719 43.0
0 -0.0301481485366821 44.0
0 -0.0319300889968872 45.0
0 0.0176113843917847 46.0
0 0.0448759794235229 47.0
0 -0.144765496253967 48.0
1.5542339903 0.26411634683609 49.0
0.084084051117 0.0758322477340698 50.0
1.8979146465 1.02003908157349 51.0
0 -0.0719331502914429 52.0
0 -0.016480565071106 53.0
0 0.0716449022293091 54.0
0 -0.0840302705764771 55.0
0 -0.0533865690231323 56.0
0 0.0740503072738647 57.0
0 -0.0502933263778687 58.0
0 -0.0264879465103149 59.0
0 0.0787335634231567 60.0
0 0.0564476251602173 61.0
0 0.835177421569824 62.0
276.7675781 254.582916259766 0.0
0 -0.0754286050796509 1.0
0 -0.118214726448059 2.0
0 -0.03083336353302 3.0
0 -0.074515700340271 4.0
0 -0.0686155557632446 5.0
0 0.221597075462341 6.0
0 -0.155683159828186 7.0
0 -0.0744274854660034 8.0
0 0.0195544958114624 9.0
0 0.0103809833526611 10.0
0 0.0285775661468506 11.0
0 -0.109628319740295 12.0
53.446777152 53.3177223205566 13.0
42.805262223 40.7868003845215 14.0
0 -0.0402189493179321 15.0
0 -0.022361159324646 16.0
32.807290346 32.2072639465332 17.0
0 -0.0425199270248413 18.0
0 0.0144559144973755 19.0
0 -0.00484693050384521 20.0
0 0.00471675395965576 21.0
0 -0.051941990852356 22.0
39.955478534 39.923267364502 23.0
0 -0.0702100992202759 24.0
0 1.10618281364441 25.0
0 0.0444859266281128 26.0
0 -0.0420715808868408 27.0
0 0.0176719427108765 28.0
0 -0.0610548257827759 29.0
0 -0.0318671464920044 30.0
0 -0.0346728563308716 31.0
0 -0.0153318643569946 32.0
0 0.0880674123764038 33.0
0 0.00840628147125244 34.0
0 -0.0430017709732056 35.0
0 0.0915964841842651 36.0
0 -0.116995692253113 37.0
0 -0.000248074531555176 38.0
0 -0.011940598487854 39.0
0 -0.0202652215957642 40.0
0 0.00886881351470947 41.0
0 -0.0319439172744751 42.0
292.623804 296.056976318359 43.0
0 -0.041725754737854 44.0
0 -0.0124582052230835 45.0
0 -0.0128747224807739 46.0
0 -0.0699313879013062 47.0
0 -0.0335556268692017 48.0
0 0.277631014585495 49.0
0.099055284191 0.0736821889877319 50.0
1.7524794653 0.9999720454216 51.0
0 -0.0623615980148315 52.0
0 -0.0313640832901001 53.0
0 0.0430947542190552 54.0
0 -0.12942373752594 55.0
0 0.00795042514801025 56.0
0 -0.0328570604324341 57.0
0 -0.0594691038131714 58.0
0 0.0117028951644897 59.0
0 -0.0780540704727173 60.0
0 0.0399256944656372 61.0
0 0.811257004737854 62.0
271.87419061 254.672393798828 0.0
0 -0.0358368158340454 1.0
0 -0.0569838285446167 2.0
0 -0.00340592861175537 3.0
0 -0.023877739906311 4.0
0 -0.0206137895584106 5.0
0 0.0526751279830933 6.0
0 -0.108401894569397 7.0
0 -0.0525667071342468 8.0
0 0.0672725439071655 9.0
0 0.017328143119812 10.0
0 0.00521445274353027 11.0
0 0.114172339439392 12.0
58.560599776 56.0462226867676 13.0
49.13534769 42.0287857055664 14.0
0 0.0207442045211792 15.0
0 -0.0368443727493286 16.0
29.051016888 28.4997615814209 17.0
0 0.0680474042892456 18.0
0 0.0352104902267456 19.0
0 -0.00429236888885498 20.0
0 -0.0418719053268433 21.0
0 -0.0141927003860474 22.0
38.277677209 37.6062965393066 23.0
0 -0.0661479234695435 24.0
0 0.576255321502686 25.0
0 -0.0133730173110962 26.0
0 -0.0482501983642578 27.0
0 -0.0989032983779907 28.0
0 -0.0139940977096558 29.0
0 -0.0175718069076538 30.0
0 -0.00394594669342041 31.0
0 0.0312885046005249 32.0
0 -0.155804753303528 33.0
0 0.0581454038619995 34.0
0 0.00666892528533936 35.0
0 0.044465184211731 36.0
0 0.107797980308533 37.0
0 0.0282086133956909 38.0
0 -0.0536507368087769 39.0
0 0.0288480520248413 40.0
0 -0.0520135164260864 41.0
0 -0.0627256631851196 42.0
295.57065636 296.293670654297 43.0
0 -0.0397001504898071 44.0
0 -0.0657178163528442 45.0
0 -0.00048363208770752 46.0
0 -0.0435434579849243 47.0
0 -0.0924011468887329 48.0
0 0.351648449897766 49.0
0.086185050852 0.0633569955825806 50.0
1.9707942146 0.8833988904953 51.0
0 -0.0142377614974976 52.0
0 -0.0335720777511597 53.0
0 0.0194050073623657 54.0
0 -0.0606628656387329 55.0
0 0.015816330909729 56.0
0 -0.0374370813369751 57.0
0 -0.0671225786209106 58.0
0 -0.0623420476913452 59.0
0 -0.0287255048751831 60.0
0 0.0652767419815063 61.0
0 0.912397503852844 62.0
264.28042817 252.377639770508 0.0
0 -0.0707279443740845 1.0
0 -0.0584594011306763 2.0
0 -0.0612024068832397 3.0
0 -0.0249112844467163 4.0
0 -0.0605679750442505 5.0
0 0.0212348699569702 6.0
0 -0.0388432741165161 7.0
0 0.0316488742828369 8.0
0 -0.0496326684951782 9.0
0 -0.00571119785308838 10.0
0 0.00800603628158569 11.0
0 -0.0323225259780884 12.0
68.738699064 68.1120223999023 13.0
22.890194531 33.5450592041016 14.0
0 -0.0440326929092407 15.0
0 0.0704931020736694 16.0
33.592704445 32.2282485961914 17.0
0 -0.0281041860580444 18.0
0 -0.0365520715713501 19.0
0 0.0335489511489868 20.0
0 -0.00886571407318115 21.0
0 -0.00165188312530518 22.0
40.165505483 39.9812355041504 23.0
0 1.78813934326172e-06 24.0
0 1.44957995414734 25.0
0 -0.0271512269973755 26.0
0 0.0471007823944092 27.0
0 0.165822625160217 28.0
0 -0.0437484979629517 29.0
0 -0.134329676628113 30.0
0 -0.0618153810501099 31.0
0 0.104116320610046 32.0
0 0.00434386730194092 33.0
0 0.0200716257095337 34.0
0 -0.00712740421295166 35.0
0 -0.0839282274246216 36.0
0 -0.00239646434783936 37.0
0 -0.00909340381622314 38.0
0 0.0407065153121948 39.0
0 0.0817254781723022 40.0
0 -0.0281116962432861 41.0
0 -0.148957848548889 42.0
288.41706292 296.724487304688 43.0
0 -0.0561691522598267 44.0
0 0.0366060733795166 45.0
0 -0.013990044593811 46.0
0 -0.0593692064285278 47.0
0 0.0645030736923218 48.0
0 0.277603983879089 49.0
0.095738469781 0.0664142370223999 50.0
1.7935757239 0.825644850730896 51.0
0 -0.0562790632247925 52.0
0 -0.0574790239334106 53.0
0 -0.0423451662063599 54.0
0 -0.149151682853699 55.0
0 0.015708327293396 56.0
0 -0.00205981731414795 57.0
0 -0.0354584455490112 58.0
0 -0.0662773847579956 59.0
0 0.127734303474426 60.0
0 0.0292781591415405 61.0
0 0.954598426818848 62.0
237.6739775 254.931350708008 0.0
0 -0.0643211603164673 1.0
0 -0.0708786249160767 2.0
0 0.0202513933181763 3.0
0 -0.0469158887863159 4.0
0 -0.0506874322891235 5.0
0 -0.0861574411392212 6.0
0 -0.0765150785446167 7.0
0 0.0136572122573853 8.0
0 -0.0087200403213501 9.0
0 -0.0156517028808594 10.0
0 0.0105854272842407 11.0
0 0.0919679403305054 12.0
64.081995144 63.2227249145508 13.0
22.910176443 35.0084953308105 14.0
0 -0.0374504327774048 15.0
0 -0.0106345415115356 16.0
26.79861857 26.6680164337158 17.0
0 0.0212193727493286 18.0
0 0.0349811315536499 19.0
0 0.0164409875869751 20.0
0 -0.0406197309494019 21.0
0 0.020288348197937 22.0
31.14661445 29.9650020599365 23.0
0 0.0774730443954468 24.0
0 0.490351915359497 25.0
0 0.0769070386886597 26.0
0 0.0153809785842896 27.0
0 -0.0149732828140259 28.0
0 -0.0608624219894409 29.0
0 -0.0441864728927612 30.0
0 -0.0414255857467651 31.0
0 -0.0311084985733032 32.0
0 -0.0851966142654419 33.0
0 -0.0719715356826782 34.0
0 -0.0891646146774292 35.0
0 -0.0501338243484497 36.0
0 -0.0446318387985229 37.0
0 -0.00801980495452881 38.0
0 0.0665715932846069 39.0
0 0.0856016874313354 40.0
0 -0.0358422994613647 41.0
0 0.0162185430526733 42.0
293.56772263 296.692138671875 43.0
0 -0.0176950693130493 44.0
0 0.014741063117981 45.0
0 -0.0624443292617798 46.0
0 -0.028560996055603 47.0
0 -0.0158816576004028 48.0
0 0.319653272628784 49.0
0.080117158891 0.0282224416732788 50.0
2.1270971768 0.993272066116333 51.0
0 -0.075318455696106 52.0
0 -0.0624939203262329 53.0
0 0.0670052766799927 54.0
0 -0.125780940055847 55.0
0 -0.0448333024978638 56.0
0 -0.017001748085022 57.0
0 -0.00434315204620361 58.0
0 0.00253379344940186 59.0
0 0.0925074815750122 60.0
0 -0.0481139421463013 61.0
0 0.841289401054382 62.0
238.08467049 260.198760986328 0.0
0 -0.14533007144928 1.0
0 -0.0144730806350708 2.0
0 0.0315960645675659 3.0
0 -0.0259050130844116 4.0
0 -0.0243030786514282 5.0
0 0.0982259511947632 6.0
0 0.008431077003479 7.0
0 -0.0661743879318237 8.0
0 0.0990175008773804 9.0
0 0.0404049158096313 10.0
0 -0.0131093263626099 11.0
0 0.0873817205429077 12.0
49.393890078 47.8701934814453 13.0
56.169972799 48.0106620788574 14.0
0 -0.0684295892715454 15.0
0 -0.0495539903640747 16.0
41.442478508 41.3101844787598 17.0
0 -0.0199161767959595 18.0
0 0.00116598606109619 19.0
0 0.0170561075210571 20.0
0 0.006522536277771 21.0
0 -0.0625549554824829 22.0
42.169465219 42.4551811218262 23.0
0 0.0213149785995483 24.0
0 2.85180425643921 25.0
0 0.0749815702438354 26.0
0 0.0508147478103638 27.0
0 -0.0850106477737427 28.0
0 -0.0883759260177612 29.0
0 -0.0842552185058594 30.0
0 -0.0190116167068481 31.0
0 -0.105883240699768 32.0
0 0.0632613897323608 33.0
0 -0.00793612003326416 34.0
0 0.0848591923713684 35.0
0 -0.0892215967178345 36.0
0 0.072393536567688 37.0
0 -0.01893150806427 38.0
0 -0.0428084135055542 39.0
0 0.0947335958480835 40.0
0 0.0511373281478882 41.0
0 -0.0920196771621704 42.0
297.45743926 296.031066894531 43.0
0 0.00134217739105225 44.0
0 0.0660815834999084 45.0
0 -0.0171786546707153 46.0
0 -0.0196510553359985 47.0
0 0.235282778739929 48.0
2.5193321356 0.325775146484375 49.0
0.088491126352 0.0665088891983032 50.0
1.7220568437 0.882296562194824 51.0
0 -0.140424370765686 52.0
0 0.0910900831222534 53.0
0 0.0943266153335571 54.0
0 -0.215143322944641 55.0
0 0.193150639533997 56.0
0 0.0825084447860718 57.0
0 -0.0237869024276733 58.0
0 -0.0357276201248169 59.0
0 0.153127074241638 60.0
0 -0.0786751508712769 61.0
0 0.845967411994934 62.0
231.69073308 258.132141113281 0.0
0 -0.0289603471755981 1.0
0 -0.0364505052566528 2.0
0 -0.0313647985458374 3.0
0 -0.014710545539856 4.0
0 0.00646984577178955 5.0
0 0.0265611410140991 6.0
0 -0.0780714750289917 7.0
0 -0.0731712579727173 8.0
0 0.0782264471054077 9.0
0 -0.00238394737243652 10.0
0 0.0444040298461914 11.0
0 -0.0303217172622681 12.0
69.329267189 69.4652557373047 13.0
34.02027327 29.5556468963623 14.0
0 0.0257982015609741 15.0
0 -0.0334786176681519 16.0
30.643068212 30.5281448364258 17.0
0 0.0273393392562866 18.0
0 -0.0121690034866333 19.0
0 0.0470436811447144 20.0
0 0.0206652879714966 21.0
0 -0.0458194017410278 22.0
27.573599633 27.2607288360596 23.0
0 0.0657132863998413 24.0
0 0.492989480495453 25.0
0 0.115320563316345 26.0
0 0.0676416158676147 27.0
0 -0.0346728563308716 28.0
0 -0.039824366569519 29.0
0 -0.0652396678924561 30.0
0 -0.0308316946029663 31.0
0 -0.0118924379348755 32.0
0 -0.0795177221298218 33.0
0 -0.0289472341537476 34.0
0 -0.0617517232894897 35.0
0 0.007698655128479 36.0
0 -0.0387426614761353 37.0
0 0.0517948865890503 38.0
0 -0.0665837526321411 39.0
0 -0.0518242120742798 40.0
0 -0.0779505968093872 41.0
0 -0.0782762765884399 42.0
304.40464963 297.142608642578 43.0
0 -0.00101625919342041 44.0
0 0.040242075920105 45.0
0 0.0083688497543335 46.0
0 -0.0549538135528564 47.0
0 -0.00691854953765869 48.0
0 0.301770925521851 49.0
0.083854484104 0.110695004463196 50.0
1.8431739396 0.914567470550537 51.0
0 -0.0248111486434937 52.0
0 -0.0416029691696167 53.0
0 0.061435341835022 54.0
0 -0.0376280546188354 55.0
0 0.0379217863082886 56.0
0 -0.0365837812423706 57.0
0 -0.0395139455795288 58.0
0 -0.068745493888855 59.0
0 -0.0399218797683716 60.0
0 -0.0307368040084839 61.0
0 0.826597332954407 62.0
278.43598285 255.31755065918 0.0
0 -0.0219341516494751 1.0
0 -0.0208758115768433 2.0
0 0.0199047327041626 3.0
0 -0.00759804248809814 4.0
0 -0.0533128976821899 5.0
0 0.0129369497299194 6.0
0 -0.0874782800674438 7.0
0 -0.0122015476226807 8.0
0 0.0752271413803101 9.0
0 0.0199459791183472 10.0
0 -0.0243611931800842 11.0
0 -0.0831543207168579 12.0
74.55616663 73.9542999267578 13.0
15.971431952 26.3501625061035 14.0
0 -0.0208879709243774 15.0
0 -0.0119904279708862 16.0
29.912285666 30.1435871124268 17.0
0 -0.0142315626144409 18.0
0 -0.00935804843902588 19.0
0 0.0246034860610962 20.0
0 0.00850427150726318 21.0
0 0.0257371664047241 22.0
29.128546305 28.1598625183105 23.0
0 0.00388944149017334 24.0
0 0.582842171192169 25.0
0 0.054097056388855 26.0
0 -0.0029219388961792 27.0
0 -0.00625336170196533 28.0
0 -0.0197006464004517 29.0
0 -0.0381485223770142 30.0
0 -0.0469402074813843 31.0
0 0.00669896602630615 32.0
0 -0.00652682781219482 33.0
0 -0.055644154548645 34.0
0 0.0165548324584961 35.0
0 -0.136952519416809 36.0
0 -0.0641471147537231 37.0
0 0.00366032123565674 38.0
0 0.0111106634140015 39.0
0 0.0763696432113647 40.0
0 -0.0131980180740356 41.0
0 -0.0228906869888306 42.0
297.30166453 297.2041015625 43.0
0 -0.00828039646148682 44.0
0 0.0695109367370605 45.0
0 0.00763118267059326 46.0
0 -0.0181013345718384 47.0
0 -0.033511757850647 48.0
0 0.282637357711792 49.0
0.096140738484 0.0788208246231079 50.0
1.7937046086 0.941498637199402 51.0
0 -0.08198082447052 52.0
0 -0.0459376573562622 53.0
0 -0.00161206722259521 54.0
0 -0.106594920158386 55.0
0 0.0150707960128784 56.0
0 -0.037483811378479 57.0
0 -0.0604561567306519 58.0
0 -0.0553964376449585 59.0
0 -0.0746067762374878 60.0
0 -0.0203100442886353 61.0
0 0.832947492599487 62.0
263.28697621 254.70539855957 0.0
0 -0.00312340259552002 1.0
0 -0.0180898904800415 2.0
0 -0.0108743906021118 3.0
0 -0.0350123643875122 4.0
0 -0.0632635354995728 5.0
0 -0.00484812259674072 6.0
0 -0.189929127693176 7.0
0 -0.0454221963882446 8.0
0 -0.00883162021636963 9.0
0 0.0373004674911499 10.0
0 0.0434607267379761 11.0
0 0.0279067754745483 12.0
50.426212613 50.6229209899902 13.0
53.90100255 46.145133972168 14.0
0 0.0130971670150757 15.0
0 0.00733363628387451 16.0
38.202167545 37.2698974609375 17.0
0 -0.0259312391281128 18.0
0 -0.0345176458358765 19.0
0 0.0350381135940552 20.0
0 0.037946343421936 21.0
0 -0.0453587770462036 22.0
38.953191523 38.4716835021973 23.0
0 -0.0745769739151001 24.0
0 0.692050218582153 25.0
0 0.0196350812911987 26.0
0 0.106638550758362 27.0
0 -0.0675770044326782 28.0
0 -0.12911331653595 29.0
0 0.141530632972717 30.0
0 -0.012102484703064 31.0
0 0.00348317623138428 32.0
0 0.123155951499939 33.0
0 -0.0528913736343384 34.0
0 -0.0358760356903076 35.0
0 -0.0189973115921021 36.0
0 -0.0614203214645386 37.0
0 0.0383263826370239 38.0
0 0.0341979265213013 39.0
0 0.0762866735458374 40.0
0 0.105193257331848 41.0
0 -0.0213991403579712 42.0
301.52837854 296.983551025391 43.0
0 -0.00293505191802979 44.0
0 0.0630874633789062 45.0
0 -0.00895655155181885 46.0
0 -0.0525079965591431 47.0
0 -0.0281778573989868 48.0
0 0.236838459968567 49.0
0.089416773702 0.211796402931213 50.0
2.2057718245 0.939453363418579 51.0
0 -0.0696958303451538 52.0
0 -0.0247315168380737 53.0
0 0.0489665269851685 54.0
0 -0.164841532707214 55.0
0 -0.0175167322158813 56.0
0 0.0459984540939331 57.0
0 -0.0407041311264038 58.0
0 -0.0301984548568726 59.0
0 -0.0123640298843384 60.0
0 0.0692399740219116 61.0
0 0.93505871295929 62.0
247.25848636 255.294586181641 0.0
0 -0.123757481575012 1.0
0 0.0321210622787476 2.0
0 -0.00700438022613525 3.0
0 -0.035453200340271 4.0
0 0.000155806541442871 5.0
0 0.0287050008773804 6.0
0 -0.0295134782791138 7.0
0 -0.0530807375907898 8.0
0 0.000101447105407715 9.0
0 0.00209271907806396 10.0
0 -0.00634002685546875 11.0
0 0.0663713216781616 12.0
66.40480806 66.3939819335938 13.0
26.769849994 30.4084911346436 14.0
0 -0.0267945528030396 15.0
0 -0.0364896059036255 16.0
35.326647863 34.9908103942871 17.0
0 0.0253614187240601 18.0
0 -0.000389218330383301 19.0
0 0.0216847658157349 20.0
0 -0.00976336002349854 21.0
0 -0.0235334634780884 22.0
35.6833838 35.1200981140137 23.0
0 -0.0840121507644653 24.0
0 1.48895263671875 25.0
0 0.0541819334030151 26.0
0 -0.0389617681503296 27.0
0 -0.00458681583404541 28.0
0 0.000911831855773926 29.0
0 -0.0598553419113159 30.0
0 -0.0270730257034302 31.0
0 -0.030823826789856 32.0
0 -0.101430296897888 33.0
0 -0.0186079740524292 34.0
0 -0.11365282535553 35.0
0 -0.0144561529159546 36.0
0 -0.0380469560623169 37.0
0 0.0488127470016479 38.0
0 0.00427877902984619 39.0
0 0.0108708143234253 40.0
0 0.0284173488616943 41.0
0 0.0275157690048218 42.0
296.08796145 297.538940429688 43.0
0 0.00877797603607178 44.0
0 0.0919153690338135 45.0
0 -0.0105551481246948 46.0
0 -0.0285824537277222 47.0
0 0.0238970518112183 48.0
0 0.294882774353027 49.0
0.090433382039 0.0710803270339966 50.0
1.6931523154 0.97013533115387 51.0
0 -0.0448306798934937 52.0
0 -0.04167640209198 53.0
0 0.0290976762771606 54.0
0 -0.0353306531906128 55.0
0 0.0159808397293091 56.0
0 -0.0261269807815552 57.0
0 -0.0425084829330444 58.0
0 -0.0278090238571167 59.0
0 -0.0856338739395142 60.0
0 0.0420283079147339 61.0
0 0.814576625823975 62.0
275.35752484 255.812225341797 0.0
0 -0.0541282892227173 1.0
0 -0.105722546577454 2.0
0 -0.0159319639205933 3.0
0 -0.0354524850845337 4.0
0 -0.115829586982727 5.0
0 0.052754282951355 6.0
0 -0.00152695178985596 7.0
0 -0.00709438323974609 8.0
0 0.08092200756073 9.0
0 -0.0692473649978638 10.0
0 -0.0211746692657471 11.0
0 -0.0970121622085571 12.0
54.670564935 54.1448097229004 13.0
49.192071519 47.9785995483398 14.0
0 -0.0498090982437134 15.0
0 0.0339382886886597 16.0
36.467767105 36.0118675231934 17.0
0 -0.0389715433120728 18.0
0 0.0383175611495972 19.0
0 -0.0132433176040649 20.0
0 -0.00482475757598877 21.0
0 -0.0453716516494751 22.0
40.173398488 39.7575187683105 23.0
0 -0.00782930850982666 24.0
0 0.510306835174561 25.0
0 0.0193585157394409 26.0
0 -0.0357271432876587 27.0
0 -0.0513268709182739 28.0
0 -0.0729819536209106 29.0
0 -0.0181223154067993 30.0
0 -0.0287445783615112 31.0
0 -0.00944101810455322 32.0
0 -0.0279186964035034 33.0
0 -0.109357237815857 34.0
0 0.00315499305725098 35.0
0 -0.00563275814056396 36.0
0 -0.034716010093689 37.0
0 0.0381914377212524 38.0
0 0.0344969034194946 39.0
0 -0.0239092111587524 40.0
0 -0.08472740650177 41.0
0 -0.0647608041763306 42.0
305.75759646 294.211730957031 43.0
0 0.007590651512146 44.0
0 0.0409827828407288 45.0
0 0.0569325685501099 46.0
0 -0.0980859994888306 47.0
0 -0.0719276666641235 48.0
0 0.281202018260956 49.0
0.10424575315 0.0944880247116089 50.0
2.206269212 0.921171069145203 51.0
0 -0.13542115688324 52.0
0 -0.0846515893936157 53.0
0 0.0967863798141479 54.0
0 -0.0271707773208618 55.0
0 -0.106480479240417 56.0
0 -0.0710684061050415 57.0
0 -0.0577281713485718 58.0
0 -0.0365792512893677 59.0
0 0.00457727909088135 60.0
0 0.0535188913345337 61.0
0 0.944741666316986 62.0
240.05350175 258.706573486328 0.0
0 -0.0612550973892212 1.0
0 0.0131114721298218 2.0
0 -0.0121437311172485 3.0
0 -0.0760208368301392 4.0
0 -0.0592204332351685 5.0
0 -0.00863277912139893 6.0
0 -0.00146043300628662 7.0
0 -0.0445132255554199 8.0
0 -0.0859056711196899 9.0
0 -0.00422227382659912 10.0
0 -0.0376302003860474 11.0
0 0.0844053030014038 12.0
63.322695604 62.6787948608398 13.0
29.663199033 36.848705291748 14.0
0 -0.0282198190689087 15.0
0 -0.029300332069397 16.0
32.535684782 32.0761260986328 17.0
0 0.0114079713821411 18.0
0 0.0296477079391479 19.0
0 0.00558841228485107 20.0
0 -0.0413800477981567 21.0
0 0.0143688917160034 22.0
38.721581019 37.042652130127 23.0
0 0.00195062160491943 24.0
0 0.533707976341248 25.0
0 0.0383104085922241 26.0
0 0.0308011770248413 27.0
0 0.122433066368103 28.0
0 -0.0101999044418335 29.0
0 -0.0339579582214355 30.0
0 -0.000547289848327637 31.0
0 0.00191390514373779 32.0
0 -0.000800490379333496 33.0
0 -0.0500048398971558 34.0
0 -0.128663897514343 35.0
0 -0.0263668298721313 36.0
0 -0.0543795824050903 37.0
0 0.0117686986923218 38.0
0 0.0205065011978149 39.0
0 0.141326308250427 40.0
0 0.0328118801116943 41.0
0 -0.105201363563538 42.0
315.08246822 294.492065429688 43.0
0 -0.00454866886138916 44.0
0 0.0476161241531372 45.0
0 -0.0542184114456177 46.0
0 0.00533449649810791 47.0
0 -0.00114357471466064 48.0
0.75763162536 0.252555966377258 49.0
0.072996309458 0.0752646923065186 50.0
1.8564770029 0.84411084651947 51.0
0 -0.0740834474563599 52.0
0 -0.030063271522522 53.0
0 0.0228766202926636 54.0
0 -0.12052857875824 55.0
0 -0.10614287853241 56.0
0 0.0098494291305542 57.0
0 0.00976979732513428 58.0
0 -0.102390646934509 59.0
0 -0.0134738683700562 60.0
0 -0.00422942638397217 61.0
0 0.804505228996277 62.0
253.12686262 258.456359863281 0.0
0 -0.0807715654373169 1.0
0 -0.0649713277816772 2.0
0 -0.00180709362030029 3.0
0 -0.0351642370223999 4.0
0 0.0645862817764282 5.0
0 0.0435956716537476 6.0
0 -0.123659491539001 7.0
0 0.0261392593383789 8.0
0 0.0565828084945679 9.0
0 -0.00457918643951416 10.0
0 0.0278578996658325 11.0
0 0.00591790676116943 12.0
50.893101907 49.8532562255859 13.0
37.764263818 38.7738304138184 14.0
0 -0.0525680780410767 15.0
0 0.0323752164840698 16.0
29.409633191 28.9119529724121 17.0
0 0.0622960329055786 18.0
0 0.02251136302948 19.0
0 -0.0276347398757935 20.0
0 -0.0546513795852661 21.0
0 0.000618815422058105 22.0
37.427133781 37.4324073791504 23.0
0 0.0670655965805054 24.0
0 0.4531409740448 25.0
0 0.0762776136398315 26.0
0 0.0605741739273071 27.0
0 0.00030982494354248 28.0
0 -0.0578387975692749 29.0
0 -0.135278582572937 30.0
0 -0.138731360435486 31.0
0 -0.0369359254837036 32.0
0 -0.0315085649490356 33.0
0 -0.0463470220565796 34.0
0 -0.0957554578781128 35.0
0 0.0609604120254517 36.0
0 -0.0350679159164429 37.0
0 -0.0150378942489624 38.0
0 0.0811198949813843 39.0
0 0.0770648717880249 40.0
0 -0.0897022485733032 41.0
0 -0.0586181879043579 42.0
285.5591431 295.279693603516 43.0
0 0.0383871793746948 44.0
0 -0.0395365953445435 45.0
0 -0.00317704677581787 46.0
0 -0.0395494699478149 47.0
0 0.0898646116256714 48.0
0 0.238276541233063 49.0
0.066117360775 0.0520020723342896 50.0
2.1400181273 0.944484233856201 51.0
0 -0.086833119392395 52.0
0 -0.0319417715072632 53.0
0 -0.0133355855941772 54.0
0 -0.0471318960189819 55.0
0 -0.0574167966842651 56.0
0 0.026964545249939 57.0
0 -0.0237699747085571 58.0
0 -0.00618946552276611 59.0
0 0.0920873880386353 60.0
0 0.0580593347549438 61.0
0 0.914033353328705 62.0
243.31665632 255.157424926758 0.0
0 -0.0593277215957642 1.0
0 -0.0218645334243774 2.0
0 0.0347682237625122 3.0
0 -0.0481010675430298 4.0
0 -0.0496991872787476 5.0
0 -0.0850702524185181 6.0
0 -0.136224627494812 7.0
0 0.00125396251678467 8.0
0 0.0240310430526733 9.0
0 0.00735962390899658 10.0
0 -0.0275266170501709 11.0
0 -0.108265519142151 12.0
56.307807227 55.7018547058105 13.0
38.135639672 40.7691917419434 14.0
0 -0.098178505897522 15.0
0 -0.0186735391616821 16.0
41.718718759 41.2650451660156 17.0
0 0.0582646131515503 18.0
0 0.0263656377792358 19.0
0 -0.0133141279220581 20.0
0 -0.0461970567703247 21.0
0 0.0284379720687866 22.0
41.364386369 40.6706237792969 23.0
0 0.0941189527511597 24.0
0 1.24193930625916 25.0
0 -0.0136820077896118 26.0
0 -0.00418287515640259 27.0
0 0.00933444499969482 28.0
0 -0.0266681909561157 29.0
0 -0.131966710090637 30.0
0 -0.0359874963760376 31.0
0 -0.0193144083023071 32.0
0 0.0248979330062866 33.0
0 -0.0333739519119263 34.0
0 0.302133917808533 35.0
0 0.0789543390274048 36.0
0 -0.00321996212005615 37.0
0 0.129054665565491 38.0
0 -0.00647413730621338 39.0
0 0.00785386562347412 40.0
0 -0.1202632188797 41.0
0 0.039267897605896 42.0
308.72303722 296.362518310547 43.0
0 0.0374044179916382 44.0
0 0.0195153951644897 45.0
0 -0.0530160665512085 46.0
0 0.0366731882095337 47.0
0 0.133770823478699 48.0
0 0.300861418247223 49.0
0.10466111364 0.0728491544723511 50.0
2.1360196043 0.858350872993469 51.0
0 -0.0608392953872681 52.0
0 -0.0143090486526489 53.0
0 0.0471946001052856 54.0
0 -0.020922064781189 55.0
0 -0.0286916494369507 56.0
0 -0.0257669687271118 57.0
0 -0.0585402250289917 58.0
0 -0.075046181678772 59.0
0 -0.0905131101608276 60.0
0 0.0702813863754272 61.0
0 0.887072205543518 62.0
240.98978493 260.177429199219 0.0
0 -0.116285920143127 1.0
0 0.0347744226455688 2.0
0 0.00345337390899658 3.0
0 -0.0411826372146606 4.0
0 -0.0594500303268433 5.0
0 -0.0723291635513306 6.0
0 -0.0796061754226685 7.0
0 0.0197287797927856 8.0
0 -0.0167967081069946 9.0
0 0.016756534576416 10.0
0 -0.0466098785400391 11.0
0 -0.0219789743423462 12.0
65.386252069 65.048210144043 13.0
37.448316432 30.4118022918701 14.0
0 -0.0993934869766235 15.0
0 0.0351530313491821 16.0
28.490731958 27.4497833251953 17.0
0 0.0395838022232056 18.0
0 0.0187455415725708 19.0
0 0.0239237546920776 20.0
0 -0.0491534471511841 21.0
0 -0.00298178195953369 22.0
36.610466664 36.1667900085449 23.0
0 0.0136922597885132 24.0
0 1.3896986246109 25.0
0 0.00486838817596436 26.0
0 0.014632523059845 27.0
0 0.0046461820602417 28.0
0 -0.0936166048049927 29.0
0 0.154524207115173 30.0
0 -0.0392566919326782 31.0
0 -0.0305701494216919 32.0
0 0.0193721055984497 33.0
0 -0.102357506752014 34.0
0 -0.0233904123306274 35.0
0 -0.102624535560608 36.0
0 -0.0299338102340698 37.0
0 0.0258678197860718 38.0
0 -0.0297900438308716 39.0
0 -0.101355195045471 40.0
0 -0.00793254375457764 41.0
0 -0.11997377872467 42.0
283.35715287 296.281372070312 43.0
0 0.00451123714447021 44.0
0 0.107431054115295 45.0
0 -0.019694447517395 46.0
0 0.0129581689834595 47.0
0 -0.00307762622833252 48.0
0 0.249767661094666 49.0
0.089966727012 0.0862512588500977 50.0
2.2483497882 0.942165493965149 51.0
0 -0.0583044290542603 52.0
0 -0.0503760576248169 53.0
0 0.0626713037490845 54.0
0 -0.121096968650818 55.0
0 -0.0444265604019165 56.0
0 -0.0125380754470825 57.0
0 -0.0677987337112427 58.0
0 -0.0888971090316772 59.0
0 -0.0512253046035767 60.0
0 0.0607532262802124 61.0
0 0.931640923023224 62.0
232.79105847 247.389297485352 0.0
0 -0.0316611528396606 1.0
0 -0.0573583841323853 2.0
0 -0.00180232524871826 3.0
0 -0.0130466222763062 4.0
0 0.0216585397720337 5.0
0 -0.0183321237564087 6.0
0 -0.0740662813186646 7.0
0 -0.0504031181335449 8.0
0 0.0407537221908569 9.0
0 -0.0272644758224487 10.0
0 -0.0492402315139771 11.0
0 0.00160634517669678 12.0
68.23933816 66.9032897949219 13.0
19.850124225 31.5854167938232 14.0
0 -0.0291625261306763 15.0
0 -0.0378514528274536 16.0
37.444178629 37.2225303649902 17.0
0 -0.0142492055892944 18.0
0 -0.0147570371627808 19.0
0 0.0316227674484253 20.0
0 -0.0515388250350952 21.0
0 -0.0299540758132935 22.0
38.713392688 38.1380882263184 23.0
0 0.0502051115036011 24.0
0 0.58931577205658 25.0
0 0.0533040761947632 26.0
0 0.039018988609314 27.0
0 0.0971788167953491 28.0
0 -0.0433462858200073 29.0
0 -0.030684232711792 30.0
0 -0.0636509656906128 31.0
0 -0.0858436822891235 32.0
0 0.0543407201766968 33.0
0 -0.0164254903793335 34.0
0 -0.0715184211730957 35.0
0 0.0302339792251587 36.0
0 -0.061200737953186 37.0
0 0.0438324213027954 38.0
0 0.114551424980164 39.0
0 0.0804990530014038 40.0
0 -0.0386180877685547 41.0
0 8.52346420288086e-05 42.0
294.58802633 299.163269042969 43.0
0 -0.0101171731948853 44.0
0 0.079857349395752 45.0
0 -0.00278007984161377 46.0
0 -0.0644193887710571 47.0
0 0.0175451040267944 48.0
0 0.316881656646729 49.0
0.095238279977 0.0533643960952759 50.0
2.1742159562 0.953604698181152 51.0
0 -0.0515850782394409 52.0
0 -0.0366030931472778 53.0
0 0.0612400770187378 54.0
0 -0.0821307897567749 55.0
0 -0.0550762414932251 56.0
0 -0.00643360614776611 57.0
0 0.00276577472686768 58.0
0 -0.0923749208450317 59.0
0 -0.079872727394104 60.0
0 -0.00172650814056396 61.0
0 0.821526527404785 62.0
273.89602809 255.744873046875 0.0
0 0.0510514974594116 1.0
0 -0.0181056261062622 2.0
0 0.0136076211929321 3.0
0 -0.0133820772171021 4.0
0 -0.102367520332336 5.0
0 -0.0215116739273071 6.0
0 -0.0845996141433716 7.0
0 -0.0276724100112915 8.0
0 -0.0108305215835571 9.0
0 -0.0728691816329956 10.0
0 -0.0319068431854248 11.0
0 -0.162634491920471 12.0
59.628445198 58.4894332885742 13.0
31.85481259 37.6287269592285 14.0
0 -0.0408321619033813 15.0
0 -0.0299323797225952 16.0
27.946023088 27.8165969848633 17.0
0 -0.0182520151138306 18.0
0 0.0083085298538208 19.0
0 0.0150953531265259 20.0
0 0.028821587562561 21.0
0 -0.0815478563308716 22.0
36.602684639 36.4353828430176 23.0
0 0.00831258296966553 24.0
0 0.509353399276733 25.0
0 0.124632716178894 26.0
0 0.0420480966567993 27.0
0 0.0598334074020386 28.0
0 -0.0144287347793579 29.0
0 -0.0132937431335449 30.0
0 -0.0364576578140259 31.0
0 -0.016886830329895 32.0
0 0.00497782230377197 33.0
0 -0.0372661352157593 34.0
0 -0.0338102579116821 35.0
0 -0.0359758138656616 36.0
0 -0.0619598627090454 37.0
0 0.00187385082244873 38.0
0 0.049568772315979 39.0
0 -0.037023663520813 40.0
0 -0.0559781789779663 41.0
0 0.0904992818832397 42.0
287.01050639 296.626281738281 43.0
0 -0.0316239595413208 44.0
0 0.0712302923202515 45.0
0 -0.00354158878326416 46.0
0 -0.0380243062973022 47.0
0 -0.00946342945098877 48.0
0 0.315373659133911 49.0
0.071999223948 0.0552247762680054 50.0
1.8788176877 0.950129210948944 51.0
0 -0.082586407661438 52.0
0 -0.0384644269943237 53.0
0 0.0628122091293335 54.0
0 -0.111737847328186 55.0
0 -0.00799524784088135 56.0
0 -0.0218032598495483 57.0
0 0.0221823453903198 58.0
0 0.0219169855117798 59.0
0 -0.0820075273513794 60.0
0 -0.0471185445785522 61.0
0 0.851595044136047 62.0
273.18578668 254.698272705078 0.0
0 0.0517925024032593 1.0
0 -0.042965292930603 2.0
0 -0.000779032707214355 3.0
0 -0.0344730615615845 4.0
0 0.0222667455673218 5.0
0 0.0752271413803101 6.0
0 -0.0767523050308228 7.0
0 -0.0324380397796631 8.0
0 -0.0599552392959595 9.0
0 0.00783455371856689 10.0
0 0.021060049533844 11.0
0 0.0583709478378296 12.0
69.813436847 68.4916229248047 13.0
27.65636747 31.7518901824951 14.0
0 -0.0276418924331665 15.0
0 0.0151103734970093 16.0
35.885270061 35.1751594543457 17.0
0 0.00112020969390869 18.0
0 -0.033173680305481 19.0
0 -0.027845025062561 20.0
0 -0.0113316774368286 21.0
0 -0.0235825777053833 22.0
34.728203494 34.2273254394531 23.0
0 -0.0441571474075317 24.0
0 0.549997568130493 25.0
0 0.0889261960983276 26.0
0 0.1280757188797 27.0
0 -0.0527673959732056 28.0
0 -0.014367938041687 29.0
0 -0.0607861280441284 30.0
0 -0.0111347436904907 31.0
0 0.0380371809005737 32.0
0 0.112747073173523 33.0
0 -0.0621086359024048 34.0
0 0.026699423789978 35.0
0 -0.051488995552063 36.0
0 -0.0954726934432983 37.0
0 -0.053402304649353 38.0
0 0.126902937889099 39.0
0 0.130492568016052 40.0
0 -0.0640543699264526 41.0
0 0.0496753454208374 42.0
308.06470882 297.96484375 43.0
0 0.0367916822433472 44.0
0 -0.0327351093292236 45.0
0 0.00359857082366943 46.0
0 -0.0291799306869507 47.0
0 0.0332268476486206 48.0
0 0.279918372631073 49.0
0.082051525957 0.0603114366531372 50.0
1.7240553443 0.890710890293121 51.0
0 -0.0928303003311157 52.0
0 -0.0366808176040649 53.0
0 0.00694382190704346 54.0
0 -0.110867142677307 55.0
0 -0.00442993640899658 56.0
0 -0.0286508798599243 57.0
0 -0.0624042749404907 58.0
0 -0.00529778003692627 59.0
0 -0.0767916440963745 60.0
0 0.0231081247329712 61.0
0 0.826030373573303 62.0
267.03090172 254.936431884766 0.0
0 -0.083318829536438 1.0
0 0.0345960855484009 2.0
0 0.00681483745574951 3.0
0 -0.0602661371231079 4.0
0 0.0164417028427124 5.0
0 -0.0389448404312134 6.0
0 -0.0267468690872192 7.0
0 -0.0219035148620605 8.0
0 -0.0194002389907837 9.0
0 0.0223120450973511 10.0
0 -0.0176217555999756 11.0
0 -0.113300919532776 12.0
52.935446864 53.1901397705078 13.0
38.251603723 37.3972473144531 14.0
0 -0.0496693849563599 15.0
0 -0.020464301109314 16.0
31.53322104 31.6649971008301 17.0
0 -0.0117834806442261 18.0
0 -0.0328487157821655 19.0
0 -0.020386815071106 20.0
0 -0.0427016019821167 21.0
0 -0.0102497339248657 22.0
38.536060328 38.0010070800781 23.0
0 -0.0224329233169556 24.0
0 0.528192222118378 25.0
0 0.0538061857223511 26.0
0 -0.00767934322357178 27.0
0 0.00347506999969482 28.0
0 -0.0316950082778931 29.0
0 -0.0461174249649048 30.0
0 -0.0260878801345825 31.0
0 -0.00376284122467041 32.0
0 -0.0201631784439087 33.0
0 -0.0594152212142944 34.0
0 0.0299721956253052 35.0
0 -0.0517591238021851 36.0
0 -0.0745292901992798 37.0
0 -0.0300811529159546 38.0
0 0.0835472345352173 39.0
0 0.111759543418884 40.0
0 -0.0533720254898071 41.0
0 0.0326370000839233 42.0
290.22042979 291.987640380859 43.0
0 -0.000102639198303223 44.0
0 -0.0586923360824585 45.0
0 0.00172674655914307 46.0
0 -0.0503534078598022 47.0
0 0.0280460119247437 48.0
0 0.332646608352661 49.0
0.088926200362 0.0672124624252319 50.0
2.1758567949 0.991863608360291 51.0
0 -0.0362788438796997 52.0
0 -0.0223466157913208 53.0
0 0.0648561716079712 54.0
0 -0.144271731376648 55.0
0 -0.023403525352478 56.0
0 0.0244482755661011 57.0
0 -0.041730523109436 58.0
0 -0.0685464143753052 59.0
0 -0.0321336984634399 60.0
0 -0.0594633817672729 61.0
0 0.824174046516418 62.0
254.48590567 258.055358886719 0.0
0 -0.0462452173233032 1.0
0 -0.0896269083023071 2.0
0 -0.00953376293182373 3.0
0 -0.0115047693252563 4.0
0 -0.0368596315383911 5.0
0 -0.028572678565979 6.0
0 0.00477921962738037 7.0
0 -0.0313345193862915 8.0
0 0.0929588079452515 9.0
0 0.00196850299835205 10.0
0 -0.0450801849365234 11.0
0 -0.135448098182678 12.0
53.620945825 51.7425079345703 13.0
43.729769586 44.8626518249512 14.0
0 -0.0515867471694946 15.0
0 -0.000788569450378418 16.0
32.924975407 32.7801704406738 17.0
0 -0.0254412889480591 18.0
0 -0.00835812091827393 19.0
0 0.0431927442550659 20.0
0 -0.0322884321212769 21.0
0 -0.0392576456069946 22.0
38.87677847 38.221736907959 23.0
0 -0.0359774827957153 24.0
0 1.20131015777588 25.0
0 -0.00970947742462158 26.0
0 0.0513910055160522 27.0
0 -0.0502544641494751 28.0
0 -0.0347586870193481 29.0
0 -0.067756175994873 30.0
0 -0.0386539697647095 31.0
0 -0.00319302082061768 32.0
0 0.105988383293152 33.0
0 -0.00602471828460693 34.0
0 -0.0396589040756226 35.0
0 0.0943571329116821 36.0
0 -0.0424994230270386 37.0
0 -0.00499999523162842 38.0
0 0.026110053062439 39.0
0 0.0469206571578979 40.0
0 -0.0223865509033203 41.0
0 0.0264896154403687 42.0
291.89860567 297.387420654297 43.0
0 4.42266464233398e-05 44.0
0 0.0454624891281128 45.0
0 -0.0347481966018677 46.0
0 -0.0401893854141235 47.0
0 -0.00716817378997803 48.0
0 0.24886691570282 49.0
0.07718011008 0.0644365549087524 50.0
2.2727167458 0.959786832332611 51.0
0 -0.0353971719741821 52.0
0 -0.0275388956069946 53.0
0 0.00237548351287842 54.0
0 -0.038756251335144 55.0
0 -0.00891458988189697 56.0
0 -0.024387001991272 57.0
0 -0.0564261674880981 58.0
0 -0.0147460699081421 59.0
0 -0.0860015153884888 60.0
0 -0.0294469594955444 61.0
0 0.815115451812744 62.0
226.70556468 258.178039550781 0.0
0 -0.110284447669983 1.0
0 -0.0267902612686157 2.0
0 -0.0518492460250854 3.0
0 -0.0146294832229614 4.0
0 -0.0241242647171021 5.0
0 -0.0288668870925903 6.0
0 -0.00610268115997314 7.0
0 -0.0203520059585571 8.0
0 -0.0439811944961548 9.0
0 0.0038529634475708 10.0
0 -0.0517348051071167 11.0
0 -0.109716057777405 12.0
59.995149779 59.1808700561523 13.0
25.132018495 35.9796257019043 14.0
0 -0.0308681726455688 15.0
0 -0.0325652360916138 16.0
36.172706389 36.1049118041992 17.0
0 0.00107014179229736 18.0
0 0.0189639329910278 19.0
0 0.032260537147522 20.0
0 -0.0409461259841919 21.0
0 0.0216697454452515 22.0
38.296255206 38.3338279724121 23.0
0 0.000432848930358887 24.0
0 0.476216554641724 25.0
0 0.0747760534286499 26.0
0 0.0143283605575562 27.0
0 -0.0228220224380493 28.0
0 -0.0395940542221069 29.0
0 -0.0431127548217773 30.0
0 -0.0532358884811401 31.0
0 0.0105899572372437 32.0
0 -0.0126322507858276 33.0
0 -0.00232112407684326 34.0
0 -0.0475019216537476 35.0
0 0.121525406837463 36.0
0 -0.0133000612258911 37.0
0 -0.0112529993057251 38.0
0 0.0418399572372437 39.0
0 0.096361517906189 40.0
0 -0.0554536581039429 41.0
0 0.0120929479598999 42.0
304.32988153 297.121673583984 43.0
0 -0.0247474908828735 44.0
0 0.0745762586593628 45.0
0 0.00283515453338623 46.0
0 -0.00378286838531494 47.0
0 0.0409730672836304 48.0
0 0.294106304645538 49.0
0.083995459468 0.042188286781311 50.0
1.9168600104 0.867667257785797 51.0
0 -0.0734473466873169 52.0
0 -0.0881439447402954 53.0
0 -0.00686943531036377 54.0
0 -0.0563141107559204 55.0
0 -0.076166033744812 56.0
0 0.00458896160125732 57.0
0 -0.086342453956604 58.0
0 -0.01508629322052 59.0
0 -0.0890246629714966 60.0
0 0.0195022821426392 61.0
0 0.790438652038574 62.0
272.88998576 244.70426940918 0.0
0 -0.101648926734924 1.0
0 -0.0477756261825562 2.0
0 -0.0315967798233032 3.0
0 0.0113238096237183 4.0
0 -0.117146849632263 5.0
0 -0.0247598886489868 6.0
0 -0.342949509620667 7.0
0 0.00544607639312744 8.0
0 -0.0677505731582642 9.0
0 -0.0360370874404907 10.0
0 0.00589358806610107 11.0
0 0.0179723501205444 12.0
72.888309015 70.2112274169922 13.0
31.394920153 30.4104919433594 14.0
0 -0.0560413599014282 15.0
0 -0.0785781145095825 16.0
41.864460265 41.909984588623 17.0
0 0.00640022754669189 18.0
0 0.0383826494216919 19.0
0 0.0309733152389526 20.0
0 -0.0187793970108032 21.0
0 -0.0089646577835083 22.0
27.170517125 27.3894157409668 23.0
0 0.0931533575057983 24.0
0 0.679673552513123 25.0
0 0.0150550603866577 26.0
0 0.0503569841384888 27.0
0 0.0101760625839233 28.0
0 -0.0470734834671021 29.0
0 -0.0926004648208618 30.0
0 0.0389117002487183 31.0
0 0.0737332105636597 32.0
0 0.201919674873352 33.0
0 -0.00052940845489502 34.0
0 -0.0786316394805908 35.0
0 0.132290244102478 36.0
0 -0.102173447608948 37.0
0 0.0449675321578979 38.0
0 -0.105593204498291 39.0
0 0.12370240688324 40.0
0 -0.0126477479934692 41.0
0 0.0209077596664429 42.0
297.81925338 293.430267333984 43.0
0 0.0125049352645874 44.0
0 0.110462486743927 45.0
0 -0.0578948259353638 46.0
0 -0.0410031080245972 47.0
0 -0.000177979469299316 48.0
0 0.251498222351074 49.0
0.097432028081 0.0567723512649536 50.0
1.3328699386 1.00411713123322 51.0
0 -0.0210283994674683 52.0
0 0.028171181678772 53.0
0 -0.00509464740753174 54.0
0 -0.0983539819717407 55.0
0 -0.0496736764907837 56.0
0 -0.0320931673049927 57.0
0 -0.0459471940994263 58.0
0 -0.0396639108657837 59.0
0 0.0122107267379761 60.0
0 -0.0188103914260864 61.0
0 0.827493786811829 62.0
243.74712382 253.974349975586 0.0
0 -0.0080181360244751 1.0
0 -0.160569310188293 2.0
0 0.025443434715271 3.0
0 -0.0514985322952271 4.0
0 0.0457800626754761 5.0
0 0.0467818975448608 6.0
0 -0.0302342176437378 7.0
0 -0.00850534439086914 8.0
0 -0.0240408182144165 9.0
0 -0.0360506772994995 10.0
0 0.0133785605430603 11.0
0 -0.0502010583877563 12.0
68.418311195 68.6396789550781 13.0
26.080632503 27.1510601043701 14.0
0 0.0178693532943726 15.0
0 -0.0407634973526001 16.0
37.331059354 36.6891288757324 17.0
0 0.00576627254486084 18.0
0 0.0136297941207886 19.0
0 -0.00137531757354736 20.0
0 -0.0258830785751343 21.0
0 0.02959144115448 22.0
35.95065135 34.6798248291016 23.0
0 0.0305212736129761 24.0
0 0.535253524780273 25.0
0 0.00872170925140381 26.0
0 0.0194180607795715 27.0
0 -0.0647755861282349 28.0
0 -0.0390383005142212 29.0
0 -0.0575705766677856 30.0
0 -0.0150908231735229 31.0
0 -0.0286246538162231 32.0
0 0.0129631757736206 33.0
0 -0.0115417242050171 34.0
0 -0.0235713720321655 35.0
0 0.0135468244552612 36.0
0 -0.0413728952407837 37.0
0 0.0198272466659546 38.0
0 0.0567477941513062 39.0
0 0.0414727926254272 40.0
0 -0.0621649026870728 41.0
0 0.0380624532699585 42.0
309.34496135 297.172180175781 43.0
0 -0.00996077060699463 44.0
0 -0.0309925079345703 45.0
0 0.00468170642852783 46.0
0 0.0191766023635864 47.0
0 0.00388705730438232 48.0
0 0.288278818130493 49.0
0.067910842071 0.0617974996566772 50.0
1.6421269724 0.8949294090271 51.0
0 -0.0516573190689087 52.0
0 -0.0519877672195435 53.0
0 -0.00323665142059326 54.0
0 -0.133577227592468 55.0
0 -0.0356827974319458 56.0
0 -0.0339981317520142 57.0
0 -0.0563076734542847 58.0
0 -0.0265318155288696 59.0
0 0.0418399572372437 60.0
0 0.0557025671005249 61.0
0 0.895873129367828 62.0
288.55106033 254.819488525391 0.0
0 -0.0349613428115845 1.0
0 -0.0106679201126099 2.0
0 -0.0653764009475708 3.0
0 -0.0239311456680298 4.0
0 0.0299943685531616 5.0
0 0.015865683555603 6.0
0 0.0133475065231323 7.0
0 -0.0132653117179871 8.0
0 0.16463315486908 9.0
0 -0.00507462024688721 10.0
0 0.0191622972488403 11.0
0 0.0102885961532593 12.0
50.735746972 50.2322883605957 13.0
50.046707319 44.1766128540039 14.0
0 -0.0587865114212036 15.0
0 -0.0582195520401001 16.0
38.625324097 39.3797988891602 17.0
0 0.0400298833847046 18.0
0 -0.0350507497787476 19.0
0 -0.0397742986679077 20.0
0 -0.0331395864486694 21.0
0 -0.0315567255020142 22.0
36.875431476 35.1748924255371 23.0
0 0.0103696584701538 24.0
0 0.858843326568604 25.0
0 0.0539878606796265 26.0
0 0.0197626352310181 27.0
0 0.0446275472640991 28.0
0 -0.0401836633682251 29.0
0 -0.0331763029098511 30.0
0 -0.0736364126205444 31.0
0 -0.0216537714004517 32.0
0 0.0291284322738647 33.0
0 -0.055768609046936 34.0
0 -0.0589390993118286 35.0
0 0.116654515266418 36.0
0 -0.0643967390060425 37.0
0 0.053686261177063 38.0
0 -0.0114545822143555 39.0
0 0.0958942174911499 40.0
0 -0.0596946477890015 41.0
0 -0.0307928323745728 42.0
286.77648037 296.452667236328 43.0
0 0.0647443532943726 44.0
0 -0.0628964900970459 45.0
0 -0.00387465953826904 46.0
0 -0.00565731525421143 47.0
0 0.0105036497116089 48.0
0 0.416273713111877 49.0
0.074694231201 0.0468944311141968 50.0
2.3228931386 1.0070972442627 51.0
0 -0.0562001466751099 52.0
0 -0.0284968614578247 53.0
0 0.0756696462631226 54.0
0 -0.142608761787415 55.0
0 -0.0048898458480835 56.0
0 -0.0708321332931519 57.0
0 -0.0476986169815063 58.0
0 -0.103305697441101 59.0
0 -0.0505725145339966 60.0
0 0.053815484046936 61.0
0 0.871483147144318 62.0
272.11374709 254.248916625977 0.0
0 -0.0464617013931274 1.0
0 0.0192302465438843 2.0
0 -0.0393117666244507 3.0
0 -0.0583502054214478 4.0
0 -0.0513368844985962 5.0
0 -0.0618778467178345 6.0
0 -0.0230571031570435 7.0
0 -0.011064887046814 8.0
0 0.0631164312362671 9.0
0 0.00445497035980225 10.0
0 0.00684487819671631 11.0
0 -0.0438696146011353 12.0
72.682594412 72.2758178710938 13.0
22.371117678 24.127965927124 14.0
0 -0.0387426614761353 15.0
0 -0.00287878513336182 16.0
40.010529099 38.5509643554688 17.0
0 -0.00319898128509521 18.0
0 0.00045621395111084 19.0
0 0.0391162633895874 20.0
0 -0.0236510038375854 21.0
0 0.00706231594085693 22.0
33.464957622 32.8507270812988 23.0
0 0.000974535942077637 24.0
0 0.467500030994415 25.0
0 -0.0210233926773071 26.0
0 0.0554500818252563 27.0
0 -0.0464836359024048 28.0
0 -0.0778087377548218 29.0
0 -0.0570316314697266 30.0
0 -0.0159314870834351 31.0
0 0.00149714946746826 32.0
0 0.139190554618835 33.0
0 -0.0547696352005005 34.0
0 -0.0622131824493408 35.0
0 -0.0784050226211548 36.0
0 0.0278779268264771 37.0
0 -0.00173532962799072 38.0
0 0.0415190458297729 39.0
0 0.0256861448287964 40.0
0 0.0127421617507935 41.0
0 -0.0265403985977173 42.0
285.89643409 295.430450439453 43.0
0 0.0449804067611694 44.0
0 0.0683181285858154 45.0
0 0.0107914209365845 46.0
0 -0.0299860239028931 47.0
0 -0.0519906282424927 48.0
0 0.283098816871643 49.0
0.088212602346 0.0657907724380493 50.0
1.6422412598 0.87593537569046 51.0
0 -0.0432242155075073 52.0
0 -0.0707932710647583 53.0
0 0.0796025991439819 54.0
0 -0.18695056438446 55.0
0 -0.0529392957687378 56.0
0 0.0261019468307495 57.0
0 -0.013222336769104 58.0
0 -0.0528522729873657 59.0
0 0.0117677450180054 60.0
0 0.00890910625457764 61.0
0 0.783995032310486 62.0
265.94262827 257.858459472656 0.0
0 -0.0323787927627563 1.0
0 0.0339707136154175 2.0
0 -0.0307339429855347 3.0
0 0.00610744953155518 4.0
0 -0.00648033618927002 5.0
0 -0.153222680091858 6.0
0 0.0176776647567749 7.0
0 0.00274109840393066 8.0
0 -0.00746405124664307 9.0
0 -0.0313040018081665 10.0
0 -0.0151102542877197 11.0
0 -0.0349603891372681 12.0
65.600087551 66.1356887817383 13.0
27.841051897 29.6099262237549 14.0
0 -0.0567747354507446 15.0
0 -0.0128582715988159 16.0
33.866469627 33.9187049865723 17.0
0 -0.0588721036911011 18.0
0 -0.0200232267379761 19.0
0 0.0558122396469116 20.0
0 -0.0199853181838989 21.0
0 -0.0376259088516235 22.0
38.329061281 36.911735534668 23.0
0 0.0463794469833374 24.0
0 0.897470355033875 25.0
0 -0.00642120838165283 26.0
0 0.0720816850662231 27.0
0 0.00089728832244873 28.0
0 -0.0626171827316284 29.0
0 0.0280443429946899 30.0
0 -0.0759075880050659 31.0
0 0.0150655508041382 32.0
0 0.0537053346633911 33.0
0 0.00296056270599365 34.0
0 -0.041296124458313 35.0
0 0.0175234079360962 36.0
0 -0.0330415964126587 37.0
0 -0.0354660749435425 38.0
0 0.0409470796585083 39.0
0 0.0563062429428101 40.0
0 -0.0183100700378418 41.0
0 0.0172713994979858 42.0
296.67025417 294.865997314453 43.0
0 -0.0101248025894165 44.0
0 0.0655043125152588 45.0
0 0.00767695903778076 46.0
0 -0.0418572425842285 47.0
0 0.0108422040939331 48.0
1.6153876527 0.272527098655701 49.0
0.075074589317 0.0875855684280396 50.0
1.8094135455 0.879250407218933 51.0
0 -0.0349112749099731 52.0
0 -0.036418080329895 53.0
0 0.0416220426559448 54.0
0 -0.0733948945999146 55.0
0 -0.0415068864822388 56.0
0 -0.0177289247512817 57.0
0 -0.0932434797286987 58.0
0 -0.0716739892959595 59.0
0 0.0124799013137817 60.0
0 -0.0110691785812378 61.0
0 0.809524297714233 62.0
242.6568557 252.25065612793 0.0
0 0.0117295980453491 1.0
0 -0.0480026006698608 2.0
0 -0.0511578321456909 3.0
0 -0.0236028432846069 4.0
0 -0.0619398355484009 5.0
0 -0.0634266138076782 6.0
0 -0.0728029012680054 7.0
0 0.043522834777832 8.0
0 0.0223201513290405 9.0
0 -0.0260177850723267 10.0
0 -0.0247889757156372 11.0
0 -0.00550806522369385 12.0
52.531789063 51.5641632080078 13.0
44.942508026 45.3422889709473 14.0
0 -0.0833584070205688 15.0
0 0.00271546840667725 16.0
28.762331657 27.4147605895996 17.0
0 0.0570510625839233 18.0
0 0.00598180294036865 19.0
0 0.0134166479110718 20.0
0 0.00448691844940186 21.0
0 -0.003456711769104 22.0
35.818889156 34.954273223877 23.0
0 -0.0824271440505981 24.0
0 0.550137221813202 25.0
0 -0.00915634632110596 26.0
0 -0.0447373390197754 27.0
0 -0.0920134782791138 28.0
0 -0.0260387659072876 29.0
0 -0.0695661306381226 30.0
0 -0.00757944583892822 31.0
0 -0.0251818895339966 32.0
0 -0.0795034170150757 33.0
0 -0.049533486366272 34.0
0 0.0162069797515869 35.0
0 -0.0844219923019409 36.0
0 -0.00494349002838135 37.0
0 -0.00580942630767822 38.0
0 0.0339201688766479 39.0
0 0.112758040428162 40.0
0 -0.0376859903335571 41.0
0 0.017412543296814 42.0
284.00416805 295.07421875 43.0
0 -0.00905859470367432 44.0
0 0.0559231042861938 45.0
0 -0.0460597276687622 46.0
0 -0.00956928730010986 47.0
0 -0.0107008218765259 48.0
0 0.321776777505875 49.0
0.10054343751 0.0681229829788208 50.0
1.5610897839 0.906445860862732 51.0
0 -0.0762516260147095 52.0
0 -0.060066819190979 53.0
0 0.0299371480941772 54.0
0 -0.044103741645813 55.0
0 -0.0109776258468628 56.0
0 -0.0628494024276733 57.0
0 -0.0119527578353882 58.0
0 -0.0260099172592163 59.0
0 -0.0617769956588745 60.0
0 0.0106683969497681 61.0
0 0.787514686584473 62.0
217.60818562 255.061309814453 0.0
0 -0.0609766244888306 1.0
0 0.034880518913269 2.0
0 0.071961522102356 3.0
0 -0.027065634727478 4.0
0 0.0078132152557373 5.0
0 0.0975021123886108 6.0
0 -0.00105834007263184 7.0
0 -0.0269206762313843 8.0
0 0.0398377180099487 9.0
0 -0.0630284547805786 10.0
0 0.0297011733055115 11.0
0 -0.0623956918716431 12.0
62.346673346 61.556526184082 13.0
22.381115289 34.8154411315918 14.0
0 -0.119134068489075 15.0
0 0.0100730657577515 16.0
37.469880983 37.6392631530762 17.0
0 0.0504409074783325 18.0
0 0.04780113697052 19.0
0 0.0010601282119751 20.0
0 -0.00787770748138428 21.0
0 -0.039798378944397 22.0
28.505442152 27.6774215698242 23.0
0 0.0883067846298218 24.0
0 0.417962789535522 25.0
0 0.0287798643112183 26.0
0 0.090675950050354 27.0
0 0.0217446088790894 28.0
0 -0.0603891611099243 29.0
0 0.17489755153656 30.0
0 -0.0673142671585083 31.0
0 -0.132749199867249 32.0
0 -0.0046311616897583 33.0
0 -0.0181378126144409 34.0
0 0.0202957391738892 35.0
0 -0.0718134641647339 36.0
0 -0.0579894781112671 37.0
0 -0.0142472982406616 38.0
0 0.0930842161178589 39.0
0 -0.021889328956604 40.0
0 -0.0616096258163452 41.0
0 0.0745443105697632 42.0
290.94223518 292.7216796875 43.0
0 -0.0810037851333618 44.0
0 0.0427713394165039 45.0
0 0.00849831104278564 46.0
0 -0.0578669309616089 47.0
0 -0.00503695011138916 48.0
0 0.263254225254059 49.0
0.095158676813 0.0394364595413208 50.0
1.6161062695 0.875897645950317 51.0
0 -0.0617402791976929 52.0
0 -0.0138624906539917 53.0
0 0.0282613039016724 54.0
0 -0.0452436208724976 55.0
0 0.0366371870040894 56.0
0 0.0380815267562866 57.0
0 -0.0460492372512817 58.0
0 -0.00224268436431885 59.0
0 -0.00216591358184814 60.0
0 -0.0551239252090454 61.0
0 0.816708087921143 62.0
288.05572503 253.3056640625 0.0
0 -0.0392695665359497 1.0
0 -0.0181359052658081 2.0
0 0.0402776002883911 3.0
0 -0.0783551931381226 4.0
0 -0.235281109809875 5.0
0 0.00912225246429443 6.0
0 -0.0930773019790649 7.0
0 0.00639104843139648 8.0
0 0.0132135152816772 9.0
0 -0.00361979007720947 10.0
0 -0.0337334871292114 11.0
0 -0.0756984949111938 12.0
55.002972775 54.0235443115234 13.0
36.63477958 40.8477249145508 14.0
0 0.00336205959320068 15.0
0 -0.0394011735916138 16.0
27.23906901 26.3005504608154 17.0
0 0.0513705015182495 18.0
0 0.0422064065933228 19.0
0 0.0054248571395874 20.0
0 -0.0105382204055786 21.0
0 0.0151938199996948 22.0
27.775229117 27.7873458862305 23.0
0 -0.0658947229385376 24.0
0 1.1228951215744 25.0
0 0.106010317802429 26.0
0 0.0457886457443237 27.0
0 -0.0251055955886841 28.0
0 -0.0531114339828491 29.0
0 -0.0359196662902832 30.0
0 -0.0212999582290649 31.0
0 -0.0832735300064087 32.0
0 0.0533219575881958 33.0
0 -0.032570481300354 34.0
0 -0.00236594676971436 35.0
0 0.0370935201644897 36.0
0 0.00501000881195068 37.0
0 -0.000693202018737793 38.0
0 0.0375193357467651 39.0
0 0.0378092527389526 40.0
0 -0.0180090665817261 41.0
0 0.0598024129867554 42.0
297.7852837 295.859436035156 43.0
0 -0.0340524911880493 44.0
0 0.0170423984527588 45.0
0 -0.0147613286972046 46.0
0 -0.0698205232620239 47.0
0 0.00113904476165771 48.0
0 0.349589079618454 49.0
0.096024796561 0.0552998781204224 50.0
1.7382468023 1.02784252166748 51.0
0 -0.0778568983078003 52.0
0 -0.018652081489563 53.0
0 0.00748336315155029 54.0
0 -0.0986601114273071 55.0
0 -0.0376776456832886 56.0
0 -0.0316649675369263 57.0
0 -0.0239797830581665 58.0
0 -0.0543464422225952 59.0
0 0.044343113899231 60.0
0 0.0149201154708862 61.0
0 0.794002771377563 62.0
232.70466059 258.654083251953 0.0
0 -0.0899349451065063 1.0
0 0.0387672185897827 2.0
0 -0.0416814088821411 3.0
0 -0.00880658626556396 4.0
0 0.0273939371109009 5.0
0 -0.0223127603530884 6.0
0 -0.0799821615219116 7.0
0 0.0218318700790405 8.0
0 0.0198811292648315 9.0
0 -0.0483421087265015 10.0
0 -0.0212101936340332 11.0
0 -0.112903237342834 12.0
62.31079969 59.7791938781738 13.0
23.603696329 34.416088104248 14.0
0 -0.153005242347717 15.0
0 0.03111732006073 16.0
36.948327195 36.0133628845215 17.0
0 0.00360524654388428 18.0
0 -0.0119234323501587 19.0
0 -0.0266696214675903 20.0
0 -0.01667320728302 21.0
0 -0.0085526704788208 22.0
34.717000304 33.7669334411621 23.0
0 -0.115500569343567 24.0
0 1.77071917057037 25.0
0 0.109957575798035 26.0
0 0.0377761125564575 27.0
0 -0.0439130067825317 28.0
0 -0.0372542142868042 29.0
0 -0.112564086914062 30.0
0 -0.0644840002059937 31.0
0 -0.0437933206558228 32.0
0 0.0243774652481079 33.0
0 -0.0270229578018188 34.0
0 -0.0193754434585571 35.0
0 0.0118027925491333 36.0
0 -0.0354586839675903 37.0
0 -0.0646015405654907 38.0
0 0.163329720497131 39.0
0 0.31219470500946 40.0
0 -0.0183671712875366 41.0
0 0.0111669301986694 42.0
302.9707171 293.964508056641 43.0
0 -0.0492507219314575 44.0
0 -0.0188509225845337 45.0
0 0.00829493999481201 46.0
0 -0.0130127668380737 47.0
0 0.0390051603317261 48.0
0 0.276843667030334 49.0
0.096712199667 0.0723479986190796 50.0
1.8086123757 0.972080111503601 51.0
0 -0.0172125101089478 52.0
0 -0.0528246164321899 53.0
0 0.0309115648269653 54.0
0 -0.110514521598816 55.0
0 -0.000941157341003418 56.0
0 -0.0527793169021606 57.0
0 -0.0164953470230103 58.0
0 -0.0240617990493774 59.0
0 -0.0473581552505493 60.0
0 0.072435736656189 61.0
0 0.946284353733063 62.0
243.09852309 258.265289306641 0.0
0 -0.0445801019668579 1.0
0 -0.000110745429992676 2.0
0 0.0148829221725464 3.0
0 -0.0225590467453003 4.0
0 0.0649994611740112 5.0
0 0.0209392309188843 6.0
0 -0.105706810951233 7.0
0 0.0105489492416382 8.0
0 0.0377916097640991 9.0
0 -0.0909351110458374 10.0
0 0.00234746932983398 11.0
0 -0.047340989112854 12.0
64.153895264 63.7341804504395 13.0
22.387253142 30.3792400360107 14.0
0 -0.0153347253799438 15.0
0 0.0332449674606323 16.0
27.974983051 27.9849605560303 17.0
0 -0.000204682350158691 18.0
0 0.0086292028427124 19.0
0 0.0453637838363647 20.0
0 -0.0209516286849976 21.0
0 -0.0207706689834595 22.0
32.175482915 31.1598262786865 23.0
0 -0.0221854448318481 24.0
0 0.549076497554779 25.0
0 0.0414241552352905 26.0
0 0.0272225141525269 27.0
0 0.0934666395187378 28.0
0 -0.0518296957015991 29.0
0 -0.0729961395263672 30.0
0 -0.0146836042404175 31.0
0 -0.00141012668609619 32.0
0 0.086300253868103 33.0
0 -0.0443910360336304 34.0
0 -0.0464035272598267 35.0
0 0.11814558506012 36.0
0 -0.0825532674789429 37.0
0 -0.036861777305603 38.0
0 0.0140777826309204 39.0
0 0.062801718711853 40.0
0 -0.0273693799972534 41.0
0 0.0302375555038452 42.0
286.60866874 298.398162841797 43.0
0 0.0622352361679077 44.0
0 -0.0455483198165894 45.0
0 -0.0491634607315063 46.0
0 -0.013178825378418 47.0
0 0.00592029094696045 48.0
0 0.30102413892746 49.0
0.068116531485 0.0400501489639282 50.0
2.0586033484 0.908615410327911 51.0
0 -0.0607272386550903 52.0
0 -0.0890346765518188 53.0
0 0.0329056978225708 54.0
0 -0.101633906364441 55.0
0 -0.03429114818573 56.0
0 -0.0130778551101685 57.0
0 -0.00444614887237549 58.0
0 -0.0237883329391479 59.0
0 -0.0801705121994019 60.0
0 -0.0495253801345825 61.0
0 0.866082668304443 62.0
285.14477651 249.21907043457 0.0
0 -0.0432130098342896 1.0
0 -0.0738078355789185 2.0
0 -0.0154191255569458 3.0
0 -0.032145619392395 4.0
0 -0.238575577735901 5.0
0 0.123437285423279 6.0
0 0.00141513347625732 7.0
0 0.006927490234375 8.0
0 0.0302032232284546 9.0
0 -0.039817214012146 10.0
0 0.0478870868682861 11.0
0 -0.0436750650405884 12.0
71.74856789 67.372314453125 13.0
23.7473168 26.4668827056885 14.0
0 -0.0124861001968384 15.0
0 -0.0602542161941528 16.0
37.177087053 37.0073776245117 17.0
0 0.0174812078475952 18.0
0 -0.0040515661239624 19.0
0 0.0323346853256226 20.0
0 -0.0219296216964722 21.0
0 -0.0357576608657837 22.0
35.397805512 34.8237495422363 23.0
0 -0.0420404672622681 24.0
0 2.45412921905518 25.0
0 0.0178247690200806 26.0
0 0.0354841947555542 27.0
0 0.0712293386459351 28.0
0 -0.0616050958633423 29.0
0 -0.0499950647354126 30.0
0 -0.0218807458877563 31.0
0 -0.0322333574295044 32.0
0 -0.0276410579681396 33.0
0 -0.0106998682022095 34.0
0 0.018775999546051 35.0
0 -0.169635415077209 36.0
0 0.0130685567855835 37.0
0 -0.0295988321304321 38.0
0 -0.0367079973220825 39.0
0 0.194985747337341 40.0
0 0.18930971622467 41.0
0 -0.0723844766616821 42.0
288.42086304 296.100952148438 43.0
0 -0.039620041847229 44.0
0 0.0817166566848755 45.0
0 -0.0490914583206177 46.0
0 -0.0535591840744019 47.0
0 -0.0440770387649536 48.0
0 0.335243850946426 49.0
0.080146593677 0.0470703840255737 50.0
2.4100387391 0.837361812591553 51.0
0 -0.0393339395523071 52.0
0 -0.0960286855697632 53.0
0 -0.138607382774353 54.0
0 0.0636762380599976 55.0
0 0.00793564319610596 56.0
0 0.00983297824859619 57.0
0 -0.0953272581100464 58.0
0 -0.0433176755905151 59.0
0 -0.0547746419906616 60.0
0 0.0224274396896362 61.0
0 0.849284410476685 62.0
279.58511863 255.219635009766 0.0
0 -0.0311628580093384 1.0
0 0.005379319190979 2.0
0 0.024635910987854 3.0
0 -0.0348464250564575 4.0
0 -0.0244885683059692 5.0
0 -0.0283938646316528 6.0
0 -0.0581182241439819 7.0
0 0.0462902784347534 8.0
0 0.019248366355896 9.0
0 -0.000395894050598145 10.0
0 -0.0155093669891357 11.0
0 -0.039314866065979 12.0
60.566279217 59.9107055664062 13.0
34.560449836 36.1681098937988 14.0
0 -0.00957739353179932 15.0
0 -0.0109215974807739 16.0
34.805906022 34.6990356445312 17.0
0 -0.058341383934021 18.0
0 -0.0104523897171021 19.0
0 -0.0153921842575073 20.0
0 0.00523674488067627 21.0
0 -0.0538650751113892 22.0
30.722654195 30.396764755249 23.0
0 -0.0569812059402466 24.0
0 0.5525141954422 25.0
0 0.0310193300247192 26.0
0 0.0289992094039917 27.0
0 0.00542581081390381 28.0
0 -0.0178612470626831 29.0
0 -0.0588569641113281 30.0
0 0.00207173824310303 31.0
0 -0.0411101579666138 32.0
0 0.0455135107040405 33.0
0 -0.0492228269577026 34.0
0 0.00110530853271484 35.0
0 0.0273159742355347 36.0
0 -0.0336345434188843 37.0
0 0.0174158811569214 38.0
0 0.02631676197052 39.0
0 0.0224655866622925 40.0
0 -0.0514208078384399 41.0
0 -0.098518967628479 42.0
290.11517056 295.75341796875 43.0
0 0.0337432622909546 44.0
0 -0.00617265701293945 45.0
0 -0.0242985486984253 46.0
0 -0.0367828607559204 47.0
0 -0.015076756477356 48.0
0 0.263859689235687 49.0
0.082650190555 0.074682354927063 50.0
1.5527584459 1.00049436092377 51.0
0 -0.0640746355056763 52.0
0 -0.0932852029800415 53.0
0 4.4703483581543e-05 54.0
0 -0.0727049112319946 55.0
0 -0.0378323793411255 56.0
0 -0.0519260168075562 57.0
0 -0.00359976291656494 58.0
0 -0.0173512697219849 59.0
0 -0.0880407094955444 60.0
0 0.0244165658950806 61.0
0 0.789759755134583 62.0
247.75085629 263.702117919922 0.0
0 -0.103049874305725 1.0
0 0.0319653749465942 2.0
0 -0.0236140489578247 3.0
0 -0.102452158927917 4.0
0 -0.124209046363831 5.0
0 -0.00480139255523682 6.0
0 -0.172454953193665 7.0
0 0.0679101943969727 8.0
0 -0.0148531198501587 9.0
0 0.000141263008117676 10.0
0 -0.0201734900474548 11.0
0 0.0508817434310913 12.0
70.969225915 69.7512054443359 13.0
25.381194152 28.081298828125 14.0
0 0.00283420085906982 15.0
0 -0.046855092048645 16.0
38.340544179 37.9890785217285 17.0
0 0.00396764278411865 18.0
0 0.00426137447357178 19.0
0 0.0260051488876343 20.0
0 -0.0335580110549927 21.0
0 -0.034709095954895 22.0
39.408788561 38.6354942321777 23.0
0 0.098565936088562 24.0
0 0.379571497440338 25.0
0 0.0392979383468628 26.0
0 0.0449048280715942 27.0
0 0.0849989652633667 28.0
0 -0.0244048833847046 29.0
0 -0.0555996894836426 30.0
0 -0.00579464435577393 31.0
0 -0.0543729066848755 32.0
0 0.0981494188308716 33.0
0 -0.0202442407608032 34.0
0 -0.0458511114120483 35.0
0 0.00168478488922119 36.0
0 -0.111815810203552 37.0
0 0.0156885385513306 38.0
0 0.0235661268234253 39.0
0 -0.0595752000808716 40.0
0 0.0467439889907837 41.0
0 -0.048627495765686 42.0
295.08684681 295.959167480469 43.0
0 0.0398610830307007 44.0
0 -0.0262712240219116 45.0
0 0.0130723714828491 46.0
0 -0.0401936769485474 47.0
0 0.0858086347579956 48.0
0 0.290376454591751 49.0
0.068893552949 0.078494668006897 50.0
2.0996022828 0.916484177112579 51.0
0 -0.0794411897659302 52.0
0 -0.0445429086685181 53.0
0 0.0107203722000122 54.0
0 -0.0619896650314331 55.0
0 -0.0990303754806519 56.0
0 0.00420892238616943 57.0
0 -0.0611151456832886 58.0
0 -0.084835410118103 59.0
0 0.0457929372787476 60.0
0 -0.0161789655685425 61.0
0 0.862048387527466 62.0
271.43766513 255.17253112793 0.0
0 -0.0323559045791626 1.0
0 -0.021729588508606 2.0
0 -0.00253355503082275 3.0
0 0.00443398952484131 4.0
0 -0.112965703010559 5.0
0 -0.039533257484436 6.0
0 -0.0818625688552856 7.0
0 0.0143983364105225 8.0
0 0.0460742712020874 9.0
0 -0.0117374658584595 10.0
0 -0.00110137462615967 11.0
0 0.0239852666854858 12.0
51.313120058 50.9202919006348 13.0
50.904797578 45.1662826538086 14.0
0 -0.144967675209045 15.0
0 0.0126296281814575 16.0
28.900113361 28.972620010376 17.0
0 0.0151633024215698 18.0
0 0.0157343149185181 19.0
0 0.0161558389663696 20.0
0 -0.0550256967544556 21.0
0 -0.0241700410842896 22.0
37.694105001 36.7392883300781 23.0
0 -0.00149500370025635 24.0
0 0.511205732822418 25.0
0 0.00608479976654053 26.0
0 0.116289973258972 27.0
0 -0.0509935617446899 28.0
0 -0.0371307134628296 29.0
0 -0.0554131269454956 30.0
0 -0.0487927198410034 31.0
0 -0.0133363008499146 32.0
0 0.0212796926498413 33.0
0 -0.0311087369918823 34.0
0 -0.0306953191757202 35.0
0 0.0978549718856812 36.0
0 0.0153950452804565 37.0
0 0.0195051431655884 38.0
0 0.0630487203598022 39.0
0 0.0383681058883667 40.0
0 -0.0467067956924438 41.0
0 0.0442508459091187 42.0
302.03226599 295.301879882812 43.0
0 -0.0259915590286255 44.0
0 -0.0250718593597412 45.0
0 -0.00149977207183838 46.0
0 -0.0293277502059937 47.0
0 0.00575482845306396 48.0
0 0.248677253723145 49.0
0.10014508264 0.135586619377136 50.0
1.5575525281 0.951386511325836 51.0
0 -0.104728817939758 52.0
0 -0.0653449296951294 53.0
0 -0.0453177690505981 54.0
0 -0.0516022443771362 55.0
0 0.0199171304702759 56.0
0 -0.0312124490737915 57.0
0 -0.0639642477035522 58.0
0 -0.0568729639053345 59.0
0 -0.107692837715149 60.0
0 0.0315178632736206 61.0
0 0.805386543273926 62.0
284.14537469 257.481475830078 0.0
0 -0.044594407081604 1.0
0 -0.0569058656692505 2.0
0 0.193795800209045 3.0
0 -0.0230494737625122 4.0
0 -0.136670708656311 5.0
0 0.0525249242782593 6.0
0 -0.0374194383621216 7.0
0 -0.081755518913269 8.0
0 0.0313624143600464 9.0
0 0.0259014368057251 10.0
0 -0.0202750563621521 11.0
0 -0.0125738382339478 12.0
76.850307876 75.6485214233398 13.0
7.2856764878 24.6196804046631 14.0
0 -0.105036854743958 15.0
0 -0.0530952215194702 16.0
35.07803141 34.0349082946777 17.0
0 0.0194915533065796 18.0
0 0.0294612646102905 19.0
0 0.0172077417373657 20.0
0 -0.0401891469955444 21.0
0 0.0271133184432983 22.0
38.930051883 38.6189041137695 23.0
0 -0.0985757112503052 24.0
0 0.756373226642609 25.0
0 0.0262731313705444 26.0
0 -0.0376702547073364 27.0
0 -0.0609346628189087 28.0
0 -0.0571748018264771 29.0
0 -0.106147646903992 30.0
0 -0.0660275220870972 31.0
0 0.0369719266891479 32.0
0 -0.0102978944778442 33.0
0 -0.0318459272384644 34.0
0 -0.0617865324020386 35.0
0 0.0299500226974487 36.0
0 0.0106793642044067 37.0
0 0.0257843732833862 38.0
0 0.0784071683883667 39.0
0 0.0417288541793823 40.0
0 -0.0252699851989746 41.0
0 -0.0749002695083618 42.0
290.0754595 296.600433349609 43.0
0 -0.0412040948867798 44.0
0 0.00995719432830811 45.0
0 -0.0203427076339722 46.0
0 -0.0516470670700073 47.0
0 -0.0391699075698853 48.0
0 0.308799743652344 49.0
0.066510878162 0.101027846336365 50.0
1.9182097686 1.01261746883392 51.0
0 -0.0666733980178833 52.0
0 -0.0369001626968384 53.0
0 0.0492593050003052 54.0
0 -0.0883487462997437 55.0
0 -0.0169917345046997 56.0
0 0.00732672214508057 57.0
0 -0.0527526140213013 58.0
0 -0.0332406759262085 59.0
0 -0.066689133644104 60.0
0 0.000676274299621582 61.0
0 0.92284893989563 62.0
233.80165969 258.060333251953 0.0
0 0.00895726680755615 1.0
0 0.0107063055038452 2.0
0 -0.0296477079391479 3.0
0 -0.0197020769119263 4.0
0 -0.0471552610397339 5.0
0 -0.0247303247451782 6.0
0 -0.114328503608704 7.0
0 -0.0748128890991211 8.0
0 0.0378049612045288 9.0
0 -0.00931155681610107 10.0
0 -0.0627483129501343 11.0
0 -0.0344969034194946 12.0
59.150218231 59.3979263305664 13.0
38.104543964 41.2179069519043 14.0
0 -0.00647318363189697 15.0
0 -0.0430198907852173 16.0
38.544214466 38.1982269287109 17.0
0 0.0392040014266968 18.0
0 0.0141328573226929 19.0
0 0.0345367193222046 20.0
0 -0.0506018400192261 21.0
0 -0.0603772401809692 22.0
35.170354878 35.2576866149902 23.0
0 -0.130414605140686 24.0
0 0.699905633926392 25.0
0 0.114361643791199 26.0
0 0.0347620248794556 27.0
0 -0.0713802576065063 28.0
0 -0.0651286840438843 29.0
0 -0.0375601053237915 30.0
0 -0.0261455774307251 31.0
0 0.0100630521774292 32.0
0 0.035176157951355 33.0
0 -0.00254356861114502 34.0
0 -0.115312933921814 35.0
0 0.0326501131057739 36.0
0 -0.0171796083450317 37.0
0 0.00254905223846436 38.0
0 0.0334123373031616 39.0
0 0.000226378440856934 40.0
0 -0.00536453723907471 41.0
0 -0.0945688486099243 42.0
308.5554924 297.514373779297 43.0
0 -0.00860178470611572 44.0
0 0.115229964256287 45.0
0 -0.0184546709060669 46.0
0 -0.0662609338760376 47.0
0 0.00746381282806396 48.0
0 0.280179440975189 49.0
0.086474174759 0.114909172058105 50.0
1.6945652072 0.881428003311157 51.0
0 -0.0360594987869263 52.0
0 -0.0533062219619751 53.0
0 -0.00885975360870361 54.0
0 -0.123936295509338 55.0
0 -0.0533314943313599 56.0
0 0.00220191478729248 57.0
0 -0.0576928853988647 58.0
0 -0.030103325843811 59.0
0 -0.0363985300064087 60.0
0 7.40289688110352e-05 61.0
0 0.852364420890808 62.0
267.97949714 257.211853027344 0.0
0 -0.0215202569961548 1.0
0 0.0321143865585327 2.0
0 -0.0242296457290649 3.0
0 -0.0746349096298218 4.0
0 -0.120494961738586 5.0
0 -0.00787699222564697 6.0
0 -0.01557457447052 7.0
0 -0.0163990259170532 8.0
0 -0.0496488809585571 9.0
0 0.0346051454544067 10.0
0 0.00134330987930298 11.0
0 -0.026960015296936 12.0
58.872971929 57.678840637207 13.0
38.985602136 40.964656829834 14.0
0 -0.0129457712173462 15.0
0 -0.0420414209365845 16.0
41.152677032 41.6990509033203 17.0
0 -0.0568786859512329 18.0
0 0.0351258516311646 19.0
0 0.0381690263748169 20.0
0 -0.0313624143600464 21.0
0 -0.0101152658462524 22.0
36.259094901 34.7749290466309 23.0
0 0.0718139410018921 24.0
0 0.485957741737366 25.0
0 0.00466954708099365 26.0
0 -0.0170073509216309 27.0
0 0.0344008207321167 28.0
0 -0.0160590410232544 29.0
0 -0.066797137260437 30.0
0 -0.055824875831604 31.0
0 0.00479638576507568 32.0
0 0.0991026163101196 33.0
0 -0.0615266561508179 34.0
0 -0.067407488822937 35.0
0 0.0960479974746704 36.0
0 -0.0379661321640015 37.0
0 -0.00149810314178467 38.0
0 0.0514544248580933 39.0
0 0.0510667562484741 40.0
0 -0.0852218866348267 41.0
0 -0.125777363777161 42.0
289.54348864 297.821380615234 43.0
0 0.00202834606170654 44.0
0 0.0675801038742065 45.0
0 -0.00273597240447998 46.0
0 -0.0259703397750854 47.0
0 -0.0681763887405396 48.0
0 0.29804801940918 49.0
0.079610501675 0.0464078187942505 50.0
1.4601166421 1.01181364059448 51.0
0 -0.046364426612854 52.0
0 -0.0116304159164429 53.0
0 -0.0829981565475464 54.0
0 -0.0454398393630981 55.0
0 -0.0156952142715454 56.0
0 -0.0438779592514038 57.0
0 0.000504136085510254 58.0
0 0.0311003923416138 59.0
0 0.124952435493469 60.0
0 0.00491392612457275 61.0
0 0.803301572799683 62.0
215.1989388 255.26106262207 0.0
0 0.0439485311508179 1.0
0 -0.0777817964553833 2.0
0 -0.0299712419509888 3.0
0 -0.0544708967208862 4.0
0 -0.0462011098861694 5.0
0 -0.0100756883621216 6.0
0 -0.0462647676467896 7.0
0 0.096437931060791 8.0
0 -0.0434290170669556 9.0
0 0.00134575366973877 10.0
0 0.0101273059844971 11.0
0 -0.051242470741272 12.0
58.886066786 59.176944732666 13.0
46.677659195 41.654483795166 14.0
0 0.049742579460144 15.0
0 -0.0126625299453735 16.0
39.510103474 39.1338577270508 17.0
0 0.0360969305038452 18.0
0 0.0151222944259644 19.0
0 -0.0335136651992798 20.0
0 0.0171357393264771 21.0
0 0.00133025646209717 22.0
38.859631182 38.2514381408691 23.0
0 -0.104835152626038 24.0
0 0.546369910240173 25.0
0 0.0790313482284546 26.0
0 -0.0237360596656799 27.0
0 0.0214000940322876 28.0
0 -0.0234767198562622 29.0
0 0.247161746025085 30.0
0 -0.219777464866638 31.0
0 -0.0142165422439575 32.0
0 -0.158078551292419 33.0
0 -0.0225657224655151 34.0
0 -0.0605059862136841 35.0
0 0.154270052909851 36.0
0 0.0116440057754517 37.0
0 0.0642589330673218 38.0
0 0.00588047504425049 39.0
0 0.00838077068328857 40.0
0 0.0258593559265137 41.0
0 0.00833165645599365 42.0
286.70841085 294.554229736328 43.0
0 -0.108729004859924 44.0
0 0.0229512453079224 45.0
0 -0.00116646289825439 46.0
0 0.00250065326690674 47.0
0 0.0361040830612183 48.0
0 0.327388852834702 49.0
0.097137871941 0.0433636903762817 50.0
1.6525891577 0.89033305644989 51.0
0 -0.0447403192520142 52.0
0 -0.0103579759597778 53.0
0 0.0700312852859497 54.0
0 -0.116946816444397 55.0
0 -0.0470293760299683 56.0
0 -0.00210678577423096 57.0
0 -0.0249232053756714 58.0
0 -0.0214458703994751 59.0
0 0.0682410001754761 60.0
0 -0.0470978021621704 61.0
0 0.840766072273254 62.0
275.36439735 255.459533691406 0.0
0 -0.0188509225845337 1.0
0 -0.0500551462173462 2.0
0 -0.0630544424057007 3.0
0 -0.0415903329849243 4.0
0 -0.0818101167678833 5.0
0 -0.0165048837661743 6.0
0 -0.111101269721985 7.0
0 -0.0590007305145264 8.0
0 -0.0144706964492798 9.0
0 -0.0334588289260864 10.0
0 -0.034455418586731 11.0
0 -0.0268846750259399 12.0
52.337996399 50.7262268066406 13.0
53.300697866 42.3497886657715 14.0
0 -0.0305801630020142 15.0
0 -0.0393761396408081 16.0
35.12402498 34.6040077209473 17.0
0 -0.0227969884872437 18.0
0 -0.0410562753677368 19.0
0 0.0764180421829224 20.0
0 0.03337562084198 21.0
0 -0.00275194644927979 22.0
26.110269326 25.8610172271729 23.0
0 0.0348490476608276 24.0
7.1634779453 0.87566614151001 25.0
0 0.0217639207839966 26.0
0 0.0368809103965759 27.0
0 0.104972720146179 28.0
0 -0.100240349769592 29.0
0 -0.0179656744003296 30.0
0 -0.0598603487014771 31.0
0 -0.0378562211990356 32.0
0 -0.0858784914016724 33.0
0 -0.0853697061538696 34.0
0 -0.0426857471466064 35.0
0 -0.0920997858047485 36.0
0 -0.0447653532028198 37.0
0 0.0308865308761597 38.0
0 0.0279325246810913 39.0
0 0.0869792699813843 40.0
0 -0.0344016551971436 41.0
0 -0.0278784036636353 42.0
289.95398947 295.240509033203 43.0
0 -0.0756403207778931 44.0
0 0.0430195331573486 45.0
0 0.0558940172195435 46.0
0 -0.0468108654022217 47.0
0 -0.117922902107239 48.0
0 0.261462807655334 49.0
0.084047460775 0.206772565841675 50.0
1.5672531393 0.886784672737122 51.0
0 -0.0658799409866333 52.0
0 -0.0654488801956177 53.0
0 -0.0183097124099731 54.0
0 -0.062186598777771 55.0
0 0.0191677808761597 56.0
0 -0.0475908517837524 57.0
0 -0.105144381523132 58.0
0 -0.0553563833236694 59.0
0 -0.012910008430481 60.0
0 -0.058987021446228 61.0
0 0.840603947639465 62.0
249.48695709 257.094390869141 0.0
0 -0.0151268243789673 1.0
0 -0.0575629472732544 2.0
0 0.016877293586731 3.0
0 -0.0351365804672241 4.0
0 -0.107211709022522 5.0
0 -0.0657964944839478 6.0
0 -0.0586909055709839 7.0
0 -0.102553606033325 8.0
0 0.0107434988021851 9.0
0 0.0286571979522705 10.0
0 -0.0399757623672485 11.0
0 -0.0186659097671509 12.0
67.115100688 65.9700164794922 13.0
37.94438071 31.0939922332764 14.0
0 0.0170043706893921 15.0
0 -0.0289098024368286 16.0
38.483770383 37.0255279541016 17.0
0 0.0232151746749878 18.0
0 -0.0101300477981567 19.0
0 0.0670291185379028 20.0
0 -0.000702500343322754 21.0
0 -0.045453667640686 22.0
32.171675982 32.1587066650391 23.0
0 -0.0444132089614868 24.0
0 0.764519691467285 25.0
0 0.0644534826278687 26.0
0 0.0648010969161987 27.0
0 0.0115212202072144 28.0
0 -0.0736492872238159 29.0
0 -0.0775542259216309 30.0
0 -0.0274487733840942 31.0
0 -0.0266257524490356 32.0
0 -0.0220984220504761 33.0
0 -0.0573288202285767 34.0
0 0.0225310325622559 35.0
0 -0.061240553855896 36.0
0 -0.0139418840408325 37.0
0 0.0511664152145386 38.0
0 0.0264252424240112 39.0
0 0.0234378576278687 40.0
0 -0.0171046257019043 41.0
0 0.0487626791000366 42.0
292.37576908 295.795623779297 43.0
0 -0.00330984592437744 44.0
0 0.04979407787323 45.0
0 -0.00280177593231201 46.0
0 -0.0458852052688599 47.0
0 0.0519527196884155 48.0
0 0.26131409406662 49.0
0.09554469375 0.0602346658706665 50.0
1.9031681127 0.889643013477325 51.0
0 -0.0676642656326294 52.0
0 -0.0232745409011841 53.0
0 0.0413893461227417 54.0
0 -0.0985778570175171 55.0
0 -0.0877190828323364 56.0
0 0.00678813457489014 57.0
0 -0.00566399097442627 58.0
0 -0.0842198133468628 59.0
0 -0.0444515943527222 60.0
0 -0.0354183912277222 61.0
0 0.877312779426575 62.0
256.49631561 252.899963378906 0.0
0 -0.0523854494094849 1.0
0 -0.0523682832717896 2.0
0 -0.00902283191680908 3.0
0 -0.0181342363357544 4.0
0 -0.0219238996505737 5.0
0 0.00149285793304443 6.0
0 -0.0217043161392212 7.0
0 -0.0275452136993408 8.0
0 -0.100914597511292 9.0
0 -0.00534236431121826 10.0
0 -0.0586909055709839 11.0
0 -0.0734049081802368 12.0
54.317112778 53.7863693237305 13.0
37.345994672 35.0994644165039 14.0
0 -0.00410759449005127 15.0
0 -0.0236114263534546 16.0
29.019435348 28.346378326416 17.0
0 -0.00867354869842529 18.0
0 -0.00022423267364502 19.0
0 0.00892150402069092 20.0
0 -0.0102752447128296 21.0
0 -0.0125199556350708 22.0
29.225982654 28.9123001098633 23.0
0 -0.0497032403945923 24.0
0 0.788626492023468 25.0
0 0.0978015661239624 26.0
0 -0.0611616373062134 27.0
0 0.00904834270477295 28.0
0 -0.0451415777206421 29.0
0 -0.0703846216201782 30.0
0 -0.0283559560775757 31.0
0 0.0449131727218628 32.0
0 0.0840181112289429 33.0
0 -0.0727769136428833 34.0
0 -0.0155719518661499 35.0
0 0.0265313386917114 36.0
0 -0.0302344560623169 37.0
0 0.0183802843093872 38.0
0 -0.0449110269546509 39.0
0 -0.0122238397598267 40.0
0 -0.0341702699661255 41.0
0 -0.0551420450210571 42.0
301.83957223 294.565368652344 43.0
0 0.0714744329452515 44.0
0 -0.0715227127075195 45.0
0 -0.00179851055145264 46.0
0 -0.0612512826919556 47.0
0 0.0654920339584351 48.0
0 0.230042278766632 49.0
0.092675354729 0.0807095766067505 50.0
1.8708407659 0.897861421108246 51.0
0 -0.0453935861587524 52.0
0 -0.0257531404495239 53.0
0 0.0437394380569458 54.0
0 -0.0465630292892456 55.0
0 0.0292774438858032 56.0
0 -0.0358062982559204 57.0
0 -0.0660499334335327 58.0
0 -0.029992938041687 59.0
0 -0.00990259647369385 60.0
0 -0.00321686267852783 61.0
0 0.799723625183105 62.0
266.41645476 258.647338867188 0.0
0 -0.0362569093704224 1.0
0 -0.0694032907485962 2.0
0 0.00331413745880127 3.0
0 -0.0342804193496704 4.0
0 4.37498092651367e-05 5.0
0 0.0462220907211304 6.0
0 0.0490082502365112 7.0
0 0.00687205791473389 8.0
0 -0.00568020343780518 9.0
0 -0.00536930561065674 10.0
0 -0.051640510559082 11.0
0 -0.0176597833633423 12.0
70.198889707 68.271728515625 13.0
20.3633639 30.9228439331055 14.0
0 -0.0824199914932251 15.0
0 -0.00829184055328369 16.0
35.07291187 34.8422622680664 17.0
0 0.0465821027755737 18.0
0 0.0274962186813354 19.0
0 0.0127519369125366 20.0
0 0.00304281711578369 21.0
0 0.0112127065658569 22.0
27.603924057 26.376802444458 23.0
0 0.03715980052948 24.0
0 0.544880211353302 25.0
0 0.113086581230164 26.0
0 0.0699824094772339 27.0
0 -0.0451275110244751 28.0
0 -0.023396372795105 29.0
0 -0.0513629913330078 30.0
0 -0.0807305574417114 31.0
0 -0.0112096071243286 32.0
0 -0.00238406658172607 33.0
0 -0.0411254167556763 34.0
0 -0.0381747484207153 35.0
0 0.112500309944153 36.0
0 -0.060207724571228 37.0
0 0.051550030708313 38.0
0 0.0858386754989624 39.0
0 0.029478907585144 40.0
0 -0.0360656976699829 41.0
0 0.0700896978378296 42.0
302.70397342 294.714111328125 43.0
0 -0.0180402994155884 44.0
0 0.0288382768630981 45.0
0 0.00955092906951904 46.0
0 -0.0206865072250366 47.0
0 -0.0113574266433716 48.0
0 0.303318023681641 49.0
0.091579183574 0.0926333665847778 50.0
1.8798391589 1.01467859745026 51.0
0 -0.0170978307723999 52.0
0 -0.0400742292404175 53.0
0 0.070438027381897 54.0
0 -0.0407017469406128 55.0
0 -0.0563615560531616 56.0
0 -0.00377810001373291 57.0
0 -0.0307332277297974 58.0
0 -0.0165764093399048 59.0
0 -0.10571563243866 60.0
0 0.0309332609176636 61.0
0 0.788079500198364 62.0
261.94938102 261.276092529297 0.0
0 -0.0378924608230591 1.0
0 0.0269578695297241 2.0
0 0.013508677482605 3.0
0 -0.0328956842422485 4.0
0 -0.090213418006897 5.0
0 -0.0106493234634399 6.0
0 -0.0660091638565063 7.0
0 -0.0310556888580322 8.0
0 0.0796760320663452 9.0
0 -0.0119377374649048 10.0
0 -0.0211864709854126 11.0
0 -0.207054734230042 12.0
61.907766411 62.8774070739746 13.0
32.505762433 32.6110305786133 14.0
0 -0.00154411792755127 15.0
0 -0.0445181131362915 16.0
36.323895019 36.5541877746582 17.0
0 0.0840147733688354 18.0
0 0.044829249382019 19.0
0 0.0624042749404907 20.0
0 -0.0266379117965698 21.0
0 -0.0602504014968872 22.0
39.19409232 38.7154731750488 23.0
0 -0.0195790529251099 24.0
0 0.465206861495972 25.0
0 0.0413025617599487 26.0
0 0.135713696479797 27.0
0 -0.0348116159439087 28.0
0 -0.146858811378479 29.0
0 -0.0979524850845337 30.0
0 -0.0405057668685913 31.0
0 0.0108312368392944 32.0
0 0.118420481681824 33.0
0 -0.0741857290267944 34.0
0 -0.0895160436630249 35.0
0 0.035091757774353 36.0
0 0.0345736742019653 37.0
0 -0.000114321708679199 38.0
0 0.0896447896957397 39.0
0 0.0752652883529663 40.0
0 -0.0604236125946045 41.0
0 -0.0614982843399048 42.0
296.90844393 294.930816650391 43.0
0 0.104092001914978 44.0
0 -0.101686358451843 45.0
0 -0.0760644674301147 46.0
0 -0.0381549596786499 47.0
0 0.00943362712860107 48.0
0 0.249344170093536 49.0
0.067493080588 0.0500472784042358 50.0
1.4845335907 0.879320323467255 51.0
0 -0.0176302194595337 52.0
0 -0.0283604860305786 53.0
0 -0.0198689699172974 54.0
0 -0.0257576704025269 55.0
0 -0.000488638877868652 56.0
0 -0.0733383893966675 57.0
0 -0.0530861616134644 58.0
0 0.0266903638839722 59.0
0 -0.0978411436080933 60.0
0 0.0472356081008911 61.0
0 0.775709867477417 62.0
273.45040258 252.262466430664 0.0
0 0.00702846050262451 1.0
0 0.00548636913299561 2.0
0 0.0328704118728638 3.0
0 -0.0616270303726196 4.0
0 -0.0899192094802856 5.0
0 -0.0342899560928345 6.0
0 -0.0401889085769653 7.0
0 -0.0767945051193237 8.0
0 0.0222934484481812 9.0
0 -0.000153422355651855 10.0
0 -0.00331401824951172 11.0
0 0.0918458700180054 12.0
66.432060698 66.7516784667969 13.0
40.102164442 33.2301216125488 14.0
0 -0.0409404039382935 15.0
0 -0.0639575719833374 16.0
35.670144624 35.3413429260254 17.0
0 -0.0323718786239624 18.0
0 -0.00274813175201416 19.0
0 0.0243583917617798 20.0
0 -0.0291672945022583 21.0
0 0.0143212080001831 22.0
33.597006326 33.5484924316406 23.0
0 0.0534213781356812 24.0
0 0.472032308578491 25.0
0 0.0707758665084839 26.0
0 0.0976253747940063 27.0
0 -0.0253502130508423 28.0
0 -0.0207713842391968 29.0
0 -0.0752496719360352 30.0
0 -0.0127826929092407 31.0
0 -0.079011082649231 32.0
0 -0.0454634428024292 33.0
0 -0.0495129823684692 34.0
0 -0.0675252676010132 35.0
0 0.0322719812393188 36.0
0 -0.0484389066696167 37.0
0 0.0588477849960327 38.0
0 0.140564799308777 39.0
0 0.0305479764938354 40.0
0 -0.030431866645813 41.0
0 -0.0138975381851196 42.0
301.21305439 297.806671142578 43.0
0 0.0114578008651733 44.0
0 0.0830482244491577 45.0
0 -0.0287741422653198 46.0
0 -0.0811649560928345 47.0
0 -0.0296694040298462 48.0
0.90558675582 0.256113469600677 49.0
0.082174053256 0.182831764221191 50.0
1.6497225689 0.93770956993103 51.0
0 -0.0705970525741577 52.0
0 -0.0687943696975708 53.0
0 0.0342003107070923 54.0
0 -0.108772158622742 55.0
0 -0.0897105932235718 56.0
0 -0.00218832492828369 57.0
0 -0.0640972852706909 58.0
0 -0.0453141927719116 59.0
0 -0.000926613807678223 60.0
0 -0.0194875001907349 61.0
0 0.803758382797241 62.0
285.9534919 254.875366210938 0.0
0 0.0205878019332886 1.0
0 -0.13769257068634 2.0
0 0.046857476234436 3.0
0 -0.0826297998428345 4.0
0 -0.0418788194656372 5.0
0 -0.0166856050491333 6.0
0 0.0128663778305054 7.0
0 -0.0719312429428101 8.0
0 0.0946696996688843 9.0
0 -0.00435411930084229 10.0
0 -0.013586163520813 11.0
0 0.0220197439193726 12.0
51.508817411 51.4028854370117 13.0
35.802671729 35.1099090576172 14.0
0 0.0120514631271362 15.0
0 -0.0462101697921753 16.0
33.153151097 33.3747520446777 17.0
0 -0.00752913951873779 18.0
0 0.00112998485565186 19.0
0 -0.000158429145812988 20.0
0 -0.0304487943649292 21.0
0 0.0177062749862671 22.0
27.182330215 27.0481357574463 23.0
0 -0.0636287927627563 24.0
0 0.635735869407654 25.0
0 0.0829814672470093 26.0
0 0.00787526369094849 27.0
0 -0.0169955492019653 28.0
0 -0.0166317224502563 29.0
0 -0.0456929206848145 30.0
0 -0.0681630373001099 31.0
0 -0.0658379793167114 32.0
0 -0.0433832406997681 33.0
0 -0.00306499004364014 34.0
0 0.1244136095047 35.0
0 0.0461050271987915 36.0
0 -0.0383762121200562 37.0
0 -0.0282977819442749 38.0
0 0.0506981611251831 39.0
0 -0.0109940767288208 40.0
0 0.0234075784683228 41.0
0 -0.060031533241272 42.0
296.47656289 296.511322021484 43.0
0 0.0363320112228394 44.0
0 0.0587412118911743 45.0
0 -0.0306388139724731 46.0
0 -0.0125702619552612 47.0
0 -0.109177708625793 48.0
0 0.306717097759247 49.0
0.07302087106 0.0404413938522339 50.0
2.1758798273 1.0009161233902 51.0
0 -0.041406512260437 52.0
0 -0.050878643989563 53.0
0 0.0642238855361938 54.0
0 -0.0932165384292603 55.0
0 -0.0139855146408081 56.0
0 0.00623071193695068 57.0
0 -0.0319405794143677 58.0
0 0.0410202741622925 59.0
0 0.021886944770813 60.0
0 -0.0720387697219849 61.0
0 0.852400779724121 62.0
240.08129185 260.465759277344 0.0
0 -0.0568228960037231 1.0
0 -0.043702244758606 2.0
0 0.00190389156341553 3.0
0 -0.0648382902145386 4.0
0 0.0200201272964478 5.0
0 -0.0175453424453735 6.0
0 -0.0706270933151245 7.0
0 -0.00265741348266602 8.0
0 0.0860046148300171 9.0
0 0.00193953514099121 10.0
0 0.0190190076828003 11.0
0 -0.00783789157867432 12.0
59.305942027 59.007453918457 13.0
43.151405036 39.2092819213867 14.0
0 -0.0701595544815063 15.0
0 -0.0764147043228149 16.0
41.227761585 42.4370803833008 17.0
0 0.0172272920608521 18.0
0 -0.0348957777023315 19.0
0 0.0463782548904419 20.0
0 -0.0190514326095581 21.0
0 -0.0285273790359497 22.0
36.003513181 35.6142272949219 23.0
0 -0.134871125221252 24.0
0 0.793159067630768 25.0
0 0.0738204717636108 26.0
0 0.100928425788879 27.0
0 -0.000852704048156738 28.0
0 -0.00271928310394287 29.0
0 -0.08855140209198 30.0
0 -0.0211116075515747 31.0
0 -0.00108397006988525 32.0
0 0.14328134059906 33.0
0 -0.052099347114563 34.0
0 0.0708873271942139 35.0
0 -0.0941725969314575 36.0
0 0.0115047693252563 37.0
0 0.0112415552139282 38.0
0 0.150731682777405 39.0
0 0.0763677358627319 40.0
0 0.21037232875824 41.0
0 -0.0222784280776978 42.0
292.02125832 291.779479980469 43.0
0 -0.0172020196914673 44.0
0 0.0221682786941528 45.0
0 -0.0550886392593384 46.0
0 -0.0442465543746948 47.0
0 0.036455512046814 48.0
2.5003941897 0.27019065618515 49.0
0.068211725641 0.103789210319519 50.0
1.9248288681 1.02423715591431 51.0
0 -0.0421925783157349 52.0
0 0.0104461908340454 53.0
0 0.0599504709243774 54.0
0 -0.0375446081161499 55.0
0 0.0461412668228149 56.0
0 -0.0445429086685181 57.0
0 -0.0588697195053101 58.0
0 0.00270998477935791 59.0
0 0.242743134498596 60.0
0 0.00348556041717529 61.0
0 0.810178995132446 62.0
234.21142557 255.203094482422 0.0
0 -0.0222564935684204 1.0
0 -0.0215507745742798 2.0
0 0.00803554058074951 3.0
0 -0.0353003740310669 4.0
0 0.0375934839248657 5.0
0 0.0554918050765991 6.0
0 -0.0546959638595581 7.0
0 -0.112702548503876 8.0
0 0.0685552358627319 9.0
0 -0.0893349647521973 10.0
0 0.0819852352142334 11.0
0 0.0843976736068726 12.0
73.901517486 71.0016098022461 13.0
42.532279673 26.2588977813721 14.0
0 0.041469931602478 15.0
0 -0.08234703540802 16.0
40.528079774 39.0362892150879 17.0
0 0.042460560798645 18.0
0 -0.0254367589950562 19.0
0 0.0334755182266235 20.0
0 0.0157333612442017 21.0
0 -0.0337759256362915 22.0
35.41475823 35.3587646484375 23.0
0 -0.0444470643997192 24.0
0 0.536112546920776 25.0
0 0.0624488592147827 26.0
0 0.0223094820976257 27.0
0 0.0487521886825562 28.0
0 -0.160142064094543 29.0
0 -0.142315149307251 30.0
0 -0.0570698976516724 31.0
0 -0.0104681253433228 32.0
0 -0.0405285358428955 33.0
0 -0.0903934240341187 34.0
0 -0.0487475395202637 35.0
0 0.0146793127059937 36.0
0 -0.0596729516983032 37.0
0 -0.0319772958755493 38.0
0 0.0179182291030884 39.0
0 0.0778411626815796 40.0
0 -0.0155556201934814 41.0
0 0.0217069387435913 42.0
310.9767671 294.897491455078 43.0
0 -0.0729528665542603 44.0
0 0.0476993918418884 45.0
0 -0.00782263278961182 46.0
0 -0.0632141828536987 47.0
0 0.0583523511886597 48.0
0 0.268485844135284 49.0
0.094776332382 0.0876127481460571 50.0
2.0539169074 1.01349020004272 51.0
0 -0.0645295381546021 52.0
0 -0.0455895662307739 53.0
0 -0.0266107320785522 54.0
0 -0.0598455667495728 55.0
0 -0.0593596696853638 56.0
0 -0.030645489692688 57.0
0 -0.0139971971511841 58.0
0 -0.080083966255188 59.0
0 -0.0752977132797241 60.0
0 0.0491713285446167 61.0
0 0.779975891113281 62.0
287.68438617 251.774597167969 0.0
0 -0.0427490472793579 1.0
0 -0.13956081867218 2.0
0 0.0113731622695923 3.0
0 -0.0501543283462524 4.0
0 -0.167923092842102 5.0
0 0.19133985042572 6.0
0 -0.181516289710999 7.0
0 0.0104342699050903 8.0
0 0.0612987279891968 9.0
0 0.0391846895217896 10.0
0 -0.0210931301116943 11.0
0 -0.0847431421279907 12.0
63.671751767 63.1545524597168 13.0
48.850657256 33.3822784423828 14.0
0 -0.104532837867737 15.0
0 -0.0886956453323364 16.0
32.548068406 32.0666275024414 17.0
0 -0.0354961156845093 18.0
0 -0.0516430139541626 19.0
0 0.100812315940857 20.0
0 0.0841161012649536 21.0
0 -0.0110689401626587 22.0
25.294959133 25.4926719665527 23.0
0 -0.0124975442886353 24.0
0 2.61425542831421 25.0
0 0.0498970746994019 26.0
0 0.0168408751487732 27.0
0 -0.0222574472427368 28.0
0 -0.0298365354537964 29.0
0 -0.0230002403259277 30.0
0 -0.0610572099685669 31.0
0 -0.01764976978302 32.0
0 0.0474218130111694 33.0
0 -0.011238694190979 34.0
0 0.0125460624694824 35.0
0 0.126219391822815 36.0
0 -0.0582355260848999 37.0
0 -0.0743969678878784 38.0
0 -0.114927172660828 39.0
0 0.165744662284851 40.0
0 0.110547304153442 41.0
0 0.079433798789978 42.0
277.52658185 297.38623046875 43.0
0 0.0253461599349976 44.0
0 0.0926703810691833 45.0
0 0.0185664892196655 46.0
0 -0.0799723863601685 47.0
0 0.0772165060043335 48.0
1.8344329307 0.342000484466553 49.0
0.085401334938 0.0481828451156616 50.0
2.1749560892 0.871500730514526 51.0
0 -0.0153821706771851 52.0
0 -0.00647485256195068 53.0
0 0.0416268110275269 54.0
0 -0.097248911857605 55.0
0 -0.0373704433441162 56.0
0 -0.0292724370956421 57.0
0 -0.113551259040833 58.0
0 -0.0927547216415405 59.0
0 -0.103095173835754 60.0
0 0.00879585742950439 61.0
0 0.825659036636353 62.0
258.89610182 257.238891601562 0.0
0 -0.0389477014541626 1.0
0 0.0378750562667847 2.0
0 -0.002541184425354 3.0
0 -0.073933482170105 4.0
0 -0.116474509239197 5.0
0 -0.0321422815322876 6.0
0 -0.0135418176651001 7.0
0 -0.0350577831268311 8.0
0 -0.00703203678131104 9.0
0 -0.00869333744049072 10.0
0 -0.0435187816619873 11.0
0 0.00880372524261475 12.0
59.249459951 58.3939628601074 13.0
43.828784991 36.3202247619629 14.0
0 0.0207394361495972 15.0
0 -0.0198081731796265 16.0
30.708807071 30.5465698242188 17.0
0 -0.0425029993057251 18.0
0 -0.0197166204452515 19.0
0 -0.0154484510421753 20.0
0 0.0120655298233032 21.0
0 -0.056280255317688 22.0
35.276215926 35.1596870422363 23.0
0 -0.0665351152420044 24.0
0 0.540528357028961 25.0
0 0.0255330801010132 26.0
0 -0.0891076326370239 27.0
0 0.000903487205505371 28.0
0 -0.0564240217208862 29.0
0 -0.0479422807693481 30.0
0 -0.055686354637146 31.0
0 -0.050392746925354 32.0
0 0.0200303792953491 33.0
0 -0.0423935651779175 34.0
0 0.0166283845901489 35.0
0 -0.0401281118392944 36.0
0 -0.00841295719146729 37.0
0 0.0159929990768433 38.0
0 0.00389254093170166 39.0
0 0.0367563962936401 40.0
0 -0.0124107599258423 41.0
0 -0.0690008401870728 42.0
288.85194651 295.509307861328 43.0
0 0.0442308187484741 44.0
0 -0.0751998424530029 45.0
0 -0.00379025936126709 46.0
0 -0.02959144115448 47.0
0 0.0221337080001831 48.0
0 0.271261036396027 49.0
0.08002008268 0.0605629682540894 50.0
2.213835677 0.890513718128204 51.0
0 -0.0878478288650513 52.0
0 -0.0576647520065308 53.0
0 0.0217303037643433 54.0
0 -0.0539885759353638 55.0
0 -0.0495010614395142 56.0
0 -0.0188537836074829 57.0
0 -0.0595763921737671 58.0
0 -0.0967921018600464 59.0
0 -0.0494557619094849 60.0
0 0.0134245157241821 61.0
0 0.839307427406311 62.0
276.56986356 254.900772094727 0.0
0 -0.0121990442276001 1.0
0 -0.0507467985153198 2.0
0 -0.00220644474029541 3.0
0 -0.0753940343856812 4.0
0 -0.05292809009552 5.0
0 0.0528362989425659 6.0
0 -0.0183831453323364 7.0
0 -0.024120569229126 8.0
0 -0.010501503944397 9.0
0 -0.0111435651779175 10.0
0 -0.0251423120498657 11.0
0 -0.0824524164199829 12.0
56.579774214 56.8484001159668 13.0
43.904792797 40.4988594055176 14.0
0 -0.0503588914871216 15.0
0 0.014147162437439 16.0
32.386267832 32.0170593261719 17.0
0 -0.0172089338302612 18.0
0 -0.00231683254241943 19.0
0 0.00313866138458252 20.0
0 0.00247299671173096 21.0
0 -0.0145164728164673 22.0
37.674452637 37.9804191589355 23.0
0 -0.115676522254944 24.0
0 1.55881750583649 25.0
0 0.0783008337020874 26.0
0 0.080976128578186 27.0
0 0.0697067975997925 28.0
0 -0.0869983434677124 29.0
0 -0.0556026697158813 30.0
0 -0.0794738531112671 31.0
0 -0.0695182085037231 32.0
0 -0.0200008153915405 33.0
0 -0.0328727960586548 34.0
0 -0.0494946241378784 35.0
0 0.105821013450623 36.0
0 -0.0774234533309937 37.0
0 0.0344361066818237 38.0
0 -0.027635931968689 39.0
0 -0.0920053720474243 40.0
0 -0.0410263538360596 41.0
0 -0.0199400186538696 42.0
291.52830217 296.006896972656 43.0
0 -0.000450730323791504 44.0
0 0.0839068293571472 45.0
0 -0.0207632780075073 46.0
0 -0.012036919593811 47.0
0 -0.0149847269058228 48.0
1.503316198 0.342716783285141 49.0
0.092497779242 0.051615834236145 50.0
1.6932715143 1.00766861438751 51.0
0 -0.0739322900772095 52.0
0 0.00517880916595459 53.0
0 -0.00132071971893311 54.0
0 -0.0540989637374878 55.0
0 0.0269221067428589 56.0
0 -0.038188099861145 57.0
0 -0.0233463048934937 58.0
0 -0.00281655788421631 59.0
0 -0.0896247625350952 60.0
0 0.0059584379196167 61.0
0 0.865492701530457 62.0
258.6234743 259.738464355469 0.0
0 -0.100200772285461 1.0
0 0.011513352394104 2.0
0 0.000579953193664551 3.0
0 -0.0520859956741333 4.0
0 0.0844699144363403 5.0
0 0.029110312461853 6.0
0 -0.188831210136414 7.0
0 0.0533241033554077 8.0
0 -0.0606609582901001 9.0
0 -0.0455964803695679 10.0
0 -0.0411249399185181 11.0
0 -0.0569025278091431 12.0
66.650757632 65.312873840332 13.0
38.460615981 32.1476593017578 14.0
0 0.00466239452362061 15.0
0 0.00332629680633545 16.0
37.283374816 35.7029685974121 17.0
0 -0.0261176824569702 18.0
0 -0.0145875215530396 19.0
0 0.0454522371292114 20.0
0 -0.0293327569961548 21.0
0 0.0159868001937866 22.0
33.991783652 33.8364372253418 23.0
0 -0.0229378938674927 24.0
0 0.686105787754059 25.0
0 -0.0490423440933228 26.0
0 0.177801489830017 27.0
0 -0.0320745706558228 28.0
0 -0.0266913175582886 29.0
0 -0.019545316696167 30.0
0 -0.0520466566085815 31.0
0 0.050406813621521 32.0
0 0.0453475713729858 33.0
0 0.0100587606430054 34.0
0 -0.0917098522186279 35.0
0 0.0770667791366577 36.0
0 -0.0156487226486206 37.0
0 0.0040053129196167 38.0
0 -0.107319474220276 39.0
0 -0.104975819587708 40.0
0 0.019727349281311 41.0
0 0.00106990337371826 42.0
300.89341092 293.322021484375 43.0
0 -0.0467926263809204 44.0
0 0.0457955598831177 45.0
0 0.000839114189147949 46.0
0 -0.0379379987716675 47.0
0 0.0271161794662476 48.0
0 0.298840045928955 49.0
0.065693903915 0.0808120965957642 50.0
1.5715701972 1.02091908454895 51.0
0 -0.103052258491516 52.0
0 -0.0434418916702271 53.0
0 0.00622451305389404 54.0
0 -0.0574122667312622 55.0
0 0.00616061687469482 56.0
0 -0.0402785539627075 57.0
0 -0.0322326421737671 58.0
0 -0.0120726823806763 59.0
0 0.0475448369979858 60.0
0 0.0162173509597778 61.0
0 0.921477735042572 62.0
231.91637109 252.820663452148 0.0
0 -0.0499025583267212 1.0
0 -0.00755727291107178 2.0
0 0.0129462480545044 3.0
0 -0.0435987710952759 4.0
0 0.00907838344573975 5.0
0 -0.000434517860412598 6.0
0 -0.1030033826828 7.0
0 -0.04489666223526 8.0
0 0.0708431005477905 9.0
0 -0.00966894626617432 10.0
0 -0.046769380569458 11.0
0 -0.010724663734436 12.0
59.488600737 57.3918037414551 13.0
51.530430294 38.9556884765625 14.0
0 0.0744456052780151 15.0
0 -0.036711573600769 16.0
34.187953013 33.691032409668 17.0
0 -0.0368508100509644 18.0
0 0.00281274318695068 19.0
0 -0.00759541988372803 20.0
0 -0.0117624998092651 21.0
0 0.00669467449188232 22.0
32.60891208 31.7159233093262 23.0
0 -0.0377031564712524 24.0
0 0.554840326309204 25.0
0 0.0495270490646362 26.0
0 -0.0306469202041626 27.0
0 -0.0231281518936157 28.0
0 -0.0328065156936646 29.0
0 -0.0426729917526245 30.0
0 -0.00314772129058838 31.0
0 -0.0626040697097778 32.0
0 -0.028633713722229 33.0
0 -0.0262068510055542 34.0
0 -0.00300133228302002 35.0
0 0.0699301958084106 36.0
0 -0.0473123788833618 37.0
0 0.0236657857894897 38.0
0 0.0417112112045288 39.0
0 -0.0325971841812134 40.0
0 0.0193487405776978 41.0
0 0.000918745994567871 42.0
302.22651091 296.621856689453 43.0
0 -0.018379807472229 44.0
0 0.0461058616638184 45.0
0 -0.00164997577667236 46.0
0 -0.0439448356628418 47.0
0 -0.0522686243057251 48.0
0 0.275794267654419 49.0
0.069869145475 0.0659700632095337 50.0
2.2762747618 0.918626546859741 51.0
0 -0.071539044380188 52.0
0 -0.10516083240509 53.0
0 -0.00863397121429443 54.0
0 -0.100741744041443 55.0
0 0.0331860780715942 56.0
0 -0.0531288385391235 57.0
0 -0.0179406404495239 58.0
0 -0.0902425050735474 59.0
0 -0.0578221082687378 60.0
0 -0.0213645696640015 61.0
0 0.805449962615967 62.0
254.61785104 257.965087890625 0.0
0 -0.0617562532424927 1.0
0 -0.0202473402023315 2.0
0 -0.0776392221450806 3.0
0 -0.027098536491394 4.0
0 -0.0684973001480103 5.0
0 -0.130983948707581 6.0
0 -0.0436955690383911 7.0
0 -0.0470819473266602 8.0
0 -0.0796447992324829 9.0
0 -0.0687919855117798 10.0
0 -0.0398825407028198 11.0
0 -0.00248587131500244 12.0
62.852801938 62.9462623596191 13.0
26.437024189 35.3721389770508 14.0
0 0.0108803510665894 15.0
0 -0.023445725440979 16.0
34.227010803 34.5478858947754 17.0
0 0.0106242895126343 18.0
0 0.0141490697860718 19.0
0 -0.00431859493255615 20.0
0 -0.0233310461044312 21.0
0 -0.0345331430435181 22.0
32.677819824 31.3361263275146 23.0
0 0.0764549970626831 24.0
0 0.43977302312851 25.0
0 0.0161117315292358 26.0
0 0.0350444316864014 27.0
0 0.0534504652023315 28.0
0 -0.0332902669906616 29.0
0 -0.061060905456543 30.0
0 -0.0402120351791382 31.0
0 -0.041939377784729 32.0
0 -0.00864994525909424 33.0
0 -0.0354942083358765 34.0
0 -0.0647622346878052 35.0
0 0.112545609474182 36.0
0 -0.00809299945831299 37.0
0 -0.000453591346740723 38.0
0 0.0963302850723267 39.0
0 0.100184798240662 40.0
0 -0.0161119699478149 41.0
0 0.0397123098373413 42.0
305.67300985 294.408905029297 43.0
0 -0.0274168252944946 44.0
0 0.043550431728363 45.0
0 0.0118788480758667 46.0
0 -0.0292860269546509 47.0
0 -0.0011674165725708 48.0
0 0.296056270599365 49.0
0.10003403417 0.0464104413986206 50.0
2.2589030731 0.920974910259247 51.0
0 -0.0223008394241333 52.0
0 -0.0208367109298706 53.0
0 0.0158478021621704 54.0
0 -0.093988299369812 55.0
0 -0.0287944078445435 56.0
0 -0.0244172811508179 57.0
0 -0.00987613201141357 58.0
0 -0.00701344013214111 59.0
0 -0.0601440668106079 60.0
0 0.0383924245834351 61.0
0 0.856375336647034 62.0
236.78590164 255.890899658203 0.0
0 -0.0101743936538696 1.0
0 -0.0344327688217163 2.0
0 -0.00167477130889893 3.0
0 -0.0479732751846313 4.0
0 0.00521433353424072 5.0
0 0.0161908864974976 6.0
0 -0.0396822690963745 7.0
0 -0.0504419207572937 8.0
0 -0.0476070642471313 9.0
0 0.0252667665481567 10.0
0 -0.0495935678482056 11.0
0 -0.0848509073257446 12.0
54.894970885 54.5062713623047 13.0
49.969043503 42.6846084594727 14.0
0 -0.0173851251602173 15.0
0 -0.0663450956344604 16.0
27.682508324 27.0735454559326 17.0
0 0.0111738443374634 18.0
0 -0.034746527671814 19.0
0 0.0717750787734985 20.0
0 -0.000780940055847168 21.0
0 0.0227559804916382 22.0
34.447314551 32.9031257629395 23.0
0 0.102008700370789 24.0
0 0.437856018543243 25.0
0 -0.0243579149246216 26.0
0 0.061853289604187 27.0
0 -0.0334793329238892 28.0
0 -0.0552080869674683 29.0
0 -0.0483338832855225 30.0
0 0.00100076198577881 31.0
0 -0.0544706583023071 32.0
0 0.0280474424362183 33.0
0 -0.0665339231491089 34.0
0 -0.029786229133606 35.0
0 0.0666946172714233 36.0
0 -0.0940412282943726 37.0
0 0.0240415334701538 38.0
0 0.0911482572555542 39.0
0 0.0158213376998901 40.0
0 0.000677227973937988 41.0
0 0.0941900014877319 42.0
311.28143529 295.576110839844 43.0
0 -0.0656687021255493 44.0
0 -0.0191071033477783 45.0
0 0.00461351871490479 46.0
0 -0.0855635404586792 47.0
0 0.0200237035751343 48.0
0 0.298453509807587 49.0
0.080158714448 0.0746473073959351 50.0
2.1896345058 0.987096011638641 51.0
0 -0.0581434965133667 52.0
0 -0.0213152170181274 53.0
0 -0.00753343105316162 54.0
0 -0.0955232381820679 55.0
0 0.0210374593734741 56.0
0 -0.0596476793289185 57.0
0 -0.0643914937973022 58.0
0 -0.0870150327682495 59.0
0 0.0295785665512085 60.0
0 0.0618723630905151 61.0
0 0.921977162361145 62.0
292.2422711 254.579116821289 0.0
0 -0.0645252466201782 1.0
0 0.00211656093597412 2.0
0 0.0384353399276733 3.0
0 -0.071226954460144 4.0
0 -0.0203844308853149 5.0
0 -0.0350533723831177 6.0
0 0.00594079494476318 7.0
0 -0.0394783616065979 8.0
0 -0.0198246240615845 9.0
0 0.0247286558151245 10.0
0 -0.0417203307151794 11.0
0 -0.0402542352676392 12.0
72.920203023 71.9597244262695 13.0
10.818918492 27.0929470062256 14.0
0 -0.0211540460586548 15.0
0 -0.0196148157119751 16.0
25.104022259 24.0546379089355 17.0
0 -0.00228464603424072 18.0
0 -0.0241323709487915 19.0
0 0.0167726278305054 20.0
0 -0.0583056211471558 21.0
0 0.0113705396652222 22.0
29.943244679 28.8145771026611 23.0
0 0.116349101066589 24.0
0 0.446630120277405 25.0
0 -0.0164810419082642 26.0
0 0.110451579093933 27.0
0 -0.0527244806289673 28.0
0 -0.0659297704696655 29.0
0 -0.113258838653564 30.0
0 -0.0148447751998901 31.0
0 -0.00620615482330322 32.0
0 0.00709807872772217 33.0
0 -0.0417047739028931 34.0
0 -0.105830788612366 35.0
0 0.0356413125991821 36.0
0 -0.0240124464035034 37.0
0 0.0285776853561401 38.0
0 0.0251525640487671 39.0
0 0.0113242864608765 40.0
0 0.0331474542617798 41.0
0 -0.0940910577774048 42.0
314.56458023 295.857788085938 43.0
0 0.0084155797958374 44.0
0 0.0349582433700562 45.0
0 -0.0264257192611694 46.0
0 0.0214656591415405 47.0
0 0.0428060293197632 48.0
0 0.313402622938156 49.0
0.083751983565 0.0445808172225952 50.0
1.8076762511 0.917534351348877 51.0
0 0.014102578163147 52.0
0 -0.0186721086502075 53.0
0 0.0627049207687378 54.0
0 -0.0232213735580444 55.0
0 0.0052109956741333 56.0
0 -0.0525957345962524 57.0
0 -0.0538049936294556 58.0
0 -0.0891720056533813 59.0
0 -0.0600475072860718 60.0
0 0.00895607471466064 61.0
0 0.826322793960571 62.0
264.12849863 256.696075439453 0.0
0 0.00406396389007568 1.0
0 -0.0927754640579224 2.0
0 -0.0274263620376587 3.0
0 0.0145817995071411 4.0
0 -0.0532633066177368 5.0
0 0.0910848379135132 6.0
0 -0.103506207466125 7.0
0 -0.0224475860595703 8.0
0 0.019436240196228 9.0
0 -0.0374568700790405 10.0
0 -0.0407346487045288 11.0
0 -0.0177351236343384 12.0
58.510001856 57.6830177307129 13.0
33.750505047 39.947940826416 14.0
0 -0.0355988740921021 15.0
0 0.0558127164840698 16.0
39.880846865 39.3009223937988 17.0
0 -0.0361768007278442 18.0
0 0.0114842653274536 19.0
0 0.00344979763031006 20.0
0 -0.0448287725448608 21.0
0 -0.00904238224029541 22.0
31.200704212 30.7451496124268 23.0
0 -0.0363007783889771 24.0
0 0.654385507106781 25.0
0 0.0741308927536011 26.0
0 0.0535303354263306 27.0
0 -0.0306316614151001 28.0
0 -0.074562668800354 29.0
0 -0.0991469621658325 30.0
0 -0.0323318243026733 31.0
0 -0.0341950654983521 32.0
0 0.0129069089889526 33.0
0 -0.0177568197250366 34.0
0 -0.0863896608352661 35.0
0 -0.0285617113113403 36.0
0 -0.0561932325363159 37.0
0 0.0214899778366089 38.0
0 0.0855304002761841 39.0
0 0.0589832067489624 40.0
0 -0.0508381128311157 41.0
0 -0.0174680948257446 42.0
285.71845451 298.130157470703 43.0
0 -0.0181561708450317 44.0
0 0.0450568199157715 45.0
0 0.000797629356384277 46.0
0 -0.0570155382156372 47.0
0 0.0219382047653198 48.0
0 0.282685846090317 49.0
0.092296352804 0.0927389860153198 50.0
1.6774711432 0.973055779933929 51.0
0 -0.0470298528671265 52.0
0 -0.0391675233840942 53.0
0 -0.0135186910629272 54.0
0 -0.0760551691055298 55.0
0 -0.0151957273483276 56.0
0 -0.00325357913970947 57.0
0 -0.067193865776062 58.0
0 0.00886285305023193 59.0
0 -0.0586062669754028 60.0
0 -0.0561052560806274 61.0
0 0.832897782325745 62.0
285.83036178 257.543060302734 0.0
0 -0.0606948137283325 1.0
0 -0.0413004159927368 2.0
0 -0.0226331949234009 3.0
0 -0.0691286325454712 4.0
0 -0.136866688728333 5.0
0 0.065144419670105 6.0
0 -0.189557671546936 7.0
0 -0.0630804300308228 8.0
0 0.0464490652084351 9.0
0 0.0104442834854126 10.0
0 -0.0541166067123413 11.0
0 -0.122467637062073 12.0
51.270367436 49.772159576416 13.0
44.579845013 46.8953514099121 14.0
0 0.0669015645980835 15.0
0 0.0278502702713013 16.0
27.57866214 26.4652709960938 17.0
0 -0.0352569818496704 18.0
0 0.00203216075897217 19.0
0 -0.00666821002960205 20.0
0 0.0146441459655762 21.0
0 -0.0407520532608032 22.0
39.730743801 39.6440238952637 23.0
0 0.0276793241500854 24.0
0 0.703416347503662 25.0
0 0.110399603843689 26.0
0 0.110394358634949 27.0
0 -0.0832282304763794 28.0
0 -0.138562560081482 29.0
0 0.0598231554031372 30.0
0 -0.0166333913803101 31.0
0 0.0256494283676147 32.0
0 -0.0356613397598267 33.0
0 0.00500071048736572 34.0
0 0.0709789991378784 35.0
0 -0.0741902589797974 36.0
0 0.00598824024200439 37.0
0 -0.027814507484436 38.0
0 0.0590847730636597 39.0
0 0.0783103704452515 40.0
0 -0.0356854200363159 41.0
0 -0.0798426866531372 42.0
305.64654629 291.284362792969 43.0
0 -0.00236189365386963 44.0
0 0.042931079864502 45.0
0 -0.0717529058456421 46.0
0 -0.0123845338821411 47.0
0 -0.0688520669937134 48.0
0 0.266042828559875 49.0
0.078425341343 0.140707850456238 50.0
1.5268959518 0.927950322628021 51.0
0 -0.0836673974990845 52.0
0 -0.0281087160110474 53.0
0 0.0520206689834595 54.0
0 -0.0294867753982544 55.0
0 -0.0377863645553589 56.0
0 -0.0195013284683228 57.0
0 -0.0609918832778931 58.0
0 -0.0781654119491577 59.0
0 -0.0710035562515259 60.0
0 0.0485767126083374 61.0
0 0.820680379867554 62.0
245.81264008 261.053131103516 0.0
0 0.0201185941696167 1.0
0 -0.0608900785446167 2.0
0 -0.0252474546432495 3.0
0 -0.0440284013748169 4.0
0 -0.0818272829055786 5.0
0 -0.0166412591934204 6.0
0 -0.0398126840591431 7.0
0 0.0168502330780029 8.0
0 -0.0571447610855103 9.0
0 0.00315773487091064 10.0
0 -0.0427016019821167 11.0
0 -0.0802685022354126 12.0
64.99393289 64.4159088134766 13.0
26.698685448 33.0518493652344 14.0
0 -0.0173946619033813 15.0
0 0.0137313604354858 16.0
37.069741489 36.168098449707 17.0
0 -0.00877010822296143 18.0
0 -0.0298973321914673 19.0
0 -0.0184568166732788 20.0
0 -0.0293096303939819 21.0
0 -0.0286928415298462 22.0
35.368887519 34.5177230834961 23.0
0 -0.0836693048477173 24.0
0 0.577097237110138 25.0
0 0.0967673063278198 26.0
0 -0.0178648233413696 27.0
0 0.0607284307479858 28.0
0 -0.0247970819473267 29.0
0 -0.0518592596054077 30.0
0 -0.0295251607894897 31.0
0 0.00324332714080811 32.0
0 0.0936363935470581 33.0
0 -0.101674437522888 34.0
0 -0.0726066827774048 35.0
0 -0.00912535190582275 36.0
0 0.0272382497787476 37.0
0 -0.0152701139450073 38.0
0 0.033744215965271 39.0
0 0.131281733512878 40.0
0 -0.0330427885055542 41.0
0 0.0553144216537476 42.0
291.33799704 296.506866455078 43.0
0 -0.00680983066558838 44.0
0 0.081933856010437 45.0
0 -0.0186945199966431 46.0
0 -0.0635782480239868 47.0
0 0.0188153982162476 48.0
0 0.32062691450119 49.0
0.099520487438 0.0512512922286987 50.0
2.1526552288 1.01369214057922 51.0
0 -0.0397442579269409 52.0
0 -0.0449475049972534 53.0
0 0.0760654211044312 54.0
0 -0.0496376752853394 55.0
0 -0.0497556924819946 56.0
0 -0.0459171533584595 57.0
0 -0.0208615064620972 58.0
0 -0.0404025316238403 59.0
0 -0.0291043519973755 60.0
0 0.0255023241043091 61.0
0 0.877556324005127 62.0
241.20453296 257.677673339844 0.0
0 -0.0906034708023071 1.0
0 0.0561815500259399 2.0
0 -0.0763646364212036 3.0
0 -0.0493015050888062 4.0
0 -0.00778830051422119 5.0
0 -0.0030285120010376 6.0
0 -0.0544027090072632 7.0
0 -0.0186595320701599 8.0
0 0.0259908437728882 9.0
0 0.0131198167800903 10.0
0 -0.0529685020446777 11.0
0 -0.153766274452209 12.0
56.83792508 55.9737548828125 13.0
41.050512659 38.0228118896484 14.0
0 -0.0106474161148071 15.0
0 0.000612139701843262 16.0
28.022522009 28.4515190124512 17.0
0 -0.00296461582183838 18.0
0 -0.0322288274765015 19.0
0 0.0546764135360718 20.0
0 -0.000537991523742676 21.0
0 -0.10497772693634 22.0
36.448437654 37.0603523254395 23.0
0 -0.125413060188293 24.0
0 0.691256403923035 25.0
0 0.0875815153121948 26.0
0 0.0431703329086304 27.0
0 0.115754246711731 28.0
0 -0.0155538320541382 29.0
0 -0.0457284450531006 30.0
0 -0.0676425695419312 31.0
0 -0.00785934925079346 32.0
0 -0.163828611373901 33.0
0 -0.000258326530456543 34.0
0 -0.0571591854095459 35.0
0 0.112458109855652 36.0
0 -0.0105401277542114 37.0
0 0.0345124006271362 38.0
0 -0.0183404684066772 39.0
0 0.0883320569992065 40.0
0 0.0500437021255493 41.0
0 -0.00783216953277588 42.0
298.58716259 297.095031738281 43.0
0 0.0498040914535522 44.0
0 -0.0609563589096069 45.0
0 -0.0584162473678589 46.0
0 0.0225111246109009 47.0
0 -0.0340644121170044 48.0
0 0.284862220287323 49.0
0.091440580127 0.0643526315689087 50.0
2.0054477443 0.866312265396118 51.0
0 -0.115831971168518 52.0
0 -0.057193398475647 53.0
0 0.0638686418533325 54.0
0 -0.104736924171448 55.0
0 -0.0292319059371948 56.0
0 -0.0210095643997192 57.0
0 -0.054510235786438 58.0
0 -0.0641282796859741 59.0
0 -0.0492302179336548 60.0
0 -0.0255681276321411 61.0
0 0.865909457206726 62.0
224.11415861 251.492401123047 0.0
0 0.0446375608444214 1.0
0 -0.0928694009780884 2.0
0 0.0172015428543091 3.0
0 -0.05131995677948 4.0
0 0.0346506834030151 5.0
0 0.00855100154876709 6.0
0 -0.00344431400299072 7.0
0 -0.0302447080612183 8.0
0 -0.0347257852554321 9.0
0 0.0173386335372925 10.0
0 -0.0434308052062988 11.0
0 -0.00173532962799072 12.0
68.655427755 67.9924087524414 13.0
25.343893349 32.3104782104492 14.0
0 -0.106037259101868 15.0
0 -0.0582104921340942 16.0
34.742793499 34.3297576904297 17.0
0 0.00964701175689697 18.0
0 0.0062633752822876 19.0
0 0.00373589992523193 20.0
0 -0.0413683652877808 21.0
0 -0.000740647315979004 22.0
30.883864894 29.4290504455566 23.0
0 0.00215709209442139 24.0
0 0.64832603931427 25.0
0 0.0815051794052124 26.0
0 0.0762628316879272 27.0
0 -0.000829815864562988 28.0
0 -0.0102328062057495 29.0
0 -0.0810494422912598 30.0
0 -0.0351980924606323 31.0
0 0.00584447383880615 32.0
0 0.0896538496017456 33.0
0 0.0187879800796509 34.0
0 -0.0271238088607788 35.0
0 0.174298167228699 36.0
0 -0.0565711259841919 37.0
0 0.000358223915100098 38.0
0 0.0668138265609741 39.0
0 0.065876841545105 40.0
0 -0.0519860982894897 41.0
0 0.0483554601669312 42.0
310.85769402 294.914764404297 43.0
0 0.00178945064544678 44.0
0 0.039345383644104 45.0
0 -0.0402451753616333 46.0
0 -0.0493634939193726 47.0
0 0.0466393232345581 48.0
0 0.420479595661163 49.0
0.063293244035 0.0730084180831909 50.0
1.7952988643 0.923586428165436 51.0
0 -0.0959440469741821 52.0
0 -0.0896912813186646 53.0
0 -0.0321460962295532 54.0
0 -0.0425115823745728 55.0
0 -0.0262190103530884 56.0
0 -0.00439846515655518 57.0
0 -0.0143595933914185 58.0
0 -0.0104771852493286 59.0
0 0.102943062782288 60.0
0 -0.0856021642684937 61.0
0 0.80229389667511 62.0
280.14880499 255.114349365234 0.0
0 -0.0418397188186646 1.0
0 -0.066327691078186 2.0
0 -0.0376251935958862 3.0
0 -0.0896440744400024 4.0
0 -0.0522466897964478 5.0
0 0.058188796043396 6.0
0 -0.0864368677139282 7.0
0 -0.0929569602012634 8.0
0 -0.0304991006851196 9.0
0 0.028719425201416 10.0
0 0.00359046459197998 11.0
0 -0.070142388343811 12.0
62.797252697 62.3888168334961 13.0
38.960147004 34.6457939147949 14.0
0 -0.0306259393692017 15.0
0 -0.0531865358352661 16.0
35.919535291 35.811653137207 17.0
0 0.0674260854721069 18.0
0 0.00744259357452393 19.0
0 -0.00492298603057861 20.0
0 0.006386399269104 21.0
0 0.0526269674301147 22.0
27.069836156 26.302921295166 23.0
0 0.0982650518417358 24.0
0 0.567738592624664 25.0
0 0.0576609373092651 26.0
0 0.0110013484954834 27.0
0 -0.106998562812805 28.0
0 -0.0723797082901001 29.0
0 -0.0234382152557373 30.0
0 -0.0826424360275269 31.0
0 0.039141058921814 32.0
0 -0.135532259941101 33.0
0 -0.146278023719788 34.0
0 -0.0369871854782104 35.0
0 -0.0327585935592651 36.0
0 -0.0101708173751831 37.0
0 -0.0141559839248657 38.0
0 0.0257436037063599 39.0
0 0.140819430351257 40.0
0 -0.0570167303085327 41.0
0 0.0556138753890991 42.0
286.00464313 295.640594482422 43.0
0 0.00974690914154053 44.0
0 0.014427125453949 45.0
0 -0.0061265230178833 46.0
0 -0.00917160511016846 47.0
0 -0.0271979570388794 48.0
0.88651629558 0.306464910507202 49.0
0.094675648822 0.0756703615188599 50.0
1.7437741584 1.01402044296265 51.0
0 -0.0854188203811646 52.0
0 -0.0808259248733521 53.0
0 0.0607806444168091 54.0
0 -0.144493460655212 55.0
0 -0.0609828233718872 56.0
0 0.00865495204925537 57.0
0 -0.0619784593582153 58.0
0 -0.0305427312850952 59.0
0 0.0815874338150024 60.0
0 0.0144296884536743 61.0
0 0.7867671251297 62.0
224.84921263 255.83903503418 0.0
0 -0.0610424280166626 1.0
0 -0.0610224008560181 2.0
0 -0.0317426919937134 3.0
0 -0.0639671087265015 4.0
0 -0.0403081178665161 5.0
0 -0.00412571430206299 6.0
0 -0.00889647006988525 7.0
0 -0.0737793445587158 8.0
0 0.0913223028182983 9.0
0 -0.0125271081924438 10.0
0 0.0125945210456848 11.0
0 0.00862252712249756 12.0
63.863949206 62.9122428894043 13.0
27.311140513 32.0247802734375 14.0
0 -0.043923020362854 15.0
0 -0.051784873008728 16.0
39.04707468 40.2245864868164 17.0
0 0.00613248348236084 18.0
0 -0.0215147733688354 19.0
0 0.0401543378829956 20.0
0 -0.0368596315383911 21.0
0 -0.0599347352981567 22.0
41.781921956 41.3456344604492 23.0
0 -0.109683156013489 24.0
0 0.653628468513489 25.0
0 0.0316756963729858 26.0
0 0.0238453149795532 27.0
0 -0.0492135286331177 28.0
0 -0.0805402994155884 29.0
0 -0.0714448690414429 30.0
0 -0.0743855237960815 31.0
0 0.0599702596664429 32.0
0 -0.0850087404251099 33.0
0 -0.0333348512649536 34.0
0 0.0570065975189209 35.0
0 -0.12625777721405 36.0
0 0.0446888208389282 37.0
0 -0.154605984687805 38.0
0 -0.241998314857483 39.0
0 -0.131308197975159 40.0
0 0.0449677705764771 41.0
0 -0.103532433509827 42.0
291.77721056 296.153839111328 43.0
0 -0.039004921913147 44.0
0 0.100400686264038 45.0
0 -0.0125507116317749 46.0
0 0.00600659847259521 47.0
0 0.000100970268249512 48.0
0 0.268181771039963 49.0
0.10685570861 0.0713812112808228 50.0
2.3194510292 0.848661363124847 51.0
0 -0.077886700630188 52.0
0 0.00515663623809814 53.0
0 -0.0158540010452271 54.0
0 -0.0325783491134644 55.0
0 -0.067292332649231 56.0
0 0.0262101888656616 57.0
0 -0.044922947883606 58.0
0 -0.0379461050033569 59.0
0 -0.0569735765457153 60.0
0 -0.0264314413070679 61.0
0 0.824937701225281 62.0
268.52452419 251.006439208984 0.0
0 0.0102633237838745 1.0
0 -0.0896009206771851 2.0
0 -0.002846360206604 3.0
0 -0.0565105676651001 4.0
0 0.0259917974472046 5.0
0 -0.032964825630188 6.0
0 -0.012635350227356 7.0
0 -0.0132619142532349 8.0
0 0.0748332738876343 9.0
0 -0.124833941459656 10.0
0 -0.0123006701469421 11.0
0 0.0134414434432983 12.0
74.758009153 74.74853515625 13.0
16.162382967 27.2658100128174 14.0
0 -0.0621534585952759 15.0
0 0.0122357606887817 16.0
30.245221427 31.3356151580811 17.0
0 -0.0270653963088989 18.0
0 -0.0137962102890015 19.0
0 0.0258256196975708 20.0
0 0.0396484136581421 21.0
0 0.00838315486907959 22.0
34.992444182 33.7352333068848 23.0
0 0.0806268453598022 24.0
0 0.605553805828094 25.0
0 0.0643442869186401 26.0
0 -0.057929515838623 27.0
0 0.0541085004806519 28.0
0 -0.0718785524368286 29.0
0 -0.0368835926055908 30.0
0 -0.0404938459396362 31.0
0 -0.00318825244903564 32.0
0 -0.0464977025985718 33.0
0 -0.135131001472473 34.0
0 0.201042294502258 35.0
0 0.0310074090957642 36.0
0 -0.0492027997970581 37.0
0 -0.0657781362533569 38.0
0 0.034943699836731 39.0
0 0.124696612358093 40.0
0 -0.0416892766952515 41.0
0 0.0355628728866577 42.0
297.29688204 297.030334472656 43.0
0 -0.0330911874771118 44.0
0 0.0320664644241333 45.0
0 -0.0147415399551392 46.0
0 0.023119330406189 47.0
0 0.0427607297897339 48.0
0.64313282151 0.282517850399017 49.0
0.10394210898 0.0674053430557251 50.0
1.934684858 0.958823502063751 51.0
0 -0.0135143995285034 52.0
0 -0.0452507734298706 53.0
0 0.035535454750061 54.0
0 -0.0444973707199097 55.0
0 -0.00181996822357178 56.0
0 -0.0234378576278687 57.0
0 -0.0672429800033569 58.0
0 -0.0284467935562134 59.0
0 -0.118153691291809 60.0
0 0.0334686040878296 61.0
0 0.875580668449402 62.0
227.12903339 254.264633178711 0.0
0 -0.0577512979507446 1.0
0 -0.0503112077713013 2.0
0 0.00369107723236084 3.0
0 -0.0535777807235718 4.0
0 -0.0650900602340698 5.0
0 0.000291228294372559 6.0
0 -0.142815232276917 7.0
0 -0.0354065895080566 8.0
0 0.0103257894515991 9.0
0 -0.0394922494888306 10.0
0 -0.0428652763366699 11.0
0 0.109924674034119 12.0
62.363471687 60.9964485168457 13.0
34.022535837 35.1813011169434 14.0
0 -0.000397801399230957 15.0
0 0.00270497798919678 16.0
32.159857575 31.851375579834 17.0
0 0.00921452045440674 18.0
0 -0.0106562376022339 19.0
0 -0.0267306566238403 20.0
0 -0.0318135023117065 21.0
0 0.0112060308456421 22.0
31.054229572 30.0160751342773 23.0
0 -0.117620587348938 24.0
0 0.723870754241943 25.0
0 0.0237723588943481 26.0
0 0.0490752458572388 27.0
0 -0.047643780708313 28.0
0 -0.0910249948501587 29.0
0 -0.0463018417358398 30.0
0 -0.066559910774231 31.0
0 0.0382207632064819 32.0
0 0.0457590818405151 33.0
0 -0.0152226686477661 34.0
0 -0.0756789445877075 35.0
0 0.141546368598938 36.0
0 0.0106052160263062 37.0
0 -0.0346115827560425 38.0
0 0.0530811548233032 39.0
0 0.10336434841156 40.0
0 -0.0131477117538452 41.0
0 0.0459235906600952 42.0
287.89130055 296.061492919922 43.0
0 0.0419849157333374 44.0
0 0.11550909280777 45.0
0 -0.0205775499343872 46.0
0 -0.0285688638687134 47.0
0 0.0386990308761597 48.0
0.6660507776 0.313115030527115 49.0
0.086564654553 0.049411416053772 50.0
1.6741268827 0.945624053478241 51.0
0 -0.101519465446472 52.0
0 -0.0499011278152466 53.0
0 -0.0220679044723511 54.0
0 -0.0420616865158081 55.0
0 -0.0618418455123901 56.0
0 0.048977255821228 57.0
0 -0.0669971704483032 58.0
0 -0.0161491632461548 59.0
0 -0.093622088432312 60.0
0 -0.0267542600631714 61.0
0 0.804193139076233 62.0
256.02880642 253.457611083984 0.0
0 -0.0925794839859009 1.0
0 -0.101240038871765 2.0
0 -0.0158308744430542 3.0
0 -0.0318726301193237 4.0
0 -0.0762659311294556 5.0
0 0.0511726140975952 6.0
0 0.0282906293869019 7.0
0 -0.0206673145294189 8.0
0 0.0600141286849976 9.0
0 -0.0176700353622437 10.0
0 -0.00102543830871582 11.0
0 0.0201190710067749 12.0
55.420499037 54.0609931945801 13.0
36.72611185 38.2284202575684 14.0
0 -0.0162039995193481 15.0
0 -0.0380839109420776 16.0
27.728908385 27.1500186920166 17.0
0 0.0184978246688843 18.0
0 -0.0235277414321899 19.0
0 0.00549876689910889 20.0
0 -0.0353177785873413 21.0
0 -0.00753223896026611 22.0
35.510548487 34.8196601867676 23.0
0 0.0192211866378784 24.0
0 0.474397301673889 25.0
0 0.0548561811447144 26.0
0 -0.00220143795013428 27.0
0 0.165225148200989 28.0
0 -0.0343791246414185 29.0
0 -0.0563967227935791 30.0
0 0.0207842588424683 31.0
0 0.0526198148727417 32.0
0 -0.0258218050003052 33.0
0 -0.007088303565979 34.0
0 -0.102917432785034 35.0
0 -0.0199524164199829 36.0
0 -0.049676775932312 37.0
0 -0.0176557302474976 38.0
0 -0.0175186395645142 39.0
0 0.0968116521835327 40.0
0 0.0686980485916138 41.0
0 -0.106855034828186 42.0
301.58470093 296.440856933594 43.0
0 0.0578454732894897 44.0
0 -0.0428115129470825 45.0
0 0.00355565547943115 46.0
0 -0.0310424566268921 47.0
0 0.0513929128646851 48.0
0 0.247817397117615 49.0
0.083739554075 0.0792673826217651 50.0
1.7598980153 0.845071315765381 51.0
0 -0.0451229810714722 52.0
0 0.0130230188369751 53.0
0 0.0109378099441528 54.0
0 -0.0922418832778931 55.0
0 -0.0831176042556763 56.0
0 0.00322186946868896 57.0
0 -0.0119718313217163 58.0
0 -0.0502492189407349 59.0
0 -0.0885041952133179 60.0
0 0.0499368906021118 61.0
0 0.892583668231964 62.0
236.47592496 258.21533203125 0.0
0 -0.0775676965713501 1.0
0 0.0349630117416382 2.0
0 -0.0431963205337524 3.0
0 -0.023438572883606 4.0
0 -0.017015814781189 5.0
0 -0.0393277406692505 6.0
0 -0.0398768186569214 7.0
0 0.00824308395385742 8.0
0 0.000279784202575684 9.0
0 -0.0687974691390991 10.0
0 0.0416926145553589 11.0
0 -0.091652512550354 12.0
57.951924482 57.4831047058105 13.0
33.361025765 36.6773529052734 14.0
0 -0.0285583734512329 15.0
0 -0.00019371509552002 16.0
39.877750025 40.2276420593262 17.0
0 0.00973498821258545 18.0
0 0.0148802995681763 19.0
0 0.0278013944625854 20.0
0 -0.054444432258606 21.0
0 0.0221408605575562 22.0
28.989765415 27.9449996948242 23.0
0 0.0843542814254761 24.0
0 0.44010978937149 25.0
0 0.061911940574646 26.0
0 0.0589488744735718 27.0
0 -0.0639134645462036 28.0
0 -0.0715094804763794 29.0
0 -0.0684224367141724 30.0
0 -0.0105575323104858 31.0
0 0.0630730390548706 32.0
0 -0.00749504566192627 33.0
0 -0.0371183156967163 34.0
0 -0.0546883344650269 35.0
0 -0.0470203161239624 36.0
0 -0.0534189939498901 37.0
0 -0.0101708173751831 38.0
0 -0.0127967596054077 39.0
0 0.019283652305603 40.0
0 0.19200587272644 41.0
0 -0.0782686471939087 42.0
297.73425418 294.937652587891 43.0
0 -0.0504366159439087 44.0
0 -0.022678017616272 45.0
0 -0.00404012203216553 46.0
0 -0.00759315490722656 47.0
0 0.0724371671676636 48.0
0 0.374559670686722 49.0
0.082918852283 0.047263503074646 50.0
1.6807894815 0.940581858158112 51.0
0 -0.0870801210403442 52.0
0 -0.080180287361145 53.0
0 0.00622069835662842 54.0
0 -0.0424708127975464 55.0
0 -0.0493742227554321 56.0
0 0.0126775503158569 57.0
0 -0.0116478204727173 58.0
0 -0.0684810876846313 59.0
0 0.0655723810195923 60.0
0 -0.0293344259262085 61.0
0 0.803969025611877 62.0
261.20913477 260.040863037109 0.0
0 -0.127426743507385 1.0
0 0.0202740430831909 2.0
0 0.015832781791687 3.0
0 -0.0622178316116333 4.0
0 -0.0559011697769165 5.0
0 0.0293735265731812 6.0
0 -0.0491472482681274 7.0
0 -0.0500646829605103 8.0
0 0.0936349630355835 9.0
0 -0.0022042989730835 10.0
0 0.021720290184021 11.0
0 0.0293639898300171 12.0
73.540565728 72.4997406005859 13.0
37.497280481 26.4302768707275 14.0
0 -0.0484691858291626 15.0
0 -0.0337082147598267 16.0
34.065180874 33.9512786865234 17.0
0 0.0145102739334106 18.0
0 -0.0240544080734253 19.0
0 0.00223171710968018 20.0
0 0.0211271047592163 21.0
0 -0.0435940027236938 22.0
39.521300559 39.0912818908691 23.0
0 -0.0564013719558716 24.0
0 0.908216655254364 25.0
0 0.0585850477218628 26.0
0 0.0813614130020142 27.0
0 0.00686538219451904 28.0
0 -0.048366904258728 29.0
0 -0.0344607830047607 30.0
0 -0.0115352869033813 31.0
0 -0.00957882404327393 32.0
0 0.0120362043380737 33.0
0 -0.00371372699737549 34.0
0 -0.0432792901992798 35.0
0 0.0330365896224976 36.0
0 -0.050782322883606 37.0
0 -0.0421906709671021 38.0
0 0.0429810285568237 39.0
0 0.0509041547775269 40.0
0 -0.044402003288269 41.0
0 -0.0457817316055298 42.0
302.88631852 297.881896972656 43.0
0 0.040127158164978 44.0
0 -0.00055992603302002 45.0
0 -0.0302995443344116 46.0
0 -0.0095142126083374 47.0
0 0.0130223035812378 48.0
0 0.270086228847504 49.0
0.090964684029 0.074736475944519 50.0
2.1009679515 0.893880188465118 51.0
0 -0.0712395906448364 52.0
0 -0.0138407945632935 53.0
0 0.0533381700515747 54.0
0 -0.119878649711609 55.0
0 -0.000770449638366699 56.0
0 -0.0200506448745728 57.0
0 -0.0663770437240601 58.0
0 -0.00286352634429932 59.0
0 -0.0695627927780151 60.0
0 0.05823814868927 61.0
0 0.852154135704041 62.0
291.39999961 254.844207763672 0.0
0 0.00895583629608154 1.0
0 -0.121650576591492 2.0
0 -0.0344556570053101 3.0
0 -0.00249898433685303 4.0
0 -0.0523139238357544 5.0
0 0.0136841535568237 6.0
0 -0.237338662147522 7.0
0 0.0506116151809692 8.0
0 0.0389283895492554 9.0
0 -0.0134633779525757 10.0
0 0.0052260160446167 11.0
0 -0.228695511817932 12.0
52.735730162 52.2147331237793 13.0
45.758196567 41.5719375610352 14.0
0 0.00862205028533936 15.0
0 0.028734564781189 16.0
36.211377639 36.5949058532715 17.0
0 -0.00833570957183838 18.0
0 -0.0110255479812622 19.0
0 0.0192314386367798 20.0
0 -0.0021442174911499 21.0
0 -0.0477243661880493 22.0
30.790039403 30.6131725311279 23.0
0 -0.0702720880508423 24.0
0 0.541232943534851 25.0
0 -0.00202453136444092 26.0
0 0.0417790412902832 27.0
0 0.0377873182296753 28.0
0 -0.0179709196090698 29.0
0 -0.0425461530685425 30.0
0 -0.0490719079971313 31.0
0 -0.0857869386672974 32.0
0 0.0084303617477417 33.0
0 0.0267654657363892 34.0
0 0.0630618333816528 35.0
0 -0.0420738458633423 36.0
0 0.0176156759262085 37.0
0 -0.0165678262710571 38.0
0 0.00365555286407471 39.0
0 0.0967339277267456 40.0
0 -0.0805854797363281 41.0
0 0.0243076086044312 42.0
297.66704426 295.010711669922 43.0
0 -0.0445057153701782 44.0
0 0.0335977077484131 45.0
0 0.0100427865982056 46.0
0 -0.0447653532028198 47.0
0 -0.0185519456863403 48.0
0 0.298160374164581 49.0
0.084100237116 0.0490690469741821 50.0
2.3381196811 0.945186257362366 51.0
0 -0.0448580980300903 52.0
0 -0.0253492593765259 53.0
0 0.0373743772506714 54.0
0 -0.129936575889587 55.0
0 -0.0601862668991089 56.0
0 -0.0271378755569458 57.0
0 -0.0867823362350464 58.0
0 -0.0159465074539185 59.0
0 -0.0980013608932495 60.0
0 -0.00128304958343506 61.0
0 0.823105812072754 62.0
224.6741042 257.335540771484 0.0
0 -0.0284658670425415 1.0
0 -0.00394284725189209 2.0
0 -0.0398842096328735 3.0
0 0.0822213888168335 4.0
0 -0.0762127637863159 5.0
0 -0.0193058252334595 6.0
0 -0.119368672370911 7.0
0 -0.0309488773345947 8.0
0 0.0344527959823608 9.0
0 -0.0117028951644897 10.0
0 -0.00123518705368042 11.0
0 0.0230382680892944 12.0
69.974356348 70.0186996459961 13.0
27.050373049 28.0736274719238 14.0
0 -0.0216065645217896 15.0
0 0.0333312749862671 16.0
31.612646398 32.0683937072754 17.0
0 0.0287574529647827 18.0
0 -0.00159776210784912 19.0
0 0.0252636671066284 20.0
0 -0.00389540195465088 21.0
0 -5.25712966918945e-05 22.0
37.724724263 36.9272270202637 23.0
0 -0.0521837472915649 24.0
0 0.560258150100708 25.0
0 0.06993567943573 26.0
0 -0.0905282497406006 27.0
0 -0.100136399269104 28.0
0 -0.0208135843276978 29.0
0 -0.0814002752304077 30.0
0 -0.0608752965927124 31.0
0 -0.00480282306671143 32.0
0 0.0477839708328247 33.0
0 -0.0206576585769653 34.0
0 0.0509179830551147 35.0
0 0.0125890970230103 36.0
0 -0.0984140634536743 37.0
0 -0.00120484828948975 38.0
0 -0.00641167163848877 39.0
0 0.124770522117615 40.0
0 -0.0335615873336792 41.0
0 -0.0850950479507446 42.0
321.01608269 294.803802490234 43.0
0 0.0640720129013062 44.0
0 -0.0094599723815918 45.0
0 -0.0219103097915649 46.0
0 -0.0553759336471558 47.0
0 0.0481113195419312 48.0
0 0.299491763114929 49.0
0.073054749805 0.0694769620895386 50.0
1.7290962862 0.885701656341553 51.0
0 -0.0775483846664429 52.0
0 -0.00602114200592041 53.0
0 -0.0304361581802368 54.0
0 -0.0548621416091919 55.0
0 0.000983357429504395 56.0
0 -0.0589271783828735 57.0
0 0.0122963190078735 58.0
0 -0.00733745098114014 59.0
0 -0.0741826295852661 60.0
0 -0.0505849123001099 61.0
0 0.825623154640198 62.0
282.49040078 247.429122924805 0.0
0 -0.0511661767959595 1.0
0 0.0247839689254761 2.0
0 -0.0409501791000366 3.0
0 -0.0637439489364624 4.0
0 -0.122303605079651 5.0
0 -0.00633156299591064 6.0
0 -0.0217033624649048 7.0
0 -0.0112746953964233 8.0
0 0.0873836278915405 9.0
0 -0.0365825891494751 10.0
0 -0.00385284423828125 11.0
0 -0.0657335519790649 12.0
51.104022208 51.1429443359375 13.0
42.638712553 41.2290687561035 14.0
0 -0.0462976694107056 15.0
0 0.0113557577133179 16.0
25.083207806 23.3375949859619 17.0
0 0.0416454076766968 18.0
0 0.0523427724838257 19.0
0 -0.0339111089706421 20.0
0 -0.0153895616531372 21.0
0 -0.0167676210403442 22.0
39.910102948 40.0739555358887 23.0
0 -0.155972599983215 24.0
14.426160808 0.531701147556305 25.0
0 -0.0412312746047974 26.0
0 -0.124857544898987 27.0
0 -0.101142525672913 28.0
0 -0.114567637443542 29.0
0 -0.0308438539505005 30.0
0 -0.0451022386550903 31.0
0 0.062430739402771 32.0
0 -0.0370169878005981 33.0
0 -0.00870692729949951 34.0
0 0.0269192457199097 35.0
0 0.143429636955261 36.0
0 -0.0237926244735718 37.0
0 0.0754557847976685 38.0
0 -0.0505026578903198 39.0
0 0.111640810966492 40.0
0 -0.0985375642776489 41.0
0 0.0149043798446655 42.0
287.29363031 296.861907958984 43.0
0 -0.114126801490784 44.0
0 0.0393061637878418 45.0
0 -0.0465754270553589 46.0
0 -0.0193263292312622 47.0
0 -0.19182550907135 48.0
0.90023994722 0.281334042549133 49.0
0.083858032355 0.089929461479187 50.0
1.4939362423 0.947248101234436 51.0
0 -0.0655208826065063 52.0
0 -0.0397630929946899 53.0
0 0.028197169303894 54.0
0 -0.000991702079772949 55.0
0 -0.0334638357162476 56.0
0 -0.0237387418746948 57.0
0 -0.0773106813430786 58.0
0 -0.0170110464096069 59.0
0 0.235669016838074 60.0
0 -0.00268924236297607 61.0
0 0.807555317878723 62.0
281.61783302 253.218627929688 0.0
0 -0.0589853525161743 1.0
0 -0.0571404695510864 2.0
0 -0.0540660619735718 3.0
0 -0.0331960916519165 4.0
0 -0.00644075870513916 5.0
0 -0.109304070472717 6.0
0 0.00524270534515381 7.0
0 0.00843489170074463 8.0
0 -0.028943657875061 9.0
0 -0.0184639692306519 10.0
0 0.0168585777282715 11.0
0 0.0696643590927124 12.0
73.382989963 72.1435775756836 13.0
16.380195098 27.5814342498779 14.0
0 -0.0556198358535767 15.0
0 0.0013660192489624 16.0
27.699902224 27.490873336792 17.0
0 -0.000518918037414551 18.0
0 0.00648581981658936 19.0
0 0.00808560848236084 20.0
0 -0.0226997137069702 21.0
0 0.000595927238464355 22.0
28.919722692 28.531644821167 23.0
0 0.00464522838592529 24.0
0 0.568659961223602 25.0
0 0.0923737287521362 26.0
0 -0.0229836702346802 27.0
0 -0.0610634088516235 28.0
0 -0.048956036567688 29.0
0 -0.00708127021789551 30.0
0 -0.0352846384048462 31.0
0 0.0182932615280151 32.0
0 -0.0182510614395142 33.0
0 -0.0063551664352417 34.0
0 -0.0757941007614136 35.0
0 -0.0544873476028442 36.0
0 -0.0229469537734985 37.0
0 0.0270963907241821 38.0
0 0.0176302194595337 39.0
0 0.00787103176116943 40.0
0 -0.0502942800521851 41.0
0 -0.0388685464859009 42.0
302.6795406 296.880218505859 43.0
0 9.33408737182617e-05 44.0
0 -0.0121331214904785 45.0
0 -0.00617468357086182 46.0
0 0.0272477865219116 47.0
0 0.0402611494064331 48.0
0 0.276189655065536 49.0
0.087304560495 0.0624195337295532 50.0
2.0037237592 1.00692343711853 51.0
0 -0.0355435609817505 52.0
0 -0.0756403207778931 53.0
0 0.0122660398483276 54.0
0 -0.0598620176315308 55.0
0 0.0131155252456665 56.0
0 -0.0342761278152466 57.0
0 -0.0378230810165405 58.0
0 -0.0273419618606567 59.0
0 0.017809271812439 60.0
0 -0.0277727842330933 61.0
0 0.826504945755005 62.0
274.78671435 254.963790893555 0.0
0 0.0217374563217163 1.0
0 0.0312341451644897 2.0
0 -0.0178495645523071 3.0
0 -0.00842034816741943 4.0
0 -0.224735856056213 5.0
0 0.0375179052352905 6.0
0 0.0347722768783569 7.0
0 -0.0784465074539185 8.0
0 0.000705122947692871 9.0
0 -0.0225832462310791 10.0
0 0.0221759080886841 11.0
0 0.0509103536605835 12.0
53.824693066 52.4290580749512 13.0
43.349507567 39.0580787658691 14.0
0 -0.0244351625442505 15.0
0 -0.0418468713760376 16.0
40.470992263 42.2489433288574 17.0
0 -0.0399554967880249 18.0
0 -0.0175322294235229 19.0
0 0.0149062871932983 20.0
0 -0.0259650945663452 21.0
0 0.0177358388900757 22.0
42.377486054 41.3244552612305 23.0
0 0.0990504026412964 24.0
0 0.584508836269379 25.0
0 0.0134204626083374 26.0
0 0.016842246055603 27.0
0 -0.00215828418731689 28.0
0 -0.0507124662399292 29.0
0 -0.0633366107940674 30.0
0 -0.0038830041885376 31.0
0 0.00979983806610107 32.0
0 0.15867030620575 33.0
0 -0.0619467496871948 34.0
0 -0.00119912624359131 35.0
0 0.0353680849075317 36.0
0 -0.0101686716079712 37.0
0 -0.0304824113845825 38.0
0 0.0868238210678101 39.0
0 0.0320662260055542 40.0
0 -0.173434376716614 41.0
0 0.0251983404159546 42.0
315.30465686 293.742279052734 43.0
0 -0.138477921485901 44.0
0 0.0261341333389282 45.0
0 -0.0137957334518433 46.0
0 -0.0689264535903931 47.0
0 0.0550154447555542 48.0
0 0.400895118713379 49.0
0.071187458495 0.0166915655136108 50.0
2.0719745164 0.957761347293854 51.0
0 -0.0423561334609985 52.0
0 -0.0104151964187622 53.0
0 0.0755933523178101 54.0
0 -0.0865720510482788 55.0
0 -0.0512846708297729 56.0
0 -0.0207830667495728 57.0
0 -0.00628340244293213 58.0
0 -0.0421906709671021 59.0
0 0.0176500082015991 60.0
0 0.0282834768295288 61.0
0 0.854607820510864 62.0
209.96597811 254.912124633789 0.0
0 -0.0632740259170532 1.0
0 -0.0783082246780396 2.0
0 -0.00108230113983154 3.0
0 -0.0467680692672729 4.0
0 0.0272890329360962 5.0
0 0.510104537010193 6.0
0 -0.0465484857559204 7.0
0 -0.0187126398086548 8.0
0 -0.0315214395523071 9.0
0 -0.0153454542160034 10.0
0 -0.206761598587036 11.0
0 -0.042946457862854 12.0
74.439340236 74.5957260131836 13.0
18.235403543 21.8338603973389 14.0
0 -0.0844932794570923 15.0
0 -0.0484620332717896 16.0
38.328187807 38.1033706665039 17.0
0 -0.0182498693466187 18.0
0 -0.0208925008773804 19.0
0 0.0177837610244751 20.0
0 -0.0506223440170288 21.0
0 -0.0630060434341431 22.0
42.001709153 40.5466346740723 23.0
0 0.0539220571517944 24.0
0 0.483745217323303 25.0
0 -0.0526062250137329 26.0
0 0.0703810453414917 27.0
0 -0.027351975440979 28.0
0 -0.108929753303528 29.0
0 0.20236074924469 30.0
0 -0.00185120105743408 31.0
0 -0.00349390506744385 32.0
0 0.0847848653793335 33.0
0 -0.0238426923751831 34.0
0 0.0254179239273071 35.0
0 0.000994086265563965 36.0
0 -0.0304573774337769 37.0
0 0.0257023572921753 38.0
0 0.0222347974777222 39.0
0 0.137376666069031 40.0
0 -0.0516675710678101 41.0
0 -0.0311495065689087 42.0
296.09550789 294.513763427734 43.0
0 -0.0136677026748657 44.0
0 0.0606379508972168 45.0
0 -0.0117934942245483 46.0
0 -0.0640057325363159 47.0
0 0.0420769453048706 48.0
0 0.215544104576111 49.0
0.095773616928 0.0799709558486938 50.0
1.5691449372 0.860180020332336 51.0
0 -0.00714147090911865 52.0
0 -0.0372155904769897 53.0
0 -0.0616284608840942 54.0
0 -0.0748637914657593 55.0
0 0.0165318250656128 56.0
0 -0.0301519632339478 57.0
0 0.0220826864242554 58.0
0 -0.0558713674545288 59.0
0 -0.0340410470962524 60.0
0 -0.0536621809005737 61.0
0 0.798815727233887 62.0
267.3387124 257.348083496094 0.0
0 -0.0173689126968384 1.0
0 -0.0929683446884155 2.0
0 -0.000356316566467285 3.0
0 -0.0133732557296753 4.0
0 -0.0412379503250122 5.0
0 -0.0838204622268677 6.0
0 -0.0234113931655884 7.0
0 -0.00997543334960938 8.0
0 0.104226469993591 9.0
0 -0.0097278356552124 10.0
0 0.0448267459869385 11.0
0 -0.0331388711929321 12.0
76.580262766 75.8909683227539 13.0
21.955859086 25.9769496917725 14.0
0 -0.0294634103775024 15.0
0 0.020774245262146 16.0
34.115091676 35.07958984375 17.0
0 0.0155094861984253 18.0
0 -0.0423368215560913 19.0
0 0.0205842256546021 20.0
0 -0.0148407220840454 21.0
0 -0.0129796266555786 22.0
25.75549732 25.0193157196045 23.0
0 0.0711325407028198 24.0
0 0.507965207099915 25.0
0 0.00758826732635498 26.0
0 -0.0452440977096558 27.0
0 -0.039958119392395 28.0
0 -0.0455001592636108 29.0
0 -0.0960253477096558 30.0
0 -0.03410804271698 31.0
0 0.113160967826843 32.0
0 0.0052562952041626 33.0
0 -0.00594651699066162 34.0
0 0.0140570402145386 35.0
0 -0.0671051740646362 36.0
0 -0.0445469617843628 37.0
0 -0.05485999584198 38.0
0 -0.0311638116836548 39.0
0 -0.0885578393936157 40.0
0 -0.000847101211547852 41.0
0 -0.104061722755432 42.0
299.51989916 296.881439208984 43.0
0 0.0302790403366089 44.0
0 -0.0351413488388062 45.0
0 -0.0103508234024048 46.0
0 -0.0683833360671997 47.0
0 0.052702784538269 48.0
0 0.286564826965332 49.0
0.063777482078 0.0528150796890259 50.0
2.3355659938 0.97152704000473 51.0
0 -0.0747989416122437 52.0
0 -0.0248731374740601 53.0
0 0.0648914575576782 54.0
0 -0.0834640264511108 55.0
0 -0.0148860216140747 56.0
0 -0.0291863679885864 57.0
0 -0.0583022832870483 58.0
0 -0.0674177408218384 59.0
0 0.00762641429901123 60.0
0 0.0537360906600952 61.0
0 0.935106456279755 62.0
275.41966082 254.266677856445 0.0
0 -0.0439649820327759 1.0
0 -0.015093207359314 2.0
0 -0.0788730382919312 3.0
0 0.00653493404388428 4.0
0 0.0221072435379028 5.0
0 -0.0756632089614868 6.0
0 -0.0594249963760376 7.0
0 -0.0381261110305786 8.0
0 0.0622791051864624 9.0
0 0.00851035118103027 10.0
0 0.0274934768676758 11.0
0 0.0934385061264038 12.0
69.107051479 68.4968032836914 13.0
36.713784803 32.1457633972168 14.0
0 -0.0225521326065063 15.0
0 -0.0109678506851196 16.0
33.26894436 32.9591026306152 17.0
0 0.0296119451522827 18.0
0 0.0118166208267212 19.0
0 0.0102952718734741 20.0
0 -0.0344945192337036 21.0
0 0.0417050123214722 22.0
28.125697556 27.5825614929199 23.0
0 0.0671433210372925 24.0
0 0.575300991535187 25.0
0 -0.00891625881195068 26.0
0 0.0144621729850769 27.0
0 -0.0621320009231567 28.0
0 -0.0573493242263794 29.0
0 0.0974107980728149 30.0
0 -0.0769754648208618 31.0
0 -0.066106915473938 32.0
0 0.039603590965271 33.0
0 -0.00441038608551025 34.0
0 -0.0640143156051636 35.0
0 -0.00381052494049072 36.0
0 0.0257171392440796 37.0
0 0.113649487495422 38.0
0 0.01103675365448 39.0
0 -0.0457097291946411 40.0
0 -0.0730440616607666 41.0
0 -0.0955711603164673 42.0
286.46455959 293.536224365234 43.0
0 -0.0153290033340454 44.0
0 0.051386296749115 45.0
0 -0.00797474384307861 46.0
0 -0.0198971033096313 47.0
0 -0.0808793306350708 48.0
0 0.229947328567505 49.0
0.063472850804 0.259387731552124 50.0
2.3516517025 0.868818521499634 51.0
0 -0.0467973947525024 52.0
0 -0.0760983228683472 53.0
0 0.0418480634689331 54.0
0 -0.0276566743850708 55.0
0 0.00759994983673096 56.0
0 -0.0414949655532837 57.0
0 -0.0215839147567749 58.0
0 -0.0081712007522583 59.0
0 0.00644910335540771 60.0
0 -0.00131309032440186 61.0
0 0.906249165534973 62.0
289.04765868 253.83723449707 0.0
0 0.0744789838790894 1.0
0 -0.010309100151062 2.0
0 0.0127836465835571 3.0
0 -0.0575257539749146 4.0
0 -0.102333903312683 5.0
0 0.103764891624451 6.0
0 -0.0506936311721802 7.0
0 -0.0752251148223877 8.0
0 0.0760349035263062 9.0
0 0.00516402721405029 10.0
0 -0.00445497035980225 11.0
0 0.074114203453064 12.0
64.776636851 64.6308364868164 13.0
37.14044377 31.292049407959 14.0
0 -0.029938817024231 15.0
0 -0.0221003293991089 16.0
32.092707593 30.8278045654297 17.0
0 0.112338900566101 18.0
0 0.0535768270492554 19.0
0 -0.029882550239563 20.0
0 -0.0039135217666626 21.0
0 -0.00696766376495361 22.0
31.646382895 31.2316722869873 23.0
0 -0.0403281450271606 24.0
0 0.954393982887268 25.0
0 0.0888203382492065 26.0
0 0.113320708274841 27.0
0 -0.00428974628448486 28.0
0 -0.0234130620956421 29.0
0 -0.0775793790817261 30.0
0 -0.0604561567306519 31.0
0 0.00779616832733154 32.0
0 -0.0290282964706421 33.0
0 -0.0602425336837769 34.0
0 0.126048684120178 35.0
0 0.0690196752548218 36.0
0 -0.0588327646255493 37.0
0 -0.00463879108428955 38.0
0 0.0814676284790039 39.0
0 0.0913423299789429 40.0
0 -0.0256460905075073 41.0
0 -0.00956261157989502 42.0
292.54840512 295.141754150391 43.0
0 0.0680292844772339 44.0
0 -0.0538918972015381 45.0
0 0.00736749172210693 46.0
0 -0.0330816507339478 47.0
0 0.0358117818832397 48.0
1.111575566 0.276405453681946 49.0
0.10143928363 0.0870500802993774 50.0
2.2296753441 0.939166307449341 51.0
0 -0.053834080696106 52.0
0 -0.0437513589859009 53.0
0 0.00144779682159424 54.0
0 -0.064426064491272 55.0
0 -0.0172837972640991 56.0
0 0.0110374689102173 57.0
0 -0.00686848163604736 58.0
0 -0.0322455167770386 59.0
0 0.0120304822921753 60.0
0 0.0137268304824829 61.0
0 0.80138885974884 62.0
258.5528846 254.647399902344 0.0
0 0.0383318662643433 1.0
0 0.00125181674957275 2.0
0 -0.0185028314590454 3.0
0 -0.0531374216079712 4.0
0 -0.0553699731826782 5.0
0 0.0141524076461792 6.0
0 -0.0369402170181274 7.0
0 -0.0365550518035889 8.0
0 -0.0269399881362915 9.0
0 -0.0996180772781372 10.0
0 -0.0278726816177368 11.0
0 0.00754296779632568 12.0
68.212114709 67.5883407592773 13.0
28.114542912 28.2073993682861 14.0
0 -0.0272384881973267 15.0
0 0.0157526731491089 16.0
31.617475974 31.1399173736572 17.0
0 -0.0538603067398071 18.0
0 0.00505030155181885 19.0
0 -0.0175453424453735 20.0
0 -0.0152376890182495 21.0
0 -0.0341650247573853 22.0
37.466239438 37.0723495483398 23.0
0 0.0386050939559937 24.0
0 0.558171927928925 25.0
0 0.0681065320968628 26.0
0 0.023318886756897 27.0
0 -0.063024640083313 28.0
0 -0.0454617738723755 29.0
0 -0.0540271997451782 30.0
0 -0.0105811357498169 31.0
0 -0.0406562089920044 32.0
0 0.0938810110092163 33.0
0 -0.0214885473251343 34.0
0 -0.0331792831420898 35.0
0 -0.00135314464569092 36.0
0 -0.0210469961166382 37.0
0 0.0360304117202759 38.0
0 0.0290693044662476 39.0
0 0.1161288022995 40.0
0 -0.0499199628829956 41.0
0 -0.0380493402481079 42.0
307.27672119 295.240478515625 43.0
0 -0.00827562808990479 44.0
0 0.0314282774925232 45.0
0 -0.00653111934661865 46.0
0 0.0179375410079956 47.0
0 -0.0186811685562134 48.0
0 0.290997266769409 49.0
0.074389333406 0.0731488466262817 50.0
2.2516711333 0.907129347324371 51.0
0 -0.049355149269104 52.0
0 -0.00664317607879639 53.0
0 0.023257851600647 54.0
0 -0.106501460075378 55.0
0 0.0299807786941528 56.0
0 -0.0480028390884399 57.0
0 -0.0429657697677612 58.0
0 0.0190457105636597 59.0
0 -0.102886319160461 60.0
0 0.0346542596817017 61.0
0 0.840823173522949 62.0
266.36361609 257.617706298828 0.0
0 0.0370081663131714 1.0
0 0.0236150026321411 2.0
0 0.00442421436309814 3.0
0 -0.046594500541687 4.0
0 0.0224050283432007 5.0
0 0.0248993635177612 6.0
0 -0.0397917032241821 7.0
0 0.0160630941390991 8.0
0 -0.0224758386611938 9.0
0 0.0126620531082153 10.0
0 0.0344331860542297 11.0
0 0.0132325887680054 12.0
66.240177227 66.7938079833984 13.0
26.088771751 32.0547752380371 14.0
0 0.0119379758834839 15.0
0 -0.0511490106582642 16.0
28.521411783 28.6740398406982 17.0
0 0.0563267469406128 18.0
0 0.0297132730484009 19.0
0 0.0432735681533813 20.0
0 -0.00792801380157471 21.0
0 -0.0417944192886353 22.0
35.263424717 34.6229705810547 23.0
0 -0.114753842353821 24.0
0 1.01290559768677 25.0
0 0.068498969078064 26.0
0 0.0473161935806274 27.0
0 -0.0653659105300903 28.0
0 -0.0530427694320679 29.0
0 -0.0568016767501831 30.0
0 -0.0402718782424927 31.0
0 -0.0338197946548462 32.0
0 0.108542084693909 33.0
0 -0.0504442453384399 34.0
0 -0.0539437532424927 35.0
0 -0.0139185190200806 36.0
0 -0.038202166557312 37.0
0 0.0132888555526733 38.0
0 0.0710757970809937 39.0
0 0.0397228002548218 40.0
0 -0.0425049066543579 41.0
0 -0.0574108362197876 42.0
290.76819492 295.992950439453 43.0
0 0.00681531429290771 44.0
0 0.0312014222145081 45.0
0 -0.00593888759613037 46.0
0 -0.0536314249038696 47.0
0 0.0268939733505249 48.0
0 0.290787100791931 49.0
0.068664587311 0.0456727743148804 50.0
2.1048909833 0.901004254817963 51.0
0 -0.0681999921798706 52.0
0 -0.0552521944046021 53.0
0 -0.0304054021835327 54.0
0 -0.0542310476303101 55.0
0 -0.0437742471694946 56.0
0 -0.0391062498092651 57.0
0 -0.0754348039627075 58.0
0 -0.00103080272674561 59.0
0 -0.0514310598373413 60.0
0 0.0574911832809448 61.0
0 0.925715446472168 62.0
290.63223739 248.894943237305 0.0
0 0.000625014305114746 1.0
0 -0.0280598402023315 2.0
0 -0.0108143091201782 3.0
0 -0.0457357168197632 4.0
0 -0.0130983591079712 5.0
0 -0.281220078468323 6.0
0 0.0218814611434937 7.0
0 -0.0688009262084961 8.0
0 -0.126447319984436 9.0
0 0.0307782888412476 10.0
0 -0.0295090675354004 11.0
0 -0.012459397315979 12.0
57.387260065 55.8396186828613 13.0
22.255530525 40.1910552978516 14.0
0 0.00209128856658936 15.0
0 -0.0401140451431274 16.0
30.17586962 31.1406860351562 17.0
0 0.0360497236251831 18.0
0 -0.0117369890213013 19.0
0 -0.0307275056838989 20.0
0 -0.00259220600128174 21.0
0 -0.0311590433120728 22.0
35.122059061 33.2380180358887 23.0
0 0.043718695640564 24.0
0 1.74435472488403 25.0
0 -0.0132361650466919 26.0
0 0.0334535837173462 27.0
0 -0.0368276834487915 28.0
0 -0.0628515481948853 29.0
0 -0.0863273143768311 30.0
0 -0.0262075662612915 31.0
0 0.136356234550476 32.0
0 -0.00692856311798096 33.0
0 -0.0918911695480347 34.0
0 -0.0419889688491821 35.0
0 -0.0638331174850464 36.0
0 -0.0615750551223755 37.0
0 0.0200825929641724 38.0
0 0.0652753114700317 39.0
0 0.109405398368835 40.0
0 -0.00671184062957764 41.0
0 -0.121043801307678 42.0
288.03750844 297.125549316406 43.0
0 -0.0308901071548462 44.0
0 0.0335416197776794 45.0
0 0.000676751136779785 46.0
0 -0.0674129724502563 47.0
0 0.0172289609909058 48.0
0 0.306920647621155 49.0
0.082724115318 0.046094536781311 50.0
2.4291772059 0.988375067710876 51.0
0 -0.0798670053482056 52.0
0 0.00288617610931396 53.0
0 0.00524556636810303 54.0
0 -0.107912182807922 55.0
0 0.00195062160491943 56.0
0 -0.0280653238296509 57.0
0 -0.110772252082825 58.0
0 0.0373424291610718 59.0
0 -0.0914822816848755 60.0
0 -0.0504082441329956 61.0
0 0.81031858921051 62.0
224.79200779 256.217407226562 0.0
0 -0.0449587106704712 1.0
0 0.0484408140182495 2.0
0 0.0255094766616821 3.0
0 -0.0591551065444946 4.0
0 -0.0175672769546509 5.0
0 -0.0244032144546509 6.0
0 -0.0327202081680298 7.0
0 -0.0248392820358276 8.0
0 0.0739673376083374 9.0
0 -0.171003222465515 10.0
0 -0.017863392829895 11.0
0 0.00431668758392334 12.0
69.024486517 68.3887176513672 13.0
22.167757293 25.6759700775146 14.0
0 -0.0501662492752075 15.0
0 -0.0367294549942017 16.0
40.242351004 41.0076675415039 17.0
0 -0.0239683389663696 18.0
0 0.00836169719696045 19.0
0 0.0229469537734985 20.0
0 -0.0694319009780884 21.0
0 0.0462177991867065 22.0
27.488591991 27.7623882293701 23.0
0 -0.0348621606826782 24.0
0 0.657031774520874 25.0
0 0.0381678342819214 26.0
0 -0.0261510610580444 27.0
0 0.00806891918182373 28.0
0 -0.0873059034347534 29.0
0 -0.0462337732315063 30.0
0 -0.012933611869812 31.0
0 0.0254796743392944 32.0
0 0.0894309282302856 33.0
0 -0.0215882062911987 34.0
0 0.0223013162612915 35.0
0 0.052085280418396 36.0
0 -0.0682390928268433 37.0
0 0.00854647159576416 38.0
0 0.0256885290145874 39.0
0 -0.0758124589920044 40.0
0 -0.049505352973938 41.0
0 0.0467180013656616 42.0
280.93342873 297.448822021484 43.0
0 0.0150927305221558 44.0
0 -0.0647156238555908 45.0
0 -0.0364795923233032 46.0
0 -0.048197865486145 47.0
0 0.021807074546814 48.0
0 0.2883580327034 49.0
0.098561729148 0.0733538866043091 50.0
2.0492601744 0.900883734226227 51.0
0 -0.0141428709030151 52.0
0 -0.00455129146575928 53.0
0 0.0876919031143188 54.0
0 -0.0530132055282593 55.0
0 -0.0502573251724243 56.0
0 -0.0198367834091187 57.0
0 -0.0113552808761597 58.0
0 -0.0979748964309692 59.0
0 -0.0487931966781616 60.0
0 0.0131205320358276 61.0
0 0.825080037117004 62.0
260.470191 259.148834228516 0.0
0 -0.122472405433655 1.0
0 0.0116037130355835 2.0
0 0.0210980176925659 3.0
0 -0.0723601579666138 4.0
0 -0.0553344488143921 5.0
0 0.0125783681869507 6.0
0 -0.139627575874329 7.0
0 0.0625454187393188 8.0
0 0.0582736730575562 9.0
0 0.00900232791900635 10.0
0 -0.0412648320198059 11.0
0 0.0200742483139038 12.0
66.688147596 66.1921997070312 13.0
20.138393068 31.9767913818359 14.0
0 -0.037418007850647 15.0
0 -0.0128322839736938 16.0
33.314832262 33.9560699462891 17.0
0 0.00843393802642822 18.0
0 0.024635910987854 19.0
0 0.0243817567825317 20.0
0 -0.0573791265487671 21.0
0 0.0454138517379761 22.0
35.160068501 33.6110343933105 23.0
0 0.0430678129196167 24.0
0 0.511305034160614 25.0
0 -0.0492435693740845 26.0
0 0.0495656728744507 27.0
0 0.0571135282516479 28.0
0 -0.043822169303894 29.0
0 0.0279499292373657 30.0
0 -0.0464349985122681 31.0
0 -0.0462270975112915 32.0
0 0.0287119150161743 33.0
0 -0.0799738168716431 34.0
0 -0.00453579425811768 35.0
0 -0.118334650993347 36.0
0 -0.0306044816970825 37.0
0 0.0412696599960327 38.0
0 -0.0149587392807007 39.0
0 -0.0423804521560669 40.0
0 -0.0125263929367065 41.0
0 0.0337651968002319 42.0
289.9213926 295.87548828125 43.0
0 -0.00784647464752197 44.0
0 0.00325465202331543 45.0
0 0.0165461301803589 46.0
0 -0.0926061868667603 47.0
0 -0.0541039705276489 48.0
1.8482635241 0.309891939163208 49.0
0.070795443155 0.0506614446640015 50.0
2.0211316999 0.899564862251282 51.0
0 -0.0450612306594849 52.0
0 0.00291717052459717 53.0
0 0.0580734014511108 54.0
0 -0.0548063516616821 55.0
0 -0.063773512840271 56.0
0 0.0239189863204956 57.0
0 -0.0158394575119019 58.0
0 0.0111347436904907 59.0
0 -0.112695574760437 60.0
0 0.00196731090545654 61.0
0 0.858028769493103 62.0
219.19496235 256.418243408203 0.0
0 0.0131691694259644 1.0
0 -0.0894507169723511 2.0
0 0.00800406932830811 3.0
0 -0.0236457586288452 4.0
0 0.0159178972244263 5.0
0 -0.0838571786880493 6.0
0 -0.0475395917892456 7.0
0 0.000956416130065918 8.0
0 -0.084450364112854 9.0
0 -0.0267165899276733 10.0
0 0.0176626443862915 11.0
0 0.078849196434021 12.0
74.223042769 74.0764541625977 13.0
11.858257154 25.2087287902832 14.0
0 -0.113731026649475 15.0
0 0.0364593267440796 16.0
28.27688234 27.1114826202393 17.0
0 0.00350368022918701 18.0
0 -0.0402461290359497 19.0
0 -0.0341471433639526 20.0
0 0.0140839815139771 21.0
0 0.00131356716156006 22.0
34.957583188 34.4358139038086 23.0
0 0.121267676353455 24.0
0 0.452700257301331 25.0
0 0.10496461391449 26.0
0 0.122106909751892 27.0
0 -0.0521284341812134 28.0
0 -0.0413111448287964 29.0
0 -0.0963202714920044 30.0
0 -0.066468358039856 31.0
0 0.0102757215499878 32.0
0 -0.138039946556091 33.0
0 -0.0651272535324097 34.0
0 -0.102028727531433 35.0
0 0.0821448564529419 36.0
0 -0.0329111814498901 37.0
0 -0.0747405290603638 38.0
0 0.00179946422576904 39.0
0 -0.113280892372131 40.0
0 -0.00639235973358154 41.0
0 0.0218433141708374 42.0
290.31785766 297.494567871094 43.0
0 0.022020697593689 44.0
0 0.0732371211051941 45.0
0 0.00691044330596924 46.0
0 -0.0624583959579468 47.0
0 -0.0854378938674927 48.0
0 0.251683712005615 49.0
0.064847797163 0.0671918392181396 50.0
2.0090568338 0.847538828849792 51.0
0 -0.0569819211959839 52.0
0 -0.0476361513137817 53.0
0 0.047493577003479 54.0
0 -0.0425933599472046 55.0
0 -0.0491641759872437 56.0
0 -0.0381141901016235 57.0
0 -0.0564624071121216 58.0
0 -0.0814884901046753 59.0
0 -0.0207687616348267 60.0
0 0.058998703956604 61.0
0 0.939477205276489 62.0
288.93721796 254.136444091797 0.0
0 -0.0520960092544556 1.0
0 -0.0634149312973022 2.0
0 -0.0277122259140015 3.0
0 -0.0308519601821899 4.0
0 0.012834906578064 5.0
0 0.182004332542419 6.0
0 -0.201332688331604 7.0
0 -0.0484911203384399 8.0
0 0.0776981115341187 9.0
0 -0.0127376317977905 10.0
0 -0.0420244932174683 11.0
0 -0.00319445133209229 12.0
65.384495083 64.9619750976562 13.0
27.501641297 32.362117767334 14.0
0 -0.0425020456314087 15.0
0 -0.0278297662734985 16.0
36.122484605 36.1218223571777 17.0
0 -0.00391304492950439 18.0
0 0.025795578956604 19.0
0 0.0222867727279663 20.0
0 0.00632178783416748 21.0
0 0.00269925594329834 22.0
32.435889857 32.5070762634277 23.0
0 -0.0284143686294556 24.0
0 0.60605525970459 25.0
0 0.0842732191085815 26.0
0 -0.0606286525726318 27.0
0 -0.0437194108963013 28.0
0 -0.0213276147842407 29.0
0 -0.050737738609314 30.0
0 -0.0645653009414673 31.0
0 0.0133780241012573 32.0
0 -0.012508749961853 33.0
0 -0.0299512147903442 34.0
0 -0.0303460359573364 35.0
0 0.0674012899398804 36.0
0 -0.0405470132827759 37.0
0 -0.0215197801589966 38.0
0 0.0762118101119995 39.0
0 0.154236197471619 40.0
0 0.00336647033691406 41.0
0 0.0300534963607788 42.0
309.01312375 298.259216308594 43.0
0 -0.0177918672561646 44.0
0 0.0128273963928223 45.0
0 0.00937163829803467 46.0
0 -0.0742374658584595 47.0
0 -0.00913393497467041 48.0
0 0.237787067890167 49.0
0.08982525606 0.0632270574569702 50.0
1.604908793 0.84327495098114 51.0
0 -0.0522009134292603 52.0
0 -0.0784827470779419 53.0
0 0.0589925050735474 54.0
0 -0.116247296333313 55.0
0 -0.0528031587600708 56.0
0 0.0202592611312866 57.0
0 0.0365327596664429 58.0
0 -0.0598345994949341 59.0
0 0.0303620100021362 60.0
0 0.0159891843795776 61.0
0 0.794919490814209 62.0
290.66777788 261.416961669922 0.0
0 -0.0420633554458618 1.0
0 0.0459643602371216 2.0
0 -0.00433075428009033 3.0
0 -0.0209070444107056 4.0
0 -0.0881439447402954 5.0
0 -0.0295573472976685 6.0
0 -0.12412703037262 7.0
0 -0.0168230533599854 8.0
0 0.00757348537445068 9.0
0 -0.0777307748794556 10.0
0 0.0530421733856201 11.0
0 0.0226567983627319 12.0
78.006602912 75.7438125610352 13.0
35.903081887 24.9082698822021 14.0
0 -0.0654417276382446 15.0
0 -0.0756083726882935 16.0
25.322285054 24.7442378997803 17.0
0 -0.0284336805343628 18.0
0 0.00365126132965088 19.0
0 0.0351959466934204 20.0
0 -0.0255130529403687 21.0
0 -0.0356432199478149 22.0
33.779773957 33.1625900268555 23.0
0 0.0317171812057495 24.0
0 2.21413969993591 25.0
0 0.0909794569015503 26.0
0 0.153211951255798 27.0
0 -0.0411187410354614 28.0
0 -0.12131941318512 29.0
0 0.0367187261581421 30.0
0 -0.0169504880905151 31.0
0 -0.0674997568130493 32.0
0 -0.0466357469558716 33.0
0 -0.115129828453064 34.0
0 -0.0190514326095581 35.0
0 0.141300797462463 36.0
0 -0.00906670093536377 37.0
0 -0.0125647783279419 38.0
0 0.0545583963394165 39.0
0 0.0887941122055054 40.0
0 -0.0500799417495728 41.0
0 0.0325120687484741 42.0
305.83467021 292.958526611328 43.0
0 -0.026898980140686 44.0
0 0.231698870658875 45.0
0 -0.0290788412094116 46.0
0 0.0330313444137573 47.0
0 -0.0318642854690552 48.0
0 0.311533808708191 49.0
0.071379511149 0.0770753622055054 50.0
2.2096223 1.02010524272919 51.0
0 -0.00439369678497314 52.0
0 -0.0515881776809692 53.0
0 -0.00434267520904541 54.0
0 -0.0746680498123169 55.0
0 0.0152992010116577 56.0
0 0.00986230373382568 57.0
0 -0.0201519727706909 58.0
0 -0.0528692007064819 59.0
0 -0.0556505918502808 60.0
0 0.0102740526199341 61.0
0 0.854726076126099 62.0
272.0474143 250.873519897461 0.0
0 -0.0917412042617798 1.0
0 0.0302687883377075 2.0
0 0.039117693901062 3.0
0 -0.0735100507736206 4.0
0 -0.167597413063049 5.0
0 0.202984213829041 6.0
0 0.0130041837692261 7.0
0 -0.0162293910980225 8.0
0 0.00509583950042725 9.0
0 0.00258386135101318 10.0
0 0.014191746711731 11.0
0 -0.24878466129303 12.0
49.314139113 49.4264907836914 13.0
36.713861648 36.0550079345703 14.0
0 -0.101173996925354 15.0
0 0.00119435787200928 16.0
40.021672577 40.7148208618164 17.0
0 -0.0130809545516968 18.0
0 0.0296002626419067 19.0
0 0.00709354877471924 20.0
0 -0.0140718221664429 21.0
0 -0.0291277170181274 22.0
27.199982913 26.3608016967773 23.0
0 0.0964916944503784 24.0
29.573486731 0.41782534122467 25.0
0 -0.00405561923980713 26.0
0 0.0387066602706909 27.0
0 0.0452827215194702 28.0
0 -0.0494576692581177 29.0
0 -0.0767863988876343 30.0
0 -0.0301293134689331 31.0
0 0.00582444667816162 32.0
0 -0.0268455743789673 33.0
0 -0.0792022943496704 34.0
0 -0.0719069242477417 35.0
0 0.0500637292861938 36.0
0 -0.0433160066604614 37.0
0 0.0396624803543091 38.0
0 -0.0718016624450684 39.0
0 0.114139914512634 40.0
0 -0.0964497327804565 41.0
0 -0.043582558631897 42.0
293.39239219 296.103240966797 43.0
0 0.0124338865280151 44.0
0 -0.0226905345916748 45.0
0 -0.0205034017562866 46.0
0 0.0825973749160767 47.0
0 0.0998481512069702 48.0
0 0.296566814184189 49.0
0.061636533876 0.0687123537063599 50.0
2.0656540364 0.881713211536407 51.0
0 -0.0633798837661743 52.0
0 -0.0498672723770142 53.0
0 0.00621354579925537 54.0
0 -0.11037003993988 55.0
0 -0.0470677614212036 56.0
0 -0.0754028558731079 57.0
0 0.0738133192062378 58.0
0 -0.0437370538711548 59.0
0 -0.0603975057601929 60.0
0 0.0461882352828979 61.0
0 0.810204982757568 62.0
254.72235854 258.529937744141 0.0
0 -0.0476928949356079 1.0
0 -0.0539520978927612 2.0
0 -0.0049976110458374 3.0
0 -0.0293821096420288 4.0
0 -0.0981041193008423 5.0
0 -0.00252878665924072 6.0
0 -0.152857899665833 7.0
0 -0.0102721452713013 8.0
0 0.0567983388900757 9.0
0 -0.0284714698791504 10.0
0 -0.0366417169570923 11.0
0 0.0551217794418335 12.0
54.012941712 52.8098335266113 13.0
42.309681284 42.5383605957031 14.0
0 -0.0642067193984985 15.0
0 0.0279558897018433 16.0
34.940489622 34.4309234619141 17.0
0 -0.0157757997512817 18.0
0 -0.0183497667312622 19.0
0 -0.0241659879684448 20.0
0 -0.0303250551223755 21.0
0 -0.0230823755264282 22.0
36.147926512 34.9776878356934 23.0
0 0.0579179525375366 24.0
0 0.926344215869904 25.0
0 0.0502570867538452 26.0
0 0.103321671485901 27.0
0 0.0457714796066284 28.0
0 -0.0611616373062134 29.0
0 -0.0665000677108765 30.0
0 -0.0267096757888794 31.0
0 -0.00858032703399658 32.0
0 0.11419141292572 33.0
0 -0.0279396772384644 34.0
0 -0.0241024494171143 35.0
0 -0.112442851066589 36.0
0 0.0250781774520874 37.0
0 -0.0617107152938843 38.0
0 0.0235797166824341 39.0
0 0.0880888700485229 40.0
0 -0.0533679723739624 41.0
0 0.0109361410140991 42.0
303.40690426 296.994293212891 43.0
0 -0.0516344308853149 44.0
0 -0.0470155477523804 45.0
0 -0.0237151384353638 46.0
0 0.0236228704452515 47.0
0 0.00463807582855225 48.0
0 0.296624183654785 49.0
0.10031826894 0.0638495683670044 50.0
1.64704654 0.992839813232422 51.0
0 -0.0561343431472778 52.0
0 -0.07764732837677 53.0
0 -0.0185967683792114 54.0
0 -0.0840929746627808 55.0
0 -0.0576666593551636 56.0
0 -0.00604498386383057 57.0
0 -0.040375828742981 58.0
0 -0.0380290746688843 59.0
0 -0.0288769006729126 60.0
0 -0.0324243307113647 61.0
0 0.824732303619385 62.0
223.30168101 255.503829956055 0.0
0 -0.0551100969314575 1.0
0 0.0448139905929565 2.0
0 0.0498141050338745 3.0
0 -0.0900989770889282 4.0
0 -0.133818507194519 5.0
0 0.0334314107894897 6.0
0 -0.0785647630691528 7.0
0 0.0518090724945068 8.0
0 -0.0281139612197876 9.0
0 -0.0208495855331421 10.0
0 0.0248973965644836 11.0
0 -0.0239015817642212 12.0
75.975722422 74.3471221923828 13.0
13.235427153 26.4027271270752 14.0
0 -0.09394371509552 15.0
0 -0.00874483585357666 16.0
32.68582305 32.0461502075195 17.0
0 0.0163058042526245 18.0
0 0.0133365392684937 19.0
0 0.00589287281036377 20.0
0 0.000929713249206543 21.0
0 0.044913649559021 22.0
24.145659417 15.7728824615479 23.0
0 0.0431488752365112 24.0
0 0.584168970584869 25.0
0 0.0680726766586304 26.0
0 -0.0561736226081848 27.0
0 -0.0999716520309448 28.0
0 0.0128220319747925 29.0
0 -0.0688760280609131 30.0
0 -0.0441626310348511 31.0
0 -0.00507843494415283 32.0
0 -0.00635111331939697 33.0
0 -0.0571819543838501 34.0
0 -0.0336370468139648 35.0
0 0.0318869352340698 36.0
0 0.00633490085601807 37.0
0 0.0412353277206421 38.0
0 0.0916520357131958 39.0
0 0.069881796836853 40.0
0 -0.132689356803894 41.0
0 0.0166724920272827 42.0
309.73542332 294.544464111328 43.0
0 -0.021947979927063 44.0
0 0.102240204811096 45.0
0 -0.0470031499862671 46.0
0 0.0327903032302856 47.0
0 0.0335434675216675 48.0
0 0.261007487773895 49.0
0.082626815004 0.084621787071228 50.0
2.2206849829 1.02447390556335 51.0
0 -0.123613476753235 52.0
0 -0.0120078325271606 53.0
0 0.0124367475509644 54.0
0 -0.0557006597518921 55.0
0 0.131746172904968 56.0
0 0.0126413106918335 57.0
0 0.0271323919296265 58.0
0 -0.0471869707107544 59.0
0 -0.0280238389968872 60.0
0 0.0702234506607056 61.0
0 0.944591999053955 62.0
259.97353359 257.226501464844 0.0
0 -0.0476347208023071 1.0
0 -0.0072866678237915 2.0
0 0.00119006633758545 3.0
0 0.0112358331680298 4.0
0 0.00350463390350342 5.0
0 0.0704721212387085 6.0
0 -0.160974144935608 7.0
0 0.003814697265625 8.0
0 0.0457037687301636 9.0
0 -0.0481477975845337 10.0
0 -0.013344407081604 11.0
0 -0.102950692176819 12.0
47.79353582 48.2311935424805 13.0
34.430654908 29.2220001220703 14.0
0 -0.000765442848205566 15.0
0 -0.0921040773391724 16.0
38.71983441 38.6114959716797 17.0
0 0.119778990745544 18.0
0 0.0580695867538452 19.0
0 0.0908154249191284 20.0
0 0.0294340848922729 21.0
0 -0.0462919473648071 22.0
28.739520123 27.4578247070312 23.0
0 0.00858581066131592 24.0
0 0.441978275775909 25.0
0 0.0597304105758667 26.0
0 0.0604910850524902 27.0
0 0.21691882610321 28.0
0 -0.0552104711532593 29.0
0 -0.0413054227828979 30.0
0 -0.0815225839614868 31.0
0 0.0106347799301147 32.0
0 0.145273089408875 33.0
0 -0.0468565225601196 34.0
0 -0.0448867082595825 35.0
0 0.032448410987854 36.0
0 0.023271918296814 37.0
0 0.117491126060486 38.0
0 -0.135677456855774 39.0
0 -0.00512421131134033 40.0
0 0.100276589393616 41.0
0 -0.0471693277359009 42.0
287.49801283 294.975463867188 43.0
0 -0.0129858255386353 44.0
0 0.0460702776908875 45.0
0 -0.0165952444076538 46.0
0 -0.0196490287780762 47.0
0 0.0192841291427612 48.0
0 0.344403833150864 49.0
0.10598188183 0.0584810972213745 50.0
2.4356406862 1.01781296730042 51.0
0 -0.109331250190735 52.0
0 -0.0362597703933716 53.0
0 0.0186485052108765 54.0
0 -0.0326992273330688 55.0
0 0.0458601713180542 56.0
0 0.000192523002624512 57.0
0 -0.0665425062179565 58.0
0 -0.023274302482605 59.0
0 0.170023798942566 60.0
0 -0.028098464012146 61.0
0 0.871131062507629 62.0
248.90488222 250.721862792969 0.0
0 0.0227483510971069 1.0
0 -0.137332081794739 2.0
0 -0.0804463624954224 3.0
0 -0.024462103843689 4.0
0 0.0265375375747681 5.0
0 0.0225852727890015 6.0
0 -0.014013409614563 7.0
0 -0.067956805229187 8.0
0 0.0315631628036499 9.0
0 -0.00236225128173828 10.0
0 -0.0255093574523926 11.0
0 -0.0900989770889282 12.0
69.551129863 68.5995788574219 13.0
32.540830326 29.6055374145508 14.0
0 -0.0373421907424927 15.0
0 0.0158919095993042 16.0
32.877838025 32.6190185546875 17.0
0 -0.00412857532501221 18.0
0 -0.0298889875411987 19.0
0 0.0180155038833618 20.0
0 -0.00996840000152588 21.0
0 -0.0252114534378052 22.0
35.404276492 34.7125778198242 23.0
0 0.087882399559021 24.0
0 0.467693507671356 25.0
0 0.102467894554138 26.0
0 0.108483195304871 27.0
0 -0.0628906488418579 28.0
0 -0.0513385534286499 29.0
0 -0.0735071897506714 30.0
0 -0.0440067052841187 31.0
0 -0.00832188129425049 32.0
0 0.0386372804641724 33.0
0 -0.0164452791213989 34.0
0 -0.0338765382766724 35.0
0 0.124412894248962 36.0
0 -0.0685211420059204 37.0
0 -0.00224077701568604 38.0
0 0.0535093545913696 39.0
0 0.0364574193954468 40.0
0 -0.0692899227142334 41.0
0 0.0417255163192749 42.0
287.09257158 296.350219726562 43.0
0 -0.0544410943984985 44.0
0 0.00379776954650879 45.0
0 0.00256478786468506 46.0
0 -0.0467933416366577 47.0
0 0.0128639936447144 48.0
0 0.283168077468872 49.0
0.089455569335 0.0589123964309692 50.0
1.5311249564 0.982780754566193 51.0
0 -0.0510634183883667 52.0
0 -0.0623141527175903 53.0
0 0.0504215955734253 54.0
0 -0.0498465299606323 55.0
0 -0.069522500038147 56.0
0 -9.21487808227539e-05 57.0
0 0.00977838039398193 58.0
0 -0.0595513582229614 59.0
0 -0.00928270816802979 60.0
0 0.0401836633682251 61.0
0 0.940266072750092 62.0
245.05914242 255.155670166016 0.0
0 -0.0885921716690063 1.0
0 0.0178662538528442 2.0
0 -0.0172683000564575 3.0
0 -0.0779517889022827 4.0
0 -0.0141431093215942 5.0
0 -0.0628095865249634 6.0
0 -0.0685180425643921 7.0
0 -0.0303391814231873 8.0
0 0.0683368444442749 9.0
0 -0.0204375982284546 10.0
0 0.0214064121246338 11.0
0 0.0801090002059937 12.0
57.872471477 58.1271057128906 13.0
44.265076846 40.1543083190918 14.0
0 -0.0543428659439087 15.0
0 -0.0363935232162476 16.0
39.172866734 39.0230903625488 17.0
0 0.0107618570327759 18.0
0 -0.00672757625579834 19.0
0 0.0221091508865356 20.0
0 -0.00510990619659424 21.0
0 -0.0694786310195923 22.0
40.812786084 41.3854446411133 23.0
0 -0.102432370185852 24.0
0 0.637636542320251 25.0
0 0.0511444807052612 26.0
0 0.0615099668502808 27.0
0 0.0208386182785034 28.0
0 -0.0846415758132935 29.0
0 0.00642359256744385 30.0
0 -0.049984335899353 31.0
0 -0.00611698627471924 32.0
0 0.0231469869613647 33.0
0 -0.0310295820236206 34.0
0 -0.00428569316864014 35.0
0 0.148633122444153 36.0
0 -0.0523508787155151 37.0
0 -0.00781929492950439 38.0
0 0.0322586297988892 39.0
0 0.114490389823914 40.0
0 -0.0291560888290405 41.0
0 -0.020485520362854 42.0
291.21246737 295.569915771484 43.0
0 0.0516308546066284 44.0
0 -0.0173921585083008 45.0
0 0.00596511363983154 46.0
0 -0.0182944536209106 47.0
0 0.00415551662445068 48.0
0 0.277324318885803 49.0
0.096199218641 0.0491822957992554 50.0
2.1336888402 0.972115755081177 51.0
0 -0.0955907106399536 52.0
0 -0.0422741174697876 53.0
0 -0.0111340284347534 54.0
0 -0.0656870603561401 55.0
0 -0.0452419519424438 56.0
0 -0.00314176082611084 57.0
0 0.0147665739059448 58.0
0 -0.0229352712631226 59.0
0 -0.0557631254196167 60.0
0 -0.00677096843719482 61.0
0 0.843021631240845 62.0
294.86048316 253.457885742188 0.0
0 -0.0171161890029907 1.0
0 -0.0961698293685913 2.0
0 0.010903000831604 3.0
0 -0.0834685564041138 4.0
0 0.00604140758514404 5.0
0 -0.17888605594635 6.0
0 -0.0297509431838989 7.0
0 -0.0540132522583008 8.0
0 0.045763373374939 9.0
0 -0.00414597988128662 10.0
0 0.0194916725158691 11.0
0 -0.0693832635879517 12.0
58.903531991 58.0941925048828 13.0
50.304253099 39.6320457458496 14.0
0 -0.029086709022522 15.0
0 -0.0345560312271118 16.0
34.69172164 34.726676940918 17.0
0 -0.00888264179229736 18.0
0 -0.00288140773773193 19.0
0 -0.04035484790802 20.0
0 -0.0191243886947632 21.0
0 0.016417384147644 22.0
29.080332385 28.3716220855713 23.0
0 -0.0512872934341431 24.0
0 0.60964435338974 25.0
0 0.0543321371078491 26.0
0 -0.0874059200286865 27.0
0 -0.0622559785842896 28.0
0 0.000101208686828613 29.0
0 -0.054203987121582 30.0
0 -0.0191015005111694 31.0
0 0.0186303853988647 32.0
0 0.0922020673751831 33.0
0 -0.0387264490127563 34.0
0 -0.0312007665634155 35.0
0 0.0466746091842651 36.0
0 0.0797313451766968 37.0
0 0.023661732673645 38.0
0 0.110327839851379 39.0
0 0.0280865430831909 40.0
0 -0.0260823965072632 41.0
0 0.00727593898773193 42.0
297.29201504 295.484527587891 43.0
0 -0.0208898782730103 44.0
0 0.0132096409797668 45.0
0 -0.0183242559432983 46.0
0 -0.0680156946182251 47.0
0 0.00122439861297607 48.0
0 0.290274441242218 49.0
0.074917853145 0.0853540897369385 50.0
1.9725949338 0.91904616355896 51.0
0 -0.0451501607894897 52.0
0 -0.0140231847763062 53.0
0 -0.0226398706436157 54.0
0 -0.0516065359115601 55.0
0 -0.111397624015808 56.0
0 -0.00793564319610596 57.0
0 -0.0502511262893677 58.0
0 -0.0264784097671509 59.0
0 -0.0155993700027466 60.0
0 0.027732253074646 61.0
0 0.826309442520142 62.0
269.13282936 256.691955566406 0.0
0 -0.0674148797988892 1.0
0 -0.0247651338577271 2.0
0 0.0643526315689087 3.0
0 -0.0562885999679565 4.0
0 -0.0642956495285034 5.0
0 0.117263197898865 6.0
0 0.0283030271530151 7.0
0 -0.0522092580795288 8.0
0 0.0281175374984741 9.0
0 0.0193041563034058 10.0
0 -0.0321033000946045 11.0
0 -0.197483658790588 12.0
49.987269414 48.8004951477051 13.0
34.364241733 33.1810302734375 14.0
0 -0.019539475440979 15.0
0 -0.0544489622116089 16.0
27.694204577 28.3101348876953 17.0
0 0.0541752576828003 18.0
0 0.0197802782058716 19.0
0 0.110481858253479 20.0
0 0.0813552141189575 21.0
0 -0.0840855836868286 22.0
38.291501287 38.4228935241699 23.0
0 0.0192908048629761 24.0
0 4.31175518035889 25.0
0 0.0911105871200562 26.0
0 0.18070113658905 27.0
0 0.202583193778992 28.0
0 -0.104103684425354 29.0
0 -0.0758037567138672 30.0
0 -0.0125066041946411 31.0
0 0.0201281309127808 32.0
0 -0.0976637601852417 33.0
0 -0.155699849128723 34.0
0 -0.0197142362594604 35.0
0 0.149727940559387 36.0
0 -0.0645681619644165 37.0
0 -0.0657309293746948 38.0
0 0.0772982835769653 39.0
0 0.093227744102478 40.0
0 -0.0614269971847534 41.0
0 -0.059491753578186 42.0
289.09630278 294.340515136719 43.0
0 -0.0281568765640259 44.0
0 -0.0205055475234985 45.0
0 0.0179237127304077 46.0
0 -0.0941191911697388 47.0
0 -0.0618373155593872 48.0
0 0.266519546508789 49.0
0.06533799866 0.0752416849136353 50.0
1.4484521583 1.01973927021027 51.0
0 -0.0586644411087036 52.0
0 -0.00844347476959229 53.0
0 -0.0575166940689087 54.0
0 -0.00343811511993408 55.0
0 -0.0214458703994751 56.0
0 0.0134705305099487 57.0
0 -0.0533138513565063 58.0
0 -0.0231088399887085 59.0
0 0.0265105962753296 60.0
0 -0.0332084894180298 61.0
0 0.864509582519531 62.0
277.47908146 259.782318115234 0.0
0 -0.098788857460022 1.0
0 0.0303674936294556 2.0
0 0.158628582954407 3.0
0 -0.0624846220016479 4.0
0 -0.133460164070129 5.0
0 -0.000239014625549316 6.0
0 -0.149166703224182 7.0
0 -0.0108891725540161 8.0
0 0.00871789455413818 9.0
0 -0.121994853019714 10.0
0 -0.00473964214324951 11.0
0 -0.0368543863296509 12.0
56.089812067 55.3436050415039 13.0
48.468637005 42.8910713195801 14.0
0 -0.138699173927307 15.0
0 0.070189356803894 16.0
33.805103268 33.7433853149414 17.0
0 -0.0238054990768433 18.0
0 0.0397454500198364 19.0
0 -0.0455054044723511 20.0
0 0.00852024555206299 21.0
0 0.00903022289276123 22.0
33.425997429 32.8342399597168 23.0
0 -0.0909768342971802 24.0
0 0.837051749229431 25.0
0 -0.0101059675216675 26.0
0 0.046906590461731 27.0
0 0.010517954826355 28.0
0 -0.012468695640564 29.0
0 -0.124464154243469 30.0
0 -0.0131655931472778 31.0
0 0.0243223905563354 32.0
0 0.00376021862030029 33.0
0 -0.0180741548538208 34.0
0 0.0930029153823853 35.0
0 -0.0565239191055298 36.0
0 -0.05451500415802 37.0
0 -0.0630897283554077 38.0
0 0.0292235612869263 39.0
0 0.114850401878357 40.0
0 -0.0331988334655762 41.0
0 0.0135473012924194 42.0
293.42126398 294.908721923828 43.0
0 0.0359691381454468 44.0
0 0.0602301359176636 45.0
0 0.00304234027862549 46.0
0 0.0112868547439575 47.0
0 0.0138353109359741 48.0
0 0.293859660625458 49.0
0.078155575475 0.0645204782485962 50.0
2.1598175894 0.9916752576828 51.0
0 -0.0640047788619995 52.0
0 -0.0497323274612427 53.0
0 0.0841459035873413 54.0
0 -0.119273066520691 55.0
0 -0.0511199235916138 56.0
0 0.0209664106369019 57.0
0 0.00627577304840088 58.0
0 -0.0171612501144409 59.0
0 0.0639852285385132 60.0
0 0.0418928861618042 61.0
0 0.769554257392883 62.0
230.44408218 256.150970458984 0.0
0 -0.013644814491272 1.0
0 -0.00871241092681885 2.0
0 -0.0303839445114136 3.0
0 -0.0755141973495483 4.0
0 0.0524140596389771 5.0
0 0.0393842458724976 6.0
0 -0.084863543510437 7.0
0 -0.0249698162078857 8.0
0 -0.114764809608459 9.0
0 -0.0483709573745728 10.0
0 0.00686788558959961 11.0
0 0.0163015127182007 12.0
65.506754529 65.5107116699219 13.0
20.182684266 32.9481391906738 14.0
0 -0.0583606958389282 15.0
0 -0.0323725938796997 16.0
34.838872478 34.2277755737305 17.0
0 -0.00519406795501709 18.0
0 0.0274034738540649 19.0
0 -0.00837337970733643 20.0
0 -0.0284727811813354 21.0
0 -0.0633751153945923 22.0
40.269232465 39.8936080932617 23.0
0 0.00163495540618896 24.0
0 0.487869203090668 25.0
0 0.0452736616134644 26.0
0 -0.0889928340911865 27.0
0 -0.0953546762466431 28.0
0 -0.0442110300064087 29.0
0 -0.0889480113983154 30.0
0 0.00328314304351807 31.0
0 -0.0298844575881958 32.0
0 -0.0172053575515747 33.0
0 0.00307786464691162 34.0
0 -0.00433123111724854 35.0
0 0.0244675874710083 36.0
0 -0.04664146900177 37.0
0 -0.040739893913269 38.0
0 0.0655752420425415 39.0
0 0.0731910467147827 40.0
0 -0.0497119426727295 41.0
0 -0.00904333591461182 42.0
297.63265979 295.884582519531 43.0
0 0.0119727849960327 44.0
0 -0.0323745012283325 45.0
0 -0.0197986364364624 46.0
0 -0.0786508321762085 47.0
0 -0.0128203630447388 48.0
0 0.302608281373978 49.0
0.087922028171 0.0652033090591431 50.0
2.1547050883 0.997198939323425 51.0
0 -0.0480128526687622 52.0
0 -0.0222800970077515 53.0
0 0.0358363389968872 54.0
0 -0.040650486946106 55.0
0 0.00876367092132568 56.0
0 -0.0428668260574341 57.0
0 -0.0717612504959106 58.0
0 -0.0809639692306519 59.0
0 -0.0258733034133911 60.0
0 0.00579750537872314 61.0
0 0.851482391357422 62.0
242.38913139 253.857543945312 0.0
0 -0.0176092386245728 1.0
0 -0.0633012056350708 2.0
0 0.00891697406768799 3.0
0 -0.0463813543319702 4.0
0 0.0578733682632446 5.0
0 0.00642812252044678 6.0
0 -0.0857146978378296 7.0
0 0.0289751291275024 8.0
0 -0.107436776161194 9.0
0 -0.0117117166519165 10.0
0 -0.0430722236633301 11.0
0 0.00732457637786865 12.0
51.610605871 51.3126182556152 13.0
38.270892513 38.5447082519531 14.0
0 -0.0171099901199341 15.0
0 -0.034673810005188 16.0
34.837471483 34.2671585083008 17.0
0 -0.00840795040130615 18.0
0 -0.0221825838088989 19.0
0 -0.0270863771438599 20.0
0 0.0203896760940552 21.0
0 0.0266412496566772 22.0
30.094018495 29.9475231170654 23.0
0 -0.0598751306533813 24.0
0 0.53372722864151 25.0
0 -0.00537073612213135 26.0
0 -0.0182558298110962 27.0
0 -0.0538002252578735 28.0
0 0.0130635499954224 29.0
0 -0.0423258543014526 30.0
0 -0.0456639528274536 31.0
0 -0.0587583780288696 32.0
0 0.00359618663787842 33.0
0 -0.0328541994094849 34.0
0 0.0103529691696167 35.0
0 0.0449138879776001 36.0
0 -0.0329548120498657 37.0
0 0.0581566095352173 38.0
0 0.119047284126282 39.0
0 0.0419191122055054 40.0
0 -0.0250886678695679 41.0
0 -0.0151973962783813 42.0
300.09787078 295.995483398438 43.0
0 0.00678765773773193 44.0
0 -0.00608175992965698 45.0
0 -0.0177313089370728 46.0
0 -0.0230211019515991 47.0
0 -0.0957051515579224 48.0
0 0.272203147411346 49.0
0.063910000148 0.0805121660232544 50.0
1.456039889 0.980679154396057 51.0
0 -0.0689679384231567 52.0
0 -0.0719529390335083 53.0
0 0.0185891389846802 54.0
0 -0.13101589679718 55.0
0 -0.0240994691848755 56.0
0 -0.0191241502761841 57.0
0 -0.082196831703186 58.0
0 0.0278040170669556 59.0
0 -0.0907925367355347 60.0
0 0.00565779209136963 61.0
0 0.815702080726624 62.0
275.22040505 259.515594482422 0.0
0 0.0487269163131714 1.0
0 -0.0237752199172974 2.0
0 0.0134402513504028 3.0
0 -0.072999119758606 4.0
0 -0.0714246034622192 5.0
0 0.0344704389572144 6.0
0 -0.000956177711486816 7.0
0 -0.0790492296218872 8.0
0 0.0415138006210327 9.0
0 -0.0124338865280151 10.0
0 -0.00658273696899414 11.0
0 0.0921529531478882 12.0
66.659907129 66.1885986328125 13.0
35.009623797 31.031608581543 14.0
0 0.0296133756637573 15.0
0 -0.0349911451339722 16.0
30.613093158 30.0523529052734 17.0
0 0.0572983026504517 18.0
0 -0.016480565071106 19.0
0 -0.0174773931503296 20.0
0 -0.0320414304733276 21.0
0 -0.0538583993911743 22.0
30.724902912 30.1810207366943 23.0
0 -0.0588794946670532 24.0
0 0.626664996147156 25.0
0 0.108361124992371 26.0
0 0.0786913633346558 27.0
0 -0.0313163995742798 28.0
0 -0.0511151552200317 29.0
0 -0.0558948516845703 30.0
0 0.00436341762542725 31.0
0 -0.0192128419876099 32.0
0 0.0784069299697876 33.0
0 -0.0502234697341919 34.0
0 -0.0693820714950562 35.0
0 0.0739861726760864 36.0
0 -0.0979577302932739 37.0
0 -0.0186353921890259 38.0
0 0.0729626417160034 39.0
0 0.0839337110519409 40.0
0 -0.0231417417526245 41.0
0 -0.0415645837783813 42.0
302.42822443 298.822021484375 43.0
0 -0.0309635400772095 44.0
0 0.0724143981933594 45.0
0 -0.0415650606155396 46.0
0 -0.0792554616928101 47.0
0 0.00206458568572998 48.0
0 0.278564393520355 49.0
0.072843792536 0.0655163526535034 50.0
1.6435388387 0.989526629447937 51.0
0 -0.0925625562667847 52.0
0 -0.104877591133118 53.0
0 0.0585278272628784 54.0
0 -0.0526968240737915 55.0
0 -0.0464650392532349 56.0
0 -0.0129877328872681 57.0
0 -0.0644699335098267 58.0
0 -0.0369919538497925 59.0
0 -0.0805889368057251 60.0
0 0.0206323862075806 61.0
0 0.808392405509949 62.0
253.53875146 258.737854003906 0.0
0 -0.191171288490295 1.0
0 0.0324064493179321 2.0
0 -0.061470627784729 3.0
0 0.0100380182266235 4.0
0 -0.08002769947052 5.0
0 -0.104679703712463 6.0
0 -0.186756730079651 7.0
0 0.000415444374084473 8.0
0 0.00205552577972412 9.0
0 -0.111078143119812 10.0
0 -0.0111079216003418 11.0
0 -0.0131222009658813 12.0
57.717651483 55.3653831481934 13.0
33.629861082 36.5398254394531 14.0
0 -0.0340415239334106 15.0
0 -0.0165940523147583 16.0
41.111115813 40.1493453979492 17.0
0 0.11709201335907 18.0
0 0.0270596742630005 19.0
0 0.00190079212188721 20.0
0 0.0643562078475952 21.0
0 0.0164779424667358 22.0
28.232476034 28.159517288208 23.0
0 0.0167444944381714 24.0
0 0.54094523191452 25.0
0 0.00588977336883545 26.0
0 0.0279121994972229 27.0
0 0.148785948753357 28.0
0 -0.0925449132919312 29.0
0 -0.0688310861587524 30.0
0 -0.020445704460144 31.0
0 -0.0105534791946411 32.0
0 -0.0584174394607544 33.0
0 -0.00658047199249268 34.0
0 -0.0683425664901733 35.0
0 0.220053553581238 36.0
0 -0.0578728914260864 37.0
0 -0.0214792490005493 38.0
0 0.115851044654846 39.0
0 -0.0406957864761353 40.0
0 -0.0158330202102661 41.0
0 0.108006834983826 42.0
299.73097047 295.539459228516 43.0
0 -0.050758957862854 44.0
0 0.0936809778213501 45.0
0 -0.0569239854812622 46.0
0 -0.0248295068740845 47.0
0 -0.0123153924942017 48.0
3.1362500336 0.277376055717468 49.0
0.065019719745 0.0767699480056763 50.0
2.4569510741 0.94830447435379 51.0
0 -0.0513273477554321 52.0
0 -0.0275954008102417 53.0
0 0.0167618989944458 54.0
0 -0.0657223463058472 55.0
0 -0.00649487972259521 56.0
0 -0.0317646265029907 57.0
0 -0.0613671541213989 58.0
0 -0.0402640104293823 59.0
0 0.0621844530105591 60.0
0 0.0601335763931274 61.0
0 0.889379501342773 62.0
232.65166051 252.728851318359 0.0
0 -0.0778318643569946 1.0
0 -0.0513888597488403 2.0
0 0.00961959362030029 3.0
0 -0.0609527826309204 4.0
0 -0.0657564401626587 5.0
0 -0.000685334205627441 6.0
0 -0.0421198606491089 7.0
0 0.0211049318313599 8.0
0 -0.0173832178115845 9.0
0 0.00625598430633545 10.0
0 -0.0465880632400513 11.0
0 0.11352002620697 12.0
60.810885982 59.8569869995117 13.0
31.929983204 36.2892913818359 14.0
0 -0.0746356248855591 15.0
0 -0.0169750452041626 16.0
36.967788076 36.1897125244141 17.0
0 -0.00885593891143799 18.0
0 0.0233279466629028 19.0
0 0.00360596179962158 20.0
0 -0.0253332853317261 21.0
0 0.000645995140075684 22.0
35.066385747 34.5346488952637 23.0
0 -0.0476623773574829 24.0
0 0.631862044334412 25.0
0 0.0395210981369019 26.0
0 -0.0641442537307739 27.0
0 -0.0968776941299438 28.0
0 -0.0465127229690552 29.0
0 -0.0927716493606567 30.0
0 -0.0689119100570679 31.0
0 -0.0989724397659302 32.0
0 0.0486921072006226 33.0
0 -0.0323351621627808 34.0
0 -0.0508016347885132 35.0
0 -0.104956030845642 36.0
0 -0.0199216604232788 37.0
0 0.0257512331008911 38.0
0 0.077630877494812 39.0
0 0.168618083000183 40.0
0 -0.0710780620574951 41.0
0 0.0210622549057007 42.0
288.5187033 295.380249023438 43.0
0 0.00307881832122803 44.0
0 0.0426840782165527 45.0
0 -0.00439774990081787 46.0
0 -0.0232785940170288 47.0
0 0.0242475271224976 48.0
2.3133490095 0.267039954662323 49.0
0.081148030894 0.0761140584945679 50.0
1.5692991252 0.965679466724396 51.0
0 -0.0735279321670532 52.0
0 -0.0378504991531372 53.0
0 -0.0282000303268433 54.0
0 -0.0725682973861694 55.0
0 -0.020742654800415 56.0
0 -0.0289579629898071 57.0
0 -0.0552483797073364 58.0
0 -0.0662602186203003 59.0
0 -0.0188847780227661 60.0
0 -0.0308464765548706 61.0
0 0.859521627426147 62.0
258.15531298 255.12434387207 0.0
0 -0.0505114793777466 1.0
0 -0.0112937688827515 2.0
0 -0.00232827663421631 3.0
0 -0.0653976202011108 4.0
0 -0.0131047964096069 5.0
0 -0.0680848360061646 6.0
0 0.00425899028778076 7.0
0 -0.0579351186752319 8.0
0 0.044643759727478 9.0
0 0.0239430665969849 10.0
0 0.0157321691513062 11.0
0 -0.187717080116272 12.0
59.823728168 59.6629753112793 13.0
36.688477536 35.6304893493652 14.0
0 -0.0901237726211548 15.0
0 -0.034870982170105 16.0
30.383769037 30.5524616241455 17.0
0 0.023335337638855 18.0
0 0.0110424757003784 19.0
0 0.00727522373199463 20.0
0 -0.00889813899993896 21.0
0 -0.0294872522354126 22.0
28.557715324 28.1460285186768 23.0
0 -0.0508652925491333 24.0
0 0.601044774055481 25.0
0 0.0425552129745483 26.0
0 -0.00441426038742065 27.0
0 -0.0209881067276001 28.0
0 -0.0375672578811646 29.0
0 -0.0440975427627563 30.0
0 -0.0159317255020142 31.0
0 0.0131868124008179 32.0
0 0.0531090497970581 33.0
0 -0.0332053899765015 34.0
0 0.0489448308944702 35.0
0 -0.0248979330062866 36.0
0 -0.0477501153945923 37.0
0 -0.00613033771514893 38.0
0 0.0596286058425903 39.0
0 -0.0258404016494751 40.0
0 -0.0229219198226929 41.0
0 -0.000224709510803223 42.0
295.10220696 297.730285644531 43.0
0 0.00159204006195068 44.0
0 0.0850556492805481 45.0
0 0.0156458616256714 46.0
0 -0.0649993419647217 47.0
0 0.00625932216644287 48.0
2.7510986584 0.261697947978973 49.0
0.073520814064 0.0822733640670776 50.0
1.5485565509 0.946314334869385 51.0
0 -0.063630223274231 52.0
0 -0.00454485416412354 53.0
0 -0.00442206859588623 54.0
0 -0.0351089239120483 55.0
0 0.0281184911727905 56.0
0 -0.0351039171218872 57.0
0 -0.0441693067550659 58.0
0 -0.0404552221298218 59.0
0 0.0568016767501831 60.0
0 0.0264350175857544 61.0
0 0.819952845573425 62.0
272.98899234 257.622497558594 0.0
0 -0.0187727212905884 1.0
0 -0.0733319520950317 2.0
0 -0.00292813777923584 3.0
0 -0.0246568918228149 4.0
0 -0.10131847858429 5.0
0 0.0483525991439819 6.0
0 -0.151502728462219 7.0
0 0.0336190462112427 8.0
0 -0.0343166589736938 9.0
0 -0.0174847841262817 10.0
0 -0.0205111503601074 11.0
0 -0.0525070428848267 12.0
58.416888787 56.9810676574707 13.0
28.1961665 37.8413505554199 14.0
0 0.00987279415130615 15.0
0 0.0117522478103638 16.0
33.943554718 34.1365089416504 17.0
0 0.0202978849411011 18.0
0 -0.0403474569320679 19.0
0 0.0116580724716187 20.0
0 -0.0232688188552856 21.0
0 -0.0351859331130981 22.0
33.742980433 32.999095916748 23.0
0 0.0604270696640015 24.0
0 0.465091586112976 25.0
0 0.0799635648727417 26.0
0 0.107796311378479 27.0
0 -0.00687611103057861 28.0
0 -0.0268830060958862 29.0
0 0.204822897911072 30.0
0 -0.0154188871383667 31.0
0 -0.052984356880188 32.0
0 0.0913048982620239 33.0
0 -0.067170262336731 34.0
0 -0.102765798568726 35.0
0 -0.0689419507980347 36.0
0 -0.0414479970932007 37.0
0 0.0194422006607056 38.0
0 0.0659199953079224 39.0
0 0.0212110280990601 40.0
0 -0.0186516046524048 41.0
0 -0.0519453287124634 42.0
306.43720587 297.301513671875 43.0
0 0.0422605276107788 44.0
0 0.0660898685455322 45.0
0 -0.00498330593109131 46.0
0 -0.0712326765060425 47.0
0 0.0380948781967163 48.0
0 0.269586145877838 49.0
0.069289000903 0.0743244886398315 50.0
1.7158084527 0.899554550647736 51.0
0 -0.0597578287124634 52.0
0 -0.082922101020813 53.0
0 0.0444718599319458 54.0
0 -0.0429521799087524 55.0
0 -0.0254831314086914 56.0
0 -0.0540758371353149 57.0
0 -0.0632692575454712 58.0
0 -0.0460835695266724 59.0
0 -0.0184715986251831 60.0
0 -0.0156537294387817 61.0
0 0.845078110694885 62.0
253.50736574 251.295181274414 0.0
0 -0.14003574848175 1.0
0 -0.0881143808364868 2.0
0 0.0387247800827026 3.0
0 -0.0627862215042114 4.0
0 -0.142354130744934 5.0
0 -0.234925866127014 6.0
0 -0.11478054523468 7.0
0 0.0325648784637451 8.0
0 0.0344547033309937 9.0
0 0.0124801397323608 10.0
0 -0.0878759622573853 11.0
0 0.0919135808944702 12.0
47.836731181 47.2991561889648 13.0
42.989432225 42.205265045166 14.0
0 -0.0404254198074341 15.0
0 -0.0146447420120239 16.0
36.004107187 36.1432075500488 17.0
0 0.0295425653457642 18.0
0 0.0501917600631714 19.0
0 0.00464403629302979 20.0
0 -0.0477570295333862 21.0
0 -0.0276147127151489 22.0
36.468433913 35.044132232666 23.0
0 -0.0320960283279419 24.0
0 0.574425876140594 25.0
0 0.00637280941009521 26.0
0 0.0435837507247925 27.0
0 0.194530844688416 28.0
0 -0.0734854936599731 29.0
0 0.214513659477234 30.0
0 -0.0517910718917847 31.0
0 -0.010022759437561 32.0
0 0.0370014905929565 33.0
0 -0.0233446359634399 34.0
0 0.0553290247917175 35.0
0 -0.116548180580139 36.0
0 0.00413143634796143 37.0
0 0.00343930721282959 38.0
0 0.0705305337905884 39.0
0 -0.0443283319473267 40.0
0 -0.00899946689605713 41.0
0 -0.037601113319397 42.0
303.94303198 292.978698730469 43.0
0 0.0379890203475952 44.0
0 -0.0903542041778564 45.0
0 -0.0392416715621948 46.0
0 -0.0348445177078247 47.0
0 -0.0565568208694458 48.0
0 0.297258734703064 49.0
0.090546972208 0.226464867591858 50.0
1.7581438165 0.933338165283203 51.0
0 -0.0892411470413208 52.0
0 -0.103282570838928 53.0
0 0.0664850473403931 54.0
0 -0.102468848228455 55.0
0 -0.0583000183105469 56.0
0 -0.0565439462661743 57.0
0 -0.0730501413345337 58.0
0 -0.0916496515274048 59.0
0 -0.0618025064468384 60.0
0 0.0540875196456909 61.0
0 0.798776268959045 62.0
213.80927252 255.348388671875 0.0
0 -0.0122843980789185 1.0
0 -0.0586932897567749 2.0
0 0.0260728597640991 3.0
0 -0.0655347108840942 4.0
0 -0.000438332557678223 5.0
0 0.0558942556381226 6.0
0 -0.0923188924789429 7.0
0 -0.0799720883369446 8.0
0 0.0234464406967163 9.0
0 -0.14591109752655 10.0
0 -0.0376154184341431 11.0
0 -0.0177551507949829 12.0
64.29922089 62.7770195007324 13.0
43.287292406 34.3832168579102 14.0
0 0.065571665763855 15.0
0 0.0110992193222046 16.0
39.032526494 39.0039482116699 17.0
0 0.0441850423812866 18.0
0 0.0285888910293579 19.0
0 0.0294946432113647 20.0
0 -0.0475984811782837 21.0
0 -0.0225120782852173 22.0
36.738975356 36.0676078796387 23.0
0 0.0672796964645386 24.0
0 0.411535680294037 25.0
0 0.0068281888961792 26.0
0 0.0298651456832886 27.0
0 -0.0315428972244263 28.0
0 -0.0202993154525757 29.0
0 -0.0404179096221924 30.0
0 -0.0587502717971802 31.0
0 0.0171560049057007 32.0
0 0.0226973295211792 33.0
0 -0.019731879234314 34.0
0 0.05014967918396 35.0
0 -0.0126835107803345 36.0
0 0.0374377965927124 37.0
0 -0.00616157054901123 38.0
0 0.0867108106613159 39.0
0 -0.00359165668487549 40.0
0 -0.0697535276412964 41.0
0 0.0138896703720093 42.0
294.4924155 296.399383544922 43.0
0 0.000409960746765137 44.0
0 -0.0717568397521973 45.0
0 0.00279843807220459 46.0
0 -0.0786298513412476 47.0
0 0.020354151725769 48.0
0 0.303343832492828 49.0
0.07196863207 0.0587092638015747 50.0
1.7036641729 0.959220409393311 51.0
0 -0.0765811204910278 52.0
0 -0.0419508218765259 53.0
0 -0.0179704427719116 54.0
0 -0.0412603616714478 55.0
0 -0.0662637948989868 56.0
0 0.0146135091781616 57.0
0 0.00251114368438721 58.0
0 -0.0769201517105103 59.0
0 0.0246173143386841 60.0
0 -0.0497735738754272 61.0
0 0.839101076126099 62.0
233.39653219 253.960357666016 0.0
0 0.0443652868270874 1.0
0 -0.0201677083969116 2.0
0 0.0258702039718628 3.0
0 0.0375539064407349 4.0
0 0.106909871101379 5.0
0 0.130245566368103 6.0
0 -0.00711894035339355 7.0
0 -0.0428767204284668 8.0
0 0.0537732839584351 9.0
0 -0.0462251901626587 10.0
0 -0.0259691476821899 11.0
0 -0.084699273109436 12.0
73.929151131 71.4306259155273 13.0
34.058096488 24.5231285095215 14.0
0 -0.0573197603225708 15.0
0 -0.00186598300933838 16.0
25.715121919 24.9187889099121 17.0
0 -0.0133770704269409 18.0
0 0.0141617059707642 19.0
0 -0.0436638593673706 20.0
0 -0.0475448369979858 21.0
0 -0.0119353532791138 22.0
26.766139073 26.2804412841797 23.0
0 -0.0508633852005005 24.0
0 0.504797756671906 25.0
0 -0.0321179628372192 26.0
0 0.0167859792709351 27.0
0 -0.0115805864334106 28.0
0 -0.00694191455841064 29.0
0 -0.104811072349548 30.0
0 -0.025246262550354 31.0
0 -0.0885955095291138 32.0
0 -0.0461026430130005 33.0
0 0.0206793546676636 34.0
0 -0.114313960075378 35.0
0 0.0590196847915649 36.0
0 0.00829684734344482 37.0
0 -0.0331896543502808 38.0
0 -0.0670863389968872 39.0
0 0.138474822044373 40.0
0 -0.0146087408065796 41.0
0 -0.082776665687561 42.0
304.31973609 292.331146240234 43.0
0 0.0250095129013062 44.0
0 -0.422122716903687 45.0
0 -0.0021442174911499 46.0
0 -0.0806816816329956 47.0
0 -0.0593825578689575 48.0
0 0.272194266319275 49.0
0.063046412882 0.0603734254837036 50.0
1.7512188062 0.999087512493134 51.0
0 -0.0701123476028442 52.0
0 -0.0932575464248657 53.0
0 0.0407239198684692 54.0
0 -0.0562509298324585 55.0
0 -0.0677460432052612 56.0
0 -0.0536984205245972 57.0
0 -0.0632270574569702 58.0
0 -0.0833760499954224 59.0
0 0.00678980350494385 60.0
0 -0.0105718374252319 61.0
0 0.852477550506592 62.0
261.08736624 258.568298339844 0.0
0 -0.165637612342834 1.0
0 0.00849235057830811 2.0
0 0.0315006971359253 3.0
0 -0.0268901586532593 4.0
0 -0.00681698322296143 5.0
0 -0.0508800745010376 6.0
0 -0.014496922492981 7.0
0 -0.0226408243179321 8.0
0 0.00991952419281006 9.0
0 0.0200052261352539 10.0
0 -0.00811505317687988 11.0
0 -0.00356161594390869 12.0
66.179953128 65.5005340576172 13.0
36.407034241 32.3825798034668 14.0
0 -0.00599253177642822 15.0
0 -0.0897926092147827 16.0
36.044385975 36.3763427734375 17.0
0 0.0369266271591187 18.0
0 -0.00932657718658447 19.0
0 0.00226080417633057 20.0
0 -0.0437067747116089 21.0
0 0.0238879919052124 22.0
40.736021611 40.6233253479004 23.0
0 -0.13887894153595 24.0
0 0.55711555480957 25.0
0 0.0340675115585327 26.0
0 0.0610054731369019 27.0
0 0.0176328420639038 28.0
0 -0.0456262826919556 29.0
0 -0.0552182197570801 30.0
0 -0.0462158918380737 31.0
0 -0.0734459161758423 32.0
0 -0.0825408697128296 33.0
0 -0.0312286615371704 34.0
0 -0.0443261861801147 35.0
0 0.0255981683731079 36.0
0 0.00886929035186768 37.0
0 0.0565739870071411 38.0
0 -0.0338340997695923 39.0
0 0.0911577939987183 40.0
0 0.00414443016052246 41.0
0 0.0279020071029663 42.0
292.26257826 295.213073730469 43.0
0 -0.0536609888076782 44.0
0 -0.0525975227355957 45.0
0 -0.00160253047943115 46.0
0 -0.0502419471740723 47.0
0 0.0208700895309448 48.0
0 0.313530921936035 49.0
0.066713174088 0.0688282251358032 50.0
1.8556383549 0.997017920017242 51.0
0 -0.0659290552139282 52.0
0 -0.0622378587722778 53.0
0 0.0385123491287231 54.0
0 -0.0492647886276245 55.0
0 -0.0231529474258423 56.0
0 -0.011597752571106 57.0
0 -0.103993535041809 58.0
0 0.00729215145111084 59.0
0 0.12407910823822 60.0
0 0.101711392402649 61.0
0 0.959606647491455 62.0
269.42602565 254.100250244141 0.0
0 -0.0666118860244751 1.0
0 -0.0317591428756714 2.0
0 0.0118132829666138 3.0
0 -0.0806699991226196 4.0
0 -0.0701063871383667 5.0
0 -0.0380855798721313 6.0
0 -0.0514904260635376 7.0
0 -0.0325238704681396 8.0
0 0.109938025474548 9.0
0 -0.0354937314987183 10.0
0 0.0297597646713257 11.0
0 -0.0341087579727173 12.0
65.799210563 65.2288284301758 13.0
41.067070204 33.35205078125 14.0
0 0.0364364385604858 15.0
0 -0.0602422952651978 16.0
33.01922309 32.7046966552734 17.0
0 -0.0192044973373413 18.0
0 -0.0103298425674438 19.0
0 -0.0169898271560669 20.0
0 -0.00490486621856689 21.0
0 -0.0223170518875122 22.0
28.015000106 27.5304584503174 23.0
0 0.0658162832260132 24.0
0 0.498426795005798 25.0
0 0.080828070640564 26.0
0 0.0284832715988159 27.0
0 -0.025063157081604 28.0
0 -0.0751391649246216 29.0
0 -0.0216323137283325 30.0
0 -0.0547527074813843 31.0
0 -0.00625765323638916 32.0
0 -0.0135148763656616 33.0
0 -0.0372096300125122 34.0
0 -0.0364540815353394 35.0
0 -0.117867827415466 36.0
0 -0.0178369283676147 37.0
0 0.0306364297866821 38.0
0 -0.00676023960113525 39.0
0 0.0392721891403198 40.0
0 -0.0171447992324829 41.0
0 0.0135215520858765 42.0
298.08422857 295.8984375 43.0
0 -0.0315672159194946 44.0
0 -0.00419139862060547 45.0
0 -0.0366519689559937 46.0
0 -0.016100287437439 47.0
0 0.0279625654220581 48.0
0.057292440523 0.306888967752457 49.0
0.093837513548 0.0582517385482788 50.0
1.5961688553 0.913136124610901 51.0
0 -0.0671440362930298 52.0
0 -0.0357333421707153 53.0
0 0.00883781909942627 54.0
0 -0.0541824102401733 55.0
0 0.0210427045822144 56.0
0 -0.0436116456985474 57.0
0 -0.045836329460144 58.0
0 -0.00753915309906006 59.0
0 0.0349990129470825 60.0
0 0.0562304258346558 61.0
0 0.845669507980347 62.0
270.1104562 258.224792480469 0.0
0 -0.0092846155166626 1.0
0 -0.0288208723068237 2.0
0 0.00927650928497314 3.0
0 -0.00842654705047607 4.0
0 -0.0650981664657593 5.0
0 -0.10191023349762 6.0
0 -0.0152601003646851 7.0
0 -0.108214318752289 8.0
0 0.0537360906600952 9.0
0 0.0185344219207764 10.0
0 -0.0101431608200073 11.0
0 -0.187735199928284 12.0
54.831854958 54.6257400512695 13.0
38.068687422 37.8757743835449 14.0
0 -0.04981529712677 15.0
0 -0.0574387311935425 16.0
36.462148554 35.6657676696777 17.0
0 0.0363799333572388 18.0
0 0.0273224115371704 19.0
0 -0.0402039289474487 20.0
0 -0.0136095285415649 21.0
0 0.00378072261810303 22.0
31.741452668 31.0602531433105 23.0
0 -0.0837608575820923 24.0
0 0.710880160331726 25.0
0 0.0567196607589722 26.0
0 0.0768526792526245 27.0
0 -0.0589252710342407 28.0
0 -0.0471252202987671 29.0
0 -0.0233659744262695 30.0
0 -0.0634742975234985 31.0
0 -0.0496994256973267 32.0
0 0.06023108959198 33.0
0 -0.0749295949935913 34.0
0 0.0331802368164062 35.0
0 -0.0217133760452271 36.0
0 0.00669181346893311 37.0
0 -0.0393956899642944 38.0
0 0.127116799354553 39.0
0 0.031755805015564 40.0
0 -0.0410603284835815 41.0
0 -0.0268036127090454 42.0
305.69173281 296.727325439453 43.0
0 -0.0203062295913696 44.0
0 -0.0505534410476685 45.0
0 -0.0332382917404175 46.0
0 -0.0388315916061401 47.0
0 -0.0053640604019165 48.0
0 0.26897931098938 49.0
0.072399664922 0.073157787322998 50.0
1.8137849334 0.882835030555725 51.0
0 -0.0604962110519409 52.0
0 -0.0760985612869263 53.0
0 -0.0266774892807007 54.0
0 -0.12516987323761 55.0
0 -0.0210922956466675 56.0
0 -0.0331908464431763 57.0
0 -0.0601428747177124 58.0
0 -0.056646466255188 59.0
0 0.0103791952133179 60.0
0 0.0674432516098022 61.0
0 0.904422581195831 62.0
276.79123355 253.204193115234 0.0
0 -0.0425969362258911 1.0
0 -0.0319346189498901 2.0
0 -0.0353666543960571 3.0
0 0.00937235355377197 4.0
0 0.0132678747177124 5.0
0 0.0554918050765991 6.0
0 -0.222583413124084 7.0
0 0.0915895700454712 8.0
0 0.014457106590271 9.0
0 -0.0235081911087036 10.0
0 -0.0300560593605042 11.0
0 0.0459331274032593 12.0
63.422758573 61.8152198791504 13.0
37.123995372 35.7994575500488 14.0
0 -0.0444504022598267 15.0
0 -0.0398391485214233 16.0
36.811108355 36.4432754516602 17.0
0 -0.0267623662948608 18.0
0 -0.00940287113189697 19.0
0 0.00867712497711182 20.0
0 -0.0589002370834351 21.0
0 -0.00439083576202393 22.0
36.051308771 35.3188133239746 23.0
0 0.00771176815032959 24.0
1.5421738041 0.610000252723694 25.0
0 0.0329707860946655 26.0
0 0.0637253522872925 27.0
0 -0.0169750452041626 28.0
0 -0.064855694770813 29.0
0 -0.0650520324707031 30.0
0 -0.0315264463424683 31.0
0 -0.0189505815505981 32.0
0 0.0446575880050659 33.0
0 -0.0396355390548706 34.0
0 -0.00491476058959961 35.0
0 0.0143450498580933 36.0
0 -0.0633233785629272 37.0
0 0.0213557481765747 38.0
0 0.0324503183364868 39.0
0 0.0427697896957397 40.0
0 -0.0254584550857544 41.0
0 -0.009285569190979 42.0
296.7365224 293.690979003906 43.0
0 0.0633252859115601 44.0
0 0.0303893089294434 45.0
0 -0.043256402015686 46.0
0 -0.0350714921951294 47.0
0 0.0169757604598999 48.0
0 0.338993221521378 49.0
0.067873108842 0.0559874773025513 50.0
2.1565564027 0.894315302371979 51.0
0 -0.03663170337677 52.0
0 0.012769341468811 53.0
0 0.0335665941238403 54.0
0 -0.0207196474075317 55.0
0 -0.03813636302948 56.0
0 0.0109919309616089 57.0
0 -0.0540949106216431 58.0
0 -0.0324853658676147 59.0
0 -0.0471423864364624 60.0
0 0.0160654783248901 61.0
0 0.802852988243103 62.0
248.81457186 255.654434204102 0.0
0 -0.0893141031265259 1.0
0 -0.140341639518738 2.0
0 0.0123742818832397 3.0
0 -0.0400243997573853 4.0
0 -0.0286327600479126 5.0
0 -0.0241352319717407 6.0
0 -0.0494118928909302 7.0
0 -0.0290497541427612 8.0
0 -0.0541063547134399 9.0
0 -0.0164512395858765 10.0
0 0.00490725040435791 11.0
0 -0.117745041847229 12.0
59.247375114 59.2679977416992 13.0
50.918427684 42.7225189208984 14.0
0 -0.0330101251602173 15.0
0 -0.0219899415969849 16.0
37.876561457 38.3650550842285 17.0
0 0.050683856010437 18.0
0 0.0472522974014282 19.0
0 0.0177277326583862 20.0
0 -0.0712140798568726 21.0
0 -0.0432130098342896 22.0
29.111025914 28.9019813537598 23.0
0 -0.0750185251235962 24.0
0 0.938556849956512 25.0
0 0.0355352163314819 26.0
0 0.00189769268035889 27.0
0 0.0661851167678833 28.0
0 -0.0822664499282837 29.0
0 -0.048067569732666 30.0
0 -0.0594049692153931 31.0
0 -0.0012131929397583 32.0
0 0.0664709806442261 33.0
0 -0.04289710521698 34.0
0 -0.117830753326416 35.0
0 -0.180901169776917 36.0
0 0.0427695512771606 37.0
0 -0.0118554830551147 38.0
0 -0.0163289308547974 39.0
0 -0.102550148963928 40.0
0 -0.0970247983932495 41.0
0 0.00727403163909912 42.0
313.29074583 290.131591796875 43.0
0 -0.0503727197647095 44.0
0 0.1587233543396 45.0
0 -0.0279892683029175 46.0
0 -0.0440751314163208 47.0
0 0.0190833806991577 48.0
0 0.263625204563141 49.0
0.09078456601 0.116189360618591 50.0
1.6286962655 1.02209842205048 51.0
0 -0.0328727960586548 52.0
0 -0.0307861566543579 53.0
0 0.0112196207046509 54.0
0 0.0333365201950073 55.0
0 0.015320897102356 56.0
0 -0.0218168497085571 57.0
0 -0.0943461656570435 58.0
0 -0.0103317499160767 59.0
0 -0.0928913354873657 60.0
0 0.0128356218338013 61.0
0 0.799432873725891 62.0
232.55699626 255.579681396484 0.0
0 -0.0409494638442993 1.0
0 -0.00114476680755615 2.0
0 0.0038067102432251 3.0
0 -0.0366193056106567 4.0
0 0.0045701265335083 5.0
0 -0.0229822397232056 6.0
0 0.000779509544372559 7.0
0 -0.00568974018096924 8.0
0 -0.0408674478530884 9.0
0 -0.0337475538253784 10.0
0 0.0146955251693726 11.0
0 -0.221075654029846 12.0
49.465981989 49.4358940124512 13.0
32.225685901 35.8736915588379 14.0
0 -0.113226532936096 15.0
0 -0.00260937213897705 16.0
33.682527315 32.7629852294922 17.0
0 -0.0110641717910767 18.0
0 -0.0142363309860229 19.0
0 0.0341435670852661 20.0
0 -0.017828106880188 21.0
0 -0.0684424638748169 22.0
35.282091723 35.1947212219238 23.0
0 0.0872272253036499 24.0
0 0.444248557090759 25.0
0 0.0346993207931519 26.0
0 0.1109858751297 27.0
0 0.0389593839645386 28.0
0 -0.0954197645187378 29.0
0 -0.013863205909729 30.0
0 -0.0600515604019165 31.0
0 -0.0204340219497681 32.0
0 0.0444108247756958 33.0
0 -0.0515578985214233 34.0
0 -0.0347055196762085 35.0
0 0.0368779897689819 36.0
0 -0.0407043695449829 37.0
0 0.0163248777389526 38.0
0 0.00283825397491455 39.0
0 -0.017518162727356 40.0
0 0.0235462188720703 41.0
0 -0.0541235208511353 42.0
298.45320865 290.722625732422 43.0
0 0.0306285619735718 44.0
0 0.121954321861267 45.0
0 0.0118802785873413 46.0
0 -0.0497766733169556 47.0
0 -0.00466644763946533 48.0
1.9239731665 0.2793830037117 49.0
0.076841748906 0.0833309888839722 50.0
1.5279097031 0.923647344112396 51.0
0 -0.0541502237319946 52.0
0 -0.036260724067688 53.0
0 0.0495644807815552 54.0
0 -0.00428998470306396 55.0
0 -0.011147141456604 56.0
0 -0.0755685567855835 57.0
0 -0.0799437761306763 58.0
0 -0.0146385431289673 59.0
0 0.00151288509368896 60.0
0 -0.0309761762619019 61.0
0 0.894759654998779 62.0
233.65799234 250.500549316406 0.0
0 -0.0124388933181763 1.0
0 -0.029150128364563 2.0
0 0.0288721323013306 3.0
0 -0.0834602117538452 4.0
0 0.0454012155532837 5.0
0 0.0188530683517456 6.0
0 -0.0147327184677124 7.0
0 -0.0338119864463806 8.0
0 0.0710405111312866 9.0
0 0.0284825563430786 10.0
0 -0.0227800607681274 11.0
0 -0.0134388208389282 12.0
54.364116177 53.8111228942871 13.0
43.894648376 41.6589851379395 14.0
0 -0.0718094110488892 15.0
0 -0.0411607027053833 16.0
33.116409708 32.4523239135742 17.0
0 -0.0189541578292847 18.0
0 -0.00595629215240479 19.0
0 0.0272899866104126 20.0
0 0.00518381595611572 21.0
0 -0.0295382738113403 22.0
36.75857316 36.584041595459 23.0
0 0.00175607204437256 24.0
0 0.853774547576904 25.0
0 0.0435632467269897 26.0
0 0.0783852338790894 27.0
0 -0.0417286157608032 28.0
0 -0.03868567943573 29.0
0 -0.0681174993515015 30.0
0 -0.0682219266891479 31.0
0 -0.0352941751480103 32.0
0 0.0446327924728394 33.0
0 -0.0559145212173462 34.0
0 -0.102565884590149 35.0
0 -0.0277479887008667 36.0
0 0.00940501689910889 37.0
0 0.0546115636825562 38.0
0 0.0586122274398804 39.0
0 -0.0288969278335571 40.0
0 0.00673651695251465 41.0
0 0.006156325340271 42.0
301.72402746 297.769134521484 43.0
0 -0.019116997718811 44.0
0 0.0494381189346313 45.0
0 -0.0197523832321167 46.0
0 -0.0461422204971313 47.0
0 0.0287774801254272 48.0
0 0.310271382331848 49.0
0.088404334203 0.0811189413070679 50.0
1.7721548014 0.910195529460907 51.0
0 -0.0617104768753052 52.0
0 -0.0672188997268677 53.0
0 0.0218445062637329 54.0
0 -0.109837889671326 55.0
0 0.0158449411392212 56.0
0 -0.00478565692901611 57.0
0 -0.0758582353591919 58.0
0 -0.0414413213729858 59.0
0 -0.0904921293258667 60.0
0 0.00487267971038818 61.0
0 0.786685347557068 62.0
225.97606024 258.311248779297 0.0
0 -0.0378705263137817 1.0
0 -0.0972124338150024 2.0
0 -0.00543797016143799 3.0
0 -0.0145152807235718 4.0
0 -0.0451964139938354 5.0
0 -0.00736343860626221 6.0
0 -0.0501841306686401 7.0
0 0.076525092124939 8.0
0 0.00092780590057373 9.0
0 -0.0186170339584351 10.0
0 0.0350608229637146 11.0
0 -0.0543323755264282 12.0
69.658659775 69.4498062133789 13.0
13.562326857 26.6208896636963 14.0
0 -0.00343239307403564 15.0
0 -0.0364328622817993 16.0
30.707251708 30.3185176849365 17.0
0 -0.0084766149520874 18.0
0 0.0231403112411499 19.0
0 0.0338577032089233 20.0
0 -0.0116015672683716 21.0
0 -0.0220576524734497 22.0
38.01278503 38.0572509765625 23.0
0 -0.0238462686538696 24.0
0 0.84806877374649 25.0
0 0.0896557569503784 26.0
0 0.0893055200576782 27.0
0 0.0687764883041382 28.0
0 -0.0566855669021606 29.0
0 -0.0458894968032837 30.0
0 -0.0494698286056519 31.0
0 -0.0194350481033325 32.0
0 -0.0478628873825073 33.0
0 -0.0428119897842407 34.0
0 0.0494775772094727 35.0
0 -0.0806628465652466 36.0
0 -0.0122615098953247 37.0
0 -0.0160635709762573 38.0
0 0.0254400968551636 39.0
0 -0.0292865037918091 40.0
0 -0.0648649930953979 41.0
0 0.0196603536605835 42.0
294.85646142 297.502868652344 43.0
0 -0.0488516092300415 44.0
0 0.0106774568557739 45.0
0 -0.0503050088882446 46.0
0 -0.0181487798690796 47.0
0 0.00700318813323975 48.0
0 0.284814476966858 49.0
0.06961985294 0.0486470460891724 50.0
1.8876776666 0.864714860916138 51.0
0 -0.114394783973694 52.0
0 -0.0469990968704224 53.0
0 -0.0417145490646362 54.0
0 -0.0604914426803589 55.0
0 -0.0333071947097778 56.0
0 -0.0379897356033325 57.0
0 -0.0896400213241577 58.0
0 -0.0245112180709839 59.0
0 -0.0871518850326538 60.0
0 -0.00720798969268799 61.0
0 0.830439329147339 62.0
268.84642055 259.178833007812 0.0
0 -0.110925316810608 1.0
0 -0.00983846187591553 2.0
0 0.0146805047988892 3.0
0 -0.0251208543777466 4.0
0 0.00934398174285889 5.0
0 -0.0533229112625122 6.0
0 -0.203533053398132 7.0
0 -0.0562211275100708 8.0
0 0.101980566978455 9.0
0 -0.080026388168335 10.0
0 -0.0394967794418335 11.0
0 0.0810407400131226 12.0
64.665590061 63.7387809753418 13.0
36.848695474 31.5584411621094 14.0
0 -0.0547720193862915 15.0
0 -0.00499927997589111 16.0
33.097476726 32.5559883117676 17.0
0 0.0535861253738403 18.0
0 0.0606421232223511 19.0
0 -0.0163251161575317 20.0
0 -0.0224553346633911 21.0
0 -0.0476046800613403 22.0
39.769121878 39.4130744934082 23.0
0 -0.0559698343276978 24.0
0 0.98659610748291 25.0
0 0.081438422203064 26.0
0 0.137429356575012 27.0
0 0.0214437246322632 28.0
0 -0.0560852289199829 29.0
0 -0.0773758888244629 30.0
0 -0.0548261404037476 31.0
0 -0.0531812906265259 32.0
0 0.0185538530349731 33.0
0 -0.118870139122009 34.0
0 -0.133644104003906 35.0
0 -0.119692921638489 36.0
0 -0.0627747774124146 37.0
0 0.0264724493026733 38.0
0 -0.0211645364761353 39.0
0 0.0232876539230347 40.0
0 -0.0282773971557617 41.0
0 -0.0425287485122681 42.0
293.94686565 295.878112792969 43.0
0 0.0417407751083374 44.0
0 -0.0400488376617432 45.0
0 -0.00906932353973389 46.0
0 -0.0444682836532593 47.0
0 0.0127981901168823 48.0
0 0.278851211071014 49.0
0.092022737967 0.080605149269104 50.0
2.0683690731 0.91912567615509 51.0
0 -0.0826269388198853 52.0
0 -0.0465883016586304 53.0
0 -0.00454556941986084 54.0
0 -0.041601300239563 55.0
0 -0.095068097114563 56.0
0 0.0234488248825073 57.0
0 0.0141767263412476 58.0
0 -0.0513676404953003 59.0
0 0.000199198722839355 60.0
0 0.0842872858047485 61.0
0 0.899663150310516 62.0
225.71843141 259.303405761719 0.0
0 -0.0526920557022095 1.0
0 0.0827072858810425 2.0
0 0.00346863269805908 3.0
0 -0.0288180112838745 4.0
0 0.104310750961304 5.0
0 0.0732778310775757 6.0
0 -0.0579242706298828 7.0
0 0.0230611562728882 8.0
0 0.0287760496139526 9.0
0 -0.0249646902084351 10.0
0 0.0290681719779968 11.0
0 -0.0745493173599243 12.0
77.126407252 77.0934448242188 13.0
3.8384771544 21.2500553131104 14.0
0 -0.0629650354385376 15.0
0 -0.0751115083694458 16.0
32.136463664 31.7652854919434 17.0
0 -0.013013482093811 18.0
0 -0.0460656881332397 19.0
0 -0.0115770101547241 20.0
0 -0.0419954061508179 21.0
0 0.0301541090011597 22.0
27.570393345 27.732364654541 23.0
0 -0.0793853998184204 24.0
0 0.689445555210114 25.0
0 0.0998185873031616 26.0
0 0.119490265846252 27.0
0 -0.00627481937408447 28.0
0 -0.00466620922088623 29.0
0 0.403519511222839 30.0
0 -0.0605138540267944 31.0
0 -0.0275470018386841 32.0
0 -0.025140643119812 33.0
0 -0.00770390033721924 34.0
0 -0.0437355041503906 35.0
0 0.0735780000686646 36.0
0 -0.0570579767227173 37.0
0 0.0580183267593384 38.0
0 0.0269159078598022 39.0
0 0.0735305547714233 40.0
0 -0.0242360830307007 41.0
0 -0.0599952936172485 42.0
291.06252088 299.883178710938 43.0
0 0.0669406652450562 44.0
0 -0.0913975238800049 45.0
0 -0.0272892713546753 46.0
0 0.00631940364837646 47.0
0 0.0461620092391968 48.0
0 0.294023871421814 49.0
0.067346752959 0.0470432043075562 50.0
2.2143386046 0.84261679649353 51.0
0 -0.0481454133987427 52.0
0 -0.0188618898391724 53.0
0 0.0545412302017212 54.0
0 -0.0976384878158569 55.0
0 -0.0348767042160034 56.0
0 0.0542365312576294 57.0
0 -0.00473511219024658 58.0
0 -0.0376428365707397 59.0
0 0.136222958564758 60.0
0 -0.00364983081817627 61.0
0 0.841882109642029 62.0
285.34146214 248.502258300781 0.0
0 0.0349596738815308 1.0
0 -0.08613121509552 2.0
0 -0.0055772066116333 3.0
0 -0.0250219106674194 4.0
0 -0.0349446535110474 5.0
0 0.0194410085678101 6.0
0 -0.0518327951431274 7.0
0 -0.0190727710723877 8.0
0 -0.0922008752822876 9.0
0 -0.10446560382843 10.0
0 -0.0207788944244385 11.0
0 0.00481641292572021 12.0
57.839089168 58.235424041748 13.0
25.341378056 41.7255668640137 14.0
0 -0.0349928140640259 15.0
0 -0.0516906976699829 16.0
37.987312129 37.0424270629883 17.0
0 -0.0154780149459839 18.0
0 -0.000486016273498535 19.0
0 -0.00412857532501221 20.0
0 -0.0365813970565796 21.0
0 0.0493024587631226 22.0
26.450075621 25.535177230835 23.0
0 0.0861848592758179 24.0
0 0.512576341629028 25.0
0 0.107006430625916 26.0
0 0.0315922498703003 27.0
0 0.0931962728500366 28.0
0 -0.0515915155410767 29.0
0 -0.0853376388549805 30.0
0 -0.0671643018722534 31.0
0 0.0283921957015991 32.0
0 0.135334849357605 33.0
0 -0.0748666524887085 34.0
0 0.0256284475326538 35.0
0 0.0261517763137817 36.0
0 -0.0311428308486938 37.0
0 -0.0291341543197632 38.0
0 0.0102082490921021 39.0
0 0.0346354246139526 40.0
0 0.0422098636627197 41.0
0 -0.00593245029449463 42.0
307.16220672 295.112213134766 43.0
0 -0.00751650333404541 44.0
0 0.224157691001892 45.0
0 -0.0185636281967163 46.0
0 -0.036604642868042 47.0
0 0.0215929746627808 48.0
0.12415470289 0.295648515224457 49.0
0.083238980263 0.17677104473114 50.0
1.8526915615 0.842913150787354 51.0
0 -0.105246424674988 52.0
0 -0.0400458574295044 53.0
0 0.055782675743103 54.0
0 -0.135722994804382 55.0
0 -0.0861538648605347 56.0
0 0.0606292486190796 57.0
0 -0.0655938386917114 58.0
0 -0.0342227220535278 59.0
0 0.0896338224411011 60.0
0 0.0611003637313843 61.0
0 0.82316255569458 62.0
283.03778031 253.693252563477 0.0
0 -0.0355583429336548 1.0
0 -0.0256489515304565 2.0
0 0.0241876840591431 3.0
0 -0.0557280778884888 4.0
0 -0.0811434984207153 5.0
0 -0.0908905267715454 6.0
0 -0.131135106086731 7.0
0 0.0687421560287476 8.0
0 -0.0683032274246216 9.0
0 -0.0612636804580688 10.0
0 -0.0147455334663391 11.0
0 0.0115035772323608 12.0
70.153369453 70.1395797729492 13.0
44.15327627 27.4450988769531 14.0
0 -0.0895687341690063 15.0
0 -0.0280729532241821 16.0
25.783204519 26.2433261871338 17.0
0 -0.0228346586227417 18.0
0 -0.0188466310501099 19.0
0 -0.0178722143173218 20.0
0 0.00701844692230225 21.0
0 -0.0698148012161255 22.0
40.269832152 40.217945098877 23.0
0 0.0322378873825073 24.0
0 0.468379735946655 25.0
0 0.0369189977645874 26.0
0 0.0528420209884644 27.0
0 0.0385726690292358 28.0
0 -0.0118891000747681 29.0
0 -0.10004198551178 30.0
0 -0.0314379930496216 31.0
0 -0.0819598436355591 32.0
0 0.0886985063552856 33.0
0 0.00522148609161377 34.0
0 0.117363810539246 35.0
0 0.100882172584534 36.0
0 -0.0579067468643188 37.0
0 0.013374924659729 38.0
0 0.0419830083847046 39.0
0 0.252248644828796 40.0
0 -0.078605055809021 41.0
0 0.00786912441253662 42.0
302.0113475 296.498138427734 43.0
0 -0.0580283403396606 44.0
0 0.141585946083069 45.0
0 0.00423872470855713 46.0
0 -0.0775097608566284 47.0
0 -0.103199601173401 48.0
0 0.315500974655151 49.0
0.082153014496 0.0701192617416382 50.0
1.5016404248 1.00444483757019 51.0
0 -0.0120702981948853 52.0
0 0.0213884115219116 53.0
0 0.0248016119003296 54.0
0 -0.0424998998641968 55.0
0 -0.0778597593307495 56.0
0 0.00825297832489014 57.0
0 -0.0711430311203003 58.0
0 -0.0939027070999146 59.0
0 0.018673300743103 60.0
0 0.116897463798523 61.0
0 0.970931172370911 62.0
252.94528798 257.844573974609 0.0
0 -0.0503793954849243 1.0
0 -0.108402609825134 2.0
0 -0.0201197862625122 3.0
0 -0.0403605699539185 4.0
0 0.0353409051895142 5.0
0 -0.023750901222229 6.0
0 -0.0676857233047485 7.0
0 -0.0762810707092285 8.0
0 0.125486731529236 9.0
0 -0.0197309255599976 10.0
0 -0.0344917774200439 11.0
0 -0.0356584787368774 12.0
56.862684157 56.4962844848633 13.0
39.05627099 37.7046585083008 14.0
0 -0.0674072504043579 15.0
0 -0.00982439517974854 16.0
37.590417237 36.5924034118652 17.0
0 0.0170165300369263 18.0
0 0.0312517881393433 19.0
0 0.0242136716842651 20.0
0 -0.00792396068572998 21.0
0 -0.0563346147537231 22.0
28.352805298 27.8362045288086 23.0
0 0.0233582258224487 24.0
0 1.39739429950714 25.0
0 0.0707024335861206 26.0
0 -0.027880072593689 27.0
0 -0.0240036249160767 28.0
0 -0.0823427438735962 29.0
0 -0.0359524488449097 30.0
0 -0.0218299627304077 31.0
0 -0.0918608903884888 32.0
0 -0.0499330759048462 33.0
0 -0.0977593660354614 34.0
0 -0.0227264165878296 35.0
0 0.0682698488235474 36.0
0 -0.0798028707504272 37.0
0 -0.000736117362976074 38.0
0 0.0595475435256958 39.0
0 0.0962971448898315 40.0
0 -0.0393719673156738 41.0
0 0.0699590444564819 42.0
296.85403669 293.616912841797 43.0
0 -0.0323330163955688 44.0
0 0.0280420780181885 45.0
0 0.0162836313247681 46.0
0 -0.0466810464859009 47.0
0 -0.0528475046157837 48.0
1.919601633 0.329600214958191 49.0
0.073686326159 0.0438851118087769 50.0
2.1239768262 0.975337564945221 51.0
0 -0.0731426477432251 52.0
0 -0.0430647134780884 53.0
0 -0.00684130191802979 54.0
0 -0.0601564645767212 55.0
0 -0.0925561189651489 56.0
0 0.0151752233505249 57.0
0 0.0208407640457153 58.0
0 -0.0640491247177124 59.0
0 -0.100906014442444 60.0
0 0.00604307651519775 61.0
0 0.821269631385803 62.0
244.38117906 257.325500488281 0.0
0 -0.0764442682266235 1.0
0 0.0428742170333862 2.0
0 -0.0403467416763306 3.0
0 -0.0413378477096558 4.0
0 0.073149561882019 5.0
0 0.0257886648178101 6.0
0 -0.0857697725296021 7.0
0 -0.0316313505172729 8.0
0 -0.020185112953186 9.0
0 0.0234197378158569 10.0
0 -0.0459836721420288 11.0
0 -0.0490413904190063 12.0
58.817908408 57.8624649047852 13.0
46.722940887 37.7081871032715 14.0
0 -0.000204682350158691 15.0
0 -0.0223840475082397 16.0
30.234262387 29.2321643829346 17.0
0 0.0230990648269653 18.0
0 -0.0164960622787476 19.0
0 0.0344778299331665 20.0
0 -0.0552302598953247 21.0
0 -0.0327063798904419 22.0
30.874178325 30.2361373901367 23.0
0 -0.055306077003479 24.0
0 0.593736231327057 25.0
0 -0.0136910676956177 26.0
0 0.0924330949783325 27.0
0 0.0270684957504272 28.0
0 -0.0487154722213745 29.0
0 -0.0834213495254517 30.0
0 -0.0361331701278687 31.0
0 0.0135449171066284 32.0
0 0.0991836786270142 33.0
0 -0.00353777408599854 34.0
0 -0.106181263923645 35.0
0 0.0422912836074829 36.0
0 -0.0521169900894165 37.0
0 0.0622204542160034 38.0
0 0.0385860204696655 39.0
0 0.0913504362106323 40.0
0 -0.0165759325027466 41.0
0 0.0171493291854858 42.0
308.17725645 295.868774414062 43.0
0 0.00851476192474365 44.0
0 -0.0197587013244629 45.0
0 -0.0066303014755249 46.0
0 -0.0456103086471558 47.0
0 0.0510290861129761 48.0
0 0.277547806501389 49.0
0.099608544695 0.0629147291183472 50.0
1.9216247472 0.955210268497467 51.0
0 -0.0315603017807007 52.0
0 0.00573551654815674 53.0
0 0.0411456823348999 54.0
0 -0.091913104057312 55.0
0 -0.0121368169784546 56.0
0 -0.0527340173721313 57.0
0 -0.0202382802963257 58.0
0 -0.0669516324996948 59.0
0 0.0238269567489624 60.0
0 -0.0381170511245728 61.0
0 0.852928161621094 62.0
266.66609717 259.207214355469 0.0
0 -0.0654741525650024 1.0
0 0.00958621501922607 2.0
0 -0.0548425912857056 3.0
0 -0.0501917600631714 4.0
0 -0.0206550359725952 5.0
0 -0.0196882486343384 6.0
0 -0.0811883211135864 7.0
0 -0.0473648309707642 8.0
0 -0.0168715715408325 9.0
0 -0.011443018913269 10.0
0 0.0053095817565918 11.0
0 -0.0112797021865845 12.0
65.383447241 64.9958114624023 13.0
22.552843657 34.2354774475098 14.0
0 -0.102294564247131 15.0
0 -0.000244975090026855 16.0
31.69747893 31.0565719604492 17.0
0 -0.0477026700973511 18.0
0 -0.00645244121551514 19.0
0 0.0237847566604614 20.0
0 -0.023518443107605 21.0
0 -0.0348578691482544 22.0
38.382268246 38.4501342773438 23.0
0 0.0506552457809448 24.0
0 0.841710567474365 25.0
0 0.024687647819519 26.0
0 0.043029248714447 27.0
0 0.0339025259017944 28.0
0 -0.011980414390564 29.0
0 -0.0525350570678711 30.0
0 -0.0484107732772827 31.0
0 -0.0843731164932251 32.0
0 0.09430992603302 33.0
0 -0.0138453245162964 34.0
0 -0.0664747953414917 35.0
0 0.0127013921737671 36.0
0 -0.0886536836624146 37.0
0 0.0159577131271362 38.0
0 -0.0257874727249146 39.0
0 0.00703132152557373 40.0
0 -0.0192313194274902 41.0
0 -0.00271093845367432 42.0
302.25395628 296.77392578125 43.0
0 -0.0482803583145142 44.0
0 -0.00036776065826416 45.0
0 0.0105479955673218 46.0
0 -0.0761371850967407 47.0
0 0.0254839658737183 48.0
0 0.275922834873199 49.0
0.088534935379 0.0939990282058716 50.0
1.8435555144 1.01728940010071 51.0
0 -0.0944229364395142 52.0
0 -0.0671535730361938 53.0
0 0.0272427797317505 54.0
0 -0.0946935415267944 55.0
0 0.0332792997360229 56.0
0 -0.0319615602493286 57.0
0 -0.0529319047927856 58.0
0 -0.00692141056060791 59.0
0 -0.0957719087600708 60.0
0 -0.0381902456283569 61.0
0 0.817939758300781 62.0
271.79683126 255.497421264648 0.0
0 -0.0575538873672485 1.0
0 -0.0140460729598999 2.0
0 -0.0901528596878052 3.0
0 -0.0191274881362915 4.0
0 -0.0298219919204712 5.0
0 -0.0865803956985474 6.0
0 -0.0115338563919067 7.0
0 -0.0342117547988892 8.0
0 0.108182311058044 9.0
0 0.00845992565155029 10.0
0 -0.00622332096099854 11.0
0 -0.041751503944397 12.0
64.138640847 63.526798248291 13.0
33.267947848 32.4504089355469 14.0
0 -0.0589834451675415 15.0
0 -0.0343183279037476 16.0
30.838204555 31.0848827362061 17.0
0 0.00948536396026611 18.0
0 0.00950682163238525 19.0
0 0.0100582838058472 20.0
0 -0.0159600973129272 21.0
0 -0.0297561883926392 22.0
40.875194813 40.687198638916 23.0
0 -0.130507588386536 24.0
0 1.85134923458099 25.0
0 0.118807673454285 26.0
0 0.0740698575973511 27.0
0 -0.0288869142532349 28.0
0 -0.0933843851089478 29.0
0 -0.0671945810317993 30.0
0 -0.0118669271469116 31.0
0 -0.0458275079727173 32.0
0 0.0391014814376831 33.0
0 -0.0780636072158813 34.0
0 -0.0494405031204224 35.0
0 0.123825669288635 36.0
0 -0.0511828660964966 37.0
0 0.0238610506057739 38.0
0 0.0905667543411255 39.0
0 0.0405625104904175 40.0
0 0.0108226537704468 41.0
0 0.0672944784164429 42.0
288.1242724 291.744567871094 43.0
0 0.012194037437439 44.0
0 0.0407494306564331 45.0
0 -0.0240689516067505 46.0
0 -0.0687886476516724 47.0
0 -0.0142422914505005 48.0
0 0.29510760307312 49.0
0.095563565059 0.0653101205825806 50.0
2.1691385168 0.920939087867737 51.0
0 -0.0772303342819214 52.0
0 -0.0489999055862427 53.0
0 -0.00607764720916748 54.0
0 -0.0579992532730103 55.0
0 -0.0113412141799927 56.0
0 -0.0362333059310913 57.0
0 -0.0486503839492798 58.0
0 -0.00232613086700439 59.0
0 -0.0185414552688599 60.0
0 -0.0148438215255737 61.0
0 0.859788060188293 62.0
275.63056689 248.265670776367 0.0
0 -0.0483499765396118 1.0
0 -0.0834523439407349 2.0
0 0.0341712236404419 3.0
0 -0.0417078733444214 4.0
0 -0.126076817512512 5.0
0 0.196985602378845 6.0
0 0.0527392625808716 7.0
0 0.00929379463195801 8.0
0 0.0581964254379272 9.0
0 -0.0132087469100952 10.0
0 -0.0299617052078247 11.0
0 0.0156692266464233 12.0
48.92124188 47.4851989746094 13.0
33.337069839 33.2814903259277 14.0
0 -0.0396305322647095 15.0
0 0.0255492925643921 16.0
24.855084947 24.0237979888916 17.0
0 0.0168818235397339 18.0
0 -0.0313011407852173 19.0
0 0.000970959663391113 20.0
0 -0.0210295915603638 21.0
0 -0.0131818056106567 22.0
43.283294724 42.2102546691895 23.0
0 -0.149850487709045 24.0
0 4.08473539352417 25.0
0 0.049888014793396 26.0
0 0.0643700361251831 27.0
0 0.0443228483200073 28.0
0 -0.0721615552902222 29.0
0 -0.0678383111953735 30.0
0 0.00457537174224854 31.0
0 -0.107261300086975 32.0
0 -0.0087425708770752 33.0
0 -0.091513991355896 34.0
0 0.00938200950622559 35.0
0 -0.201245665550232 36.0
0 0.0261765718460083 37.0
0 -0.050017237663269 38.0
0 0.108899474143982 39.0
0 0.0645283460617065 40.0
0 -0.0581891536712646 41.0
0 -0.101559281349182 42.0
312.24680318 296.424713134766 43.0
0 -0.0128418207168579 44.0
0 0.0118150115013123 45.0
0 -0.0178643465042114 46.0
0 -0.0694111585617065 47.0
0 0.131152033805847 48.0
0 0.219805181026459 49.0
0.099569927761 0.0748981237411499 50.0
1.9400882412 1.00599813461304 51.0
0 -0.074039101600647 52.0
0 -0.054651141166687 53.0
0 0.0204488039016724 54.0
0 -0.0734392404556274 55.0
0 -0.118459105491638 56.0
0 0.0199130773544312 57.0
0 -0.0364152193069458 58.0
0 -0.0699132680892944 59.0
0 -0.0623992681503296 60.0
0 -0.0303481817245483 61.0
0 0.802160382270813 62.0
276.42782907 256.665679931641 0.0
0 -0.0534301996231079 1.0
0 -0.0557225942611694 2.0
0 -0.0344864130020142 3.0
0 -0.0651222467422485 4.0
0 -0.0154272317886353 5.0
0 0.00256335735321045 6.0
0 -0.0348097085952759 7.0
0 -0.0828326940536499 8.0
0 0.102622389793396 9.0
0 -0.039009690284729 10.0
0 -0.0265487432479858 11.0
0 0.0711997747421265 12.0
73.507077504 70.8033828735352 13.0
22.789538882 24.0718002319336 14.0
0 -0.173109650611877 15.0
0 0.00778281688690186 16.0
41.661853572 42.2402305603027 17.0
0 -0.0333060026168823 18.0
0 -0.00166046619415283 19.0
0 0.00935399532318115 20.0
0 -0.0447510480880737 21.0
0 0.035213828086853 22.0
32.744999983 32.7133636474609 23.0
0 0.0780996084213257 24.0
0 0.502658784389496 25.0
0 0.0234411954879761 26.0
0 0.052817702293396 27.0
0 0.0991619825363159 28.0
0 -0.062296986579895 29.0
0 -0.0768365859985352 30.0
0 -0.0624006986618042 31.0
0 0.015281081199646 32.0
0 0.106759905815125 33.0
0 -0.044798731803894 34.0
0 -0.0649710893630981 35.0
0 -0.0228866338729858 36.0
0 -0.0936332941055298 37.0
0 -0.0360184907913208 38.0
0 0.056531548500061 39.0
0 0.178513407707214 40.0
0 -0.0807749032974243 41.0
0 -0.100242257118225 42.0
309.86963318 297.009887695312 43.0
0 0.0355538129806519 44.0
0 -0.079706072807312 45.0
0 -0.00778782367706299 46.0
0 0.0250428915023804 47.0
0 0.0507372617721558 48.0
0 0.376006901264191 49.0
0.097215947996 0.0557404756546021 50.0
1.5379566443 1.02401614189148 51.0
0 -0.0152992010116577 52.0
0 -0.0551613569259644 53.0
0 0.0505739450454712 54.0
0 -0.126572728157043 55.0
0 -0.0419560670852661 56.0
0 0.00619876384735107 57.0
0 -0.0610357522964478 58.0
0 -0.0813542604446411 59.0
0 -0.112648606300354 60.0
0 0.0508114099502563 61.0
0 0.840018749237061 62.0
239.56547112 253.967819213867 0.0
0 -0.0192948579788208 1.0
0 -0.011722207069397 2.0
0 0.0259302854537964 3.0
0 -0.0417574644088745 4.0
0 0.0567363500595093 5.0
0 0.0124820470809937 6.0
0 -0.0733782052993774 7.0
0 -0.0284465551376343 8.0
0 -0.0347000360488892 9.0
0 0.0391675233840942 10.0
0 0.0189839601516724 11.0
0 0.0747274160385132 12.0
48.949389317 45.0098724365234 13.0
39.698927028 45.325252532959 14.0
0 -0.0547877550125122 15.0
0 -0.031408429145813 16.0
31.294059177 31.5667476654053 17.0
0 -0.0112158060073853 18.0
0 -0.000699400901794434 19.0
0 0.0849152803421021 20.0
0 0.0405000448226929 21.0
0 -0.0154973268508911 22.0
25.58027467 24.8812294006348 23.0
0 -0.0177861452102661 24.0
0 0.934187710285187 25.0
0 0.0054856538772583 26.0
0 -0.117198348045349 27.0
0 0.0498532056808472 28.0
0 -0.103936314582825 29.0
0 -0.0674487352371216 30.0
0 -0.0280622243881226 31.0
0 0.0222429037094116 32.0
0 0.00379025936126709 33.0
0 0.00454843044281006 34.0
0 -0.0242025852203369 35.0
0 0.00406181812286377 36.0
0 -0.0706168413162231 37.0
0 -0.0500539541244507 38.0
0 -0.0411924123764038 39.0
0 0.166307330131531 40.0
0 -0.0171948671340942 41.0
0 0.0264543294906616 42.0
301.28417734 293.139923095703 43.0
0 -0.00782454013824463 44.0
0 -0.0483907461166382 45.0
0 0.00824034214019775 46.0
0 -0.0687986612319946 47.0
0 -0.0734878778457642 48.0
0 0.351131975650787 49.0
0.070410603483 0.0482760667800903 50.0
1.5471955211 0.915026485919952 51.0
0 -0.0517901182174683 52.0
0 -0.026971697807312 53.0
0 0.0689877271652222 54.0
0 -0.0890017747879028 55.0
0 -0.0682352781295776 56.0
0 -0.0379856824874878 57.0
0 -0.00172007083892822 58.0
0 -0.0131410360336304 59.0
0 0.110465884208679 60.0
0 0.0218585729598999 61.0
0 0.894650876522064 62.0
280.00719244 258.276824951172 0.0
0 -0.139309525489807 1.0
0 -0.072273850440979 2.0
0 -0.00946414470672607 3.0
0 -0.0539034605026245 4.0
0 -0.116579651832581 5.0
0 -0.047272801399231 6.0
0 -0.112201809883118 7.0
0 0.00597870349884033 8.0
0 0.0301889181137085 9.0
0 -0.0426181554794312 10.0
0 -0.0265827775001526 11.0
0 -0.0999952554702759 12.0
53.790237666 53.1135749816895 13.0
31.410668941 41.515754699707 14.0
0 -0.0149751901626587 15.0
0 -0.00891077518463135 16.0
33.25132231 33.6083564758301 17.0
0 -0.0611914396286011 18.0
0 -0.00546061992645264 19.0
0 0.000504851341247559 20.0
0 -0.0514310598373413 21.0
0 0.0289210081100464 22.0
32.638287812 31.4311084747314 23.0
0 0.00914180278778076 24.0
0 0.539139032363892 25.0
0 0.118454813957214 26.0
0 -0.0528603792190552 27.0
0 -0.0116015672683716 28.0
0 0.00277268886566162 29.0
0 0.121621251106262 30.0
0 -0.0727418661117554 31.0
0 0.000567317008972168 32.0
0 -0.0320903062820435 33.0
0 -0.024687647819519 34.0
0 -0.0170186758041382 35.0
0 0.0257710218429565 36.0
0 -0.000483155250549316 37.0
0 -0.0014575719833374 38.0
0 0.0235134363174438 39.0
0 0.0414255857467651 40.0
0 -0.0620822906494141 41.0
0 0.0286086797714233 42.0
290.34130433 295.149841308594 43.0
0 0.0235980749130249 44.0
0 0.0548765659332275 45.0
0 -0.0222169160842896 46.0
0 -0.0521031618118286 47.0
0 -0.0277739763259888 48.0
0 0.318236500024796 49.0
0.089386096543 0.0673168897628784 50.0
1.6631265882 0.946956694126129 51.0
0 -0.0453420877456665 52.0
0 -0.0276983976364136 53.0
0 -0.0184410810470581 54.0
0 -0.0672065019607544 55.0
0 0.0382841825485229 56.0
0 -0.0342603921890259 57.0
0 -0.0508929491043091 58.0
0 0.00849664211273193 59.0
0 -0.106635689735413 60.0
0 0.0405622720718384 61.0
0 0.867687940597534 62.0
223.9914769 257.657501220703 0.0
0 -0.0476795434951782 1.0
0 -0.0415350198745728 2.0
0 0.0163673162460327 3.0
0 -0.005928635597229 4.0
0 -0.0188561677932739 5.0
0 -0.303609490394592 6.0
0 -0.0688034296035767 7.0
0 -0.00754654407501221 8.0
0 0.0939763784408569 9.0
0 -0.0820854902267456 10.0
0 0.138464331626892 11.0
0 0.0654882192611694 12.0
77.911143784 74.9172592163086 13.0
5.7892204574 25.7761325836182 14.0
0 -0.0844131708145142 15.0
0 -0.00750124454498291 16.0
37.335076731 37.3204345703125 17.0
0 0.0548382997512817 18.0
0 0.0414539575576782 19.0
0 -0.01219642162323 20.0
0 -0.0517255067825317 21.0
0 0.0938786268234253 22.0
33.287056097 32.2031745910645 23.0
0 -0.0965815782546997 24.0
0 0.857538640499115 25.0
0 -0.0179773569107056 26.0
0 0.0125812292098999 27.0
0 -0.0607115030288696 28.0
0 -0.0130969285964966 29.0
0 -0.145840883255005 30.0
0 0.00888502597808838 31.0
0 -0.187341332435608 32.0
0 -0.0846291780471802 33.0
0 -0.107542872428894 34.0
0 0.237842917442322 35.0
0 0.0557140111923218 36.0
0 0.0357199907302856 37.0
0 0.0352438688278198 38.0
0 0.068569540977478 39.0
0 0.0531672239303589 40.0
0 0.401475191116333 41.0
0 -0.116133332252502 42.0
303.74530842 291.778198242188 43.0
0 -0.0294967889785767 44.0
0 -0.00483620166778564 45.0
0 -0.00640642642974854 46.0
0 -0.0573357343673706 47.0
0 -0.0720280408859253 48.0
0.51427872687 0.343034744262695 49.0
0.082703252759 0.248214721679688 50.0
2.0670842045 0.838998079299927 51.0
0 -0.0459069013595581 52.0
0 -0.0549131631851196 53.0
0 0.066752552986145 54.0
0 -0.0871418714523315 55.0
0 -0.113493084907532 56.0
0 -0.0205823183059692 57.0
0 -0.0418885946273804 58.0
0 -0.104641556739807 59.0
0 -0.235396504402161 60.0
0 0.0188952684402466 61.0
0 0.865670084953308 62.0
257.47549706 253.539398193359 0.0
0 -0.00197756290435791 1.0
0 -0.0364786386489868 2.0
0 -0.0030440092086792 3.0
0 -0.0759788751602173 4.0
0 0.0230876207351685 5.0
0 -0.157508492469788 6.0
0 -0.0109487771987915 7.0
0 -0.0306951999664307 8.0
0 0.00971925258636475 9.0
0 -0.004189133644104 10.0
0 0.0474979877471924 11.0
0 0.0750287771224976 12.0
70.792350397 70.2211456298828 13.0
37.630556774 24.7339305877686 14.0
0 -0.0757623910903931 15.0
0 0.00980889797210693 16.0
39.604461073 39.2265281677246 17.0
0 0.0861028432846069 18.0
0 0.0172103643417358 19.0
0 0.0389848947525024 20.0
0 -0.00546777248382568 21.0
0 0.0377925634384155 22.0
27.41008405 26.5394325256348 23.0
0 0.0853770971298218 24.0
0 0.469604790210724 25.0
0 -0.0369340181350708 26.0
0 0.067745566368103 27.0
0 -0.0049515962600708 28.0
0 -0.0207031965255737 29.0
0 -0.0510910749435425 30.0
0 -0.0685778856277466 31.0
0 0.0439938306808472 32.0
0 0.0404039621353149 33.0
0 -0.069865345954895 34.0
0 0.174074292182922 35.0
0 -0.0338379144668579 36.0
0 -0.0230809450149536 37.0
0 0.0268524885177612 38.0
0 0.0562845468521118 39.0
0 0.27008855342865 40.0
0 -0.0478127002716064 41.0
0 -0.0972887277603149 42.0
293.10661013 296.447021484375 43.0
0 0.00361859798431396 44.0
0 0.0774071216583252 45.0
0 -0.0610290765762329 46.0
0 -0.0293194055557251 47.0
0 -0.0438371896743774 48.0
2.4965545754 0.284857749938965 49.0
0.069006927485 0.0913609266281128 50.0
1.6443350752 0.960287928581238 51.0
0 -0.110608696937561 52.0
0 -0.0485793352127075 53.0
0 0.020433783531189 54.0
0 -0.00310957431793213 55.0
0 0.0266021490097046 56.0
0 -0.0465439558029175 57.0
0 -0.0679720640182495 58.0
0 -0.0278912782669067 59.0
0 0.0666779279708862 60.0
0 7.66515731811523e-05 61.0
0 0.887785315513611 62.0
264.94160441 258.215637207031 0.0
0 -0.0713006258010864 1.0
0 -0.0118707418441772 2.0
0 -0.0645309686660767 3.0
0 -0.00836431980133057 4.0
0 -0.0126749277114868 5.0
0 0.00190961360931396 6.0
0 -0.073438286781311 7.0
0 0.0482181310653687 8.0
0 0.0156368017196655 9.0
0 0.013312816619873 10.0
0 -0.0408152341842651 11.0
0 0.00977170467376709 12.0
53.983733149 52.6457061767578 13.0
48.837777674 43.1013717651367 14.0
0 0.00731313228607178 15.0
0 -0.0307623147964478 16.0
34.295588413 33.6569557189941 17.0
0 -0.0199741125106812 18.0
0 0.0328367948532104 19.0
0 0.0214635133743286 20.0
0 0.0123540163040161 21.0
0 -0.00544130802154541 22.0
30.130073022 29.416467666626 23.0
0 0.00490081310272217 24.0
0 0.555021226406097 25.0
0 0.0353015661239624 26.0
0 0.0193053483963013 27.0
0 -0.00081169605255127 28.0
0 -0.038490891456604 29.0
0 -0.0368435382843018 30.0
0 -0.047538161277771 31.0
0 -0.0273395776748657 32.0
0 0.119757771492004 33.0
0 0.00604617595672607 34.0
0 -0.076384425163269 35.0
0 0.000438094139099121 36.0
0 0.00662529468536377 37.0
0 0.000849366188049316 38.0
0 0.0281633138656616 39.0
0 0.112801432609558 40.0
0 -0.0572725534439087 41.0
0 -0.0692435503005981 42.0
296.98357959 296.565979003906 43.0
0 -0.0197316408157349 44.0
0 0.0171095132827759 45.0
0 -0.0047067403793335 46.0
0 -0.0196958780288696 47.0
0 0.0392268896102905 48.0
0 0.23270046710968 49.0
0.084731397219 0.0662347078323364 50.0
1.9770899799 0.930837750434875 51.0
0 -0.0535537004470825 52.0
0 -0.0372036695480347 53.0
0 0.0204225778579712 54.0
0 -0.0668855905532837 55.0
0 -0.0353080034255981 56.0
0 -0.0225063562393188 57.0
0 -0.0426567792892456 58.0
0 -0.0143626928329468 59.0
0 0.00685441493988037 60.0
0 0.0557371377944946 61.0
0 0.865107417106628 62.0
234.76436467 258.271667480469 0.0
0 -0.0853830575942993 1.0
0 -0.030769944190979 2.0
0 0.00554573535919189 3.0
0 -0.0409282445907593 4.0
0 -0.0193313360214233 5.0
0 0.100964426994324 6.0
0 0.0225814580917358 7.0
0 -0.00908505916595459 8.0
0 0.0880283117294312 9.0
0 0.00729978084564209 10.0
0 -0.0379574298858643 11.0
0 -0.201044678688049 12.0
48.440328131 47.5387763977051 13.0
52.436244207 46.4447975158691 14.0
0 -0.105160355567932 15.0
0 -0.012494683265686 16.0
37.514577482 36.3415107727051 17.0
0 -0.0122283697128296 18.0
0 0.0137585401535034 19.0
0 0.0132473707199097 20.0
0 -0.0425773859024048 21.0
0 0.0174044370651245 22.0
37.230078733 36.8972015380859 23.0
0 -0.124567151069641 24.0
0 1.71255731582642 25.0
0 0.0485256910324097 26.0
0 0.0219358205795288 27.0
0 0.081323504447937 28.0
0 -0.0308417081832886 29.0
0 0.084628701210022 30.0
0 -0.139375329017639 31.0
0 -0.0593091249465942 32.0
0 -0.118854165077209 33.0
0 -0.0115331411361694 34.0
0 -0.0768691301345825 35.0
0 -0.0138872861862183 36.0
0 -0.030103325843811 37.0
0 0.000184893608093262 38.0
0 0.0593284368515015 39.0
0 0.0544222593307495 40.0
0 0.0882760286331177 41.0
0 -0.0319133996963501 42.0
299.91259139 293.969146728516 43.0
0 -0.0306092500686646 44.0
0 0.0379451513290405 45.0
0 0.00503575801849365 46.0
0 -0.0755738019943237 47.0
0 -0.0211626291275024 48.0
1.2566849593 0.258073925971985 49.0
0.080848219987 0.0958913564682007 50.0
1.6261997037 0.869746625423431 51.0
0 -0.0427485704421997 52.0
0 0.00431358814239502 53.0
0 -0.0897399187088013 54.0
0 -0.0802778005599976 55.0
0 -0.0627919435501099 56.0
0 -0.005745530128479 57.0
0 -0.0783127546310425 58.0
0 -0.0260533094406128 59.0
0 -0.0632506608963013 60.0
0 -0.0252753496170044 61.0
0 0.801645517349243 62.0
268.2770799 251.552352905273 0.0
0 -0.0468164682388306 1.0
0 -0.0430179834365845 2.0
0 -0.0312486886978149 3.0
0 -0.0482276678085327 4.0
0 -0.00233328342437744 5.0
0 -0.176762223243713 6.0
0 -0.217422842979431 7.0
0 -0.0787098407745361 8.0
0 0.0900167226791382 9.0
0 0.00317728519439697 10.0
0 -0.0396658182144165 11.0
0 0.00909936428070068 12.0
63.496752842 62.4969940185547 13.0
40.536821234 35.0706214904785 14.0
0 -0.0256716012954712 15.0
0 0.010553240776062 16.0
39.234571726 38.7651176452637 17.0
0 -0.0497454404830933 18.0
0 0.00750768184661865 19.0
0 -0.012548565864563 20.0
0 -0.0182303190231323 21.0
0 -0.0103431940078735 22.0
33.091375035 33.0125846862793 23.0
0 -0.0759474039077759 24.0
0 0.587981522083282 25.0
0 0.0557059049606323 26.0
0 0.0191854238510132 27.0
0 -0.0280009508132935 28.0
0 -0.0981212854385376 29.0
0 -0.0595576763153076 30.0
0 -0.0320264101028442 31.0
0 0.0601714849472046 32.0
0 0.0509723424911499 33.0
0 -0.00064241886138916 34.0
0 0.0457043647766113 35.0
0 -0.0082319974899292 36.0
0 0.0880998373031616 37.0
0 0.0445584058761597 38.0
0 0.0464378595352173 39.0
0 0.0287617444992065 40.0
0 -0.0171449184417725 41.0
0 0.00752294063568115 42.0
283.01646596 293.189392089844 43.0
0 -0.0145751237869263 44.0
0 0.0311325788497925 45.0
0 -0.0308221578598022 46.0
0 -0.0602313280105591 47.0
0 -0.007515549659729 48.0
0 0.324356764554977 49.0
0.091112931927 0.131214737892151 50.0
1.9567162238 0.887820780277252 51.0
0 -0.0556532144546509 52.0
0 0.0141195058822632 53.0
0 0.00874364376068115 54.0
0 -0.024908185005188 55.0
0 -0.13433563709259 56.0
0 0.024558424949646 57.0
0 -0.0564562082290649 58.0
0 -0.0377241373062134 59.0
0 -0.0816465616226196 60.0
0 0.0150531530380249 61.0
0 0.845453143119812 62.0
281.00113385 256.009735107422 0.0
0 -0.078992486000061 1.0
0 0.0167790651321411 2.0
0 -0.00407969951629639 3.0
0 -0.0666927099227905 4.0
0 0.0398391485214233 5.0
0 -0.0579639673233032 6.0
0 0.0341581106185913 7.0
0 0.0139907598495483 8.0
0 0.0655254125595093 9.0
0 -0.00903427600860596 10.0
0 0.0369641780853271 11.0
0 -0.0253578424453735 12.0
74.753302787 75.1272201538086 13.0
24.489793822 22.8140048980713 14.0
0 -0.0254536867141724 15.0
0 -0.0219548940658569 16.0
32.552392045 32.7081260681152 17.0
0 0.0176812410354614 18.0
0 -0.0114811658859253 19.0
0 -0.0191332101821899 20.0
0 -0.0166774988174438 21.0
0 -0.0601121187210083 22.0
38.698858006 37.6498870849609 23.0
0 0.0541962385177612 24.0
0 1.0059220790863 25.0
0 0.0092395544052124 26.0
0 -0.0352044105529785 27.0
0 0.00608670711517334 28.0
0 -0.040617823600769 29.0
0 -0.0550609827041626 30.0
0 -0.032253623008728 31.0
0 0.0146440267562866 32.0
0 -0.0242463350296021 33.0
0 -0.00377428531646729 34.0
0 -0.0384334325790405 35.0
0 0.0372070074081421 36.0
0 -0.0502392053604126 37.0
0 -0.013410210609436 38.0
0 0.112464547157288 39.0
0 0.0960577726364136 40.0
0 -0.0575605630874634 41.0
0 -0.0992695093154907 42.0
303.23468453 295.588775634766 43.0
0 0.038718581199646 44.0
0 0.00927853584289551 45.0
0 -0.00494968891143799 46.0
0 -0.0457836389541626 47.0
0 0.000344157218933105 48.0
0 0.342276334762573 49.0
0.092712506686 0.0706163644790649 50.0
1.5301851507 0.987078189849854 51.0
0 -0.079349160194397 52.0
0 -0.0670057535171509 53.0
0 0.00204765796661377 54.0
0 -0.0944885015487671 55.0
0 -0.0332366228103638 56.0
0 -0.0170432329177856 57.0
0 -0.091363787651062 58.0
0 -0.101377844810486 59.0
0 -0.0298532247543335 60.0
0 0.00296056270599365 61.0
0 0.831891059875488 62.0
291.04612406 262.024688720703 0.0
0 -0.0379840135574341 1.0
0 -0.0030747652053833 2.0
0 0.0213698148727417 3.0
0 -0.0395756959915161 4.0
0 0.0235017538070679 5.0
0 0.167866110801697 6.0
0 -0.0398951768875122 7.0
0 -0.039704442024231 8.0
0 -0.101317048072815 9.0
0 -0.00773751735687256 10.0
0 -0.0162851810455322 11.0
0 -0.0763565301895142 12.0
50.210531035 48.7804069519043 13.0
36.662032654 36.919605255127 14.0
0 -0.0946818590164185 15.0
0 -0.024553656578064 16.0
33.256550229 33.574592590332 17.0
0 0.0703710317611694 18.0
0 0.0245813131332397 19.0
0 0.0120712518692017 20.0
0 -0.0182706117630005 21.0
0 -0.0506159067153931 22.0
30.045406534 30.1266193389893 23.0
0 -0.0573245286941528 24.0
0 0.587943136692047 25.0
0 0.0963181257247925 26.0
0 -0.0383901596069336 27.0
0 -0.0414725542068481 28.0
0 -0.0342530012130737 29.0
0 -0.0941640138626099 30.0
0 -0.0604721307754517 31.0
0 0.0592759847640991 32.0
0 -0.00165379047393799 33.0
0 -0.0134862661361694 34.0
0 0.0310451984405518 35.0
0 -0.0304597616195679 36.0
0 -0.0020526647567749 37.0
0 -0.0626662969589233 38.0
0 -0.0283743143081665 39.0
0 0.000147700309753418 40.0
0 -0.0285285711288452 41.0
0 0.00229203701019287 42.0
305.51659408 297.101959228516 43.0
0 -0.00221312046051025 44.0
0 0.0484688282012939 45.0
0 -0.00613534450531006 46.0
0 0.0134247541427612 47.0
0 0.0162149667739868 48.0
0 0.307569742202759 49.0
0.10127332003 0.0572541952133179 50.0
1.5492713777 0.881357371807098 51.0
0 -0.123183846473694 52.0
0 -0.0441385507583618 53.0
0 0.043757438659668 54.0
0 -0.0388680696487427 55.0
0 -0.0466021299362183 56.0
0 0.00599372386932373 57.0
0 -0.0459481477737427 58.0
0 -0.0995622873306274 59.0
0 -0.0337475538253784 60.0
0 0.0386534929275513 61.0
0 0.84857839345932 62.0
251.11943256 255.580184936523 0.0
0 -0.0383988618850708 1.0
0 -0.0206843614578247 2.0
0 -0.0783673524856567 3.0
0 -0.067721962928772 4.0
0 0.0386255979537964 5.0
0 -0.146258473396301 6.0
0 -0.109269738197327 7.0
0 -0.019834041595459 8.0
0 -0.0296093225479126 9.0
0 0.025926947593689 10.0
0 -0.00243759155273438 11.0
0 -0.0658327341079712 12.0
57.3738703 56.3140525817871 13.0
51.459239401 41.4130783081055 14.0
0 -0.0960136651992798 15.0
0 0.0443590879440308 16.0
38.996225241 37.6327438354492 17.0
0 -0.00921738147735596 18.0
0 0.0195075273513794 19.0
0 0.0226999521255493 20.0
0 -0.0428386926651001 21.0
0 0.00392663478851318 22.0
34.594830323 34.5214805603027 23.0
0 -0.0604768991470337 24.0
0 0.664617598056793 25.0
0 0.0596822500228882 26.0
0 0.062616229057312 27.0
0 -0.0676923990249634 28.0
0 -0.0619474649429321 29.0
0 -0.0673549175262451 30.0
0 -0.0549749135971069 31.0
0 0.00474965572357178 32.0
0 -0.0110331773757935 33.0
0 0.025081992149353 34.0
0 -0.0824095010757446 35.0
0 -0.00648868083953857 36.0
0 0.0849038362503052 37.0
0 -0.0132194757461548 38.0
0 0.00485312938690186 39.0
0 -0.121224045753479 40.0
0 -0.0308382511138916 41.0
0 -0.0613590478897095 42.0
287.75546475 297.051605224609 43.0
0 -0.00036776065826416 44.0
0 0.081673800945282 45.0
0 -0.0291320085525513 46.0
0 -0.0129549503326416 47.0
0 -0.0457741022109985 48.0
0 0.276125609874725 49.0
0.079434909767 0.0600560903549194 50.0
1.7852753187 0.930935919284821 51.0
0 -0.00986683368682861 52.0
0 0.0167208909988403 53.0
0 0.0141762495040894 54.0
0 -0.066508412361145 55.0
0 -0.0560086965560913 56.0
0 -0.0547207593917847 57.0
0 -0.0654116868972778 58.0
0 -0.10461699962616 59.0
0 0.030643105506897 60.0
0 0.0324066877365112 61.0
0 0.861297607421875 62.0
242.28158491 247.249069213867 0.0
0 0.054216742515564 1.0
0 -0.0803459882736206 2.0
0 -0.0386086702346802 3.0
0 -0.0553520917892456 4.0
0 0.0704487562179565 5.0
0 0.0864948034286499 6.0
0 -0.0253015756607056 7.0
0 -0.0204437971115112 8.0
0 -0.0680819749832153 9.0
0 -0.0943297147750854 10.0
0 -0.000296473503112793 11.0
0 -0.111980080604553 12.0
59.514924938 59.5399856567383 13.0
32.06230178 40.8734474182129 14.0
0 -0.204371094703674 15.0
0 0.00879037380218506 16.0
25.793300633 24.9912700653076 17.0
0 0.0412954092025757 18.0
0 0.0186280012130737 19.0
0 0.0156251192092896 20.0
0 -0.0351048707962036 21.0
0 -0.0271760225296021 22.0
36.813159216 36.786563873291 23.0
0 0.167842745780945 24.0
0 0.427934050559998 25.0
0 -0.0514274835586548 26.0
0 0.143586277961731 27.0
0 -0.0682693719863892 28.0
0 -0.00612890720367432 29.0
0 -0.192452073097229 30.0
0 0.00729119777679443 31.0
0 -0.0786515474319458 32.0
0 -0.0190719366073608 33.0
0 -0.138676524162292 34.0
0 -0.00986814498901367 35.0
0 0.0218547582626343 36.0
0 -0.0297070741653442 37.0
0 0.0246189832687378 38.0
0 0.0218335390090942 39.0
0 0.0212539434432983 40.0
0 -0.0730746984481812 41.0
0 -0.0648213624954224 42.0
278.08586279 293.890380859375 43.0
0 0.0576761960983276 44.0
0 -0.0398138761520386 45.0
0 -0.0595372915267944 46.0
0 -0.0840610265731812 47.0
0 -0.0567938089370728 48.0
2.666721913 0.218105614185333 49.0
0.073982131637 0.133763194084167 50.0
1.9455021058 0.906672716140747 51.0
0 -0.0460816621780396 52.0
0 -0.0258427858352661 53.0
0 0.00169074535369873 54.0
0 -0.0588840246200562 55.0
0 0.0220233201980591 56.0
0 -0.000148892402648926 57.0
0 -0.0541855096817017 58.0
0 0.018950343132019 59.0
0 -0.0404888391494751 60.0
0 0.00603163242340088 61.0
0 0.80104386806488 62.0
311.87771469 255.148422241211 0.0
0 0.021321177482605 1.0
0 0.01427161693573 2.0
0 -0.0316764116287231 3.0
0 -0.017295241355896 4.0
0 -0.17242968082428 5.0
0 -0.148531556129456 6.0
0 -0.168502449989319 7.0
0 0.00703012943267822 8.0
0 0.0457442998886108 9.0
0 -0.00621545314788818 10.0
0 -0.0774879455566406 11.0
0 -0.195188164710999 12.0
52.283006122 51.3025245666504 13.0
35.526330907 36.2133560180664 14.0
0 -0.0883046388626099 15.0
0 -0.0135828256607056 16.0
31.976197706 31.5752811431885 17.0
0 -0.0224436521530151 18.0
0 0.013337254524231 19.0
0 0.0185052156448364 20.0
0 -0.0498300790786743 21.0
0 0.000325560569763184 22.0
30.535223792 29.6299018859863 23.0
0 0.0559777021408081 24.0
0 0.487795650959015 25.0
0 0.0756863355636597 26.0
0 0.106834053993225 27.0
0 -0.0322209596633911 28.0
0 0.0110474824905396 29.0
0 -0.0950385332107544 30.0
0 -0.085384726524353 31.0
0 0.0128792524337769 32.0
0 -0.0246269702911377 33.0
0 -0.195545077323914 34.0
0 -0.0340467691421509 35.0
0 0.0578480958938599 36.0
0 -0.0695339441299438 37.0
0 0.0100597143173218 38.0
0 0.0400210618972778 39.0
0 0.060205340385437 40.0
0 -0.0451933145523071 41.0
0 -0.0386415719985962 42.0
295.04952334 296.693084716797 43.0
0 0.0154908895492554 44.0
0 0.0762006044387817 45.0
0 0.00733101367950439 46.0
0 -0.0768827199935913 47.0
0 0.0211085081100464 48.0
0 0.22373640537262 49.0
0.093484530431 0.063069224357605 50.0
2.253202797 0.934819102287292 51.0
0 -0.0833498239517212 52.0
0 -0.0827561616897583 53.0
0 0.00927460193634033 54.0
0 -0.106850504875183 55.0
0 -0.00404822826385498 56.0
0 -0.0246902704238892 57.0
0 0.000429749488830566 58.0
0 -0.00631487369537354 59.0
0 -0.0654803514480591 60.0
0 0.00113499164581299 61.0
0 0.861058473587036 62.0
219.31567005 254.439376831055 0.0
0 0.00581300258636475 1.0
0 -0.0518065690994263 2.0
0 -0.0060659646987915 3.0
0 -0.00822842121124268 4.0
0 0.00885188579559326 5.0
0 0.0181196928024292 6.0
0 -0.0588864088058472 7.0
0 0.0386717319488525 8.0
0 0.0214227437973022 9.0
0 0.0427910089492798 10.0
0 0.0860167741775513 11.0
0 0.146528601646423 12.0
70.379393024 69.8405456542969 13.0
34.207273733 29.5416812896729 14.0
0 -0.0155521631240845 15.0
0 0.0166124105453491 16.0
40.474224841 40.5901374816895 17.0
0 0.0260881185531616 18.0
0 -0.0131443738937378 19.0
0 0.122003197669983 20.0
0 0.0634375810623169 21.0
0 -0.108539223670959 22.0
42.523834472 42.0304679870605 23.0
0 -0.0148869752883911 24.0
0 0.460713922977448 25.0
0 0.117008566856384 26.0
0 -0.0125530958175659 27.0
0 -0.0491014719009399 28.0
0 -0.00116050243377686 29.0
0 0.147079348564148 30.0
0 -0.106031060218811 31.0
0 -0.0904475450515747 32.0
0 0.0315831899642944 33.0
0 -0.00558221340179443 34.0
0 0.0231751203536987 35.0
0 0.0687016248703003 36.0
0 -0.0887423753738403 37.0
0 -0.00493180751800537 38.0
0 0.00143539905548096 39.0
0 0.174920916557312 40.0
0 -0.0884159803390503 41.0
0 -0.0538583993911743 42.0
295.88285251 297.256317138672 43.0
0 0.0926040410995483 44.0
0 -0.0770854949951172 45.0
0 -0.0297363996505737 46.0
0 -0.0203016996383667 47.0
0 0.0376437902450562 48.0
0 0.28965100646019 49.0
0.090197974917 0.0645219087600708 50.0
2.1474421058 0.976892411708832 51.0
0 -0.0815564393997192 52.0
0 -0.0796452760696411 53.0
0 0.0276368856430054 54.0
0 -0.0646084547042847 55.0
0 -0.102541327476501 56.0
0 0.0414165258407593 57.0
0 -0.0844632387161255 58.0
0 0.00234520435333252 59.0
0 0.0637325048446655 60.0
0 -0.0605770349502563 61.0
0 0.877133131027222 62.0
220.67157005 259.094940185547 0.0
0 -0.0399261713027954 1.0
0 -0.101155400276184 2.0
0 -0.0482279062271118 3.0
0 0.00346982479095459 4.0
0 -0.0828894376754761 5.0
0 0.0271342992782593 6.0
0 -0.0598968267440796 7.0
0 0.0122308135032654 8.0
0 0.0287164449691772 9.0
0 0.00552773475646973 10.0
0 0.0189310908317566 11.0
0 -0.146349549293518 12.0
73.584740272 73.4171752929688 13.0
21.265505822 23.8398532867432 14.0
0 -0.0511480569839478 15.0
0 0.00239837169647217 16.0
39.070011874 38.8717765808105 17.0
0 -0.0127571821212769 18.0
0 -0.0254017114639282 19.0
0 0.0296984910964966 20.0
0 -0.0380181074142456 21.0
0 0.0103868246078491 22.0
32.967187562 32.4412994384766 23.0
0 -0.0265527963638306 24.0
0 0.662521958351135 25.0
0 -0.0322049856185913 26.0
0 0.01967453956604 27.0
0 0.0281132459640503 28.0
0 -0.033875584602356 29.0
0 -0.0948853492736816 30.0
0 -0.0265306234359741 31.0
0 0.0022815465927124 32.0
0 -0.0321623086929321 33.0
0 -0.0172065496444702 34.0
0 0.0826576948165894 35.0
0 -0.101877808570862 36.0
0 -0.0409194231033325 37.0
0 -0.0158411264419556 38.0
0 0.115864396095276 39.0
0 0.081720232963562 40.0
0 -0.0277261734008789 41.0
0 0.0352033376693726 42.0
306.65425679 295.6943359375 43.0
0 0.0131992101669312 44.0
0 0.0121493339538574 45.0
0 -0.0079425573348999 46.0
0 -0.054533839225769 47.0
0 0.0722688436508179 48.0
0 0.337078601121902 49.0
0.072621856579 0.0763393640518188 50.0
2.005245894 1.02283835411072 51.0
0 -0.0761362314224243 52.0
0 -0.0758925676345825 53.0
0 0.0537325143814087 54.0
0 -0.126760840415955 55.0
0 -0.0545538663864136 56.0
0 0.0132354497909546 57.0
0 -0.0637062788009644 58.0
0 -0.0233751535415649 59.0
0 -0.0648008584976196 60.0
0 0.00380909442901611 61.0
0 0.800063729286194 62.0
238.04106374 259.477020263672 0.0
0 -0.057592511177063 1.0
0 -0.160697817802429 2.0
0 -0.0556668043136597 3.0
0 -0.0446807146072388 4.0
0 0.0782920122146606 5.0
0 0.0729073286056519 6.0
0 -0.00409567356109619 7.0
0 0.0500578880310059 8.0
0 -0.0259519815444946 9.0
0 0.0492430925369263 10.0
0 -0.0340533256530762 11.0
0 0.0402196645736694 12.0
58.689453163 56.3555641174316 13.0
47.01825198 38.4289665222168 14.0
0 -0.111229538917542 15.0
0 -0.041291356086731 16.0
42.515756931 43.5026588439941 17.0
0 0.0992401838302612 18.0
0 0.0680829286575317 19.0
0 0.000434041023254395 20.0
0 -0.0219739675521851 21.0
0 0.0178512334823608 22.0
27.366168107 26.9853439331055 23.0
0 -0.0646849870681763 24.0
0 0.658274173736572 25.0
0 0.0289195775985718 26.0
0 -0.0983390808105469 27.0
0 -0.0724536180496216 28.0
0 -0.0383771657943726 29.0
0 -0.0453259944915771 30.0
0 0.00529301166534424 31.0
0 -0.0260268449783325 32.0
0 0.0304725170135498 33.0
0 -0.0813354253768921 34.0
0 -0.0576196908950806 35.0
0 0.118397355079651 36.0
0 0.0474280118942261 37.0
0 -0.0189377069473267 38.0
0 0.0981849431991577 39.0
0 0.0412520170211792 40.0
0 -0.108462929725647 41.0
0 -0.0214630365371704 42.0
295.99044964 295.633392333984 43.0
0 0.0122946500778198 44.0
0 0.0609999895095825 45.0
0 0.00288331508636475 46.0
0 -0.0585962533950806 47.0
0 0.0171321630477905 48.0
0 0.333367645740509 49.0
0.067907379296 0.0627659559249878 50.0
1.4784783034 1.00724506378174 51.0
0 -0.0153590440750122 52.0
0 -0.0948048830032349 53.0
0 0.0097426176071167 54.0
0 -0.0807667970657349 55.0
0 -0.0587896108627319 56.0
0 -0.0111812353134155 57.0
0 0.0140529870986938 58.0
0 -0.0300835371017456 59.0
0 0.0412794351577759 60.0
0 0.00256335735321045 61.0
0 0.809002041816711 62.0
259.92483504 249.327011108398 0.0
0 -0.0956379175186157 1.0
0 0.0501405000686646 2.0
0 -0.00969183444976807 3.0
0 -0.0538321733474731 4.0
0 -0.195891499519348 5.0
0 0.154744505882263 6.0
0 -0.0127664804458618 7.0
0 -0.0548869371414185 8.0
0 0.107810378074646 9.0
0 -0.0684572458267212 10.0
0 -0.00479447841644287 11.0
0 -0.126353859901428 12.0
47.341977477 46.0104675292969 13.0
50.880785745 44.563777923584 14.0
0 -0.0366379022598267 15.0
0 0.0193833112716675 16.0
34.269138489 34.6965141296387 17.0
0 0.0158883333206177 18.0
0 -0.0216690301895142 19.0
0 0.00832641124725342 20.0
0 -0.0119363069534302 21.0
0 -0.0326048135757446 22.0
31.189564932 30.804500579834 23.0
0 -0.133060097694397 24.0
36.875099932 0.732602775096893 25.0
0 0.0471838712692261 26.0
0 0.0558140873908997 27.0
0 0.089621901512146 28.0
0 -0.034183144569397 29.0
0 -0.0904186964035034 30.0
0 -0.0258628129959106 31.0
0 -0.0190531015396118 32.0
0 -0.122243523597717 33.0
0 0.00358808040618896 34.0
0 0.0482571125030518 35.0
0 -0.000830292701721191 36.0
0 -0.0376518964767456 37.0
0 0.0559805631637573 38.0
0 0.129477858543396 39.0
0 0.00599849224090576 40.0
0 -0.0203433036804199 41.0
0 -0.00820696353912354 42.0
281.15557654 296.975646972656 43.0
0 0.000193476676940918 44.0
0 -0.0103058815002441 45.0
0 -0.00720274448394775 46.0
0 -0.111700415611267 47.0
0 -0.0229545831680298 48.0
0 0.249161183834076 49.0
0.063974411647 0.077940821647644 50.0
2.0972755789 0.946989774703979 51.0
0 -0.0607057809829712 52.0
0 -0.00288355350494385 53.0
0 0.0207537412643433 54.0
0 -0.0956853628158569 55.0
0 0.0164378881454468 56.0
0 -0.0389996767044067 57.0
0 -0.0662730932235718 58.0
0 -0.0905464887619019 59.0
0 -0.0930401086807251 60.0
0 -0.0137845277786255 61.0
0 0.794188857078552 62.0
302.1335631 255.573822021484 0.0
0 -0.133930325508118 1.0
0 0.0416353940963745 2.0
0 -0.0696510076522827 3.0
0 -0.0158027410507202 4.0
0 -0.0440641641616821 5.0
0 -0.132696747779846 6.0
0 -0.00987112522125244 7.0
0 -0.0571359992027283 8.0
0 0.0149539709091187 9.0
0 -0.00514733791351318 10.0
0 -0.0245097875595093 11.0
0 0.0491431951522827 12.0
52.019086093 49.2216415405273 13.0
52.186249782 43.2329292297363 14.0
0 -0.0333162546157837 15.0
0 0.0307191610336304 16.0
28.657082208 27.57639503479 17.0
0 -0.00322568416595459 18.0
0 -0.00675427913665771 19.0
0 -0.0120726823806763 20.0
0 -0.0565382242202759 21.0
0 -0.0136696100234985 22.0
27.74360472 27.2419452667236 23.0
0 0.0645560026168823 24.0
0 0.498844146728516 25.0
0 0.0125468969345093 26.0
0 0.000608086585998535 27.0
0 0.0272549390792847 28.0
0 -0.074273943901062 29.0
0 -0.0796384811401367 30.0
0 -0.0391412973403931 31.0
0 -0.109182000160217 32.0
0 0.0638383626937866 33.0
0 -0.063146710395813 34.0
0 -0.0752211809158325 35.0
0 0.107308506965637 36.0
0 -0.0234285593032837 37.0
0 0.0416041612625122 38.0
0 -0.0489665269851685 39.0
0 -0.0409361124038696 40.0
0 -0.0283132791519165 41.0
0 0.0323337316513062 42.0
300.64079956 297.782470703125 43.0
0 -0.0107895135879517 44.0
0 0.0324742794036865 45.0
0 -0.00592553615570068 46.0
0 -0.053780198097229 47.0
0 0.0548394918441772 48.0
0 0.246150732040405 49.0
0.074691647384 0.119387030601501 50.0
1.6097897844 0.896836280822754 51.0
0 -0.0649241209030151 52.0
0 -0.0882350206375122 53.0
0 -0.009452223777771 54.0
0 -0.0784748792648315 55.0
0 -0.0215667486190796 56.0
0 -0.0298722982406616 57.0
0 -0.00216639041900635 58.0
0 -0.0610843896865845 59.0
0 -0.0266267061233521 60.0
0 0.0219539403915405 61.0
0 0.838551044464111 62.0
239.31525595 260.449829101562 0.0
0 0.0400956869125366 1.0
0 -0.0491002798080444 2.0
0 -0.0464218854904175 3.0
0 -0.00492393970489502 4.0
0 0.0416260957717896 5.0
0 0.0333226919174194 6.0
0 -0.018640398979187 7.0
0 -0.049958348274231 8.0
0 -0.0237985849380493 9.0
0 0.00944161415100098 10.0
0 0.00160163640975952 11.0
0 -0.0230375528335571 12.0
72.656255727 70.6521682739258 13.0
35.667377076 28.4997310638428 14.0
0 -0.0486017465591431 15.0
0 -0.0716220140457153 16.0
32.997486584 33.7045211791992 17.0
0 0.00153267383575439 18.0
0 0.0160266160964966 19.0
0 -0.0243200063705444 20.0
0 -0.0359801054000854 21.0
0 -0.0497852563858032 22.0
37.35378644 37.145580291748 23.0
0 -0.0752550363540649 24.0
0 0.738402962684631 25.0
0 0.0445822477340698 26.0
0 -0.081081748008728 27.0
0 -0.0897089242935181 28.0
0 -0.0533908605575562 29.0
0 0.00939631462097168 30.0
0 -0.0704151391983032 31.0
0 0.0283282995223999 32.0
0 0.0116418600082397 33.0
0 0.00160443782806396 34.0
0 -0.0189124345779419 35.0
0 0.107048630714417 36.0
0 -0.074891209602356 37.0
0 -0.0466245412826538 38.0
0 0.133910059928894 39.0
0 0.0984300374984741 40.0
0 -0.0798027515411377 41.0
0 -0.025363564491272 42.0
302.08609943 295.353454589844 43.0
0 0.0749868154525757 44.0
0 -0.0170414447784424 45.0
0 0.0141319036483765 46.0
0 -0.0517810583114624 47.0
0 -0.0311895608901978 48.0
0 0.310714781284332 49.0
0.094985358033 0.0624023675918579 50.0
2.0550890314 0.956079542636871 51.0
0 -0.0654903650283813 52.0
0 -0.0354815721511841 53.0
0 -0.00595700740814209 54.0
0 -0.0278705358505249 55.0
0 -0.0406979322433472 56.0
0 0.00889575481414795 57.0
0 -0.0563796758651733 58.0
0 -0.0254741907119751 59.0
0 -0.0877863168716431 60.0
0 0.00580751895904541 61.0
0 0.836202144622803 62.0
246.39874901 255.811553955078 0.0
0 -0.0417343378067017 1.0
0 -0.0729535818099976 2.0
0 -0.0234404802322388 3.0
0 0.015499472618103 4.0
0 0.0327895879745483 5.0
0 0.0599907636642456 6.0
0 -0.0552359819412231 7.0
0 -0.0136731863021851 8.0
0 -0.0533777475357056 9.0
0 -0.0255943536758423 10.0
0 -0.0323712825775146 11.0
0 -0.0292593240737915 12.0
71.923019271 72.0624008178711 13.0
17.97802291 29.6273746490479 14.0
0 -0.136205792427063 15.0
0 0.0419687032699585 16.0
38.94124841 38.0954437255859 17.0
0 0.050047755241394 18.0
0 -0.00724494457244873 19.0
0 0.0029376745223999 20.0
0 -0.0331918001174927 21.0
0 -0.0243669748306274 22.0
37.488216216 36.4623146057129 23.0
0 -0.112037777900696 24.0
0 0.771492660045624 25.0
0 -0.0272260904312134 26.0
0 0.093669056892395 27.0
0 -0.0472637414932251 28.0
0 -0.0634726285934448 29.0
0 -0.0832507610321045 30.0
0 -0.0251232385635376 31.0
0 0.0256699323654175 32.0
0 0.0718346834182739 33.0
0 -0.032284140586853 34.0
0 -0.023561954498291 35.0
0 -0.0519737005233765 36.0
0 -0.0246227979660034 37.0
0 -0.0325425863265991 38.0
0 -0.0014268159866333 39.0
0 0.0670403242111206 40.0
0 0.0290426015853882 41.0
0 0.0139755010604858 42.0
286.29513927 294.541259765625 43.0
0 -0.0352426767349243 44.0
0 0.0560939311981201 45.0
0 -0.0278981924057007 46.0
0 -0.0360897779464722 47.0
0 -0.0415817499160767 48.0
0 0.283656001091003 49.0
0.096211713991 0.0644971132278442 50.0
1.769375075 0.912777781486511 51.0
0 -0.0269232988357544 52.0
0 -0.0846444368362427 53.0
0 0.00133383274078369 54.0
0 -0.0453568696975708 55.0
0 -0.0653825998306274 56.0
0 -0.0594328641891479 57.0
0 -0.0457068681716919 58.0
0 -0.0191804170608521 59.0
0 -0.000251173973083496 60.0
0 -0.00840699672698975 61.0
0 0.846750378608704 62.0
253.70460744 256.242309570312 0.0
0 -0.0737901926040649 1.0
0 0.0190995931625366 2.0
0 -0.0735634565353394 3.0
0 -0.00668704509735107 4.0
0 -0.00307047367095947 5.0
0 -0.0201288461685181 6.0
0 -0.0438450574874878 7.0
0 -0.0172052383422852 8.0
0 0.143483519554138 9.0
0 0.000532031059265137 10.0
0 -0.000669479370117188 11.0
0 -0.024936318397522 12.0
64.974404875 64.8232650756836 13.0
38.518314503 31.7737827301025 14.0
0 -0.0115333795547485 15.0
0 -0.0454891920089722 16.0
30.329998196 29.765926361084 17.0
0 -0.0111786127090454 18.0
0 0.0095289945602417 19.0
0 0.0332471132278442 20.0
0 -0.0146747827529907 21.0
0 -0.0918141603469849 22.0
33.12737726 32.311206817627 23.0
0 -0.0527526140213013 24.0
0 0.613101482391357 25.0
0 -0.005226731300354 26.0
0 -0.0172542333602905 27.0
0 0.0447720289230347 28.0
0 -0.0337411165237427 29.0
0 -0.0793538093566895 30.0
0 -0.035204291343689 31.0
0 0.0170153379440308 32.0
0 0.0712043046951294 33.0
0 -0.0643595457077026 34.0
0 0.0159232616424561 35.0
0 0.0509132146835327 36.0
0 -0.0784121751785278 37.0
0 -0.00714790821075439 38.0
0 0.0388051271438599 39.0
0 0.0934861898422241 40.0
0 0.130799531936646 41.0
0 -0.0489345788955688 42.0
303.35380131 295.627716064453 43.0
0 0.044669508934021 44.0
0 -0.0627185106277466 45.0
0 -0.015238881111145 46.0
0 -0.0535722970962524 47.0
0 0.0442417860031128 48.0
0 0.320085048675537 49.0
0.095997323273 0.070393443107605 50.0
2.0867652098 0.96625679731369 51.0
0 -0.0902811288833618 52.0
0 -0.0664397478103638 53.0
0 0.0392557382583618 54.0
0 -0.124258399009705 55.0
0 -0.0458287000656128 56.0
0 0.00212085247039795 57.0
0 -0.0737923383712769 58.0
0 -0.0569645166397095 59.0
0 -0.0713652372360229 60.0
0 0.0294727087020874 61.0
0 0.831875324249268 62.0
232.56218415 249.329498291016 0.0
0 -0.127437710762024 1.0
0 -0.00215590000152588 2.0
0 -0.044918417930603 3.0
0 -0.0539778470993042 4.0
0 -0.0337830781936646 5.0
0 -0.0313082933425903 6.0
0 0.00603640079498291 7.0
0 0.0528476238250732 8.0
0 0.0338376760482788 9.0
0 0.03044593334198 10.0
0 -0.0403499603271484 11.0
0 0.0340989828109741 12.0
66.756231875 65.6515655517578 13.0
15.038318469 31.985164642334 14.0
0 0.0335205793380737 15.0
0 -0.0218688249588013 16.0
42.733016726 43.2809906005859 17.0
0 0.0663870573043823 18.0
0 0.0250746011734009 19.0
0 -0.0324138402938843 20.0
0 0.0841392278671265 21.0
0 0.0232580900192261 22.0
40.884618414 40.7776031494141 23.0
0 -0.0461326837539673 24.0
0 0.533320128917694 25.0
0 0.0357526540756226 26.0
0 0.029328465461731 27.0
0 -0.0452009439468384 28.0
0 -0.0577784776687622 29.0
0 -0.171640634536743 30.0
0 0.0147253274917603 31.0
0 -0.0119329690933228 32.0
0 0.0734316110610962 33.0
0 -0.0853101015090942 34.0
0 -0.0941771268844604 35.0
0 0.0685652494430542 36.0
0 -0.0860735177993774 37.0
0 -0.0268162488937378 38.0
0 0.060580849647522 39.0
0 -0.0521374940872192 40.0
0 -0.0614371299743652 41.0
0 -0.0282245874404907 42.0
278.3503852 293.657806396484 43.0
0 0.0344051122665405 44.0
0 0.02590012550354 45.0
0 -0.0592783689498901 46.0
0 -0.0696481466293335 47.0
0 -0.047488808631897 48.0
2.9809180946 0.248126149177551 49.0
0.080866151121 0.242881178855896 50.0
1.7815458483 0.858181953430176 51.0
0 -0.0786253213882446 52.0
0 -0.0871478319168091 53.0
0 0.0220261812210083 54.0
0 -0.0843905210494995 55.0
0 -0.0929328203201294 56.0
0 -0.000371575355529785 57.0
0 -0.0714846849441528 58.0
0 -0.0730866193771362 59.0
0 0.0195177793502808 60.0
0 0.0478631258010864 61.0
0 0.844792246818542 62.0
255.10287453 254.783645629883 0.0
0 0.0218223333358765 1.0
0 -0.027265191078186 2.0
0 -0.0635544061660767 3.0
0 -0.0823856592178345 4.0
0 0.00143730640411377 5.0
0 -0.0168005228042603 6.0
0 -0.0775052309036255 7.0
0 0.00651192665100098 8.0
0 -0.0514570474624634 9.0
0 0.00792300701141357 10.0
0 -0.0371193885803223 11.0
0 -0.063205361366272 12.0
50.880412886 49.5965232849121 13.0
38.691059923 45.1467742919922 14.0
0 0.0197151899337769 15.0
0 -0.00568163394927979 16.0
34.2039283 34.3825531005859 17.0
0 -0.0325287580490112 18.0
0 0.0112344026565552 19.0
0 0.0295971632003784 20.0
0 -0.011316180229187 21.0
0 -0.0271197557449341 22.0
40.519624266 39.5199699401855 23.0
0 0.0757907629013062 24.0
0 0.425655603408813 25.0
0 0.0451658964157104 26.0
0 -0.0538111925125122 27.0
0 -0.0627485513687134 28.0
0 -0.0829025506973267 29.0
0 -0.0265485048294067 30.0
0 -0.0451763868331909 31.0
0 0.0234988927841187 32.0
0 -0.0294100046157837 33.0
0 -0.028011679649353 34.0
0 0.0606230497360229 35.0
0 0.00435292720794678 36.0
0 -0.0331732034683228 37.0
0 -0.00244724750518799 38.0
0 0.073332667350769 39.0
0 0.0927590131759644 40.0
0 -0.029350757598877 41.0
0 -0.0340219736099243 42.0
288.24456602 297.767120361328 43.0
0 0.0372356176376343 44.0
0 0.000740528106689453 45.0
0 -0.0448907613754272 46.0
0 -0.0562001466751099 47.0
0 0.00952994823455811 48.0
0 0.2727170586586 49.0
0.073702805731 0.08156418800354 50.0
1.5960864304 0.933975517749786 51.0
0 -0.057000994682312 52.0
0 -0.021544337272644 53.0
0 0.0500584840774536 54.0
0 -0.0757404565811157 55.0
0 0.0373333692550659 56.0
0 -0.00372397899627686 57.0
0 -0.123350024223328 58.0
0 -0.059113621711731 59.0
0 0.0383173227310181 60.0
0 -0.0258942842483521 61.0
0 0.825545310974121 62.0
249.46328796 254.599182128906 0.0
0 -0.0876151323318481 1.0
0 0.00407445430755615 2.0
0 0.00807011127471924 3.0
0 -0.0893257856369019 4.0
0 -0.0258899927139282 5.0
0 -0.0261112451553345 6.0
0 -0.0392385721206665 7.0
0 0.071664571762085 8.0
0 0.0986479520797729 9.0
0 -0.051484227180481 10.0
0 0.114564627408981 11.0
0 -0.129261612892151 12.0
75.946328884 74.6154861450195 13.0
3.9468699412 26.3777027130127 14.0
0 -0.0259000062942505 15.0
0 -0.0282896757125854 16.0
30.69990668 30.7156085968018 17.0
0 -0.0378400087356567 18.0
0 -0.0193780660629272 19.0
0 0.0319322347640991 20.0
0 0.0285955667495728 21.0
0 -0.00529873371124268 22.0
26.930817774 26.45361328125 23.0
0 0.0240687131881714 24.0
0 0.597016036510468 25.0
0 0.0628193616867065 26.0
0 0.0278171300888062 27.0
0 0.0876463651657104 28.0
0 -0.0552247762680054 29.0
0 -0.0697294473648071 30.0
0 -0.00654780864715576 31.0
0 -0.004921555519104 32.0
0 0.0994144678115845 33.0
0 -0.0249291658401489 34.0
0 0.11890971660614 35.0
0 -0.0217522382736206 36.0
0 -0.0280169248580933 37.0
0 0.0250676870346069 38.0
0 -0.0427666902542114 39.0
0 0.0550583600997925 40.0
0 -0.0306800603866577 41.0
0 -0.0220037698745728 42.0
291.84882837 293.480346679688 43.0
0 0.0250886678695679 44.0
0 -0.0122621059417725 45.0
0 -0.0469566583633423 46.0
0 -0.0545238256454468 47.0
0 0.0439623594284058 48.0
0 0.364597022533417 49.0
0.090237204057 0.0371302366256714 50.0
1.9243110082 0.936371684074402 51.0
0 -0.0828241109848022 52.0
0 0.0409506559371948 53.0
0 -0.0189856290817261 54.0
0 -0.0277148485183716 55.0
0 -0.0766614675521851 56.0
0 -0.010712742805481 57.0
0 -0.0651448965072632 58.0
0 -0.0310364961624146 59.0
0 0.0536607503890991 60.0
0 0.0568519830703735 61.0
0 0.933684051036835 62.0
256.92590835 254.607696533203 0.0
0 -0.112012505531311 1.0
0 0.0128613710403442 2.0
0 -0.0654033422470093 3.0
0 0.00675475597381592 4.0
0 0.00114095211029053 5.0
0 -0.117507576942444 6.0
0 -0.0516449213027954 7.0
0 -0.0424413681030273 8.0
0 0.0596822500228882 9.0
0 -0.0168980360031128 10.0
0 -0.0157521963119507 11.0
0 0.0392574071884155 12.0
55.223623362 54.6534843444824 13.0
36.956893699 34.9605484008789 14.0
0 -0.0107094049453735 15.0
0 -0.0385178327560425 16.0
28.186577854 28.1838283538818 17.0
0 0.0112863779067993 18.0
0 -0.0123149156570435 19.0
0 0.0222762823104858 20.0
0 -0.00894510746002197 21.0
0 0.0372093915939331 22.0
37.526342396 36.4229583740234 23.0
0 -0.0278850793838501 24.0
0 0.53749805688858 25.0
0 -0.0049208402633667 26.0
0 0.0452752113342285 27.0
0 0.0446032285690308 28.0
0 -0.0470317602157593 29.0
0 0.203822731971741 30.0
0 -0.0132871866226196 31.0
0 -0.0144287347793579 32.0
0 0.114065766334534 33.0
0 -0.054052472114563 34.0
0 -0.0363954305648804 35.0
0 0.0279139280319214 36.0
0 -0.0250755548477173 37.0
0 0.00442564487457275 38.0
0 0.121284604072571 39.0
0 0.0122722387313843 40.0
0 -0.0988730192184448 41.0
0 -0.150232911109924 42.0
303.32930529 295.018768310547 43.0
0 0.0303128957748413 44.0
0 -0.0297261476516724 45.0
0 0.002388596534729 46.0
0 -0.0444544553756714 47.0
0 -0.0116273164749146 48.0
0 0.280811131000519 49.0
0.083098356939 0.0463641881942749 50.0
2.2434371246 0.89533144235611 51.0
0 -0.0580233335494995 52.0
0 -0.00421369075775146 53.0
0 0.00969564914703369 54.0
0 -0.0427230596542358 55.0
0 -0.0898574590682983 56.0
0 0.00749719142913818 57.0
0 -0.0599490404129028 58.0
0 -0.0655244588851929 59.0
0 0.0455416440963745 60.0
0 0.0530649423599243 61.0
0 0.786074042320251 62.0
256.19397997 257.069732666016 0.0
0 -0.0492435693740845 1.0
0 -0.00287497043609619 2.0
0 0.0129338502883911 3.0
0 -0.0217493772506714 4.0
0 0.0288945436477661 5.0
0 0.0213321447372437 6.0
0 -0.0302180051803589 7.0
0 0.0204557180404663 8.0
0 0.0740045309066772 9.0
0 0.032463550567627 10.0
0 -0.0296467542648315 11.0
0 0.0271543264389038 12.0
72.559710283 71.4077987670898 13.0
22.283979656 25.711763381958 14.0
0 -0.0412966012954712 15.0
0 0.0044788122177124 16.0
36.266881649 36.2760047912598 17.0
0 0.019551157951355 18.0
0 -0.0321975946426392 19.0
0 -0.0336891412734985 20.0
0 -0.0239630937576294 21.0
0 0.00882422924041748 22.0
30.190988998 29.346887588501 23.0
0 0.00407636165618896 24.0
0 0.847023487091064 25.0
0 0.0694831609725952 26.0
0 -0.0110530257225037 27.0
0 -0.064949631690979 28.0
0 -0.00656616687774658 29.0
0 -0.00260651111602783 30.0
0 -0.0449391603469849 31.0
0 0.0399950742721558 32.0
0 0.064311146736145 33.0
0 -0.0421441793441772 34.0
0 0.0709980726242065 35.0
0 0.0223609209060669 36.0
0 -0.0314649343490601 37.0
0 -0.0193232297897339 38.0
0 0.0573931932449341 39.0
0 0.00698411464691162 40.0
0 0.0165836811065674 41.0
0 -0.0710979700088501 42.0
298.23558351 295.975036621094 43.0
0 0.00546586513519287 44.0
0 0.0673831701278687 45.0
0 -0.0311506986618042 46.0
0 -0.0267823934555054 47.0
0 0.0282095670700073 48.0
0.35345542079 0.260895252227783 49.0
0.083548187829 0.0665503740310669 50.0
2.1847563736 0.977513134479523 51.0
0 -0.0705453157424927 52.0
0 -0.0675927400588989 53.0
0 0.0345414876937866 54.0
0 -0.068205714225769 55.0
0 -0.078522801399231 56.0
0 0.0163711309432983 57.0
0 -0.0584508180618286 58.0
0 -0.0718444585800171 59.0
0 -0.0600756406784058 60.0
0 -0.0282019376754761 61.0
0 0.851925015449524 62.0
269.30185065 258.017700195312 0.0
0 -0.0261827707290649 1.0
0 -0.0321997404098511 2.0
0 0.00286877155303955 3.0
0 -0.0294855833053589 4.0
0 -0.0647827386856079 5.0
0 0.0264695882797241 6.0
0 -0.095765233039856 7.0
0 -0.0696941614151001 8.0
0 0.0682014226913452 9.0
0 0.00375974178314209 10.0
0 -0.00671720504760742 11.0
0 0.0327914953231812 12.0
52.849592136 51.7840957641602 13.0
52.868304174 44.3376808166504 14.0
0 -0.0230003595352173 15.0
0 -0.0579546689987183 16.0
29.833932817 29.8892459869385 17.0
0 0.00508558750152588 18.0
0 0.0079268217086792 19.0
0 0.0229774713516235 20.0
0 0.00331389904022217 21.0
0 -0.0359793901443481 22.0
39.382427681 39.0761375427246 23.0
0 -0.0098954439163208 24.0
0 0.545250356197357 25.0
0 0.0628856420516968 26.0
0 -0.021344780921936 27.0
0 0.0107120275497437 28.0
0 -0.0344403982162476 29.0
0 -0.0264136791229248 30.0
0 -0.0616828203201294 31.0
0 -0.0125333070755005 32.0
0 0.0540505647659302 33.0
0 -0.0677458047866821 34.0
0 -0.0248326063156128 35.0
0 -0.00628244876861572 36.0
0 -0.0679601430892944 37.0
0 0.0144551992416382 38.0
0 0.0427778959274292 39.0
0 -0.0106021165847778 40.0
0 -0.0608259439468384 41.0
0 -0.0722619295120239 42.0
283.30942408 295.236297607422 43.0
0 0.0494526624679565 44.0
0 0.0332788228988647 45.0
0 -0.00221621990203857 46.0
0 -0.0975750684738159 47.0
0 0.00308215618133545 48.0
0 0.254546225070953 49.0
0.096816779611 0.0665837526321411 50.0
2.0557149422 0.893245160579681 51.0
0 -0.052329421043396 52.0
0 0.0232893228530884 53.0
0 -0.00680744647979736 54.0
0 -0.142085671424866 55.0
0 -0.0181738138198853 56.0
0 -0.02930748462677 57.0
0 -0.0719541311264038 58.0
0 -0.0114609003067017 59.0
0 0.0258182287216187 60.0
0 -0.0150030851364136 61.0
0 0.814446926116943 62.0
265.80910048 257.547973632812 0.0
0 -0.0720750093460083 1.0
0 -0.0661543607711792 2.0
0 -0.0170842409133911 3.0
0 -0.0361379384994507 4.0
0 -0.0772367715835571 5.0
0 -0.0154954195022583 6.0
0 -0.122405409812927 7.0
0 -0.0557949542999268 8.0
0 0.0572942495346069 9.0
0 -0.0416985750198364 10.0
0 -0.00966787338256836 11.0
0 -0.111552834510803 12.0
47.139764741 47.3245048522949 13.0
50.39038906 45.5405197143555 14.0
0 -0.0505138635635376 15.0
0 -0.0528982877731323 16.0
24.60814755 24.0776615142822 17.0
0 0.00836622714996338 18.0
0 0.0776416063308716 19.0
0 0.050171971321106 20.0
0 0.079877495765686 21.0
0 -0.061836838722229 22.0
33.249849595 32.0267219543457 23.0
0 -0.0898133516311646 24.0
0 1.44852566719055 25.0
0 0.104519724845886 26.0
0 0.0355817079544067 27.0
0 0.165810465812683 28.0
0 0.0462409257888794 29.0
0 0.371063828468323 30.0
0 -0.0761350393295288 31.0
0 0.000163435935974121 32.0
0 0.0636342763900757 33.0
0 -0.018208384513855 34.0
0 0.093561053276062 35.0
0 0.0363701581954956 36.0
0 0.00673496723175049 37.0
0 -0.149494051933289 38.0
0 0.0485714673995972 39.0
0 -0.00397169589996338 40.0
0 0.142757058143616 41.0
0 -0.0827394723892212 42.0
299.45936428 296.806762695312 43.0
0 -0.0371938943862915 44.0
0 0.0651246309280396 45.0
0 -0.0205577611923218 46.0
0 -0.034420371055603 47.0
0 0.103657126426697 48.0
1.1341277689 0.290122807025909 49.0
0.091726057961 0.0642043352127075 50.0
1.8212389282 1.02854239940643 51.0
0 -0.112695336341858 52.0
0 -0.108664631843567 53.0
0 -0.0462616682052612 54.0
0 -0.12265408039093 55.0
0 -0.052991509437561 56.0
0 0.0359596014022827 57.0
0 -0.0418878793716431 58.0
0 -0.0633913278579712 59.0
0 -0.0715376138687134 60.0
0 -0.0728746652603149 61.0
0 0.873709917068481 62.0
240.22955994 254.83122253418 0.0
0 -0.0252038240432739 1.0
0 -0.0197049379348755 2.0
0 0.00325882434844971 3.0
0 -0.0620111227035522 4.0
0 0.0369406938552856 5.0
0 -0.0104724168777466 6.0
0 -0.00477516651153564 7.0
0 0.102402806282043 8.0
0 0.0409702062606812 9.0
0 0.0032961368560791 10.0
0 0.0328200459480286 11.0
0 0.0299981832504272 12.0
69.636107372 70.33740234375 13.0
14.307446932 29.1385822296143 14.0
0 0.0317071676254272 15.0
0 -0.026887059211731 16.0
39.270926501 39.1993713378906 17.0
0 0.00456106662750244 18.0
0 0.0207449197769165 19.0
0 -0.0159896612167358 20.0
0 -0.0332933664321899 21.0
0 -0.0368343591690063 22.0
28.931644896 28.0774936676025 23.0
0 0.152470469474792 24.0
0 3.71036195755005 25.0
0 -0.0203167200088501 26.0
0 -0.0140434503555298 27.0
0 0.00110232830047607 28.0
0 -0.0865899324417114 29.0
0 -0.0789287090301514 30.0
0 -0.0346161127090454 31.0
0 -0.0160595178604126 32.0
0 0.012640118598938 33.0
0 -0.0556937456130981 34.0
0 0.0912623405456543 35.0
0 -0.0790838003158569 36.0
0 -0.0274621248245239 37.0
0 0.0531936883926392 38.0
0 -0.0425149202346802 39.0
0 0.149446845054626 40.0
0 -0.0756698846817017 41.0
0 -0.0237118005752563 42.0
308.65474658 296.232574462891 43.0
0 0.0630048513412476 44.0
0 -0.0559746026992798 45.0
0 -0.0741533041000366 46.0
0 -0.0456668138504028 47.0
0 -0.000182271003723145 48.0
0 0.287153482437134 49.0
0.087223187366 0.0730229616165161 50.0
1.8949571688 1.01260352134705 51.0
0 -0.0919197797775269 52.0
0 0.0450726747512817 53.0
0 0.023816704750061 54.0
0 -0.0247422456741333 55.0
0 -0.000742316246032715 56.0
0 -0.00551998615264893 57.0
0 -0.0544980764389038 58.0
0 -6.31809234619141e-06 59.0
0 0.0586975812911987 60.0
0 0.073550820350647 61.0
0 0.904367327690125 62.0
260.35723117 255.641937255859 0.0
0 -0.0532127618789673 1.0
0 -0.0473634004592896 2.0
0 0.0131012201309204 3.0
0 -0.0412200689315796 4.0
0 -0.0693546533584595 5.0
0 -0.0767713785171509 6.0
0 -0.0860387086868286 7.0
0 -0.00617563724517822 8.0
0 0.0173658132553101 9.0
0 -0.0273295640945435 10.0
0 0.0168522596359253 11.0
0 0.0194562673568726 12.0
48.179879009 44.8978157043457 13.0
38.927279617 44.6509666442871 14.0
0 -0.00216448307037354 15.0
0 -0.0477675199508667 16.0
27.590744074 26.7486705780029 17.0
0 0.038655161857605 18.0
0 0.0145941972732544 19.0
0 0.0348352193832397 20.0
0 -0.00871646404266357 21.0
0 -0.0450388193130493 22.0
35.001658069 33.9175643920898 23.0
0 -0.0467945337295532 24.0
0 1.54915797710419 25.0
0 0.00969254970550537 26.0
0 0.0164632797241211 27.0
0 -0.0614038705825806 28.0
0 -0.0154093503952026 29.0
0 -0.0621281862258911 30.0
0 -0.0176092386245728 31.0
0 -0.0354663133621216 32.0
0 0.08250892162323 33.0
0 -0.0583726167678833 34.0
0 -0.0329328775405884 35.0
0 0.0123258829116821 36.0
0 -0.016264796257019 37.0
0 -0.00863134860992432 38.0
0 -0.0601626634597778 39.0
0 0.103989958763123 40.0
0 -0.00910437107086182 41.0
0 0.0501989126205444 42.0
297.93995323 296.591156005859 43.0
0 0.052416205406189 44.0
0 0.0482081174850464 45.0
0 0.0146001577377319 46.0
0 -0.0306211709976196 47.0
0 0.0333510637283325 48.0
0 0.237474918365479 49.0
0.076489205824 0.239457368850708 50.0
2.0169821015 0.869531512260437 51.0
0 -0.0516871213912964 52.0
0 -0.0299938917160034 53.0
0 0.017614483833313 54.0
0 -0.0531436204910278 55.0
0 0.00466692447662354 56.0
0 -0.0296789407730103 57.0
0 -0.0652588605880737 58.0
0 -0.0951241254806519 59.0
0 -0.0153142213821411 60.0
0 0.0580192804336548 61.0
0 0.93643718957901 62.0
247.45731025 254.015670776367 0.0
0 -0.0525165796279907 1.0
0 -0.0334640741348267 2.0
0 0.0464917421340942 3.0
0 -0.0940355062484741 4.0
0 -0.023429274559021 5.0
0 0.0554364919662476 6.0
0 0.0178178548812866 7.0
0 -0.0539929866790771 8.0
0 -0.00951635837554932 9.0
0 0.0415022373199463 10.0
0 0.0291332602500916 11.0
0 0.0247892141342163 12.0
74.734776812 75.1647644042969 13.0
28.251256154 19.1703910827637 14.0
0 0.014726996421814 15.0
0 -0.0120140314102173 16.0
42.244453167 40.9987754821777 17.0
0 0.012424111366272 18.0
0 0.0353156328201294 19.0
0 0.0223764181137085 20.0
0 -0.0354589223861694 21.0
0 -0.0783249139785767 22.0
24.043058012 24.1110553741455 23.0
0 -0.0438286066055298 24.0
0 0.516317963600159 25.0
0 0.030793309211731 26.0
0 0.0404281616210938 27.0
0 0.0168269872665405 28.0
0 -0.0152665376663208 29.0
0 -0.0239583253860474 30.0
0 -0.0141667127609253 31.0
0 0.0133017301559448 32.0
0 0.01282799243927 33.0
0 -0.0497993230819702 34.0
0 -0.148480534553528 35.0
0 -0.0254665613174438 36.0
0 -0.0537523031234741 37.0
0 0.0633984804153442 38.0
0 0.140619158744812 39.0
0 0.146883368492126 40.0
0 -0.0550150871276855 41.0
0 0.0282644033432007 42.0
299.88042464 294.449645996094 43.0
0 -0.121835827827454 44.0
0 0.0502384901046753 45.0
0 -0.00461685657501221 46.0
0 -0.00227916240692139 47.0
0 0.0945781469345093 48.0
0 0.241641461849213 49.0
0.099544338957 0.0551911592483521 50.0
1.6187271156 0.877901077270508 51.0
0 -0.0508238077163696 52.0
0 0.0256928205490112 53.0
0 0.0722280740737915 54.0
0 -0.0945051908493042 55.0
0 -0.0468965768814087 56.0
0 -0.0287629365921021 57.0
0 -0.0457197427749634 58.0
0 -0.11169445514679 59.0
0 0.0282992124557495 60.0
0 0.0247129201889038 61.0
0 0.775083899497986 62.0
246.69066752 251.929580688477 0.0
0 -0.0831543207168579 1.0
0 -0.0586556196212769 2.0
0 0.0147916078567505 3.0
0 -0.0760065317153931 4.0
0 -0.0514291524887085 5.0
0 -0.206655144691467 6.0
0 -0.0111263990402222 7.0
0 -0.0131319761276245 8.0
0 -0.0116106271743774 9.0
0 0.00720345973968506 10.0
0 -0.0366084575653076 11.0
0 -0.159397721290588 12.0
53.966757899 53.0201530456543 13.0
48.433397519 43.4382057189941 14.0
0 -0.0379320383071899 15.0
0 -0.0547713041305542 16.0
31.213370155 31.3898296356201 17.0
0 -0.00734269618988037 18.0
0 0.0178214311599731 19.0
0 0.0248681306838989 20.0
0 -0.0432533025741577 21.0
0 0.0276979207992554 22.0
32.067007033 30.6997566223145 23.0
0 0.0727657079696655 24.0
0 0.477068066596985 25.0
0 0.0597695112228394 26.0
0 -0.0498369932174683 27.0
0 0.0527433156967163 28.0
0 -0.0490235090255737 29.0
0 0.0875864028930664 30.0
0 -0.0619274377822876 31.0
0 0.0315083265304565 32.0
0 -0.187402248382568 33.0
0 -0.0540539026260376 34.0
0 -0.0235456228256226 35.0
0 -0.0207592248916626 36.0
0 -0.0390437841415405 37.0
0 -0.0685943365097046 38.0
0 0.0701054334640503 39.0
0 0.0617350339889526 40.0
0 0.0440443754196167 41.0
0 0.0273164510726929 42.0
292.27003107 296.010864257812 43.0
0 0.00603616237640381 44.0
0 0.0718556642532349 45.0
0 -0.0333043336868286 46.0
0 0.0170787572860718 47.0
0 0.0125530958175659 48.0
0 0.251657128334045 49.0
0.065155186409 0.0773993730545044 50.0
2.3143531112 0.905164659023285 51.0
0 -0.0416005849838257 52.0
0 -0.0225883722305298 53.0
0 -0.0477434396743774 54.0
0 0.0173190832138062 55.0
0 -0.112465262413025 56.0
0 0.00243508815765381 57.0
0 -0.0280734300613403 58.0
0 -0.0198096036911011 59.0
0 -0.0255731344223022 60.0
0 0.0397638082504272 61.0
0 0.797117590904236 62.0
266.99238228 255.761703491211 0.0
0 -0.0342308282852173 1.0
0 -0.14199697971344 2.0
0 -0.00874817371368408 3.0
0 -0.0378841161727905 4.0
0 -0.10932457447052 5.0
0 0.0403741598129272 6.0
0 -0.0541998147964478 7.0
0 0.0384112596511841 8.0
0 -0.0305353403091431 9.0
0 0.00759851932525635 10.0
0 0.00295448303222656 11.0
0 -0.00357401371002197 12.0
58.021837003 57.685962677002 13.0
32.947656746 40.7826728820801 14.0
0 -0.00945389270782471 15.0
0 -0.0818105936050415 16.0
40.787496385 40.2860298156738 17.0
0 0.0172313451766968 18.0
0 0.0436359643936157 19.0
0 0.0179094076156616 20.0
0 -0.029607892036438 21.0
0 -0.0713177919387817 22.0
37.103448925 37.372501373291 23.0
0 0.109063029289246 24.0
0 0.55686354637146 25.0
0 0.0597971677780151 26.0
0 0.000339388847351074 27.0
0 0.0195893049240112 28.0
0 -0.035335898399353 29.0
0 -0.0751296281814575 30.0
0 -0.05540931224823 31.0
0 0.0123647451400757 32.0
0 0.0110288858413696 33.0
0 -0.0825067758560181 34.0
0 -0.00341999530792236 35.0
0 -0.0219656229019165 36.0
0 -0.0763405561447144 37.0
0 0.0450409650802612 38.0
0 0.0404118299484253 39.0
0 0.0591048002243042 40.0
0 -0.026556134223938 41.0
0 0.0253747701644897 42.0
297.93071406 293.722534179688 43.0
0 -0.0205808877944946 44.0
0 -0.0255236625671387 45.0
0 -0.00710070133209229 46.0
0 -0.00661838054656982 47.0
0 0.0328463315963745 48.0
0 0.308587551116943 49.0
0.089120521593 0.0526624917984009 50.0
2.2522056604 0.924654185771942 51.0
0 -0.0794154405593872 52.0
0 -0.0260387659072876 53.0
0 -0.0340162515640259 54.0
0 -0.0635093450546265 55.0
0 -0.104508280754089 56.0
0 0.0206230878829956 57.0
0 -0.0542806386947632 58.0
0 0.00467455387115479 59.0
0 -0.0460585355758667 60.0
0 -0.0381327867507935 61.0
0 0.809791207313538 62.0
234.02942146 254.765869140625 0.0
0 0.0113815069198608 1.0
0 -0.0816749334335327 2.0
0 -0.00438988208770752 3.0
0 0.0132056474685669 4.0
0 -0.0559264421463013 5.0
0 0.0121215581893921 6.0
0 -0.0760201215744019 7.0
0 -0.0900830030441284 8.0
0 -0.0565167665481567 9.0
0 -0.00310671329498291 10.0
0 0.0499986410140991 11.0
0 -0.0167537927627563 12.0
60.630432255 60.129825592041 13.0
39.253370751 36.8469352722168 14.0
0 -1.10864639282227e-05 15.0
0 -0.00550520420074463 16.0
39.092942463 38.355224609375 17.0
0 0.00814259052276611 18.0
0 -0.0269678831100464 19.0
0 0.0607908964157104 20.0
0 -0.0160108804702759 21.0
0 0.0224770307540894 22.0
32.935321283 32.1655654907227 23.0
0 -0.0777784585952759 24.0
0 1.63987195491791 25.0
0 -0.0125409364700317 26.0
0 0.0155137777328491 27.0
0 0.0035322904586792 28.0
0 -0.0426872968673706 29.0
0 -0.0401222705841064 30.0
0 -0.0140267610549927 31.0
0 0.0260370969772339 32.0
0 0.0672227144241333 33.0
0 -0.0381850004196167 34.0
0 -0.00547707080841064 35.0
0 -0.0199738740921021 36.0
0 -0.0247019529342651 37.0
0 0.0261815786361694 38.0
0 -0.0247484445571899 39.0
0 -0.10377562046051 40.0
0 0.0626817941665649 41.0
0 -0.13706362247467 42.0
290.69798825 289.669555664062 43.0
0 -0.00919783115386963 44.0
0 0.064747154712677 45.0
0 -0.0574842691421509 46.0
0 -0.0393010377883911 47.0
0 0.0972198247909546 48.0
4.1611504403 0.363356977701187 49.0
0.10066955157 0.0826724767684937 50.0
1.9154200387 0.889748394489288 51.0
0 -0.0662790536880493 52.0
0 -0.0635339021682739 53.0
0 -0.0341445207595825 54.0
0 -0.0344628095626831 55.0
0 0.0168477296829224 56.0
0 -0.0300217866897583 57.0
0 -0.0685993432998657 58.0
0 -0.0265055894851685 59.0
0 0.002890944480896 60.0
0 -0.0544785261154175 61.0
0 0.837509512901306 62.0
238.63475866 257.614776611328 0.0
0 -0.0343624353408813 1.0
0 -0.0396734476089478 2.0
0 -0.0207034349441528 3.0
0 -0.0196715593338013 4.0
0 -0.0610474348068237 5.0
0 -0.0738083124160767 6.0
0 -0.122975945472717 7.0
0 -0.0795760154724121 8.0
0 -0.00230944156646729 9.0
0 0.0141011476516724 10.0
0 0.0873306393623352 11.0
0 -0.0272966623306274 12.0
71.951354519 71.9957427978516 13.0
19.015484427 25.7732181549072 14.0
0 0.000778555870056152 15.0
0 -0.0129622220993042 16.0
35.376981317 35.1482810974121 17.0
0 -0.0252054929733276 18.0
0 0.00053560733795166 19.0
0 0.0286282300949097 20.0
0 0.00045168399810791 21.0
0 -0.0237103700637817 22.0
33.600307126 33.0226898193359 23.0
0 0.0944732427597046 24.0
0 0.739874482154846 25.0
0 0.00724208354949951 26.0
0 -0.0719060897827148 27.0
0 0.0167850255966187 28.0
0 -0.0199795961380005 29.0
0 -0.0927199125289917 30.0
0 -0.0444461107254028 31.0
0 0.0227521657943726 32.0
0 0.0368896722793579 33.0
0 -0.00277101993560791 34.0
0 -0.0481373071670532 35.0
0 0.0895861387252808 36.0
0 -0.0322607755661011 37.0
0 -0.0572148561477661 38.0
0 0.0158563852310181 39.0
0 0.106491446495056 40.0
0 -0.0313736200332642 41.0
0 -0.0485016107559204 42.0
293.1893796 297.893402099609 43.0
0 0.0623286962509155 44.0
0 -0.0350649356842041 45.0
0 -0.0213464498519897 46.0
0 -0.0384165048599243 47.0
0 -0.0011979341506958 48.0
0 0.300618350505829 49.0
0.07959358186 0.0541592836380005 50.0
2.0187404117 0.907039642333984 51.0
0 -0.0915662050247192 52.0
0 -0.0373407602310181 53.0
0 0.0062328577041626 54.0
0 -0.0750390291213989 55.0
0 -0.0373560190200806 56.0
0 -0.0686343908309937 57.0
0 -0.0617576837539673 58.0
0 -0.0382534265518188 59.0
0 -0.0394836664199829 60.0
0 0.0488771200180054 61.0
0 0.787204504013062 62.0
221.67063151 259.262939453125 0.0
0 -0.0647126436233521 1.0
0 0.0200947523117065 2.0
0 -0.0222300291061401 3.0
0 -0.0152169466018677 4.0
0 0.04981529712677 5.0
0 -0.021060585975647 6.0
0 -0.0386759042739868 7.0
0 -0.0245931148529053 8.0
0 -0.0401102304458618 9.0
0 -0.0129410028457642 10.0
0 -0.0320949554443359 11.0
0 0.0847858190536499 12.0
67.13595327 66.2431716918945 13.0
26.766804939 32.9700546264648 14.0
0 0.00535094738006592 15.0
0 0.0319017171859741 16.0
37.589080986 37.2064208984375 17.0
0 0.00794351100921631 18.0
0 0.00385153293609619 19.0
0 0.0184279680252075 20.0
0 -0.0347248315811157 21.0
0 -0.0606304407119751 22.0
39.765662904 39.199348449707 23.0
0 -0.0471497774124146 24.0
0 0.513598382472992 25.0
0 0.0308722257614136 26.0
0 0.0123320817947388 27.0
0 0.0143011808395386 28.0
0 -0.0354760885238647 29.0
0 -0.0477567911148071 30.0
0 -0.0376569032669067 31.0
0 0.00828349590301514 32.0
0 0.00881493091583252 33.0
0 -0.00395858287811279 34.0
0 -0.0110507011413574 35.0
0 0.116084456443787 36.0
0 -0.0706509351730347 37.0
0 -0.0241295099258423 38.0
0 0.0364207029342651 39.0
0 0.0640586614608765 40.0
0 -0.00568795204162598 41.0
0 -0.0376945734024048 42.0
290.05078228 296.691101074219 43.0
0 0.0149425268173218 44.0
0 0.0199841856956482 45.0
0 -0.0119329690933228 46.0
0 -0.0512810945510864 47.0
0 -0.0361462831497192 48.0
0 0.257522106170654 49.0
0.078756182425 0.125587105751038 50.0
1.8235164658 0.875988841056824 51.0
0 -0.130998730659485 52.0
0 -0.0384713411331177 53.0
0 0.0358222723007202 54.0
0 -0.0620065927505493 55.0
0 -0.0264180898666382 56.0
0 -0.0646928548812866 57.0
0 -0.0675169229507446 58.0
0 -0.075292706489563 59.0
0 0.0180875062942505 60.0
0 -0.0259655714035034 61.0
0 0.817938685417175 62.0
269.60218236 257.397644042969 0.0
0 -0.0864325761795044 1.0
0 0.0292884111404419 2.0
0 -0.0512098073959351 3.0
0 -0.0363980531692505 4.0
0 -0.0244013071060181 5.0
0 -0.0176960229873657 6.0
0 0.0328167676925659 7.0
0 0.0183330774307251 8.0
0 0.0519460439682007 9.0
0 -0.00720417499542236 10.0
0 0.0369399189949036 11.0
0 -0.0411230325698853 12.0
64.710139171 65.1602935791016 13.0
34.017302934 32.5082397460938 14.0
0 -0.0512009859085083 15.0
0 0.00625026226043701 16.0
31.216245485 30.6121520996094 17.0
0 -0.00996577739715576 18.0
0 0.00740087032318115 19.0
0 -0.0231603384017944 20.0
0 -0.024816632270813 21.0
0 -0.0239497423171997 22.0
34.839461676 33.7761192321777 23.0
0 0.0820716619491577 24.0
0 0.571333885192871 25.0
0 -0.0283137559890747 26.0
0 -0.00560009479522705 27.0
0 -0.0187035799026489 28.0
0 -0.0222200155258179 29.0
0 -0.0392348766326904 30.0
0 -0.0204027891159058 31.0
0 -0.00437462329864502 32.0
0 0.000958800315856934 33.0
0 -0.0138505697250366 34.0
0 -0.0740832090377808 35.0
0 0.104894995689392 36.0
0 -0.0983289480209351 37.0
0 0.0375479459762573 38.0
0 0.0466519594192505 39.0
0 0.00776517391204834 40.0
0 -0.0322777032852173 41.0
0 0.0413860082626343 42.0
295.9116752 296.3525390625 43.0
0 -0.0138298273086548 44.0
0 0.0371401309967041 45.0
0 0.00727283954620361 46.0
0 -0.0840855836868286 47.0
0 0.0285905599594116 48.0
0 0.295197784900665 49.0
0.070130376699 0.0448685884475708 50.0
2.093771922 0.867775142192841 51.0
0 -0.0874415636062622 52.0
0 -0.0593563318252563 53.0
0 -0.0219935178756714 54.0
0 -0.0999130010604858 55.0
0 0.0275708436965942 56.0
0 -0.0209780931472778 57.0
0 -0.0229917764663696 58.0
0 -0.0722588300704956 59.0
0 -0.0750702619552612 60.0
0 0.0261088609695435 61.0
0 0.865854859352112 62.0
254.52474018 257.554656982422 0.0
0 -0.077649712562561 1.0
0 0.0173555612564087 2.0
0 0.0160282850265503 3.0
0 -0.0446411371231079 4.0
0 -0.0431782007217407 5.0
0 -0.0146995782852173 6.0
0 -0.119651198387146 7.0
0 -0.0355627536773682 8.0
0 -0.0292764902114868 9.0
0 -0.0316132307052612 10.0
0 -0.0496751070022583 11.0
0 0.0273126363754272 12.0
70.03709014 69.5671768188477 13.0
34.893110645 30.0125751495361 14.0
0 0.0168427228927612 15.0
0 -0.0783313512802124 16.0
36.13946642 36.117561340332 17.0
0 0.0111464262008667 18.0
0 0.0239087343215942 19.0
0 -0.00454986095428467 20.0
0 -0.026835560798645 21.0
0 -0.0464435815811157 22.0
40.064822834 39.8910713195801 23.0
0 -0.116110444068909 24.0
0 1.79581820964813 25.0
0 -0.0183087587356567 26.0
0 -0.0514459609985352 27.0
0 0.0317038297653198 28.0
0 -0.0346173048019409 29.0
0 -0.078218936920166 30.0
0 -0.00434315204620361 31.0
0 0.0222262144088745 32.0
0 0.0721372365951538 33.0
0 0.00266730785369873 34.0
0 -0.058346152305603 35.0
0 0.050566554069519 36.0
0 -0.0890723466873169 37.0
0 0.0590969324111938 38.0
0 0.059378981590271 39.0
0 0.134133219718933 40.0
0 -0.0803326368331909 41.0
0 0.0174564123153687 42.0
285.89570968 295.624298095703 43.0
0 0.0172666311264038 44.0
0 0.0963352918624878 45.0
0 -0.042516827583313 46.0
0 -0.0343209505081177 47.0
0 -0.027225136756897 48.0
1.5905183078 0.31610631942749 49.0
0.068831653867 0.0507687330245972 50.0
2.0531431832 0.95407098531723 51.0
0 -0.0232878923416138 52.0
0 0.0135709047317505 53.0
0 0.0505903959274292 54.0
0 -0.133029580116272 55.0
0 -0.0257934331893921 56.0
0 -0.0334616899490356 57.0
0 -0.052484393119812 58.0
0 -0.0537320375442505 59.0
0 -0.0787137746810913 60.0
0 -0.0188480615615845 61.0
0 0.805809617042542 62.0
253.20000185 256.183135986328 0.0
0 -0.0345340967178345 1.0
0 -0.028344988822937 2.0
0 0.055461049079895 3.0
0 -0.0730503797531128 4.0
0 -0.0570813417434692 5.0
0 -0.0339313745498657 6.0
0 -0.036267876625061 7.0
0 -0.0101997852325439 8.0
0 0.0859140157699585 9.0
0 -0.519813537597656 10.0
0 0.0458968877792358 11.0
0 -0.171096444129944 12.0
78.400270341 76.5929489135742 13.0
30.45134704 23.1457462310791 14.0
0 -0.1304851770401 15.0
0 -0.0273910760879517 16.0
28.155920514 27.4144229888916 17.0
0 0.125816226005554 18.0
0 0.0362883806228638 19.0
0 -0.0287207365036011 20.0
0 -0.0487626791000366 21.0
0 0.0225781202316284 22.0
25.551756648 25.6826343536377 23.0
0 -0.0744200944900513 24.0
0 2.55950498580933 25.0
0 0.0770043134689331 26.0
0 0.0966726541519165 27.0
0 0.106190085411072 28.0
0 -0.105779051780701 29.0
0 -0.0523109436035156 30.0
0 0.00852954387664795 31.0
0 -0.045868992805481 32.0
0 -0.0106897354125977 33.0
0 -0.0224679708480835 34.0
0 -0.0582438707351685 35.0
0 0.118748784065247 36.0
0 0.0270100831985474 37.0
0 0.0791362524032593 38.0
0 -0.0649915933609009 39.0
0 -0.107843518257141 40.0
0 -0.0966898202896118 41.0
0 -0.0255075693130493 42.0
280.91359608 295.732543945312 43.0
0 -0.0469862222671509 44.0
0 -0.0345815420150757 45.0
0 -0.0451186895370483 46.0
0 -0.0386019945144653 47.0
0 -0.00462448596954346 48.0
2.9527220798 0.270916372537613 49.0
0.087160510679 0.0965253114700317 50.0
1.5003556838 0.948094427585602 51.0
0 -0.127816796302795 52.0
0 -0.0454059839248657 53.0
0 0.0415948629379272 54.0
0 -0.0968700647354126 55.0
0 0.0224136114120483 56.0
0 -0.00514018535614014 57.0
0 -0.0831640958786011 58.0
0 -0.0401238203048706 59.0
0 0.19914448261261 60.0
0 -0.0350838899612427 61.0
0 0.838636517524719 62.0
264.47989877 252.938674926758 0.0
0 -0.0296355485916138 1.0
0 -0.000351548194885254 2.0
0 -0.0419658422470093 3.0
0 -0.00427782535552979 4.0
0 -0.0775624513626099 5.0
0 -0.00441038608551025 6.0
0 -0.0534206628799438 7.0
0 -0.0398606061935425 8.0
0 0.0283921957015991 9.0
0 -0.0261796712875366 10.0
0 0.0105470418930054 11.0
0 -0.0748773813247681 12.0
61.562181259 61.0849914550781 13.0
30.04947805 38.4319152832031 14.0
0 -0.0126124620437622 15.0
0 -0.0725597143173218 16.0
29.63005733 29.3421154022217 17.0
0 -0.0411397218704224 18.0
0 0.00671303272247314 19.0
0 0.0138262510299683 20.0
0 -0.0251716375350952 21.0
0 -0.0112015008926392 22.0
31.069100403 30.6732578277588 23.0
0 0.071802020072937 24.0
0 0.521521091461182 25.0
0 0.0732992887496948 26.0
0 0.0902262926101685 27.0
0 -0.0398479700088501 28.0
0 -0.057592511177063 29.0
0 -0.0450495481491089 30.0
0 -0.069548487663269 31.0
0 -0.0299783945083618 32.0
0 0.113800883293152 33.0
0 -0.0114058256149292 34.0
0 0.00716769695281982 35.0
0 0.00262916088104248 36.0
0 -0.0449100732803345 37.0
0 0.0280393362045288 38.0
0 0.0132910013198853 39.0
0 0.033199667930603 40.0
0 -0.0332702398300171 41.0
0 0.0640548467636108 42.0
293.03229757 294.768859863281 43.0
0 -0.0619798898696899 44.0
0 -0.0171034336090088 45.0
0 -0.0357006788253784 46.0
0 -0.0172706842422485 47.0
0 -0.030861496925354 48.0
0 0.292476892471313 49.0
0.082113524514 0.0834153890609741 50.0
2.1334813876 1.01364231109619 51.0
0 -0.0815526247024536 52.0
0 -0.0654140710830688 53.0
0 0.0197421312332153 54.0
0 -0.117718815803528 55.0
0 -0.0613874197006226 56.0
0 -0.00908195972442627 57.0
0 0.0111199617385864 58.0
0 -0.0297621488571167 59.0
0 -0.00380575656890869 60.0
0 0.0493036508560181 61.0
0 0.86655205488205 62.0
248.62109118 256.429595947266 0.0
0 -0.00631392002105713 1.0
0 -0.0258969068527222 2.0
0 -0.0458391904830933 3.0
0 -0.0172170400619507 4.0
0 0.0393856763839722 5.0
0 -0.0128833055496216 6.0
0 -0.0556765794754028 7.0
0 0.00387477874755859 8.0
0 -0.046573281288147 9.0
0 -0.0328866243362427 10.0
0 -0.0571048259735107 11.0
0 -0.0390621423721313 12.0
61.562914844 60.7247123718262 13.0
25.064835695 37.4173469543457 14.0
0 -0.0366874933242798 15.0
0 -0.0174576044082642 16.0
32.794741914 32.5772972106934 17.0
0 -0.0213519334793091 18.0
0 0.009377121925354 19.0
0 0.027096152305603 20.0
0 0.00742566585540771 21.0
0 0.00558364391326904 22.0
29.866216802 28.9607028961182 23.0
0 0.0689986944198608 24.0
0 0.510633647441864 25.0
0 0.0418299436569214 26.0
0 -0.0595791339874268 27.0
0 0.0388234853744507 28.0
0 -0.0885118246078491 29.0
0 0.0162988901138306 30.0
0 -0.0145343542098999 31.0
0 -0.0147548913955688 32.0
0 0.0821706056594849 33.0
0 -0.0309072732925415 34.0
0 0.0240142345428467 35.0
0 0.0484343767166138 36.0
0 -0.0395346879959106 37.0
0 -0.0231064558029175 38.0
0 0.0765756368637085 39.0
0 0.0543087720870972 40.0
0 -0.0646557807922363 41.0
0 -0.066175103187561 42.0
293.00686217 294.215667724609 43.0
0 -0.0695921182632446 44.0
0 0.0196298360824585 45.0
0 0.0272341966629028 46.0
0 -0.0813745260238647 47.0
0 -0.0976377725601196 48.0
0 0.280641555786133 49.0
0.08402808032 0.0788429975509644 50.0
2.1993978314 1.01315426826477 51.0
0 -0.0492881536483765 52.0
0 -0.0340868234634399 53.0
0 0.068076491355896 54.0
0 -0.0536855459213257 55.0
0 -0.0445889234542847 56.0
0 -0.0112963914871216 57.0
0 -0.0610004663467407 58.0
0 -0.0334557294845581 59.0
0 0.0185006856918335 60.0
0 0.0577858686447144 61.0
0 0.879652380943298 62.0
281.61087054 258.860412597656 0.0
0 -0.0289760828018188 1.0
0 -0.0335541963577271 2.0
0 -0.0172442197799683 3.0
0 -0.0704821348190308 4.0
0 0.0162347555160522 5.0
0 0.0622633695602417 6.0
0 -0.0518733263015747 7.0
0 0.0217819213867188 8.0
0 -0.0050283670425415 9.0
0 -0.0330132246017456 10.0
0 -0.00547623634338379 11.0
0 -0.118826508522034 12.0
59.159966396 58.2182426452637 13.0
30.17111157 36.7454948425293 14.0
0 -0.0105959177017212 15.0
0 0.0208307504653931 16.0
30.985182362 30.5867595672607 17.0
0 0.0101832151412964 18.0
0 -0.0224889516830444 19.0
0 0.00146114826202393 20.0
0 -0.047092080116272 21.0
0 0.01378333568573 22.0
36.34464829 35.7570495605469 23.0
0 0.0717309713363647 24.0
0 0.455083250999451 25.0
0 0.0843366384506226 26.0
0 0.0809077024459839 27.0
0 -0.00725090503692627 28.0
0 -0.073121190071106 29.0
0 -0.0969949960708618 30.0
0 -0.0194193124771118 31.0
0 -0.0217844247817993 32.0
0 -0.050740122795105 33.0
0 0.0234626531600952 34.0
0 -0.0500423908233643 35.0
0 0.0416625738143921 36.0
0 -0.00857651233673096 37.0
0 -0.0126041173934937 38.0
0 0.0680822134017944 39.0
0 0.0710023641586304 40.0
0 -0.0507856607437134 41.0
0 0.00242555141448975 42.0
306.70099068 296.679412841797 43.0
0 -0.0552850961685181 44.0
0 0.0162911415100098 45.0
0 0.00406396389007568 46.0
0 -0.0193525552749634 47.0
0 0.0161656141281128 48.0
0 0.274179697036743 49.0
0.069129266665 0.0866416692733765 50.0
1.8001557747 0.936575710773468 51.0
0 -0.0654054880142212 52.0
0 -0.0766264200210571 53.0
0 0.0455137491226196 54.0
0 -0.0734101533889771 55.0
0 0.00820028781890869 56.0
0 -0.0374373197555542 57.0
0 -0.038622260093689 58.0
0 -0.0260595083236694 59.0
0 -0.0395220518112183 60.0
0 -0.0233777761459351 61.0
0 0.821455955505371 62.0
245.42931705 256.743560791016 0.0
0 -3.68356704711914e-05 1.0
0 0.0136376619338989 2.0
0 -0.0281323194503784 3.0
0 -0.0243998765945435 4.0
0 -0.0316475629806519 5.0
0 0.0181573629379272 6.0
0 -0.0896009206771851 7.0
0 -0.0378415584564209 8.0
0 -0.0473657846450806 9.0
0 -0.0171850919723511 10.0
0 -0.0287880897521973 11.0
0 0.0324834585189819 12.0
54.965823083 53.6785888671875 13.0
35.444878905 41.3319664001465 14.0
0 -0.0359393358230591 15.0
0 -0.0430845022201538 16.0
30.791429567 30.0089797973633 17.0
0 0.0285028219223022 18.0
0 0.0233453512191772 19.0
0 0.00279223918914795 20.0
0 -0.0289705991744995 21.0
0 -0.0222011804580688 22.0
31.39693113 30.7862377166748 23.0
0 0.0422185659408569 24.0
0 0.540239989757538 25.0
0 -0.0232974290847778 26.0
0 -0.0147988796234131 27.0
0 -0.0604149103164673 28.0
0 -0.0383003950119019 29.0
0 -0.0858801603317261 30.0
0 0.00489151477813721 31.0
0 0.0219238996505737 32.0
0 -0.0183831453323364 33.0
0 -0.0119336843490601 34.0
0 -0.0357794761657715 35.0
0 -0.0266047716140747 36.0
0 -0.0864938497543335 37.0
0 -0.0264993906021118 38.0
0 0.0103286504745483 39.0
0 -0.0265012979507446 40.0
0 -0.0636099576950073 41.0
0 0.003257155418396 42.0
301.41947472 297.089935302734 43.0
0 0.0125468969345093 44.0
0 -0.0458837747573853 45.0
0 -0.0204216241836548 46.0
0 -0.0148266553878784 47.0
0 0.032373309135437 48.0
0 0.27493280172348 49.0
0.086372717601 0.0674494504928589 50.0
1.9782131787 0.942559003829956 51.0
0 -0.0236665010452271 52.0
0 -0.0566340684890747 53.0
0 -0.0225104093551636 54.0
0 -0.0395327806472778 55.0
0 -0.072618842124939 56.0
0 0.0114270448684692 57.0
0 -0.0594590902328491 58.0
0 -0.0888320207595825 59.0
0 -0.0314928293228149 60.0
0 0.0507355928421021 61.0
0 0.801658987998962 62.0
247.30722208 255.382583618164 0.0
0 -0.0657192468643188 1.0
0 -0.0936750173568726 2.0
0 -0.0130296945571899 3.0
0 -0.0321933031082153 4.0
0 -0.00579655170440674 5.0
0 -0.0976558923721313 6.0
0 -0.0157822370529175 7.0
0 0.00214612483978271 8.0
0 0.0117944478988647 9.0
0 -0.0323499441146851 10.0
0 -0.05100417137146 11.0
0 -0.0623623132705688 12.0
51.76171997 51.6031150817871 13.0
38.01176415 42.4973297119141 14.0
0 -0.00107443332672119 15.0
0 -0.022701621055603 16.0
33.669348058 33.3848686218262 17.0
0 0.00576412677764893 18.0
0 0.0100988149642944 19.0
0 0.02808678150177 20.0
0 -0.011609673500061 21.0
0 0.0293115377426147 22.0
31.110092225 30.5400142669678 23.0
0 -0.0293980836868286 24.0
0 0.571751534938812 25.0
0 0.0124377012252808 26.0
0 -0.0203977823257446 27.0
0 -0.00573408603668213 28.0
0 -0.0613759756088257 29.0
0 -0.0111626386642456 30.0
0 -0.030469536781311 31.0
0 0.0106595754623413 32.0
0 0.0489245653152466 33.0
0 0.00158131122589111 34.0
0 -0.0135531425476074 35.0
0 0.0254043340682983 36.0
0 -0.0922437906265259 37.0
0 -0.0355831384658813 38.0
0 0.0731667280197144 39.0
0 -0.0170718431472778 40.0
0 -0.0399321317672729 41.0
0 0.0817102193832397 42.0
300.2608113 296.914276123047 43.0
0 -0.0391727685928345 44.0
0 0.0663763284683228 45.0
0 -0.0127147436141968 46.0
0 -0.0427745580673218 47.0
0 -0.0196195840835571 48.0
0 0.308420062065125 49.0
0.072681129229 0.0690988302230835 50.0
1.955695114 0.904031574726105 51.0
0 -0.0818740129470825 52.0
0 -0.0273014307022095 53.0
0 0.0503472089767456 54.0
0 -0.0413318872451782 55.0
0 -0.0323828458786011 56.0
0 -0.0338388681411743 57.0
0 -0.0657466650009155 58.0
0 -0.0190151929855347 59.0
0 -0.0936208963394165 60.0
0 0.0687583684921265 61.0
0 0.893003165721893 62.0
276.19814493 257.785278320312 0.0
0 -0.105660080909729 1.0
0 0.0166829824447632 2.0
0 -0.027483344078064 3.0
0 -0.0303832292556763 4.0
0 -0.0605274438858032 5.0
0 -0.0104490518569946 6.0
0 -0.0165344476699829 7.0
0 -0.0428628921508789 8.0
0 0.122971892356873 9.0
0 0.00490152835845947 10.0
0 -0.0211089253425598 11.0
0 0.0425437688827515 12.0
62.047479561 61.0357398986816 13.0
37.360148293 35.8150367736816 14.0
0 -0.0525232553482056 15.0
0 0.011034369468689 16.0
32.145155216 31.7075157165527 17.0
0 -0.0188759565353394 18.0
0 -0.0321422815322876 19.0
0 -0.0405775308609009 20.0
0 -0.00861847400665283 21.0
0 -0.0708409547805786 22.0
30.360410063 29.5978984832764 23.0
0 -0.0292054414749146 24.0
0 0.590352892875671 25.0
0 0.115601420402527 26.0
0 0.033990740776062 27.0
0 -0.0338269472122192 28.0
0 -0.0731204748153687 29.0
0 -0.08785080909729 30.0
0 -0.0274564027786255 31.0
0 0.00920188426971436 32.0
0 0.0996612310409546 33.0
0 -0.046918511390686 34.0
0 -0.075334906578064 35.0
0 0.026636004447937 36.0
0 0.0384918451309204 37.0
0 0.0187145471572876 38.0
0 0.0689772367477417 39.0
0 0.0728973150253296 40.0
0 -0.0385105609893799 41.0
0 0.0680736303329468 42.0
293.27436776 294.494812011719 43.0
0 -0.0279756784439087 44.0
0 -0.00683259963989258 45.0
0 -8.34465026855469e-07 46.0
0 -0.0349808931350708 47.0
0 0.0374187231063843 48.0
0.17418863458 0.262979805469513 49.0
0.098072052474 0.0627270936965942 50.0
1.7359085401 0.883898973464966 51.0
0 -0.0508285760879517 52.0
0 -0.0292679071426392 53.0
0 0.0477689504623413 54.0
0 -0.0680757761001587 55.0
0 -0.0617219209671021 56.0
0 0.004660964012146 57.0
0 0.0306893587112427 58.0
0 -0.0624333620071411 59.0
0 0.109198451042175 60.0
0 -0.0298091173171997 61.0
0 0.833971977233887 62.0
210.08432496 217.90055847168 0.0
0 0.000171065330505371 1.0
0 0.0315054655075073 2.0
0 -0.0214251279830933 3.0
0 0.018785834312439 4.0
0 0.117320895195007 5.0
0 0.00492227077484131 6.0
0 -0.0204993486404419 7.0
0 -0.050147533416748 8.0
0 -0.0376583337783813 9.0
0 0.0831190347671509 10.0
0 0.0571930706501007 11.0
0 0.0175861120223999 12.0
73.376851198 71.9848403930664 13.0
32.887369122 28.8371067047119 14.0
0 -0.041364312171936 15.0
0 0.0505112409591675 16.0
29.330682953 30.2528476715088 17.0
0 -0.0169600248336792 18.0
0 0.0270687341690063 19.0
0 0.00449669361114502 20.0
0 -0.0421205759048462 21.0
0 -0.0758582353591919 22.0
42.076584507 42.0703086853027 23.0
0 -0.117456555366516 24.0
0 0.554310739040375 25.0
0 0.0391196012496948 26.0
0 0.0467270016670227 27.0
0 0.0990198850631714 28.0
0 -0.0295978784561157 29.0
0 -0.0931763648986816 30.0
0 -0.0216261148452759 31.0
0 0.12417733669281 32.0
0 -0.0047379732131958 33.0
0 -0.0294691324234009 34.0
0 0.0879148840904236 35.0
0 -0.103678822517395 36.0
0 -0.00248420238494873 37.0
0 -0.0483661890029907 38.0
0 -0.0246182680130005 39.0
0 0.128212332725525 40.0
0 -0.0472925901412964 41.0
0 -0.0830274820327759 42.0
298.38481952 295.071838378906 43.0
0 0.00282943248748779 44.0
0 0.0767039060592651 45.0
0 -0.00997292995452881 46.0
0 0.0834749937057495 47.0
0 -0.0268121957778931 48.0
0 0.336628913879395 49.0
0.077927456927 0.0672320127487183 50.0
1.3317238482 0.918374121189117 51.0
0 -0.0679823160171509 52.0
0 -0.0750833749771118 53.0
0 0.0216044187545776 54.0
0 -0.0168522596359253 55.0
0 -0.0496228933334351 56.0
0 -0.0376435518264771 57.0
0 -0.0344327688217163 58.0
0 -0.0229424238204956 59.0
0 0.0951532125473022 60.0
0 -0.0272549390792847 61.0
0 0.820741415023804 62.0
264.63435244 258.518432617188 0.0
0 -0.0499249696731567 1.0
0 -0.0968478918075562 2.0
0 -0.023466944694519 3.0
0 -0.0392247438430786 4.0
0 -0.0867384672164917 5.0
0 -0.0287777185440063 6.0
0 -0.0432335138320923 7.0
0 0.0075531005859375 8.0
0 0.0504134893417358 9.0
0 0.00467431545257568 10.0
0 -0.0511397123336792 11.0
0 -0.134331345558167 12.0
51.7203565 51.0412940979004 13.0
44.075659543 43.7045478820801 14.0
0 -0.0545235872268677 15.0
0 0.00826680660247803 16.0
39.335482198 37.984073638916 17.0
0 -0.0327681303024292 18.0
0 0.00544726848602295 19.0
0 -0.0160869359970093 20.0
0 0.00513851642608643 21.0
0 -0.0514150857925415 22.0
31.898294025 30.8760852813721 23.0
0 0.0872300863265991 24.0
0 0.424474596977234 25.0
0 0.0367954969406128 26.0
0 -0.0374091863632202 27.0
0 -0.0865098237991333 28.0
0 -0.062442421913147 29.0
0 -0.0484164953231812 30.0
0 -0.0217012166976929 31.0
0 -0.007698655128479 32.0
0 -0.00939023494720459 33.0
0 -0.0311499834060669 34.0
0 -0.035831093788147 35.0
0 0.0805600881576538 36.0
0 -0.0192514657974243 37.0
0 -0.000450968742370605 38.0
0 0.0493589639663696 39.0
0 0.03258216381073 40.0
0 -0.0222996473312378 41.0
0 0.001120924949646 42.0
297.74596701 297.722808837891 43.0
0 -0.0516229867935181 44.0
0 -0.0535831451416016 45.0
0 -0.0170584917068481 46.0
0 0.0177534818649292 47.0
0 -0.0235401391983032 48.0
0 0.285350739955902 49.0
0.083566976261 0.0630170106887817 50.0
1.9863997017 0.927021026611328 51.0
0 -0.0802901983261108 52.0
0 -0.0301426649093628 53.0
0 0.0292068719863892 54.0
0 -0.12978732585907 55.0
0 0.00426375865936279 56.0
0 -0.0478535890579224 57.0
0 -0.0665055513381958 58.0
0 -0.0431228876113892 59.0
0 -0.0297578573226929 60.0
0 0.0610264539718628 61.0
0 0.832905769348145 62.0
237.77470758 248.711608886719 0.0
0 -0.130511403083801 1.0
0 0.0213888883590698 2.0
0 -0.0397292375564575 3.0
0 -0.0101455450057983 4.0
0 -0.024065375328064 5.0
0 0.0225766897201538 6.0
0 -0.127757668495178 7.0
0 0.0372695922851562 8.0
0 -0.00260889530181885 9.0
0 0.00221860408782959 10.0
0 -0.030733585357666 11.0
0 -0.0818830728530884 12.0
75.145997498 73.7209854125977 13.0
23.448961281 24.01025390625 14.0
0 -0.0520650148391724 15.0
0 -0.0439480543136597 16.0
30.32079587 30.6903171539307 17.0
0 -0.0469478368759155 18.0
0 0.0337783098220825 19.0
0 0.0254195928573608 20.0
0 -0.0235825777053833 21.0
0 -0.0759221315383911 22.0
41.2078355 42.1311073303223 23.0
0 -0.0773302316665649 24.0
0 2.77037048339844 25.0
0 0.0300201177597046 26.0
0 0.104712128639221 27.0
0 0.26032292842865 28.0
0 -0.0770982503890991 29.0
0 -0.0433437824249268 30.0
0 -0.0334481000900269 31.0
0 0.048850417137146 32.0
0 -0.00372278690338135 33.0
0 -0.0945991277694702 34.0
0 0.0321133136749268 35.0
0 0.113549828529358 36.0
0 0.144861102104187 37.0
0 0.0210155248641968 38.0
0 -0.0670491456985474 39.0
0 0.0742214918136597 40.0
0 0.017656683921814 41.0
0 -0.161612153053284 42.0
277.78384492 297.508850097656 43.0
0 -0.0235153436660767 44.0
0 0.094325065612793 45.0
0 -0.013628363609314 46.0
0 -0.0472155809402466 47.0
0 -0.0569306612014771 48.0
0 0.295494258403778 49.0
0.098542726474 0.0428928136825562 50.0
1.3397887989 0.812048554420471 51.0
0 -0.105818867683411 52.0
0 -0.0381442308425903 53.0
0 -0.00542294979095459 54.0
0 -0.0110670328140259 55.0
0 -0.0990315675735474 56.0
0 0.0121454000473022 57.0
0 -0.0785809755325317 58.0
0 -0.0585297346115112 59.0
0 0.0732969045639038 60.0
0 -0.0405460596084595 61.0
0 0.881135225296021 62.0
261.69516383 260.667388916016 0.0
0 -0.0508195161819458 1.0
0 -0.0658916234970093 2.0
0 -0.0183671712875366 3.0
0 -0.0323666334152222 4.0
0 0.0243920087814331 5.0
0 -0.0168243646621704 6.0
0 0.0175231695175171 7.0
0 0.000954985618591309 8.0
0 -0.0402504205703735 9.0
0 0.0224143266677856 10.0
0 0.0418887734413147 11.0
0 -0.0355325937271118 12.0
75.495362196 75.6262969970703 13.0
12.579817416 22.0568561553955 14.0
0 0.0298885107040405 15.0
0 -0.0371366739273071 16.0
29.229829856 29.3421096801758 17.0
0 -0.0299099683761597 18.0
0 -0.0120681524276733 19.0
0 0.0474718809127808 20.0
0 -0.0183156728744507 21.0
0 0.00110280513763428 22.0
27.969855091 27.2317752838135 23.0
0 0.0901082754135132 24.0
0 0.49133563041687 25.0
0 0.0987395048141479 26.0
0 -0.0933517217636108 27.0
0 -0.027962327003479 28.0
0 -0.125846028327942 29.0
0 0.0517743825912476 30.0
0 -0.072783350944519 31.0
0 0.0194672346115112 32.0
0 0.0935550928115845 33.0
0 -0.0166662931442261 34.0
0 0.0126168727874756 35.0
0 -0.0388127565383911 36.0
0 -0.0973566770553589 37.0
0 0.0131114721298218 38.0
0 0.0525268316268921 39.0
0 0.00174939632415771 40.0
0 -0.0330387353897095 41.0
0 -0.0329815149307251 42.0
292.41731407 295.732238769531 43.0
0 0.0465067625045776 44.0
0 -0.0181469917297363 45.0
0 0.0124586820602417 46.0
0 -0.0207667350769043 47.0
0 -0.0723778009414673 48.0
0 0.269733965396881 49.0
0.070407563472 0.0798641443252563 50.0
2.1042127785 0.955182194709778 51.0
0 -0.0592612028121948 52.0
0 -0.0529142618179321 53.0
0 -0.00343477725982666 54.0
0 -0.0304595232009888 55.0
0 -0.0457676649093628 56.0
0 0.0131820440292358 57.0
0 -0.0298851728439331 58.0
0 -0.0587233304977417 59.0
0 -0.0534669160842896 60.0
0 -0.0350333452224731 61.0
0 0.837423920631409 62.0
251.95961803 257.535827636719 0.0
0 -0.0619851350784302 1.0
0 0.0294097661972046 2.0
0 -0.0140923261642456 3.0
0 -0.0427318811416626 4.0
0 -0.0426641702651978 5.0
0 -0.030708909034729 6.0
0 -0.0760997533798218 7.0
0 -0.0468730926513672 8.0
0 -0.099673867225647 9.0
0 0.0253769159317017 10.0
0 0.0187886953353882 11.0
0 -0.0464760065078735 12.0
70.956213442 71.1253890991211 13.0
28.192665005 26.8727798461914 14.0
0 -0.0859400033950806 15.0
0 -0.0374492406845093 16.0
25.295494207 23.8909645080566 17.0
0 0.00528824329376221 18.0
0 0.00443828105926514 19.0
0 -0.0441855192184448 20.0
0 0.0123296976089478 21.0
0 -0.0463486909866333 22.0
42.260446544 41.7356643676758 23.0
0 -0.118005394935608 24.0
0 0.67157781124115 25.0
0 0.0702899694442749 26.0
0 0.0689078569412231 27.0
0 -0.0641552209854126 28.0
0 -0.0855535268783569 29.0
0 -0.0732804536819458 30.0
0 -0.0378192663192749 31.0
0 -0.151420712471008 32.0
0 0.040213942527771 33.0
0 -0.00503838062286377 34.0
0 -0.0422521829605103 35.0
0 -0.101526856422424 36.0
0 -0.0958882570266724 37.0
0 -0.0488468408584595 38.0
0 0.0186556577682495 39.0
0 0.123545527458191 40.0
0 -0.0472675561904907 41.0
0 -0.0226341485977173 42.0
294.51108639 296.640777587891 43.0
0 -0.0586792230606079 44.0
0 -0.00764924287796021 45.0
0 -0.015184760093689 46.0
0 -0.00742256641387939 47.0
0 -0.0459333658218384 48.0
0 0.26500391960144 49.0
0.094241441293 0.11860191822052 50.0
2.1252169638 0.852744579315186 51.0
0 -0.0657805204391479 52.0
0 -0.0409213304519653 53.0
0 0.0406404733657837 54.0
0 -0.0244959592819214 55.0
0 0.0234012603759766 56.0
0 -0.0268906354904175 57.0
0 -0.0518509149551392 58.0
0 -0.0893343687057495 59.0
0 -0.0765410661697388 60.0
0 0.0770272016525269 61.0
0 0.910463929176331 62.0
226.65796938 254.561569213867 0.0
0 0.00478065013885498 1.0
0 -0.0252367258071899 2.0
0 0.0169159173965454 3.0
0 -0.0165761709213257 4.0
0 -0.00450122356414795 5.0
0 0.16317880153656 6.0
0 -0.0608519315719604 7.0
0 -0.161543607711792 8.0
0 0.0882514715194702 9.0
0 0.0283557176589966 10.0
0 -0.049167275428772 11.0
0 -0.00810015201568604 12.0
63.560308703 63.2857551574707 13.0
42.924527383 35.7150802612305 14.0
0 -0.0145455598831177 15.0
0 -0.0841497182846069 16.0
31.009641071 30.6495666503906 17.0
0 -0.0228918790817261 18.0
0 0.0259281396865845 19.0
0 -0.0424376726150513 20.0
0 0.0153020620346069 21.0
0 -0.0240164995193481 22.0
38.335664262 38.1045417785645 23.0
0 -0.00813114643096924 24.0
0 0.535970032215118 25.0
0 0.0295184850692749 26.0
0 0.0453945398330688 27.0
0 -0.0496746301651001 28.0
0 -0.0549062490463257 29.0
0 -0.0569761991500854 30.0
0 -0.01856529712677 31.0
0 -0.0131970643997192 32.0
0 -0.0179067850112915 33.0
0 -0.0291463136672974 34.0
0 -0.0451942682266235 35.0
0 0.030610203742981 36.0
0 -0.0299664735794067 37.0
0 0.0113943815231323 38.0
0 0.0234500169754028 39.0
0 -0.0797730684280396 40.0
0 -0.0694006681442261 41.0
0 0.117100119590759 42.0
288.52182452 293.834991455078 43.0
0 -0.0274816751480103 44.0
0 0.0242382287979126 45.0
0 -0.0613845586776733 46.0
0 0.0309182405471802 47.0
0 0.00108802318572998 48.0
0.26651186107 0.26366138458252 49.0
0.093873115245 0.090961217880249 50.0
2.1969197565 0.87056028842926 51.0
0 -0.0635777711868286 52.0
0 0.000519156455993652 53.0
0 0.00495707988739014 54.0
0 -0.110156655311584 55.0
0 -0.0619779825210571 56.0
0 -0.0133754014968872 57.0
0 -0.0567334890365601 58.0
0 -0.00920712947845459 59.0
0 -0.0195623636245728 60.0
0 0.0234755277633667 61.0
0 0.861513257026672 62.0
237.51783235 252.085296630859 0.0
0 -0.0385137796401978 1.0
0 -0.0119067430496216 2.0
0 -0.0160053968429565 3.0
0 -0.0533050298690796 4.0
0 0.0680817365646362 5.0
0 0.000258803367614746 6.0
0 -0.0547972917556763 7.0
0 -0.0585743188858032 8.0
0 -0.0891252756118774 9.0
0 0.0273834466934204 10.0
0 0.0163393020629883 11.0
0 0.0504382848739624 12.0
64.927611907 63.9392623901367 13.0
36.307376942 28.6597442626953 14.0
0 -0.00533449649810791 15.0
0 0.0423046350479126 16.0
25.120222883 24.6484642028809 17.0
0 -0.0471600294113159 18.0
0 -0.0371824502944946 19.0
0 0.019677996635437 20.0
0 -0.0543285608291626 21.0
0 -0.0530005693435669 22.0
40.689263114 40.3899421691895 23.0
0 0.0664037466049194 24.0
0 0.470597624778748 25.0
0 0.0549010038375854 26.0
0 -0.0119296908378601 27.0
0 -0.0744277238845825 28.0
0 -0.0256956815719604 29.0
0 0.0926285982131958 30.0
0 -0.0199660062789917 31.0
0 -0.100058674812317 32.0
0 0.0276685953140259 33.0
0 -0.0738984346389771 34.0
0 0.583624958992004 35.0
0 0.10064709186554 36.0
0 -0.0139397382736206 37.0
0 0.0642162561416626 38.0
0 0.00734078884124756 39.0
0 0.100168585777283 40.0
0 -0.0877656936645508 41.0
0 -0.218271851539612 42.0
295.37218523 295.280395507812 43.0
0 0.0139659643173218 44.0
0 0.122646808624268 45.0
0 -0.0018993616104126 46.0
0 -0.0708879232406616 47.0
0 0.112907767295837 48.0
2.6238521576 0.31345397233963 49.0
0.082208944562 0.0559440851211548 50.0
2.3178064489 0.95858633518219 51.0
0 -0.024657130241394 52.0
0 -0.0260745286941528 53.0
0 0.00179016590118408 54.0
0 -0.0942298173904419 55.0
0 -0.0848345756530762 56.0
0 -0.0297142267227173 57.0
0 -0.109580874443054 58.0
0 -0.058651328086853 59.0
0 0.0918530225753784 60.0
0 0.00482308864593506 61.0
0 0.791584849357605 62.0
232.93067687 256.897521972656 0.0
0 -0.0890814065933228 1.0
0 0.0383092164993286 2.0
0 -0.0449172258377075 3.0
0 -0.0197254419326782 4.0
0 -0.0459507703781128 5.0
0 -0.0356117486953735 6.0
0 -0.0583635568618774 7.0
0 -0.0354825258255005 8.0
0 0.0285223722457886 9.0
0 -0.045491099357605 10.0
0 -0.0453245639801025 11.0
0 -0.0199772119522095 12.0
67.076604449 66.557258605957 13.0
27.355363573 32.2348442077637 14.0
0 -0.0263925790786743 15.0
0 -0.0671578645706177 16.0
27.127755128 26.0665836334229 17.0
0 0.0337213277816772 18.0
0 0.0349417924880981 19.0
0 -0.0293229818344116 20.0
0 0.0019453763961792 21.0
0 -0.0195413827896118 22.0
38.906369125 38.9374351501465 23.0
0 0.084072470664978 24.0
0 0.438959836959839 25.0
0 0.000178217887878418 26.0
0 0.0275173783302307 27.0
0 0.0825661420822144 28.0
0 -0.00591170787811279 29.0
0 -0.0769684314727783 30.0
0 -0.00774800777435303 31.0
0 -0.00568687915802002 32.0
0 -0.00454676151275635 33.0
0 -0.026373028755188 34.0
0 -0.0514920949935913 35.0
0 0.0714861154556274 36.0
0 -0.0500141382217407 37.0
0 0.0138336420059204 38.0
0 0.0564945936203003 39.0
0 0.0513986349105835 40.0
0 -0.0672439336776733 41.0
0 0.0334485769271851 42.0
293.22306812 297.230163574219 43.0
0 0.0666273832321167 44.0
0 -0.0536313056945801 45.0
0 -0.0422474145889282 46.0
0 -0.011741042137146 47.0
0 0.00800216197967529 48.0
0 0.3027463555336 49.0
0.069960196202 0.0681971311569214 50.0
1.6420656098 0.890386700630188 51.0
0 -0.0616251230239868 52.0
0 -0.0221511125564575 53.0
0 -0.0221768617630005 54.0
0 -0.0556024312973022 55.0
0 -0.0389264822006226 56.0
0 -0.018964409828186 57.0
0 -0.0240193605422974 58.0
0 -0.0321050882339478 59.0
0 0.0355988740921021 60.0
0 0.0178216695785522 61.0
0 0.886384844779968 62.0
262.67040224 261.362548828125 0.0
0 -0.111036896705627 1.0
0 -0.0476137399673462 2.0
0 -0.0341662168502808 3.0
0 0.0158127546310425 4.0
0 -0.0144301652908325 5.0
0 0.0132983922958374 6.0
0 -0.0120688676834106 7.0
0 0.000713586807250977 8.0
0 -0.0397993326187134 9.0
0 0.0207985639572144 10.0
0 0.00522762537002563 11.0
0 0.082267165184021 12.0
74.541289994 74.5586624145508 13.0
12.444886739 29.6739311218262 14.0
0 -0.034986138343811 15.0
0 -0.0184327363967896 16.0
36.371880693 35.6742515563965 17.0
0 -0.0236009359359741 18.0
0 0.0401431322097778 19.0
0 -0.0315138101577759 20.0
0 -0.0190585851669312 21.0
0 0.0274366140365601 22.0
29.040941702 28.6784648895264 23.0
0 -0.0689798593521118 24.0
0 0.632644832134247 25.0
0 0.0427173376083374 26.0
0 -0.000944733619689941 27.0
0 -0.0370761156082153 28.0
0 -0.0450083017349243 29.0
0 -0.0993115901947021 30.0
0 -0.0484455823898315 31.0
0 0.116634726524353 32.0
0 0.13939368724823 33.0
0 -0.0722674131393433 34.0
0 0.0495234727859497 35.0
0 0.0701810121536255 36.0
0 0.0340224504470825 37.0
0 -0.00887095928192139 38.0
0 0.0747305154800415 39.0
0 0.0689691305160522 40.0
0 -0.0592514276504517 41.0
0 -0.114562153816223 42.0
284.53842756 292.624908447266 43.0
0 0.0205429792404175 44.0
0 0.0110658407211304 45.0
0 -0.0189276933670044 46.0
0 -0.0513356924057007 47.0
0 0.035692572593689 48.0
0 0.270214140415192 49.0
0.096328519372 0.0481878519058228 50.0
2.3150551281 0.944635152816772 51.0
0 -0.0380016565322876 52.0
0 -0.00391685962677002 53.0
0 0.0285738706588745 54.0
0 -0.00740540027618408 55.0
0 -0.106672644615173 56.0
0 0.0278753042221069 57.0
0 -0.0458608865737915 58.0
0 -0.0415066480636597 59.0
0 -0.0571001768112183 60.0
0 0.0396953821182251 61.0
0 0.863426208496094 62.0
264.75550282 256.152099609375 0.0
0 -0.0391641855239868 1.0
0 -0.0275920629501343 2.0
0 -0.0267740488052368 3.0
0 -0.00290262699127197 4.0
0 -0.0301345586776733 5.0
0 -0.0878235101699829 6.0
0 -0.0563102960586548 7.0
0 -0.0115231275558472 8.0
0 0.0186504125595093 9.0
0 0.00799250602722168 10.0
0 -0.0334309339523315 11.0
0 -0.00288259983062744 12.0
54.01039412 52.6929664611816 13.0
38.664463813 37.951847076416 14.0
0 -0.0908161401748657 15.0
0 0.0143572092056274 16.0
40.095653999 40.4924278259277 17.0
0 -0.0405830144882202 18.0
0 -0.0404471158981323 19.0
0 -0.0334757566452026 20.0
0 -0.0348730087280273 21.0
0 -0.0956140756607056 22.0
42.485921193 41.968936920166 23.0
0 -0.136638283729553 24.0
0 1.24046492576599 25.0
0 0.0220330953598022 26.0
0 0.0210479497909546 27.0
0 0.0945628881454468 28.0
0 -0.00231039524078369 29.0
0 -0.0209974050521851 30.0
0 -0.0211352109909058 31.0
0 -0.0359765291213989 32.0
0 0.0242542028427124 33.0
0 -0.00247490406036377 34.0
0 0.0309802293777466 35.0
0 0.04474937915802 36.0
0 -0.0429965257644653 37.0
0 -0.0153154134750366 38.0
0 -0.00336349010467529 39.0
0 0.126109004020691 40.0
0 -0.00781834125518799 41.0
0 -0.0624691247940063 42.0
293.9107329 296.159698486328 43.0
0 0.0482343435287476 44.0
0 0.00454527139663696 45.0
0 -0.0116137266159058 46.0
0 -0.0291422605514526 47.0
0 0.0586823225021362 48.0
0 0.250382125377655 49.0
0.077491245465 0.0685616731643677 50.0
2.072437679 0.975372850894928 51.0
0 -0.0771325826644897 52.0
0 -0.0384312868118286 53.0
0 0.0477184057235718 54.0
0 -0.108252167701721 55.0
0 0.0392097234725952 56.0
0 -0.0325723886489868 57.0
0 -0.0391625165939331 58.0
0 -0.0217920541763306 59.0
0 0.158228278160095 60.0
0 0.0263892412185669 61.0
0 0.828346967697144 62.0
280.28484605 256.605041503906 0.0
0 -0.101818203926086 1.0
0 -0.0601030588150024 2.0
0 -0.0159374475479126 3.0
0 0.00293004512786865 4.0
0 -0.0635696649551392 5.0
0 0.0793031454086304 6.0
0 0.0382035970687866 7.0
0 -0.0777497291564941 8.0
0 0.10324227809906 9.0
0 0.018104076385498 10.0
0 -0.0398030281066895 11.0
0 -0.105510354042053 12.0
50.998664849 50.818058013916 13.0
41.999509436 43.4341201782227 14.0
0 -0.118689179420471 15.0
0 0.02715003490448 16.0
32.276948366 31.3049755096436 17.0
0 0.0106374025344849 18.0
0 0.0655561685562134 19.0
0 -0.00440657138824463 20.0
0 -0.00142443180084229 21.0
0 -0.0161772966384888 22.0
39.279484566 38.2159652709961 23.0
0 0.00918805599212646 24.0
0 0.53213769197464 25.0
0 0.0637849569320679 26.0
0 0.0552664995193481 27.0
0 0.0544260740280151 28.0
0 -0.058369517326355 29.0
0 -0.0840635299682617 30.0
0 -0.0445460081100464 31.0
0 -0.0713258981704712 32.0
0 0.0110403299331665 33.0
0 -0.0426055192947388 34.0
0 -0.031190037727356 35.0
0 -0.0133477449417114 36.0
0 -0.0580688714981079 37.0
0 -0.0177346467971802 38.0
0 0.126840949058533 39.0
0 0.0584648847579956 40.0
0 -0.0222567319869995 41.0
0 -0.0315548181533813 42.0
299.6503347 296.426727294922 43.0
0 -0.0084606409072876 44.0
0 -0.00871825218200684 45.0
0 -0.0207568407058716 46.0
0 -0.0506948232650757 47.0
0 0.0774549245834351 48.0
0 0.311455339193344 49.0
0.069007167321 0.0568894147872925 50.0
1.9052116522 0.998243272304535 51.0
0 -0.0630887746810913 52.0
0 -0.00878727436065674 53.0
0 0.0256015062332153 54.0
0 -0.119084000587463 55.0
0 -0.0599559545516968 56.0
0 -0.0224412679672241 57.0
0 -0.0181137323379517 58.0
0 -0.0733598470687866 59.0
0 0.00491702556610107 60.0
0 0.00747287273406982 61.0
0 0.80702817440033 62.0
217.56839131 248.407409667969 0.0
0 0.0214132070541382 1.0
0 -0.00581085681915283 2.0
0 0.000695586204528809 3.0
0 -0.00767672061920166 4.0
0 0.0639992952346802 5.0
0 0.0371650457382202 6.0
0 -0.0031970739364624 7.0
0 -0.0870287418365479 8.0
0 0.00597226619720459 9.0
0 0.0128563642501831 10.0
0 -0.0310869216918945 11.0
0 0.0164216756820679 12.0
67.224249853 66.9737167358398 13.0
30.493504799 27.7048892974854 14.0
0 0.00350654125213623 15.0
0 -0.0212162733078003 16.0
30.959584783 31.4841117858887 17.0
0 0.00818955898284912 18.0
0 0.056688666343689 19.0
0 0.0103596448898315 20.0
0 -0.0317515134811401 21.0
0 0.0575135946273804 22.0
33.288480266 32.8959846496582 23.0
0 -0.0350915193557739 24.0
0 0.610930681228638 25.0
0 0.0696471929550171 26.0
0 0.0818345546722412 27.0
0 0.114726901054382 28.0
0 -0.0168184041976929 29.0
0 -0.00448060035705566 30.0
0 -0.0353668928146362 31.0
0 0.116377234458923 32.0
0 0.0707052946090698 33.0
0 -0.0333586931228638 34.0
0 -0.0657240152359009 35.0
0 0.0629113912582397 36.0
0 0.0299731492996216 37.0
0 -0.0183147192001343 38.0
0 0.0799921751022339 39.0
0 0.000322699546813965 40.0
0 0.0159963369369507 41.0
0 -0.0803772211074829 42.0
302.25549224 290.180908203125 43.0
0 0.00655877590179443 44.0
0 0.0649338960647583 45.0
0 -0.0301653146743774 46.0
0 -0.0294262170791626 47.0
0 -0.109538674354553 48.0
0 0.273976147174835 49.0
0.10002265775 0.0634206533432007 50.0
1.7015037391 1.00738930702209 51.0
0 -0.127308011054993 52.0
0 0.000689387321472168 53.0
0 -0.00502622127532959 54.0
0 -0.0467675924301147 55.0
0 -0.0582708120346069 56.0
0 0.055745005607605 57.0
0 -0.0882507562637329 58.0
0 -0.0316554307937622 59.0
0 0.0185543298721313 60.0
0 0.0685278177261353 61.0
0 0.809153199195862 62.0
259.34352174 255.957809448242 0.0
0 -0.0366555452346802 1.0
0 -0.0463210344314575 2.0
0 -0.01410973072052 3.0
0 -0.0252097845077515 4.0
0 -0.0300549268722534 5.0
0 0.00833117961883545 6.0
0 -0.00615608692169189 7.0
0 -0.0280297994613647 8.0
0 0.0676225423812866 9.0
0 0.0237160921096802 10.0
0 0.0242360830307007 11.0
0 -0.0141235589981079 12.0
64.870012892 65.0138854980469 13.0
27.012132241 32.8321418762207 14.0
0 -0.000174641609191895 15.0
0 -0.0756567716598511 16.0
28.272072668 27.8206081390381 17.0
0 0.0392163991928101 18.0
0 0.00560534000396729 19.0
0 -0.00649797916412354 20.0
0 -0.0321663618087769 21.0
0 -0.0625296831130981 22.0
31.273759965 29.8119258880615 23.0
0 0.0117830038070679 24.0
0 0.508258402347565 25.0
0 -0.0121713876724243 26.0
0 0.0617761611938477 27.0
0 0.0504449605941772 28.0
0 -0.0640448331832886 29.0
0 -0.0205212831497192 30.0
0 -0.0397578477859497 31.0
0 0.00790536403656006 32.0
0 0.112229228019714 33.0
0 0.00557434558868408 34.0
0 -0.0147091150283813 35.0
0 0.0103484392166138 36.0
0 -0.0197468996047974 37.0
0 0.0172957181930542 38.0
0 0.106819272041321 39.0
0 0.09142005443573 40.0
0 -0.0170422792434692 41.0
0 -0.0202022790908813 42.0
307.11549337 298.308868408203 43.0
0 -0.00834810733795166 44.0
0 0.0854353308677673 45.0
0 -0.0488613843917847 46.0
0 -0.0127500295639038 47.0
0 -0.0640784502029419 48.0
0 0.329574525356293 49.0
0.095771582441 0.0737758874893188 50.0
2.1251155727 1.00282573699951 51.0
0 -0.0657709836959839 52.0
0 -0.0546413660049438 53.0
0 0.0437406301498413 54.0
0 -0.0240305662155151 55.0
0 -0.00300586223602295 56.0
0 -0.0446597337722778 57.0
0 -0.133362174034119 58.0
0 -0.00257956981658936 59.0
0 0.0272077322006226 60.0
0 -0.0569876432418823 61.0
0 0.835340142250061 62.0
239.78732382 257.073760986328 0.0
0 -0.0542980432510376 1.0
0 -0.0775266885757446 2.0
0 -0.00514519214630127 3.0
0 -0.03441321849823 4.0
0 -0.0203200578689575 5.0
0 -0.032173752784729 6.0
0 -0.11050546169281 7.0
0 -0.0749006867408752 8.0
0 0.0806726217269897 9.0
0 0.0240797996520996 10.0
0 -0.0270792245864868 11.0
0 0.109042525291443 12.0
61.200613139 60.4706497192383 13.0
39.764265483 36.0355339050293 14.0
0 -0.0480533838272095 15.0
0 0.0320395231246948 16.0
34.274509081 33.8749847412109 17.0
0 0.0271972417831421 18.0
0 -0.0049978494644165 19.0
0 -0.0221329927444458 20.0
0 -0.0384460687637329 21.0
0 -0.00363600254058838 22.0
30.347505698 29.0523357391357 23.0
0 0.00663983821868896 24.0
0 0.564832389354706 25.0
0 0.109558463096619 26.0
0 0.110305666923523 27.0
0 -0.0350896120071411 28.0
0 -0.0793541669845581 29.0
0 0.0172237157821655 30.0
0 -0.0306860208511353 31.0
0 0.0100325345993042 32.0
0 0.0228594541549683 33.0
0 -0.00328290462493896 34.0
0 -0.112937331199646 35.0
0 0.107184290885925 36.0
0 0.064792275428772 37.0
0 -0.0543051958084106 38.0
0 0.00207793712615967 39.0
0 -0.0298148393630981 40.0
0 0.0594550371170044 41.0
0 -0.0380760431289673 42.0
290.0538782 296.83642578125 43.0
0 0.0305804014205933 44.0
0 -0.0522853136062622 45.0
0 -0.0292266607284546 46.0
0 -0.00959217548370361 47.0
0 -0.0573965311050415 48.0
0 0.31143182516098 49.0
0.072487331402 0.0234655141830444 50.0
1.8163158111 0.87448787689209 51.0
0 -0.0845221281051636 52.0
0 -0.0198818445205688 53.0
0 0.0511151552200317 54.0
0 -0.105211853981018 55.0
0 -0.0118526220321655 56.0
0 -0.0184887647628784 57.0
0 -0.0487147569656372 58.0
0 -0.0254610776901245 59.0
0 0.0186029672622681 60.0
0 0.0850731134414673 61.0
0 0.908706784248352 62.0
230.47627669 256.579925537109 0.0
0 -0.174169182777405 1.0
0 0.00666844844818115 2.0
0 -0.00331485271453857 3.0
0 -0.0540159940719604 4.0
0 -0.0231865644454956 5.0
0 0.0258687734603882 6.0
0 -0.0712531805038452 7.0
0 -0.00230813026428223 8.0
0 -0.0304752588272095 9.0
0 -0.00227510929107666 10.0
0 -0.0211844444274902 11.0
0 -0.0635610818862915 12.0
65.137025965 64.5341186523438 13.0
23.248803719 32.6821136474609 14.0
0 -0.0543972253799438 15.0
0 0.0419524908065796 16.0
38.266280008 37.7413215637207 17.0
0 0.00685179233551025 18.0
0 0.00888121128082275 19.0
0 -0.0102044343948364 20.0
0 -0.0174325704574585 21.0
0 -0.0262666940689087 22.0
28.494969988 27.9683361053467 23.0
0 0.111070990562439 24.0
0 0.438634097576141 25.0
0 0.0612238645553589 26.0
0 0.0892888307571411 27.0
0 0.0284079313278198 28.0
0 -0.0728906393051147 29.0
0 0.141794085502625 30.0
0 -0.00971806049346924 31.0
0 0.074358344078064 32.0
0 0.0807203054428101 33.0
0 0.00172317028045654 34.0
0 -0.0867356061935425 35.0
0 0.084795355796814 36.0
0 -0.00630390644073486 37.0
0 0.020511269569397 38.0
0 0.0243641138076782 39.0
0 -0.0161319971084595 40.0
0 -0.0398228168487549 41.0
0 0.00413835048675537 42.0
292.29046354 296.005523681641 43.0
0 -0.0531622171401978 44.0
0 -0.0411598682403564 45.0
0 -0.0019458532333374 46.0
0 -0.0471736192703247 47.0
0 0.00828063488006592 48.0
1.4625533864 0.251176595687866 49.0
0.087135632418 0.171585202217102 50.0
2.2131491394 0.913606464862823 51.0
0 -0.0694066286087036 52.0
0 -0.0280565023422241 53.0
0 0.0329357385635376 54.0
0 -0.0355511903762817 55.0
0 0.00441968441009521 56.0
0 -0.0319939851760864 57.0
0 -0.0553516149520874 58.0
0 -0.0663601160049438 59.0
0 -0.0957826375961304 60.0
0 0.0322393178939819 61.0
0 0.844653487205505 62.0
249.38223153 254.285186767578 0.0
0 -0.0600253343582153 1.0
0 -0.013043999671936 2.0
0 -0.0306917428970337 3.0
0 -0.0621191263198853 4.0
0 0.0438796281814575 5.0
0 0.0383447408676147 6.0
0 -0.0814312696456909 7.0
0 0.00550675392150879 8.0
0 -0.0685287714004517 9.0
0 0.00497329235076904 10.0
0 -0.00866472721099854 11.0
0 -0.164406418800354 12.0
58.357653569 57.5827674865723 13.0
40.868829352 38.3952598571777 14.0
0 -0.0418311357498169 15.0
0 -0.0641568899154663 16.0
27.539115201 26.8778247833252 17.0
0 0.0605713129043579 18.0
0 0.00874364376068115 19.0
0 -0.00691330432891846 20.0
0 -0.0165590047836304 21.0
0 -0.00954353809356689 22.0
32.431771665 31.6092281341553 23.0
0 -0.0836745500564575 24.0
0 0.604640305042267 25.0
0 0.00998389720916748 26.0
0 -0.00399744510650635 27.0
0 0.0390089750289917 28.0
0 -0.058263897895813 29.0
0 -0.0793999433517456 30.0
0 -0.0157195329666138 31.0
0 -0.050028920173645 32.0
0 0.0619724988937378 33.0
0 -0.0255364179611206 34.0
0 0.0514415502548218 35.0
0 -0.0106161832809448 36.0
0 -0.074379563331604 37.0
0 -0.0150023698806763 38.0
0 0.0172532796859741 39.0
0 0.0658410787582397 40.0
0 -0.0516366958618164 41.0
0 -0.009804368019104 42.0
305.68967484 296.829498291016 43.0
0 -0.000571846961975098 44.0
0 0.0249872207641602 45.0
0 -0.0213907957077026 46.0
0 -0.00317871570587158 47.0
0 0.0467652082443237 48.0
0 0.274416595697403 49.0
0.091698441537 0.0650442838668823 50.0
1.525319138 0.902375876903534 51.0
0 -0.0446015596389771 52.0
0 -0.0348387956619263 53.0
0 0.012438178062439 54.0
0 -0.0556830167770386 55.0
0 -0.0799201726913452 56.0
0 0.00419414043426514 57.0
0 -0.0416179895401001 58.0
0 -0.0302914381027222 59.0
0 -0.119562268257141 60.0
0 0.000185608863830566 61.0
0 0.854565620422363 62.0
232.3109274 257.5830078125 0.0
0 -0.0229755640029907 1.0
0 0.0160499811172485 2.0
0 -0.0156859159469604 3.0
0 -0.064576268196106 4.0
0 -0.0755032300949097 5.0
0 -0.0559889078140259 6.0
0 -0.0408934354782104 7.0
0 -0.0505952835083008 8.0
0 0.0387719869613647 9.0
0 -0.0920323133468628 10.0
0 -0.0808799266815186 11.0
0 0.127160429954529 12.0
65.478391051 64.7419815063477 13.0
24.03198437 33.5068092346191 14.0
0 -0.133529305458069 15.0
0 -0.0229672193527222 16.0
30.692681519 30.7376537322998 17.0
0 -0.0202008485794067 18.0
0 -0.0341392755508423 19.0
0 -0.00239551067352295 20.0
0 -0.0125864744186401 21.0
0 -0.0348778963088989 22.0
34.26777897 33.7776527404785 23.0
0 0.0405572652816772 24.0
0 0.475376665592194 25.0
0 0.0856989622116089 26.0
0 -0.0293437242507935 27.0
0 -0.107316136360168 28.0
0 -0.0330418348312378 29.0
0 -0.0426399707794189 30.0
0 -0.0116876363754272 31.0
0 0.0398091077804565 32.0
0 0.0103093385696411 33.0
0 -0.00403702259063721 34.0
0 0.0108785629272461 35.0
0 -0.108212113380432 36.0
0 -0.0281869173049927 37.0
0 0.00571882724761963 38.0
0 0.0320261716842651 39.0
0 0.0793417692184448 40.0
0 -0.028508186340332 41.0
0 0.0116370916366577 42.0
298.62863694 295.840484619141 43.0
0 0.0232852697372437 44.0
0 0.0919480919837952 45.0
0 -0.040716290473938 46.0
0 -0.0320180654525757 47.0
0 0.0256990194320679 48.0
0 0.293221652507782 49.0
0.096521692643 0.070203423500061 50.0
2.2012805908 0.928871035575867 51.0
0 -0.024203896522522 52.0
0 -0.0373150110244751 53.0
0 0.0500701665878296 54.0
0 -0.105119824409485 55.0
0 -0.0425018072128296 56.0
0 -0.0142208337783813 57.0
0 -0.106947779655457 58.0
0 -0.0605486631393433 59.0
0 0.000483870506286621 60.0
0 -0.021532416343689 61.0
0 0.801161170005798 62.0
272.13189377 257.599578857422 0.0
0 -0.0890375375747681 1.0
0 0.03288733959198 2.0
0 0.00293457508087158 3.0
0 -0.0312899351119995 4.0
0 -0.0150882005691528 5.0
0 0.0402835607528687 6.0
0 -0.183086514472961 7.0
0 -0.0161305665969849 8.0
0 -0.0308700799942017 9.0
0 0.0603947639465332 10.0
0 0.00552290678024292 11.0
0 -0.033115029335022 12.0
65.029524617 65.1135787963867 13.0
43.37505515 33.5587997436523 14.0
0 -0.0568968057632446 15.0
0 -0.0326720476150513 16.0
38.910189204 38.3257598876953 17.0
0 -0.029086709022522 18.0
0 -0.0164510011672974 19.0
0 -0.010717511177063 20.0
0 -0.00938117504119873 21.0
0 0.0172528028488159 22.0
39.104963414 38.720287322998 23.0
0 -0.00291216373443604 24.0
0 1.84827768802643 25.0
0 -0.0230675935745239 26.0
0 0.0628165006637573 27.0
0 0.0233796834945679 28.0
0 -0.0639034509658813 29.0
0 0.0755165815353394 30.0
0 0.00059807300567627 31.0
0 -0.0488172769546509 32.0
0 0.00811946392059326 33.0
0 -0.0462383031845093 34.0
0 0.0358569622039795 35.0
0 -0.0124231576919556 36.0
0 -0.0490888357162476 37.0
0 -0.0326539278030396 38.0
0 -0.0111404657363892 39.0
0 -0.153578877449036 40.0
0 -0.0214709043502808 41.0
0 -0.0325542688369751 42.0
305.85476575 295.008087158203 43.0
0 0.0227612257003784 44.0
0 -0.060096263885498 45.0
0 -0.0475760698318481 46.0
0 -0.0120354890823364 47.0
0 0.0436495542526245 48.0
0 0.363731443881989 49.0
0.075277228732 0.0490075349807739 50.0
1.6232026091 0.996834278106689 51.0
0 -0.042793869972229 52.0
0 -0.065376877784729 53.0
0 0.00758492946624756 54.0
0 -0.124331593513489 55.0
0 0.0111242532730103 56.0
0 -0.0497287511825562 57.0
0 -0.0402237176895142 58.0
0 -0.0369447469711304 59.0
0 0.00742614269256592 60.0
0 -0.000744223594665527 61.0
0 0.876569867134094 62.0
237.52721954 252.842758178711 0.0
0 -0.0158959627151489 1.0
0 -0.103041768074036 2.0
0 0.00351130962371826 3.0
0 -0.0668607950210571 4.0
0 -0.0783776044845581 5.0
0 -0.111713051795959 6.0
0 -0.0719298124313354 7.0
0 -0.0848879814147949 8.0
0 -0.0313955545425415 9.0
0 0.00397682189941406 10.0
0 -0.0338802337646484 11.0
0 0.0919564962387085 12.0
57.484695243 57.0224342346191 13.0
42.971066646 39.8648948669434 14.0
0 -0.0439873933792114 15.0
0 -0.00606644153594971 16.0
39.718516674 38.6395263671875 17.0
0 -0.00222647190093994 18.0
0 0.0394750833511353 19.0
0 0.0117238759994507 20.0
0 -0.0154627561569214 21.0
0 -0.0346986055374146 22.0
33.886835177 33.4737777709961 23.0
0 0.0474580526351929 24.0
0 0.455631196498871 25.0
0 0.0953325033187866 26.0
0 0.0327523946762085 27.0
0 -0.049324631690979 28.0
0 -0.0616213083267212 29.0
0 -0.0581275224685669 30.0
0 -0.0604497194290161 31.0
0 -0.0279322862625122 32.0
0 -0.0506902933120728 33.0
0 -0.0945461988449097 34.0
0 -0.055288553237915 35.0
0 -0.0910631418228149 36.0
0 -0.0417150259017944 37.0
0 0.0348323583602905 38.0
0 0.0557571649551392 39.0
0 0.142820239067078 40.0
0 -0.0639903545379639 41.0
0 -0.0463577508926392 42.0
288.13440796 297.270202636719 43.0
0 -0.0222307443618774 44.0
0 0.0778711438179016 45.0
0 -0.000402092933654785 46.0
0 -0.0476702451705933 47.0
0 0.0289129018783569 48.0
0 0.305118024349213 49.0
0.082844985084 0.0637694597244263 50.0
2.2802321592 1.01916480064392 51.0
0 -0.0705214738845825 52.0
0 -0.0175789594650269 53.0
0 0.0668925046920776 54.0
0 -0.112037062644958 55.0
0 0.0209919214248657 56.0
0 -0.034803032875061 57.0
0 0.00880920886993408 58.0
0 -0.0800672769546509 59.0
0 -0.0350683927536011 60.0
0 -0.0190976858139038 61.0
0 0.842844843864441 62.0
246.05935244 255.884246826172 0.0
0 -0.0357095003128052 1.0
0 -0.00189435482025146 2.0
0 -0.0614277124404907 3.0
0 -0.0104383230209351 4.0
0 -0.137004494667053 5.0
0 -0.00424730777740479 6.0
0 -0.0945726633071899 7.0
0 -0.00059354305267334 8.0
0 -0.000248551368713379 9.0
0 -0.0254083871841431 10.0
0 -0.00106650590896606 11.0
0 -0.0377727746963501 12.0
60.648794993 60.1620903015137 13.0
30.297955667 38.1693115234375 14.0
0 0.0152968168258667 15.0
0 -0.000105977058410645 16.0
40.023682347 39.9673728942871 17.0
0 0.006034255027771 18.0
0 -0.0270289182662964 19.0
0 -0.0160871744155884 20.0
0 0.0179265737533569 21.0
0 0.00395190715789795 22.0
31.872507403 31.6510334014893 23.0
0 -0.0947304964065552 24.0
0 0.953590512275696 25.0
0 0.123564600944519 26.0
0 0.0319539308547974 27.0
0 0.0165618658065796 28.0
0 -0.0828341245651245 29.0
0 -0.0939853191375732 30.0
0 -0.0364013910293579 31.0
0 -0.0160247087478638 32.0
0 0.111480593681335 33.0
0 -0.0290580987930298 34.0
0 0.0113486051559448 35.0
0 0.0606695413589478 36.0
0 -0.0371593236923218 37.0
0 0.011879563331604 38.0
0 0.104130625724792 39.0
0 0.0352686643600464 40.0
0 0.0166195631027222 41.0
0 -0.00703966617584229 42.0
293.72206789 297.184967041016 43.0
0 0.0278714895248413 44.0
0 -0.0631552934646606 45.0
0 0.0106068849563599 46.0
0 -0.043211817741394 47.0
0 -0.0239964723587036 48.0
0 0.290677130222321 49.0
0.069468918632 0.0604931116104126 50.0
1.9536012898 1.00506675243378 51.0
0 -0.0426229238510132 52.0
0 -0.0871015787124634 53.0
0 0.0399855375289917 54.0
0 -0.0241852998733521 55.0
0 0.0156561136245728 56.0
0 -0.00901639461517334 57.0
0 -0.0668932199478149 58.0
0 -0.1054927110672 59.0
0 -0.0899859666824341 60.0
0 0.0515800714492798 61.0
0 0.901494801044464 62.0
220.32226889 252.862182617188 0.0
0 0.00532615184783936 1.0
0 -0.0836769342422485 2.0
0 0.0372713804244995 3.0
0 -0.0154350996017456 4.0
0 -0.0285613536834717 5.0
0 0.0186771154403687 6.0
0 -0.136201143264771 7.0
0 0.0388342142105103 8.0
0 -0.0598350763320923 9.0
0 -0.0699328184127808 10.0
0 -0.0355449914932251 11.0
0 -0.0946365594863892 12.0
48.320742735 46.9725761413574 13.0
56.588202567 47.4778785705566 14.0
0 -0.00149023532867432 15.0
0 0.00705945491790771 16.0
36.02255836 35.6811180114746 17.0
0 -0.0264025926589966 18.0
0 -0.0455986261367798 19.0
0 0.00719368457794189 20.0
0 -0.0377770662307739 21.0
0 0.0334862470626831 22.0
41.213348433 39.5165328979492 23.0
0 -0.0622855424880981 24.0
0 0.541804611682892 25.0
0 0.0538743734359741 26.0
0 0.102558016777039 27.0
0 0.0740731954574585 28.0
0 -0.0683289766311646 29.0
0 0.1583651304245 30.0
0 -0.0171619653701782 31.0
0 0.0582026243209839 32.0
0 0.0766628980636597 33.0
0 -0.0354830026626587 34.0
0 0.0555388927459717 35.0
0 -0.023032546043396 36.0
0 -0.0991843938827515 37.0
0 0.0116316080093384 38.0
0 -0.0320557355880737 39.0
0 0.0884064435958862 40.0
0 -0.0934275388717651 41.0
0 -0.0394960641860962 42.0
282.03177645 296.117736816406 43.0
0 -0.009590744972229 44.0
0 -0.0380150079727173 45.0
0 0.00723159313201904 46.0
0 -0.0196553468704224 47.0
0 0.0109075307846069 48.0
0 0.262323915958405 49.0
0.084078616469 0.177457571029663 50.0
2.1349774065 0.911589980125427 51.0
0 -0.0271388292312622 52.0
0 -0.0562688112258911 53.0
0 0.00336158275604248 54.0
0 -0.0558346509933472 55.0
0 -0.0169135332107544 56.0
0 -0.0160967111587524 57.0
0 -0.0572770833969116 58.0
0 0.0115870237350464 59.0
0 -0.0690935850143433 60.0
0 0.0400575399398804 61.0
0 0.868867993354797 62.0
287.78096686 256.730804443359 0.0
0 -0.0472484827041626 1.0
0 -0.101263165473938 2.0
0 -0.0869897603988647 3.0
0 -0.0270782709121704 4.0
0 -0.00121080875396729 5.0
0 -0.0818196535110474 6.0
0 -0.00257980823516846 7.0
0 -0.0316283702850342 8.0
0 -0.0395790338516235 9.0
0 -0.00410664081573486 10.0
0 0.00631833076477051 11.0
0 -0.0254484415054321 12.0
73.753106042 72.559455871582 13.0
20.9075625 22.3598823547363 14.0
0 0.0572980642318726 15.0
0 -0.0219789743423462 16.0
28.153379787 27.9662799835205 17.0
0 -0.022236704826355 18.0
0 -0.0121206045150757 19.0
0 -0.0391682386398315 20.0
0 0.0074995756149292 21.0
0 0.0171035528182983 22.0
29.328173564 28.8062229156494 23.0
0 -0.00883400440216064 24.0
6.1047708807 0.616324722766876 25.0
0 0.0819543600082397 26.0
0 0.0721691846847534 27.0
0 0.0651549100875854 28.0
0 -0.0590485334396362 29.0
0 -0.0301998853683472 30.0
0 -0.0704060792922974 31.0
0 0.034016489982605 32.0
0 0.0802730321884155 33.0
0 -0.00470006465911865 34.0
0 0.0800046920776367 35.0
0 -0.0262840986251831 36.0
0 -0.0739890336990356 37.0
0 -0.0204447507858276 38.0
0 0.0409690141677856 39.0
0 0.0998057126998901 40.0
0 -0.0297102928161621 41.0
0 -0.0650383234024048 42.0
291.71875663 295.34228515625 43.0
0 -0.040021538734436 44.0
0 -0.0427502393722534 45.0
0 0.0205219984054565 46.0
0 -0.0698174238204956 47.0
0 0.0103458166122437 48.0
0 0.285510540008545 49.0
0.069292825703 0.0901108980178833 50.0
2.202704643 0.91420304775238 51.0
0 -0.0948280096054077 52.0
0 -0.00373613834381104 53.0
0 0.000997662544250488 54.0
0 -0.0924509763717651 55.0
0 0.00686252117156982 56.0
0 -0.0180989503860474 57.0
0 -0.0343080759048462 58.0
0 -0.00609362125396729 59.0
0 -0.265528798103333 60.0
0 -0.0317069292068481 61.0
0 0.809368014335632 62.0
251.86766432 254.511077880859 0.0
0 -0.0344969034194946 1.0
0 -0.110408425331116 2.0
0 -0.0201469659805298 3.0
0 -0.0429359674453735 4.0
0 -0.0455471277236938 5.0
0 -0.584822773933411 6.0
0 -0.212715029716492 7.0
0 -0.0245434045791626 8.0
0 0.0254510641098022 9.0
0 -0.0710574388504028 10.0
0 0.0383136868476868 11.0
0 -0.0141893625259399 12.0
78.150987325 80.0374603271484 13.0
7.9669064394 16.2386074066162 14.0
0 -0.0456854104995728 15.0
0 0.0361522436141968 16.0
38.452334072 37.9769706726074 17.0
0 0.0135608911514282 18.0
0 0.0743013620376587 19.0
0 -0.00506484508514404 20.0
0 0.0136936902999878 21.0
0 -0.0770622491836548 22.0
32.905687815 31.952615737915 23.0
0 -0.0743318796157837 24.0
0 0.636402487754822 25.0
0 0.127638220787048 26.0
0 0.0421603918075562 27.0
0 -0.0279079675674438 28.0
0 -0.0748287439346313 29.0
0 -0.070706844329834 30.0
0 -0.0531297922134399 31.0
0 0.0741266012191772 32.0
0 -0.115634560585022 33.0
0 -0.0573323965072632 34.0
0 -0.0723029375076294 35.0
0 0.112017512321472 36.0
0 0.0426942110061646 37.0
0 0.055477499961853 38.0
0 -0.0151416063308716 39.0
0 -0.0229159593582153 40.0
0 0.261908650398254 41.0
0 -0.0211111307144165 42.0
291.18965567 292.246765136719 43.0
0 -0.00375950336456299 44.0
0 -0.0541322231292725 45.0
0 -0.0309206247329712 46.0
0 -0.0353549718856812 47.0
0 -0.0102611780166626 48.0
2.2963563198 0.29742506146431 49.0
0.074811963145 0.0619696378707886 50.0
1.46082485 0.937680780887604 51.0
0 -0.070496678352356 52.0
0 -0.0582014322280884 53.0
0 -0.119531273841858 54.0
0 0.0214303731918335 55.0
0 -0.0171011686325073 56.0
0 -0.0647231340408325 57.0
0 -0.108722567558289 58.0
0 -0.0653203725814819 59.0
0 -0.0165623426437378 60.0
0 -0.044563889503479 61.0
0 0.804234027862549 62.0
271.67658987 249.590774536133 0.0
0 -0.0390793085098267 1.0
0 -0.0187250375747681 2.0
0 -0.0353966951370239 3.0
0 -0.0354303121566772 4.0
0 0.0024183988571167 5.0
0 0.19898784160614 6.0
0 -0.0448151826858521 7.0
0 -0.0936252474784851 8.0
0 -0.00332510471343994 9.0
0 -0.236336946487427 10.0
0 0.0209475159645081 11.0
0 -0.00821268558502197 12.0
77.972772209 79.8217544555664 13.0
39.491035465 16.7569904327393 14.0
0 -0.0197798013687134 15.0
0 -0.0202232599258423 16.0
36.071717401 36.1696624755859 17.0
0 -0.0101794004440308 18.0
0 -0.00851094722747803 19.0
0 -0.0395621061325073 20.0
0 -0.0136960744857788 21.0
0 0.0155652761459351 22.0
29.54791832 29.0882701873779 23.0
0 0.00959384441375732 24.0
1.0647807124 0.648918926715851 25.0
0 0.0635217428207397 26.0
0 0.00362467765808105 27.0
0 -0.0304514169692993 28.0
0 -0.0710657835006714 29.0
0 -0.0301820039749146 30.0
0 -0.0438562631607056 31.0
0 0.0314635038375854 32.0
0 0.130254864692688 33.0
0 -0.0322288274765015 34.0
0 -0.0408178567886353 35.0
0 0.127612471580505 36.0
0 -0.0325368642807007 37.0
0 0.0214728116989136 38.0
0 0.0252941846847534 39.0
0 0.0346578359603882 40.0
0 0.0396891832351685 41.0
0 -0.0205122232437134 42.0
289.94363973 294.330108642578 43.0
0 -0.0524097681045532 44.0
0 0.250205039978027 45.0
0 -0.00502574443817139 46.0
0 -0.076025128364563 47.0
0 -0.0645290613174438 48.0
0 0.238178372383118 49.0
0.099570813844 0.109760642051697 50.0
2.3310354819 0.97285783290863 51.0
0 -0.0825766324996948 52.0
0 -0.0355440378189087 53.0
0 -0.0538202524185181 54.0
0 -0.136499762535095 55.0
0 -0.0296396017074585 56.0
0 -0.0371428728103638 57.0
0 -0.0578004121780396 58.0
0 -0.0958133935928345 59.0
0 0.0768882036209106 60.0
0 0.0562437772750854 61.0
0 0.858263671398163 62.0
248.62534282 255.461303710938 0.0
0 -0.105635285377502 1.0
0 0.0428105592727661 2.0
0 0.0287517309188843 3.0
0 -0.0425597429275513 4.0
0 -0.0879231691360474 5.0
0 -0.0512405633926392 6.0
0 -0.085485577583313 7.0
0 -0.0472139120101929 8.0
0 0.0149883031845093 9.0
0 0.019258975982666 10.0
0 -0.0210011601448059 11.0
0 -0.194606423377991 12.0
59.32989404 58.2949523925781 13.0
41.547820596 37.2035865783691 14.0
0 -0.0851227045059204 15.0
0 0.00494229793548584 16.0
40.848720742 41.7058486938477 17.0
0 -0.00702035427093506 18.0
0 -0.00422489643096924 19.0
0 0.0208717584609985 20.0
0 -0.0123811960220337 21.0
0 -0.0367685556411743 22.0
29.915444503 29.1288547515869 23.0
0 0.0967305898666382 24.0
0 0.499568998813629 25.0
0 0.035753607749939 26.0
0 -0.0694707632064819 27.0
0 -0.00082695484161377 28.0
0 0.038598895072937 29.0
0 -0.0229434967041016 30.0
0 -0.0219720602035522 31.0
0 -0.0104633569717407 32.0
0 0.146578669548035 33.0
0 -0.0193532705307007 34.0
0 0.0276949405670166 35.0
0 -0.0552221536636353 36.0
0 -0.00513660907745361 37.0
0 -0.0278126001358032 38.0
0 0.0344189405441284 39.0
0 0.00681483745574951 40.0
0 -0.0374857187271118 41.0
0 0.0325053930282593 42.0
289.04453648 291.846343994141 43.0
0 0.00477492809295654 44.0
0 0.0352438688278198 45.0
0 -0.038965106010437 46.0
0 -0.00581324100494385 47.0
0 0.0620836019515991 48.0
0 0.307713627815247 49.0
0.082713070301 0.0674360990524292 50.0
2.1216175769 1.01680040359497 51.0
0 -0.0831738710403442 52.0
0 0.0243772268295288 53.0
0 -0.0348173379898071 54.0
0 -0.0356827974319458 55.0
0 -0.0307518243789673 56.0
0 -0.00314962863922119 57.0
0 -0.0282505750656128 58.0
0 -0.0332413911819458 59.0
0 -0.0020831823348999 60.0
0 0.0607026815414429 61.0
0 0.890080511569977 62.0
286.63029923 260.868682861328 0.0
0 -0.0421463251113892 1.0
0 -0.106763243675232 2.0
0 0.0347288846969604 3.0
0 -0.0162366628646851 4.0
0 -0.0774317979812622 5.0
0 0.0102385282516479 6.0
0 0.0614770650863647 7.0
0 0.0283892154693604 8.0
0 0.0496515035629272 9.0
0 -0.0134264230728149 10.0
0 -0.0384201407432556 11.0
0 0.00646817684173584 12.0
72.591744301 71.4655838012695 13.0
23.790355498 30.4502029418945 14.0
0 -0.0463529825210571 15.0
0 -0.00271260738372803 16.0
30.986996245 30.1308345794678 17.0
0 -0.00903642177581787 18.0
0 0.0284644365310669 19.0
0 -0.0367401838302612 20.0
0 -0.0180591344833374 21.0
0 0.00815808773040771 22.0
26.001174025 25.2318096160889 23.0
0 0.0678437948226929 24.0
0 0.502790331840515 25.0
0 0.0591710805892944 26.0
0 0.137986779212952 27.0
0 -0.0455104112625122 28.0
0 -0.145921349525452 29.0
0 0.115442156791687 30.0
0 -0.0503665208816528 31.0
0 -0.0189357995986938 32.0
0 0.063562273979187 33.0
0 -0.0783981084823608 34.0
0 0.0432535409927368 35.0
0 -0.129814267158508 36.0
0 0.0131720304489136 37.0
0 -0.0107396841049194 38.0
0 0.0405250787734985 39.0
0 0.108260989189148 40.0
0 0.0802723169326782 41.0
0 -0.119757294654846 42.0
303.19393774 293.864440917969 43.0
0 0.038479208946228 44.0
0 0.00972729921340942 45.0
0 -0.072370171546936 46.0
0 -0.0530847311019897 47.0
0 0.0880507230758667 48.0
0 0.346669048070908 49.0
0.081068813634 0.0563865900039673 50.0
1.4305229253 0.881633520126343 51.0
0 -0.126883625984192 52.0
0 -0.0691009759902954 53.0
0 0.025389552116394 54.0
0 -0.0643783807754517 55.0
0 -0.0336480140686035 56.0
0 -0.0248743295669556 57.0
0 -0.00122106075286865 58.0
0 -0.00084841251373291 59.0
0 0.0804892778396606 60.0
0 0.0468968152999878 61.0
0 0.823288440704346 62.0
243.02479961 260.93798828125 0.0
0 0.0460518598556519 1.0
0 -0.128603577613831 2.0
0 -0.00608313083648682 3.0
0 -0.0447231531143188 4.0
0 -0.135985255241394 5.0
0 -0.0214201211929321 6.0
0 -0.0143173933029175 7.0
0 0.0877496004104614 8.0
0 0.0256341695785522 9.0
0 -0.0490201711654663 10.0
0 -0.0348215103149414 11.0
0 -0.0492883920669556 12.0
73.33205714 71.9423751831055 13.0
15.807538868 27.9591941833496 14.0
0 0.00650203227996826 15.0
0 -0.0580602884292603 16.0
36.673572127 35.783390045166 17.0
0 -0.0173231363296509 18.0
0 -0.000758767127990723 19.0
0 0.0225211381912231 20.0
0 -0.0516594648361206 21.0
0 0.0282586812973022 22.0
30.497911887 29.2296390533447 23.0
0 0.0839732885360718 24.0
0 0.465991616249084 25.0
0 -0.0295192003250122 26.0
0 0.0353926420211792 27.0
0 0.084459662437439 28.0
0 -0.11329185962677 29.0
0 0.0157824754714966 30.0
0 -0.0592061281204224 31.0
0 0.0316594839096069 32.0
0 -0.0645676851272583 33.0
0 0.0148466825485229 34.0
0 -0.0568948984146118 35.0
0 0.0598324537277222 36.0
0 0.110218405723572 37.0
0 -0.0187417268753052 38.0
0 0.00407874584197998 39.0
0 0.0232518911361694 40.0
0 -0.00103664398193359 41.0
0 -0.0073782205581665 42.0
308.90230742 295.668792724609 43.0
0 -0.0200573205947876 44.0
0 -0.0306234955787659 45.0
0 -0.0371688604354858 46.0
0 -0.0696033239364624 47.0
0 0.0542052984237671 48.0
0 0.287261247634888 49.0
0.067390503591 0.0455647706985474 50.0
1.5566688125 0.940601706504822 51.0
0 -0.0698815584182739 52.0
0 0.000883936882019043 53.0
0 -0.0825940370559692 54.0
0 -0.0130437612533569 55.0
0 -0.0212463140487671 56.0
0 -0.041040301322937 57.0
0 -0.0439168214797974 58.0
0 -0.0331419706344604 59.0
0 -0.0583823919296265 60.0
0 -0.0219994783401489 61.0
0 0.808091163635254 62.0
257.20001115 257.198608398438 0.0
0 -0.106134057044983 1.0
0 0.0317445993423462 2.0
0 -0.00786912441253662 3.0
0 -0.0468462705612183 4.0
0 0.0932260751724243 5.0
0 0.229671835899353 6.0
0 0.0350807905197144 7.0
0 0.0437412261962891 8.0
0 0.0778869390487671 9.0
0 0.00755488872528076 10.0
0 -0.0377435088157654 11.0
0 -0.103005051612854 12.0
77.431067014 75.7425003051758 13.0
34.231020978 23.7374973297119 14.0
0 -0.0400835275650024 15.0
0 -0.055660605430603 16.0
29.417780872 28.8803043365479 17.0
0 0.0530732870101929 18.0
0 0.00887191295623779 19.0
0 -0.0351499319076538 20.0
0 -0.00780069828033447 21.0
0 -0.0190349817276001 22.0
36.473307099 35.9041976928711 23.0
0 0.0466612577438354 24.0
0 1.53240275382996 25.0
0 -0.0153228044509888 26.0
0 0.026745080947876 27.0
0 -0.0563856363296509 28.0
0 -0.0379534959793091 29.0
0 -0.0968585014343262 30.0
0 -0.0428403615951538 31.0
0 0.024349570274353 32.0
0 0.025321364402771 33.0
0 -0.0226491689682007 34.0
0 -0.0222903490066528 35.0
0 -0.114930033683777 36.0
0 -0.0336295366287231 37.0
0 -0.00583422183990479 38.0
0 0.0933679342269897 39.0
0 0.0371745824813843 40.0
0 -0.032794713973999 41.0
0 0.00484883785247803 42.0
297.67759516 297.661956787109 43.0
0 0.0334318876266479 44.0
0 0.0363627672195435 45.0
0 -0.00628602504730225 46.0
0 -0.0275155305862427 47.0
0 0.0322445631027222 48.0
0 0.277700424194336 49.0
0.096062353499 0.0735450983047485 50.0
2.3217491514 0.975613236427307 51.0
0 -0.0925909280776978 52.0
0 -0.0586644411087036 53.0
0 0.0336199998855591 54.0
0 -0.100981831550598 55.0
0 -0.0581308603286743 56.0
0 -0.0122882127761841 57.0
0 -0.0550268888473511 58.0
0 -0.037145733833313 59.0
0 -0.0864983797073364 60.0
0 -0.0545858144760132 61.0
0 0.84962797164917 62.0
257.99846208 254.231063842773 0.0
0 -0.0430361032485962 1.0
0 -0.0200001001358032 2.0
0 0.0272184610366821 3.0
0 -0.0478583574295044 4.0
0 -0.0629101991653442 5.0
0 -0.0957499742507935 6.0
0 -0.0433355569839478 7.0
0 -0.0768177509307861 8.0
0 0.115373015403748 9.0
0 0.0228317975997925 10.0
0 -0.0380810499191284 11.0
0 0.0192931890487671 12.0
55.399678833 55.0205612182617 13.0
53.889953377 44.0670394897461 14.0
0 0.0317620038986206 15.0
0 -0.0486160516738892 16.0
35.892887873 35.563777923584 17.0
0 0.0125952959060669 18.0
0 0.0269080400466919 19.0
0 0.0233556032180786 20.0
0 -0.0336563587188721 21.0
0 -0.00149881839752197 22.0
39.067789787 38.2244606018066 23.0
0 0.0719345808029175 24.0
0 0.449046552181244 25.0
0 0.0304888486862183 26.0
0 0.0173168182373047 27.0
0 0.0310238599777222 28.0
0 -0.0246521234512329 29.0
0 -0.0870378017425537 30.0
0 0.00662481784820557 31.0
0 -0.00879490375518799 32.0
0 0.0226150751113892 33.0
0 -0.0617109537124634 34.0
0 0.00968241691589355 35.0
0 0.0288716554641724 36.0
0 -0.0566180944442749 37.0
0 -0.0403693914413452 38.0
0 0.0483096837997437 39.0
0 -0.0401903390884399 40.0
0 -0.0748831033706665 41.0
0 0.00859963893890381 42.0
303.13362215 293.7861328125 43.0
0 -0.0290790796279907 44.0
0 0.0176076889038086 45.0
0 -0.00311124324798584 46.0
0 -0.0170445442199707 47.0
0 -0.0208760499954224 48.0
0 0.282709240913391 49.0
0.097882409403 0.0792211294174194 50.0
1.9692237301 0.977794408798218 51.0
0 -0.0830255746841431 52.0
0 -0.0374912023544312 53.0
0 0.0332614183425903 54.0
0 -0.0582822561264038 55.0
0 0.0426336526870728 56.0
0 -0.0384713411331177 57.0
0 -0.0526615381240845 58.0
0 -0.0205117464065552 59.0
0 0.133439660072327 60.0
0 -0.0273927450180054 61.0
0 0.826664209365845 62.0
221.18963464 253.35205078125 0.0
0 -0.0307193994522095 1.0
0 -0.0267311334609985 2.0
0 -0.036258339881897 3.0
0 -0.0580302476882935 4.0
0 -0.0350998640060425 5.0
0 -0.039075493812561 6.0
0 -0.138361096382141 7.0
0 -0.00455987453460693 8.0
0 0.0822213888168335 9.0
0 -0.0117205381393433 10.0
0 0.0118210315704346 11.0
0 0.0894521474838257 12.0
72.297798706 71.3746185302734 13.0
23.612076894 27.6320762634277 14.0
0 -0.0410791635513306 15.0
0 0.0113567113876343 16.0
36.469931345 34.8221397399902 17.0
0 0.0222412347793579 18.0
0 0.0220593214035034 19.0
0 -0.00728476047515869 20.0
0 -0.0343281030654907 21.0
0 -0.0543566942214966 22.0
40.70308422 40.4339714050293 23.0
0 -0.111677289009094 24.0
0 0.648269653320312 25.0
0 -0.0284997224807739 26.0
0 -0.039858341217041 27.0
0 -0.0713340044021606 28.0
0 -0.0684272050857544 29.0
0 -0.0874627828598022 30.0
0 0.00892817974090576 31.0
0 0.000503897666931152 32.0
0 -0.0133281946182251 33.0
0 -0.0436655282974243 34.0
0 -0.0622707605361938 35.0
0 0.0827785730361938 36.0
0 -0.0357035398483276 37.0
0 0.0383609533309937 38.0
0 0.0338457822799683 39.0
0 0.0301240682601929 40.0
0 -0.0839241743087769 41.0
0 -0.059517502784729 42.0
293.36798227 295.742065429688 43.0
0 0.00722110271453857 44.0
0 -0.0748605728149414 45.0
0 -0.0117272138595581 46.0
0 -0.0737282037734985 47.0
0 0.00119102001190186 48.0
0 0.351166367530823 49.0
0.098220466819 0.0554553270339966 50.0
2.2695639013 0.949415385723114 51.0
0 -0.0719407796859741 52.0
0 0.00757801532745361 53.0
0 -0.00504624843597412 54.0
0 -0.0524278879165649 55.0
0 -0.00952160358428955 56.0
0 -0.0303026437759399 57.0
0 -0.0637134313583374 58.0
0 -0.0842362642288208 59.0
0 0.0192192792892456 60.0
0 0.00271356105804443 61.0
0 0.874402523040771 62.0
279.17660995 256.383850097656 0.0
0 -0.114074349403381 1.0
0 -0.0141090154647827 2.0
0 -0.0408107042312622 3.0
0 -0.0567187070846558 4.0
0 0.0321673154830933 5.0
0 0.0278772115707397 6.0
0 -0.0458852052688599 7.0
0 0.030016303062439 8.0
0 -0.000598549842834473 9.0
0 -0.00621509552001953 10.0
0 -0.0308822393417358 11.0
0 0.0633500814437866 12.0
73.536429152 72.5762023925781 13.0
25.242637195 26.6736488342285 14.0
0 0.0133641958236694 15.0
0 -0.0894330739974976 16.0
33.620101475 33.1095237731934 17.0
0 -0.0230346918106079 18.0
0 -0.0112625360488892 19.0
0 0.0364545583724976 20.0
0 -0.0261188745498657 21.0
0 0.00634658336639404 22.0
30.838792679 29.8494281768799 23.0
0 -0.0609041452407837 24.0
0 0.58233642578125 25.0
0 0.00290524959564209 26.0
0 0.000280499458312988 27.0
0 0.0778454542160034 28.0
0 -0.0415827035903931 29.0
0 -0.101229310035706 30.0
0 -0.0212236642837524 31.0
0 -0.028908371925354 32.0
0 0.000887751579284668 33.0
0 -0.0615972280502319 34.0
0 -0.0468989610671997 35.0
0 -0.00973665714263916 36.0
0 -0.0676950216293335 37.0
0 0.0520311594009399 38.0
0 -0.021638035774231 39.0
0 -0.0358697175979614 40.0
0 -0.0042346715927124 41.0
0 0.0347732305526733 42.0
301.51073201 297.017272949219 43.0
0 0.0192006826400757 44.0
0 -0.0244647264480591 45.0
0 -0.0662175416946411 46.0
0 -0.0313071012496948 47.0
0 -0.0484796762466431 48.0
0 0.329183280467987 49.0
0.079042540691 0.0688713788986206 50.0
1.6246454731 0.995133817195892 51.0
0 -0.0958515405654907 52.0
0 -0.0179888010025024 53.0
0 -0.0516408681869507 54.0
0 -0.0767217874526978 55.0
0 0.0258949995040894 56.0
0 -0.0550233125686646 57.0
0 -0.0601538419723511 58.0
0 -0.0856064558029175 59.0
0 -0.0383685827255249 60.0
0 0.0620237588882446 61.0
0 0.919826567173004 62.0
289.13540036 257.403656005859 0.0
0 -0.0311552286148071 1.0
0 -0.112440228462219 2.0
0 -0.0168625116348267 3.0
0 -0.0838552713394165 4.0
0 -0.1809002161026 5.0
0 0.174898028373718 6.0
0 -0.0684243440628052 7.0
0 -0.0250455141067505 8.0
0 0.110035300254822 9.0
0 0.0277742147445679 10.0
0 -0.0226478576660156 11.0
0 0.053305983543396 12.0
68.866693791 69.4694519042969 13.0
41.288334264 34.1355819702148 14.0
0 -0.0206519365310669 15.0
0 -0.100031495094299 16.0
38.383454877 37.7509384155273 17.0
0 -0.0426183938980103 18.0
0 -0.0476447343826294 19.0
0 0.0387047529220581 20.0
0 -0.0295702219009399 21.0
0 -0.0359698534011841 22.0
30.047276578 29.5266799926758 23.0
0 -0.0643316507339478 24.0
0 1.86524569988251 25.0
0 0.0330365896224976 26.0
0 -0.0133935213088989 27.0
0 -0.109828591346741 28.0
0 -0.184431195259094 29.0
0 0.073123574256897 30.0
0 0.0109275579452515 31.0
0 0.00541865825653076 32.0
0 -0.212844133377075 33.0
0 -0.0555912256240845 34.0
0 0.0310111045837402 35.0
0 0.00644481182098389 36.0
0 -0.00605237483978271 37.0
0 -0.0383995771408081 38.0
0 -0.0361541509628296 39.0
0 0.0853389501571655 40.0
0 -0.179188847541809 41.0
0 -0.185434460639954 42.0
299.41961875 292.493560791016 43.0
0 0.00181996822357178 44.0
0 -0.0137683153152466 45.0
0 0.0623708963394165 46.0
0 -0.142796635627747 47.0
0 -0.0838133096694946 48.0
0 0.358524769544601 49.0
0.095425536491 0.0203763246536255 50.0
2.251991223 0.9744673371315 51.0
0 -0.0737415552139282 52.0
0 -0.0738240480422974 53.0
0 0.0182570219039917 54.0
0 -0.0782500505447388 55.0
0 0.0326520204544067 56.0
0 -0.0512882471084595 57.0
0 0.162637591362 58.0
0 0.00351798534393311 59.0
0 -0.207866787910461 60.0
0 0.0631102323532104 61.0
0 0.960477232933044 62.0
220.71939728 264.5888671875 0.0
0 -0.0423132181167603 1.0
0 0.00522387027740479 2.0
0 0.0631916522979736 3.0
0 -0.0107461214065552 4.0
0 -0.0195013284683228 5.0
0 -0.0517536401748657 6.0
0 -0.0576266050338745 7.0
0 -0.0175484418869019 8.0
0 0.0424212217330933 9.0
0 -0.0437778234481812 10.0
0 -0.00724917650222778 11.0
0 -0.0421320199966431 12.0
74.884040347 72.1463165283203 13.0
19.861244368 28.226110458374 14.0
0 -0.0788952112197876 15.0
0 -0.0183807611465454 16.0
40.464876681 41.2999954223633 17.0
0 0.0295699834823608 18.0
0 0.00405395030975342 19.0
0 0.00126063823699951 20.0
0 0.00425899028778076 21.0
0 -0.0619927644729614 22.0
29.094090129 29.2649745941162 23.0
0 0.040737509727478 24.0
0 1.08126711845398 25.0
0 0.00888431072235107 26.0
0 0.0883630514144897 27.0
0 -0.0680357217788696 28.0
0 -0.0499523878097534 29.0
0 -0.0774625539779663 30.0
0 -0.0253955125808716 31.0
0 -0.0115257501602173 32.0
0 0.0750049352645874 33.0
0 -0.0308536291122437 34.0
0 -0.106382727622986 35.0
0 -0.0858670473098755 36.0
0 -0.075499415397644 37.0
0 -0.0454062223434448 38.0
0 0.0414594411849976 39.0
0 0.12345016002655 40.0
0 0.00741255283355713 41.0
0 -0.0164800882339478 42.0
300.80218241 294.113830566406 43.0
0 0.0109723806381226 44.0
0 0.112966060638428 45.0
0 -0.0102585554122925 46.0
0 -0.091891884803772 47.0
0 -0.0159088373184204 48.0
0 0.327841252088547 49.0
0.084481283998 0.0702821016311646 50.0
2.2467770895 1.01398825645447 51.0
0 -0.0530132055282593 52.0
0 -0.0360761880874634 53.0
0 0.0122417211532593 54.0
0 -0.0432084798812866 55.0
0 -0.0290297269821167 56.0
0 -0.00654399394989014 57.0
0 -0.0401018857955933 58.0
0 0.00158059597015381 59.0
0 0.0669184923171997 60.0
0 0.000482439994812012 61.0
0 0.842249035835266 62.0
207.41855917 226.14289855957 0.0
0 -0.066261887550354 1.0
0 0.0896760225296021 2.0
0 -0.0453935861587524 3.0
0 -0.0701271295547485 4.0
0 0.0167936086654663 5.0
0 -0.0583134889602661 6.0
0 -0.0150619745254517 7.0
0 -0.000914096832275391 8.0
0 -0.0722461938858032 9.0
0 0.0397461652755737 10.0
0 -0.04310142993927 11.0
0 -0.207702279090881 12.0
57.291286378 56.7583503723145 13.0
22.617069621 35.6558837890625 14.0
0 -0.0544239282608032 15.0
0 0.0480550527572632 16.0
37.04270841 37.9643936157227 17.0
0 -0.0360461473464966 18.0
0 -0.0378414392471313 19.0
0 0.00136172771453857 20.0
0 -0.0433341264724731 21.0
0 -0.032849907875061 22.0
39.755573851 40.0582389831543 23.0
0 0.099514365196228 24.0
0 0.426907896995544 25.0
0 -0.0405102968215942 26.0
0 0.0597106218338013 27.0
0 0.148127913475037 28.0
0 -0.153757691383362 29.0
0 0.0824424028396606 30.0
0 0.0229684114456177 31.0
0 -0.0685182809829712 32.0
0 -0.0216492414474487 33.0
0 -0.180751204490662 34.0
0 0.217014670372009 35.0
0 0.0171531438827515 36.0
0 -0.0801488161087036 37.0
0 -0.0598911046981812 38.0
0 -0.0228391885757446 39.0
0 0.0205773115158081 40.0
0 -0.160313010215759 41.0
0 0.0902541875839233 42.0
293.73933739 293.675140380859 43.0
0 0.0372575521469116 44.0
0 0.0241731405258179 45.0
0 -0.0338085889816284 46.0
0 -0.0986431837081909 47.0
0 -0.0835596323013306 48.0
0 0.255366444587708 49.0
0.080225425828 0.200227379798889 50.0
2.1768402116 0.971144139766693 51.0
0 -0.0122040510177612 52.0
0 -0.0123530626296997 53.0
0 -0.0302034616470337 54.0
0 0.00824463367462158 55.0
0 -0.0711334943771362 56.0
0 -0.0493041276931763 57.0
0 0.0483812093734741 58.0
0 -0.0548139810562134 59.0
0 -0.00808322429656982 60.0
0 0.0749212503433228 61.0
0 0.902137935161591 62.0
250.08172929 265.74755859375 0.0
0 -0.0635553598403931 1.0
0 0.0178986787796021 2.0
0 0.44095504283905 3.0
0 -0.0186859369277954 4.0
0 -0.190977215766907 5.0
0 0.0646752119064331 6.0
0 -0.13549792766571 7.0
0 -0.0494046211242676 8.0
0 -0.0255810022354126 9.0
0 0.0273510217666626 10.0
0 -0.00619816780090332 11.0
0 -0.0478531122207642 12.0
50.54904622 50.1703224182129 13.0
35.276093557 46.2519340515137 14.0
0 -0.0752111673355103 15.0
0 0.0182646512985229 16.0
35.694720552 34.3006019592285 17.0
0 0.0157967805862427 18.0
0 0.0325700044631958 19.0
0 0.0322679281234741 20.0
0 0.0159882307052612 21.0
0 -0.0406605005264282 22.0
38.65859477 37.6344795227051 23.0
0 0.0834311246871948 24.0
0 0.362805902957916 25.0
0 0.0972880125045776 26.0
0 0.0681782960891724 27.0
0 -0.105840802192688 28.0
0 -0.0582104921340942 29.0
0 0.14933967590332 30.0
0 -0.0149933099746704 31.0
0 -0.0868126153945923 32.0
0 -0.140240550041199 33.0
0 -0.0777541399002075 34.0
0 -0.0412694215774536 35.0
0 0.142816424369812 36.0
0 -0.0184494256973267 37.0
0 0.0139666795730591 38.0
0 -0.0534484386444092 39.0
0 0.151240229606628 40.0
0 -0.0803986787796021 41.0
0 -0.0601574182510376 42.0
287.03245529 296.892913818359 43.0
0 0.0139058828353882 44.0
0 0.0706170797348022 45.0
0 -0.0875884294509888 46.0
0 0.0879162549972534 47.0
0 0.0641978979110718 48.0
0 0.278550654649734 49.0
0.10180198242 0.0651859045028687 50.0
2.0508946936 1.00713896751404 51.0
0 -0.124330639839172 52.0
0 -0.0841656923294067 53.0
0 0.0388864278793335 54.0
0 -0.101431488990784 55.0
0 0.0266364812850952 56.0
0 -0.0499268770217896 57.0
0 -0.0914832353591919 58.0
0 0.0527471303939819 59.0
0 0.128423094749451 60.0
0 -0.0626174211502075 61.0
0 0.920520663261414 62.0
258.68463186 257.176452636719 0.0
0 0.020476222038269 1.0
0 0.0562998056411743 2.0
0 -0.0209606885910034 3.0
0 -0.064689040184021 4.0
0 0.043242335319519 5.0
0 0.049765944480896 6.0
0 -0.0938469171524048 7.0
0 -0.0444403886795044 8.0
0 0.091020941734314 9.0
0 -0.144120573997498 10.0
0 -0.030229926109314 11.0
0 0.098232626914978 12.0
46.141564417 45.5418968200684 13.0
43.307929872 41.3330612182617 14.0
0 -0.0664921998977661 15.0
0 0.0445398092269897 16.0
25.364805385 23.3548984527588 17.0
0 -0.00926244258880615 18.0
0 0.00521171092987061 19.0
0 0.0434027910232544 20.0
0 -0.0363162755966187 21.0
0 0.0178521871566772 22.0
30.114389469 29.3688793182373 23.0
0 0.00928294658660889 24.0
0 1.01905870437622 25.0
0 0.0147000551223755 26.0
0 0.0527746081352234 27.0
0 0.000323176383972168 28.0
0 -0.0534130334854126 29.0
0 -0.0683772563934326 30.0
0 -0.00103318691253662 31.0
0 -0.0091930627822876 32.0
0 -0.0494498014450073 33.0
0 -0.0283008813858032 34.0
0 -0.104308724403381 35.0
0 0.0602003335952759 36.0
0 -0.0796955823898315 37.0
0 -0.0279780626296997 38.0
0 0.00931298732757568 39.0
0 -0.00674498081207275 40.0
0 0.0164707899093628 41.0
0 0.0217679738998413 42.0
282.98633165 295.170654296875 43.0
0 -0.129292130470276 44.0
0 0.0458046197891235 45.0
0 -0.00335061550140381 46.0
0 -0.0247524976730347 47.0
0 -0.110651135444641 48.0
3.810066187 0.261138200759888 49.0
0.10348498179 0.157413482666016 50.0
1.7098376902 0.874295353889465 51.0
0 -0.100083708763123 52.0
0 0.000885844230651855 53.0
0 0.0511363744735718 54.0
0 -0.024262547492981 55.0
0 -0.0544737577438354 56.0
0 0.017755389213562 57.0
0 -0.0714994668960571 58.0
0 -0.0319181680679321 59.0
0 -0.0066605806350708 60.0
0 -0.00921618938446045 61.0
0 0.831765174865723 62.0
283.5009113 253.638046264648 0.0
0 0.00343167781829834 1.0
0 0.0256894826889038 2.0
0 -0.0598310232162476 3.0
0 -0.0209945440292358 4.0
0 0.0342448949813843 5.0
0 -0.0832029581069946 6.0
0 -0.0699931383132935 7.0
0 -0.0342457294464111 8.0
0 -0.0475422143936157 9.0
0 0.017717719078064 10.0
0 -0.0665155649185181 11.0
0 -0.0339971780776978 12.0
55.690441054 54.9562110900879 13.0
39.88448581 38.4896469116211 14.0
0 -0.0749970674514771 15.0
0 -0.0609146356582642 16.0
35.411546351 34.8908843994141 17.0
0 0.0285810232162476 18.0
0 -0.00992047786712646 19.0
0 0.00658094882965088 20.0
0 -0.0122588872909546 21.0
0 -0.0136348009109497 22.0
30.055030089 29.1663265228271 23.0
0 0.0696786642074585 24.0
0 0.491457521915436 25.0
0 0.0840123891830444 26.0
0 0.00365984439849854 27.0
0 -0.00741827487945557 28.0
0 -0.0759974718093872 29.0
0 -0.0866619348526001 30.0
0 -0.0666669607162476 31.0
0 0.016377329826355 32.0
0 0.0396991968154907 33.0
0 0.00457823276519775 34.0
0 0.0072401762008667 35.0
0 0.0249897241592407 36.0
0 -0.0503596067428589 37.0
0 0.00592386722564697 38.0
0 0.123839497566223 39.0
0 0.0894488096237183 40.0
0 -0.0309488773345947 41.0
0 -0.00924360752105713 42.0
297.11379986 296.441528320312 43.0
0 0.0220445394515991 44.0
0 -0.0157989263534546 45.0
0 -0.00924217700958252 46.0
0 -0.056505560874939 47.0
0 -0.0264583826065063 48.0
0 0.269673585891724 49.0
0.082478329306 0.102096199989319 50.0
2.1049435022 0.898402094841003 51.0
0 -0.084872841835022 52.0
0 -0.0732985734939575 53.0
0 0.0100210905075073 54.0
0 -0.0876837968826294 55.0
0 0.000156283378601074 56.0
0 -0.0472618341445923 57.0
0 -0.0314639806747437 58.0
0 -0.0547980070114136 59.0
0 -0.00140845775604248 60.0
0 0.0286794900894165 61.0
0 0.809034824371338 62.0
252.38042851 258.028778076172 0.0
0 -0.0611749887466431 1.0
0 0.0465172529220581 2.0
0 0.0154405832290649 3.0
0 -0.0488570928573608 4.0
0 -0.00192320346832275 5.0
0 -0.0636645555496216 6.0
0 0.00620114803314209 7.0
0 -0.0730186700820923 8.0
0 0.0413316488265991 9.0
0 -0.0255652666091919 10.0
0 -0.0254461765289307 11.0
0 0.0414870977401733 12.0
59.439065846 59.0594291687012 13.0
36.295059662 38.7870254516602 14.0
0 -0.0773168802261353 15.0
0 -0.0317219495773315 16.0
36.204838544 35.7666893005371 17.0
0 -0.0170208215713501 18.0
0 0.00797760486602783 19.0
0 -0.0289958715438843 20.0
0 -0.00492823123931885 21.0
0 -0.0645300149917603 22.0
27.925007411 27.3905048370361 23.0
0 0.0784257650375366 24.0
0 1.2289400100708 25.0
0 0.105555415153503 26.0
0 0.0604516267776489 27.0
0 0.0418380498886108 28.0
0 -0.0392199754714966 29.0
0 -0.0165691375732422 30.0
0 -0.0881818532943726 31.0
0 -0.023016095161438 32.0
0 0.0327223539352417 33.0
0 -0.0364364385604858 34.0
0 -0.103988289833069 35.0
0 -0.0741745233535767 36.0
0 -0.044303297996521 37.0
0 0.0336035490036011 38.0
0 0.0275682210922241 39.0
0 0.053888201713562 40.0
0 -0.0106674432754517 41.0
0 0.0542434453964233 42.0
299.19094952 294.745086669922 43.0
0 -0.0625597238540649 44.0
0 -0.00950372219085693 45.0
0 0.00649178028106689 46.0
0 -0.0351675748825073 47.0
0 0.0555609464645386 48.0
0 0.303118765354156 49.0
0.085345708936 0.053113579750061 50.0
1.9300965745 0.891948938369751 51.0
0 -0.0737146139144897 52.0
0 -0.0581656694412231 53.0
0 -0.0301069021224976 54.0
0 -0.0752445459365845 55.0
0 -0.00136160850524902 56.0
0 -0.0834263563156128 57.0
0 -0.0705219507217407 58.0
0 -0.0280946493148804 59.0
0 -0.0508466958999634 60.0
0 -0.0219227075576782 61.0
0 0.819139122962952 62.0
245.88010282 255.000503540039 0.0
0 -0.0602718591690063 1.0
0 -0.0462037324905396 2.0
0 -0.0103110074996948 3.0
0 -0.0509756803512573 4.0
0 0.00220763683319092 5.0
0 -0.0469166040420532 6.0
0 -0.0818790197372437 7.0
0 -0.000825047492980957 8.0
0 0.0434383153915405 9.0
0 0.00896060466766357 10.0
0 -0.0409587621688843 11.0
0 -0.0587078332901001 12.0
66.12451197 65.7917785644531 13.0
27.060630449 32.8167915344238 14.0
0 0.0533851385116577 15.0
0 -0.010161280632019 16.0
35.394268756 35.6402931213379 17.0
0 0.0608290433883667 18.0
0 0.00914680957794189 19.0
0 -0.00621402263641357 20.0
0 -0.0321539640426636 21.0
0 0.00849425792694092 22.0
36.764392787 36.0624465942383 23.0
0 -0.0345298051834106 24.0
0 0.626469314098358 25.0
0 0.065825343132019 26.0
0 0.0556364059448242 27.0
0 0.111587405204773 28.0
0 -0.0464659929275513 29.0
0 -0.0673040151596069 30.0
0 -0.0141993761062622 31.0
0 -0.00244200229644775 32.0
0 0.0481904745101929 33.0
0 -0.0539008378982544 34.0
0 -0.0774294137954712 35.0
0 0.0992788076400757 36.0
0 -0.0873197317123413 37.0
0 -0.0271428823471069 38.0
0 0.0351954698562622 39.0
0 0.0986937284469604 40.0
0 -0.0367591381072998 41.0
0 0.009116530418396 42.0
302.78565925 296.130401611328 43.0
0 0.0294293165206909 44.0
0 -0.0104820728302002 45.0
0 -0.00489246845245361 46.0
0 -0.0381017923355103 47.0
0 -0.020997166633606 48.0
0.67679063142 0.309421837329865 49.0
0.088097603945 0.0618273019790649 50.0
1.7242136023 0.965698778629303 51.0
0 -0.0553363561630249 52.0
0 -0.0630174875259399 53.0
0 -0.0233999490737915 54.0
0 -0.0537999868392944 55.0
0 0.0160912275314331 56.0
0 -0.0441504716873169 57.0
0 0.0205475091934204 58.0
0 -0.0485848188400269 59.0
0 0.0133041143417358 60.0
0 -0.0180841684341431 61.0
0 0.801257967948914 62.0
242.19384869 251.475402832031 0.0
0 0.00405824184417725 1.0
0 0.00195109844207764 2.0
0 -0.0108629465103149 3.0
0 -0.0522559881210327 4.0
0 0.0331264734268188 5.0
0 0.000296950340270996 6.0
0 -0.0732918977737427 7.0
0 0.0708030462265015 8.0
0 0.0920265913009644 9.0
0 0.0139306783676147 10.0
0 -0.0227357149124146 11.0
0 -0.134995102882385 12.0
73.228093622 72.4416046142578 13.0
9.7815139041 24.6520709991455 14.0
0 -0.0409203767776489 15.0
0 -0.00573551654815674 16.0
26.744888326 26.2352619171143 17.0
0 -0.0246702432632446 18.0
0 0.0116811990737915 19.0
0 0.0249582529067993 20.0
0 0.00846445560455322 21.0
0 -0.0957909822463989 22.0
41.89800623 42.0992546081543 23.0
0 -0.100697636604309 24.0
0 0.674813866615295 25.0
0 0.00205886363983154 26.0
0 0.000236153602600098 27.0
0 -0.0641261339187622 28.0
0 -0.0714524984359741 29.0
0 -0.0861608982086182 30.0
0 -0.0513185262680054 31.0
0 -0.00503695011138916 32.0
0 -0.0152546167373657 33.0
0 -0.050561785697937 34.0
0 -0.0671689510345459 35.0
0 -0.129352927207947 36.0
0 -0.0410655736923218 37.0
0 0.0114408731460571 38.0
0 0.0116270780563354 39.0
0 0.0442126989364624 40.0
0 -0.0365930795669556 41.0
0 -0.0696092844009399 42.0
305.26233141 294.622985839844 43.0
0 0.0136741399765015 44.0
0 -0.00800901651382446 45.0
0 0.00614678859710693 46.0
0 -0.0525227785110474 47.0
0 0.0300439596176147 48.0
0.32470042567 0.27220356464386 49.0
0.079557474583 0.0791388750076294 50.0
1.5291629244 1.021817445755 51.0
0 -0.0629957914352417 52.0
0 -0.0514680147171021 53.0
0 0.00768792629241943 54.0
0 -0.063071608543396 55.0
0 -0.126803278923035 56.0
0 0.0258826017379761 57.0
0 -0.0747824907302856 58.0
0 -0.00897848606109619 59.0
0 0.00856554508209229 60.0
0 -0.00841701030731201 61.0
0 0.794486165046692 62.0
222.53475815 256.212493896484 0.0
0 -0.0279556512832642 1.0
0 -0.100991129875183 2.0
0 -0.0123645067214966 3.0
0 -0.0216842889785767 4.0
0 -0.0124424695968628 5.0
0 -0.031232476234436 6.0
0 -0.0747700929641724 7.0
0 0.00786590576171875 8.0
0 0.0473064184188843 9.0
0 -0.0332673788070679 10.0
0 0.0159088373184204 11.0
0 0.0924524068832397 12.0
74.282548098 74.3523101806641 13.0
23.315430257 23.1867656707764 14.0
0 -0.0120989084243774 15.0
0 -0.00735199451446533 16.0
41.436963763 40.5930862426758 17.0
0 -0.0105587244033813 18.0
0 -0.0302280187606812 19.0
0 0.0610672235488892 20.0
0 -0.0108515024185181 21.0
0 0.00908935070037842 22.0
37.793914061 37.3783874511719 23.0
0 -0.110921025276184 24.0
0 0.672526717185974 25.0
0 0.0736640691757202 26.0
0 0.107261300086975 27.0
0 0.00404822826385498 28.0
0 -0.0891784429550171 29.0
0 -0.0539627075195312 30.0
0 -0.0136367082595825 31.0
0 -0.03901207447052 32.0
0 0.0279487371444702 33.0
0 -0.00491607189178467 34.0
0 -0.0756911039352417 35.0
0 0.138340592384338 36.0
0 -0.0946012735366821 37.0
0 0.00502383708953857 38.0
0 0.0182727575302124 39.0
0 0.0403774976730347 40.0
0 0.0606354475021362 41.0
0 -0.0142837762832642 42.0
294.20229469 295.410003662109 43.0
0 -0.11481249332428 44.0
0 0.00970911979675293 45.0
0 -0.0544260740280151 46.0
0 -0.0419906377792358 47.0
0 0.0532878637313843 48.0
0 0.312949061393738 49.0
0.083547178043 0.081196665763855 50.0
2.0829406106 1.00656747817993 51.0
0 -0.0597509145736694 52.0
0 -0.0335565805435181 53.0
0 -0.0135015249252319 54.0
0 -0.0950876474380493 55.0
0 -0.0838559865951538 56.0
0 -0.0154246091842651 57.0
0 -0.0754364728927612 58.0
0 -0.0868576765060425 59.0
0 0.000588536262512207 60.0
0 -0.011454701423645 61.0
0 0.84839940071106 62.0
255.16083761 255.344497680664 0.0
0 -0.0491300821304321 1.0
0 -0.0460623502731323 2.0
0 0.0022817850112915 3.0
0 -0.00319802761077881 4.0
0 0.00834596157073975 5.0
0 -0.0201927423477173 6.0
0 0.0219382047653198 7.0
0 -0.0437455773353577 8.0
0 0.0719465017318726 9.0
0 0.0407273769378662 10.0
0 0.0427100658416748 11.0
0 0.0921071767807007 12.0
72.669480389 71.979866027832 13.0
27.773062189 24.2391834259033 14.0
0 -0.0272499322891235 15.0
0 -0.0455256700515747 16.0
40.682801768 40.0785102844238 17.0
0 -0.00574982166290283 18.0
0 0.00919973850250244 19.0
0 -0.000919938087463379 20.0
0 -0.0628923177719116 21.0
0 -0.00799477100372314 22.0
35.658010278 34.9892158508301 23.0
0 -0.116522431373596 24.0
0 0.652062177658081 25.0
0 0.0321491956710815 26.0
0 0.00199270248413086 27.0
0 -0.069337010383606 28.0
0 -0.0262590646743774 29.0
0 -0.0880358219146729 30.0
0 -0.0411449670791626 31.0
0 -0.0405622720718384 32.0
0 0.0523442029953003 33.0
0 -0.042448878288269 34.0
0 0.00943952798843384 35.0
0 -0.18318498134613 36.0
0 -0.0702360868453979 37.0
0 -0.0403534173965454 38.0
0 0.0123487710952759 39.0
0 0.174156069755554 40.0
0 -0.0589854717254639 41.0
0 -0.00657999515533447 42.0
302.83771367 295.110778808594 43.0
0 -0.0245672464370728 44.0
0 0.00048673152923584 45.0
0 -0.0685995817184448 46.0
0 -0.0534473657608032 47.0
0 0.0261305570602417 48.0
0.54247811833 0.297827363014221 49.0
0.098496921765 0.0631300210952759 50.0
2.2923141728 0.954421043395996 51.0
0 -0.0342491865158081 52.0
0 0.00451576709747314 53.0
0 -0.0711086988449097 54.0
0 -0.0295435190200806 55.0
0 -0.0293375253677368 56.0
0 -0.0512264966964722 57.0
0 -0.0203512907028198 58.0
0 -0.0236262083053589 59.0
0 -0.110267043113708 60.0
0 0.0686913728713989 61.0
0 0.862363040447235 62.0
237.87281016 258.826324462891 0.0
0 -0.0682154893875122 1.0
0 0.0345460176467896 2.0
0 -0.0423909425735474 3.0
0 -0.0301634073257446 4.0
0 -0.0842705965042114 5.0
0 -0.0461965799331665 6.0
0 -0.114142298698425 7.0
0 -0.00966811180114746 8.0
0 -0.00828993320465088 9.0
0 -0.00116455554962158 10.0
0 0.0115154981613159 11.0
0 0.128840804100037 12.0
64.824007446 64.3917083740234 13.0
23.009797594 34.1476440429688 14.0
0 -0.025612473487854 15.0
0 -0.00859320163726807 16.0
30.819566998 31.0712585449219 17.0
0 -0.0120569467544556 18.0
0 -0.0132600069046021 19.0
0 0.0143483877182007 20.0
0 -0.0453413724899292 21.0
0 -0.00348198413848877 22.0
30.730806784 29.8077354431152 23.0
0 0.0234283208847046 24.0
0 0.596030592918396 25.0
0 0.013304591178894 26.0
0 0.0254992246627808 27.0
0 0.0487445592880249 28.0
0 -0.0449761152267456 29.0
0 -0.092052698135376 30.0
0 -0.0459834337234497 31.0
0 0.0301302671432495 32.0
0 -0.00319874286651611 33.0
0 -0.0373238325119019 34.0
0 -0.0449260473251343 35.0
0 0.0968095064163208 36.0
0 -0.0315858125686646 37.0
0 0.0208231210708618 38.0
0 0.0334714651107788 39.0
0 0.0578917264938354 40.0
0 -0.0378460884094238 41.0
0 0.0273679494857788 42.0
297.38788536 291.720092773438 43.0
0 -0.00210249423980713 44.0
0 -0.00970232486724854 45.0
0 0.00260508060455322 46.0
0 -0.0289424657821655 47.0
0 0.0175966024398804 48.0
0 0.279394328594208 49.0
0.089327612676 0.0408741235733032 50.0
1.8685452927 0.912756741046906 51.0
0 -0.0455797910690308 52.0
0 -0.0410343408584595 53.0
0 0.031049370765686 54.0
0 -0.00403678417205811 55.0
0 -0.0774914026260376 56.0
0 0.0236297845840454 57.0
0 0.0328596830368042 58.0
0 -0.0648757219314575 59.0
0 0.108224272727966 60.0
0 0.0385919809341431 61.0
0 0.853506088256836 62.0
264.33470731 259.264892578125 0.0
0 -0.187972187995911 1.0
0 -0.0335005521774292 2.0
0 -0.0393480062484741 3.0
0 0.0511170625686646 4.0
0 0.060083270072937 5.0
0 0.0480178594589233 6.0
0 0.0204957723617554 7.0
0 0.0669809579849243 8.0
0 -0.536150574684143 9.0
0 0.0176535844802856 10.0
0 0.0265070199966431 11.0
0 0.0626415014266968 12.0
65.994429092 66.52880859375 13.0
23.132654157 33.4934196472168 14.0
0 -0.227323651313782 15.0
0 -0.00443613529205322 16.0
37.835481072 37.3130683898926 17.0
0 0.0313366651535034 18.0
0 -0.0265747308731079 19.0
0 0.0483342409133911 20.0
0 0.002205491065979 21.0
0 -0.0734015703201294 22.0
44.059005567 43.5084190368652 23.0
0 -0.108306050300598 24.0
0 0.667694926261902 25.0
0 -0.0157848596572876 26.0
0 0.049824595451355 27.0
0 -0.0374351739883423 28.0
0 -0.0697208642959595 29.0
0 -0.0491093397140503 30.0
0 -0.041556715965271 31.0
0 0.0406240224838257 32.0
0 -0.0648404359817505 33.0
0 -0.00468909740447998 34.0
0 -0.0164505243301392 35.0
0 0.163141131401062 36.0
0 0.00125968456268311 37.0
0 0.0129400491714478 38.0
0 0.0516339540481567 39.0
0 0.0881670713424683 40.0
0 -0.0549522638320923 41.0
0 0.0383666753768921 42.0
296.4681872 294.109771728516 43.0
0 0.0222047567367554 44.0
0 0.0913403630256653 45.0
0 0.00913846492767334 46.0
0 -0.0332592725753784 47.0
0 -0.0201359987258911 48.0
0 0.237046658992767 49.0
0.084110514746 0.0518327951431274 50.0
1.8160922236 0.908463776111603 51.0
0 -0.0916186571121216 52.0
0 -0.0503817796707153 53.0
0 0.0137690305709839 54.0
0 -0.18522846698761 55.0
0 0.116215109825134 56.0
0 0.045406699180603 57.0
0 -0.0531619787216187 58.0
0 -0.0110722780227661 59.0
0 0.0175095796585083 60.0
0 0.129120230674744 61.0
0 0.984516143798828 62.0
239.11522201 256.266754150391 0.0
0 0.0413764715194702 1.0
0 0.0409098863601685 2.0
0 0.0116163492202759 3.0
0 -0.0441957712173462 4.0
0 -0.0193959474563599 5.0
0 -0.0343414545059204 6.0
0 -0.0637267827987671 7.0
0 0.0657038688659668 8.0
0 0.0420717000961304 9.0
0 0.0157898664474487 10.0
0 -0.0435835719108582 11.0
0 -0.0317398309707642 12.0
68.169675423 68.259391784668 13.0
26.454188959 33.3523712158203 14.0
0 -0.100447297096252 15.0
0 -0.00951540470123291 16.0
39.486161947 38.993293762207 17.0
0 -0.0135430097579956 18.0
0 -0.0178245306015015 19.0
0 -0.0243161916732788 20.0
0 -0.0276361703872681 21.0
0 0.0119465589523315 22.0
29.900615106 30.0687789916992 23.0
0 -0.00135385990142822 24.0
0 0.680835902690887 25.0
0 0.0660847425460815 26.0
0 -0.0120798349380493 27.0
0 -0.0753856897354126 28.0
0 -0.0358730554580688 29.0
0 -0.0327906608581543 30.0
0 -0.0488911867141724 31.0
0 0.00586831569671631 32.0
0 0.0468431711196899 33.0
0 -0.0480169057846069 34.0
0 -0.00529742240905762 35.0
0 0.0632728338241577 36.0
0 -0.0762315988540649 37.0
0 0.0680338144302368 38.0
0 0.0250653028488159 39.0
0 -0.0284353494644165 40.0
0 -0.070959210395813 41.0
0 -0.00306570529937744 42.0
294.9843684 293.666961669922 43.0
0 0.00971066951751709 44.0
0 0.0017542839050293 45.0
0 0.00924336910247803 46.0
0 -0.0641647577285767 47.0
0 -0.00243914127349854 48.0
0.19979740263 0.241312503814697 49.0
0.096007441204 0.0577765703201294 50.0
2.0951637449 0.909021019935608 51.0
0 -0.106486678123474 52.0
0 0.0125550031661987 53.0
0 0.000155091285705566 54.0
0 -0.120872616767883 55.0
0 -0.0103188753128052 56.0
0 -0.0225940942764282 57.0
0 -0.0596121549606323 58.0
0 -0.045350193977356 59.0
0 -0.00758516788482666 60.0
0 -0.0191429853439331 61.0
0 0.807766556739807 62.0
217.3160003 245.463226318359 0.0
0 -0.040418267250061 1.0
0 -0.08790123462677 2.0
0 0.0127009153366089 3.0
0 -0.0453258752822876 4.0
0 -0.0937312841415405 5.0
0 0.0937188863754272 6.0
0 0.0968400239944458 7.0
0 -0.0032990574836731 8.0
0 0.145784735679626 9.0
0 -0.00716710090637207 10.0
0 0.0143537521362305 11.0
0 -0.0976196527481079 12.0
70.623865828 70.7483673095703 13.0
36.874925793 26.5130023956299 14.0
0 -0.0453411340713501 15.0
0 -0.0196219682693481 16.0
34.427348197 34.507495880127 17.0
0 0.0267916917800903 18.0
0 0.0542193651199341 19.0
0 0.016443133354187 20.0
0 0.00914347171783447 21.0
0 -0.0229541063308716 22.0
31.511185911 31.0974178314209 23.0
0 -0.100949883460999 24.0
0 0.605302453041077 25.0
0 0.0925806760787964 26.0
0 -0.0449234247207642 27.0
0 -0.077090859413147 28.0
0 -0.0339113473892212 29.0
0 -0.0535998344421387 30.0
0 -0.116856932640076 31.0
0 -0.044251561164856 32.0
0 0.123701453208923 33.0
0 -0.0209845304489136 34.0
0 0.07303786277771 35.0
0 0.00503289699554443 36.0
0 0.126602530479431 37.0
0 0.0755022764205933 38.0
0 0.0326999425888062 39.0
0 0.282817721366882 40.0
0 -0.133727431297302 41.0
0 -0.0111662149429321 42.0
297.53187305 291.875762939453 43.0
0 -0.0091172456741333 44.0
0 0.0636699199676514 45.0
0 0.0258105993270874 46.0
0 -0.094901442527771 47.0
0 0.127325892448425 48.0
0 0.361053824424744 49.0
0.058845978523 0.0818196535110474 50.0
1.5574254181 1.01031637191772 51.0
0 -0.0539580583572388 52.0
0 -0.0255295038223267 53.0
0 0.000501751899719238 54.0
0 -0.124393105506897 55.0
0 -0.00612056255340576 56.0
0 -0.0509260892868042 57.0
0 -0.0734313726425171 58.0
0 -0.0524216890335083 59.0
0 -0.0311928987503052 60.0
0 0.0101255178451538 61.0
0 0.830488443374634 62.0
263.67565341 255.978134155273 0.0
0 -0.0672918558120728 1.0
0 -0.0457085371017456 2.0
0 -0.0200945138931274 3.0
0 -0.0241090059280396 4.0
0 -0.036626935005188 5.0
0 -0.0673772096633911 6.0
0 -0.00146472454071045 7.0
0 0.000139236450195312 8.0
0 0.117246985435486 9.0
0 -0.0274127721786499 10.0
0 -0.0499304533004761 11.0
0 -0.180824875831604 12.0
52.311710681 52.6325950622559 13.0
37.083596385 37.9645004272461 14.0
0 -0.026614785194397 15.0
0 -0.009529709815979 16.0
30.759255512 30.8695621490479 17.0
0 0.00105345249176025 18.0
0 0.0402966737747192 19.0
0 -0.00652182102203369 20.0
0 -0.0424116849899292 21.0
0 0.0489648580551147 22.0
35.578611312 34.3698120117188 23.0
0 -0.00570309162139893 24.0
0 0.582196235656738 25.0
0 0.0661228895187378 26.0
0 0.00117099285125732 27.0
0 0.0341528654098511 28.0
0 -0.0570393800735474 29.0
0 -0.0948672294616699 30.0
0 -0.0233598947525024 31.0
0 -0.0970360040664673 32.0
0 0.0126522779464722 33.0
0 -0.0257564783096313 34.0
0 -0.0458647012710571 35.0
0 0.122306227684021 36.0
0 -0.0945369005203247 37.0
0 -0.0237301588058472 38.0
0 0.0732709169387817 39.0
0 0.184653639793396 40.0
0 -0.0393897294998169 41.0
0 0.0164121389389038 42.0
302.5665184 297.745910644531 43.0
0 0.0247100591659546 44.0
0 0.0462510585784912 45.0
0 0.00461828708648682 46.0
0 -0.00927150249481201 47.0
0 -0.117185711860657 48.0
0 0.269997000694275 49.0
0.068799032607 0.0889362096786499 50.0
1.5304762444 0.914959669113159 51.0
0 -0.0643872022628784 52.0
0 -0.0119916200637817 53.0
0 -0.0255447626113892 54.0
0 -0.0453764200210571 55.0
0 -0.0433362722396851 56.0
0 -0.0189763307571411 57.0
0 0.0151513814926147 58.0
0 -0.0302962064743042 59.0
0 0.0104836225509644 60.0
0 -0.0161322355270386 61.0
0 0.868056416511536 62.0
232.3307681 257.821136474609 0.0
0 -0.142158150672913 1.0
0 -0.112302660942078 2.0
0 -0.0474423170089722 3.0
0 -0.0517724752426147 4.0
0 0.0752745866775513 5.0
0 0.119266867637634 6.0
0 -0.0049813985824585 7.0
0 -0.0426274538040161 8.0
0 0.000416159629821777 9.0
0 -0.0889259576797485 10.0
0 -0.0213658809661865 11.0
0 -0.0119949579238892 12.0
48.15489493 46.1347923278809 13.0
53.035406725 45.7982788085938 14.0
0 -0.00532877445220947 15.0
0 0.0427602529525757 16.0
42.231424022 43.4455261230469 17.0
0 0.00233542919158936 18.0
0 0.03156578540802 19.0
0 -0.00146853923797607 20.0
0 -0.0054241418838501 21.0
0 -0.0128813982009888 22.0
25.01101635 25.2751312255859 23.0
0 0.160364508628845 24.0
0 4.03620529174805 25.0
0 0.0419310331344604 26.0
0 0.0240696668624878 27.0
0 0.0887554883956909 28.0
0 -0.0672975778579712 29.0
0 -0.0638008117675781 30.0
0 -0.0340217351913452 31.0
0 -0.0116034746170044 32.0
0 0.0819867849349976 33.0
0 -0.0160330533981323 34.0
0 -0.0223451852798462 35.0
0 -0.124854922294617 36.0
0 -0.0296074151992798 37.0
0 -0.0305393934249878 38.0
0 0.0676755905151367 39.0
0 0.237629294395447 40.0
0 0.000206112861633301 41.0
0 -0.000583291053771973 42.0
291.21315359 295.380249023438 43.0
0 0.0454491376876831 44.0
0 -0.0385408401489258 45.0
0 -0.00414812564849854 46.0
0 -0.0454118251800537 47.0
0 0.0311874151229858 48.0
0.67752155086 0.254977583885193 49.0
0.066172989946 0.0927853584289551 50.0
2.2350539324 0.870200634002686 51.0
0 -0.0619543790817261 52.0
0 -0.0342041254043579 53.0
0 -0.0522300004959106 54.0
0 -0.158254981040955 55.0
0 0.164599537849426 56.0
0 0.148609519004822 57.0
0 -0.0671349763870239 58.0
0 -0.0539213418960571 59.0
0 -0.0453864336013794 60.0
0 -0.0158287286758423 61.0
0 0.852684736251831 62.0
220.38595671 255.262512207031 0.0
0 -0.0565086603164673 1.0
0 0.00435245037078857 2.0
0 0.028800368309021 3.0
0 -0.0661581754684448 4.0
0 -0.0339735746383667 5.0
0 0.0318130254745483 6.0
0 0.0796030759811401 7.0
0 -0.0766149759292603 8.0
0 -0.110888123512268 9.0
0 0.000422835350036621 10.0
0 0.020386815071106 11.0
0 -0.0530973672866821 12.0
75.809005303 74.1075439453125 13.0
23.098544417 26.1585559844971 14.0
0 -0.0487533807754517 15.0
0 0.00298821926116943 16.0
40.625354281 40.3934288024902 17.0
0 -0.060027003288269 18.0
0 -0.0144976377487183 19.0
0 0.052669882774353 20.0
0 -0.025603175163269 21.0
0 0.04216468334198 22.0
28.186593851 28.5456409454346 23.0
0 -0.017346978187561 24.0
0 0.642506182193756 25.0
0 -0.0122681856155396 26.0
0 0.000667214393615723 27.0
0 0.0480388402938843 28.0
0 -0.168772339820862 29.0
0 0.0138810873031616 30.0
0 -0.0504258871078491 31.0
0 -0.0105763673782349 32.0
0 0.0356062650680542 33.0
0 -0.0609444379806519 34.0
0 -0.0560215711593628 35.0
0 -0.159808993339539 36.0
0 -0.0768057107925415 37.0
0 0.0292192697525024 38.0
0 0.0315624475479126 39.0
0 0.0710514783859253 40.0
0 0.00188624858856201 41.0
0 0.0326894521713257 42.0
287.7375369 295.855834960938 43.0
0 -0.116239190101624 44.0
0 -0.00107502937316895 45.0
0 -0.0293928384780884 46.0
0 -0.00683391094207764 47.0
0 -0.0109869241714478 48.0
0 0.296391248703003 49.0
0.068954088859 0.037601113319397 50.0
1.9287466489 0.961310923099518 51.0
0 -0.101557374000549 52.0
0 -0.0301998853683472 53.0
0 0.0374170541763306 54.0
0 -0.101948142051697 55.0
0 -0.0998026132583618 56.0
0 0.0379847288131714 57.0
0 0.0167220830917358 58.0
0 -0.0256050825119019 59.0
0 -0.0758253335952759 60.0
0 0.0403186082839966 61.0
0 0.841696262359619 62.0
235.26173313 256.967407226562 0.0
0 -0.0725976228713989 1.0
0 0.0289252996444702 2.0
0 0.025969386100769 3.0
0 -0.0672591924667358 4.0
0 0.0195459127426147 5.0
0 0.064264178276062 6.0
0 -0.00535953044891357 7.0
0 0.0151945352554321 8.0
0 -0.0254389047622681 9.0
0 0.0365513563156128 10.0
0 0.0358620882034302 11.0
0 -0.0590959787368774 12.0
60.171260524 59.9101600646973 13.0
37.644956247 36.5196876525879 14.0
0 -0.100855946540833 15.0
0 0.00897300243377686 16.0
36.278208065 36.0699081420898 17.0
0 -0.0298193693161011 18.0
0 -0.031976580619812 19.0
0 0.0406891107559204 20.0
0 -0.0187619924545288 21.0
0 -0.0402399301528931 22.0
37.437973804 36.9222793579102 23.0
0 0.0738133192062378 24.0
0 0.487813949584961 25.0
0 0.026598334312439 26.0
0 0.0533968210220337 27.0
0 -0.0500274896621704 28.0
0 -0.0144006013870239 29.0
0 -0.0198918581008911 30.0
0 -0.0373369455337524 31.0
0 0.0271834135055542 32.0
0 0.0744622945785522 33.0
0 -0.033119797706604 34.0
0 -0.0786365270614624 35.0
0 0.113003611564636 36.0
0 0.015926718711853 37.0
0 0.0227583646774292 38.0
0 0.0371519327163696 39.0
0 0.10942018032074 40.0
0 0.00372219085693359 41.0
0 -0.000874161720275879 42.0
302.52394381 296.425445556641 43.0
0 -0.0887600183486938 44.0
0 0.0294580459594727 45.0
0 0.0102862119674683 46.0
0 -0.0496323108673096 47.0
0 -0.0921703577041626 48.0
0 0.249351143836975 49.0
0.067407977203 0.071515679359436 50.0
1.889961565 0.86248242855072 51.0
0 -0.0698076486587524 52.0
0 -0.0103980302810669 53.0
0 0.0117861032485962 54.0
0 -0.129091858863831 55.0
0 -0.0400835275650024 56.0
0 -0.0256272554397583 57.0
0 -0.0531958341598511 58.0
0 -0.00435721874237061 59.0
0 -0.019100546836853 60.0
0 0.0523163080215454 61.0
0 0.890887677669525 62.0
298.05816727 257.807373046875 0.0
0 -0.146772503852844 1.0
0 -0.030427098274231 2.0
0 0.0131570100784302 3.0
0 -0.0957959890365601 4.0
0 -0.257585167884827 5.0
0 0.0567153692245483 6.0
0 0.013853907585144 7.0
0 0.0836207866668701 8.0
0 -0.0141569375991821 9.0
0 0.00501525402069092 10.0
0 -0.00401425361633301 11.0
0 -0.0568767786026001 12.0
52.219233525 51.6209030151367 13.0
43.000396797 38.584400177002 14.0
0 -0.198574662208557 15.0
0 0.00819504261016846 16.0
31.235369923 32.1309585571289 17.0
0 0.0184684991836548 18.0
0 0.0138062238693237 19.0
0 -0.0103458166122437 20.0
0 0.0245925188064575 21.0
0 0.0448435544967651 22.0
34.983959259 34.4737854003906 23.0
0 0.0788248777389526 24.0
0 0.47409462928772 25.0
0 -0.00201952457427979 26.0
0 -0.00156837701797485 27.0
0 -0.097148060798645 28.0
0 -0.0346881151199341 29.0
0 -0.0532708168029785 30.0
0 -0.0636981725692749 31.0
0 0.0420397520065308 32.0
0 0.119261145591736 33.0
0 -0.0255166292190552 34.0
0 0.00967645645141602 35.0
0 -0.0385862588882446 36.0
0 -0.018919825553894 37.0
0 -0.126260161399841 38.0
0 -0.110346436500549 39.0
0 -0.11169445514679 40.0
0 -0.066336989402771 41.0
0 -0.0825830698013306 42.0
296.56821776 295.362182617188 43.0
0 -0.0105358362197876 44.0
0 -0.000748753547668457 45.0
0 0.00926792621612549 46.0
0 -0.0805879831314087 47.0
0 -0.0106660127639771 48.0
3.6351245034 0.22235095500946 49.0
0.068282151705 0.256266832351685 50.0
1.8112243603 0.854484558105469 51.0
0 -0.0912286043167114 52.0
0 -0.0632849931716919 53.0
0 -0.011921763420105 54.0
0 -0.0887249708175659 55.0
0 -0.0107849836349487 56.0
0 0.00399863719940186 57.0
0 -0.0463706254959106 58.0
0 -0.0903602838516235 59.0
0 -0.0572699308395386 60.0
0 0.0482531785964966 61.0
0 0.81551992893219 62.0
270.23927835 252.670913696289 0.0
0 -0.0658074617385864 1.0
0 -0.0131796598434448 2.0
0 -0.028403639793396 3.0
0 -0.0662964582443237 4.0
0 -0.114679932594299 5.0
0 -0.00388872623443604 6.0
0 -0.0837126970291138 7.0
0 0.0113754272460938 8.0
0 -0.00590193271636963 9.0
0 -0.0227993726730347 10.0
0 -0.0372547507286072 11.0
0 -0.0736805200576782 12.0
56.860920445 56.6013450622559 13.0
51.046141537 40.1702079772949 14.0
0 -0.13849937915802 15.0
0 0.0758265256881714 16.0
39.209754044 39.6843185424805 17.0
0 -0.0464431047439575 18.0
0 0.0180240869522095 19.0
0 0.0445431470870972 20.0
0 -0.00837385654449463 21.0
0 -0.00193750858306885 22.0
30.397317481 30.3171329498291 23.0
0 -0.0206714868545532 24.0
0 0.600813448429108 25.0
0 0.0589760541915894 26.0
0 -0.0502548217773438 27.0
0 0.0245398283004761 28.0
0 -0.0371550321578979 29.0
0 -0.0760459899902344 30.0
0 -0.0201569795608521 31.0
0 -0.00389111042022705 32.0
0 -0.00746524333953857 33.0
0 -0.0538903474807739 34.0
0 -0.0632096529006958 35.0
0 0.104321122169495 36.0
0 -0.0780419111251831 37.0
0 0.047920823097229 38.0
0 0.0590366125106812 39.0
0 0.0474199056625366 40.0
0 -0.000357866287231445 41.0
0 -0.00729620456695557 42.0
298.39678256 294.352996826172 43.0
0 0.0207637548446655 44.0
0 0.0546128153800964 45.0
0 -0.0132936239242554 46.0
0 -0.0632897615432739 47.0
0 -0.0552579164505005 48.0
0 0.255558729171753 49.0
0.082274691007 0.0816240310668945 50.0
2.178440716 0.856557965278625 51.0
0 -0.0616906881332397 52.0
0 -0.0572563409805298 53.0
0 0.0131796598434448 54.0
0 -0.0486105680465698 55.0
0 -0.00872910022735596 56.0
0 -0.00487625598907471 57.0
0 -0.0728038549423218 58.0
0 -0.0702003240585327 59.0
0 -0.037270188331604 60.0
0 0.0693627595901489 61.0
0 0.858768582344055 62.0
240.59310271 257.031158447266 0.0
0 0.00800216197967529 1.0
0 -0.0870212316513062 2.0
0 -0.00145542621612549 3.0
0 -0.0305145978927612 4.0
0 -0.123282551765442 5.0
0 0.00106847286224365 6.0
0 -0.102054238319397 7.0
0 0.0412876605987549 8.0
0 0.0333799123764038 9.0
0 -0.0336252450942993 10.0
0 0.0155443549156189 11.0
0 -0.0522047281265259 12.0
77.619795451 76.5537261962891 13.0
13.211694308 21.9432621002197 14.0
0 0.00340735912322998 15.0
0 0.0316427946090698 16.0
32.241471181 31.5369243621826 17.0
0 -0.00514686107635498 18.0
0 0.0294367074966431 19.0
0 -0.0267645120620728 20.0
0 -0.0309752225875854 21.0
0 0.0453289747238159 22.0
33.003388115 31.7040634155273 23.0
0 0.153749346733093 24.0
0 0.84463906288147 25.0
0 -0.0239230394363403 26.0
0 0.0582197904586792 27.0
0 0.110738158226013 28.0
0 -0.067935585975647 29.0
0 -0.0776602029800415 30.0
0 0.00141441822052002 31.0
0 0.0637601613998413 32.0
0 0.0898069143295288 33.0
0 -0.099467396736145 34.0
0 -0.065937876701355 35.0
0 0.0582033395767212 36.0
0 -0.108708500862122 37.0
0 -0.0488513708114624 38.0
0 -0.00196182727813721 39.0
0 0.112193942070007 40.0
0 -0.0529245138168335 41.0
0 -0.075905442237854 42.0
280.85618501 294.9775390625 43.0
0 0.0299433469772339 44.0
0 -0.0592492818832397 45.0
0 -0.0165027379989624 46.0
0 0.0202113389968872 47.0
0 -0.045606255531311 48.0
0 0.248455941677094 49.0
0.087094521659 0.0646907091140747 50.0
2.1225821682 0.932504296302795 51.0
0 -0.0962387323379517 52.0
0 -0.109156250953674 53.0
0 0.0440183877944946 54.0
0 -0.0620256662368774 55.0
0 0.0154269933700562 56.0
0 0.0267938375473022 57.0
0 0.0126363039016724 58.0
0 -0.0613590478897095 59.0
0 -0.00996148586273193 60.0
0 -0.0513781309127808 61.0
0 0.797938704490662 62.0
288.68310947 259.192169189453 0.0
0 -0.0221716165542603 1.0
0 0.00913488864898682 2.0
0 -0.0512925386428833 3.0
0 -0.0121558904647827 4.0
0 -0.136440873146057 5.0
0 -0.0308147668838501 6.0
0 -0.0635396242141724 7.0
0 0.000926434993743896 8.0
0 0.0908998250961304 9.0
0 -0.0443195104598999 10.0
0 -0.0613163709640503 11.0
0 0.0031970739364624 12.0
54.927716053 54.4053916931152 13.0
42.515306541 40.4506301879883 14.0
0 0.0515426397323608 15.0
0 0.0131508111953735 16.0
28.038908665 27.9881401062012 17.0
0 0.00110328197479248 18.0
0 -0.0132187604904175 19.0
0 0.026222825050354 20.0
0 -0.0167571306228638 21.0
0 0.00269496440887451 22.0
29.083724455 29.2018051147461 23.0
0 -0.0528675317764282 24.0
0 1.49318957328796 25.0
0 0.121333479881287 26.0
0 0.0543397665023804 27.0
0 -0.0514098405838013 28.0
0 -0.0961462259292603 29.0
0 0.0200737714767456 30.0
0 -0.0279709100723267 31.0
0 0.0161975622177124 32.0
0 -0.01520836353302 33.0
0 -0.000324606895446777 34.0
0 0.106852889060974 35.0
0 0.0364090204238892 36.0
0 0.0250653028488159 37.0
0 0.00517499446868896 38.0
0 0.0231658220291138 39.0
0 0.0437620878219604 40.0
0 -0.0248817205429077 41.0
0 -0.00563395023345947 42.0
294.89439326 293.198181152344 43.0
0 0.0111860036849976 44.0
0 0.102861762046814 45.0
0 -0.00079643726348877 46.0
0 -0.0232664346694946 47.0
0 0.0531295537948608 48.0
0 0.382752597332001 49.0
0.092006087383 0.0550380945205688 50.0
2.1377905596 1.00402164459229 51.0
0 -0.0464776754379272 52.0
0 -0.0233110189437866 53.0
0 0.0294681787490845 54.0
0 -0.0536080598831177 55.0
0 -0.0346635580062866 56.0
0 -0.0123130083084106 57.0
0 -0.0285071134567261 58.0
0 -0.0566412210464478 59.0
0 0.0127125978469849 60.0
0 0.0664345026016235 61.0
0 0.901488363742828 62.0
241.01445311 256.367065429688 0.0
0 -0.0531965494155884 1.0
0 -0.0473827123641968 2.0
0 -0.00349390506744385 3.0
0 -0.0241409540176392 4.0
0 0.128177046775818 5.0
0 0.017753005027771 6.0
0 -0.0928357839584351 7.0
0 0.0312454700469971 8.0
0 -0.0720154047012329 9.0
0 -0.0367649793624878 10.0
0 0.0132639408111572 11.0
0 -0.0190759897232056 12.0
69.131995679 69.3090667724609 13.0
17.662589933 30.1896419525146 14.0
0 -0.0348879098892212 15.0
0 -0.0404497385025024 16.0
28.411521536 28.3959083557129 17.0
0 0.0257595777511597 18.0
0 0.0398491621017456 19.0
0 -0.0301676988601685 20.0
0 0.00955069065093994 21.0
0 0.00713145732879639 22.0
35.740490402 35.0099868774414 23.0
0 -0.0194445848464966 24.0
0 0.513082444667816 25.0
0 0.0491980314254761 26.0
0 0.0157319307327271 27.0
0 -0.0658060312271118 28.0
0 -0.0605531930923462 29.0
0 -0.0498552322387695 30.0
0 0.00218307971954346 31.0
0 -0.00752365589141846 32.0
0 -0.00981485843658447 33.0
0 -0.0373579263687134 34.0
0 -0.00713944435119629 35.0
0 0.0201252698898315 36.0
0 -0.00958526134490967 37.0
0 -0.0479046106338501 38.0
0 0.00600063800811768 39.0
0 0.077022910118103 40.0
0 -0.0434916019439697 41.0
0 -0.0848461389541626 42.0
297.30877898 295.491516113281 43.0
0 0.0200470685958862 44.0
0 -0.0456873178482056 45.0
0 0.000341057777404785 46.0
0 -0.0658136606216431 47.0
0 0.0732101202011108 48.0
0 0.342719286680222 49.0
0.078903344481 0.0392047166824341 50.0
2.0753104941 0.905776023864746 51.0
0 -0.0672668218612671 52.0
0 -0.0309568643569946 53.0
0 -0.0553525686264038 54.0
0 -0.140729784965515 55.0
0 -0.0578681230545044 56.0
0 -0.000496506690979004 57.0
0 -0.0788189172744751 58.0
0 -0.0607069730758667 59.0
0 0.0233784914016724 60.0
0 -0.0701144933700562 61.0
0 0.813161373138428 62.0
219.85847512 260.517303466797 0.0
0 -0.116357445716858 1.0
0 -0.0842739343643188 2.0
0 0.0219348669052124 3.0
0 -0.0216923952102661 4.0
0 -0.0189810991287231 5.0
0 -0.131912350654602 6.0
0 -0.0953859090805054 7.0
0 -0.0869439840316772 8.0
0 0.00946462154388428 9.0
0 0.0482257604598999 10.0
0 -0.0446417331695557 11.0
0 -0.0446325540542603 12.0
54.373589458 53.598217010498 13.0
45.400741121 45.3642730712891 14.0
0 -0.029488205909729 15.0
0 0.052855372428894 16.0
34.824889582 35.1966743469238 17.0
0 0.0266770124435425 18.0
0 -0.0339473485946655 19.0
0 0.0489832162857056 20.0
0 0.0376075506210327 21.0
0 -0.039277195930481 22.0
39.860077407 38.3893356323242 23.0
0 0.0285986661911011 24.0
0 0.508388578891754 25.0
0 -0.0112959146499634 26.0
0 0.0582293272018433 27.0
0 -0.0189100503921509 28.0
0 -0.053125262260437 29.0
0 -0.0572408437728882 30.0
0 -0.04753577709198 31.0
0 0.0255602598190308 32.0
0 0.0899742841720581 33.0
0 -0.0271066427230835 34.0
0 0.00567424297332764 35.0
0 0.0268617868423462 36.0
0 0.107263445854187 37.0
0 0.0577801465988159 38.0
0 0.0909239053726196 39.0
0 0.0565136671066284 40.0
0 0.0310831069946289 41.0
0 -0.0678993463516235 42.0
302.41016237 295.396148681641 43.0
0 0.0124591588973999 44.0
0 0.0735231041908264 45.0
0 -0.0623687505722046 46.0
0 -0.050665020942688 47.0
0 0.108898520469666 48.0
2.5023707677 0.267467200756073 49.0
0.065299455483 0.0497637987136841 50.0
1.5048429251 0.899904668331146 51.0
0 -0.0344749689102173 52.0
0 -0.0320380926132202 53.0
0 0.0705209970474243 54.0
0 -0.095439076423645 55.0
0 -0.0474611520767212 56.0
0 -0.0126062631607056 57.0
0 -0.00590312480926514 58.0
0 -0.0268207788467407 59.0
0 -0.00677168369293213 60.0
0 0.0659867525100708 61.0
0 0.91569846868515 62.0
212.26781159 255.468704223633 0.0
0 -0.0749956369400024 1.0
0 -0.0972839593887329 2.0
0 0.000664114952087402 3.0
0 -0.0691694021224976 4.0
0 -0.11175811290741 5.0
0 0.0402730703353882 6.0
0 -0.0207720994949341 7.0
0 -0.0239511132240295 8.0
0 0.103009581565857 9.0
0 -0.032991886138916 10.0
0 -0.0666753053665161 11.0
0 0.0760120153427124 12.0
69.331706772 67.9982757568359 13.0
34.542107074 30.2001113891602 14.0
0 -0.0158325433731079 15.0
0 -0.0219312906265259 16.0
40.854444328 40.7578506469727 17.0
0 -0.0111459493637085 18.0
0 0.0127075910568237 19.0
0 0.0348876714706421 20.0
0 -0.0336519479751587 21.0
0 0.0376147031784058 22.0
35.644893517 34.039119720459 23.0
0 -0.0400673151016235 24.0
0 0.817991137504578 25.0
0 -0.0291949510574341 26.0
0 0.0323876142501831 27.0
0 0.00994193553924561 28.0
0 -0.100441098213196 29.0
0 -0.0492032766342163 30.0
0 0.0899296998977661 31.0
0 -0.0132070779800415 32.0
0 0.0959652662277222 33.0
0 -4.56571578979492e-05 34.0
0 -0.0614632368087769 35.0
0 0.0300732851028442 36.0
0 0.0161958932876587 37.0
0 -0.0392636060714722 38.0
0 0.131372094154358 39.0
0 0.028680682182312 40.0
0 -0.0619829893112183 41.0
0 -0.0664840936660767 42.0
295.20429435 297.537719726562 43.0
0 -0.0257107019424438 44.0
0 0.0301533937454224 45.0
0 -0.0487526655197144 46.0
0 -0.0323265790939331 47.0
0 -0.052709698677063 48.0
0 0.311076492071152 49.0
0.082006154749 0.0434678792953491 50.0
1.7727399108 0.95056164264679 51.0
0 -0.0552364587783813 52.0
0 -0.0196360349655151 53.0
0 0.0519989728927612 54.0
0 -0.0941072702407837 55.0
0 -0.0644482374191284 56.0
0 -0.00803911685943604 57.0
0 -0.0538657903671265 58.0
0 -0.0481668710708618 59.0
0 -0.0360532999038696 60.0
0 -0.0594574213027954 61.0
0 0.844719648361206 62.0
286.87036867 249.182907104492 0.0
0 -0.0395699739456177 1.0
0 -0.0594953298568726 2.0
0 -0.0293096303939819 3.0
0 0.00879251956939697 4.0
0 -0.0522357225418091 5.0
0 0.174208521842957 6.0
0 0.0642369985580444 7.0
0 0.0323923826217651 8.0
0 0.041739821434021 9.0
0 -0.0242255926132202 10.0
0 0.00908458232879639 11.0
0 -0.0257049798965454 12.0
46.68744948 45.3694343566895 13.0
42.538988478 40.3886413574219 14.0
0 -0.104539513587952 15.0
0 0.0681780576705933 16.0
41.726721964 40.8582725524902 17.0
0 0.056904673576355 18.0
0 -0.0158146619796753 19.0
0 0.0388361215591431 20.0
0 0.0383785963058472 21.0
0 -0.069891095161438 22.0
31.800116824 31.960657119751 23.0
0 -0.0611578226089478 24.0
6.0105823476 0.529408514499664 25.0
0 0.0550936460494995 26.0
0 0.020159125328064 27.0
0 -0.0516930818557739 28.0
0 -0.00828731060028076 29.0
0 -0.118024706840515 30.0
0 0.0289763212203979 31.0
0 0.0822609663009644 32.0
0 0.0672043561935425 33.0
0 -0.0302613973617554 34.0
0 -0.0780788660049438 35.0
0 -0.0786594152450562 36.0
0 -0.0584789514541626 37.0
0 -0.0305322408676147 38.0
0 0.0815004110336304 39.0
0 0.351662039756775 40.0
0 0.0434364080429077 41.0
0 -0.154787659645081 42.0
268.67405968 294.032897949219 43.0
0 0.0305541753768921 44.0
0 0.0455721616744995 45.0
0 0.00151145458221436 46.0
0 -0.0426315069198608 47.0
0 0.0539983510971069 48.0
0.68782398798 0.44253945350647 49.0
0.059818228005 0.146080374717712 50.0
1.664009326 1.00553250312805 51.0
0 -0.12184202671051 52.0
0 -0.0456539392471313 53.0
0 0.0991572141647339 54.0
0 -0.0797246694564819 55.0
0 -0.0659216642379761 56.0
0 -0.0283998250961304 57.0
0 -0.0442284345626831 58.0
0 -0.00239932537078857 59.0
0 -0.0469440221786499 60.0
0 0.0303648710250854 61.0
0 0.840306639671326 62.0
260.32997917 254.974655151367 0.0
0 -0.0764261484146118 1.0
0 -0.0317126512527466 2.0
0 0.175824522972107 3.0
0 -0.0206290483474731 4.0
0 0.0124596357345581 5.0
0 0.0805987119674683 6.0
0 0.0281051397323608 7.0
0 -0.0359913110733032 8.0
0 -0.0108157396316528 9.0
0 -0.0256839990615845 10.0
0 0.0471983551979065 11.0
0 -0.117671608924866 12.0
53.794217428 53.30322265625 13.0
29.555356257 42.9166221618652 14.0
0 -0.0620394945144653 15.0
0 -0.0613940954208374 16.0
36.132153611 35.1274642944336 17.0
0 0.0523771047592163 18.0
0 0.00691115856170654 19.0
0 -0.0118292570114136 20.0
0 -0.0103229284286499 21.0
0 -0.0490633249282837 22.0
32.684494684 32.3469009399414 23.0
0 -0.0896211862564087 24.0
0 0.687761425971985 25.0
0 0.0384162664413452 26.0
0 0.016898512840271 27.0
0 0.0424503087997437 28.0
0 -0.0225082635879517 29.0
0 -0.00472474098205566 30.0
0 -0.0268062353134155 31.0
0 -0.0101867914199829 32.0
0 0.18911874294281 33.0
0 -0.100137591362 34.0
0 0.0400484800338745 35.0
0 -0.0310686826705933 36.0
0 0.0965996980667114 37.0
0 -0.0705329179763794 38.0
0 -0.0587784051895142 39.0
0 0.0174411535263062 40.0
0 -0.074039101600647 41.0
0 -0.0515638589859009 42.0
298.25777384 299.012084960938 43.0
0 0.0371359586715698 44.0
0 -0.0750389099121094 45.0
0 -0.00238549709320068 46.0
0 -0.0836089849472046 47.0
0 -0.0431634187698364 48.0
0 0.308436095714569 49.0
0.10499294142 0.0462206602096558 50.0
1.8857461447 0.944174408912659 51.0
0 -0.0656906366348267 52.0
0 -0.074687123298645 53.0
0 0.0379190444946289 54.0
0 -0.0355070829391479 55.0
0 -0.0167449712753296 56.0
0 -0.0189021825790405 57.0
0 -0.112332701683044 58.0
0 -0.0240000486373901 59.0
0 0.0481294393539429 60.0
0 -0.0053020715713501 61.0
0 0.91425222158432 62.0
276.44899054 256.59716796875 0.0
0 -0.0255972146987915 1.0
0 -0.0880473852157593 2.0
0 -0.0461817979812622 3.0
0 -0.0378991365432739 4.0
0 -0.0221978425979614 5.0
0 -0.0740891695022583 6.0
0 -0.0133830308914185 7.0
0 -0.045198917388916 8.0
0 0.026577353477478 9.0
0 -0.0305644273757935 10.0
0 -0.0405479669570923 11.0
0 -0.0733734369277954 12.0
51.067754552 49.728816986084 13.0
38.532063453 39.1149635314941 14.0
0 0.0505536794662476 15.0
0 -0.0603545904159546 16.0
39.585584893 38.845588684082 17.0
0 0.00242483615875244 18.0
0 -0.0210062265396118 19.0
0 -0.0465527772903442 20.0
0 -0.0165773630142212 21.0
0 -0.0204211473464966 22.0
27.558684383 27.5015716552734 23.0
0 -0.0168634653091431 24.0
0 0.50989305973053 25.0
0 0.0457366704940796 26.0
0 0.0296810865402222 27.0
0 -0.0382572412490845 28.0
0 -0.0198968648910522 29.0
0 -0.0279308557510376 30.0
0 -0.0694109201431274 31.0
0 0.02635657787323 32.0
0 0.00292575359344482 33.0
0 -0.0138112306594849 34.0
0 0.00353562831878662 35.0
0 0.0934475660324097 36.0
0 -0.0376826524734497 37.0
0 -0.0330818891525269 38.0
0 0.0845683813095093 39.0
0 0.0923384428024292 40.0
0 -0.075268030166626 41.0
0 -0.0356289148330688 42.0
288.39837811 296.401916503906 43.0
0 0.0600842237472534 44.0
0 -0.0686542987823486 45.0
0 0.0010303258895874 46.0
0 -0.0627387762069702 47.0
0 0.0289057493209839 48.0
1.2085313439 0.299822956323624 49.0
0.08918143139 0.0669664144515991 50.0
1.5537229536 0.944699466228485 51.0
0 -0.0713740587234497 52.0
0 -0.0575131177902222 53.0
0 0.05219566822052 54.0
0 -0.0572477579116821 55.0
0 -0.0804191827774048 56.0
0 0.0123552083969116 57.0
0 -0.0641387701034546 58.0
0 -0.0602058172225952 59.0
0 -0.0137652158737183 60.0
0 -0.0321072340011597 61.0
0 0.82669472694397 62.0
286.04089005 256.494171142578 0.0
0 -0.0530511140823364 1.0
0 -0.0605043172836304 2.0
0 0.0335289239883423 3.0
0 0.020166277885437 4.0
0 -0.0380698442459106 5.0
0 -0.0560027360916138 6.0
0 -0.0142809152603149 7.0
0 0.0149096250534058 8.0
0 -0.0748015642166138 9.0
0 0.00306665897369385 10.0
0 -0.0165442228317261 11.0
0 -0.0279985666275024 12.0
67.496237578 66.0958023071289 13.0
21.767624332 30.6990432739258 14.0
0 -0.0805660486221313 15.0
0 -0.00490033626556396 16.0
28.503457818 27.9883041381836 17.0
0 0.0216602087020874 18.0
0 -0.0322285890579224 19.0
0 -0.0138286352157593 20.0
0 0.0256346464157104 21.0
0 -0.0695940256118774 22.0
28.919316228 27.693151473999 23.0
0 0.0651277303695679 24.0
0 0.535855412483215 25.0
0 0.00367796421051025 26.0
0 0.135670065879822 27.0
0 0.045124888420105 28.0
0 -0.0699397325515747 29.0
0 0.140351414680481 30.0
0 -0.0310860872268677 31.0
0 2.89678573608398e-05 32.0
0 -0.0941764116287231 33.0
0 -0.027441143989563 34.0
0 -0.178986072540283 35.0
0 -0.107413649559021 36.0
0 -0.0360716581344604 37.0
0 -0.0135177373886108 38.0
0 -0.0162471532821655 39.0
0 0.108281493186951 40.0
0 0.00549721717834473 41.0
0 0.0408881902694702 42.0
298.82843799 297.785614013672 43.0
0 -0.0144573450088501 44.0
0 0.111433625221252 45.0
0 -0.0512713193893433 46.0
0 -0.0450409650802612 47.0
0 -0.0105654001235962 48.0
0 0.276061534881592 49.0
0.090661758896 0.1297447681427 50.0
1.620437118 0.864548087120056 51.0
0 -0.101343750953674 52.0
0 -0.00842726230621338 53.0
0 -0.0273200273513794 54.0
0 -0.118294358253479 55.0
0 -0.0995038747787476 56.0
0 0.00664651393890381 57.0
0 -0.00133597850799561 58.0
0 -0.0811980962753296 59.0
0 -0.0519198179244995 60.0
0 -0.0186730623245239 61.0
0 0.895458817481995 62.0
262.06197997 257.563171386719 0.0
0 -0.122519612312317 1.0
0 -0.0872961282730103 2.0
0 0.00361812114715576 3.0
0 -0.0290306806564331 4.0
0 0.00896584987640381 5.0
0 -0.158955216407776 6.0
0 0.0699185132980347 7.0
0 -0.0694426298141479 8.0
0 -0.0794359445571899 9.0
0 -0.0296984910964966 10.0
0 -0.0282889604568481 11.0
0 -0.128122925758362 12.0
52.012106648 52.2091903686523 13.0
41.3491415 42.4242744445801 14.0
0 -0.0300670862197876 15.0
0 0.0566724538803101 16.0
35.292382721 33.1318130493164 17.0
0 -0.0202735662460327 18.0
0 0.00158679485321045 19.0
0 -0.0154541730880737 20.0
0 -0.0481914281845093 21.0
0 -0.000110745429992676 22.0
29.988754544 28.7600593566895 23.0
0 0.011867880821228 24.0
0 1.26874375343323 25.0
0 0.0592964887619019 26.0
0 0.0536972284317017 27.0
0 -0.0407558679580688 28.0
0 -0.0520280599594116 29.0
0 -0.0413797497749329 30.0
0 -0.0139380693435669 31.0
0 0.116114020347595 32.0
0 0.102980017662048 33.0
0 -0.0838218927383423 34.0
0 -0.00230801105499268 35.0
0 0.042136549949646 36.0
0 -0.114351630210876 37.0
0 0.0215176343917847 38.0
0 0.0502623319625854 39.0
0 -0.0284839868545532 40.0
0 -0.0426565408706665 41.0
0 0.0440906286239624 42.0
306.312585 296.211242675781 43.0
0 -0.00857317447662354 44.0
0 0.0365154147148132 45.0
0 -0.0110198259353638 46.0
0 -0.123771071434021 47.0
0 -0.0409308671951294 48.0
0 0.251649260520935 49.0
0.09814811083 0.213762044906616 50.0
1.8141370145 0.859546422958374 51.0
0 -0.0822616815567017 52.0
0 -0.00910651683807373 53.0
0 -0.0032273530960083 54.0
0 -0.0593606233596802 55.0
0 -0.0747696161270142 56.0
0 0.022365927696228 57.0
0 -0.00224697589874268 58.0
0 -0.0319935083389282 59.0
0 -0.00494182109832764 60.0
0 0.0262881517410278 61.0
0 0.821298360824585 62.0
282.29800653 255.657699584961 0.0
0 -0.0803581476211548 1.0
0 -0.0205718278884888 2.0
0 -0.0521920919418335 3.0
0 -0.0362566709518433 4.0
0 -0.0200968980789185 5.0
0 0.124482035636902 6.0
0 -0.168305516242981 7.0
0 -0.0164570808410645 8.0
0 0.0528777837753296 9.0
0 0.00267493724822998 10.0
0 0.0120260715484619 11.0
0 -0.0428968667984009 12.0
64.522614861 64.7330017089844 13.0
26.340123109 32.4533500671387 14.0
0 0.0169833898544312 15.0
0 -0.0406230688095093 16.0
27.440309454 27.3812255859375 17.0
0 0.0207191705703735 18.0
0 -0.0206142663955688 19.0
0 0.027812123298645 20.0
0 0.00370895862579346 21.0
0 -0.0201574563980103 22.0
38.725814425 38.074649810791 23.0
0 0.0181750059127808 24.0
0 0.511634588241577 25.0
0 -0.0263062715530396 26.0
0 0.0321397185325623 27.0
0 0.0784176588058472 28.0
0 -0.0399934053421021 29.0
0 -0.06260085105896 30.0
0 -0.068292498588562 31.0
0 0.00036466121673584 32.0
0 0.0547500848770142 33.0
0 0.0127214193344116 34.0
0 -0.00662720203399658 35.0
0 0.0746716260910034 36.0
0 -0.0637596845626831 37.0
0 0.00548660755157471 38.0
0 0.00332963466644287 39.0
0 0.0609792470932007 40.0
0 -0.0439552068710327 41.0
0 -0.0142427682876587 42.0
295.22529275 295.486694335938 43.0
0 0.0166267156600952 44.0
0 0.0583740472793579 45.0
0 -0.0559285879135132 46.0
0 -0.0305360555648804 47.0
0 -0.0869718790054321 48.0
0 0.25745964050293 49.0
0.084778115279 0.0760613679885864 50.0
1.9927972116 0.982480645179749 51.0
0 -0.084903359413147 52.0
0 -0.00382602214813232 53.0
0 0.0254437923431396 54.0
0 -0.0682462453842163 55.0
0 -0.0458301305770874 56.0
0 -0.0393162965774536 57.0
0 -0.0571831464767456 58.0
0 -0.0567752122879028 59.0
0 0.0239511728286743 60.0
0 -0.00200140476226807 61.0
0 0.827014446258545 62.0
281.0586102 253.771484375 0.0
0 -0.033298134803772 1.0
0 -0.0711365938186646 2.0
0 0.00043332576751709 3.0
0 -0.0203155279159546 4.0
0 -0.09540855884552 5.0
0 0.0623449087142944 6.0
0 -0.00439846515655518 7.0
0 -0.054167628288269 8.0
0 0.105441451072693 9.0
0 0.034969687461853 10.0
0 -0.0152429342269897 11.0
0 -0.102641701698303 12.0
53.325408415 54.0800971984863 13.0
37.395221491 37.9774742126465 14.0
0 0.00294339656829834 15.0
0 -0.0414067506790161 16.0
32.285681142 31.8384056091309 17.0
0 0.0472851991653442 18.0
0 -0.0107539892196655 19.0
0 -0.0209888219833374 20.0
0 -0.0429137945175171 21.0
0 0.0136617422103882 22.0
38.741625692 37.8661155700684 23.0
0 -0.106185078620911 24.0
0 0.700532197952271 25.0
0 -0.0280381441116333 26.0
0 0.00883638858795166 27.0
0 -0.0170632600784302 28.0
0 -0.0230602025985718 29.0
0 -0.0542900562286377 30.0
0 -0.0480514764785767 31.0
0 0.0784710645675659 32.0
0 0.0131500959396362 33.0
0 -0.0326067209243774 34.0
0 -0.0347213745117188 35.0
0 0.000506758689880371 36.0
0 0.0624393224716187 37.0
0 -0.0322376489639282 38.0
0 0.0931891202926636 39.0
0 0.0579508543014526 40.0
0 -0.0360230207443237 41.0
0 -0.0268169641494751 42.0
298.47339062 296.48779296875 43.0
0 0.0109542608261108 44.0
0 0.0192062854766846 45.0
0 -0.034410834312439 46.0
0 -0.0186175107955933 47.0
0 0.0220788717269897 48.0
0 0.243896186351776 49.0
0.074228196102 0.0937658548355103 50.0
1.714387499 1.01033544540405 51.0
0 -0.0633841753005981 52.0
0 -0.0424274206161499 53.0
0 -0.00475394725799561 54.0
0 -0.0953003168106079 55.0
0 -0.00340092182159424 56.0
0 -0.022466778755188 57.0
0 -0.0498300790786743 58.0
0 -0.0191155672073364 59.0
0 0.0761951208114624 60.0
0 -0.0466970205307007 61.0
0 0.840783834457397 62.0
267.36616709 255.405776977539 0.0
0 -0.125655293464661 1.0
0 0.0127142667770386 2.0
0 -0.0673490762710571 3.0
0 -0.0359312295913696 4.0
0 -0.0795363187789917 5.0
0 -0.160844922065735 6.0
0 -0.0261713266372681 7.0
0 0.0191556215286255 8.0
0 -0.0497585535049438 9.0
0 -0.0145708322525024 10.0
0 0.0314761400222778 11.0
0 -0.0834766626358032 12.0
59.51851307 58.4854850769043 13.0
32.089554859 37.3556861877441 14.0
0 0.0365308523178101 15.0
0 -0.0289736986160278 16.0
32.613319972 32.0527153015137 17.0
0 0.0274523496627808 18.0
0 0.0347424745559692 19.0
0 0.0253092050552368 20.0
0 0.00853431224822998 21.0
0 0.00950610637664795 22.0
28.455497503 27.3769702911377 23.0
0 0.0165075063705444 24.0
0 0.508569955825806 25.0
0 0.0930866003036499 26.0
0 0.0755137205123901 27.0
0 -0.0476962327957153 28.0
0 -0.0262638330459595 29.0
0 -0.0862493515014648 30.0
0 -0.020433783531189 31.0
0 -0.0108901262283325 32.0
0 0.0966018438339233 33.0
0 0.0235098600387573 34.0
0 -0.0837360620498657 35.0
0 0.0185548067092896 36.0
0 0.0213998556137085 37.0
0 -0.0197411775588989 38.0
0 0.0675657987594604 39.0
0 0.0345267057418823 40.0
0 -0.0637422800064087 41.0
0 0.0712417364120483 42.0
295.09069198 294.392272949219 43.0
0 0.0114949941635132 44.0
0 0.0493676662445068 45.0
0 0.0116368532180786 46.0
0 -0.0565756559371948 47.0
0 -0.000534653663635254 48.0
0 0.294281184673309 49.0
0.077222902259 0.08582603931427 50.0
2.2243827985 1.01860928535461 51.0
0 -0.0914934873580933 52.0
0 -0.0104542970657349 53.0
0 0.022612452507019 54.0
0 -0.0759516954421997 55.0
0 -0.0416489839553833 56.0
0 -0.00947606563568115 57.0
0 -0.0376816987991333 58.0
0 -0.00787413120269775 59.0
0 0.000312209129333496 60.0
0 -0.00591456890106201 61.0
0 0.794553279876709 62.0
279.69863451 255.80876159668 0.0
0 0.00460994243621826 1.0
0 -0.061862587928772 2.0
0 -0.0960882902145386 3.0
0 -0.0472513437271118 4.0
0 -0.0296269655227661 5.0
0 0.0440934896469116 6.0
0 -0.109561085700989 7.0
0 -0.000579416751861572 8.0
0 0.043362021446228 9.0
0 0.00915253162384033 10.0
0 -0.0055009126663208 11.0
0 0.0728696584701538 12.0
52.722140636 50.9147605895996 13.0
49.177139343 49.7826766967773 14.0
0 -0.0547944307327271 15.0
0 -0.0213414430618286 16.0
26.778336194 28.2771625518799 17.0
0 0.00420200824737549 18.0
0 -0.012717604637146 19.0
0 0.0307968854904175 20.0
0 -0.0154820680618286 21.0
0 -0.00298035144805908 22.0
24.585865192 24.8051929473877 23.0
0 -0.105185627937317 24.0
0 0.759682536125183 25.0
0 0.0373066663742065 26.0
0 0.00444638729095459 27.0
0 -0.0807467699050903 28.0
0 0.0178872346878052 29.0
0 -0.0894912481307983 30.0
0 -0.0636073350906372 31.0
0 0.0836461782455444 32.0
0 -0.0342427492141724 33.0
0 0.0115681886672974 34.0
0 0.0615721940994263 35.0
0 0.0182400941848755 36.0
0 -0.134029507637024 37.0
0 -0.0583435297012329 38.0
0 0.0711148977279663 39.0
0 0.07506263256073 40.0
0 0.137745141983032 41.0
0 -0.0642610788345337 42.0
282.13151882 293.318115234375 43.0
0 -0.0273281335830688 44.0
0 0.0411696434020996 45.0
0 0.00809395313262939 46.0
0 -0.0591593980789185 47.0
0 0.032678484916687 48.0
0 0.298319041728973 49.0
0.064330153962 0.0631126165390015 50.0
1.73731272 1.02044153213501 51.0
0 -0.0369518995285034 52.0
0 0.00657975673675537 53.0
0 0.0494717359542847 54.0
0 -0.109314799308777 55.0
0 -0.0907317399978638 56.0
0 0.0494774580001831 57.0
0 -0.0806807279586792 58.0
0 -0.00974190235137939 59.0
0 -0.00307786464691162 60.0
0 -0.0531107187271118 61.0
0 0.857807278633118 62.0
247.90429988 254.137451171875 0.0
0 -0.0245720148086548 1.0
0 -0.00392544269561768 2.0
0 0.0454906225204468 3.0
0 -0.0368314981460571 4.0
0 -0.0152920484542847 5.0
0 -0.0351415872573853 6.0
0 -0.0137232542037964 7.0
0 0.0321581363677979 8.0
0 0.0403722524642944 9.0
0 0.00702905654907227 10.0
0 -0.0377731919288635 11.0
0 -0.0574451684951782 12.0
67.228804805 67.5553283691406 13.0
31.005208712 30.526668548584 14.0
0 -0.182921528816223 15.0
0 0.0419682264328003 16.0
29.020012376 29.8198184967041 17.0
0 -0.0298110246658325 18.0
0 0.0346416234970093 19.0
0 -0.00142872333526611 20.0
0 -0.0321484804153442 21.0
0 -0.017220139503479 22.0
29.093371863 28.5788631439209 23.0
0 -0.117791295051575 24.0
0 0.624041318893433 25.0
0 0.117148756980896 26.0
0 0.0939596891403198 27.0
0 0.216269850730896 28.0
0 -0.0389364957809448 29.0
0 -0.0308955907821655 30.0
0 -0.0642591714859009 31.0
0 -0.0661345720291138 32.0
0 0.17190420627594 33.0
0 -0.0433667898178101 34.0
0 -0.0128817558288574 35.0
0 0.041653037071228 36.0
0 0.000727295875549316 37.0
0 -0.141344904899597 38.0
0 -0.0382460355758667 39.0
0 0.00682961940765381 40.0
0 -0.0316864252090454 41.0
0 -0.0287262201309204 42.0
305.5731127 293.4013671875 43.0
0 0.0404952764511108 44.0
0 0.0392992496490479 45.0
0 -0.0355578660964966 46.0
0 -0.076367974281311 47.0
0 -0.0234333276748657 48.0
0 0.308216035366058 49.0
0.094667944011 0.0748189687728882 50.0
1.4407456639 0.973816335201263 51.0
0 -0.0877183675765991 52.0
0 -0.0205103158950806 53.0
0 -0.000241875648498535 54.0
0 -0.0717333555221558 55.0
0 -0.0983873605728149 56.0
0 0.0133508443832397 57.0
0 -0.0184944868087769 58.0
0 -0.0139414072036743 59.0
0 -0.135028958320618 60.0
0 0.0621153116226196 61.0
0 0.939949095249176 62.0
278.63259851 259.044677734375 0.0
0 -0.0703097581863403 1.0
0 -0.0478435754776001 2.0
0 -0.0274237394332886 3.0
0 -0.073517918586731 4.0
0 -0.109848618507385 5.0
0 0.00819146633148193 6.0
0 -0.183600544929504 7.0
0 -0.0909026265144348 8.0
0 -0.0104919672012329 9.0
0 -0.0184487104415894 10.0
0 0.0458089113235474 11.0
0 0.0454686880111694 12.0
52.502901386 52.2059631347656 13.0
35.312734919 33.461254119873 14.0
0 -0.0232993364334106 15.0
0 -0.00697863101959229 16.0
33.47810802 33.9430923461914 17.0
0 0.00754368305206299 18.0
0 -0.0473097562789917 19.0
0 -0.0293833017349243 20.0
0 -0.0168880224227905 21.0
0 0.0238412618637085 22.0
37.645697034 36.9412879943848 23.0
0 0.0351637601852417 24.0
0 0.544121325016022 25.0
0 0.081384539604187 26.0
0 0.0777033567428589 27.0
0 -0.0805188417434692 28.0
0 -0.004402756690979 29.0
0 -0.109715938568115 30.0
0 -0.0507038831710815 31.0
0 0.00374925136566162 32.0
0 -0.0512814521789551 33.0
0 -0.00722229480743408 34.0
0 0.0127847194671631 35.0
0 0.044049859046936 36.0
0 -0.0563896894454956 37.0
0 0.0321260690689087 38.0
0 0.0921682119369507 39.0
0 0.161925196647644 40.0
0 -0.0155901908874512 41.0
0 -0.0445343255996704 42.0
275.12698703 295.317749023438 43.0
0 0.0193160772323608 44.0
0 0.0715713500976562 45.0
0 -0.0168114900588989 46.0
0 -0.00225365161895752 47.0
0 -0.0054394006729126 48.0
1.3918829858 0.343426495790482 49.0
0.095072127127 0.0831505060195923 50.0
2.0717044307 0.883623480796814 51.0
0 -0.0326424837112427 52.0
0 -0.0448287725448608 53.0
0 -0.020605206489563 54.0
0 0.000653862953186035 55.0
0 0.0169378519058228 56.0
0 0.0344225168228149 57.0
0 -0.0719095468521118 58.0
0 -0.0194810628890991 59.0
0 -0.0223592519760132 60.0
0 0.0275846719741821 61.0
0 0.892852425575256 62.0
283.24356847 254.561889648438 0.0
0 0.0159090757369995 1.0
0 0.0494295358657837 2.0
0 -0.0449672937393188 3.0
0 -0.0220485925674438 4.0
0 -0.282414555549622 5.0
0 17.3479480743408 6.0
0 -0.178012013435364 7.0
0 0.0738468170166016 8.0
0 0.0818008184432983 9.0
0 -0.0409530401229858 10.0
0 -0.0515080690383911 11.0
0 -0.0320688486099243 12.0
78.793043392 77.8468399047852 13.0
5.9688303395 22.1471347808838 14.0
0 -0.00824129581451416 15.0
0 -0.0155812501907349 16.0
37.401295111 37.9038162231445 17.0
0 0.105312705039978 18.0
0 -0.018743634223938 19.0
0 0.036925196647644 20.0
0 0.0369025468826294 21.0
0 -0.0571824312210083 22.0
37.515664868 36.4768600463867 23.0
0 -0.107249855995178 24.0
0 2.28882646560669 25.0
0 0.0792092084884644 26.0
0 0.16162383556366 27.0
0 -0.0343862771987915 28.0
0 -0.0543447732925415 29.0
0 -0.0975613594055176 30.0
0 -0.03168785572052 31.0
0 -0.0321007966995239 32.0
0 -0.0119825601577759 33.0
0 -0.0339168310165405 34.0
0 -0.0644890069961548 35.0
0 0.0092545747756958 36.0
0 0.0325225591659546 37.0
0 -0.00544774532318115 38.0
0 0.00625741481781006 39.0
0 0.0880941152572632 40.0
0 -0.0486897230148315 41.0
0 -0.167999386787415 42.0
287.09574607 293.266876220703 43.0
0 0.0595206022262573 44.0
0 -0.0811586380004883 45.0
0 -0.0626004934310913 46.0
0 -0.0576306581497192 47.0
0 0.0258330106735229 48.0
0 0.400069177150726 49.0
0.073718297104 0.0464328527450562 50.0
1.4351162071 1.00965857505798 51.0
0 -0.0783547163009644 52.0
0 -0.0535227060317993 53.0
0 0.0298179388046265 54.0
0 -0.091436505317688 55.0
0 -0.044247031211853 56.0
0 -0.0142122507095337 57.0
0 -0.049704909324646 58.0
0 -0.0203930139541626 59.0
0 -0.0544201135635376 60.0
0 0.0725506544113159 61.0
0 0.951801896095276 62.0
287.03912195 265.234924316406 0.0
0 -0.0894628763198853 1.0
0 -0.146519303321838 2.0
0 0.0409806966781616 3.0
0 0.0416406393051147 4.0
0 -0.0258208513259888 5.0
0 0.548276305198669 6.0
0 -0.087774395942688 7.0
0 -0.0724334716796875 8.0
0 0.0281480550765991 9.0
0 0.025076150894165 10.0
0 0.00354564189910889 11.0
0 -0.0576902627944946 12.0
74.904855998 74.1834411621094 13.0
35.309664555 23.705659866333 14.0
0 -0.0429240465164185 15.0
0 0.0158447027206421 16.0
31.201515712 31.3031425476074 17.0
0 -0.0629144906997681 18.0
0 -0.0361305475234985 19.0
0 -0.0130425691604614 20.0
0 -0.0383292436599731 21.0
0 -0.0652023553848267 22.0
33.624895939 32.835506439209 23.0
0 0.163848757743835 24.0
0 0.431499361991882 25.0
0 0.0664657354354858 26.0
0 0.0423709154129028 27.0
0 0.320572733879089 28.0
0 -0.0853801965713501 29.0
0 0.098355770111084 30.0
0 -0.0570584535598755 31.0
0 -0.00851690769195557 32.0
0 0.000650525093078613 33.0
0 -0.00318801403045654 34.0
0 0.0218592882156372 35.0
0 0.0290991067886353 36.0
0 -0.0550490617752075 37.0
0 0.0285400152206421 38.0
0 0.146520018577576 39.0
0 0.262939810752869 40.0
0 -0.0258013010025024 41.0
0 -0.0394960641860962 42.0
290.01885354 296.877166748047 43.0
0 0.0453060865402222 44.0
0 -0.0598816871643066 45.0
0 -0.0157221555709839 46.0
0 -0.0257846117019653 47.0
0 0.0973776578903198 48.0
2.38711896 0.271843433380127 49.0
0.10105636817 0.0735448598861694 50.0
1.8364142282 1.01069974899292 51.0
0 -0.0480571985244751 52.0
0 -0.0561081171035767 53.0
0 0.00245344638824463 54.0
0 -0.0890268087387085 55.0
0 0.0167649984359741 56.0
0 -0.0394488573074341 57.0
0 -0.0316585302352905 58.0
0 -0.0526994466781616 59.0
0 0.0510767698287964 60.0
0 -0.0178877115249634 61.0
0 0.855759382247925 62.0
210.3732136 254.203048706055 0.0
0 -0.0683656930923462 1.0
0 0.00906264781951904 2.0
0 -0.0410431623458862 3.0
0 -0.0485466718673706 4.0
0 -0.0248191356658936 5.0
0 0.0936673879623413 6.0
0 0.144901394844055 7.0
0 -0.055451512336731 8.0
0 -0.0376125574111938 9.0
0 -0.0727325677871704 10.0
0 -0.0309776067733765 11.0
0 -0.0163618326187134 12.0
70.222261391 69.9111099243164 13.0
9.5193853789 29.003719329834 14.0
0 -0.1884685754776 15.0
0 0.00574672222137451 16.0
32.995929292 32.5170555114746 17.0
0 -0.0276833772659302 18.0
0 -0.0161341428756714 19.0
0 -0.0166388750076294 20.0
0 0.00257766246795654 21.0
0 0.0547159910202026 22.0
28.556578551 28.526762008667 23.0
0 -0.0140753984451294 24.0
0 1.0903605222702 25.0
0 0.0387152433395386 26.0
0 -0.0248827934265137 27.0
0 -0.00288116931915283 28.0
0 0.0184861421585083 29.0
0 0.0261949300765991 30.0
0 -0.0706771612167358 31.0
0 0.0830711126327515 32.0
0 -0.100840210914612 33.0
0 -0.0053640604019165 34.0
0 -0.00570166110992432 35.0
0 0.116679310798645 36.0
0 -0.0159653425216675 37.0
0 -0.0503894090652466 38.0
0 -0.0343486070632935 39.0
0 0.077846884727478 40.0
0 0.320860862731934 41.0
0 -0.0846096277236938 42.0
291.7244426 293.514251708984 43.0
0 -0.0717712640762329 44.0
0 0.0221196413040161 45.0
0 -0.0494734048843384 46.0
0 0.0153261423110962 47.0
0 0.0669130086898804 48.0
0 0.386417269706726 49.0
0.068969803772 0.0427128076553345 50.0
1.505824475 0.848531365394592 51.0
0 -0.0376869440078735 52.0
0 0.0424538850784302 53.0
0 -0.0184682607650757 54.0
0 -0.0669709444046021 55.0
0 -0.0761500597000122 56.0
0 0.0319851636886597 57.0
0 -0.0776787996292114 58.0
0 0.0314043760299683 59.0
0 -0.114129900932312 60.0
0 -0.00687539577484131 61.0
0 0.783629298210144 62.0
280.13984062 255.229721069336 0.0
0 -0.0486079454421997 1.0
0 0.05778968334198 2.0
0 0.00637495517730713 3.0
0 -0.0377591848373413 4.0
0 -0.000117897987365723 5.0
0 -0.0349909067153931 6.0
0 -0.0577825307846069 7.0
0 -0.0307309627532959 8.0
0 -0.0346132516860962 9.0
0 -0.0100897550582886 10.0
0 -0.026795506477356 11.0
0 -0.121345162391663 12.0
76.869278494 76.7574234008789 13.0
8.516960043 22.7731018066406 14.0
0 -0.0527864694595337 15.0
0 0.0145171880722046 16.0
31.69258874 31.4174098968506 17.0
0 -0.048406720161438 18.0
0 -0.00131547451019287 19.0
0 -0.000917315483093262 20.0
0 0.0112456083297729 21.0
0 0.0210965871810913 22.0
28.345926367 28.9964294433594 23.0
0 -0.0454674959182739 24.0
0 0.75506329536438 25.0
0 0.0762370824813843 26.0
0 -0.0259107351303101 27.0
0 0.0271538496017456 28.0
0 -0.0265520811080933 29.0
0 -0.0433415174484253 30.0
0 -0.0198568105697632 31.0
0 0.0274127721786499 32.0
0 0.10866105556488 33.0
0 -0.0174700021743774 34.0
0 -0.0629850625991821 35.0
0 -0.0234731435775757 36.0
0 -0.0507751703262329 37.0
0 -0.00785386562347412 38.0
0 0.0152722597122192 39.0
0 0.075825572013855 40.0
0 -0.0593172311782837 41.0
0 0.0540202856063843 42.0
281.95434016 291.939544677734 43.0
0 -0.0920072793960571 44.0
0 0.0580977201461792 45.0
0 0.00339972972869873 46.0
0 -0.0328634977340698 47.0
0 -0.0565677881240845 48.0
2.1983389161 0.253013253211975 49.0
0.093345219357 0.102083086967468 50.0
2.0193049116 1.0041606426239 51.0
0 -0.019189715385437 52.0
0 -0.0456973314285278 53.0
0 0.0117107629776001 54.0
0 -0.096316933631897 55.0
0 -0.0739115476608276 56.0
0 -0.00307810306549072 57.0
0 0.00439155101776123 58.0
0 0.0380829572677612 59.0
0 0.116627812385559 60.0
0 -0.0314055681228638 61.0
0 0.844404697418213 62.0
242.57443679 256.022216796875 0.0
0 0.000644087791442871 1.0
0 -0.0589624643325806 2.0
0 -0.0160263776779175 3.0
0 -0.019356369972229 4.0
0 0.0864590406417847 5.0
0 0.0475367307662964 6.0
0 -0.0199505090713501 7.0
0 0.00204992294311523 8.0
0 0.00130689144134521 9.0
0 0.00129401683807373 10.0
0 -0.0231837034225464 11.0
0 0.0105584859848022 12.0
72.261032236 71.1392517089844 13.0
16.752060994 27.2465915679932 14.0
0 0.0276545286178589 15.0
0 -0.0418235063552856 16.0
39.82912236 39.5591316223145 17.0
0 0.0103346109390259 18.0
0 0.0365067720413208 19.0
0 0.00268542766571045 20.0
0 0.0024489164352417 21.0
0 0.0245970487594604 22.0
39.829513031 39.3053092956543 23.0
0 0.0079725980758667 24.0
0 0.851018905639648 25.0
0 0.0191353559494019 26.0
0 -0.0821419954299927 27.0
0 -0.0960065126419067 28.0
0 -0.0390020608901978 29.0
0 -0.0845218896865845 30.0
0 -0.0173717737197876 31.0
0 -5.30481338500977e-05 32.0
0 0.0543879270553589 33.0
0 -0.0160404443740845 34.0
0 -0.038793683052063 35.0
0 0.13983166217804 36.0
0 -0.0437475442886353 37.0
0 -0.00161659717559814 38.0
0 0.0229891538619995 39.0
0 -0.0508719682693481 40.0
0 -0.0206953287124634 41.0
0 0.0492318868637085 42.0
287.70999018 296.267303466797 43.0
0 -0.00914442539215088 44.0
0 0.0475060343742371 45.0
0 0.00535976886749268 46.0
0 -0.0458487272262573 47.0
0 0.0146831274032593 48.0
0 0.340406179428101 49.0
0.074529071282 0.0465801954269409 50.0
2.020271849 0.898918211460114 51.0
0 -0.0579029321670532 52.0
0 -0.00976979732513428 53.0
0 0.0215966701507568 54.0
0 -0.064989447593689 55.0
0 0.00988781452178955 56.0
0 -0.027807354927063 57.0
0 -0.0618587732315063 58.0
0 -0.0834239721298218 59.0
0 0.0172210931777954 60.0
0 0.0629428625106812 61.0
0 0.916980803012848 62.0
220.34798063 256.983184814453 0.0
0 -0.0934902429580688 1.0
0 -0.0151785612106323 2.0
0 -0.0140517950057983 3.0
0 -0.0815340280532837 4.0
0 -0.0195926427841187 5.0
0 -0.0196596384048462 6.0
0 -0.0894829034805298 7.0
0 -0.0421659350395203 8.0
0 -0.019748330116272 9.0
0 0.0123826265335083 10.0
0 -0.0117483139038086 11.0
0 -0.0655256509780884 12.0
52.579990729 51.6919364929199 13.0
44.160734846 43.3308906555176 14.0
0 -0.0130754709243774 15.0
0 -0.0386964082717896 16.0
38.715976435 37.6441535949707 17.0
0 -0.00199615955352783 18.0
0 -0.00453674793243408 19.0
0 -0.0440100431442261 20.0
0 -0.0557471513748169 21.0
0 0.00450217723846436 22.0
39.910497891 38.0427665710449 23.0
0 -0.0833321809768677 24.0
0 0.566765546798706 25.0
0 -0.00545036792755127 26.0
0 0.0301658511161804 27.0
0 -0.0205265283584595 28.0
0 -0.0112398862838745 29.0
0 -0.0550593137741089 30.0
0 -0.0223361253738403 31.0
0 -0.0311084985733032 32.0
0 0.048524022102356 33.0
0 -0.0397559404373169 34.0
0 -0.011667013168335 35.0
0 -0.092279314994812 36.0
0 -0.0410648584365845 37.0
0 0.0441325902938843 38.0
0 0.0241426229476929 39.0
0 0.00791299343109131 40.0
0 0.00825870037078857 41.0
0 0.0310100317001343 42.0
291.37576827 295.238433837891 43.0
0 0.00217092037200928 44.0
0 0.0352436304092407 45.0
0 -0.00360238552093506 46.0
0 0.0203880071640015 47.0
0 -0.00727903842926025 48.0
0 0.314878940582275 49.0
0.096834131342 0.0613218545913696 50.0
1.8362021857 0.940388917922974 51.0
0 -0.0617958307266235 52.0
0 -0.0497432947158813 53.0
0 0.0571244955062866 54.0
0 -0.0512462854385376 55.0
0 -0.0428017377853394 56.0
0 -0.000495076179504395 57.0
0 -0.0615614652633667 58.0
0 -0.0923613309860229 59.0
0 -0.0371707677841187 60.0
0 0.0475150346755981 61.0
0 0.807563424110413 62.0
281.46447196 258.091857910156 0.0
0 -0.112778306007385 1.0
0 -0.0189186334609985 2.0
0 -0.0584250688552856 3.0
0 -0.0358422994613647 4.0
0 -0.0391274690628052 5.0
0 -0.00331747531890869 6.0
0 -0.0212069749832153 7.0
0 -0.00641441345214844 8.0
0 -0.00881731510162354 9.0
0 -0.0177561044692993 10.0
0 0.0367393493652344 11.0
0 0.0275729894638062 12.0
64.784267831 64.6799011230469 13.0
29.040946745 32.6198425292969 14.0
0 -0.131889462471008 15.0
0 0.048949122428894 16.0
29.357539053 28.8366394042969 17.0
0 -0.0275024175643921 18.0
0 0.00119602680206299 19.0
0 0.0119560956954956 20.0
0 0.00471317768096924 21.0
0 -0.0286246538162231 22.0
28.80285437 27.8328075408936 23.0
0 -0.0478302240371704 24.0
0 0.949704825878143 25.0
0 0.00353062152862549 26.0
0 0.076651930809021 27.0
0 -0.0323587656021118 28.0
0 -0.0132471323013306 29.0
0 -0.0709140300750732 30.0
0 -0.0594655275344849 31.0
0 0.0428265333175659 32.0
0 0.0620020627975464 33.0
0 -0.0234946012496948 34.0
0 -0.029424786567688 35.0
0 0.0382195711135864 36.0
0 -0.0547837018966675 37.0
0 -0.0108214616775513 38.0
0 0.0233720541000366 39.0
0 0.0427473783493042 40.0
0 -0.0669785737991333 41.0
0 -0.0869127511978149 42.0
297.54156386 295.549560546875 43.0
0 -0.0327459573745728 44.0
0 0.0961883068084717 45.0
0 -0.0597077608108521 46.0
0 -0.0390361547470093 47.0
0 0.00504910945892334 48.0
0 0.269934833049774 49.0
0.070555030227 0.0755542516708374 50.0
2.0991595382 0.865421295166016 51.0
0 -0.0729562044143677 52.0
0 -0.0503705739974976 53.0
0 0.0558761358261108 54.0
0 -0.0897403955459595 55.0
0 -0.0101436376571655 56.0
0 -0.00885164737701416 57.0
0 0.0223406553268433 58.0
0 -0.0019305944442749 59.0
0 -0.10128915309906 60.0
0 0.0282422304153442 61.0
0 0.879460096359253 62.0
262.90394992 260.109283447266 0.0
0 0.00351274013519287 1.0
0 -0.0353778600692749 2.0
0 0.0269912481307983 3.0
0 -0.0457960367202759 4.0
0 -0.0440527200698853 5.0
0 -0.019931435585022 6.0
0 -0.0337306261062622 7.0
0 -0.0338379144668579 8.0
0 0.132687926292419 9.0
0 -0.0322407484054565 10.0
0 0.0173972845077515 11.0
0 -0.005013108253479 12.0
74.733737372 73.6688385009766 13.0
26.065559514 23.9233379364014 14.0
0 -0.0156713724136353 15.0
0 -0.0314139127731323 16.0
27.352612065 27.5233879089355 17.0
0 -0.0156444311141968 18.0
0 0.0316578149795532 19.0
0 0.00494015216827393 20.0
0 -0.0243245363235474 21.0
0 -0.0264807939529419 22.0
35.317924428 34.4483833312988 23.0
0 -0.00705492496490479 24.0
0 1.09812045097351 25.0
0 0.113601088523865 26.0
0 0.113815903663635 27.0
0 -0.0418030023574829 28.0
0 -0.0374749898910522 29.0
0 -0.0565406084060669 30.0
0 -0.0660718679428101 31.0
0 -0.0101796388626099 32.0
0 -0.0289281606674194 33.0
0 -0.0301393270492554 34.0
0 0.105573296546936 35.0
0 0.0258306264877319 36.0
0 -0.0669416189193726 37.0
0 -0.0587080717086792 38.0
0 0.0712450742721558 39.0
0 0.114887595176697 40.0
0 -0.0484684705734253 41.0
0 0.00418508052825928 42.0
308.1921571 295.675964355469 43.0
0 0.00219523906707764 44.0
0 0.0203683376312256 45.0
0 -0.0559250116348267 46.0
0 -0.0478452444076538 47.0
0 0.0306323766708374 48.0
0 0.310342669487 49.0
0.07315531581 0.055235743522644 50.0
2.2179871765 0.984966576099396 51.0
0 -0.0885912179946899 52.0
0 -0.0689107179641724 53.0
0 0.0151151418685913 54.0
0 -0.0985478162765503 55.0
0 -0.089556097984314 56.0
0 0.0194486379623413 57.0
0 -0.0419729948043823 58.0
0 -0.0738695859909058 59.0
0 -0.0661591291427612 60.0
0 0.0768288373947144 61.0
0 0.922189652919769 62.0
229.90229003 250.954574584961 0.0
0 0.0181010961532593 1.0
0 -0.0611177682876587 2.0
0 -0.0113183259963989 3.0
0 -0.0854102373123169 4.0
0 -0.16361129283905 5.0
0 -0.00136053562164307 6.0
0 -0.0843266248703003 7.0
0 -0.0209372043609619 8.0
0 0.0349773168563843 9.0
0 0.0160350799560547 10.0
0 0.022739589214325 11.0
0 0.0349382162094116 12.0
66.560621126 66.5985336303711 13.0
28.904167579 30.8824996948242 14.0
0 0.0178664922714233 15.0
0 0.0398620367050171 16.0
34.630979768 34.2595672607422 17.0
0 -0.041150689125061 18.0
0 0.0187424421310425 19.0
0 0.000579714775085449 20.0
0 -0.0381003618240356 21.0
0 0.014548659324646 22.0
33.474994478 32.2299156188965 23.0
0 0.0610803365707397 24.0
0 0.514657497406006 25.0
0 0.0995429754257202 26.0
0 0.0583304166793823 27.0
0 0.0573948621749878 28.0
0 -0.0486575365066528 29.0
0 -0.0583268404006958 30.0
0 -0.0539113283157349 31.0
0 -0.085457444190979 32.0
0 0.0373440980911255 33.0
0 -0.0475460290908813 34.0
0 -0.0984979867935181 35.0
0 0.0135699510574341 36.0
0 -0.0824829339981079 37.0
0 0.0205384492874146 38.0
0 0.0342906713485718 39.0
0 0.0880507230758667 40.0
0 -0.0169973373413086 41.0
0 0.0049365758895874 42.0
305.3693217 297.182312011719 43.0
0 -0.0060659646987915 44.0
0 -0.00258111953735352 45.0
0 -0.00841724872589111 46.0
0 -0.0707646608352661 47.0
0 0.00698506832122803 48.0
0 0.285734295845032 49.0
0.096074437849 0.0548015832901001 50.0
1.9362114934 0.8677898645401 51.0
0 -0.0693274736404419 52.0
0 -0.0219908952713013 53.0
0 0.00842630863189697 54.0
0 -0.120198130607605 55.0
0 -0.0917614698410034 56.0
0 0.00646913051605225 57.0
0 -0.0622690916061401 58.0
0 -0.030861496925354 59.0
0 -0.0342494249343872 60.0
0 -0.0133339166641235 61.0
0 0.861282467842102 62.0
262.85456753 255.928787231445 0.0
0 -0.144975304603577 1.0
0 -0.0737498998641968 2.0
0 0.0139464139938354 3.0
0 -0.0449017286300659 4.0
0 -0.14973509311676 5.0
0 0.0776313543319702 6.0
0 -0.0161691904067993 7.0
0 -0.013176441192627 8.0
0 0.0481065511703491 9.0
0 0.0217458009719849 10.0
0 -0.0434081554412842 11.0
0 -0.171705842018127 12.0
65.716654485 65.7412338256836 13.0
31.158174152 29.6672821044922 14.0
0 -0.0534282922744751 15.0
0 -0.00649464130401611 16.0
40.854153771 40.6835632324219 17.0
0 -0.00836050510406494 18.0
0 0.0356603860855103 19.0
0 0.0389388799667358 20.0
0 0.0433073043823242 21.0
0 -0.0217033624649048 22.0
33.320946837 32.1190948486328 23.0
0 0.0434716939926147 24.0
0 0.476426720619202 25.0
0 0.0499807596206665 26.0
0 0.0391796827316284 27.0
0 0.0843256711959839 28.0
0 -0.0581265687942505 29.0
0 -0.0705041885375977 30.0
0 0.00209486484527588 31.0
0 -0.0683127641677856 32.0
0 -0.0519992113113403 33.0
0 -0.0642443895339966 34.0
0 -0.0419620275497437 35.0
0 0.0593150854110718 36.0
0 -0.046113133430481 37.0
0 -0.0296553373336792 38.0
0 -0.126205563545227 39.0
0 0.101531386375427 40.0
0 -0.009712815284729 41.0
0 -0.0343958139419556 42.0
277.70491776 291.972534179688 43.0
0 0.0463298559188843 44.0
0 0.0502749681472778 45.0
0 -0.0219124555587769 46.0
0 -0.00717270374298096 47.0
0 -0.0190011262893677 48.0
0 0.254541158676147 49.0
0.094162097161 0.25022280216217 50.0
1.5907494099 0.843256235122681 51.0
0 -0.0110045671463013 52.0
0 -0.0204161405563354 53.0
0 0.0535622835159302 54.0
0 -0.103735566139221 55.0
0 -0.0527495145797729 56.0
0 -0.0109425783157349 57.0
0 -0.00346672534942627 58.0
0 -0.0510035753250122 59.0
0 0.187562823295593 60.0
0 -0.0136367082595825 61.0
0 0.92509651184082 62.0
247.27688166 248.969848632812 0.0
0 -0.0305134057998657 1.0
0 -0.128059267997742 2.0
0 -0.0394498109817505 3.0
0 -0.0345803499221802 4.0
0 0.0240851640701294 5.0
0 -0.0298892259597778 6.0
0 0.00590121746063232 7.0
0 0.112791180610657 8.0
0 0.0783704519271851 9.0
0 -0.0361608266830444 10.0
0 0.0644265413284302 11.0
0 0.0559886693954468 12.0
47.8586967 47.1240234375 13.0
41.494657128 39.5937995910645 14.0
0 -0.0183454751968384 15.0
0 -0.0872169733047485 16.0
40.943948714 40.9632606506348 17.0
0 -0.0584992170333862 18.0
0 0.0229357481002808 19.0
0 0.0354777574539185 20.0
0 -0.0154291391372681 21.0
0 -0.0626403093338013 22.0
41.820883601 41.5447120666504 23.0
0 -0.0930854082107544 24.0
0 0.606903433799744 25.0
0 -0.0195909738540649 26.0
0 0.0389693975448608 27.0
0 0.0239790678024292 28.0
0 -0.0213571786880493 29.0
0 -0.0742274522781372 30.0
0 -0.0241526365280151 31.0
0 -0.0146504640579224 32.0
0 -0.0682328939437866 33.0
0 -0.0919321775436401 34.0
0 -0.0411204099655151 35.0
0 0.0750495195388794 36.0
0 0.0428670644760132 37.0
0 -0.0135499238967896 38.0
0 0.0561004877090454 39.0
0 0.0622342824935913 40.0
0 -0.0310736894607544 41.0
0 0.0471004247665405 42.0
287.34909198 296.757995605469 43.0
0 -0.0508323907852173 44.0
0 -0.0208061933517456 45.0
0 -0.0367580652236938 46.0
0 -0.016840934753418 47.0
0 -0.0114947557449341 48.0
0 0.277321994304657 49.0
0.092769818517 0.0849481821060181 50.0
1.8816234448 1.02450370788574 51.0
0 -0.0661908388137817 52.0
0 -0.0630894899368286 53.0
0 0.0404385328292847 54.0
0 -0.0576757192611694 55.0
0 -0.0501335859298706 56.0
0 -0.0171345472335815 57.0
0 -0.0682929754257202 58.0
0 -0.0618873834609985 59.0
0 0.128302216529846 60.0
0 0.0199021100997925 61.0
0 0.81007993221283 62.0
199.56853871 257.738647460938 0.0
0 -0.0940748453140259 1.0
0 -0.107936501502991 2.0
0 0.0372697114944458 3.0
0 0.0345674753189087 4.0
0 0.0629109144210815 5.0
0 0.0327571630477905 6.0
0 -0.056543231010437 7.0
0 -0.052446722984314 8.0
0 -0.108808159828186 9.0
0 -0.0600277185440063 10.0
0 0.00399291515350342 11.0
0 -0.0154567956924438 12.0
73.914887701 74.8430480957031 13.0
23.192194968 19.3252925872803 14.0
0 -0.0750232934951782 15.0
0 0.0591477155685425 16.0
34.787687303 33.8095512390137 17.0
0 0.00254547595977783 18.0
0 0.0371648073196411 19.0
0 -0.0164016485214233 20.0
0 -0.031915545463562 21.0
0 -0.113953232765198 22.0
40.311613214 39.6883583068848 23.0
0 -0.172508835792542 24.0
0 0.745597064495087 25.0
0 0.0453838109970093 26.0
0 0.0216808319091797 27.0
0 -0.0416494607925415 28.0
0 -0.0694688558578491 29.0
0 -0.0450618267059326 30.0
0 -0.0612703561782837 31.0
0 -0.0498524904251099 32.0
0 0.00810062885284424 33.0
0 -0.0126956701278687 34.0
0 -0.0420371294021606 35.0
0 0.131510853767395 36.0
0 -0.0280665159225464 37.0
0 0.00740730762481689 38.0
0 0.0759707689285278 39.0
0 0.148814558982849 40.0
0 -0.02805495262146 41.0
0 0.00586450099945068 42.0
276.72008582 293.483612060547 43.0
0 0.0383609533309937 44.0
0 -0.0250144004821777 45.0
0 -0.00908243656158447 46.0
0 0.0577327013015747 47.0
0 0.00364124774932861 48.0
0 0.264178693294525 49.0
0.10273295143 0.0939501523971558 50.0
1.5686037941 0.902683973312378 51.0
0 -0.0899745225906372 52.0
0 -0.0708776712417603 53.0
0 0.0224000215530396 54.0
0 -0.0889064073562622 55.0
0 -0.00894558429718018 56.0
0 -0.0096665620803833 57.0
0 -0.0428029298782349 58.0
0 -0.0779250860214233 59.0
0 -0.0333956480026245 60.0
0 -4.99486923217773e-05 61.0
0 0.794453740119934 62.0
208.32376007 256.220977783203 0.0
0 -0.0435057878494263 1.0
0 -0.10833752155304 2.0
0 -0.0451315641403198 3.0
0 -0.0426455736160278 4.0
0 -0.100256204605103 5.0
0 0.0534051656723022 6.0
0 0.0765341520309448 7.0
0 -0.000505447387695312 8.0
0 0.0456470251083374 9.0
0 -0.0393010377883911 10.0
0 0.0229188203811646 11.0
0 0.0543035268783569 12.0
65.046757542 64.7272644042969 13.0
36.941752993 30.4036636352539 14.0
0 -0.0807076692581177 15.0
0 -0.0575116872787476 16.0
34.306826137 33.7372245788574 17.0
0 -0.0217248201370239 18.0
0 -0.0149208307266235 19.0
0 0.0220621824264526 20.0
0 0.0128272771835327 21.0
0 -0.0675646066665649 22.0
41.839379136 41.0869026184082 23.0
0 -0.109007000923157 24.0
0 1.77496886253357 25.0
0 0.0525630712509155 26.0
0 0.0503815412521362 27.0
0 -0.057038426399231 28.0
0 -0.0612431764602661 29.0
0 -0.0749940872192383 30.0
0 -0.033659815788269 31.0
0 0.060932993888855 32.0
0 -0.0114775896072388 33.0
0 -0.0735019445419312 34.0
0 0.0607579946517944 35.0
0 0.020677924156189 36.0
0 -0.0482674837112427 37.0
0 -0.0256232023239136 38.0
0 -0.0450774431228638 39.0
0 0.0919417142868042 40.0
0 -0.0704942941665649 41.0
0 0.023459792137146 42.0
293.60079286 295.451385498047 43.0
0 0.0122765302658081 44.0
0 0.0151728391647339 45.0
0 -0.0532318353652954 46.0
0 -0.02208411693573 47.0
0 0.0609458684921265 48.0
0 0.30362468957901 49.0
0.081214047268 0.0695041418075562 50.0
1.5010535613 0.925613403320312 51.0
0 -0.0662499666213989 52.0
0 -0.0739289522171021 53.0
0 0.00949966907501221 54.0
0 -0.0716363191604614 55.0
0 -0.0794965028762817 56.0
0 -0.0410405397415161 57.0
0 -0.0622392892837524 58.0
0 -0.0616101026535034 59.0
0 -0.12203061580658 60.0
0 0.0509337186813354 61.0
0 0.967040181159973 62.0
248.49726433 253.386840820312 0.0
0 -0.0128179788589478 1.0
0 0.0278245210647583 2.0
0 -0.0248116254806519 3.0
0 -0.0505019426345825 4.0
0 -0.0264507532119751 5.0
0 -0.146825909614563 6.0
0 -0.063947319984436 7.0
0 -0.0436736941337585 8.0
0 0.102293372154236 9.0
0 -0.00648295879364014 10.0
0 0.0174152851104736 11.0
0 -0.0171495676040649 12.0
59.267151492 58.910270690918 13.0
43.789416536 35.3912963867188 14.0
0 -0.187850594520569 15.0
0 -0.0121587514877319 16.0
38.552934444 37.9051971435547 17.0
0 0.0629581212997437 18.0
0 0.0184386968612671 19.0
0 0.0288373231887817 20.0
0 -0.0438860654830933 21.0
0 0.0247644186019897 22.0
37.980961885 37.4179763793945 23.0
0 0.0471848249435425 24.0
0 0.515880823135376 25.0
0 0.0472506284713745 26.0
0 0.0956708192825317 27.0
0 -0.0475307703018188 28.0
0 -0.0551532506942749 29.0
0 -0.0489381551742554 30.0
0 -0.0379844903945923 31.0
0 0.00859630107879639 32.0
0 0.065576434135437 33.0
0 -0.0144520998001099 34.0
0 -0.0596338510513306 35.0
0 -0.013541579246521 36.0
0 -0.0998522043228149 37.0
0 0.0495315790176392 38.0
0 0.0589076280593872 39.0
0 -0.0226813554763794 40.0
0 -0.0473505258560181 41.0
0 -0.054782509803772 42.0
295.5148869 294.678253173828 43.0
0 0.0663622617721558 44.0
0 -0.0439068078994751 45.0
0 -0.0687695741653442 46.0
0 -0.0451072454452515 47.0
0 0.0191143751144409 48.0
1.6844036159 0.318869233131409 49.0
0.074911347609 0.0648714303970337 50.0
2.1596089522 1.00801467895508 51.0
0 -0.0532516241073608 52.0
0 -0.0562311410903931 53.0
0 0.0704320669174194 54.0
0 -0.115883469581604 55.0
0 -0.054906964302063 56.0
0 0.00253713130950928 57.0
0 -0.0671824216842651 58.0
0 -0.0913151502609253 59.0
0 -0.115829586982727 60.0
0 0.00468790531158447 61.0
0 0.871228456497192 62.0
275.17389549 256.887939453125 0.0
0 -0.0358816385269165 1.0
0 -0.0340989828109741 2.0
0 0.00104320049285889 3.0
0 -0.010633111000061 4.0
0 -0.0409170389175415 5.0
0 0.121385931968689 6.0
0 -0.202340722084045 7.0
0 -0.0486127138137817 8.0
0 -0.0610853433609009 9.0
0 -0.004738450050354 10.0
0 0.0193475484848022 11.0
0 0.0528167486190796 12.0
71.317386717 71.4718933105469 13.0
23.301003826 26.4343338012695 14.0
0 -0.027377724647522 15.0
0 -0.0149956941604614 16.0
36.429389996 35.9556427001953 17.0
0 -0.00087893009185791 18.0
0 0.00924146175384521 19.0
0 0.00540220737457275 20.0
0 -0.0301207304000854 21.0
0 0.041284441947937 22.0
36.725830706 35.956371307373 23.0
0 -0.075778603553772 24.0
0 1.48367834091187 25.0
0 0.0682162046432495 26.0
0 0.0955523252487183 27.0
0 -0.0714970827102661 28.0
0 -0.0419555902481079 29.0
0 0.0934749841690063 30.0
0 -0.0390633344650269 31.0
0 -0.0460821390151978 32.0
0 -0.0365041494369507 33.0
0 0.0128453969955444 34.0
0 -0.0831860303878784 35.0
0 0.0337353944778442 36.0
0 -0.0271707773208618 37.0
0 -0.00873363018035889 38.0
0 0.0971750020980835 39.0
0 0.0706828832626343 40.0
0 -0.0129789113998413 41.0
0 0.0135005712509155 42.0
287.06527813 295.168029785156 43.0
0 0.0119146108627319 44.0
0 0.0918445587158203 45.0
0 -0.0329002141952515 46.0
0 -0.0585235357284546 47.0
0 -0.0232969522476196 48.0
0 0.299686074256897 49.0
0.099655138521 0.0578066110610962 50.0
1.6582109955 0.970301806926727 51.0
0 -0.096356987953186 52.0
0 -0.0500560998916626 53.0
0 -0.032056450843811 54.0
0 -0.0456584692001343 55.0
0 -0.0271576642990112 56.0
0 -0.0633019208908081 57.0
0 -0.0541008710861206 58.0
0 -0.0579208135604858 59.0
0 -0.0476583242416382 60.0
0 -0.0330489873886108 61.0
0 0.818382263183594 62.0
271.60358718 255.970962524414 0.0
0 -0.0954653024673462 1.0
0 0.0318547487258911 2.0
0 -0.085096001625061 3.0
0 0.00342786312103271 4.0
0 -0.0175787210464478 5.0
0 -0.0135689973831177 6.0
0 -0.083203911781311 7.0
0 -0.0088038444519043 8.0
0 0.0196851491928101 9.0
0 -0.00429809093475342 10.0
0 0.0215222835540771 11.0
0 0.112419486045837 12.0
56.984253009 55.0429763793945 13.0
37.774192673 39.5655326843262 14.0
0 -0.0566908121109009 15.0
0 -0.00233733654022217 16.0
26.070424668 25.7254695892334 17.0
0 -0.0643552541732788 18.0
0 0.014802098274231 19.0
0 0.0194729566574097 20.0
0 -0.0305594205856323 21.0
0 -0.0692845582962036 22.0
37.981946431 38.70263671875 23.0
0 -0.107453942298889 24.0
0 0.666161954402924 25.0
0 0.06303870677948 26.0
0 0.068632960319519 27.0
0 0.0110749006271362 28.0
0 -0.0191584825515747 29.0
0 -0.0404573678970337 30.0
0 -0.0423470735549927 31.0
0 -0.0943528413772583 32.0
0 0.0226472616195679 33.0
0 -0.11105477809906 34.0
0 -0.0959018468856812 35.0
0 -0.00593745708465576 36.0
0 -0.00891721248626709 37.0
0 0.0584250688552856 38.0
0 0.128773093223572 39.0
0 0.0660470724105835 40.0
0 -0.0299274921417236 41.0
0 -0.00906431674957275 42.0
291.07342298 296.346466064453 43.0
0 -0.0572992563247681 44.0
0 0.0715841054916382 45.0
0 0.0122352838516235 46.0
0 -0.0962530374526978 47.0
0 -0.0256544351577759 48.0
0 0.280035614967346 49.0
0.083522871053 0.12975287437439 50.0
1.4431618393 0.854708790779114 51.0
0 -0.126529812812805 52.0
0 0.00173509120941162 53.0
0 0.00320732593536377 54.0
0 -0.040460467338562 55.0
0 -0.0230585336685181 56.0
0 -0.0695410966873169 57.0
0 -0.0459815263748169 58.0
0 -0.0486832857131958 59.0
0 -0.113820433616638 60.0
0 -0.000337004661560059 61.0
0 0.849839329719543 62.0
232.20106629 255.81999206543 0.0
0 -0.081790566444397 1.0
0 -0.044861912727356 2.0
0 -0.00505435466766357 3.0
0 -0.0504387617111206 4.0
0 0.00489556789398193 5.0
0 0.032915472984314 6.0
0 -0.0360773801803589 7.0
0 -0.025576114654541 8.0
0 0.0696223974227905 9.0
0 0.0155128240585327 10.0
0 -0.053125262260437 11.0
0 -0.0315853357315063 12.0
70.230648464 68.7947692871094 13.0
15.313269025 29.4597492218018 14.0
0 -0.0672060251235962 15.0
0 -0.0996049642562866 16.0
31.917085252 31.4779930114746 17.0
0 0.0401979684829712 18.0
0 -0.0181657075881958 19.0
0 0.00694811344146729 20.0
0 -0.0256298780441284 21.0
0 0.00896775722503662 22.0
30.614824325 29.0821781158447 23.0
0 0.0920141935348511 24.0
0 0.41058486700058 25.0
0 0.0759328603744507 26.0
0 0.036370575428009 27.0
0 -0.0591813325881958 28.0
0 -0.0455225706100464 29.0
0 0.0727874040603638 30.0
0 -0.00124609470367432 31.0
0 0.00658786296844482 32.0
0 -0.0186196565628052 33.0
0 -0.103028893470764 34.0
0 0.140937924385071 35.0
0 -0.0517188310623169 36.0
0 -0.10160768032074 37.0
0 0.013445258140564 38.0
0 0.0819143056869507 39.0
0 0.0686477422714233 40.0
0 -0.0690665245056152 41.0
0 -0.0391680002212524 42.0
305.61089148 293.719909667969 43.0
0 -0.0374299287796021 44.0
0 0.0300778150558472 45.0
0 0.0132318735122681 46.0
0 -0.0356758832931519 47.0
0 0.0404905080795288 48.0
0 0.223954916000366 49.0
0.073289204953 0.0772377252578735 50.0
1.8032371736 0.981206893920898 51.0
0 -0.0291904211044312 52.0
0 -0.048230767250061 53.0
0 -0.0353862047195435 54.0
0 -0.125563979148865 55.0
0 -0.0350102186203003 56.0
0 -0.0356491804122925 57.0
0 0.0282419919967651 58.0
0 -0.0371493101119995 59.0
0 0.0176008939743042 60.0
0 -0.026236891746521 61.0
0 0.860437870025635 62.0
275.78057615 261.035888671875 0.0
0 -0.0214892625808716 1.0
0 -0.0170618295669556 2.0
0 0.00323331356048584 3.0
0 -0.0236774682998657 4.0
0 0.0144914388656616 5.0
0 0.0371402502059937 6.0
0 -0.0071483850479126 7.0
0 0.0539466738700867 8.0
0 0.0179780721664429 9.0
0 0.0168739557266235 10.0
0 0.0170477032661438 11.0
0 -0.0075603723526001 12.0
74.801876637 74.1949081420898 13.0
27.012304854 24.471363067627 14.0
0 0.000241637229919434 15.0
0 -0.0730265378952026 16.0
34.852657472 34.1430931091309 17.0
0 0.00820624828338623 18.0
0 0.0134552717208862 19.0
0 0.00390827655792236 20.0
0 -0.0211554765701294 21.0
0 -0.0217324495315552 22.0
35.290274454 34.3814430236816 23.0
0 -0.0922189950942993 24.0
0 1.0730813741684 25.0
0 -0.0128296613693237 26.0
0 0.0378538370132446 27.0
0 0.0874084234237671 28.0
0 -0.082588791847229 29.0
0 0.150068998336792 30.0
0 -0.0462163686752319 31.0
0 -0.00365793704986572 32.0
0 0.0211278200149536 33.0
0 -0.0192579030990601 34.0
0 -0.00450408458709717 35.0
0 -0.0363098382949829 36.0
0 -0.0290225744247437 37.0
0 0.0201319456100464 38.0
0 0.156676888465881 39.0
0 0.0630005598068237 40.0
0 -0.0790066719055176 41.0
0 -0.0772587060928345 42.0
292.06700254 296.333038330078 43.0
0 0.0055011510848999 44.0
0 0.0248464345932007 45.0
0 -0.00467312335968018 46.0
0 -0.0190463066101074 47.0
0 0.00564181804656982 48.0
0.48186238659 0.274184852838516 49.0
0.073508409272 0.0765417814254761 50.0
1.9355381709 1.00305473804474 51.0
0 -0.0625127553939819 52.0
0 -0.0478936433792114 53.0
0 0.00776684284210205 54.0
0 -0.0882269144058228 55.0
0 -0.011776328086853 56.0
0 -0.00121414661407471 57.0
0 -0.0656677484512329 58.0
0 -0.0181699991226196 59.0
0 0.0578175783157349 60.0
0 -0.0056687593460083 61.0
0 0.839388251304626 62.0
257.579695 252.004257202148 0.0
0 -0.00770533084869385 1.0
0 -0.0354810953140259 2.0
0 0.0264524221420288 3.0
0 -0.041321873664856 4.0
0 0.0119894742965698 5.0
0 0.056784987449646 6.0
0 -0.159102559089661 7.0
0 -0.0201603174209595 8.0
0 0.0892280340194702 9.0
0 -0.0270951986312866 10.0
0 0.0526227951049805 11.0
0 -0.0345255136489868 12.0
55.457659519 54.7855911254883 13.0
46.405566784 36.3381729125977 14.0
0 -0.0408226251602173 15.0
0 -0.0401824712753296 16.0
30.202759877 29.6786212921143 17.0
0 0.0432039499282837 18.0
0 0.0645071268081665 19.0
0 -0.0150278806686401 20.0
0 -0.0337940454483032 21.0
0 -0.0184189081192017 22.0
34.108071713 33.3193016052246 23.0
0 0.091145396232605 24.0
29.516487592 0.487234652042389 25.0
0 0.0404990911483765 26.0
0 0.0551844835281372 27.0
0 -0.0570474863052368 28.0
0 -0.0407959222793579 29.0
0 -0.0424249172210693 30.0
0 -0.0456289052963257 31.0
0 0.0424717664718628 32.0
0 0.116936564445496 33.0
0 0.0193723440170288 34.0
0 -0.0454086065292358 35.0
0 -0.011017918586731 36.0
0 -0.037326455116272 37.0
0 0.0427383184432983 38.0
0 0.0434736013412476 39.0
0 -0.000400185585021973 40.0
0 -0.0729337930679321 41.0
0 -0.0658222436904907 42.0
297.18644464 294.19580078125 43.0
0 -0.0360819101333618 44.0
0 0.0737015604972839 45.0
0 -0.0464764833450317 46.0
0 0.0113521814346313 47.0
0 0.0528172254562378 48.0
0 0.345433235168457 49.0
0.083608823224 0.0522588491439819 50.0
1.8884394892 0.931208670139313 51.0
0 -0.032629132270813 52.0
0 -0.098279595375061 53.0
0 0.000594258308410645 54.0
0 -0.0588911771774292 55.0
0 -0.0530527830123901 56.0
0 -0.024577260017395 57.0
0 0.00579178333282471 58.0
0 -0.0329130887985229 59.0
0 0.0978256464004517 60.0
0 0.0466805696487427 61.0
0 0.89924430847168 62.0
270.99625891 258.116516113281 0.0
0 -0.0485883951187134 1.0
0 -0.0674806833267212 2.0
0 -0.0682252645492554 3.0
0 -0.0695143938064575 4.0
0 -0.0144706964492798 5.0
0 -0.0164390802383423 6.0
0 -0.0178095102310181 7.0
0 -0.0464290380477905 8.0
0 -0.00735533237457275 9.0
0 0.00405275821685791 10.0
0 -0.059148907661438 11.0
0 -0.0259300470352173 12.0
48.072261826 47.7828140258789 13.0
36.684633278 38.200023651123 14.0
0 -0.0499769449234009 15.0
0 -0.0466238260269165 16.0
37.841888144 37.8988227844238 17.0
0 -0.00570785999298096 18.0
0 -0.0448395013809204 19.0
0 0.00121033191680908 20.0
0 -0.00541794300079346 21.0
0 -0.118507504463196 22.0
39.616845866 38.8701515197754 23.0
0 -0.0655413866043091 24.0
18.564394511 0.606914281845093 25.0
0 0.0413416624069214 26.0
0 0.0394980311393738 27.0
0 0.0380328893661499 28.0
0 -0.0808087587356567 29.0
0 -0.0495383739471436 30.0
0 -0.00958836078643799 31.0
0 0.0208853483200073 32.0
0 0.0938543081283569 33.0
0 -0.0156909227371216 34.0
0 -0.0544363260269165 35.0
0 -0.030492901802063 36.0
0 -0.0641227960586548 37.0
0 -0.000441908836364746 38.0
0 -0.0761018991470337 39.0
0 -0.118813157081604 40.0
0 -0.0353038311004639 41.0
0 0.0610851049423218 42.0
289.37334476 294.361236572266 43.0
0 -0.078652024269104 44.0
0 0.0616482496261597 45.0
0 -0.0168014764785767 46.0
0 -0.0520437955856323 47.0
0 0.0447539091110229 48.0
0 0.280054777860641 49.0
0.07558203861 0.063421368598938 50.0
1.8300725009 0.984476685523987 51.0
0 -0.117296814918518 52.0
0 -0.0229078531265259 53.0
0 0.0415774583816528 54.0
0 -0.100551247596741 55.0
0 -0.106556534767151 56.0
0 -0.0530492067337036 57.0
0 -0.0325928926467896 58.0
0 -0.0210322141647339 59.0
0 -0.00760376453399658 60.0
0 0.0256963968276978 61.0
0 0.847505927085876 62.0
241.916803 262.867370605469 0.0
0 -0.0278583765029907 1.0
0 0.0185211896896362 2.0
0 0.029266357421875 3.0
0 -0.035683274269104 4.0
0 -0.077515721321106 5.0
0 -0.0164715051651001 6.0
0 -0.0527793169021606 7.0
0 -0.0533566474914551 8.0
0 -0.0456863641738892 9.0
0 0.0536776781082153 10.0
0 -0.0344029068946838 11.0
0 0.00187051296234131 12.0
71.601108604 71.0214614868164 13.0
28.549216602 28.3901653289795 14.0
0 0.0340665578842163 15.0
0 -0.0319837331771851 16.0
34.395313311 34.2549705505371 17.0
0 -0.0295339822769165 18.0
0 0.0221160650253296 19.0
0 0.0623520612716675 20.0
0 0.00120699405670166 21.0
0 0.00243604183197021 22.0
28.545406097 27.2984790802002 23.0
0 -0.000728726387023926 24.0
0 0.57113790512085 25.0
0 0.00865662097930908 26.0
0 0.0363780856132507 27.0
0 -0.0512367486953735 28.0
0 -0.031091570854187 29.0
0 -0.0788074731826782 30.0
0 0.00544512271881104 31.0
0 -0.0702730417251587 32.0
0 0.0188537836074829 33.0
0 -0.0101557970046997 34.0
0 -0.0672658681869507 35.0
0 0.011252760887146 36.0
0 -0.0359803438186646 37.0
0 -0.0144990682601929 38.0
0 0.00863778591156006 39.0
0 0.0632671117782593 40.0
0 -0.059362530708313 41.0
0 0.0367969274520874 42.0
289.98488078 296.543334960938 43.0
0 -0.0607525110244751 44.0
0 -0.0310169458389282 45.0
0 -0.000158429145812988 46.0
0 -0.00804030895233154 47.0
0 -0.0247193574905396 48.0
0 0.268682360649109 49.0
0.084098265965 0.133503198623657 50.0
1.7053764041 0.941543281078339 51.0
0 -0.0911904573440552 52.0
0 -0.077126145362854 53.0
0 0.0428858995437622 54.0
0 -0.0914188623428345 55.0
0 -0.079367995262146 56.0
0 -0.00121223926544189 57.0
0 -0.012792706489563 58.0
0 -0.0717495679855347 59.0
0 0.049059271812439 60.0
0 0.0508307218551636 61.0
0 0.899957001209259 62.0
234.92622212 255.548034667969 0.0
0 -0.0503436326980591 1.0
0 0.0214461088180542 2.0
0 -0.0362757444381714 3.0
0 -0.00298893451690674 4.0
0 -0.0177580118179321 5.0
0 -0.00824606418609619 6.0
0 -0.0137072801589966 7.0
0 0.0226584672927856 8.0
0 -0.0341192483901978 9.0
0 -0.0242170095443726 10.0
0 0.0297813415527344 11.0
0 -0.0584665536880493 12.0
73.078557418 72.9567108154297 13.0
15.170612388 26.3912372589111 14.0
0 0.0106719732284546 15.0
0 -0.0322505235671997 16.0
37.527367898 36.6984329223633 17.0
0 0.00616848468780518 18.0
0 0.00028383731842041 19.0
0 -0.011478066444397 20.0
0 -0.0192335844039917 21.0
0 0.0274828672409058 22.0
31.171631065 30.357666015625 23.0
0 -0.0461779832839966 24.0
0 0.59461772441864 25.0
0 0.0696524381637573 26.0
0 0.0122985243797302 27.0
0 0.0698984861373901 28.0
0 -0.0206218957901001 29.0
0 -0.0589803457260132 30.0
0 -0.052115797996521 31.0
0 0.0283025503158569 32.0
0 0.0283526182174683 33.0
0 -0.0882352590560913 34.0
0 -0.0253461599349976 35.0
0 0.0849896669387817 36.0
0 -0.0339714288711548 37.0
0 -0.0275818109512329 38.0
0 0.0348862409591675 39.0
0 0.136349081993103 40.0
0 0.0112849473953247 41.0
0 -0.0430132150650024 42.0
299.45085629 298.501770019531 43.0
0 -0.006935715675354 44.0
0 -0.0187243223190308 45.0
0 -0.00514662265777588 46.0
0 -0.0320373773574829 47.0
0 -0.0302411317825317 48.0
0 0.327520281076431 49.0
0.083773669426 0.0632299184799194 50.0
1.8235104933 0.903010547161102 51.0
0 -0.0598810911178589 52.0
0 0.0146981477737427 53.0
0 0.0134059190750122 54.0
0 -0.0536116361618042 55.0
0 -0.0673328638076782 56.0
0 0.0106452703475952 57.0
0 -0.0309368371963501 58.0
0 -0.0646544694900513 59.0
0 -0.0716649293899536 60.0
0 0.0588370561599731 61.0
0 0.926041066646576 62.0
265.53335386 258.760864257812 0.0
0 0.0183714628219604 1.0
0 -0.0214051008224487 2.0
0 0.0186930894851685 3.0
0 -0.0143319368362427 4.0
0 -0.0956488847732544 5.0
0 0.00969016551971436 6.0
0 -0.0268036127090454 7.0
0 0.0177126526832581 8.0
0 0.0352681875228882 9.0
0 -0.0113706588745117 10.0
0 -0.0505087375640869 11.0
0 -0.0256372690200806 12.0
59.468570686 58.6230049133301 13.0
44.364680904 39.008674621582 14.0
0 0.0519431829452515 15.0
0 -0.0292681455612183 16.0
34.817014922 34.0083274841309 17.0
0 -0.00671803951263428 18.0
0 -0.00983703136444092 19.0
0 0.0331403017044067 20.0
0 -0.00286233425140381 21.0
0 -0.0634962320327759 22.0
38.239443516 37.338493347168 23.0
0 -0.0498377084732056 24.0
0 0.514761686325073 25.0
0 0.113786578178406 26.0
0 0.0124245882034302 27.0
0 0.0447262525558472 28.0
0 -0.0959311723709106 29.0
0 -0.0226308107376099 30.0
0 -0.0109878778457642 31.0
0 -0.0416218042373657 32.0
0 0.11416232585907 33.0
0 -0.0363341569900513 34.0
0 -0.0711053609848022 35.0
0 0.110798478126526 36.0
0 0.0622881650924683 37.0
0 0.0445798635482788 38.0
0 0.101430535316467 39.0
0 0.0164312124252319 40.0
0 0.00103926658630371 41.0
0 0.0319665670394897 42.0
284.06596081 298.124145507812 43.0
0 0.00542008876800537 44.0
0 0.0889813899993896 45.0
0 0.00821912288665771 46.0
0 -0.0634168386459351 47.0
0 0.0419243574142456 48.0
0 0.266317486763 49.0
0.078807583439 0.0580254793167114 50.0
1.6498847923 1.00919926166534 51.0
0 -0.0661031007766724 52.0
0 -0.0377947092056274 53.0
0 -0.0544105768203735 54.0
0 -0.104767918586731 55.0
0 -0.0127052068710327 56.0
0 -0.0378090143203735 57.0
0 -0.0611077547073364 58.0
0 -0.0750869512557983 59.0
0 0.0303069353103638 60.0
0 0.0629163980484009 61.0
0 0.901572525501251 62.0
231.37290218 251.905181884766 0.0
0 -0.0322233438491821 1.0
0 0.00865590572357178 2.0
0 -0.0139962434768677 3.0
0 -0.0250564813613892 4.0
0 -0.0211833715438843 5.0
0 -0.0427175760269165 6.0
0 -0.0648692846298218 7.0
0 -0.0368725061416626 8.0
0 0.0536283254623413 9.0
0 -0.00131094455718994 10.0
0 -0.00524771213531494 11.0
0 0.11176335811615 12.0
73.729266251 73.2673873901367 13.0
23.803730181 27.8336334228516 14.0
0 0.0385192632675171 15.0
0 -0.0203348398208618 16.0
31.429577842 31.1552562713623 17.0
0 -0.0177599191665649 18.0
0 0.0171195268630981 19.0
0 0.0118926763534546 20.0
0 -0.0148993730545044 21.0
0 -0.027906060218811 22.0
37.704617022 36.4420967102051 23.0
0 -0.0656176805496216 24.0
0 0.549708008766174 25.0
0 0.0366882085800171 26.0
0 0.0828870534896851 27.0
0 0.0594476461410522 28.0
0 -0.0545037984848022 29.0
0 -0.0566089153289795 30.0
0 -0.0667482614517212 31.0
0 0.0591286420822144 32.0
0 -0.00796186923980713 33.0
0 -0.0313495397567749 34.0
0 -0.0523002147674561 35.0
0 -0.0986260175704956 36.0
0 -0.0830918550491333 37.0
0 -0.0109552145004272 38.0
0 0.0258513689041138 39.0
0 0.100059866905212 40.0
0 -0.0489490032196045 41.0
0 -0.0823169946670532 42.0
284.30898456 295.044494628906 43.0
0 -0.00225555896759033 44.0
0 -0.00194597244262695 45.0
0 0.008186936378479 46.0
0 -0.0848242044448853 47.0
0 -0.0251709222793579 48.0
1.2882250007 0.273666322231293 49.0
0.072113733817 0.109750747680664 50.0
2.1939200668 0.892701864242554 51.0
0 -0.0957170724868774 52.0
0 -0.0381814241409302 53.0
0 0.0355602502822876 54.0
0 -0.0999060869216919 55.0
0 -0.0206941366195679 56.0
0 -0.0395489931106567 57.0
0 -0.0338488817214966 58.0
0 0.0129462480545044 59.0
0 -0.122908473014832 60.0
0 0.0432587862014771 61.0
0 0.900839328765869 62.0
294.95874107 254.376190185547 0.0
0 -0.0303236246109009 1.0
0 -0.0799576044082642 2.0
0 -0.0272911787033081 3.0
0 -0.0522257089614868 4.0
0 -0.0339633226394653 5.0
0 0.136012434959412 6.0
0 -0.192722201347351 7.0
0 -0.0709267854690552 8.0
0 -0.000693917274475098 9.0
0 -0.0463498830795288 10.0
0 0.0346298813819885 11.0
0 -0.0852018594741821 12.0
76.259398208 74.8948516845703 13.0
30.880485354 24.8182468414307 14.0
0 -0.0874038934707642 15.0
0 -0.0268994569778442 16.0
26.489415048 25.6558208465576 17.0
0 0.0882662534713745 18.0
0 0.0480386018753052 19.0
0 0.00719130039215088 20.0
0 -0.0308982133865356 21.0
0 0.00953328609466553 22.0
28.909933772 28.0873203277588 23.0
0 0.0393303632736206 24.0
0 0.49286949634552 25.0
0 0.0585407018661499 26.0
0 0.0707262754440308 27.0
0 -0.0835500955581665 28.0
0 -0.0445905923843384 29.0
0 -0.0488268136978149 30.0
0 -0.0151222944259644 31.0
0 0.0485624074935913 32.0
0 -0.00921165943145752 33.0
0 -0.0539933443069458 34.0
0 0.135645508766174 35.0
0 0.250308394432068 36.0
0 -0.0135263204574585 37.0
0 0.00604498386383057 38.0
0 0.0519939661026001 39.0
0 0.166645407676697 40.0
0 -0.0729503631591797 41.0
0 -0.0646873712539673 42.0
295.63098587 294.68505859375 43.0
0 0.0219444036483765 44.0
0 -0.00334763526916504 45.0
0 0.0113371610641479 46.0
0 0.000776767730712891 47.0
0 0.0184973478317261 48.0
0 0.369735568761826 49.0
0.080492369965 0.0445271730422974 50.0
2.2720643893 0.957671403884888 51.0
0 -0.0485013723373413 52.0
0 -0.0512405633926392 53.0
0 -0.00213515758514404 54.0
0 -0.136172652244568 55.0
0 -0.0146380662918091 56.0
0 -0.0528756380081177 57.0
0 -0.0483771562576294 58.0
0 -0.027962327003479 59.0
0 0.150529742240906 60.0
0 -0.00977671146392822 61.0
0 0.810588598251343 62.0
235.9834042 252.189270019531 0.0
0 -0.0547944307327271 1.0
0 -0.113679766654968 2.0
0 -0.00748121738433838 3.0
0 -0.0540133714675903 4.0
0 -0.0339962244033813 5.0
0 -0.0114771127700806 6.0
0 -0.0389670133590698 7.0
0 -0.0277209281921387 8.0
0 -0.0554300546646118 9.0
0 0.0347503423690796 10.0
0 -0.0588936805725098 11.0
0 0.0792583227157593 12.0
60.453966266 59.508716583252 13.0
42.07636244 37.4448928833008 14.0
0 -0.0395580530166626 15.0
0 -0.0620073080062866 16.0
32.55161438 32.063060760498 17.0
0 0.0195995569229126 18.0
0 0.0144411325454712 19.0
0 0.0326699018478394 20.0
0 -0.0140856504440308 21.0
0 -0.0117789506912231 22.0
28.175531348 27.1331844329834 23.0
0 0.0127805471420288 24.0
0 0.585150063037872 25.0
0 0.0815380811691284 26.0
0 -0.0233396291732788 27.0
0 -0.00124657154083252 28.0
0 -0.0549300909042358 29.0
0 -0.0550723075866699 30.0
0 -0.039711594581604 31.0
0 -0.00854122638702393 32.0
0 0.0230468511581421 33.0
0 -0.0417224168777466 34.0
0 -0.0764786005020142 35.0
0 0.0251065492630005 36.0
0 -0.0470765829086304 37.0
0 0.0379818677902222 38.0
0 0.077269434928894 39.0
0 0.0134896039962769 40.0
0 0.0107535123825073 41.0
0 0.00813519954681396 42.0
302.53042299 296.551513671875 43.0
0 -0.0454212427139282 44.0
0 0.0348494052886963 45.0
0 0.00621473789215088 46.0
0 -0.0500205755233765 47.0
0 0.0431832075119019 48.0
0 0.280554324388504 49.0
0.085883718722 0.0482538938522339 50.0
1.805869249 0.909223318099976 51.0
0 -0.0631493330001831 52.0
0 0.00204670429229736 53.0
0 -0.0195351839065552 54.0
0 -0.13075053691864 55.0
0 -0.0710817575454712 56.0
0 -0.010417103767395 57.0
0 -0.0594617128372192 58.0
0 -0.0175856351852417 59.0
0 0.0781816244125366 60.0
0 0.0260523557662964 61.0
0 0.802384734153748 62.0
278.9552749 259.183868408203 0.0
0 -0.136513829231262 1.0
0 -0.0864483118057251 2.0
0 0.00963389873504639 3.0
0 -0.0589689016342163 4.0
0 -0.102955460548401 5.0
0 0.010677695274353 6.0
0 -0.0893491506576538 7.0
0 -0.0279008150100708 8.0
0 0.0922979116439819 9.0
0 -3.63588333129883e-05 10.0
0 0.0167771577835083 11.0
0 -0.148958802223206 12.0
48.988235784 48.7090072631836 13.0
50.913591413 46.7927703857422 14.0
0 -0.0497547388076782 15.0
0 0.0567010641098022 16.0
33.498128009 33.5028038024902 17.0
0 -0.0442992448806763 18.0
0 -0.0530568361282349 19.0
0 0.0630489587783813 20.0
0 0.0346235036849976 21.0
0 -0.0628229379653931 22.0
38.712898969 38.2924156188965 23.0
0 -0.149911999702454 24.0
9.1277749745 0.587804198265076 25.0
0 0.0559014081954956 26.0
0 -0.00589227676391602 27.0
0 -0.0681363344192505 28.0
0 -0.0101979970932007 29.0
0 -0.0115853548049927 30.0
0 -0.102823615074158 31.0
0 0.0164850950241089 32.0
0 0.033598780632019 33.0
0 -0.0152472257614136 34.0
0 -0.0259252786636353 35.0
0 0.124727606773376 36.0
0 -0.175471425056458 37.0
0 0.0237997770309448 38.0
0 0.0357030630111694 39.0
0 -0.00742542743682861 40.0
0 -0.0120788812637329 41.0
0 -0.0676866769790649 42.0
285.31678736 295.608978271484 43.0
0 0.015123724937439 44.0
0 0.0153284072875977 45.0
0 0.0149480104446411 46.0
0 0.00102889537811279 47.0
0 -0.11639130115509 48.0
0 0.254399716854095 49.0
0.080393239348 0.0897508859634399 50.0
1.5151120086 1.01255583763123 51.0
0 -0.0724281072616577 52.0
0 -0.0722895860671997 53.0
0 -0.0191947221755981 54.0
0 -0.0234204530715942 55.0
0 -0.0667537450790405 56.0
0 -0.00761187076568604 57.0
0 -0.0623255968093872 58.0
0 -0.0245422124862671 59.0
0 0.00192463397979736 60.0
0 0.0123215913772583 61.0
0 0.87827742099762 62.0
236.79043036 256.583709716797 0.0
0 0.00170457363128662 1.0
0 -0.0427380800247192 2.0
0 -0.061346173286438 3.0
0 -0.0385962724685669 4.0
0 0.000478625297546387 5.0
0 -0.0904289484024048 6.0
0 -0.0652099847793579 7.0
0 0.0278573036193848 8.0
0 -0.0441356897354126 9.0
0 0.016313910484314 10.0
0 -0.0344148874282837 11.0
0 -0.101011872291565 12.0
52.196457312 51.8125381469727 13.0
34.316352077 43.6829795837402 14.0
0 -0.00172102451324463 15.0
0 -0.0736857652664185 16.0
33.637875312 32.7665977478027 17.0
0 0.0798724889755249 18.0
0 0.0410674810409546 19.0
0 -0.00747501850128174 20.0
0 -0.0426398515701294 21.0
0 -0.0245763063430786 22.0
36.838440484 36.1089210510254 23.0
0 -0.0913139581680298 24.0
0 0.622694492340088 25.0
0 0.0926398038864136 26.0
0 -0.0151015520095825 27.0
0 0.0697296857833862 28.0
0 -0.000547051429748535 29.0
0 -0.129319071769714 30.0
0 -0.0489047765731812 31.0
0 -0.100904583930969 32.0
0 -0.0470536947250366 33.0
0 -0.0548983812332153 34.0
0 -0.0315204858779907 35.0
0 0.0833975076675415 36.0
0 -0.0446711778640747 37.0
0 -0.00541627407073975 38.0
0 0.0856627225875854 39.0
0 0.0498536825180054 40.0
0 -0.0600550174713135 41.0
0 0.0435184240341187 42.0
302.01659948 295.838348388672 43.0
0 0.003684401512146 44.0
0 0.0354141592979431 45.0
0 -0.0158208608627319 46.0
0 -0.0241470336914062 47.0
0 -0.0881630182266235 48.0
0 0.288093507289886 49.0
0.087202340172 0.0635348558425903 50.0
2.1894028871 0.912960886955261 51.0
0 -0.0279825925827026 52.0
0 -0.0115333795547485 53.0
0 0.052202582359314 54.0
0 -0.112258315086365 55.0
0 0.0395807027816772 56.0
0 -0.0478516817092896 57.0
0 -0.0680636167526245 58.0
0 -0.107596039772034 59.0
0 0.0203770399093628 60.0
0 0.0435339212417603 61.0
0 0.810285091400146 62.0
275.52811002 253.609512329102 0.0
0 -0.0786319971084595 1.0
0 -0.0990573167800903 2.0
0 -0.0272761583328247 3.0
0 -0.0308834314346313 4.0
0 0.0182875394821167 5.0
0 -0.124619126319885 6.0
0 -0.0289760828018188 7.0
0 0.0605158805847168 8.0
0 -0.00850355625152588 9.0
0 -0.00609982013702393 10.0
0 -0.0314863920211792 11.0
0 -0.0197769403457642 12.0
69.258033846 69.0236358642578 13.0
30.641860792 28.0196743011475 14.0
0 0.00213897228240967 15.0
0 -0.0357075929641724 16.0
26.915620124 26.68869972229 17.0
0 0.0148402452468872 18.0
0 0.0216108560562134 19.0
0 0.00448215007781982 20.0
0 -0.0105656385421753 21.0
0 -0.0268336534500122 22.0
35.785623654 35.0940055847168 23.0
0 -0.0996066331863403 24.0
0 1.03985047340393 25.0
0 -0.0338410139083862 26.0
0 0.0828138589859009 27.0
0 -0.0315266847610474 28.0
0 -0.0305031538009644 29.0
0 -0.0546609163284302 30.0
0 -0.0709627866744995 31.0
0 0.0128787755966187 32.0
0 0.0363789796829224 33.0
0 -0.039476752281189 34.0
0 0.0108450651168823 35.0
0 0.102755904197693 36.0
0 0.00397622585296631 37.0
0 0.025483250617981 38.0
0 0.025957465171814 39.0
0 0.0591577291488647 40.0
0 -0.042043924331665 41.0
0 -0.0508514642715454 42.0
298.85453126 296.455505371094 43.0
0 -0.047402024269104 44.0
0 -0.026500940322876 45.0
0 -0.0527349710464478 46.0
0 0.0369852781295776 47.0
0 -0.0346683263778687 48.0
0 0.251499772071838 49.0
0.070408304074 0.061210036277771 50.0
2.2216969395 0.939977407455444 51.0
0 -0.0401598215103149 52.0
0 -0.0241669416427612 53.0
0 0.037211537361145 54.0
0 -0.0407077074050903 55.0
0 0.0238651037216187 56.0
0 -0.0512501001358032 57.0
0 -0.0592163801193237 58.0
0 -0.0448287725448608 59.0
0 -0.0187968015670776 60.0
0 -0.00509345531463623 61.0
0 0.821603178977966 62.0
258.0898583 258.338104248047 0.0
0 0.0271409749984741 1.0
0 -0.0589920282363892 2.0
0 -0.0203496217727661 3.0
0 -0.0643304586410522 4.0
0 -0.151582598686218 5.0
0 0.00392663478851318 6.0
0 -0.000251173973083496 7.0
0 0.0558773279190063 8.0
0 -0.0631786584854126 9.0
0 -0.0367931127548218 10.0
0 -0.0643947124481201 11.0
0 -0.126420617103577 12.0
58.684710823 57.3199920654297 13.0
28.362083797 36.5040016174316 14.0
0 -0.0280557870864868 15.0
0 -0.0839284658432007 16.0
28.00511878 28.3194751739502 17.0
0 0.0249322652816772 18.0
0 -0.0265930891036987 19.0
0 0.00299990177154541 20.0
0 -0.0266395807266235 21.0
0 0.0412915945053101 22.0
26.136116014 25.5366039276123 23.0
0 0.0828999280929565 24.0
0 0.437870383262634 25.0
0 0.089718222618103 26.0
0 -0.00349706411361694 27.0
0 0.0718773603439331 28.0
0 -0.0507370233535767 29.0
0 -0.0797460079193115 30.0
0 0.0033189058303833 31.0
0 -0.0881906747817993 32.0
0 0.118277907371521 33.0
0 -0.0417400598526001 34.0
0 0.0724729299545288 35.0
0 0.0295964479446411 36.0
0 -0.0506249666213989 37.0
0 -0.0283654928207397 38.0
0 0.047319769859314 39.0
0 0.177985072135925 40.0
0 0.00583040714263916 41.0
0 -0.115365624427795 42.0
303.41825155 296.384185791016 43.0
0 -0.000683426856994629 44.0
0 0.0594072937965393 45.0
0 -0.0449815988540649 46.0
0 -0.0177844762802124 47.0
0 -0.0513092279434204 48.0
0 0.254330039024353 49.0
0.079684317409 0.132764935493469 50.0
1.684521442 0.908187925815582 51.0
0 -0.0674253702163696 52.0
0 -0.0298973321914673 53.0
0 0.0363026857376099 54.0
0 -0.064388632774353 55.0
0 -0.0153862237930298 56.0
0 -0.0327121019363403 57.0
0 -0.08472740650177 58.0
0 -0.0838803052902222 59.0
0 0.0205363035202026 60.0
0 0.0187453031539917 61.0
0 0.891480803489685 62.0
259.2429644 253.617156982422 0.0
0 -0.0805865526199341 1.0
0 0.0115920305252075 2.0
0 0.0300697088241577 3.0
0 -0.0296381711959839 4.0
0 -0.0181180238723755 5.0
0 0.0112375020980835 6.0
0 -0.0410705804824829 7.0
0 -0.0181132555007935 8.0
0 0.0136560201644897 9.0
0 0.0294629335403442 10.0
0 0.0130388140678406 11.0
0 0.0668996572494507 12.0
72.491194817 71.3011932373047 13.0
18.87755827 25.4371318817139 14.0
0 -0.0545574426651001 15.0
0 -0.0134605169296265 16.0
35.266665616 34.6976699829102 17.0
0 0.0143893957138062 18.0
0 -0.0176388025283813 19.0
0 0.03480064868927 20.0
0 -0.0258301496505737 21.0
0 0.0300444364547729 22.0
29.261050709 29.2402400970459 23.0
0 0.0171922445297241 24.0
0 0.558754742145538 25.0
0 0.131736636161804 26.0
0 0.0931657552719116 27.0
0 -0.0306903123855591 28.0
0 -0.0657137632369995 29.0
0 -0.0617725849151611 30.0
0 -0.0446254014968872 31.0
0 0.0135959386825562 32.0
0 -0.0330625772476196 33.0
0 -0.0390652418136597 34.0
0 -0.00958740711212158 35.0
0 0.108959555625916 36.0
0 -0.0327519178390503 37.0
0 0.00178492069244385 38.0
0 0.0508052110671997 39.0
0 0.0276273488998413 40.0
0 -0.0607472658157349 41.0
0 0.0155309438705444 42.0
282.50309314 298.201080322266 43.0
0 -0.0194317102432251 44.0
0 0.0531754493713379 45.0
0 0.00766551494598389 46.0
0 -0.0639127492904663 47.0
0 -0.0179907083511353 48.0
0 0.288338899612427 49.0
0.072925460211 0.0697098970413208 50.0
2.2388166397 0.923782348632812 51.0
0 -0.0687440633773804 52.0
0 0.0186504125595093 53.0
0 -0.0184532403945923 54.0
0 -0.0543824434280396 55.0
0 -0.0732287168502808 56.0
0 0.000516295433044434 57.0
0 -0.029152512550354 58.0
0 -0.0593231916427612 59.0
0 -0.0230828523635864 60.0
0 -0.0410212278366089 61.0
0 0.814813494682312 62.0
229.46714368 255.995590209961 0.0
0 -0.0394083261489868 1.0
0 -0.137884736061096 2.0
0 0.000667214393615723 3.0
0 -0.0181602239608765 4.0
0 0.0933946371078491 5.0
0 0.0310806035995483 6.0
0 -0.073330283164978 7.0
0 -0.037688136100769 8.0
0 0.00800836086273193 9.0
0 0.0549259185791016 10.0
0 0.0123074054718018 11.0
0 0.0445894002914429 12.0
81.816986617 78.7236938476562 13.0
15.768766602 23.113826751709 14.0
0 -0.00509178638458252 15.0
0 -0.0781384706497192 16.0
43.190744436 42.1284141540527 17.0
0 0.032068133354187 18.0
0 0.00882279872894287 19.0
0 -0.000513195991516113 20.0
0 0.00341188907623291 21.0
0 0.015344500541687 22.0
27.095534136 27.030632019043 23.0
0 -0.0535022020339966 24.0
0 0.604202568531036 25.0
0 0.0651577711105347 26.0
0 0.0190017223358154 27.0
0 -0.0357233285903931 28.0
0 -0.0158900022506714 29.0
0 -0.0485731363296509 30.0
0 -0.0213562250137329 31.0
0 0.0241903066635132 32.0
0 -0.0329519510269165 33.0
0 -0.021307110786438 34.0
0 0.00812733173370361 35.0
0 -0.170186638832092 36.0
0 -0.0182007551193237 37.0
0 -0.0350297689437866 38.0
0 -0.0429037809371948 39.0
0 0.121419787406921 40.0
0 0.0381151437759399 41.0
0 -0.0931364297866821 42.0
289.39495836 295.299865722656 43.0
0 0.00249326229095459 44.0
0 0.039201557636261 45.0
0 -0.0119143724441528 46.0
0 -0.0698095560073853 47.0
0 -0.0570131540298462 48.0
0 0.339956164360046 49.0
0.091526673166 0.047577977180481 50.0
1.4856117511 0.838272511959076 51.0
0 -0.0498615503311157 52.0
0 -0.0506423711776733 53.0
0 0.0681065320968628 54.0
0 -0.101790308952332 55.0
0 0.00343155860900879 56.0
0 -0.0397061109542847 57.0
0 -0.0191272497177124 58.0
0 0.0193649530410767 59.0
0 -0.0508273839950562 60.0
0 0.0223146677017212 61.0
0 0.900046944618225 62.0
278.75643369 254.988006591797 0.0
0 -0.00156223773956299 1.0
0 -0.126896262168884 2.0
0 -2.51531600952148e-05 3.0
0 -0.0124658346176147 4.0
0 -0.0274072885513306 5.0
0 -0.0378471612930298 6.0
0 -0.0335036516189575 7.0
0 -0.0462126731872559 8.0
0 0.0319150686264038 9.0
0 0.00915682315826416 10.0
0 -0.00294578075408936 11.0
0 -0.0888162851333618 12.0
60.591985627 60.7050437927246 13.0
40.756077199 34.6944999694824 14.0
0 -0.0370475053787231 15.0
0 -0.0560866594314575 16.0
34.426570234 34.7895240783691 17.0
0 0.0172518491744995 18.0
0 -0.0114957094192505 19.0
0 0.0302976369857788 20.0
0 -0.0286353826522827 21.0
0 0.0737341642379761 22.0
29.591181045 28.5957126617432 23.0
0 0.051810622215271 24.0
0 0.47111302614212 25.0
0 0.0719417333602905 26.0
0 0.0841845273971558 27.0
0 -0.040990948677063 28.0
0 -0.0439120531082153 29.0
0 -0.0606955289840698 30.0
0 -0.0247870683670044 31.0
0 -0.0164681673049927 32.0
0 0.00453341007232666 33.0
0 -0.0201631784439087 34.0
0 0.0319981575012207 35.0
0 -0.00100481510162354 36.0
0 -0.0878287553787231 37.0
0 0.00961554050445557 38.0
0 0.0217269659042358 39.0
0 -0.0589462518692017 40.0
0 -0.0748904943466187 41.0
0 0.0422662496566772 42.0
293.26135685 297.238372802734 43.0
0 -0.00402987003326416 44.0
0 0.0610730648040771 45.0
0 -0.0507065057754517 46.0
0 -0.0176280736923218 47.0
0 -0.0535885095596313 48.0
0 0.288827836513519 49.0
0.071721666677 0.0766397714614868 50.0
1.6649417791 0.891909658908844 51.0
0 -0.0555139780044556 52.0
0 0.00716912746429443 53.0
0 0.00130212306976318 54.0
0 -0.0877078771591187 55.0
0 0.0244835615158081 56.0
0 -0.0482531785964966 57.0
0 -0.00548470020294189 58.0
0 -0.03492271900177 59.0
0 0.0675753355026245 60.0
0 0.0129066705703735 61.0
0 0.851017475128174 62.0
279.70088099 251.439254760742 0.0
0 -0.11172354221344 1.0
0 -0.031352162361145 2.0
0 -0.0225092172622681 3.0
0 -0.0371185541152954 4.0
0 -0.0911537408828735 5.0
0 -0.0390759706497192 6.0
0 -0.0303889513015747 7.0
0 -0.0938466787338257 8.0
0 0.103317618370056 9.0
0 -0.121621012687683 10.0
0 -0.0347284078598022 11.0
0 -0.083661675453186 12.0
63.959198209 62.5635223388672 13.0
44.11608437 32.5435256958008 14.0
0 -0.0343214273452759 15.0
0 -0.0200308561325073 16.0
29.07866159 28.5092887878418 17.0
0 -0.0154563188552856 18.0
0 0.0230542421340942 19.0
0 0.0246375799179077 20.0
0 -0.00883209705352783 21.0
0 -0.0386601686477661 22.0
38.388323032 38.7963600158691 23.0
0 -0.0741630792617798 24.0
0.22530124412 0.655091762542725 25.0
0 0.0138391256332397 26.0
0 0.00141388177871704 27.0
0 -0.0755048990249634 28.0
0 -0.0305835008621216 29.0
0 -0.0576844215393066 30.0
0 -0.0105997323989868 31.0
0 -0.00834381580352783 32.0
0 0.0252841711044312 33.0
0 -0.0347722768783569 34.0
0 -0.0188513994216919 35.0
0 0.0660432577133179 36.0
0 -0.0274440050125122 37.0
0 0.00822722911834717 38.0
0 0.0281692743301392 39.0
0 0.120727896690369 40.0
0 -0.0275974273681641 41.0
0 0.015499472618103 42.0
294.06010271 296.728637695312 43.0
0 -0.0492340326309204 44.0
0 -0.00578689575195312 45.0
0 -0.133355021476746 46.0
0 0.0490392446517944 47.0
0 0.0749515295028687 48.0
0 0.239128291606903 49.0
0.077057333571 0.060278058052063 50.0
2.064094059 0.874144077301025 51.0
0 -0.0675612688064575 52.0
0 -0.0178793668746948 53.0
0 -0.0185216665267944 54.0
0 -0.0377649068832397 55.0
0 -0.0658837556838989 56.0
0 -0.00195682048797607 57.0
0 0.0262209177017212 58.0
0 -0.0168665647506714 59.0
0 0.100191950798035 60.0
0 0.0634614229202271 61.0
0 0.916008293628693 62.0
285.87677014 254.712310791016 0.0
0 -0.198690056800842 1.0
0 -0.0634375810623169 2.0
0 -0.0129903554916382 3.0
0 -0.0525916814804077 4.0
0 -0.097725510597229 5.0
0 0.103518843650818 6.0
0 -0.0325106382369995 7.0
0 -0.0417042970657349 8.0
0 0.130245566368103 9.0
0 0.0417388677597046 10.0
0 0.0465866923332214 11.0
0 0.0226281881332397 12.0
67.682685537 66.1621475219727 13.0
23.908782334 31.5597019195557 14.0
0 -0.0998121500015259 15.0
0 -0.0172363519668579 16.0
34.699577358 34.2666130065918 17.0
0 -0.0128926038742065 18.0
0 -0.0167165994644165 19.0
0 -0.0085529088973999 20.0
0 -0.0429607629776001 21.0
0 -0.0496574640274048 22.0
34.914879306 33.9641036987305 23.0
0 -0.0447417497634888 24.0
0 0.490119397640228 25.0
0 0.0583041906356812 26.0
0 0.0208163261413574 27.0
0 -0.0589085817337036 28.0
0 -0.0580297708511353 29.0
0 -0.111321449279785 30.0
0 -0.0379549264907837 31.0
0 -0.00244486331939697 32.0
0 0.131206393241882 33.0
0 -0.0751134157180786 34.0
0 -0.0258241891860962 35.0
0 0.101891875267029 36.0
0 -0.109826922416687 37.0
0 -0.00935208797454834 38.0
0 0.0207852125167847 39.0
0 0.0148090124130249 40.0
0 -0.045496940612793 41.0
0 -0.0888906717300415 42.0
294.97744891 295.6806640625 43.0
0 -0.058754563331604 44.0
0 -0.0319780111312866 45.0
0 -0.0344361066818237 46.0
0 -0.0538095235824585 47.0
0 0.0156883001327515 48.0
0 0.27554002404213 49.0
0.10859238404 0.10049831867218 50.0
1.943095784 1.00417482852936 51.0
0 -0.0661503076553345 52.0
0 -0.0648001432418823 53.0
0 0.0792077779769897 54.0
0 -0.0755792856216431 55.0
0 -0.0594686269760132 56.0
0 -0.0460256338119507 57.0
0 -0.09235680103302 58.0
0 0.0510605573654175 59.0
0 0.12999165058136 60.0
0 -0.0838476419448853 61.0
0 0.839580297470093 62.0
233.74491392 250.931686401367 0.0
0 0.0425022840499878 1.0
0 -0.0226060152053833 2.0
0 0.00238525867462158 3.0
0 -0.0323711633682251 4.0
0 0.0267156362533569 5.0
0 0.0974277257919312 6.0
0 -0.0740038156509399 7.0
0 0.0138274431228638 8.0
0 -0.0221744775772095 9.0
0 -0.0284255743026733 10.0
0 0.0442582368850708 11.0
0 -0.0578361749649048 12.0
63.226397139 62.995922088623 13.0
40.629856699 34.2290382385254 14.0
0 -0.0455285310745239 15.0
0 0.0102370977401733 16.0
36.937310964 36.6580238342285 17.0
0 0.017952561378479 18.0
0 -0.0345855951309204 19.0
0 -0.00925314426422119 20.0
0 -0.0340479612350464 21.0
0 -0.0296961069107056 22.0
28.357222253 27.610818862915 23.0
0 0.0570138692855835 24.0
0 0.473492383956909 25.0
0 0.0650573968887329 26.0
0 0.0774146318435669 27.0
0 0.176062941551208 28.0
0 -0.0525511503219604 29.0
0 -0.0767867565155029 30.0
0 0.0103923082351685 31.0
0 -0.0240136384963989 32.0
0 0.0123475790023804 33.0
0 -0.0401045083999634 34.0
0 -0.0303133726119995 35.0
0 0.156605124473572 36.0
0 0.004996657371521 37.0
0 0.0149468183517456 38.0
0 0.0714668035507202 39.0
0 0.0224951505661011 40.0
0 0.00685226917266846 41.0
0 -0.0146585702896118 42.0
290.8565828 296.799682617188 43.0
0 -0.0070730447769165 44.0
0 0.0422322154045105 45.0
0 -0.11093270778656 46.0
0 -0.0291160345077515 47.0
0 0.00521409511566162 48.0
0 0.301045298576355 49.0
0.092757093424 0.0654095411300659 50.0
1.5847780696 1.00412654876709 51.0
0 -0.168671250343323 52.0
0 -0.0499330759048462 53.0
0 0.00326573848724365 54.0
0 -0.0520559549331665 55.0
0 -0.040981650352478 56.0
0 -0.0251923799514771 57.0
0 0.00169932842254639 58.0
0 -0.0481201410293579 59.0
0 -0.107261061668396 60.0
0 -0.0176295042037964 61.0
0 0.883471131324768 62.0
292.32594393 255.997695922852 0.0
0 -0.0213404893875122 1.0
0 -0.0788646936416626 2.0
0 -0.00582921504974365 3.0
0 -0.0187546014785767 4.0
0 -0.0915776491165161 5.0
0 0.0328296422958374 6.0
0 -0.103191018104553 7.0
0 0.0137812495231628 8.0
0 0.0407518148422241 9.0
0 0.00519645214080811 10.0
0 -0.0490796566009521 11.0
0 0.00233113765716553 12.0
69.523303678 68.9706192016602 13.0
26.563109338 29.8792209625244 14.0
0 0.00751721858978271 15.0
0 -0.0837777853012085 16.0
28.255373994 28.1195507049561 17.0
0 0.00190460681915283 18.0
0 8.33272933959961e-05 19.0
0 0.050707221031189 20.0
0 0.00998413562774658 21.0
0 0.0184482336044312 22.0
28.809435015 27.5094623565674 23.0
0 0.0968774557113647 24.0
0 0.439456760883331 25.0
0 0.0628942251205444 26.0
0 -0.0494205951690674 27.0
0 -0.069316029548645 28.0
0 -0.0506170988082886 29.0
0 -0.0498307943344116 30.0
0 -0.0203770399093628 31.0
0 0.000787615776062012 32.0
0 0.0442577600479126 33.0
0 0.00693953037261963 34.0
0 0.0443379878997803 35.0
0 -0.0703386068344116 36.0
0 -0.00780713558197021 37.0
0 -0.0215021371841431 38.0
0 0.0207475423812866 39.0
0 0.0757559537887573 40.0
0 -0.0228463411331177 41.0
0 -0.017616868019104 42.0
292.9634541 295.36181640625 43.0
0 0.0743669271469116 44.0
0 -0.0444004535675049 45.0
0 0.00959360599517822 46.0
0 -0.0494529008865356 47.0
0 0.0334094762802124 48.0
0.4262285813 0.347268551588058 49.0
0.093864799732 0.0575705766677856 50.0
1.7240773046 0.86673241853714 51.0
0 -0.0542827844619751 52.0
0 -0.0221167802810669 53.0
0 -0.00288999080657959 54.0
0 -0.104451775550842 55.0
0 0.0297430753707886 56.0
0 -0.0369542837142944 57.0
0 -0.0528258085250854 58.0
0 -0.0666600465774536 59.0
0 0.00707042217254639 60.0
0 0.0496968030929565 61.0
0 0.949601531028748 62.0
239.1078801 255.249282836914 0.0
0 -0.0770069360733032 1.0
0 -0.0989340543746948 2.0
0 -0.0424762964248657 3.0
0 -0.00355327129364014 4.0
0 -0.0179840326309204 5.0
0 -0.0320588350296021 6.0
0 -0.198328852653503 7.0
0 0.0143889188766479 8.0
0 0.00341546535491943 9.0
0 0.0310331583023071 10.0
0 0.0403939485549927 11.0
0 0.0879300832748413 12.0
66.235542845 66.3211364746094 13.0
32.112309833 32.7131881713867 14.0
0 -0.0837494134902954 15.0
0 0.0158907175064087 16.0
41.504587676 40.6095542907715 17.0
0 0.014539361000061 18.0
0 0.00797450542449951 19.0
0 0.00625860691070557 20.0
0 -0.0199202299118042 21.0
0 -0.0171180963516235 22.0
30.037984899 28.7568798065186 23.0
0 -0.0295382738113403 24.0
0 0.516092598438263 25.0
0 0.0899685621261597 26.0
0 0.0612038373947144 27.0
0 0.112028956413269 28.0
0 -0.0670183897018433 29.0
0 -0.0193228721618652 30.0
0 -0.0420123338699341 31.0
0 -0.0641576051712036 32.0
0 0.0108681917190552 33.0
0 -0.0446163415908813 34.0
0 -0.0433748960494995 35.0
0 0.177858948707581 36.0
0 -0.00926649570465088 37.0
0 0.0219074487686157 38.0
0 0.0343910455703735 39.0
0 0.00738990306854248 40.0
0 0.0328680276870728 41.0
0 -0.138499855995178 42.0
301.61068488 297.424621582031 43.0
0 0.0372837781906128 44.0
0 -0.0534369945526123 45.0
0 -0.0464617013931274 46.0
0 -0.0837253332138062 47.0
0 -0.0214744806289673 48.0
0 0.266729533672333 49.0
0.075579692182 0.135901808738708 50.0
1.6644130673 0.96534276008606 51.0
0 -0.0911906957626343 52.0
0 -0.110609173774719 53.0
0 0.0281938314437866 54.0
0 -0.0591660737991333 55.0
0 -0.0423232316970825 56.0
0 -0.0193754434585571 57.0
0 -0.0493940114974976 58.0
0 -0.0860732793807983 59.0
0 0.0266188383102417 60.0
0 0.0465234518051147 61.0
0 0.841558814048767 62.0
269.65509803 259.601470947266 0.0
0 -0.0917297601699829 1.0
0 -0.0575450658798218 2.0
0 -0.0498524904251099 3.0
0 -0.0446764230728149 4.0
0 -0.0171395540237427 5.0
0 0.129329562187195 6.0
0 -0.14928662776947 7.0
0 0.0508279800415039 8.0
0 0.0773738622665405 9.0
0 -0.0574086904525757 10.0
0 0.00738251209259033 11.0
0 0.0493606328964233 12.0
61.084048008 60.7963371276855 13.0
37.771865436 35.8726539611816 14.0
0 -0.0398565530776978 15.0
0 -0.0282958745956421 16.0
33.559687325 33.2427940368652 17.0
0 -0.00465571880340576 18.0
0 0.00348174571990967 19.0
0 0.0166972875595093 20.0
0 -0.0523034334182739 21.0
0 0.034964919090271 22.0
35.250780522 34.5575981140137 23.0
0 -0.0720292329788208 24.0
0 0.613769292831421 25.0
0 0.0657972097396851 26.0
0 0.129168629646301 27.0
0 -0.0424638986587524 28.0
0 -0.092814564704895 29.0
0 -0.0315068960189819 30.0
0 -0.0139282941818237 31.0
0 0.00383174419403076 32.0
0 0.0802339315414429 33.0
0 -0.0486716032028198 34.0
0 0.00742542743682861 35.0
0 -0.0249060392379761 36.0
0 -0.098692774772644 37.0
0 0.029765248298645 38.0
0 0.0765308141708374 39.0
0 0.0170649290084839 40.0
0 -0.0225434303283691 41.0
0 -0.0646263360977173 42.0
290.19431974 296.422668457031 43.0
0 -0.043617844581604 44.0
0 -0.0231448411941528 45.0
0 0.00743472576141357 46.0
0 -0.00414478778839111 47.0
0 0.00119197368621826 48.0
0 0.263933777809143 49.0
0.095594335305 0.116199851036072 50.0
1.6282663224 0.92182058095932 51.0
0 -0.0875595808029175 52.0
0 -0.0552221536636353 53.0
0 0.082166314125061 54.0
0 -0.120495438575745 55.0
0 -0.0642482042312622 56.0
0 0.015947699546814 57.0
0 -0.0293773412704468 58.0
0 -0.0921410322189331 59.0
0 -0.0682493448257446 60.0
0 -0.0115174055099487 61.0
0 0.860755324363708 62.0
251.96702745 255.2939453125 0.0
0 -0.0115524530410767 1.0
0 -0.013032078742981 2.0
0 -0.0185478925704956 3.0
0 -0.013135552406311 4.0
0 -0.0297843217849731 5.0
0 0.0798186063766479 6.0
0 -0.145007729530334 7.0
0 0.0298632383346558 8.0
0 -0.0895344018936157 9.0
0 -0.0609050989151001 10.0
0 0.0218291282653809 11.0
0 -0.0899358987808228 12.0
69.992569034 69.1586837768555 13.0
38.122754273 27.7796306610107 14.0
0 0.000399947166442871 15.0
0 -0.0407058000564575 16.0
39.512086424 39.4134826660156 17.0
0 -0.0476392507553101 18.0
0 -0.0278395414352417 19.0
0 0.0949941873550415 20.0
0 -0.00745213031768799 21.0
0 -0.111168026924133 22.0
37.853688804 37.3456649780273 23.0
0 -0.0466948747634888 24.0
0 0.642143309116364 25.0
0 0.023772120475769 26.0
0 0.0271722674369812 27.0
0 0.0547078847885132 28.0
0 -0.0588635206222534 29.0
0 -0.0788075923919678 30.0
0 -0.0489000082015991 31.0
0 0.0252994298934937 32.0
0 0.0391031503677368 33.0
0 -0.0022660493850708 34.0
0 -0.0163501501083374 35.0
0 0.148421406745911 36.0
0 -0.0247405767440796 37.0
0 0.000170111656188965 38.0
0 0.058008074760437 39.0
0 0.112050414085388 40.0
0 -0.0122402906417847 41.0
0 -0.064761757850647 42.0
290.23449131 296.172607421875 43.0
0 0.0144089460372925 44.0
0 -0.0038597583770752 45.0
0 0.00814282894134521 46.0
0 -0.0665925741195679 47.0
0 0.0172079801559448 48.0
0 0.324770987033844 49.0
0.098581653162 0.0659047365188599 50.0
2.2157008611 0.999594926834106 51.0
0 -0.0354992151260376 52.0
0 -0.0874449014663696 53.0
0 -0.0023425817489624 54.0
0 -0.0595406293869019 55.0
0 -0.0704611539840698 56.0
0 -0.0140961408615112 57.0
0 -0.080919623374939 58.0
0 -0.0576931238174438 59.0
0 0.04743492603302 60.0
0 0.00222814083099365 61.0
0 0.908241391181946 62.0
221.87361643 250.858062744141 0.0
0 -0.20181143283844 1.0
0 0.0312269926071167 2.0
0 0.0402220487594604 3.0
0 -0.0424343347549438 4.0
0 0.0177978277206421 5.0
0 0.0421994924545288 6.0
0 0.0092545747756958 7.0
0 0.0436838865280151 8.0
0 -0.0977932214736938 9.0
0 -0.0293251276016235 10.0
0 -0.058889627456665 11.0
0 0.0972751379013062 12.0
59.034370612 57.6324462890625 13.0
27.619327639 35.7447204589844 14.0
0 -0.000832200050354004 15.0
0 -0.0554031133651733 16.0
30.852876702 30.5728034973145 17.0
0 0.0521899461746216 18.0
0 0.0551606416702271 19.0
0 -0.0248152017593384 20.0
0 -0.0419727563858032 21.0
0 -0.0397745370864868 22.0
39.781319642 39.8574066162109 23.0
0 -0.153972268104553 24.0
0 1.29342782497406 25.0
0 0.0337423086166382 26.0
0 -0.0842113494873047 27.0
0 0.0115588903427124 28.0
0 -0.0395597219467163 29.0
0 -0.109152436256409 30.0
0 -0.0304535627365112 31.0
0 -0.0135732889175415 32.0
0 -0.0542198419570923 33.0
0 -0.0194472074508667 34.0
0 0.0177513360977173 35.0
0 0.0954197645187378 36.0
0 -0.035056471824646 37.0
0 -0.0847991704940796 38.0
0 0.0128587484359741 39.0
0 0.0196107625961304 40.0
0 0.0758084058761597 41.0
0 -0.0130783319473267 42.0
295.29156438 294.800415039062 43.0
0 0.0198034048080444 44.0
0 0.0807697176933289 45.0
0 -0.0116175413131714 46.0
0 -0.0926425457000732 47.0
0 -0.00444662570953369 48.0
0 0.299502193927765 49.0
0.065545535818 0.0681103467941284 50.0
1.8379005543 0.907299280166626 51.0
0 -0.0774804353713989 52.0
0 -0.0269654989242554 53.0
0 0.0269545316696167 54.0
0 -0.0933605432510376 55.0
0 0.0186430215835571 56.0
0 -0.0408540964126587 57.0
0 0.0186566114425659 58.0
0 -0.0178381204605103 59.0
0 0.0521858930587769 60.0
0 -0.0221694707870483 61.0
0 0.79572331905365 62.0
214.98327155 260.39990234375 0.0
0 -0.096993088722229 1.0
0 -0.0237008333206177 2.0
0 -0.00161516666412354 3.0
0 -0.0174456834793091 4.0
0 -0.0192168951034546 5.0
0 -0.0840774774551392 6.0
0 -0.0322657823562622 7.0
0 -0.0190842151641846 8.0
0 -0.0212026834487915 9.0
0 -0.0586148500442505 10.0
0 0.0205058455467224 11.0
0 -0.0470567941665649 12.0
71.887816988 71.121955871582 13.0
30.595490687 26.8010921478271 14.0
0 -0.0112186670303345 15.0
0 -0.00659596920013428 16.0
33.049350699 34.7201728820801 17.0
0 -0.031147837638855 18.0
0 0.0150641202926636 19.0
0 0.0956341028213501 20.0
0 -0.00992357730865479 21.0
0 -0.06934654712677 22.0
39.875402078 38.846263885498 23.0
0 0.101323485374451 24.0
0 0.415640115737915 25.0
0 0.132080435752869 26.0
0 -0.0130187273025513 27.0
0 -0.0223323106765747 28.0
0 -0.0319687128067017 29.0
0 -0.0846406221389771 30.0
0 -0.0498119592666626 31.0
0 0.0715793371200562 32.0
0 0.0502403974533081 33.0
0 -0.0724531412124634 34.0
0 -0.0227102041244507 35.0
0 0.0624231100082397 36.0
0 -0.0744141340255737 37.0
0 0.0285576581954956 38.0
0 0.00923216342926025 39.0
0 0.00445544719696045 40.0
0 -0.0293760299682617 41.0
0 -0.0249143838882446 42.0
298.53518202 297.97998046875 43.0
0 -0.0403820276260376 44.0
0 -0.00565958023071289 45.0
0 0.00547945499420166 46.0
0 -0.00282728672027588 47.0
0 -0.0417090654373169 48.0
0 0.286807775497437 49.0
0.065279441021 0.111547589302063 50.0
1.684512634 0.996189892292023 51.0
0 -0.0754841566085815 52.0
0 -0.0916053056716919 53.0
0 -0.0476983785629272 54.0
0 -0.0491400957107544 55.0
0 -0.110941290855408 56.0
0 0.0175474882125854 57.0
0 -0.0128051042556763 58.0
0 -0.00401151180267334 59.0
0 0.181239724159241 60.0
0 0.0500894784927368 61.0
0 0.919889152050018 62.0
238.28621966 254.349304199219 0.0
0 -0.04944908618927 1.0
0 -0.00940191745758057 2.0
0 -0.0310515165328979 3.0
0 -0.0110782384872437 4.0
0 0.100500464439392 5.0
0 0.0345906019210815 6.0
0 -0.0487145185470581 7.0
0 -0.0700160264968872 8.0
0 0.0132812261581421 9.0
0 0.00399577617645264 10.0
0 -0.0637401342391968 11.0
0 0.0781062841415405 12.0
54.90366638 54.1390762329102 13.0
52.777318959 45.6223526000977 14.0
0 0.0187696218490601 15.0
0 -0.0450600385665894 16.0
32.32894565 32.2428817749023 17.0
0 0.0297712087631226 18.0
0 0.0479799509048462 19.0
0 0.0247474908828735 20.0
0 -0.0532666444778442 21.0
0 0.0165184736251831 22.0
27.986443286 27.2485179901123 23.0
0 0.0212610960006714 24.0
0 0.530200719833374 25.0
0 0.0314754247665405 26.0
0 0.0248492360115051 27.0
0 0.0332096815109253 28.0
0 -0.0482867956161499 29.0
0 -0.0832673311233521 30.0
0 -0.0497344732284546 31.0
0 -0.0747014284133911 32.0
0 0.0284725427627563 33.0
0 -0.0135747194290161 34.0
0 3.06367874145508e-05 35.0
0 0.000362992286682129 36.0
0 -0.0602704286575317 37.0
0 -0.0239309072494507 38.0
0 0.0429891347885132 39.0
0 0.0571655035018921 40.0
0 -0.0159242153167725 41.0
0 0.0114120244979858 42.0
297.56626058 295.739013671875 43.0
0 0.00166165828704834 44.0
0 0.0370775461196899 45.0
0 0.00691878795623779 46.0
0 -0.0578644275665283 47.0
0 -0.0501290559768677 48.0
1.485817682 0.257978796958923 49.0
0.076859983049 0.129126310348511 50.0
1.7325096791 0.875061988830566 51.0
0 -0.0737920999526978 52.0
0 -0.0373994112014771 53.0
0 0.00810444355010986 54.0
0 -0.0349797010421753 55.0
0 -0.0234354734420776 56.0
0 -0.0133711099624634 57.0
0 -0.0163496732711792 58.0
0 -0.0806790590286255 59.0
0 -0.075334906578064 60.0
0 -0.0310345888137817 61.0
0 0.842981576919556 62.0
268.73683534 254.528213500977 0.0
0 -0.0120707750320435 1.0
0 -0.0684617757797241 2.0
0 -0.00221455097198486 3.0
0 -0.00886666774749756 4.0
0 0.030669093132019 5.0
0 0.0261930227279663 6.0
0 -0.0743247270584106 7.0
0 0.0577479600906372 8.0
0 0.0139602422714233 9.0
0 0.020399808883667 10.0
0 0.0196655988693237 11.0
0 -0.158234238624573 12.0
60.984075135 60.5113830566406 13.0
38.439238265 36.0104713439941 14.0
0 -0.0435396432876587 15.0
0 -0.0699506998062134 16.0
39.179407764 38.0995483398438 17.0
0 0.0162221193313599 18.0
0 0.0253814458847046 19.0
0 0.038343071937561 20.0
0 -0.0131963491439819 21.0
0 -0.0193401575088501 22.0
33.611903438 33.0427742004395 23.0
0 -0.0524822473526001 24.0
0 0.540285229682922 25.0
0 0.0113972425460815 26.0
0 0.0789550542831421 27.0
0 -0.00841104984283447 28.0
0 0.0121558904647827 29.0
0 -0.0908530950546265 30.0
0 -0.0218108892440796 31.0
0 -0.00808918476104736 32.0
0 0.110394835472107 33.0
0 -0.032305121421814 34.0
0 0.0524091720581055 35.0
0 -0.0169707536697388 36.0
0 -0.107645869255066 37.0
0 -0.0250808000564575 38.0
0 0.0868169069290161 39.0
0 0.106121897697449 40.0
0 -0.0595428943634033 41.0
0 0.0290435552597046 42.0
287.1059853 297.802185058594 43.0
0 -0.0222889184951782 44.0
0 0.00153601169586182 45.0
0 -0.00960648059844971 46.0
0 -0.047176718711853 47.0
0 0.00998103618621826 48.0
0 0.306509256362915 49.0
0.091702639074 0.0631076097488403 50.0
2.1224256986 0.867105185985565 51.0
0 -0.0331059694290161 52.0
0 -0.0615895986557007 53.0
0 0.00503313541412354 54.0
0 -0.0649746656417847 55.0
0 -0.0730208158493042 56.0
0 -0.0141993761062622 57.0
0 -0.0741733312606812 58.0
0 -0.00620949268341064 59.0
0 -0.0312753915786743 60.0
0 -0.0203112363815308 61.0
0 0.8266282081604 62.0
255.53472348 254.783157348633 0.0
0 -0.0714308023452759 1.0
0 -0.0325287580490112 2.0
0 -0.0042644739151001 3.0
0 -0.0658615827560425 4.0
0 -0.0666824579238892 5.0
0 -0.0549699068069458 6.0
0 -0.0122758150100708 7.0
0 0.0393425226211548 8.0
0 0.00312650203704834 9.0
0 -0.0715795755386353 10.0
0 -0.0565141439437866 11.0
0 -0.246550679206848 12.0
47.318277851 45.3212471008301 13.0
32.79229526 31.0349922180176 14.0
0 -0.0467754602432251 15.0
0 0.0100158452987671 16.0
25.453134098 24.8191909790039 17.0
0 -0.0632339715957642 18.0
0 -0.035309910774231 19.0
0 -0.0437105894088745 20.0
0 7.11679458618164e-05 21.0
0 -0.0481773614883423 22.0
37.581278578 37.6908531188965 23.0
0 -0.163291573524475 24.0
0 0.662212312221527 25.0
0 0.0448745489120483 26.0
0 -0.0462123155593872 27.0
0 0.0217746496200562 28.0
0 -0.0746723413467407 29.0
0 -0.0732070207595825 30.0
0 0.0236269235610962 31.0
0 0.0592674016952515 32.0
0 0.0305267572402954 33.0
0 -0.0632587671279907 34.0
0 0.0720080137252808 35.0
0 0.0211421251296997 36.0
0 -0.0745960474014282 37.0
0 0.013262152671814 38.0
0 -0.0279062986373901 39.0
0 0.0764502286911011 40.0
0 -0.00298070907592773 41.0
0 -0.188463807106018 42.0
304.65070894 287.613647460938 43.0
0 -0.0169540643692017 44.0
0 -0.803588628768921 45.0
0 0.0128442049026489 46.0
0 -0.0680464506149292 47.0
0 -0.0779539346694946 48.0
0 0.285985589027405 49.0
0.065633409226 0.0656055212020874 50.0
1.6199125903 0.951838731765747 51.0
0 -0.123941779136658 52.0
0 -0.0493694543838501 53.0
0 -0.0454093217849731 54.0
0 -0.139754414558411 55.0
0 0.126927971839905 56.0
0 0.0416916608810425 57.0
0 0.0120984315872192 58.0
0 -0.0399287939071655 59.0
0 -0.169296622276306 60.0
0 -0.0180004835128784 61.0
0 0.780012607574463 62.0
251.31625314 257.016723632812 0.0
0 -0.134858727455139 1.0
0 0.046146035194397 2.0
0 -0.0299860239028931 3.0
0 -0.0454438924789429 4.0
0 -0.138558268547058 5.0
0 -0.0341192483901978 6.0
0 -0.16635525226593 7.0
0 0.0168900489807129 8.0
0 0.00792920589447021 9.0
0 0.0352017879486084 10.0
0 -0.0544013977050781 11.0
0 0.0743449926376343 12.0
57.708967037 56.9092864990234 13.0
43.747324778 39.8457832336426 14.0
0 -0.0981842279434204 15.0
0 -0.0300651788711548 16.0
40.036539983 40.886604309082 17.0
0 0.00432074069976807 18.0
0 -0.0273457765579224 19.0
0 0.00760567188262939 20.0
0 -0.0445355176925659 21.0
0 -0.0130316019058228 22.0
34.16217594 33.5119667053223 23.0
0 -0.0543819665908813 24.0
0 0.583140134811401 25.0
0 0.0648263692855835 26.0
0 0.0540362596511841 27.0
0 0.0503896474838257 28.0
0 -0.0351206064224243 29.0
0 -0.0767058134078979 30.0
0 -0.033697247505188 31.0
0 -0.0134392976760864 32.0
0 -0.0659335851669312 33.0
0 -0.047540545463562 34.0
0 -0.0373023748397827 35.0
0 0.0797897577285767 36.0
0 -0.0217753648757935 37.0
0 0.0246552228927612 38.0
0 -0.00720036029815674 39.0
0 0.0200508832931519 40.0
0 -0.0313528776168823 41.0
0 0.0453017950057983 42.0
300.39455794 295.943572998047 43.0
0 -0.0222584009170532 44.0
0 -0.0440536737442017 45.0
0 -0.055768609046936 46.0
0 -0.00871741771697998 47.0
0 0.0306476354598999 48.0
0 0.273856222629547 49.0
0.082316863142 0.0819500684738159 50.0
1.6393939244 0.936725735664368 51.0
0 -0.0821701288223267 52.0
0 -0.0914274454116821 53.0
0 0.0293434858322144 54.0
0 -0.0333753824234009 55.0
0 -0.0974842309951782 56.0
0 0.038359522819519 57.0
0 -0.00876080989837646 58.0
0 -0.0585166215896606 59.0
0 -0.0395885705947876 60.0
0 0.00862395763397217 61.0
0 0.88965916633606 62.0
263.81034548 254.686294555664 0.0
0 -0.106966614723206 1.0
0 -0.0185388326644897 2.0
0 -0.0357426404953003 3.0
0 -0.0416945219039917 4.0
0 -0.100853562355042 5.0
0 0.00698697566986084 6.0
0 -0.1753751039505 7.0
0 -0.0187737941741943 8.0
0 -0.101905465126038 9.0
0 -0.0996798276901245 10.0
0 -0.0340784788131714 11.0
0 0.0472463369369507 12.0
73.294769378 73.1206817626953 13.0
34.665665358 28.8159637451172 14.0
0 -0.0611306428909302 15.0
0 -0.00946676731109619 16.0
33.22317322 33.239070892334 17.0
0 0.0206927061080933 18.0
0 0.0076676607131958 19.0
0 0.0142377614974976 20.0
0 0.013863205909729 21.0
0 -0.037788987159729 22.0
27.7619274 27.5455532073975 23.0
0 -0.0860453844070435 24.0
20.552521613 0.598031282424927 25.0
0 0.0735019445419312 26.0
0 0.0115019083023071 27.0
0 0.0514158010482788 28.0
0 -0.0497709512710571 29.0
0 -0.0637829303741455 30.0
0 -0.0308958292007446 31.0
0 0.0169304609298706 32.0
0 0.144644618034363 33.0
0 -0.00722038745880127 34.0
0 -0.0145422220230103 35.0
0 0.0874837636947632 36.0
0 -0.0445436239242554 37.0
0 0.0143381357192993 38.0
0 0.0285400152206421 39.0
0 0.025112509727478 40.0
0 -0.00786030292510986 41.0
0 0.0069965124130249 42.0
298.56473116 293.94775390625 43.0
0 -0.0128027200698853 44.0
0 0.0325554609298706 45.0
0 -0.00719201564788818 46.0
0 -0.0299087762832642 47.0
0 0.0303291082382202 48.0
0.035281019878 0.289616942405701 49.0
0.068100821363 0.0739468336105347 50.0
1.8890982422 1.00237095355988 51.0
0 -0.0925747156143188 52.0
0 -0.0420318841934204 53.0
0 0.080222487449646 54.0
0 -0.0365043878555298 55.0
0 -0.0546702146530151 56.0
0 0.0107392072677612 57.0
0 -0.0302809476852417 58.0
0 -0.0612715482711792 59.0
0 0.0843354463577271 60.0
0 0.0824559926986694 61.0
0 0.941876769065857 62.0
228.55439973 256.064117431641 0.0
0 -0.0239721536636353 1.0
0 -0.0729974508285522 2.0
0 -0.0138410329818726 3.0
0 -0.00964987277984619 4.0
0 0.0915366411209106 5.0
0 0.0401686429977417 6.0
0 -0.0229476690292358 7.0
0 -0.0303810834884644 8.0
0 -0.044782280921936 9.0
0 0.00624597072601318 10.0
0 -0.0374701023101807 11.0
0 0.00825631618499756 12.0
65.119129974 64.4143981933594 13.0
29.687186289 34.6911849975586 14.0
0 -0.0350557565689087 15.0
0 0.0146735906600952 16.0
33.280789397 33.3570785522461 17.0
0 0.0134464502334595 18.0
0 0.0140794515609741 19.0
0 0.0276180505752563 20.0
0 0.0263561010360718 21.0
0 -0.0276647806167603 22.0
36.42474548 36.0947227478027 23.0
0 -0.0538707971572876 24.0
0 1.40564239025116 25.0
0 0.000244736671447754 26.0
0 0.0166367292404175 27.0
0 -0.0544883012771606 28.0
0 -0.0265351533889771 29.0
0 -0.0750707387924194 30.0
0 -0.00568258762359619 31.0
0 -0.00899088382720947 32.0
0 0.0361851453781128 33.0
0 -0.0328625440597534 34.0
0 -0.0852864980697632 35.0
0 -0.00544130802154541 36.0
0 -0.0613912343978882 37.0
0 0.0247989892959595 38.0
0 0.0485802888870239 39.0
0 -0.00247585773468018 40.0
0 -0.0653327703475952 41.0
0 -0.005318284034729 42.0
298.18324209 294.063201904297 43.0
0 0.023764967918396 44.0
0 -0.0577312707901001 45.0
0 -0.0686074495315552 46.0
0 -0.0232222080230713 47.0
0 -0.017884373664856 48.0
0 0.285026252269745 49.0
0.073575693848 0.0773911476135254 50.0
2.2216393747 0.891307651996613 51.0
0 -0.0699595212936401 52.0
0 -0.064095139503479 53.0
0 0.0201762914657593 54.0
0 -0.0556250810623169 55.0
0 0.0243848562240601 56.0
0 -0.0415269136428833 57.0
0 -0.0492066144943237 58.0
0 -0.0767771005630493 59.0
0 -0.0639208555221558 60.0
0 -0.0259007215499878 61.0
0 0.845465064048767 62.0
222.34075331 253.718811035156 0.0
0 -0.0436640977859497 1.0
0 -0.0161586999893188 2.0
0 0.0119465589523315 3.0
0 -0.078539252281189 4.0
0 -0.0253390073776245 5.0
0 -0.0111595392227173 6.0
0 -0.121204257011414 7.0
0 -0.00697606801986694 8.0
0 0.0195678472518921 9.0
0 0.0250877141952515 10.0
0 0.0345271825790405 11.0
0 0.0380328893661499 12.0
68.797360723 68.2078399658203 13.0
40.570969191 30.6455936431885 14.0
0 0.0183004140853882 15.0
0 -0.017194390296936 16.0
38.213385301 37.7975997924805 17.0
0 -0.00415956974029541 18.0
0 -0.0103393793106079 19.0
0 0.0218511819839478 20.0
0 0.00792300701141357 21.0
0 -0.0354281663894653 22.0
36.283005157 35.4257392883301 23.0
0 -0.0964380502700806 24.0
0 0.820586740970612 25.0
0 -0.0158010721206665 26.0
0 0.0062636137008667 27.0
0 0.0498346090316772 28.0
0 -0.0412517786026001 29.0
0 -0.0690251588821411 30.0
0 -0.00758635997772217 31.0
0 0.0275486707687378 32.0
0 0.0904151201248169 33.0
0 -0.0204130411148071 34.0
0 -0.0384069681167603 35.0
0 0.116302609443665 36.0
0 0.00391566753387451 37.0
0 -0.0303095579147339 38.0
0 0.0786136388778687 39.0
0 0.0364755392074585 40.0
0 0.0180395841598511 41.0
0 0.000840544700622559 42.0
304.72779736 293.900390625 43.0
0 0.0114291906356812 44.0
0 0.110236287117004 45.0
0 -0.0347069501876831 46.0
0 -0.002799391746521 47.0
0 -0.0287119150161743 48.0
0 0.291032314300537 49.0
0.086192040543 0.0506540536880493 50.0
1.5608557562 0.922599971294403 51.0
0 -0.0534135103225708 52.0
0 -0.0124944448471069 53.0
0 0.0671552419662476 54.0
0 -0.0963324308395386 55.0
0 -0.0122860670089722 56.0
0 -0.0311280488967896 57.0
0 -0.0445297956466675 58.0
0 -0.058590292930603 59.0
0 0.03984534740448 60.0
0 0.0650078058242798 61.0
0 0.848669648170471 62.0
264.82383215 256.196594238281 0.0
0 -0.0553613901138306 1.0
0 0.0615705251693726 2.0
0 -0.0328911542892456 3.0
0 -0.0841027498245239 4.0
0 -0.0853556394577026 5.0
0 0.0441025495529175 6.0
0 0.00578582286834717 7.0
0 0.0121089220046997 8.0
0 -0.119381546974182 9.0
0 -0.0040661096572876 10.0
0 0.0222809314727783 11.0
0 -0.0207067728042603 12.0
70.282789219 69.7996673583984 13.0
10.808473721 27.6077537536621 14.0
0 -0.119575619697571 15.0
0 0.0671137571334839 16.0
42.127497532 41.7322425842285 17.0
0 0.0203040838241577 18.0
0 -0.00945365428924561 19.0
0 0.0437592267990112 20.0
0 0.0548228025436401 21.0
0 -0.00207006931304932 22.0
27.470765527 25.6113739013672 23.0
0 0.0392173528671265 24.0
0 0.584818184375763 25.0
0 0.00248801708221436 26.0
0 0.021895170211792 27.0
0 0.0536555051803589 28.0
0 -0.0656460523605347 29.0
0 0.124970555305481 30.0
0 -0.033328652381897 31.0
0 0.0127905607223511 32.0
0 -0.170934319496155 33.0
0 -0.136268734931946 34.0
0 0.0268739461898804 35.0
0 0.0190454721450806 36.0
0 -0.170328974723816 37.0
0 -0.0179113149642944 38.0
0 0.0938115119934082 39.0
0 0.103092074394226 40.0
0 -0.0271626710891724 41.0
0 -0.101909756660461 42.0
293.70366515 295.087097167969 43.0
0 0.0411680936813354 44.0
0 0.116019010543823 45.0
0 -0.00154149532318115 46.0
0 -0.0278176069259644 47.0
0 0.0414961576461792 48.0
0 0.29227203130722 49.0
0.096470687751 0.0394104719161987 50.0
2.1292153747 0.867745280265808 51.0
0 -0.0484679937362671 52.0
0 0.0413702726364136 53.0
0 0.0699371099472046 54.0
0 -0.12848436832428 55.0
0 0.0356451272964478 56.0
0 -0.0434027910232544 57.0
0 -0.0537725687026978 58.0
0 -0.0603638887405396 59.0
0 -0.0602642297744751 60.0
0 0.0779591798782349 61.0
0 0.861158967018127 62.0
240.99431336 256.263702392578 0.0
0 -0.0737406015396118 1.0
0 0.00580728054046631 2.0
0 0.00940001010894775 3.0
0 -0.0595604181289673 4.0
0 -0.0294040441513062 5.0
0 0.0192769765853882 6.0
0 -0.153672575950623 7.0
0 0.0343785285949707 8.0
0 -0.0190368890762329 9.0
0 0.0146169662475586 10.0
0 0.0289205312728882 11.0
0 -0.0665613412857056 12.0
74.845624911 73.383674621582 13.0
32.84370916 25.9000816345215 14.0
0 -0.0683609247207642 15.0
0 -0.0203555822372437 16.0
30.540146745 30.5600700378418 17.0
0 -0.00843894481658936 18.0
0 -0.0134905576705933 19.0
0 -0.00139915943145752 20.0
0 -0.0458685159683228 21.0
0 -0.018135666847229 22.0
38.431047004 38.6350746154785 23.0
0 -0.0335971117019653 24.0
0 0.528913676738739 25.0
0 0.0594815015792847 26.0
0 0.00442337989807129 27.0
0 0.0189574956893921 28.0
0 -0.0661658048629761 29.0
0 -0.0538417100906372 30.0
0 -0.0623418092727661 31.0
0 -0.0301476716995239 32.0
0 0.119513154029846 33.0
0 -0.0232678651809692 34.0
0 0.13021981716156 35.0
0 0.0966418981552124 36.0
0 -0.0359944105148315 37.0
0 -0.0283514261245728 38.0
0 0.10736095905304 39.0
0 0.108323931694031 40.0
0 -0.0574511289596558 41.0
0 -0.068952202796936 42.0
302.85766885 297.539367675781 43.0
0 -0.0124255418777466 44.0
0 0.105297565460205 45.0
0 0.00521981716156006 46.0
0 -0.0357047319412231 47.0
0 -0.0497232675552368 48.0
0 0.286354899406433 49.0
0.083547671646 0.0499759912490845 50.0
2.0818586082 0.887424767017365 51.0
0 -0.0531278848648071 52.0
0 -0.0242129564285278 53.0
0 0.0465148687362671 54.0
0 -0.0417875051498413 55.0
0 -0.0709563493728638 56.0
0 0.0233930349349976 57.0
0 -0.0631939172744751 58.0
0 0.00156033039093018 59.0
0 -0.0747054815292358 60.0
0 -0.00727689266204834 61.0
0 0.802203893661499 62.0
227.73235607 257.321960449219 0.0
0 -0.0982290506362915 1.0
0 -0.0720640420913696 2.0
0 0.00778853893280029 3.0
0 -0.0171045064926147 4.0
0 -0.0546458959579468 5.0
0 0.00558888912200928 6.0
0 0.0292245149612427 7.0
0 0.0431135892868042 8.0
0 -0.0196367502212524 9.0
0 -0.0382305383682251 10.0
0 0.0100666284561157 11.0
0 -0.109330773353577 12.0
61.260581122 60.8512001037598 13.0
33.386442718 35.4715003967285 14.0
0 -0.1024409532547 15.0
0 -0.0234752893447876 16.0
26.778130921 26.3180751800537 17.0
0 0.0584684610366821 18.0
0 0.0681113004684448 19.0
0 0.0110846757888794 20.0
0 -0.022173285484314 21.0
0 -0.0295835733413696 22.0
42.401683417 41.4421920776367 23.0
0 -0.124643445014954 24.0
0 0.759482204914093 25.0
0 0.0324691534042358 26.0
0 -0.00970125198364258 27.0
0 0.0680149793624878 28.0
0 -0.0941139459609985 29.0
0 0.00378024578094482 30.0
0 -0.0613032579421997 31.0
0 -0.00487387180328369 32.0
0 0.0488866567611694 33.0
0 -0.0028756856918335 34.0
0 0.0631905794143677 35.0
0 -0.0660308599472046 36.0
0 -0.114624381065369 37.0
0 0.0456706285476685 38.0
0 0.0469359159469604 39.0
0 0.0413445234298706 40.0
0 -0.0983208417892456 41.0
0 -0.0228782892227173 42.0
297.8409968 295.907928466797 43.0
0 -0.0931879281997681 44.0
0 -0.0121433734893799 45.0
0 0.0201107263565063 46.0
0 -0.0619275569915771 47.0
0 0.0769170522689819 48.0
0 0.268651068210602 49.0
0.085741025074 0.0372234582901001 50.0
1.8075028649 0.926099121570587 51.0
0 -0.0311795473098755 52.0
0 -0.018558144569397 53.0
0 0.00706803798675537 54.0
0 -0.0193458795547485 55.0
0 -0.0225149393081665 56.0
0 -0.0720080137252808 57.0
0 0.040998101234436 58.0
0 -0.0118976831436157 59.0
0 -0.0347086191177368 60.0
0 0.0728057622909546 61.0
0 0.979838788509369 62.0
252.02164271 256.077270507812 0.0
0 0.0367535352706909 1.0
0 -0.0607455968856812 2.0
0 0.014055609703064 3.0
0 -0.072008490562439 4.0
0 -0.0861479043960571 5.0
0 -0.0398193597793579 6.0
0 -0.086910605430603 7.0
0 -0.0407745838165283 8.0
0 0.0966485738754272 9.0
0 0.0159415006637573 10.0
0 -0.0296974182128906 11.0
0 -0.0281254053115845 12.0
53.911576808 53.6190071105957 13.0
46.392411857 44.9598770141602 14.0
0 -0.0600100755691528 15.0
0 -0.0305663347244263 16.0
39.476462652 38.757698059082 17.0
0 0.0371397733688354 18.0
0 0.0347491502761841 19.0
0 -0.0478836297988892 20.0
0 -0.0548332929611206 21.0
0 0.0708392858505249 22.0
30.240960458 29.1311683654785 23.0
0 0.0755132436752319 24.0
0 0.452335953712463 25.0
0 0.0920356512069702 26.0
0 0.0718346834182739 27.0
0 -0.0373440980911255 28.0
0 -0.0279380083084106 29.0
0 -0.0563130378723145 30.0
0 0.00397741794586182 31.0
0 0.0162338018417358 32.0
0 0.0770565271377563 33.0
0 -0.0447064638137817 34.0
0 -0.0369404554367065 35.0
0 0.126095652580261 36.0
0 -0.0416477918624878 37.0
0 0.00157392024993896 38.0
0 0.0220004320144653 39.0
0 0.109418272972107 40.0
0 -0.0196175575256348 41.0
0 0.0407747030258179 42.0
290.55793085 296.508483886719 43.0
0 -0.0624061822891235 44.0
0 -0.00666475296020508 45.0
0 0.00385963916778564 46.0
0 0.000472664833068848 47.0
0 0.0225785970687866 48.0
0 0.313626557588577 49.0
0.077159891457 0.0658434629440308 50.0
1.5222842832 0.90338408946991 51.0
0 -0.0772548913955688 52.0
0 -0.0355583429336548 53.0
0 -0.0195471048355103 54.0
0 -0.0357347726821899 55.0
0 0.0256541967391968 56.0
0 -0.0276533365249634 57.0
0 -0.0607782602310181 58.0
0 -0.0326176881790161 59.0
0 -0.0677675008773804 60.0
0 -0.0161508321762085 61.0
0 0.828456282615662 62.0
270.80122478 258.111022949219 0.0
0 -0.0445247888565063 1.0
0 0.0145996809005737 2.0
0 0.00927913188934326 3.0
0 -0.0498286485671997 4.0
0 -0.0015794038772583 5.0
0 0.00173699855804443 6.0
0 0.0581988096237183 7.0
0 0.0372122526168823 8.0
0 0.0180095434188843 9.0
0 0.000496745109558105 10.0
0 -0.0322362780570984 11.0
0 -0.144835114479065 12.0
59.501070188 58.6183052062988 13.0
48.97467817 38.4351425170898 14.0
0 4.32729721069336e-05 15.0
0 -0.0778883695602417 16.0
30.661504065 30.8931655883789 17.0
0 -0.0243974924087524 18.0
0 -0.0193139314651489 19.0
0 -0.0234726667404175 20.0
0 -0.00769126415252686 21.0
0 -0.00994646549224854 22.0
29.212725305 27.7146377563477 23.0
0 0.0166815519332886 24.0
0 0.706988155841827 25.0
0 -0.0103937387466431 26.0
0 0.00585424900054932 27.0
0 -0.00809776782989502 28.0
0 -0.0984255075454712 29.0
0 -0.0652238130569458 30.0
0 -0.0140546560287476 31.0
0 -0.0172172784805298 32.0
0 0.0333975553512573 33.0
0 -0.0381637811660767 34.0
0 -0.0567219257354736 35.0
0 -0.0521663427352905 36.0
0 -0.0265012979507446 37.0
0 -0.0146304368972778 38.0
0 0.137733817100525 39.0
0 0.0907644033432007 40.0
0 -0.0690220594406128 41.0
0 0.0408700704574585 42.0
292.32051973 296.754425048828 43.0
0 -0.0646501779556274 44.0
0 -0.00954395532608032 45.0
0 -0.010210394859314 46.0
0 -0.0581780672073364 47.0
0 0.0162109136581421 48.0
0 0.295319557189941 49.0
0.073333910532 0.0656954050064087 50.0
2.0356751961 0.859820365905762 51.0
0 -0.0679420232772827 52.0
0 -0.0579987764358521 53.0
0 0.0456696748733521 54.0
0 -0.0833891630172729 55.0
0 -0.00194919109344482 56.0
0 -0.0518194437026978 57.0
0 -0.0639275312423706 58.0
0 -0.0503441095352173 59.0
0 -0.0698355436325073 60.0
0 0.0334399938583374 61.0
0 0.868710577487946 62.0
231.87285439 252.825881958008 0.0
0 -0.0477176904678345 1.0
0 0.0319522619247437 2.0
0 -0.00151669979095459 3.0
0 -0.0209563970565796 4.0
0 0.0918229818344116 5.0
0 0.0685399770736694 6.0
0 -0.108452677726746 7.0
0 -0.0779938697814941 8.0
0 -0.0272518396377563 9.0
0 -0.0964628458023071 10.0
0 -0.0374355316162109 11.0
0 -0.0850225687026978 12.0
67.909543857 68.0704727172852 13.0
22.837573375 29.5157108306885 14.0
0 -0.173847317695618 15.0
0 0.0368221998214722 16.0
36.520247142 36.5898208618164 17.0
0 -0.00277459621429443 18.0
0 -0.00567710399627686 19.0
0 -0.0477396249771118 20.0
0 -0.0292977094650269 21.0
0 -0.0229612588882446 22.0
32.951719865 31.5099353790283 23.0
0 -0.057287335395813 24.0
0 0.622407853603363 25.0
0 -0.0264879465103149 26.0
0 0.118958115577698 27.0
0 0.0812300443649292 28.0
0 -0.0384947061538696 29.0
0 -0.115061283111572 30.0
0 -0.0533660650253296 31.0
0 -0.0616871118545532 32.0
0 0.0772372484207153 33.0
0 -0.0168088674545288 34.0
0 -0.00217461585998535 35.0
0 0.0273400545120239 36.0
0 -0.111093640327454 37.0
0 -0.00669276714324951 38.0
0 -0.0251914262771606 39.0
0 -0.116103291511536 40.0
0 0.0122164487838745 41.0
0 0.0106652975082397 42.0
305.99982188 296.066741943359 43.0
0 0.0227521657943726 44.0
0 0.0486153364181519 45.0
0 -0.0240832567214966 46.0
0 -0.0559719800949097 47.0
0 0.0382179021835327 48.0
1.2446015156 0.365016222000122 49.0
0.076453735783 0.0731574296951294 50.0
1.8098506405 1.01492130756378 51.0
0 -0.0401972532272339 52.0
0 -0.0236862897872925 53.0
0 0.0657798051834106 54.0
0 -0.047995924949646 55.0
0 -0.059810996055603 56.0
0 -0.0155214071273804 57.0
0 -0.0519474744796753 58.0
0 -0.00803530216217041 59.0
0 0.23154604434967 60.0
0 0.0474265813827515 61.0
0 0.794238328933716 62.0
249.17938777 257.304565429688 0.0
0 0.00859630107879639 1.0
0 -0.0728603601455688 2.0
0 -0.00841081142425537 3.0
0 -0.0331305265426636 4.0
0 -0.0794645547866821 5.0
0 0.0225757360458374 6.0
0 -0.0465852022171021 7.0
0 -0.0122725367546082 8.0
0 -0.0576416254043579 9.0
0 0.0178869962692261 10.0
0 -0.015226423740387 11.0
0 -0.0442339181900024 12.0
61.093667433 59.4773178100586 13.0
37.050920359 37.6451148986816 14.0
0 -0.0687776803970337 15.0
0 0.00669896602630615 16.0
28.45714564 28.2798633575439 17.0
0 -0.02283775806427 18.0
0 -0.0350512266159058 19.0
0 -0.0220872163772583 20.0
0 0.00445568561553955 21.0
0 -0.0527302026748657 22.0
34.826109181 34.2887115478516 23.0
0 -0.0741978883743286 24.0
0 0.700815737247467 25.0
0 -0.00108587741851807 26.0
0 0.0143555402755737 27.0
0 -0.0150490999221802 28.0
0 0.00808203220367432 29.0
0 0.210488200187683 30.0
0 -0.0361741781234741 31.0
0 -0.00137770175933838 32.0
0 0.0341833829879761 33.0
0 -0.0604540109634399 34.0
0 -0.0281678438186646 35.0
0 0.120105147361755 36.0
0 -0.0425070524215698 37.0
0 0.000947833061218262 38.0
0 0.0884073972702026 39.0
0 0.0473479032516479 40.0
0 -0.0996578931808472 41.0
0 -0.00576555728912354 42.0
288.60034068 297.586853027344 43.0
0 0.0106834173202515 44.0
0 0.0244289636611938 45.0
0 -0.0217753648757935 46.0
0 0.006874680519104 47.0
0 0.0401910543441772 48.0
0 0.244852244853973 49.0
0.073481875478 0.0563348531723022 50.0
2.1610603202 0.936021447181702 51.0
0 -0.095894455909729 52.0
0 -0.0401474237442017 53.0
0 0.00117886066436768 54.0
0 -0.104928374290466 55.0
0 0.00307667255401611 56.0
0 -0.063346266746521 57.0
0 -0.0316649675369263 58.0
0 -0.0117181539535522 59.0
0 -0.00477063655853271 60.0
0 0.0552648305892944 61.0
0 0.896082401275635 62.0
253.22285958 254.820129394531 0.0
0 -0.10614550113678 1.0
0 -0.114967465400696 2.0
0 -0.0527459383010864 3.0
0 -0.0226594209671021 4.0
0 0.0764070749282837 5.0
0 0.024800181388855 6.0
0 0.0387071371078491 7.0
0 -0.044054388999939 8.0
0 0.0892857313156128 9.0
0 -0.0228108167648315 10.0
0 -0.0178949236869812 11.0
0 -0.0216737985610962 12.0
68.006981795 67.7256393432617 13.0
20.445209871 28.1949367523193 14.0
0 -0.0103718042373657 15.0
0 -0.0477036237716675 16.0
39.691836515 38.8714790344238 17.0
0 -0.0277684926986694 18.0
0 -0.00511133670806885 19.0
0 -0.0511401891708374 20.0
0 0.022997260093689 21.0
0 -0.0563584566116333 22.0
37.841063406 37.8693428039551 23.0
0 0.0929807424545288 24.0
0 0.419811844825745 25.0
0 0.0450562238693237 26.0
0 -0.0124146938323975 27.0
0 0.0238646268844604 28.0
0 -0.0332566499710083 29.0
0 -0.074626088142395 30.0
0 -0.0184141397476196 31.0
0 -0.037481427192688 32.0
0 0.02033531665802 33.0
0 -0.0236712694168091 34.0
0 0.0178371667861938 35.0
0 -0.144757628440857 36.0
0 -0.031042218208313 37.0
0 -0.00211632251739502 38.0
0 0.11432158946991 39.0
0 0.0864514112472534 40.0
0 -0.0121704339981079 41.0
0 -0.0193582773208618 42.0
307.09163169 296.986297607422 43.0
0 -0.0103336572647095 44.0
0 0.0114505290985107 45.0
0 -0.0288642644882202 46.0
0 -0.0463205575942993 47.0
0 0.0274180173873901 48.0
0 0.353705883026123 49.0
0.071420927419 0.0551742315292358 50.0
1.836011505 0.854140281677246 51.0
0 -0.0677217245101929 52.0
0 0.0191410779953003 53.0
0 0.0893181562423706 54.0
0 -0.0632883310317993 55.0
0 -0.0585476160049438 56.0
0 0.000856280326843262 57.0
0 -0.0590668916702271 58.0
0 -0.00277793407440186 59.0
0 -0.0844370126724243 60.0
0 -0.0327974557876587 61.0
0 0.802868723869324 62.0
268.64522096 253.818344116211 0.0
0 -0.0259720087051392 1.0
0 -0.0887068510055542 2.0
0 -0.0644921064376831 3.0
0 0.0156873464584351 4.0
0 -0.0425664186477661 5.0
0 -0.0369163751602173 6.0
0 0.0108183622360229 7.0
0 -0.0185261368751526 8.0
0 0.0746310949325562 9.0
0 -0.0224364995956421 10.0
0 -0.0409005880355835 11.0
0 0.00207078456878662 12.0
66.672308843 65.2376022338867 13.0
45.606255479 34.6291465759277 14.0
0 0.0354365110397339 15.0
0 -0.00613534450531006 16.0
34.678730911 35.2213249206543 17.0
0 -0.0197576284408569 18.0
0 0.0111387968063354 19.0
0 0.00705921649932861 20.0
0 -0.0483008623123169 21.0
0 -0.0186268091201782 22.0
31.468319249 30.7980499267578 23.0
0 -0.0065385103225708 24.0
0 0.515490829944611 25.0
0 0.0606740713119507 26.0
0 0.0591381788253784 27.0
0 0.013271689414978 28.0
0 -0.08668053150177 29.0
0 -0.0690360069274902 30.0
0 -0.0309725999832153 31.0
0 0.00930678844451904 32.0
0 0.0683780908584595 33.0
0 -0.0104399919509888 34.0
0 -0.13013756275177 35.0
0 -0.0285552740097046 36.0
0 -0.0705322027206421 37.0
0 0.0538970232009888 38.0
0 0.0539733171463013 39.0
0 0.0367854833602905 40.0
0 -0.0218788385391235 41.0
0 -0.0461012125015259 42.0
294.55961386 297.245178222656 43.0
0 -0.00789797306060791 44.0
0 0.0617367029190063 45.0
0 0.00490224361419678 46.0
0 -0.0527197122573853 47.0
0 -0.0684319734573364 48.0
0 0.245744585990906 49.0
0.094246789642 0.180613279342651 50.0
1.5733816559 0.858872652053833 51.0
0 -0.0688506364822388 52.0
0 -0.0421558618545532 53.0
0 0.0702370405197144 54.0
0 -0.0926402807235718 55.0
0 -0.0791469812393188 56.0
0 0.00464236736297607 57.0
0 -0.0547410249710083 58.0
0 -0.0621803998947144 59.0
0 -0.0708879232406616 60.0
0 0.00689613819122314 61.0
0 0.88386058807373 62.0
266.78658714 256.407012939453 0.0
0 -0.147321343421936 1.0
0 0.0384331941604614 2.0
0 -0.0305391550064087 3.0
0 -0.0348914861679077 4.0
0 -0.21419632434845 5.0
0 -0.0928198099136353 6.0
0 -0.0741866827011108 7.0
0 -0.0611206293106079 8.0
0 0.0176771879196167 9.0
0 0.00892412662506104 10.0
0 -0.0409497022628784 11.0
0 -0.0654531717300415 12.0
71.1798157 70.3553619384766 13.0
8.9644641545 27.1945934295654 14.0
0 -0.0606905221939087 15.0
0 0.0173829793930054 16.0
39.520555938 38.2744178771973 17.0
0 -0.0213226079940796 18.0
0 -0.0100554227828979 19.0
0 -0.0233293771743774 20.0
0 0.00905883312225342 21.0
0 0.0498665571212769 22.0
38.212851516 37.2167816162109 23.0
0 -0.00105154514312744 24.0
0 0.467306852340698 25.0
0 0.0626648664474487 26.0
0 -0.00115072727203369 27.0
0 -0.0502935647964478 28.0
0 -0.0474096536636353 29.0
0 -0.032171368598938 30.0
0 -0.0535184144973755 31.0
0 -0.0574837923049927 32.0
0 0.0194011926651001 33.0
0 -0.0776993036270142 34.0
0 -0.0596683025360107 35.0
0 -0.086891770362854 36.0
0 -0.0223203897476196 37.0
0 -0.0255588293075562 38.0
0 -0.075595498085022 39.0
0 0.0590819120407104 40.0
0 -0.00249612331390381 41.0
0 0.0554946660995483 42.0
292.92229748 293.816467285156 43.0
0 0.0422900915145874 44.0
0 0.114021301269531 45.0
0 -0.0420669317245483 46.0
0 -0.0658072233200073 47.0
0 -0.00523388385772705 48.0
0 0.311318457126617 49.0
0.080275186023 0.0395361185073853 50.0
1.909064239 0.994367122650146 51.0
0 -0.0922585725784302 52.0
0 -0.0157262086868286 53.0
0 -0.0919424295425415 54.0
0 -0.0493060350418091 55.0
0 -0.0645824670791626 56.0
0 0.00768935680389404 57.0
0 -0.08021080493927 58.0
0 -0.0635098218917847 59.0
0 -0.0593675374984741 60.0
0 -0.0699878931045532 61.0
0 0.809049963951111 62.0
261.67836324 255.558013916016 0.0
0 -0.0578385591506958 1.0
0 -0.0429846048355103 2.0
0 0.00314557552337646 3.0
0 -0.0270503759384155 4.0
0 -0.117896199226379 5.0
0 -0.00452768802642822 6.0
0 -0.0465694665908813 7.0
0 -0.0789132118225098 8.0
0 0.110609412193298 9.0
0 0.00746047496795654 10.0
0 0.0856813490390778 11.0
0 -0.0770937204360962 12.0
70.987775366 71.851188659668 13.0
40.743132975 25.4312152862549 14.0
0 0.0114210844039917 15.0
0 -0.0584152936935425 16.0
26.909981327 26.2722225189209 17.0
0 0.036396861076355 18.0
0 0.0168489217758179 19.0
0 -0.00609362125396729 20.0
0 0.0249840021133423 21.0
0 -0.0223513841629028 22.0
40.142785367 39.3046760559082 23.0
0 0.0860885381698608 24.0
0 0.856810688972473 25.0
0 0.0254553556442261 26.0
0 0.0914407968521118 27.0
0 0.0325883626937866 28.0
0 -0.0349136590957642 29.0
0 -0.0831823348999023 30.0
0 -0.0670861005783081 31.0
0 -0.0177303552627563 32.0
0 0.0436385869979858 33.0
0 -0.0651136636734009 34.0
0 -0.0782631635665894 35.0
0 0.0744374990463257 36.0
0 0.0642160177230835 37.0
0 -0.0583699941635132 38.0
0 0.0601261854171753 39.0
0 0.00750863552093506 40.0
0 -0.0255957841873169 41.0
0 -0.0193639993667603 42.0
289.19542161 292.855804443359 43.0
0 -0.00354635715484619 44.0
0 0.078759491443634 45.0
0 0.0152775049209595 46.0
0 -0.0283774137496948 47.0
0 -0.00683748722076416 48.0
0 0.297360301017761 49.0
0.095731972915 0.0638402700424194 50.0
2.1552807351 0.94042980670929 51.0
0 -0.109512448310852 52.0
0 -0.00816833972930908 53.0
0 0.0187815427780151 54.0
0 -0.0523673295974731 55.0
0 -0.0344158411026001 56.0
0 -0.0206691026687622 57.0
0 -0.127689719200134 58.0
0 -0.0644983053207397 59.0
0 -0.00207936763763428 60.0
0 -0.052885890007019 61.0
0 0.869198799133301 62.0
228.7724457 256.743194580078 0.0
0 -0.0810791254043579 1.0
0 -0.137419819831848 2.0
0 -0.0363329648971558 3.0
0 -0.0666297674179077 4.0
0 0.0274680852890015 5.0
0 -0.0375372171401978 6.0
0 -0.0453554391860962 7.0
0 -0.068756103515625 8.0
0 0.0206722021102905 9.0
0 -0.030436635017395 10.0
0 -0.0343269109725952 11.0
0 0.00576817989349365 12.0
44.189536873 43.6828498840332 13.0
45.346981184 48.3338165283203 14.0
0 -0.0390454530715942 15.0
0 0.0137451887130737 16.0
33.125635603 32.9248542785645 17.0
0 -0.026549220085144 18.0
0 0.0041426420211792 19.0
0 0.00462329387664795 20.0
0 -0.0389112234115601 21.0
0 -0.0275846719741821 22.0
37.031766391 36.2803840637207 23.0
0 -0.00668442249298096 24.0
0 0.628003001213074 25.0
0 0.0222595930099487 26.0
0 0.035993754863739 27.0
0 0.141814589500427 28.0
0 -0.06413733959198 29.0
0 -0.0454022884368896 30.0
0 -0.0320380926132202 31.0
0 -0.00628769397735596 32.0
0 0.00365245342254639 33.0
0 -0.0188859701156616 34.0
0 -0.0696362257003784 35.0
0 0.0445486307144165 36.0
0 -0.0255144834518433 37.0
0 -0.0057753324508667 38.0
0 0.0202959775924683 39.0
0 0.0805162191390991 40.0
0 -0.0517996549606323 41.0
0 -0.0770927667617798 42.0
283.51323095 293.533386230469 43.0
0 0.0268052816390991 44.0
0 0.062049388885498 45.0
0 -0.0513104200363159 46.0
0 -0.0178245306015015 47.0
0 0.0487321615219116 48.0
0 0.351726174354553 49.0
0.070220588432 0.0847011804580688 50.0
2.0563324054 0.895829558372498 51.0
0 -0.0232111215591431 52.0
0 -0.0206590890884399 53.0
0 0.117657542228699 54.0
0 -0.14243471622467 55.0
0 -0.0079423189163208 56.0
0 0.00947129726409912 57.0
0 -0.0404545068740845 58.0
0 0.0211321115493774 59.0
0 -0.0605727434158325 60.0
0 0.0828412771224976 61.0
0 0.882950246334076 62.0
280.53253657 255.699996948242 0.0
0 -0.0108400583267212 1.0
0 -0.100898385047913 2.0
0 -0.00326716899871826 3.0
0 -0.060853123664856 4.0
0 -0.12007200717926 5.0
0 0.0225013494491577 6.0
0 -0.123280644416809 7.0
0 -0.0683529376983643 8.0
0 -0.0292574167251587 9.0
0 -0.0234214067459106 10.0
0 -0.0281027555465698 11.0
0 -0.0476537942886353 12.0
73.231345313 71.189094543457 13.0
12.805478206 26.2144947052002 14.0
0 -0.0367799997329712 15.0
0 -0.0184265375137329 16.0
26.260972395 25.4594802856445 17.0
0 -0.0209885835647583 18.0
0 -0.00731098651885986 19.0
0 0.0441540479660034 20.0
0 0.00410807132720947 21.0
0 0.0368365049362183 22.0
39.075018611 38.4537315368652 23.0
0 0.024588942527771 24.0
0 0.558176577091217 25.0
0 0.0263770818710327 26.0
0 0.0758289098739624 27.0
0 0.0384453535079956 28.0
0 -0.0581213235855103 29.0
0 0.0935468673706055 30.0
0 -0.0236488580703735 31.0
0 -0.0352879762649536 32.0
0 -0.0131794214248657 33.0
0 -0.059418797492981 34.0
0 -0.0925834178924561 35.0
0 -0.153809428215027 36.0
0 -0.0552994012832642 37.0
0 0.0122703313827515 38.0
0 0.097978949546814 39.0
0 0.101447939872742 40.0
0 0.0807162523269653 41.0
0 -0.0991474390029907 42.0
289.02295542 294.197204589844 43.0
0 0.0193437337875366 44.0
0 0.0838611125946045 45.0
0 0.00134575366973877 46.0
0 -0.0523484945297241 47.0
0 -0.0299807786941528 48.0
0 0.268904685974121 49.0
0.090703303809 0.129966616630554 50.0
2.2094275485 0.926744520664215 51.0
0 -0.0920802354812622 52.0
0 -0.0395108461380005 53.0
0 0.0681668519973755 54.0
0 -0.029084324836731 55.0
0 -0.0461691617965698 56.0
0 -0.0256942510604858 57.0
0 -0.0410619974136353 58.0
0 -0.0435968637466431 59.0
0 -0.0491486787796021 60.0
0 0.0581432580947876 61.0
0 0.920712471008301 62.0
244.6339829 258.385040283203 0.0
0 -0.034253716468811 1.0
0 -0.0178908109664917 2.0
0 -0.0167244672775269 3.0
0 -0.057245135307312 4.0
0 0.0219072103500366 5.0
0 -0.0415235757827759 6.0
0 0.00915753841400146 7.0
0 0.0154174566268921 8.0
0 -0.0386134386062622 9.0
0 0.0216687917709351 10.0
0 0.0120992660522461 11.0
0 0.0232747793197632 12.0
71.37361797 71.8121871948242 13.0
20.437904387 27.4585609436035 14.0
0 -0.0690056085586548 15.0
0 -0.00239264965057373 16.0
32.822877275 32.2251358032227 17.0
0 0.0403622388839722 18.0
0 -0.00359022617340088 19.0
0 0.00233376026153564 20.0
0 -0.0324023962020874 21.0
0 -0.00370895862579346 22.0
27.43871654 26.8756160736084 23.0
0 0.0738929510116577 24.0
0 0.509900331497192 25.0
0 0.101940512657166 26.0
0 -0.0272876024246216 27.0
0 -0.0563284158706665 28.0
0 -0.0432835817337036 29.0
0 -0.0639535188674927 30.0
0 -0.0686897039413452 31.0
0 0.00968348979949951 32.0
0 0.0488210916519165 33.0
0 -0.0378223657608032 34.0
0 -0.0191363096237183 35.0
0 0.137279152870178 36.0
0 -0.0374423265457153 37.0
0 -0.00732481479644775 38.0
0 0.0338295698165894 39.0
0 0.00192534923553467 40.0
0 -0.0309915542602539 41.0
0 -0.0521961450576782 42.0
288.4585863 297.082214355469 43.0
0 -0.0556627511978149 44.0
0 0.0232445001602173 45.0
0 -0.0150121450424194 46.0
0 -0.0699392557144165 47.0
0 -0.00912249088287354 48.0
0 0.277894020080566 49.0
0.089233224846 0.0544453859329224 50.0
2.0497031 0.856030464172363 51.0
0 -0.112940907478333 52.0
0 -0.0737582445144653 53.0
0 0.0236917734146118 54.0
0 -0.103347659111023 55.0
0 0.0147954225540161 56.0
0 -0.0539218187332153 57.0
0 -0.0628608465194702 58.0
0 -0.0544494390487671 59.0
0 -0.0690275430679321 60.0
0 0.0512014627456665 61.0
0 0.830345511436462 62.0
242.51087869 257.357666015625 0.0
0 0.0196303129196167 1.0
0 0.0274547338485718 2.0
0 0.147915005683899 3.0
0 -0.0520263910293579 4.0
0 0.0127884149551392 5.0
0 0.0288852453231812 6.0
0 -0.0708974599838257 7.0
0 -0.0431312322616577 8.0
0 0.00670850276947021 9.0
0 0.000770926475524902 10.0
0 -0.0275630950927734 11.0
0 -0.136091828346252 12.0
52.080455426 50.3377456665039 13.0
52.775834398 41.4314498901367 14.0
0 -0.0237270593643188 15.0
0 -0.0462750196456909 16.0
24.857065148 24.3188037872314 17.0
0 0.0325409173965454 18.0
0 -0.0300401449203491 19.0
0 0.0129979848861694 20.0
0 0.0633853673934937 21.0
0 -0.101194500923157 22.0
43.119778152 43.9008750915527 23.0
0 -0.00843632221221924 24.0
0 0.470848560333252 25.0
0 0.0861829519271851 26.0
0 0.113149046897888 27.0
0 -0.0520111322402954 28.0
0 -0.0532232522964478 29.0
0 -0.0288437604904175 30.0
0 -0.0413581132888794 31.0
0 -0.00578892230987549 32.0
0 0.0521272420883179 33.0
0 -0.0181013345718384 34.0
0 -0.00985300540924072 35.0
0 0.175909399986267 36.0
0 -0.0128978490829468 37.0
0 -0.0872093439102173 38.0
0 -0.0606957674026489 39.0
0 -0.053561806678772 40.0
0 -0.076018214225769 41.0
0 -0.0870109796524048 42.0
292.22120951 297.863525390625 43.0
0 0.031762957572937 44.0
0 0.0371831655502319 45.0
0 -0.0287982225418091 46.0
0 -0.0497549772262573 47.0
0 -0.0394669771194458 48.0
0 0.317679554224014 49.0
0.075099622862 0.0506709814071655 50.0
2.4670316479 0.912439167499542 51.0
0 -0.0378900766372681 52.0
0 0.0116220712661743 53.0
0 0.0370088815689087 54.0
0 -0.0141111612319946 55.0
0 0.00851309299468994 56.0
0 0.0183151960372925 57.0
0 0.0887013673782349 58.0
0 -0.0352436304092407 59.0
0 0.00309336185455322 60.0
0 0.0530143976211548 61.0
0 0.945387125015259 62.0
239.32061051 254.105514526367 0.0
0 -0.0338865518569946 1.0
0 -0.0220705270767212 2.0
0 -0.063625693321228 3.0
0 -0.00939071178436279 4.0
0 0.0797351598739624 5.0
0 0.0653904676437378 6.0
0 -0.0480159521102905 7.0
0 -0.0405763387680054 8.0
0 0.076285719871521 9.0
0 0.0073312520980835 10.0
0 -0.0430166721343994 11.0
0 -0.0171667337417603 12.0
65.232969204 64.7889938354492 13.0
31.66158515 32.9624099731445 14.0
0 -0.0297032594680786 15.0
0 -0.0584613084793091 16.0
39.737265476 39.188648223877 17.0
0 0.00114524364471436 18.0
0 0.0282622575759888 19.0
0 0.00842487812042236 20.0
0 -0.01506507396698 21.0
0 -0.00122511386871338 22.0
36.526522588 36.0697898864746 23.0
0 0.0914639234542847 24.0
0 0.622633934020996 25.0
0 0.116473078727722 26.0
0 0.0784040689468384 27.0
0 0.019978404045105 28.0
0 -0.0604122877120972 29.0
0 0.0939353704452515 30.0
0 -0.0514758825302124 31.0
0 -0.00230181217193604 32.0
0 0.0778888463973999 33.0
0 -0.00236093997955322 34.0
0 -0.097112774848938 35.0
0 0.109515786170959 36.0
0 -0.00639355182647705 37.0
0 -0.00843465328216553 38.0
0 0.0467840433120728 39.0
0 0.0789755582809448 40.0
0 -0.032274603843689 41.0
0 0.0556586980819702 42.0
287.74965554 295.96630859375 43.0
0 0.00253713130950928 44.0
0 0.033627986907959 45.0
0 0.0105029344558716 46.0
0 0.0296897888183594 47.0
0 0.00961101055145264 48.0
0 0.267891883850098 49.0
0.087298901602 0.06824791431427 50.0
1.9383259302 1.01862776279449 51.0
0 -0.0898488759994507 52.0
0 -0.0661231279373169 53.0
0 0.0168000459671021 54.0
0 -0.060458779335022 55.0
0 -0.0662442445755005 56.0
0 0.0336896181106567 57.0
0 -0.0521739721298218 58.0
0 -0.0801266431808472 59.0
0 -0.0403388738632202 60.0
0 -0.0234361886978149 61.0
0 0.842656135559082 62.0
242.84022059 254.437301635742 0.0
0 -0.0484625101089478 1.0
0 -0.0320016145706177 2.0
0 -0.00328433513641357 3.0
0 -0.0640994310379028 4.0
0 -0.0257631540298462 5.0
0 -0.00239908695220947 6.0
0 -0.0150352716445923 7.0
0 -0.0257947444915771 8.0
0 -0.0408579111099243 9.0
0 -0.00953543186187744 10.0
0 -0.0423719882965088 11.0
0 0.0709651708602905 12.0
59.076364698 58.6776466369629 13.0
45.733434644 39.3345718383789 14.0
0 0.0289038419723511 15.0
0 -0.07691490650177 16.0
29.589494977 29.1640453338623 17.0
0 0.0375510454177856 18.0
0 -0.0181950330734253 19.0
0 0.0101760625839233 20.0
0 -0.0383011102676392 21.0
0 0.0303429365158081 22.0
38.333210999 37.6941337585449 23.0
0 -0.0746375322341919 24.0
0 1.38078713417053 25.0
0 0.0701507329940796 26.0
0 0.0539253950119019 27.0
0 -0.0219069719314575 28.0
0 -0.0446103811264038 29.0
0 -0.0360642671585083 30.0
0 -0.00237429141998291 31.0
0 -0.0463043451309204 32.0
0 0.0755854845046997 33.0
0 0.000362277030944824 34.0
0 0.0966874361038208 35.0
0 0.0660175085067749 36.0
0 -0.0927523374557495 37.0
0 0.0356622934341431 38.0
0 0.0117992162704468 39.0
0 -0.0536199808120728 40.0
0 -0.083224892616272 41.0
0 0.0236390829086304 42.0
298.16822197 295.498809814453 43.0
0 -0.0249201059341431 44.0
0 -0.0150212049484253 45.0
0 -0.0530353784561157 46.0
0 -0.000856757164001465 47.0
0 -0.0582882165908813 48.0
0 0.239092350006104 49.0
0.094011662225 0.0956375598907471 50.0
1.5220511172 0.917622148990631 51.0
0 -0.0574785470962524 52.0
0 -0.0816184282302856 53.0
0 -0.000816464424133301 54.0
0 -0.120657563209534 55.0
0 -0.106481909751892 56.0
0 0.0410760641098022 57.0
0 -0.0339020490646362 58.0
0 0.00273072719573975 59.0
0 0.0369924306869507 60.0
0 0.00119936466217041 61.0
0 0.780608892440796 62.0
248.52516911 247.887145996094 0.0
0 -0.0534263849258423 1.0
0 0.065332293510437 2.0
0 0.0230759382247925 3.0
0 -0.0595916509628296 4.0
0 -0.0844844579696655 5.0
0 0.0338176488876343 6.0
0 -0.0624848604202271 7.0
0 -0.0413633584976196 8.0
0 0.137718558311462 9.0
0 0.00639843940734863 10.0
0 0.0408076047897339 11.0
0 -0.0108171701431274 12.0
50.906049038 50.3712997436523 13.0
37.72052794 35.6860694885254 14.0
0 -0.139483571052551 15.0
0 0.0674451589584351 16.0
37.090920197 34.9743537902832 17.0
0 -0.00108683109283447 18.0
0 0.00997722148895264 19.0
0 -0.0415409803390503 20.0
0 -0.0498238801956177 21.0
0 0.00169551372528076 22.0
29.270662451 28.818380355835 23.0
0 0.0673912763595581 24.0
18.383009797 0.485285460948944 25.0
0 -0.00990664958953857 26.0
0 0.103498339653015 27.0
0 -0.0260287523269653 28.0
0 -0.0682274103164673 29.0
0 0.0202814340591431 30.0
0 -0.0647424459457397 31.0
0 0.00398910045623779 32.0
0 0.0280927419662476 33.0
0 -0.164263367652893 34.0
0 0.266113638877869 35.0
0 0.183205485343933 36.0
0 -0.0116623640060425 37.0
0 -0.0029679536819458 38.0
0 0.0370622873306274 39.0
0 0.0237201452255249 40.0
0 0.010596752166748 41.0
0 -0.0618826150894165 42.0
302.59218598 293.980133056641 43.0
0 0.0378736257553101 44.0
0 0.0643931031227112 45.0
0 -0.121347069740295 46.0
0 -0.00270569324493408 47.0
0 0.109460711479187 48.0
0 0.291167974472046 49.0
0.065175261791 0.0137766599655151 50.0
1.9167984978 0.896426498889923 51.0
0 -0.0584477186203003 52.0
0 0.0135473012924194 53.0
0 0.0224946737289429 54.0
0 -0.0635925531387329 55.0
0 -0.0126694440841675 56.0
0 0.0027230978012085 57.0
0 -0.0600272417068481 58.0
0 -0.0975703001022339 59.0
0 -0.0787512063980103 60.0
0 0.0624309778213501 61.0
0 0.860838651657104 62.0
241.76568572 257.172058105469 0.0
0 -0.0946933031082153 1.0
0 0.0450762510299683 2.0
0 0.0266507863998413 3.0
0 -0.0520795583724976 4.0
0 -0.0331350564956665 5.0
0 0.0381721258163452 6.0
0 -0.00136101245880127 7.0
0 -0.0259054899215698 8.0
0 0.0605729818344116 9.0
0 -0.0164065361022949 10.0
0 -0.0412056446075439 11.0
0 0.102785468101501 12.0
58.868076311 57.9478149414062 13.0
46.200559404 40.911376953125 14.0
0 -0.0124183893203735 15.0
0 -0.0452302694320679 16.0
38.372324709 38.249324798584 17.0
0 0.0362018346786499 18.0
0 0.00207984447479248 19.0
0 0.023307204246521 20.0
0 -0.0146571397781372 21.0
0 -0.0156751871109009 22.0
36.076071049 35.355884552002 23.0
0 0.0299962759017944 24.0
0 0.513521254062653 25.0
0 0.00070655345916748 26.0
0 0.0796047449111938 27.0
0 -0.00615370273590088 28.0
0 -0.0478140115737915 29.0
0 -0.06926429271698 30.0
0 -0.0144582986831665 31.0
0 -0.00823843479156494 32.0
0 -0.0673600435256958 33.0
0 -0.0104387998580933 34.0
0 -0.109683394432068 35.0
0 -0.000221133232116699 36.0
0 -0.0347877740859985 37.0
0 0.0412653684616089 38.0
0 0.0123745203018188 39.0
0 0.0929855108261108 40.0
0 -0.0201539993286133 41.0
0 0.0418723821640015 42.0
294.10897261 296.821685791016 43.0
0 -0.025551438331604 44.0
0 0.0274476408958435 45.0
0 -0.00568759441375732 46.0
0 -0.0246052742004395 47.0
0 0.0146611928939819 48.0
0 0.274307012557983 49.0
0.074706904832 0.0643752813339233 50.0
2.2516411669 0.998245358467102 51.0
0 -0.0453180074691772 52.0
0 -0.0505286455154419 53.0
0 0.0462840795516968 54.0
0 -0.0651057958602905 55.0
0 0.0326718091964722 56.0
0 -0.0448280572891235 57.0
0 -0.0480509996414185 58.0
0 -0.0593444108963013 59.0
0 -0.0319780111312866 60.0
0 0.00306546688079834 61.0
0 0.809829115867615 62.0
223.17712182 262.898162841797 0.0
0 -0.0612322092056274 1.0
0 -0.0183871984481812 2.0
0 -0.00433385372161865 3.0
0 0.0126174688339233 4.0
0 -0.0480948686599731 5.0
0 -0.135769486427307 6.0
0 -0.0407730340957642 7.0
0 0.0405178070068359 8.0
0 0.0680125951766968 9.0
0 -0.0102580785751343 10.0
0 0.0161190032958984 11.0
0 -0.110226273536682 12.0
69.239895581 68.7927856445312 13.0
27.325769563 28.3171634674072 14.0
0 0.0283793210983276 15.0
0 -0.031927227973938 16.0
39.012309105 38.1662559509277 17.0
0 -0.00561010837554932 18.0
0 -0.00430786609649658 19.0
0 0.0267502069473267 20.0
0 -0.0106860399246216 21.0
0 -0.00497639179229736 22.0
37.521861794 37.6850051879883 23.0
0 -0.108468174934387 24.0
0 0.551861763000488 25.0
0 0.0504273176193237 26.0
0 0.0356067419052124 27.0
0 -0.075605034828186 28.0
0 -0.0588542222976685 29.0
0 -0.102335929870605 30.0
0 -0.0273436307907104 31.0
0 -0.0749174356460571 32.0
0 -0.0745145082473755 33.0
0 0.0553559064865112 34.0
0 0.153689503669739 35.0
0 -0.0520726442337036 36.0
0 0.0737365484237671 37.0
0 -0.0199872255325317 38.0
0 0.145760655403137 39.0
0 0.110634207725525 40.0
0 -0.0845283269882202 41.0
0 -0.0284310579299927 42.0
301.03927732 293.567626953125 43.0
0 0.00294387340545654 44.0
0 0.142137885093689 45.0
0 -0.0171202421188354 46.0
0 -0.0360435247421265 47.0
0 -0.0553189516067505 48.0
0.63362981642 0.262672424316406 49.0
0.100625172 0.0908154249191284 50.0
2.0060420618 0.852123737335205 51.0
0 -0.0868006944656372 52.0
0 0.0483402013778687 53.0
0 -0.0122896432876587 54.0
0 -0.0121797323226929 55.0
0 -0.0232971906661987 56.0
0 -0.0216718912124634 57.0
0 -0.00796425342559814 58.0
0 0.0309792757034302 59.0
0 -0.00710475444793701 60.0
0 -0.0426503419876099 61.0
0 0.83789324760437 62.0
245.91799878 258.506561279297 0.0
0 -0.0120431184768677 1.0
0 -0.0829316377639771 2.0
0 0.000452637672424316 3.0
0 -0.0446635484695435 4.0
0 -0.0743318796157837 5.0
0 -0.064984917640686 6.0
0 -0.0641387701034546 7.0
0 0.00416207313537598 8.0
0 -0.0108476877212524 9.0
0 -0.101451277732849 10.0
0 0.00820982456207275 11.0
0 -0.0301805734634399 12.0
58.516824428 57.6555786132812 13.0
40.202772132 38.3212547302246 14.0
0 -0.189767956733704 15.0
0 0.000223994255065918 16.0
33.188625311 33.1608085632324 17.0
0 0.0191425085067749 18.0
0 -0.0284339189529419 19.0
0 0.0738440752029419 20.0
0 0.00229060649871826 21.0
0 -0.0203882455825806 22.0
38.565008501 37.9685859680176 23.0
0 0.0709184408187866 24.0
0 0.453361988067627 25.0
0 0.0240830183029175 26.0
0 -0.0230735540390015 27.0
0 0.00484931468963623 28.0
0 -0.0228573083877563 29.0
0 -0.0639849901199341 30.0
0 -0.0326515436172485 31.0
0 -0.000916123390197754 32.0
0 0.0324324369430542 33.0
0 -0.00745642185211182 34.0
0 0.0716096162796021 35.0
0 -0.117722392082214 36.0
0 -0.0506232976913452 37.0
0 0.0267459154129028 38.0
0 0.084173321723938 39.0
0 0.0376709699630737 40.0
0 -0.0309873819351196 41.0
0 -0.0439201593399048 42.0
292.17769438 297.725891113281 43.0
0 0.00839221477508545 44.0
0 0.0503702163696289 45.0
0 -0.048155665397644 46.0
0 -0.0469509363174438 47.0
0 -0.05707848072052 48.0
0 0.319350898265839 49.0
0.069654888426 0.0267504453659058 50.0
2.1826425875 0.990603625774384 51.0
0 -0.0539594888687134 52.0
0 -0.02357017993927 53.0
0 0.0333513021469116 54.0
0 -0.0597046613693237 55.0
0 -0.0706576108932495 56.0
0 0.00392234325408936 57.0
0 -0.0318049192428589 58.0
0 -0.0317062139511108 59.0
0 -0.023401141166687 60.0
0 0.0329605340957642 61.0
0 0.837388753890991 62.0
215.67279661 254.572219848633 0.0
0 -0.0510550737380981 1.0
0 0.00196278095245361 2.0
0 0.0226644277572632 3.0
0 -0.0228301286697388 4.0
0 0.0433608293533325 5.0
0 0.0242694616317749 6.0
0 -0.0879577398300171 7.0
0 -0.0422039031982422 8.0
0 0.0500749349594116 9.0
0 0.0165355205535889 10.0
0 0.000287830829620361 11.0
0 -0.106567978858948 12.0
75.495040757 74.5524368286133 13.0
39.319202175 21.2959232330322 14.0
0 -0.062477707862854 15.0
0 0.020525336265564 16.0
39.744508752 38.2457237243652 17.0
0 -0.0513259172439575 18.0
0 -0.0138324499130249 19.0
0 0.0186713933944702 20.0
0 -0.0177217721939087 21.0
0 -0.0458718538284302 22.0
40.030031598 39.6586265563965 23.0
0 0.0016406774520874 24.0
0 0.526849389076233 25.0
0 0.0459107160568237 26.0
0 0.0136892795562744 27.0
0 -0.0546828508377075 28.0
0 -0.0656450986862183 29.0
0 -0.0840393304824829 30.0
0 -0.0926359891891479 31.0
0 -0.0466920137405396 32.0
0 0.0259217023849487 33.0
0 -0.0393227338790894 34.0
0 -0.0938421487808228 35.0
0 0.0468360185623169 36.0
0 -0.047826886177063 37.0
0 -0.0430303812026978 38.0
0 0.0117496252059937 39.0
0 0.0543230772018433 40.0
0 -0.038105845451355 41.0
0 -0.0529676675796509 42.0
288.8494564 296.366638183594 43.0
0 0.0623406171798706 44.0
0 -0.087710976600647 45.0
0 -0.0286260843276978 46.0
0 0.0515478849411011 47.0
0 0.00669944286346436 48.0
0 0.308451741933823 49.0
0.10131861186 0.0300194025039673 50.0
1.8214781665 0.851389348506927 51.0
0 -0.0250035524368286 52.0
0 -0.0193551778793335 53.0
0 -0.00748884677886963 54.0
0 -0.0272878408432007 55.0
0 -0.0505789518356323 56.0
0 0.0211414098739624 57.0
0 -0.0929704904556274 58.0
0 0.0219863653182983 59.0
0 -0.0849579572677612 60.0
0 -0.0702120065689087 61.0
0 0.796349167823792 62.0
282.62822862 259.333343505859 0.0
0 -0.123684525489807 1.0
0 -0.0597134828567505 2.0
0 0.0323020219802856 3.0
0 -0.0652986764907837 4.0
0 -0.00603401660919189 5.0
0 0.0590537786483765 6.0
0 -0.079004168510437 7.0
0 -0.0163060426712036 8.0
0 -0.007179856300354 9.0
0 -0.0903133153915405 10.0
0 -0.0529531240463257 11.0
0 0.0682824850082397 12.0
49.397152635 48.2788925170898 13.0
34.934617835 34.6806831359863 14.0
0 -0.0392500162124634 15.0
0 0.00154197216033936 16.0
34.718311237 34.6911964416504 17.0
0 0.0213998556137085 18.0
0 0.0346382856369019 19.0
0 -0.0276490449905396 20.0
0 -0.0208355188369751 21.0
0 -0.0029531717300415 22.0
29.797624182 29.1686248779297 23.0
0 0.0260952711105347 24.0
0 0.56798654794693 25.0
0 0.0300787687301636 26.0
0 0.0398544073104858 27.0
0 0.0822328329086304 28.0
0 -0.0634015798568726 29.0
0 -0.0587677955627441 30.0
0 -0.0686761140823364 31.0
0 0.012560248374939 32.0
0 -0.0336767435073853 33.0
0 -0.0856910943984985 34.0
0 -0.0317182540893555 35.0
0 -0.149643063545227 36.0
0 -0.0449148416519165 37.0
0 -0.0269204378128052 38.0
0 -0.0170141458511353 39.0
0 -0.0463830232620239 40.0
0 0.0873831510543823 41.0
0 -0.0519520044326782 42.0
296.17525446 294.473968505859 43.0
0 0.0157703161239624 44.0
0 0.162662744522095 45.0
0 -0.00311553478240967 46.0
0 -0.0505942106246948 47.0
0 0.100067496299744 48.0
0 0.313183724880219 49.0
0.074654459631 0.0412031412124634 50.0
2.3276120766 0.943709135055542 51.0
0 -0.112225651741028 52.0
0 -0.0235000848770142 53.0
0 0.0315836668014526 54.0
0 -0.054146409034729 55.0
0 -0.0680524110794067 56.0
0 -0.0146480798721313 57.0
0 -0.0369089841842651 58.0
0 -0.0555318593978882 59.0
0 -0.00050055980682373 60.0
0 0.00737416744232178 61.0
0 0.806963205337524 62.0
248.11337624 250.50944519043 0.0
0 -0.0296279191970825 1.0
0 -0.0783206224441528 2.0
0 -0.0388633012771606 3.0
0 -0.0275241136550903 4.0
0 -0.0712205171585083 5.0
0 -0.0914970636367798 6.0
0 -0.048697829246521 7.0
0 -0.011461615562439 8.0
0 0.0106261968612671 9.0
0 -0.00689804553985596 10.0
0 0.00945347547531128 11.0
0 0.0491145849227905 12.0
62.600649148 61.4922790527344 13.0
39.057579714 34.0849189758301 14.0
0 0.000847220420837402 15.0
0 -0.0443688631057739 16.0
39.642478925 39.0552139282227 17.0
0 -0.0323852300643921 18.0
0 0.0305279493331909 19.0
0 0.0252710580825806 20.0
0 0.00839459896087646 21.0
0 -0.0323950052261353 22.0
29.294634714 29.037914276123 23.0
0 0.0635179281234741 24.0
0 0.484204411506653 25.0
0 0.0800718069076538 26.0
0 0.0156872868537903 27.0
0 -0.0280224084854126 28.0
0 -0.0495485067367554 29.0
0 -0.0603071451187134 30.0
0 -0.0296343564987183 31.0
0 -0.0244284868240356 32.0
0 0.0664759874343872 33.0
0 -0.0342262983322144 34.0
0 -0.0495487451553345 35.0
0 0.0697952508926392 36.0
0 -0.0711327791213989 37.0
0 0.0293530225753784 38.0
0 0.0333293676376343 39.0
0 -0.104449868202209 40.0
0 -0.0467904806137085 41.0
0 -0.00206911563873291 42.0
300.07997214 296.077301025391 43.0
0 -0.0608707666397095 44.0
0 -0.018632173538208 45.0
0 -0.00758445262908936 46.0
0 -0.00413548946380615 47.0
0 0.00515210628509521 48.0
1.7667891302 0.293838024139404 49.0
0.086356501254 0.0879019498825073 50.0
1.6979166454 0.937180459499359 51.0
0 -0.0146580934524536 52.0
0 -0.0230501890182495 53.0
0 -0.00732791423797607 54.0
0 -0.0601946115493774 55.0
0 0.0125261545181274 56.0
0 -0.0329655408859253 57.0
0 -0.034110426902771 58.0
0 -0.0832277536392212 59.0
0 -0.0395423173904419 60.0
0 0.0484799146652222 61.0
0 0.859512209892273 62.0
267.33503675 261.394317626953 0.0
0 -0.0970789194107056 1.0
0 0.010658860206604 2.0
0 -0.0324631929397583 3.0
0 -0.0010145902633667 4.0
0 -0.038798451423645 5.0
0 -0.0999718904495239 6.0
0 -0.0262104272842407 7.0
0 -0.0514271259307861 8.0
0 -0.0303713083267212 9.0
0 -0.00241768360137939 10.0
0 -0.0246928930282593 11.0
0 -0.231760621070862 12.0
57.528379983 57.9477233886719 13.0
45.357883924 39.132740020752 14.0
0 -0.0366445779800415 15.0
0 -0.000522971153259277 16.0
38.009591225 37.4628944396973 17.0
0 -0.00608289241790771 18.0
0 0.0115450620651245 19.0
0 0.0569154024124146 20.0
0 0.00787484645843506 21.0
0 -0.0292388200759888 22.0
33.413340572 33.485179901123 23.0
0 -0.0298482179641724 24.0
0 0.477870106697083 25.0
0 0.0280531644821167 26.0
0 0.109403729438782 27.0
0 -0.0637918710708618 28.0
0 -0.0572360754013062 29.0
0 -0.0495978593826294 30.0
0 -0.0482364892959595 31.0
0 0.0295037031173706 32.0
0 0.12977135181427 33.0
0 -0.0325056314468384 34.0
0 -0.0518602132797241 35.0
0 -0.115288376808167 36.0
0 -0.0131107568740845 37.0
0 -0.0091623067855835 38.0
0 -0.00506222248077393 39.0
0 0.0335859060287476 40.0
0 -0.072393536567688 41.0
0 -0.103963494300842 42.0
297.98101369 294.924926757812 43.0
0 0.0167211294174194 44.0
0 -0.0972734689712524 45.0
0 0.0198041200637817 46.0
0 -0.082044243812561 47.0
0 0.000686526298522949 48.0
0 0.286134272813797 49.0
0.098001075448 0.0705865621566772 50.0
2.0491327275 0.88796591758728 51.0
0 -0.06163489818573 52.0
0 -0.0559545755386353 53.0
0 0.0349327325820923 54.0
0 -0.116381049156189 55.0
0 0.0215808153152466 56.0
0 -0.0157414674758911 57.0
0 -0.0828579664230347 58.0
0 -0.0390268564224243 59.0
0 -0.0577307939529419 60.0
0 -0.0503782033920288 61.0
0 0.877510666847229 62.0
282.1134423 251.185684204102 0.0
0 0.000880122184753418 1.0
0 0.0233930349349976 2.0
0 -0.00646078586578369 3.0
0 -0.0445941686630249 4.0
0 -0.0434433221817017 5.0
0 -0.0184017419815063 6.0
0 -0.0607401132583618 7.0
0 0.0361686944961548 8.0
0 -0.00820696353912354 9.0
0 0.0383679866790771 10.0
0 -0.0111018419265747 11.0
0 -0.0734249353408813 12.0
56.673653393 56.9022789001465 13.0
41.178294141 39.0218811035156 14.0
0 0.0581473112106323 15.0
0 -0.0854045152664185 16.0
29.127740126 28.3539638519287 17.0
0 0.0209437608718872 18.0
0 -0.00504004955291748 19.0
0 -0.0161131620407104 20.0
0 -0.0175138711929321 21.0
0 -0.0413781404495239 22.0
35.140980611 34.8020210266113 23.0
0 -0.0833388566970825 24.0
0 1.24261569976807 25.0
0 0.0755981206893921 26.0
0 0.0591288805007935 27.0
0 0.00203979015350342 28.0
0 -0.0358327627182007 29.0
0 -0.0728806257247925 30.0
0 -0.011217474937439 31.0
0 0.0211185216903687 32.0
0 -0.0420244932174683 33.0
0 -0.0456463098526001 34.0
0 -0.000297665596008301 35.0
0 0.118412375450134 36.0
0 -0.0250157117843628 37.0
0 -0.0213617086410522 38.0
0 0.00754153728485107 39.0
0 0.0991290807723999 40.0
0 0.0338764190673828 41.0
0 -0.0559746026992798 42.0
288.47798426 296.848510742188 43.0
0 0.0331743955612183 44.0
0 -0.00445020198822021 45.0
0 -0.0243996381759644 46.0
0 -0.0647763013839722 47.0
0 0.000676989555358887 48.0
0 0.311356574296951 49.0
0.097042098806 0.0495468378067017 50.0
1.6554447004 0.970852613449097 51.0
0 -0.104456543922424 52.0
0 -0.0307759046554565 53.0
0 0.0376886129379272 54.0
0 -0.0468927621841431 55.0
0 -0.0632685422897339 56.0
0 -0.0386859178543091 57.0
0 -0.0501092672348022 58.0
0 -0.0742291212081909 59.0
0 -0.0192068815231323 60.0
0 -0.00112473964691162 61.0
0 0.795429110527039 62.0
240.78681227 254.162719726562 0.0
0 0.0158671140670776 1.0
0 0.0503641366958618 2.0
0 0.0126186609268188 3.0
0 -0.0537046194076538 4.0
0 0.0551356077194214 5.0
0 -0.0160418748855591 6.0
0 -0.0093158483505249 7.0
0 -0.0169305801391602 8.0
0 0.119614958763123 9.0
0 -0.0589138269424438 10.0
0 0.0348874926567078 11.0
0 0.0108283758163452 12.0
75.206888943 75.2652587890625 13.0
20.899782081 22.3044090270996 14.0
0 -0.0164662599563599 15.0
0 0.0131844282150269 16.0
37.470895907 36.8857116699219 17.0
0 0.0526818037033081 18.0
0 0.0151506662368774 19.0
0 0.0301090478897095 20.0
0 -0.0195671319961548 21.0
0 -0.0219637155532837 22.0
38.561854171 38.5662879943848 23.0
0 -0.0259138345718384 24.0
0 0.573257267475128 25.0
0 0.092612624168396 26.0
0 0.00346708297729492 27.0
0 0.0601109266281128 28.0
0 -0.0862311124801636 29.0
0 -0.0350691080093384 30.0
0 -0.0130418539047241 31.0
0 -0.000516533851623535 32.0
0 0.0135153532028198 33.0
0 -0.036528468132019 34.0
0 -0.0628982782363892 35.0
0 -0.0049978494644165 36.0
0 -0.0450867414474487 37.0
0 0.053907036781311 38.0
0 0.0828379392623901 39.0
0 -0.0256839990615845 40.0
0 -0.0314028263092041 41.0
0 -0.0507160425186157 42.0
289.14350512 296.713623046875 43.0
0 0.0852450132369995 44.0
0 -0.0274646282196045 45.0
0 -0.0336869955062866 46.0
0 -0.0730825662612915 47.0
0 0.056606650352478 48.0
0 0.370752483606339 49.0
0.086084604431 0.062041163444519 50.0
2.1870333365 0.918673038482666 51.0
0 -0.0518049001693726 52.0
0 -0.043028712272644 53.0
0 0.0303159952163696 54.0
0 -0.10021960735321 55.0
0 0.034828782081604 56.0
0 -0.0434409379959106 57.0
0 -0.0169020891189575 58.0
0 -0.0355147123336792 59.0
0 -0.0477176904678345 60.0
0 -0.0261498689651489 61.0
0 0.844200611114502 62.0
278.34523574 259.873992919922 0.0
0 -0.0481134653091431 1.0
0 -0.0311640501022339 2.0
0 -0.00650370121002197 3.0
0 -0.0303705930709839 4.0
0 -0.155341744422913 5.0
0 0.0944122076034546 6.0
0 -0.117377877235413 7.0
0 -0.0560752749443054 8.0
0 0.0177806615829468 9.0
0 0.0128084421157837 10.0
0 -0.0220642685890198 11.0
0 0.00236070156097412 12.0
74.10057323 72.589225769043 13.0
34.108813467 25.7431144714355 14.0
0 -0.0817409753799438 15.0
0 -0.0630899667739868 16.0
36.425239961 35.8437118530273 17.0
0 0.0449663400650024 18.0
0 0.0258220434188843 19.0
0 -0.0476778745651245 20.0
0 -0.00602924823760986 21.0
0 -0.00523865222930908 22.0
31.026102014 30.5237483978271 23.0
0 0.0902389287948608 24.0
0 0.460026443004608 25.0
0 0.0953620672225952 26.0
0 0.00700211524963379 27.0
0 0.0412791967391968 28.0
0 0.0277935266494751 29.0
0 -0.0489829778671265 30.0
0 -0.0104299783706665 31.0
0 0.0280579328536987 32.0
0 -0.0264760255813599 33.0
0 -0.00497472286224365 34.0
0 0.0542131662368774 35.0
0 -0.00313293933868408 36.0
0 -0.0389207601547241 37.0
0 -0.0264676809310913 38.0
0 0.0436142683029175 39.0
0 -0.040925145149231 40.0
0 -0.0938338041305542 41.0
0 -0.024905800819397 42.0
305.83049434 297.913909912109 43.0
0 0.0164240598678589 44.0
0 -0.0140793323516846 45.0
0 0.00120842456817627 46.0
0 -0.0255931615829468 47.0
0 -0.0808607339859009 48.0
0 0.257001161575317 49.0
0.091873957003 0.0910650491714478 50.0
1.8961682634 0.866866648197174 51.0
0 -0.0478593111038208 52.0
0 -0.0293627977371216 53.0
0 0.0422395467758179 54.0
0 -0.0863186120986938 55.0
0 -0.0140038728713989 56.0
0 -0.0123869180679321 57.0
0 -0.0540221929550171 58.0
0 0.021417498588562 59.0
0 0.0148316621780396 60.0
0 -0.0350791215896606 61.0
0 0.844735622406006 62.0
250.48217647 255.026077270508 0.0
0 -0.0743213891983032 1.0
0 -0.0880569219589233 2.0
0 0.00357973575592041 3.0
0 -0.0517135858535767 4.0
0 -0.174388766288757 5.0
0 -0.0228277444839478 6.0
0 -0.0187736749649048 7.0
0 0.00186604261398315 8.0
0 -0.0745645761489868 9.0
0 0.0452483892440796 10.0
0 -0.0142581462860107 11.0
0 -0.128127694129944 12.0
57.753628372 58.3626365661621 13.0
38.659660508 36.3303260803223 14.0
0 -0.0611025094985962 15.0
0 -0.0204826593399048 16.0
34.428022302 33.9701232910156 17.0
0 0.0523780584335327 18.0
0 0.0304006338119507 19.0
0 0.0220898389816284 20.0
0 -0.0117059946060181 21.0
0 -0.0634700059890747 22.0
41.165432125 41.0877647399902 23.0
0 -0.111209988594055 24.0
0 0.654368817806244 25.0
0 0.00702798366546631 26.0
0 0.0366118550300598 27.0
0 -0.0512124300003052 28.0
0 -0.0384405851364136 29.0
0 -0.065657377243042 30.0
0 -0.0270208120346069 31.0
0 0.0451444387435913 32.0
0 -0.118445038795471 33.0
0 -0.0054088830947876 34.0
0 -0.0170100927352905 35.0
0 0.0875822305679321 36.0
0 0.0775073766708374 37.0
0 -0.0241783857345581 38.0
0 -0.100139260292053 39.0
0 -0.0995179414749146 40.0
0 -0.0224295854568481 41.0
0 0.0093311071395874 42.0
284.03060757 296.709838867188 43.0
0 -0.0203768014907837 44.0
0 0.108253836631775 45.0
0 0.019041895866394 46.0
0 0.0034860372543335 47.0
0 0.0348483324050903 48.0
0 0.265531063079834 49.0
0.095986899865 0.0865167379379272 50.0
1.4737200266 0.84911048412323 51.0
0 -0.0541654825210571 52.0
0 0.014288067817688 53.0
0 0.0100988149642944 54.0
0 -0.0752760171890259 55.0
0 0.0333002805709839 56.0
0 -0.0320965051651001 57.0
0 -0.0673702955245972 58.0
0 -0.0969926118850708 59.0
0 0.0306328535079956 60.0
0 0.0586041212081909 61.0
0 0.888232350349426 62.0
271.68861281 252.250411987305 0.0
0 -0.0485903024673462 1.0
0 -0.00233757495880127 2.0
0 0.0256832838058472 3.0
0 -0.0726422071456909 4.0
0 0.00413167476654053 5.0
0 0.105765700340271 6.0
0 -0.0333701372146606 7.0
0 0.0452330112457275 8.0
0 0.0567601919174194 9.0
0 -0.0502550601959229 10.0
0 -0.0201963782310486 11.0
0 -0.157213807106018 12.0
59.477389131 58.6524543762207 13.0
48.418345172 38.6375312805176 14.0
0 0.0036700963973999 15.0
0 0.00225889682769775 16.0
32.315475183 32.591178894043 17.0
0 0.0115963220596313 18.0
0 -0.0363866090774536 19.0
0 -0.0395911931991577 20.0
0 -0.0206304788589478 21.0
0 0.0699838399887085 22.0
26.343315095 26.2857036590576 23.0
0 0.0212029218673706 24.0
0 0.603324353694916 25.0
0 0.102631449699402 26.0
0 0.0712422132492065 27.0
0 -0.0443259477615356 28.0
0 -0.0896304845809937 29.0
0 -0.106947064399719 30.0
0 -0.0490792989730835 31.0
0 -0.0935269594192505 32.0
0 0.0189343690872192 33.0
0 -0.0773061513900757 34.0
0 -0.0648247003555298 35.0
0 -0.036159873008728 36.0
0 -0.0176881551742554 37.0
0 -0.0429271459579468 38.0
0 0.0551868677139282 39.0
0 0.0348910093307495 40.0
0 -0.00701677799224854 41.0
0 0.0277389287948608 42.0
288.55103347 295.248962402344 43.0
0 -0.0572677850723267 44.0
0 0.00135123729705811 45.0
0 0.00901210308074951 46.0
0 -0.0633696317672729 47.0
0 0.0968259572982788 48.0
0 0.338711798191071 49.0
0.075545697364 0.086058497428894 50.0
2.0645924029 0.888276398181915 51.0
0 -0.00626838207244873 52.0
0 0.00175487995147705 53.0
0 0.0601264238357544 54.0
0 -0.0971354246139526 55.0
0 -0.0356289148330688 56.0
0 -0.00559008121490479 57.0
0 -0.0846518278121948 58.0
0 -0.010492205619812 59.0
0 0.027809739112854 60.0
0 -0.0765560865402222 61.0
0 0.807050943374634 62.0
273.58160435 256.157897949219 0.0
0 -0.0898815393447876 1.0
0 -0.0260392427444458 2.0
0 0.0234593152999878 3.0
0 -0.037627100944519 4.0
0 -0.113142609596252 5.0
0 0.0278695821762085 6.0
0 -0.0225669145584106 7.0
0 -0.0262222290039062 8.0
0 -0.0191713571548462 9.0
0 0.0242220163345337 10.0
0 0.0137633681297302 11.0
0 -0.0720211267471313 12.0
71.155582163 71.3925323486328 13.0
37.947552344 25.4142475128174 14.0
0 -0.0820232629776001 15.0
0 0.033079981803894 16.0
39.317887367 38.7568588256836 17.0
0 0.0366181135177612 18.0
0 0.022990345954895 19.0
0 0.0256744623184204 20.0
0 -0.0323473215103149 21.0
0 -0.0247098207473755 22.0
36.02562392 36.0799331665039 23.0
0 -0.0783253908157349 24.0
0 0.80095911026001 25.0
0 0.0486735105514526 26.0
0 -0.077156662940979 27.0
0 -0.0315724611282349 28.0
0 -0.0290743112564087 29.0
0 -0.0263433456420898 30.0
0 -0.0591651201248169 31.0
0 0.0280617475509644 32.0
0 -0.080222487449646 33.0
0 -0.0378445386886597 34.0
0 0.0769412517547607 35.0
0 -0.0768755674362183 36.0
0 -0.0250681638717651 37.0
0 0.0693389177322388 38.0
0 0.0444115400314331 39.0
0 0.00414741039276123 40.0
0 -0.187794327735901 41.0
0 -0.0502916574478149 42.0
294.12687127 294.341033935547 43.0
0 -0.0232635736465454 44.0
0 0.0284358263015747 45.0
0 0.00943315029144287 46.0
0 -0.0622237920761108 47.0
0 0.0534671545028687 48.0
0 0.286359786987305 49.0
0.076329561421 0.0979627370834351 50.0
1.9961822968 0.958032131195068 51.0
0 -0.0330883264541626 52.0
0 -0.0254362821578979 53.0
0 0.0193084478378296 54.0
0 -0.0839484930038452 55.0
0 -0.00779879093170166 56.0
0 -0.0550085306167603 57.0
0 -0.0564993619918823 58.0
0 -0.0132781267166138 59.0
0 -0.104154467582703 60.0
0 -0.00275695323944092 61.0
0 0.81218147277832 62.0
242.01835932 252.525573730469 0.0
0 -0.0278526544570923 1.0
0 -0.0402134656906128 2.0
0 0.208996176719666 3.0
0 -0.00965631008148193 4.0
0 0.0833755731582642 5.0
0 -0.0358835458755493 6.0
0 -0.0506633520126343 7.0
0 0.0317103862762451 8.0
0 0.0061410665512085 9.0
0 0.0271693468093872 10.0
0 0.00247669219970703 11.0
0 0.0222305059432983 12.0
66.657444029 65.862190246582 13.0
39.755328297 31.439733505249 14.0
0 -0.001961350440979 15.0
0 -0.0235289335250854 16.0
30.77732024 30.8146381378174 17.0
0 0.0287290811538696 18.0
0 -0.0118571519851685 19.0
0 0.0317560434341431 20.0
0 0.00456202030181885 21.0
0 -0.075940728187561 22.0
40.508005282 39.6775360107422 23.0
0 0.168213725090027 24.0
0 0.508806586265564 25.0
0 0.0264414548873901 26.0
0 0.0436620712280273 27.0
0 0.172099947929382 28.0
0 -0.064372181892395 29.0
0 0.0507262945175171 30.0
0 -0.269314408302307 31.0
0 -0.0603152513504028 32.0
0 -0.0112987756729126 33.0
0 -0.140088438987732 34.0
0 0.19182550907135 35.0
0 0.162380576133728 36.0
0 0.00330197811126709 37.0
0 0.113471388816833 38.0
0 0.0838216543197632 39.0
0 0.131924033164978 40.0
0 -0.0826340913772583 41.0
0 -0.0454088449478149 42.0
300.97383752 296.641754150391 43.0
0 0.0383151769638062 44.0
0 0.035547137260437 45.0
0 0.0100311040878296 46.0
0 -0.0551842451095581 47.0
0 0.0225175619125366 48.0
0 0.245071649551392 49.0
0.072223119636 0.0334769487380981 50.0
1.3742987298 0.988513708114624 51.0
0 -0.0488945245742798 52.0
0 -0.0418845415115356 53.0
0 -0.00450241565704346 54.0
0 -0.0734168291091919 55.0
0 -0.00490248203277588 56.0
0 -0.0270620584487915 57.0
0 -0.0418926477432251 58.0
0 -0.0326937437057495 59.0
0 0.147542595863342 60.0
0 -0.0273894071578979 61.0
0 0.860261797904968 62.0
239.98168467 251.561172485352 0.0
0 -0.051392674446106 1.0
0 -0.0557607412338257 2.0
0 -0.00690019130706787 3.0
0 -0.0237339735031128 4.0
0 -0.0879989862442017 5.0
0 -0.0292125940322876 6.0
0 -0.105912089347839 7.0
0 0.000481128692626953 8.0
0 -0.0422931909561157 9.0
0 0.0437859296798706 10.0
0 -0.00422775745391846 11.0
0 0.00399720668792725 12.0
50.039153767 49.3897895812988 13.0
36.427092006 37.1642417907715 14.0
0 -0.0200487375259399 15.0
0 -0.0776301622390747 16.0
33.525782324 33.8782844543457 17.0
0 -0.0221661329269409 18.0
0 -0.0486921072006226 19.0
0 0.0435539484024048 20.0
0 -0.0390952825546265 21.0
0 -0.0752726793289185 22.0
26.085415411 26.2363052368164 23.0
0 0.0119589567184448 24.0
0 0.562544763088226 25.0
0 0.0129920244216919 26.0
0 0.0316755771636963 27.0
0 0.0402415990829468 28.0
0 -0.0507022142410278 29.0
0 -0.101046562194824 30.0
0 -0.0389937162399292 31.0
0 0.0353277921676636 32.0
0 -0.0588256120681763 33.0
0 -0.0746976137161255 34.0
0 -0.107205510139465 35.0
0 -0.0516849756240845 36.0
0 -0.0351530313491821 37.0
0 0.0882705450057983 38.0
0 -0.00805723667144775 39.0
0 -0.020907998085022 40.0
0 -0.143640398979187 41.0
0 0.0793641805648804 42.0
277.49014027 298.851257324219 43.0
0 -0.00193846225738525 44.0
0 0.0730719566345215 45.0
0 0.0154701471328735 46.0
0 -0.0457509756088257 47.0
0 -0.0252794027328491 48.0
0.2701626643 0.226310610771179 49.0
0.10474411195 0.226103782653809 50.0
2.3354762464 0.936583936214447 51.0
0 -0.0380877256393433 52.0
0 -0.0486327409744263 53.0
0 0.0369036197662354 54.0
0 -0.0398350954055786 55.0
0 -0.0514472723007202 56.0
0 -0.00223982334136963 57.0
0 -0.0633946657180786 58.0
0 0.0144189596176147 59.0
0 -0.098080039024353 60.0
0 0.0244947671890259 61.0
0 0.907427310943604 62.0
215.52495939 259.798034667969 0.0
0 -0.0376921892166138 1.0
0 0.0540374517440796 2.0
0 -0.0345050096511841 3.0
0 -0.0449172258377075 4.0
0 0.0472592115402222 5.0
0 0.0191277265548706 6.0
0 -0.0284689664840698 7.0
0 0.00753104686737061 8.0
0 -0.0892606973648071 9.0
0 -0.026213526725769 10.0
0 -0.0384384393692017 11.0
0 -0.162141442298889 12.0
55.794266121 55.3749504089355 13.0
29.13528326 42.0010757446289 14.0
0 -0.035251259803772 15.0
0 -0.0663508176803589 16.0
39.255764081 40.0438270568848 17.0
0 0.0179594755172729 18.0
0 -0.00622379779815674 19.0
0 -0.00739395618438721 20.0
0 -0.000699639320373535 21.0
0 -0.0329686403274536 22.0
39.606053873 39.447868347168 23.0
0 -0.110701203346252 24.0
0 0.717161774635315 25.0
0 0.0184630155563354 26.0
0 0.0213580131530762 27.0
0 -0.0387865304946899 28.0
0 -0.0778111219406128 29.0
0 -0.0351427793502808 30.0
0 -0.0628589391708374 31.0
0 -0.00866663455963135 32.0
0 0.0534783601760864 33.0
0 -0.0135709047317505 34.0
0 -0.00173211097717285 35.0
0 -0.0759066343307495 36.0
0 -0.0108596086502075 37.0
0 -0.00102269649505615 38.0
0 0.0385738611221313 39.0
0 0.022154688835144 40.0
0 -0.0396175384521484 41.0
0 0.0364459753036499 42.0
288.45124051 295.869415283203 43.0
0 0.0118211507797241 44.0
0 0.0402870178222656 45.0
0 -0.0325068235397339 46.0
0 -0.0679513216018677 47.0
0 0.0174325704574585 48.0
0.47773281517 0.289173483848572 49.0
0.10042677391 0.0581506490707397 50.0
2.2089702451 1.01314032077789 51.0
0 -0.0456467866897583 52.0
0 -0.0168505907058716 53.0
0 0.0496367216110229 54.0
0 -0.0892068147659302 55.0
0 -0.0883501768112183 56.0
0 -0.000764012336730957 57.0
0 -0.0462690591812134 58.0
0 -0.0247727632522583 59.0
0 -0.035845160484314 60.0
0 0.0108939409255981 61.0
0 0.804094910621643 62.0
246.20674477 260.346527099609 0.0
0 -0.145823121070862 1.0
0 -0.123745799064636 2.0
0 0.0243173837661743 3.0
0 -0.0442506074905396 4.0
0 -0.11157763004303 5.0
0 0.00267159938812256 6.0
0 -0.0183354616165161 7.0
0 0.0435646772384644 8.0
0 0.0750030279159546 9.0
0 0.018213152885437 10.0
0 0.0421873331069946 11.0
0 -0.0265718698501587 12.0
76.606186672 76.7388458251953 13.0
21.163892638 20.8814849853516 14.0
0 -0.139285206794739 15.0
0 -0.00991714000701904 16.0
28.177117665 26.2448101043701 17.0
0 -0.0436230897903442 18.0
0 -0.0239688158035278 19.0
0 0.00374853610992432 20.0
0 -0.0204316377639771 21.0
0 -0.0136209726333618 22.0
29.834672084 28.4247303009033 23.0
0 0.0336860418319702 24.0
0 0.560397624969482 25.0
0 0.0735791921615601 26.0
0 0.13898503780365 27.0
0 -0.0559898614883423 28.0
0 -0.07966148853302 29.0
0 -0.0661079883575439 30.0
0 -0.0335444211959839 31.0
0 -0.00139772891998291 32.0
0 -0.00141394138336182 33.0
0 -0.101622462272644 34.0
0 0.0932220220565796 35.0
0 0.0135329961776733 36.0
0 -0.0182102918624878 37.0
0 -0.00960218906402588 38.0
0 0.0181797742843628 39.0
0 0.0852168798446655 40.0
0 -0.0346838235855103 41.0
0 0.0257076025009155 42.0
302.78527312 292.446044921875 43.0
0 -0.0113064050674438 44.0
0 0.0562905073165894 45.0
0 -0.0131202936172485 46.0
0 -0.0586448907852173 47.0
0 -0.0242410898208618 48.0
3.5005886028 0.303832113742828 49.0
0.093187626515 0.0678597688674927 50.0
2.0253616759 1.0106840133667 51.0
0 -0.0686951875686646 52.0
0 -0.0535835027694702 53.0
0 0.0346134901046753 54.0
0 -0.0380736589431763 55.0
0 -0.0119835138320923 56.0
0 -0.0586704015731812 57.0
0 0.0756014585494995 58.0
0 -0.00177061557769775 59.0
0 -0.00782310962677002 60.0
0 0.0701535940170288 61.0
0 0.86000919342041 62.0
249.97135326 260.877746582031 0.0
0 -0.170642971992493 1.0
0 0.0306652784347534 2.0
0 0.0491670370101929 3.0
0 -0.0658069849014282 4.0
0 -0.0635696649551392 5.0
0 -0.0382877588272095 6.0
0 -0.135430932044983 7.0
0 0.0141429305076599 8.0
0 0.0178960561752319 9.0
0 -0.0206466913223267 10.0
0 -0.0261143445968628 11.0
0 0.0801643133163452 12.0
68.542152303 67.2253494262695 13.0
32.960545583 25.8854923248291 14.0
0 -0.19143807888031 15.0
0 0.0955618619918823 16.0
39.452684336 38.0536766052246 17.0
0 0.00104773044586182 18.0
0 0.0201901197433472 19.0
0 0.0227361917495728 20.0
0 -0.0255547761917114 21.0
0 0.00948512554168701 22.0
29.283652282 28.7767333984375 23.0
0 -0.0208717584609985 24.0
0 0.563864171504974 25.0
0 -0.0100852251052856 26.0
0 0.0260027647018433 27.0
0 -0.0651527643203735 28.0
0 -0.0142537355422974 29.0
0 -0.0825146436691284 30.0
0 -0.0479635000228882 31.0
0 -0.00616466999053955 32.0
0 0.0347274541854858 33.0
0 -0.00631868839263916 34.0
0 -0.0368752479553223 35.0
0 -0.122413754463196 36.0
0 0.0171798467636108 37.0
0 -0.00467860698699951 38.0
0 -0.19527542591095 39.0
0 0.169684767723083 40.0
0 -0.0336513519287109 41.0
0 -0.0242973566055298 42.0
281.32166198 291.049865722656 43.0
0 0.0368541479110718 44.0
0 0.124637126922607 45.0
0 0.011644721031189 46.0
0 0.0379685163497925 47.0
0 0.0715063810348511 48.0
0 0.298094272613525 49.0
0.071041341186 0.0569921731948853 50.0
2.2134749445 0.975636661052704 51.0
0 -0.00708997249603271 52.0
0 -0.043423056602478 53.0
0 -0.0539716482162476 54.0
0 -0.145519852638245 55.0
0 -0.0564848184585571 56.0
0 0.0124629735946655 57.0
0 -0.0253545045852661 58.0
0 -0.0539158582687378 59.0
0 -0.0317012071609497 60.0
0 -0.0788153409957886 61.0
0 0.897909998893738 62.0
272.50304604 260.65478515625 0.0
0 -0.128618836402893 1.0
0 -0.153074145317078 2.0
0 -0.0715805292129517 3.0
0 -0.0452553033828735 4.0
0 -0.0549944639205933 5.0
0 -0.100596070289612 6.0
0 -0.10153067111969 7.0
0 -0.0882582664489746 8.0
0 0.0778592824935913 9.0
0 0.0666221380233765 10.0
0 0.00245273113250732 11.0
0 -0.0765787363052368 12.0
44.234207789 44.3325805664062 13.0
33.727151645 33.6307678222656 14.0
0 -0.122218251228333 15.0
0 -0.0592392683029175 16.0
35.080437188 34.3543930053711 17.0
0 -0.039270281791687 18.0
0 -0.0144332647323608 19.0
0 -0.0183111429214478 20.0
0 -0.00877487659454346 21.0
0 -0.0165368318557739 22.0
32.150843258 32.0653266906738 23.0
0 0.0312365293502808 24.0
0 0.503076195716858 25.0
0 0.125923037528992 26.0
0 0.0649217367172241 27.0
0 -0.0888744592666626 28.0
0 -0.0460764169692993 29.0
0 -0.0696402192115784 30.0
0 -0.00361955165863037 31.0
0 -0.0099865198135376 32.0
0 -0.00543463230133057 33.0
0 0.00937020778656006 34.0
0 -0.0204172134399414 35.0
0 -0.10228431224823 36.0
0 -0.0261884927749634 37.0
0 0.0417536497116089 38.0
0 -0.150523781776428 39.0
0 -0.117252469062805 40.0
0 -0.0397661924362183 41.0
0 -0.0468231439590454 42.0
300.78900995 292.747894287109 43.0
0 -0.00918686389923096 44.0
0 0.0281723737716675 45.0
0 -0.0630837678909302 46.0
0 0.0255435705184937 47.0
0 0.0244892835617065 48.0
0 0.238929450511932 49.0
0.064249910472 0.0824555158615112 50.0
2.2288691353 0.901019155979156 51.0
0 -0.00346195697784424 52.0
0 -0.0757509469985962 53.0
0 0.0848344564437866 54.0
0 -0.0747109651565552 55.0
0 0.0631872415542603 56.0
0 -0.0307677984237671 57.0
0 -0.057299017906189 58.0
0 -0.0157228708267212 59.0
0 0.0132907629013062 60.0
0 -0.0819078683853149 61.0
0 0.900568604469299 62.0
244.36375251 255.101852416992 0.0
0 0.0633648633956909 1.0
0 -0.0928715467453003 2.0
0 -0.0405229330062866 3.0
0 0.00470983982086182 4.0
0 -0.016213059425354 5.0
0 -0.0359688997268677 6.0
0 -0.0117501020431519 7.0
0 0.0889896154403687 8.0
0 -0.0458904504776001 9.0
0 0.0022052526473999 10.0
0 -0.0253528356552124 11.0
0 -0.116306900978088 12.0
59.109724165 59.4702186584473 13.0
38.727902611 35.0058250427246 14.0
0 -0.142483353614807 15.0
0 0.0123261213302612 16.0
32.281848932 32.1178016662598 17.0
0 -0.0226658582687378 18.0
0 -0.0374389886856079 19.0
0 0.0627630949020386 20.0
0 -0.0163208246231079 21.0
0 0.00760018825531006 22.0
30.154971234 29.7633781433105 23.0
0 -0.0602675676345825 24.0
0 0.705092668533325 25.0
0 -0.0439578294754028 26.0
0 0.00289022922515869 27.0
0 0.00873458385467529 28.0
0 -0.0323868989944458 29.0
0 -0.072466254234314 30.0
0 -0.0115100145339966 31.0
0 -0.0200845003128052 32.0
0 0.147822022438049 33.0
0 -0.0825594663619995 34.0
0 -0.0709444284439087 35.0
0 0.0904890298843384 36.0
0 -0.0345908403396606 37.0
0 -0.0277849435806274 38.0
0 0.0568171739578247 39.0
0 0.0800026655197144 40.0
0 -0.047710657119751 41.0
0 0.0296396017074585 42.0
294.8443707 297.283111572266 43.0
0 0.0265082120895386 44.0
0 -0.06878662109375 45.0
0 0.0104106664657593 46.0
0 -0.0934461355209351 47.0
0 -0.094025731086731 48.0
0 0.280803918838501 49.0
0.074048340663 0.0583668947219849 50.0
2.2253786567 0.914396405220032 51.0
0 -0.0765513181686401 52.0
0 -0.00903308391571045 53.0
0 -0.000433802604675293 54.0
0 -0.0932949781417847 55.0
0 -0.00681531429290771 56.0
0 -0.0424531698226929 57.0
0 -0.085208535194397 58.0
0 -0.0624712705612183 59.0
0 -0.0849291086196899 60.0
0 -0.0185784101486206 61.0
0 0.833675742149353 62.0
225.98772482 254.945846557617 0.0
0 -0.0398632287979126 1.0
0 -0.021180272102356 2.0
0 -0.00496494770050049 3.0
0 -0.0275696516036987 4.0
0 -0.0669211149215698 5.0
0 0.0158666372299194 6.0
0 -0.0682684183120728 7.0
0 0.021687388420105 8.0
0 0.00576627254486084 9.0
0 0.0023801326751709 10.0
0 0.0697492361068726 11.0
0 -0.00450193881988525 12.0
58.656289413 56.9871826171875 13.0
36.373793884 40.4690818786621 14.0
0 -0.0468984842300415 15.0
0 -0.00704061985015869 16.0
31.464186454 30.5950355529785 17.0
0 -0.0387312173843384 18.0
0 -0.025225043296814 19.0
0 0.0130430459976196 20.0
0 -0.056566596031189 21.0
0 -0.0564509630203247 22.0
36.488292164 35.6638259887695 23.0
0 -0.108843922615051 24.0
0 0.566650807857513 25.0
0 -0.00714576244354248 26.0
0 0.0796288251876831 27.0
0 -0.028546929359436 28.0
0 -0.0468038320541382 29.0
0 -0.0755956172943115 30.0
0 -0.0456639528274536 31.0
0 -0.0794459581375122 32.0
0 -0.0252515077590942 33.0
0 -0.0306309461593628 34.0
0 -0.0362603664398193 35.0
0 -0.0672773122787476 36.0
0 -0.0130287408828735 37.0
0 -0.0260974168777466 38.0
0 0.0027998685836792 39.0
0 -0.159640908241272 40.0
0 0.0420917272567749 41.0
0 -0.0281463861465454 42.0
271.9449345 296.158538818359 43.0
0 0.00399100780487061 44.0
0 0.0657765865325928 45.0
0 -0.00977885723114014 46.0
0 0.00160324573516846 47.0
0 0.0458406209945679 48.0
1.6328395511 0.307458937168121 49.0
0.063136153826 0.0929025411605835 50.0
1.8547471443 0.947473168373108 51.0
0 -0.110060095787048 52.0
0 -0.0720440149307251 53.0
0 -0.0439594984054565 54.0
0 -0.0246671438217163 55.0
0 0.00258791446685791 56.0
0 -0.0445142984390259 57.0
0 -0.045448899269104 58.0
0 -0.00171411037445068 59.0
0 -0.0147536993026733 60.0
0 0.055543065071106 61.0
0 0.83727753162384 62.0
259.02738298 252.945358276367 0.0
0 -0.0145894289016724 1.0
0 0.016166090965271 2.0
0 -0.00825870037078857 3.0
0 -0.0476347208023071 4.0
0 -0.0510510206222534 5.0
0 -0.0114618539810181 6.0
0 -0.0585383176803589 7.0
0 -0.0332950353622437 8.0
0 -0.0392671823501587 9.0
0 0.0180577039718628 10.0
0 -0.0551282167434692 11.0
0 0.117170691490173 12.0
54.086684197 52.8778762817383 13.0
36.648849385 35.5524635314941 14.0
0 -0.0671931505203247 15.0
0 -0.0211936235427856 16.0
30.509370584 30.262939453125 17.0
0 -0.0125056505203247 18.0
0 -0.0423649549484253 19.0
0 -0.0195802450180054 20.0
0 -0.0289541482925415 21.0
0 -0.000136971473693848 22.0
42.159636048 41.321834564209 23.0
0 -0.0333682298660278 24.0
0 0.545074880123138 25.0
0 0.0208977460861206 26.0
0 0.0883084535598755 27.0
0 -0.0162755250930786 28.0
0 -0.0730026960372925 29.0
0 -0.104655742645264 30.0
0 -0.0044475793838501 31.0
0 0.000168204307556152 32.0
0 0.099565863609314 33.0
0 -0.0569039583206177 34.0
0 -0.0598059892654419 35.0
0 0.0261582136154175 36.0
0 0.0208703279495239 37.0
0 0.0312801599502563 38.0
0 0.00990259647369385 39.0
0 0.154337286949158 40.0
0 -0.00565922260284424 41.0
0 0.0437072515487671 42.0
285.55075658 295.759979248047 43.0
0 -0.0725246667861938 44.0
0 0.0449714660644531 45.0
0 -0.0545727014541626 46.0
0 -0.0874916315078735 47.0
0 0.0364850759506226 48.0
0 0.271545469760895 49.0
0.065499378212 0.0747209787368774 50.0
1.8006695232 0.941533446311951 51.0
0 -0.0382834672927856 52.0
0 -0.0421537160873413 53.0
0 0.0107742547988892 54.0
0 -0.078581690788269 55.0
0 0.0211449861526489 56.0
0 0.000655055046081543 57.0
0 -0.0331810712814331 58.0
0 0.00110185146331787 59.0
0 -0.193206667900085 60.0
0 0.0569318532943726 61.0
0 0.821080684661865 62.0
287.83676607 255.097564697266 0.0
0 -0.0881887674331665 1.0
0 -0.105300307273865 2.0
0 -0.00225436687469482 3.0
0 -0.0680235624313354 4.0
0 0.0288466215133667 5.0
0 -0.033180832862854 6.0
0 0.0565890073776245 7.0
0 0.173575401306152 8.0
0 -0.232630372047424 9.0
0 -0.0304552316665649 10.0
0 -0.0331352949142456 11.0
0 -0.0629202127456665 12.0
60.337048269 59.7883796691895 13.0
24.870050508 33.7709693908691 14.0
0 -0.0299841165542603 15.0
0 -0.075884222984314 16.0
37.495261248 37.0767707824707 17.0
0 0.0260831117630005 18.0
0 0.0132433176040649 19.0
0 -0.0408357381820679 20.0
0 0.0217238664627075 21.0
0 -0.0290094614028931 22.0
33.381015837 32.7004737854004 23.0
0 0.0223764181137085 24.0
0 0.62187123298645 25.0
0 0.0466212034225464 26.0
0 0.0362011194229126 27.0
0 -0.0518347024917603 28.0
0 -0.0330215692520142 29.0
0 -0.00220286846160889 30.0
0 -0.0426074266433716 31.0
0 -0.00350677967071533 32.0
0 -0.163658142089844 33.0
0 -0.0145665407180786 34.0
0 -0.0190078020095825 35.0
0 0.130357384681702 36.0
0 0.0239568948745728 37.0
0 0.00439107418060303 38.0
0 0.00375449657440186 39.0
0 0.188747763633728 40.0
0 0.0958672761917114 41.0
0 0.0410779714584351 42.0
293.33065665 290.637054443359 43.0
0 -0.00250303745269775 44.0
0 0.180321931838989 45.0
0 0.00623548030853271 46.0
0 -0.0138524770736694 47.0
0 -0.110903859138489 48.0
0 0.256842851638794 49.0
0.10523408049 0.236224055290222 50.0
2.1242517802 0.91849946975708 51.0
0 -0.0676447153091431 52.0
0 -0.0586754083633423 53.0
0 0.122460007667542 54.0
0 -0.0733948945999146 55.0
0 -0.0668307542800903 56.0
0 -0.0385085344314575 57.0
0 -0.0603374242782593 58.0
0 -0.0514570474624634 59.0
0 0.0416256189346313 60.0
0 -0.0116318464279175 61.0
0 0.840526342391968 62.0
243.86735991 255.808944702148 0.0
0 -0.0343862771987915 1.0
0 0.0248483419418335 2.0
0 0.108512759208679 3.0
0 -0.0078502893447876 4.0
0 -0.107365727424622 5.0
0 0.0390933752059937 6.0
0 0.00691759586334229 7.0
0 -0.0219215154647827 8.0
0 0.0236104726791382 9.0
0 -0.00686633586883545 10.0
0 -0.0169990062713623 11.0
0 0.0771421194076538 12.0
67.127002544 66.299690246582 13.0
37.158914484 30.1019096374512 14.0
0 -0.0240417718887329 15.0
0 0.0015338659286499 16.0
33.329475029 33.2837104797363 17.0
0 0.0159121751785278 18.0
0 -0.0369390249252319 19.0
0 0.0400835275650024 20.0
0 0.0701922178268433 21.0
0 -0.00997507572174072 22.0
25.899205427 25.2893657684326 23.0
0 0.0680330991744995 24.0
0 0.469417333602905 25.0
0 0.102888464927673 26.0
0 -0.0161217451095581 27.0
0 -0.00427448749542236 28.0
0 0.00511372089385986 29.0
0 0.192661881446838 30.0
0 -0.0152040719985962 31.0
0 -0.0104495286941528 32.0
0 0.0912922620773315 33.0
0 -0.0855225324630737 34.0
0 0.0590305328369141 35.0
0 -0.00676310062408447 36.0
0 -0.0755168199539185 37.0
0 -0.0634373426437378 38.0
0 -0.0657488107681274 39.0
0 0.0907720327377319 40.0
0 -0.0309736728668213 41.0
0 0.0924257040023804 42.0
295.98959212 296.059356689453 43.0
0 0.0135663747787476 44.0
0 0.047881543636322 45.0
0 -0.0657967329025269 46.0
0 -0.0773235559463501 47.0
0 -0.0425482988357544 48.0
0 0.293608546257019 49.0
0.068121909443 0.105939149856567 50.0
1.4359158275 0.86475670337677 51.0
0 -0.0781909227371216 52.0
0 -0.0880647897720337 53.0
0 0.00964510440826416 54.0
0 -0.0626481771469116 55.0
0 -0.0729984045028687 56.0
0 0.00177514553070068 57.0
0 -0.0452238321304321 58.0
0 0.0252863168716431 59.0
0 0.16097104549408 60.0
0 -0.0796462297439575 61.0
0 0.824802756309509 62.0
204.27638936 256.823425292969 0.0
0 -0.0679184198379517 1.0
0 -0.0383509397506714 2.0
0 -0.0143364667892456 3.0
0 -0.0631130933761597 4.0
0 -0.0692943334579468 5.0
0 0.0387576818466187 6.0
0 -0.0243378877639771 7.0
0 0.0036848783493042 8.0
0 -0.0112968683242798 9.0
0 -0.00691425800323486 10.0
0 -0.0534439086914062 11.0
0 -0.0836244821548462 12.0
65.610447414 64.5538635253906 13.0
24.386461509 35.1973037719727 14.0
0 -0.011081337928772 15.0
0 -0.0136138200759888 16.0
34.923882534 35.1062278747559 17.0
0 0.00762689113616943 18.0
0 -0.0286208391189575 19.0
0 -0.0331598520278931 20.0
0 0.00367057323455811 21.0
0 -0.0328880548477173 22.0
40.262979718 40.5078926086426 23.0
0 -0.107463955879211 24.0
0 0.726552128791809 25.0
0 0.0624459981918335 26.0
0 -0.0401418209075928 27.0
0 -0.0835977792739868 28.0
0 -0.0170074701309204 29.0
0 -0.0739486217498779 30.0
0 -0.0227383375167847 31.0
0 -0.0166312456130981 32.0
0 0.0676120519638062 33.0
0 -0.0110665559768677 34.0
0 -0.000440716743469238 35.0
0 -0.0884557962417603 36.0
0 -0.0929423570632935 37.0
0 0.0323992967605591 38.0
0 0.0453430414199829 39.0
0 0.0816015005111694 40.0
0 -0.0226789712905884 41.0
0 0.0193876028060913 42.0
305.85007787 295.97314453125 43.0
0 0.0411914587020874 44.0
0 0.108671605587006 45.0
0 -0.00975930690765381 46.0
0 0.051702618598938 47.0
0 0.00275313854217529 48.0
1.1637784227 0.277348816394806 49.0
0.097492309361 0.0721551179885864 50.0
1.6658639051 0.852851867675781 51.0
0 -0.0428615808486938 52.0
0 -0.0681315660476685 53.0
0 0.0570718050003052 54.0
0 -0.0712686777114868 55.0
0 0.00775253772735596 56.0
0 -0.0132566690444946 57.0
0 -0.0795785188674927 58.0
0 -0.0599435567855835 59.0
0 -0.114887595176697 60.0
0 -0.0314177274703979 61.0
0 0.892667651176453 62.0
278.2407868 251.899276733398 0.0
0 -0.0548654794692993 1.0
0 -0.104794383049011 2.0
0 -0.0171111822128296 3.0
0 -0.0743561983108521 4.0
0 -0.003456711769104 5.0
0 -0.00220119953155518 6.0
0 0.0181573629379272 7.0
0 -0.00497639179229736 8.0
0 0.107765555381775 9.0
0 -0.0182336568832397 10.0
0 0.0313218235969543 11.0
0 -0.0409246683120728 12.0
72.807957719 71.8110733032227 13.0
25.519022516 23.958683013916 14.0
0 0.0455688238143921 15.0
0 0.00635635852813721 16.0
32.95020058 32.9165077209473 17.0
0 0.0194741487503052 18.0
0 -0.0263699293136597 19.0
0 0.0141750574111938 20.0
0 0.00157272815704346 21.0
0 -0.025976300239563 22.0
37.99715404 36.996395111084 23.0
0 -0.0394107103347778 24.0
0 0.54901134967804 25.0
0 0.0118204355239868 26.0
0 0.0378321409225464 27.0
0 -0.0218254327774048 28.0
0 -0.0616327524185181 29.0
0 -0.0858782529830933 30.0
0 0.00985562801361084 31.0
0 -0.0242348909378052 32.0
0 0.0131436586380005 33.0
0 -0.0608443021774292 34.0
0 -0.0594660043716431 35.0
0 0.0499674081802368 36.0
0 -0.013018012046814 37.0
0 -0.0280553102493286 38.0
0 0.0594068765640259 39.0
0 0.112737536430359 40.0
0 -0.0295190811157227 41.0
0 0.00473439693450928 42.0
302.16587076 295.114410400391 43.0
0 0.0187925100326538 44.0
0 0.0279163122177124 45.0
0 -0.0329545736312866 46.0
0 -0.0542289018630981 47.0
0 -0.00243246555328369 48.0
1.2473462301 0.327904760837555 49.0
0.078781832126 0.0367709398269653 50.0
2.0193618966 0.905791938304901 51.0
0 -0.0286847352981567 52.0
0 -0.0153971910476685 53.0
0 0.0270425081253052 54.0
0 -0.108466029167175 55.0
0 -0.0460604429244995 56.0
0 -0.0410808324813843 57.0
0 -0.0291837453842163 58.0
0 -0.0393034219741821 59.0
0 -0.0769819021224976 60.0
0 0.00424015522003174 61.0
0 0.815705299377441 62.0
280.18100927 256.379730224609 0.0
0 -0.00598680973052979 1.0
0 -0.0591264963150024 2.0
0 -0.039648175239563 3.0
0 -0.0363618135452271 4.0
0 -0.0143224000930786 5.0
0 0.0746163129806519 6.0
0 0.00438272953033447 7.0
0 -0.0383967161178589 8.0
0 -0.0355516672134399 9.0
0 0.000625252723693848 10.0
0 -0.00961518287658691 11.0
0 -0.103091835975647 12.0
64.034107254 62.7108306884766 13.0
27.453889419 32.7958183288574 14.0
0 -0.00083315372467041 15.0
0 -0.0565344095230103 16.0
31.922249331 31.639892578125 17.0
0 -0.0170222520828247 18.0
0 -0.0259193181991577 19.0
0 -0.0247052907943726 20.0
0 -0.0303899049758911 21.0
0 -0.102229714393616 22.0
31.277426049 31.1464557647705 23.0
0 0.0350559949874878 24.0
0 0.483800947666168 25.0
0 0.100191950798035 26.0
0 0.0207031965255737 27.0
0 0.00822103023529053 28.0
0 -0.0719965696334839 29.0
0 -0.0767742395401001 30.0
0 -0.0707944631576538 31.0
0 -0.018257737159729 32.0
0 0.0118604898452759 33.0
0 -0.0456308126449585 34.0
0 0.0245673656463623 35.0
0 -0.0127326250076294 36.0
0 -0.054254412651062 37.0
0 -0.043948769569397 38.0
0 0.0392028093338013 39.0
0 0.00344550609588623 40.0
0 -0.0915259122848511 41.0
0 0.0150035619735718 42.0
297.14844558 296.728607177734 43.0
0 0.0240319967269897 44.0
0 -0.0021291971206665 45.0
0 0.0148385763168335 46.0
0 -0.0188404321670532 47.0
0 0.0103529691696167 48.0
0 0.252377390861511 49.0
0.07284401556 0.0686885118484497 50.0
1.5912764728 1.00166118144989 51.0
0 -0.082353949546814 52.0
0 -0.0246855020523071 53.0
0 -0.0224272012710571 54.0
0 -0.0875853300094604 55.0
0 -0.00715076923370361 56.0
0 -0.0129753351211548 57.0
0 -0.0507045984268188 58.0
0 -0.0650190114974976 59.0
0 -0.0503724813461304 60.0
0 0.0126138925552368 61.0
0 0.850553512573242 62.0
238.99100309 259.807098388672 0.0
0 -0.0765148401260376 1.0
0 0.0590585470199585 2.0
0 -0.0293408632278442 3.0
0 -0.0617469549179077 4.0
0 -0.157857298851013 5.0
0 -0.00694620609283447 6.0
0 -0.0392481088638306 7.0
0 0.0915075540542603 8.0
0 -0.0237318277359009 9.0
0 -0.0699011087417603 10.0
0 -0.0615658760070801 11.0
0 0.0330747365951538 12.0
56.874574283 56.0599746704102 13.0
26.222699521 40.8927536010742 14.0
0 -0.0741573572158813 15.0
0 -0.0826307535171509 16.0
35.490898359 35.3100624084473 17.0
0 0.0167568922042847 18.0
0 0.0367585420608521 19.0
0 0.0145653486251831 20.0
0 -0.042883038520813 21.0
0 0.0570911169052124 22.0
27.382919413 26.8348426818848 23.0
0 -0.00970041751861572 24.0
0 0.546171009540558 25.0
0 0.0571752786636353 26.0
0 -0.0115964412689209 27.0
0 -0.0469533205032349 28.0
0 -0.0639976263046265 29.0
0 0.0550075769424438 30.0
0 -0.0343915224075317 31.0
0 -0.0195313692092896 32.0
0 -0.0238956212997437 33.0
0 -0.00659024715423584 34.0
0 -0.0141857862472534 35.0
0 -0.0857425928115845 36.0
0 -0.0185378789901733 37.0
0 0.0332447290420532 38.0
0 0.0855699777603149 39.0
0 0.119663596153259 40.0
0 -0.0387660264968872 41.0
0 0.00266778469085693 42.0
310.10594086 294.969696044922 43.0
0 -0.0334984064102173 44.0
0 -0.0138156414031982 45.0
0 -0.00329601764678955 46.0
0 -0.0108400583267212 47.0
0 0.110772967338562 48.0
0 0.453289151191711 49.0
0.073393748773 0.0149713754653931 50.0
2.4182322292 1.01260042190552 51.0
0 -0.0965467691421509 52.0
0 -0.0395523309707642 53.0
0 0.0128599405288696 54.0
0 -0.121191620826721 55.0
0 -0.0579649209976196 56.0
0 0.0158976316452026 57.0
0 -0.0426703691482544 58.0
0 -0.0146456956863403 59.0
0 -0.184839844703674 60.0
0 -0.0344716310501099 61.0
0 0.801191568374634 62.0
290.4737658 257.451507568359 0.0
0 0.000783801078796387 1.0
0 -0.102507710456848 2.0
0 -0.0256360769271851 3.0
0 -0.0311659574508667 4.0
0 -0.0435687303543091 5.0
0 0.0894359350204468 6.0
0 -0.0680199861526489 7.0
0 -0.0185031890869141 8.0
0 0.103427290916443 9.0
0 0.0145496129989624 10.0
0 -0.0244529247283936 11.0
0 -0.0512176752090454 12.0
74.47712597 74.5555953979492 13.0
31.018134641 21.275785446167 14.0
0 -0.00838911533355713 15.0
0 -0.0459574460983276 16.0
28.377817682 28.0962734222412 17.0
0 0.0230559110641479 18.0
0 0.0192188024520874 19.0
0 -0.00611984729766846 20.0
0 -0.0259417295455933 21.0
0 -0.0277100801467896 22.0
35.202012653 35.3483619689941 23.0
0 -0.122179627418518 24.0
0 0.849756717681885 25.0
0 0.11484706401825 26.0
0 0.115053534507751 27.0
0 -0.0317879915237427 28.0
0 -0.0544265508651733 29.0
0 -0.0816463232040405 30.0
0 -0.00402414798736572 31.0
0 0.0200036764144897 32.0
0 -0.00340187549591064 33.0
0 -0.0314024686813354 34.0
0 -0.112580418586731 35.0
0 -0.0352312326431274 36.0
0 -0.0221718549728394 37.0
0 -0.0012136697769165 38.0
0 -0.0292099714279175 39.0
0 0.118520140647888 40.0
0 -0.0665063858032227 41.0
0 -0.00209105014801025 42.0
289.39060655 295.433715820312 43.0
0 0.0208162069320679 44.0
0 0.062129020690918 45.0
0 -0.0447477102279663 46.0
0 -0.0440288782119751 47.0
0 -0.0101872682571411 48.0
0 0.290906518697739 49.0
0.079677720901 0.0799700021743774 50.0
1.6928243377 1.01649332046509 51.0
0 -0.0779539346694946 52.0
0 -0.0369764566421509 53.0
0 0.0262655019760132 54.0
0 -0.128301739692688 55.0
0 -0.0384658575057983 56.0
0 -0.0115228891372681 57.0
0 -0.04936683177948 58.0
0 -0.0379713773727417 59.0
0 -0.0928035974502563 60.0
0 -0.042364239692688 61.0
0 0.831300854682922 62.0
248.43958602 261.163940429688 0.0
0 -0.0383039712905884 1.0
0 -0.0555118322372437 2.0
0 -0.0228070020675659 3.0
0 -0.0280107259750366 4.0
0 -0.0308796167373657 5.0
0 -0.0547271966934204 6.0
0 -0.0747495889663696 7.0
0 0.0277197360992432 8.0
0 -0.0485130548477173 9.0
0 -0.0342036485671997 10.0
0 -0.045688271522522 11.0
0 0.0456594228744507 12.0
63.766245353 63.3759384155273 13.0
22.956768577 34.3327331542969 14.0
0 -0.0234171152114868 15.0
0 -0.0530005693435669 16.0
39.839276995 38.6862487792969 17.0
0 0.00608408451080322 18.0
0 -0.0215214490890503 19.0
0 0.0137654542922974 20.0
0 -0.0271025896072388 21.0
0 -0.0165449380874634 22.0
38.350925468 37.9360237121582 23.0
0 -0.0739575624465942 24.0
0 1.17808079719543 25.0
0 0.0412448644638062 26.0
0 0.107891201972961 27.0
0 -0.0345436334609985 28.0
0 -0.0444318056106567 29.0
0 -0.091614842414856 30.0
0 0.00442588329315186 31.0
0 -0.00625765323638916 32.0
0 0.0473014116287231 33.0
0 -0.0475302934646606 34.0
0 -0.0921789407730103 35.0
0 0.0288430452346802 36.0
0 -0.0176109075546265 37.0
0 0.0153313875198364 38.0
0 0.0130895376205444 39.0
0 0.00137841701507568 40.0
0 -0.0563943386077881 41.0
0 0.0203403234481812 42.0
292.8071353 294.601104736328 43.0
0 -0.0228582620620728 44.0
0 0.0592303276062012 45.0
0 0.00448572635650635 46.0
0 -0.0781522989273071 47.0
0 0.0764364004135132 48.0
0 0.282496362924576 49.0
0.085439234167 0.0604673624038696 50.0
1.7366201606 0.866601884365082 51.0
0 -0.0654646158218384 52.0
0 -0.0542446374893188 53.0
0 0.0610700845718384 54.0
0 -0.0983244180679321 55.0
0 -0.118039727210999 56.0
0 0.00663268566131592 57.0
0 -0.0809122323989868 58.0
0 -0.014828085899353 59.0
0 -0.00802648067474365 60.0
0 -0.0119127035140991 61.0
0 0.812178254127502 62.0
278.46626545 255.857925415039 0.0
0 0.0314663648605347 1.0
0 -0.00729382038116455 2.0
0 -0.0462592840194702 3.0
0 -0.0448085069656372 4.0
0 -0.0979663133621216 5.0
0 -0.0591398477554321 6.0
0 -0.130854725837708 7.0
0 -0.0478684902191162 8.0
0 -0.00533163547515869 9.0
0 -0.0481570959091187 10.0
0 0.0162906646728516 11.0
0 -0.196163773536682 12.0
51.008070864 49.8084831237793 13.0
42.164999155 41.3473815917969 14.0
0 -0.130436539649963 15.0
0 0.036641001701355 16.0
30.013560124 29.3569221496582 17.0
0 0.00169861316680908 18.0
0 -0.0250214338302612 19.0
0 0.0176340341567993 20.0
0 -0.0134435892105103 21.0
0 0.041953444480896 22.0
27.988556239 27.4730186462402 23.0
0 0.00691783428192139 24.0
0 0.611918091773987 25.0
0 0.0130125284194946 26.0
0 -0.0374047756195068 27.0
0 -0.0864578485488892 28.0
0 -0.0559166669845581 29.0
0 -0.0631288290023804 30.0
0 -0.0246354341506958 31.0
0 0.0272616147994995 32.0
0 0.0444256067276001 33.0
0 -0.0336626768112183 34.0
0 -0.0461002588272095 35.0
0 0.0827268362045288 36.0
0 -0.017119288444519 37.0
0 -0.0285850763320923 38.0
0 0.0529204607009888 39.0
0 0.00257575511932373 40.0
0 -0.0903154611587524 41.0
0 -0.0678936243057251 42.0
290.07483721 296.47900390625 43.0
0 0.0269168615341187 44.0
0 -0.00420141220092773 45.0
0 -0.00231349468231201 46.0
0 -0.0687576532363892 47.0
0 0.00864303112030029 48.0
2.7750411744 0.270717263221741 49.0
0.081625496609 0.0753493309020996 50.0
1.7007950898 0.901666283607483 51.0
0 -0.0893288850784302 52.0
0 -0.0813208818435669 53.0
0 0.0688730478286743 54.0
0 -0.0659502744674683 55.0
0 -0.00587666034698486 56.0
0 -0.0389429330825806 57.0
0 -0.0305434465408325 58.0
0 -0.0248392820358276 59.0
0 -0.0205510854721069 60.0
0 0.0559526681900024 61.0
0 0.86976820230484 62.0
211.85605503 260.482452392578 0.0
0 -0.182421326637268 1.0
0 -0.0238333940505981 2.0
0 0.00375235080718994 3.0
0 -0.0670346021652222 4.0
0 -0.158775806427002 5.0
0 0.0753759145736694 6.0
0 -0.0144034624099731 7.0
0 -0.0289331674575806 8.0
0 -0.00669634342193604 9.0
0 0.0142413377761841 10.0
0 0.0131715536117554 11.0
0 -0.079837441444397 12.0
55.314151002 54.7370643615723 13.0
38.009230018 39.2815742492676 14.0
0 -0.110040783882141 15.0
0 -0.0046764612197876 16.0
35.889405338 36.3391456604004 17.0
0 0.0194824934005737 18.0
0 -0.0232471227645874 19.0
0 -0.0158172845840454 20.0
0 0.0319143533706665 21.0
0 0.00515639781951904 22.0
43.183188086 42.6737251281738 23.0
0 -0.10783588886261 24.0
0 1.11399829387665 25.0
0 0.0595468282699585 26.0
0 -0.0277197360992432 27.0
0 -0.08039391040802 28.0
0 -0.0317221879959106 29.0
0 -0.0523210763931274 30.0
0 -0.167498707771301 31.0
0 -0.104652047157288 32.0
0 0.00723874568939209 33.0
0 0.0232404470443726 34.0
0 0.052960991859436 35.0
0 -0.118309140205383 36.0
0 -0.0186985731124878 37.0
0 0.0426641702651978 38.0
0 0.00227892398834229 39.0
0 -0.0493651628494263 40.0
0 -0.0194088220596313 41.0
0 0.0299557447433472 42.0
302.40578242 295.795837402344 43.0
0 0.0212754011154175 44.0
0 0.0323053598403931 45.0
0 -0.121507525444031 46.0
0 0.102090716362 47.0
0 -0.00758755207061768 48.0
0 0.317113250494003 49.0
0.1044434842 0.133916735649109 50.0
2.202523898 0.819109797477722 51.0
0 -0.0993391275405884 52.0
0 -0.100579857826233 53.0
0 0.0348478555679321 54.0
0 -0.0672746896743774 55.0
0 -0.0034099817276001 56.0
0 0.00882375240325928 57.0
0 0.0920428037643433 58.0
0 -0.0489329099655151 59.0
0 0.0257874727249146 60.0
0 -0.057054877281189 61.0
0 0.849045753479004 62.0
302.27147038 255.216278076172 0.0
0 -0.00266134738922119 1.0
0 -0.0268350839614868 2.0
0 0.0030134916305542 3.0
0 0.00176060199737549 4.0
0 -0.0290018320083618 5.0
0 0.056115984916687 6.0
0 -0.248526215553284 7.0
0 0.0507166385650635 8.0
0 0.146759390830994 9.0
0 -0.0751495361328125 10.0
0 0.0277811288833618 11.0
0 0.0591338872909546 12.0
56.885010396 56.2244415283203 13.0
48.455075711 41.1291313171387 14.0
0 -0.240366578102112 15.0
0 0.0566145181655884 16.0
31.571768759 30.1376895904541 17.0
0 0.0319386720657349 18.0
0 0.0171257257461548 19.0
0 -0.0278936624526978 20.0
0 -0.0423439741134644 21.0
0 0.0421804189682007 22.0
27.090642875 27.1837863922119 23.0
0 -0.0685349702835083 24.0
0 1.1546905040741 25.0
0 0.0862101316452026 26.0
0 0.106622099876404 27.0
0 0.0831707715988159 28.0
0 -0.112017273902893 29.0
0 -0.0476064682006836 30.0
0 -0.0154231786727905 31.0
0 -0.107085347175598 32.0
0 -0.0208667516708374 33.0
0 -0.0307835340499878 34.0
0 -0.0496896505355835 35.0
0 -0.160615801811218 36.0
0 -0.0460797548294067 37.0
0 -0.0630570650100708 38.0
0 0.0900238752365112 39.0
0 0.111574530601501 40.0
0 -0.00977873802185059 41.0
0 0.0175222158432007 42.0
304.213328 295.893615722656 43.0
0 0.0394443273544312 44.0
0 0.140425324440002 45.0
0 -0.00307905673980713 46.0
0 -0.0202435255050659 47.0
0 0.0157660245895386 48.0
1.415409114 0.314859628677368 49.0
0.096039305992 0.0539671182632446 50.0
1.743682165 0.957815587520599 51.0
0 -0.0763357877731323 52.0
0 -0.000537514686584473 53.0
0 0.00968420505523682 54.0
0 -0.0826162099838257 55.0
0 -0.166669249534607 56.0
0 0.0248607397079468 57.0
0 -0.0573283433914185 58.0
0 0.0058978796005249 59.0
0 -0.23259961605072 60.0
0 -0.0141838788986206 61.0
0 0.83514392375946 62.0
224.12577429 262.139556884766 0.0
0 0.00198209285736084 1.0
0 -0.0444742441177368 2.0
0 0.0353633165359497 3.0
0 -0.0482741594314575 4.0
0 0.0821360349655151 5.0
0 0.042228102684021 6.0
0 -0.0435663461685181 7.0
0 0.0849013328552246 8.0
0 0.0848010778427124 9.0
0 -0.0488346815109253 10.0
0 0.102183818817139 11.0
0 -0.0588003396987915 12.0
80.760436044 78.1164169311523 13.0
11.578269525 25.3558616638184 14.0
0 0.0106953382492065 15.0
0 -0.0493496656417847 16.0
25.819864654 26.2491703033447 17.0
0 0.0197740793228149 18.0
0 0.00279748439788818 19.0
0 -0.0149699449539185 20.0
0 -0.0380412340164185 21.0
0 -0.0863257646560669 22.0
43.05718695 42.2237358093262 23.0
0 -0.145617127418518 24.0
0 3.80010032653809 25.0
0 0.0559576749801636 26.0
0 -0.11941385269165 27.0
0 0.0358642339706421 28.0
0 -0.0438200235366821 29.0
0 -0.0937070846557617 30.0
0 -0.0446242094039917 31.0
0 0.0292443037033081 32.0
0 -0.0556234121322632 33.0
0 -0.0536631345748901 34.0
0 -0.00426995754241943 35.0
0 0.104600071907043 36.0
0 -0.0254982709884644 37.0
0 0.0467475652694702 38.0
0 -0.0280030965805054 39.0
0 -0.00411474704742432 40.0
0 -0.0373686552047729 41.0
0 -0.0489898920059204 42.0
282.29057351 296.459167480469 43.0
0 0.030176043510437 44.0
0 -0.0876607894897461 45.0
0 -0.00752937793731689 46.0
0 -0.0168129205703735 47.0
0 -0.0963784456253052 48.0
1.482510048 0.263559341430664 49.0
0.10706010674 0.199285387992859 50.0
1.8007778272 0.964559018611908 51.0
0 -0.0265108346939087 52.0
0 -0.00281369686126709 53.0
0 0.089913010597229 54.0
0 -0.0889757871627808 55.0
0 -0.0704184770584106 56.0
0 -0.0582391023635864 57.0
0 -0.0347260236740112 58.0
0 -0.0294421911239624 59.0
0 -0.00423133373260498 60.0
0 -0.020285964012146 61.0
0 0.768069386482239 62.0
285.68938743 250.625122070312 0.0
0 -0.0648132562637329 1.0
0 -0.0621830224990845 2.0
0 -0.0132144689559937 3.0
0 0.00613987445831299 4.0
0 -0.106474041938782 5.0
0 0.0367845296859741 6.0
0 0.053545355796814 7.0
0 -0.0276341438293457 8.0
0 0.0830720663070679 9.0
0 -0.0643684864044189 10.0
0 -0.0485938787460327 11.0
0 -0.0141445398330688 12.0
60.224962003 58.8148536682129 13.0
34.50476733 32.4419860839844 14.0
0 -0.088664174079895 15.0
0 0.00967657566070557 16.0
30.038763068 29.7824840545654 17.0
0 0.0373889207839966 18.0
0 -0.00345289707183838 19.0
0 0.0464903116226196 20.0
0 -0.0374442338943481 21.0
0 -0.0239125490188599 22.0
31.937172262 31.470422744751 23.0
0 0.0606240034103394 24.0
26.303015044 0.997235417366028 25.0
0 0.00703418254852295 26.0
0 0.117498517036438 27.0
0 -0.0594054460525513 28.0
0 -0.0390483140945435 29.0
0 -0.142650961875916 30.0
0 -0.0257734060287476 31.0
0 -0.0256792306900024 32.0
0 0.0740562677383423 33.0
0 -0.0181177854537964 34.0
0 -0.0736511945724487 35.0
0 -0.0186148881912231 36.0
0 -0.0422965288162231 37.0
0 -0.021844744682312 38.0
0 0.00530397891998291 39.0
0 0.106571555137634 40.0
0 -0.0608493089675903 41.0
0 0.0510114431381226 42.0
299.48295738 297.143127441406 43.0
0 0.0411566495895386 44.0
0 0.00634133815765381 45.0
0 -0.0437744855880737 46.0
0 -0.0799602270126343 47.0
0 0.0314968824386597 48.0
0 0.311696410179138 49.0
0.063905944587 0.0586286783218384 50.0
1.4339563034 0.922831296920776 51.0
0 -0.119576811790466 52.0
0 -0.0126863718032837 53.0
0 -0.000708699226379395 54.0
0 -0.0857213735580444 55.0
0 -0.00896799564361572 56.0
0 0.0212877988815308 57.0
0 -0.0089491605758667 58.0
0 -0.0166062116622925 59.0
0 -0.00408089160919189 60.0
0 -0.0300294160842896 61.0
0 0.816321611404419 62.0
243.47223828 230.529373168945 0.0
0 -0.0773245096206665 1.0
0 -0.048517107963562 2.0
0 0.00669753551483154 3.0
0 -0.021600604057312 4.0
0 -0.121259808540344 5.0
0 0.0712894201278687 6.0
0 -0.0150233507156372 7.0
0 -0.0260990858078003 8.0
0 0.146849036216736 9.0
0 -0.0696390867233276 10.0
0 0.119029998779297 11.0
0 0.0721067190170288 12.0
62.264105959 59.0320243835449 13.0
55.784385164 36.5168380737305 14.0
0 -0.0424610376358032 15.0
0 -0.0584694147109985 16.0
24.170314652 22.0125370025635 17.0
0 0.0241070985794067 18.0
0 0.00603258609771729 19.0
0 0.0128530263900757 20.0
0 -0.0250149965286255 21.0
0 0.00641381740570068 22.0
27.019698714 26.5341472625732 23.0
0 -0.0575944185256958 24.0
33.024345954 0.659399271011353 25.0
0 -0.0534893274307251 26.0
0 0.0941988229751587 27.0
0 0.0169452428817749 28.0
0 -0.0863550901412964 29.0
0 -0.084861159324646 30.0
0 -0.0698543787002563 31.0
0 0.0924586057662964 32.0
0 -0.00392377376556396 33.0
0 -0.0430082082748413 34.0
0 0.0141547918319702 35.0
0 0.103071570396423 36.0
0 -0.0894023180007935 37.0
0 -0.00719726085662842 38.0
0 0.108229041099548 39.0
0 0.171307921409607 40.0
0 -0.0372906923294067 41.0
0 -0.0210663080215454 42.0
291.10179427 293.816955566406 43.0
0 0.0031818151473999 44.0
0 0.0797575712203979 45.0
0 -0.0145455598831177 46.0
0 -0.0230063199996948 47.0
0 -0.0768004655838013 48.0
0 0.284429848194122 49.0
0.072210044653 0.265035271644592 50.0
2.0891500799 0.873659193515778 51.0
0 -0.118723034858704 52.0
0 -0.0887900590896606 53.0
0 0.0212122201919556 54.0
0 -0.0554543733596802 55.0
0 -0.0181847810745239 56.0
0 -0.0384894609451294 57.0
0 -0.0476778745651245 58.0
0 -0.0641177892684937 59.0
0 0.0668262243270874 60.0
0 0.0226062536239624 61.0
0 0.801490902900696 62.0
231.44172736 258.513092041016 0.0
0 -0.107283234596252 1.0
0 -0.0386601686477661 2.0
0 0.0348323583602905 3.0
0 -0.112599968910217 4.0
0 0.00741088390350342 5.0
0 0.00967872142791748 6.0
0 0.00947272777557373 7.0
0 0.0670292377471924 8.0
0 0.0342859029769897 9.0
0 -0.00957000255584717 10.0
0 0.100690960884094 11.0
0 -0.091621994972229 12.0
66.979039259 66.9278030395508 13.0
36.082329342 40.6854476928711 14.0
0 -0.160102486610413 15.0
0 -0.00493514537811279 16.0
36.355899886 36.0652084350586 17.0
0 0.11043393611908 18.0
0 0.0122002363204956 19.0
0 -0.0411585569381714 20.0
0 -0.0276352167129517 21.0
0 0.0290955305099487 22.0
40.479611254 39.643913269043 23.0
0 -0.116443753242493 24.0
0 0.585520803928375 25.0
0 0.00492846965789795 26.0
0 0.0867766141891479 27.0
0 -0.11676561832428 28.0
0 -0.0560721158981323 29.0
0 -0.0360883474349976 30.0
0 -0.0422574281692505 31.0
0 0.0407627820968628 32.0
0 0.0766361951828003 33.0
0 -0.0348714590072632 34.0
0 -0.110398411750793 35.0
0 0.0451813936233521 36.0
0 -0.0822576284408569 37.0
0 -0.0187104940414429 38.0
0 0.0258103609085083 39.0
0 0.0297564268112183 40.0
0 -0.0331732034683228 41.0
0 -0.0941892862319946 42.0
282.16731331 293.475067138672 43.0
0 0.0658220052719116 44.0
0 -0.0615890622138977 45.0
0 0.0136231184005737 46.0
0 -0.0313047170639038 47.0
0 -0.101860642433167 48.0
0 0.247553110122681 49.0
0.098510456341 0.0767145156860352 50.0
1.8216671832 0.950998842716217 51.0
0 -0.0329204797744751 52.0
0 -0.0567861795425415 53.0
0 0.0424126386642456 54.0
0 -0.0730830430984497 55.0
0 -0.0895270109176636 56.0
0 0.0120500326156616 57.0
0 -0.0292493104934692 58.0
0 -0.00392735004425049 59.0
0 -0.086800217628479 60.0
0 0.0432626008987427 61.0
0 0.844970226287842 62.0
256.04335226 260.512298583984 0.0
0 -0.0531126260757446 1.0
0 -0.0637563467025757 2.0
0 0.0177816152572632 3.0
0 0.0551072359085083 4.0
0 -0.000993847846984863 5.0
0 0.0839346647262573 6.0
0 -0.112469792366028 7.0
0 -0.0642929077148438 8.0
0 0.0503171682357788 9.0
0 0.0184061527252197 10.0
0 -0.0399283170700073 11.0
0 -0.00275957584381104 12.0
49.58600719 48.7064208984375 13.0
50.933527123 46.4687385559082 14.0
0 -0.0296789407730103 15.0
0 -0.0234271287918091 16.0
37.832984153 37.2553215026855 17.0
0 -0.0229164361953735 18.0
0 -0.0352085828781128 19.0
0 -0.0172892808914185 20.0
0 -0.0201801061630249 21.0
0 -0.0197755098342896 22.0
35.199212953 34.2139930725098 23.0
0 0.0178831815719604 24.0
0 0.510720074176788 25.0
0 0.0586999654769897 26.0
0 0.0122387409210205 27.0
0 -0.0609451532363892 28.0
0 -0.0596657991409302 29.0
0 -0.0554823875427246 30.0
0 -0.0316637754440308 31.0
0 -0.0383812189102173 32.0
0 0.0289477109909058 33.0
0 -0.0219799280166626 34.0
0 0.0312449932098389 35.0
0 -0.0688832998275757 36.0
0 -0.0415621995925903 37.0
0 0.00619804859161377 38.0
0 0.0276385545730591 39.0
0 0.0485771894454956 40.0
0 -0.0107107162475586 41.0
0 0.0460084676742554 42.0
285.94291385 296.026336669922 43.0
0 -0.0801006555557251 44.0
0 0.00455588102340698 45.0
0 -0.00525557994842529 46.0
0 -0.0013195276260376 47.0
0 -0.0463205575942993 48.0
0.1069576001 0.301766514778137 49.0
0.10204606399 0.0590204000473022 50.0
1.735372669 0.891682028770447 51.0
0 -0.0468839406967163 52.0
0 -0.0527368783950806 53.0
0 0.0162469148635864 54.0
0 -0.145825982093811 55.0
0 -0.0656634569168091 56.0
0 0.0550669431686401 57.0
0 0.0132957696914673 58.0
0 -0.0565603971481323 59.0
0 -0.103569865226746 60.0
0 0.0818077325820923 61.0
0 0.941280484199524 62.0
249.99431079 256.145904541016 0.0
0 -0.0787035226821899 1.0
0 -0.148711323738098 2.0
0 -0.00533521175384521 3.0
0 -0.0419462919235229 4.0
0 0.0414198637008667 5.0
0 0.0326656103134155 6.0
0 -0.133455872535706 7.0
0 -0.00664567947387695 8.0
0 0.0168269872665405 9.0
0 0.0247815847396851 10.0
0 -0.0508664846420288 11.0
0 -0.10955011844635 12.0
54.466281631 52.9103813171387 13.0
33.833278203 41.1482315063477 14.0
0 -0.0804325342178345 15.0
0 -0.00474226474761963 16.0
27.226103391 27.0314903259277 17.0
0 0.0653609037399292 18.0
0 0.0246454477310181 19.0
0 -0.0232232809066772 20.0
0 0.0048675537109375 21.0
0 0.0550307035446167 22.0
33.033023389 31.8336086273193 23.0
0 -0.0292578935623169 24.0
0 0.860289096832275 25.0
0 -0.0233303308486938 26.0
0 0.0331109762191772 27.0
0 -0.0212947130203247 28.0
0 -0.0529595613479614 29.0
0 -0.0373020172119141 30.0
0 -0.0575898885726929 31.0
0 -0.0178147554397583 32.0
0 0.0398420095443726 33.0
0 -0.0860494375228882 34.0
0 -0.0188083648681641 35.0
0 -0.113991379737854 36.0
0 -0.0169790983200073 37.0
0 0.0143905878067017 38.0
0 0.0440987348556519 39.0
0 0.0427626371383667 40.0
0 -0.020097017288208 41.0
0 0.00204694271087646 42.0
294.14701013 297.749908447266 43.0
0 -0.0361887216567993 44.0
0 0.0492539405822754 45.0
0 -0.0125769376754761 46.0
0 -0.0348623991012573 47.0
0 -0.0142608880996704 48.0
0 0.265724748373032 49.0
0.071906328711 0.0519646406173706 50.0
1.8575115052 0.873092770576477 51.0
0 -0.0655590295791626 52.0
0 -0.0323039293289185 53.0
0 -0.0159605741500854 54.0
0 -0.121509671211243 55.0
0 -0.0767136812210083 56.0
0 0.0315364599227905 57.0
0 -0.0649400949478149 58.0
0 -0.0129884481430054 59.0
0 0.00729906558990479 60.0
0 -0.0316227674484253 61.0
0 0.878618717193604 62.0
229.55700833 252.356582641602 0.0
0 -0.0370408296585083 1.0
0 0.00110518932342529 2.0
0 -0.0228191614151001 3.0
0 -0.0191177129745483 4.0
0 0.112236618995667 5.0
0 0.0482944250106812 6.0
0 -0.0789088010787964 7.0
0 0.0298483371734619 8.0
0 0.062156081199646 9.0
0 0.0188288688659668 10.0
0 -0.0228875875473022 11.0
0 0.100133299827576 12.0
73.270746549 73.229621887207 13.0
25.213872912 22.7916450500488 14.0
0 -0.0206328630447388 15.0
0 -0.0388160943984985 16.0
30.659179873 31.0741024017334 17.0
0 0.0219649076461792 18.0
0 -0.0301152467727661 19.0
0 0.00835621356964111 20.0
0 -0.0467661619186401 21.0
0 0.0137957334518433 22.0
34.007752525 33.520320892334 23.0
0 -0.0696274042129517 24.0
0 0.715881943702698 25.0
0 0.00297296047210693 26.0
0 -0.00242912769317627 27.0
0 0.0626734495162964 28.0
0 -0.0582045316696167 29.0
0 0.199043393135071 30.0
0 0.0115529298782349 31.0
0 -0.0447546243667603 32.0
0 0.0839365720748901 33.0
0 -0.116434931755066 34.0
0 -0.0393379926681519 35.0
0 0.033542275428772 36.0
0 -0.101905226707458 37.0
0 0.000618338584899902 38.0
0 0.0287016630172729 39.0
0 0.00974166393280029 40.0
0 -0.0646659135818481 41.0
0 -0.0122934579849243 42.0
297.53709612 295.562408447266 43.0
0 -0.0502811670303345 44.0
0 -0.000929713249206543 45.0
0 0.0170141458511353 46.0
0 -0.0300689935684204 47.0
0 -0.0104347467422485 48.0
0 0.301671743392944 49.0
0.076997497906 0.0505856275558472 50.0
1.7632970171 0.901454627513885 51.0
0 -0.0865355730056763 52.0
0 0.025046706199646 53.0
0 0.0338889360427856 54.0
0 -0.0540026426315308 55.0
0 -0.0356324911117554 56.0
0 -0.036507248878479 57.0
0 -0.0678015947341919 58.0
0 -0.00152051448822021 59.0
0 -0.0055614709854126 60.0
0 -0.0141931772232056 61.0
0 0.837347269058228 62.0
290.32108164 262.559173583984 0.0
0 -0.0515066385269165 1.0
0 -0.0432533025741577 2.0
0 0.0162512063980103 3.0
0 -0.0412129163742065 4.0
0 -0.0700007677078247 5.0
0 -0.0205408334732056 6.0
0 0.0545710325241089 7.0
0 -0.0842212438583374 8.0
0 -0.0394879579544067 9.0
0 0.00309550762176514 10.0
0 0.00647103786468506 11.0
0 0.0919593572616577 12.0
64.363623498 62.7545509338379 13.0
23.731238707 31.6962852478027 14.0
0 0.0725935697555542 15.0
0 0.0141714811325073 16.0
41.973632891 41.3168678283691 17.0
0 -0.0698968172073364 18.0
0 -0.0238784551620483 19.0
0 0.00694739818572998 20.0
0 -0.0322638750076294 21.0
0 0.0349606275558472 22.0
37.798567681 37.7457962036133 23.0
0 -0.0611754655838013 24.0
0 1.40815830230713 25.0
0 0.057613730430603 26.0
0 0.00502234697341919 27.0
0 0.026786208152771 28.0
0 -0.112894415855408 29.0
0 0.0686250925064087 30.0
0 -0.0113600492477417 31.0
0 -0.0116444826126099 32.0
0 0.0779784917831421 33.0
0 -0.0980125665664673 34.0
0 0.109073519706726 35.0
0 0.161599040031433 36.0
0 -0.0453082323074341 37.0
0 0.00600683689117432 38.0
0 0.0325189828872681 39.0
0 0.0879095792770386 40.0
0 -0.0991019010543823 41.0
0 -0.125323414802551 42.0
292.94971953 295.841949462891 43.0
0 -0.0128728151321411 44.0
0 -0.0136854648590088 45.0
0 -0.0291508436203003 46.0
0 -0.00402200222015381 47.0
0 -0.0331541299819946 48.0
0 0.2812539935112 49.0
0.10854584856 0.0538803339004517 50.0
2.2162483298 0.869841814041138 51.0
0 -0.0988370180130005 52.0
0 -0.00134003162384033 53.0
0 -0.00657474994659424 54.0
0 -0.0855494737625122 55.0
0 -0.0577267408370972 56.0
0 -0.057634711265564 57.0
0 -0.0997887849807739 58.0
0 -0.0625382661819458 59.0
0 -0.032572865486145 60.0
0 -0.067898154258728 61.0
0 0.866406440734863 62.0
270.70358841 251.712448120117 0.0
0 -0.0338054895401001 1.0
0 -0.0596615076065063 2.0
0 -0.0203570127487183 3.0
0 -0.0220755338668823 4.0
0 -0.0591732263565063 5.0
0 -0.160477757453918 6.0
0 -0.180890917778015 7.0
0 0.0166639089584351 8.0
0 -0.0547596216201782 9.0
0 0.00735557079315186 10.0
0 -0.0477100610733032 11.0
0 -0.0570360422134399 12.0
58.730379266 57.3951835632324 13.0
44.024796473 37.263786315918 14.0
0 -0.0409265756607056 15.0
0 0.00167453289031982 16.0
27.685896229 27.3398303985596 17.0
0 -0.00363695621490479 18.0
0 0.0166412591934204 19.0
0 -0.0335904359817505 20.0
0 -0.0556405782699585 21.0
0 0.0202668905258179 22.0
30.61702817 30.5848369598389 23.0
0 -0.0162492990493774 24.0
0 0.645628094673157 25.0
0 0.0665639638900757 26.0
0 0.0997840166091919 27.0
0 -0.0600935220718384 28.0
0 -0.0450254678726196 29.0
0 -0.0534912347793579 30.0
0 -0.0706042051315308 31.0
0 0.00845372676849365 32.0
0 0.0637763738632202 33.0
0 -0.0367342233657837 34.0
0 -0.0252491235733032 35.0
0 0.106812596321106 36.0
0 0.0111502408981323 37.0
0 -0.0403765439987183 38.0
0 0.060757040977478 39.0
0 0.0780681371688843 40.0
0 -0.0330849885940552 41.0
0 0.00485646724700928 42.0
290.83558994 296.197784423828 43.0
0 -0.0208059549331665 44.0
0 0.0112294554710388 45.0
0 -0.0146692991256714 46.0
0 -0.0218652486801147 47.0
0 -0.0468064546585083 48.0
0 0.284135580062866 49.0
0.098612944162 0.072156548500061 50.0
2.2175752334 0.979586720466614 51.0
0 -0.0839320421218872 52.0
0 -0.00173676013946533 53.0
0 0.012248158454895 54.0
0 -0.0686029195785522 55.0
0 -0.0750082731246948 56.0
0 -0.00720036029815674 57.0
0 -0.0612825155258179 58.0
0 -0.0587528944015503 59.0
0 -0.000136733055114746 60.0
0 0.0208417177200317 61.0
0 0.849435567855835 62.0
229.67722721 257.475433349609 0.0
0 -0.0722643136978149 1.0
0 -0.0623480081558228 2.0
0 -0.0951944589614868 3.0
0 0.0522416830062866 4.0
0 0.060505747795105 5.0
0 0.076708197593689 6.0
0 -0.0121541023254395 7.0
0 -0.0228623151779175 8.0
0 -0.0401598215103149 9.0
0 -0.0325080156326294 10.0
0 0.00116217136383057 11.0
0 0.0388330221176147 12.0
72.197980483 71.1004104614258 13.0
22.10056248 27.7902946472168 14.0
0 -0.0453792810440063 15.0
0 -0.0377825498580933 16.0
30.645359998 29.8187561035156 17.0
0 0.0282713174819946 18.0
0 0.0252090692520142 19.0
0 -0.0140031576156616 20.0
0 -0.00660693645477295 21.0
0 -0.0223528146743774 22.0
34.989811965 34.6175422668457 23.0
0 -0.14703094959259 24.0
0 0.707002639770508 25.0
0 0.0613263845443726 26.0
0 -0.0121468305587769 27.0
0 0.163230776786804 28.0
0 -0.0691806077957153 29.0
0 -0.091153621673584 30.0
0 -0.0108622312545776 31.0
0 0.013152003288269 32.0
0 0.0322927236557007 33.0
0 -0.0350903272628784 34.0
0 -0.0533000230789185 35.0
0 0.104922890663147 36.0
0 -0.050564169883728 37.0
0 -0.0320426225662231 38.0
0 0.0321866273880005 39.0
0 0.118270754814148 40.0
0 -0.0383857488632202 41.0
0 -0.0988670587539673 42.0
274.15371495 293.530426025391 43.0
0 0.0124839544296265 44.0
0 0.0703582763671875 45.0
0 -0.053061842918396 46.0
0 0.0232423543930054 47.0
0 -0.00429356098175049 48.0
0 0.340611338615417 49.0
0.059528125774 0.0178068876266479 50.0
2.3204037312 0.906486809253693 51.0
0 -0.00668346881866455 52.0
0 0.0300575494766235 53.0
0 0.0346225500106812 54.0
0 -0.0395597219467163 55.0
0 -0.00397872924804688 56.0
0 -0.0205157995223999 57.0
0 -0.0690635442733765 58.0
0 -0.0548018217086792 59.0
0 0.110800862312317 60.0
0 0.0289536714553833 61.0
0 0.910132646560669 62.0
284.55288592 256.015258789062 0.0
0 -0.0212494134902954 1.0
0 0.0714734792709351 2.0
0 -0.00216305255889893 3.0
0 -0.0907045602798462 4.0
0 -0.033882737159729 5.0
0 0.261155009269714 6.0
0 -0.104882836341858 7.0
0 0.00376921892166138 8.0
0 0.0714925527572632 9.0
0 -0.0396188497543335 10.0
0 -0.0264091491699219 11.0
0 0.0242046117782593 12.0
64.612632853 64.2227401733398 13.0
41.664207064 32.716438293457 14.0
0 -0.0431276559829712 15.0
0 -0.0548104047775269 16.0
33.14027959 33.3432846069336 17.0
0 0.0223101377487183 18.0
0 0.0084155797958374 19.0
0 0.0143100023269653 20.0
0 0.0483092069625854 21.0
0 -0.0757619142532349 22.0
27.384997931 27.3184967041016 23.0
0 -0.0437852144241333 24.0
2.4481295822 0.594678997993469 25.0
0 0.0427588224411011 26.0
0 0.0753387212753296 27.0
0 0.0218051671981812 28.0
0 -0.0579487085342407 29.0
0 -0.0805226564407349 30.0
0 -0.0648981332778931 31.0
0 -0.0512849092483521 32.0
0 0.0736905336380005 33.0
0 -0.0288487672805786 34.0
0 -0.0235744714736938 35.0
0 0.115412831306458 36.0
0 -0.0209568738937378 37.0
0 -0.0397211313247681 38.0
0 0.0330387353897095 39.0
0 0.137828230857849 40.0
0 -0.0807980298995972 41.0
0 -0.142564415931702 42.0
304.46962616 296.199523925781 43.0
0 0.0707143545150757 44.0
0 -0.029572606086731 45.0
0 -0.0330790281295776 46.0
0 -0.0484625101089478 47.0
0 0.124703764915466 48.0
0 0.42088520526886 49.0
0.097826748779 0.053735613822937 50.0
1.6617190146 0.985452771186829 51.0
0 -0.0620890855789185 52.0
0 -0.0645905733108521 53.0
0 0.0630794763565063 54.0
0 -0.116409659385681 55.0
0 0.00501024723052979 56.0
0 -0.0281902551651001 57.0
0 -0.0463219881057739 58.0
0 0.00982820987701416 59.0
0 0.102720141410828 60.0
0 -0.0300852060317993 61.0
0 0.903495192527771 62.0
254.6715707 254.890594482422 0.0
0 -0.0305048227310181 1.0
0 -0.0490509271621704 2.0
0 -0.0373526811599731 3.0
0 -0.0304809808731079 4.0
0 -0.0471364259719849 5.0
0 -0.0753742456436157 6.0
0 -0.0292612314224243 7.0
0 -0.0742282867431641 8.0
0 -0.044965386390686 9.0
0 -0.0120807886123657 10.0
0 -0.0248135328292847 11.0
0 0.0217736959457397 12.0
55.91474104 55.6118049621582 13.0
39.143796666 36.5044555664062 14.0
0 -0.0369125604629517 15.0
0 -0.0107487440109253 16.0
29.39783079 29.942590713501 17.0
0 -0.00843799114227295 18.0
0 0.0130482912063599 19.0
0 0.0173290967941284 20.0
0 -0.055916428565979 21.0
0 -0.0266572237014771 22.0
35.489715219 34.9048500061035 23.0
0 0.0102847814559937 24.0
0 0.552396535873413 25.0
0 0.049761176109314 26.0
0 0.00189846754074097 27.0
0 -0.0730000734329224 28.0
0 -0.0144752264022827 29.0
0 -0.0751256942749023 30.0
0 -0.0360935926437378 31.0
0 -0.0186706781387329 32.0
0 0.0344005823135376 33.0
0 -0.0259402990341187 34.0
0 -0.000810384750366211 35.0
0 -0.0199674367904663 36.0
0 -0.041155219078064 37.0
0 0.0105925798416138 38.0
0 2.9444694519043e-05 39.0
0 -0.0654550790786743 40.0
0 0.0157345533370972 41.0
0 0.00170457363128662 42.0
293.14026509 294.207977294922 43.0
0 -0.0251709222793579 44.0
0 0.0926251411437988 45.0
0 -0.0577970743179321 46.0
0 -0.0684880018234253 47.0
0 0.0778487920761108 48.0
0 0.372744381427765 49.0
0.098867996991 0.0435670614242554 50.0
1.7472478128 0.899189472198486 51.0
0 -0.09013831615448 52.0
0 -0.0309096574783325 53.0
0 -0.0215669870376587 54.0
0 -0.0275388956069946 55.0
0 0.0388766527175903 56.0
0 -0.0379611253738403 57.0
0 -0.0840789079666138 58.0
0 -0.108951210975647 59.0
0 -0.0708919763565063 60.0
0 0.0213004350662231 61.0
0 0.811290144920349 62.0
212.20393165 253.473648071289 0.0
0 0.0502785444259644 1.0
0 -0.0341449975967407 2.0
0 0.105651497840881 3.0
0 -0.00936758518218994 4.0
0 0.0241141319274902 5.0
0 -0.000939011573791504 6.0
0 -0.0943236351013184 7.0
0 -0.0325598120689392 8.0
0 0.128252387046814 9.0
0 8.16583633422852e-05 10.0
0 -0.0225437879562378 11.0
0 0.0110858678817749 12.0
71.128505893 68.6848754882812 13.0
40.901673931 31.4402942657471 14.0
0 0.020281195640564 15.0
0 -0.0124243497848511 16.0
27.293677358 27.1503715515137 17.0
0 -0.043791651725769 18.0
0 0.0139557123184204 19.0
0 0.0142508745193481 20.0
0 -0.0462580919265747 21.0
0 0.0526217222213745 22.0
37.440146235 36.4715919494629 23.0
0 0.113077998161316 24.0
0 0.461219549179077 25.0
0 0.0542105436325073 26.0
0 0.174614071846008 27.0
0 -0.0234261751174927 28.0
0 -0.0439578294754028 29.0
0 -0.0531522035598755 30.0
0 -0.0505224466323853 31.0
0 -0.0185805559158325 32.0
0 -0.103410243988037 33.0
0 -0.0510889291763306 34.0
0 0.0803546905517578 35.0
0 -0.0780924558639526 36.0
0 -0.000398039817810059 37.0
0 -0.00573170185089111 38.0
0 0.0701504945755005 39.0
0 0.114384531974792 40.0
0 -0.0113248825073242 41.0
0 -0.1137615442276 42.0
298.30976256 294.114654541016 43.0
0 0.0558675527572632 44.0
0 -0.014986515045166 45.0
0 -0.0188382863998413 46.0
0 -0.0107556581497192 47.0
0 -0.0667763948440552 48.0
0 0.293091297149658 49.0
0.073350875899 0.0523873567581177 50.0
1.4500164192 0.917349219322205 51.0
0 -0.109744191169739 52.0
0 0.0457051992416382 53.0
0 -0.0201884508132935 54.0
0 -0.0525485277175903 55.0
0 -0.0369089841842651 56.0
0 0.0039898157119751 57.0
0 -0.0465580224990845 58.0
0 -0.0304335355758667 59.0
0 0.203679919242859 60.0
0 0.05412757396698 61.0
0 0.857503712177277 62.0
247.88849941 258.555114746094 0.0
0 -0.093845009803772 1.0
0 -0.0509649515151978 2.0
0 0.0223344564437866 3.0
0 -0.00753343105316162 4.0
0 -0.0455652475357056 5.0
0 -0.028821587562561 6.0
0 -0.0742896795272827 7.0
0 0.0718022584915161 8.0
0 -0.00515997409820557 9.0
0 0.0188268423080444 10.0
0 -0.076877236366272 11.0
0 0.151380896568298 12.0
50.733748978 49.8353614807129 13.0
37.92975737 36.2653541564941 14.0
0 -0.0807482004165649 15.0
0 -0.0399597883224487 16.0
32.540317113 31.9161052703857 17.0
0 0.0684036016464233 18.0
0 0.0705152750015259 19.0
0 0.00758016109466553 20.0
0 -0.0686722993850708 21.0
0 -0.0549856424331665 22.0
42.542608801 40.993968963623 23.0
0 0.0453256368637085 24.0
0 0.536481916904449 25.0
0 0.00671803951263428 26.0
0 0.00651156902313232 27.0
0 -0.0203815698623657 28.0
0 -0.0167356729507446 29.0
0 0.008994460105896 30.0
0 -0.0110388994216919 31.0
0 0.0606001615524292 32.0
0 -0.141007542610168 33.0
0 -0.0897709131240845 34.0
0 0.16029965877533 35.0
0 -0.0537410974502563 36.0
0 0.102974772453308 37.0
0 0.0385621786117554 38.0
0 0.0403138399124146 39.0
0 0.120065093040466 40.0
0 0.168853640556335 41.0
0 0.00888192653656006 42.0
301.97600572 296.272216796875 43.0
0 0.0304960012435913 44.0
0 0.0543769598007202 45.0
0 -0.00276815891265869 46.0
0 -0.0221010446548462 47.0
0 0.0299943685531616 48.0
0 0.295852690935135 49.0
0.090966138698 0.0274280309677124 50.0
2.0947647829 0.878329038619995 51.0
0 -0.0771507024765015 52.0
0 0.0190454721450806 53.0
0 -0.00778520107269287 54.0
0 -0.0434447526931763 55.0
0 -0.0890880823135376 56.0
0 0.0333350896835327 57.0
0 -0.0442467927932739 58.0
0 0.0794905424118042 59.0
0 0.0633655786514282 60.0
0 -0.0562993288040161 61.0
0 0.783125877380371 62.0
260.89285463 257.509674072266 0.0
0 -0.00125372409820557 1.0
0 -0.0210839509963989 2.0
0 0.188271164894104 3.0
0 -0.0476974248886108 4.0
0 -0.311301350593567 5.0
0 0.0904301404953003 6.0
0 -0.0143123865127563 7.0
0 -0.0400738716125488 8.0
0 -0.000986695289611816 9.0
0 -0.0626860857009888 10.0
0 -0.00472903251647949 11.0
0 -0.133206963539124 12.0
58.285007757 57.6713371276855 13.0
32.113945564 31.8629302978516 14.0
0 -0.0431734323501587 15.0
0 -0.0352455377578735 16.0
36.008457059 36.1359977722168 17.0
0 0.008841872215271 18.0
0 0.00296413898468018 19.0
0 0.0360811948776245 20.0
0 -0.0396864414215088 21.0
0 -0.013703465461731 22.0
42.147832913 41.7146949768066 23.0
0 -0.122065186500549 24.0
0 1.05938506126404 25.0
0 -0.00895321369171143 26.0
0 0.107622265815735 27.0
0 -0.0238710641860962 28.0
0 -0.00170576572418213 29.0
0 -0.0491558313369751 30.0
0 -0.068062424659729 31.0
0 0.00651872158050537 32.0
0 -0.145640134811401 33.0
0 -0.0536738634109497 34.0
0 -0.0484441518783569 35.0
0 0.12087070941925 36.0
0 0.0174013376235962 37.0
0 0.0963529348373413 38.0
0 0.0560718774795532 39.0
0 0.0751012563705444 40.0
0 -0.0684123039245605 41.0
0 -0.0654093027114868 42.0
294.83808443 297.224395751953 43.0
0 0.00198686122894287 44.0
0 0.18674111366272 45.0
0 -0.0296965837478638 46.0
0 -0.00657904148101807 47.0
0 0.0246595144271851 48.0
0.17172879965 0.266851186752319 49.0
0.060800106113 0.102807879447937 50.0
1.9995505506 0.898768901824951 51.0
0 -0.053780198097229 52.0
0 -0.0475879907608032 53.0
0 0.0262564420700073 54.0
0 -0.117992043495178 55.0
0 0.314693808555603 56.0
0 0.0403097867965698 57.0
0 -0.0273896455764771 58.0
0 0.0312196016311646 59.0
0 0.218079686164856 60.0
0 -0.0290414094924927 61.0
0 0.79991340637207 62.0
214.57075343 258.664489746094 0.0
0 -0.0947529077529907 1.0
0 -0.0362225770950317 2.0
0 -0.00888383388519287 3.0
0 -0.0397757291793823 4.0
0 -0.0604642629623413 5.0
0 -0.0478168725967407 6.0
0 -0.0822993516921997 7.0
0 -0.069361686706543 8.0
0 0.071092963218689 9.0
0 -0.0197579860687256 10.0
0 -0.00920403003692627 11.0
0 -0.115932106971741 12.0
57.599323047 55.7209739685059 13.0
40.177904194 39.5836868286133 14.0
0 -0.0451904535293579 15.0
0 0.014644980430603 16.0
40.63475257 40.7292671203613 17.0
0 -0.062308669090271 18.0
0 0.000470280647277832 19.0
0 -0.00606977939605713 20.0
0 0.000340819358825684 21.0
0 -0.0673614740371704 22.0
41.297173024 42.2254905700684 23.0
0 -0.137621998786926 24.0
0 0.741455674171448 25.0
0 0.0804861783981323 26.0
0 0.0925711393356323 27.0
0 -0.0677119493484497 28.0
0 -0.0384980440139771 29.0
0 -0.0550532341003418 30.0
0 -0.026814341545105 31.0
0 0.0464409589767456 32.0
0 0.0170744657516479 33.0
0 -0.00860178470611572 34.0
0 -0.0695980787277222 35.0
0 0.0109325647354126 36.0
0 -0.0647314786911011 37.0
0 -0.0398114919662476 38.0
0 0.0446606874465942 39.0
0 0.0892099142074585 40.0
0 -0.0649691820144653 41.0
0 -0.0982376337051392 42.0
295.99639626 296.634063720703 43.0
0 0.0145238637924194 44.0
0 0.0501176118850708 45.0
0 -0.0456768274307251 46.0
0 -0.0859049558639526 47.0
0 -0.0323530435562134 48.0
1.9146032212 0.281377851963043 49.0
0.096800124531 0.0651770830154419 50.0
2.0968843075 0.924152076244354 51.0
0 -0.0396283864974976 52.0
0 -0.0517107248306274 53.0
0 0.00529634952545166 54.0
0 -0.101649880409241 55.0
0 -0.0531085729598999 56.0
0 -0.0232292413711548 57.0
0 -0.0427721738815308 58.0
0 -0.0045546293258667 59.0
0 0.0178741216659546 60.0
0 0.05341637134552 61.0
0 0.981018841266632 62.0
217.91946451 256.151611328125 0.0
0 -0.0497561693191528 1.0
0 -0.0200725793838501 2.0
0 0.0436619520187378 3.0
0 -0.0166245698928833 4.0
0 0.0116456747055054 5.0
0 0.0212587118148804 6.0
0 0.0415467023849487 7.0
0 -0.0191290974617004 8.0
0 0.0741571187973022 9.0
0 0.0069965124130249 10.0
0 0.0118168592453003 11.0
0 -0.0129648447036743 12.0
59.478834199 58.6771812438965 13.0
46.409562916 39.1844177246094 14.0
0 -0.0668336153030396 15.0
0 -0.0325385332107544 16.0
28.692105235 28.9441146850586 17.0
0 0.0187050104141235 18.0
0 -0.00117242336273193 19.0
0 0.0437599420547485 20.0
0 0.0374869108200073 21.0
0 -0.0656405687332153 22.0
37.997745539 37.4440879821777 23.0
0 -0.100567936897278 24.0
0 1.63202178478241 25.0
0 0.0546187162399292 26.0
0 0.0239876508712769 27.0
0 0.0354026556015015 28.0
0 -0.0717693567276001 29.0
0 -0.102812647819519 30.0
0 -0.0381513833999634 31.0
0 -0.00435411930084229 32.0
0 -0.0415619611740112 33.0
0 -0.0105413198471069 34.0
0 0.0820246934890747 35.0
0 -0.00839555263519287 36.0
0 0.0504947900772095 37.0
0 0.0136774778366089 38.0
0 0.0127509832382202 39.0
0 -0.0247727632522583 40.0
0 0.00465595722198486 41.0
0 -0.0511270761489868 42.0
291.23321184 296.352691650391 43.0
0 -0.0238605737686157 44.0
0 0.0826528668403625 45.0
0 -0.0184174776077271 46.0
0 -0.066149115562439 47.0
0 0.0173963308334351 48.0
0 0.356951415538788 49.0
0.063417660552 0.0579980611801147 50.0
2.0206333439 1.02412796020508 51.0
0 -0.0713373422622681 52.0
0 0.00150668621063232 53.0
0 -0.0221225023269653 54.0
0 -0.0827986001968384 55.0
0 -0.0639711618423462 56.0
0 0.0350049734115601 57.0
0 -0.0791672468185425 58.0
0 -0.0267230272293091 59.0
0 -0.0933493375778198 60.0
0 -0.06645667552948 61.0
0 0.817799210548401 62.0
262.76678226 252.685684204102 0.0
0 -0.00747358798980713 1.0
0 -0.00527346134185791 2.0
0 -0.0083233118057251 3.0
0 -0.023647665977478 4.0
0 -0.0626417398452759 5.0
0 0.0405887365341187 6.0
0 -0.00625145435333252 7.0
0 0.0128002166748047 8.0
0 0.0431751012802124 9.0
0 0.00940990447998047 10.0
0 0.028535008430481 11.0
0 0.00971829891204834 12.0
57.824283741 56.9955253601074 13.0
45.487282754 41.975959777832 14.0
0 0.0235642194747925 15.0
0 -0.0313969850540161 16.0
39.603523949 39.4969291687012 17.0
0 -0.0319298505783081 18.0
0 0.00243055820465088 19.0
0 0.0626260042190552 20.0
0 -0.00738942623138428 21.0
0 -0.0255242586135864 22.0
31.806319326 31.1056079864502 23.0
0 0.047514796257019 24.0
0 0.739019691944122 25.0
0 0.142240881919861 26.0
0 0.000798583030700684 27.0
0 -0.0330649614334106 28.0
0 -0.0808202028274536 29.0
0 -0.00916659832000732 30.0
0 -0.0215727090835571 31.0
0 0.0045851469039917 32.0
0 0.071745753288269 33.0
0 -0.0485774278640747 34.0
0 -0.0554542541503906 35.0
0 -0.0629326105117798 36.0
0 -0.101357340812683 37.0
0 -0.00861203670501709 38.0
0 0.0218311548233032 39.0
0 0.0347851514816284 40.0
0 0.00262212753295898 41.0
0 0.0201486349105835 42.0
304.30449262 292.726684570312 43.0
0 -0.0656229257583618 44.0
0 -0.000860929489135742 45.0
0 -0.0361908674240112 46.0
0 -0.0668298006057739 47.0
0 -0.0185128450393677 48.0
0 0.299219489097595 49.0
0.07597002613 0.0716146230697632 50.0
1.5939782681 0.889585256576538 51.0
0 -0.0315040349960327 52.0
0 -0.0613752603530884 53.0
0 -0.0103636980056763 54.0
0 -0.135264039039612 55.0
0 0.023772120475769 56.0
0 -0.0343014001846313 57.0
0 -0.0901373624801636 58.0
0 -0.0516337156295776 59.0
0 -0.0907238721847534 60.0
0 -0.0109280347824097 61.0
0 0.879419684410095 62.0
218.4108762 256.841064453125 0.0
0 -0.135962128639221 1.0
0 -0.0187824964523315 2.0
0 -0.0104457139968872 3.0
0 -0.0379210710525513 4.0
0 -0.05193030834198 5.0
0 0.0237702131271362 6.0
0 0.010372519493103 7.0
0 0.00357639789581299 8.0
0 0.139988303184509 9.0
0 0.0312837362289429 10.0
0 -0.0328632593154907 11.0
0 0.00245797634124756 12.0
78.369453694 75.9019241333008 13.0
26.678807942 22.9127044677734 14.0
0 -0.0610576868057251 15.0
0 0.0589355230331421 16.0
33.882306077 33.7184448242188 17.0
0 -0.00271427631378174 18.0
0 0.019321084022522 19.0
0 -0.0257176160812378 20.0
0 0.0259679555892944 21.0
0 -0.0385161638259888 22.0
36.920556958 36.4481964111328 23.0
0 0.0778535604476929 24.0
0 0.524407982826233 25.0
0 -0.0107289552688599 26.0
0 0.0571190118789673 27.0
0 -0.0026775598526001 28.0
0 -0.0259331464767456 29.0
0 -0.103551506996155 30.0
0 -0.0389667749404907 31.0
0 0.0188325643539429 32.0
0 -0.0416380167007446 33.0
0 -0.0203992128372192 34.0
0 -0.030457615852356 35.0
0 0.10983145236969 36.0
0 -0.0393205881118774 37.0
0 -0.0419925451278687 38.0
0 -0.0863953828811646 39.0
0 0.0576714277267456 40.0
0 -0.0246748924255371 41.0
0 0.00998818874359131 42.0
293.48660797 296.883819580078 43.0
0 0.0237792730331421 44.0
0 0.0875738263130188 45.0
0 0.000246167182922363 46.0
0 -0.0926221609115601 47.0
0 -0.0379940271377563 48.0
0 0.272007942199707 49.0
0.077486446423 0.0581501722335815 50.0
1.8858513395 1.00048303604126 51.0
0 -0.0215746164321899 52.0
0 -0.0358809232711792 53.0
0 0.0686300992965698 54.0
0 -0.105288624763489 55.0
0 -0.0611988306045532 56.0
0 0.0473898649215698 57.0
0 -0.0624023675918579 58.0
0 -0.0289846658706665 59.0
0 -0.172064661979675 60.0
0 -0.0117200613021851 61.0
0 0.829455256462097 62.0
225.35068032 237.013916015625 0.0
0 -0.119820237159729 1.0
0 0.00894296169281006 2.0
0 -0.0197890996932983 3.0
0 -0.0597482919692993 4.0
0 0.0858153104782104 5.0
0 0.0511516332626343 6.0
0 -0.0571783781051636 7.0
0 -0.0160934925079346 8.0
0 0.104221701622009 9.0
0 -0.216099858283997 10.0
0 0.14157634973526 11.0
0 -0.723875641822815 12.0
78.746372517 79.9811248779297 13.0
42.900920484 26.214147567749 14.0
0 -0.00738680362701416 15.0
0 -0.0585662126541138 16.0
41.654272338 41.7911033630371 17.0
0 -0.0271908044815063 18.0
0 0.0277327299118042 19.0
0 0.057087779045105 20.0
0 -0.0370527505874634 21.0
0 0.0574244260787964 22.0
25.618625159 25.4088230133057 23.0
0 -0.111860394477844 24.0
0 1.30098783969879 25.0
0 0.00677859783172607 26.0
0 -0.00398027896881104 27.0
0 -0.0450111627578735 28.0
0 -0.0590258836746216 29.0
0 -0.0565407276153564 30.0
0 -0.147226929664612 31.0
0 -0.0362511873245239 32.0
0 -0.0456554889678955 33.0
0 -0.0882128477096558 34.0
0 0.0235837697982788 35.0
0 0.188403010368347 36.0
0 -0.0875118970870972 37.0
0 -0.0239783525466919 38.0
0 -0.0293880701065063 39.0
0 0.129845499992371 40.0
0 -0.0886774063110352 41.0
0 -0.0786291360855103 42.0
291.88263004 298.428344726562 43.0
0 0.0668953657150269 44.0
0 -0.0404179096221924 45.0
0 -0.109919905662537 46.0
0 0.0337363481521606 47.0
0 -0.0022202730178833 48.0
0 0.354279100894928 49.0
0.098550668453 0.0620747804641724 50.0
1.3765129248 1.015212059021 51.0
0 0.00668179988861084 52.0
0 0.0206187963485718 53.0
0 -0.0566602945327759 54.0
0 -0.076198935508728 55.0
0 -0.0463460683822632 56.0
0 -0.0175119638442993 57.0
0 -0.0749911069869995 58.0
0 -0.012715220451355 59.0
0 -0.0443600416183472 60.0
0 0.0791348218917847 61.0
0 0.827625274658203 62.0
238.68326094 260.618469238281 0.0
0 -0.0795055627822876 1.0
0 -0.00581157207489014 2.0
0 0.0457853078842163 3.0
0 -0.00342643260955811 4.0
0 0.0162192583084106 5.0
0 0.0191658735275269 6.0
0 -0.0488187074661255 7.0
0 0.0380039215087891 8.0
0 0.0297473669052124 9.0
0 -0.0375722646713257 10.0
0 -0.00654143095016479 11.0
0 0.0309451818466187 12.0
64.238030054 64.5092086791992 13.0
26.748580019 33.3807601928711 14.0
0 -0.0193411111831665 15.0
0 -0.0886291265487671 16.0
37.640256677 37.5597686767578 17.0
0 0.00439631938934326 18.0
0 -0.0192278623580933 19.0
0 0.0313671827316284 20.0
0 -0.0207811594009399 21.0
0 -0.0581256151199341 22.0
40.562912515 39.7785835266113 23.0
0 -0.0353618860244751 24.0
0 0.49333256483078 25.0
0 -0.017306923866272 26.0
0 0.0279242098331451 27.0
0 -0.000523686408996582 28.0
0 -0.0720330476760864 29.0
0 -0.0732687711715698 30.0
0 -0.0772408246994019 31.0
0 0.0333126783370972 32.0
0 0.0865927934646606 33.0
0 -0.0179108381271362 34.0
0 0.0373660326004028 35.0
0 -0.0932272672653198 36.0
0 -0.0428932905197144 37.0
0 0.0186260938644409 38.0
0 0.0860036611557007 39.0
0 0.0262998342514038 40.0
0 -0.0522663593292236 41.0
0 -0.0979639291763306 42.0
285.16804358 296.769317626953 43.0
0 0.039610743522644 44.0
0 -0.00559711456298828 45.0
0 0.0578650236129761 46.0
0 0.00291752815246582 47.0
0 0.00950515270233154 48.0
1.3722630632 0.302397429943085 49.0
0.088298272681 0.0916949510574341 50.0
2.0439038349 1.01967477798462 51.0
0 -0.00692307949066162 52.0
0 -0.0750938653945923 53.0
0 -0.0195049047470093 54.0
0 -0.0751656293869019 55.0
0 0.0332518815994263 56.0
0 0.149716019630432 57.0
0 0.0334566831588745 58.0
0 0.0110589265823364 59.0
0 -0.0481985807418823 60.0
0 0.0527418851852417 61.0
0 0.864993929862976 62.0
280.56752019 260.096160888672 0.0
0 -0.0422641038894653 1.0
0 0.0595136880874634 2.0
0 0.0298639535903931 3.0
0 0.00353968143463135 4.0
0 -0.168102860450745 5.0
0 0.0298203229904175 6.0
0 -0.0887683629989624 7.0
0 -0.00366735458374023 8.0
0 0.0178226232528687 9.0
0 -0.00909101963043213 10.0
0 0.0529602468013763 11.0
0 -0.0441614389419556 12.0
75.068642675 74.289665222168 13.0
25.632167587 24.7787551879883 14.0
0 -0.0285717248916626 15.0
0 -0.00802576541900635 16.0
32.104267336 31.9937362670898 17.0
0 -0.0404430627822876 18.0
0 -0.0109244585037231 19.0
0 0.0351418256759644 20.0
0 -0.0556677579879761 21.0
0 0.0120338201522827 22.0
36.117439943 35.1909103393555 23.0
0 0.0344585180282593 24.0
0 0.570320010185242 25.0
0 0.0969632863998413 26.0
0 0.0295306444168091 27.0
0 -0.054383397102356 28.0
0 -0.00141894817352295 29.0
0 -0.0473062992095947 30.0
0 -0.0569874048233032 31.0
0 0.0244902372360229 32.0
0 -0.0426822900772095 33.0
0 0.0196844339370728 34.0
0 0.0274547338485718 35.0
0 0.00530946254730225 36.0
0 -0.0689831972122192 37.0
0 0.0378099679946899 38.0
0 0.0249863862991333 39.0
0 0.0433710813522339 40.0
0 -0.0806772708892822 41.0
0 0.0147517919540405 42.0
299.69281027 296.718292236328 43.0
0 -0.0236979722976685 44.0
0 0.0787804126739502 45.0
0 -0.0407549142837524 46.0
0 -0.0444921255111694 47.0
0 0.0349429845809937 48.0
0 0.340459525585175 49.0
0.068228669897 0.0367871522903442 50.0
1.9021331776 0.880738019943237 51.0
0 -0.0459913015365601 52.0
0 0.00235879421234131 53.0
0 0.0083237886428833 54.0
0 -0.074492335319519 55.0
0 -0.026009202003479 56.0
0 0.00175654888153076 57.0
0 0.0123230218887329 58.0
0 -0.0166233777999878 59.0
0 0.0223089456558228 60.0
0 0.101655125617981 61.0
0 0.944615542888641 62.0
254.54534295 260.702087402344 0.0
0 -0.099307656288147 1.0
0 -0.0209270715713501 2.0
0 -0.0269430875778198 3.0
0 -0.0194185972213745 4.0
0 -0.0448840856552124 5.0
0 0.0192693471908569 6.0
0 -0.12666928768158 7.0
0 -0.0400182008743286 8.0
0 0.0727533102035522 9.0
0 -0.0278648138046265 10.0
0 0.00645995140075684 11.0
0 0.0953811407089233 12.0
67.208876715 66.9057388305664 13.0
31.372426399 33.1186408996582 14.0
0 -0.0822464227676392 15.0
0 0.0407146215438843 16.0
34.5289667 34.1089515686035 17.0
0 -0.0117169618606567 18.0
0 -0.0202659368515015 19.0
0 0.00726926326751709 20.0
0 -0.00544917583465576 21.0
0 -0.00382435321807861 22.0
29.077970191 29.574390411377 23.0
0 0.00325429439544678 24.0
0 0.671113610267639 25.0
0 0.0845135450363159 26.0
0 0.0192655324935913 27.0
0 0.0946563482284546 28.0
0 -0.0684341192245483 29.0
0 0.0619007349014282 30.0
0 -0.0191758871078491 31.0
0 -0.0751129388809204 32.0
0 0.103055119514465 33.0
0 -0.0530422925949097 34.0
0 -0.023154616355896 35.0
0 0.0715934038162231 36.0
0 -0.0372351408004761 37.0
0 -0.0483938455581665 38.0
0 0.0679200887680054 39.0
0 0.0457471609115601 40.0
0 -0.0415669679641724 41.0
0 -0.0850178003311157 42.0
301.66498052 294.851989746094 43.0
0 -0.0194541215896606 44.0
0 0.0686944723129272 45.0
0 -0.0483928918838501 46.0
0 0.010097861289978 47.0
0 -0.0270005464553833 48.0
0 0.330454915761948 49.0
0.06734684049 0.0397876501083374 50.0
2.1559368052 0.868916690349579 51.0
0 -0.0823193788528442 52.0
0 0.00910699367523193 53.0
0 0.00534713268280029 54.0
0 -0.0176700353622437 55.0
0 -0.0327450037002563 56.0
0 -0.0010608434677124 57.0
0 -0.0183945894241333 58.0
0 -0.0531274080276489 59.0
0 -0.0696734189987183 60.0
0 -0.0258532762527466 61.0
0 0.835953712463379 62.0
217.62329508 253.009704589844 0.0
0 -0.0928431749343872 1.0
0 -0.0847502946853638 2.0
0 -0.0251883268356323 3.0
0 -0.0112928152084351 4.0
0 0.000117063522338867 5.0
0 0.00625360012054443 6.0
0 -0.103489756584167 7.0
0 0.0156988501548767 8.0
0 0.0365546941757202 9.0
0 0.00285816192626953 10.0
0 -0.0544447898864746 11.0
0 -0.0610395669937134 12.0
68.580553107 68.5822906494141 13.0
24.059072432 29.4305667877197 14.0
0 -0.13506281375885 15.0
0 -0.0118085145950317 16.0
40.604433122 40.0275535583496 17.0
0 -0.0215190649032593 18.0
0 -0.0354479551315308 19.0
0 -9.71555709838867e-05 20.0
0 -0.000858902931213379 21.0
0 -0.115328907966614 22.0
33.390634093 33.661075592041 23.0
0 -0.0593324899673462 24.0
0 1.36315786838531 25.0
0 0.0340155363082886 26.0
0 0.0369340181350708 27.0
0 0.0899366140365601 28.0
0 -0.129403710365295 29.0
0 -0.0149232149124146 30.0
0 -0.0521615743637085 31.0
0 -0.0272222757339478 32.0
0 -0.019450306892395 33.0
0 -0.0533226728439331 34.0
0 0.0326061248779297 35.0
0 -0.0901044607162476 36.0
0 -0.0728670358657837 37.0
0 -0.0120843648910522 38.0
0 -0.0692950487136841 39.0
0 -0.0639678239822388 40.0
0 0.121772885322571 41.0
0 -0.0267425775527954 42.0
290.0880264 295.400756835938 43.0
0 0.00432813167572021 44.0
0 0.0915284156799316 45.0
0 0.00113546848297119 46.0
0 -0.0369142293930054 47.0
0 -0.0479446649551392 48.0
0 0.285271108150482 49.0
0.083890348419 0.0783575773239136 50.0
2.1308860983 0.893976211547852 51.0
0 -0.0705934762954712 52.0
0 -0.0328472852706909 53.0
0 0.00283610820770264 54.0
0 -0.0900880098342896 55.0
0 -0.0495589971542358 56.0
0 -0.0357755422592163 57.0
0 -0.0283526182174683 58.0
0 -0.0532480478286743 59.0
0 0.0140613317489624 60.0
0 0.0459932088851929 61.0
0 0.911181449890137 62.0
252.8776666 257.697601318359 0.0
0 -0.0466305017471313 1.0
0 0.00862467288970947 2.0
0 0.0269910097122192 3.0
0 -0.0705157518386841 4.0
0 -0.0313097238540649 5.0
0 -0.0907570123672485 6.0
0 -0.0196329355239868 7.0
0 0.0133970975875854 8.0
0 0.00201165676116943 9.0
0 0.00151550769805908 10.0
0 -0.0211846828460693 11.0
0 -0.0432878732681274 12.0
53.371839696 53.0472602844238 13.0
36.182797605 43.0049934387207 14.0
0 -0.00816643238067627 15.0
0 0.00986659526824951 16.0
35.488146765 35.6395149230957 17.0
0 -6.40153884887695e-05 18.0
0 -0.000705599784851074 19.0
0 0.0307925939559937 20.0
0 0.00461471080780029 21.0
0 -0.0127545595169067 22.0
35.464251111 33.882984161377 23.0
0 -0.0153080224990845 24.0
0 1.40685987472534 25.0
0 0.0501312017440796 26.0
0 0.0981889963150024 27.0
0 -0.0154277086257935 28.0
0 -0.0442692041397095 29.0
0 -0.0513929128646851 30.0
0 -0.0203450918197632 31.0
0 -0.0264459848403931 32.0
0 0.0926765203475952 33.0
0 0.00682008266448975 34.0
0 0.00140476226806641 35.0
0 -0.0430220365524292 36.0
0 -0.0053330659866333 37.0
0 -0.0245152711868286 38.0
0 0.0566762685775757 39.0
0 0.076473593711853 40.0
0 -0.0611207485198975 41.0
0 -0.0779424905776978 42.0
295.73657671 295.799560546875 43.0
0 0.00149190425872803 44.0
0 0.0484437942504883 45.0
0 0.0158966779708862 46.0
0 -0.0550957918167114 47.0
0 -0.0611382722854614 48.0
0 0.258135676383972 49.0
0.075506082823 0.0694500207901001 50.0
2.2164233124 0.983938694000244 51.0
0 -0.0481013059616089 52.0
0 -0.0477036237716675 53.0
0 0.0402206182479858 54.0
0 -0.129914164543152 55.0
0 -0.0428088903427124 56.0
0 -0.0256303548812866 57.0
0 -0.0500615835189819 58.0
0 -0.0598906278610229 59.0
0 0.0418025255203247 60.0
0 0.00131022930145264 61.0
0 0.879169821739197 62.0
274.19193798 255.892288208008 0.0
0 -0.00386345386505127 1.0
0 -0.0305143594741821 2.0
0 -0.0143917798995972 3.0
0 -0.0188297033309937 4.0
0 -0.0150605440139771 5.0
0 0.0425781011581421 6.0
0 0.00255334377288818 7.0
0 -0.0709712505340576 8.0
0 0.0585144758224487 9.0
0 0.00242125988006592 10.0
0 0.0345682799816132 11.0
0 -0.0476919412612915 12.0
75.643082264 74.5821228027344 13.0
13.75747602 24.5124073028564 14.0
0 -0.041786789894104 15.0
0 -0.00953304767608643 16.0
34.773097815 34.5906562805176 17.0
0 0.0102387666702271 18.0
0 -0.0253390073776245 19.0
0 -0.0295273065567017 20.0
0 0.025861382484436 21.0
0 0.00874221324920654 22.0
27.666336732 27.358362197876 23.0
0 -0.0033189058303833 24.0
0 0.560618877410889 25.0
0 0.00407540798187256 26.0
0 -0.0770043134689331 27.0
0 0.0519670248031616 28.0
0 -0.0364874601364136 29.0
0 -0.0549079179763794 30.0
0 -0.0509940385818481 31.0
0 -0.0084763765335083 32.0
0 0.0757955312728882 33.0
0 -0.0130907297134399 34.0
0 -0.0297390222549438 35.0
0 0.0596991777420044 36.0
0 -0.0208357572555542 37.0
0 -0.0054391622543335 38.0
0 0.0146764516830444 39.0
0 0.183091044425964 40.0
0 -0.0230751037597656 41.0
0 -0.0151573419570923 42.0
297.56864332 297.243469238281 43.0
0 -0.0159856081008911 44.0
0 0.0554789304733276 45.0
0 -0.00217545032501221 46.0
0 -0.0264097452163696 47.0
0 -0.00974142551422119 48.0
0 0.336274832487106 49.0
0.085654161353 0.0707067251205444 50.0
1.7178663352 0.957155227661133 51.0
0 -0.0398956537246704 52.0
0 -0.0269159078598022 53.0
0 0.0272132158279419 54.0
0 -0.0262099504470825 55.0
0 0.0234075784683228 56.0
0 -0.0253893136978149 57.0
0 -0.0779327154159546 58.0
0 -0.0731068849563599 59.0
0 0.00896275043487549 60.0
0 0.0535992383956909 61.0
0 0.900635957717896 62.0
232.90973734 253.777328491211 0.0
0 0.0145324468612671 1.0
0 -0.0834039449691772 2.0
0 -0.00697624683380127 3.0
0 -0.0213919878005981 4.0
0 -0.0479122400283813 5.0
0 -0.0216623544692993 6.0
0 -0.0410631895065308 7.0
0 -0.0222184658050537 8.0
0 0.0185874700546265 9.0
0 0.00932157039642334 10.0
0 0.00512552261352539 11.0
0 -0.156104683876038 12.0
51.189231602 50.8096733093262 13.0
50.818993875 45.0810241699219 14.0
0 -0.0540753602981567 15.0
0 0.0267537832260132 16.0
36.602084839 36.0977478027344 17.0
0 -0.0379914045333862 18.0
0 0.0042647123336792 19.0
0 0.000583529472351074 20.0
0 -0.0266245603561401 21.0
0 -0.027044415473938 22.0
34.919519712 34.2558708190918 23.0
0 -0.0122057199478149 24.0
0 0.575722754001617 25.0
0 0.0186647176742554 26.0
0 -0.0521243810653687 27.0
0 -0.0287328958511353 28.0
0 -0.0585149526596069 29.0
0 -0.0872479677200317 30.0
0 -0.0476747751235962 31.0
0 0.0353258848190308 32.0
0 0.0981718301773071 33.0
0 -0.0481253862380981 34.0
0 -0.0282009840011597 35.0
0 0.0372403860092163 36.0
0 -0.0626214742660522 37.0
0 -0.0246602296829224 38.0
0 0.0745099782943726 39.0
0 0.0121620893478394 40.0
0 -0.0751577615737915 41.0
0 0.0503534078598022 42.0
300.74704362 296.227386474609 43.0
0 -0.044101357460022 44.0
0 0.0734254121780396 45.0
0 -0.0151985883712769 46.0
0 -0.0675805807113647 47.0
0 -0.00952780246734619 48.0
0 0.326284229755402 49.0
0.082471374992 0.0630544424057007 50.0
1.6308822718 0.970908105373383 51.0
0 -0.096011757850647 52.0
0 -0.0510450601577759 53.0
0 0.0723363161087036 54.0
0 -0.0650593042373657 55.0
0 0.00956308841705322 56.0
0 -0.0357891321182251 57.0
0 -0.0317865610122681 58.0
0 -0.0272630453109741 59.0
0 -0.0409306287765503 60.0
0 -0.0516766309738159 61.0
0 0.886041164398193 62.0
294.20739513 259.795166015625 0.0
0 -0.0896373987197876 1.0
0 0.0274578332901001 2.0
0 -0.0739954710006714 3.0
0 0.0186270475387573 4.0
0 0.0195735692977905 5.0
0 -0.117852330207825 6.0
0 0.0375736951828003 7.0
0 0.0937333106994629 8.0
0 0.00645005702972412 9.0
0 -0.0247188806533813 10.0
0 -0.404529929161072 11.0
0 -0.00544607639312744 12.0
81.741061226 81.1712417602539 13.0
18.484079879 18.5971584320068 14.0
0 -0.0455814599990845 15.0
0 -0.0390890836715698 16.0
30.218778637 31.0386257171631 17.0
0 0.028112530708313 18.0
0 -0.0101414918899536 19.0
0 0.0295144319534302 20.0
0 0.000324368476867676 21.0
0 -0.108757138252258 22.0
40.358921969 42.3698692321777 23.0
0 0.0554479360580444 24.0
0 0.471402168273926 25.0
0 0.0484241247177124 26.0
0 0.0132511258125305 27.0
0 0.0855640172958374 28.0
0 -0.163334488868713 29.0
0 -0.0469077825546265 30.0
0 -0.0289181470870972 31.0
0 0.0315808057785034 32.0
0 -0.008431077003479 33.0
0 -0.0403763055801392 34.0
0 0.0577638149261475 35.0
0 -0.101812958717346 36.0
0 -0.039167046546936 37.0
0 0.0828837156295776 38.0
0 0.0697149038314819 39.0
0 0.0542887449264526 40.0
0 0.0726786851882935 41.0
0 -0.00896799564361572 42.0
275.13017626 296.627471923828 43.0
0 -0.112175583839417 44.0
0 0.0245115756988525 45.0
0 -0.152132391929626 46.0
0 -0.0462843179702759 47.0
0 0.00764930248260498 48.0
0 0.429957687854767 49.0
0.086390391679 0.0417078733444214 50.0
1.5119311696 0.866796672344208 51.0
0 -0.0958541631698608 52.0
0 0.0241652727127075 53.0
0 0.0228399038314819 54.0
0 -0.0650433301925659 55.0
0 0.00758016109466553 56.0
0 -0.0222452878952026 57.0
0 -0.0875893831253052 58.0
0 -0.0519939661026001 59.0
0 0.00332391262054443 60.0
0 0.00979983806610107 61.0
0 0.799383759498596 62.0
244.8816621 254.607635498047 0.0
0 -0.0176841020584106 1.0
0 -0.115109801292419 2.0
0 -0.00433576107025146 3.0
0 -0.0358551740646362 4.0
0 -0.111269593238831 5.0
0 -0.016309380531311 6.0
0 -0.0543419122695923 7.0
0 0.00516915321350098 8.0
0 -0.0410867929458618 9.0
0 0.0128847360610962 10.0
0 0.0396540760993958 11.0
0 -0.0856205224990845 12.0
67.759926245 67.1994857788086 13.0
21.004208216 31.7858219146729 14.0
0 -0.00823700428009033 15.0
0 0.0178960561752319 16.0
26.933622107 27.5959033966064 17.0
0 -0.015372633934021 18.0
0 -0.00600135326385498 19.0
0 0.00419533252716064 20.0
0 0.0269430875778198 21.0
0 -0.0191603899002075 22.0
37.033500591 36.8013610839844 23.0
0 -0.110381722450256 24.0
0 0.9824378490448 25.0
0 0.0858610868453979 26.0
0 0.00436872243881226 27.0
0 0.0556896924972534 28.0
0 -0.0307444334030151 29.0
0 -0.0934803485870361 30.0
0 -0.012114405632019 31.0
0 -0.0161162614822388 32.0
0 0.0710228681564331 33.0
0 -0.0529285669326782 34.0
0 -0.0244060754776001 35.0
0 0.0276733636856079 36.0
0 0.00752007961273193 37.0
0 -0.0219367742538452 38.0
0 0.112943768501282 39.0
0 0.12096107006073 40.0
0 -0.00736880302429199 41.0
0 -0.0767198801040649 42.0
305.32437519 297.319488525391 43.0
0 -0.00424540042877197 44.0
0 0.0769824385643005 45.0
0 0.00433504581451416 46.0
0 -0.0739942789077759 47.0
0 -0.010348916053772 48.0
0 0.32088702917099 49.0
0.081088773646 0.143612027168274 50.0
1.8786827144 0.892883777618408 51.0
0 -0.085044264793396 52.0
0 -0.0438681840896606 53.0
0 0.0379973649978638 54.0
0 -0.0730465650558472 55.0
0 -0.0170787572860718 56.0
0 -0.0476866960525513 57.0
0 -0.0246461629867554 58.0
0 -0.00997269153594971 59.0
0 0.00987684726715088 60.0
0 -0.0289198160171509 61.0
0 0.878529787063599 62.0
237.13006186 248.901473999023 0.0
0 -0.0192896127700806 1.0
0 -0.09717857837677 2.0
0 -0.027107834815979 3.0
0 -0.0193675756454468 4.0
0 -0.0524014234542847 5.0
0 -0.018680214881897 6.0
0 -0.0470234155654907 7.0
0 -0.0298606753349304 8.0
0 0.12019670009613 9.0
0 0.0196655988693237 10.0
0 0.02979576587677 11.0
0 -0.0560060739517212 12.0
54.571975238 53.4956321716309 13.0
46.119781581 42.5797004699707 14.0
0 -0.18610155582428 15.0
0 0.0144585371017456 16.0
36.110648646 35.2150573730469 17.0
0 -0.0416325330734253 18.0
0 -0.0224014520645142 19.0
0 0.0162369012832642 20.0
0 -0.00107991695404053 21.0
0 -0.0249015092849731 22.0
27.160436254 26.7036323547363 23.0
0 0.0664757490158081 24.0
0 0.466249406337738 25.0
0 0.0323518514633179 26.0
0 0.0309821367263794 27.0
0 0.0261896848678589 28.0
0 -0.0993431806564331 29.0
0 -0.0291250944137573 30.0
0 -0.0352629423141479 31.0
0 -0.00881016254425049 32.0
0 -0.0340362787246704 33.0
0 -0.00339710712432861 34.0
0 0.022687554359436 35.0
0 0.00201046466827393 36.0
0 -0.0435510873794556 37.0
0 -0.0116499662399292 38.0
0 0.0101078748703003 39.0
0 0.112191081047058 40.0
0 -0.152225375175476 41.0
0 -0.0432392358779907 42.0
298.08076804 295.125335693359 43.0
0 0.0670865774154663 44.0
0 -0.00697708129882812 45.0
0 0.0172923803329468 46.0
0 -0.0597609281539917 47.0
0 0.00378596782684326 48.0
0 0.273358047008514 49.0
0.098746881887 0.135284543037415 50.0
1.9460942729 0.914548397064209 51.0
0 -0.0666242837905884 52.0
0 -0.0204237699508667 53.0
0 0.0256909132003784 54.0
0 -0.0791014432907104 55.0
0 -0.0845152139663696 56.0
0 -0.00348913669586182 57.0
0 -0.0227953195571899 58.0
0 0.000908255577087402 59.0
0 -0.00897777080535889 60.0
0 0.0191525220870972 61.0
0 0.858046412467957 62.0
234.65716883 255.002304077148 0.0
0 -0.0717817544937134 1.0
0 -0.0022810697555542 2.0
0 -0.0335046052932739 3.0
0 -0.0925818681716919 4.0
0 -0.00703585147857666 5.0
0 0.0502532720565796 6.0
0 -0.00531232357025146 7.0
0 -0.156226217746735 8.0
0 0.12771737575531 9.0
0 -0.105409860610962 10.0
0 -0.0085301399230957 11.0
0 -0.0349575281143188 12.0
47.298295101 46.4409332275391 13.0
50.720897389 43.8272819519043 14.0
0 -0.0453993082046509 15.0
0 -0.0155185461044312 16.0
35.364516817 34.541015625 17.0
0 0.0258780717849731 18.0
0 0.0241461992263794 19.0
0 -0.00530850887298584 20.0
0 -0.00960409641265869 21.0
0 0.0633929967880249 22.0
32.200745388 31.915283203125 23.0
0 0.116611361503601 24.0
0 0.480844020843506 25.0
0 0.00650441646575928 26.0
0 0.173084139823914 27.0
0 0.00409066677093506 28.0
0 -0.140865445137024 29.0
0 0.243676066398621 30.0
0 -0.0379732847213745 31.0
0 -0.0058523416519165 32.0
0 -0.0871129035949707 33.0
0 -0.0356293916702271 34.0
0 -0.0647660493850708 35.0
0 0.200650811195374 36.0
0 0.00857198238372803 37.0
0 -0.0221110582351685 38.0
0 -0.00272476673126221 39.0
0 0.212760806083679 40.0
0 0.0146297216415405 41.0
0 -0.0665699243545532 42.0
307.96706574 296.921112060547 43.0
0 -0.000769257545471191 44.0
0 0.0183718204498291 45.0
0 -0.055428147315979 46.0
0 0.0244013071060181 47.0
0 0.000107884407043457 48.0
0 0.252898573875427 49.0
0.087987982218 0.09168541431427 50.0
1.344756985 0.893151462078094 51.0
0 -0.0431977510452271 52.0
0 -0.0470494031906128 53.0
0 0.0557268857955933 54.0
0 -0.0892153978347778 55.0
0 0.0577000379562378 56.0
0 0.0145095586776733 57.0
0 -0.134742856025696 58.0
0 -0.00707948207855225 59.0
0 -0.13432252407074 60.0
0 0.0578292608261108 61.0
0 0.820434927940369 62.0
258.88237442 246.304428100586 0.0
0 -0.1120525598526 1.0
0 0.043498158454895 2.0
0 0.0137642621994019 3.0
0 -0.03785240650177 4.0
0 0.0171509981155396 5.0
0 0.0212796926498413 6.0
0 -0.0963732004165649 7.0
0 -0.0102431774139404 8.0
0 0.147306323051453 9.0
0 0.0191751718521118 10.0
0 -0.0345396995544434 11.0
0 -0.155479073524475 12.0
47.88675572 43.9197387695312 13.0
47.92879095 38.8853225708008 14.0
0 -0.0593258142471313 15.0
0 -0.0197123289108276 16.0
32.812988538 34.5271759033203 17.0
0 -0.00313842296600342 18.0
0 -0.0354107618331909 19.0
0 0.0401209592819214 20.0
0 -0.0112003087997437 21.0
0 -0.0224758386611938 22.0
34.549200188 33.7726860046387 23.0
0 -0.186172127723694 24.0
5.4426552722 0.604838073253632 25.0
0 0.00231039524078369 26.0
0 0.0856651067733765 27.0
0 0.169093012809753 28.0
0 -0.121371388435364 29.0
0 -0.190367102622986 30.0
0 -0.0763188600540161 31.0
0 -0.100587010383606 32.0
0 -0.0222357511520386 33.0
0 -0.0331758260726929 34.0
0 0.26029908657074 35.0
0 0.101427912712097 36.0
0 -0.0281277894973755 37.0
0 -0.00193560123443604 38.0
0 0.0202502012252808 39.0
0 0.00977504253387451 40.0
0 0.161796927452087 41.0
0 -0.149998307228088 42.0
295.43668419 288.942138671875 43.0
0 -0.0193639993667603 44.0
0 -0.618471384048462 45.0
0 -0.0449261665344238 46.0
0 -0.00221288204193115 47.0
0 0.0615619421005249 48.0
0 0.237040042877197 49.0
0.081433129384 0.0728400945663452 50.0
1.7621144139 0.826170682907104 51.0
0 -0.0812731981277466 52.0
0 -0.0890985727310181 53.0
0 0.0842651128768921 54.0
0 -0.0925835371017456 55.0
0 -0.0460084676742554 56.0
0 -0.0322743654251099 57.0
0 -0.0244029760360718 58.0
0 -0.066031813621521 59.0
0 0.105273365974426 60.0
0 -0.0342868566513062 61.0
0 0.814053058624268 62.0
325.45467795 258.905090332031 0.0
0 0.020128607749939 1.0
0 -0.0580767393112183 2.0
0 0.0510522127151489 3.0
0 -0.0252865552902222 4.0
0 0.0225847959518433 5.0
0 -0.215417504310608 6.0
0 0.073058009147644 7.0
0 -0.0823590159416199 8.0
0 0.136542677879333 9.0
0 -0.0696266889572144 10.0
0 0.0260013341903687 11.0
0 -0.277111649513245 12.0
52.444244042 51.7595634460449 13.0
34.823352652 34.4428558349609 14.0
0 -0.071872353553772 15.0
0 -0.0559860467910767 16.0
27.263400709 27.0503883361816 17.0
0 0.0356851816177368 18.0
0 -0.0384970903396606 19.0
0 -0.0109301805496216 20.0
0 0.00613677501678467 21.0
0 -0.0290967226028442 22.0
39.698222495 38.3493270874023 23.0
0 0.00226056575775146 24.0
0 0.482140839099884 25.0
0 0.115900874137878 26.0
0 0.044939398765564 27.0
0 0.0197209119796753 28.0
0 -0.0500785112380981 29.0
0 -0.185442328453064 30.0
0 -0.0246692895889282 31.0
0 0.047087550163269 32.0
0 -0.0229364633560181 33.0
0 -0.0038381814956665 34.0
0 -0.026903510093689 35.0
0 -0.181933522224426 36.0
0 -0.00434887409210205 37.0
0 -0.0495797395706177 38.0
0 0.00725257396697998 39.0
0 0.241108775138855 40.0
0 0.0432292222976685 41.0
0 0.0580428838729858 42.0
304.23807588 291.892944335938 43.0
0 -0.141028046607971 44.0
0 0.0267534255981445 45.0
0 0.00404977798461914 46.0
0 0.00341212749481201 47.0
0 0.033748984336853 48.0
0 0.327025949954987 49.0
0.093613550149 0.0606054067611694 50.0
2.4171198018 0.959826350212097 51.0
0 -0.0936700105667114 52.0
0 -0.0466533899307251 53.0
0 -0.00487399101257324 54.0
0 -0.100896239280701 55.0
0 -0.0727790594100952 56.0
0 0.0265134572982788 57.0
0 0.0350137948989868 58.0
0 -0.0751591920852661 59.0
0 -0.0849305391311646 60.0
0 -0.0171517133712769 61.0
0 0.812237501144409 62.0
254.03495808 254.613723754883 0.0
0 -0.0418654680252075 1.0
0 -0.0752881765365601 2.0
0 -0.0146929025650024 3.0
0 -0.0451251268386841 4.0
0 -0.109360814094543 5.0
0 -0.0605188608169556 6.0
0 -0.0919018983840942 7.0
0 -0.0472866296768188 8.0
0 0.0487502813339233 9.0
0 0.0356245040893555 10.0
0 -0.0538567304611206 11.0
0 -0.00890028476715088 12.0
53.61746442 52.2930145263672 13.0
47.018244347 42.2613830566406 14.0
0 -0.0953527688980103 15.0
0 0.00456130504608154 16.0
39.013527275 38.8231086730957 17.0
0 0.0108398199081421 18.0
0 -0.0107880830764771 19.0
0 -0.00517594814300537 20.0
0 -0.0214196443557739 21.0
0 -0.0736976861953735 22.0
36.303448129 36.3013877868652 23.0
0 -0.0370436906814575 24.0
0 0.522781133651733 25.0
0 0.130109190940857 26.0
0 0.0641320943832397 27.0
0 0.0294703245162964 28.0
0 -0.0199171304702759 29.0
0 -0.0717751979827881 30.0
0 -0.0525244474411011 31.0
0 -0.0183507204055786 32.0
0 0.0621975660324097 33.0
0 0.00932276248931885 34.0
0 -0.0774334669113159 35.0
0 -0.0513643026351929 36.0
0 0.0537606477737427 37.0
0 -0.0216270685195923 38.0
0 0.047235369682312 39.0
0 0.151016116142273 40.0
0 -0.0181372165679932 41.0
0 0.0242494344711304 42.0
291.74951304 295.876739501953 43.0
0 0.00294387340545654 44.0
0 0.00725221633911133 45.0
0 -0.065578818321228 46.0
0 -0.0255714654922485 47.0
0 -0.0110055208206177 48.0
0.49257156538 0.267235994338989 49.0
0.08159813432 0.0651999711990356 50.0
2.2415451044 0.98039323091507 51.0
0 -0.0534993410110474 52.0
0 -0.0322455167770386 53.0
0 0.0135139226913452 54.0
0 -0.0191999673843384 55.0
0 -0.0432466268539429 56.0
0 0.0304585695266724 57.0
0 -0.00391995906829834 58.0
0 -0.0724233388900757 59.0
0 0.0071033239364624 60.0
0 0.0161296129226685 61.0
0 0.91509348154068 62.0
236.04417897 254.581954956055 0.0
0 -0.00268948078155518 1.0
0 -0.0630096197128296 2.0
0 0.0240164995193481 3.0
0 -0.0317057371139526 4.0
0 0.0424407720565796 5.0
0 -0.00299370288848877 6.0
0 -0.0712980031967163 7.0
0 0.0175907611846924 8.0
0 0.0506623983383179 9.0
0 -0.00798690319061279 10.0
0 -0.00830936431884766 11.0
0 -0.0940700769424438 12.0
60.331493039 60.4744873046875 13.0
35.566921063 37.2943077087402 14.0
0 0.0360344648361206 15.0
0 0.000367283821105957 16.0
33.109169994 33.3595771789551 17.0
0 -0.0147143602371216 18.0
0 -0.0532234907150269 19.0
0 0.00260961055755615 20.0
0 -0.0378884077072144 21.0
0 0.0269311666488647 22.0
36.1682862 35.0601844787598 23.0
0 0.102902293205261 24.0
0 0.427889704704285 25.0
0 0.0595816373825073 26.0
0 0.0434645414352417 27.0
0 -0.0544096231460571 28.0
0 -0.0510455369949341 29.0
0 0.029105544090271 30.0
0 -0.0783275365829468 31.0
0 -0.0259062051773071 32.0
0 0.0318704843521118 33.0
0 -0.0296581983566284 34.0
0 0.0572192668914795 35.0
0 -0.112316250801086 36.0
0 -0.0880891084671021 37.0
0 -0.0158149003982544 38.0
0 0.00372302532196045 39.0
0 0.0301426649093628 40.0
0 -0.0285773277282715 41.0
0 0.0371431112289429 42.0
295.78077173 295.760772705078 43.0
0 -0.00260031223297119 44.0
0 0.0397506356239319 45.0
0 -0.00544154644012451 46.0
0 -0.0493730306625366 47.0
0 0.0592516660690308 48.0
0 0.324695557355881 49.0
0.099450149283 0.145698666572571 50.0
1.9770940485 0.874365091323853 51.0
0 -0.0879007577896118 52.0
0 -0.093725323677063 53.0
0 0.0316044092178345 54.0
0 -0.108770489692688 55.0
0 -0.0548168420791626 56.0
0 0.0031818151473999 57.0
0 -0.0661996603012085 58.0
0 -0.0304402112960815 59.0
0 0.0845698118209839 60.0
0 -0.027527928352356 61.0
0 0.819696068763733 62.0
257.03468776 254.899536132812 0.0
0 -0.0457159280776978 1.0
0 0.00134265422821045 2.0
0 0.0104566812515259 3.0
0 -0.0450862646102905 4.0
0 -0.0674082040786743 5.0
0 -0.0641223192214966 6.0
0 -0.04816734790802 7.0
0 -0.0646818876266479 8.0
0 0.0833467245101929 9.0
0 -0.094723105430603 10.0
0 -0.0227556228637695 11.0
0 -0.0619465112686157 12.0
75.561863144 74.5167388916016 13.0
33.25320531 24.011739730835 14.0
0 -0.0528923273086548 15.0
0 -0.0148955583572388 16.0
36.912263212 36.6487846374512 17.0
0 -0.0337022542953491 18.0
0 0.00155866146087646 19.0
0 -0.00416052341461182 20.0
0 -0.00165188312530518 21.0
0 0.032579779624939 22.0
35.852809871 34.8007164001465 23.0
0 -0.108474850654602 24.0
0 0.759220361709595 25.0
0 -0.0115939378738403 26.0
0 0.0756186246871948 27.0
0 -0.0360742807388306 28.0
0 -0.0420624017715454 29.0
0 -0.0786288976669312 30.0
0 -0.0181866884231567 31.0
0 0.0266050100326538 32.0
0 0.048564076423645 33.0
0 -0.0667146444320679 34.0
0 -0.0791525840759277 35.0
0 -0.0756815671920776 36.0
0 -0.0389522314071655 37.0
0 0.0298417806625366 38.0
0 0.12923538684845 39.0
0 0.0473412275314331 40.0
0 -0.0378317832946777 41.0
0 -0.0447994470596313 42.0
291.67765658 296.011810302734 43.0
0 -0.00739061832427979 44.0
0 0.0292861461639404 45.0
0 -0.00740683078765869 46.0
0 -0.00351297855377197 47.0
0 0.0644716024398804 48.0
0 0.260762274265289 49.0
0.07345652707 0.066283106803894 50.0
1.7718718791 0.866422832012177 51.0
0 -0.0633894205093384 52.0
0 -0.0361822843551636 53.0
0 0.0140208005905151 54.0
0 -0.0985583066940308 55.0
0 -0.13679563999176 56.0
0 -0.00558006763458252 57.0
0 -0.0623005628585815 58.0
0 -0.0846012830734253 59.0
0 2.50339508056641e-06 60.0
0 0.0455683469772339 61.0
0 0.813013434410095 62.0
247.14978929 258.045623779297 0.0
0 -0.0336366891860962 1.0
0 -0.121161818504333 2.0
0 -0.00516331195831299 3.0
0 -0.0212615728378296 4.0
0 -0.0404602289199829 5.0
0 0.0558212995529175 6.0
0 -0.116750121116638 7.0
0 0.0277184247970581 8.0
0 0.1001056432724 9.0
0 -0.0206559896469116 10.0
0 0.0101728439331055 11.0
0 -0.146595597267151 12.0
70.269927493 69.1871032714844 13.0
26.019537196 30.1788959503174 14.0
0 -0.0478307008743286 15.0
0 -0.00937879085540771 16.0
37.112595429 37.8287239074707 17.0
0 0.0101627111434937 18.0
0 -0.03217613697052 19.0
0 -0.0196577310562134 20.0
0 0.00514686107635498 21.0
0 -0.0771113634109497 22.0
34.357191016 34.0625495910645 23.0
0 0.00563561916351318 24.0
0 0.573914766311646 25.0
0 0.0824512243270874 26.0
0 -0.049106240272522 27.0
0 0.102305769920349 28.0
0 -0.0324174165725708 29.0
0 -0.0904406309127808 30.0
0 -0.0398210287094116 31.0
0 -0.0648714303970337 32.0
0 -0.0120965242385864 33.0
0 -0.0480204820632935 34.0
0 -0.0368393659591675 35.0
0 0.0577384233474731 36.0
0 -0.0453518629074097 37.0
0 -0.023349404335022 38.0
0 0.0378049612045288 39.0
0 0.0773838758468628 40.0
0 -0.0313133001327515 41.0
0 0.011278510093689 42.0
304.49214157 292.076934814453 43.0
0 -0.0167509317398071 44.0
0 0.07454514503479 45.0
0 -0.00134456157684326 46.0
0 0.018555760383606 47.0
0 -0.0149661302566528 48.0
0 0.324806272983551 49.0
0.086462505248 0.0652922391891479 50.0
2.0149638687 0.995886147022247 51.0
0 -0.0504728555679321 52.0
0 -0.0131555795669556 53.0
0 0.0518537759780884 54.0
0 -0.131165623664856 55.0
0 0.0136665105819702 56.0
0 -0.0460604429244995 57.0
0 -0.0671714544296265 58.0
0 0.00797402858734131 59.0
0 0.145683407783508 60.0
0 -0.000591635704040527 61.0
0 0.818594694137573 62.0
248.40144179 253.15251159668 0.0
0 0.06151282787323 1.0
0 0.0199722051620483 2.0
0 0.0211280584335327 3.0
0 0.00728499889373779 4.0
0 -0.00519835948944092 5.0
0 -0.137926697731018 6.0
0 -0.0433541536331177 7.0
0 0.019428014755249 8.0
0 0.058122992515564 9.0
0 -0.0531729459762573 10.0
0 -0.0343225002288818 11.0
0 -0.0122791528701782 12.0
72.444532492 72.1035461425781 13.0
19.209369275 28.8546371459961 14.0
0 -0.0303508043289185 15.0
0 0.01909339427948 16.0
28.940090047 28.5692691802979 17.0
0 -0.0257381200790405 18.0
0 0.012062668800354 19.0
0 0.00823986530303955 20.0
0 -0.0302704572677612 21.0
0 -0.0501285791397095 22.0
28.966577482 28.6436424255371 23.0
0 -0.0850979089736938 24.0
0 0.612242817878723 25.0
0 0.0648559331893921 26.0
0 0.0605159997940063 27.0
0 -0.0401393175125122 28.0
0 -0.0756765604019165 29.0
0 -0.0291622877120972 30.0
0 -0.017769455909729 31.0
0 -0.0191494226455688 32.0
0 0.0462902784347534 33.0
0 -0.0365577936172485 34.0
0 0.0378715991973877 35.0
0 -0.00166952610015869 36.0
0 -0.0140968561172485 37.0
0 -0.0992709398269653 38.0
0 0.0373471975326538 39.0
0 -0.0106073617935181 40.0
0 0.172534823417664 41.0
0 -0.0695406198501587 42.0
294.08510851 294.085815429688 43.0
0 0.0222228765487671 44.0
0 0.0491991639137268 45.0
0 -0.0322211980819702 46.0
0 -0.0885424613952637 47.0
0 0.0203684568405151 48.0
0 0.308610767126083 49.0
0.088052253603 0.0558053255081177 50.0
1.695596824 0.940064787864685 51.0
0 -0.0404766798019409 52.0
0 -0.0625125169754028 53.0
0 0.0776408910751343 54.0
0 -0.169527411460876 55.0
0 -0.0591977834701538 56.0
0 0.00488889217376709 57.0
0 -0.0888324975967407 58.0
0 -0.0515340566635132 59.0
0 -0.0484145879745483 60.0
0 -0.0143548250198364 61.0
0 0.798934817314148 62.0
272.30572715 254.41047668457 0.0
0 -0.0865989923477173 1.0
0 -0.0629481077194214 2.0
0 -0.053892970085144 3.0
0 0.002754807472229 4.0
0 0.0342725515365601 5.0
0 -0.147387623786926 6.0
0 -0.0647116899490356 7.0
0 -0.0145940184593201 8.0
0 0.0613149404525757 9.0
0 0.0137468576431274 10.0
0 0.00643181800842285 11.0
0 0.0223087072372437 12.0
62.622463872 62.1503295898438 13.0
33.431457743 34.2240447998047 14.0
0 -0.137042164802551 15.0
0 -0.0503131151199341 16.0
33.761975149 33.817798614502 17.0
0 0.0216244459152222 18.0
0 -0.0320109128952026 19.0
0 0.00939285755157471 20.0
0 -0.00973212718963623 21.0
0 -0.0345450639724731 22.0
36.881935514 36.5273475646973 23.0
0 -0.0104843378067017 24.0
0 0.539185047149658 25.0
0 0.0395001173019409 26.0
0 0.0807260274887085 27.0
0 -0.0199199914932251 28.0
0 -0.0689746141433716 29.0
0 0.0749720335006714 30.0
0 -0.0315099954605103 31.0
0 -0.00942099094390869 32.0
0 -0.0230897665023804 33.0
0 0.00228703022003174 34.0
0 -0.0671824216842651 35.0
0 0.102764964103699 36.0
0 -0.0395792722702026 37.0
0 0.0397638082504272 38.0
0 -0.00967657566070557 39.0
0 0.0677770376205444 40.0
0 -0.0436185598373413 41.0
0 0.00112760066986084 42.0
297.01918654 295.982299804688 43.0
0 0.0253142118453979 44.0
0 0.0444198846817017 45.0
0 -0.0408147573471069 46.0
0 -0.0815967321395874 47.0
0 -0.0198994874954224 48.0
0 0.283121526241302 49.0
0.097947078953 0.0723620653152466 50.0
1.9221338643 0.952421188354492 51.0
0 -0.087389349937439 52.0
0 -0.0235868692398071 53.0
0 0.0204423666000366 54.0
0 -0.0711706876754761 55.0
0 -0.0443586111068726 56.0
0 0.00240123271942139 57.0
0 -0.0339316129684448 58.0
0 -0.0360850095748901 59.0
0 -0.162543892860413 60.0
0 0.0297168493270874 61.0
0 0.896277248859406 62.0
266.25209636 256.008575439453 0.0
0 -0.0528723001480103 1.0
0 -0.0532370805740356 2.0
0 -0.00886785984039307 3.0
0 0.0154587030410767 4.0
0 -0.0575982332229614 5.0
0 -0.0888692140579224 6.0
0 -0.0780140161514282 7.0
0 -0.0324392318725586 8.0
0 0.103624701499939 9.0
0 -0.0228201150894165 10.0
0 -0.0455005168914795 11.0
0 0.0823034048080444 12.0
71.704330614 70.634651184082 13.0
35.84333892 30.9639568328857 14.0
0 -0.0895572900772095 15.0
0 -0.068578839302063 16.0
42.220579872 42.3112335205078 17.0
0 0.085828423500061 18.0
0 0.102154850959778 19.0
0 -0.0217429399490356 20.0
0 -0.047216534614563 21.0
0 -0.0461932420730591 22.0
42.791119017 42.7025337219238 23.0
0 -0.0684982538223267 24.0
0 0.643864512443542 25.0
0 0.0621174573898315 26.0
0 0.0154432058334351 27.0
0 0.0311540365219116 28.0
0 -0.0224224328994751 29.0
0 -0.0459598302841187 30.0
0 -0.00432288646697998 31.0
0 0.0310739278793335 32.0
0 0.0736464262008667 33.0
0 0.008750319480896 34.0
0 -0.0328296422958374 35.0
0 0.13870108127594 36.0
0 -0.020389199256897 37.0
0 -0.00148952007293701 38.0
0 -0.00997364521026611 39.0
0 0.250258326530457 40.0
0 0.405206203460693 41.0
0 -0.127260804176331 42.0
282.79239019 291.50830078125 43.0
0 -0.0595270395278931 44.0
0 0.0368980765342712 45.0
0 -0.020856499671936 46.0
0 -0.00994408130645752 47.0
0 0.013637900352478 48.0
0.83089773726 0.255160868167877 49.0
0.10567670745 0.0643757581710815 50.0
2.1006475019 0.974481284618378 51.0
0 -0.0831271409988403 52.0
0 -0.0765930414199829 53.0
0 -0.00734031200408936 54.0
0 -0.110600829124451 55.0
0 0.00779211521148682 56.0
0 -0.0380264520645142 57.0
0 -0.0468460321426392 58.0
0 -0.0153573751449585 59.0
0 0.00385963916778564 60.0
0 0.0599857568740845 61.0
0 0.85133683681488 62.0
273.27337162 253.699203491211 0.0
0 -0.082955002784729 1.0
0 0.00597798824310303 2.0
0 0.00544106960296631 3.0
0 -0.0539950132369995 4.0
0 -0.115614056587219 5.0
0 -0.00247251987457275 6.0
0 -0.0418616533279419 7.0
0 0.0094459056854248 8.0
0 0.0871518850326538 9.0
0 0.00206005573272705 10.0
0 -0.0478783845901489 11.0
0 -0.0766226053237915 12.0
66.652338872 65.8834609985352 13.0
36.787301651 30.750186920166 14.0
0 -0.0239988565444946 15.0
0 0.00662052631378174 16.0
34.949145668 34.135066986084 17.0
0 0.0284394025802612 18.0
0 -0.00395882129669189 19.0
0 -0.0212951898574829 20.0
0 -0.0391675233840942 21.0
0 0.000323653221130371 22.0
28.85372796 27.8492527008057 23.0
0 -0.0147000551223755 24.0
0 0.54620748758316 25.0
0 0.0972155332565308 26.0
0 0.0248886346817017 27.0
0 -0.0256348848342896 28.0
0 -0.0357919931411743 29.0
0 -0.0749427080154419 30.0
0 -0.00477921962738037 31.0
0 0.00948750972747803 32.0
0 0.0209659337997437 33.0
0 -0.0589901208877563 34.0
0 0.00880622863769531 35.0
0 0.00575363636016846 36.0
0 -0.0829373598098755 37.0
0 0.0253840684890747 38.0
0 0.0425947904586792 39.0
0 -0.0379616022109985 40.0
0 -0.0878769159317017 41.0
0 0.0137847661972046 42.0
283.22351272 297.084930419922 43.0
0 -0.049878716468811 44.0
0 0.0242600440979004 45.0
0 -0.00791823863983154 46.0
0 -0.0797535181045532 47.0
0 -0.0287867784500122 48.0
0 0.310787498950958 49.0
0.076424670483 0.0638018846511841 50.0
2.2403455676 0.884015500545502 51.0
0 -0.0581201314926147 52.0
0 -0.00053250789642334 53.0
0 0.00920999050140381 54.0
0 -0.0745781660079956 55.0
0 -0.0545800924301147 56.0
0 -0.0142375230789185 57.0
0 -0.068078875541687 58.0
0 -0.0528501272201538 59.0
0 0.0231248140335083 60.0
0 0.0593429803848267 61.0
0 0.891967535018921 62.0
240.19196184 258.5517578125 0.0
0 -0.117092251777649 1.0
0 -0.0305465459823608 2.0
0 0.0101929903030396 3.0
0 -0.0322376489639282 4.0
0 -0.0415052175521851 5.0
0 0.104963660240173 6.0
0 -0.0831722021102905 7.0
0 0.0422184467315674 8.0
0 0.00541245937347412 9.0
0 -0.0338236093521118 10.0
0 0.0914725065231323 11.0
0 0.0149377584457397 12.0
68.300996291 67.4522171020508 13.0
24.313683288 33.5243606567383 14.0
0 -0.178400635719299 15.0
0 0.0407098531723022 16.0
37.184641099 36.6195793151855 17.0
0 -0.0487061738967896 18.0
0 -0.0316656827926636 19.0
0 -0.00570666790008545 20.0
0 -0.0507215261459351 21.0
0 -0.0195666551589966 22.0
37.381731222 35.6986808776855 23.0
0 -0.12327778339386 24.0
0 2.47745752334595 25.0
0 -0.0203191041946411 26.0
0 0.0601335763931274 27.0
0 -0.0132743120193481 28.0
0 -0.0412729978561401 29.0
0 -0.0772331953048706 30.0
0 -0.0319439172744751 31.0
0 0.112136244773865 32.0
0 0.0310086011886597 33.0
0 -0.0406056642532349 34.0
0 -0.0799533128738403 35.0
0 0.0720807313919067 36.0
0 -0.01484215259552 37.0
0 -0.0564688444137573 38.0
0 -0.0364142656326294 39.0
0 -0.206366181373596 40.0
0 -0.0316559076309204 41.0
0 0.0316575765609741 42.0
284.07177981 295.113098144531 43.0
0 0.0486549139022827 44.0
0 0.131866991519928 45.0
0 -0.0170279741287231 46.0
0 -0.0237849950790405 47.0
0 0.0421367883682251 48.0
0 0.253625631332397 49.0
0.083536194658 0.174939036369324 50.0
1.4643799391 0.887193202972412 51.0
0 -0.0749977827072144 52.0
0 0.0117408037185669 53.0
0 -0.0312165021896362 54.0
0 -0.111464381217957 55.0
0 0.138309597969055 56.0
0 0.0469011068344116 57.0
0 -0.0808261632919312 58.0
0 -0.0685232877731323 59.0
0 -0.112661242485046 60.0
0 -0.0635675191879272 61.0
0 0.795415997505188 62.0
279.91525713 259.189636230469 0.0
0 -0.0643097162246704 1.0
0 0.0357645750045776 2.0
0 0.00800788402557373 3.0
0 -0.0571247339248657 4.0
0 0.0261744260787964 5.0
0 0.0208309888839722 6.0
0 -0.131521344184875 7.0
0 -0.0279726982116699 8.0
0 0.00348317623138428 9.0
0 0.0228714942932129 10.0
0 -0.0323978662490845 11.0
0 -0.0311418771743774 12.0
61.099661207 60.7948150634766 13.0
44.151335023 36.0447196960449 14.0
0 -0.0479961633682251 15.0
0 -0.0412851572036743 16.0
29.401995711 29.3011341094971 17.0
0 -0.0134938955307007 18.0
0 0.00561511516571045 19.0
0 -0.00705397129058838 20.0
0 -0.0191823244094849 21.0
0 -0.0113188028335571 22.0
31.183532442 30.5001811981201 23.0
0 -0.0823451280593872 24.0
16.127765537 0.645384073257446 25.0
0 0.117225527763367 26.0
0 0.0539780855178833 27.0
0 0.0669173002243042 28.0
0 -0.0750755071640015 29.0
0 -0.0738430023193359 30.0
0 -0.0103319883346558 31.0
0 0.0186208486557007 32.0
0 0.0418103933334351 33.0
0 -0.0124162435531616 34.0
0 -0.0176197290420532 35.0
0 0.0832523107528687 36.0
0 -0.0580881834030151 37.0
0 -0.0129855871200562 38.0
0 0.029598593711853 39.0
0 0.0334481000900269 40.0
0 -0.021068811416626 41.0
0 0.0407966375350952 42.0
297.37118817 296.8671875 43.0
0 0.0541356801986694 44.0
0 -0.0634058713912964 45.0
0 -0.0217236280441284 46.0
0 -0.0568796396255493 47.0
0 0.0370463132858276 48.0
0 0.238167643547058 49.0
0.06952100797 0.0588287115097046 50.0
1.6299858697 0.846112132072449 51.0
0 -0.0703877210617065 52.0
0 -0.0431033372879028 53.0
0 0.0122197866439819 54.0
0 -0.0708664655685425 55.0
0 -0.0418235063552856 56.0
0 -0.0252610445022583 57.0
0 -0.0604618787765503 58.0
0 -0.0448154211044312 59.0
0 -0.0464118719100952 60.0
0 0.0260390043258667 61.0
0 0.803323745727539 62.0
299.84376297 248.842651367188 0.0
0 0.0301889181137085 1.0
0 0.010607123374939 2.0
0 -0.0066598653793335 3.0
0 -0.041695237159729 4.0
0 -0.0454994440078735 5.0
0 -0.159208416938782 6.0
0 -0.069064736366272 7.0
0 -0.996725559234619 8.0
0 0.0122207403182983 9.0
0 -0.369226336479187 10.0
0 -0.025049090385437 11.0
0 0.0688575506210327 12.0
72.077804465 70.7562484741211 13.0
34.644952896 30.4997615814209 14.0
0 -0.0610305070877075 15.0
0 -0.0838390588760376 16.0
24.921083892 24.2190265655518 17.0
0 0.0804153680801392 18.0
0 0.0051809549331665 19.0
0 -0.0586599111557007 20.0
0 0.0388931035995483 21.0
0 -0.0569645166397095 22.0
25.851399123 25.9132690429688 23.0
0 0.0733774900436401 24.0
0 0.887797594070435 25.0
0 0.0660122632980347 26.0
0 0.00102603435516357 27.0
0 -0.0107032060623169 28.0
0 -0.0454941987991333 29.0
0 -0.065642237663269 30.0
0 0.026983380317688 31.0
0 -0.0967522859573364 32.0
0 -0.0266855955123901 33.0
0 -0.13354480266571 34.0
0 -0.0374907255172729 35.0
0 0.308569550514221 36.0
0 0.00683176517486572 37.0
0 0.0378206968307495 38.0
0 -0.234239935874939 39.0
0 -0.0456281900405884 40.0
0 -0.0206917524337769 41.0
0 -0.0233703851699829 42.0
302.29095183 297.156646728516 43.0
0 0.0545042753219604 44.0
0 -0.0117521286010742 45.0
0 -0.0448757410049438 46.0
0 0.0252867937088013 47.0
0 0.0299299955368042 48.0
0 0.273198962211609 49.0
0.10358520651 0.197888970375061 50.0
1.7617632978 0.811967372894287 51.0
0 -0.085509181022644 52.0
0 -0.045451283454895 53.0
0 -0.0190469026565552 54.0
0 -0.0773504972457886 55.0
0 -0.12327778339386 56.0
0 0.0129364728927612 57.0
0 0.00444614887237549 58.0
0 -0.0294455289840698 59.0
0 0.0345975160598755 60.0
0 -0.00652706623077393 61.0
0 0.919109582901001 62.0
256.63018531 254.970306396484 0.0
0 0.0150998830795288 1.0
0 -0.0352779626846313 2.0
0 -0.0226794481277466 3.0
0 -0.0870276689529419 4.0
0 -0.0158628225326538 5.0
0 -0.0563336610794067 6.0
0 0.00262820720672607 7.0
0 -0.00632345676422119 8.0
0 0.0990737676620483 9.0
0 -0.0167151689529419 10.0
0 -0.0270758867263794 11.0
0 -0.0413748025894165 12.0
56.862618449 57.4268798828125 13.0
36.920438112 34.2631721496582 14.0
0 0.0243715047836304 15.0
0 0.010658860206604 16.0
31.667790177 31.9697494506836 17.0
0 0.0271080732345581 18.0
0 0.0266591310501099 19.0
0 -0.00438797473907471 20.0
0 -0.012055516242981 21.0
0 -0.0167967081069946 22.0
37.712980342 36.8376846313477 23.0
0 -0.0398560762405396 24.0
0 0.503219127655029 25.0
0 0.0552190542221069 26.0
0 -0.0424050092697144 27.0
0 -0.0803147554397583 28.0
0 -0.011933445930481 29.0
0 -0.0992511510848999 30.0
0 -0.0570636987686157 31.0
0 -0.0230971574783325 32.0
0 0.0537697076797485 33.0
0 -0.0282570123672485 34.0
0 -0.0693775415420532 35.0
0 -0.0319696664810181 36.0
0 -0.0359350442886353 37.0
0 -0.0295027494430542 38.0
0 0.0132795572280884 39.0
0 0.0237306356430054 40.0
0 0.0567642450332642 41.0
0 -0.0171667337417603 42.0
290.44649166 296.413421630859 43.0
0 0.0880016088485718 44.0
0 -0.0722522735595703 45.0
0 -0.0354858636856079 46.0
0 -0.021215558052063 47.0
0 -0.0349198579788208 48.0
0 0.26715749502182 49.0
0.095250515062 0.0673222541809082 50.0
1.7892221208 0.879132866859436 51.0
0 -0.110780835151672 52.0
0 -0.0508724451065063 53.0
0 -0.0411728620529175 54.0
0 -0.044859766960144 55.0
0 -0.0997167825698853 56.0
0 0.0130661725997925 57.0
0 -0.0612858533859253 58.0
0 -0.0607939958572388 59.0
0 0.0421301126480103 60.0
0 0.00463187694549561 61.0
0 0.809286594390869 62.0
243.62646505 259.055267333984 0.0
0 -0.0770994424819946 1.0
0 -0.00623214244842529 2.0
0 -0.0151277780532837 3.0
0 -0.0627831220626831 4.0
0 -0.0416089296340942 5.0
0 -0.134012341499329 6.0
0 -0.0928813219070435 7.0
0 -0.0257714986801147 8.0
0 -0.0293194055557251 9.0
0 -0.0304070711135864 10.0
0 0.015685498714447 11.0
0 0.114735960960388 12.0
69.219982648 68.3299179077148 13.0
22.810244942 30.0466346740723 14.0
0 -0.0660902261734009 15.0
0 -0.0636576414108276 16.0
31.60101736 31.6661148071289 17.0
0 0.0322850942611694 18.0
0 -0.0161479711532593 19.0
0 -0.0155805349349976 20.0
0 -0.00901138782501221 21.0
0 -0.0377579927444458 22.0
39.728784695 39.5787315368652 23.0
0 -0.0410720109939575 24.0
0 0.546428263187408 25.0
0 0.0468662977218628 26.0
0 0.0440109968185425 27.0
0 -0.00404036045074463 28.0
0 -0.0450202226638794 29.0
0 -0.0567178726196289 30.0
0 -0.00943052768707275 31.0
0 0.0376499891281128 32.0
0 -0.0242668390274048 33.0
0 0.0175665616989136 34.0
0 -0.0340896844863892 35.0
0 0.0137518644332886 36.0
0 -0.094967246055603 37.0
0 -0.048427939414978 38.0
0 0.0432585477828979 39.0
0 0.144267439842224 40.0
0 -0.0395493507385254 41.0
0 -0.0587173700332642 42.0
291.62949503 297.003692626953 43.0
0 -0.0179105997085571 44.0
0 -0.0378768444061279 45.0
0 -0.0225712060928345 46.0
0 -0.0142861604690552 47.0
0 0.0779598951339722 48.0
0 0.421635121107101 49.0
0.095931872684 0.0551313161849976 50.0
2.2548959947 0.939840376377106 51.0
0 -0.0564268827438354 52.0
0 -0.0373722314834595 53.0
0 0.0653716325759888 54.0
0 -0.103065371513367 55.0
0 0.0172911882400513 56.0
0 -0.0261985063552856 57.0
0 -0.0486863851547241 58.0
0 -0.0128544569015503 59.0
0 -0.0165451765060425 60.0
0 -0.0216645002365112 61.0
0 0.84849488735199 62.0
289.97585618 255.15217590332 0.0
0 -0.00625479221343994 1.0
0 -0.162082076072693 2.0
0 -0.0021437406539917 3.0
0 -0.0223227739334106 4.0
0 -0.0165396928787231 5.0
0 -0.134614109992981 6.0
0 -0.0323091745376587 7.0
0 0.0053364634513855 8.0
0 -0.0780512094497681 9.0
0 0.0191754102706909 10.0
0 0.0183228254318237 11.0
0 0.030535101890564 12.0
64.908904284 64.396240234375 13.0
21.872592549 34.4274978637695 14.0
0 0.00216042995452881 15.0
0 -0.0398565530776978 16.0
25.787085088 25.7763805389404 17.0
0 -0.0363062620162964 18.0
0 -0.0135122537612915 19.0
0 0.00260555744171143 20.0
0 -0.0215061902999878 21.0
0 -0.0400973558425903 22.0
39.69505896 38.3736534118652 23.0
0 0.109396815299988 24.0
0 0.398540139198303 25.0
0 0.023131251335144 26.0
0 0.0394346714019775 27.0
0 -0.0573903322219849 28.0
0 -0.0551060438156128 29.0
0 -0.0177685022354126 30.0
0 -0.00183951854705811 31.0
0 0.0732063055038452 32.0
0 0.109634280204773 33.0
0 -0.00185143947601318 34.0
0 0.0625009536743164 35.0
0 -0.0293632745742798 36.0
0 0.0337637662887573 37.0
0 -0.0121399164199829 38.0
0 -0.0392371416091919 39.0
0 -0.0978914499282837 40.0
0 0.0242537260055542 41.0
0 -0.146795868873596 42.0
294.42834998 295.59716796875 43.0
0 0.117331385612488 44.0
0 -0.119089484214783 45.0
0 -0.0683742761611938 46.0
0 -0.0150077342987061 47.0
0 -0.0385357141494751 48.0
1.6124177994 0.273939073085785 49.0
0.087749272705 0.0776553153991699 50.0
2.0730645242 0.864301323890686 51.0
0 -0.0697797536849976 52.0
0 0.0112887620925903 53.0
0 -0.066336989402771 54.0
0 -0.0387581586837769 55.0
0 -0.00592291355133057 56.0
0 -0.00600016117095947 57.0
0 0.0125752687454224 58.0
0 -0.000221371650695801 59.0
0 -0.0366145372390747 60.0
0 0.0304385423660278 61.0
0 0.809562206268311 62.0
293.12166541 253.203857421875 0.0
0 0.00728023052215576 1.0
0 -0.07935631275177 2.0
0 -0.022229790687561 3.0
0 -0.0312131643295288 4.0
0 -0.0485583543777466 5.0
0 0.0935744047164917 6.0
0 -0.0046621561050415 7.0
0 -0.0411191582679749 8.0
0 0.0784085988998413 9.0
0 -0.0275754928588867 10.0
0 -0.0201857089996338 11.0
0 0.0415729284286499 12.0
59.574156399 58.0430107116699 13.0
36.741091148 35.7913703918457 14.0
0 -0.100158333778381 15.0
0 -0.0549708604812622 16.0
34.454393456 33.6180686950684 17.0
0 -0.00539767742156982 18.0
0 -0.00427782535552979 19.0
0 -0.0115286111831665 20.0
0 0.00139892101287842 21.0
0 0.0101221799850464 22.0
28.591421896 28.0179595947266 23.0
0 -0.0665432214736938 24.0
0 0.538200795650482 25.0
0 0.0895828008651733 26.0
0 0.00510430335998535 27.0
0 0.0479139089584351 28.0
0 -0.088614821434021 29.0
0 -0.0417859554290771 30.0
0 -0.0144418478012085 31.0
0 0.025297999382019 32.0
0 0.0736645460128784 33.0
0 -0.020633339881897 34.0
0 -0.0549825429916382 35.0
0 0.00712788105010986 36.0
0 -0.0902923345565796 37.0
0 0.0109504461288452 38.0
0 0.00316011905670166 39.0
0 -0.0690990686416626 40.0
0 -0.0240945816040039 41.0
0 -0.0377355813980103 42.0
313.48518079 296.705688476562 43.0
0 0.019832968711853 44.0
0 -0.00184661149978638 45.0
0 -0.0263608694076538 46.0
0 -0.0499147176742554 47.0
0 -0.0868650674819946 48.0
0.93406215243 0.299879968166351 49.0
0.079948515492 0.0850616693496704 50.0
1.9847203711 0.936159431934357 51.0
0 -0.0613590478897095 52.0
0 -0.0592581033706665 53.0
0 -0.0108767747879028 54.0
0 -0.00487673282623291 55.0
0 -0.0388456583023071 56.0
0 -0.0297659635543823 57.0
0 -0.0221091508865356 58.0
0 -0.0743194818496704 59.0
0 -0.00471603870391846 60.0
0 0.0837353467941284 61.0
0 0.895139336585999 62.0
226.42842874 255.443618774414 0.0
0 0.00887906551361084 1.0
0 -0.0297020673751831 2.0
0 -0.0278452634811401 3.0
0 -0.0514024496078491 4.0
0 -0.127814888954163 5.0
0 0.00110852718353271 6.0
0 -0.157995700836182 7.0
0 0.0558363199234009 8.0
0 -0.0122796297073364 9.0
0 -0.0152469873428345 10.0
0 -0.0148776769638062 11.0
0 -0.0790954828262329 12.0
57.420917904 57.418342590332 13.0
31.106988255 38.702995300293 14.0
0 -0.0120445489883423 15.0
0 -0.0261865854263306 16.0
30.423663646 30.2688941955566 17.0
0 -0.0289832353591919 18.0
0 -0.0040658712387085 19.0
0 -0.00888049602508545 20.0
0 -0.0286320447921753 21.0
0 -0.0509353876113892 22.0
28.300460891 27.7634410858154 23.0
0 0.0570410490036011 24.0
0 0.500203788280487 25.0
0 0.093401312828064 26.0
0 0.0504883527755737 27.0
0 0.143236517906189 28.0
0 -0.0416284799575806 29.0
0 -0.0641344785690308 30.0
0 -0.0456510782241821 31.0
0 -0.0106316804885864 32.0
0 -0.00829613208770752 33.0
0 -0.0511437654495239 34.0
0 -0.0301576852798462 35.0
0 0.125055193901062 36.0
0 -0.0304442644119263 37.0
0 0.0111681222915649 38.0
0 0.0419679880142212 39.0
0 0.173232913017273 40.0
0 -0.00410735607147217 41.0
0 0.0357040166854858 42.0
306.42735889 296.806427001953 43.0
0 0.0138581991195679 44.0
0 0.0727173686027527 45.0
0 -0.0222669839859009 46.0
0 0.0311386585235596 47.0
0 -0.0481516122817993 48.0
0.14862723358 0.337533235549927 49.0
0.092112695056 0.028450608253479 50.0
1.4814205284 0.93499892950058 51.0
0 -0.0979846715927124 52.0
0 -0.0515996217727661 53.0
0 0.0200601816177368 54.0
0 -0.0516551733016968 55.0
0 -0.0453323125839233 56.0
0 -0.0427695512771606 57.0
0 0.0061107873916626 58.0
0 0.00239408016204834 59.0
0 0.0533376932144165 60.0
0 -0.0468012094497681 61.0
0 0.839223504066467 62.0
284.91632861 256.181976318359 0.0
0 -0.113191246986389 1.0
0 0.0298417806625366 2.0
0 -0.0165601968765259 3.0
0 -0.0718539953231812 4.0
0 -0.018262505531311 5.0
0 0.101800322532654 6.0
0 0.0484627485275269 7.0
0 -0.0272467136383057 8.0
0 0.0638307332992554 9.0
0 -0.120042681694031 10.0
0 -0.0429513454437256 11.0
0 0.0380691289901733 12.0
54.934044737 54.060131072998 13.0
37.188702488 34.5100059509277 14.0
0 0.0693835020065308 15.0
0 0.0210450887680054 16.0
28.338153492 28.128002166748 17.0
0 0.00841629505157471 18.0
0 -0.0157047510147095 19.0
0 0.0452560186386108 20.0
0 -0.0607978105545044 21.0
0 -0.0138055086135864 22.0
38.643895006 37.4596290588379 23.0
0 -0.0149050951004028 24.0
0 0.538165271282196 25.0
0 -0.0459448099136353 26.0
0 0.0556133985519409 27.0
0 -0.0315619707107544 28.0
0 -0.0505038499832153 29.0
0 -0.0196539163589478 30.0
0 -0.0198382139205933 31.0
0 -0.00549232959747314 32.0
0 0.060484766960144 33.0
0 -0.0325831174850464 34.0
0 -0.0251621007919312 35.0
0 0.0296398401260376 36.0
0 -0.0597940683364868 37.0
0 0.00568187236785889 38.0
0 0.017612099647522 39.0
0 0.114470362663269 40.0
0 -0.0329095125198364 41.0
0 -0.0518537759780884 42.0
283.81026563 294.245910644531 43.0
0 -0.0216575860977173 44.0
0 -0.0155385732650757 45.0
0 -0.0661064386367798 46.0
0 -0.0246835947036743 47.0
0 -0.00139486789703369 48.0
0 0.250506579875946 49.0
0.098263669205 0.215486407279968 50.0
2.1297101723 0.912007391452789 51.0
0 -0.0738221406936646 52.0
0 -0.0568982362747192 53.0
0 0.0606449842453003 54.0
0 -0.145756840705872 55.0
0 -0.0119818449020386 56.0
0 -0.0462714433670044 57.0
0 -0.0547014474868774 58.0
0 -0.0578645467758179 59.0
0 0.141300559043884 60.0
0 0.061853289604187 61.0
0 0.907151162624359 62.0
267.16641274 253.273651123047 0.0
0 -0.044592022895813 1.0
0 0.00832545757293701 2.0
0 0.00911009311676025 3.0
0 -0.0804277658462524 4.0
0 0.00773847103118896 5.0
0 -0.00574314594268799 6.0
0 -0.0031660795211792 7.0
0 -0.0349997282028198 8.0
0 0.0240501165390015 9.0
0 -0.077318549156189 10.0
0 0.0126291513442993 11.0
0 0.0302833318710327 12.0
73.622815665 72.8218231201172 13.0
12.585848687 23.5954647064209 14.0
0 0.0168837308883667 15.0
0 -0.0742472410202026 16.0
28.171901681 28.006814956665 17.0
0 0.059670090675354 18.0
0 0.00318849086761475 19.0
0 0.00496900081634521 20.0
0 -0.0311967134475708 21.0
0 -0.0238105058670044 22.0
35.519281615 34.444221496582 23.0
0 -0.0644479990005493 24.0
0 0.647244095802307 25.0
0 0.0208224058151245 26.0
0 -0.015809178352356 27.0
0 0.0677040815353394 28.0
0 -0.0405119657516479 29.0
0 -0.0733679533004761 30.0
0 -0.044216513633728 31.0
0 -0.00239241123199463 32.0
0 0.0774415731430054 33.0
0 0.00109851360321045 34.0
0 -0.0791908502578735 35.0
0 0.110922932624817 36.0
0 -0.00546491146087646 37.0
0 -0.0238384008407593 38.0
0 0.0422872304916382 39.0
0 0.0981868505477905 40.0
0 -0.0100574493408203 41.0
0 0.0523933172225952 42.0
299.72437057 296.033020019531 43.0
0 0.0337728261947632 44.0
0 0.0501973628997803 45.0
0 -0.0401862859725952 46.0
0 0.00484836101531982 47.0
0 -0.0155731439590454 48.0
0 0.289196193218231 49.0
0.06770095198 0.115407347679138 50.0
2.1275261575 0.882575631141663 51.0
0 -0.0694423913955688 52.0
0 -0.0643078088760376 53.0
0 -0.021527886390686 54.0
0 -0.0643855333328247 55.0
0 0.0248552560806274 56.0
0 -0.0457445383071899 57.0
0 0.00434744358062744 58.0
0 -0.0448287725448608 59.0
0 -0.0947495698928833 60.0
0 -0.0593854188919067 61.0
0 0.828004837036133 62.0
255.56707141 261.476837158203 0.0
0 -0.0275341272354126 1.0
0 -0.053855299949646 2.0
0 -0.0654975175857544 3.0
0 -0.063829779624939 4.0
0 0.0287002325057983 5.0
0 -0.00162899494171143 6.0
0 -0.0262209177017212 7.0
0 0.00968050956726074 8.0
0 0.012412428855896 9.0
0 0.0159924030303955 10.0
0 -0.0362981557846069 11.0
0 0.113184332847595 12.0
54.39590443 54.2331771850586 13.0
33.204580577 43.3961601257324 14.0
0 -0.0703684091567993 15.0
0 0.038542628288269 16.0
33.145094906 33.6490478515625 17.0
0 -0.0280090570449829 18.0
0 0.0453401803970337 19.0
0 -0.00311911106109619 20.0
0 -0.0188905000686646 21.0
0 -0.00596821308135986 22.0
33.915489764 32.8308906555176 23.0
0 0.0646437406539917 24.0
0 0.486566841602325 25.0
0 0.00390565395355225 26.0
0 0.01195228099823 27.0
0 -0.046204686164856 28.0
0 -0.0102850198745728 29.0
0 -0.128933787345886 30.0
0 0.0184932947158813 31.0
0 0.0271652936935425 32.0
0 -0.0166887044906616 33.0
0 -0.0230518579483032 34.0
0 -0.101742625236511 35.0
0 0.0386303663253784 36.0
0 -0.0262864828109741 37.0
0 -0.0305513143539429 38.0
0 0.0646010637283325 39.0
0 0.087706446647644 40.0
0 -0.0937432050704956 41.0
0 0.0354055166244507 42.0
303.20201703 295.27490234375 43.0
0 -0.0198055505752563 44.0
0 0.0289138555526733 45.0
0 -0.050998330116272 46.0
0 -0.0997222661972046 47.0
0 0.0422052145004272 48.0
0 0.344011783599854 49.0
0.090061249676 0.061968207359314 50.0
1.7849062669 0.99649178981781 51.0
0 -0.0539335012435913 52.0
0 -0.0514041185379028 53.0
0 0.0219484567642212 54.0
0 -0.0999361276626587 55.0
0 0.020983099937439 56.0
0 -0.0544663667678833 57.0
0 -0.0399407148361206 58.0
0 -0.00789773464202881 59.0
0 0.0833278894424438 60.0
0 -0.0419613122940063 61.0
0 0.813862919807434 62.0
249.90293714 256.368957519531 0.0
0 -0.0319501161575317 1.0
0 -0.0495404005050659 2.0
0 0.025208592414856 3.0
0 -0.0521072149276733 4.0
0 -0.0144003629684448 5.0
0 -0.0160824060440063 6.0
0 0.00941383838653564 7.0
0 -0.054957389831543 8.0
0 0.0488628149032593 9.0
0 -0.00898110866546631 10.0
0 -0.00957286357879639 11.0
0 0.022762656211853 12.0
52.495871358 51.8560256958008 13.0
40.528860485 41.277400970459 14.0
0 -0.0318175554275513 15.0
0 -0.0587207078933716 16.0
36.52351367 35.968692779541 17.0
0 0.00328266620635986 18.0
0 -0.0295947790145874 19.0
0 0.0208474397659302 20.0
0 -0.0113712549209595 21.0
0 -0.0405818223953247 22.0
38.16165703 37.1334609985352 23.0
0 -0.0151258707046509 24.0
0 0.541183173656464 25.0
0 0.0347250699996948 26.0
0 0.0547863245010376 27.0
0 -0.00674545764923096 28.0
0 -0.0632086992263794 29.0
0 -0.0823128223419189 30.0
0 -0.0424233675003052 31.0
0 0.024107813835144 32.0
0 -0.00818073749542236 33.0
0 -0.0268875360488892 34.0
0 -0.0933047533035278 35.0
0 0.0613139867782593 36.0
0 0.00302731990814209 37.0
0 -0.0567203760147095 38.0
0 0.0552617311477661 39.0
0 0.119352698326111 40.0
0 -0.0433787107467651 41.0
0 -0.0076519250869751 42.0
291.58404701 294.950103759766 43.0
0 -0.00742685794830322 44.0
0 0.00149476528167725 45.0
0 0.00956237316131592 46.0
0 -0.0831140279769897 47.0
0 0.0330713987350464 48.0
0 0.315969169139862 49.0
0.072260930937 0.0587774515151978 50.0
2.02134151 1.00700664520264 51.0
0 -0.0519169569015503 52.0
0 -0.0609918832778931 53.0
0 0.00314819812774658 54.0
0 -0.106447577476501 55.0
0 -0.0816746950149536 56.0
0 0.00957667827606201 57.0
0 -0.0562032461166382 58.0
0 -0.0139657258987427 59.0
0 0.0143440961837769 60.0
0 0.0324255228042603 61.0
0 0.843194484710693 62.0
246.47853885 256.876403808594 0.0
0 0.00289428234100342 1.0
0 -0.0738698244094849 2.0
0 -0.018189549446106 3.0
0 0.00984799861907959 4.0
0 0.0405465364456177 5.0
0 0.0261272192001343 6.0
0 0.0253692865371704 7.0
0 -0.0575176477432251 8.0
0 -0.000812172889709473 9.0
0 -0.04883873462677 10.0
0 -0.0513391494750977 11.0
0 -0.0151878595352173 12.0
55.153480377 53.6465263366699 13.0
39.001223103 37.7599678039551 14.0
0 -0.0630046129226685 15.0
0 -0.0224472284317017 16.0
29.890045797 29.898063659668 17.0
0 0.00784766674041748 18.0
0 -0.0379079580307007 19.0
0 0.0566805601119995 20.0
0 -0.00275170803070068 21.0
0 -0.0364514589309692 22.0
35.594433896 35.0987548828125 23.0
0 -0.108038544654846 24.0
0 0.590648412704468 25.0
0 0.0495284795761108 26.0
0 0.123308062553406 27.0
0 -0.0712481737136841 28.0
0 -0.0798114538192749 29.0
0 -0.0824623107910156 30.0
0 -0.0592876672744751 31.0
0 0.0296815633773804 32.0
0 0.0293096303939819 33.0
0 -0.0319637060165405 34.0
0 -0.0986613035202026 35.0
0 0.0227509737014771 36.0
0 -0.0190075635910034 37.0
0 0.00161874294281006 38.0
0 0.055420994758606 39.0
0 0.159361720085144 40.0
0 -0.0675760507583618 41.0
0 -0.0543466806411743 42.0
308.96250714 295.893615722656 43.0
0 -0.016327977180481 44.0
0 0.0159111022949219 45.0
0 -0.00388920307159424 46.0
0 -0.0420686006546021 47.0
0 0.0348414182662964 48.0
0 0.301136791706085 49.0
0.076557151485 0.0631464719772339 50.0
2.1349900561 0.951079130172729 51.0
0 -0.0614217519760132 52.0
0 -0.0255774259567261 53.0
0 0.0717440843582153 54.0
0 -0.0852624177932739 55.0
0 0.00994408130645752 56.0
0 -0.00121438503265381 57.0
0 -0.0831302404403687 58.0
0 -0.0332368612289429 59.0
0 0.0556894540786743 60.0
0 0.0724438428878784 61.0
0 0.903861224651337 62.0
246.27384936 250.424942016602 0.0
0 -0.0778614282608032 1.0
0 0.0669358968734741 2.0
0 -0.0100070238113403 3.0
0 -0.0336576700210571 4.0
0 -0.0195742845535278 5.0
0 -0.511479020118713 6.0
0 -0.14878261089325 7.0
0 -0.0306349992752075 8.0
0 0.0438960790634155 9.0
0 -0.0379692316055298 10.0
0 0.0821632146835327 11.0
0 0.0823053121566772 12.0
67.764320873 66.6309967041016 13.0
36.06674367 35.6157341003418 14.0
0 -0.0128790140151978 15.0
0 -0.0174242258071899 16.0
28.879855975 29.2889404296875 17.0
0 0.0681954622268677 18.0
0 -0.0470956563949585 19.0
0 0.0268567800521851 20.0
0 -0.0862294435501099 21.0
0 -0.0102745294570923 22.0
33.674342195 33.020679473877 23.0
0 0.0795248746871948 24.0
29.904366056 0.460075497627258 25.0
0 0.0587171316146851 26.0
0 -0.0527759790420532 27.0
0 0.143064856529236 28.0
0 -0.131913304328918 29.0
0 0.162092328071594 30.0
0 -0.00493466854095459 31.0
0 -0.00456297397613525 32.0
0 0.0222064256668091 33.0
0 -0.0157622098922729 34.0
0 -0.00489652156829834 35.0
0 -0.117859959602356 36.0
0 0.00519979000091553 37.0
0 -0.0611635446548462 38.0
0 0.0712438821792603 39.0
0 0.00235021114349365 40.0
0 -0.0316376686096191 41.0
0 -0.0241304636001587 42.0
279.49976993 297.178863525391 43.0
0 -0.0767217874526978 44.0
0 0.00734579563140869 45.0
0 -0.00363337993621826 46.0
0 -0.021844744682312 47.0
0 0.0344866514205933 48.0
0 0.252477824687958 49.0
0.074540763721 0.194666981697083 50.0
1.9977340937 0.864042162895203 51.0
0 -0.0873452425003052 52.0
0 0.0246762037277222 53.0
0 -0.00315845012664795 54.0
0 -0.067867636680603 55.0
0 -0.0636783838272095 56.0
0 0.0192192792892456 57.0
0 -0.0399690866470337 58.0
0 -0.0551005601882935 59.0
0 0.0416978597640991 60.0
0 0.0258883237838745 61.0
0 0.786144495010376 62.0
223.26609178 261.882659912109 0.0
0 -0.266867756843567 1.0
0 -0.0549970865249634 2.0
0 0.0420068502426147 3.0
0 -0.0536030530929565 4.0
0 -0.00807344913482666 5.0
0 -0.0872384309768677 6.0
0 -0.125654101371765 7.0
0 -0.116293787956238 8.0
0 -0.0518175363540649 9.0
0 0.0398054122924805 10.0
0 -0.0259634256362915 11.0
0 0.0660985708236694 12.0
54.585999735 53.9367523193359 13.0
51.400147099 40.1301040649414 14.0
0 -0.00498449802398682 15.0
0 0.00550878047943115 16.0
41.964693888 41.9099922180176 17.0
0 -0.0825947523117065 18.0
0 -0.0577861070632935 19.0
0 0.0418511629104614 20.0
0 0.0217605829238892 21.0
0 0.000298857688903809 22.0
33.234306113 32.3907012939453 23.0
0 0.043815016746521 24.0
0 0.529991745948792 25.0
0 0.0224403142929077 26.0
0 0.171601891517639 27.0
0 -0.0820451974868774 28.0
0 -0.0491582155227661 29.0
0 0.0775359869003296 30.0
0 -0.050775408744812 31.0
0 0.0210665464401245 32.0
0 0.0654817819595337 33.0
0 -0.0230475664138794 34.0
0 0.0784069299697876 35.0
0 0.128813624382019 36.0
0 -0.0886057615280151 37.0
0 0.034852147102356 38.0
0 0.139043688774109 39.0
0 0.117749094963074 40.0
0 -0.0499162673950195 41.0
0 -0.088849663734436 42.0
304.85191717 296.556335449219 43.0
0 -0.071694016456604 44.0
0 0.0329656600952148 45.0
0 -0.0307174921035767 46.0
0 0.0780407190322876 47.0
0 -0.0142661333084106 48.0
2.0246724542 0.36624613404274 49.0
0.078205828447 0.0545598268508911 50.0
1.7566524133 0.834974050521851 51.0
0 -0.06556236743927 52.0
0 -0.0250808000564575 53.0
0 -0.0432990789413452 54.0
0 -0.0354920625686646 55.0
0 -0.102473855018616 56.0
0 0.0237997770309448 57.0
0 -0.0020216703414917 58.0
0 -0.00194084644317627 59.0
0 -0.127440094947815 60.0
0 0.0641182661056519 61.0
0 0.986886858940125 62.0
253.78856919 256.268005371094 0.0
0 -0.0427414178848267 1.0
0 -0.0767220258712769 2.0
0 0.00978648662567139 3.0
0 -0.0386329889297485 4.0
0 -0.12775456905365 5.0
0 -0.0198255777359009 6.0
0 -0.109013199806213 7.0
0 -0.0450946688652039 8.0
0 -0.019748330116272 9.0
0 -0.00122106075286865 10.0
0 -0.0302438735961914 11.0
0 -0.0900331735610962 12.0
58.856230718 58.1942443847656 13.0
46.66904074 38.1323204040527 14.0
0 -0.0329757928848267 15.0
0 -0.0257002115249634 16.0
28.581460582 28.3505191802979 17.0
0 0.0352483987808228 18.0
0 0.0132647752761841 19.0
0 -0.0415142774581909 20.0
0 0.0200208425521851 21.0
0 0.0318969488143921 22.0
30.850875241 29.8538017272949 23.0
0 0.0312341451644897 24.0
0 0.520775198936462 25.0
0 0.069818377494812 26.0
0 -0.0269478559494019 27.0
0 -0.079053521156311 28.0
0 -0.0496295690536499 29.0
0 -0.00768792629241943 30.0
0 -0.0538688898086548 31.0
0 -0.0172215700149536 32.0
0 0.0680867433547974 33.0
0 -0.0272589921951294 34.0
0 -0.0425175428390503 35.0
0 0.111275792121887 36.0
0 -0.00788366794586182 37.0
0 -0.0539077520370483 38.0
0 0.0267988443374634 39.0
0 0.0795081853866577 40.0
0 -0.0561556816101074 41.0
0 0.0567325353622437 42.0
286.17129642 296.356140136719 43.0
0 0.0109838247299194 44.0
0 0.0664446353912354 45.0
0 0.0174888372421265 46.0
0 -0.0130232572555542 47.0
0 0.00515830516815186 48.0
1.3629653983 0.322335839271545 49.0
0.076665818564 0.0578943490982056 50.0
2.0756467539 1.00028991699219 51.0
0 -0.0440787076950073 52.0
0 -0.0366061925888062 53.0
0 0.0111955404281616 54.0
0 -0.0913630723953247 55.0
0 -0.0081857442855835 56.0
0 -0.0405300855636597 57.0
0 -0.0815397500991821 58.0
0 -0.0484393835067749 59.0
0 -0.0971692800521851 60.0
0 0.0691772699356079 61.0
0 0.896136105060577 62.0
239.10623981 258.738189697266 0.0
0 -0.0316721200942993 1.0
0 -0.099725604057312 2.0
0 0.0486797094345093 3.0
0 -0.00208985805511475 4.0
0 -0.0348578691482544 5.0
0 -0.130855202674866 6.0
0 -0.0839933156967163 7.0
0 -0.026503324508667 8.0
0 0.0786975622177124 9.0
0 -0.0280424356460571 10.0
0 -0.0470128059387207 11.0
0 -0.0101772546768188 12.0
72.884736446 72.2322616577148 13.0
36.719566195 27.2256927490234 14.0
0 -0.0278068780899048 15.0
0 -0.0146082639694214 16.0
34.726962542 34.5596618652344 17.0
0 0.0465985536575317 18.0
0 0.0580204725265503 19.0
0 -0.023074746131897 20.0
0 -0.0489464998245239 21.0
0 -0.016053318977356 22.0
38.2427777 37.0562210083008 23.0
0 0.0701926946640015 24.0
0 0.438478350639343 25.0
0 0.0179041624069214 26.0
0 -0.0661743879318237 27.0
0 -0.0046769380569458 28.0
0 0.00869262218475342 29.0
0 -0.0296856164932251 30.0
0 -0.0223537683486938 31.0
0 0.0101422071456909 32.0
0 -0.0486308336257935 33.0
0 0.00839722156524658 34.0
0 0.0499590635299683 35.0
0 -0.000271439552307129 36.0
0 -0.00443756580352783 37.0
0 0.0257636308670044 38.0
0 0.0690380334854126 39.0
0 -0.0127841234207153 40.0
0 -0.0586687326431274 41.0
0 0.00523030757904053 42.0
307.23998536 295.231323242188 43.0
0 -0.0191026926040649 44.0
0 -0.0371434688568115 45.0
0 0.0116225481033325 46.0
0 -0.0126172304153442 47.0
0 0.0305742025375366 48.0
0 0.337760359048843 49.0
0.093856312152 0.0406802892684937 50.0
2.0102677936 0.974798083305359 51.0
0 -0.111766934394836 52.0
0 -0.0447579622268677 53.0
0 -0.00913679599761963 54.0
0 -0.106907725334167 55.0
0 -0.0296350717544556 56.0
0 0.0654562711715698 57.0
0 -0.068219780921936 58.0
0 -0.0191093683242798 59.0
0 -0.0787409543991089 60.0
0 0.0829805135726929 61.0
0 0.960049331188202 62.0
249.47224269 262.568725585938 0.0
0 -0.114864468574524 1.0
0 0.00188100337982178 2.0
0 -0.0118895769119263 3.0
0 -0.0603519678115845 4.0
0 0.0370825529098511 5.0
0 -0.0617262125015259 6.0
0 -0.217750430107117 7.0
0 -0.0798224210739136 8.0
0 -0.011724591255188 9.0
0 -0.00715720653533936 10.0
0 -0.0245469808578491 11.0
0 -0.0558077096939087 12.0
44.530182814 43.3031539916992 13.0
39.326904163 48.0668792724609 14.0
0 -0.0789638757705688 15.0
0 -0.0601476430892944 16.0
30.126958638 31.1387557983398 17.0
0 0.0227464437484741 18.0
0 -0.0213395357131958 19.0
0 -0.0199576616287231 20.0
0 -0.0367571115493774 21.0
0 0.0204237699508667 22.0
39.192553873 38.8073348999023 23.0
0 -0.0379577875137329 24.0
0 0.606485784053802 25.0
0 0.0194915533065796 26.0
0 0.0226408243179321 27.0
0 0.021776556968689 28.0
0 -0.0279790163040161 29.0
0 -0.0646114349365234 30.0
0 -0.042319655418396 31.0
0 -0.0873657464981079 32.0
0 0.0354617834091187 33.0
0 -0.0781713724136353 34.0
0 -0.0271334648132324 35.0
0 -0.147777438163757 36.0
0 -0.017961859703064 37.0
0 0.0157462358474731 38.0
0 0.0265365839004517 39.0
0 -0.0355826616287231 40.0
0 -0.0559624433517456 41.0
0 -0.150717377662659 42.0
292.42597538 295.725250244141 43.0
0 0.066405177116394 44.0
0 -0.0392841100692749 45.0
0 -0.0524684190750122 46.0
0 -0.023736834526062 47.0
0 0.0354336500167847 48.0
0 0.388350009918213 49.0
0.09665012652 0.0859028100967407 50.0
2.4211588333 0.965044677257538 51.0
0 -0.0710774660110474 52.0
0 -0.00521314144134521 53.0
0 0.027370810508728 54.0
0 -0.169944882392883 55.0
0 0.164151549339294 56.0
0 0.0301069021224976 57.0
0 0.00492966175079346 58.0
0 -0.0649136304855347 59.0
0 0.0833061933517456 60.0
0 0.0812383890151978 61.0
0 0.85956734418869 62.0
255.99925825 258.074554443359 0.0
0 0.033476710319519 1.0
0 -0.0775488615036011 2.0
0 0.0201660394668579 3.0
0 -0.0288249254226685 4.0
0 0.00135219097137451 5.0
0 0.0269559621810913 6.0
0 -0.0496689081192017 7.0
0 0.0622082948684692 8.0
0 0.00949704647064209 9.0
0 -0.00728833675384521 10.0
0 0.0816659331321716 11.0
0 0.0552228689193726 12.0
63.378122859 63.3727073669434 13.0
35.461713515 32.1217079162598 14.0
0 -0.0254184007644653 15.0
0 -0.0406750440597534 16.0
23.823333818 23.8477554321289 17.0
0 -0.0369638204574585 18.0
0 -0.0203589200973511 19.0
0 0.0869053602218628 20.0
0 0.00435554981231689 21.0
0 -0.0163742303848267 22.0
31.601187126 30.2144680023193 23.0
0 0.0941518545150757 24.0
0 0.484046697616577 25.0
0 0.0218921899795532 26.0
0 0.0360795259475708 27.0
0 -0.0232726335525513 28.0
0 -0.0838648080825806 29.0
0 -0.073498010635376 30.0
0 -0.0166655778884888 31.0
0 0.0356658697128296 32.0
0 0.129876494407654 33.0
0 -0.0284267663955688 34.0
0 -0.0767709016799927 35.0
0 -0.0293110609054565 36.0
0 0.0731238126754761 37.0
0 -0.00371658802032471 38.0
0 0.0360094308853149 39.0
0 -0.0355159044265747 40.0
0 -0.0196636915206909 41.0
0 -0.00074160099029541 42.0
281.42504354 295.561065673828 43.0
0 0.0173510313034058 44.0
0 -0.0339958667755127 45.0
0 0.0214184522628784 46.0
0 -0.00462925434112549 47.0
0 -0.00507795810699463 48.0
3.8216077332 0.304774105548859 49.0
0.064380623846 0.0701895952224731 50.0
2.0002402086 0.846070170402527 51.0
0 -0.0746299028396606 52.0
0 -0.0435949563980103 53.0
0 0.0955578088760376 54.0
0 -0.0824300050735474 55.0
0 -0.0480443239212036 56.0
0 0.0269778966903687 57.0
0 -0.0996025800704956 58.0
0 -0.018640398979187 59.0
0 0.0840319395065308 60.0
0 -0.0689135789871216 61.0
0 0.838287591934204 62.0
277.8785134 256.918853759766 0.0
0 -0.223138451576233 1.0
0 -0.00517332553863525 2.0
0 0.0127030611038208 3.0
0 -0.0516308546066284 4.0
0 -0.0137139558792114 5.0
0 0.12111508846283 6.0
0 0.0752187967300415 7.0
0 0.0546877384185791 8.0
0 -0.00242006778717041 9.0
0 0.0111013650894165 10.0
0 -0.0186578035354614 11.0
0 -0.0143600702285767 12.0
52.149117455 50.7182579040527 13.0
43.921465091 42.9252853393555 14.0
0 -0.184540390968323 15.0
0 0.0222979784011841 16.0
30.038375892 30.5334129333496 17.0
0 -0.0319916009902954 18.0
0 0.0554255247116089 19.0
0 0.0667475461959839 20.0
0 0.019976019859314 21.0
0 -0.00344240665435791 22.0
35.006707538 33.4382057189941 23.0
0 0.124102473258972 24.0
0 0.419576287269592 25.0
0 -0.0289880037307739 26.0
0 0.0761433839797974 27.0
0 0.0795491933822632 28.0
0 -0.0486525297164917 29.0
0 -0.0910110473632812 30.0
0 0.000997662544250488 31.0
0 0.135488390922546 32.0
0 0.0394409894943237 33.0
0 -0.117539763450623 34.0
0 0.00151026248931885 35.0
0 0.163546442985535 36.0
0 0.107240796089172 37.0
0 0.0497196912765503 38.0
0 0.0727266073226929 39.0
0 0.0227245092391968 40.0
0 -0.00243270397186279 41.0
0 -0.0590540170669556 42.0
280.03533859 296.037872314453 43.0
0 0.0332998037338257 44.0
0 0.054707407951355 45.0
0 0.00405395030975342 46.0
0 -0.031407356262207 47.0
0 -0.0238596200942993 48.0
0 0.278266847133636 49.0
0.10704007417 0.0454751253128052 50.0
1.7283463559 1.01711297035217 51.0
0 -0.0404039621353149 52.0
0 -0.0348020792007446 53.0
0 0.0318318605422974 54.0
0 -0.0634113550186157 55.0
0 -0.0517314672470093 56.0
0 0.0215331315994263 57.0
0 -0.00876080989837646 58.0
0 -0.0859028100967407 59.0
0 -0.052320122718811 60.0
0 0.0100036859512329 61.0
0 0.815010786056519 62.0
224.26623383 249.46760559082 0.0
0 -0.057348370552063 1.0
0 -0.119505524635315 2.0
0 0.0141946077346802 3.0
0 -0.106148838996887 4.0
0 -0.0118049383163452 5.0
0 -0.00471270084381104 6.0
0 -0.0154765844345093 7.0
0 0.0180699825286865 8.0
0 -0.0318046808242798 9.0
0 -0.00685012340545654 10.0
0 -0.0376861095428467 11.0
0 -0.122422814369202 12.0
54.995673976 53.9621963500977 13.0
35.408442778 39.6068000793457 14.0
0 0.0130718946456909 15.0
0 0.0202640295028687 16.0
29.342136396 30.703462600708 17.0
0 -0.0196532011032104 18.0
0 0.028753399848938 19.0
0 0.0164879560470581 20.0
0 -0.0122061967849731 21.0
0 -0.0178195238113403 22.0
40.57176485 39.9656791687012 23.0
0 0.0190109014511108 24.0
0 0.548383891582489 25.0
0 0.0818203687667847 26.0
0 0.0454941987991333 27.0
0 -0.0221205949783325 28.0
0 -0.0540550947189331 29.0
0 -0.0502477884292603 30.0
0 -0.0748962163925171 31.0
0 0.0460160970687866 32.0
0 0.0487140417098999 33.0
0 -0.0848428010940552 34.0
0 0.0643292665481567 35.0
0 -0.0612350702285767 36.0
0 -0.0699902772903442 37.0
0 0.0530308485031128 38.0
0 0.0800555944442749 39.0
0 0.186832785606384 40.0
0 0.0591009855270386 41.0
0 -0.0695406198501587 42.0
293.95017357 297.627197265625 43.0
0 0.0142282247543335 44.0
0 0.0844858884811401 45.0
0 0.0619326829910278 46.0
0 -0.0922728776931763 47.0
0 0.0376516580581665 48.0
0 0.255945563316345 49.0
0.10203926264 0.186161279678345 50.0
2.2384175158 0.862667441368103 51.0
0 -0.0837980508804321 52.0
0 -0.00784790515899658 53.0
0 0.0858312845230103 54.0
0 -0.0444349050521851 55.0
0 -0.0494230985641479 56.0
0 -0.0244461297988892 57.0
0 0.00678002834320068 58.0
0 -0.0155755281448364 59.0
0 -0.0434461832046509 60.0
0 0.0607389211654663 61.0
0 0.965568125247955 62.0
237.9749393 254.638580322266 0.0
0 -0.0968776941299438 1.0
0 0.00306928157806396 2.0
0 0.0190056562423706 3.0
0 -0.0408545732498169 4.0
0 -0.00625741481781006 5.0
0 0.0203746557235718 6.0
0 -0.0199590921401978 7.0
0 -0.0658307075500488 8.0
0 -0.0333915948867798 9.0
0 -0.0202466249465942 10.0
0 0.0196131467819214 11.0
0 0.00757920742034912 12.0
53.01598376 52.9162139892578 13.0
42.106576182 45.2529792785645 14.0
0 -0.0519381761550903 15.0
0 -0.0177937746047974 16.0
30.24209057 30.6955432891846 17.0
0 0.0321990251541138 18.0
0 -0.0070723295211792 19.0
0 0.00925314426422119 20.0
0 -0.0334457159042358 21.0
0 0.0157307386398315 22.0
37.721346758 36.6794967651367 23.0
0 -0.0600486993789673 24.0
0 0.861435651779175 25.0
0 0.0475805997848511 26.0
0 0.018168032169342 27.0
0 0.0852335691452026 28.0
0 -0.050775408744812 29.0
0 0.0826860666275024 30.0
0 -0.0282794237136841 31.0
0 -0.0293704271316528 32.0
0 -0.00794732570648193 33.0
0 -0.0569623708724976 34.0
0 -0.0802502632141113 35.0
0 -0.133643746376038 36.0
0 -0.0888708829879761 37.0
0 -0.0292471647262573 38.0
0 0.0310467481613159 39.0
0 0.107656836509705 40.0
0 -0.0429188013076782 41.0
0 0.0268863439559937 42.0
305.97915306 294.745147705078 43.0
0 -0.0496326684951782 44.0
0 0.0277810096740723 45.0
0 0.0159345865249634 46.0
0 -0.066679835319519 47.0
0 -0.00362312793731689 48.0
0 0.362991780042648 49.0
0.089228017656 0.0542038679122925 50.0
1.8739509055 0.96475738286972 51.0
0 -0.0266953706741333 52.0
0 -0.0524107217788696 53.0
0 0.042728066444397 54.0
0 -0.0357447862625122 55.0
0 -0.050737738609314 56.0
0 -0.0548621416091919 57.0
0 -0.0107413530349731 58.0
0 -0.0701018571853638 59.0
0 -0.00326311588287354 60.0
0 -0.000356793403625488 61.0
0 0.887453436851501 62.0
243.5053937 257.391357421875 0.0
0 -0.0691944360733032 1.0
0 -0.0730737447738647 2.0
0 0.015116810798645 3.0
0 -0.0404039621353149 4.0
0 0.0140863656997681 5.0
0 -0.147577404975891 6.0
0 -0.0444134473800659 7.0
0 -0.065102219581604 8.0
0 0.0522836446762085 9.0
0 0.0161309242248535 10.0
0 -0.0489585399627686 11.0
0 -0.153098702430725 12.0
58.517422758 57.2604103088379 13.0
44.983247631 37.8514099121094 14.0
0 -0.00418627262115479 15.0
0 -0.0104571580886841 16.0
39.213407233 38.9291000366211 17.0
0 0.0100349187850952 18.0
0 -0.00417625904083252 19.0
0 -0.0228739976882935 20.0
0 0.00395381450653076 21.0
0 -0.0340663194656372 22.0
33.424482537 33.218936920166 23.0
0 -0.00310099124908447 24.0
0 0.49303263425827 25.0
0 0.0662964582443237 26.0
0 0.00615894794464111 27.0
0 -0.0726915597915649 28.0
0 -0.0915166139602661 29.0
0 -0.0617672204971313 30.0
0 0.000701785087585449 31.0
0 0.0462769269943237 32.0
0 0.0412100553512573 33.0
0 0.00959599018096924 34.0
0 -0.0762046575546265 35.0
0 3.73125076293945e-05 36.0
0 -0.0824891328811646 37.0
0 0.021239161491394 38.0
0 0.0585333108901978 39.0
0 0.0364001989364624 40.0
0 -0.0693491697311401 41.0
0 0.00568234920501709 42.0
302.23032285 297.361602783203 43.0
0 0.0335677862167358 44.0
0 0.025287389755249 45.0
0 0.0041123628616333 46.0
0 -0.0944539308547974 47.0
0 0.0360864400863647 48.0
0.33193961482 0.285253077745438 49.0
0.079595171134 0.049219012260437 50.0
1.9698338108 0.902969717979431 51.0
0 -0.0297452211380005 52.0
0 0.00519979000091553 53.0
0 -0.00758016109466553 54.0
0 -0.0632294416427612 55.0
0 0.0200473070144653 56.0
0 -0.0395032167434692 57.0
0 -0.0761831998825073 58.0
0 -0.0746122598648071 59.0
0 -0.0212088823318481 60.0
0 0.0412272214889526 61.0
0 0.896820783615112 62.0
252.29878881 254.122482299805 0.0
0 0.0108741521835327 1.0
0 -0.113366007804871 2.0
0 -0.00153911113739014 3.0
0 -0.100686192512512 4.0
0 -0.0855394601821899 5.0
0 0.043819785118103 6.0
0 -0.0535813570022583 7.0
0 -0.0280908346176147 8.0
0 0.0621904134750366 9.0
0 -0.221813917160034 10.0
0 0.0203911066055298 11.0
0 0.145504355430603 12.0
65.30730968 65.3088607788086 13.0
43.929372721 33.343090057373 14.0
0 0.00470244884490967 15.0
0 -0.0277310609817505 16.0
29.692732258 30.1173820495605 17.0
0 -0.00173032283782959 18.0
0 -0.0206934213638306 19.0
0 0.0818592309951782 20.0
0 0.0182760953903198 21.0
0 -0.0528484582901001 22.0
40.252212846 40.108154296875 23.0
0 -0.0785442590713501 24.0
0 2.32326769828796 25.0
0 0.0543640851974487 26.0
0 0.045326828956604 27.0
0 -0.0351325273513794 28.0
0 -0.00798642635345459 29.0
0 -0.0114119052886963 30.0
0 -0.0404537916183472 31.0
0 0.00995481014251709 32.0
0 -0.0364960432052612 33.0
0 0.00951516628265381 34.0
0 -0.0640586614608765 35.0
0 -0.0230275392532349 36.0
0 -0.0495675802230835 37.0
0 0.0494683980941772 38.0
0 -0.0839120149612427 39.0
0 -0.0874854326248169 40.0
0 -0.0820204019546509 41.0
0 -0.035831093788147 42.0
290.52591934 296.011535644531 43.0
0 0.0515979528427124 44.0
0 -0.0678852796554565 45.0
0 -0.0385786294937134 46.0
0 -0.0393799543380737 47.0
0 0.0852483510971069 48.0
0 0.245246112346649 49.0
0.082579872985 0.0730108022689819 50.0
1.8766716864 0.926250278949738 51.0
0 -0.0788227319717407 52.0
0 -0.0489364862442017 53.0
0 0.0341428518295288 54.0
0 -0.0889462232589722 55.0
0 -0.0840400457382202 56.0
0 0.0250495672225952 57.0
0 -0.0432075262069702 58.0
0 -0.0122131109237671 59.0
0 0.037744402885437 60.0
0 -0.0535227060317993 61.0
0 0.850230574607849 62.0
228.19421449 257.428436279297 0.0
0 -0.0515762567520142 1.0
0 -0.0413295030593872 2.0
0 0.00455605983734131 3.0
0 -0.0523425340652466 4.0
0 -0.0887335538864136 5.0
0 -0.0706120729446411 6.0
0 -0.0430241823196411 7.0
0 -0.0517780780792236 8.0
0 -0.0275427103042603 9.0
0 0.0294321775436401 10.0
0 -0.047937273979187 11.0
0 -0.0249792337417603 12.0
55.992410422 55.0322723388672 13.0
44.106995702 40.5211524963379 14.0
0 0.00991380214691162 15.0
0 -0.00883376598358154 16.0
32.027255686 31.5535945892334 17.0
0 0.0278276205062866 18.0
0 -0.0237590074539185 19.0
0 0.0253251791000366 20.0
0 -0.00795304775238037 21.0
0 -0.0339056253433228 22.0
34.324535744 32.7838745117188 23.0
0 0.0851982831954956 24.0
0 0.411089777946472 25.0
0 0.0890082120895386 26.0
0 0.0283260345458984 27.0
0 0.0562080144882202 28.0
0 -0.0271779298782349 29.0
0 -0.0450184345245361 30.0
0 -0.0449355840682983 31.0
0 0.0658940076828003 32.0
0 0.0175284147262573 33.0
0 -0.00859177112579346 34.0
0 -0.0275570154190063 35.0
0 0.00230300426483154 36.0
0 0.0632084608078003 37.0
0 0.0385481119155884 38.0
0 0.00218713283538818 39.0
0 -0.0207592248916626 40.0
0 -0.0391618013381958 41.0
0 -0.110595345497131 42.0
306.1345352 295.262817382812 43.0
0 0.0351270437240601 44.0
0 -0.0656009912490845 45.0
0 0.00643503665924072 46.0
0 -0.00844681262969971 47.0
0 -0.0122820138931274 48.0
1.4974476211 0.296167254447937 49.0
0.093620291271 0.0788401365280151 50.0
1.8684514663 0.998825252056122 51.0
0 -0.0411592721939087 52.0
0 -0.0396889448165894 53.0
0 0.00667846202850342 54.0
0 -0.156330943107605 55.0
0 0.0218874216079712 56.0
0 -0.0484699010848999 57.0
0 -0.0632535219192505 58.0
0 -0.0927537679672241 59.0
0 -0.0431421995162964 60.0
0 0.0533357858657837 61.0
0 0.882676422595978 62.0
246.10036775 252.153366088867 0.0
0 -0.0896183252334595 1.0
0 -0.00669991970062256 2.0
0 -0.0406807661056519 3.0
0 -0.019992470741272 4.0
0 -0.073224663734436 5.0
0 -0.0367590188980103 6.0
0 -0.0556904077529907 7.0
0 -0.0555577278137207 8.0
0 0.048550009727478 9.0
0 -0.00152671337127686 10.0
0 -0.0475577116012573 11.0
0 -0.0396171808242798 12.0
60.16591839 59.213752746582 13.0
42.230332396 38.1709060668945 14.0
0 -0.0284792184829712 15.0
0 -0.0407415628433228 16.0
33.765235929 34.111686706543 17.0
0 -0.0362151861190796 18.0
0 -0.0404411554336548 19.0
0 0.0746146440505981 20.0
0 0.0155240297317505 21.0
0 -0.0240613222122192 22.0
31.095887022 30.704517364502 23.0
0 -0.0305472612380981 24.0
0 0.630960881710052 25.0
0 0.080527663230896 26.0
0 0.0169626474380493 27.0
0 0.0179661512374878 28.0
0 -0.0790008306503296 29.0
0 -0.0595779418945312 30.0
0 -0.0335515737533569 31.0
0 -0.0696808099746704 32.0
0 -0.048833966255188 33.0
0 -0.0027388334274292 34.0
0 -0.0780261754989624 35.0
0 0.0717419385910034 36.0
0 -0.0571149587631226 37.0
0 0.0365160703659058 38.0
0 -0.011487603187561 39.0
0 -0.0316492319107056 40.0
0 0.00977957248687744 41.0
0 -0.0321098566055298 42.0
299.23666777 296.339599609375 43.0
0 0.00781190395355225 44.0
0 0.0645663142204285 45.0
0 -0.0447677373886108 46.0
0 -0.0160435438156128 47.0
0 -0.032143235206604 48.0
0 0.271331369876862 49.0
0.078311438954 0.0630851984024048 50.0
1.8185687446 0.915862381458282 51.0
0 -0.0950044393539429 52.0
0 -0.0369836091995239 53.0
0 -0.0597995519638062 54.0
0 -0.129286646842957 55.0
0 -0.0510598421096802 56.0
0 0.0301116704940796 57.0
0 -0.0645145177841187 58.0
0 -0.0654648542404175 59.0
0 -0.0498336553573608 60.0
0 -0.0159071683883667 61.0
0 0.797182083129883 62.0
246.95520436 256.001525878906 0.0
0 -0.0994354486465454 1.0
0 0.00920188426971436 2.0
0 -0.0716768503189087 3.0
0 -0.065635085105896 4.0
0 -0.00152075290679932 5.0
0 -0.0246092081069946 6.0
0 -0.0386511087417603 7.0
0 -0.0123690366744995 8.0
0 0.081875205039978 9.0
0 -0.0030587911605835 10.0
0 -0.0291944742202759 11.0
0 0.0192312002182007 12.0
53.989704928 52.9809989929199 13.0
47.414005905 44.7695159912109 14.0
0 0.0112780332565308 15.0
0 -0.0798395872116089 16.0
33.393887828 32.3422203063965 17.0
0 0.00632679462432861 18.0
0 0.00328123569488525 19.0
0 0.0259557962417603 20.0
0 -0.00450170040130615 21.0
0 0.00762498378753662 22.0
35.250246598 34.5985298156738 23.0
0 0.00227200984954834 24.0
0 0.535783708095551 25.0
0 0.0357297658920288 26.0
0 0.0705868005752563 27.0
0 0.041521430015564 28.0
0 -0.0968905687332153 29.0
0 -0.0510932207107544 30.0
0 -0.0211230516433716 31.0
0 -0.00691330432891846 32.0
0 0.0183867216110229 33.0
0 -0.0371531248092651 34.0
0 -0.0621774196624756 35.0
0 -0.032110333442688 36.0
0 -0.0425323247909546 37.0
0 0.0500270128250122 38.0
0 0.0390774011611938 39.0
0 -0.0574108362197876 40.0
0 -0.0492675304412842 41.0
0 0.0545734167098999 42.0
295.17538214 295.548309326172 43.0
0 0.0267757177352905 44.0
0 0.0625790357589722 45.0
0 -0.00266659259796143 46.0
0 -0.0523160696029663 47.0
0 0.00918233394622803 48.0
0 0.336515516042709 49.0
0.079214648542 0.0705074071884155 50.0
1.548652431 0.929699957370758 51.0
0 -0.0517688989639282 52.0
0 -0.0493446588516235 53.0
0 0.0393384695053101 54.0
0 -0.0502163171768188 55.0
0 -0.0562518835067749 56.0
0 0.0176161527633667 57.0
0 -0.107540965080261 58.0
0 -0.0491472482681274 59.0
0 -0.00059044361114502 60.0
0 -0.0644298791885376 61.0
0 0.821759819984436 62.0
214.77891698 259.363098144531 0.0
0 -0.0755716562271118 1.0
0 -0.0563904047012329 2.0
0 0.0132769346237183 3.0
0 -0.0755153894424438 4.0
0 0.0545636415481567 5.0
0 0.0824168920516968 6.0
0 -0.0419031381607056 7.0
0 0.0268265008926392 8.0
0 0.0127266645431519 9.0
0 -0.0561264753341675 10.0
0 -0.0112487077713013 11.0
0 0.0775798559188843 12.0
61.401887446 60.3607139587402 13.0
23.041916514 33.8038330078125 14.0
0 -0.0325723886489868 15.0
0 -0.0423127412796021 16.0
38.097280504 37.559009552002 17.0
0 -0.0231660604476929 18.0
0 -0.0249220132827759 19.0
0 0.0195950269699097 20.0
0 -0.0433443784713745 21.0
0 0.0117515325546265 22.0
37.038688915 36.1839942932129 23.0
0 -0.05231773853302 24.0
0 0.651773154735565 25.0
0 0.0268296003341675 26.0
0 0.131189227104187 27.0
0 0.0782188177108765 28.0
0 -0.0992523431777954 29.0
0 0.211296916007996 30.0
0 -0.0153740644454956 31.0
0 0.0351370573043823 32.0
0 0.0191687345504761 33.0
0 -0.0418676137924194 34.0
0 -0.0941528081893921 35.0
0 0.0907143354415894 36.0
0 0.0213257074356079 37.0
0 0.0122510194778442 38.0
0 0.0285559892654419 39.0
0 0.0931485891342163 40.0
0 -0.0489672422409058 41.0
0 -0.0605999231338501 42.0
293.23891824 294.9228515625 43.0
0 -0.0414005517959595 44.0
0 -0.00369870662689209 45.0
0 -0.0087658166885376 46.0
0 -0.0691255331039429 47.0
0 -0.0253030061721802 48.0
0.56460484691 0.300189316272736 49.0
0.089891975643 0.0914372205734253 50.0
1.9454678792 0.897586643695831 51.0
0 -0.0480538606643677 52.0
0 -0.00932562351226807 53.0
0 0.0212281942367554 54.0
0 -0.0870572328567505 55.0
0 -0.0556727647781372 56.0
0 -0.0373378992080688 57.0
0 -0.034633994102478 58.0
0 -0.00945365428924561 59.0
0 -0.0157192945480347 60.0
0 -0.0324269533157349 61.0
0 0.849274396896362 62.0
227.1505211 259.355377197266 0.0
0 -0.0157648324966431 1.0
0 -0.0966347455978394 2.0
0 -0.0404239892959595 3.0
0 -0.099141001701355 4.0
0 -0.0338040590286255 5.0
0 0.0709508657455444 6.0
0 -0.084511399269104 7.0
0 0.058153510093689 8.0
0 0.0617598295211792 9.0
0 -0.143507361412048 10.0
0 -0.0426468849182129 11.0
0 -0.00215685367584229 12.0
54.629164531 54.0626068115234 13.0
30.547711843 34.2195816040039 14.0
0 -0.0148025751113892 15.0
0 -0.0136004686355591 16.0
39.448261244 40.0533638000488 17.0
0 -0.0568424463272095 18.0
0 0.0296241044998169 19.0
0 0.00928008556365967 20.0
0 -0.0328248739242554 21.0
0 0.00486171245574951 22.0
37.44736164 36.1665077209473 23.0
0 0.0787256956100464 24.0
0 0.409643292427063 25.0
0 0.0510715246200562 26.0
0 0.0124961137771606 27.0
0 -0.0293623208999634 28.0
0 -0.0973678827285767 29.0
0 0.0322099924087524 30.0
0 -0.00651252269744873 31.0
0 0.00863444805145264 32.0
0 -0.0026247501373291 33.0
0 -0.0612515211105347 34.0
0 0.137369275093079 35.0
0 0.0440412759780884 36.0
0 -0.0305217504501343 37.0
0 0.0326672792434692 38.0
0 0.129710793495178 39.0
0 0.0708059072494507 40.0
0 0.00247704982757568 41.0
0 0.0113948583602905 42.0
288.43737327 298.314605712891 43.0
0 -0.0318838357925415 44.0
0 0.0469704270362854 45.0
0 0.0312899351119995 46.0
0 -0.0518776178359985 47.0
0 0.0900973081588745 48.0
0 0.323975682258606 49.0
0.097052147027 0.039730429649353 50.0
2.0553872314 0.964807391166687 51.0
0 -0.0947061777114868 52.0
0 -0.0551401376724243 53.0
0 0.0416725873947144 54.0
0 -0.0948973894119263 55.0
0 -0.0978797674179077 56.0
0 0.0164006948471069 57.0
0 -0.0201150178909302 58.0
0 -0.0383893251419067 59.0
0 -0.0464557409286499 60.0
0 -0.021173357963562 61.0
0 0.823115348815918 62.0
253.37155263 257.074584960938 0.0
0 -0.108672738075256 1.0
0 0.00647938251495361 2.0
0 -0.053923487663269 3.0
0 0.0105265378952026 4.0
0 0.0250831842422485 5.0
0 0.0708154439926147 6.0
0 -0.02061927318573 7.0
0 -0.00313794612884521 8.0
0 0.0563262701034546 9.0
0 -0.015363335609436 10.0
0 -0.0364789366722107 11.0
0 0.0269197225570679 12.0
67.207264462 67.6800155639648 13.0
28.29809793 31.1129055023193 14.0
0 0.0222505331039429 15.0
0 0.00276267528533936 16.0
26.532257296 26.0689964294434 17.0
0 0.0470913648605347 18.0
0 0.053500771522522 19.0
0 -0.0214487314224243 20.0
0 -0.0177329778671265 21.0
0 -0.0819793939590454 22.0
32.482596425 32.5024948120117 23.0
0 0.0169171094894409 24.0
0 0.470386803150177 25.0
0 0.0894221067428589 26.0
0 0.0548309087753296 27.0
0 -0.0810810327529907 28.0
0 -0.0335177183151245 29.0
0 -0.0846264362335205 30.0
0 -0.0318008661270142 31.0
0 -0.0273114442825317 32.0
0 0.0114432573318481 33.0
0 -0.0146273374557495 34.0
0 -0.0393497943878174 35.0
0 0.0279415845870972 36.0
0 -0.0663849115371704 37.0
0 -0.0222822427749634 38.0
0 0.0571564435958862 39.0
0 0.0394605398178101 40.0
0 -0.0193648338317871 41.0
0 0.0197614431381226 42.0
305.21051439 293.903625488281 43.0
0 -0.00399744510650635 44.0
0 0.0702563524246216 45.0
0 -0.0132967233657837 46.0
0 -0.0194882154464722 47.0
0 0.00680482387542725 48.0
0 0.353034168481827 49.0
0.064116596024 0.0387266874313354 50.0
1.6610570194 0.89832216501236 51.0
0 -0.0564080476760864 52.0
0 -0.0583344697952271 53.0
0 -0.00574862957000732 54.0
0 -0.0955420732498169 55.0
0 0.0356539487838745 56.0
0 0.0253080129623413 57.0
0 0.0190151929855347 58.0
0 -0.00184428691864014 59.0
0 -0.0270606279373169 60.0
0 -0.00192129611968994 61.0
0 0.809369802474976 62.0
248.55034147 253.668884277344 0.0
0 -0.0611072778701782 1.0
0 -0.12087619304657 2.0
0 -0.0182338953018188 3.0
0 -0.0688549280166626 4.0
0 0.106003642082214 5.0
0 0.0726960897445679 6.0
0 -0.0917149782180786 7.0
0 -0.0199655294418335 8.0
0 0.0535081624984741 9.0
0 0.0467890501022339 10.0
0 -0.00462174415588379 11.0
0 -0.126365303993225 12.0
78.786265315 78.3009567260742 13.0
25.028728706 19.4680461883545 14.0
0 -0.0503422021865845 15.0
0 0.0239483118057251 16.0
23.565910309 22.3796997070312 17.0
0 0.0604122877120972 18.0
0 0.101992726325989 19.0
0 -0.00499355792999268 20.0
0 0.0366162061691284 21.0
0 0.018243670463562 22.0
31.229943222 30.190486907959 23.0
0 0.170839667320251 24.0
0 0.41377055644989 25.0
0 0.0964926481246948 26.0
0 -0.0551108121871948 27.0
0 0.118980765342712 28.0
0 -0.154280304908752 29.0
0 0.217584490776062 30.0
0 -0.0405601263046265 31.0
0 0.0177029371261597 32.0
0 0.0232056379318237 33.0
0 0.015159010887146 34.0
0 -0.100488543510437 35.0
0 0.00649631023406982 36.0
0 -0.0051189661026001 37.0
0 -0.0859099626541138 38.0
0 0.149714589118958 39.0
0 0.0914419889450073 40.0
0 -0.0767103433609009 41.0
0 -0.0431104898452759 42.0
281.8352294 296.975738525391 43.0
0 -0.0146723985671997 44.0
0 -0.0776468515396118 45.0
0 0.0124524831771851 46.0
0 -0.0700944662094116 47.0
0 0.198269724845886 48.0
0 0.517253398895264 49.0
0.10367579902 0.0879660844802856 50.0
2.3842427632 0.92536336183548 51.0
0 -0.0892177820205688 52.0
0 -0.0604933500289917 53.0
0 0.00295579433441162 54.0
0 -0.0232433080673218 55.0
0 0.0269061326980591 56.0
0 -0.0393792390823364 57.0
0 -0.0732754468917847 58.0
0 -0.0204654932022095 59.0
0 -0.0356661081314087 60.0
0 -0.00325477123260498 61.0
0 0.876163244247437 62.0
274.71338076 261.428649902344 0.0
0 -0.1037517786026 1.0
0 -0.0648690462112427 2.0
0 0.105085015296936 3.0
0 0.0327321290969849 4.0
0 -0.0620375871658325 5.0
0 0.118240714073181 6.0
0 -0.0230811834335327 7.0
0 -0.099271297454834 8.0
0 -0.0382372140884399 9.0
0 -0.00446617603302002 10.0
0 -0.0241944789886475 11.0
0 -0.0593305826187134 12.0
48.592168345 46.8874816894531 13.0
48.066042103 44.7575607299805 14.0
0 -0.10952627658844 15.0
0 -0.0270029306411743 16.0
31.925967301 30.7501277923584 17.0
0 0.00580060482025146 18.0
0 -0.037397027015686 19.0
0 -0.0221065282821655 20.0
0 0.0189913511276245 21.0
0 -0.0219560861587524 22.0
36.00091564 35.9883842468262 23.0
0 -0.0759869813919067 24.0
0 0.733140826225281 25.0
0 0.0169366598129272 26.0
0 0.106364846229553 27.0
0 -0.0808645486831665 28.0
0 -0.0189124345779419 29.0
0 -0.0305465459823608 30.0
0 -0.0502399206161499 31.0
0 0.0566242933273315 32.0
0 -0.060158371925354 33.0
0 -0.0317384004592896 34.0
0 -0.0684964656829834 35.0
0 0.0210846662521362 36.0
0 -0.0111209154129028 37.0
0 -0.0474041700363159 38.0
0 0.0178033113479614 39.0
0 0.0556811094284058 40.0
0 -0.0310178995132446 41.0
0 -0.136102318763733 42.0
299.78474926 294.552520751953 43.0
0 0.035216212272644 44.0
0 0.00119459629058838 45.0
0 -0.0194834470748901 46.0
0 0.0157746076583862 47.0
0 0.0249284505844116 48.0
0 0.262695074081421 49.0
0.082233883225 0.0930294990539551 50.0
2.1052938168 0.862134575843811 51.0
0 -0.0452603101730347 52.0
0 -0.00442278385162354 53.0
0 0.0607491731643677 54.0
0 -0.0407487154006958 55.0
0 -0.0995453596115112 56.0
0 0.017234206199646 57.0
0 -0.0530987977981567 58.0
0 -0.0586730241775513 59.0
0 0.0035477876663208 60.0
0 0.0312591791152954 61.0
0 0.887750804424286 62.0
245.92235752 251.342849731445 0.0
0 -0.00667345523834229 1.0
0 -0.0829375982284546 2.0
0 0.00967037677764893 3.0
0 -0.0533279180526733 4.0
0 -0.113062739372253 5.0
0 -0.0654188394546509 6.0
0 -0.0738049745559692 7.0
0 -0.0713615417480469 8.0
0 0.0337556600570679 9.0
0 -0.00923395156860352 10.0
0 0.0048830509185791 11.0
0 -0.0812698602676392 12.0
63.611338926 61.8316535949707 13.0
43.487659217 33.5792617797852 14.0
0 -0.0287934541702271 15.0
0 -0.0290788412094116 16.0
28.186473688 29.0283737182617 17.0
0 0.00216257572174072 18.0
0 -0.0336626768112183 19.0
0 0.015252947807312 20.0
0 -0.0376187562942505 21.0
0 -0.022828221321106 22.0
36.111356254 35.5706100463867 23.0
0 -0.0457488298416138 24.0
0 0.608387410640717 25.0
0 0.10434901714325 26.0
0 0.0419241189956665 27.0
0 -0.0517202615737915 28.0
0 -0.0718220472335815 29.0
0 0.143905997276306 30.0
0 -0.00304782390594482 31.0
0 0.032561182975769 32.0
0 0.0237945318222046 33.0
0 -0.0375806093215942 34.0
0 -0.0571860074996948 35.0
0 -0.0649548768997192 36.0
0 -0.189451575279236 37.0
0 -0.0372425317764282 38.0
0 -0.0849233865737915 39.0
0 0.063862681388855 40.0
0 0.0883349180221558 41.0
0 -0.0889631509780884 42.0
283.44922613 295.424102783203 43.0
0 -0.0290476083755493 44.0
0 -0.0563962459564209 45.0
0 -0.00431787967681885 46.0
0 -0.00542104244232178 47.0
0 -0.0357528924942017 48.0
2.4850982791 0.283018231391907 49.0
0.094109062831 0.180280327796936 50.0
2.3624543252 0.910643994808197 51.0
0 -0.112699866294861 52.0
0 -0.0274406671524048 53.0
0 -0.0458899736404419 54.0
0 -0.0545316934585571 55.0
0 -0.0775309801101685 56.0
0 -0.00209963321685791 57.0
0 0.0132871866226196 58.0
0 -0.0721036195755005 59.0
0 -0.034932017326355 60.0
0 0.0249422788619995 61.0
0 0.822871088981628 62.0
236.23977496 253.327163696289 0.0
0 -0.0460830926895142 1.0
0 -0.0236626863479614 2.0
0 0.0240534543991089 3.0
0 -0.0326207876205444 4.0
0 0.101459383964539 5.0
0 0.192227721214294 6.0
0 -0.0281013250350952 7.0
0 -0.0237804651260376 8.0
0 -0.0711380243301392 9.0
0 -0.0214923620223999 10.0
0 -0.00358808040618896 11.0
0 -0.0180813074111938 12.0
62.305242826 60.8668403625488 13.0
38.160287119 34.9017715454102 14.0
0 -0.00499927997589111 15.0
0 -0.0227943658828735 16.0
41.464854029 40.6149673461914 17.0
0 0.0543168783187866 18.0
0 0.0551868677139282 19.0
0 0.011935830116272 20.0
0 -0.0256396532058716 21.0
0 0.0505937337875366 22.0
28.961416808 28.2658615112305 23.0
0 -0.0411430597305298 24.0
0 0.611379027366638 25.0
0 0.0338987112045288 26.0
0 -0.0225809216499329 27.0
0 -0.00930988788604736 28.0
0 -0.038082480430603 29.0
0 0.0287140607833862 30.0
0 -0.0444759130477905 31.0
0 -0.0284644365310669 32.0
0 0.0728861093521118 33.0
0 -0.00450551509857178 34.0
0 -0.138896346092224 35.0
0 0.0325249433517456 36.0
0 -0.0631266832351685 37.0
0 -0.0146292448043823 38.0
0 0.0378412008285522 39.0
0 0.153215765953064 40.0
0 -0.0207333564758301 41.0
0 -0.00847208499908447 42.0
300.86026577 292.785491943359 43.0
0 -0.0194154977798462 44.0
0 0.055025577545166 45.0
0 0.0188037157058716 46.0
0 -0.0652784109115601 47.0
0 0.0239242315292358 48.0
0 0.304878175258636 49.0
0.083252628919 0.0405443906784058 50.0
2.1624799006 0.926485180854797 51.0
0 -0.0944687128067017 52.0
0 -0.0574346780776978 53.0
0 0.00259578227996826 54.0
0 -0.0779894590377808 55.0
0 -0.0233396291732788 56.0
0 -0.0225929021835327 57.0
0 -0.0734151601791382 58.0
0 -0.0191959142684937 59.0
0 0.0303300619125366 60.0
0 0.0682781934738159 61.0
0 0.811705589294434 62.0
251.6335405 256.385467529297 0.0
0 0.0221308469772339 1.0
0 -0.0831023454666138 2.0
0 -0.00785887241363525 3.0
0 -0.0510121583938599 4.0
0 -0.0912915468215942 5.0
0 -0.11130964756012 6.0
0 -0.108339190483093 7.0
0 0.0344308614730835 8.0
0 0.0148957967758179 9.0
0 -0.119305014610291 10.0
0 -0.0349434614181519 11.0
0 -0.0887285470962524 12.0
68.743829998 68.8474884033203 13.0
26.677010757 28.3919677734375 14.0
0 0.0248764753341675 15.0
0 0.00308620929718018 16.0
29.639555081 28.978687286377 17.0
0 -0.0162667036056519 18.0
0 -0.0316680669784546 19.0
0 0.0103217363357544 20.0
0 0.00933587551116943 21.0
0 -0.0881158113479614 22.0
36.124095328 35.293586730957 23.0
0 0.0910004377365112 24.0
0 0.400435507297516 25.0
0 -0.00036013126373291 26.0
0 -0.0181905031204224 27.0
0 -0.0274606943130493 28.0
0 -0.0993715524673462 29.0
0 -0.0805807113647461 30.0
0 0.0245510339736938 31.0
0 0.00357568264007568 32.0
0 -0.0106765031814575 33.0
0 -0.0377448797225952 34.0
0 -0.0898518562316895 35.0
0 -0.110892415046692 36.0
0 -0.0408138036727905 37.0
0 -0.121257185935974 38.0
0 -0.0339921712875366 39.0
0 -0.0979019403457642 40.0
0 0.0551420450210571 41.0
0 -0.0030590295791626 42.0
287.72703475 293.584777832031 43.0
0 0.0334814786911011 44.0
0 0.0969648361206055 45.0
0 0.013774037361145 46.0
0 -0.0360205173492432 47.0
0 0.0136488676071167 48.0
0 0.255615174770355 49.0
0.083055145228 0.117410063743591 50.0
2.3711110216 1.00145792961121 51.0
0 -0.0552595853805542 52.0
0 -0.0455118417739868 53.0
0 -0.00464737415313721 54.0
0 -0.0566226243972778 55.0
0 -0.0628660917282104 56.0
0 -0.0226126909255981 57.0
0 -0.0577312707901001 58.0
0 -0.0181881189346313 59.0
0 -0.140734314918518 60.0
0 0.0680474042892456 61.0
0 0.950489401817322 62.0
275.88544297 253.15592956543 0.0
0 -0.0701533555984497 1.0
0 0.0320008993148804 2.0
0 -0.00319349765777588 3.0
0 -0.0479964017868042 4.0
0 0.0485666990280151 5.0
0 -0.179434895515442 6.0
0 -0.0770889520645142 7.0
0 -0.0224705934524536 8.0
0 0.117688536643982 9.0
0 0.0102531909942627 10.0
0 -0.0442963838577271 11.0
0 0.0166124105453491 12.0
70.721540296 70.4016265869141 13.0
35.210300513 27.5172882080078 14.0
0 -0.000838875770568848 15.0
0 -0.0328332185745239 16.0
30.22864456 29.7543659210205 17.0
0 0.0047379732131958 18.0
0 0.0144914388656616 19.0
0 0.0106793642044067 20.0
0 0.0231722593307495 21.0
0 -0.0321160554885864 22.0
29.910647956 29.389253616333 23.0
0 -0.0543752908706665 24.0
0 0.558488726615906 25.0
0 0.0398577451705933 26.0
0 0.0554300546646118 27.0
0 0.0425566434860229 28.0
0 -0.021788477897644 29.0
0 -0.0692754983901978 30.0
0 -0.048601508140564 31.0
0 -0.0315409898757935 32.0
0 -0.0207690000534058 33.0
0 -0.0733731985092163 34.0
0 -0.0116389989852905 35.0
0 0.0933908224105835 36.0
0 -0.01421058177948 37.0
0 0.0322328805923462 38.0
0 0.0611635446548462 39.0
0 -0.0196739435195923 40.0
0 0.0248124599456787 41.0
0 -0.0323997735977173 42.0
305.76546092 293.640350341797 43.0
0 -0.004219651222229 44.0
0 -0.000404477119445801 45.0
0 -0.0382117033004761 46.0
0 -0.039406418800354 47.0
0 0.059508204460144 48.0
0 0.295237749814987 49.0
0.09097757744 0.0434685945510864 50.0
1.6970982364 0.87777704000473 51.0
0 -0.0709348917007446 52.0
0 -0.0468078851699829 53.0
0 -0.00177037715911865 54.0
0 -0.0434938669204712 55.0
0 -0.0979911088943481 56.0
0 0.0227583646774292 57.0
0 0.00325524806976318 58.0
0 -0.062050461769104 59.0
0 0.00320065021514893 60.0
0 0.0242050886154175 61.0
0 0.845889925956726 62.0
256.11859023 254.306427001953 0.0
0 0.0255769491195679 1.0
0 0.000481963157653809 2.0
0 0.0046459436416626 3.0
0 -0.0687371492385864 4.0
0 0.0171200037002563 5.0
0 -0.0287739038467407 6.0
0 -0.0438977479934692 7.0
0 -0.0216484069824219 8.0
0 0.0794657468795776 9.0
0 -0.0603913068771362 10.0
0 -0.0414998531341553 11.0
0 0.0105232000350952 12.0
61.406157636 60.9865798950195 13.0
32.150833402 37.3355026245117 14.0
0 0.0355867147445679 15.0
0 -0.0662001371383667 16.0
30.327468976 30.4483509063721 17.0
0 -0.020006537437439 18.0
0 -0.0058666467666626 19.0
0 -0.0488322973251343 20.0
0 -0.0351977348327637 21.0
0 -0.0101829767227173 22.0
40.707992043 39.3492393493652 23.0
0 0.0301183462142944 24.0
0 0.49200701713562 25.0
0 0.0462511777877808 26.0
0 0.0385648012161255 27.0
0 -0.0609222650527954 28.0
0 -0.0453134775161743 29.0
0 -0.0756158828735352 30.0
0 -0.0425828695297241 31.0
0 0.0248368978500366 32.0
0 0.0710581541061401 33.0
0 -0.0321582555770874 34.0
0 0.0136367082595825 35.0
0 -0.0115963220596313 36.0
0 -0.110098004341125 37.0
0 0.0106097459793091 38.0
0 0.0760053396224976 39.0
0 -0.0109916925430298 40.0
0 -0.0731052160263062 41.0
0 0.0233949422836304 42.0
289.76088901 295.03564453125 43.0
0 -0.0142718553543091 44.0
0 -0.0337406396865845 45.0
0 -0.0509837865829468 46.0
0 0.00449240207672119 47.0
0 0.0196322202682495 48.0
1.0488815448 0.303538918495178 49.0
0.080977034108 0.0893675088882446 50.0
2.2198448049 0.904333531856537 51.0
0 -0.0995475053787231 52.0
0 -0.0110822916030884 53.0
0 0.00685751438140869 54.0
0 -0.0974870920181274 55.0
0 -0.0204199552536011 56.0
0 -0.0276414155960083 57.0
0 -0.0593007802963257 58.0
0 -0.00868046283721924 59.0
0 0.014184832572937 60.0
0 0.0363062620162964 61.0
0 0.89179652929306 62.0
225.35791818 252.481872558594 0.0
0 -0.0948482751846313 1.0
0 -0.0153223276138306 2.0
0 0.00905311107635498 3.0
0 -0.0408991575241089 4.0
0 0.0181130170822144 5.0
0 0.0284126996994019 6.0
0 -0.0366209745407104 7.0
0 -0.00223278999328613 8.0
0 -0.0678454637527466 9.0
0 -0.114575028419495 10.0
0 0.0203022956848145 11.0
0 0.0592378377914429 12.0
74.140283542 74.3320770263672 13.0
8.2389699424 23.0391082763672 14.0
0 0.0369642972946167 15.0
0 -0.0211030244827271 16.0
27.284019722 27.1898498535156 17.0
0 0.0417388677597046 18.0
0 0.0356062650680542 19.0
0 0.030981183052063 20.0
0 -0.0433756113052368 21.0
0 -0.0207444429397583 22.0
33.992217249 33.2529792785645 23.0
0 0.0835598707199097 24.0
0 0.437699854373932 25.0
0 0.0440739393234253 26.0
0 0.0853055715560913 27.0
0 0.00623738765716553 28.0
0 -0.0449310541152954 29.0
0 0.0409106016159058 30.0
0 -0.0632263422012329 31.0
0 0.0153335332870483 32.0
0 0.128549456596375 33.0
0 -0.0103608369827271 34.0
0 -0.0462124347686768 35.0
0 0.0529288053512573 36.0
0 -0.0196067094802856 37.0
0 0.0371159315109253 38.0
0 -0.0051954984664917 39.0
0 -0.00826609134674072 40.0
0 -0.0045245885848999 41.0
0 -0.0680557489395142 42.0
301.78185481 292.691619873047 43.0
0 0.00520789623260498 44.0
0 0.0201557874679565 45.0
0 -0.0160356760025024 46.0
0 -0.0330307483673096 47.0
0 -0.0508538484573364 48.0
0 0.276742488145828 49.0
0.082599040763 0.085985541343689 50.0
2.1902961354 1.00512361526489 51.0
0 -0.0430694818496704 52.0
0 -0.0144833326339722 53.0
0 -0.00854265689849854 54.0
0 -0.0918921232223511 55.0
0 -0.0215142965316772 56.0
0 -0.0458959341049194 57.0
0 -0.110866904258728 58.0
0 -0.00486695766448975 59.0
0 -0.191485285758972 60.0
0 -0.064674973487854 61.0
0 0.859198331832886 62.0
222.2919626 257.114166259766 0.0
0 -0.018201470375061 1.0
0 -0.0968495607376099 2.0
0 0.0112954378128052 3.0
0 -0.0263475179672241 4.0
0 -0.0710326433181763 5.0
0 0.0327895879745483 6.0
0 -0.0558856725692749 7.0
0 0.0604056119918823 8.0
0 0.0243915319442749 9.0
0 -0.0337779521942139 10.0
0 0.0578691959381104 11.0
0 -0.0164095163345337 12.0
77.473098842 75.9316635131836 13.0
12.334580639 22.8996028900146 14.0
0 0.0197409391403198 15.0
0 -0.0148197412490845 16.0
35.764464649 35.3321380615234 17.0
0 -0.0379019975662231 18.0
0 -0.000342249870300293 19.0
0 -0.0402470827102661 20.0
0 -0.00728011131286621 21.0
0 0.0289982557296753 22.0
32.733605826 32.4054222106934 23.0
0 -0.0906754732131958 24.0
0 1.06714487075806 25.0
0 0.0893625020980835 26.0
0 0.0164493918418884 27.0
0 -0.032807469367981 28.0
0 -0.0524748563766479 29.0
0 -0.0625882148742676 30.0
0 -0.0341399908065796 31.0
0 -0.0306893587112427 32.0
0 -0.0163999795913696 33.0
0 -0.000935912132263184 34.0
0 -0.0513581037521362 35.0
0 0.195072293281555 36.0
0 -0.0471011400222778 37.0
0 -0.0478156805038452 38.0
0 0.0838450193405151 39.0
0 0.126171469688416 40.0
0 -0.0306545495986938 41.0
0 -0.0965648889541626 42.0
284.17695069 296.876922607422 43.0
0 0.00317370891571045 44.0
0 -0.00341904163360596 45.0
0 0.00467193126678467 46.0
0 -0.0324335098266602 47.0
0 -0.000351548194885254 48.0
0 0.289094388484955 49.0
0.075431432874 0.0856114625930786 50.0
2.0171134421 0.953196942806244 51.0
0 -0.0104225873947144 52.0
0 0.00161278247833252 53.0
0 -0.00211751461029053 54.0
0 -0.0646213293075562 55.0
0 -0.0245963335037231 56.0
0 -0.0171595811843872 57.0
0 -0.0809110403060913 58.0
0 -0.0483843088150024 59.0
0 0.0171233415603638 60.0
0 -0.0583816766738892 61.0
0 0.826791167259216 62.0
250.40868616 253.015426635742 0.0
0 -0.0341650247573853 1.0
0 -0.0671232938766479 2.0
0 -0.0504025220870972 3.0
0 -0.0633493661880493 4.0
0 0.0274444818496704 5.0
0 0.162406325340271 6.0
0 -0.0216819047927856 7.0
0 0.00946915149688721 8.0
0 0.181489825248718 9.0
0 0.0120283365249634 10.0
0 -0.0102478265762329 11.0
0 0.134099364280701 12.0
48.986813036 49.3432540893555 13.0
34.521160553 33.220947265625 14.0
0 -0.0150829553604126 15.0
0 -0.0574487447738647 16.0
28.579983834 29.0309944152832 17.0
0 -0.0299960374832153 18.0
0 -0.0192440748214722 19.0
0 0.0810734033584595 20.0
0 0.079074501991272 21.0
0 -0.0371071100234985 22.0
32.437606246 32.9155616760254 23.0
0 -0.110056042671204 24.0
20.612312119 0.683636546134949 25.0
0 -0.0125848054885864 26.0
0 0.12347424030304 27.0
0 0.0154274702072144 28.0
0 -0.147916913032532 29.0
0 0.162450194358826 30.0
0 -0.00585663318634033 31.0
0 0.0793622732162476 32.0
0 0.06462562084198 33.0
0 -0.0670567750930786 34.0
0 -0.00140666961669922 35.0
0 -0.119982361793518 36.0
0 -0.037483811378479 37.0
0 -0.00354468822479248 38.0
0 0.0466779470443726 39.0
0 0.176999926567078 40.0
0 -0.0239043235778809 41.0
0 0.030661940574646 42.0
309.11879795 296.523010253906 43.0
0 -0.0887037515640259 44.0
0 -0.0191956758499146 45.0
0 -0.0375441312789917 46.0
0 -0.12183678150177 47.0
0 -0.0394474267959595 48.0
0 0.280462026596069 49.0
0.061378010895 0.0494879484176636 50.0
1.3499419302 1.0175678730011 51.0
0 -0.0850554704666138 52.0
0 -0.0336436033248901 53.0
0 0.0170317888259888 54.0
0 -0.138328433036804 55.0
0 0.00979959964752197 56.0
0 -0.00612390041351318 57.0
0 -0.0554856061935425 58.0
0 -0.0543226003646851 59.0
0 0.0735131502151489 60.0
0 0.0627685785293579 61.0
0 0.907724380493164 62.0
250.49109626 254.520126342773 0.0
0 -0.028689980506897 1.0
0 -0.0212804079055786 2.0
0 -0.0110443830490112 3.0
0 0.00877606868743896 4.0
0 0.0474947690963745 5.0
0 -0.038730263710022 6.0
0 -0.00311863422393799 7.0
0 -0.044718861579895 8.0
0 -0.0106083154678345 9.0
0 -0.00708544254302979 10.0
0 -0.00157809257507324 11.0
0 -0.0283762216567993 12.0
65.127533516 65.3778991699219 13.0
36.304656101 32.5302200317383 14.0
0 -0.00873386859893799 15.0
0 -0.0176690816879272 16.0
32.931399479 32.4547729492188 17.0
0 -0.00596725940704346 18.0
0 0.0216783285140991 19.0
0 0.0144027471542358 20.0
0 -0.0351389646530151 21.0
0 -0.0468593835830688 22.0
36.4068468 35.0574607849121 23.0
0 0.135439276695251 24.0
0 0.327566206455231 25.0
0 -0.00268447399139404 26.0
0 0.0914984941482544 27.0
0 0.0184911489486694 28.0
0 -0.0630837678909302 29.0
0 -0.0609188079833984 30.0
0 -0.0474966764450073 31.0
0 -0.0279513597488403 32.0
0 -0.0255976915359497 33.0
0 -0.0449942350387573 34.0
0 -0.0704225301742554 35.0
0 0.0735365152359009 36.0
0 -0.0336822271347046 37.0
0 -0.00550162792205811 38.0
0 0.00236952304840088 39.0
0 0.0815504789352417 40.0
0 -0.0681127309799194 41.0
0 0.0812801122665405 42.0
303.72890241 295.902526855469 43.0
0 -0.0357280969619751 44.0
0 0.0350404977798462 45.0
0 -0.0420688390731812 46.0
0 -0.0175954103469849 47.0
0 0.0284956693649292 48.0
0 0.247058808803558 49.0
0.082687637955 0.0639303922653198 50.0
1.948681639 0.941990852355957 51.0
0 -0.0293279886245728 52.0
0 -0.00843131542205811 53.0
0 0.0054699182510376 54.0
0 -0.0281721353530884 55.0
0 0.0257154703140259 56.0
0 -0.00573694705963135 57.0
0 -0.0836976766586304 58.0
0 -0.0738128423690796 59.0
0 -0.0149701833724976 60.0
0 -0.0141891241073608 61.0
0 0.838004469871521 62.0
229.26889286 252.23698425293 0.0
0 -0.0592924356460571 1.0
0 -0.0733567476272583 2.0
0 0.0681313276290894 3.0
0 -0.0753451585769653 4.0
0 -0.0205129384994507 5.0
0 -0.135383248329163 6.0
0 -0.00813329219818115 7.0
0 -0.0591171979904175 8.0
0 0.0309432744979858 9.0
0 0.0564893484115601 10.0
0 0.0903646945953369 11.0
0 -0.0227991342544556 12.0
76.716671706 77.3034362792969 13.0
33.1084879 20.0373668670654 14.0
0 -0.0790258646011353 15.0
0 0.0276817083358765 16.0
28.217203973 27.8323020935059 17.0
0 -0.0213505029678345 18.0
0 0.0239299535751343 19.0
0 0.036298394203186 20.0
0 -0.00673842430114746 21.0
0 -0.022107720375061 22.0
33.852899347 33.503288269043 23.0
0 -0.0644818544387817 24.0
0 0.649494409561157 25.0
0 -0.00295794010162354 26.0
0 0.0175074338912964 27.0
0 -0.0519601106643677 28.0
0 -0.0391846895217896 29.0
0 0.0946258306503296 30.0
0 -0.0464862585067749 31.0
0 -0.0120526552200317 32.0
0 -0.0475989580154419 33.0
0 0.0150078535079956 34.0
0 -0.00158321857452393 35.0
0 0.145542025566101 36.0
0 -0.0153964757919312 37.0
0 0.0181256532669067 38.0
0 0.0186790227890015 39.0
0 0.0329397916793823 40.0
0 0.0562452077865601 41.0
0 -0.0688396692276001 42.0
293.81677313 294.633239746094 43.0
0 0.0316404104232788 44.0
0 0.0270277261734009 45.0
0 -0.0264273881912231 46.0
0 -0.0318621397018433 47.0
0 0.0153430700302124 48.0
0 0.244905233383179 49.0
0.10014390631 0.0706242322921753 50.0
1.8850201977 0.929517507553101 51.0
0 -0.111276030540466 52.0
0 -0.0669022798538208 53.0
0 0.0303796529769897 54.0
0 0.00374186038970947 55.0
0 -0.068968653678894 56.0
0 -0.0440267324447632 57.0
0 -0.0555452108383179 58.0
0 -0.0871354341506958 59.0
0 0.0442990064620972 60.0
0 0.0577069520950317 61.0
0 0.984948396682739 62.0
269.68141378 257.089263916016 0.0
0 -0.0557256937026978 1.0
0 0.000577569007873535 2.0
0 -0.073825478553772 3.0
0 -0.0505968332290649 4.0
0 0.0361112356185913 5.0
0 -0.0497876405715942 6.0
0 0.0130981206893921 7.0
0 -0.0202866792678833 8.0
0 0.0452102422714233 9.0
0 -0.00084376335144043 10.0
0 0.0111526250839233 11.0
0 0.0797790288925171 12.0
55.552462707 55.4926223754883 13.0
39.319588548 36.1761474609375 14.0
0 -0.0353199243545532 15.0
0 -0.0666881799697876 16.0
33.709039357 33.4376754760742 17.0
0 0.0104550123214722 18.0
0 -0.00786125659942627 19.0
0 -0.00106251239776611 20.0
0 -0.0317357778549194 21.0
0 0.0541332960128784 22.0
37.436243462 36.5938911437988 23.0
0 -0.0577322244644165 24.0
0 0.482788980007172 25.0
0 -0.0409923791885376 26.0
0 0.0175856351852417 27.0
0 -0.0143738985061646 28.0
0 -0.0549243688583374 29.0
0 -0.0642626285552979 30.0
0 -0.0191575288772583 31.0
0 -0.0513859987258911 32.0
0 0.0351566076278687 33.0
0 -0.0622228384017944 34.0
0 -0.0174119472503662 35.0
0 -0.0879162549972534 36.0
0 -0.048242449760437 37.0
0 -0.0197604894638062 38.0
0 0.0326091051101685 39.0
0 0.0348080396652222 40.0
0 0.0583430528640747 41.0
0 -0.0290037393569946 42.0
294.64026527 294.246643066406 43.0
0 -0.0133100748062134 44.0
0 0.0500149726867676 45.0
0 0.0124906301498413 46.0
0 -0.0639487504959106 47.0
0 0.00733268260955811 48.0
0.59764420091 0.326647698879242 49.0
0.086474010776 0.0671290159225464 50.0
1.7747647144 0.858973383903503 51.0
0 -0.102436184883118 52.0
0 -0.0724369287490845 53.0
0 -0.0205279588699341 54.0
0 -0.12108314037323 55.0
0 0.0187891721725464 56.0
0 -0.0485531091690063 57.0
0 -0.015351414680481 58.0
0 -0.0222238302230835 59.0
0 -0.00356209278106689 60.0
0 0.0702356100082397 61.0
0 0.910251021385193 62.0
291.08171095 259.249725341797 0.0
0 -0.169043660163879 1.0
0 0.0527530908584595 2.0
0 0.0156432390213013 3.0
0 -0.0372351408004761 4.0
0 0.0079272985458374 5.0
0 0.0510929822921753 6.0
0 -0.0490490198135376 7.0
0 0.00488519668579102 8.0
0 -0.0186887979507446 9.0
0 -0.0167053937911987 10.0
0 0.00903308391571045 11.0
0 0.0438436269760132 12.0
53.78858457 53.1011428833008 13.0
27.039991377 40.7431106567383 14.0
0 -0.000908970832824707 15.0
0 -0.035054087638855 16.0
27.122923807 26.3192138671875 17.0
0 0.0948077440261841 18.0
0 0.0132828950881958 19.0
0 0.00987327098846436 20.0
0 0.00368607044219971 21.0
0 -0.0200291872024536 22.0
26.368936495 26.8611278533936 23.0
0 -0.092382550239563 24.0
0 1.19620895385742 25.0
0 0.108315348625183 26.0
0 0.0733906030654907 27.0
0 0.0742791891098022 28.0
0 -0.0214279890060425 29.0
0 -0.120002865791321 30.0
0 -0.0520285367965698 31.0
0 -0.0632287263870239 32.0
0 0.0631676912307739 33.0
0 -0.087530255317688 34.0
0 0.2399982213974 35.0
0 0.0556102991104126 36.0
0 -0.0526272058486938 37.0
0 -0.0515567064285278 38.0
0 0.0870925188064575 39.0
0 -0.02161705493927 40.0
0 -0.027837872505188 41.0
0 -0.010266900062561 42.0
305.87768424 295.399841308594 43.0
0 0.00785386562347412 44.0
0 0.0812807083129883 45.0
0 -0.00132524967193604 46.0
0 -0.0415047407150269 47.0
0 -0.0177747011184692 48.0
0 0.255532741546631 49.0
0.068253031613 0.0930582284927368 50.0
1.5096010009 1.03306746482849 51.0
0 -0.10971987247467 52.0
0 -0.0851598978042603 53.0
0 -0.0179096460342407 54.0
0 -0.105715155601501 55.0
0 0.0740922689437866 56.0
0 0.0187662839889526 57.0
0 0.0239821672439575 58.0
0 -0.000666260719299316 59.0
0 -0.0693718194961548 60.0
0 -0.0392793416976929 61.0
0 0.82319974899292 62.0
307.75593221 261.457305908203 0.0
0 -0.0581756830215454 1.0
0 -0.0113962888717651 2.0
0 -0.0591238737106323 3.0
0 -5.68628311157227e-05 4.0
0 -0.0934773683547974 5.0
0 0.0616191625595093 6.0
0 -0.232673287391663 7.0
0 -0.0264661312103271 8.0
0 0.102459311485291 9.0
0 -0.0312193632125854 10.0
0 -0.0449627637863159 11.0
0 0.0416892766952515 12.0
51.086684797 49.2899131774902 13.0
33.310005547 29.2816257476807 14.0
0 -0.0492416620254517 15.0
0 -0.0503700971603394 16.0
30.008862643 30.463415145874 17.0
0 -0.0314804315567017 18.0
0 -0.0303019285202026 19.0
0 0.0668851137161255 20.0
0 0.0220638513565063 21.0
0 -0.0497614145278931 22.0
30.729808814 30.7759342193604 23.0
0 -0.120635151863098 24.0
0 0.989591956138611 25.0
0 0.0350226163864136 26.0
0 0.0378558039665222 27.0
0 0.0384052991867065 28.0
0 -0.0471960306167603 29.0
0 0.103101491928101 30.0
0 -0.0285323858261108 31.0
0 -0.0299040079116821 32.0
0 -0.139223694801331 33.0
0 -0.0130959749221802 34.0
0 -0.0468306541442871 35.0
0 -0.133565068244934 36.0
0 -0.0252534151077271 37.0
0 -0.00207149982452393 38.0
0 0.00205433368682861 39.0
0 0.140456080436707 40.0
0 -0.0117201805114746 41.0
0 0.0239289999008179 42.0
307.12761265 293.592742919922 43.0
0 0.0210088491439819 44.0
0 0.14370584487915 45.0
0 -0.0244308710098267 46.0
0 0.0101350545883179 47.0
0 -0.0193129777908325 48.0
0.97733821256 0.310120284557343 49.0
0.092395832605 0.0599173307418823 50.0
2.2745948555 0.835910797119141 51.0
0 -0.0628780126571655 52.0
0 -0.0583368539810181 53.0
0 -0.0588880777359009 54.0
0 0.0500575304031372 55.0
0 -0.121140360832214 56.0
0 0.0138481855392456 57.0
0 0.0156310796737671 58.0
0 -0.0629004240036011 59.0
0 -0.0167845487594604 60.0
0 0.0873502492904663 61.0
0 0.989185273647308 62.0
279.76466269 260.232757568359 0.0
0 -0.0199748277664185 1.0
0 -0.0505388975143433 2.0
0 -0.0119866132736206 3.0
0 -0.0225228071212769 4.0
0 0.01811683177948 5.0
0 0.147832274436951 6.0
0 -0.087161660194397 7.0
0 0.0393067598342896 8.0
0 0.0236457586288452 9.0
0 -0.0057605504989624 10.0
0 -0.0374618768692017 11.0
0 -0.138394951820374 12.0
56.702884893 56.1654396057129 13.0
30.215250916 38.6306495666504 14.0
0 -0.0301166772842407 15.0
0 -0.0196574926376343 16.0
32.583872401 32.3160705566406 17.0
0 0.0469385385513306 18.0
0 0.0574065446853638 19.0
0 -0.0125671625137329 20.0
0 -0.0393143892288208 21.0
0 -0.0175758600234985 22.0
38.783805736 37.8622627258301 23.0
0 -0.0357667207717896 24.0
0 0.554769098758698 25.0
0 0.107063174247742 26.0
0 0.0326613187789917 27.0
0 0.0844095945358276 28.0
0 -0.0369099378585815 29.0
0 -0.0550171136856079 30.0
0 -0.0425628423690796 31.0
0 -0.0463544130325317 32.0
0 0.0108350515365601 33.0
0 -0.0621205568313599 34.0
0 0.0524042844772339 35.0
0 -0.0887571573257446 36.0
0 -0.0263525247573853 37.0
0 0.00334560871124268 38.0
0 0.0324889421463013 39.0
0 0.061367392539978 40.0
0 -0.0601068735122681 41.0
0 -0.105129837989807 42.0
287.83828545 295.286102294922 43.0
0 0.0846394300460815 44.0
0 -0.0698293447494507 45.0
0 -0.0420528650283813 46.0
0 -0.0415621995925903 47.0
0 0.0174921751022339 48.0
0 0.29664933681488 49.0
0.065079268487 0.0737318992614746 50.0
1.6267071442 0.996636688709259 51.0
0 -0.0644522905349731 52.0
0 0.011522650718689 53.0
0 -0.0753234624862671 54.0
0 -0.0257276296615601 55.0
0 -0.0511034727096558 56.0
0 -0.00397932529449463 57.0
0 -0.0647844076156616 58.0
0 -0.0850211381912231 59.0
0 0.0303655862808228 60.0
0 0.00518858432769775 61.0
0 0.834989190101624 62.0
274.38931176 260.305358886719 0.0
0 -0.0826736688613892 1.0
0 -0.0392180681228638 2.0
0 0.0107690095901489 3.0
0 -0.0277453660964966 4.0
0 -0.141919255256653 5.0
0 -0.00743448734283447 6.0
0 -0.101207375526428 7.0
0 0.0473027229309082 8.0
0 -0.0289713144302368 9.0
0 0.0304557085037231 10.0
0 -0.0511618852615356 11.0
0 -0.115606904029846 12.0
50.596074853 49.5022583007812 13.0
37.448987404 46.0441932678223 14.0
0 -0.0901695489883423 15.0
0 -0.0245422124862671 16.0
36.513332711 35.9125556945801 17.0
0 -0.0335367918014526 18.0
0 -0.0458523035049438 19.0
0 0.0833290815353394 20.0
0 -0.0268839597702026 21.0
0 -0.0546432733535767 22.0
34.967581683 34.4165229797363 23.0
0 -0.106927514076233 24.0
0 1.1234438419342 25.0
0 -0.0340760946273804 26.0
0 0.0218466520309448 27.0
0 0.000425219535827637 28.0
0 -0.0244423151016235 29.0
0 -0.070926308631897 30.0
0 -0.017612099647522 31.0
0 0.0091012716293335 32.0
0 0.0048748254776001 33.0
0 -0.0368803739547729 34.0
0 -0.0205636024475098 35.0
0 0.0157455205917358 36.0
0 -0.0303610563278198 37.0
0 -0.0543261766433716 38.0
0 0.0778745412826538 39.0
0 0.149029135704041 40.0
0 -0.0495561361312866 41.0
0 -0.0306469202041626 42.0
307.84354457 296.211395263672 43.0
0 0.0102952718734741 44.0
0 -0.0300517082214355 45.0
0 -0.0609818696975708 46.0
0 -0.0580791234970093 47.0
0 0.00863635540008545 48.0
0 0.26338267326355 49.0
0.097997714581 0.0797845125198364 50.0
2.2543396636 0.921557247638702 51.0
0 -0.101392388343811 52.0
0 -0.0768967866897583 53.0
0 -0.0277241468429565 54.0
0 -0.0353909730911255 55.0
0 -0.0819994211196899 56.0
0 0.0104864835739136 57.0
0 -0.0910342931747437 58.0
0 -0.0784338712692261 59.0
0 -0.096443772315979 60.0
0 0.0799177885055542 61.0
0 0.92611563205719 62.0
255.92779805 256.865661621094 0.0
0 -0.0823208093643188 1.0
0 -0.0305625200271606 2.0
0 -0.0319951772689819 3.0
0 -0.030023455619812 4.0
0 -0.0553303956985474 5.0
0 0.0249465703964233 6.0
0 -0.0291742086410522 7.0
0 -0.0632367730140686 8.0
0 0.0598052740097046 9.0
0 -0.0872882604598999 10.0
0 -0.0418716669082642 11.0
0 -0.123346924781799 12.0
54.109720708 53.9986000061035 13.0
47.114327095 42.3964347839355 14.0
0 0.0236881971359253 15.0
0 -0.0100423097610474 16.0
39.72562017 39.2433166503906 17.0
0 0.0229846239089966 18.0
0 0.00856673717498779 19.0
0 -0.0343793630599976 20.0
0 -0.00201690196990967 21.0
0 -0.0597821474075317 22.0
29.470716346 29.3138790130615 23.0
0 0.0110253095626831 24.0
0 0.552685678005219 25.0
0 0.0982049703598022 26.0
0 0.0893930196762085 27.0
0 -0.0340644121170044 28.0
0 -0.0397063493728638 29.0
0 -0.0738492012023926 30.0
0 -0.0475853681564331 31.0
0 -0.0073472261428833 32.0
0 0.0755611658096313 33.0
0 -0.0384589433670044 34.0
0 -0.0300823450088501 35.0
0 0.143012166023254 36.0
0 -0.0166150331497192 37.0
0 -0.0240484476089478 38.0
0 0.0672076940536499 39.0
0 0.0792292356491089 40.0
0 -0.0199328660964966 41.0
0 0.0621370077133179 42.0
296.15998686 297.290466308594 43.0
0 -0.0435367822647095 44.0
0 0.00970220565795898 45.0
0 0.014985203742981 46.0
0 -0.0685340166091919 47.0
0 0.0367835760116577 48.0
0 0.249451518058777 49.0
0.0681768052 0.0518084764480591 50.0
2.206507185 0.873971283435822 51.0
0 -0.0545238256454468 52.0
0 -0.0342141389846802 53.0
0 0.063611626625061 54.0
0 -0.0648266077041626 55.0
0 -0.0532008409500122 56.0
0 -0.0314284563064575 57.0
0 -0.0389722585678101 58.0
0 -0.0765999555587769 59.0
0 -0.0288239717483521 60.0
0 0.0346323251724243 61.0
0 0.794465065002441 62.0
252.34967923 258.967559814453 0.0
0 -0.0756937265396118 1.0
0 -0.0383251905441284 2.0
0 -0.00920498371124268 3.0
0 -0.0255037546157837 4.0
0 -0.0907388925552368 5.0
0 0.0616153478622437 6.0
0 -0.0657666921615601 7.0
0 -0.0192485451698303 8.0
0 0.120386481285095 9.0
0 -0.0126461982727051 10.0
0 -0.0216916799545288 11.0
0 -0.224688649177551 12.0
46.762597964 46.6699104309082 13.0
35.003813457 35.150447845459 14.0
0 -0.143509984016418 15.0
0 -0.0371271371841431 16.0
39.053064695 38.5253982543945 17.0
0 -0.00894558429718018 18.0
0 0.0106614828109741 19.0
0 0.0519524812698364 20.0
0 -0.00815212726593018 21.0
0 -0.0304275751113892 22.0
40.406945709 41.8594627380371 23.0
0 0.0986231565475464 24.0
0 0.402438879013062 25.0
0 -0.0528723001480103 26.0
0 -0.0449384450912476 27.0
0 0.538394331932068 28.0
0 -0.0642009973526001 29.0
0 -0.0484871864318848 30.0
0 -0.129235625267029 31.0
0 -0.0374222993850708 32.0
0 -0.184230923652649 33.0
0 -0.0584396123886108 34.0
0 0.0295398235321045 35.0
0 0.0242620706558228 36.0
0 0.0788470506668091 37.0
0 0.0321923494338989 38.0
0 -0.0939701795578003 39.0
0 0.0116556882858276 40.0
0 0.00114834308624268 41.0
0 -0.0625635385513306 42.0
292.8233387 296.289215087891 43.0
0 0.0215734243392944 44.0
0 0.0885300040245056 45.0
0 0.000302672386169434 46.0
0 -0.0221303701400757 47.0
0 -0.0617491006851196 48.0
0 0.239889144897461 49.0
0.099050126349 0.0985110998153687 50.0
1.6664045094 0.889166831970215 51.0
0 0.00585401058197021 52.0
0 -0.024112343788147 53.0
0 -0.0398122072219849 54.0
0 -0.16941511631012 55.0
0 0.136183142662048 56.0
0 0.0322960615158081 57.0
0 -0.154811978340149 58.0
0 0.00760972499847412 59.0
0 0.0915807485580444 60.0
0 -0.0944696664810181 61.0
0 0.82637894153595 62.0
275.3043677 246.826873779297 0.0
0 -0.0578275918960571 1.0
0 0.0277806520462036 2.0
0 0.0304585695266724 3.0
0 0.011492133140564 4.0
0 0.00623476505279541 5.0
0 0.102135539054871 6.0
0 0.0532945394515991 7.0
0 -0.0336641073226929 8.0
0 0.0501440763473511 9.0
0 -0.023440957069397 10.0
0 0.0202564001083374 11.0
0 -0.00609362125396729 12.0
64.600019834 63.9132957458496 13.0
36.268573339 29.9376392364502 14.0
0 -0.0238977670669556 15.0
0 -0.070858359336853 16.0
30.887518839 30.9625949859619 17.0
0 0.0491617918014526 18.0
0 -0.00256741046905518 19.0
0 0.00838911533355713 20.0
0 -0.00309312343597412 21.0
0 -0.0151630640029907 22.0
32.934773826 31.9541320800781 23.0
0 -0.047153115272522 24.0
5.4103287499 0.941912055015564 25.0
0 0.00282895565032959 26.0
0 0.0373314619064331 27.0
0 0.0451616048812866 28.0
0 -0.0370396375656128 29.0
0 -0.0563100576400757 30.0
0 -0.0569378137588501 31.0
0 -0.0651484727859497 32.0
0 -0.151281476020813 33.0
0 -0.00726854801177979 34.0
0 -0.0254441499710083 35.0
0 0.120865702629089 36.0
0 0.182897925376892 37.0
0 0.0385242700576782 38.0
0 0.0399323701858521 39.0
0 -0.0299197435379028 40.0
0 -0.0164450407028198 41.0
0 -0.0338245630264282 42.0
300.4473475 296.307800292969 43.0
0 0.0765522718429565 44.0
0 -0.0844569206237793 45.0
0 -0.0269421339035034 46.0
0 -0.0489100217819214 47.0
0 0.0257881879806519 48.0
1.9204469612 0.360146790742874 49.0
0.083858651558 0.0794786214828491 50.0
1.7433263219 0.926989495754242 51.0
0 -0.000299572944641113 52.0
0 -0.023145318031311 53.0
0 0.00922620296478271 54.0
0 0.00582778453826904 55.0
0 -0.0594247579574585 56.0
0 -0.0575410127639771 57.0
0 -0.0388749837875366 58.0
0 -0.0352853536605835 59.0
0 0.0540064573287964 60.0
0 -0.0331689119338989 61.0
0 0.854846715927124 62.0
284.68101815 258.901580810547 0.0
0 -0.0260540246963501 1.0
0 0.0737987756729126 2.0
0 -0.0793753862380981 3.0
0 -0.0518292188644409 4.0
0 -0.184414982795715 5.0
0 0.023668646812439 6.0
0 -0.124837517738342 7.0
0 -0.0395179390907288 8.0
0 0.114602446556091 9.0
0 0.015870213508606 10.0
0 -0.0137847661972046 11.0
0 -0.0977073907852173 12.0
65.063950928 63.8441390991211 13.0
39.603872045 35.3249855041504 14.0
0 -0.094882607460022 15.0
0 -0.020746111869812 16.0
31.704376234 31.4143924713135 17.0
0 0.00415408611297607 18.0
0 0.0152739286422729 19.0
0 0.0345858335494995 20.0
0 0.000833392143249512 21.0
0 0.0180281400680542 22.0
34.678551321 34.0354881286621 23.0
0 -0.0417824983596802 24.0
0 0.593817353248596 25.0
0 0.092150092124939 26.0
0 -0.0432034730911255 27.0
0 -0.0302101373672485 28.0
0 -0.0454698801040649 29.0
0 -0.0378183126449585 30.0
0 -0.00688087940216064 31.0
0 -0.00835192203521729 32.0
0 -0.0368388891220093 33.0
0 -0.0258001089096069 34.0
0 -0.00655591487884521 35.0
0 0.0373281240463257 36.0
0 -0.021694540977478 37.0
0 0.0619012117385864 38.0
0 0.0210071802139282 39.0
0 0.0876845121383667 40.0
0 -0.0418198108673096 41.0
0 -0.125180840492249 42.0
288.75249874 294.778228759766 43.0
0 0.030504584312439 44.0
0 0.115955889225006 45.0
0 0.00947797298431396 46.0
0 -0.0372768640518188 47.0
0 0.0475195646286011 48.0
0 0.286017835140228 49.0
0.10433613882 0.0811623334884644 50.0
1.9666064809 0.994478404521942 51.0
0 -0.0717393159866333 52.0
0 0.023584246635437 53.0
0 -0.00352585315704346 54.0
0 -0.0260342359542847 55.0
0 -0.0479253530502319 56.0
0 0.0100640058517456 57.0
0 0.0669792890548706 58.0
0 -0.09907066822052 59.0
0 -0.0967670679092407 60.0
0 0.0724965333938599 61.0
0 0.897851169109344 62.0
287.28037552 255.635360717773 0.0
0 -0.0404287576675415 1.0
0 0.0171157121658325 2.0
0 -0.0424724817276001 3.0
0 -0.0225614309310913 4.0
0 0.0595935583114624 5.0
0 0.0117925405502319 6.0
0 0.0137442350387573 7.0
0 -0.0648764371871948 8.0
0 -0.0278764963150024 9.0
0 -0.00555956363677979 10.0
0 -0.0219087600708008 11.0
0 0.0432599782943726 12.0
67.935250884 66.7700576782227 13.0
15.197371828 29.1997737884521 14.0
0 -0.192175507545471 15.0
0 -0.0130249261856079 16.0
31.122765614 32.009593963623 17.0
0 -0.0420972108840942 18.0
0 0.0118728876113892 19.0
0 -0.0211268663406372 20.0
0 -0.00216567516326904 21.0
0 0.0209044218063354 22.0
29.325986481 28.2342700958252 23.0
0 -0.0677405595779419 24.0
0 0.698125541210175 25.0
0 0.0683640241622925 26.0
0 0.0903340578079224 27.0
0 -0.00708353519439697 28.0
0 -0.0679060220718384 29.0
0 0.187001347541809 30.0
0 -0.0551072359085083 31.0
0 -0.0404015779495239 32.0
0 0.0149821043014526 33.0
0 -0.0450347661972046 34.0
0 -0.0442889928817749 35.0
0 0.0946213006973267 36.0
0 -0.0689421892166138 37.0
0 0.025732159614563 38.0
0 0.0603474378585815 39.0
0 0.0347107648849487 40.0
0 -0.0775511264801025 41.0
0 0.0088123083114624 42.0
280.86975552 296.158599853516 43.0
0 0.0129770040512085 44.0
0 0.0942848324775696 45.0
0 -0.0240017175674438 46.0
0 0.0146974325180054 47.0
0 0.033079981803894 48.0
2.5469392214 0.263319671154022 49.0
0.091455845681 0.0585533380508423 50.0
1.6066944094 0.992668390274048 51.0
0 -0.0364047288894653 52.0
0 -0.059083104133606 53.0
0 0.0266939401626587 54.0
0 -0.0222688913345337 55.0
0 -0.0536245107650757 56.0
0 -0.0031358003616333 57.0
0 -0.0497616529464722 58.0
0 0.00433361530303955 59.0
0 -0.068299412727356 60.0
0 0.0778266191482544 61.0
0 0.930407583713531 62.0
237.54105769 254.555221557617 0.0
0 0.0233315229415894 1.0
0 -0.0753494501113892 2.0
0 -0.00557243824005127 3.0
0 -0.060956597328186 4.0
0 -0.0106164216995239 5.0
0 -0.11357319355011 6.0
0 -0.00164282321929932 7.0
0 0.0293240547180176 8.0
0 0.0521081686019897 9.0
0 0.0159986019134521 10.0
0 -0.00112950801849365 11.0
0 0.0345834493637085 12.0
56.893944802 56.167594909668 13.0
43.526331043 42.8901901245117 14.0
0 0.00237452983856201 15.0
0 -0.0785282850265503 16.0
37.807135045 36.7952194213867 17.0
0 -0.000367283821105957 18.0
0 0.0146335363388062 19.0
0 0.0401245355606079 20.0
0 -0.032107949256897 21.0
0 -0.0502077341079712 22.0
39.210994689 38.7867088317871 23.0
0 -0.0948334932327271 24.0
0 0.682648777961731 25.0
0 0.0965327024459839 26.0
0 0.128164172172546 27.0
0 -0.017856240272522 28.0
0 -0.0671225786209106 29.0
0 -0.0588408708572388 30.0
0 -0.0317906141281128 31.0
0 -0.019201397895813 32.0
0 0.0360258817672729 33.0
0 -0.0482715368270874 34.0
0 0.00205504894256592 35.0
0 0.0754820108413696 36.0
0 0.0259901285171509 37.0
0 0.0246320962905884 38.0
0 0.00981581211090088 39.0
0 0.0708955526351929 40.0
0 -0.0470770597457886 41.0
0 0.0936979055404663 42.0
292.56560845 292.093383789062 43.0
0 -0.023389458656311 44.0
0 0.0894653797149658 45.0
0 -0.0310107469558716 46.0
0 -0.0294817686080933 47.0
0 0.0317095518112183 48.0
0 0.273635149002075 49.0
0.08942268068 0.0747684240341187 50.0
2.1812558595 0.906634390354156 51.0
0 -0.0652924776077271 52.0
0 -0.0821744203567505 53.0
0 0.00872695446014404 54.0
0 -0.0549012422561646 55.0
0 -0.0586396455764771 56.0
0 -0.00155985355377197 57.0
0 -0.0824240446090698 58.0
0 -0.0657681226730347 59.0
0 -0.0184248685836792 60.0
0 -0.00673925876617432 61.0
0 0.832450866699219 62.0
275.98517783 259.739807128906 0.0
0 -0.145191788673401 1.0
0 0.00622403621673584 2.0
0 -0.0466498136520386 3.0
0 -0.0228837728500366 4.0
0 -0.00653278827667236 5.0
0 -0.0327259302139282 6.0
0 -0.0572377443313599 7.0
0 0.00746643543243408 8.0
0 -0.0277420282363892 9.0
0 -0.0341850519180298 10.0
0 -0.0529111623764038 11.0
0 0.106916785240173 12.0
50.994106635 48.5594215393066 13.0
48.216071814 43.5886459350586 14.0
0 -0.0340439081192017 15.0
0 -0.030856728553772 16.0
25.931175343 26.2485637664795 17.0
0 -0.0392931699752808 18.0
0 0.00178039073944092 19.0
0 -0.00655472278594971 20.0
0 0.038429856300354 21.0
0 -0.00586950778961182 22.0
41.861145546 41.6601104736328 23.0
0 -0.11260998249054 24.0
0 0.570592999458313 25.0
0 0.0896071195602417 26.0
0 -0.0454345941543579 27.0
0 0.130347609519958 28.0
0 -0.118497967720032 29.0
0 0.0333434343338013 30.0
0 -0.0733636617660522 31.0
0 -0.00240004062652588 32.0
0 0.0291670560836792 33.0
0 -0.00239050388336182 34.0
0 -0.0318697690963745 35.0
0 0.177024245262146 36.0
0 -0.0579763650894165 37.0
0 0.0106703042984009 38.0
0 0.0407568216323853 39.0
0 -0.0514284372329712 40.0
0 -0.0193330049514771 41.0
0 0.124143958091736 42.0
278.74188031 295.725646972656 43.0
0 0.0257591009140015 44.0
0 0.0591777563095093 45.0
0 0.0197631120681763 46.0
0 -0.0607196092605591 47.0
0 0.00449740886688232 48.0
0 0.23787921667099 49.0
0.098592901533 0.0941811800003052 50.0
1.9757061643 1.0236622095108 51.0
0 -0.0647321939468384 52.0
0 -0.0306555032730103 53.0
0 -0.0183640718460083 54.0
0 -0.0556102991104126 55.0
0 0.0139139890670776 56.0
0 0.102195858955383 57.0
0 -0.0603805780410767 58.0
0 -0.0434738397598267 59.0
0 -0.0784052610397339 60.0
0 0.0303694009780884 61.0
0 0.891870379447937 62.0
285.22617903 257.683410644531 0.0
0 -0.0782076120376587 1.0
0 -0.00450193881988525 2.0
0 -0.0562032461166382 3.0
0 -0.0739625692367554 4.0
0 -0.00421297550201416 5.0
0 -0.0445677042007446 6.0
0 -0.0770188570022583 7.0
0 -0.0340814590454102 8.0
0 -0.0961481332778931 9.0
0 -0.0395079851150513 10.0
0 -0.0506559610366821 11.0
0 0.173687338829041 12.0
66.956987539 65.9508819580078 13.0
33.427380462 32.7684440612793 14.0
0 -0.000118851661682129 15.0
0 -0.0475093126296997 16.0
34.300763357 33.4848098754883 17.0
0 -0.019400954246521 18.0
0 0.0094527006149292 19.0
0 -0.0315717458724976 20.0
0 0.00199282169342041 21.0
0 -0.00379002094268799 22.0
30.935732544 30.3408622741699 23.0
0 -0.0495859384536743 24.0
0 0.604509830474854 25.0
0 0.00189578533172607 26.0
0 0.139757513999939 27.0
0 0.15517270565033 28.0
0 -0.0541096925735474 29.0
0 -0.108965516090393 30.0
0 -0.0456386804580688 31.0
0 -0.0244122743606567 32.0
0 -0.118520021438599 33.0
0 -0.0540703535079956 34.0
0 -0.0690909624099731 35.0
0 0.155498385429382 36.0
0 0.00419437885284424 37.0
0 0.0289222002029419 38.0
0 -0.0362614393234253 39.0
0 0.0166577100753784 40.0
0 -0.0328251123428345 41.0
0 -0.0591961145401001 42.0
306.61975992 293.516448974609 43.0
0 -0.0254684686660767 44.0
0 0.104455947875977 45.0
0 -0.00710618495941162 46.0
0 -0.00250470638275146 47.0
0 0.0523241758346558 48.0
0 0.263414263725281 49.0
0.094456536044 0.147371768951416 50.0
2.0822127351 0.852276921272278 51.0
0 -0.0675889253616333 52.0
0 -0.0418239831924438 53.0
0 0.0815736055374146 54.0
0 -0.0967415571212769 55.0
0 -0.0607725381851196 56.0
0 0.0699533224105835 57.0
0 -0.042876124382019 58.0
0 -0.000906109809875488 59.0
0 -0.0286546945571899 60.0
0 0.0220719575881958 61.0
0 0.929210007190704 62.0
249.84386924 256.821563720703 0.0
0 -0.0516611337661743 1.0
0 -0.0110887289047241 2.0
0 0.00306022167205811 3.0
0 -0.0167208909988403 4.0
0 -0.074029803276062 5.0
0 0.00899732112884521 6.0
0 0.045106053352356 7.0
0 -0.0132136344909668 8.0
0 0.0591262578964233 9.0
0 -0.0200101137161255 10.0
0 -0.0235608816146851 11.0
0 0.0283664464950562 12.0
61.609053655 60.3053741455078 13.0
37.073592711 32.0479469299316 14.0
0 -0.0564600229263306 15.0
0 -0.0304726362228394 16.0
37.484953156 37.0804595947266 17.0
0 0.0112632513046265 18.0
0 -0.00779187679290771 19.0
0 -0.0169740915298462 20.0
0 -0.0533140897750854 21.0
0 0.0283607244491577 22.0
30.947753272 30.0774822235107 23.0
0 -0.0722037553787231 24.0
0 0.60084855556488 25.0
0 -0.00018608570098877 26.0
0 -0.017087459564209 27.0
0 -0.0576730966567993 28.0
0 -0.0784121751785278 29.0
0 0.167429089546204 30.0
0 -0.0375329256057739 31.0
0 0.00515115261077881 32.0
0 0.0490559339523315 33.0
0 -0.0511034727096558 34.0
0 -0.0176892280578613 35.0
0 -0.13968551158905 36.0
0 -0.0412136316299438 37.0
0 -0.0205479860305786 38.0
0 0.0281633138656616 39.0
0 0.065144419670105 40.0
0 -0.0610703229904175 41.0
0 0.032701849937439 42.0
294.59599448 293.689666748047 43.0
0 0.022600531578064 44.0
0 0.0594675540924072 45.0
0 0.00661957263946533 46.0
0 -0.00912559032440186 47.0
0 0.0272058248519897 48.0
0 0.272435069084167 49.0
0.079013306321 0.0667902231216431 50.0
2.1399526795 0.95334404706955 51.0
0 -0.104493975639343 52.0
0 -0.0288769006729126 53.0
0 -0.0353380441665649 54.0
0 -0.084417462348938 55.0
0 -0.0403071641921997 56.0
0 -0.0378285646438599 57.0
0 -0.0127607583999634 58.0
0 -0.0383051633834839 59.0
0 -0.00499308109283447 60.0
0 0.0130172967910767 61.0
0 0.813262462615967 62.0
269.05819337 252.802825927734 0.0
0 0.00938546657562256 1.0
0 -0.0609678030014038 2.0
0 0.0296772718429565 3.0
0 -0.0691791772842407 4.0
0 -0.055420994758606 5.0
0 0.0738452672958374 6.0
0 -0.0533701181411743 7.0
0 0.0864841938018799 8.0
0 -0.0685688257217407 9.0
0 -0.0865606069564819 10.0
0 0.00775611400604248 11.0
0 -0.187217354774475 12.0
54.319430824 55.0201988220215 13.0
41.365266406 40.68212890625 14.0
0 -0.0639525651931763 15.0
0 -0.0116473436355591 16.0
32.865457466 30.982400894165 17.0
0 0.0733515024185181 18.0
0 0.0271204710006714 19.0
0 0.0133758783340454 20.0
0 -0.00567638874053955 21.0
0 -0.00507605075836182 22.0
24.165976484 24.4120121002197 23.0
0 -0.00409281253814697 24.0
24.947118847 0.615971028804779 25.0
0 0.0585697889328003 26.0
0 -0.000226378440856934 27.0
0 0.0544871091842651 28.0
0 -0.0456947088241577 29.0
0 -0.0690306425094604 30.0
0 -0.0372203588485718 31.0
0 0.00573670864105225 32.0
0 0.00700652599334717 33.0
0 -0.193284630775452 34.0
0 0.0379430055618286 35.0
0 0.157540678977966 36.0
0 -0.125331521034241 37.0
0 0.0747729539871216 38.0
0 -0.0509918928146362 39.0
0 0.00964725017547607 40.0
0 -0.123060345649719 41.0
0 -0.0091167688369751 42.0
276.33555039 296.659759521484 43.0
0 -0.060977578163147 44.0
0 0.0317947864532471 45.0
0 0.012311577796936 46.0
0 -0.0769046545028687 47.0
0 0.151646971702576 48.0
0 0.40408992767334 49.0
0.073028407159 0.0432740449905396 50.0
2.4295335331 0.884581685066223 51.0
0 -0.0614234209060669 52.0
0 -0.0228697061538696 53.0
0 -0.0604373216629028 54.0
0 -0.144676804542542 55.0
0 -0.017234206199646 56.0
0 0.015766978263855 57.0
0 -0.0708476305007935 58.0
0 -0.0343884229660034 59.0
0 0.0989056825637817 60.0
0 0.0271111726760864 61.0
0 0.787873148918152 62.0
221.91762877 263.136901855469 0.0
0 -0.0835729837417603 1.0
0 -0.103581547737122 2.0
0 0.00107657909393311 3.0
0 -0.0355719327926636 4.0
0 0.0578896999359131 5.0
0 0.0852779150009155 6.0
0 -0.0313616991043091 7.0
0 0.0203330516815186 8.0
0 0.0366615056991577 9.0
0 -0.00260627269744873 10.0
0 0.0371665358543396 11.0
0 0.144063353538513 12.0
64.538315837 65.8368682861328 13.0
39.189849814 35.0520324707031 14.0
0 -0.0840116739273071 15.0
0 -0.0334702730178833 16.0
40.842555222 40.0967292785645 17.0
0 -0.0522016286849976 18.0
0 0.0288950204849243 19.0
0 0.0842894315719604 20.0
0 0.0293139219284058 21.0
0 -0.002663254737854 22.0
31.832918477 31.4688549041748 23.0
0 -0.0451709032058716 24.0
0 0.888342022895813 25.0
0 0.0359534025192261 26.0
0 -0.0404967069625854 27.0
0 -0.0065990686416626 28.0
0 -0.0526481866836548 29.0
0 -0.0601539611816406 30.0
0 -0.0376600027084351 31.0
0 0.0288709402084351 32.0
0 -0.000668168067932129 33.0
0 -0.0299986600875854 34.0
0 -0.0658682584762573 35.0
0 0.147520661354065 36.0
0 -0.00131618976593018 37.0
0 -0.0249143838882446 38.0
0 0.0444940328598022 39.0
0 0.0887907743453979 40.0
0 -0.0565781593322754 41.0
0 -0.0459467172622681 42.0
284.78024272 297.172088623047 43.0
0 -0.0335651636123657 44.0
0 0.036463737487793 45.0
0 0.043465256690979 46.0
0 -0.0201176404953003 47.0
0 -0.0106850862503052 48.0
4.3289940177 0.25923228263855 49.0
0.099792830817 0.0930423736572266 50.0
2.0006023392 0.881125748157501 51.0
0 -0.0624369382858276 52.0
0 -0.0105012655258179 53.0
0 -0.132504820823669 54.0
0 -0.0130840539932251 55.0
0 -0.100278496742249 56.0
0 0.0284780263900757 57.0
0 -0.0561243295669556 58.0
0 -0.0570224523544312 59.0
0 -0.0241507291793823 60.0
0 -0.0454679727554321 61.0
0 0.822966337203979 62.0
249.48345762 259.789428710938 0.0
0 -0.118526101112366 1.0
0 -0.0424593687057495 2.0
0 -0.0182982683181763 3.0
0 -0.0409678220748901 4.0
0 0.0196813344955444 5.0
0 0.0846141576766968 6.0
0 0.0115262269973755 7.0
0 -0.0267362594604492 8.0
0 0.0332359075546265 9.0
0 -0.0890642404556274 10.0
0 0.0360797643661499 11.0
0 0.0458101034164429 12.0
58.77821622 58.8737716674805 13.0
48.153231664 39.0305786132812 14.0
0 -0.0288540124893188 15.0
0 -0.0130618810653687 16.0
26.235728915 25.325496673584 17.0
0 0.0360885858535767 18.0
0 -0.0187596082687378 19.0
0 -0.0111202001571655 20.0
0 -0.00218403339385986 21.0
0 -0.0314489603042603 22.0
32.439661467 31.7324466705322 23.0
0 -0.0495907068252563 24.0
0 0.696946144104004 25.0
0 0.129933714866638 26.0
0 0.0257571935653687 27.0
0 -0.0179721117019653 28.0
0 -0.0157886743545532 29.0
0 -0.117718458175659 30.0
0 -0.0614327192306519 31.0
0 -0.0103975534439087 32.0
0 0.0533286333084106 33.0
0 -0.0212479829788208 34.0
0 0.0774359703063965 35.0
0 -0.0402594804763794 36.0
0 -0.0223983526229858 37.0
0 -0.0161415338516235 38.0
0 0.00672233104705811 39.0
0 0.137068152427673 40.0
0 -0.133269548416138 41.0
0 -0.0302425622940063 42.0
283.70113661 295.387786865234 43.0
0 0.039324164390564 44.0
0 0.112311065196991 45.0
0 -0.0755561590194702 46.0
0 -0.0358973741531372 47.0
0 -0.0157452821731567 48.0
0 0.321643948554993 49.0
0.095829782669 0.0762711763381958 50.0
2.179861404 0.892805814743042 51.0
0 -0.0122896432876587 52.0
0 -0.0539709329605103 53.0
0 -0.0263117551803589 54.0
0 -0.0711745023727417 55.0
0 -0.0143789052963257 56.0
0 -0.0483647584915161 57.0
0 -0.0671600103378296 58.0
0 -0.000824093818664551 59.0
0 0.123719334602356 60.0
0 -0.00463235378265381 61.0
0 0.8559410572052 62.0
286.59959479 254.412384033203 0.0
0 0.0186418294906616 1.0
0 -0.0830610990524292 2.0
0 -0.00620567798614502 3.0
0 -0.0761257410049438 4.0
0 -0.0142337083816528 5.0
0 -0.0193759202957153 6.0
0 0.00956475734710693 7.0
0 -0.0206320285797119 8.0
0 0.0720056295394897 9.0
0 -0.0346716642379761 10.0
0 -0.0496255159378052 11.0
0 -0.136489510536194 12.0
53.49258881 52.5840530395508 13.0
37.796194569 36.7282409667969 14.0
0 0.0797661542892456 15.0
0 -0.0436230897903442 16.0
31.263005395 31.2695503234863 17.0
0 0.0254591703414917 18.0
0 0.0313771963119507 19.0
0 0.0333081483840942 20.0
0 0.0209575891494751 21.0
0 -0.0172492265701294 22.0
38.506253573 37.7241744995117 23.0
0 0.0529264211654663 24.0
0 1.31467616558075 25.0
0 0.0370930433273315 26.0
0 0.0899029970169067 27.0
0 0.0576733350753784 28.0
0 -0.105091214179993 29.0
0 0.122351765632629 30.0
0 -0.009665846824646 31.0
0 0.030453085899353 32.0
0 -0.0300508737564087 33.0
0 -0.0189749002456665 34.0
0 0.0264126062393188 35.0
0 0.0399810075759888 36.0
0 -0.119341731071472 37.0
0 0.0114103555679321 38.0
0 -0.0418070554733276 39.0
0 -0.00549566745758057 40.0
0 -0.0325403213500977 41.0
0 -0.032025933265686 42.0
304.25826801 297.203247070312 43.0
0 -0.0126386880874634 44.0
0 0.0530073642730713 45.0
0 -0.00636661052703857 46.0
0 -0.0237013101577759 47.0
0 0.00880897045135498 48.0
0.33061588588 0.273047626018524 49.0
0.081300359236 0.0740408897399902 50.0
1.5505018716 0.873162031173706 51.0
0 -0.0893584489822388 52.0
0 -0.039188027381897 53.0
0 -0.0106090307235718 54.0
0 -0.145838618278503 55.0
0 0.000261902809143066 56.0
0 -0.0247935056686401 57.0
0 -0.0718950033187866 58.0
0 0.000159621238708496 59.0
0 -0.0331915616989136 60.0
0 -0.0350919961929321 61.0
0 0.808375716209412 62.0
231.44689297 211.654022216797 0.0
0 0.00148046016693115 1.0
0 -0.0639446973800659 2.0
0 -0.0334745645523071 3.0
0 0.0612500905990601 4.0
0 0.0321875810623169 5.0
0 0.08952796459198 6.0
0 0.0594915151596069 7.0
0 0.0666321516036987 8.0
0 -0.13816511631012 9.0
0 0.0316909551620483 10.0
0 0.0023643970489502 11.0
0 -0.106693863868713 12.0
68.702625864 68.1688766479492 13.0
39.240581123 30.1171169281006 14.0
0 -0.0709167718887329 15.0
0 0.00522267818450928 16.0
23.826472374 21.8459072113037 17.0
0 0.035948634147644 18.0
0 0.00749266147613525 19.0
0 0.000188946723937988 20.0
0 0.0702692270278931 21.0
0 -0.0886255502700806 22.0
33.8809781 33.9485054016113 23.0
0 -0.0798927545547485 24.0
0 0.571595311164856 25.0
0 0.078520655632019 26.0
0 0.00182771682739258 27.0
0 -0.00128567218780518 28.0
0 -0.0486952066421509 29.0
0 -0.024539589881897 30.0
0 -0.0509034395217896 31.0
0 0.00643241405487061 32.0
0 0.0277813673019409 33.0
0 -0.110529541969299 34.0
0 -0.0608469247817993 35.0
0 -0.0208765268325806 36.0
0 0.0434271097183228 37.0
0 -0.009407639503479 38.0
0 0.102625250816345 39.0
0 0.193710684776306 40.0
0 -0.037479043006897 41.0
0 -0.0308243036270142 42.0
296.79678915 295.747650146484 43.0
0 0.0104812383651733 44.0
0 0.258159518241882 45.0
0 -0.0496922731399536 46.0
0 -0.0261279344558716 47.0
0 -0.136633992195129 48.0
0 0.216978549957275 49.0
0.073471162123 0.229214429855347 50.0
2.3957648299 0.850221514701843 51.0
0 -0.07215416431427 52.0
0 -0.0133897066116333 53.0
0 -0.0124357938766479 54.0
0 -0.0167075395584106 55.0
0 0.00166738033294678 56.0
0 -0.0103636980056763 57.0
0 0.0211635828018188 58.0
0 -0.0281287431716919 59.0
0 -0.0368438959121704 60.0
0 0.0551685094833374 61.0
0 0.945357501506805 62.0
281.15404842 258.942321777344 0.0
0 -0.103525757789612 1.0
0 -0.114318013191223 2.0
0 0.0048748254776001 3.0
0 -0.0550040006637573 4.0
0 -0.197603344917297 5.0
0 0.0275634527206421 6.0
0 -0.149542450904846 7.0
0 0.0266039371490479 8.0
0 0.0215467214584351 9.0
0 -0.0221856832504272 10.0
0 -0.0118956565856934 11.0
0 0.0636266469955444 12.0
57.680016296 56.5851783752441 13.0
51.890261069 43.144115447998 14.0
0 -0.0289102792739868 15.0
0 -0.0235539674758911 16.0
39.748491369 38.4265327453613 17.0
0 0.0128706693649292 18.0
0 -0.0319377183914185 19.0
0 -0.0297642946243286 20.0
0 0.00738847255706787 21.0
0 -0.0372549295425415 22.0
33.378792628 32.8543930053711 23.0
0 0.0130208730697632 24.0
0 0.554508984088898 25.0
0 0.0243891477584839 26.0
0 -0.0430761575698853 27.0
0 0.0170739889144897 28.0
0 -0.0791236162185669 29.0
0 -0.0316401720046997 30.0
0 -0.0484670400619507 31.0
0 0.0183976888656616 32.0
0 -0.00518333911895752 33.0
0 -0.0107775926589966 34.0
0 0.00701630115509033 35.0
0 0.108655333518982 36.0
0 -0.0404578447341919 37.0
0 0.00612795352935791 38.0
0 -0.0247875452041626 39.0
0 -0.086922287940979 40.0
0 0.143284440040588 41.0
0 -0.154704689979553 42.0
285.4502082 296.219085693359 43.0
0 0.0346421003341675 44.0
0 0.016060471534729 45.0
0 -0.049963116645813 46.0
0 -0.0559297800064087 47.0
0 0.0103157758712769 48.0
0 0.277367949485779 49.0
0.066171708892 0.0679012537002563 50.0
2.1785227876 1.01284539699554 51.0
0 -0.0660501718521118 52.0
0 -0.0608206987380981 53.0
0 0.0113126039505005 54.0
0 -0.0642675161361694 55.0
0 -0.0785490274429321 56.0
0 0.0067061185836792 57.0
0 -0.050451397895813 58.0
0 -0.0222378969192505 59.0
0 0.0535925626754761 60.0
0 0.0303696393966675 61.0
0 0.767535090446472 62.0
287.24195977 262.445068359375 0.0
0 -0.0638853311538696 1.0
0 0.0623880624771118 2.0
0 -0.0272418260574341 3.0
0 -0.0529724359512329 4.0
0 -0.0628682374954224 5.0
0 0.0873492956161499 6.0
0 -0.0749067068099976 7.0
0 -0.0250568985939026 8.0
0 0.00568711757659912 9.0
0 0.0371168851852417 10.0
0 0.0250967741012573 11.0
0 0.0328983068466187 12.0
53.982494033 53.8928451538086 13.0
42.644261488 38.2697715759277 14.0
0 -0.0587259531021118 15.0
0 0.00595533847808838 16.0
29.584683627 29.6479606628418 17.0
0 0.0162378549575806 18.0
0 0.0173090696334839 19.0
0 0.0135401487350464 20.0
0 -0.0221589803695679 21.0
0 -0.0238157510757446 22.0
33.223247148 32.0153465270996 23.0
0 -0.0302459001541138 24.0
0 0.85677969455719 25.0
0 0.0043872594833374 26.0
0 0.0879362821578979 27.0
0 -0.0315400362014771 28.0
0 -0.0843876600265503 29.0
0 0.0458101034164429 30.0
0 -0.0100001096725464 31.0
0 -0.0155655145645142 32.0
0 -0.107184052467346 33.0
0 -0.0979372262954712 34.0
0 -0.127727150917053 35.0
0 -0.108531832695007 36.0
0 0.0351384878158569 37.0
0 -0.0298563241958618 38.0
0 0.0199736356735229 39.0
0 -0.0637398958206177 40.0
0 -0.0627316236495972 41.0
0 0.0174449682235718 42.0
297.85780169 298.111175537109 43.0
0 -0.0526138544082642 44.0
0 0.220736861228943 45.0
0 0.0906525850296021 46.0
0 -0.107149481773376 47.0
0 -0.0136500597000122 48.0
0 0.400048196315765 49.0
0.062641527489 0.0863991975784302 50.0
1.6283256948 1.00736212730408 51.0
0 -0.110366225242615 52.0
0 0.0368586778640747 53.0
0 -0.0736445188522339 54.0
0 -0.0884109735488892 55.0
0 -0.0862554311752319 56.0
0 -0.00849545001983643 57.0
0 0.0113018751144409 58.0
0 -0.0166503190994263 59.0
0 0.0440667867660522 60.0
0 0.0569347143173218 61.0
0 0.85818076133728 62.0
229.42819507 260.551177978516 0.0
0 -0.096227765083313 1.0
0 -0.00197684764862061 2.0
0 0.0432316064834595 3.0
0 -0.0437945127487183 4.0
0 0.110891938209534 5.0
0 0.0448974370956421 6.0
0 -0.0480948686599731 7.0
0 -0.0456498861312866 8.0
0 0.0247844457626343 9.0
0 -0.097197413444519 10.0
0 -0.0475606918334961 11.0
0 -0.00579321384429932 12.0
69.241795975 68.8228607177734 13.0
24.117330889 31.8379726409912 14.0
0 -0.10939610004425 15.0
0 -0.0377887487411499 16.0
31.095099259 31.282751083374 17.0
0 -0.035725474357605 18.0
0 -0.0465847253799438 19.0
0 0.137642502784729 20.0
0 0.0373340845108032 21.0
0 -0.00617086887359619 22.0
33.522673275 32.7624282836914 23.0
0 -0.0666490793228149 24.0
0 0.637828707695007 25.0
0 -0.0622068643569946 26.0
0 0.0911813974380493 27.0
0 -0.0339041948318481 28.0
0 -0.0217677354812622 29.0
0 -0.122213244438171 30.0
0 0.0147131681442261 31.0
0 0.00550544261932373 32.0
0 0.0953150987625122 33.0
0 -0.034143328666687 34.0
0 -0.0894776582717896 35.0
0 0.0492454767227173 36.0
0 -0.049646258354187 37.0
0 -0.043678879737854 38.0
0 -0.0527015924453735 39.0
0 0.040122389793396 40.0
0 -0.00560462474822998 41.0
0 0.00290143489837646 42.0
294.8248876 291.669219970703 43.0
0 0.0159982442855835 44.0
0 0.0556567907333374 45.0
0 -0.0290607213973999 46.0
0 -0.0884984731674194 47.0
0 -0.0339699983596802 48.0
0.076642592144 0.277901232242584 49.0
0.095903779645 0.0754712820053101 50.0
2.0353398607 0.991232931613922 51.0
0 -0.104913353919983 52.0
0 -0.0446745157241821 53.0
0 0.0307286977767944 54.0
0 -0.0405641794204712 55.0
0 -0.0859314203262329 56.0
0 -0.0328172445297241 57.0
0 -0.0116780996322632 58.0
0 -0.0797466039657593 59.0
0 -0.0199466943740845 60.0
0 0.0769640207290649 61.0
0 0.949421584606171 62.0
235.4123612 258.260955810547 0.0
0 -0.0834099054336548 1.0
0 -0.0640391111373901 2.0
0 0.000236153602600098 3.0
0 -0.0665816068649292 4.0
0 -0.075442910194397 5.0
0 0.0394238233566284 6.0
0 -0.048193097114563 7.0
0 0.00880157947540283 8.0
0 0.045012354850769 9.0
0 0.0190554857254028 10.0
0 0.101444005966187 11.0
0 0.0528386831283569 12.0
77.712714085 74.3214950561523 13.0
16.915932079 36.3514099121094 14.0
0 -0.107412934303284 15.0
0 0.0501636266708374 16.0
32.179464189 34.1803588867188 17.0
0 0.086081862449646 18.0
0 -0.0203932523727417 19.0
0 -0.0466829538345337 20.0
0 -0.0178545713424683 21.0
0 -0.0276323556900024 22.0
40.398705858 39.8202056884766 23.0
0 0.0202573537826538 24.0
0 2.061443567276 25.0
0 0.0805529356002808 26.0
0 -0.0232189893722534 27.0
0 -0.053202748298645 28.0
0 -0.0323585271835327 29.0
0 -0.0554846525192261 30.0
0 -0.0162268877029419 31.0
0 -0.0591193437576294 32.0
0 0.0720595121383667 33.0
0 -0.038446307182312 34.0
0 -0.0423842668533325 35.0
0 0.0341752767562866 36.0
0 -0.0587042570114136 37.0
0 -0.101985573768616 38.0
0 0.116573691368103 39.0
0 -0.00736057758331299 40.0
0 -0.0810080766677856 41.0
0 0.0550779104232788 42.0
311.82351695 297.658325195312 43.0
0 0.00499379634857178 44.0
0 -0.0409870147705078 45.0
0 -0.00619804859161377 46.0
0 -0.0201870203018188 47.0
0 -0.055759072303772 48.0
0 0.251896142959595 49.0
0.090250072652 0.116509914398193 50.0
1.9950449487 0.88577127456665 51.0
0 -3.75509262084961e-05 52.0
0 -0.0143126249313354 53.0
0 -0.0407904386520386 54.0
0 0.0334042310714722 55.0
0 0.00553524494171143 56.0
0 -0.0648971796035767 57.0
0 -0.0375372171401978 58.0
0 -0.103983044624329 59.0
0 -0.0525330305099487 60.0
0 0.0109411478042603 61.0
0 0.872422337532043 62.0
217.5594009 252.896255493164 0.0
0 -0.0310236215591431 1.0
0 -0.0531526803970337 2.0
0 -0.000400424003601074 3.0
0 -0.047418475151062 4.0
0 0.0407334566116333 5.0
0 0.0522598028182983 6.0
0 -0.159089684486389 7.0
0 -0.00549602508544922 8.0
0 0.0774511098861694 9.0
0 -0.0357592105865479 10.0
0 -0.0072782039642334 11.0
0 -0.0337215662002563 12.0
53.558440821 52.708610534668 13.0
56.777720657 44.1560821533203 14.0
0 -0.111412167549133 15.0
0 -0.0792697668075562 16.0
31.342182789 30.8802833557129 17.0
0 0.0320755243301392 18.0
0 0.0128587484359741 19.0
0 0.0261541604995728 20.0
0 0.00330650806427002 21.0
0 -0.0408922433853149 22.0
32.813352055 31.976957321167 23.0
0 -0.0111204385757446 24.0
0 0.482435286045074 25.0
0 0.0121649503707886 26.0
0 0.0062633752822876 27.0
0 0.0614327192306519 28.0
0 -0.0775593519210815 29.0
0 -0.0757389068603516 30.0
0 -0.0430079698562622 31.0
0 0.000195860862731934 32.0
0 -0.0840120315551758 33.0
0 -0.0730687379837036 34.0
0 -0.0899921655654907 35.0
0 0.0936442613601685 36.0
0 0.00148856639862061 37.0
0 0.0532485246658325 38.0
0 0.0890382528305054 39.0
0 0.0390485525131226 40.0
0 0.0710147619247437 41.0
0 -0.131699204444885 42.0
296.42640978 295.490447998047 43.0
0 -0.0245109796524048 44.0
0 0.0593156814575195 45.0
0 0.0011066198348999 46.0
0 -0.0968259572982788 47.0
0 0.0468605756759644 48.0
0 0.265984356403351 49.0
0.088495673637 0.0746594667434692 50.0
1.905946706 0.955968976020813 51.0
0 -0.0518451929092407 52.0
0 -0.0319186449050903 53.0
0 -0.0153192281723022 54.0
0 -0.0633763074874878 55.0
0 0.00859081745147705 56.0
0 -0.0448867082595825 57.0
0 -0.0329486131668091 58.0
0 -0.0165344476699829 59.0
0 -0.083375096321106 60.0
0 0.00430977344512939 61.0
0 0.877132058143616 62.0
223.30155042 252.0224609375 0.0
0 0.0437477827072144 1.0
0 -0.0872946977615356 2.0
0 -0.00745928287506104 3.0
0 -0.047387957572937 4.0
0 -0.00150668621063232 5.0
0 -0.0681463479995728 6.0
0 -0.0462990999221802 7.0
0 -0.0151574015617371 8.0
0 0.0414670705795288 9.0
0 0.0266988277435303 10.0
0 -0.0315407514572144 11.0
0 -0.0989261865615845 12.0
51.667320304 51.3030090332031 13.0
40.205274185 45.048770904541 14.0
0 -0.0528098344802856 15.0
0 -0.0153971910476685 16.0
34.868075732 34.5660133361816 17.0
0 -0.0276640653610229 18.0
0 0.00620186328887939 19.0
0 -0.0439378023147583 20.0
0 -0.0394421815872192 21.0
0 -0.0402446985244751 22.0
35.37909763 34.7924957275391 23.0
0 0.0129841566085815 24.0
0 0.434269309043884 25.0
0 0.102393984794617 26.0
0 0.0864828824996948 27.0
0 0.125999331474304 28.0
0 -0.0410059690475464 29.0
0 -0.067617654800415 30.0
0 -0.0200814008712769 31.0
0 -0.00761520862579346 32.0
0 -0.00219309329986572 33.0
0 -0.0268696546554565 34.0
0 -0.0411370992660522 35.0
0 0.119450211524963 36.0
0 -0.0642510652542114 37.0
0 -0.0233379602432251 38.0
0 0.010797381401062 39.0
0 0.0648678541183472 40.0
0 -0.043649435043335 41.0
0 -0.0284935235977173 42.0
300.58826483 295.824584960938 43.0
0 0.0324500799179077 44.0
0 -0.04442298412323 45.0
0 0.00670230388641357 46.0
0 -0.0551186800003052 47.0
0 -0.0427020788192749 48.0
0 0.246827006340027 49.0
0.070607258342 0.139410257339478 50.0
2.1217518689 0.915355801582336 51.0
0 -0.0863724946975708 52.0
0 -0.0579310655593872 53.0
0 0.0171358585357666 54.0
0 -0.0759705305099487 55.0
0 -0.0188044309616089 56.0
0 -0.0387126207351685 57.0
0 -0.0141648054122925 58.0
0 -0.0108953714370728 59.0
0 0.0141333341598511 60.0
0 0.00381052494049072 61.0
0 0.852762341499329 62.0
263.04232814 258.100708007812 0.0
0 -0.0597292184829712 1.0
0 -0.0264822244644165 2.0
0 -0.0357617139816284 3.0
0 -0.0430258512496948 4.0
0 -0.0641580820083618 5.0
0 -0.0751219987869263 6.0
0 -0.0265246629714966 7.0
0 0.0234417915344238 8.0
0 -0.0187002420425415 9.0
0 0.0258451700210571 10.0
0 0.027418315410614 11.0
0 0.0494703054428101 12.0
71.877396091 71.8004150390625 13.0
34.873003447 27.2883033752441 14.0
0 -0.0531030893325806 15.0
0 -0.0397058725357056 16.0
28.481891263 28.2593936920166 17.0
0 0.0327740907669067 18.0
0 0.0224977731704712 19.0
0 0.0142132043838501 20.0
0 -0.0180262327194214 21.0
0 -0.01795494556427 22.0
30.807792881 29.6512355804443 23.0
0 -0.0611492395401001 24.0
0 0.566073894500732 25.0
0 -0.00521481037139893 26.0
0 0.110937237739563 27.0
0 -0.0747586488723755 28.0
0 -0.0872529745101929 29.0
0 0.0170482397079468 30.0
0 -0.0268222093582153 31.0
0 0.0177682638168335 32.0
0 0.0349048376083374 33.0
0 -0.0503143072128296 34.0
0 0.0840751528739929 35.0
0 -0.0586735010147095 36.0
0 0.0494166612625122 37.0
0 -0.0148061513900757 38.0
0 0.00650513172149658 39.0
0 0.176380515098572 40.0
0 -0.108599424362183 41.0
0 -0.0785280466079712 42.0
304.19256538 293.880615234375 43.0
0 -0.00545275211334229 44.0
0 -0.00145971775054932 45.0
0 0.0492562055587769 46.0
0 0.00230693817138672 47.0
0 0.10025680065155 48.0
3.083003142 0.410428404808044 49.0
0.086866421568 0.0479584932327271 50.0
1.5623424096 0.979915261268616 51.0
0 -0.0786672830581665 52.0
0 -0.0347310304641724 53.0
0 -0.00386655330657959 54.0
0 0.0136421918869019 55.0
0 -0.0564943552017212 56.0
0 0.0427464246749878 57.0
0 -0.0317968130111694 58.0
0 -0.033748984336853 59.0
0 0.0698305368423462 60.0
0 -0.00833714008331299 61.0
0 0.829531669616699 62.0
239.55765498 256.182647705078 0.0
0 -0.139000058174133 1.0
0 0.00263655185699463 2.0
0 -0.00153219699859619 3.0
0 -0.0387288331985474 4.0
0 0.0152653455734253 5.0
0 -0.00774872303009033 6.0
0 -0.0371745824813843 7.0
0 -0.00136077404022217 8.0
0 -0.00398790836334229 9.0
0 0.0308384895324707 10.0
0 -0.0245984792709351 11.0
0 -0.110801339149475 12.0
52.849057059 51.540096282959 13.0
49.746252464 42.4901809692383 14.0
0 -0.0750566720962524 15.0
0 -0.00463449954986572 16.0
37.48258718 37.1193962097168 17.0
0 0.0446149110794067 18.0
0 0.0282315015792847 19.0
0 -0.0207751989364624 20.0
0 -0.0539225339889526 21.0
0 0.0779265165328979 22.0
29.142831035 27.8601455688477 23.0
0 0.0506519079208374 24.0
0 0.55719131231308 25.0
0 0.00162804126739502 26.0
0 0.0280388593673706 27.0
0 0.0279501676559448 28.0
0 -0.0887647867202759 29.0
0 -0.0536031723022461 30.0
0 -0.0469266176223755 31.0
0 -0.00536930561065674 32.0
0 0.0641788244247437 33.0
0 -0.0219763517379761 34.0
0 -0.141142249107361 35.0
0 0.0232733488082886 36.0
0 -0.0106443166732788 37.0
0 0.0235797166824341 38.0
0 0.0098336935043335 39.0
0 -0.0159760713577271 40.0
0 -0.0362428426742554 41.0
0 0.0337623357772827 42.0
301.09598584 298.304321289062 43.0
0 -0.000920891761779785 44.0
0 0.0471220016479492 45.0
0 -0.00168979167938232 46.0
0 -0.0323948860168457 47.0
0 -0.0392442941665649 48.0
1.283928521 0.312625706195831 49.0
0.072443526567 0.0736650228500366 50.0
1.8559469499 0.98406845331192 51.0
0 -0.0514663457870483 52.0
0 -0.0628327131271362 53.0
0 0.0369242429733276 54.0
0 -0.0675193071365356 55.0
0 -0.0550163984298706 56.0
0 -0.0288232564926147 57.0
0 -0.0647908449172974 58.0
0 -0.10154664516449 59.0
0 -0.0679706335067749 60.0
0 0.0673166513442993 61.0
0 0.926821053028107 62.0
256.966466 254.933410644531 0.0
0 -0.0217486619949341 1.0
0 -0.0376666784286499 2.0
0 -0.0239075422286987 3.0
0 -0.00105035305023193 4.0
0 -0.000955700874328613 5.0
0 -0.0497876405715942 6.0
0 -0.115786194801331 7.0
0 -0.014443039894104 8.0
0 -0.00825655460357666 9.0
0 0.00980961322784424 10.0
0 -0.0431137084960938 11.0
0 -0.00122320652008057 12.0
62.205015268 61.1932640075684 13.0
40.365267653 37.122730255127 14.0
0 -0.0832549333572388 15.0
0 -0.0144838094711304 16.0
31.143264192 31.7349872589111 17.0
0 -0.0382233858108521 18.0
0 0.0227102041244507 19.0
0 -0.0149508714675903 20.0
0 -0.0146476030349731 21.0
0 -0.0590516328811646 22.0
31.626512779 31.0318393707275 23.0
0 -0.0448185205459595 24.0
0 0.547066986560822 25.0
0 0.0831860303878784 26.0
0 -0.010738730430603 27.0
0 0.0450509786605835 28.0
0 -0.051587700843811 29.0
0 -0.0507258176803589 30.0
0 -0.0549174547195435 31.0
0 0.0256284475326538 32.0
0 0.0432373285293579 33.0
0 -0.11437976360321 34.0
0 -0.018006443977356 35.0
0 0.00703656673431396 36.0
0 -0.0796133279800415 37.0
0 -0.0436323881149292 38.0
0 0.0650557279586792 39.0
0 0.167107939720154 40.0
0 -0.0258172750473022 41.0
0 0.0374377965927124 42.0
282.57474048 295.770172119141 43.0
0 -0.0329843759536743 44.0
0 -0.046802282333374 45.0
0 -0.0218356847763062 46.0
0 -0.0485113859176636 47.0
0 0.00846564769744873 48.0
0.11916375769 0.259180009365082 49.0
0.073328214887 0.159749150276184 50.0
1.9828109933 0.872612357139587 51.0
0 -0.0245794057846069 52.0
0 -0.0277730226516724 53.0
0 -0.0206516981124878 54.0
0 -0.0920847654342651 55.0
0 -0.0476937294006348 56.0
0 -0.0485085248947144 57.0
0 -0.0640517473220825 58.0
0 -0.08228600025177 59.0
0 0.032037615776062 60.0
0 -0.0135434865951538 61.0
0 0.822900772094727 62.0
273.32145127 256.251037597656 0.0
0 -0.102712273597717 1.0
0 -0.00171840190887451 2.0
0 -0.0772742033004761 3.0
0 -0.0386749505996704 4.0
0 -0.129897713661194 5.0
0 0.0492733716964722 6.0
0 -0.00674211978912354 7.0
0 0.0886968374252319 8.0
0 -0.00446236133575439 9.0
0 -0.0391682386398315 10.0
0 -0.0145103931427002 11.0
0 -0.0776010751724243 12.0
74.242447016 74.6185150146484 13.0
34.473087534 26.0114727020264 14.0
0 0.0106357336044312 15.0
0 -0.0287939310073853 16.0
28.091003229 27.1443786621094 17.0
0 0.048423171043396 18.0
0 0.0324746370315552 19.0
0 -0.0333906412124634 20.0
0 -0.0394188165664673 21.0
0 -0.0132027864456177 22.0
30.687873131 29.8918228149414 23.0
0 -0.011121392250061 24.0
0 0.497254371643066 25.0
0 0.0345371961593628 26.0
0 0.0462442636489868 27.0
0 -0.0137912034988403 28.0
0 -0.0559076070785522 29.0
0 -0.0867395401000977 30.0
0 -0.0251299142837524 31.0
0 -0.00543129444122314 32.0
0 -0.0612972974777222 33.0
0 -0.0359140634536743 34.0
0 0.105839371681213 35.0
0 0.0409075021743774 36.0
0 -0.0594264268875122 37.0
0 -0.0491930246353149 38.0
0 -0.0235904455184937 39.0
0 0.258571982383728 40.0
0 -0.00312864780426025 41.0
0 -0.0762850046157837 42.0
304.30228807 295.989715576172 43.0
0 0.0120466947555542 44.0
0 0.0317550897598267 45.0
0 -0.0596355199813843 46.0
0 -0.00759422779083252 47.0
0 -0.00800192356109619 48.0
1.1598405993 0.322699338197708 49.0
0.10030573585 0.0602294206619263 50.0
2.015265386 1.02234244346619 51.0
0 -0.0420986413955688 52.0
0 -0.0608965158462524 53.0
0 0.0297483205795288 54.0
0 -0.0834561586380005 55.0
0 0.0336109399795532 56.0
0 -0.047986626625061 57.0
0 -0.100080609321594 58.0
0 -0.0399798154830933 59.0
0 0.0243023633956909 60.0
0 0.0287069082260132 61.0
0 0.813634991645813 62.0
229.21611805 248.382949829102 0.0
0 -0.0846554040908813 1.0
0 -0.0458050966262817 2.0
0 -0.0485376119613647 3.0
0 -0.0579980611801147 4.0
0 -0.0416914224624634 5.0
0 0.0500510931015015 6.0
0 -0.0708554983139038 7.0
0 -0.0096900463104248 8.0
0 0.0507692098617554 9.0
0 -0.00194168090820312 10.0
0 -0.0241085290908813 11.0
0 0.0269502401351929 12.0
70.358368493 70.1369934082031 13.0
33.193124011 28.927526473999 14.0
0 -0.130298256874084 15.0
0 -0.051782488822937 16.0
36.435208471 35.7304954528809 17.0
0 -0.00195133686065674 18.0
0 0.00981771945953369 19.0
0 0.0036691427230835 20.0
0 -0.0279961824417114 21.0
0 -0.0408025979995728 22.0
28.975753635 28.0500755310059 23.0
0 -0.0502554178237915 24.0
0 0.591231226921082 25.0
0 0.00195705890655518 26.0
0 -0.0235298871994019 27.0
0 -0.0590502023696899 28.0
0 -0.0556966066360474 29.0
0 -0.0519142150878906 30.0
0 -0.0421050786972046 31.0
0 -0.0286670923233032 32.0
0 -0.0266879796981812 33.0
0 -0.091863751411438 34.0
0 -0.0247448682785034 35.0
0 0.000915884971618652 36.0
0 -0.0613473653793335 37.0
0 0.0157307386398315 38.0
0 0.0172828435897827 39.0
0 0.0701912641525269 40.0
0 -0.0296542644500732 41.0
0 -0.0528026819229126 42.0
308.01557923 296.526519775391 43.0
0 -0.0392338037490845 44.0
0 -0.0169336795806885 45.0
0 -0.0578020811080933 46.0
0 -0.0256468057632446 47.0
0 0.0618079900741577 48.0
0 0.2804354429245 49.0
0.085043832368 0.0544711351394653 50.0
1.6958797211 0.919146478176117 51.0
0 -0.0923653841018677 52.0
0 -0.0438486337661743 53.0
0 0.015544056892395 54.0
0 -0.055968165397644 55.0
0 -0.0456100702285767 56.0
0 0.00562798976898193 57.0
0 -0.0402995347976685 58.0
0 -0.0626143217086792 59.0
0 0.0157362222671509 60.0
0 -0.0386906862258911 61.0
0 0.815104007720947 62.0
233.80079462 257.854461669922 0.0
0 -0.0727220773696899 1.0
0 -0.0485366582870483 2.0
0 -0.00984227657318115 3.0
0 -0.00582993030548096 4.0
0 -0.0594485998153687 5.0
0 -0.0238691568374634 6.0
0 -0.0167065858840942 7.0
0 -0.0343891382217407 8.0
0 -0.0602432489395142 9.0
0 -0.00722897052764893 10.0
0 -0.0258018970489502 11.0
0 0.00531327724456787 12.0
70.068538987 69.9043502807617 13.0
26.101550786 27.134220123291 14.0
0 -0.0627437829971313 15.0
0 0.0454109907150269 16.0
38.28403119 38.7716102600098 17.0
0 0.00181686878204346 18.0
0 -0.0414952039718628 19.0
0 0.00155007839202881 20.0
0 -0.00280249118804932 21.0
0 0.0106362104415894 22.0
29.348629225 28.9715976715088 23.0
0 -0.0829559564590454 24.0
0 0.647342503070831 25.0
0 -0.0304132699966431 26.0
0 0.0124558210372925 27.0
0 -0.0190526247024536 28.0
0 -0.0472723245620728 29.0
0 -0.00020897388458252 30.0
0 -0.00597774982452393 31.0
0 -0.00648224353790283 32.0
0 0.110265135765076 33.0
0 0.00522506237030029 34.0
0 -0.0310636758804321 35.0
0 0.0979882478713989 36.0
0 -0.0744699239730835 37.0
0 -0.00491130352020264 38.0
0 0.0313204526901245 39.0
0 0.0754607915878296 40.0
0 -0.0763187408447266 41.0
0 -0.113884568214417 42.0
294.51245268 296.634094238281 43.0
0 0.00332152843475342 44.0
0 0.0741606950759888 45.0
0 -0.0329781770706177 46.0
0 -0.00363743305206299 47.0
0 0.0205873250961304 48.0
0 0.315100163221359 49.0
0.07149536779 0.0644232034683228 50.0
1.7674595022 1.01081871986389 51.0
0 -0.0302649736404419 52.0
0 -0.0411490201950073 53.0
0 0.0532113313674927 54.0
0 -0.0695351362228394 55.0
0 0.0432113409042358 56.0
0 -0.0280038118362427 57.0
0 -0.0872660875320435 58.0
0 -0.0298515558242798 59.0
0 0.111210465431213 60.0
0 -0.0632568597793579 61.0
0 0.822969555854797 62.0
250.78670267 265.253051757812 0.0
0 -0.180281281471252 1.0
0 -0.0995327234268188 2.0
0 -0.0144165754318237 3.0
0 -0.076128363609314 4.0
0 0.0400925874710083 5.0
0 0.0189050436019897 6.0
0 -0.137377858161926 7.0
0 -0.0710985064506531 8.0
0 -0.0588365793228149 9.0
0 0.00430810451507568 10.0
0 -0.0763781070709229 11.0
0 -0.0420862436294556 12.0
55.073028492 54.929069519043 13.0
39.595488293 37.4910736083984 14.0
0 -0.0895096063613892 15.0
0 -0.00288546085357666 16.0
43.159996453 42.2038421630859 17.0
0 -0.0387700796127319 18.0
0 0.0174585580825806 19.0
0 0.00305211544036865 20.0
0 -0.0283344984054565 21.0
0 -0.0533124208450317 22.0
31.336650876 30.9902362823486 23.0
0 -0.00885307788848877 24.0
0 0.567788183689117 25.0
0 0.00263917446136475 26.0
0 0.0403663516044617 27.0
0 0.119188189506531 28.0
0 -0.0810743570327759 29.0
0 0.170534491539001 30.0
0 -0.0382305383682251 31.0
0 -0.0154248476028442 32.0
0 -0.00215268135070801 33.0
0 -0.120785593986511 34.0
0 0.00628817081451416 35.0
0 -0.0302830934524536 36.0
0 -0.115825772285461 37.0
0 -0.00756990909576416 38.0
0 0.00452601909637451 39.0
0 -0.0270487070083618 40.0
0 0.0948790311813354 41.0
0 -0.0779262781143188 42.0
305.31578846 296.586120605469 43.0
0 -0.004372239112854 44.0
0 0.0710503458976746 45.0
0 0.00046384334564209 46.0
0 -0.0196603536605835 47.0
0 0.0424250364303589 48.0
0 0.260178327560425 49.0
0.097085774175 0.0334790945053101 50.0
1.953680014 0.836503386497498 51.0
0 -0.0503565073013306 52.0
0 -0.0295959711074829 53.0
0 -0.0346139669418335 54.0
0 -0.0375679731369019 55.0
0 -0.0611416101455688 56.0
0 -0.00377476215362549 57.0
0 -0.0812009572982788 58.0
0 -0.0857068300247192 59.0
0 -0.0539239645004272 60.0
0 -0.00422227382659912 61.0
0 0.811172962188721 62.0
243.13752788 257.033813476562 0.0
0 -0.0209242105484009 1.0
0 -0.070177435874939 2.0
0 -0.037016749382019 3.0
0 -0.0789157152175903 4.0
0 -0.0890523195266724 5.0
0 -0.103242993354797 6.0
0 -0.0395716428756714 7.0
0 0.0229566097259521 8.0
0 0.0282586812973022 9.0
0 0.0104600191116333 10.0
0 0.00579172372817993 11.0
0 -0.0763517618179321 12.0
73.745884554 72.83154296875 13.0
20.783112889 24.7603626251221 14.0
0 0.00792109966278076 15.0
0 -0.0176728963851929 16.0
31.324940445 31.7921447753906 17.0
0 -0.0440374612808228 18.0
0 -0.0264304876327515 19.0
0 -0.0110098123550415 20.0
0 -0.00347411632537842 21.0
0 -0.017041802406311 22.0
36.482584146 35.533519744873 23.0
0 0.02141273021698 24.0
0 0.511264979839325 25.0
0 -0.000254273414611816 26.0
0 0.0364481210708618 27.0
0 -0.0153356790542603 28.0
0 -0.0942045450210571 29.0
0 -0.0304919481277466 30.0
0 -0.0114849805831909 31.0
0 -0.0252572298049927 32.0
0 0.0557700395584106 33.0
0 -0.00353538990020752 34.0
0 -0.0381978750228882 35.0
0 0.0245486497879028 36.0
0 -0.0707248449325562 37.0
0 0.0409802198410034 38.0
0 0.144596695899963 39.0
0 -0.0300246477127075 40.0
0 -0.0126618146896362 41.0
0 0.00204789638519287 42.0
299.93648471 297.236450195312 43.0
0 0.0084766149520874 44.0
0 0.0168889760971069 45.0
0 -0.0607925653457642 46.0
0 -0.0560613870620728 47.0
0 -0.0355898141860962 48.0
0 0.294463336467743 49.0
0.094606658523 0.0543087720870972 50.0
1.7452614474 0.901306986808777 51.0
0 -0.104858040809631 52.0
0 -0.030016303062439 53.0
0 0.0344878435134888 54.0
0 -0.0936278104782104 55.0
0 -0.0152604579925537 56.0
0 -0.0552641153335571 57.0
0 -0.0141803026199341 58.0
0 -0.00854790210723877 59.0
0 0.00703370571136475 60.0
0 0.0578509569168091 61.0
0 0.862537443637848 62.0
260.21596519 253.905334472656 0.0
0 0.00190532207489014 1.0
0 -0.102142214775085 2.0
0 -0.0295392274856567 3.0
0 -0.112160563468933 4.0
0 -0.0269643068313599 5.0
0 -0.0449692010879517 6.0
0 -0.0354524850845337 7.0
0 0.00903522968292236 8.0
0 0.0422929525375366 9.0
0 0.00791788101196289 10.0
0 -0.032049298286438 11.0
0 0.0624498128890991 12.0
53.099206087 53.413444519043 13.0
41.356023468 39.0064582824707 14.0
0 -0.0145527124404907 15.0
0 -0.0432096719741821 16.0
33.019139068 32.5802726745605 17.0
0 0.0125876665115356 18.0
0 0.0303014516830444 19.0
0 -0.0162080526351929 20.0
0 -0.0504132509231567 21.0
0 0.0547908544540405 22.0
31.756504593 30.9573822021484 23.0
0 0.0795501470565796 24.0
0 0.520965218544006 25.0
0 -0.00656759738922119 26.0
0 -0.00461781024932861 27.0
0 -0.0293704271316528 28.0
0 -0.0782965421676636 29.0
0 0.0206824541091919 30.0
0 -0.0343436002731323 31.0
0 -0.018109917640686 32.0
0 0.0899671316146851 33.0
0 -0.0355616807937622 34.0
0 -0.0462347269058228 35.0
0 0.0898808240890503 36.0
0 -0.0264872312545776 37.0
0 -0.00218188762664795 38.0
0 0.0268620252609253 39.0
0 0.0648211240768433 40.0
0 -0.0588918924331665 41.0
0 0.000948309898376465 42.0
288.6975651 297.135955810547 43.0
0 -0.000434517860412598 44.0
0 0.0704842805862427 45.0
0 0.019473671913147 46.0
0 -0.012805700302124 47.0
0 0.025875449180603 48.0
0 0.283326923847198 49.0
0.069620648676 0.0810128450393677 50.0
1.892026531 0.927376508712769 51.0
0 -0.0585266351699829 52.0
0 -0.0248254537582397 53.0
0 -0.0282825231552124 54.0
0 0.0276862382888794 55.0
0 -0.0578418970108032 56.0
0 -0.035535454750061 57.0
0 -0.0572816133499146 58.0
0 -0.0673564672470093 59.0
0 -0.0730799436569214 60.0
0 -0.00342786312103271 61.0
0 0.843988418579102 62.0
258.84835025 256.089996337891 0.0
0 -0.125251412391663 1.0
0 0.0250283479690552 2.0
0 0.00666272640228271 3.0
0 -0.0443571805953979 4.0
0 -0.00396788120269775 5.0
0 -0.0458289384841919 6.0
0 -0.0513001680374146 7.0
0 -0.0583224296569824 8.0
0 -0.0312458276748657 9.0
0 -0.0129042863845825 10.0
0 0.020368218421936 11.0
0 -0.00898897647857666 12.0
72.231083023 72.0071258544922 13.0
22.391961858 27.1335887908936 14.0
0 -0.0498949289321899 15.0
0 -0.0103045701980591 16.0
35.064090269 35.1071510314941 17.0
0 0.0120762586593628 18.0
0 -0.0269745588302612 19.0
0 -0.0134516954421997 20.0
0 -0.0191205739974976 21.0
0 -0.0325495004653931 22.0
38.664490381 37.5695152282715 23.0
0 0.00127017498016357 24.0
0 0.498419642448425 25.0
0 0.0542534589767456 26.0
0 0.0198389291763306 27.0
0 0.0529874563217163 28.0
0 -0.0376836061477661 29.0
0 -0.0703186988830566 30.0
0 -0.0385857820510864 31.0
0 -0.0373688936233521 32.0
0 0.00318872928619385 33.0
0 0.00669443607330322 34.0
0 0.0246530771255493 35.0
0 -0.0182331800460815 36.0
0 -0.0499230623245239 37.0
0 0.0878912210464478 38.0
0 -0.0176030397415161 39.0
0 -0.127573609352112 40.0
0 -0.0244467258453369 41.0
0 -0.0119119882583618 42.0
294.03980122 296.728485107422 43.0
0 0.0157970190048218 44.0
0 0.00604647397994995 45.0
0 -0.0611103773117065 46.0
0 -0.0412346124649048 47.0
0 0.0297788381576538 48.0
0 0.294717818498611 49.0
0.078676989074 0.065503716468811 50.0
2.1521124148 0.933363258838654 51.0
0 -0.0703481435775757 52.0
0 -0.0500026941299438 53.0
0 -0.00612437725067139 54.0
0 -0.121385216712952 55.0
0 -0.0529211759567261 56.0
0 -0.00891077518463135 57.0
0 -0.0248066186904907 58.0
0 -0.0950521230697632 59.0
0 -0.0719248056411743 60.0
0 -0.0237623453140259 61.0
0 0.807959198951721 62.0
278.14158706 257.056335449219 0.0
0 -0.135376572608948 1.0
0 -0.00831234455108643 2.0
0 0.000793337821960449 3.0
0 -0.0496491193771362 4.0
0 0.0199450254440308 5.0
0 0.0281246900558472 6.0
0 -0.155445694923401 7.0
0 -0.0145068168640137 8.0
0 0.0393584966659546 9.0
0 0.00206553936004639 10.0
0 0.0248383283615112 11.0
0 -0.0324541330337524 12.0
53.599903491 53.0258178710938 13.0
38.219984152 36.7380905151367 14.0
0 -0.014051079750061 15.0
0 -0.0386656522750854 16.0
36.402517354 35.4705276489258 17.0
0 -0.0289961099624634 18.0
0 -0.0211783647537231 19.0
0 0.0185765027999878 20.0
0 -0.0361589193344116 21.0
0 0.05174720287323 22.0
30.137476528 28.8020343780518 23.0
0 0.0839565992355347 24.0
0 0.49482786655426 25.0
0 0.102726817131042 26.0
0 0.0257943868637085 27.0
0 -0.0103694200515747 28.0
0 -0.0682333707809448 29.0
0 0.0212620496749878 30.0
0 0.0089346170425415 31.0
0 -0.0251137018203735 32.0
0 0.0788394212722778 33.0
0 -0.0626233816146851 34.0
0 -0.0655277967453003 35.0
0 0.0235923528671265 36.0
0 -0.0688871145248413 37.0
0 0.00295770168304443 38.0
0 -0.0118182897567749 39.0
0 0.0121845006942749 40.0
0 -0.0273574590682983 41.0
0 -0.0118356943130493 42.0
301.40603488 296.008697509766 43.0
0 -0.0141302347183228 44.0
0 0.0798835158348083 45.0
0 0.00121605396270752 46.0
0 -0.0564764738082886 47.0
0 0.0315450429916382 48.0
0 0.277338564395905 49.0
0.070955344363 0.0749624967575073 50.0
2.0021030415 1.00542378425598 51.0
0 -0.0728393793106079 52.0
0 -0.0290099382400513 53.0
0 -0.0189923048019409 54.0
0 -0.0353559255599976 55.0
0 -0.0254675149917603 56.0
0 -0.0152901411056519 57.0
0 -0.000673174858093262 58.0
0 -0.00102007389068604 59.0
0 -0.0123752355575562 60.0
0 0.0458501577377319 61.0
0 0.799266934394836 62.0
222.53694425 255.131683349609 0.0
0 0.0282939672470093 1.0
0 -0.105360865592957 2.0
0 -0.036155104637146 3.0
0 -0.0511053800582886 4.0
0 0.0230618715286255 5.0
0 0.0347036123275757 6.0
0 -0.00935041904449463 7.0
0 0.0219182968139648 8.0
0 -0.040174126625061 9.0
0 0.0167931318283081 10.0
0 -0.00360465049743652 11.0
0 -0.109643578529358 12.0
58.566110583 58.7614212036133 13.0
31.244729912 37.0641555786133 14.0
0 -0.0229426622390747 15.0
0 -0.00670969486236572 16.0
31.916165464 32.3196487426758 17.0
0 0.0127543210983276 18.0
0 -0.0411869287490845 19.0
0 0.0892266035079956 20.0
0 0.00170361995697021 21.0
0 -0.0331898927688599 22.0
37.946938735 37.4202003479004 23.0
0 -0.0142995119094849 24.0
0 0.628006935119629 25.0
0 0.0440677404403687 26.0
0 0.118629097938538 27.0
0 -0.061921238899231 28.0
0 -0.0846890211105347 29.0
0 -0.0766583681106567 30.0
0 -0.00914299488067627 31.0
0 -0.00552618503570557 32.0
0 -0.0109010934829712 33.0
0 -0.0014265775680542 34.0
0 -0.0059126615524292 35.0
0 0.0381754636764526 36.0
0 -0.0915926694869995 37.0
0 0.0278223752975464 38.0
0 0.0244640111923218 39.0
0 -0.0360723733901978 40.0
0 -0.0517128705978394 41.0
0 0.00364053249359131 42.0
300.48097757 296.893829345703 43.0
0 0.0562695264816284 44.0
0 -0.0668551921844482 45.0
0 -0.0451148748397827 46.0
0 -0.0321923494338989 47.0
0 -0.0123711824417114 48.0
0 0.323690980672836 49.0
0.095123967199 0.0663353204727173 50.0
1.8145255127 0.928605556488037 51.0
0 -0.0423070192337036 52.0
0 -0.0312546491622925 53.0
0 0.0238596200942993 54.0
0 -0.0547727346420288 55.0
0 -0.106389164924622 56.0
0 -0.00113308429718018 57.0
0 -0.0513423681259155 58.0
0 -0.0444411039352417 59.0
0 -0.016710638999939 60.0
0 0.00440895557403564 61.0
0 0.853752255439758 62.0
212.06917478 253.45100402832 0.0
0 -0.0566164255142212 1.0
0 0.021553635597229 2.0
0 0.0258635282516479 3.0
0 -0.115773797035217 4.0
0 -0.0762253999710083 5.0
0 0.0772413015365601 6.0
0 0.142667174339294 7.0
0 0.0326986312866211 8.0
0 -0.071877121925354 9.0
0 -0.0213261842727661 10.0
0 -0.00704681873321533 11.0
0 0.0365155935287476 12.0
63.615569412 63.3040351867676 13.0
22.844709003 36.512996673584 14.0
0 -0.0495811700820923 15.0
0 -0.0304313898086548 16.0
35.260481058 35.1232490539551 17.0
0 -0.00122988224029541 18.0
0 -0.0255552530288696 19.0
0 0.0422874689102173 20.0
0 0.00464832782745361 21.0
0 -0.0449029207229614 22.0
34.333765282 32.8116569519043 23.0
0 0.0863722562789917 24.0
0 0.493327677249908 25.0
0 -0.0133849382400513 26.0
0 0.0295678377151489 27.0
0 -0.0742193460464478 28.0
0 -0.100157618522644 29.0
0 -0.0438083410263062 30.0
0 -0.0679932832717896 31.0
0 0.0538028478622437 32.0
0 -0.00945150852203369 33.0
0 -0.115100979804993 34.0
0 0.381076693534851 35.0
0 0.259123682975769 36.0
0 0.133452773094177 37.0
0 0.0558990240097046 38.0
0 0.107234358787537 39.0
0 0.0611432790756226 40.0
0 -0.000389933586120605 41.0
0 -0.0904356241226196 42.0
309.06944081 295.804931640625 43.0
0 0.0788244009017944 44.0
0 -0.0908188819885254 45.0
0 -0.0414527654647827 46.0
0 -0.00892364978790283 47.0
0 0.00840127468109131 48.0
0.43309061828 0.279368042945862 49.0
0.068300639146 0.110077977180481 50.0
2.0357855881 0.869958996772766 51.0
0 -0.0611649751663208 52.0
0 -0.0491930246353149 53.0
0 0.018532395362854 54.0
0 -0.0865339040756226 55.0
0 -0.0946606397628784 56.0
0 0.0228317975997925 57.0
0 -0.0528687238693237 58.0
0 0.0465737581253052 59.0
0 0.0580917596817017 60.0
0 -0.0797468423843384 61.0
0 0.849903702735901 62.0
277.3372159 248.991668701172 0.0
0 -0.0400296449661255 1.0
0 -0.0647433996200562 2.0
0 -0.0153354406356812 3.0
0 -0.0437060594558716 4.0
0 -0.136897206306458 5.0
0 0.00704705715179443 6.0
0 -0.170036911964417 7.0
0 0.00465857982635498 8.0
0 0.0762647390365601 9.0
0 0.0244731903076172 10.0
0 -0.00496304035186768 11.0
0 -0.064003586769104 12.0
52.135840533 51.7312507629395 13.0
45.26284574 42.7227592468262 14.0
0 -0.0188919305801392 15.0
0 -0.00466263294219971 16.0
32.806505844 32.9659614562988 17.0
0 -0.0345367193222046 18.0
0 -0.0201510190963745 19.0
0 -0.00850784778594971 20.0
0 -0.00736391544342041 21.0
0 0.0128768682479858 22.0
28.305452457 27.5483512878418 23.0
0 0.0951846837997437 24.0
16.213087482 0.478242993354797 25.0
0 0.110962271690369 26.0
0 0.07329261302948 27.0
0 0.0476802587509155 28.0
0 -0.0231207609176636 29.0
0 -0.0287476778030396 30.0
0 -0.0646408796310425 31.0
0 0.0152863264083862 32.0
0 0.0036245584487915 33.0
0 -0.0396932363510132 34.0
0 -0.0132390260696411 35.0
0 0.077104926109314 36.0
0 -0.0134032964706421 37.0
0 -0.0445955991744995 38.0
0 0.0372477769851685 39.0
0 0.0996125936508179 40.0
0 -0.0462604761123657 41.0
0 -0.0284954309463501 42.0
292.152561 296.735168457031 43.0
0 -0.0219370126724243 44.0
0 -0.0487790107727051 45.0
0 -0.0261427164077759 46.0
0 -0.00738847255706787 47.0
0 -0.0106440782546997 48.0
0 0.253090977668762 49.0
0.076956602856 0.0833638906478882 50.0
1.5658015475 0.983588516712189 51.0
0 -0.0794821977615356 52.0
0 -0.0154439210891724 53.0
0 0.00761902332305908 54.0
0 -0.116100192070007 55.0
0 -0.0630422830581665 56.0
0 -0.00132405757904053 57.0
0 -0.0507320165634155 58.0
0 -0.0721067190170288 59.0
0 -0.0404626131057739 60.0
0 -0.0265988111495972 61.0
0 0.879546523094177 62.0
240.35396961 256.674774169922 0.0
0 0.0212382078170776 1.0
0 -0.0855094194412231 2.0
0 -0.0252774953842163 3.0
0 -0.0205997228622437 4.0
0 0.0616377592086792 5.0
0 -0.0223070383071899 6.0
0 -0.0831311941146851 7.0
0 -0.0552512407302856 8.0
0 0.0227397680282593 9.0
0 -0.00677669048309326 10.0
0 -0.0185943841934204 11.0
0 -0.00180304050445557 12.0
70.913581622 70.491096496582 13.0
25.855697556 28.1298580169678 14.0
0 -0.0175830125808716 15.0
0 -0.0202928781509399 16.0
27.858357745 27.5977268218994 17.0
0 0.00629937648773193 18.0
0 -0.00472152233123779 19.0
0 0.0199614763259888 20.0
0 -0.0317565202713013 21.0
0 -0.0255204439163208 22.0
35.305742439 34.6162033081055 23.0
0 -0.0900822877883911 24.0
0 0.719706833362579 25.0
0 0.00772225856781006 26.0
0 0.0620940923690796 27.0
0 -0.00380289554595947 28.0
0 -0.0334926843643188 29.0
0 -0.0917849540710449 30.0
0 -0.00814688205718994 31.0
0 -0.0532981157302856 32.0
0 0.0729678869247437 33.0
0 -0.0278815031051636 34.0
0 0.00584065914154053 35.0
0 0.0212994813919067 36.0
0 -0.0718168020248413 37.0
0 0.00963461399078369 38.0
0 0.090412974357605 39.0
0 0.0672272443771362 40.0
0 -0.0270863771438599 41.0
0 -0.00844252109527588 42.0
303.54388244 297.356872558594 43.0
0 -0.037265419960022 44.0
0 0.00219321250915527 45.0
0 -0.0515626668930054 46.0
0 -0.0893248319625854 47.0
0 0.0150412321090698 48.0
0.93647173163 0.275835275650024 49.0
0.096348310978 0.0825890302658081 50.0
2.280504512 0.940860033035278 51.0
0 -0.0640279054641724 52.0
0 -0.0532232522964478 53.0
0 0.0729695558547974 54.0
0 -0.102344393730164 55.0
0 0.0550826787948608 56.0
0 -0.0413471460342407 57.0
0 -0.0175875425338745 58.0
0 -0.0176941156387329 59.0
0 0.112474083900452 60.0
0 0.0210355520248413 61.0
0 0.868913173675537 62.0
263.2494073 256.245513916016 0.0
0 -0.0650659799575806 1.0
0 -0.0863865613937378 2.0
0 0.0108777284622192 3.0
0 -0.0347853899002075 4.0
0 -0.012637734413147 5.0
0 0.0601791143417358 6.0
0 -0.0275264978408813 7.0
0 -0.0712605714797974 8.0
0 -0.0884519815444946 9.0
0 0.0249086618423462 10.0
0 -0.033608078956604 11.0
0 -0.142855286598206 12.0
50.264681029 49.7918128967285 13.0
38.649297015 38.9628562927246 14.0
0 -0.053134560585022 15.0
0 -0.0583065748214722 16.0
29.322728733 29.6816463470459 17.0
0 -0.027413010597229 18.0
0 0.0395761728286743 19.0
0 -0.0391956567764282 20.0
0 0.0117548704147339 21.0
0 -0.035530686378479 22.0
38.673965994 38.1049346923828 23.0
0 -0.0870219469070435 24.0
0 0.598655104637146 25.0
0 0.0334590673446655 26.0
0 0.0943096876144409 27.0
0 0.0895994901657104 28.0
0 -0.10977041721344 29.0
0 0.0217341184616089 30.0
0 -0.0252147912979126 31.0
0 0.00567233562469482 32.0
0 -0.121985793113708 33.0
0 -0.0183957815170288 34.0
0 -0.0272237062454224 35.0
0 0.177325844764709 36.0
0 -0.0707718133926392 37.0
0 -0.0354987382888794 38.0
0 0.171476244926453 39.0
0 0.263253092765808 40.0
0 -0.0715144872665405 41.0
0 0.0520137548446655 42.0
307.18118865 291.164703369141 43.0
0 -0.0162864923477173 44.0
0 -0.013302206993103 45.0
0 -0.0222393274307251 46.0
0 -0.0331512689590454 47.0
0 0.0283831357955933 48.0
1.7545256831 0.270776212215424 49.0
0.1032949599 0.0610047578811646 50.0
1.4554644357 0.912261307239532 51.0
0 -0.0188618898391724 52.0
0 -0.0622414350509644 53.0
0 0.0763391256332397 54.0
0 -0.094091534614563 55.0
0 -0.025478720664978 56.0
0 0.0733808279037476 57.0
0 -0.0840440988540649 58.0
0 -0.0715776681900024 59.0
0 -0.0913814306259155 60.0
0 -0.0564655065536499 61.0
0 0.900645971298218 62.0
222.59798319 254.858383178711 0.0
0 -0.0276132822036743 1.0
0 -0.0752240419387817 2.0
0 0.0108274221420288 3.0
0 -0.0341984033584595 4.0
0 -0.00155389308929443 5.0
0 0.0872834920883179 6.0
0 -0.0292733907699585 7.0
0 -0.0243204832077026 8.0
0 -0.0189253091812134 9.0
0 -0.0183264017105103 10.0
0 -0.00733983516693115 11.0
0 0.143710494041443 12.0
67.648901167 66.5083999633789 13.0
16.925311353 32.1311721801758 14.0
0 0.0350641012191772 15.0
0 -0.0371905565261841 16.0
33.533115246 33.1542549133301 17.0
0 0.00803458690643311 18.0
0 -0.0173441171646118 19.0
0 0.0444766283035278 20.0
0 -0.00924742221832275 21.0
0 0.00786817073822021 22.0
32.663695397 31.8820686340332 23.0
0 -0.0558019876480103 24.0
0 0.568179428577423 25.0
0 -0.0298856496810913 26.0
0 0.0797275304794312 27.0
0 0.0185707807540894 28.0
0 -0.0452562570571899 29.0
0 -0.0572130680084229 30.0
0 -0.0632582902908325 31.0
0 -0.0216208696365356 32.0
0 -0.0264631509780884 33.0
0 0.00786697864532471 34.0
0 -0.0286246538162231 35.0
0 0.151603102684021 36.0
0 -0.0213903188705444 37.0
0 0.0233045816421509 38.0
0 0.0973201990127563 39.0
0 -0.0148383378982544 40.0
0 -0.0465050935745239 41.0
0 0.00122487545013428 42.0
295.29786484 295.996398925781 43.0
0 -0.0071183443069458 44.0
0 -0.0712344646453857 45.0
0 0.000428557395935059 46.0
0 -0.058454155921936 47.0
0 -0.0199476480484009 48.0
0 0.259100705385208 49.0
0.078235822518 0.0771998167037964 50.0
2.0603740868 0.976485252380371 51.0
0 -0.0914913415908813 52.0
0 -0.0919648408889771 53.0
0 0.0286513566970825 54.0
0 -0.053547739982605 55.0
0 -0.0131732225418091 56.0
0 -0.0282307863235474 57.0
0 -0.0509864091873169 58.0
0 -0.0241092443466187 59.0
0 0.0632208585739136 60.0
0 0.00493347644805908 61.0
0 0.825740218162537 62.0
246.6890976 255.375686645508 0.0
0 -0.0547634363174438 1.0
0 0.0242372751235962 2.0
0 -0.0755420923233032 3.0
0 -0.02580726146698 4.0
0 -0.0718849897384644 5.0
0 -0.0638958215713501 6.0
0 -0.0705560445785522 7.0
0 0.00887501239776611 8.0
0 -0.0269733667373657 9.0
0 -0.00223028659820557 10.0
0 0.0478225946426392 11.0
0 0.00466287136077881 12.0
75.44972482 74.9027557373047 13.0
13.304151842 27.2658710479736 14.0
0 -0.006600022315979 15.0
0 -0.027276873588562 16.0
31.968047318 31.3285751342773 17.0
0 -0.0626176595687866 18.0
0 -0.00149500370025635 19.0
0 -0.0367270708084106 20.0
0 -0.0241559743881226 21.0
0 -0.0436984300613403 22.0
43.245574436 42.5573844909668 23.0
0 0.0200666189193726 24.0
0 0.51275771856308 25.0
0 -0.0146547555923462 26.0
0 0.0387464761734009 27.0
0 0.084288477897644 28.0
0 -0.0125218629837036 29.0
0 0.0584627389907837 30.0
0 -0.00729835033416748 31.0
0 0.106192469596863 32.0
0 -0.000670075416564941 33.0
0 -0.0107506513595581 34.0
0 -0.0320721864700317 35.0
0 0.1098393201828 36.0
0 -0.12011992931366 37.0
0 -0.0116449594497681 38.0
0 -0.0141443014144897 39.0
0 0.0771759748458862 40.0
0 -0.00059962272644043 41.0
0 0.0320676565170288 42.0
297.94637097 295.085632324219 43.0
0 -0.0157829523086548 44.0
0 0.00965166091918945 45.0
0 -0.0241750478744507 46.0
0 -0.0218931436538696 47.0
0 -0.0317369699478149 48.0
0 0.261502385139465 49.0
0.077762557051 0.0516370534896851 50.0
1.9892170678 1.01889252662659 51.0
0 -0.0686575174331665 52.0
0 -0.0658770799636841 53.0
0 -0.00243723392486572 54.0
0 -0.122302651405334 55.0
0 -0.0890628099441528 56.0
0 -0.0117846727371216 57.0
0 -0.0165842771530151 58.0
0 -0.0764657258987427 59.0
0 0.0836058855056763 60.0
0 -0.0314017534255981 61.0
0 0.854527354240417 62.0
267.16714832 256.147552490234 0.0
0 -0.0962082147598267 1.0
0 0.0255075693130493 2.0
0 -0.00826752185821533 3.0
0 -0.0316787958145142 4.0
0 -0.0140289068222046 5.0
0 0.00342977046966553 6.0
0 -0.0479737520217896 7.0
0 -0.0390570163726807 8.0
0 0.0274175405502319 9.0
0 0.0207165479660034 10.0
0 -0.00253808498382568 11.0
0 0.0908273458480835 12.0
69.653781677 68.755729675293 13.0
22.436049422 31.52490234375 14.0
0 -0.103510499000549 15.0
0 -0.0374380350112915 16.0
33.012393054 32.7673797607422 17.0
0 -0.0251108407974243 18.0
0 -0.025053858757019 19.0
0 -0.0172621011734009 20.0
0 -0.00512087345123291 21.0
0 -0.0762420892715454 22.0
33.770314693 33.3609008789062 23.0
0 0.0574935674667358 24.0
0 0.45893520116806 25.0
0 0.109104037284851 26.0
0 -0.00555503368377686 27.0
0 -0.0226727724075317 28.0
0 -0.0944794416427612 29.0
0 -0.0738791227340698 30.0
0 -0.0454391241073608 31.0
0 0.053491473197937 32.0
0 0.0585802793502808 33.0
0 -0.0452145338058472 34.0
0 0.0526479482650757 35.0
0 -0.0464898347854614 36.0
0 -0.110267758369446 37.0
0 -0.0165969133377075 38.0
0 0.106193900108337 39.0
0 0.103987097740173 40.0
0 -0.0369313955307007 41.0
0 -0.0327821969985962 42.0
288.85092307 296.569702148438 43.0
0 -0.0153242349624634 44.0
0 0.0278751850128174 45.0
0 -0.0371192693710327 46.0
0 -0.051017165184021 47.0
0 0.0314291715621948 48.0
0 0.310911238193512 49.0
0.081741139749 0.0470401048660278 50.0
1.6631941818 1.00514578819275 51.0
0 -0.100142598152161 52.0
0 -0.077241063117981 53.0
0 0.043418288230896 54.0
0 -0.177025675773621 55.0
0 -0.000927329063415527 56.0
0 0.0189131498336792 57.0
0 -0.0367382764816284 58.0
0 -0.0087047815322876 59.0
0 -0.00512421131134033 60.0
0 0.0545943975448608 61.0
0 0.90191924571991 62.0
230.31567264 255.122924804688 0.0
0 0.0160998106002808 1.0
0 -0.077872633934021 2.0
0 -0.0141023397445679 3.0
0 -0.0116738080978394 4.0
0 0.0643783807754517 5.0
0 0.0172628164291382 6.0
0 0.0109580755233765 7.0
0 -0.0218697786331177 8.0
0 -0.0283495187759399 9.0
0 0.0269011259078979 10.0
0 -0.059098482131958 11.0
0 -0.0357013940811157 12.0
65.267520376 63.7836151123047 13.0
34.263413919 30.5250949859619 14.0
0 -0.0403920412063599 15.0
0 0.0614608526229858 16.0
37.098704112 35.244068145752 17.0
0 -0.0295835733413696 18.0
0 -0.0370916128158569 19.0
0 0.0422884225845337 20.0
0 -0.0249100923538208 21.0
0 -0.000535130500793457 22.0
34.595929362 34.3544578552246 23.0
0 -0.0200139284133911 24.0
0 0.608217597007751 25.0
0 -0.0436087846755981 26.0
0 -0.0210064649581909 27.0
0 -0.0831643342971802 28.0
0 -0.0561031103134155 29.0
0 0.213653206825256 30.0
0 -0.0687712430953979 31.0
0 -0.105186104774475 32.0
0 -0.00632774829864502 33.0
0 -0.0375608205795288 34.0
0 0.0779792070388794 35.0
0 -0.0468038320541382 36.0
0 -0.0129300355911255 37.0
0 0.00268137454986572 38.0
0 0.0325521230697632 39.0
0 -0.0303264856338501 40.0
0 0.16974937915802 41.0
0 -0.030007004737854 42.0
313.37392863 296.342681884766 43.0
0 -0.0222259759902954 44.0
0 0.0446684956550598 45.0
0 -0.040035605430603 46.0
0 -0.0362659692764282 47.0
0 0.0284823179244995 48.0
0 0.258043646812439 49.0
0.067200362493 0.0785068273544312 50.0
2.2521492529 0.994417369365692 51.0
0 -0.025037407875061 52.0
0 -0.0447570085525513 53.0
0 -0.0205396413803101 54.0
0 -0.0456937551498413 55.0
0 -0.0332285165786743 56.0
0 -0.0511616468429565 57.0
0 -0.0745292901992798 58.0
0 -0.00279748439788818 59.0
0 0.143909811973572 60.0
0 0.0416866540908813 61.0
0 0.897417426109314 62.0
275.44364401 250.476684570312 0.0
0 -0.0343443155288696 1.0
0 -0.0499383211135864 2.0
0 -0.0252135992050171 3.0
0 -0.026868462562561 4.0
0 -0.0768734216690063 5.0
0 -0.0454927682876587 6.0
0 -0.131081700325012 7.0
0 0.0580575466156006 8.0
0 0.00023496150970459 9.0
0 0.00303995609283447 10.0
0 -0.0437151193618774 11.0
0 0.0397504568099976 12.0
57.672905438 57.5721092224121 13.0
39.881812559 37.7059211730957 14.0
0 -0.0153199434280396 15.0
0 -0.0248855352401733 16.0
35.573422543 35.1383514404297 17.0
0 0.00539815425872803 18.0
0 0.0449680089950562 19.0
0 0.0404096841812134 20.0
0 -0.0125917196273804 21.0
0 0.0186361074447632 22.0
36.965364417 37.1880340576172 23.0
0 0.10797393321991 24.0
0 0.435157835483551 25.0
0 0.0468047857284546 26.0
0 0.0257140398025513 27.0
0 0.0715831518173218 28.0
0 -0.143374562263489 29.0
0 -0.0697990655899048 30.0
0 -0.071581244468689 31.0
0 -0.0211511850357056 32.0
0 -0.0182498693466187 33.0
0 -0.0228837728500366 34.0
0 -0.100082516670227 35.0
0 0.0160399675369263 36.0
0 -0.0429805517196655 37.0
0 0.00265419483184814 38.0
0 0.0386902093887329 39.0
0 0.142692923545837 40.0
0 -0.0108097791671753 41.0
0 0.0770086050033569 42.0
282.71161507 296.869750976562 43.0
0 0.0200322866439819 44.0
0 -0.00298744440078735 45.0
0 -0.0206557512283325 46.0
0 -0.0470253229141235 47.0
0 -0.0646651983261108 48.0
0 0.255639791488647 49.0
0.067063441153 0.0655325651168823 50.0
2.1714743454 0.907233953475952 51.0
0 -0.0595318078994751 52.0
0 -0.0404444932937622 53.0
0 0.00344407558441162 54.0
0 -0.0681897401809692 55.0
0 -0.00997579097747803 56.0
0 0.0208185911178589 57.0
0 -0.0736914873123169 58.0
0 -0.107647776603699 59.0
0 0.00257670879364014 60.0
0 -0.00980126857757568 61.0
0 0.818911552429199 62.0
244.75775225 262.857757568359 0.0
0 -0.0681196451187134 1.0
0 -0.0703831911087036 2.0
0 0.0214318037033081 3.0
0 -0.0842560529708862 4.0
0 -0.012947678565979 5.0
0 -0.183631062507629 6.0
0 0.0225909948348999 7.0
0 0.0578999519348145 8.0
0 0.0637916326522827 9.0
0 -0.0210109949111938 10.0
0 0.0385686755180359 11.0
0 -0.00314581394195557 12.0
76.700689116 75.9307861328125 13.0
6.3886619918 26.1963748931885 14.0
0 -0.0433493852615356 15.0
0 -0.00248038768768311 16.0
28.483837565 28.5849571228027 17.0
0 0.0177468061447144 18.0
0 0.014057993888855 19.0
0 0.0321191549301147 20.0
0 -0.00364077091217041 21.0
0 -0.118240475654602 22.0
38.299228571 38.5777816772461 23.0
0 -0.00637543201446533 24.0
0 2.45319366455078 25.0
0 0.0255435705184937 26.0
0 0.107666611671448 27.0
0 -0.0626012086868286 28.0
0 -0.0504695177078247 29.0
0 -0.0577208995819092 30.0
0 -0.0678838491439819 31.0
0 0.0228546857833862 32.0
0 0.043155312538147 33.0
0 -0.0112859010696411 34.0
0 -0.0475286245346069 35.0
0 0.00783884525299072 36.0
0 -0.0902270078659058 37.0
0 0.0533000230789185 38.0
0 -0.0456959009170532 39.0
0 0.0621265172958374 40.0
0 -0.0123422145843506 41.0
0 -0.0354553461074829 42.0
291.61028005 296.042541503906 43.0
0 0.0168145895004272 44.0
0 0.015688955783844 45.0
0 -0.00773227214813232 46.0
0 -0.0069507360458374 47.0
0 0.0237025022506714 48.0
0 0.306209653615952 49.0
0.08486051157 0.0567373037338257 50.0
2.2265903285 0.986043751239777 51.0
0 -0.0236588716506958 52.0
0 -0.0477262735366821 53.0
0 0.0767754316329956 54.0
0 -0.0872412919998169 55.0
0 -0.00321900844573975 56.0
0 -0.0617343187332153 57.0
0 -0.0568674802780151 58.0
0 -0.0739732980728149 59.0
0 0.000263094902038574 60.0
0 0.059482216835022 61.0
0 0.854815423488617 62.0
288.10627486 261.377166748047 0.0
0 0.0180634260177612 1.0
0 -0.06908118724823 2.0
0 -0.0258997678756714 3.0
0 -0.0320757627487183 4.0
0 -0.00183117389678955 5.0
0 0.0584896802902222 6.0
0 -0.0472980737686157 7.0
0 -0.0520333051681519 8.0
0 0.00578725337982178 9.0
0 -0.00579512119293213 10.0
0 -0.0243725776672363 11.0
0 -0.00641119480133057 12.0
62.914499448 61.667537689209 13.0
38.025730684 36.1431198120117 14.0
0 0.000847220420837402 15.0
0 -0.0332971811294556 16.0
29.391276531 29.2069931030273 17.0
0 -0.00269448757171631 18.0
0 0.00778520107269287 19.0
0 0.00502383708953857 20.0
0 -0.0285121202468872 21.0
0 0.00866353511810303 22.0
29.343137579 27.9615135192871 23.0
0 0.00659692287445068 24.0
0 0.570730805397034 25.0
0 0.0173567533493042 26.0
0 -0.0478847026824951 27.0
0 0.0383862257003784 28.0
0 -0.0149255990982056 29.0
0 -0.067246675491333 30.0
0 -0.0335260629653931 31.0
0 -0.00470268726348877 32.0
0 0.0437294244766235 33.0
0 -0.0231350660324097 34.0
0 -0.0701614618301392 35.0
0 0.102623105049133 36.0
0 -0.0291303396224976 37.0
0 0.00735509395599365 38.0
0 0.0255802869796753 39.0
0 0.088701605796814 40.0
0 -0.0148634910583496 41.0
0 -0.0109173059463501 42.0
293.55354108 299.047424316406 43.0
0 -0.00253593921661377 44.0
0 0.0210343599319458 45.0
0 0.00343549251556396 46.0
0 -0.000640988349914551 47.0
0 0.0194529294967651 48.0
0 0.280448466539383 49.0
0.097549822087 0.0675731897354126 50.0
1.9293970625 1.01375460624695 51.0
0 -0.0330420732498169 52.0
0 -0.0358811616897583 53.0
0 -0.000135540962219238 54.0
0 -0.0913492441177368 55.0
0 -0.0499218702316284 56.0
0 -0.0104690790176392 57.0
0 -0.0225111246109009 58.0
0 -0.0583125352859497 59.0
0 -0.105817437171936 60.0
0 0.00287854671478271 61.0
0 0.863359332084656 62.0
269.35934311 252.690246582031 0.0
0 -0.108025670051575 1.0
0 0.0472208261489868 2.0
0 -0.0207027196884155 3.0
0 -0.0696221590042114 4.0
0 0.0490168333053589 5.0
0 0.0473431348800659 6.0
0 -0.0263553857803345 7.0
0 0.0503242015838623 8.0
0 0.0276645421981812 9.0
0 0.00657987594604492 10.0
0 -0.0420656204223633 11.0
0 -0.130241990089417 12.0
62.960999471 61.753231048584 13.0
36.710074915 34.2466087341309 14.0
0 -0.114352822303772 15.0
0 -0.0250927209854126 16.0
40.197456205 38.8850631713867 17.0
0 -0.0167616605758667 18.0
0 -0.0156708955764771 19.0
0 -0.0329388380050659 20.0
0 -0.0194040536880493 21.0
0 -0.0156151056289673 22.0
34.603412945 33.8929672241211 23.0
0 -0.0291038751602173 24.0
0 0.559053182601929 25.0
0 0.0398398637771606 26.0
0 0.0328345894813538 27.0
0 -0.0112873315811157 28.0
0 -0.0108758211135864 29.0
0 -0.162580132484436 30.0
0 0.0420466661453247 31.0
0 0.0454195737838745 32.0
0 -0.0379413366317749 33.0
0 0.0134865045547485 34.0
0 -0.0512791872024536 35.0
0 0.081612229347229 36.0
0 -0.0204707384109497 37.0
0 0.00789344310760498 38.0
0 -0.00931251049041748 39.0
0 0.065151572227478 40.0
0 -0.03688645362854 41.0
0 -0.0512157678604126 42.0
291.21260012 296.981018066406 43.0
0 0.0550369024276733 44.0
0 -0.00305652618408203 45.0
0 -0.109611630439758 46.0
0 -0.0492650270462036 47.0
0 0.0402010679244995 48.0
0.54675711017 0.326269328594208 49.0
0.070668795436 0.0530341863632202 50.0
2.3651984301 1.01294481754303 51.0
0 -0.0743368864059448 52.0
0 0.00733506679534912 53.0
0 0.0276161432266235 54.0
0 -0.0863126516342163 55.0
0 -0.0843137502670288 56.0
0 0.0110315084457397 57.0
0 -0.0617576837539673 58.0
0 -0.102012991905212 59.0
0 -0.00390136241912842 60.0
0 0.026128888130188 61.0
0 0.784632921218872 62.0
241.0293595 255.422897338867 0.0
0 -0.00724565982818604 1.0
0 -0.112024903297424 2.0
0 -0.00558149814605713 3.0
0 -0.0369762182235718 4.0
0 0.0562931299209595 5.0
0 0.0205091238021851 6.0
0 -0.0964759588241577 7.0
0 0.0079912543296814 8.0
0 0.0401080846786499 9.0
0 -0.016008734703064 10.0
0 -0.044903039932251 11.0
0 -0.117805123329163 12.0
62.809096643 62.1561279296875 13.0
31.011153532 34.7977523803711 14.0
0 -0.00997889041900635 15.0
0 -0.0478419065475464 16.0
33.532459082 33.1092109680176 17.0
0 0.0539628267288208 18.0
0 0.000945210456848145 19.0
0 -0.00485098361968994 20.0
0 -0.0145119428634644 21.0
0 -0.0252572298049927 22.0
33.546000417 32.572940826416 23.0
0 0.0211218595504761 24.0
0 0.532090604305267 25.0
0 0.0547295808792114 26.0
0 0.0197286009788513 27.0
0 0.0236324071884155 28.0
0 -0.0484944581985474 29.0
0 -0.0778404474258423 30.0
0 -0.0560253858566284 31.0
0 -0.0581690073013306 32.0
0 0.0138190984725952 33.0
0 -0.00359523296356201 34.0
0 -0.0657943487167358 35.0
0 -0.0102428197860718 36.0
0 -0.0423964262008667 37.0
0 -0.0211597681045532 38.0
0 0.0522807836532593 39.0
0 -0.0252972841262817 40.0
0 -0.0851404666900635 41.0
0 -0.0595031976699829 42.0
292.51363609 295.149230957031 43.0
0 -0.00153696537017822 44.0
0 -0.0416439771652222 45.0
0 0.00798380374908447 46.0
0 -0.0201684236526489 47.0
0 -0.00948774814605713 48.0
0 0.260287463665009 49.0
0.075020841311 0.0896583795547485 50.0
2.1502454019 0.944823443889618 51.0
0 -0.0782397985458374 52.0
0 -0.0256367921829224 53.0
0 0.0321890115737915 54.0
0 -0.0999523401260376 55.0
0 -0.014989972114563 56.0
0 -0.0158177614212036 57.0
0 -0.0583647489547729 58.0
0 -0.017957329750061 59.0
0 -0.106893420219421 60.0
0 0.0515586137771606 61.0
0 0.805577397346497 62.0
257.60342654 247.339492797852 0.0
0 -0.0308777093887329 1.0
0 0.00492680072784424 2.0
0 0.0247422456741333 3.0
0 -0.0213688611984253 4.0
0 0.00711166858673096 5.0
0 0.220268607139587 6.0
0 -0.0878416299819946 7.0
0 -0.0198565721511841 8.0
0 -0.0200930833816528 9.0
0 -0.0599690675735474 10.0
0 -0.0192159414291382 11.0
0 -0.00658094882965088 12.0
79.472037348 80.4418106079102 13.0
29.734155259 18.6649570465088 14.0
0 -0.0381571054458618 15.0
0 -0.0142484903335571 16.0
30.444792609 29.6549301147461 17.0
0 0.0815674066543579 18.0
0 0.0497912168502808 19.0
0 -0.0402413606643677 20.0
0 -0.0595158338546753 21.0
0 -0.102848649024963 22.0
26.815846394 25.8754653930664 23.0
0 0.0663484334945679 24.0
0 0.489053249359131 25.0
0 0.0491994619369507 26.0
0 0.040532648563385 27.0
0 0.0740727186203003 28.0
0 -0.0730890035629272 29.0
0 -0.0170589685440063 30.0
0 -0.00621116161346436 31.0
0 -0.0239397287368774 32.0
0 -0.101765394210815 33.0
0 -0.0500143766403198 34.0
0 -0.0356036424636841 35.0
0 0.0902620553970337 36.0
0 0.00715196132659912 37.0
0 -0.0365835428237915 38.0
0 -0.00746893882751465 39.0
0 0.120398879051208 40.0
0 0.0376211404800415 41.0
0 -0.0301843881607056 42.0
281.87258828 295.627685546875 43.0
0 0.0236247777938843 44.0
0 0.0618716478347778 45.0
0 -0.0144656896591187 46.0
0 -0.026136040687561 47.0
0 0.0314697027206421 48.0
0 0.346793711185455 49.0
0.082652401439 0.0464259386062622 50.0
1.5683646733 1.00413489341736 51.0
0 -0.0860964059829712 52.0
0 -0.0523890256881714 53.0
0 -0.115781188011169 54.0
0 -0.142585396766663 55.0
0 -0.0791705846786499 56.0
0 -0.0248872041702271 57.0
0 -0.0512071847915649 58.0
0 -0.00906383991241455 59.0
0 0.0270975828170776 60.0
0 0.0778795480728149 61.0
0 0.939576148986816 62.0
274.11611659 261.316528320312 0.0
0 -0.109190583229065 1.0
0 0.0463420152664185 2.0
0 -0.0314203500747681 3.0
0 -0.0128875970840454 4.0
0 -0.129786610603333 5.0
0 -0.0160204172134399 6.0
0 -0.0931230783462524 7.0
0 -0.0550789833068848 8.0
0 -0.0541015863418579 9.0
0 0.00239574909210205 10.0
0 -0.026789665222168 11.0
0 0.0384315252304077 12.0
74.933474076 73.7892074584961 13.0
21.708000001 23.5068740844727 14.0
0 -0.0249172449111938 15.0
0 -0.0632034540176392 16.0
34.476638225 33.7848205566406 17.0
0 0.052404522895813 18.0
0 0.0617083311080933 19.0
0 0.0189744234085083 20.0
0 -0.0570648908615112 21.0
0 -0.0566936731338501 22.0
41.994809507 40.8271217346191 23.0
0 -0.14394199848175 24.0
0 0.658069550991058 25.0
0 -0.0415836572647095 26.0
0 0.0384919047355652 27.0
0 -0.0339885950088501 28.0
0 0.00490128993988037 29.0
0 -0.0514558553695679 30.0
0 -0.0269643068313599 31.0
0 -0.009132981300354 32.0
0 -0.0235196352005005 33.0
0 -0.0115801095962524 34.0
0 0.119556784629822 35.0
0 0.0261880159378052 36.0
0 -0.0642889738082886 37.0
0 -0.00211489200592041 38.0
0 -0.00349581241607666 39.0
0 0.131309390068054 40.0
0 -0.0155435800552368 41.0
0 0.0483773946762085 42.0
304.05714558 296.997802734375 43.0
0 0.0235809087753296 44.0
0 0.102507174015045 45.0
0 -0.0133353471755981 46.0
0 -0.0172132253646851 47.0
0 0.0631042718887329 48.0
0 0.295515418052673 49.0
0.097559138507 0.0237418413162231 50.0
2.2657307849 0.860731780529022 51.0
0 -0.0372945070266724 52.0
0 0.0296057462692261 53.0
0 0.0286058187484741 54.0
0 -0.0486406087875366 55.0
0 -0.0769530534744263 56.0
0 -0.0168324708938599 57.0
0 -0.0550028085708618 58.0
0 -0.0374222993850708 59.0
0 -0.0207120180130005 60.0
0 0.0807501077651978 61.0
0 0.980608761310577 62.0
254.8697006 252.387817382812 0.0
0 -0.00964748859405518 1.0
0 -0.0846453905105591 2.0
0 0.0093458890914917 3.0
0 -0.0110944509506226 4.0
0 0.0233656167984009 5.0
0 0.0613998174667358 6.0
0 -0.0457698106765747 7.0
0 -0.00841766595840454 8.0
0 0.0419172048568726 9.0
0 0.0471252202987671 10.0
0 -0.0443924069404602 11.0
0 -0.113291382789612 12.0
72.287598451 71.9456634521484 13.0
38.217445805 25.2835693359375 14.0
0 0.0349401235580444 15.0
0 -0.0328425168991089 16.0
26.550774978 26.6122379302979 17.0
0 0.000548243522644043 18.0
0 -0.0141204595565796 19.0
0 0.0160363912582397 20.0
0 0.0240849256515503 21.0
0 -0.0486708879470825 22.0
34.696864364 34.5947570800781 23.0
0 -0.0821424722671509 24.0
0 1.13273310661316 25.0
0 0.0403636693954468 26.0
0 0.0551186800003052 27.0
0 0.0760186910629272 28.0
0 -0.0550252199172974 29.0
0 -0.085063099861145 30.0
0 -0.00630271434783936 31.0
0 -0.0127078294754028 32.0
0 -0.0209585428237915 33.0
0 -0.0339361429214478 34.0
0 -0.0112004280090332 35.0
0 0.0598567724227905 36.0
0 -0.0332728624343872 37.0
0 -0.00110948085784912 38.0
0 0.0408428907394409 39.0
0 0.0888770818710327 40.0
0 0.00686812400817871 41.0
0 0.0685733556747437 42.0
295.24219041 294.565307617188 43.0
0 -0.0342890024185181 44.0
0 0.0633821487426758 45.0
0 -0.0329135656356812 46.0
0 -0.0333056449890137 47.0
0 0.0120581388473511 48.0
0 0.335274904966354 49.0
0.10146427836 0.0587650537490845 50.0
2.3399098852 1.00629305839539 51.0
0 -0.094274640083313 52.0
0 -0.036023736000061 53.0
0 0.064726710319519 54.0
0 -0.154645323753357 55.0
0 0.142714142799377 56.0
0 0.123831629753113 57.0
0 -0.0361922979354858 58.0
0 -0.0466212034225464 59.0
0 0.0934163331985474 60.0
0 -0.0924347639083862 61.0
0 0.839187026023865 62.0
250.97533467 257.592071533203 0.0
0 -0.15733015537262 1.0
0 -0.0172191858291626 2.0
0 0.0585006475448608 3.0
0 -0.0344928503036499 4.0
0 0.000101447105407715 5.0
0 0.0670708417892456 6.0
0 0.0543433427810669 7.0
0 0.0673640966415405 8.0
0 -0.0852380990982056 9.0
0 -0.0251590013504028 10.0
0 -0.00883889198303223 11.0
0 -0.267891526222229 12.0
57.215070168 57.2203598022461 13.0
32.617684704 33.9310531616211 14.0
0 -0.261553406715393 15.0
0 0.0352967977523804 16.0
33.134670835 34.081657409668 17.0
0 0.00921571254730225 18.0
0 0.0327848196029663 19.0
0 0.0627511739730835 20.0
0 -0.0532070398330688 21.0
0 0.00994527339935303 22.0
28.009451411 26.9385509490967 23.0
0 0.0748442411422729 24.0
6.3752186108 0.496312499046326 25.0
0 -0.0352360010147095 26.0
0 0.0270422697067261 27.0
0 0.0564678907394409 28.0
0 -0.141536593437195 29.0
0 -0.0511442422866821 30.0
0 -0.0292822122573853 31.0
0 0.0516265630722046 32.0
0 -0.0335170030593872 33.0
0 -0.017513632774353 34.0
0 -0.0548475980758667 35.0
0 0.085121750831604 36.0
0 -0.0259114503860474 37.0
0 0.0145174264907837 38.0
0 -0.0165261030197144 39.0
0 -0.0918303728103638 40.0
0 0.226746320724487 41.0
0 -0.21235716342926 42.0
287.02234003 284.260833740234 43.0
0 0.00891435146331787 44.0
0 -0.440905034542084 45.0
0 -0.0358680486679077 46.0
0 -0.0878142118453979 47.0
0 -0.0345584154129028 48.0
0 0.324992001056671 49.0
0.061394233851 0.0437673330307007 50.0
2.4102527111 0.89005833864212 51.0
0 -0.0614193677902222 52.0
0 -0.043584942817688 53.0
0 0.0115922689437866 54.0
0 -0.12663733959198 55.0
0 -0.0624538660049438 56.0
0 -0.0520273447036743 57.0
0 -0.0223194360733032 58.0
0 -0.0124181509017944 59.0
0 0.153496384620667 60.0
0 0.0101169347763062 61.0
0 0.79494035243988 62.0
231.1341196 253.415740966797 0.0
0 -0.0515323877334595 1.0
0 -0.0890551805496216 2.0
0 -0.0296376943588257 3.0
0 -0.0336605310440063 4.0
0 -0.0713149309158325 5.0
0 -0.0491186380386353 6.0
0 -0.0234102010726929 7.0
0 0.0179328918457031 8.0
0 0.0463727712631226 9.0
0 -0.0289944410324097 10.0
0 -0.0206024646759033 11.0
0 -0.0953251123428345 12.0
56.166767391 56.0138397216797 13.0
36.51423982 40.4372978210449 14.0
0 -0.0100818872451782 15.0
0 -0.0240374803543091 16.0
38.633350611 38.4166488647461 17.0
0 0.0242658853530884 18.0
0 -0.0162214040756226 19.0
0 0.00221693515777588 20.0
0 -0.00579988956451416 21.0
0 -0.0514765977859497 22.0
40.767065161 39.6300354003906 23.0
0 -0.0205627679824829 24.0
0 0.516294717788696 25.0
0 0.0352596044540405 26.0
0 -0.0752662420272827 27.0
0 -0.0538874864578247 28.0
0 -0.0329815149307251 29.0
0 -0.0702557563781738 30.0
0 -0.0618120431900024 31.0
0 0.0561841726303101 32.0
0 0.0937536954879761 33.0
0 0.0119267702102661 34.0
0 0.00806939601898193 35.0
0 0.0553358793258667 36.0
0 -0.0308381319046021 37.0
0 -0.0205255746841431 38.0
0 -0.0163496732711792 39.0
0 0.111033320426941 40.0
0 0.0401535034179688 41.0
0 0.0253928899765015 42.0
303.98146919 297.312805175781 43.0
0 -0.00243151187896729 44.0
0 0.0935052633285522 45.0
0 -0.0399118661880493 46.0
0 -0.0170226097106934 47.0
0 -0.0740553140640259 48.0
0 0.298195421695709 49.0
0.084321224124 0.0528246164321899 50.0
2.135597622 0.977249920368195 51.0
0 -0.0510128736495972 52.0
0 -0.0560160875320435 53.0
0 0.0369044542312622 54.0
0 -0.105579733848572 55.0
0 -0.0460458993911743 56.0
0 0.0469094514846802 57.0
0 -0.037152886390686 58.0
0 -0.0495320558547974 59.0
0 -0.00681030750274658 60.0
0 0.0313009023666382 61.0
0 0.797249555587769 62.0
213.99382493 255.367034912109 0.0
0 -0.0277544260025024 1.0
0 -0.0647836923599243 2.0
0 -0.048596978187561 3.0
0 -0.00826394557952881 4.0
0 0.00700294971466064 5.0
0 -0.0390621423721313 6.0
0 -0.0268722772598267 7.0
0 0.0167802572250366 8.0
0 0.0598405599594116 9.0
0 -0.0157526731491089 10.0
0 0.028209924697876 11.0
0 0.185821890830994 12.0
57.947298952 58.369815826416 13.0
37.894288682 40.1499328613281 14.0
0 -0.0158883333206177 15.0
0 -0.00542819499969482 16.0
29.35575339 28.4550361633301 17.0
0 0.0552340745925903 18.0
0 -0.0092085599899292 19.0
0 0.0462328195571899 20.0
0 0.0287381410598755 21.0
0 -0.0131560564041138 22.0
36.63369869 36.104793548584 23.0
0 -0.0229655504226685 24.0
0 0.880105197429657 25.0
0 0.0195430517196655 26.0
0 0.03052818775177 27.0
0 0.0211623907089233 28.0
0 -0.0356904268264771 29.0
0 -0.054293155670166 30.0
0 -0.0472403764724731 31.0
0 0.010276198387146 32.0
0 -0.160422801971436 33.0
0 -0.0546187162399292 34.0
0 -0.0720335245132446 35.0
0 0.119105696678162 36.0
0 -0.0529974699020386 37.0
0 -0.0110307931900024 38.0
0 -0.0595005750656128 39.0
0 0.126829504966736 40.0
0 -0.0802491903305054 41.0
0 0.0183843374252319 42.0
301.13895711 296.011840820312 43.0
0 0.0245555639266968 44.0
0 0.0761387348175049 45.0
0 -0.00121915340423584 46.0
0 -0.0297752618789673 47.0
0 -0.0413447618484497 48.0
0.60406110081 0.275209188461304 49.0
0.072416596431 0.061015248298645 50.0
1.7912737136 1.00699853897095 51.0
0 -0.0781100988388062 52.0
0 -0.0482178926467896 53.0
0 0.0156437158584595 54.0
0 -0.0486069917678833 55.0
0 -0.0269993543624878 56.0
0 -0.0320688486099243 57.0
0 -0.00516188144683838 58.0
0 0.0289355516433716 59.0
0 -0.0310934782028198 60.0
0 -0.0477226972579956 61.0
0 0.840914845466614 62.0
256.23920654 260.224578857422 0.0
0 -0.0737787485122681 1.0
0 0.0361326932907104 2.0
0 0.00587284564971924 3.0
0 -0.0266624689102173 4.0
0 0.0522791147232056 5.0
0 0.0114024877548218 6.0
0 -0.0318111181259155 7.0
0 0.00630676746368408 8.0
0 -0.00842249393463135 9.0
0 0.0184239149093628 10.0
0 -0.0369472503662109 11.0
0 -0.037697434425354 12.0
62.003353772 60.5055618286133 13.0
27.473281213 37.4094848632812 14.0
0 -0.040753960609436 15.0
0 -0.0309544801712036 16.0
28.877149546 28.8521690368652 17.0
0 0.077900767326355 18.0
0 -0.0123807191848755 19.0
0 -0.015257716178894 20.0
0 -0.0380083322525024 21.0
0 -0.0445171594619751 22.0
40.674969883 39.5679550170898 23.0
0 0.0520514249801636 24.0
0 0.558357417583466 25.0
0 0.0538777112960815 26.0
0 0.104592442512512 27.0
0 -0.0197917222976685 28.0
0 -0.0151609182357788 29.0
0 -0.060204029083252 30.0
0 -0.0410689115524292 31.0
0 -0.0351244211196899 32.0
0 0.0177594423294067 33.0
0 -0.00959169864654541 34.0
0 -0.052254319190979 35.0
0 -0.0653077363967896 36.0
0 -0.075276255607605 37.0
0 0.00985229015350342 38.0
0 0.148373961448669 39.0
0 0.10046660900116 40.0
0 -0.0601392984390259 41.0
0 0.0115712881088257 42.0
301.38190128 293.791442871094 43.0
0 -0.0170480012893677 44.0
0 0.0303341150283813 45.0
0 -0.021030068397522 46.0
0 -0.0462275743484497 47.0
0 -0.0102683305740356 48.0
0 0.301744163036346 49.0
0.074101118079 0.0639210939407349 50.0
1.8553044493 0.899121165275574 51.0
0 -0.0930029153823853 52.0
0 -0.0466936826705933 53.0
0 0.0230871438980103 54.0
0 -0.0766321420669556 55.0
0 -0.00857222080230713 56.0
0 -0.0396510362625122 57.0
0 -0.0418347120285034 58.0
0 -0.0421870946884155 59.0
0 -0.0929387807846069 60.0
0 0.0345498323440552 61.0
0 0.880070090293884 62.0
283.15171873 254.711822509766 0.0
0 -0.019025444984436 1.0
0 -0.00223433971405029 2.0
0 0.00999486446380615 3.0
0 -0.0429631471633911 4.0
0 0.0190175771713257 5.0
0 -0.152942299842834 6.0
0 -0.0144764184951782 7.0
0 0.0243834257125854 8.0
0 0.027239203453064 9.0
0 -0.00218808650970459 10.0
0 0.00724554061889648 11.0
0 -0.183017373085022 12.0
51.179446404 51.1212120056152 13.0
43.690931356 42.4461898803711 14.0
0 -0.0912081003189087 15.0
0 -0.0430275201797485 16.0
36.246890975 35.5291557312012 17.0
0 0.0263139009475708 18.0
0 -0.00758969783782959 19.0
0 0.0231536626815796 20.0
0 0.0372353792190552 21.0
0 -0.0136820077896118 22.0
29.861079349 29.2037887573242 23.0
0 -0.0182849168777466 24.0
0 0.53277051448822 25.0
0 -0.00138604640960693 26.0
0 0.0514432191848755 27.0
0 0.0130113363265991 28.0
0 -0.0287817716598511 29.0
0 -0.0370447635650635 30.0
0 -0.0388761758804321 31.0
0 -0.00115406513214111 32.0
0 0.0472582578659058 33.0
0 -0.0391815900802612 34.0
0 -0.0228615999221802 35.0
0 0.134413599967957 36.0
0 -0.062024712562561 37.0
0 0.0365082025527954 38.0
0 0.0264579057693481 39.0
0 -0.00400173664093018 40.0
0 0.00494885444641113 41.0
0 0.000817656517028809 42.0
286.19624746 297.517700195312 43.0
0 -0.024236798286438 44.0
0 -0.0607573986053467 45.0
0 -0.0146263837814331 46.0
0 -0.0184105634689331 47.0
0 -0.00461781024932861 48.0
0 0.274610877037048 49.0
0.076541066447 0.0713819265365601 50.0
2.2538349655 0.915647387504578 51.0
0 -0.0292898416519165 52.0
0 0.0190702676773071 53.0
0 0.0413576364517212 54.0
0 -0.051784873008728 55.0
0 -0.0540131330490112 56.0
0 -0.00281250476837158 57.0
0 -0.0743273496627808 58.0
0 -0.0665611028671265 59.0
0 -0.053195595741272 60.0
0 -0.0048903226852417 61.0
0 0.802796959877014 62.0
270.64992015 255.45947265625 0.0
0 -0.0262323617935181 1.0
0 -0.111508727073669 2.0
0 -0.00808298587799072 3.0
0 -0.0386954545974731 4.0
0 0.0173276662826538 5.0
0 0.0143617391586304 6.0
0 -0.000670552253723145 7.0
0 0.00480842590332031 8.0
0 -0.0208727121353149 9.0
0 -0.0182825326919556 10.0
0 -0.0304199457168579 11.0
0 0.119672179222107 12.0
68.909100336 68.4333190917969 13.0
28.212666377 29.8983573913574 14.0
0 -0.014224648475647 15.0
0 0.000388026237487793 16.0
36.712224368 35.9794883728027 17.0
0 -0.0177189111709595 18.0
0 -0.00637471675872803 19.0
0 -0.0439223051071167 20.0
0 -0.0114833116531372 21.0
0 -0.016584038734436 22.0
33.23723737 32.0079307556152 23.0
0 0.0115402936935425 24.0
0 0.53547614812851 25.0
0 0.0743921995162964 26.0
0 0.0794187784194946 27.0
0 -0.0615264177322388 28.0
0 -0.059857964515686 29.0
0 -0.0688291788101196 30.0
0 -0.0176259279251099 31.0
0 -0.00806200504302979 32.0
0 -0.0331763029098511 33.0
0 -0.0105100870132446 34.0
0 -0.0289108753204346 35.0
0 -0.112367510795593 36.0
0 -0.059904932975769 37.0
0 0.0197380781173706 38.0
0 0.0630334615707397 39.0
0 0.106267809867859 40.0
0 -0.0459479093551636 41.0
0 -0.0267215967178345 42.0
297.31427227 294.635925292969 43.0
0 -0.0200382471084595 44.0
0 -0.0735217332839966 45.0
0 -0.0504461526870728 46.0
0 -0.0106583833694458 47.0
0 -0.00716888904571533 48.0
0.28523358932 0.309435218572617 49.0
0.076343919732 0.0862430334091187 50.0
1.7036314385 0.90805584192276 51.0
0 -0.0444997549057007 52.0
0 0.0107179880142212 53.0
0 0.0494276285171509 54.0
0 -0.115646004676819 55.0
0 -0.0364905595779419 56.0
0 -0.0292593240737915 57.0
0 -0.0309983491897583 58.0
0 -0.0844720602035522 59.0
0 -0.0165098905563354 60.0
0 -0.0133000612258911 61.0
0 0.854236602783203 62.0
253.44551137 253.811416625977 0.0
0 -0.0432125329971313 1.0
0 -0.0827032327651978 2.0
0 -0.00735056400299072 3.0
0 -0.000210404396057129 4.0
0 -0.0995022058486938 5.0
0 -0.0999666452407837 6.0
0 -0.0545634031295776 7.0
0 -0.00757169723510742 8.0
0 -0.0224658250808716 9.0
0 -0.00285279750823975 10.0
0 -0.018596887588501 11.0
0 0.00243031978607178 12.0
72.732976148 72.3288955688477 13.0
11.426996931 25.2307510375977 14.0
0 -0.0455328226089478 15.0
0 -0.00257408618927002 16.0
31.043602909 30.9084339141846 17.0
0 -0.0330146551132202 18.0
0 0.00622141361236572 19.0
0 0.0218073129653931 20.0
0 -0.0206583738327026 21.0
0 -0.0232001543045044 22.0
35.518940528 34.8403129577637 23.0
0 -0.076727032661438 24.0
0 0.707884788513184 25.0
0 0.0519756078720093 26.0
0 0.0814458131790161 27.0
0 -0.0630704164505005 28.0
0 -0.0410205125808716 29.0
0 -0.0840667486190796 30.0
0 -0.0163975954055786 31.0
0 -0.0262991189956665 32.0
0 -0.023340106010437 33.0
0 -0.0334955453872681 34.0
0 -0.061484694480896 35.0
0 0.0257920026779175 36.0
0 -0.0154160261154175 37.0
0 -0.00349581241607666 38.0
0 0.0610274076461792 39.0
0 0.0698546171188354 40.0
0 -0.039057731628418 41.0
0 -0.0389419794082642 42.0
307.5000653 297.778350830078 43.0
0 -0.013344407081604 44.0
0 0.0430310368537903 45.0
0 0.00934255123138428 46.0
0 -0.0659948587417603 47.0
0 0.0168269872665405 48.0
0 0.301617741584778 49.0
0.096742985543 0.0512813329696655 50.0
2.0028143471 1.00335454940796 51.0
0 -0.0748859643936157 52.0
0 -0.0464383363723755 53.0
0 0.0649608373641968 54.0
0 -0.0605584383010864 55.0
0 -0.00933802127838135 56.0
0 -0.000192761421203613 57.0
0 -0.056954026222229 58.0
0 -0.00997602939605713 59.0
0 0.0721369981765747 60.0
0 0.0593317747116089 61.0
0 0.865655124187469 62.0
240.65918219 259.088684082031 0.0
0 -0.054547905921936 1.0
0 -0.0890206098556519 2.0
0 -0.0206283330917358 3.0
0 -0.0262447595596313 4.0
0 0.00957238674163818 5.0
0 -0.066780686378479 6.0
0 -0.0845454931259155 7.0
0 -0.0310767889022827 8.0
0 0.009696364402771 9.0
0 -0.00963842868804932 10.0
0 0.0231483578681946 11.0
0 -0.00665724277496338 12.0
73.915328619 72.9887390136719 13.0
33.540645481 28.7082920074463 14.0
0 -0.0534883737564087 15.0
0 -0.0582455396652222 16.0
31.743220704 32.5483894348145 17.0
0 0.0184034109115601 18.0
0 -9.38177108764648e-05 19.0
0 -0.0265008211135864 20.0
0 -0.0121282339096069 21.0
0 -0.024293065071106 22.0
37.486363232 36.8148384094238 23.0
0 0.121931433677673 24.0
0 1.01988983154297 25.0
0 0.07335364818573 26.0
0 0.0633159875869751 27.0
0 -0.0633779764175415 28.0
0 -0.0791341066360474 29.0
0 -0.0547689199447632 30.0
0 -0.0488239526748657 31.0
0 -0.00224554538726807 32.0
0 0.123487114906311 33.0
0 -0.00634419918060303 34.0
0 -0.0839804410934448 35.0
0 0.070494532585144 36.0
0 -0.0719717741012573 37.0
0 0.0162423849105835 38.0
0 0.0276230573654175 39.0
0 0.132035613059998 40.0
0 -0.0518876314163208 41.0
0 0.00809800624847412 42.0
295.79157961 295.798034667969 43.0
0 -0.0395742654800415 44.0
0 -0.0340741872787476 45.0
0 -0.0110906362533569 46.0
0 -0.0153855085372925 47.0
0 0.0126675367355347 48.0
0 0.284409463405609 49.0
0.075284133411 0.0607155561447144 50.0
2.2420668996 0.871640086174011 51.0
0 -0.0170162916183472 52.0
0 -0.0485841035842896 53.0
0 0.0725013017654419 54.0
0 -0.032835841178894 55.0
0 -0.0420390367507935 56.0
0 -0.00698196887969971 57.0
0 0.00130140781402588 58.0
0 -0.0665851831436157 59.0
0 -0.0468138456344604 60.0
0 0.0160311460494995 61.0
0 0.828452229499817 62.0
236.21554762 253.556503295898 0.0
0 -0.076546311378479 1.0
0 0.0074080228805542 2.0
0 -0.0457297563552856 3.0
0 -0.0576914548873901 4.0
0 -0.0592836141586304 5.0
0 -0.0807687044143677 6.0
0 -0.0667787790298462 7.0
0 -0.0459083318710327 8.0
0 0.0562509298324585 9.0
0 -0.0153416395187378 10.0
0 0.0140643715858459 11.0
0 0.0766423940658569 12.0
67.704846618 66.8857955932617 13.0
28.673826264 32.5033912658691 14.0
0 0.0372766256332397 15.0
0 -0.0578709840774536 16.0
33.144127623 33.1289291381836 17.0
0 -0.0310894250869751 18.0
0 0.020281195640564 19.0
0 -0.011067271232605 20.0
0 -0.0109297037124634 21.0
0 -0.0252343416213989 22.0
40.277228746 39.5737419128418 23.0
0 -0.00810921192169189 24.0
0 0.662513017654419 25.0
0 0.0478090047836304 26.0
0 0.0607582330703735 27.0
0 0.0363682508468628 28.0
0 -0.0390063524246216 29.0
0 -0.0701388120651245 30.0
0 0.00729596614837646 31.0
0 -0.0148602724075317 32.0
0 -0.0178576707839966 33.0
0 -0.0123509168624878 34.0
0 -0.0798734426498413 35.0
0 0.0555559396743774 36.0
0 -0.00922238826751709 37.0
0 -0.0176516771316528 38.0
0 0.144646763801575 39.0
0 0.129957556724548 40.0
0 -0.0554260015487671 41.0
0 -0.0315033197402954 42.0
298.92320284 295.817321777344 43.0
0 -0.0147548913955688 44.0
0 0.0219703912734985 45.0
0 -0.0401891469955444 46.0
0 -0.0328959226608276 47.0
0 -0.0248466730117798 48.0
0 0.281561136245728 49.0
0.074537723492 0.0800930261611938 50.0
2.211995118 0.909788072109222 51.0
0 -0.0422741174697876 52.0
0 -0.0517288446426392 53.0
0 -0.00310957431793213 54.0
0 -0.125605940818787 55.0
0 -0.0126444101333618 56.0
0 -0.0206438302993774 57.0
0 -0.0297547578811646 58.0
0 -0.0596081018447876 59.0
0 -0.0188194513320923 60.0
0 -0.055705189704895 61.0
0 0.807969450950623 62.0
226.21907454 255.417266845703 0.0
0 -0.0561081171035767 1.0
0 0.0669814348220825 2.0
0 -0.0150300264358521 3.0
0 -0.0508135557174683 4.0
0 -0.00699007511138916 5.0
0 -0.141537308692932 6.0
0 -0.039375901222229 7.0
0 -0.0313007831573486 8.0
0 0.0287984609603882 9.0
0 -0.0213016271591187 10.0
0 -0.0245064496994019 11.0
0 -0.0597015619277954 12.0
65.1229249 63.9349937438965 13.0
33.023361695 32.4409713745117 14.0
0 -0.0895925760269165 15.0
0 -0.00695216655731201 16.0
38.797380035 38.0652046203613 17.0
0 -0.011313796043396 18.0
0 0.0353075265884399 19.0
0 -0.0108579397201538 20.0
0 -0.00908219814300537 21.0
0 -0.0414444208145142 22.0
37.532445479 36.8961868286133 23.0
0 -0.036725640296936 24.0
0 0.540397524833679 25.0
0 0.0182658433914185 26.0
0 0.0234708189964294 27.0
0 0.0501354932785034 28.0
0 -0.0142349004745483 29.0
0 -0.0905587673187256 30.0
0 -0.00210630893707275 31.0
0 -0.028448224067688 32.0
0 0.036587119102478 33.0
0 -0.014386773109436 34.0
0 0.0200387239456177 35.0
0 -0.00271594524383545 36.0
0 -0.0777150392532349 37.0
0 0.0922781229019165 38.0
0 0.029258131980896 39.0
0 0.0479949712753296 40.0
0 0.0923715829849243 41.0
0 -0.0423437356948853 42.0
307.42168115 296.881622314453 43.0
0 -0.00358641147613525 44.0
0 0.0629830360412598 45.0
0 0.00718462467193604 46.0
0 -0.0430749654769897 47.0
0 -0.0287495851516724 48.0
0 0.267666041851044 49.0
0.093473233093 0.0715728998184204 50.0
2.0775558148 0.891213655471802 51.0
0 -0.0618089437484741 52.0
0 -0.0523043870925903 53.0
0 -0.00953757762908936 54.0
0 -0.0309449434280396 55.0
0 -0.0245856046676636 56.0
0 -0.0485643148422241 57.0
0 -0.0468624830245972 58.0
0 -0.0730854272842407 59.0
0 -0.0194627046585083 60.0
0 0.0491088628768921 61.0
0 0.800379872322083 62.0
293.42562446 255.355529785156 0.0
0 -0.0647264719009399 1.0
0 0.029004693031311 2.0
0 0.00037539005279541 3.0
0 -0.0562406778335571 4.0
0 0.0203412771224976 5.0
0 -0.208202481269836 6.0
0 0.0320538282394409 7.0
0 -0.0247080326080322 8.0
0 -0.0346008539199829 9.0
0 -0.0480462312698364 10.0
0 0.000586271286010742 11.0
0 -0.0480152368545532 12.0
60.797147896 60.1030158996582 13.0
43.438183544 37.2148551940918 14.0
0 -0.0109554529190063 15.0
0 0.00641453266143799 16.0
31.033777536 30.6418743133545 17.0
0 0.0459533929824829 18.0
0 0.0674973726272583 19.0
0 -0.0272954702377319 20.0
0 -0.029096245765686 21.0
0 0.0172637701034546 22.0
40.679160118 39.012866973877 23.0
0 -0.0945836305618286 24.0
0 0.823029696941376 25.0
0 -0.0761138200759888 26.0
0 0.0245243310928345 27.0
0 -0.0315204858779907 28.0
0 -0.0716296434402466 29.0
0 -0.0903488397598267 30.0
0 -0.0790525674819946 31.0
0 0.0626739263534546 32.0
0 0.0625261068344116 33.0
0 -0.0195947885513306 34.0
0 -0.0522283315658569 35.0
0 -0.0205205678939819 36.0
0 -0.00582587718963623 37.0
0 0.019898533821106 38.0
0 0.0285693407058716 39.0
0 0.309553503990173 40.0
0 -0.0619230270385742 41.0
0 -0.0106092691421509 42.0
286.04620846 297.896514892578 43.0
0 0.0366247892379761 44.0
0 -0.0267823934555054 45.0
0 -0.0246051549911499 46.0
0 -0.0173994302749634 47.0
0 -0.0583893060684204 48.0
0 0.258488416671753 49.0
0.07243459247 0.0597195625305176 50.0
1.9127589685 0.902773380279541 51.0
0 -0.0100756883621216 52.0
0 -0.0345367193222046 53.0
0 -0.0230506658554077 54.0
0 -0.0140422582626343 55.0
0 -0.114238142967224 56.0
0 -0.0222998857498169 57.0
0 0.275910258293152 58.0
0 -0.0739918947219849 59.0
0 0.0591813325881958 60.0
0 -0.0479410886764526 61.0
0 0.790636897087097 62.0
224.05559464 257.896087646484 0.0
0 -0.0216423273086548 1.0
0 -0.000326752662658691 2.0
0 0.00253236293792725 3.0
0 -0.057134747505188 4.0
0 0.0168215036392212 5.0
0 0.0398811101913452 6.0
0 -0.033988356590271 7.0
0 -0.0347715616226196 8.0
0 0.0499156713485718 9.0
0 0.0123733282089233 10.0
0 -0.00901496410369873 11.0
0 0.0495980978012085 12.0
61.094894998 60.7585792541504 13.0
36.132673421 36.6842842102051 14.0
0 0.0179609060287476 15.0
0 0.0413199663162231 16.0
31.305123228 30.9325065612793 17.0
0 0.00740087032318115 18.0
0 -0.0472017526626587 19.0
0 0.031734824180603 20.0
0 -0.00211656093597412 21.0
0 0.0124243497848511 22.0
32.831461818 32.0583229064941 23.0
0 0.0588172674179077 24.0
0 0.481246948242188 25.0
0 -0.00786888599395752 26.0
0 0.0136239528656006 27.0
0 0.0152281522750854 28.0
0 -0.0319164991378784 29.0
0 -0.102293968200684 30.0
0 -0.0570257902145386 31.0
0 0.0257681608200073 32.0
0 0.0620483160018921 33.0
0 -0.0406469106674194 34.0
0 -0.0086435079574585 35.0
0 -0.0333026647567749 36.0
0 -0.0578590631484985 37.0
0 0.00872743129730225 38.0
0 0.0831764936447144 39.0
0 0.245616316795349 40.0
0 -0.0858930349349976 41.0
0 -0.0481706857681274 42.0
302.45417644 294.684722900391 43.0
0 -0.0296856164932251 44.0
0 -0.0486904382705688 45.0
0 0.00396335124969482 46.0
0 -0.0528815984725952 47.0
0 -0.0268417596817017 48.0
0 0.280493855476379 49.0
0.082952498376 0.0616761445999146 50.0
2.0574994956 0.865251779556274 51.0
0 -0.0448504686355591 52.0
0 -0.0649267435073853 53.0
0 -0.00687563419342041 54.0
0 -0.0554150342941284 55.0
0 -0.0651692152023315 56.0
0 0.0183542966842651 57.0
0 0.0178741216659546 58.0
0 -0.0301955938339233 59.0
0 -0.0178934335708618 60.0
0 -0.030115008354187 61.0
0 0.808634996414185 62.0
248.39687961 254.445114135742 0.0
0 -0.0325747728347778 1.0
0 -0.00382673740386963 2.0
0 0.0076373815536499 3.0
0 0.012787938117981 4.0
0 0.0608838796615601 5.0
0 0.0574191808700562 6.0
0 -0.0996280908584595 7.0
0 0.0275905132293701 8.0
0 -0.00607073307037354 9.0
0 -0.00213134288787842 10.0
0 -0.00249326229095459 11.0
0 -0.138620018959045 12.0
62.275924241 60.9503974914551 13.0
32.766061485 35.1396369934082 14.0
0 -0.0210238695144653 15.0
0 0.0125342607498169 16.0
28.213173378 27.8664646148682 17.0
0 -0.00181806087493896 18.0
0 0.0100175142288208 19.0
0 0.0115400552749634 20.0
0 0.0136836767196655 21.0
0 0.0010141134262085 22.0
35.840460596 35.0331573486328 23.0
0 0.0983394384384155 24.0
0 0.421905159950256 25.0
0 0.0626353025436401 26.0
0 0.038885235786438 27.0
0 -0.0619436502456665 28.0
0 -0.0588232278823853 29.0
0 -0.0592834949493408 30.0
0 -0.0594972372055054 31.0
0 -0.0932894945144653 32.0
0 -0.0201948881149292 33.0
0 0.00225245952606201 34.0
0 0.0120205879211426 35.0
0 0.0184022188186646 36.0
0 -0.0157254934310913 37.0
0 0.0170975923538208 38.0
0 0.00898110866546631 39.0
0 0.074461817741394 40.0
0 -0.101208090782166 41.0
0 -0.0429245233535767 42.0
294.44592481 296.254943847656 43.0
0 0.0275958776473999 44.0
0 -0.0347167253494263 45.0
0 -0.0186587572097778 46.0
0 -0.0635024309158325 47.0
0 0.0173152685165405 48.0
0 0.330894380807877 49.0
0.10029351321 0.0698386430740356 50.0
2.1883315265 0.977603256702423 51.0
0 -0.0656915903091431 52.0
0 -0.0338903665542603 53.0
0 -0.0178107023239136 54.0
0 -0.0523542165756226 55.0
0 -0.0152970552444458 56.0
0 -0.0638836622238159 57.0
0 -0.0486413240432739 58.0
0 -0.0787891149520874 59.0
0 -0.00937306880950928 60.0
0 0.0109654664993286 61.0
0 0.861313581466675 62.0
265.50025976 255.336669921875 0.0
0 -0.0609928369522095 1.0
0 -0.162433505058289 2.0
0 0.0384241342544556 3.0
0 -0.0947862863540649 4.0
0 -0.372810959815979 5.0
0 0.220996737480164 6.0
0 0.160595774650574 7.0
0 -0.0364127159118652 8.0
0 0.0671828985214233 9.0
0 0.0109658241271973 10.0
0 7.45058059692383e-05 11.0
0 -0.0715786218643188 12.0
68.813804404 68.0304107666016 13.0
42.034308852 30.877124786377 14.0
0 0.0436705350875854 15.0
0 -0.0354820489883423 16.0
42.130653 43.3699226379395 17.0
0 0.0133160352706909 18.0
0 0.00761020183563232 19.0
0 -0.012668251991272 20.0
0 -0.0207372903823853 21.0
0 0.0193213224411011 22.0
31.840193682 31.0300636291504 23.0
0 -0.025065541267395 24.0
0 0.551169037818909 25.0
0 0.0305331945419312 26.0
0 -0.0329915285110474 27.0
0 -0.0586978197097778 28.0
0 -0.0477210283279419 29.0
0 -0.0477374792098999 30.0
0 -0.0464414358139038 31.0
0 0.0188087224960327 32.0
0 0.110212922096252 33.0
0 -0.0779949426651001 34.0
0 -0.0843782424926758 35.0
0 -0.158344864845276 36.0
0 -0.0142849683761597 37.0
0 0.00718009471893311 38.0
0 -0.0331417322158813 39.0
0 0.0784791707992554 40.0
0 -0.0583051443099976 41.0
0 -0.00990355014801025 42.0
306.81070046 295.055358886719 43.0
0 0.0125855207443237 44.0
0 0.0789096355438232 45.0
0 0.00964057445526123 46.0
0 -0.0895897150039673 47.0
0 0.0734666585922241 48.0
0 0.272735387086868 49.0
0.070712532429 0.0835329294204712 50.0
1.5085296887 1.00233137607574 51.0
0 -0.00205481052398682 52.0
0 -0.0110980272293091 53.0
0 0.110163807868958 54.0
0 -0.099521279335022 55.0
0 0.0544482469558716 56.0
0 -0.0301061868667603 57.0
0 0.0245627164840698 58.0
0 -0.0438994169235229 59.0
0 0.0717674493789673 60.0
0 -0.0722848176956177 61.0
0 0.877043724060059 62.0
241.58928825 259.600616455078 0.0
0 -0.0977026224136353 1.0
0 0.0101748704910278 2.0
0 -0.0122357606887817 3.0
0 -0.0501590967178345 4.0
0 -0.0276936292648315 5.0
0 0.0712417364120483 6.0
0 3.01599502563477e-05 7.0
0 -0.0646545886993408 8.0
0 0.121423125267029 9.0
0 0.0101728439331055 10.0
0 -0.049080491065979 11.0
0 0.0308355093002319 12.0
53.502340535 54.1101989746094 13.0
39.818146014 39.1353416442871 14.0
0 -0.0619016885757446 15.0
0 0.0302141904830933 16.0
27.810410691 27.5999298095703 17.0
0 0.0291451215744019 18.0
0 -0.00470292568206787 19.0
0 -0.0572842359542847 20.0
0 -0.00173985958099365 21.0
0 0.00284135341644287 22.0
40.870808402 40.2001495361328 23.0
0 -0.133605122566223 24.0
0 0.642868757247925 25.0
0 0.00667321681976318 26.0
0 -0.095156192779541 27.0
0 -0.0939046144485474 28.0
0 -0.0657557249069214 29.0
0 -0.0888413190841675 30.0
0 -0.0569690465927124 31.0
0 -0.0298711061477661 32.0
0 0.0720328092575073 33.0
0 -0.0637367963790894 34.0
0 0.0544381141662598 35.0
0 0.0287586450576782 36.0
0 -0.0408247709274292 37.0
0 -0.0491386651992798 38.0
0 0.0562249422073364 39.0
0 0.0832594633102417 40.0
0 -0.0483094453811646 41.0
0 -0.063088059425354 42.0
289.7219725 294.246124267578 43.0
0 -0.044599175453186 44.0
0 0.0586347579956055 45.0
0 -0.0328625440597534 46.0
0 -0.016859769821167 47.0
0 0.0385383367538452 48.0
1.023356564 0.26831904053688 49.0
0.083489362205 0.0664757490158081 50.0
1.9078581876 0.85809338092804 51.0
0 -0.0468641519546509 52.0
0 -0.0432931184768677 53.0
0 0.0524417161941528 54.0
0 -0.0762313604354858 55.0
0 0.0177751779556274 56.0
0 -0.0200985670089722 57.0
0 -0.0557543039321899 58.0
0 -0.0383270978927612 59.0
0 -0.0182691812515259 60.0
0 -0.0317448377609253 61.0
0 0.804116606712341 62.0
282.60905189 258.733337402344 0.0
0 -0.0371166467666626 1.0
0 0.0171617269515991 2.0
0 -0.0512679815292358 3.0
0 -0.0559109449386597 4.0
0 -0.123669743537903 5.0
0 0.0473045110702515 6.0
0 -0.109766602516174 7.0
0 -0.0178366899490356 8.0
0 -0.0688463449478149 9.0
0 -0.0129495859146118 10.0
0 -0.0307922959327698 11.0
0 0.036033034324646 12.0
73.98314465 71.7630081176758 13.0
38.991697736 27.1761341094971 14.0
0 -0.0393692255020142 15.0
0 -0.0882459878921509 16.0
37.868908354 37.4561233520508 17.0
0 0.0301131010055542 18.0
0 -0.0185157060623169 19.0
0 0.0196019411087036 20.0
0 0.0413650274276733 21.0
0 -0.0637255907058716 22.0
33.287120718 32.4451751708984 23.0
0 -0.153976082801819 24.0
0 1.32706892490387 25.0
0 0.0331496000289917 26.0
0 -0.0364935398101807 27.0
0 0.21054208278656 28.0
0 -0.0480278730392456 29.0
0 -0.0799020528793335 30.0
0 -0.0185815095901489 31.0
0 0.036251425743103 32.0
0 0.0010758638381958 33.0
0 -0.0272473096847534 34.0
0 -0.0206586122512817 35.0
0 -0.000187039375305176 36.0
0 -0.0459302663803101 37.0
0 0.0126618146896362 38.0
0 0.0686556100845337 39.0
0 0.00619542598724365 40.0
0 -0.00715827941894531 41.0
0 -0.0168472528457642 42.0
304.56667036 297.910217285156 43.0
0 -0.058230996131897 44.0
0 0.013466477394104 45.0
0 -0.0611056089401245 46.0
0 -0.0231558084487915 47.0
0 -0.0190583467483521 48.0
0 0.303844422101974 49.0
0.08638238824 0.0679212808609009 50.0
1.6765776913 0.905104041099548 51.0
0 -0.0720611810684204 52.0
0 -0.0814701318740845 53.0
0 0.025849461555481 54.0
0 -0.0756078958511353 55.0
0 -0.137695670127869 56.0
0 0.0372365713119507 57.0
0 0.00339615345001221 58.0
0 -0.064552903175354 59.0
0 -0.100056052207947 60.0
0 -0.00962769985198975 61.0
0 0.893907427787781 62.0
223.40470918 256.617767333984 0.0
0 -0.0890289545059204 1.0
0 -0.00822269916534424 2.0
0 0.0101979970932007 3.0
0 -0.000756382942199707 4.0
0 -0.0172246694564819 5.0
0 -0.0107909440994263 6.0
0 0.009604811668396 7.0
0 0.0931150913238525 8.0
0 0.0740026235580444 9.0
0 0.0267184972763062 10.0
0 -0.0547048449516296 11.0
0 -0.13462221622467 12.0
71.541785175 69.3212509155273 13.0
11.349998831 28.3489227294922 14.0
0 -0.0696064233779907 15.0
0 -0.0494745969772339 16.0
32.923070611 34.1253204345703 17.0
0 0.00265538692474365 18.0
0 0.0227862596511841 19.0
0 0.0645841360092163 20.0
0 0.0873874425888062 21.0
0 -0.0714150667190552 22.0
44.093301122 43.4825706481934 23.0
0 -0.154143452644348 24.0
0 0.661305785179138 25.0
0 -0.0142303705215454 26.0
0 0.0286577939987183 27.0
0 -0.0433346033096313 28.0
0 -0.0537158250808716 29.0
0 0.0172766447067261 30.0
0 -0.053921103477478 31.0
0 0.00957095623016357 32.0
0 0.0175067186355591 33.0
0 -0.0920134782791138 34.0
0 -0.0405842065811157 35.0
0 -0.0050584077835083 36.0
0 -0.0383919477462769 37.0
0 -0.0136560201644897 38.0
0 0.0367358922958374 39.0
0 0.0978368520736694 40.0
0 -0.0630460977554321 41.0
0 -0.0845485925674438 42.0
270.51918219 296.812713623047 43.0
0 -0.0537310838699341 44.0
0 0.115085601806641 45.0
0 -0.0342346429824829 46.0
0 -0.0678979158401489 47.0
0 0.0408438444137573 48.0
1.3709660702 0.218768417835236 49.0
0.06752003309 0.0994044542312622 50.0
1.5543782603 0.941400229930878 51.0
0 -0.0716406106948853 52.0
0 0.0497459173202515 53.0
0 -0.00631225109100342 54.0
0 -0.0956300497055054 55.0
0 0.0122283697128296 56.0
0 -0.0356837511062622 57.0
0 -0.0592585802078247 58.0
0 -0.0392323732376099 59.0
0 0.132662177085876 60.0
0 0.107094407081604 61.0
0 0.977811574935913 62.0
285.48498988 256.159942626953 0.0
0 -0.0294910669326782 1.0
0 -0.0450071096420288 2.0
0 -0.00723707675933838 3.0
0 -0.0488137006759644 4.0
0 -0.108409523963928 5.0
0 0.0619386434555054 6.0
0 0.023074746131897 7.0
0 -0.00156879425048828 8.0
0 0.0989965200424194 9.0
0 -0.00689780712127686 10.0
0 0.0416058897972107 11.0
0 -0.00669729709625244 12.0
66.04390803 65.6830825805664 13.0
46.092653103 33.4492835998535 14.0
0 -0.0600639581680298 15.0
0 -0.0281354188919067 16.0
27.74666352 28.2829341888428 17.0
0 0.0228484869003296 18.0
0 -0.0270460844039917 19.0
0 0.0590769052505493 20.0
0 -0.01799476146698 21.0
0 0.0299562215805054 22.0
33.760387342 33.0034484863281 23.0
0 0.0842803716659546 24.0
0 0.452398777008057 25.0
0 0.115702986717224 26.0
0 0.096631646156311 27.0
0 -0.0631052255630493 28.0
0 -0.0616663694381714 29.0
0 -0.0625463724136353 30.0
0 -0.0335010290145874 31.0
0 -0.0261040925979614 32.0
0 -0.0919977426528931 33.0
0 3.68356704711914e-05 34.0
0 -0.0117522478103638 35.0
0 0.100855946540833 36.0
0 -0.0570508241653442 37.0
0 0.0174189805984497 38.0
0 0.00494396686553955 39.0
0 0.018757700920105 40.0
0 -0.0263090133666992 41.0
0 -0.0368295907974243 42.0
296.2529259 296.958953857422 43.0
0 -0.00935518741607666 44.0
0 0.0342046022415161 45.0
0 -0.0268286466598511 46.0
0 -0.0551273822784424 47.0
0 0.00181138515472412 48.0
0.29221340826 0.298080384731293 49.0
0.075475703301 0.0680726766586304 50.0
2.2286424598 0.943123817443848 51.0
0 -0.0632010698318481 52.0
0 -0.0620070695877075 53.0
0 -0.0444608926773071 54.0
0 -0.0447014570236206 55.0
0 -0.0629581212997437 56.0
0 0.0312041044235229 57.0
0 -0.0662077665328979 58.0
0 -0.0746983289718628 59.0
0 0.0573800802230835 60.0
0 -0.0671826601028442 61.0
0 0.845214009284973 62.0
271.00144083 249.061737060547 0.0
0 -0.0642076730728149 1.0
0 0.0235494375228882 2.0
0 0.00942933559417725 3.0
0 -0.0544000864028931 4.0
0 -0.0397492647171021 5.0
0 0.0701950788497925 6.0
0 -0.0334717035293579 7.0
0 -0.095439076423645 8.0
0 0.037676215171814 9.0
0 -0.00548136234283447 10.0
0 -0.0482193231582642 11.0
0 -0.014839768409729 12.0
54.786238493 53.6807289123535 13.0
40.357366933 39.0255126953125 14.0
0 -0.00430166721343994 15.0
0 -0.0103996992111206 16.0
29.386375666 29.1307544708252 17.0
0 0.0126756429672241 18.0
0 0.00367653369903564 19.0
0 -0.0355533361434937 20.0
0 -0.0449725389480591 21.0
0 -0.0349317789077759 22.0
26.271691248 25.5665645599365 23.0
0 0.179325938224792 24.0
13.919405726 0.832013964653015 25.0
0 0.0258258581161499 26.0
0 -0.0460497140884399 27.0
0 -0.058536171913147 28.0
0 -0.0274349451065063 29.0
0 -0.0638890266418457 30.0
0 -0.019595742225647 31.0
0 0.0380333662033081 32.0
0 0.0845605134963989 33.0
0 -0.0196579694747925 34.0
0 -0.0256506204605103 35.0
0 0.0239239931106567 36.0
0 0.0103789567947388 37.0
0 0.0180541276931763 38.0
0 0.0362187623977661 39.0
0 -0.0338727235794067 40.0
0 -0.00886011123657227 41.0
0 -0.0212455987930298 42.0
292.036164 296.876953125 43.0
0 -0.00131046772003174 44.0
0 -0.0412696599960327 45.0
0 -0.0445138216018677 46.0
0 0.0549753904342651 47.0
0 0.0498870611190796 48.0
0.27722198621 0.344258338212967 49.0
0.073426801253 0.102581381797791 50.0
1.8559008831 0.948763370513916 51.0
0 -0.0614544153213501 52.0
0 -0.0544568300247192 53.0
0 0.0931800603866577 54.0
0 -0.0834428071975708 55.0
0 -0.0772501230239868 56.0
0 -0.0080493688583374 57.0
0 -0.0904463529586792 58.0
0 -0.0464152097702026 59.0
0 0.205645680427551 60.0
0 -0.0187956094741821 61.0
0 0.829623222351074 62.0
214.52766973 254.648971557617 0.0
0 0.0513347387313843 1.0
0 -0.0535365343093872 2.0
0 0.01079261302948 3.0
0 -0.0196384191513062 4.0
0 -0.063578724861145 5.0
0 0.0431379079818726 6.0
0 0.0600947141647339 7.0
0 -0.00608313083648682 8.0
0 -0.10471785068512 9.0
0 -0.0159119367599487 10.0
0 0.0199272632598877 11.0
0 -0.0689769983291626 12.0
57.515316571 57.2310104370117 13.0
40.714787251 40.0424575805664 14.0
0 0.00648200511932373 15.0
0 -0.0637580156326294 16.0
38.779347539 37.2370071411133 17.0
0 0.0516084432601929 18.0
0 0.0917152166366577 19.0
0 -0.0395418405532837 20.0
0 0.000779986381530762 21.0
0 0.00298774242401123 22.0
30.724533826 30.2248630523682 23.0
0 -0.0413724184036255 24.0
0 0.623829483985901 25.0
0 0.0832861661911011 26.0
0 -0.0536652803421021 27.0
0 0.0130156278610229 28.0
0 -0.0482097864151001 29.0
0 -0.0802074670791626 30.0
0 -0.032056450843811 31.0
0 -0.0190926790237427 32.0
0 -0.000489115715026855 33.0
0 -0.127699255943298 34.0
0 -0.0841783285140991 35.0
0 -0.0642133951187134 36.0
0 -0.0822025537490845 37.0
0 -0.0491646528244019 38.0
0 0.153946042060852 39.0
0 0.138426184654236 40.0
0 -0.0707492828369141 41.0
0 0.0484346151351929 42.0
301.68348717 295.433349609375 43.0
0 0.00805413722991943 44.0
0 0.0685035586357117 45.0
0 0.00410878658294678 46.0
0 -0.00982356071472168 47.0
0 0.0114287137985229 48.0
1.3204330239 0.282949090003967 49.0
0.099268470486 0.0902911424636841 50.0
2.0982922507 1.01086091995239 51.0
0 -0.0597847700119019 52.0
0 -0.0683251619338989 53.0
0 -0.0407370328903198 54.0
0 -0.0814801454544067 55.0
0 -0.0687655210494995 56.0
0 -0.0281192064285278 57.0
0 -0.033197283744812 58.0
0 0.00433862209320068 59.0
0 -0.0387860536575317 60.0
0 0.0639013051986694 61.0
0 0.814596652984619 62.0
228.63026125 255.839874267578 0.0
0 -0.0529066324234009 1.0
0 -0.00933659076690674 2.0
0 -0.00806629657745361 3.0
0 -0.0214904546737671 4.0
0 0.123836159706116 5.0
0 0.0998948812484741 6.0
0 -0.0933130979537964 7.0
0 -0.0426355600357056 8.0
0 0.0219124555587769 9.0
0 -0.0181666612625122 10.0
0 0.0405391454696655 11.0
0 0.105967879295349 12.0
57.945354588 57.3722076416016 13.0
23.781682019 37.3770294189453 14.0
0 -0.0167738199234009 15.0
0 0.0521553754806519 16.0
35.802414904 35.5406532287598 17.0
0 -0.00446403026580811 18.0
0 -0.0360103845596313 19.0
0 0.00349628925323486 20.0
0 0.0327180624008179 21.0
0 -0.0134741067886353 22.0
31.770109286 30.5392284393311 23.0
0 -0.0113726854324341 24.0
0 0.720303595066071 25.0
0 0.105019450187683 26.0
0 0.0242787599563599 27.0
0 0.0297583341598511 28.0
0 -0.076250433921814 29.0
0 -0.0344057083129883 30.0
0 -0.00603663921356201 31.0
0 -0.0109010934829712 32.0
0 0.0438569784164429 33.0
0 -0.0101262331008911 34.0
0 0.00778257846832275 35.0
0 0.0202330350875854 36.0
0 -0.0982712507247925 37.0
0 0.00442707538604736 38.0
0 0.0749481916427612 39.0
0 0.126185774803162 40.0
0 0.0138258934020996 41.0
0 0.00572621822357178 42.0
292.61727791 294.026641845703 43.0
0 0.0184320211410522 44.0
0 0.0845524668693542 45.0
0 0.00376689434051514 46.0
0 -0.0314854383468628 47.0
0 0.0520848035812378 48.0
0 0.273432314395905 49.0
0.094657686223 0.0847986936569214 50.0
1.4716266321 0.980053722858429 51.0
0 -0.0492278337478638 52.0
0 -0.033161997795105 53.0
0 0.0554291009902954 54.0
0 -0.0658317804336548 55.0
0 0.0156046152114868 56.0
0 -0.0333946943283081 57.0
0 -0.0354644060134888 58.0
0 0.0186008214950562 59.0
0 0.12437379360199 60.0
0 -0.034326434135437 61.0
0 0.881317496299744 62.0
267.86063716 260.925109863281 0.0
0 -0.0605455636978149 1.0
0 -0.0351301431655884 2.0
0 0.00558674335479736 3.0
0 -0.0402499437332153 4.0
0 0.0470479726791382 5.0
0 -0.128464818000793 6.0
0 0.0604552030563354 7.0
0 -0.0704224109649658 8.0
0 -0.00372040271759033 9.0
0 -0.0252903699874878 10.0
0 -0.0456522703170776 11.0
0 -0.257753014564514 12.0
49.680704203 48.7779083251953 13.0
31.138162122 43.2778739929199 14.0
0 -0.100030541419983 15.0
0 0.0299695730209351 16.0
38.705343062 37.7827377319336 17.0
0 -0.0553570985794067 18.0
0 -0.0042957067489624 19.0
0 0.000699877738952637 20.0
0 -0.0288692712783813 21.0
0 -0.00275528430938721 22.0
23.907913078 23.4816951751709 23.0
0 0.194706320762634 24.0
0 3.53897190093994 25.0
0 0.0212396383285522 26.0
0 0.016833484172821 27.0
0 0.041953444480896 28.0
0 -0.0415815114974976 29.0
0 -0.0808509588241577 30.0
0 0.00311839580535889 31.0
0 -0.0381094217300415 32.0
0 -0.0932848453521729 33.0
0 -0.0230802297592163 34.0
0 -0.105214476585388 35.0
0 0.00957095623016357 36.0
0 -0.0549794435501099 37.0
0 0.0230604410171509 38.0
0 -0.0203710794448853 39.0
0 -0.0673319101333618 40.0
0 -0.002005934715271 41.0
0 -0.108960747718811 42.0
303.39003554 295.25048828125 43.0
0 0.00737226009368896 44.0
0 0.0845541954040527 45.0
0 -0.0858129262924194 46.0
0 -0.0882500410079956 47.0
0 0.0824621915817261 48.0
0 0.246682465076447 49.0
0.080973326939 0.0825983285903931 50.0
1.823735673 1.01029014587402 51.0
0 -0.0132089853286743 52.0
0 -0.049324631690979 53.0
0 0.0639020204544067 54.0
0 -0.0634866952896118 55.0
0 -0.0308054685592651 56.0
0 -0.0298582315444946 57.0
0 -0.0689135789871216 58.0
0 -0.0748716592788696 59.0
0 0.132591605186462 60.0
0 0.0327955484390259 61.0
0 0.82786500453949 62.0
233.66574661 256.156921386719 0.0
0 -0.0587431192398071 1.0
0 -0.0482767820358276 2.0
0 0.0774670839309692 3.0
0 -0.0251132249832153 4.0
0 -0.0660449266433716 5.0
0 -0.0364853143692017 6.0
0 -0.0315786600112915 7.0
0 -0.0360978245735168 8.0
0 0.0306819677352905 9.0
0 0.0224889516830444 10.0
0 -0.0501874685287476 11.0
0 -0.0333925485610962 12.0
55.318754283 54.0764389038086 13.0
53.099984013 44.955192565918 14.0
0 -0.0216671228408813 15.0
0 0.0300806760787964 16.0
39.810743542 37.9108200073242 17.0
0 -0.00981080532073975 18.0
0 0.0139478445053101 19.0
0 -0.0116857290267944 20.0
0 -0.0161274671554565 21.0
0 0.0106581449508667 22.0
31.926482239 30.4712810516357 23.0
0 0.0310734510421753 24.0
0 0.751103639602661 25.0
0 -0.0339099168777466 26.0
0 0.0650275945663452 27.0
0 -0.0233503580093384 28.0
0 -0.0416535139083862 29.0
0 0.0434702634811401 30.0
0 -0.00194418430328369 31.0
0 -0.0100594758987427 32.0
0 0.130371451377869 33.0
0 -0.0322145223617554 34.0
0 0.067379355430603 35.0
0 -0.0712858438491821 36.0
0 -0.106280207633972 37.0
0 -0.0648492574691772 38.0
0 0.00875174999237061 39.0
0 -0.0963526964187622 40.0
0 0.0498431921005249 41.0
0 0.0276216268539429 42.0
293.57096095 296.041870117188 43.0
0 -0.0485349893569946 44.0
0 -0.00448954105377197 45.0
0 -0.0240951776504517 46.0
0 -0.03205406665802 47.0
0 -0.039467453956604 48.0
0 0.268659055233002 49.0
0.078673569841 0.21780002117157 50.0
2.0643049163 0.903701186180115 51.0
0 -0.0543602705001831 52.0
0 0.0116780996322632 53.0
0 0.0156878232955933 54.0
0 -0.0417791604995728 55.0
0 0.0291470289230347 56.0
0 -0.0284110307693481 57.0
0 -0.0361150503158569 58.0
0 -0.0630747079849243 59.0
0 -0.0945810079574585 60.0
0 -0.0354987382888794 61.0
0 0.838778257369995 62.0
279.89236076 253.570373535156 0.0
0 -0.0400701761245728 1.0
0 -0.00590097904205322 2.0
0 -0.0166796445846558 3.0
0 -0.0486382246017456 4.0
0 -0.0265864133834839 5.0
0 -0.0624681711196899 6.0
0 -0.0650788545608521 7.0
0 -0.0415490865707397 8.0
0 0.0597242116928101 9.0
0 -0.0551666021347046 10.0
0 0.00326621532440186 11.0
0 -0.159043908119202 12.0
47.464357266 47.5104141235352 13.0
55.771398927 47.8380508422852 14.0
0 -0.131543755531311 15.0
0 0.0343540906906128 16.0
30.956313966 30.2938365936279 17.0
0 -0.00938951969146729 18.0
0 0.055115818977356 19.0
0 0.0266762971878052 20.0
0 0.00482332706451416 21.0
0 0.044669508934021 22.0
33.074831409 33.2521553039551 23.0
0 0.0309022665023804 24.0
2.7247727146 1.08352446556091 25.0
0 0.0702346563339233 26.0
0 0.0816198587417603 27.0
0 0.0999282598495483 28.0
0 -0.130863785743713 29.0
0 0.139600396156311 30.0
0 -0.0300220251083374 31.0
0 -0.0418959856033325 32.0
0 -0.103193163871765 33.0
0 -0.0237523317337036 34.0
0 -0.00471556186676025 35.0
0 0.109123826026917 36.0
0 -0.0534106492996216 37.0
0 0.14292585849762 38.0
0 -0.0626903772354126 39.0
0 -0.0538660287857056 40.0
0 0.0297378301620483 41.0
0 -0.0783859491348267 42.0
295.72236269 293.007476806641 43.0
0 -0.0540167093276978 44.0
0 0.0712399482727051 45.0
0 -0.0413528680801392 46.0
0 -0.0928703546524048 47.0
0 -0.0553890466690063 48.0
0 0.263730406761169 49.0
0.079396926519 0.252085447311401 50.0
1.6879972655 0.946596622467041 51.0
0 -0.0395098924636841 52.0
0 -0.0705372095108032 53.0
0 0.0553468465805054 54.0
0 -0.0772813558578491 55.0
0 -0.049528956413269 56.0
0 -0.0414415597915649 57.0
0 -0.03264319896698 58.0
0 0.0104659795761108 59.0
0 -0.0517534017562866 60.0
0 -0.0219379663467407 61.0
0 0.787437677383423 62.0
268.53906409 257.299987792969 0.0
0 -0.124348759651184 1.0
0 -0.109580397605896 2.0
0 -0.0302811861038208 3.0
0 -0.0297642946243286 4.0
0 -0.0779863595962524 5.0
0 0.0817855596542358 6.0
0 0.00174319744110107 7.0
0 0.0056537389755249 8.0
0 -0.772751450538635 9.0
0 0.0387102365493774 10.0
0 -0.0218034982681274 11.0
0 -0.118251442909241 12.0
62.190257203 62.0721015930176 13.0
35.618354132 32.1883010864258 14.0
0 0.00613677501678467 15.0
0 -0.0542408227920532 16.0
28.347314208 28.6343593597412 17.0
0 -0.0142165422439575 18.0
0 0.0293396711349487 19.0
0 -0.0391920804977417 20.0
0 -0.0707184076309204 21.0
0 0.00624406337738037 22.0
30.200406904 29.5439052581787 23.0
0 0.0855110883712769 24.0
0 0.462913453578949 25.0
0 0.0205479860305786 26.0
0 0.105125784873962 27.0
0 -0.0627442598342896 28.0
0 -0.098326563835144 29.0
0 -0.0852223634719849 30.0
0 -0.00391662120819092 31.0
0 -0.0162268877029419 32.0
0 -0.0670880079269409 33.0
0 -0.044340968132019 34.0
0 0.0695815086364746 35.0
0 0.0457874536514282 36.0
0 -0.0114513635635376 37.0
0 0.0568121671676636 38.0
0 0.00766360759735107 39.0
0 -0.0282250642776489 40.0
0 -0.00177681446075439 41.0
0 0.0288270711898804 42.0
284.76938136 295.305938720703 43.0
0 -0.0757538080215454 44.0
0 0.0883030891418457 45.0
0 -0.00308692455291748 46.0
0 -0.0408378839492798 47.0
0 0.0275343656539917 48.0
0 0.360277056694031 49.0
0.06767479772 0.0474652051925659 50.0
2.277999865 1.01962113380432 51.0
0 0.00343644618988037 52.0
0 -0.0306965112686157 53.0
0 0.0250695943832397 54.0
0 -0.158999085426331 55.0
0 -0.0850969552993774 56.0
0 0.0801883935928345 57.0
0 -0.0989753007888794 58.0
0 -0.0607441663742065 59.0
0 -0.00807654857635498 60.0
0 -0.069623589515686 61.0
0 0.802311062812805 62.0
254.01407252 251.608657836914 0.0
0 -0.0234161615371704 1.0
0 -0.0503495931625366 2.0
0 -0.0255042314529419 3.0
0 0.0486916303634644 4.0
0 0.109289526939392 5.0
0 0.0702918767929077 6.0
0 -0.0377837419509888 7.0
0 0.0302978754043579 8.0
0 -0.00949442386627197 9.0
0 -0.0852166414260864 10.0
0 -0.0522615909576416 11.0
0 -0.0687834024429321 12.0
60.753999644 60.3507461547852 13.0
36.027191319 36.2740211486816 14.0
0 -0.0271793603897095 15.0
0 -0.0295294523239136 16.0
23.804559827 21.9303417205811 17.0
0 -0.0220621824264526 18.0
0 -0.0378469228744507 19.0
0 -0.0284782648086548 20.0
0 0.0643247365951538 21.0
0 -0.0108472108840942 22.0
36.208347868 35.6525573730469 23.0
0 0.101896166801453 24.0
0 0.434255957603455 25.0
0 0.00915610790252686 26.0
0 0.0348057150840759 27.0
0 -0.0203472375869751 28.0
0 -0.0551444292068481 29.0
0 -0.0169093608856201 30.0
0 -0.0205608606338501 31.0
0 -0.00810110569000244 32.0
0 -0.0724459886550903 33.0
0 0.00248253345489502 34.0
0 -0.0248063802719116 35.0
0 -0.06556236743927 36.0
0 -0.0256127119064331 37.0
0 0.0403715372085571 38.0
0 0.0554555654525757 39.0
0 0.131400465965271 40.0
0 -0.0694676637649536 41.0
0 -0.0671402215957642 42.0
286.2246376 296.504791259766 43.0
0 0.0114444494247437 44.0
0 0.119756519794464 45.0
0 -0.0477911233901978 46.0
0 -0.0233012437820435 47.0
0 -0.0122817754745483 48.0
0 0.259938836097717 49.0
0.064525577131 0.0382503271102905 50.0
1.7539313445 0.915170669555664 51.0
0 -0.110796093940735 52.0
0 -0.0260850191116333 53.0
0 -0.00500428676605225 54.0
0 -0.0526860952377319 55.0
0 -0.0383301973342896 56.0
0 -0.0294324159622192 57.0
0 -0.0419348478317261 58.0
0 -0.0516766309738159 59.0
0 0.0113600492477417 60.0
0 0.0409063100814819 61.0
0 0.886010527610779 62.0
241.84263471 259.670013427734 0.0
0 0.0360068082809448 1.0
0 0.00984132289886475 2.0
0 0.00236523151397705 3.0
0 -0.0385679006576538 4.0
0 -0.0456477403640747 5.0
0 -0.0915695428848267 6.0
0 -0.0733529329299927 7.0
0 0.0302978754043579 8.0
0 -0.0295654535293579 9.0
0 0.0445271730422974 10.0
0 -0.0020759105682373 11.0
0 0.0931543111801147 12.0
78.429406507 76.9459228515625 13.0
32.166202418 21.1166229248047 14.0
0 -0.02649986743927 15.0
0 0.00165927410125732 16.0
27.245902491 27.2321815490723 17.0
0 0.02521812915802 18.0
0 -0.00594437122344971 19.0
0 -0.0203393697738647 20.0
0 -0.0347009897232056 21.0
0 -0.00208723545074463 22.0
33.283640356 32.9086761474609 23.0
0 0.0304383039474487 24.0
0 1.19772291183472 25.0
0 -0.0192900896072388 26.0
0 0.0571655035018921 27.0
0 0.189696669578552 28.0
0 -0.135779976844788 29.0
0 0.0402694940567017 30.0
0 -0.0539368391036987 31.0
0 0.021023154258728 32.0
0 -0.0628467798233032 33.0
0 -0.087746262550354 34.0
0 0.00615274906158447 35.0
0 -0.0154465436935425 36.0
0 0.106435656547546 37.0
0 0.0469094514846802 38.0
0 -0.0628383159637451 39.0
0 0.0929330587387085 40.0
0 0.0264734029769897 41.0
0 0.0233100652694702 42.0
300.14380934 294.340087890625 43.0
0 0.0377429723739624 44.0
0 0.091039776802063 45.0
0 0.000812172889709473 46.0
0 -0.0470019578933716 47.0
0 0.00551211833953857 48.0
0 0.34469211101532 49.0
0.092362456615 0.0593498945236206 50.0
2.2606329454 0.977299451828003 51.0
0 -0.029488205909729 52.0
0 0.00123131275177002 53.0
0 -0.0220834016799927 54.0
0 0.0171128511428833 55.0
0 -0.0246928930282593 56.0
0 -0.0423847436904907 57.0
0 -0.0513325929641724 58.0
0 0.00462186336517334 59.0
0 -0.00886952877044678 60.0
0 0.00833046436309814 61.0
0 0.844288110733032 62.0
234.03854131 253.922622680664 0.0
0 -0.0268203020095825 1.0
0 -0.0407637357711792 2.0
0 -0.0478214025497437 3.0
0 0.000255465507507324 4.0
0 -0.00256931781768799 5.0
0 -0.136185765266418 6.0
0 -0.118770241737366 7.0
0 -0.0154273509979248 8.0
0 -0.0368505716323853 9.0
0 -0.0595039129257202 10.0
0 0.00631159543991089 11.0
0 -0.115797638893127 12.0
76.838364452 73.633186340332 13.0
9.1102989106 24.3610744476318 14.0
0 0.0551031827926636 15.0
0 0.0367764234542847 16.0
35.053977073 34.7468757629395 17.0
0 -0.0296884775161743 18.0
0 0.0148493051528931 19.0
0 -0.0226575136184692 20.0
0 -0.0289512872695923 21.0
0 -0.0491009950637817 22.0
37.34908796 36.4921493530273 23.0
0 -0.0144020318984985 24.0
0 1.95737779140472 25.0
0 -0.0115610361099243 26.0
0 0.0755628347396851 27.0
0 -0.00656473636627197 28.0
0 -0.039298415184021 29.0
0 -0.0602337121963501 30.0
0 -0.0120290517807007 31.0
0 0.0775092840194702 32.0
0 0.146530985832214 33.0
0 -0.0164896249771118 34.0
0 -0.0492788553237915 35.0
0 0.114662289619446 36.0
0 -0.0560799837112427 37.0
0 -0.0206786394119263 38.0
0 0.101119875907898 39.0
0 0.141367793083191 40.0
0 -0.026584267616272 41.0
0 -0.000876069068908691 42.0
303.81540708 297.996368408203 43.0
0 -3.20672988891602e-05 44.0
0 0.0616774559020996 45.0
0 -0.024408221244812 46.0
0 -0.0224101543426514 47.0
0 -0.0267354249954224 48.0
0.1147134639 0.288197636604309 49.0
0.081908723024 0.0531164407730103 50.0
2.1369753677 0.899784624576569 51.0
0 -0.0432802438735962 52.0
0 -0.0355826616287231 53.0
0 0.00400817394256592 54.0
0 -0.0861817598342896 55.0
0 -0.0480970144271851 56.0
0 -0.000904679298400879 57.0
0 -0.104064583778381 58.0
0 -0.0033344030380249 59.0
0 -0.0377777814865112 60.0
0 -0.0595790147781372 61.0
0 0.874700665473938 62.0
221.46793065 257.413635253906 0.0
0 -0.0371662378311157 1.0
0 -0.0539764165878296 2.0
0 -0.0085684061050415 3.0
0 -0.04218590259552 4.0
0 0.0503994226455688 5.0
0 -0.0152708292007446 6.0
0 -0.0696765184402466 7.0
0 -0.0210248231887817 8.0
0 -0.00674021244049072 9.0
0 -0.0308915376663208 10.0
0 -0.0301302671432495 11.0
0 0.0329002141952515 12.0
71.824143599 70.7025451660156 13.0
31.99489351 28.0427589416504 14.0
0 -0.0506535768508911 15.0
0 -0.0350476503372192 16.0
40.414019712 39.4185066223145 17.0
0 0.0291255712509155 18.0
0 0.0472649335861206 19.0
0 -0.0173767805099487 20.0
0 -0.0367563962936401 21.0
0 0.00804650783538818 22.0
37.061387067 36.0316619873047 23.0
0 0.077875018119812 24.0
0 3.48087453842163 25.0
0 -0.0154420137405396 26.0
0 0.00830376148223877 27.0
0 -0.0612145662307739 28.0
0 -0.0358201265335083 29.0
0 -0.0340784788131714 30.0
0 -0.0190576314926147 31.0
0 0.079283595085144 32.0
0 0.00405395030975342 33.0
0 -0.0220094919204712 34.0
0 -0.0015113353729248 35.0
0 -0.10965621471405 36.0
0 0.111395001411438 37.0
0 -0.0416923761367798 38.0
0 0.0120474100112915 39.0
0 0.104543089866638 40.0
0 -0.0533113479614258 41.0
0 -0.0370746850967407 42.0
291.81780057 296.163879394531 43.0
0 0.0902818441390991 44.0
0 -0.0335477590560913 45.0
0 -0.0697544813156128 46.0
0 0.019964337348938 47.0
0 -0.0797044038772583 48.0
0 0.260338544845581 49.0
0.10256904065 0.0606075525283813 50.0
1.5964032373 0.928387105464935 51.0
0 -0.10557234287262 52.0
0 -0.0102118253707886 53.0
0 0.00259935855865479 54.0
0 -0.0635775327682495 55.0
0 -0.0511733293533325 56.0
0 -0.029180645942688 57.0
0 0.00350987911224365 58.0
0 -0.0153266191482544 59.0
0 0.184333682060242 60.0
0 0.0496374368667603 61.0
0 0.959457814693451 62.0
247.6131174 257.752105712891 0.0
0 -0.112393975257874 1.0
0 0.0421351194381714 2.0
0 -0.0660039186477661 3.0
0 -0.0164312124252319 4.0
0 0.0158122777938843 5.0
0 0.00307595729827881 6.0
0 -0.0671733617782593 7.0
0 -0.0543466806411743 8.0
0 0.0681976079940796 9.0
0 -0.0727139711380005 10.0
0 0.0417812466621399 11.0
0 0.0137361288070679 12.0
76.319203029 75.2580795288086 13.0
25.649087238 21.666202545166 14.0
0 -0.0601679086685181 15.0
0 -0.0293122529983521 16.0
25.341914732 25.8316020965576 17.0
0 0.0270012617111206 18.0
0 0.033291220664978 19.0
0 0.012609601020813 20.0
0 -0.0391956567764282 21.0
0 0.0416039228439331 22.0
34.531093199 33.9949493408203 23.0
0 -0.0489007234573364 24.0
0 0.596081972122192 25.0
0 -0.0614777803421021 26.0
0 -0.00221002101898193 27.0
0 -0.00410759449005127 28.0
0 -0.0640445947647095 29.0
0 0.161571383476257 30.0
0 -0.0166829824447632 31.0
0 -0.0162636041641235 32.0
0 0.0228999853134155 33.0
0 -0.0512984991073608 34.0
0 -0.0670139789581299 35.0
0 0.00915205478668213 36.0
0 -0.0410631895065308 37.0
0 -0.0391644239425659 38.0
0 0.0535143613815308 39.0
0 0.0420163869857788 40.0
0 -0.0453537702560425 41.0
0 -0.0401273965835571 42.0
314.28944471 296.400787353516 43.0
0 0.0313018560409546 44.0
0 0.0229144096374512 45.0
0 -0.00299584865570068 46.0
0 0.00538694858551025 47.0
0 0.0194896459579468 48.0
0 0.285309493541718 49.0
0.095151660427 0.084739089012146 50.0
2.2334348031 0.957065045833588 51.0
0 -0.0380657911300659 52.0
0 -0.0216497182846069 53.0
0 -0.00307643413543701 54.0
0 -0.058885931968689 55.0
0 0.0174990892410278 56.0
0 -0.00838577747344971 57.0
0 -0.0849772691726685 58.0
0 -0.0873926877975464 59.0
0 -0.00709474086761475 60.0
0 0.00751006603240967 61.0
0 0.858212828636169 62.0
261.74610948 256.243621826172 0.0
0 -0.0703941583633423 1.0
0 0.00831782817840576 2.0
0 -0.0598646402359009 3.0
0 -0.0216411352157593 4.0
0 -0.0166712999343872 5.0
0 -0.134721875190735 6.0
0 0.0440076589584351 7.0
0 -0.0287591218948364 8.0
0 0.0497058629989624 9.0
0 0.0166052579879761 10.0
0 0.00953567028045654 11.0
0 0.00889909267425537 12.0
69.53284277 69.5822601318359 13.0
38.059677214 31.2534961700439 14.0
0 -0.0456863641738892 15.0
0 -0.0387316942214966 16.0
31.594816246 31.8471946716309 17.0
0 -0.0285543203353882 18.0
0 -0.0214616060256958 19.0
0 0.0239659547805786 20.0
0 0.0122023820877075 21.0
0 -0.028417706489563 22.0
35.149564525 34.3431549072266 23.0
0 0.0161761045455933 24.0
0 0.470877528190613 25.0
0 -0.0181299448013306 26.0
0 0.065226674079895 27.0
0 -0.0595904588699341 28.0
0 -0.0523432493209839 29.0
0 -0.0261467695236206 30.0
0 -0.0178765058517456 31.0
0 -0.0215293169021606 32.0
0 0.0104924440383911 33.0
0 -0.0872138738632202 34.0
0 0.0330454111099243 35.0
0 -0.0920826196670532 36.0
0 -0.0215564966201782 37.0
0 0.00713884830474854 38.0
0 -0.0170663595199585 39.0
0 0.0415447950363159 40.0
0 -0.00193440914154053 41.0
0 -0.036969780921936 42.0
294.81175233 296.001983642578 43.0
0 0.00565183162689209 44.0
0 -0.00696647167205811 45.0
0 -0.0191797018051147 46.0
0 -0.00229728221893311 47.0
0 0.0160764455795288 48.0
0 0.25645512342453 49.0
0.070426020313 0.0701509714126587 50.0
2.2325551085 0.950523138046265 51.0
0 -0.0772587060928345 52.0
0 -0.0609637498855591 53.0
0 0.0196284055709839 54.0
0 -0.103996396064758 55.0
0 -0.042455792427063 56.0
0 -0.0176273584365845 57.0
0 -0.0490289926528931 58.0
0 -0.0132104158401489 59.0
0 -0.113829970359802 60.0
0 0.00463712215423584 61.0
0 0.870612621307373 62.0
277.09290868 256.888275146484 0.0
0 0.00251519680023193 1.0
0 -0.041784405708313 2.0
0 -0.0215526819229126 3.0
0 -0.0365356206893921 4.0
0 -0.121551156044006 5.0
0 -0.00265467166900635 6.0
0 0.00643050670623779 7.0
0 0.122130632400513 8.0
0 -0.705836892127991 9.0
0 -0.0336788892745972 10.0
0 -0.0724892616271973 11.0
0 -0.106019616127014 12.0
53.477481834 52.9091300964355 13.0
36.912436926 36.1450500488281 14.0
0 -0.0224453210830688 15.0
0 0.0376220941543579 16.0
25.912651484 25.9916286468506 17.0
0 0.00229537487030029 18.0
0 -0.000928997993469238 19.0
0 0.0472720861434937 20.0
0 0.00661146640777588 21.0
0 -0.0135589838027954 22.0
28.335864102 27.6559219360352 23.0
0 -0.0678902864456177 24.0
0 0.698790311813354 25.0
0 0.130953669548035 26.0
0 0.0343219041824341 27.0
0 0.100315928459167 28.0
0 -0.0110548734664917 29.0
0 -0.0894734859466553 30.0
0 -0.0746692419052124 31.0
0 -0.0355678796768188 32.0
0 0.000742077827453613 33.0
0 -0.0435925722122192 34.0
0 -0.0160273313522339 35.0
0 0.0778306722640991 36.0
0 -0.0181289911270142 37.0
0 -0.0451222658157349 38.0
0 -0.181577801704407 39.0
0 -0.09675133228302 40.0
0 0.0533829927444458 41.0
0 -0.0788732767105103 42.0
290.95705833 296.15771484375 43.0
0 0.0634907484054565 44.0
0 -0.0899426937103271 45.0
0 -0.00323688983917236 46.0
0 -0.0128329992294312 47.0
0 0.0756033658981323 48.0
2.108941273 0.312307059764862 49.0
0.096002606855 0.0721663236618042 50.0
1.9648751656 0.901982307434082 51.0
0 -0.0568591356277466 52.0
0 -0.0612469911575317 53.0
0 -0.0247722864151001 54.0
0 -0.0885125398635864 55.0
0 -0.149860739707947 56.0
0 0.0124319791793823 57.0
0 -0.0412269830703735 58.0
0 -0.0532861948013306 59.0
0 -0.0846089124679565 60.0
0 -0.0315314531326294 61.0
0 0.812001585960388 62.0
249.45492195 225.708770751953 0.0
0 -0.0296388864517212 1.0
0 -0.0659445524215698 2.0
0 -0.0322204828262329 3.0
0 -0.036105751991272 4.0
0 0.0351389646530151 5.0
0 -0.131374478340149 6.0
0 0.00156629085540771 7.0
0 0.0552546977996826 8.0
0 0.0352205038070679 9.0
0 0.0427438020706177 10.0
0 0.0955287218093872 11.0
0 0.055956244468689 12.0
55.251927053 56.4067077636719 13.0
51.677657302 44.0389976501465 14.0
0 -0.0133072137832642 15.0
0 -0.0816155672073364 16.0
40.790354674 40.1670989990234 17.0
0 -0.0230375528335571 18.0
0 0.0324071645736694 19.0
0 -0.00578868389129639 20.0
0 -0.0049511194229126 21.0
0 0.0187777280807495 22.0
26.034732853 26.3711223602295 23.0
0 0.0763849020004272 24.0
25.222853634 2.87144613265991 25.0
0 0.0367251634597778 26.0
0 0.0178351998329163 27.0
0 0.0598825216293335 28.0
0 -0.026408314704895 29.0
0 -0.064411997795105 30.0
0 -0.0592354536056519 31.0
0 -0.119952321052551 32.0
0 0.0363553762435913 33.0
0 -0.00691020488739014 34.0
0 -0.0673567056655884 35.0
0 0.00370919704437256 36.0
0 -0.000953078269958496 37.0
0 -0.0350617170333862 38.0
0 -0.0699759721755981 39.0
0 -0.109686017036438 40.0
0 0.0628408193588257 41.0
0 -0.0449353456497192 42.0
301.2916209 297.818481445312 43.0
0 0.0352019071578979 44.0
0 0.0610418319702148 45.0
0 0.0218197107315063 46.0
0 -0.0214804410934448 47.0
0 0.0501860380172729 48.0
2.2964556695 0.451121628284454 49.0
0.094805283665 0.0833121538162231 50.0
2.0505827376 1.01754057407379 51.0
0 -0.080464243888855 52.0
0 0.00597274303436279 53.0
0 -0.0154415369033813 54.0
0 -0.0636903047561646 55.0
0 -0.101434826850891 56.0
0 0.0791834592819214 57.0
0 0.0117009878158569 58.0
0 -0.00781309604644775 59.0
0 -0.106212735176086 60.0
0 0.0214439630508423 61.0
0 0.850253462791443 62.0
243.21563184 255.35905456543 0.0
0 -0.0590320825576782 1.0
0 -0.0271924734115601 2.0
0 -0.0344313383102417 3.0
0 -0.0477820634841919 4.0
0 0.0182849168777466 5.0
0 -0.106843590736389 6.0
0 0.0318366289138794 7.0
0 -0.081812858581543 8.0
0 0.00882470607757568 9.0
0 -0.0357562303543091 10.0
0 0.101417243480682 11.0
0 -0.0806299448013306 12.0
78.088979864 77.6702117919922 13.0
9.8632372385 23.3285884857178 14.0
0 0.00307214260101318 15.0
0 -0.0364145040512085 16.0
25.826757671 25.2711200714111 17.0
0 -0.0203903913497925 18.0
0 -0.00402414798736572 19.0
0 -0.032168984413147 20.0
0 -0.0428684949874878 21.0
0 0.0532110929489136 22.0
26.298612433 25.8936023712158 23.0
0 0.0416363477706909 24.0
0 0.621346592903137 25.0
0 0.0626415014266968 26.0
0 0.00876128673553467 27.0
0 -0.0264283418655396 28.0
0 -0.0740660429000854 29.0
0 -0.0419604778289795 30.0
0 -0.0684171915054321 31.0
0 -0.0754081010818481 32.0
0 0.0922855138778687 33.0
0 -0.0318201780319214 34.0
0 0.00380480289459229 35.0
0 -0.0661019086837769 36.0
0 -0.043138861656189 37.0
0 -0.0127080678939819 38.0
0 0.139748930931091 39.0
0 0.233280062675476 40.0
0 -0.00604236125946045 41.0
0 -0.00394785404205322 42.0
299.96133095 294.239624023438 43.0
0 0.0181916952133179 44.0
0 0.0141549110412598 45.0
0 0.00806701183319092 46.0
0 -0.0675555467605591 47.0
0 0.00799787044525146 48.0
3.9638102895 0.31487300992012 49.0
0.097987304558 0.0548189878463745 50.0
2.0080214324 0.952229261398315 51.0
0 -0.0381416082382202 52.0
0 -0.0244165658950806 53.0
0 0.0543442964553833 54.0
0 -0.141185164451599 55.0
0 0.0324622392654419 56.0
0 -0.0481754541397095 57.0
0 -0.0231722593307495 58.0
0 -0.0202337503433228 59.0
0 0.0670648813247681 60.0
0 -0.041258692741394 61.0
0 0.849415898323059 62.0
250.29175705 259.294769287109 0.0
0 -0.128783822059631 1.0
0 -0.12048614025116 2.0
0 0.0195575952529907 3.0
0 -0.0749374628067017 4.0
0 -0.107114195823669 5.0
0 -0.0474191904067993 6.0
0 -0.0844377279281616 7.0
0 -0.0295048952102661 8.0
0 0.150796294212341 9.0
0 -0.0622900724411011 10.0
0 -0.000163555145263672 11.0
0 -0.124539017677307 12.0
78.279068904 79.114013671875 13.0
28.558928874 21.3057613372803 14.0
0 -0.0437837839126587 15.0
0 0.00352704524993896 16.0
32.921552998 32.9957695007324 17.0
0 -0.00677311420440674 18.0
0 -0.0124388933181763 19.0
0 0.00811326503753662 20.0
0 -0.0183185338973999 21.0
0 -0.0479999780654907 22.0
37.093937737 37.0717964172363 23.0
0 -0.129278779029846 24.0
0 0.729964852333069 25.0
0 0.0718940496444702 26.0
0 0.0740457773208618 27.0
0 -0.0249630212783813 28.0
0 -0.0675269365310669 29.0
0 -0.0447101593017578 30.0
0 -0.0762890577316284 31.0
0 -0.0580765008926392 32.0
0 0.107900023460388 33.0
0 -0.0272349119186401 34.0
0 -0.077170729637146 35.0
0 0.0788112878799438 36.0
0 0.028532862663269 37.0
0 -0.0299183130264282 38.0
0 0.0548117160797119 39.0
0 0.251038908958435 40.0
0 0.017017126083374 41.0
0 -0.0762077569961548 42.0
291.13352056 294.493957519531 43.0
0 0.0447238683700562 44.0
0 -0.0831350088119507 45.0
0 -0.0188237428665161 46.0
0 -0.0536042451858521 47.0
0 -0.0638290643692017 48.0
1.5982567953 0.271938860416412 49.0
0.097149398972 0.0807908773422241 50.0
1.7920967951 0.85857880115509 51.0
0 -0.0777159929275513 52.0
0 -0.0546542406082153 53.0
0 -0.00185954570770264 54.0
0 -0.0454384088516235 55.0
0 -0.0661388635635376 56.0
0 0.015288233757019 57.0
0 -0.0881563425064087 58.0
0 -0.0227209329605103 59.0
0 0.05266273021698 60.0
0 -0.0691415071487427 61.0
0 0.814101815223694 62.0
239.51127979 256.273010253906 0.0
0 -0.0662561655044556 1.0
0 0.00539767742156982 2.0
0 0.0230940580368042 3.0
0 -0.0312219858169556 4.0
0 -0.0696207284927368 5.0
0 -0.045596718788147 6.0
0 -0.0655995607376099 7.0
0 -0.00761497020721436 8.0
0 0.0407670736312866 9.0
0 0.0334118604660034 10.0
0 0.00326061248779297 11.0
0 -0.0610672235488892 12.0
64.485923547 64.1761474609375 13.0
42.325195667 34.2059173583984 14.0
0 -0.0025402307510376 15.0
0 -0.022579550743103 16.0
29.925599919 29.522331237793 17.0
0 0.0239788293838501 18.0
0 -0.0247715711593628 19.0
0 -0.00443637371063232 20.0
0 0.00539910793304443 21.0
0 -0.0169421434402466 22.0
33.488868177 33.2219543457031 23.0
0 -0.0843964815139771 24.0
0 0.547119379043579 25.0
0 0.0187734365463257 26.0
0 0.0250257253646851 27.0
0 -0.0787864923477173 28.0
0 -0.0303088426589966 29.0
0 -0.0799872875213623 30.0
0 -0.0377756357192993 31.0
0 -0.0720111131668091 32.0
0 0.069703221321106 33.0
0 -0.0278688669204712 34.0
0 -0.0175344944000244 35.0
0 -0.0049668550491333 36.0
0 -0.0221570730209351 37.0
0 0.0534056425094604 38.0
0 0.117776274681091 39.0
0 0.0834358930587769 40.0
0 -0.0648702383041382 41.0
0 0.0570167303085327 42.0
291.15549498 297.947174072266 43.0
0 0.0102866888046265 44.0
0 0.0125552415847778 45.0
0 -0.0619281530380249 46.0
0 -0.0725895166397095 47.0
0 0.0408796072006226 48.0
0.36362271719 0.268895089626312 49.0
0.071924548879 0.0590404272079468 50.0
1.9133313534 0.903506934642792 51.0
0 -0.0599290132522583 52.0
0 -0.0040518045425415 53.0
0 0.00680696964263916 54.0
0 -0.0815736055374146 55.0
0 -0.0273455381393433 56.0
0 -0.0130096673965454 57.0
0 -0.0771390199661255 58.0
0 -0.0333002805709839 59.0
0 -0.0763803720474243 60.0
0 -0.0639380216598511 61.0
0 0.838030576705933 62.0
284.18896347 255.966995239258 0.0
0 0.00819146633148193 1.0
0 -0.100614666938782 2.0
0 0.0330067873001099 3.0
0 -0.0496886968612671 4.0
0 -0.0905762910842896 5.0
0 -0.148305535316467 6.0
0 -0.0288664102554321 7.0
0 0.0213702917098999 8.0
0 -0.0605255365371704 9.0
0 -0.0314708948135376 10.0
0 0.0147600769996643 11.0
0 0.0520728826522827 12.0
72.138429721 70.2524337768555 13.0
10.4424253 29.7187995910645 14.0
0 -0.00785171985626221 15.0
0 -0.0130387544631958 16.0
25.136219035 24.3865451812744 17.0
0 0.0356994867324829 18.0
0 -0.0287426710128784 19.0
0 -0.00845777988433838 20.0
0 -0.021964430809021 21.0
0 0.00618350505828857 22.0
35.850716914 35.858715057373 23.0
0 -0.0772405862808228 24.0
0 0.674262464046478 25.0
0 0.0195201635360718 26.0
0 -0.00522089004516602 27.0
0 0.00266969203948975 28.0
0 -0.0597876310348511 29.0
0 -0.1046541929245 30.0
0 -0.122157454490662 31.0
0 0.06017005443573 32.0
0 -0.0766998529434204 33.0
0 -0.00220024585723877 34.0
0 0.088394820690155 35.0
0 -0.178108096122742 36.0
0 0.0481616258621216 37.0
0 -0.0621064901351929 38.0
0 0.158550620079041 39.0
0 0.0874994993209839 40.0
0 -0.0715128183364868 41.0
0 -0.0400935411453247 42.0
295.85704234 295.116333007812 43.0
0 -0.033208966255188 44.0
0 0.0959013700485229 45.0
0 -0.0195139646530151 46.0
0 -0.0619426965713501 47.0
0 -0.01032555103302 48.0
0 0.2625373005867 49.0
0.089656547463 0.0561076402664185 50.0
1.6011260186 0.846039175987244 51.0
0 -0.062653660774231 52.0
0 -0.0721532106399536 53.0
0 -0.0278457403182983 54.0
0 0.0323783159255981 55.0
0 -0.0624707937240601 56.0
0 -0.0385414361953735 57.0
0 -0.0305818319320679 58.0
0 -0.0461345911026001 59.0
0 -0.228098034858704 60.0
0 0.0399106740951538 61.0
0 0.850122451782227 62.0
261.86699589 252.335205078125 0.0
0 -0.0894380807876587 1.0
0 0.0603638887405396 2.0
0 -0.0397576093673706 3.0
0 -0.0593119859695435 4.0
0 0.0223182439804077 5.0
0 -0.0711246728897095 6.0
0 0.0909308195114136 7.0
0 0.076408863067627 8.0
0 -0.103214859962463 9.0
0 0.0161000490188599 10.0
0 -0.0692309141159058 11.0
0 0.163387656211853 12.0
44.901317496 42.9965476989746 13.0
35.760039936 48.2840919494629 14.0
0 -0.0033031702041626 15.0
0 -0.0482488870620728 16.0
29.486854089 29.8259029388428 17.0
0 -0.0297337770462036 18.0
0 0.00416195392608643 19.0
0 -0.0138794183731079 20.0
0 -0.0271240472793579 21.0
0 -0.106166958808899 22.0
33.336136368 32.0652542114258 23.0
0 0.0405400991439819 24.0
0 0.794720232486725 25.0
0 0.0827301740646362 26.0
0 -0.0505859851837158 27.0
0 -0.0899649858474731 28.0
0 0.00578320026397705 29.0
0 -0.083432674407959 30.0
0 -0.0749467611312866 31.0
0 -0.113592267036438 32.0
0 -0.025478720664978 33.0
0 -0.0620030164718628 34.0
0 -0.0815762281417847 35.0
0 0.0939368009567261 36.0
0 0.0151084661483765 37.0
0 -0.0253034830093384 38.0
0 0.0356465578079224 39.0
0 0.133726954460144 40.0
0 -0.031701922416687 41.0
0 -0.0747562646865845 42.0
283.00956435 297.748901367188 43.0
0 0.0519975423812866 44.0
0 0.0336113572120667 45.0
0 0.0166858434677124 46.0
0 -0.0285404920578003 47.0
0 -0.0682440996170044 48.0
1.8665259843 0.242960929870605 49.0
0.074070759966 0.0958749055862427 50.0
2.183104921 0.945481181144714 51.0
0 -0.0392469167709351 52.0
0 -0.0300575494766235 53.0
0 -0.0591341257095337 54.0
0 -0.0131670236587524 55.0
0 -0.0998698472976685 56.0
0 0.0470489263534546 57.0
0 -0.0423170328140259 58.0
0 -0.0344411134719849 59.0
0 0.16321074962616 60.0
0 -0.0227092504501343 61.0
0 0.825569868087769 62.0
265.67200758 257.322143554688 0.0
0 -0.00828278064727783 1.0
0 0.0222162008285522 2.0
0 -0.0192109346389771 3.0
0 -0.0411981344223022 4.0
0 -0.0694700479507446 5.0
0 0.0270971059799194 6.0
0 -0.0108577013015747 7.0
0 -0.00419938564300537 8.0
0 -0.0667539834976196 9.0
0 -0.0803908109664917 10.0
0 0.0430021286010742 11.0
0 0.0371679067611694 12.0
76.710259106 75.7039489746094 13.0
11.391431096 28.7131862640381 14.0
0 -0.00750839710235596 15.0
0 0.00329601764678955 16.0
31.994360087 31.3069686889648 17.0
0 0.0203424692153931 18.0
0 -0.0143777132034302 19.0
0 -0.0101715326309204 20.0
0 0.0142563581466675 21.0
0 -0.0903493165969849 22.0
35.111561049 34.289981842041 23.0
0 -0.0544883012771606 24.0
0 0.631759881973267 25.0
0 0.0690377950668335 26.0
0 0.0857652425765991 27.0
0 0.138687968254089 28.0
0 -0.0137621164321899 29.0
0 -0.0437003374099731 30.0
0 -0.00445401668548584 31.0
0 -0.00032198429107666 32.0
0 -0.0294114351272583 33.0
0 -0.00588095188140869 34.0
0 -0.104806303977966 35.0
0 0.0528095960617065 36.0
0 0.0359519720077515 37.0
0 0.00577151775360107 38.0
0 -0.0246380567550659 39.0
0 0.0430721044540405 40.0
0 -0.0467973947525024 41.0
0 -0.0295530557632446 42.0
283.58730026 296.036560058594 43.0
0 0.0582655668258667 44.0
0 -0.0247375965118408 45.0
0 0.00255310535430908 46.0
0 -0.0794352293014526 47.0
0 -0.0747805833816528 48.0
0 0.348079681396484 49.0
0.10643368626 0.225070953369141 50.0
2.1690377987 1.00738799571991 51.0
0 -0.037293553352356 52.0
0 0.00828111171722412 53.0
0 0.000486254692077637 54.0
0 -0.0697892904281616 55.0
0 0.0121911764144897 56.0
0 0.00936782360076904 57.0
0 -0.117915511131287 58.0
0 0.0432752370834351 59.0
0 0.0788732767105103 60.0
0 0.013403058052063 61.0
0 0.857312977313995 62.0
267.48054756 256.41943359375 0.0
0 -0.0201517343521118 1.0
0 -0.0936568975448608 2.0
0 -0.00870883464813232 3.0
0 -0.0519927740097046 4.0
0 0.0473107099533081 5.0
0 -0.169816613197327 6.0
0 0.0365899801254272 7.0
0 0.0187007188796997 8.0
0 0.0708354711532593 9.0
0 -0.0123013257980347 10.0
0 -0.0151035785675049 11.0
0 0.00187146663665771 12.0
57.703916354 57.0638999938965 13.0
37.020759439 36.7838706970215 14.0
0 -0.00507843494415283 15.0
0 -0.02501380443573 16.0
27.115647354 26.2777099609375 17.0
0 0.0236657857894897 18.0
0 0.0105668306350708 19.0
0 0.0190213918685913 20.0
0 -0.0329567193984985 21.0
0 -0.0151740312576294 22.0
30.978534807 30.4942493438721 23.0
0 -0.0975672006607056 24.0
0 0.694423019886017 25.0
0 0.0466426610946655 26.0
0 0.0195392370223999 27.0
0 -0.0348249673843384 28.0
0 -0.0365303754806519 29.0
0 -0.0466896295547485 30.0
0 -0.0546671152114868 31.0
0 -0.0125805139541626 32.0
0 0.00189435482025146 33.0
0 -0.0173627138137817 34.0
0 -0.0546797513961792 35.0
0 0.0513588190078735 36.0
0 -0.00356400012969971 37.0
0 0.0229984521865845 38.0
0 0.0277820825576782 39.0
0 0.114770293235779 40.0
0 0.022119402885437 41.0
0 -0.00395548343658447 42.0
286.01325435 296.807830810547 43.0
0 0.00949180126190186 44.0
0 -0.0286084413528442 45.0
0 -0.00619757175445557 46.0
0 -0.0521384477615356 47.0
0 0.0191367864608765 48.0
0 0.28021827340126 49.0
0.077024026516 0.0479472875595093 50.0
1.7915774925 0.918763518333435 51.0
0 -0.0634948015213013 52.0
0 -0.0520297288894653 53.0
0 0.0222753286361694 54.0
0 -0.0910030603408813 55.0
0 0.0335880517959595 56.0
0 -0.0396734476089478 57.0
0 -0.0385006666183472 58.0
0 -0.0867851972579956 59.0
0 -0.0802689790725708 60.0
0 -0.0315672159194946 61.0
0 0.830655217170715 62.0
227.07067814 252.916687011719 0.0
0 0.0198625326156616 1.0
0 -0.0147241353988647 2.0
0 -0.028079628944397 3.0
0 -0.0273052453994751 4.0
0 -0.0129662752151489 5.0
0 -0.00576889514923096 6.0
0 0.00479328632354736 7.0
0 -0.0537302494049072 8.0
0 -0.0173145532608032 9.0
0 0.00554955005645752 10.0
0 -0.0452815294265747 11.0
0 -0.0710445642471313 12.0
72.16094778 70.9722061157227 13.0
30.151610644 27.21994972229 14.0
0 -0.0145403146743774 15.0
0 -0.0541454553604126 16.0
38.252456121 37.4797668457031 17.0
0 0.0274919271469116 18.0
0 0.0298553705215454 19.0
0 -0.0258208513259888 20.0
0 -0.0434738397598267 21.0
0 0.00880229473114014 22.0
32.388080696 31.8831272125244 23.0
0 -0.0856109857559204 24.0
0 0.654475331306458 25.0
0 0.0164769887924194 26.0
0 0.030214786529541 27.0
0 0.0213512182235718 28.0
0 -0.0446470975875854 29.0
0 -0.0693424940109253 30.0
0 -0.040859580039978 31.0
0 -0.0167170763015747 32.0
0 0.0693815946578979 33.0
0 0.00720345973968506 34.0
0 -0.074496865272522 35.0
0 0.0136953592300415 36.0
0 -0.0286833047866821 37.0
0 -0.041291356086731 38.0
0 0.0480846166610718 39.0
0 0.147331118583679 40.0
0 -0.0154945850372314 41.0
0 -0.0169597864151001 42.0
286.70055721 296.845367431641 43.0
0 -0.0306650400161743 44.0
0 0.00147026777267456 45.0
0 0.0103274583816528 46.0
0 -0.00933468341827393 47.0
0 0.0148056745529175 48.0
0 0.332823038101196 49.0
0.071867333211 0.0420364141464233 50.0
2.2525413002 0.91063666343689 51.0
0 -0.0388776063919067 52.0
0 -0.012525200843811 53.0
0 0.003853440284729 54.0
0 -0.0369552373886108 55.0
0 -0.120573401451111 56.0
0 0.0508488416671753 57.0
0 -0.0600420236587524 58.0
0 0.00120198726654053 59.0
0 -0.108595728874207 60.0
0 -0.0142351388931274 61.0
0 0.830352902412415 62.0
261.14042368 255.667266845703 0.0
0 0.00663316249847412 1.0
0 -0.0746098756790161 2.0
0 0.0407613515853882 3.0
0 -0.0800155401229858 4.0
0 -0.0965801477432251 5.0
0 0.127151370048523 6.0
0 -0.0435200929641724 7.0
0 -0.0475760698318481 8.0
0 0.0612815618515015 9.0
0 0.0360883474349976 10.0
0 0.032809853553772 11.0
0 -0.0200077295303345 12.0
71.083858121 71.7323532104492 13.0
50.826591861 30.4892444610596 14.0
0 -0.0134421586990356 15.0
0 -0.0122162103652954 16.0
36.651673303 36.4074592590332 17.0
0 -0.0345925092697144 18.0
0 -0.0475658178329468 19.0
0 -0.0540007352828979 20.0
0 -0.00631725788116455 21.0
0 -0.0174847841262817 22.0
31.494727069 30.5367679595947 23.0
0 -0.0113488435745239 24.0
26.743126621 0.565366148948669 25.0
0 0.0955876111984253 26.0
0 -0.0605376958847046 27.0
0 -0.0261293649673462 28.0
0 -0.108734726905823 29.0
0 -0.0144726037979126 30.0
0 -0.0527161359786987 31.0
0 0.038382887840271 32.0
0 -0.0154067277908325 33.0
0 -0.0477255582809448 34.0
0 -0.12373411655426 35.0
0 0.0613211393356323 36.0
0 -0.0799871683120728 37.0
0 0.0260099172592163 38.0
0 0.017605185508728 39.0
0 0.0448306798934937 40.0
0 -0.0942617654800415 41.0
0 -0.0105692148208618 42.0
311.11158598 293.462371826172 43.0
0 -0.0510126352310181 44.0
0 -0.0573527812957764 45.0
0 -0.0190259218215942 46.0
0 -0.0420702695846558 47.0
0 0.0100477933883667 48.0
0 0.241887092590332 49.0
0.096624606515 0.0741997957229614 50.0
2.1237207782 0.964558303356171 51.0
0 -0.0298720598220825 52.0
0 -0.0550528764724731 53.0
0 0.0367799997329712 54.0
0 -0.0607372522354126 55.0
0 -0.0675398111343384 56.0
0 0.00993573665618896 57.0
0 -0.062616229057312 58.0
0 -0.00693404674530029 59.0
0 0.0371955633163452 60.0
0 0.0271629095077515 61.0
0 0.872891187667847 62.0
245.12148855 255.212356567383 0.0
0 -0.00346720218658447 1.0
0 -0.0926421880722046 2.0
0 -0.0336602926254272 3.0
0 -0.063721776008606 4.0
0 0.085945725440979 5.0
0 0.0219444036483765 6.0
0 -0.0350733995437622 7.0
0 0.0320281982421875 8.0
0 0.0483888387680054 9.0
0 -0.0135225057601929 10.0
0 -0.0325173139572144 11.0
0 -0.0995556116104126 12.0
65.895475576 64.759407043457 13.0
41.657727304 33.1823387145996 14.0
0 -0.0467516183853149 15.0
0 0.00691282749176025 16.0
28.640773079 27.9439220428467 17.0
0 0.0371500253677368 18.0
0 0.0131915807723999 19.0
0 -0.0139087438583374 20.0
0 -0.0177820920944214 21.0
0 0.0190485715866089 22.0
34.636532129 34.1971168518066 23.0
0 -0.0582810640335083 24.0
0 0.639118194580078 25.0
0 0.0984095335006714 26.0
0 0.133557438850403 27.0
0 -0.0647569894790649 28.0
0 -0.0147038698196411 29.0
0 -0.111832976341248 30.0
0 0.00729715824127197 31.0
0 -0.0164657831192017 32.0
0 -0.00775229930877686 33.0
0 -0.0360161066055298 34.0
0 -0.0702074766159058 35.0
0 0.0583125352859497 36.0
0 -0.00301659107208252 37.0
0 -0.00980627536773682 38.0
0 0.0215831995010376 39.0
0 0.0638617277145386 40.0
0 -0.0420249700546265 41.0
0 -0.0308271646499634 42.0
290.75527972 297.259735107422 43.0
0 0.0578383207321167 44.0
0 -0.0330688953399658 45.0
0 -0.0202480554580688 46.0
0 -0.043282151222229 47.0
0 0.0141304731369019 48.0
1.0477668826 0.319437175989151 49.0
0.090740649412 0.0640672445297241 50.0
1.5957167434 0.994903385639191 51.0
0 -0.0549660921096802 52.0
0 -0.036704421043396 53.0
0 0.0621274709701538 54.0
0 -0.106230854988098 55.0
0 -0.0300701856613159 56.0
0 0.0179630517959595 57.0
0 -0.058557391166687 58.0
0 -0.0751806497573853 59.0
0 -0.0188955068588257 60.0
0 0.0310183763504028 61.0
0 0.879095554351807 62.0
253.52106179 256.410797119141 0.0
0 -0.0483471155166626 1.0
0 0.0340889692306519 2.0
0 -0.0169264078140259 3.0
0 -0.0844334363937378 4.0
0 -0.00164473056793213 5.0
0 -0.104365468025208 6.0
0 -0.0356463193893433 7.0
0 -0.00438487529754639 8.0
0 0.0064462423324585 9.0
0 -0.00637185573577881 10.0
0 -0.0298656225204468 11.0
0 -0.0332294702529907 12.0
71.875574123 70.9300994873047 13.0
12.838606711 26.775728225708 14.0
0 -0.101044297218323 15.0
0 0.000546097755432129 16.0
27.05299929 26.7412242889404 17.0
0 0.0146969556808472 18.0
0 0.0178679227828979 19.0
0 0.0602854490280151 20.0
0 -0.0353268384933472 21.0
0 -0.0284544229507446 22.0
40.274733803 39.5686531066895 23.0
0 -0.0942379236221313 24.0
0 0.69760262966156 25.0
0 -0.0615302324295044 26.0
0 0.0156697034835815 27.0
0 -0.0212646722793579 28.0
0 -0.0543030500411987 29.0
0 -0.0405126810073853 30.0
0 -0.0342968702316284 31.0
0 0.0428351163864136 32.0
0 -0.0615822076797485 33.0
0 -0.00527417659759521 34.0
0 0.0736881494522095 35.0
0 -0.0273464918136597 36.0
0 -0.0350106954574585 37.0
0 -0.0193675756454468 38.0
0 0.065653920173645 39.0
0 0.0137332677841187 40.0
0 -0.0941215753555298 41.0
0 -0.037061333656311 42.0
298.75882548 297.184265136719 43.0
0 0.0289510488510132 44.0
0 -0.0351468324661255 45.0
0 0.0170642137527466 46.0
0 -0.0452940464019775 47.0
0 0.0132492780685425 48.0
0 0.336974680423737 49.0
0.096344271167 0.0427466630935669 50.0
1.9984780734 0.94541472196579 51.0
0 -0.0483812093734741 52.0
0 -0.0807682275772095 53.0
0 0.0301176309585571 54.0
0 -0.0736182928085327 55.0
0 -0.0122197866439819 56.0
0 0.00822103023529053 57.0
0 -0.0653101205825806 58.0
0 -0.0604885816574097 59.0
0 0.00645434856414795 60.0
0 -0.08466637134552 61.0
0 0.817015528678894 62.0
267.2703099 261.662536621094 0.0
0 -0.11495316028595 1.0
0 -0.0156456232070923 2.0
0 -0.0940762758255005 3.0
0 -0.0585144758224487 4.0
0 -0.00271952152252197 5.0
0 -0.296145558357239 6.0
0 0.0633739233016968 7.0
0 0.00275552272796631 8.0
0 -0.00650656223297119 9.0
0 -0.0994411706924438 10.0
0 -0.0513893365859985 11.0
0 -0.036777138710022 12.0
52.553203475 50.9272079467773 13.0
33.5564501 32.2781791687012 14.0
0 -0.0324317216873169 15.0
0 0.00234353542327881 16.0
27.2933085 26.5161685943604 17.0
0 0.0194920301437378 18.0
0 -0.0184932947158813 19.0
0 -0.0163596868515015 20.0
0 -0.0151849985122681 21.0
0 -0.0642443895339966 22.0
37.442280897 39.4140892028809 23.0
0 -0.149412274360657 24.0
0 2.28361845016479 25.0
0 -0.0523988008499146 26.0
0 0.0924519300460815 27.0
0 -0.0645204782485962 28.0
0 -0.0769084692001343 29.0
0 0.175858855247498 30.0
0 -0.0596269369125366 31.0
0 -0.0603328943252563 32.0
0 0.0388456583023071 33.0
0 -0.0251246690750122 34.0
0 0.0457563400268555 35.0
0 -0.153333783149719 36.0
0 -0.0026012659072876 37.0
0 0.0324157476425171 38.0
0 0.051770806312561 39.0
0 0.194366812705994 40.0
0 -0.0559767484664917 41.0
0 -0.0405198335647583 42.0
287.3955766 296.371032714844 43.0
0 0.0384997129440308 44.0
0 -0.0797073841094971 45.0
0 0.013384222984314 46.0
0 -0.0503407716751099 47.0
0 -0.0863410234451294 48.0
0 0.244477272033691 49.0
0.063175023323 0.127689838409424 50.0
1.6377028112 0.860933780670166 51.0
0 -0.0379868745803833 52.0
0 -0.0276628732681274 53.0
0 0.0197373628616333 54.0
0 -0.0321613550186157 55.0
0 -0.0244215726852417 56.0
0 -0.0399624109268188 57.0
0 -0.0501123666763306 58.0
0 0.0275589227676392 59.0
0 0.0276795625686646 60.0
0 -0.0184460878372192 61.0
0 0.900703072547913 62.0
251.16310876 253.146820068359 0.0
0 -0.0334035158157349 1.0
0 -0.0628954172134399 2.0
0 0.0255802869796753 3.0
0 -0.0627886056900024 4.0
0 0.0319367647171021 5.0
0 0.0660752058029175 6.0
0 -0.023577094078064 7.0
0 -0.04266357421875 8.0
0 0.023760199546814 9.0
0 -0.0587464570999146 10.0
0 -0.0677157640457153 11.0
0 -0.0240265130996704 12.0
50.813518549 49.6347236633301 13.0
40.036298479 40.2671699523926 14.0
0 0.00212609767913818 15.0
0 -0.0219837427139282 16.0
32.836430912 32.538990020752 17.0
0 -0.00747811794281006 18.0
0 -0.0134462118148804 19.0
0 0.0176163911819458 20.0
0 -0.0257056951522827 21.0
0 -0.0031658411026001 22.0
38.982384552 38.3666725158691 23.0
0 0.0446761846542358 24.0
2.1279160358 0.704302251338959 25.0
0 0.0490444898605347 26.0
0 0.00361979007720947 27.0
0 0.2623690366745 28.0
0 -0.0450540781021118 29.0
0 0.0214678049087524 30.0
0 -0.201572060585022 31.0
0 0.002066969871521 32.0
0 0.113076567649841 33.0
0 -0.0346885919570923 34.0
0 0.0464857816696167 35.0
0 0.114359736442566 36.0
0 -0.00642073154449463 37.0
0 0.0115271806716919 38.0
0 -0.0426164865493774 39.0
0 0.0893720388412476 40.0
0 0.138366103172302 41.0
0 0.0271667242050171 42.0
294.89315889 292.506561279297 43.0
0 0.0629075765609741 44.0
0 0.138457298278809 45.0
0 0.0196372270584106 46.0
0 0.00872552394866943 47.0
0 0.0675357580184937 48.0
1.4800181918 0.288585215806961 49.0
0.068843932195 0.0631078481674194 50.0
1.9172830808 0.961363971233368 51.0
0 -0.025079607963562 52.0
0 0.00369870662689209 53.0
0 0.0581938028335571 54.0
0 -0.00976598262786865 55.0
0 0.0408328771591187 56.0
0 0.141639590263367 57.0
0 -0.0252400636672974 58.0
0 0.0301491022109985 59.0
0 0.0129374265670776 60.0
0 -0.027269721031189 61.0
0 0.871934771537781 62.0
246.18008873 259.089233398438 0.0
0 -0.2236248254776 1.0
0 0.0511997938156128 2.0
0 -0.00937449932098389 3.0
0 -0.00939476490020752 4.0
0 0.0300775766372681 5.0
0 -0.136388421058655 6.0
0 -0.0939146280288696 7.0
0 -0.0763250589370728 8.0
0 0.126110434532166 9.0
0 -0.0269197225570679 10.0
0 0.0209015607833862 11.0
0 0.0763057470321655 12.0
51.925210294 52.6357192993164 13.0
47.576853963 44.0812225341797 14.0
0 -0.0928398370742798 15.0
0 -0.0161932706832886 16.0
29.227506981 29.6496868133545 17.0
0 0.0658947229385376 18.0
0 0.0162581205368042 19.0
0 0.0857502222061157 20.0
0 0.0102964639663696 21.0
0 -0.0623527765274048 22.0
40.726369818 38.6715240478516 23.0
0 0.0115894079208374 24.0
0 0.448722779750824 25.0
0 0.0725139379501343 26.0
0 0.0560336112976074 27.0
0 0.155779242515564 28.0
0 -0.0479806661605835 29.0
0 -0.0940711498260498 30.0
0 -0.0235446691513062 31.0
0 -0.083476185798645 32.0
0 -0.0169025659561157 33.0
0 -0.0408943891525269 34.0
0 -0.0039292573928833 35.0
0 0.0410267114639282 36.0
0 -0.044664740562439 37.0
0 0.045432448387146 38.0
0 0.064353346824646 39.0
0 0.256232142448425 40.0
0 -0.0712183713912964 41.0
0 -0.0075908899307251 42.0
303.15273324 296.165679931641 43.0
0 -0.0316396951675415 44.0
0 0.00929975509643555 45.0
0 0.000969052314758301 46.0
0 -0.0648590326309204 47.0
0 0.0274695158004761 48.0
0 0.316918015480042 49.0
0.070269781269 0.0494519472122192 50.0
1.5756982828 0.942407965660095 51.0
0 -0.108300566673279 52.0
0 -0.0521875619888306 53.0
0 -0.0493119955062866 54.0
0 -0.149256825447083 55.0
0 -0.0666755437850952 56.0
0 0.00852477550506592 57.0
0 -0.0414911508560181 58.0
0 -0.0483685731887817 59.0
0 0.00893962383270264 60.0
0 -0.0627728700637817 61.0
0 0.844065427780151 62.0
280.56167198 255.365051269531 0.0
0 -0.0399771928787231 1.0
0 -0.0365563631057739 2.0
0 -0.0542334318161011 3.0
0 -0.0261446237564087 4.0
0 -0.12597382068634 5.0
0 0.0919874906539917 6.0
0 0.0433143377304077 7.0
0 0.0308936834335327 8.0
0 0.0313504934310913 9.0
0 -0.0204514265060425 10.0
0 -0.0108897686004639 11.0
0 -0.0868545770645142 12.0
64.059476262 64.0795059204102 13.0
40.988447534 33.8592147827148 14.0
0 0.0432237386703491 15.0
0 -0.00979292392730713 16.0
37.877956099 38.0691184997559 17.0
0 0.0100451707839966 18.0
0 -0.0333744287490845 19.0
0 -0.0213350057601929 20.0
0 -0.0325990915298462 21.0
0 -0.0203896760940552 22.0
32.491973167 31.4414882659912 23.0
0 -0.00511372089385986 24.0
0 0.561072766780853 25.0
0 0.0192354917526245 26.0
0 0.0321601033210754 27.0
0 -0.0168310403823853 28.0
0 -0.015159010887146 29.0
0 -0.0513797998428345 30.0
0 -0.0299831628799438 31.0
0 -0.0180574655532837 32.0
0 -0.0479110479354858 33.0
0 -0.0695375204086304 34.0
0 -0.0599392652511597 35.0
0 0.111340165138245 36.0
0 -0.0349997282028198 37.0
0 -0.0224283933639526 38.0
0 0.00206887722015381 39.0
0 0.0676602125167847 40.0
0 -0.0281050205230713 41.0
0 0.0137656927108765 42.0
285.93045757 295.141418457031 43.0
0 -0.0322381258010864 44.0
0 -0.0270166397094727 45.0
0 -0.00722134113311768 46.0
0 -0.0200992822647095 47.0
0 -0.0170680284500122 48.0
0 0.289401352405548 49.0
0.072908591894 0.0835500955581665 50.0
1.9118310413 0.862716674804688 51.0
0 -0.0582574605941772 52.0
0 -0.00463521480560303 53.0
0 -0.0265272855758667 54.0
0 -0.0618442296981812 55.0
0 0.0230339765548706 56.0
0 -0.0382624864578247 57.0
0 -0.048672080039978 58.0
0 -0.0691603422164917 59.0
0 0.0331159830093384 60.0
0 0.0415838956832886 61.0
0 0.834053874015808 62.0
268.84593865 255.780227661133 0.0
0 -0.0564600229263306 1.0
0 -0.0718499422073364 2.0
0 -0.00776660442352295 3.0
0 0.0133222341537476 4.0
0 0.0569652318954468 5.0
0 0.097432017326355 6.0
0 -0.00184595584869385 7.0
0 -0.00212687253952026 8.0
0 0.0950502157211304 9.0
0 0.0105221271514893 10.0
0 0.030984491109848 11.0
0 -0.179749131202698 12.0
73.288245755 72.0638885498047 13.0
33.653586836 26.6071109771729 14.0
0 0.0171421766281128 15.0
0 -0.0392776727676392 16.0
38.056911978 37.0763130187988 17.0
0 -0.00877964496612549 18.0
0 -0.0131193399429321 19.0
0 -0.0277377367019653 20.0
0 -0.0548003911972046 21.0
0 -0.0454651117324829 22.0
37.427885457 36.8376922607422 23.0
0 0.0429285764694214 24.0
0 0.462038636207581 25.0
0 0.0993193387985229 26.0
0 0.0699895620346069 27.0
0 -0.00578320026397705 28.0
0 -0.118000626564026 29.0
0 0.036620020866394 30.0
0 -0.0622111558914185 31.0
0 -0.024091362953186 32.0
0 0.03907310962677 33.0
0 -0.0808995962142944 34.0
0 0.0234216451644897 35.0
0 -0.113194108009338 36.0
0 -0.054021954536438 37.0
0 0.0359083414077759 38.0
0 0.000788331031799316 39.0
0 0.0467208623886108 40.0
0 -0.036851167678833 41.0
0 -0.0447651147842407 42.0
286.67563441 297.979217529297 43.0
0 -0.0474907159805298 44.0
0 0.00507187843322754 45.0
0 -0.0417262315750122 46.0
0 -0.0702940225601196 47.0
0 -0.035253643989563 48.0
2.8441207667 0.3280029296875 49.0
0.09403044084 0.0703977346420288 50.0
1.852969128 1.00296878814697 51.0
0 -0.065367579460144 52.0
0 -0.0339814424514771 53.0
0 0.0236774682998657 54.0
0 -0.111520648002625 55.0
0 -0.0727311372756958 56.0
0 0.000338435173034668 57.0
0 -0.0436638593673706 58.0
0 -0.0829592943191528 59.0
0 -0.0426064729690552 60.0
0 -0.0227967500686646 61.0
0 0.822863101959229 62.0
259.76311628 258.528015136719 0.0
0 -0.0816713571548462 1.0
0 0.0226548910140991 2.0
0 -0.0350188016891479 3.0
0 -0.0642787218093872 4.0
0 -0.101110816001892 5.0
0 0.00323379039764404 6.0
0 -0.0958412885665894 7.0
0 -0.0956841707229614 8.0
0 0.013785719871521 9.0
0 -0.00907528400421143 10.0
0 -0.0390263795852661 11.0
0 -0.102989792823792 12.0
59.980435683 58.6242179870605 13.0
40.125500029 37.7891654968262 14.0
0 -0.0513979196548462 15.0
0 -0.00625574588775635 16.0
36.309769656 36.321605682373 17.0
0 0.0318702459335327 18.0
0 -0.0168584585189819 19.0
0 -0.0281199216842651 20.0
0 -0.0151900053024292 21.0
0 -0.0213419198989868 22.0
33.802271857 33.5781745910645 23.0
0 -0.106656193733215 24.0
0 0.714805424213409 25.0
0 0.0791162252426147 26.0
0 0.00129592418670654 27.0
0 0.0127609968185425 28.0
0 -0.127725005149841 29.0
0 -0.0449231863021851 30.0
0 -0.0522161722183228 31.0
0 0.0188006162643433 32.0
0 0.0877083539962769 33.0
0 -0.0144096612930298 34.0
0 -0.055472731590271 35.0
0 0.0071183443069458 36.0
0 -0.0358825922012329 37.0
0 0.00488054752349854 38.0
0 0.035319447517395 39.0
0 0.0995525121688843 40.0
0 -0.0168323516845703 41.0
0 0.038998007774353 42.0
306.11527063 294.660705566406 43.0
0 -0.0199738740921021 44.0
0 0.0230250358581543 45.0
0 -0.0473181009292603 46.0
0 -0.0566263198852539 47.0
0 0.0251110792160034 48.0
0 0.261084347963333 49.0
0.090106932242 0.0719610452651978 50.0
2.2742264445 1.01946616172791 51.0
0 -0.0493072271347046 52.0
0 -0.0468844175338745 53.0
0 0.00699055194854736 54.0
0 -0.10178816318512 55.0
0 -0.0303713083267212 56.0
0 -0.0230447053909302 57.0
0 0.00590384006500244 58.0
0 -0.0728167295455933 59.0
0 -0.0475591421127319 60.0
0 0.00696814060211182 61.0
0 0.835038661956787 62.0
273.58449628 252.731491088867 0.0
0 -0.0230900049209595 1.0
0 -0.0289677381515503 2.0
0 0.00995337963104248 3.0
0 -0.0675243139266968 4.0
0 -0.178992867469788 5.0
0 -0.0605837106704712 6.0
0 -0.117348790168762 7.0
0 0.0178649425506592 8.0
0 -0.00069582462310791 9.0
0 0.039171576499939 10.0
0 0.0384832322597504 11.0
0 -0.0079275369644165 12.0
71.739535599 71.6738967895508 13.0
36.289904094 24.5955371856689 14.0
0 -0.0128799676895142 15.0
0 -0.0191255807876587 16.0
38.336049513 38.413028717041 17.0
0 0.00447428226470947 18.0
0 -0.0195854902267456 19.0
0 0.0285755395889282 20.0
0 -0.0205656290054321 21.0
0 -0.0561710596084595 22.0
37.069804395 36.5179977416992 23.0
0 -0.0445338487625122 24.0
0 0.803491652011871 25.0
0 0.122345805168152 26.0
0 0.0454806089401245 27.0
0 -0.0380698442459106 28.0
0 -0.068724513053894 29.0
0 -0.041174054145813 30.0
0 -0.0112851858139038 31.0
0 -0.0203162431716919 32.0
0 0.0631574392318726 33.0
0 -0.0136336088180542 34.0
0 -0.00547230243682861 35.0
0 0.0760191679000854 36.0
0 -0.0361694097518921 37.0
0 -0.0341523885726929 38.0
0 0.040382981300354 39.0
0 0.0951169729232788 40.0
0 -0.0699261426925659 41.0
0 0.00816380977630615 42.0
289.85898582 296.023101806641 43.0
0 -0.0212236642837524 44.0
0 0.0254182815551758 45.0
0 0.00723040103912354 46.0
0 -0.0997309684753418 47.0
0 -0.0130206346511841 48.0
0 0.273079991340637 49.0
0.082269033937 0.0985735654830933 50.0
2.1728415722 1.00126123428345 51.0
0 -0.0461219549179077 52.0
0 -0.0651699304580688 53.0
0 -0.012325644493103 54.0
0 -0.0413438081741333 55.0
0 0.00429332256317139 56.0
0 -0.0486832857131958 57.0
0 -0.0678929090499878 58.0
0 -0.0783756971359253 59.0
0 0.040406346321106 60.0
0 0.0644413232803345 61.0
0 0.892603635787964 62.0
255.3310723 252.303787231445 0.0
0 -0.0248996019363403 1.0
0 -0.0830453634262085 2.0
0 0.00593578815460205 3.0
0 -0.0402671098709106 4.0
0 -0.0266591310501099 5.0
0 0.0211385488510132 6.0
0 0.0383914709091187 7.0
0 -0.052422046661377 8.0
0 0.0133651494979858 9.0
0 -0.133942246437073 10.0
0 0.0222070217132568 11.0
0 0.088823676109314 12.0
68.725793606 67.4761276245117 13.0
32.966682747 30.7041816711426 14.0
0 -0.0212393999099731 15.0
0 -0.0100089311599731 16.0
35.656370849 35.0315780639648 17.0
0 0.00178277492523193 18.0
0 -0.043378472328186 19.0
0 0.0748449563980103 20.0
0 0.0280910730361938 21.0
0 -0.0796266794204712 22.0
39.390824651 38.3666801452637 23.0
0 -0.0701457262039185 24.0
0 0.557254672050476 25.0
0 -0.0351606607437134 26.0
0 0.0698086023330688 27.0
0 -0.0126086473464966 28.0
0 -0.0427964925765991 29.0
0 -0.0402225255966187 30.0
0 -0.044115424156189 31.0
0 0.135268092155457 32.0
0 0.0802744626998901 33.0
0 -0.0907877683639526 34.0
0 0.231047034263611 35.0
0 0.0701643228530884 36.0
0 -0.0809599161148071 37.0
0 -0.0246788263320923 38.0
0 0.0273932218551636 39.0
0 -0.0194026231765747 40.0
0 -0.0234595537185669 41.0
0 -0.0287691354751587 42.0
297.47305988 293.107116699219 43.0
0 -0.0509029626846313 44.0
0 0.0631846189498901 45.0
0 -0.0011671781539917 46.0
0 -0.0720165967941284 47.0
0 -0.00354301929473877 48.0
0 0.279099524021149 49.0
0.085130585563 0.0645612478256226 50.0
2.1527465318 0.90686821937561 51.0
0 -0.0421532392501831 52.0
0 -0.0461698770523071 53.0
0 0.0485073328018188 54.0
0 -0.0710636377334595 55.0
0 -0.00173556804656982 56.0
0 -0.0460413694381714 57.0
0 -0.0719138383865356 58.0
0 -0.0448960065841675 59.0
0 -0.0563820600509644 60.0
0 -0.0226694345474243 61.0
0 0.81630265712738 62.0
230.6589112 254.059158325195 0.0
0 -0.075442910194397 1.0
0 -0.0395487546920776 2.0
0 0.00657498836517334 3.0
0 -0.0341340303421021 4.0
0 0.1302889585495 5.0
0 0.0391992330551147 6.0
0 -0.0395020246505737 7.0
0 0.00626218318939209 8.0
0 -0.0291706323623657 9.0
0 -0.0630968809127808 10.0
0 -0.0324290990829468 11.0
0 -0.0415598154067993 12.0
52.067495891 51.7773551940918 13.0
40.420871628 39.1680183410645 14.0
0 -0.00420725345611572 15.0
0 -0.0545564889907837 16.0
29.560458525 29.0495376586914 17.0
0 0.00477206707000732 18.0
0 -0.0219272375106812 19.0
0 0.0274475812911987 20.0
0 -0.0124228000640869 21.0
0 -0.0767017602920532 22.0
41.093404166 41.0005302429199 23.0
0 -0.0784603357315063 24.0
0 0.605883717536926 25.0
0 0.0588735342025757 26.0
0 0.0323771238327026 27.0
0 0.0338739156723022 28.0
0 -0.054834246635437 29.0
0 -0.0863020420074463 30.0
0 -0.0111786127090454 31.0
0 0.00170266628265381 32.0
0 0.00978529453277588 33.0
0 0.00530326366424561 34.0
0 -0.0299371480941772 35.0
0 0.0359004735946655 36.0
0 -0.0278853178024292 37.0
0 -0.00693404674530029 38.0
0 -0.0122545957565308 39.0
0 0.0422309637069702 40.0
0 0.0488268136978149 41.0
0 -0.0427013635635376 42.0
292.01538521 293.421936035156 43.0
0 -0.0673443078994751 44.0
0 0.00895589590072632 45.0
0 -0.0269747972488403 46.0
0 -0.0454474687576294 47.0
0 -0.0793968439102173 48.0
0.18665609771 0.261996626853943 49.0
0.10455569011 0.10184645652771 50.0
1.4587178627 0.942134976387024 51.0
0 -0.0462607145309448 52.0
0 -0.0325115919113159 53.0
0 0.0797172784805298 54.0
0 -0.0421677827835083 55.0
0 -0.0787860155105591 56.0
0 0.00515735149383545 57.0
0 -0.0542508363723755 58.0
0 -0.0231596231460571 59.0
0 -0.0671995878219604 60.0
0 0.0125993490219116 61.0
0 0.806474924087524 62.0
256.91643352 254.664642333984 0.0
0 -0.0937200784683228 1.0
0 0.0176795721054077 2.0
0 0.0182119607925415 3.0
0 -0.0704134702682495 4.0
0 0.0190180540084839 5.0
0 -0.116382241249084 6.0
0 -0.0512276887893677 7.0
0 -0.0180058479309082 8.0
0 0.0239275693893433 9.0
0 -0.0320206880569458 10.0
0 -0.0285141468048096 11.0
0 -0.00875246524810791 12.0
54.374880093 53.0913009643555 13.0
34.997839413 41.9307861328125 14.0
0 -0.0848451852798462 15.0
0 0.00916206836700439 16.0
32.398470798 32.2397422790527 17.0
0 0.0262573957443237 18.0
0 -0.0217822790145874 19.0
0 -0.0441325902938843 20.0
0 -0.0534936189651489 21.0
0 -0.000842690467834473 22.0
34.515914595 34.0989570617676 23.0
0 -0.0547815561294556 24.0
0 1.53241193294525 25.0
0 0.0281451940536499 26.0
0 0.00763273239135742 27.0
0 -0.016487717628479 28.0
0 -0.0110379457473755 29.0
0 -0.0761218070983887 30.0
0 -0.0701380968093872 31.0
0 0.0257176160812378 32.0
0 0.0286635160446167 33.0
0 -0.026495099067688 34.0
0 0.0253646373748779 35.0
0 -0.102770686149597 36.0
0 -0.0212570428848267 37.0
0 -0.0282727479934692 38.0
0 0.0851510763168335 39.0
0 0.105461001396179 40.0
0 -0.0367151498794556 41.0
0 0.0244563817977905 42.0
300.16829508 295.479827880859 43.0
0 -0.0230695009231567 44.0
0 0.0470874309539795 45.0
0 -0.0356582403182983 46.0
0 -0.0301781892776489 47.0
0 -0.00893270969390869 48.0
0 0.284135639667511 49.0
0.095201597017 0.0829211473464966 50.0
2.1267604477 0.896848082542419 51.0
0 -0.0752896070480347 52.0
0 -0.0326262712478638 53.0
0 -0.00450384616851807 54.0
0 -0.0438088178634644 55.0
0 -0.070381760597229 56.0
0 -0.0629128217697144 57.0
0 -0.00658524036407471 58.0
0 -0.0349327325820923 59.0
0 0.0833269357681274 60.0
0 -0.0296169519424438 61.0
0 0.792204022407532 62.0
244.15121697 256.079071044922 0.0
0 0.0120648145675659 1.0
0 -0.0414131879806519 2.0
0 0.0177613496780396 3.0
0 -0.0580567121505737 4.0
0 -0.0688468217849731 5.0
0 -0.0576845407485962 6.0
0 -0.046862006187439 7.0
0 -0.0452392101287842 8.0
0 -0.0143142938613892 9.0
0 -0.014346718788147 10.0
0 -0.0163666009902954 11.0
0 0.0695289373397827 12.0
57.201152815 55.4296646118164 13.0
36.618320941 39.6429405212402 14.0
0 -0.0381580591201782 15.0
0 -0.0586329698562622 16.0
35.737651117 35.3213806152344 17.0
0 0.00131833553314209 18.0
0 0.032823920249939 19.0
0 0.0274726152420044 20.0
0 -0.0488699674606323 21.0
0 0.000582575798034668 22.0
33.328296936 32.7268333435059 23.0
0 0.0813044309616089 24.0
0 0.448258936405182 25.0
0 0.0778583288192749 26.0
0 0.0532559156417847 27.0
0 -0.00635635852813721 28.0
0 -0.0534564256668091 29.0
0 -0.0681501626968384 30.0
0 -0.0326515436172485 31.0
0 -0.0201917886734009 32.0
0 0.112729430198669 33.0
0 -0.0300039052963257 34.0
0 0.045487642288208 35.0
0 0.0203555822372437 36.0
0 0.0215941667556763 37.0
0 0.00799596309661865 38.0
0 0.110724091529846 39.0
0 0.126963496208191 40.0
0 -0.0414377450942993 41.0
0 0.021715521812439 42.0
290.09975398 296.964202880859 43.0
0 -0.00718176364898682 44.0
0 0.0464408993721008 45.0
0 0.0061877965927124 46.0
0 -0.0551456212997437 47.0
0 0.0142611265182495 48.0
0 0.241777181625366 49.0
0.086474966022 0.0687078237533569 50.0
2.2054620147 0.898462533950806 51.0
0 -0.0743314027786255 52.0
0 0.000525593757629395 53.0
0 -0.00184094905853271 54.0
0 -0.0894006490707397 55.0
0 -0.0403822660446167 56.0
0 -0.000743269920349121 57.0
0 -0.0315872430801392 58.0
0 -0.0457116365432739 59.0
0 -0.00397622585296631 60.0
0 0.0422497987747192 61.0
0 0.792007803916931 62.0
267.84318832 261.179748535156 0.0
0 -0.0860220193862915 1.0
0 -0.0810390710830688 2.0
0 0.00184118747711182 3.0
0 -0.00427162647247314 4.0
0 0.0228389501571655 5.0
0 -0.215880990028381 6.0
0 -0.0795711278915405 7.0
0 -0.0468496084213257 8.0
0 0.0676006078720093 9.0
0 0.0228291749954224 10.0
0 -0.00480747222900391 11.0
0 -0.153815865516663 12.0
57.727400957 57.3203430175781 13.0
41.444529192 39.1347579956055 14.0
0 -0.0157724618911743 15.0
0 -0.0306359529495239 16.0
33.959410074 33.7640380859375 17.0
0 -0.0311235189437866 18.0
0 0.0272282361984253 19.0
0 -0.0105103254318237 20.0
0 -0.0511223077774048 21.0
0 0.0294464826583862 22.0
26.137877313 25.9149208068848 23.0
0 -0.0106483697891235 24.0
0 0.548742234706879 25.0
0 0.0929754972457886 26.0
0 -0.0349401235580444 27.0
0 0.0793150663375854 28.0
0 -0.0773166418075562 29.0
0 -0.0894731283187866 30.0
0 -0.0125025510787964 31.0
0 -0.0635768175125122 32.0
0 -0.00236666202545166 33.0
0 -0.0454543828964233 34.0
0 -0.0754450559616089 35.0
0 -0.0548480749130249 36.0
0 -0.00043785572052002 37.0
0 0.00283849239349365 38.0
0 0.042906641960144 39.0
0 -0.126530289649963 40.0
0 -0.0466197729110718 41.0
0 0.00834882259368896 42.0
314.75526169 296.321960449219 43.0
0 0.0107053518295288 44.0
0 -0.0146738290786743 45.0
0 -0.00054776668548584 46.0
0 -0.034218430519104 47.0
0 -0.02900230884552 48.0
0 0.272864937782288 49.0
0.070525096769 0.0581914186477661 50.0
2.1869748781 1.03924453258514 51.0
0 -0.0772911310195923 52.0
0 -0.0388966798782349 53.0
0 0.0132983922958374 54.0
0 -0.102957367897034 55.0
0 -0.0903877019882202 56.0
0 0.0316489934921265 57.0
0 0.0125676393508911 58.0
0 -0.0142925977706909 59.0
0 -0.0190283060073853 60.0
0 0.0602649450302124 61.0
0 0.924300789833069 62.0
289.26783722 256.803741455078 0.0
0 -0.0670000314712524 1.0
0 -0.0649808645248413 2.0
0 -0.0170086622238159 3.0
0 -0.060771107673645 4.0
0 -0.158619046211243 5.0
0 0.032427191734314 6.0
0 -0.0679699182510376 7.0
0 -0.00481522083282471 8.0
0 0.0179437398910522 9.0
0 -0.0288602113723755 10.0
0 -0.0254384279251099 11.0
0 -0.0173937082290649 12.0
52.878613904 52.1824073791504 13.0
38.986217854 42.7764663696289 14.0
0 -0.0596939325332642 15.0
0 -0.00586116313934326 16.0
30.489808975 30.8348941802979 17.0
0 -0.0433262586593628 18.0
0 -0.0272215604782104 19.0
0 0.0356928110122681 20.0
0 -0.027513861656189 21.0
0 0.00437057018280029 22.0
28.77247142 27.9166851043701 23.0
0 -0.0425883531570435 24.0
0 0.589669346809387 25.0
0 0.0280042886734009 26.0
0 -0.0042959451675415 27.0
0 -0.0455247163772583 28.0
0 -0.00836074352264404 29.0
0 -0.0721447467803955 30.0
0 -0.0311509370803833 31.0
0 -0.0732985734939575 32.0
0 0.0151342153549194 33.0
0 -0.0683082342147827 34.0
0 -0.000439286231994629 35.0
0 -0.00447070598602295 36.0
0 -0.0308107137680054 37.0
0 -0.0557905435562134 38.0
0 0.0324043035507202 39.0
0 0.151025652885437 40.0
0 0.000149369239807129 41.0
0 0.0350269079208374 42.0
297.18569359 296.825988769531 43.0
0 0.00428330898284912 44.0
0 0.0204648971557617 45.0
0 -0.0421541929244995 46.0
0 -0.04716956615448 47.0
0 -0.0197597742080688 48.0
0.28612007582 0.258867621421814 49.0
0.072037425039 0.112055420875549 50.0
2.0593312865 0.894084393978119 51.0
0 -0.0323745012283325 52.0
0 -0.0324491262435913 53.0
0 0.0151990652084351 54.0
0 -0.0637568235397339 55.0
0 -0.00297558307647705 56.0
0 -0.0246363878250122 57.0
0 -0.0629819631576538 58.0
0 -0.0866883993148804 59.0
0 -0.0176910161972046 60.0
0 0.0597094297409058 61.0
0 0.943932890892029 62.0
233.97170705 259.984619140625 0.0
0 -0.0439015626907349 1.0
0 0.0311049222946167 2.0
0 0.0165129899978638 3.0
0 -0.0650004148483276 4.0
0 0.0616554021835327 5.0
0 0.0507529973983765 6.0
0 -0.0312288999557495 7.0
0 -0.0411348342895508 8.0
0 0.0563873052597046 9.0
0 0.000851988792419434 10.0
0 -0.00974583625793457 11.0
0 -0.0360265970230103 12.0
51.823540481 48.4248008728027 13.0
40.010832852 41.2103881835938 14.0
0 -0.0051342248916626 15.0
0 -0.0743395090103149 16.0
40.487353553 39.5616111755371 17.0
0 0.0957502126693726 18.0
0 0.0910137891769409 19.0
0 -0.0258022546768188 20.0
0 0.00285804271697998 21.0
0 -0.022779107093811 22.0
26.191972839 25.9397506713867 23.0
0 -0.107077240943909 24.0
0 0.594505667686462 25.0
0 0.019830584526062 26.0
0 0.110459446907043 27.0
0 0.0371249914169312 28.0
0 -0.0647777318954468 29.0
0 0.126471638679504 30.0
0 -0.136766791343689 31.0
0 -0.0295363664627075 32.0
0 0.0970948934555054 33.0
0 -0.0333939790725708 34.0
0 0.062518835067749 35.0
0 0.104031920433044 36.0
0 0.010032057762146 37.0
0 0.0285979509353638 38.0
0 -0.139124035835266 39.0
0 -0.165270924568176 40.0
0 0.206907391548157 41.0
0 -0.0802561044692993 42.0
290.76195089 295.125671386719 43.0
0 -0.0465575456619263 44.0
0 -0.0213236808776855 45.0
0 -0.0186237096786499 46.0
0 -0.0530613660812378 47.0
0 0.0922878980636597 48.0
0 0.265908658504486 49.0
0.10372269847 0.104276299476624 50.0
1.5850435502 1.02495646476746 51.0
0 -0.021126389503479 52.0
0 -0.0658289194107056 53.0
0 -0.0419162511825562 54.0
0 -0.0566993951797485 55.0
0 -0.0315573215484619 56.0
0 -0.0507732629776001 57.0
0 -0.0346969366073608 58.0
0 -0.0220023393630981 59.0
0 -0.0910123586654663 60.0
0 -0.0493687391281128 61.0
0 0.872081160545349 62.0
267.59447758 256.340209960938 0.0
0 0.0315202474594116 1.0
0 -0.0252233743667603 2.0
0 -0.0257352590560913 3.0
0 -0.090448260307312 4.0
0 0.00742757320404053 5.0
0 0.00726974010467529 6.0
0 9.90629196166992e-05 7.0
0 -0.0228630900382996 8.0
0 -0.0876013040542603 9.0
0 0.0120774507522583 10.0
0 -0.0107306241989136 11.0
0 -0.130359292030334 12.0
59.547623861 58.2777862548828 13.0
41.228472187 38.7091903686523 14.0
0 -0.0282903909683228 15.0
0 -0.0567761659622192 16.0
36.228725719 35.5999450683594 17.0
0 0.0416332483291626 18.0
0 0.0265723466873169 19.0
0 -0.0157846212387085 20.0
0 -0.0333033800125122 21.0
0 0.0350064039230347 22.0
32.949529401 31.6356678009033 23.0
0 -0.0292378664016724 24.0
0 0.655496716499329 25.0
0 -0.0253716707229614 26.0
0 -0.00365841388702393 27.0
0 0.0494855642318726 28.0
0 -0.0511246919631958 29.0
0 -0.0675774812698364 30.0
0 -0.0154792070388794 31.0
0 0.000677943229675293 32.0
0 -0.0044940710067749 33.0
0 -0.0162383317947388 34.0
0 -0.0643454790115356 35.0
0 -0.00946009159088135 36.0
0 -0.10639226436615 37.0
0 0.0284472703933716 38.0
0 0.0156658887863159 39.0
0 0.0913890600204468 40.0
0 -0.0734080076217651 41.0
0 0.00313317775726318 42.0
303.55499756 296.746002197266 43.0
0 0.0366300344467163 44.0
0 0.0364229083061218 45.0
0 0.00413262844085693 46.0
0 -0.0633901357650757 47.0
0 -0.00691139698028564 48.0
0 0.294759035110474 49.0
0.083083736066 0.0625878572463989 50.0
2.1818918748 0.957060694694519 51.0
0 -0.0727211236953735 52.0
0 -0.0193699598312378 53.0
0 0.0361858606338501 54.0
0 -0.124106764793396 55.0
0 -0.0723365545272827 56.0
0 -0.00656211376190186 57.0
0 -0.0482555627822876 58.0
0 -0.0192633867263794 59.0
0 -0.0196212530136108 60.0
0 0.0194178819656372 61.0
0 0.821311235427856 62.0
268.15967924 260.851684570312 0.0
0 -0.0730835199356079 1.0
0 -0.0759838819503784 2.0
0 -0.013032078742981 3.0
0 -0.0322252511978149 4.0
0 0.0198525190353394 5.0
0 -0.0271726846694946 6.0
0 -0.0432358980178833 7.0
0 0.00978362560272217 8.0
0 -0.00519716739654541 9.0
0 0.0161112546920776 10.0
0 -0.0351169109344482 11.0
0 -0.0204273462295532 12.0
67.023692763 66.2522048950195 13.0
35.795601632 32.1708106994629 14.0
0 -0.0750662088394165 15.0
0 -0.0119668245315552 16.0
36.646952669 36.0101013183594 17.0
0 0.0137039422988892 18.0
0 0.0203267335891724 19.0
0 0.0322462320327759 20.0
0 -0.0239008665084839 21.0
0 -0.0153976678848267 22.0
36.47325566 36.3316612243652 23.0
0 -0.0526787042617798 24.0
0 0.683931589126587 25.0
0 0.0640634298324585 26.0
0 0.00546669960021973 27.0
0 0.0106900930404663 28.0
0 -0.0245965719223022 29.0
0 -0.0281429290771484 30.0
0 -0.00318276882171631 31.0
0 -0.00846827030181885 32.0
0 0.0595391988754272 33.0
0 0.00557816028594971 34.0
0 0.0684375762939453 35.0
0 -0.0506900548934937 36.0
0 0.0173100233078003 37.0
0 0.0187209844589233 38.0
0 0.0731354951858521 39.0
0 -0.0723510980606079 40.0
0 -0.0520864725112915 41.0
0 0.00979745388031006 42.0
304.64355556 298.087890625 43.0
0 0.0315216779708862 44.0
0 0.0606027841567993 45.0
0 -0.0223971605300903 46.0
0 -0.0713692903518677 47.0
0 -0.0206786394119263 48.0
0 0.317651987075806 49.0
0.068784346755 0.060564398765564 50.0
1.6468977015 0.994338095188141 51.0
0 -0.0830334424972534 52.0
0 -0.00049889087677002 53.0
0 0.0337785482406616 54.0
0 -0.0879055261611938 55.0
0 -0.0185846090316772 56.0
0 0.0183957815170288 57.0
0 0.00796687602996826 58.0
0 -0.0367563962936401 59.0
0 0.084173321723938 60.0
0 -0.0381332635879517 61.0
0 0.831258773803711 62.0
250.91901102 252.746688842773 0.0
0 -0.0696583986282349 1.0
0 -0.042075514793396 2.0
0 -0.0214790105819702 3.0
0 -0.0696572065353394 4.0
0 -0.0576170682907104 5.0
0 -0.0642162561416626 6.0
0 -0.0915385484695435 7.0
0 -0.0157073736190796 8.0
0 -0.00534546375274658 9.0
0 -0.0422250032424927 10.0
0 -0.0342796444892883 11.0
0 -0.0107866525650024 12.0
64.888911081 64.6159210205078 13.0
23.468201469 35.55810546875 14.0
0 0.0176037549972534 15.0
0 -0.0267332792282104 16.0
31.837037613 31.5814971923828 17.0
0 -0.0137051343917847 18.0
0 -0.0060504674911499 19.0
0 0.0195835828781128 20.0
0 -0.000751376152038574 21.0
0 0.00279176235198975 22.0
27.739125931 27.2649459838867 23.0
0 0.0303815603256226 24.0
0 0.524038791656494 25.0
0 0.0619505643844604 26.0
0 0.130222916603088 27.0
0 -0.0255118608474731 28.0
0 -0.0445796251296997 29.0
0 -0.0282517671585083 30.0
0 -0.0485993623733521 31.0
0 -0.0207411050796509 32.0
0 0.0291155576705933 33.0
0 -0.0351234674453735 34.0
0 -0.0641218423843384 35.0
0 0.0405584573745728 36.0
0 -0.0353642702102661 37.0
0 0.0379840135574341 38.0
0 0.00715148448944092 39.0
0 -0.0328055620193481 40.0
0 -0.0190588235855103 41.0
0 0.00134503841400146 42.0
287.88521548 295.462463378906 43.0
0 -0.0509964227676392 44.0
0 0.0263135433197021 45.0
0 -0.0102630853652954 46.0
0 -0.0701929330825806 47.0
0 0.0154322385787964 48.0
0 0.282686740159988 49.0
0.10046419202 0.0433398485183716 50.0
2.2455310135 0.938373029232025 51.0
0 -0.0720099210739136 52.0
0 -0.0548955202102661 53.0
0 0.00501406192779541 54.0
0 -0.0721396207809448 55.0
0 -0.024696946144104 56.0
0 -0.015755295753479 57.0
0 -0.0740717649459839 58.0
0 -0.0351310968399048 59.0
0 -0.0247601270675659 60.0
0 -0.0284327268600464 61.0
0 0.837926030158997 62.0
234.64110996 255.433624267578 0.0
0 -0.0832005739212036 1.0
0 0.0437880754470825 2.0
0 -0.0333300828933716 3.0
0 -0.0336848497390747 4.0
0 -0.131815075874329 5.0
0 -0.008156418800354 6.0
0 -0.0494030714035034 7.0
0 0.0381226539611816 8.0
0 0.0405706167221069 9.0
0 -0.0312908887863159 10.0
0 -0.0108596086502075 11.0
0 -0.0646606683731079 12.0
62.498959374 61.9302787780762 13.0
26.562161582 37.4181327819824 14.0
0 -0.0629101991653442 15.0
0 0.026943564414978 16.0
37.924949753 37.8162956237793 17.0
0 -0.0169442892074585 18.0
0 0.00812041759490967 19.0
0 -0.0277301073074341 20.0
0 0.0160657167434692 21.0
0 0.0365055799484253 22.0
32.203883507 31.6360034942627 23.0
0 -0.0966345071792603 24.0
0 0.62368106842041 25.0
0 0.0715774297714233 26.0
0 0.0319308042526245 27.0
0 -0.0271130800247192 28.0
0 -0.0203207731246948 29.0
0 -0.0420759916305542 30.0
0 -0.0627435445785522 31.0
0 -0.0499740839004517 32.0
0 -0.000618338584899902 33.0
0 -0.0588468313217163 34.0
0 -0.0279185771942139 35.0
0 -0.0662437677383423 36.0
0 -0.0441137552261353 37.0
0 -0.0582665205001831 38.0
0 0.137279391288757 39.0
0 0.121419310569763 40.0
0 -0.0155948400497437 41.0
0 0.0116647481918335 42.0
289.35397848 297.550018310547 43.0
0 0.0185130834579468 44.0
0 0.0542775988578796 45.0
0 -0.003928542137146 46.0
0 -0.0546141862869263 47.0
0 -0.0110665559768677 48.0
0 0.308794647455215 49.0
0.079800267759 0.0717266798019409 50.0
2.0561708833 0.989347159862518 51.0
0 -0.0871821641921997 52.0
0 -0.0900527238845825 53.0
0 0.00548088550567627 54.0
0 -0.0384629964828491 55.0
0 0.011694073677063 56.0
0 -0.0221363306045532 57.0
0 -0.0369099378585815 58.0
0 -0.0593808889389038 59.0
0 -0.0307344198226929 60.0
0 -0.057486891746521 61.0
0 0.809870004653931 62.0
239.90514893 255.258560180664 0.0
0 -0.0924321413040161 1.0
0 -0.0633529424667358 2.0
0 -0.0128217935562134 3.0
0 -0.0312031507492065 4.0
0 0.000912308692932129 5.0
0 -0.147385239601135 6.0
0 -0.0474098920822144 7.0
0 -0.0649737119674683 8.0
0 0.0401748418807983 9.0
0 -0.0177382230758667 10.0
0 0.00260937213897705 11.0
0 -0.0301557779312134 12.0
52.337781603 51.7683982849121 13.0
46.169391446 43.6898803710938 14.0
0 -0.120079636573792 15.0
0 -0.0029522180557251 16.0
40.0065843 39.8356704711914 17.0
0 -0.0478395223617554 18.0
0 0.0170022249221802 19.0
0 -0.0313700437545776 20.0
0 0.00840151309967041 21.0
0 0.0167573690414429 22.0
29.790150676 29.087574005127 23.0
0 -0.00680744647979736 24.0
0 0.584058284759521 25.0
0 0.0425890684127808 26.0
0 0.0283735990524292 27.0
0 -0.0697423219680786 28.0
0 -0.0734845399856567 29.0
0 -0.0786498785018921 30.0
0 -0.0139302015304565 31.0
0 0.00300586223602295 32.0
0 0.072205662727356 33.0
0 -0.00547635555267334 34.0
0 -0.0241659879684448 35.0
0 0.117101550102234 36.0
0 0.00593793392181396 37.0
0 -0.028577446937561 38.0
0 0.08314049243927 39.0
0 0.0955837965011597 40.0
0 -0.0330318212509155 41.0
0 0.0306504964828491 42.0
290.9599611 294.90185546875 43.0
0 0.00323235988616943 44.0
0 0.0324477553367615 45.0
0 -0.00252926349639893 46.0
0 -0.068548321723938 47.0
0 0.0206331014633179 48.0
0 0.291546821594238 49.0
0.078253458353 0.0638829469680786 50.0
1.571422133 0.981956481933594 51.0
0 -0.0937041044235229 52.0
0 -0.0561524629592896 53.0
0 0.0562554597854614 54.0
0 -0.0526624917984009 55.0
0 -0.0297139883041382 56.0
0 0.0271519422531128 57.0
0 0.0124107599258423 58.0
0 -0.0655919313430786 59.0
0 -0.10861337184906 60.0
0 0.0124107599258423 61.0
0 0.891276240348816 62.0
266.26660172 255.260818481445 0.0
0 -0.011910080909729 1.0
0 -0.0122557878494263 2.0
0 0.000745415687561035 3.0
0 -0.00338542461395264 4.0
0 0.0381416082382202 5.0
0 -0.129787087440491 6.0
0 -0.0263439416885376 7.0
0 -0.0684268474578857 8.0
0 0.0982860326766968 9.0
0 0.00804603099822998 10.0
0 0.0299900770187378 11.0
0 -0.0906845331192017 12.0
68.044549987 68.1633987426758 13.0
36.453784743 26.5240306854248 14.0
0 -0.100785851478577 15.0
0 0.0344833135604858 16.0
27.876166939 26.5721664428711 17.0
0 0.0379637479782104 18.0
0 0.092499852180481 19.0
0 0.0179773569107056 20.0
0 0.0194100141525269 21.0
0 -0.0308490991592407 22.0
30.561186441 30.4535045623779 23.0
0 0.0456842184066772 24.0
0 0.492878794670105 25.0
0 0.0571161508560181 26.0
0 0.0372967123985291 27.0
0 0.070677638053894 28.0
0 -0.113131642341614 29.0
0 -0.0668522119522095 30.0
0 0.0113362073898315 31.0
0 -0.0373483896255493 32.0
0 0.0987311601638794 33.0
0 -0.133390784263611 34.0
0 -0.0750913619995117 35.0
0 -0.0987728834152222 36.0
0 -0.0266751050949097 37.0
0 -0.010900616645813 38.0
0 0.0927423238754272 39.0
0 0.201226592063904 40.0
0 -0.00259959697723389 41.0
0 0.0101865530014038 42.0
290.74070957 299.140289306641 43.0
0 -0.0178600549697876 44.0
0 0.0558050870895386 45.0
0 -0.0534483194351196 46.0
0 -0.00437867641448975 47.0
0 0.0393470525741577 48.0
0 0.296787887811661 49.0
0.089725810217 0.0694540739059448 50.0
1.5642982562 1.01639938354492 51.0
0 -0.0474718809127808 52.0
0 -0.0423485040664673 53.0
0 0.00690734386444092 54.0
0 -0.124397397041321 55.0
0 0.00565612316131592 56.0
0 -0.0346609354019165 57.0
0 -0.0663453340530396 58.0
0 -0.0478252172470093 59.0
0 0.014682412147522 60.0
0 0.00368320941925049 61.0
0 0.848947525024414 62.0
228.80937746 257.200347900391 0.0
0 -0.0391660928726196 1.0
0 -0.0606180429458618 2.0
0 0.00390231609344482 3.0
0 -0.0298973321914673 4.0
0 -0.0464435815811157 5.0
0 -0.0511099100112915 6.0
0 -0.0305081605911255 7.0
0 -0.0607571601867676 8.0
0 0.0163291692733765 9.0
0 -0.025887131690979 10.0
0 -0.0438133478164673 11.0
0 0.0413650274276733 12.0
55.28209034 53.9537391662598 13.0
43.691123642 40.8898162841797 14.0
0 -0.0451608896255493 15.0
0 -0.00707328319549561 16.0
31.536494247 31.4962902069092 17.0
0 0.0233701467514038 18.0
0 0.00283873081207275 19.0
0 0.00158441066741943 20.0
0 -0.0499118566513062 21.0
0 0.0219014883041382 22.0
40.151034909 39.454662322998 23.0
0 -0.0368086099624634 24.0
0 0.50217467546463 25.0
0 -0.00102579593658447 26.0
0 0.0279093384742737 27.0
0 0.0161522626876831 28.0
0 -0.0416332483291626 29.0
0 -0.0930289030075073 30.0
0 -0.0408333539962769 31.0
0 0.0155047178268433 32.0
0 0.109081149101257 33.0
0 -0.0269192457199097 34.0
0 -0.101030826568604 35.0
0 -0.0539525747299194 36.0
0 -0.0618880987167358 37.0
0 0.0349467992782593 38.0
0 0.0733073949813843 39.0
0 0.0299719572067261 40.0
0 -0.0719305276870728 41.0
0 -0.0134245157241821 42.0
298.01936031 294.951629638672 43.0
0 0.00801026821136475 44.0
0 0.00411605834960938 45.0
0 -0.0250862836837769 46.0
0 -0.0468598604202271 47.0
0 0.034725546836853 48.0
0 0.313819766044617 49.0
0.091773301616 0.064012885093689 50.0
2.1862461828 0.919539093971252 51.0
0 -0.0794914960861206 52.0
0 -0.0248595476150513 53.0
0 0.0200036764144897 54.0
0 -0.0924988985061646 55.0
0 -0.0219477415084839 56.0
0 -0.016907811164856 57.0
0 -0.00706827640533447 58.0
0 -0.0720819234848022 59.0
0 -0.0520340204238892 60.0
0 -0.0250524282455444 61.0
0 0.862472534179688 62.0
294.17577292 253.41325378418 0.0
0 -0.0368739366531372 1.0
0 0.00394737720489502 2.0
0 -0.0818110704421997 3.0
0 -0.00434863567352295 4.0
0 -0.0112005472183228 5.0
0 0.095725417137146 6.0
0 -0.0583277940750122 7.0
0 0.0291923880577087 8.0
0 -0.000198006629943848 9.0
0 -0.0111546516418457 10.0
0 0.0124393701553345 11.0
0 -0.0176931619644165 12.0
53.643704598 53.1632347106934 13.0
40.391699893 40.1926460266113 14.0
0 -0.0250788927078247 15.0
0 -0.042284369468689 16.0
37.112968431 37.4618759155273 17.0
0 -0.0126620531082153 18.0
0 0.0458709001541138 19.0
0 0.0184005498886108 20.0
0 -0.0182098150253296 21.0
0 -0.026922345161438 22.0
34.292670167 33.4863471984863 23.0
0 -0.0208064317703247 24.0
0 1.70073449611664 25.0
0 0.0244821310043335 26.0
0 0.026792049407959 27.0
0 0.0530451536178589 28.0
0 -0.0856033563613892 29.0
0 0.0120147466659546 30.0
0 -0.114792943000793 31.0
0 -0.0135294198989868 32.0
0 0.112634778022766 33.0
0 -0.0458952188491821 34.0
0 0.05852210521698 35.0
0 -0.0465680360794067 36.0
0 -0.0387510061264038 37.0
0 0.000678658485412598 38.0
0 0.0196973085403442 39.0
0 0.159995913505554 40.0
0 -0.0288468599319458 41.0
0 0.0545963048934937 42.0
299.79970784 293.338012695312 43.0
0 -0.044167160987854 44.0
0 -0.00190639495849609 45.0
0 0.0134636163711548 46.0
0 -0.0382734537124634 47.0
0 -0.0734583139419556 48.0
0 0.264704883098602 49.0
0.068864734936 0.0901622772216797 50.0
1.553257107 0.937667369842529 51.0
0 -0.0510438680648804 52.0
0 -0.0254455804824829 53.0
0 0.0297268629074097 54.0
0 0.0287600755691528 55.0
0 -0.0304821729660034 56.0
0 0.00235307216644287 57.0
0 -0.0982257127761841 58.0
0 -0.0750206708908081 59.0
0 -0.0652421712875366 60.0
0 -0.0294643640518188 61.0
0 0.824270009994507 62.0
289.36849264 259.001556396484 0.0
0 0.0178006887435913 1.0
0 -0.0445395708084106 2.0
0 -0.0266264677047729 3.0
0 -0.0369175672531128 4.0
0 -0.0298682451248169 5.0
0 0.0125092267990112 6.0
0 -0.0654059648513794 7.0
0 -0.0289574861526489 8.0
0 -0.0495153665542603 9.0
0 -0.015630841255188 10.0
0 0.0490038990974426 11.0
0 -0.0275160074234009 12.0
75.334707876 74.4197845458984 13.0
12.044067138 25.6229686737061 14.0
0 -0.0814963579177856 15.0
0 0.00722920894622803 16.0
26.981781413 26.8086967468262 17.0
0 -0.00906813144683838 18.0
0 0.0141178369522095 19.0
0 0.0299628973007202 20.0
0 -0.0168331861495972 21.0
0 -0.0529628992080688 22.0
36.238728744 35.8470840454102 23.0
0 -0.0228968858718872 24.0
0 0.545027375221252 25.0
0 0.135156512260437 26.0
0 -0.0384482145309448 27.0
0 0.124216914176941 28.0
0 -0.100248456001282 29.0
0 -0.0699559450149536 30.0
0 -0.0140243768692017 31.0
0 0.0349968671798706 32.0
0 -0.0394426584243774 33.0
0 -0.023232102394104 34.0
0 0.119975805282593 35.0
0 -0.0782748460769653 36.0
0 0.0724138021469116 37.0
0 -0.0265136957168579 38.0
0 0.0635091066360474 39.0
0 0.0935462713241577 40.0
0 -0.00932514667510986 41.0
0 0.0183366537094116 42.0
288.81512864 294.621520996094 43.0
0 -0.0102688074111938 44.0
0 0.0344899892807007 45.0
0 -0.0296965837478638 46.0
0 -0.0666395425796509 47.0
0 -0.0347744226455688 48.0
0 0.270979225635529 49.0
0.098769944511 0.0813454389572144 50.0
1.8146554525 1.01243054866791 51.0
0 -0.0383456945419312 52.0
0 -0.00907981395721436 53.0
0 0.0495618581771851 54.0
0 -0.0804542303085327 55.0
0 -0.086797833442688 56.0
0 0.0244892835617065 57.0
0 -0.0417908430099487 58.0
0 0.0109158754348755 59.0
0 -0.0253335237503052 60.0
0 -0.0394097566604614 61.0
0 0.851532340049744 62.0
248.86043466 253.906066894531 0.0
0 0.0791014432907104 1.0
0 0.0193675756454468 2.0
0 0.0408357381820679 3.0
0 -0.0341392755508423 4.0
0 -0.0269895792007446 5.0
0 -0.0862518548965454 6.0
0 -0.00383007526397705 7.0
0 0.00773906707763672 8.0
0 0.0322870016098022 9.0
0 0.0173544883728027 10.0
0 0.0193253755569458 11.0
0 -0.0604511499404907 12.0
74.073636845 73.0078125 13.0
14.612720101 25.3910827636719 14.0
0 -0.00613844394683838 15.0
0 -0.0290437936782837 16.0
39.334098123 40.0481452941895 17.0
0 0.00463306903839111 18.0
0 -0.00855934619903564 19.0
0 0.0163909196853638 20.0
0 -0.0215963125228882 21.0
0 -0.044927716255188 22.0
32.577972335 31.7388896942139 23.0
0 -0.104038834571838 24.0
0 0.677508115768433 25.0
0 0.0872229337692261 26.0
0 0.0411657094955444 27.0
0 0.0579046010971069 28.0
0 -0.0414141416549683 29.0
0 -0.0931376218795776 30.0
0 -0.0632635354995728 31.0
0 0.0300568342208862 32.0
0 0.0313881635665894 33.0
0 -0.0726407766342163 34.0
0 -0.0353453159332275 35.0
0 -0.122742772102356 36.0
0 -0.00828826427459717 37.0
0 0.0260168313980103 38.0
0 0.0706783533096313 39.0
0 0.028316855430603 40.0
0 -0.102172493934631 41.0
0 -0.0173870325088501 42.0
311.88450141 295.462677001953 43.0
0 0.0940030813217163 44.0
0 -0.0988590717315674 45.0
0 0.00131595134735107 46.0
0 -0.0372520685195923 47.0
0 -0.00837337970733643 48.0
1.0286332098 0.285181164741516 49.0
0.087048106312 0.0644479990005493 50.0
1.8618221487 0.910273432731628 51.0
0 -0.0496314764022827 52.0
0 -0.0778352022171021 53.0
0 0.00957071781158447 54.0
0 -0.112520813941956 55.0
0 0.00118577480316162 56.0
0 -0.0381189584732056 57.0
0 -0.0679136514663696 58.0
0 -0.0753401517868042 59.0
0 0.0352529287338257 60.0
0 0.0374602079391479 61.0
0 0.8962322473526 62.0
215.81166283 256.261932373047 0.0
0 -0.0760771036148071 1.0
0 0.0298482179641724 2.0
0 -0.0153597593307495 3.0
0 -0.0885103940963745 4.0
0 0.0925530195236206 5.0
0 0.0862282514572144 6.0
0 -0.0196465253829956 7.0
0 -0.0578911900520325 8.0
0 0.125569701194763 9.0
0 -0.047397255897522 10.0
0 -0.011630654335022 11.0
0 0.152917265892029 12.0
76.194331866 74.7282791137695 13.0
36.855815904 27.55419921875 14.0
0 -0.0285154581069946 15.0
0 0.0175975561141968 16.0
35.022344943 35.1615867614746 17.0
0 0.0354942083358765 18.0
0 0.0535022020339966 19.0
0 -0.0510026216506958 20.0
0 -0.0534740686416626 21.0
0 0.0168088674545288 22.0
27.975272082 27.7140159606934 23.0
0 -0.0756617784500122 24.0
0 0.770396709442139 25.0
0 0.0836714506149292 26.0
0 0.0221229791641235 27.0
0 -0.0252004861831665 28.0
0 -0.107126116752625 29.0
0 -0.05408775806427 30.0
0 0.00828421115875244 31.0
0 -0.0815244913101196 32.0
0 -0.0801130533218384 33.0
0 -0.0220948457717896 34.0
0 -0.0408004522323608 35.0
0 -0.0599595308303833 36.0
0 -0.0285090208053589 37.0
0 -0.115956664085388 38.0
0 0.0292538404464722 39.0
0 -0.0349894762039185 40.0
0 -0.0781892538070679 41.0
0 0.0263646841049194 42.0
294.10007913 294.717193603516 43.0
0 -0.0209275484085083 44.0
0 0.0393013954162598 45.0
0 -0.00905382633209229 46.0
0 -0.0191715955734253 47.0
0 0.0206154584884644 48.0
0 0.307912588119507 49.0
0.065448348059 0.0428618192672729 50.0
1.7024858596 0.907089054584503 51.0
0 -0.0973716974258423 52.0
0 -0.062421441078186 53.0
0 0.00010216236114502 54.0
0 -0.0389281511306763 55.0
0 -0.0262075662612915 56.0
0 0.0288671255111694 57.0
0 -0.0496312379837036 58.0
0 -0.0944966077804565 59.0
0 0.0835727453231812 60.0
0 0.0447815656661987 61.0
0 0.773353099822998 62.0
281.5221094 256.431182861328 0.0
0 -0.0796381235122681 1.0
0 -0.0924283266067505 2.0
0 0.0142050981521606 3.0
0 -0.0281535387039185 4.0
0 0.033072829246521 5.0
0 -0.00392782688140869 6.0
0 -0.0371695756912231 7.0
0 0.0967434644699097 8.0
0 0.0618098974227905 9.0
0 -0.0640889406204224 10.0
0 -0.0580943822860718 11.0
0 -0.0337663888931274 12.0
64.093319474 62.9321670532227 13.0
27.029944624 35.7933120727539 14.0
0 -0.041669487953186 15.0
0 -0.00759971141815186 16.0
29.937958665 30.0563411712646 17.0
0 0.0643600225448608 18.0
0 -0.00976359844207764 19.0
0 -0.0224379301071167 20.0
0 -0.0353425741195679 21.0
0 -0.0113550424575806 22.0
37.625776526 36.5117416381836 23.0
0 -0.00526869297027588 24.0
0 0.817119061946869 25.0
0 0.0101298093795776 26.0
0 0.0650612115859985 27.0
0 -0.0225688219070435 28.0
0 -0.0414754152297974 29.0
0 0.28113067150116 30.0
0 -0.0622454881668091 31.0
0 0.0179456472396851 32.0
0 -5.8293342590332e-05 33.0
0 -0.0663219690322876 34.0
0 -0.0290969610214233 35.0
0 0.0860275030136108 36.0
0 -0.057709813117981 37.0
0 -0.0599304437637329 38.0
0 0.0234185457229614 39.0
0 0.118089079856873 40.0
0 -0.0393315553665161 41.0
0 0.00173985958099365 42.0
293.28736897 295.905914306641 43.0
0 0.0507644414901733 44.0
0 0.0107241868972778 45.0
0 -0.0273205041885376 46.0
0 -0.0290650129318237 47.0
0 0.0122851133346558 48.0
1.6012509281 0.301609218120575 49.0
0.08581378977 0.0524476766586304 50.0
1.7005494871 1.00879549980164 51.0
0 -0.07862389087677 52.0
0 -0.0458441972732544 53.0
0 -0.0285619497299194 54.0
0 -0.0415154695510864 55.0
0 -0.0604013204574585 56.0
0 -0.0224593877792358 57.0
0 -0.0628894567489624 58.0
0 -0.0351945161819458 59.0
0 -0.0371338129043579 60.0
0 -0.0208503007888794 61.0
0 0.821328401565552 62.0
241.54487034 257.252502441406 0.0
0 -0.0277549028396606 1.0
0 -0.061766505241394 2.0
0 -0.0309127569198608 3.0
0 -0.0895551443099976 4.0
0 -0.0215927362442017 5.0
0 -0.0959998369216919 6.0
0 -0.191142201423645 7.0
0 0.0539082288742065 8.0
0 0.0388110876083374 9.0
0 -0.042012095451355 10.0
0 0.0312882661819458 11.0
0 0.028464674949646 12.0
62.907480898 62.68603515625 13.0
35.446862978 34.6461219787598 14.0
0 -0.0429130792617798 15.0
0 -0.0175515413284302 16.0
40.864363228 40.7476387023926 17.0
0 -0.0617924928665161 18.0
0 -0.032007098197937 19.0
0 0.0435940027236938 20.0
0 0.0418905019760132 21.0
0 -0.00201809406280518 22.0
36.655961474 36.8338088989258 23.0
0 -0.0350648164749146 24.0
0 0.649562239646912 25.0
0 0.034239649772644 26.0
0 -0.0902427434921265 27.0
0 -0.00582611560821533 28.0
0 -0.0168386697769165 29.0
0 0.143616557121277 30.0
0 -0.137658953666687 31.0
0 -0.0764566659927368 32.0
0 0.0172799825668335 33.0
0 -0.0282400846481323 34.0
0 -0.0489782094955444 35.0
0 0.129771828651428 36.0
0 -0.0291234254837036 37.0
0 0.000759482383728027 38.0
0 0.012580394744873 39.0
0 0.166853308677673 40.0
0 -0.0415730476379395 41.0
0 -0.0876680612564087 42.0
289.87361607 292.352478027344 43.0
0 0.0175479650497437 44.0
0 -0.0158731937408447 45.0
0 0.0256227254867554 46.0
0 -0.0182353258132935 47.0
0 0.0350755453109741 48.0
0 0.318038582801819 49.0
0.063718342578 0.0697695016860962 50.0
2.2509597032 0.868453085422516 51.0
0 -0.0372065305709839 52.0
0 -0.00412023067474365 53.0
0 0.00606381893157959 54.0
0 0.0274814367294312 55.0
0 -0.0622419118881226 56.0
0 0.0192657709121704 57.0
0 -0.0659111738204956 58.0
0 0.0080331563949585 59.0
0 -0.0619374513626099 60.0
0 0.000689148902893066 61.0
0 0.792349100112915 62.0
269.24685086 257.595062255859 0.0
0 -0.109322190284729 1.0
0 0.0141817331314087 2.0
0 -0.002158522605896 3.0
0 -0.0780342817306519 4.0
0 -0.0458914041519165 5.0
0 -0.162451386451721 6.0
0 0.0617774724960327 7.0
0 -0.04212486743927 8.0
0 0.0470327138900757 9.0
0 -0.0282834768295288 10.0
0 -0.0198782682418823 11.0
0 0.00728833675384521 12.0
48.943975668 48.8760643005371 13.0
35.301209358 35.328556060791 14.0
0 -0.125258564949036 15.0
0 0.0589421987533569 16.0
40.580190546 39.3269958496094 17.0
0 -0.0201042890548706 18.0
0 -0.000579714775085449 19.0
0 0.00630438327789307 20.0
0 -0.00488483905792236 21.0
0 -0.0458546876907349 22.0
41.398914492 41.4151191711426 23.0
0 0.0373052358627319 24.0
0 2.55566835403442 25.0
0 0.0303393602371216 26.0
0 0.0548213720321655 27.0
0 0.176766753196716 28.0
0 -0.0879291296005249 29.0
0 -0.141857028007507 30.0
0 -0.0465060472488403 31.0
0 0.019558310508728 32.0
0 -0.0115913152694702 33.0
0 -0.00158631801605225 34.0
0 -0.061672568321228 35.0
0 0.0534442663192749 36.0
0 -0.00728499889373779 37.0
0 0.045862078666687 38.0
0 0.0765618085861206 39.0
0 0.0459460020065308 40.0
0 -0.0501902103424072 41.0
0 -0.127246499061584 42.0
293.34797824 296.622802734375 43.0
0 0.0590585470199585 44.0
0 -0.023308277130127 45.0
0 -0.00309669971466064 46.0
0 -0.0356520414352417 47.0
0 0.0183166265487671 48.0
0 0.368592768907547 49.0
0.069019708511 0.0443102121353149 50.0
2.3446128202 0.945521414279938 51.0
0 -0.0630344152450562 52.0
0 -0.0503350496292114 53.0
0 0.00253045558929443 54.0
0 -0.0685142278671265 55.0
0 -0.00557148456573486 56.0
0 -0.0270897150039673 57.0
0 -0.00763046741485596 58.0
0 -0.0290898084640503 59.0
0 -0.0322684049606323 60.0
0 -0.045040488243103 61.0
0 0.79190719127655 62.0
219.03880523 255.836380004883 0.0
0 0.0911401510238647 1.0
0 0.00915229320526123 2.0
0 -0.0148926973342896 3.0
0 -0.0846847295761108 4.0
0 -0.0881091356277466 5.0
0 -0.0152841806411743 6.0
0 -0.0262235403060913 7.0
0 -0.00270044803619385 8.0
0 0.0993074178695679 9.0
0 -0.0109120607376099 10.0
0 -0.0453851819038391 11.0
0 -0.035744309425354 12.0
65.768556534 64.3666915893555 13.0
20.281544045 33.0243377685547 14.0
0 -0.038180947303772 15.0
0 -0.0634099245071411 16.0
38.421209201 38.7471084594727 17.0
0 -0.0134302377700806 18.0
0 -0.0407012701034546 19.0
0 -0.0158601999282837 20.0
0 -0.0463002920150757 21.0
0 -0.0258151292800903 22.0
34.384870097 34.4413833618164 23.0
0 0.0686653852462769 24.0
0 0.455404162406921 25.0
0 0.0148161649703979 26.0
0 0.0289351940155029 27.0
0 -0.074102520942688 28.0
0 -0.0638614892959595 29.0
0 -0.0639578104019165 30.0
0 -0.0506645441055298 31.0
0 0.0149253606796265 32.0
0 -0.0783308744430542 33.0
0 -0.0175200700759888 34.0
0 -0.0596083402633667 35.0
0 0.0659841299057007 36.0
0 -0.0507570505142212 37.0
0 0.0432146787643433 38.0
0 -0.0599287748336792 39.0
0 0.00811612606048584 40.0
0 -0.031652569770813 41.0
0 -0.00333559513092041 42.0
300.77188498 296.819213867188 43.0
0 -0.121480584144592 44.0
0 0.0153411626815796 45.0
0 -0.0658310651779175 46.0
0 -0.0643125772476196 47.0
0 0.0891817808151245 48.0
2.5255057647 0.266469746828079 49.0
0.071621489294 0.0870963335037231 50.0
2.0981095452 0.959470152854919 51.0
0 -0.0779410600662231 52.0
0 -0.0482674837112427 53.0
0 0.0778008699417114 54.0
0 -0.102097392082214 55.0
0 0.051719069480896 56.0
0 -0.0404173135757446 57.0
0 -0.0368009805679321 58.0
0 -0.0402733087539673 59.0
0 0.0233808755874634 60.0
0 -0.0521746873855591 61.0
0 0.853776812553406 62.0
269.02673603 256.265930175781 0.0
0 -0.0465213060379028 1.0
0 -0.0736986398696899 2.0
0 -0.0252271890640259 3.0
0 -0.0460358858108521 4.0
0 -0.109303593635559 5.0
0 -0.114842057228088 6.0
0 -0.037335991859436 7.0
0 -0.0528755187988281 8.0
0 0.0746101140975952 9.0
0 -0.00404262542724609 10.0
0 -0.0398625135421753 11.0
0 -0.0155788660049438 12.0
61.629423576 60.4881019592285 13.0
43.934907988 35.2291831970215 14.0
0 -0.0280495882034302 15.0
0 -0.0157920122146606 16.0
40.108860822 40.8627662658691 17.0
0 -0.0290323495864868 18.0
0 -0.0298804044723511 19.0
0 -0.0163575410842896 20.0
0 0.000232815742492676 21.0
0 -0.00824320316314697 22.0
29.108944914 28.4535846710205 23.0
0 -0.0391031503677368 24.0
0 0.560660719871521 25.0
0 0.0913370847702026 26.0
0 0.0915201902389526 27.0
0 0.0475715398788452 28.0
0 -0.0753074884414673 29.0
0 -0.0434224605560303 30.0
0 -0.0416349172592163 31.0
0 -0.000745892524719238 32.0
0 0.0179187059402466 33.0
0 -0.0224946737289429 34.0
0 -0.0436769723892212 35.0
0 0.0774091482162476 36.0
0 -0.0180879831314087 37.0
0 -0.0126248598098755 38.0
0 0.0887163877487183 39.0
0 0.0878232717514038 40.0
0 -0.0397835969924927 41.0
0 0.0240129232406616 42.0
298.52734435 294.480590820312 43.0
0 0.0108798742294312 44.0
0 0.0784697532653809 45.0
0 -0.0180233716964722 46.0
0 -0.00527942180633545 47.0
0 -0.0384517908096313 48.0
0 0.330907642841339 49.0
0.081479054694 0.0557485818862915 50.0
1.6282451454 1.00718832015991 51.0
0 -0.0880783796310425 52.0
0 -0.0487748384475708 53.0
0 0.0263797044754028 54.0
0 -0.0407804250717163 55.0
0 0.0406452417373657 56.0
0 -0.0400763750076294 57.0
0 -0.0629266500473022 58.0
0 -0.0676208734512329 59.0
0 -0.0407084226608276 60.0
0 -0.018297553062439 61.0
0 0.832902550697327 62.0
269.09940342 253.972808837891 0.0
0 -0.0801817178726196 1.0
0 0.0139285326004028 2.0
0 -0.0341414213180542 3.0
0 -0.0256458520889282 4.0
0 -0.0300489664077759 5.0
0 0.000651717185974121 6.0
0 -0.0240365266799927 7.0
0 -0.0433962345123291 8.0
0 0.125195860862732 9.0
0 -0.0163339376449585 10.0
0 -0.0477656126022339 11.0
0 -0.0498033761978149 12.0
53.660829229 52.2937355041504 13.0
35.803408955 34.6202812194824 14.0
0 -0.0926386117935181 15.0
0 0.00411427021026611 16.0
33.267401752 33.7364311218262 17.0
0 -0.0172058343887329 18.0
0 -0.00297033786773682 19.0
0 0.101728081703186 20.0
0 0.0720063447952271 21.0
0 -0.0161987543106079 22.0
24.772446718 22.8386535644531 23.0
0 0.0874465703964233 24.0
14.00034613 0.490381896495819 25.0
0 0.149967551231384 26.0
0 -0.0198771953582764 27.0
0 0.19649612903595 28.0
0 -0.00221741199493408 29.0
0 0.102392077445984 30.0
0 -0.039033055305481 31.0
0 -0.0487657785415649 32.0
0 0.054728627204895 33.0
0 -0.00908911228179932 34.0
0 0.0411044359207153 35.0
0 -0.181123614311218 36.0
0 -0.0303493738174438 37.0
0 0.0558909177780151 38.0
0 -0.00908839702606201 39.0
0 0.0869048833847046 40.0
0 0.0563211441040039 41.0
0 -0.00262415409088135 42.0
305.42972198 292.408386230469 43.0
0 0.0460723638534546 44.0
0 0.16924649477005 45.0
0 -0.0132886171340942 46.0
0 -0.0226800441741943 47.0
0 -0.0300060510635376 48.0
0 0.254290580749512 49.0
0.060243533815 0.0956336259841919 50.0
1.447844503 0.855721116065979 51.0
0 -0.0683234930038452 52.0
0 0.0191813707351685 53.0
0 0.00952637195587158 54.0
0 -0.0260158777236938 55.0
0 -0.0295208692550659 56.0
0 -0.0271273851394653 57.0
0 0.170182585716248 58.0
0 -0.0523098707199097 59.0
0 -0.0109361410140991 60.0
0 -0.0372859239578247 61.0
0 0.851650953292847 62.0
265.98582083 255.841278076172 0.0
0 -0.118900895118713 1.0
0 -0.0898994207382202 2.0
0 0.0433803796768188 3.0
0 -0.0754059553146362 4.0
0 -0.0430933237075806 5.0
0 0.0325483083724976 6.0
0 -0.0861154794692993 7.0
0 0.00676995515823364 8.0
0 -0.00714743137359619 9.0
0 0.0367013216018677 10.0
0 0.0144033432006836 11.0
0 0.0153859853744507 12.0
54.655463424 53.5130233764648 13.0
57.185213586 38.7889747619629 14.0
0 -0.214996933937073 15.0
0 0.0247424840927124 16.0
27.896524894 25.7734069824219 17.0
0 0.0815740823745728 18.0
0 0.0660723447799683 19.0
0 -0.00367200374603271 20.0
0 0.0474656820297241 21.0
0 -0.0542889833450317 22.0
33.068486558 31.9952564239502 23.0
0 -0.0582457780838013 24.0
0 1.89759039878845 25.0
0 0.0879875421524048 26.0
0 0.0794428586959839 27.0
0 -0.00537598133087158 28.0
0 -0.124727368354797 29.0
0 0.177234530448914 30.0
0 -0.118048071861267 31.0
0 -0.0655924081802368 32.0
0 0.127830862998962 33.0
0 0.00186574459075928 34.0
0 0.104983806610107 35.0
0 -0.0083390474319458 36.0
0 0.0540114641189575 37.0
0 -0.0567575693130493 38.0
0 0.00700509548187256 39.0
0 -0.0349231958389282 40.0
0 0.041861891746521 41.0
0 -0.151568055152893 42.0
292.06523289 292.188781738281 43.0
0 0.0133441686630249 44.0
0 0.0636717081069946 45.0
0 0.0111407041549683 46.0
0 -0.0719932317733765 47.0
0 -0.0657602548599243 48.0
0 0.290254533290863 49.0
0.10544921492 0.0576325654983521 50.0
1.8552489341 0.990622282028198 51.0
0 -0.0473521947860718 52.0
0 -0.0442963838577271 53.0
0 0.0134693384170532 54.0
0 -0.140003561973572 55.0
0 -0.00529825687408447 56.0
0 0.021417498588562 57.0
0 0.0337549448013306 58.0
0 0.0182427167892456 59.0
0 -0.0895167589187622 60.0
0 0.0694280862808228 61.0
0 0.959096431732178 62.0
247.32602249 251.921813964844 0.0
0 -0.00532352924346924 1.0
0 -0.0643476247787476 2.0
0 -0.0121363401412964 3.0
0 -0.0398706197738647 4.0
0 -0.0572687387466431 5.0
0 -0.0591760873794556 6.0
0 -0.0977743864059448 7.0
0 -0.0482898950576782 8.0
0 0.0726388692855835 9.0
0 0.0213592052459717 10.0
0 0.025223970413208 11.0
0 -0.222797989845276 12.0
56.092974634 56.4070892333984 13.0
46.283204416 40.4021186828613 14.0
0 -0.0480061769485474 15.0
0 0.0195986032485962 16.0
27.739093856 28.19091796875 17.0
0 -0.0340121984481812 18.0
0 0.0248359441757202 19.0
0 0.0358465909957886 20.0
0 0.0142422914505005 21.0
0 -0.0237027406692505 22.0
29.972609585 28.7846908569336 23.0
0 0.0584176778793335 24.0
0 0.504677832126617 25.0
0 0.0186923742294312 26.0
0 0.0221112966537476 27.0
0 0.132255911827087 28.0
0 -0.0445548295974731 29.0
0 -0.05219566822052 30.0
0 -0.0139058828353882 31.0
0 -0.0445882081985474 32.0
0 0.0105434656143188 33.0
0 -0.031408429145813 34.0
0 -0.0811871290206909 35.0
0 0.063233494758606 36.0
0 -0.0235549211502075 37.0
0 0.0118535757064819 38.0
0 0.0865651369094849 39.0
0 0.034967303276062 40.0
0 -0.015916109085083 41.0
0 -0.032840371131897 42.0
296.07320777 296.023010253906 43.0
0 0.0693258047103882 44.0
0 -0.0589749813079834 45.0
0 -0.0385640859603882 46.0
0 -0.0292283296585083 47.0
0 -0.00432407855987549 48.0
0 0.266164541244507 49.0
0.068173736242 0.0788215398788452 50.0
1.8145119985 0.872924327850342 51.0
0 -0.0386921167373657 52.0
0 -0.0153564214706421 53.0
0 -0.00311625003814697 54.0
0 -0.0873633623123169 55.0
0 -0.00847232341766357 56.0
0 -0.00567924976348877 57.0
0 -0.0669258832931519 58.0
0 -0.0464202165603638 59.0
0 0.0529898405075073 60.0
0 -0.0607980489730835 61.0
0 0.826971650123596 62.0
256.96971009 259.031188964844 0.0
0 -0.0831781625747681 1.0
0 0.0138851404190063 2.0
0 2.22921371459961e-05 3.0
0 0.0114854574203491 4.0
0 -0.021833062171936 5.0
0 0.0706847906112671 6.0
0 -0.0474244356155396 7.0
0 -0.0878657102584839 8.0
0 0.0593494176864624 9.0
0 -0.0503238439559937 10.0
0 -0.0326415300369263 11.0
0 0.0102113485336304 12.0
58.80702744 57.9583625793457 13.0
33.249885907 30.4064655303955 14.0
0 -0.280466675758362 15.0
0 0.0290135145187378 16.0
24.218607905 22.6591262817383 17.0
0 -0.0121821165084839 18.0
0 0.00572264194488525 19.0
0 0.00287234783172607 20.0
0 0.00192654132843018 21.0
0 0.0231422185897827 22.0
29.712272429 29.3400192260742 23.0
0 -0.104297280311584 24.0
0 0.72854083776474 25.0
0 -0.0262657403945923 26.0
0 0.0526551008224487 27.0
0 -0.048622727394104 28.0
0 -0.0259543657302856 29.0
0 -0.0846526622772217 30.0
0 0.00568997859954834 31.0
0 -0.0266720056533813 32.0
0 0.02452552318573 33.0
0 -0.0186232328414917 34.0
0 0.0283160209655762 35.0
0 0.00464046001434326 36.0
0 -0.0286203622817993 37.0
0 0.0174156427383423 38.0
0 -0.0047687292098999 39.0
0 0.102825999259949 40.0
0 -0.0268738269805908 41.0
0 -0.00665915012359619 42.0
299.02450993 292.798126220703 43.0
0 0.0150798559188843 44.0
0 0.12786078453064 45.0
0 -0.0422028303146362 46.0
0 -0.086434006690979 47.0
0 -0.108456254005432 48.0
0 0.292164087295532 49.0
0.083887248505 0.0326710939407349 50.0
1.5296717106 0.983439445495605 51.0
0 -0.00865805149078369 52.0
0 0.0324829816818237 53.0
0 0.0134700536727905 54.0
0 -0.060782790184021 55.0
0 0.108075499534607 56.0
0 0.012947678565979 57.0
0 -0.0364505052566528 58.0
0 -0.013170599937439 59.0
0 0.105603337287903 60.0
0 0.020185112953186 61.0
0 0.87361204624176 62.0
207.64070398 253.147857666016 0.0
0 -0.00870096683502197 1.0
0 -0.0493227243423462 2.0
0 0.0528329610824585 3.0
0 -0.039913535118103 4.0
0 0.0118492841720581 5.0
0 0.0510662794113159 6.0
0 0.0388401746749878 7.0
0 -0.015738308429718 8.0
0 0.0162109136581421 9.0
0 0.0341370701789856 10.0
0 -0.0480996370315552 11.0
0 -0.0211693048477173 12.0
75.015263001 74.3228302001953 13.0
18.181752466 21.152551651001 14.0
0 -0.0158072710037231 15.0
0 -0.0139394998550415 16.0
39.771118013 39.9889183044434 17.0
0 -0.0354562997817993 18.0
0 0.0491632223129272 19.0
0 0.0609208345413208 20.0
0 -0.00941503047943115 21.0
0 -0.0155297517776489 22.0
32.604031648 31.8738307952881 23.0
0 -0.0242396593093872 24.0
0 0.643956542015076 25.0
0 -0.0270339250564575 26.0
0 0.0424007177352905 27.0
0 -0.0271164178848267 28.0
0 -0.12510359287262 29.0
0 -0.0261721611022949 30.0
0 -0.0261057615280151 31.0
0 0.0205925703048706 32.0
0 -0.0265675783157349 33.0
0 -0.0491966009140015 34.0
0 0.277011871337891 35.0
0 -0.00533163547515869 36.0
0 -0.223849654197693 37.0
0 0.0347002744674683 38.0
0 -0.0157700777053833 39.0
0 0.0235546827316284 40.0
0 -0.144478440284729 41.0
0 -0.0525103807449341 42.0
293.75868599 295.890991210938 43.0
0 0.0791662931442261 44.0
0 -0.0671935081481934 45.0
0 0.0732139348983765 46.0
0 -0.0800341367721558 47.0
0 -0.0463477373123169 48.0
2.4189310742 0.447493880987167 49.0
0.10746107231 0.0607987642288208 50.0
1.6621454437 1.01130831241608 51.0
0 -0.0528198480606079 52.0
0 -0.00114524364471436 53.0
0 -0.0177627801895142 54.0
0 -0.0790606737136841 55.0
0 0.0128031969070435 56.0
0 0.112285017967224 57.0
0 -0.0453704595565796 58.0
0 0.00900018215179443 59.0
0 -0.096129298210144 60.0
0 0.0024036169052124 61.0
0 0.835941910743713 62.0
280.951121 251.881256103516 0.0
0 -0.0427385568618774 1.0
0 -0.0303252935409546 2.0
0 -0.0235785245895386 3.0
0 -0.0375651121139526 4.0
0 -0.043831467628479 5.0
0 0.00589168071746826 6.0
0 -0.00767147541046143 7.0
0 -0.0219789743423462 8.0
0 0.0394786596298218 9.0
0 -2.90870666503906e-05 10.0
0 0.0076754093170166 11.0
0 -0.190725922584534 12.0
53.434081889 53.238410949707 13.0
37.406246049 36.9567832946777 14.0
0 0.0299686193466187 15.0
0 0.0395640134811401 16.0
28.347029314 28.1518077850342 17.0
0 -0.0326966047286987 18.0
0 -0.0183576345443726 19.0
0 -0.0150738954544067 20.0
0 0.00945556163787842 21.0
0 -0.0253216028213501 22.0
40.765449803 39.7237701416016 23.0
0 -0.0285440683364868 24.0
0 0.567422032356262 25.0
0 -0.00311958789825439 26.0
0 0.00621795654296875 27.0
0 0.0107840299606323 28.0
0 -0.0101474523544312 29.0
0 -0.0763169527053833 30.0
0 -0.0605171918869019 31.0
0 0.0200971364974976 32.0
0 -0.0118051767349243 33.0
0 -0.0377477407455444 34.0
0 -0.0647770166397095 35.0
0 -0.00357186794281006 36.0
0 0.0232721567153931 37.0
0 0.0181866884231567 38.0
0 0.0250998735427856 39.0
0 0.105265498161316 40.0
0 0.0132721662521362 41.0
0 0.002494215965271 42.0
298.20542997 294.699127197266 43.0
0 -0.00194418430328369 44.0
0 0.0493719577789307 45.0
0 -0.0397292375564575 46.0
0 0.00146687030792236 47.0
0 -0.0169130563735962 48.0
0 0.281697809696198 49.0
0.07749536628 0.0531612634658813 50.0
1.5808263779 0.85864782333374 51.0
0 -0.0486365556716919 52.0
0 0.0195688009262085 53.0
0 0.0499521493911743 54.0
0 -0.114270567893982 55.0
0 0.00708687305450439 56.0
0 -0.00285351276397705 57.0
0 -0.0280770063400269 58.0
0 -0.0279818773269653 59.0
0 -0.0578519105911255 60.0
0 0.0761581659317017 61.0
0 0.915201842784882 62.0
278.31612159 260.336853027344 0.0
0 -0.0718656778335571 1.0
0 0.0461121797561646 2.0
0 0.010891318321228 3.0
0 -0.0440410375595093 4.0
0 -0.10139000415802 5.0
0 0.142707228660583 6.0
0 0.0306237936019897 7.0
0 0.0388951301574707 8.0
0 0.00772225856781006 9.0
0 -0.024929404258728 10.0
0 -0.916029214859009 11.0
0 -0.0723730325698853 12.0
76.833462599 75.3936538696289 13.0
8.760186625 22.9653205871582 14.0
0 -0.0282140970230103 15.0
0 -0.0488888025283813 16.0
41.777726903 41.4043235778809 17.0
0 0.00298035144805908 18.0
0 -0.0079120397567749 19.0
0 0.0281103849411011 20.0
0 -0.00559449195861816 21.0
0 -0.00486624240875244 22.0
39.078663899 38.3046493530273 23.0
0 -0.0555182695388794 24.0
0 0.574587643146515 25.0
0 0.0723026990890503 26.0
0 -0.0957684516906738 27.0
0 -0.0330368280410767 28.0
0 0.0273343324661255 29.0
0 -0.0956401824951172 30.0
0 -0.00118029117584229 31.0
0 -0.0201650857925415 32.0
0 -0.0608171224594116 33.0
0 -0.138579249382019 34.0
0 -0.0732346773147583 35.0
0 -0.0485607385635376 36.0
0 -0.0469216108322144 37.0
0 0.125700354576111 38.0
0 -0.0889874696731567 39.0
0 -0.0703355073928833 40.0
0 -0.144102692604065 41.0
0 -0.0666977167129517 42.0
297.69275938 293.597381591797 43.0
0 0.000950217247009277 44.0
0 -0.000815153121948242 45.0
0 -0.0159912109375 46.0
0 -0.027290940284729 47.0
0 -0.0321246385574341 48.0
0 0.263979911804199 49.0
0.093646781367 0.0688560009002686 50.0
2.0923895741 1.01271796226501 51.0
0 -0.0925296545028687 52.0
0 -0.0370715856552124 53.0
0 -0.0171138048171997 54.0
0 -0.0532330274581909 55.0
0 -0.0740689039230347 56.0
0 -0.0456703901290894 57.0
0 -0.0666314363479614 58.0
0 -0.0251616239547729 59.0
0 0.129228711128235 60.0
0 0.0184563398361206 61.0
0 0.850737690925598 62.0
298.84193652 260.502410888672 0.0
0 -0.0806633234024048 1.0
0 0.0176879167556763 2.0
0 -0.0619553327560425 3.0
0 0.0423625707626343 4.0
0 0.0161656141281128 5.0
0 0.0715135335922241 6.0
0 -0.0354553461074829 7.0
0 -0.134816646575928 8.0
0 -0.00818789005279541 9.0
0 0.0026552677154541 10.0
0 0.0195732712745667 11.0
0 -0.0350996255874634 12.0
72.911870594 71.8584442138672 13.0
33.78222214 24.2236766815186 14.0
0 0.0548590421676636 15.0
0 -0.0271643400192261 16.0
25.689592062 25.2611560821533 17.0
0 0.0669683218002319 18.0
0 0.0579227209091187 19.0
0 -0.0325020551681519 20.0
0 -0.0213519334793091 21.0
0 0.0110095739364624 22.0
25.849205605 26.5851078033447 23.0
0 -0.0485044717788696 24.0
0 1.54559195041656 25.0
0 0.0114392042160034 26.0
0 -0.0133885145187378 27.0
0 0.016046404838562 28.0
0 -0.0692943334579468 29.0
0 -0.0833641290664673 30.0
0 0.00889742374420166 31.0
0 0.0897253751754761 32.0
0 0.0286935567855835 33.0
0 -0.0458368062973022 34.0
0 -0.0318158864974976 35.0
0 0.0549646615982056 36.0
0 -0.0423208475112915 37.0
0 0.0303424596786499 38.0
0 0.0642834901809692 39.0
0 0.0951560735702515 40.0
0 -0.0509806871414185 41.0
0 -0.0152612924575806 42.0
295.85851779 297.138336181641 43.0
0 -0.0221821069717407 44.0
0 0.0209929943084717 45.0
0 -0.0252114534378052 46.0
0 0.0149656534194946 47.0
0 0.0273054838180542 48.0
0 0.330333322286606 49.0
0.098874716655 0.11919105052948 50.0
1.7657961462 0.916902661323547 51.0
0 -0.0757118463516235 52.0
0 -0.06102454662323 53.0
0 -0.00676119327545166 54.0
0 -0.0449756383895874 55.0
0 -0.0567106008529663 56.0
0 0.0274194478988647 57.0
0 -0.0504356622695923 58.0
0 -0.071118950843811 59.0
0 -0.0516401529312134 60.0
0 0.0149139165878296 61.0
0 0.796554803848267 62.0
270.04327089 255.813034057617 0.0
0 -0.0549913644790649 1.0
0 0.0392142534255981 2.0
0 -0.0335859060287476 3.0
0 -0.0996845960617065 4.0
0 0.0527194738388062 5.0
0 -0.175714135169983 6.0
0 -0.0406213998794556 7.0
0 -0.0149229764938354 8.0
0 -0.0242325067520142 9.0
0 -0.0298072099685669 10.0
0 -0.0353881120681763 11.0
0 -0.136969208717346 12.0
62.477462346 62.4727973937988 13.0
26.067242464 33.8150291442871 14.0
0 -0.0654865503311157 15.0
0 -0.00265419483184814 16.0
37.879259581 37.2469139099121 17.0
0 -0.0288800001144409 18.0
0 0.00333607196807861 19.0
0 0.0186430215835571 20.0
0 -0.0264269113540649 21.0
0 0.0243414640426636 22.0
29.818677147 28.4017467498779 23.0
0 0.0384777784347534 24.0
0 0.522015929222107 25.0
0 0.0687083005905151 26.0
0 0.0457311868667603 27.0
0 0.0711230039596558 28.0
0 -0.049094557762146 29.0
0 0.11788809299469 30.0
0 -0.0528498888015747 31.0
0 -0.117290616035461 32.0
0 0.034964919090271 33.0
0 -0.0574971437454224 34.0
0 0.00589668750762939 35.0
0 0.0128110647201538 36.0
0 -0.112211108207703 37.0
0 -0.0304838418960571 38.0
0 -0.0667880773544312 39.0
0 0.116890788078308 40.0
0 -0.0599936246871948 41.0
0 -0.0110169649124146 42.0
301.06496028 295.55419921875 43.0
0 0.0125068426132202 44.0
0 0.0506654977798462 45.0
0 -0.0126127004623413 46.0
0 -0.0907763242721558 47.0
0 0.0677641630172729 48.0
0 0.366750121116638 49.0
0.080111170894 0.0379241704940796 50.0
2.3327080822 0.984931647777557 51.0
0 -0.0545603036880493 52.0
0 -0.0405975580215454 53.0
0 -0.018701434135437 54.0
0 -0.0772339105606079 55.0
0 0.026036262512207 56.0
0 -0.0233107805252075 57.0
0 -0.0430663824081421 58.0
0 -0.0444287061691284 59.0
0 0.0738643407821655 60.0
0 -0.0437568426132202 61.0
0 0.870433330535889 62.0
266.67511651 257.201904296875 0.0
0 -0.0500160455703735 1.0
0 -0.0295997858047485 2.0
0 -0.0168746709823608 3.0
0 -0.036171555519104 4.0
0 0.0270498991012573 5.0
0 0.0208543539047241 6.0
0 -0.0461350679397583 7.0
0 -0.0436989068984985 8.0
0 0.0851167440414429 9.0
0 -0.0183724164962769 10.0
0 -0.0581951141357422 11.0
0 -0.164083123207092 12.0
50.891133219 49.7621192932129 13.0
51.400086423 45.2165412902832 14.0
0 -0.049899697303772 15.0
0 -0.0135270357131958 16.0
38.825851899 38.2400588989258 17.0
0 -0.0582143068313599 18.0
0 0.0183058977127075 19.0
0 0.0352603197097778 20.0
0 -0.0221070051193237 21.0
0 -0.0196396112442017 22.0
33.862619708 32.6176948547363 23.0
0 -0.0536261796951294 24.0
0 0.520882129669189 25.0
0 0.0427635908126831 26.0
0 -0.0654287338256836 27.0
0 0.0360764265060425 28.0
0 -0.0492583513259888 29.0
0 0.0572043657302856 30.0
0 -0.0166705846786499 31.0
0 -0.000812172889709473 32.0
0 0.0792495012283325 33.0
0 -0.038892388343811 34.0
0 -0.0343352556228638 35.0
0 0.00755465030670166 36.0
0 -0.095441460609436 37.0
0 0.0406697988510132 38.0
0 0.119431376457214 39.0
0 0.00573098659515381 40.0
0 0.0144065618515015 41.0
0 0.0815137624740601 42.0
302.24791077 295.954223632812 43.0
0 -0.0289264917373657 44.0
0 0.00258326530456543 45.0
0 0.000907778739929199 46.0
0 -0.0305068492889404 47.0
0 -0.0395647287368774 48.0
0 0.273407459259033 49.0
0.0877416668 0.114259004592896 50.0
1.9574083831 0.902349770069122 51.0
0 -0.0517393350601196 52.0
0 -0.0577908754348755 53.0
0 0.0258280038833618 54.0
0 -0.0907858610153198 55.0
0 -0.0812448263168335 56.0
0 -0.000837445259094238 57.0
0 -0.0339521169662476 58.0
0 -0.0393944978713989 59.0
0 0.0201183557510376 60.0
0 0.0253161191940308 61.0
0 0.809560537338257 62.0
240.78857796 257.232452392578 0.0
0 -0.097385048866272 1.0
0 -0.002372145652771 2.0
0 0.0021592378616333 3.0
0 -0.0861026048660278 4.0
0 0.0390156507492065 5.0
0 0.00114667415618896 6.0
0 -0.0574647188186646 7.0
0 0.00227838754653931 8.0
0 0.0234140157699585 9.0
0 0.00820589065551758 10.0
0 0.0368226170539856 11.0
0 -0.0284839868545532 12.0
62.782662062 62.7188911437988 13.0
33.559625782 35.3380699157715 14.0
0 -0.0248132944107056 15.0
0 0.0208317041397095 16.0
34.468654705 34.8443260192871 17.0
0 -0.013018012046814 18.0
0 0.0346833467483521 19.0
0 0.0178290605545044 20.0
0 -0.057146430015564 21.0
0 -0.0177637338638306 22.0
36.506732829 35.862174987793 23.0
0 -0.00519812107086182 24.0
0 0.544998049736023 25.0
0 0.0574826002120972 26.0
0 0.0813497304916382 27.0
0 -0.0508538484573364 28.0
0 -0.0474406480789185 29.0
0 -0.0662020444869995 30.0
0 -0.0587712526321411 31.0
0 -0.101339936256409 32.0
0 0.0405048131942749 33.0
0 -0.0332039594650269 34.0
0 0.0704672336578369 35.0
0 -0.0428005456924438 36.0
0 -0.0349470376968384 37.0
0 0.0332974195480347 38.0
0 0.0358160734176636 39.0
0 0.0462716817855835 40.0
0 -0.0359126329421997 41.0
0 0.0138782262802124 42.0
292.22652869 292.504730224609 43.0
0 0.0199397802352905 44.0
0 -0.00698482990264893 45.0
0 -0.0166763067245483 46.0
0 -0.0477703809738159 47.0
0 0.043113112449646 48.0
0 0.23163914680481 49.0
0.096966011451 0.0650066137313843 50.0
1.7970417546 0.965561330318451 51.0
0 -0.0980678796768188 52.0
0 -0.033650279045105 53.0
0 -0.0864051580429077 54.0
0 0.0211383104324341 55.0
0 0.0246001482009888 56.0
0 -0.0390034914016724 57.0
0 -0.0403860807418823 58.0
0 -0.00993287563323975 59.0
0 0.0972236394882202 60.0
0 -0.0454224348068237 61.0
0 0.822578549385071 62.0
265.6707009 258.376434326172 0.0
0 -0.0348941087722778 1.0
0 -0.0628987550735474 2.0
0 0.0682936906814575 3.0
0 -0.0629478693008423 4.0
0 -0.035406231880188 5.0
0 -0.101526856422424 6.0
0 -0.0741046667098999 7.0
0 0.0653040409088135 8.0
0 -0.0367618799209595 9.0
0 0.0177067518234253 10.0
0 -0.0208292603492737 11.0
0 -0.0515028238296509 12.0
78.359867536 76.8883056640625 13.0
11.933209304 22.8116245269775 14.0
0 -0.0917073488235474 15.0
0 -0.0795661211013794 16.0
27.772835339 27.8369464874268 17.0
0 -0.0254398584365845 18.0
0 0.0149108171463013 19.0
0 0.0314618349075317 20.0
0 0.0133851766586304 21.0
0 -0.0284959077835083 22.0
27.704634855 26.4550151824951 23.0
0 0.093889594078064 24.0
0 0.451951563358307 25.0
0 0.00246107578277588 26.0
0 0.0338846445083618 27.0
0 0.114654898643494 28.0
0 -0.151779055595398 29.0
0 0.0493317842483521 30.0
0 -0.0481380224227905 31.0
0 -0.000511288642883301 32.0
0 0.0570708513259888 33.0
0 -0.0727647542953491 34.0
0 0.0055229663848877 35.0
0 0.0378943681716919 36.0
0 -0.0166999101638794 37.0
0 -0.0261920690536499 38.0
0 0.0163520574569702 39.0
0 0.00470912456512451 40.0
0 0.171395421028137 41.0
0 -0.0660587549209595 42.0
310.5191666 295.866790771484 43.0
0 0.00207841396331787 44.0
0 0.00484049320220947 45.0
0 -0.0583902597427368 46.0
0 0.0302168130874634 47.0
0 0.0512140989303589 48.0
0 0.254892230033875 49.0
0.07203800088 0.0652247667312622 50.0
1.4037075818 0.853809356689453 51.0
0 -0.109820246696472 52.0
0 0.0472425222396851 53.0
0 0.0360941886901855 54.0
0 -0.106708884239197 55.0
0 0.0330518484115601 56.0
0 0.0216983556747437 57.0
0 -0.050850510597229 58.0
0 -0.0709072351455688 59.0
0 0.00667202472686768 60.0
0 -0.032823920249939 61.0
0 0.810847878456116 62.0
255.42207413 259.351776123047 0.0
0 -0.0325514078140259 1.0
0 -0.0581446886062622 2.0
0 -0.030791163444519 3.0
0 -0.0957525968551636 4.0
0 -0.0309435129165649 5.0
0 -0.0369030237197876 6.0
0 -0.0130053758621216 7.0
0 -0.0322952270507812 8.0
0 0.0236934423446655 9.0
0 0.00630152225494385 10.0
0 -0.00484371185302734 11.0
0 -0.0115915536880493 12.0
68.838099062 68.3672714233398 13.0
16.27012008 29.3959560394287 14.0
0 -0.0689464807510376 15.0
0 -0.0163034200668335 16.0
34.960859225 34.7210998535156 17.0
0 0.00356400012969971 18.0
0 0.0268365144729614 19.0
0 -0.0166565179824829 20.0
0 -0.00332486629486084 21.0
0 0.00043189525604248 22.0
28.963336219 28.23410987854 23.0
0 -0.0177851915359497 24.0
0 0.523274302482605 25.0
0 0.0123261213302612 26.0
0 0.00324022769927979 27.0
0 -0.0905004739761353 28.0
0 -0.0283678770065308 29.0
0 -0.0599099397659302 30.0
0 -0.0175608396530151 31.0
0 -0.0197497606277466 32.0
0 -0.139538764953613 33.0
0 -0.112058997154236 34.0
0 -0.0276330709457397 35.0
0 -0.0587669610977173 36.0
0 0.00575625896453857 37.0
0 0.0538338422775269 38.0
0 0.0148283243179321 39.0
0 0.0609978437423706 40.0
0 -0.00955724716186523 41.0
0 0.0980123281478882 42.0
286.9343581 294.958190917969 43.0
0 -0.0361653566360474 44.0
0 0.034451961517334 45.0
0 -0.0497995615005493 46.0
0 -0.0434087514877319 47.0
0 -0.0423985719680786 48.0
0 0.249138236045837 49.0
0.067615316952 0.142736911773682 50.0
2.064808949 0.928950309753418 51.0
0 -0.0851937532424927 52.0
0 -0.0911194086074829 53.0
0 0.0433022975921631 54.0
0 -0.0523339509963989 55.0
0 -0.051817774772644 56.0
0 0.00160157680511475 57.0
0 -0.0780080556869507 58.0
0 -0.0730148553848267 59.0
0 -0.0783263444900513 60.0
0 -0.0371168851852417 61.0
0 0.828100323677063 62.0
226.81730217 255.183334350586 0.0
0 -0.0784164667129517 1.0
0 -0.0182873010635376 2.0
0 0.0283416509628296 3.0
0 -0.0636640787124634 4.0
0 -0.0226298570632935 5.0
0 -0.00548899173736572 6.0
0 -0.0347882509231567 7.0
0 0.071081280708313 8.0
0 -0.0164705514907837 9.0
0 -0.0228480100631714 10.0
0 -0.0376826524734497 11.0
0 0.0615438222885132 12.0
73.038313919 73.0054244995117 13.0
10.172848308 25.2854328155518 14.0
0 -0.166296124458313 15.0
0 -0.0265923738479614 16.0
33.702681249 33.5334854125977 17.0
0 -0.0196424722671509 18.0
0 -0.0363422632217407 19.0
0 -0.0314828157424927 20.0
0 -0.00749123096466064 21.0
0 -0.0117093324661255 22.0
34.617372759 34.688777923584 23.0
0 -0.089913010597229 24.0
0 1.91600441932678 25.0
0 0.118241190910339 26.0
0 -0.0401870012283325 27.0
0 -0.0120359659194946 28.0
0 -0.0524958372116089 29.0
0 -0.0482559204101562 30.0
0 -0.0820411443710327 31.0
0 0.0653265714645386 32.0
0 0.0362168550491333 33.0
0 -0.011435866355896 34.0
0 -0.0623883008956909 35.0
0 0.101321816444397 36.0
0 -0.0104917287826538 37.0
0 0.0732582807540894 38.0
0 0.053127646446228 39.0
0 0.103205561637878 40.0
0 0.197350382804871 41.0
0 -0.0618358850479126 42.0
291.41294636 294.037292480469 43.0
0 -0.0250588655471802 44.0
0 0.130284547805786 45.0
0 -0.0441526174545288 46.0
0 -0.0695911645889282 47.0
0 0.00315797328948975 48.0
0 0.295279681682587 49.0
0.07352751765 0.0523356199264526 50.0
1.9170223032 0.955632448196411 51.0
0 -0.0149065256118774 52.0
0 -0.00271332263946533 53.0
0 -0.00537788867950439 54.0
0 -0.0186396837234497 55.0
0 0.0305589437484741 56.0
0 -0.039833664894104 57.0
0 -0.0652252435684204 58.0
0 -0.013384222984314 59.0
0 0.0318359136581421 60.0
0 0.0499058961868286 61.0
0 0.823698163032532 62.0
254.22729991 254.670928955078 0.0
0 -0.0474663972854614 1.0
0 -0.0306302309036255 2.0
0 0.00622165203094482 3.0
0 0.0123080015182495 4.0
0 0.0393680334091187 5.0
0 0.00173985958099365 6.0
0 -0.0596252679824829 7.0
0 -0.0387322902679443 8.0
0 -0.0226999521255493 9.0
0 -0.0936683416366577 10.0
0 0.101023316383362 11.0
0 -0.00417196750640869 12.0
71.906615193 69.5878829956055 13.0
20.144743924 30.0273132324219 14.0
0 -0.0228787660598755 15.0
0 -0.0308343172073364 16.0
40.685579683 39.8233299255371 17.0
0 0.0467880964279175 18.0
0 0.0225247144699097 19.0
0 0.00332581996917725 20.0
0 -0.0448073148727417 21.0
0 0.0246633291244507 22.0
37.29534209 36.7082824707031 23.0
0 -0.0832215547561646 24.0
0 0.704127907752991 25.0
0 -0.00584328174591064 26.0
0 0.0411626100540161 27.0
0 0.081323504447937 28.0
0 -0.0470565557479858 29.0
0 -0.0650148391723633 30.0
0 -0.0496288537979126 31.0
0 0.0372475385665894 32.0
0 0.0559662580490112 33.0
0 -0.0182198286056519 34.0
0 0.0822970867156982 35.0
0 -0.0877779722213745 36.0
0 -0.0469433069229126 37.0
0 -0.034004807472229 38.0
0 0.0869606733322144 39.0
0 0.196237444877625 40.0
0 0.0772024393081665 41.0
0 -0.0377832651138306 42.0
285.05606859 295.104797363281 43.0
0 0.0313667058944702 44.0
0 0.00544273853302002 45.0
0 -0.0271812677383423 46.0
0 -0.00814807415008545 47.0
0 0.0282348394393921 48.0
0 0.25809308886528 49.0
0.075947219563 0.0451165437698364 50.0
1.5336980308 0.865305125713348 51.0
0 -0.108582615852356 52.0
0 -0.0653291940689087 53.0
0 0.0250245332717896 54.0
0 -0.0129276514053345 55.0
0 -0.0637353658676147 56.0
0 0.0139273405075073 57.0
0 -0.0687156915664673 58.0
0 -0.0455330610275269 59.0
0 -0.116414666175842 60.0
0 0.0667632818222046 61.0
0 0.901808321475983 62.0
211.77044955 255.821884155273 0.0
0 -0.033490777015686 1.0
0 -0.0131880044937134 2.0
0 -0.0642431974411011 3.0
0 -0.0838943719863892 4.0
0 0.00474655628204346 5.0
0 -0.21394407749176 6.0
0 -0.0376356840133667 7.0
0 -0.0027916431427002 8.0
0 0.100542426109314 9.0
0 -0.00826740264892578 10.0
0 0.0986031293869019 11.0
0 0.0683444738388062 12.0
63.442485258 62.7323913574219 13.0
24.833422147 32.7915267944336 14.0
0 -0.0230971574783325 15.0
0 0.0502527952194214 16.0
36.432652912 35.7965774536133 17.0
0 0.00530111789703369 18.0
0 0.0208076238632202 19.0
0 0.00553333759307861 20.0
0 -0.0679794549942017 21.0
0 0.0381768941879272 22.0
42.400400962 41.3138694763184 23.0
0 0.00510966777801514 24.0
0 1.33839380741119 25.0
0 0.0569089651107788 26.0
0 0.0934375524520874 27.0
0 -0.0623494386672974 28.0
0 -0.0710645914077759 29.0
0 -0.100097298622131 30.0
0 -0.0361050367355347 31.0
0 0.00628697872161865 32.0
0 -0.107399821281433 33.0
0 -0.02332603931427 34.0
0 -0.00666987895965576 35.0
0 -0.163950562477112 36.0
0 -0.0262123346328735 37.0
0 -0.0372205972671509 38.0
0 0.0403114557266235 39.0
0 0.125955939292908 40.0
0 -0.124445557594299 41.0
0 -0.103529572486877 42.0
299.67491447 296.003448486328 43.0
0 0.033960223197937 44.0
0 0.0144996643066406 45.0
0 -0.0229588747024536 46.0
0 0.0756474733352661 47.0
0 -0.094521164894104 48.0
0 0.272450268268585 49.0
0.066256797848 0.0596994161605835 50.0
1.8653538742 0.99866247177124 51.0
0 -0.056007981300354 52.0
0 -0.0622292757034302 53.0
0 0.0374394655227661 54.0
0 -0.0830341577529907 55.0
0 0.00121057033538818 56.0
0 0.0274583101272583 57.0
0 -0.0559102296829224 58.0
0 -0.068999171257019 59.0
0 0.0654498338699341 60.0
0 0.0180083513259888 61.0
0 0.873356938362122 62.0
254.96947951 253.604598999023 0.0
0 -0.0523830652236938 1.0
0 -0.11112105846405 2.0
0 0.0222746133804321 3.0
0 -0.0477713346481323 4.0
0 0.0396169424057007 5.0
0 -0.0304232835769653 6.0
0 -0.086586594581604 7.0
0 0.0333287715911865 8.0
0 0.0412935018539429 9.0
0 -0.0283917188644409 10.0
0 0.00572776794433594 11.0
0 0.0179809331893921 12.0
56.718667316 56.1577606201172 13.0
34.980680886 41.1825141906738 14.0
0 -0.0318495035171509 15.0
0 -0.0137282609939575 16.0
32.716913425 32.9326210021973 17.0
0 0.000812411308288574 18.0
0 -0.0248910188674927 19.0
0 0.0359040498733521 20.0
0 -0.0415571928024292 21.0
0 0.0135649442672729 22.0
33.71086084 32.7393379211426 23.0
0 0.0657542943954468 24.0
0 0.539235293865204 25.0
0 -0.00154554843902588 26.0
0 -0.026841402053833 27.0
0 -0.00262463092803955 28.0
0 -0.0650252103805542 29.0
0 -0.0833989381790161 30.0
0 -0.0269895792007446 31.0
0 0.00590360164642334 32.0
0 0.00678765773773193 33.0
0 -0.0478707551956177 34.0
0 -0.0508862733840942 35.0
0 -0.00360643863677979 36.0
0 -0.0598174333572388 37.0
0 0.0207613706588745 38.0
0 -0.0092092752456665 39.0
0 -0.0175620317459106 40.0
0 -0.0359770059585571 41.0
0 -0.0303570032119751 42.0
288.65431029 295.933013916016 43.0
0 0.0035017728805542 44.0
0 -0.0253924131393433 45.0
0 -0.0227710008621216 46.0
0 -0.0229507684707642 47.0
0 0.007895827293396 48.0
0 0.29130893945694 49.0
0.073767664733 0.0740948915481567 50.0
1.6558579922 1.00369572639465 51.0
0 -0.0953704118728638 52.0
0 -0.072001576423645 53.0
0 0.0559383630752563 54.0
0 -0.09064781665802 55.0
0 0.0128775835037231 56.0
0 -0.0277148485183716 57.0
0 0.0233381986618042 58.0
0 -0.0368489027023315 59.0
0 -0.0457180738449097 60.0
0 0.0127912759780884 61.0
0 0.872947812080383 62.0
281.08630745 254.955215454102 0.0
0 -0.100461602210999 1.0
0 0.0373152494430542 2.0
0 0.0323227643966675 3.0
0 -0.070618748664856 4.0
0 -0.171122670173645 5.0
0 18.4752769470215 6.0
0 -0.0691272020339966 7.0
0 -0.0190020799636841 8.0
0 -0.051430344581604 9.0
0 -0.000472187995910645 10.0
0 -0.018466591835022 11.0
0 -0.0105587244033813 12.0
69.248218313 67.3941879272461 13.0
40.129497209 28.6679573059082 14.0
0 -0.0521780252456665 15.0
0 -0.0724071264266968 16.0
36.221627729 35.0697212219238 17.0
0 -0.0469187498092651 18.0
0 0.00982797145843506 19.0
0 0.0465279817581177 20.0
0 0.01990807056427 21.0
0 -0.000185608863830566 22.0
30.496907509 29.6963710784912 23.0
0 0.160404086112976 24.0
0 1.1541736125946 25.0
0 -0.0137788057327271 26.0
0 0.0653215646743774 27.0
0 -0.0215507745742798 28.0
0 -0.00760138034820557 29.0
0 0.236123919487 30.0
0 -0.0142327547073364 31.0
0 0.0672115087509155 32.0
0 0.0450049638748169 33.0
0 -0.0283676385879517 34.0
0 -0.103549599647522 35.0
0 0.135294079780579 36.0
0 -0.0336388349533081 37.0
0 0.00884664058685303 38.0
0 -0.00761687755584717 39.0
0 0.0981744527816772 40.0
0 -0.0226715803146362 41.0
0 -0.168056130409241 42.0
275.43817852 295.800903320312 43.0
0 -0.012759804725647 44.0
0 0.00741857290267944 45.0
0 -0.000192046165466309 46.0
0 0.0405455827713013 47.0
0 -0.0226873159408569 48.0
3.2979221618 0.242435216903687 49.0
0.081460762242 0.0774161815643311 50.0
1.9573401666 0.941605985164642 51.0
0 -0.00408542156219482 52.0
0 -0.0422917604446411 53.0
0 0.0683573484420776 54.0
0 -0.0567995309829712 55.0
0 -0.089513897895813 56.0
0 0.0212806463241577 57.0
0 -0.0565363168716431 58.0
0 -0.064996600151062 59.0
0 -0.00940001010894775 60.0
0 -0.0199559926986694 61.0
0 0.901281833648682 62.0
235.65309346 258.680541992188 0.0
0 -0.0123759508132935 1.0
0 -0.10111391544342 2.0
0 0.00729501247406006 3.0
0 -0.0189467668533325 4.0
0 0.0467792749404907 5.0
0 0.0492485761642456 6.0
0 -0.00569570064544678 7.0
0 -0.0324065685272217 8.0
0 0.0498584508895874 9.0
0 -0.0173028707504272 10.0
0 0.0407400727272034 11.0
0 0.0508130788803101 12.0
50.631545966 46.2862281799316 13.0
53.85550055 39.789421081543 14.0
0 -0.003975510597229 15.0
0 0.010435938835144 16.0
34.503528592 34.3909454345703 17.0
0 0.017452597618103 18.0
0 0.023992657661438 19.0
0 0.0465458631515503 20.0
0 0.00778162479400635 21.0
0 0.0153158903121948 22.0
26.495236597 26.2984256744385 23.0
0 -0.0434871912002563 24.0
0 1.17897510528564 25.0
0 0.0846027135848999 26.0
0 0.0571738481521606 27.0
0 0.050284743309021 28.0
0 -0.0513457059860229 29.0
0 -0.126366853713989 30.0
0 -0.017276406288147 31.0
0 0.0085834264755249 32.0
0 0.0785654783248901 33.0
0 0.0254315137863159 34.0
0 -0.0564924478530884 35.0
0 0.0942810773849487 36.0
0 0.0343927145004272 37.0
0 0.00346028804779053 38.0
0 0.115900635719299 39.0
0 0.0930355787277222 40.0
0 -0.0846098661422729 41.0
0 0.0308383703231812 42.0
272.00117304 289.987884521484 43.0
0 -0.0338865518569946 44.0
0 0.101135492324829 45.0
0 0.0148340463638306 46.0
0 -0.0500208139419556 47.0
0 -0.0337811708450317 48.0
0.85863454833 0.256398856639862 49.0
0.079931329352 0.108638405799866 50.0
1.4301202321 0.903143107891083 51.0
0 -0.0482436418533325 52.0
0 0.0256539583206177 53.0
0 0.0168894529342651 54.0
0 0.00906026363372803 55.0
0 0.0077444314956665 56.0
0 -0.0498322248458862 57.0
0 0.0484522581100464 58.0
0 -0.0897051095962524 59.0
0 -0.0415898561477661 60.0
0 0.0559369325637817 61.0
0 0.988194465637207 62.0
266.23269751 244.638046264648 0.0
0 0.0122946500778198 1.0
0 0.0263532400131226 2.0
0 -0.00214040279388428 3.0
0 -0.0578912496566772 4.0
0 0.0649269819259644 5.0
0 -0.191125988960266 6.0
0 0.0248526334762573 7.0
0 0.0265409648418427 8.0
0 0.0373061895370483 9.0
0 -0.0166201591491699 10.0
0 -0.0369687080383301 11.0
0 -0.0374561548233032 12.0
49.919664279 49.4302101135254 13.0
49.815782402 46.4481658935547 14.0
0 0.00738656520843506 15.0
0 -0.00213921070098877 16.0
33.79741348 33.1541442871094 17.0
0 -0.0538402795791626 18.0
0 -0.0432835817337036 19.0
0 -0.0105513334274292 20.0
0 -0.0330477952957153 21.0
0 0.0232895612716675 22.0
28.942475729 27.9789333343506 23.0
0 0.0138386487960815 24.0
27.813546009 0.568551599979401 25.0
0 0.0300248861312866 26.0
0 0.0375307202339172 27.0
0 0.0349034070968628 28.0
0 -0.056122899055481 29.0
0 -0.106261372566223 30.0
0 0.0272735357284546 31.0
0 0.0364259481430054 32.0
0 -0.180961966514587 33.0
0 -0.153935074806213 34.0
0 -0.0753226280212402 35.0
0 -0.10452401638031 36.0
0 -0.00399434566497803 37.0
0 -0.0433861017227173 38.0
0 0.00368702411651611 39.0
0 -0.0409284830093384 40.0
0 -0.0998679399490356 41.0
0 -0.035962700843811 42.0
295.17425392 296.254333496094 43.0
0 -0.0137954950332642 44.0
0 -0.0241284370422363 45.0
0 -0.059536337852478 46.0
0 -0.0102490186691284 47.0
0 -0.0307694673538208 48.0
0 0.250014901161194 49.0
0.1040052238 0.054547905921936 50.0
1.8392533256 0.908038675785065 51.0
0 -0.0793811082839966 52.0
0 -0.0275627374649048 53.0
0 0.0492392778396606 54.0
0 -0.042016863822937 55.0
0 -0.0278834104537964 56.0
0 -0.0514265298843384 57.0
0 -0.0408586263656616 58.0
0 -0.0305005311965942 59.0
0 -0.0705252885818481 60.0
0 0.0227152109146118 61.0
0 0.795641899108887 62.0
268.79205694 253.195373535156 0.0
0 -0.0382272005081177 1.0
0 0.0640372037887573 2.0
0 0.0145443677902222 3.0
0 -0.0773929357528687 4.0
0 -0.0033794641494751 5.0
0 0.0600608587265015 6.0
0 -0.0596247911453247 7.0
0 0.0390113592147827 8.0
0 0.0685151815414429 9.0
0 -0.00386703014373779 10.0
0 -0.00610768795013428 11.0
0 -0.107134461402893 12.0
60.231424158 59.5610809326172 13.0
34.475105244 36.1901588439941 14.0
0 -0.111627221107483 15.0
0 0.0242036581039429 16.0
35.378015577 34.9945907592773 17.0
0 0.0426803827285767 18.0
0 0.041225790977478 19.0
0 -0.00338494777679443 20.0
0 -0.0463932752609253 21.0
0 -0.0203157663345337 22.0
36.143409194 35.3416519165039 23.0
0 0.00421607494354248 24.0
0 0.473815679550171 25.0
0 0.0882905721664429 26.0
0 0.0630911588668823 27.0
0 -0.0534754991531372 28.0
0 -0.0753861665725708 29.0
0 -0.000525951385498047 30.0
0 -0.00896704196929932 31.0
0 0.0359591245651245 32.0
0 -0.0447055101394653 33.0
0 -0.0488089323043823 34.0
0 0.0402182340621948 35.0
0 -0.0949043035507202 36.0
0 0.0386408567428589 37.0
0 -0.0318759679794312 38.0
0 0.112629532814026 39.0
0 0.0185564756393433 40.0
0 -0.0355656147003174 41.0
0 0.068869948387146 42.0
301.81205815 295.850341796875 43.0
0 0.0124248266220093 44.0
0 0.0711789131164551 45.0
0 0.00992047786712646 46.0
0 -0.0932542085647583 47.0
0 0.026692271232605 48.0
0 0.259066164493561 49.0
0.08544656885 0.0643481016159058 50.0
1.8297222574 0.937680244445801 51.0
0 -0.0932391881942749 52.0
0 -0.0337477922439575 53.0
0 -0.00338494777679443 54.0
0 -0.052791953086853 55.0
0 0.0119794607162476 56.0
0 -0.00475537776947021 57.0
0 0.0128403902053833 58.0
0 -0.0427578687667847 59.0
0 0.109822869300842 60.0
0 -0.0381506681442261 61.0
0 0.809521913528442 62.0
256.60244087 255.818161010742 0.0
0 0.0284336805343628 1.0
0 -0.0943249464035034 2.0
0 0.00708615779876709 3.0
0 -0.0237134695053101 4.0
0 -0.00849640369415283 5.0
0 -0.00808966159820557 6.0
0 -0.0840672254562378 7.0
0 -0.041960597038269 8.0
0 0.0536397695541382 9.0
0 -0.00102078914642334 10.0
0 0.014211118221283 11.0
0 -0.0268217325210571 12.0
69.917635869 70.1914215087891 13.0
30.147755862 28.3524303436279 14.0
0 -0.00906717777252197 15.0
0 -0.0557361841201782 16.0
32.485373732 32.6508712768555 17.0
0 -0.0213176012039185 18.0
0 0.00709664821624756 19.0
0 0.0218809843063354 20.0
0 -0.0357204675674438 21.0
0 0.0164932012557983 22.0
27.996445706 27.2415351867676 23.0
0 0.012386679649353 24.0
0 0.57389223575592 25.0
0 0.0200279951095581 26.0
0 0.0666817426681519 27.0
0 -0.0690118074417114 28.0
0 -0.0139845609664917 29.0
0 -0.0610940456390381 30.0
0 0.000578522682189941 31.0
0 0.0361379384994507 32.0
0 0.0848969221115112 33.0
0 -0.0485502481460571 34.0
0 -0.0933655500411987 35.0
0 -0.00557339191436768 36.0
0 0.0535928010940552 37.0
0 0.0265394449234009 38.0
0 0.0369731187820435 39.0
0 0.0474618673324585 40.0
0 0.0410503149032593 41.0
0 0.0605148077011108 42.0
297.96687429 297.511688232422 43.0
0 -0.0341373682022095 44.0
0 0.0480904579162598 45.0
0 -0.0346659421920776 46.0
0 -0.073639988899231 47.0
0 0.00259721279144287 48.0
0 0.357094645500183 49.0
0.083651493473 0.0690006017684937 50.0
2.259056085 1.00455689430237 51.0
0 -0.0583711862564087 52.0
0 -0.0658198595046997 53.0
0 0.0123273134231567 54.0
0 -0.0643450021743774 55.0
0 -0.0203260183334351 56.0
0 -0.0369025468826294 57.0
0 -0.0141829252243042 58.0
0 -0.0730646848678589 59.0
0 -0.0191792249679565 60.0
0 0.0681561231613159 61.0
0 0.854417264461517 62.0
272.96606454 252.317092895508 0.0
0 -0.0578151941299438 1.0
0 -0.0323148965835571 2.0
0 -0.0638011693954468 3.0
0 -0.0642043352127075 4.0
0 0.000619769096374512 5.0
0 0.0453065633773804 6.0
0 -0.0475326776504517 7.0
0 0.0115964412689209 8.0
0 0.0374301671981812 9.0
0 0.0125299692153931 10.0
0 -0.00588536262512207 11.0
0 -0.0559135675430298 12.0
70.041546531 70.4448776245117 13.0
38.802781104 24.3686828613281 14.0
0 -0.00970423221588135 15.0
0 -0.0218411684036255 16.0
30.823773941 30.4988422393799 17.0
0 0.018065333366394 18.0
0 0.0218852758407593 19.0
0 -0.00710713863372803 20.0
0 -0.0210236310958862 21.0
0 0.0390022993087769 22.0
27.68642223 27.5339908599854 23.0
0 -0.0351306200027466 24.0
0 0.796520888805389 25.0
0 0.0435339212417603 26.0
0 0.0763388872146606 27.0
0 -0.031866192817688 28.0
0 -0.0234819650650024 29.0
0 -0.0526374578475952 30.0
0 -0.0221537351608276 31.0
0 0.0205200910568237 32.0
0 -0.0169850587844849 33.0
0 -0.0845164060592651 34.0
0 -0.0939465761184692 35.0
0 -0.0653332471847534 36.0
0 -0.0114635229110718 37.0
0 0.0113168954849243 38.0
0 0.00336277484893799 39.0
0 0.014543890953064 40.0
0 -0.0288723707199097 41.0
0 -0.0222870111465454 42.0
300.16613316 296.518951416016 43.0
0 0.0440467596054077 44.0
0 -0.000389695167541504 45.0
0 -0.0416413545608521 46.0
0 -0.0366212129592896 47.0
0 -0.00743687152862549 48.0
0 0.279950797557831 49.0
0.087766014177 0.0706278085708618 50.0
1.5782738992 1.00696611404419 51.0
0 -0.130725741386414 52.0
0 -0.0144597291946411 53.0
0 0.00844037532806396 54.0
0 -0.0540753602981567 55.0
0 -0.00995910167694092 56.0
0 -0.0277794599533081 57.0
0 0.033624529838562 58.0
0 -0.0355039834976196 59.0
0 0.0291410684585571 60.0
0 0.00487315654754639 61.0
0 0.84040367603302 62.0
272.10312066 254.266723632812 0.0
0 -0.0948883295059204 1.0
0 -0.00530922412872314 2.0
0 -0.0266267061233521 3.0
0 -0.0421477556228638 4.0
0 -0.0523804426193237 5.0
0 -0.00800716876983643 6.0
0 0.032981276512146 7.0
0 0.093579888343811 8.0
0 -0.0156413316726685 9.0
0 -0.014140248298645 10.0
0 -0.060167670249939 11.0
0 0.081224799156189 12.0
49.290461569 47.9947891235352 13.0
33.459932538 34.9034004211426 14.0
0 -0.0875059366226196 15.0
0 0.0201653242111206 16.0
25.341852774 24.8946628570557 17.0
0 0.0652807950973511 18.0
0 0.0333377122879028 19.0
0 0.00135481357574463 20.0
0 -0.0294612646102905 21.0
0 0.0226634740829468 22.0
26.594372985 25.2860908508301 23.0
0 0.0509542226791382 24.0
0 0.612597465515137 25.0
0 0.0243877172470093 26.0
0 0.0261905789375305 27.0
0 -0.0777279138565063 28.0
0 -0.0282143354415894 29.0
0 0.272865176200867 30.0
0 -0.0436617136001587 31.0
0 0.160637259483337 32.0
0 0.0459789037704468 33.0
0 -0.114357590675354 34.0
0 0.0263251066207886 35.0
0 0.0238634347915649 36.0
0 -0.0548621416091919 37.0
0 0.139982581138611 38.0
0 -0.0995827913284302 39.0
0 0.160917162895203 40.0
0 0.0239623785018921 41.0
0 -0.0277239084243774 42.0
282.67945155 294.810699462891 43.0
0 0.00671041011810303 44.0
0 0.0308206081390381 45.0
0 -0.0375813245773315 46.0
0 -0.0879504680633545 47.0
0 -0.0460792779922485 48.0
0 0.259609460830688 49.0
0.079732273256 0.0836659669876099 50.0
1.7811991652 0.99291467666626 51.0
0 -0.0346194505691528 52.0
0 -0.0296310186386108 53.0
0 -0.121753811836243 54.0
0 -0.0672639608383179 55.0
0 0.0416940450668335 56.0
0 -0.0322557687759399 57.0
0 -0.0924166440963745 58.0
0 0.0358778238296509 59.0
0 0.082832932472229 60.0
0 0.0160759687423706 61.0
0 0.759741902351379 62.0
222.25829195 254.595230102539 0.0
0 -0.0622421503067017 1.0
0 -0.00899183750152588 2.0
0 0.00224745273590088 3.0
0 -0.0318268537521362 4.0
0 -0.0038067102432251 5.0
0 0.0357755422592163 6.0
0 -0.123783707618713 7.0
0 -0.00637829303741455 8.0
0 0.0829967260360718 9.0
0 -0.0155946016311646 10.0
0 -0.0272113680839539 11.0
0 0.0575670003890991 12.0
70.232569249 69.6509857177734 13.0
31.057192014 26.4090728759766 14.0
0 -0.149580597877502 15.0
0 -0.017539381980896 16.0
34.056307764 33.7388076782227 17.0
0 0.00138819217681885 18.0
0 0.0175036191940308 19.0
0 -0.0031808614730835 20.0
0 -0.0220471620559692 21.0
0 -0.0799709558486938 22.0
35.485392222 35.3021087646484 23.0
0 -0.105711102485657 24.0
0 0.620934784412384 25.0
0 0.0967472791671753 26.0
0 0.0774859189987183 27.0
0 -0.0599299669265747 28.0
0 -0.0512510538101196 29.0
0 -0.0815023183822632 30.0
0 -0.0558162927627563 31.0
0 -0.0499850511550903 32.0
0 0.0690804719924927 33.0
0 -0.0445355176925659 34.0
0 -0.038166880607605 35.0
0 0.0662370920181274 36.0
0 -0.0755919218063354 37.0
0 0.0536378622055054 38.0
0 0.138455510139465 39.0
0 -0.0292340517044067 40.0
0 -0.00872552394866943 41.0
0 -0.0245643854141235 42.0
300.08842362 295.767456054688 43.0
0 -0.0540395975112915 44.0
0 0.0555644035339355 45.0
0 -0.0521234273910522 46.0
0 -0.0619384050369263 47.0
0 0.0290268659591675 48.0
0 0.270851641893387 49.0
0.09150960594 0.061970591545105 50.0
2.1570394777 0.976708829402924 51.0
0 -0.0733987092971802 52.0
0 -0.0454303026199341 53.0
0 -0.0277212858200073 54.0
0 -0.057794451713562 55.0
0 -0.00379931926727295 56.0
0 -0.0149260759353638 57.0
0 -0.0687943696975708 58.0
0 -0.05677330493927 59.0
0 -0.0704630613327026 60.0
0 0.0110481977462769 61.0
0 0.869427084922791 62.0
263.36362287 255.278656005859 0.0
0 -0.0247789621353149 1.0
0 -0.0241380929946899 2.0
0 -0.0753566026687622 3.0
0 -0.0755935907363892 4.0
0 -0.0601655244827271 5.0
0 -0.00498545169830322 6.0
0 0.0263246297836304 7.0
0 0.0399807691574097 8.0
0 0.042441725730896 9.0
0 -0.12742006778717 10.0
0 0.0186562538146973 11.0
0 0.0193055868148804 12.0
67.275669575 65.8373107910156 13.0
24.536927009 33.7673835754395 14.0
0 -0.0475841760635376 15.0
0 0.0184706449508667 16.0
26.608947155 25.8904685974121 17.0
0 0.0264190435409546 18.0
0 -0.00544297695159912 19.0
0 -0.037352442741394 20.0
0 -0.0347877740859985 21.0
0 0.0112084150314331 22.0
29.998122641 29.2519435882568 23.0
0 -0.0894819498062134 24.0
0 2.39726781845093 25.0
0 0.0650161504745483 26.0
0 0.052198052406311 27.0
0 0.0916813611984253 28.0
0 -0.0724698305130005 29.0
0 -0.137436747550964 30.0
0 0.035932183265686 31.0
0 0.1395343542099 32.0
0 -0.0428205728530884 33.0
0 -0.0584002733230591 34.0
0 0.0229448080062866 35.0
0 0.0306929349899292 36.0
0 0.00391209125518799 37.0
0 -0.0582515001296997 38.0
0 0.0837596654891968 39.0
0 0.0840400457382202 40.0
0 0.015687108039856 41.0
0 0.0142854452133179 42.0
286.73249986 294.016265869141 43.0
0 -0.0703612565994263 44.0
0 0.00190401077270508 45.0
0 -0.00128233432769775 46.0
0 -0.0270699262619019 47.0
0 -0.039106011390686 48.0
0 0.272423446178436 49.0
0.069692944211 0.0765229463577271 50.0
2.238508584 0.971995413303375 51.0
0 -0.0889462232589722 52.0
0 -0.0189195871353149 53.0
0 0.0598026514053345 54.0
0 -0.0475848913192749 55.0
0 -0.0486882925033569 56.0
0 -0.026708722114563 57.0
0 -0.0077742338180542 58.0
0 0.00782716274261475 59.0
0 0.14831268787384 60.0
0 -0.0503827333450317 61.0
0 0.798621892929077 62.0
292.60448179 252.75715637207 0.0
0 -0.0504494905471802 1.0
0 -0.0356382131576538 2.0
0 0.0089036226272583 3.0
0 -0.0452417135238647 4.0
0 -0.129575848579407 5.0
0 -0.00129187107086182 6.0
0 -0.0963531732559204 7.0
0 0.00139081478118896 8.0
0 0.0769580602645874 9.0
0 0.0032426118850708 10.0
0 -0.0321037769317627 11.0
0 -0.0010601282119751 12.0
57.901475166 57.6744194030762 13.0
43.752026033 39.9580612182617 14.0
0 0.0414527654647827 15.0
0 0.0440698862075806 16.0
28.216139949 28.9433269500732 17.0
0 0.0295394659042358 18.0
0 -0.0149015188217163 19.0
0 -0.0320552587509155 20.0
0 -0.0263341665267944 21.0
0 0.00873076915740967 22.0
34.912379346 34.0283584594727 23.0
0 -0.104323506355286 24.0
0.18583781012 1.12476420402527 25.0
0 -0.013191819190979 26.0
0 0.038288950920105 27.0
0 -0.00569689273834229 28.0
0 -0.0572136640548706 29.0
0 -0.0354175567626953 30.0
0 -0.0205005407333374 31.0
0 -0.0544387102127075 32.0
0 0.00114762783050537 33.0
0 -0.0594927072525024 34.0
0 -0.0274144411087036 35.0
0 0.0381134748458862 36.0
0 -0.035462498664856 37.0
0 -0.0186687707901001 38.0
0 0.0635992288589478 39.0
0 0.00910842418670654 40.0
0 -0.0418976545333862 41.0
0 -0.0498625040054321 42.0
308.35272351 297.024230957031 43.0
0 -0.0636459589004517 44.0
0 0.0294603109359741 45.0
0 0.00147902965545654 46.0
0 -0.0144028663635254 47.0
0 -0.0942927598953247 48.0
0 0.31311970949173 49.0
0.07752868841 0.0523039102554321 50.0
2.1447460787 1.01908504962921 51.0
0 -0.0507146120071411 52.0
0 -0.0858513116836548 53.0
0 -0.0123125314712524 54.0
0 -0.0916203260421753 55.0
0 0.0171860456466675 56.0
0 -0.00758707523345947 57.0
0 -0.0683614015579224 58.0
0 -0.00129091739654541 59.0
0 0.00119554996490479 60.0
0 0.030906081199646 61.0
0 0.882928311824799 62.0
255.13725937 258.173248291016 0.0
0 -0.0378996133804321 1.0
0 -0.0447419881820679 2.0
0 -0.00582230091094971 3.0
0 -0.0492812395095825 4.0
0 -0.0290619134902954 5.0
0 -0.0841199159622192 6.0
0 -0.0310629606246948 7.0
0 0.0436716079711914 8.0
0 0.0276006460189819 9.0
0 -0.0500632524490356 10.0
0 -0.0365701913833618 11.0
0 0.0079425573348999 12.0
54.453909777 53.3470115661621 13.0
45.174893065 42.3856391906738 14.0
0 -0.12690794467926 15.0
0 -0.062796950340271 16.0
36.926493281 36.0220222473145 17.0
0 0.0412043333053589 18.0
0 0.0196453332901001 19.0
0 -0.02894127368927 20.0
0 -0.00805389881134033 21.0
0 -0.00395119190216064 22.0
34.781931745 34.3502082824707 23.0
0 -0.0662742853164673 24.0
0 0.654908895492554 25.0
0 0.102485537528992 26.0
0 0.036518931388855 27.0
0 -0.0281621217727661 28.0
0 -0.0973807573318481 29.0
0 -0.0340038537979126 30.0
0 0.00832641124725342 31.0
0 -0.0282202959060669 32.0
0 0.0175234079360962 33.0
0 0.014830470085144 34.0
0 -0.0102397203445435 35.0
0 0.058071494102478 36.0
0 -0.0283521413803101 37.0
0 0.0237647294998169 38.0
0 0.0750678777694702 39.0
0 -0.0012589693069458 40.0
0 -0.0635682344436646 41.0
0 0.0328420400619507 42.0
293.90768573 295.081420898438 43.0
0 -0.011121392250061 44.0
0 -0.0208083391189575 45.0
0 -0.0255917310714722 46.0
0 -0.0794939994812012 47.0
0 -0.0116978883743286 48.0
0 0.276291370391846 49.0
0.086099480986 0.0925376415252686 50.0
1.8814101777 0.876879215240479 51.0
0 -0.0365010499954224 52.0
0 -0.0481687784194946 53.0
0 0.0876461267471313 54.0
0 -0.0858103036880493 55.0
0 -0.0252466201782227 56.0
0 -0.0265089273452759 57.0
0 -0.0402926206588745 58.0
0 -0.0725430250167847 59.0
0 0.000382542610168457 60.0
0 -0.0163465738296509 61.0
0 0.837716341018677 62.0
213.019486 258.233581542969 0.0
0 -0.0247818231582642 1.0
0 -0.0415953397750854 2.0
0 0.0327175855636597 3.0
0 -0.0371285676956177 4.0
0 0.175155758857727 5.0
0 -0.0499616861343384 6.0
0 -0.180288791656494 7.0
0 -0.0620365142822266 8.0
0 0.0730193853378296 9.0
0 0.0045771598815918 10.0
0 0.0238298773765564 11.0
0 -0.0473600625991821 12.0
73.783142672 72.7415390014648 13.0
22.265905218 29.8054103851318 14.0
0 -0.00368702411651611 15.0
0 -0.0490075349807739 16.0
38.841657731 38.3164596557617 17.0
0 -0.0151640176773071 18.0
0 0.075865626335144 19.0
0 -0.0140722990036011 20.0
0 -0.0322946310043335 21.0
0 -0.00855600833892822 22.0
38.845232914 38.2458839416504 23.0
0 0.021349310874939 24.0
0 1.81669461727142 25.0
0 0.0981820821762085 26.0
0 0.110434174537659 27.0
0 0.0405687093734741 28.0
0 -0.0454310178756714 29.0
0 -0.0764474868774414 30.0
0 -0.0217832326889038 31.0
0 -0.0750795602798462 32.0
0 0.13987410068512 33.0
0 -0.05915367603302 34.0
0 -0.0240192413330078 35.0
0 0.0763047933578491 36.0
0 0.0244897603988647 37.0
0 0.0656198263168335 38.0
0 0.123763680458069 39.0
0 0.0538228750228882 40.0
0 -0.0352612733840942 41.0
0 -0.000410676002502441 42.0
294.32722516 297.343170166016 43.0
0 0.0653270483016968 44.0
0 -0.060737133026123 45.0
0 0.00878036022186279 46.0
0 -0.0878628492355347 47.0
0 -0.0438801050186157 48.0
0 0.25488007068634 49.0
0.10141859103 0.18001115322113 50.0
1.80237172 0.897299647331238 51.0
0 -0.0903412103652954 52.0
0 0.000711321830749512 53.0
0 0.0539647340774536 54.0
0 -0.0670541524887085 55.0
0 -0.0637811422348022 56.0
0 -0.0137978792190552 57.0
0 -0.0376018285751343 58.0
0 -0.0723439455032349 59.0
0 0.00445425510406494 60.0
0 -0.024842381477356 61.0
0 0.810120701789856 62.0
271.52311652 251.883285522461 0.0
0 0.0128024816513062 1.0
0 -0.0246690511703491 2.0
0 -0.0509628057479858 3.0
0 -0.0700379610061646 4.0
0 -0.0185185670852661 5.0
0 0.0327457189559937 6.0
0 -0.0889943838119507 7.0
0 -0.0123946070671082 8.0
0 0.0416005849838257 9.0
0 -0.0205323696136475 10.0
0 -0.00110757350921631 11.0
0 -0.102238297462463 12.0
70.82903193 70.1320953369141 13.0
25.536232397 26.3970069885254 14.0
0 -0.0414267778396606 15.0
0 -0.0392664670944214 16.0
30.653274688 30.5573749542236 17.0
0 -0.00947797298431396 18.0
0 0.0156956911087036 19.0
0 0.00853574275970459 20.0
0 -0.00782883167266846 21.0
0 0.00100982189178467 22.0
29.466170284 28.4698600769043 23.0
0 -0.0475207567214966 24.0
0 0.593851327896118 25.0
0 0.063801646232605 26.0
0 0.00394999980926514 27.0
0 0.0151771306991577 28.0
0 -0.0668808221817017 29.0
0 -0.0607844591140747 30.0
0 -0.00827133655548096 31.0
0 -0.01435387134552 32.0
0 0.00267517566680908 33.0
0 0.00494158267974854 34.0
0 -0.00605618953704834 35.0
0 0.0315357446670532 36.0
0 -0.0348979234695435 37.0
0 -0.0382660627365112 38.0
0 0.0708237886428833 39.0
0 0.11174476146698 40.0
0 -0.0419759750366211 41.0
0 -0.0164743661880493 42.0
304.04595173 297.173522949219 43.0
0 0.00264585018157959 44.0
0 -0.0663049221038818 45.0
0 -0.0335291624069214 46.0
0 -0.0281261205673218 47.0
0 0.00415360927581787 48.0
0.72893303034 0.253866910934448 49.0
0.095219236659 0.0897356271743774 50.0
2.2450697539 0.959121406078339 51.0
0 -0.0735749006271362 52.0
0 -0.043157696723938 53.0
0 0.0280107259750366 54.0
0 -0.0795732736587524 55.0
0 -0.050904393196106 56.0
0 -0.0507429838180542 57.0
0 -0.0284675359725952 58.0
0 -0.0635453462600708 59.0
0 0.0348283052444458 60.0
0 0.0736857652664185 61.0
0 0.923834562301636 62.0
234.01249151 257.884582519531 0.0
0 -0.158177018165588 1.0
0 -0.0544883012771606 2.0
0 0.0568569898605347 3.0
0 -0.0929926633834839 4.0
0 0.0580899715423584 5.0
0 -0.116885781288147 6.0
0 0.00890254974365234 7.0
0 0.0650815963745117 8.0
0 -0.0231806039810181 9.0
0 -0.0161604881286621 10.0
0 0.0321159362792969 11.0
0 -0.186363816261292 12.0
57.855332768 57.9686508178711 13.0
48.617548726 38.9733619689941 14.0
0 -0.0749608278274536 15.0
0 -0.00499117374420166 16.0
34.474307471 34.000732421875 17.0
0 -0.00839102268218994 18.0
0 0.0147010087966919 19.0
0 0.0357381105422974 20.0
0 -0.0234724283218384 21.0
0 0.0508307218551636 22.0
24.736601216 18.708589553833 23.0
0 0.0959576368331909 24.0
0 0.505953729152679 25.0
0 0.0726741552352905 26.0
0 0.0893441438674927 27.0
0 -0.116053700447083 28.0
0 -0.00939977169036865 29.0
0 -0.122432708740234 30.0
0 -0.0451914072036743 31.0
0 -0.132587075233459 32.0
0 0.0261677503585815 33.0
0 0.000319123268127441 34.0
0 -0.0871878862380981 35.0
0 0.0531214475631714 36.0
0 -0.04790198802948 37.0
0 0.0255073308944702 38.0
0 -0.0414239168167114 39.0
0 -0.00198805332183838 40.0
0 0.201291799545288 41.0
0 -0.0437303781509399 42.0
309.78640244 297.342346191406 43.0
0 0.00290095806121826 44.0
0 0.06752610206604 45.0
0 -0.0106357336044312 46.0
0 -0.035457968711853 47.0
0 0.0500921010971069 48.0
0 0.272490799427032 49.0
0.093012964264 0.0664602518081665 50.0
1.4458678613 0.990902543067932 51.0
0 -0.101495862007141 52.0
0 0.0103234052658081 53.0
0 -0.0250846147537231 54.0
0 -0.0542076826095581 55.0
0 -0.0807770490646362 56.0
0 0.0549620389938354 57.0
0 -0.0712481737136841 58.0
0 -0.0723263025283813 59.0
0 -0.0816081762313843 60.0
0 0.0644975900650024 61.0
0 0.982172250747681 62.0
288.61563253 250.084533691406 0.0
0 -0.0472952127456665 1.0
0 0.0213876962661743 2.0
0 -0.0107120275497437 3.0
0 -0.0556236505508423 4.0
0 -0.0597063302993774 5.0
0 -0.0352393388748169 6.0
0 -0.0986658334732056 7.0
0 -0.071519136428833 8.0
0 0.0220645666122437 9.0
0 -0.138643145561218 10.0
0 -0.024319052696228 11.0
0 0.0810703039169312 12.0
56.716801511 56.6239242553711 13.0
35.499780911 33.3746795654297 14.0
0 -0.120509266853333 15.0
0 0.0781072378158569 16.0
34.03683588 33.3078536987305 17.0
0 -0.045751690864563 18.0
0 -0.0208162069320679 19.0
0 -0.0403245687484741 20.0
0 0.00859200954437256 21.0
0 -0.078349232673645 22.0
33.182667881 32.6280746459961 23.0
0 -0.046033501625061 24.0
0 0.570369005203247 25.0
0 0.075579047203064 26.0
0 0.069395899772644 27.0
0 0.0409144163131714 28.0
0 -0.165487408638 29.0
0 0.0155781507492065 30.0
0 -0.0327905416488647 31.0
0 -0.0513912439346313 32.0
0 -0.121660351753235 33.0
0 -0.0347205400466919 34.0
0 -0.0315085649490356 35.0
0 -0.000203728675842285 36.0
0 -0.00497734546661377 37.0
0 -0.00111401081085205 38.0
0 -0.00513374805450439 39.0
0 -0.124074101448059 40.0
0 0.013245701789856 41.0
0 -0.0614858865737915 42.0
279.58703095 295.944427490234 43.0
0 -0.0469719171524048 44.0
0 0.0922861099243164 45.0
0 -0.0730983018875122 46.0
0 -0.00957405567169189 47.0
0 -0.0297242403030396 48.0
0 0.279407560825348 49.0
0.10559023829 0.0419496297836304 50.0
1.3893158872 0.844692349433899 51.0
0 -0.0861226320266724 52.0
0 -0.0150185823440552 53.0
0 0.0867472887039185 54.0
0 -0.0707567930221558 55.0
0 -0.0443154573440552 56.0
0 -0.0122443437576294 57.0
0 -0.049939751625061 58.0
0 0.0570122003555298 59.0
0 0.0325576066970825 60.0
0 -0.0264822244644165 61.0
0 0.786905288696289 62.0
250.32358406 252.845611572266 0.0
0 -0.0192772150039673 1.0
0 -0.036110520362854 2.0
0 -0.0110076665878296 3.0
0 -0.0530518293380737 4.0
0 -0.0910040140151978 5.0
0 -0.081607460975647 6.0
0 -0.0433615446090698 7.0
0 -0.0148836374282837 8.0
0 -0.0481773614883423 9.0
0 -0.0251783132553101 10.0
0 -0.0200046300888062 11.0
0 0.0874751806259155 12.0
60.267495789 59.2129287719727 13.0
36.83290804 33.0344505310059 14.0
0 -0.0219055414199829 15.0
0 -0.00626575946807861 16.0
33.940891289 33.9881858825684 17.0
0 -0.0550276041030884 18.0
0 0.0201603174209595 19.0
0 0.0510181188583374 20.0
0 0.0400542020797729 21.0
0 -0.0701075792312622 22.0
37.323872616 36.7886848449707 23.0
0 0.0808814764022827 24.0
0 0.512298405170441 25.0
0 0.00941741466522217 26.0
0 -0.0635808706283569 27.0
0 -0.0580698251724243 28.0
0 -0.0344444513320923 29.0
0 -0.0841649770736694 30.0
0 -0.035673975944519 31.0
0 0.0352972745895386 32.0
0 0.00578963756561279 33.0
0 -0.0320109128952026 34.0
0 -0.0503108501434326 35.0
0 -0.0253368616104126 36.0
0 -0.0354045629501343 37.0
0 -0.0162035226821899 38.0
0 0.0591157674789429 39.0
0 0.0837639570236206 40.0
0 -0.0119160413742065 41.0
0 0.0423330068588257 42.0
298.56314066 295.509887695312 43.0
0 0.0189727544784546 44.0
0 0.0344489812850952 45.0
0 0.0103620290756226 46.0
0 -0.06474769115448 47.0
0 -0.00780594348907471 48.0
0.35393219234 0.266937255859375 49.0
0.098564822749 0.074323296546936 50.0
1.5658450335 0.906597554683685 51.0
0 -0.0741726160049438 52.0
0 -0.0236769914627075 53.0
0 0.0741754770278931 54.0
0 -0.115674376487732 55.0
0 -0.0794137716293335 56.0
0 0.0287817716598511 57.0
0 -0.0347393751144409 58.0
0 -0.0817965269088745 59.0
0 -0.0179940462112427 60.0
0 -0.0381537675857544 61.0
0 0.841450095176697 62.0
214.1020204 254.78727722168 0.0
0 -0.0221964120864868 1.0
0 -0.0253952741622925 2.0
0 -0.0187095403671265 3.0
0 -0.0166159868240356 4.0
0 -0.0704308748245239 5.0
0 -0.117914795875549 6.0
0 -0.205002188682556 7.0
0 0.000470280647277832 8.0
0 0.062743067741394 9.0
0 -0.0937868356704712 10.0
0 0.00420218706130981 11.0
0 0.0895818471908569 12.0
67.385368973 66.8566284179688 13.0
13.399399728 28.2434883117676 14.0
0 -0.0404249429702759 15.0
0 -0.0374032258987427 16.0
41.613555255 42.2983016967773 17.0
0 0.00477373600006104 18.0
0 -0.032586932182312 19.0
0 -0.0332530736923218 20.0
0 -0.00779831409454346 21.0
0 0.075886607170105 22.0
27.063185977 25.2364673614502 23.0
0 0.0806254148483276 24.0
0 1.36449337005615 25.0
0 0.0133436918258667 26.0
0 0.0370901823043823 27.0
0 0.0963600873947144 28.0
0 -0.0620182752609253 29.0
0 -0.133632898330688 30.0
0 0.0303345918655396 31.0
0 0.00819051265716553 32.0
0 -0.00476968288421631 33.0
0 -0.00476968288421631 34.0
0 -0.059450626373291 35.0
0 0.0855907201766968 36.0
0 -0.0570772886276245 37.0
0 -0.00113499164581299 38.0
0 0.0401443243026733 39.0
0 0.0512799024581909 40.0
0 -0.0471744537353516 41.0
0 0.013689398765564 42.0
311.35902477 297.182464599609 43.0
0 -0.00677740573883057 44.0
0 0.0144574642181396 45.0
0 -0.0673009157180786 46.0
0 -0.0190426111221313 47.0
0 0.00899779796600342 48.0
0 0.330860882997513 49.0
0.091164836751 0.0598706007003784 50.0
2.0366008191 1.01744341850281 51.0
0 -0.0499550104141235 52.0
0 -0.00050055980682373 53.0
0 0.0432237386703491 54.0
0 -0.0617986917495728 55.0
0 -0.101741671562195 56.0
0 -0.00050652027130127 57.0
0 -0.0531867742538452 58.0
0 -0.0152539014816284 59.0
0 -0.00923454761505127 60.0
0 0.0107258558273315 61.0
0 0.879919767379761 62.0
266.58729332 253.858001708984 0.0
0 -0.0404778718948364 1.0
0 -0.0262724161148071 2.0
0 0.0282000303268433 3.0
0 -0.0386301279067993 4.0
0 -0.0319515466690063 5.0
0 -0.183863759040833 6.0
0 0.0498999357223511 7.0
0 -0.0902945995330811 8.0
0 0.0561965703964233 9.0
0 -0.071205735206604 10.0
0 -0.0204541087150574 11.0
0 -0.0427118539810181 12.0
66.590593742 65.8467483520508 13.0
46.924865732 34.7061958312988 14.0
0 -0.0187755823135376 15.0
0 -0.0263229608535767 16.0
28.054461186 27.5216579437256 17.0
0 0.0223363637924194 18.0
0 -0.0239955186843872 19.0
0 -0.0105546712875366 20.0
0 -0.0371493101119995 21.0
0 -0.0343753099441528 22.0
38.251210974 38.1610832214355 23.0
0 0.0152429342269897 24.0
0 0.494191825389862 25.0
0 0.126563906669617 26.0
0 0.104569554328918 27.0
0 -0.0199285745620728 28.0
0 -0.0728942155838013 29.0
0 -0.0783084630966187 30.0
0 -0.0559107065200806 31.0
0 -0.0411092042922974 32.0
0 0.0678662061691284 33.0
0 -0.00857388973236084 34.0
0 -0.0604890584945679 35.0
0 -0.02546226978302 36.0
0 -0.0251716375350952 37.0
0 0.00792872905731201 38.0
0 0.100631594657898 39.0
0 0.106661200523376 40.0
0 -0.108505249023438 41.0
0 0.0158876180648804 42.0
309.92790288 296.806213378906 43.0
0 -0.019971489906311 44.0
0 0.0456327199935913 45.0
0 -0.0425561666488647 46.0
0 -0.0794582366943359 47.0
0 0.0233687162399292 48.0
0 0.287926256656647 49.0
0.078719991196 0.0581752061843872 50.0
1.8622598671 0.928947150707245 51.0
0 -0.0513197183609009 52.0
0 0.00414597988128662 53.0
0 -0.00242388248443604 54.0
0 -0.0789297819137573 55.0
0 0.0124768018722534 56.0
0 -0.0340286493301392 57.0
0 -0.0516279935836792 58.0
0 -0.0375450849533081 59.0
0 -0.0785983800888062 60.0
0 0.0145388841629028 61.0
0 0.844532012939453 62.0
261.96181176 253.344482421875 0.0
0 -0.0515533685684204 1.0
0 -0.0608187913894653 2.0
0 -0.0222576856613159 3.0
0 -0.0141094923019409 4.0
0 0.00287234783172607 5.0
0 0.0581125020980835 6.0
0 -0.0868765115737915 7.0
0 -0.0318037867546082 8.0
0 -0.0463348627090454 9.0
0 -0.0302706956863403 10.0
0 -0.0117244124412537 11.0
0 -0.0602670907974243 12.0
76.270477982 74.0157089233398 13.0
32.316074149 25.8599166870117 14.0
0 0.0558537244796753 15.0
0 -0.00450980663299561 16.0
25.12726935 23.1794166564941 17.0
0 -0.00666248798370361 18.0
0 -0.037279486656189 19.0
0 -0.0336006879806519 20.0
0 -0.0284057855606079 21.0
0 -0.0516592264175415 22.0
41.131144405 41.0509490966797 23.0
0 -0.108281254768372 24.0
19.832536234 1.70188355445862 25.0
0 0.0743507146835327 26.0
0 0.0923918485641479 27.0
0 0.00489366054534912 28.0
0 -0.0320078134536743 29.0
0 -0.105317711830139 30.0
0 -0.0219169855117798 31.0
0 -0.0153313875198364 32.0
0 -0.150274515151978 33.0
0 -0.123365521430969 34.0
0 -0.135000348091125 35.0
0 -0.102460265159607 36.0
0 -0.0420151948928833 37.0
0 0.0476771593093872 38.0
0 0.0932856798171997 39.0
0 0.0305346250534058 40.0
0 -0.0367357730865479 41.0
0 -0.0308672189712524 42.0
291.34830089 294.744812011719 43.0
0 0.0386327505111694 44.0
0 0.129452049732208 45.0
0 -0.00857484340667725 46.0
0 -0.0835281610488892 47.0
0 -0.0862866640090942 48.0
0 0.255035758018494 49.0
0.078672534299 0.113392353057861 50.0
2.1571340247 0.883008718490601 51.0
0 -0.022962212562561 52.0
0 -0.0582848787307739 53.0
0 0.0754381418228149 54.0
0 -0.0234519243240356 55.0
0 -0.0237816572189331 56.0
0 -0.0191723108291626 57.0
0 -0.0374032258987427 58.0
0 -0.0212820768356323 59.0
0 -0.0500980615615845 60.0
0 -0.0135542154312134 61.0
0 0.850086688995361 62.0
255.58420524 256.967926025391 0.0
0 -0.0390845537185669 1.0
0 -0.132288098335266 2.0
0 0.0214515924453735 3.0
0 -0.0508443117141724 4.0
0 -0.0756062269210815 5.0
0 -0.0625444650650024 6.0
0 -0.0545412302017212 7.0
0 -0.0166445970535278 8.0
0 0.0203460454940796 9.0
0 -0.0395447015762329 10.0
0 -0.045337438583374 11.0
0 0.0109809637069702 12.0
52.563922912 52.2606048583984 13.0
41.889489708 40.6119003295898 14.0
0 -0.0176631212234497 15.0
0 0.0328329801559448 16.0
33.898973105 33.6110610961914 17.0
0 0.0435754060745239 18.0
0 0.0177589654922485 19.0
0 -0.0216811895370483 20.0
0 -0.0321294069290161 21.0
0 -0.0194846391677856 22.0
35.872247347 35.6716384887695 23.0
0 -0.0976454019546509 24.0
0 0.571056187152863 25.0
0 0.055693507194519 26.0
0 0.0437361001968384 27.0
0 -0.0668197870254517 28.0
0 -0.110817313194275 29.0
0 -0.095973014831543 30.0
0 -0.0239678621292114 31.0
0 0.0414947271347046 32.0
0 0.0835999250411987 33.0
0 -0.00178539752960205 34.0
0 -0.0705515146255493 35.0
0 0.0245286226272583 36.0
0 0.0377074480056763 37.0
0 0.000550150871276855 38.0
0 0.0615075826644897 39.0
0 -0.0775605440139771 40.0
0 -0.034701943397522 41.0
0 -0.0314689874649048 42.0
295.26043613 298.149749755859 43.0
0 -0.000983357429504395 44.0
0 -0.0684328079223633 45.0
0 -0.0226017236709595 46.0
0 -0.0464061498641968 47.0
0 -0.011935830116272 48.0
0 0.254231870174408 49.0
0.090542566347 0.204952120780945 50.0
1.8092060272 0.890791356563568 51.0
0 -0.0569688081741333 52.0
0 -0.029105544090271 53.0
0 0.065545916557312 54.0
0 -0.040552020072937 55.0
0 -0.00512921810150146 56.0
0 -0.0267428159713745 57.0
0 -0.0588256120681763 58.0
0 0.00367701053619385 59.0
0 -0.0775381326675415 60.0
0 0.0607117414474487 61.0
0 0.90614265203476 62.0
216.12449871 253.477066040039 0.0
0 0.051683783531189 1.0
0 0.0463718175888062 2.0
0 -0.0445798635482788 3.0
0 -0.0502530336380005 4.0
0 -0.0840414762496948 5.0
0 0.0195354223251343 6.0
0 -0.0643535852432251 7.0
0 0.00124800205230713 8.0
0 -0.000332474708557129 9.0
0 0.000707030296325684 10.0
0 0.0663596987724304 11.0
0 0.0216096639633179 12.0
71.999606471 71.2798538208008 13.0
16.026230369 24.314754486084 14.0
0 0.0221437215805054 15.0
0 0.0207666158676147 16.0
33.734032877 33.6337394714355 17.0
0 0.0371478796005249 18.0
0 0.0147603750228882 19.0
0 -0.0267010927200317 20.0
0 -0.019184947013855 21.0
0 -0.0610948801040649 22.0
40.43445633 40.2326507568359 23.0
0 0.0347460508346558 24.0
0 0.492920815944672 25.0
0 0.0744527578353882 26.0
0 0.0701367855072021 27.0
0 0.00498473644256592 28.0
0 -0.0530732870101929 29.0
0 -0.0465112924575806 30.0
0 -0.00403296947479248 31.0
0 -0.0870767831802368 32.0
0 0.0149906873703003 33.0
0 -0.00622403621673584 34.0
0 -0.0264896154403687 35.0
0 0.0460883378982544 36.0
0 -0.0448874235153198 37.0
0 -0.0403708219528198 38.0
0 0.0162898302078247 39.0
0 0.0803035497665405 40.0
0 -0.0316678285598755 41.0
0 -0.0550543069839478 42.0
289.96589452 296.479125976562 43.0
0 -0.0155903100967407 44.0
0 0.0713331699371338 45.0
0 0.0146845579147339 46.0
0 -0.0172985792160034 47.0
0 -0.0801092386245728 48.0
1.4913021635 0.282058119773865 49.0
0.068805983454 0.131736397743225 50.0
1.9151616894 0.928069949150085 51.0
0 -0.0646401643753052 52.0
0 -0.0382062196731567 53.0
0 0.0674663782119751 54.0
0 -0.124148726463318 55.0
0 0.0276018381118774 56.0
0 -0.0556408166885376 57.0
0 -0.0568135976791382 58.0
0 -0.0451799631118774 59.0
0 -0.0893300771713257 60.0
0 0.0557409524917603 61.0
0 0.884583294391632 62.0
281.76062801 256.557159423828 0.0
0 -0.038239598274231 1.0
0 -0.0901492834091187 2.0
0 -0.0151163339614868 3.0
0 -0.016626238822937 4.0
0 0.0259507894515991 5.0
0 -0.181460499763489 6.0
0 0.0171216726303101 7.0
0 -0.0577287673950195 8.0
0 -0.0179535150527954 9.0
0 0.0239473581314087 10.0
0 -0.0350241661071777 11.0
0 0.0644181966781616 12.0
71.713724232 71.0811386108398 13.0
37.222719945 25.2381992340088 14.0
0 -0.0407325029373169 15.0
0 0.00111043453216553 16.0
33.239477933 33.5899314880371 17.0
0 0.0121155977249146 18.0
0 0.00682651996612549 19.0
0 -0.0170217752456665 20.0
0 -0.0175696611404419 21.0
0 -0.0333229303359985 22.0
34.03275686 33.5622520446777 23.0
0 -0.0542103052139282 24.0
0 0.554351449012756 25.0
0 0.07762610912323 26.0
0 0.114872574806213 27.0
0 0.0633729696273804 28.0
0 -0.0404549837112427 29.0
0 -0.0803395509719849 30.0
0 -0.00803649425506592 31.0
0 0.0128215551376343 32.0
0 0.0969692468643188 33.0
0 -0.0352848768234253 34.0
0 -0.0663970708847046 35.0
0 -0.085248589515686 36.0
0 -0.00631272792816162 37.0
0 -0.0175603628158569 38.0
0 0.0280596017837524 39.0
0 0.0262964963912964 40.0
0 -0.0143991708755493 41.0
0 0.00337541103363037 42.0
295.35717102 294.037445068359 43.0
0 -0.0381218194961548 44.0
0 -0.00677835941314697 45.0
0 -0.052859902381897 46.0
0 -0.0415457487106323 47.0
0 0.00326287746429443 48.0
0 0.236226260662079 49.0
0.088910956254 0.0762442350387573 50.0
2.0012011238 1.01874554157257 51.0
0 -0.0795623064041138 52.0
0 -0.0356777906417847 53.0
0 0.0501848459243774 54.0
0 -0.0735474824905396 55.0
0 -0.105639100074768 56.0
0 -0.00850069522857666 57.0
0 -0.0365759134292603 58.0
0 -0.0305770635604858 59.0
0 -0.15218722820282 60.0
0 0.0477319955825806 61.0
0 0.889240562915802 62.0
248.4974182 254.523803710938 0.0
0 -0.028356671333313 1.0
0 -0.102473855018616 2.0
0 -0.0109602212905884 3.0
0 -0.0694230794906616 4.0
0 0.103146195411682 5.0
0 0.0446971654891968 6.0
0 -0.0707031488418579 7.0
0 0.0376627445220947 8.0
0 0.0112565755844116 9.0
0 0.014944314956665 10.0
0 -0.0462410449981689 11.0
0 -0.104971528053284 12.0
53.73577113 52.8973083496094 13.0
40.033022614 39.1927223205566 14.0
0 -0.0489403009414673 15.0
0 0.0280331373214722 16.0
33.320084664 33.77392578125 17.0
0 -0.00721395015716553 18.0
0 0.0112780332565308 19.0
0 -0.00989735126495361 20.0
0 -0.0264014005661011 21.0
0 -0.089917778968811 22.0
29.539669225 29.0638751983643 23.0
0 0.000996947288513184 24.0
0 0.605912685394287 25.0
0 0.0733598470687866 26.0
0 0.00662124156951904 27.0
0 -0.020630955696106 28.0
0 -0.0332502126693726 29.0
0 -0.0384137630462646 30.0
0 -0.0316413640975952 31.0
0 -0.0227786302566528 32.0
0 0.00539672374725342 33.0
0 -0.0252338647842407 34.0
0 0.0194541215896606 35.0
0 -0.0108710527420044 36.0
0 -0.0481292009353638 37.0
0 0.0112260580062866 38.0
0 0.019722580909729 39.0
0 0.0323375463485718 40.0
0 -0.0430974960327148 41.0
0 -0.039706826210022 42.0
302.90434079 294.857543945312 43.0
0 -0.00956308841705322 44.0
0 0.0227501392364502 45.0
0 -0.0283075571060181 46.0
0 -0.0203589200973511 47.0
0 -0.014499306678772 48.0
0 0.321932911872864 49.0
0.080782961659 0.0445533990859985 50.0
1.9569118886 0.972177088260651 51.0
0 -0.0371462106704712 52.0
0 -0.0371938943862915 53.0
0 0.059087872505188 54.0
0 -0.116060376167297 55.0
0 -0.00295341014862061 56.0
0 -0.0224238634109497 57.0
0 -0.0277272462844849 58.0
0 -0.0719059705734253 59.0
0 -0.0195926427841187 60.0
0 0.024170994758606 61.0
0 0.871173143386841 62.0
240.74163718 250.729110717773 0.0
0 -0.0885082483291626 1.0
0 0.0182391405105591 2.0
0 0.0448151826858521 3.0
0 -0.0426682233810425 4.0
0 -0.0712164640426636 5.0
0 -0.0570665597915649 6.0
0 0.00544703006744385 7.0
0 -0.00934040546417236 8.0
0 0.00438868999481201 9.0
0 -0.0491193532943726 10.0
0 -0.0198879241943359 11.0
0 -0.0472398996353149 12.0
55.298184691 54.395191192627 13.0
29.44232705 42.0331153869629 14.0
0 0.0238322019577026 15.0
0 -0.0580371618270874 16.0
34.202829277 34.2009429931641 17.0
0 0.0145314931869507 18.0
0 -0.00784385204315186 19.0
0 0.0230299234390259 20.0
0 -0.0327125787734985 21.0
0 -0.0109440088272095 22.0
41.510042373 40.3643989562988 23.0
0 0.0500949621200562 24.0
0 0.445614874362946 25.0
0 0.0316356420516968 26.0
0 0.0759516954421997 27.0
0 0.0327152013778687 28.0
0 -0.0328420400619507 29.0
0 0.00833475589752197 30.0
0 -0.0410431623458862 31.0
0 0.0357812643051147 32.0
0 -0.0646544694900513 33.0
0 -0.0507053136825562 34.0
0 0.0301892757415771 35.0
0 0.013097882270813 36.0
0 -0.0508791208267212 37.0
0 -0.0100857019424438 38.0
0 -0.0612462759017944 39.0
0 -0.00712120532989502 40.0
0 -0.0710543394088745 41.0
0 -0.00458967685699463 42.0
296.43050365 296.400268554688 43.0
0 0.0267671346664429 44.0
0 0.0166723728179932 45.0
0 -0.0683339834213257 46.0
0 -0.0900501012802124 47.0
0 0.0487474203109741 48.0
0 0.350399166345596 49.0
0.078597635548 0.0756813287734985 50.0
1.9626192259 0.998274743556976 51.0
0 -0.0569945573806763 52.0
0 -0.0371721982955933 53.0
0 0.0275882482528687 54.0
0 -0.158353209495544 55.0
0 0.138070702552795 56.0
0 0.0415033102035522 57.0
0 -0.0758069753646851 58.0
0 -0.00682628154754639 59.0
0 -0.145645499229431 60.0
0 -0.0259982347488403 61.0
0 0.830961942672729 62.0
281.47580065 258.960632324219 0.0
0 -0.144383549690247 1.0
0 0.0749257802963257 2.0
0 0.0526903867721558 3.0
0 -0.0249749422073364 4.0
0 0.0467504262924194 5.0
0 -0.314604878425598 6.0
0 0.00710093975067139 7.0
0 -0.0399622321128845 8.0
0 0.142677664756775 9.0
0 -0.0577841997146606 10.0
0 -0.0383013486862183 11.0
0 0.0880120992660522 12.0
65.369333217 64.6034393310547 13.0
38.10009818 31.8267574310303 14.0
0 -0.0999199151992798 15.0
0 -0.0440222024917603 16.0
36.107295901 36.6852073669434 17.0
0 0.0212932825088501 18.0
0 -0.0164881944656372 19.0
0 0.0428265333175659 20.0
0 0.0180425643920898 21.0
0 -0.0877991914749146 22.0
41.036021393 41.1304016113281 23.0
0 0.0390242338180542 24.0
0 0.487443745136261 25.0
0 0.0542258024215698 26.0
0 0.00750726461410522 27.0
0 0.000338912010192871 28.0
0 -0.0200883150100708 29.0
0 -0.0829373598098755 30.0
0 -0.0382140874862671 31.0
0 0.0182446241378784 32.0
0 0.0570353269577026 33.0
0 -0.00674140453338623 34.0
0 -0.0470123291015625 35.0
0 -0.0110393762588501 36.0
0 -0.205522179603577 37.0
0 0.0243328809738159 38.0
0 0.0378297567367554 39.0
0 0.136190295219421 40.0
0 -0.0146812200546265 41.0
0 0.0287755727767944 42.0
293.33861992 295.711029052734 43.0
0 0.0232003927230835 44.0
0 0.0852696895599365 45.0
0 -0.0342875719070435 46.0
0 -0.0125628709793091 47.0
0 0.0649679899215698 48.0
0 0.350294232368469 49.0
0.069263331402 0.0359033346176147 50.0
2.2690501695 0.876551628112793 51.0
0 -0.0861855745315552 52.0
0 0.000350356101989746 53.0
0 0.0267409086227417 54.0
0 -0.0642613172531128 55.0
0 -0.0411616563796997 56.0
0 0.0379732847213745 57.0
0 -0.115710139274597 58.0
0 0.0355356931686401 59.0
0 0.0757769346237183 60.0
0 -0.0651136636734009 61.0
0 0.850106000900269 62.0
265.08575898 252.336212158203 0.0
0 -0.121332764625549 1.0
0 -0.0433577299118042 2.0
0 -0.0257729291915894 3.0
0 -0.0693100690841675 4.0
0 -0.0742374658584595 5.0
0 0.1046541929245 6.0
0 0.0237573385238647 7.0
0 0.00504779815673828 8.0
0 0.109445929527283 9.0
0 -0.0232712030410767 10.0
0 0.00878119468688965 11.0
0 -0.176327347755432 12.0
47.541737455 46.761962890625 13.0
37.588862226 37.8098754882812 14.0
0 -0.121408104896545 15.0
0 -0.0458015203475952 16.0
26.832490607 25.8554592132568 17.0
0 -0.01909339427948 18.0
0 -0.00526416301727295 19.0
0 0.00895082950592041 20.0
0 -0.0185502767562866 21.0
0 0.006095290184021 22.0
36.183459938 35.1911659240723 23.0
0 0.051085352897644 24.0
18.43676522 0.725603878498077 25.0
0 0.0911301374435425 26.0
0 0.0274041891098022 27.0
0 0.0655654668807983 28.0
0 -0.0918599367141724 29.0
0 0.16307270526886 30.0
0 -0.0136553049087524 31.0
0 -0.0648180246353149 32.0
0 0.022497296333313 33.0
0 -0.041540265083313 34.0
0 -0.007438063621521 35.0
0 0.104570984840393 36.0
0 -0.0367776155471802 37.0
0 -0.0336576700210571 38.0
0 0.00949752330780029 39.0
0 0.227906107902527 40.0
0 -0.0376167297363281 41.0
0 -0.0322366952896118 42.0
296.09064114 295.175872802734 43.0
0 0.0284169912338257 44.0
0 0.00955009460449219 45.0
0 -0.0281451940536499 46.0
0 0.0134001970291138 47.0
0 0.0140160322189331 48.0
0 0.235744833946228 49.0
0.092621249111 0.193917751312256 50.0
2.0179596507 0.87404191493988 51.0
0 -0.0406695604324341 52.0
0 -0.0770150423049927 53.0
0 -0.0193194150924683 54.0
0 -0.151004195213318 55.0
0 -0.046662449836731 56.0
0 0.0259970426559448 57.0
0 0.00764358043670654 58.0
0 -0.0168617963790894 59.0
0 -0.0243188142776489 60.0
0 0.0142961740493774 61.0
0 0.770442008972168 62.0
220.98176151 261.944488525391 0.0
0 -0.0736271142959595 1.0
0 0.036852240562439 2.0
0 -0.017253041267395 3.0
0 -0.0401183366775513 4.0
0 -0.0329679250717163 5.0
0 -0.0826975107192993 6.0
0 -0.0415948629379272 7.0
0 -0.0219336748123169 8.0
0 0.0689100027084351 9.0
0 -0.0623239278793335 10.0
0 -0.025036096572876 11.0
0 -0.0448528528213501 12.0
60.664408657 59.5506057739258 13.0
35.626502929 36.268253326416 14.0
0 -0.065005898475647 15.0
0 -0.0699838399887085 16.0
25.177379079 23.5588684082031 17.0
0 0.0604938268661499 18.0
0 0.0237950086593628 19.0
0 -0.00583302974700928 20.0
0 -0.0408657789230347 21.0
0 0.0404757261276245 22.0
42.389579004 40.7377014160156 23.0
0 0.116267085075378 24.0
0 0.384083211421967 25.0
0 -0.0393053293228149 26.0
0 0.00532424449920654 27.0
0 -0.042756199836731 28.0
0 -0.073372483253479 29.0
0 -0.00606286525726318 30.0
0 -0.0124906301498413 31.0
0 -0.0174151659011841 32.0
0 -0.0565773248672485 33.0
0 -0.0535582304000854 34.0
0 0.0402873754501343 35.0
0 0.211750864982605 36.0
0 -0.0553334951400757 37.0
0 0.0430573225021362 38.0
0 0.0520836114883423 39.0
0 0.0732158422470093 40.0
0 0.0552730560302734 41.0
0 -0.0528923273086548 42.0
278.13717784 297.990478515625 43.0
0 -0.0285297632217407 44.0
0 0.0745669007301331 45.0
0 0.00293982028961182 46.0
0 -0.0462936162948608 47.0
0 0.00375258922576904 48.0
0 0.235567688941956 49.0
0.077020632143 0.219655156135559 50.0
1.6556282596 0.968852400779724 51.0
0 -0.0980626344680786 52.0
0 -0.0278993844985962 53.0
0 0.0198935270309448 54.0
0 -0.0531033277511597 55.0
0 -0.0658344030380249 56.0
0 0.0214685201644897 57.0
0 -0.0529457330703735 58.0
0 -0.0230172872543335 59.0
0 0.0133002996444702 60.0
0 0.0338584184646606 61.0
0 0.786791205406189 62.0
262.1767559 249.3681640625 0.0
0 -0.0588055849075317 1.0
0 0.0932713747024536 2.0
0 0.0247172117233276 3.0
0 -0.0550574064254761 4.0
0 -0.0457297563552856 5.0
0 0.0412801504135132 6.0
0 0.00716578960418701 7.0
0 -0.0109487771987915 8.0
0 0.0895437002182007 9.0
0 0.0351814031600952 10.0
0 0.0230817794799805 11.0
0 0.161584258079529 12.0
66.614808386 67.1088180541992 13.0
37.375906697 33.0085792541504 14.0
0 0.0332659482955933 15.0
0 0.01543128490448 16.0
38.921121013 37.523868560791 17.0
0 -0.0463129281997681 18.0
0 -0.0123296976089478 19.0
0 0.0153411626815796 20.0
0 -0.0580844879150391 21.0
0 -0.0328322649002075 22.0
37.602040355 36.8422660827637 23.0
0 -0.0854355096817017 24.0
0 0.6351637840271 25.0
0 -0.0111464262008667 26.0
0 0.0882595777511597 27.0
0 0.0182875394821167 28.0
0 -0.188344597816467 29.0
0 0.0523356199264526 30.0
0 -0.0540369749069214 31.0
0 -0.0476056337356567 32.0
0 -0.0502344369888306 33.0
0 -0.0793918371200562 34.0
0 0.00691568851470947 35.0
0 0.0972851514816284 36.0
0 -0.0293818712234497 37.0
0 -0.0676513910293579 38.0
0 -0.037378191947937 39.0
0 0.0285314321517944 40.0
0 0.124067664146423 41.0
0 -0.11585009098053 42.0
296.76839063 292.048645019531 43.0
0 -0.0629783868789673 44.0
0 0.027856171131134 45.0
0 0.00452196598052979 46.0
0 -0.0502680540084839 47.0
0 0.0158184766769409 48.0
0 0.26190197467804 49.0
0.075718564897 0.110155582427979 50.0
2.4441437361 0.864972829818726 51.0
0 -0.0290170907974243 52.0
0 -0.0375293493270874 53.0
0 -0.0108269453048706 54.0
0 -0.0380409955978394 55.0
0 -0.0715054273605347 56.0
0 -0.00148594379425049 57.0
0 -0.00602686405181885 58.0
0 -0.0129517316818237 59.0
0 -0.0470470190048218 60.0
0 0.073870062828064 61.0
0 0.814108729362488 62.0
253.67249239 262.412933349609 0.0
0 0.00270020961761475 1.0
0 -0.0935968160629272 2.0
0 -0.0107313394546509 3.0
0 -0.0566636323928833 4.0
0 0.030605673789978 5.0
0 -0.0224109888076782 6.0
0 -0.0506104230880737 7.0
0 0.0540872812271118 8.0
0 0.0215953588485718 9.0
0 -0.04857337474823 10.0
0 -0.0310574769973755 11.0
0 -0.156211495399475 12.0
54.805968869 55.0510215759277 13.0
33.350198973 38.0120162963867 14.0
0 0.00389659404754639 15.0
0 -0.02283775806427 16.0
39.058830177 38.7985076904297 17.0
0 0.0127750635147095 18.0
0 -0.013064980506897 19.0
0 0.0106269121170044 20.0
0 -0.0495949983596802 21.0
0 0.0326794385910034 22.0
29.392543309 28.4237422943115 23.0
0 -0.0863786935806274 24.0
0 0.724169969558716 25.0
0 0.0401557683944702 26.0
0 0.0480626821517944 27.0
0 -0.0390135049819946 28.0
0 -0.0416404008865356 29.0
0 -0.0511136054992676 30.0
0 -0.0282987356185913 31.0
0 0.055785059928894 32.0
0 -0.0210584402084351 33.0
0 -0.00274693965911865 34.0
0 -0.0358041524887085 35.0
0 0.0991379022598267 36.0
0 -0.0678137540817261 37.0
0 -0.00687181949615479 38.0
0 0.0332337617874146 39.0
0 0.0193580389022827 40.0
0 0.0390311479568481 41.0
0 -0.00340902805328369 42.0
292.36241154 293.550415039062 43.0
0 -0.0606476068496704 44.0
0 -0.0226621627807617 45.0
0 -0.00492751598358154 46.0
0 -0.0212210416793823 47.0
0 -0.00839054584503174 48.0
0 0.34510001540184 49.0
0.098122924048 0.0643733739852905 50.0
1.5516535049 0.999877691268921 51.0
0 -0.0650912523269653 52.0
0 -0.0557099580764771 53.0
0 0.0779029130935669 54.0
0 -0.110310912132263 55.0
0 -0.0824962854385376 56.0
0 0.0128940343856812 57.0
0 -0.030485987663269 58.0
0 -0.0380429029464722 59.0
0 0.0189980268478394 60.0
0 -0.00755846500396729 61.0
0 0.825412631034851 62.0
};
\addlegendentry{$R^2$=0.996}
\end{axis}

\end{tikzpicture}
}}
        \subfloat [Actual vs predicted edge flows.] {\label{fig:ye_col_base}\resizebox{\figurewidth}{\figureheight}{% This file was created with tikzplotlib v0.10.1.
\begin{tikzpicture}

\definecolor{darkgray176}{RGB}{176,176,176}
\definecolor{lightgray204}{RGB}{204,204,204}

\begin{axis}[
colorbar,
colorbar style={ylabel={edge\_id}},
colormap={mymap}{[1pt]
 rgb(0pt)=(0.12156862745098,0.466666666666667,0.705882352941177);
  rgb(1pt)=(1,0.498039215686275,0.0549019607843137);
  rgb(2pt)=(0.172549019607843,0.627450980392157,0.172549019607843);
  rgb(3pt)=(0.83921568627451,0.152941176470588,0.156862745098039);
  rgb(4pt)=(0.580392156862745,0.403921568627451,0.741176470588235);
  rgb(5pt)=(0.549019607843137,0.337254901960784,0.294117647058824);
  rgb(6pt)=(0.890196078431372,0.466666666666667,0.76078431372549);
  rgb(7pt)=(0.498039215686275,0.498039215686275,0.498039215686275);
  rgb(8pt)=(0.737254901960784,0.741176470588235,0.133333333333333);
  rgb(9pt)=(0.0901960784313725,0.745098039215686,0.811764705882353)
},
legend cell align={left},
legend style={fill opacity=0.8, draw opacity=1, text opacity=1, draw=lightgray204},
point meta max=61,
point meta min=0,
tick align=outside,
tick pos=left,
title={},
x grid style={darkgray176},
xlabel={},
xmajorgrids,
xmin=-102.83625613855, xmax=292.38127904755,
xtick style={color=black},
y grid style={darkgray176},
ylabel={},
ymajorgrids,
ymin=-0.74451944231987, ymax=12.9457553327084,
ytick style={color=black}
]
\addplot [
  colormap={mymap}{[1pt]
 rgb(0pt)=(0.12156862745098,0.466666666666667,0.705882352941177);
  rgb(1pt)=(1,0.498039215686275,0.0549019607843137);
  rgb(2pt)=(0.172549019607843,0.627450980392157,0.172549019607843);
  rgb(3pt)=(0.83921568627451,0.152941176470588,0.156862745098039);
  rgb(4pt)=(0.580392156862745,0.403921568627451,0.741176470588235);
  rgb(5pt)=(0.549019607843137,0.337254901960784,0.294117647058824);
  rgb(6pt)=(0.890196078431372,0.466666666666667,0.76078431372549);
  rgb(7pt)=(0.498039215686275,0.498039215686275,0.498039215686275);
  rgb(8pt)=(0.737254901960784,0.741176470588235,0.133333333333333);
  rgb(9pt)=(0.0901960784313725,0.745098039215686,0.811764705882353)
},
  only marks,
  scatter,
  scatter src=explicit
]
table [x=x, y=y, meta=colordata]{%
x  y  colordata
178.39565096 0.897000193595886 0.0
178.39565096 1.58873641490936 1.0
177.36234042 1.78282642364502 2.0
172.57323517 3.49410843849182 3.0
0 6.80334329605103 4.0
0 4.22865533828735 5.0
-55.368893774 2.16872072219849 6.0
64.842040756 0.87657642364502 7.0
43.010433349 0.456119120121002 8.0
21.831607407 0.319691866636276 9.0
88.077596202 0.729063034057617 10.0
74.438379742 0.0476290099322796 11.0
83.453489772 0.101507313549519 12.0
83.453489772 0.772419452667236 13.0
83.453489772 0.0201497934758663 14.0
0 -0.00479909591376781 15.0
0 0.238034218549728 16.0
0 0.661749958992004 17.0
61.175274948 1.06764626502991 18.0
-61.175274948 1.58335995674133 19.0
-61.175274948 0.423932492733002 20.0
-61.175274948 -0.00932323187589645 21.0
82.736292819 1.72612082958221 22.0
138.89442508 1.59348499774933 23.0
100.87966965 0.653319776058197 24.0
22.295170322 0.116139702498913 25.0
221.6307179 1.24721455574036 26.0
221.6307179 1.33655071258545 27.0
25.676334436 1.57345581054688 28.0
247.30705233 0.793877482414246 29.0
247.30705233 1.72025763988495 30.0
46.259497021 0.994421720504761 31.0
9.3444474078 0.507080376148224 32.0
21.561017871 2.24926161766052 33.0
21.561017871 0.686103045940399 34.0
21.561017871 0.152279183268547 35.0
9.0950162499 0.0275715570896864 36.0
7.8317986219 0.300566673278809 37.0
0 0.0612945854663849 38.0
0 0.14923894405365 39.0
-11.184365553 0.423949927091599 40.0
-25.431737542 0.712782502174377 41.0
4.8101391205 0.430850803852081 42.0
-32.533753475 0.465479046106339 43.0
88.187809462 0.718395888805389 44.0
88.187809462 0.10070925205946 45.0
7.5007979522 0.499637484550476 46.0
2.3579510428 0.314678192138672 47.0
178.39565096 0.410423457622528 48.0
172.57323517 0.154523432254791 49.0
0 0.296533107757568 50.0
64.842040756 0.109616011381149 51.0
88.077596202 0.18127503991127 52.0
83.453489772 0.0382119305431843 53.0
83.453489772 0.132040902972221 54.0
83.453489772 0.199535027146339 55.0
0 3.40774583816528 56.0
82.736292819 0.165195465087891 57.0
221.6307179 0.196117267012596 58.0
247.30705233 0.914599180221558 59.0
21.561017871 -0.0178356319665909 60.0
0 0.0984464138746262 61.0
151.7806869 0.871843338012695 0.0
151.7806869 1.3622442483902 1.0
150.68648537 1.87114548683167 2.0
145.89738009 4.35521030426025 3.0
0 6.4837121963501 4.0
0 6.28897762298584 5.0
-68.015093435 1.06816267967224 6.0
78.946125328 0.454810321331024 7.0
51.577065425 1.07185232639313 8.0
27.369059902 1.04780805110931 9.0
96.593243649 0.988122463226318 10.0
83.932546669 0.333460837602615 11.0
86.985796792 0.131665825843811 12.0
86.985796792 0.268179416656494 13.0
86.985796792 0.0553309060633183 14.0
0 0.0412081256508827 15.0
0 0.208200722932816 16.0
0 1.48119807243347 17.0
59.241963957 1.22124290466309 18.0
-59.241963957 0.848500370979309 19.0
-59.241963957 0.407872825860977 20.0
-59.241963957 0.0195685885846615 21.0
79.052768513 2.97873687744141 22.0
149.5589469 2.93737697601318 23.0
116.37385361 0.380156368017197 24.0
23.497989837 0.152869433164597 25.0
228.61171542 1.75458312034607 26.0
228.61171542 1.42241609096527 27.0
29.161494715 1.36827600002289 28.0
257.77321013 0.596694827079773 29.0
257.77321013 1.73875272274017 30.0
46.580305778 1.05483412742615 31.0
9.232600896 0.641607284545898 32.0
19.810804556 1.53558957576752 33.0
19.810804556 0.884600222110748 34.0
19.810804556 0.156497299671173 35.0
8.9607346528 0.0308963637799025 36.0
7.6933615125 0.375395119190216 37.0
0 0.0192975401878357 38.0
0 0.16846826672554 39.0
-12.027592984 0.763296663761139 40.0
-25.283303353 0.4151591360569 41.0
4.6972652423 0.552306056022644 42.0
-32.881825449 0.715434551239014 43.0
88.799819304 0.585601925849915 44.0
88.799819304 0.0905023962259293 45.0
6.926168941 0.58091151714325 46.0
2.3359134488 0.396687120199203 47.0
151.7806869 0.0994008705019951 48.0
145.89738009 0.372869163751602 49.0
0 0.403727293014526 50.0
78.946125328 0.269232243299484 51.0
96.593243649 0.365414053201675 52.0
86.985796792 0.0756639093160629 53.0
86.985796792 0.049858495593071 54.0
86.985796792 0.135806038975716 55.0
0 1.27156758308411 56.0
79.052768513 -0.0791871547698975 57.0
228.61171542 0.527593553066254 58.0
257.77321013 0.842501521110535 59.0
19.810804556 -0.01637938991189 60.0
0 0.112772114574909 61.0
220.44420911 0.751521408557892 0.0
220.44420911 1.62539339065552 1.0
219.26712736 1.56012260913849 2.0
213.80245664 4.10502290725708 3.0
29.628054306 6.46486806869507 4.0
29.628054306 5.9404878616333 5.0
-49.999667094 0.703748464584351 6.0
62.395740147 0.53276801109314 7.0
36.404889462 1.06928491592407 8.0
25.990850685 1.24133098125458 9.0
81.930439042 0.768099904060364 10.0
67.09510441 0.255156248807907 11.0
69.672388923 0.375814616680145 12.0
69.672388923 0.881547570228577 13.0
69.672388923 0.178788632154465 14.0
0 0.0986429452896118 15.0
0 0.152785912156105 16.0
0 1.81322455406189 17.0
57.960148681 1.74480271339417 18.0
-57.960148681 1.40162110328674 19.0
-57.960148681 1.06624400615692 20.0
-57.960148681 0.00882621109485626 21.0
80.675216949 0.834544897079468 22.0
138.25559502 0.395111203193665 23.0
105.79775451 0.703991413116455 24.0
10.283836638 0.073946014046669 25.0
218.93081197 0.766755282878876 26.0
218.93081197 1.7447155714035 27.0
26.718184325 0.963796198368073 28.0
245.64899629 0.882391095161438 29.0
245.64899629 1.71883320808411 30.0
46.172755472 0.871015727519989 31.0
9.215190993 0.436854809522629 32.0
22.715068268 0.928985238075256 33.0
22.715068268 1.28578317165375 34.0
22.715068268 0.0891638994216919 35.0
9.8790107264 0.0339881107211113 36.0
8.712229798 0.389713078737259 37.0
0 0.011670857667923 38.0
0 0.0934180617332458 39.0
-11.575250235 0.61806333065033 40.0
-24.997617742 0.835859537124634 41.0
4.6501096316 0.48990660905838 42.0
-31.961017249 0.577277898788452 43.0
95.19618413 0.733626902103424 44.0
95.19618413 0.0183601155877113 45.0
5.9880715378 0.67223584651947 46.0
2.2988183924 0.46032190322876 47.0
220.44420911 0.102386221289635 48.0
213.80245664 0.177382171154022 49.0
29.628054306 0.219151049852371 50.0
62.395740147 0.293311893939972 51.0
81.930439042 0.225199609994888 52.0
69.672388923 0.0359030403196812 53.0
69.672388923 0.141960144042969 54.0
69.672388923 0.096864327788353 55.0
0 8.10684967041016 56.0
80.675216949 0.165379345417023 57.0
218.93081197 0.364276170730591 58.0
245.64899629 1.06316041946411 59.0
22.715068268 0.0223172791302204 60.0
0 0.0678385049104691 61.0
195.32705319 0.603199124336243 0.0
195.32705319 0.665219604969025 1.0
194.20585615 0.642011702060699 2.0
193.14081947 3.8685941696167 3.0
0 10.6769104003906 4.0
0 5.3331127166748 5.0
-61.791654647 0.441094160079956 6.0
72.330981955 1.00442123413086 7.0
46.750045633 1.32094502449036 8.0
25.580936321 1.14503300189972 9.0
70.564264988 0.806765556335449 10.0
58.148165375 0.337265133857727 11.0
52.224527271 1.12565994262695 12.0
52.224527271 0.726451814174652 13.0
52.224527271 0.144951686263084 14.0
31.524665793 0.28086456656456 15.0
31.524665793 0.17372240126133 16.0
0 1.56986105442047 17.0
58.618262874 0.943359017372131 18.0
-58.618262874 0.701743364334106 19.0
-59.390517986 0.864457666873932 20.0
-59.464975355 0.0373794659972191 21.0
80.66210494 1.75585174560547 22.0
143.22863055 1.43997704982758 23.0
109.8360426 0.450239986181259 24.0
20.305304487 0.289081305265427 25.0
223.89073549 0.890799045562744 26.0
223.89073549 1.98843622207642 27.0
28.459008743 1.4215669631958 28.0
252.34974424 0.359516650438309 29.0
252.34974424 1.3664972782135 30.0
46.930376966 0.965012848377228 31.0
9.4577481024 0.456047594547272 32.0
21.197129585 1.36874532699585 33.0
21.197129585 1.00366234779358 34.0
21.197129585 0.119264639914036 35.0
8.9730452589 0.0248632431030273 36.0
7.7726427384 0.349719673395157 37.0
0 0.0384063757956028 38.0
0 0.149125948548317 39.0
-11.512642313 0.512692511081696 40.0
-24.458352297 0.514821171760559 41.0
5.2777171949 0.575326085090637 42.0
-32.464477055 0.572813272476196 43.0
93.130523104 0.678594887256622 44.0
93.130523104 0.0705392211675644 45.0
7.4361482286 0.6156325340271 46.0
2.2886657932 0.422527492046356 47.0
195.32705319 0.187349632382393 48.0
193.14081947 0.0406466461718082 49.0
0 0.716127216815948 50.0
72.330981955 0.295210868120193 51.0
70.564264988 0.415303915739059 52.0
52.224527271 0.0348702482879162 53.0
52.224527271 0.429040044546127 54.0
52.224527271 0.0338524580001831 55.0
0 2.01581478118896 56.0
80.66210494 -0.0556417666375637 57.0
223.89073549 0.344884723424911 58.0
252.34974424 0.0809889882802963 59.0
21.197129585 -0.0249373354017735 60.0
0 0.113750673830509 61.0
161.34073922 1.03725433349609 0.0
161.34073922 0.64987325668335 1.0
160.33532858 0.636067688465118 2.0
155.54622352 2.64092016220093 3.0
0 6.24757099151611 4.0
0 3.48536729812622 5.0
-55.101175941 1.36698186397552 6.0
63.988308112 1.31666874885559 7.0
56.698021865 0.420548856258392 8.0
7.2902862474 0.332475543022156 9.0
91.169445775 0.956353724002838 10.0
79.619116321 0.121571384370327 11.0
89.697383113 0.707952678203583 12.0
89.697383113 0.78271496295929 13.0
89.697383113 0.0961966961622238 14.0
0 0.480320274829865 15.0
0 0.188452571630478 16.0
0 1.91979825496674 17.0
59.914500193 1.74893236160278 18.0
-59.914500193 0.934071242809296 19.0
-59.914500193 0.878089070320129 20.0
-59.914500193 -0.0964147001504898 21.0
78.082906466 1.8915331363678 22.0
151.65463326 1.52468132972717 23.0
113.35219831 0.220784276723862 24.0
22.575781225 0.407794773578644 25.0
229.73753972 1.42656171321869 26.0
229.73753972 2.37674140930176 27.0
23.221503893 1.01831293106079 28.0
252.95904362 0.396793335676193 29.0
252.95904362 1.65413677692413 30.0
47.172672539 1.1913366317749 31.0
9.2247697112 0.688685178756714 32.0
18.168406273 2.03089237213135 33.0
18.168406273 1.00418674945831 34.0
18.168406273 0.109481632709503 35.0
8.4775773272 0.0187963303178549 36.0
7.164937557 0.37841933965683 37.0
0 -0.0108158066868782 38.0
0 0.2249715924263 39.0
-10.783338732 0.246066987514496 40.0
-23.99407369 0.761885941028595 41.0
5.5888440999 0.509201526641846 42.0
-32.077895107 0.744647264480591 43.0
108.18474435 0.681315958499908 44.0
108.18474435 0.135600045323372 45.0
7.5457975382 0.542852997779846 46.0
3.0653121431 0.364731132984161 47.0
161.34073922 0.167678132653236 48.0
155.54622352 0.0236894153058529 49.0
0 0.389586925506592 50.0
63.988308112 0.130337759852409 51.0
91.169445775 0.00733706122264266 52.0
89.697383113 0.0951224565505981 53.0
89.697383113 0.112129129469395 54.0
89.697383113 0.261157333850861 55.0
0 0.412387758493423 56.0
78.082906466 0.177047073841095 57.0
229.73753972 1.12328469753265 58.0
252.95904362 0.494286328554153 59.0
18.168406273 0.0395279489457607 60.0
0 0.103622496128082 61.0
166.97422001 1.20312917232513 0.0
166.97422001 1.06542026996613 1.0
166.02940557 1.10116398334503 2.0
161.24029972 4.2230396270752 3.0
0 6.40469598770142 4.0
0 4.25156784057617 5.0
-69.806116249 2.03881239891052 6.0
79.176157958 0.335633903741837 7.0
39.021656612 0.728392839431763 8.0
40.154501347 0.673066020011902 9.0
96.682881853 1.28715431690216 10.0
83.830953626 0.0864252671599388 11.0
91.955454241 0.477617502212524 12.0
91.955454241 0.373677641153336 13.0
91.955454241 0.150549665093422 14.0
0 0.312334150075912 15.0
0 0.195252165198326 16.0
0 1.06927299499512 17.0
59.946366827 1.55511486530304 18.0
-59.946366827 1.50705635547638 19.0
-59.946366827 0.245269849896431 20.0
-59.946366827 0.00848552957177162 21.0
79.377039945 1.83418190479279 22.0
149.15103921 1.74442219734192 23.0
113.9987639 0.268942534923553 24.0
22.486225681 0.173304289579391 25.0
228.52807916 1.57918739318848 26.0
228.52807916 2.20780277252197 27.0
22.439533669 1.52512431144714 28.0
250.96761283 0.4606893658638 29.0
250.96761283 1.81427824497223 30.0
46.439591137 1.13293695449829 31.0
9.4025667196 0.657417237758636 32.0
19.430673119 1.80233788490295 33.0
19.430673119 0.976195335388184 34.0
19.430673119 0.151799440383911 35.0
8.9263180875 0.0332607477903366 36.0
7.6007414471 0.392228960990906 37.0
0 0.0130760706961155 38.0
0 0.182092979550362 39.0
-11.732139243 0.649827301502228 40.0
-25.68331321 0.538100183010101 41.0
4.9819087657 0.569990158081055 42.0
-33.026522574 0.610204935073853 43.0
77.481280102 0.377537965774536 44.0
77.481280102 0.0978177338838577 45.0
7.4427724893 0.596290469169617 46.0
2.3401765537 0.464383393526077 47.0
166.97422001 0.137433007359505 48.0
161.24029972 0.228321850299835 49.0
0 0.260633111000061 50.0
79.176157958 0.0933363437652588 51.0
96.682881853 0.153594270348549 52.0
91.955454241 0.0720410197973251 53.0
91.955454241 0.120955608785152 54.0
91.955454241 0.0472984313964844 55.0
0 0.64152455329895 56.0
79.377039945 0.250206470489502 57.0
228.52807916 0.799765765666962 58.0
250.96761283 0.449103057384491 59.0
19.430673119 -0.00309654325246811 60.0
0 0.0934446454048157 61.0
154.72846636 1.05815899372101 0.0
154.72846636 0.834416151046753 1.0
153.72217413 1.85645234584808 2.0
148.93310418 4.00859785079956 3.0
0 6.90768194198608 4.0
0 4.64365386962891 5.0
-48.807213809 1.30537736415863 6.0
59.57265615 0.752476453781128 7.0
52.333214227 0.891521155834198 8.0
7.2394419226 0.7694171667099 9.0
93.953957069 0.998188495635986 10.0
80.655982373 0.0615474097430706 11.0
83.517803764 1.03492784500122 12.0
83.517803764 0.705233991146088 13.0
83.517803764 0.0927074626088142 14.0
0 0.17844545841217 15.0
0 0.211487650871277 16.0
0 0.788682699203491 17.0
56.69912129 1.51886236667633 18.0
-56.69912129 1.52568244934082 19.0
-56.69912129 0.361927360296249 20.0
-56.69912129 0.0278707146644592 21.0
79.18429626 2.26186728477478 22.0
137.08232007 2.04655146598816 23.0
102.55924244 0.290651738643646 24.0
21.151251899 0.26269942522049 25.0
216.26661633 1.29465591907501 26.0
216.26661633 1.77511930465698 27.0
24.676023226 1.17771625518799 28.0
240.94263956 0.574552118778229 29.0
240.94263956 2.05631232261658 30.0
47.169555776 1.26879262924194 31.0
9.4270442028 0.576708137989044 32.0
22.48517497 1.25213885307312 33.0
22.48517497 1.12153291702271 34.0
22.48517497 0.0886764079332352 35.0
10.679339859 0.0345434211194515 36.0
9.3546978041 0.372572422027588 37.0
0 0.0281027294695377 38.0
0 0.129728987812996 39.0
-13.450109003 0.389599621295929 40.0
-27.983746178 0.657602310180664 41.0
5.2729518685 0.350582122802734 42.0
-35.754934264 0.355579376220703 43.0
101.68599922 0.604291677474976 44.0
101.68599922 0.05299361795187 45.0
6.3389399388 0.501432120800018 46.0
2.5460454983 0.415658354759216 47.0
154.72846636 0.137430891394615 48.0
148.93310418 0.379914313554764 49.0
0 0.903941690921783 50.0
59.57265615 0.278475910425186 51.0
93.953957069 0.0350186340510845 52.0
83.517803764 0.0673813074827194 53.0
83.517803764 0.272886365652084 54.0
83.517803764 0.0438274294137955 55.0
0 4.0982723236084 56.0
79.18429626 0.0971790626645088 57.0
216.26661633 0.686363875865936 58.0
240.94263956 0.591013252735138 59.0
22.48517497 -0.0207401402294636 60.0
0 0.100482411682606 61.0
191.7046677 0.912593245506287 0.0
191.7046677 0.746473133563995 1.0
190.53594969 0.891219556331635 2.0
185.07130069 2.84857368469238 3.0
9.6610313443 6.08230352401733 4.0
9.6610313443 5.9383397102356 5.0
-64.727907633 7.0285496711731 6.0
74.815240487 0.530332505702972 7.0
46.574273535 0.732660531997681 8.0
28.240966952 0.921031177043915 9.0
89.88141087 0.94540947675705 10.0
77.189285348 0.0815571621060371 11.0
85.636802943 0.485065013170242 12.0
85.636802943 0.632179379463196 13.0
85.636802943 0.0273796021938324 14.0
0 0.147766873240471 15.0
0 0.200619548559189 16.0
0 0.992217540740967 17.0
55.737240828 1.35970914363861 18.0
-55.737240828 1.41021299362183 19.0
-55.737240828 0.311050534248352 20.0
-55.737240828 0.130816370248795 21.0
75.130748627 2.66880464553833 22.0
135.15299562 2.79338121414185 23.0
99.879142185 0.256517261266708 24.0
19.871042471 0.0868163108825684 25.0
210.28374425 1.16077554225922 26.0
210.28374425 1.40992641448975 27.0
27.378975598 2.03029537200928 28.0
237.66271984 1.28552806377411 29.0
237.66271984 1.92965352535248 30.0
46.194516648 1.12934148311615 31.0
9.3964847935 0.471316963434219 32.0
19.393507799 0.987707793712616 33.0
19.393507799 0.990523517131805 34.0
19.393507799 0.144243285059929 35.0
8.5877342676 0.0506174191832542 36.0
7.5192816643 0.422235429286957 37.0
0 -0.00163649767637253 38.0
0 0.145024389028549 39.0
-11.219414294 0.818245708942413 40.0
-25.028392223 0.378479301929474 41.0
4.8801162352 0.472111582756042 42.0
-32.186427207 0.698368430137634 43.0
91.120624531 0.592942953109741 44.0
91.120624531 0.126396849751472 45.0
7.1260670416 0.462452411651611 46.0
2.3194071529 0.398118913173676 47.0
191.7046677 0.160127356648445 48.0
185.07130069 0.0588992349803448 49.0
9.6610313443 0.27007594704628 50.0
74.815240487 0.282154977321625 51.0
89.88141087 0.0978361368179321 52.0
85.636802943 0.0412316136062145 53.0
85.636802943 0.061195645481348 54.0
85.636802943 0.286332458257675 55.0
0 2.82615184783936 56.0
75.130748627 0.0663185715675354 57.0
210.28374425 0.396311402320862 58.0
237.66271984 0.669688522815704 59.0
19.393507799 0.00476035848259926 60.0
0 0.0962195247411728 61.0
162.77884065 0.819650530815125 0.0
162.77884065 0.761602759361267 1.0
161.47908355 1.11395907402039 2.0
156.68997834 3.24514818191528 3.0
0 7.32492876052856 4.0
0 5.39952278137207 5.0
-67.758158239 1.34766662120819 6.0
79.768764002 0.527172267436981 7.0
51.523049355 0.623293876647949 8.0
28.245714647 0.741532206535339 9.0
77.671526048 0.817725837230682 10.0
63.639967669 0.0451573990285397 11.0
76.086908859 0.168185964226723 12.0
76.086908859 0.417200267314911 13.0
76.086908859 0.089668296277523 14.0
0 0.091422937810421 15.0
0 0.201183378696442 16.0
0 1.61033034324646 17.0
57.63155618 0.908748984336853 18.0
-57.63155618 0.632748663425446 19.0
-57.63155618 0.530742883682251 20.0
-57.63155618 0.0125975497066975 21.0
79.323768894 0.979764580726624 22.0
136.21546417 0.590459525585175 23.0
103.89756857 0.929272174835205 24.0
22.566168208 0.0692633986473083 25.0
215.53923306 1.37550985813141 26.0
215.53923306 1.85365891456604 27.0
23.380389437 1.74994313716888 28.0
238.9196225 0.538911163806915 29.0
238.9196225 1.78426802158356 30.0
46.459362396 0.959371209144592 31.0
9.2082431199 0.52911388874054 32.0
21.692212714 1.81334757804871 33.0
21.692212714 1.12609899044037 34.0
21.692212714 0.0953348428010941 35.0
8.7239085318 0.025116603821516 36.0
7.3865546297 0.357100605964661 37.0
0 0.0298520289361477 38.0
0 0.236958876252174 39.0
-11.42656928 0.551432847976685 40.0
-24.393008509 0.458888620138168 41.0
4.6347399973 0.355821967124939 42.0
-31.623476097 0.525248885154724 43.0
104.76119682 0.747435510158539 44.0
104.76119682 0.131441652774811 45.0
6.9550380218 0.474743545055389 46.0
2.7458668384 0.331246852874756 47.0
162.77884065 0.179165825247765 48.0
156.68997834 0.163743585348129 49.0
0 0.583790123462677 50.0
79.768764002 0.117831625044346 51.0
77.671526048 0.0704993307590485 52.0
76.086908859 0.0187991820275784 53.0
76.086908859 0.0681578665971756 54.0
76.086908859 0.172682747244835 55.0
0 4.71998643875122 56.0
79.323768894 0.177330419421196 57.0
215.53923306 0.806072235107422 58.0
238.9196225 0.644359111785889 59.0
21.692212714 -0.00869690999388695 60.0
0 0.0966393351554871 61.0
162.31315976 0.738220393657684 0.0
162.31315976 1.7538548707962 1.0
161.24322198 1.75482761859894 2.0
155.7785513 5.22496509552002 3.0
7.8440668032 7.5547661781311 4.0
7.8440668032 5.39344215393066 5.0
-66.051986455 0.754261016845703 6.0
76.013756312 0.410464376211166 7.0
44.019240789 1.0461061000824 8.0
31.994515523 1.249436378479 9.0
95.105153716 0.863513827323914 10.0
81.592100467 0.377345621585846 11.0
84.39659651 0.840428709983826 12.0
84.39659651 0.649193406105042 13.0
84.39659651 0.0679553151130676 14.0
0 0.171674400568008 15.0
0 0.191333174705505 16.0
0 1.22501146793365 17.0
57.710399227 0.9173304438591 18.0
-57.710399227 0.877446472644806 19.0
-57.710399227 0.49879115819931 20.0
-57.710399227 0.0257939770817757 21.0
78.395055817 1.49520206451416 22.0
146.26215536 1.06700694561005 23.0
115.42089318 0.697436630725861 24.0
21.462856437 0.160861283540726 25.0
224.65721118 1.2199581861496 26.0
224.65721118 1.58814132213593 27.0
24.148228904 1.54786431789398 28.0
248.80544008 1.06691539287567 29.0
248.80544008 1.83701682090759 30.0
47.176224431 1.05236113071442 31.0
9.4369746165 0.396675884723663 32.0
20.684656591 1.56874406337738 33.0
20.684656591 1.0819376707077 34.0
20.684656591 0.106627188622952 35.0
8.764367795 0.0261815767735243 36.0
7.4116452931 0.386825710535049 37.0
0 0.0317695774137974 38.0
0 0.17571796476841 39.0
-11.723084848 0.505690097808838 40.0
-24.97665589 0.491222441196442 41.0
5.5920881626 0.570249676704407 42.0
-33.329860775 0.858883917331696 43.0
99.169970158 0.632143139839172 44.0
99.169970158 0.121445618569851 45.0
6.4991577196 0.549937605857849 46.0
2.5866986538 0.39716237783432 47.0
162.31315976 0.109293520450592 48.0
155.7785513 0.189426764845848 49.0
7.8440668032 0.554151475429535 50.0
76.013756312 0.161573141813278 51.0
95.105153716 0.383453249931335 52.0
84.39659651 0.0798417255282402 53.0
84.39659651 0.175593361258507 54.0
84.39659651 0.100454621016979 55.0
0 3.585618019104 56.0
78.395055817 0.133874893188477 57.0
224.65721118 0.266453862190247 58.0
248.80544008 0.501870274543762 59.0
20.684656591 -0.00633961521089077 60.0
0 0.0817824006080627 61.0
193.79210233 1.12779009342194 0.0
193.79210233 1.60015523433685 1.0
192.5180401 1.57032251358032 2.0
187.72897015 6.75783252716064 3.0
0 4.8304615020752 4.0
0 3.6425621509552 5.0
-58.986484654 1.63412022590637 6.0
68.081376502 0.486709475517273 7.0
47.584478224 0.969291388988495 8.0
20.496898278 1.11990308761597 9.0
83.39218524 1.1974778175354 10.0
70.388091581 0.183262899518013 11.0
78.124055737 0.170857146382332 12.0
78.124055737 0.620007157325745 13.0
78.124055737 0.144566714763641 14.0
0 0.0861561894416809 15.0
0 0.198059290647507 16.0
0 1.62755715847015 17.0
57.748360353 1.24402403831482 18.0
-57.748360353 1.49060988426208 19.0
-57.748360353 0.218959271907806 20.0
-57.748360353 0.165827423334122 21.0
78.937131703 1.19099652767181 22.0
137.61173805 0.595765650272369 23.0
104.6452573 0.678064584732056 24.0
22.084336993 0.0877426117658615 25.0
216.54886975 1.05308508872986 26.0
216.54886975 1.61519169807434 27.0
24.846832967 1.3044068813324 28.0
241.39570272 0.257294863462448 29.0
241.39570272 1.50437188148499 30.0
47.141638273 1.07817029953003 31.0
9.2381366476 0.494590640068054 32.0
21.18877135 1.39350342750549 33.0
21.18877135 0.832301497459412 34.0
21.18877135 0.110802240669727 35.0
10.034221705 0.0232088156044483 36.0
8.8323157111 0.413834661245346 37.0
0 0.0270418785512447 38.0
0 0.153616517782211 39.0
-11.822048758 0.532086670398712 40.0
-24.173029904 0.59990406036377 41.0
5.6349324444 0.379547089338303 42.0
-32.286818172 0.49642425775528 43.0
88.131092292 0.590924382209778 44.0
88.131092292 0.0966349616646767 45.0
6.4327748012 0.498488306999207 46.0
1.7055679606 0.393896192312241 47.0
193.79210233 0.128092646598816 48.0
187.72897015 0.271022289991379 49.0
0 0.22571687400341 50.0
68.081376502 0.115371122956276 51.0
83.39218524 0.225663363933563 52.0
78.124055737 0.0596039332449436 53.0
78.124055737 0.0918773636221886 54.0
78.124055737 0.0511950477957726 55.0
0 4.17138147354126 56.0
78.937131703 0.0518117696046829 57.0
216.54886975 0.547520041465759 58.0
241.39570272 0.275984972715378 59.0
21.18877135 0.01072290353477 60.0
0 0.0952836349606514 61.0
136.19195524 0.775245547294617 0.0
136.19195524 2.06276273727417 1.0
135.12119168 1.60635781288147 2.0
130.33208645 2.89579343795776 3.0
0 5.97834634780884 4.0
0 4.3723521232605 5.0
-55.978356664 1.05098450183868 6.0
64.864426898 0.504904866218567 7.0
38.652852061 1.05494487285614 8.0
26.211574837 1.109135389328 9.0
83.445672481 0.937134742736816 10.0
69.932921761 0.46319505572319 11.0
77.493303584 0.589049816131592 12.0
77.493303584 0.358164548873901 13.0
77.493303584 0.0431699156761169 14.0
0 0.149498045444489 15.0
0 0.211169883608818 16.0
0 0.9103142619133 17.0
57.489351552 1.65205800533295 18.0
-57.489351552 1.54749464988708 19.0
-57.489351552 0.526916980743408 20.0
-57.489351552 0.00682202726602554 21.0
78.835478381 1.21256577968597 22.0
149.87259799 1.51239168643951 23.0
114.23447087 0.492879867553711 24.0
21.955909371 0.203295648097992 25.0
228.70807637 0.8635213971138 26.0
228.70807637 2.3397536277771 27.0
28.471910104 2.3579273223877 28.0
257.17998647 1.16990041732788 29.0
257.17998647 1.96367621421814 30.0
46.43882685 1.02124738693237 31.0
9.2106402385 0.400682955980301 32.0
21.346126829 1.42879772186279 33.0
21.346126829 1.21270644664764 34.0
21.346126829 0.152341663837433 35.0
10.093168834 0.0417637005448341 36.0
8.8563919375 0.364458084106445 37.0
0 0.0250324681401253 38.0
0 0.178564012050629 39.0
-11.733398698 0.717064082622528 40.0
-23.939848486 0.552588164806366 41.0
4.8339815765 0.48657438158989 42.0
-31.183108218 0.580268263816833 43.0
91.519881952 0.719646036624908 44.0
91.519881952 0.079052560031414 45.0
6.4058019718 0.60661393404007 46.0
2.2531729845 0.369420170783997 47.0
136.19195524 0.209949344396591 48.0
130.33208645 0.135687425732613 49.0
0 0.404732406139374 50.0
64.864426898 0.171044290065765 51.0
83.445672481 0.425220429897308 52.0
77.493303584 0.129249513149261 53.0
77.493303584 0.151779770851135 54.0
77.493303584 0.140048623085022 55.0
0 3.96750736236572 56.0
78.835478381 0.0629912763834 57.0
228.70807637 0.55138772726059 58.0
257.17998647 0.652452707290649 59.0
21.346126829 -0.0163786225020885 60.0
0 0.110538117587566 61.0
210.73380155 0.792632281780243 0.0
210.73380155 0.918968796730042 1.0
209.49684175 1.00102019309998 2.0
204.06293178 3.30815577507019 3.0
11.978665577 7.76412105560303 4.0
11.978665577 5.16236591339111 5.0
-63.534244979 1.66471338272095 6.0
72.977379928 0.557409703731537 7.0
51.814506621 0.628303647041321 8.0
21.162873307 0.560734033584595 9.0
66.291836894 0.96876859664917 10.0
54.242134132 0.752441048622131 11.0
55.516107708 0.835130333900452 12.0
55.516107708 0.106793284416199 13.0
55.516107708 0.0680265426635742 14.0
26.791545875 0.125735193490982 15.0
26.791545875 0.221669837832451 16.0
0 1.24758350849152 17.0
50.931766784 1.53998649120331 18.0
-50.931766784 1.72960913181305 19.0
-51.675190621 0.489895761013031 20.0
-51.748030154 0.0650260373950005 21.0
74.544768876 2.95483064651489 22.0
140.57024919 3.00307607650757 23.0
106.08033731 0.538199484348297 24.0
14.642941637 0.083908274769783 25.0
215.11501807 2.05462121963501 26.0
215.11501807 2.38407826423645 27.0
25.959338107 2.198073387146 28.0
241.07435618 1.20533990859985 29.0
241.07435618 2.15770816802979 30.0
46.432133648 1.12539112567902 31.0
9.5015955913 0.571937203407288 32.0
22.796738722 0.84996771812439 33.0
22.796738722 1.14000415802002 34.0
22.796738722 0.0771782174706459 35.0
8.9566614774 0.02450767531991 36.0
7.5787945342 0.406580001115799 37.0
0 0.0182093530893326 38.0
0 0.0966282784938812 39.0
-12.949427812 0.339345395565033 40.0
-25.380110803 0.968964695930481 41.0
5.6758281815 0.421035647392273 42.0
-33.72659647 0.641927242279053 43.0
85.882385089 0.594341278076172 44.0
85.882385089 0.0897263884544373 45.0
6.6891433204 0.59262752532959 46.0
2.5124167771 0.45561945438385 47.0
210.73380155 0.0837395638227463 48.0
204.06293178 0.0352105349302292 49.0
11.978665577 0.603363156318665 50.0
72.977379928 0.104045338928699 51.0
66.291836894 0.50175815820694 52.0
55.516107708 0.194507747888565 53.0
55.516107708 0.0162782873958349 54.0
55.516107708 0.068321168422699 55.0
0 3.05494976043701 56.0
74.544768876 0.0668855831027031 57.0
215.11501807 0.941341936588287 58.0
241.07435618 0.847794711589813 59.0
22.796738722 -0.0152255687862635 60.0
0 0.135966926813126 61.0
184.44566872 0.937111794948578 0.0
184.44566872 1.22135591506958 1.0
182.98615719 1.25799369812012 2.0
178.19705207 4.08820152282715 3.0
0 8.60493278503418 4.0
0 2.42505407333374 5.0
-60.013350955 0.248403653502464 6.0
72.688915374 0.565589427947998 7.0
55.857236473 0.968792319297791 8.0
16.831678901 1.13584566116333 9.0
86.293110239 1.10711026191711 10.0
72.559124947 0.0714810416102409 11.0
83.281568749 0.927276253700256 12.0
83.281568749 0.594218850135803 13.0
83.281568749 0.0802421569824219 14.0
0 0.152954921126366 15.0
0 0.207465425133705 16.0
0 1.23981285095215 17.0
51.732459665 1.36963081359863 18.0
-51.732459665 1.16150939464569 19.0
-51.732459665 0.334911495447159 20.0
-51.732459665 0.0220670849084854 21.0
71.424537315 1.57769620418549 22.0
144.61216323 0.750544607639313 23.0
107.95857266 0.478187024593353 24.0
20.403313852 0.107569962739944 25.0
216.03670054 1.49362361431122 26.0
216.03670054 1.73610126972198 27.0
24.997491533 1.80531704425812 28.0
241.03419208 0.506998479366302 29.0
241.03419208 1.44820868968964 30.0
46.264935932 1.06960797309875 31.0
9.3328065988 0.267366588115692 32.0
19.692077649 1.45471668243408 33.0
19.692077649 1.36997926235199 34.0
19.692077649 0.122296743094921 35.0
8.3054949708 0.0286614261567593 36.0
6.9249748799 0.339000701904297 37.0
0 0.00306982547044754 38.0
0 0.200252503156662 39.0
-11.94185057 0.605828523635864 40.0
-25.592128762 0.670734524726868 41.0
5.358767706 0.595102906227112 42.0
-33.302821924 0.745101153850555 43.0
90.518912244 0.598672568798065 44.0
90.518912244 0.104176670312881 45.0
6.6505819037 0.497605264186859 46.0
2.4060836576 0.336621701717377 47.0
184.44566872 0.157759457826614 48.0
178.19705207 0.170615389943123 49.0
0 0.537286639213562 50.0
72.688915374 0.220576226711273 51.0
86.293110239 0.355727434158325 52.0
83.281568749 0.113056905567646 53.0
83.281568749 0.063385121524334 54.0
83.281568749 0.127753049135208 55.0
0 3.91953706741333 56.0
71.424537315 0.160269558429718 57.0
216.03670054 0.521997332572937 58.0
241.03419208 0.433189362287521 59.0
19.692077649 -0.0116695584729314 60.0
0 0.0808552950620651 61.0
134.78553407 0.849090397357941 0.0
134.78553407 1.65209925174713 1.0
133.49366915 1.74820613861084 2.0
128.70456402 4.25547981262207 3.0
0 5.79339838027954 4.0
0 5.58835935592651 5.0
-70.930851405 2.62094950675964 6.0
80.565261918 0.447545081377029 7.0
51.411549216 0.764748334884644 8.0
29.153712702 0.933320462703705 9.0
96.413323874 0.955195069313049 10.0
84.318876265 0.262353181838989 11.0
87.807791282 0.373795181512833 12.0
87.807791282 0.101632237434387 13.0
87.807791282 0.126602217555046 14.0
0 0.288624942302704 15.0
0 0.141457289457321 16.0
0 1.68076407909393 17.0
59.407022433 1.24286234378815 18.0
-59.407022433 1.08660960197449 19.0
-59.407022433 0.44538289308548 20.0
-59.407022433 0.00671727955341339 21.0
82.21408013 1.32500350475311 22.0
153.19722612 1.41277670860291 23.0
114.3469143 0.254419267177582 24.0
21.668947388 0.256350934505463 25.0
235.41130625 1.0400607585907 26.0
235.41130625 1.34954631328583 27.0
23.74409822 1.52678120136261 28.0
259.15540447 0.605855226516724 29.0
259.15540447 1.80811297893524 30.0
46.307578434 1.10341286659241 31.0
9.2154890945 0.397806793451309 32.0
22.807057697 1.70113658905029 33.0
22.807057697 0.809662103652954 34.0
22.807057697 0.0681781396269798 35.0
9.9606526889 0.0247736647725105 36.0
8.6827856039 0.383650869131088 37.0
0 0.0275175645947456 38.0
0 0.124180614948273 39.0
-11.177159258 0.440786868333817 40.0
-24.319976343 0.876962184906006 41.0
5.4908753829 0.425019264221191 42.0
-32.325184773 0.634947597980499 43.0
91.388110069 0.726804435253143 44.0
91.388110069 0.081166610121727 45.0
7.0237285675 0.515099167823792 46.0
2.4182920096 0.351556092500687 47.0
134.78553407 0.119799621403217 48.0
128.70456402 0.320823282003403 49.0
0 0.764109194278717 50.0
80.565261918 0.0785816535353661 51.0
96.413323874 0.278705894947052 52.0
87.807791282 0.056299015879631 53.0
87.807791282 0.0478473603725433 54.0
87.807791282 0.0630412995815277 55.0
0 1.7685512304306 56.0
82.21408013 0.138872712850571 57.0
235.41130625 0.476856350898743 58.0
259.15540447 0.519667863845825 59.0
22.807057697 -0.0112681724131107 60.0
0 0.106378987431526 61.0
183.47580046 1.03303360939026 0.0
183.47580046 1.581209897995 1.0
182.36718364 1.82259953022003 2.0
177.57807786 4.25572967529297 3.0
0 5.14929628372192 4.0
0 4.52081108093262 5.0
-54.686298324 1.3604336977005 6.0
63.451775665 1.07837617397308 7.0
48.112628206 0.752574920654297 8.0
15.33914746 0.494898468255997 9.0
91.018163397 1.16223680973053 10.0
79.110912175 0.23300564289093 11.0
80.723558055 0.602774918079376 12.0
80.723558055 0.927148163318634 13.0
80.723558055 0.0618318282067776 14.0
0 0.0460779629647732 15.0
0 0.226767495274544 16.0
0 1.51120781898499 17.0
49.086246469 1.19240021705627 18.0
-49.086246469 0.717473566532135 19.0
-49.086246469 0.839857041835785 20.0
-49.086246469 -0.0820645689964294 21.0
68.044438599 1.2950052022934 22.0
131.57301212 1.0403288602829 23.0
101.09506121 0.393839955329895 24.0
19.357005237 0.100831009447575 25.0
199.61745072 0.77452290058136 26.0
199.61745072 1.87265741825104 27.0
24.12061019 2.4096462726593 28.0
223.73806091 1.16040325164795 29.0
223.73806091 1.97637641429901 30.0
46.031947321 1.0776698589325 31.0
9.3503447099 0.427089184522629 32.0
18.95819213 1.58963251113892 33.0
18.95819213 0.918364763259888 34.0
18.95819213 0.117194138467312 35.0
8.8585430497 0.0358225852251053 36.0
7.7489865203 0.368931204080582 37.0
0 0.0266098938882351 38.0
0 0.158375889062881 39.0
-10.550378701 0.847361028194427 40.0
-22.268211267 0.364006161689758 41.0
4.5568688906 0.546893417835236 42.0
-29.324636941 0.596198320388794 43.0
87.154365029 0.739007651805878 44.0
87.154365029 0.126503646373749 45.0
6.6838155199 0.638674378395081 46.0
2.5334171131 0.453231692314148 47.0
183.47580046 0.124319076538086 48.0
177.57807786 0.140756323933601 49.0
0 0.242346704006195 50.0
63.451775665 0.120525345206261 51.0
91.018163397 0.350719004869461 52.0
80.723558055 0.0163246430456638 53.0
80.723558055 0.166189298033714 54.0
80.723558055 0.289958268404007 55.0
0 3.2481369972229 56.0
68.044438599 0.212623685598373 57.0
199.61745072 0.460266590118408 58.0
223.73806091 1.02865183353424 59.0
18.95819213 0.0130809852853417 60.0
0 0.0937335416674614 61.0
191.69207584 0.718208134174347 0.0
191.69207584 1.1894052028656 1.0
190.30499491 0.825340986251831 2.0
185.73124367 2.88587713241577 3.0
0 7.51268005371094 4.0
0 4.3911657333374 5.0
-69.658849709 1.74569523334503 6.0
79.575641746 0.513293564319611 7.0
45.976259806 0.716474056243896 8.0
33.59938194 0.829150080680847 9.0
70.497780787 0.742122948169708 10.0
57.693303419 0.126519531011581 11.0
60.636300973 1.05400836467743 12.0
60.636300973 0.715066909790039 13.0
60.636300973 0.047685481607914 14.0
29.879480622 0.265599012374878 15.0
29.879480622 0.19958284497261 16.0
0 1.02629792690277 17.0
61.014362604 1.08150410652161 18.0
-61.014362604 1.09042465686798 19.0
-61.014362604 0.570236325263977 20.0
-61.014362604 0.0209720283746719 21.0
81.973937314 1.39662826061249 22.0
145.21749633 2.1008882522583 23.0
112.62926269 0.371569454669952 24.0
24.417861521 0.0969413816928864 25.0
227.19143364 0.319050252437592 26.0
227.19143364 1.71572625637054 27.0
23.450426273 2.00440526008606 28.0
250.64185991 0.663511991500854 29.0
250.64185991 1.65127897262573 30.0
46.696116596 1.09008491039276 31.0
9.5010403237 0.434442937374115 32.0
20.95957471 1.20990979671478 33.0
20.95957471 1.14837217330933 34.0
20.95957471 0.124607875943184 35.0
9.1082102205 0.0497509241104126 36.0
7.7590031034 0.366989612579346 37.0
0 -0.00883424282073975 38.0
0 0.212752625346184 39.0
-11.795560915 0.357474654912949 40.0
-23.928797341 0.782558619976044 41.0
4.6865734872 0.435233473777771 42.0
-31.013728978 0.573823750019073 43.0
88.379716903 0.837380588054657 44.0
88.379716903 0.0836100056767464 45.0
7.565636985 0.671779751777649 46.0
2.715071495 0.376258283853531 47.0
191.69207584 0.125482246279716 48.0
185.73124367 0.000102928228443488 49.0
0 0.187465727329254 50.0
79.575641746 0.161946296691895 51.0
70.497780787 0.219779580831528 52.0
60.636300973 0.0548067800700665 53.0
60.636300973 0.131503388285637 54.0
60.636300973 0.073535144329071 55.0
0 0.773260056972504 56.0
81.973937314 0.066847488284111 57.0
227.19143364 0.19060243666172 58.0
250.64185991 0.309104979038239 59.0
20.95957471 -0.00443005748093128 60.0
0 0.0884017869830132 61.0
179.58012758 0.932492256164551 0.0
179.58012758 1.16333782672882 1.0
178.20898541 1.46374225616455 2.0
172.74431476 4.18384790420532 3.0
6.7874597894 7.96307229995728 4.0
6.7874597894 4.77950286865234 5.0
-67.219455476 0.878912389278412 6.0
79.547904202 0.504559397697449 7.0
45.926745361 0.993200242519379 8.0
33.621158841 0.897962152957916 9.0
78.879891935 1.06877291202545 10.0
66.317062888 0.350277066230774 11.0
74.329788749 0.878965497016907 12.0
74.329788749 0.88913905620575 13.0
74.329788749 0.133844718337059 14.0
0 0.193439662456512 15.0
0 0.13076651096344 16.0
0 1.28804051876068 17.0
51.913368809 1.21234667301178 18.0
-51.913368809 0.936457216739655 19.0
-51.913368809 1.01060593128204 20.0
-51.913368809 -0.0901441276073456 21.0
72.463988169 1.57202804088593 22.0
146.43455777 1.15198183059692 23.0
109.30018011 0.554245948791504 24.0
25.460420071 0.16654446721077 25.0
218.89854594 0.96087235212326 26.0
218.89854594 1.47328960895538 27.0
29.86557461 2.31919360160828 28.0
248.76412055 0.937713265419006 29.0
248.76412055 1.89797401428223 30.0
46.635297917 1.00014865398407 31.0
9.3164012699 0.503281533718109 32.0
20.55061936 2.08689904212952 33.0
20.55061936 0.924225211143494 34.0
20.55061936 0.125832363963127 35.0
9.7095863901 0.0346406772732735 36.0
8.4527611274 0.411132305860519 37.0
0 0.0105335861444473 38.0
0 0.128410845994949 39.0
-13.228672424 0.486759662628174 40.0
-26.242508366 0.530815660953522 41.0
5.2453724483 0.415664047002792 42.0
-34.19722901 0.420917809009552 43.0
106.29876058 0.607405424118042 44.0
106.29876058 0.0835027396678925 45.0
7.3006328917 0.658689320087433 46.0
2.6582956207 0.406009316444397 47.0
179.58012758 0.102780312299728 48.0
172.74431476 0.0150267761200666 49.0
6.7874597894 0.302544504404068 50.0
79.547904202 0.16318841278553 51.0
78.879891935 0.308447450399399 52.0
74.329788749 0.124094903469086 53.0
74.329788749 0.302323192358017 54.0
74.329788749 0.207822099328041 55.0
0 5.76062202453613 56.0
72.463988169 0.15707141160965 57.0
218.89854594 0.247264429926872 58.0
248.76412055 0.582798540592194 59.0
20.55061936 -0.00610827002674341 60.0
0 0.0968185365200043 61.0
167.64017102 0.686731576919556 0.0
167.64017102 2.20942354202271 1.0
166.32337541 1.91867554187775 2.0
161.53427023 4.71050691604614 3.0
0 7.7962646484375 4.0
0 6.74806976318359 5.0
-71.388832888 0.967522501945496 6.0
82.45778327 0.583082497119904 7.0
63.73616098 1.48529815673828 8.0
18.72162229 1.82260227203369 9.0
93.200712698 0.793847799301147 10.0
79.393829135 0.418423026800156 11.0
75.903555152 0.711595177650452 12.0
75.903555152 0.636203169822693 13.0
75.903555152 0.114196471869946 14.0
0 0.050703726708889 15.0
0 0.198582798242569 16.0
0 1.15638411045074 17.0
56.510646544 1.95290207862854 18.0
-56.510646544 2.9148895740509 19.0
-56.510646544 0.316897094249725 20.0
-56.510646544 -0.0273313000798225 21.0
77.808006577 2.0791482925415 22.0
143.10994099 1.98867511749268 23.0
103.46962327 0.611148655414581 24.0
22.484528961 0.152214199304581 25.0
220.91794757 1.39668846130371 26.0
220.91794757 1.94353485107422 27.0
30.286377793 1.82089745998383 28.0
251.20432536 1.26904511451721 29.0
251.20432536 1.9282203912735 30.0
46.986794278 1.02245044708252 31.0
9.4052123541 0.416409224271774 32.0
21.297360032 1.49811947345734 33.0
21.297360032 1.05483424663544 34.0
21.297360032 0.110940836369991 35.0
9.5746609029 0.0527143180370331 36.0
8.1802421526 0.411773025989532 37.0
0 0.0616582408547401 38.0
0 0.0839406102895737 39.0
-11.037077773 0.707005977630615 40.0
-22.843340524 0.37863752245903 41.0
5.0696165202 0.363981068134308 42.0
-30.67628413 0.454649746417999 43.0
86.216392767 0.511613309383392 44.0
86.216392767 0.105426073074341 45.0
6.6776504878 0.539609849452972 46.0
2.2028867 0.42197260260582 47.0
167.64017102 0.137006476521492 48.0
161.53427023 0.204143106937408 49.0
0 0.596436202526093 50.0
82.45778327 0.150595754384995 51.0
93.200712698 0.192959561944008 52.0
75.903555152 0.140940830111504 53.0
75.903555152 0.0685967057943344 54.0
75.903555152 0.130738779902458 55.0
0 9.59357738494873 56.0
77.808006577 0.0207053124904633 57.0
220.91794757 0.949498176574707 58.0
251.20432536 1.12683725357056 59.0
21.297360032 0.0162498652935028 60.0
0 0.077429935336113 61.0
135.40046637 1.12925636768341 0.0
135.40046637 1.38367605209351 1.0
133.93659608 1.55932581424713 2.0
129.14749091 5.03285121917725 3.0
0 4.14965915679932 4.0
0 4.17035865783691 5.0
-61.484626107 4.26067447662354 6.0
71.23174434 0.333965748548508 7.0
52.855553319 1.31180596351624 8.0
18.376191021 1.49418306350708 9.0
95.103390323 1.33315706253052 10.0
79.902458212 0.00930536072701216 11.0
89.838271667 1.3918149471283 12.0
89.838271667 0.913492858409882 13.0
89.838271667 0.0506305173039436 14.0
0 0.145385935902596 15.0
0 0.247993528842926 16.0
0 0.868750154972076 17.0
56.080271802 1.39647102355957 18.0
-56.080271802 2.54913997650146 19.0
-56.080271802 0.217497661709785 20.0
-56.080271802 0.0172119420021772 21.0
79.413965028 1.84278416633606 22.0
136.59891118 1.94237577915192 23.0
106.41119161 0.156675860285759 24.0
23.844433964 0.218965366482735 25.0
216.01287621 1.35025227069855 26.0
216.01287621 1.78444266319275 27.0
24.582921214 1.18552112579346 28.0
240.59579743 0.822231352329254 29.0
240.59579743 2.17515659332275 30.0
47.390177748 0.989883184432983 31.0
9.4312690526 0.512803077697754 32.0
23.333693225 0.888185560703278 33.0
23.333693225 1.31919240951538 34.0
23.333693225 0.0911866426467896 35.0
9.694668824 0.0286378134042025 36.0
8.350263646 0.390055865049362 37.0
0 -0.0336879380047321 38.0
0 0.343721598386765 39.0
-11.480382629 0.200821340084076 40.0
-23.859618629 0.813709378242493 41.0
5.2996284725 0.341973125934601 42.0
-32.143780991 0.489946007728577 43.0
83.318664154 1.07966804504395 44.0
83.318664154 0.0905070602893829 45.0
6.4612648577 0.887285172939301 46.0
2.5089226752 0.379189372062683 47.0
135.40046637 0.153311878442764 48.0
129.14749091 0.407374203205109 49.0
0 0.457481414079666 50.0
71.23174434 0.0950138568878174 51.0
95.103390323 0.288127422332764 52.0
89.838271667 0.0287070851773024 53.0
89.838271667 0.374810129404068 54.0
89.838271667 0.116590529680252 55.0
0 2.4544939994812 56.0
79.413965028 0.0233550854027271 57.0
216.01287621 0.388069301843643 58.0
240.59579743 1.35962808132172 59.0
23.333693225 0.0488425493240356 60.0
0 0.108586154878139 61.0
138.89245231 0.849211513996124 0.0
138.89245231 2.08242464065552 1.0
137.43875624 1.72543060779572 2.0
132.64968629 4.91206359863281 3.0
0 5.32908535003662 4.0
0 5.81851673126221 5.0
-70.822429739 4.45916604995728 6.0
82.256983626 1.20066475868225 7.0
67.251928484 0.804560780525208 8.0
15.005055142 0.58005690574646 9.0
87.310616443 0.921614408493042 10.0
74.741725824 0.0525823831558228 11.0
83.397327699 0.490203380584717 12.0
83.397327699 0.237944006919861 13.0
83.397327699 0.128908291459084 14.0
0 0.148127034306526 15.0
0 0.118571683764458 16.0
0 1.40257155895233 17.0
51.949275442 1.36239850521088 18.0
-51.949275442 1.13572299480438 19.0
-51.949275442 0.569319367408752 20.0
-51.949275442 -0.054731510579586 21.0
73.456416142 1.95791172981262 22.0
136.32384238 1.69312512874603 23.0
97.945129654 0.360513359308243 24.0
20.365203387 0.0810968279838562 25.0
209.78025852 1.05982565879822 26.0
209.78025852 2.15858101844788 27.0
29.338120605 2.50213623046875 28.0
239.11837912 1.10773634910583 29.0
239.11837912 1.81604063510895 30.0
46.259074989 1.02141869068146 31.0
9.2062134706 0.413187563419342 32.0
21.507140701 1.44828915596008 33.0
21.507140701 0.814528107643127 34.0
21.507140701 0.133658021688461 35.0
9.0272644709 0.0292743295431137 36.0
7.7097619081 0.298924714326859 37.0
0 0.0561955720186234 38.0
0 0.110315531492233 39.0
-11.731114675 0.777722060680389 40.0
-24.657412833 0.617259919643402 41.0
5.4126618209 0.542828261852264 42.0
-32.526578293 0.560883760452271 43.0
100.71356522 0.77107173204422 44.0
100.71356522 0.0972417220473289 45.0
6.7427658228 0.642256557941437 46.0
2.4708231766 0.354839593172073 47.0
138.89245231 0.0315590240061283 48.0
132.64968629 0.336218625307083 49.0
0 0.387545257806778 50.0
82.256983626 0.177785471081734 51.0
87.310616443 0.0917340591549873 52.0
83.397327699 0.0235598124563694 53.0
83.397327699 0.147079020738602 54.0
83.397327699 0.0785122439265251 55.0
0 5.69673204421997 56.0
73.456416142 0.168767884373665 57.0
209.78025852 0.237940087914467 58.0
239.11837912 0.918729901313782 59.0
21.507140701 -0.00797668099403381 60.0
0 0.0989310070872307 61.0
180.77311606 0.811296761035919 0.0
180.77311606 2.2559757232666 1.0
179.41890062 1.8440203666687 2.0
174.62979549 3.03498935699463 3.0
0 7.2815523147583 4.0
0 2.81131505966187 5.0
-60.724549852 0.442487627267838 6.0
71.088933537 0.314925014972687 7.0
39.091413438 0.709510385990143 8.0
31.997520098 0.72883677482605 9.0
92.600934016 0.945599496364594 10.0
79.307373083 0.530849874019623 11.0
81.262705094 0.824078142642975 12.0
81.262705094 0.11023560911417 13.0
81.262705094 0.193191647529602 14.0
0 0.275844842195511 15.0
0 0.215866297483444 16.0
0 1.75959658622742 17.0
55.041962455 1.23639988899231 18.0
-55.041962455 1.30762660503387 19.0
-55.041962455 0.711891889572144 20.0
-55.041962455 0.0265590772032738 21.0
76.006205468 1.37401700019836 22.0
154.78445297 0.808846116065979 23.0
120.48057516 0.633202135562897 24.0
25.674650545 0.162595838308334 25.0
230.79065844 0.969108283519745 26.0
230.79065844 2.54430460929871 27.0
29.872781905 3.11645579338074 28.0
260.66344035 1.17420208454132 29.0
260.66344035 1.88954925537109 30.0
47.41485976 1.06205976009369 31.0
9.4426657744 0.590652525424957 32.0
20.964243013 1.19517815113068 33.0
20.964243013 0.659630060195923 34.0
20.964243013 0.0849277675151825 35.0
8.7897023381 0.0204242076724768 36.0
7.5536109827 0.433423668146133 37.0
0 0.0569651052355766 38.0
0 0.151603192090988 39.0
-12.169140723 0.686854660511017 40.0
-25.035754589 0.70138281583786 41.0
4.8437159926 0.407344847917557 42.0
-32.371023328 0.407945364713669 43.0
98.602608762 0.79826545715332 44.0
98.602608762 0.10167457908392 45.0
6.4546248663 0.605568408966064 46.0
2.1507631098 0.416137874126434 47.0
180.77311606 0.0752943456172943 48.0
174.62979549 0.0594911240041256 49.0
0 0.164315342903137 50.0
71.088933537 0.0813108086585999 51.0
92.600934016 0.378185570240021 52.0
81.262705094 0.111314855515957 53.0
81.262705094 0.134360566735268 54.0
81.262705094 0.0558550655841827 55.0
0 4.39775037765503 56.0
76.006205468 0.0497052110731602 57.0
230.79065844 0.151736468076706 58.0
260.66344035 0.556181907653809 59.0
20.964243013 -0.0180930886417627 60.0
0 0.122567147016525 61.0
130.32294901 0.866731882095337 0.0
130.32294901 1.77807509899139 1.0
129.07805226 1.78992307186127 2.0
124.288947 4.66209697723389 3.0
0 5.46263885498047 4.0
0 4.59429788589478 5.0
-52.239840771 0.512606680393219 6.0
62.748155516 1.00691628456116 7.0
50.448143599 1.30450248718262 8.0
12.300011916 1.1713525056839 9.0
87.830748786 1.11121773719788 10.0
75.018336322 0.557977378368378 11.0
83.974141728 0.926695108413696 12.0
83.974141728 0.554048776626587 13.0
83.974141728 0.135875761508942 14.0
0 0.094812422990799 15.0
0 0.135817974805832 16.0
0 1.46477460861206 17.0
53.862425902 1.10783839225769 18.0
-53.862425902 0.928307771682739 19.0
-53.862425902 0.560627698898315 20.0
-53.862425902 -0.00961919128894806 21.0
76.139673054 1.16472983360291 22.0
144.54832179 0.439427196979523 23.0
106.8048423 1.07589960098267 24.0
21.174152271 0.0568195506930351 25.0
220.68799484 0.494076520204544 26.0
220.68799484 1.10304772853851 27.0
25.40953699 1.24319970607758 28.0
246.09753183 0.578298151493073 29.0
246.09753183 1.47761833667755 30.0
47.203744962 0.929930686950684 31.0
9.2581356089 0.386926025152206 32.0
22.277247152 0.846245288848877 33.0
22.277247152 1.47821688652039 34.0
22.277247152 0.0794827118515968 35.0
9.3375874139 0.025366859510541 36.0
8.1317065182 0.344791263341904 37.0
0 0.0161220468580723 38.0
0 0.318559944629669 39.0
-10.888705247 0.295834481716156 40.0
-23.362774666 0.822654902935028 41.0
4.7833637486 0.391553908586502 42.0
-30.6396785 0.572321355342865 43.0
89.242652442 0.654993116855621 44.0
89.242652442 0.138404205441475 45.0
7.1698581126 0.361676186323166 46.0
2.3988377494 0.35977765917778 47.0
130.32294901 0.382737725973129 48.0
124.288947 0.199080467224121 49.0
0 0.423748880624771 50.0
62.748155516 0.328582406044006 51.0
87.830748786 0.776929080486298 52.0
83.974141728 0.0893828347325325 53.0
83.974141728 0.134644001722336 54.0
83.974141728 0.085611417889595 55.0
0 4.66149854660034 56.0
76.139673054 -0.00595263112336397 57.0
220.68799484 0.0533181391656399 58.0
246.09753183 0.245041400194168 59.0
22.277247152 -0.0194294117391109 60.0
0 0.109070912003517 61.0
211.70939637 0.848608434200287 0.0
211.70939637 0.753962755203247 1.0
210.6375261 0.885496079921722 2.0
205.17285532 4.89628505706787 3.0
9.2573247158 9.68085289001465 4.0
9.2573247158 2.62348413467407 5.0
-53.631234473 0.462907314300537 6.0
66.644894356 0.510686814785004 7.0
41.505493866 0.757724523544312 8.0
25.13940049 0.860538065433502 9.0
90.3233333 1.02405714988708 10.0
77.090748988 0.171608835458755 11.0
81.664147468 0.815401077270508 12.0
81.664147468 0.285315960645676 13.0
81.664147468 0.0438075140118599 14.0
0 0.17310819029808 15.0
0 0.182185009121895 16.0
0 1.12339127063751 17.0
50.417050595 1.51723170280457 18.0
-50.417050595 1.49504733085632 19.0
-50.417050595 0.535109639167786 20.0
-50.417050595 -0.0423426106572151 21.0
70.463480821 1.55702221393585 22.0
140.52760588 1.30777478218079 23.0
109.61823875 0.569371223449707 24.0
18.890917573 0.072565920650959 25.0
210.9910867 1.06753265857697 26.0
210.9910867 1.49125969409943 27.0
27.471224585 1.51302945613861 28.0
238.46231129 1.20762979984283 29.0
238.46231129 2.14452719688416 30.0
46.8902309 0.985612154006958 31.0
9.4721482683 0.603666126728058 32.0
20.046430225 1.82566606998444 33.0
20.046430225 0.711434185504913 34.0
20.046430225 0.129243642091751 35.0
9.1173737823 0.025543512776494 36.0
7.7019458304 0.370999872684479 37.0
0 0.0266881808638573 38.0
0 0.186263099312782 39.0
-11.469024175 0.742788136005402 40.0
-25.161505516 0.530533611774445 41.0
5.4140634537 0.426859855651855 42.0
-33.126385403 0.702741086483002 43.0
76.41109574 0.455055564641953 44.0
76.41109574 0.0985858589410782 45.0
6.6116567328 0.410163581371307 46.0
2.802068798 0.38257908821106 47.0
211.70939637 0.0293849594891071 48.0
205.17285532 -0.00905398558825254 49.0
9.2573247158 0.280370116233826 50.0
66.644894356 0.0583736822009087 51.0
90.3233333 0.0830974504351616 52.0
81.664147468 0.0327111221849918 53.0
81.664147468 0.0503186695277691 54.0
81.664147468 0.083639070391655 55.0
0 2.79180026054382 56.0
70.463480821 0.102141365408897 57.0
210.9910867 0.532349407672882 58.0
238.46231129 1.48530411720276 59.0
20.046430225 -0.0171111784875393 60.0
0 0.0890945643186569 61.0
191.66461821 1.00098669528961 0.0
191.66461821 0.786422371864319 1.0
190.33063843 0.749431371688843 2.0
185.54153331 4.28438854217529 3.0
0 8.32187652587891 4.0
0 4.36181259155273 5.0
-57.378751115 0.322116941213608 6.0
69.664955534 1.27839910984039 7.0
56.885731963 0.528330981731415 8.0
12.779223571 0.466250360012054 9.0
89.989215703 1.08569085597992 10.0
77.64776593 0.127796217799187 11.0
85.940045682 0.274537742137909 12.0
85.940045682 0.669096529483795 13.0
85.940045682 0.0537652224302292 14.0
0 0.119842171669006 15.0
0 0.176026374101639 16.0
0 0.730887055397034 17.0
54.675880287 1.67794501781464 18.0
-54.675880287 1.84204578399658 19.0
-54.675880287 0.380776733160019 20.0
-54.675880287 -0.0307357348501682 21.0
76.450520311 1.46139931678772 22.0
139.94730352 1.3204174041748 23.0
103.10272416 0.368164360523224 24.0
20.275102986 0.207401663064957 25.0
216.39782384 0.906457126140594 26.0
216.39782384 1.11088967323303 27.0
24.04516865 1.09897184371948 28.0
240.44299249 0.450269430875778 29.0
240.44299249 1.61190330982208 30.0
47.327921449 1.10446536540985 31.0
9.4451379751 0.393895268440247 32.0
21.774640023 1.52348864078522 33.0
21.774640023 0.986359059810638 34.0
21.774640023 0.117095291614532 35.0
9.649144512 0.0295111052691936 36.0
8.3588150077 0.414514034986496 37.0
0 0.0268261469900608 38.0
0 0.179591804742813 39.0
-11.932523568 0.491461634635925 40.0
-25.088422563 0.628137528896332 41.0
4.6605463949 0.401142776012421 42.0
-32.012708473 0.408178508281708 43.0
97.67403992 0.977030575275421 44.0
97.67403992 0.0690576359629631 45.0
6.7587672025 0.638370633125305 46.0
2.5733209698 0.338607102632523 47.0
191.66461821 0.0865573361515999 48.0
185.54153331 0.00403053779155016 49.0
0 0.238264575600624 50.0
69.664955534 0.207186102867126 51.0
89.989215703 0.161677226424217 52.0
85.940045682 0.0252022463828325 53.0
85.940045682 0.130143567919731 54.0
85.940045682 0.147584959864616 55.0
0 2.49605059623718 56.0
76.450520311 -0.00450381403788924 57.0
216.39782384 0.350575983524323 58.0
240.44299249 0.181869432330132 59.0
21.774640023 -0.0220203343778849 60.0
0 0.100212678313255 61.0
156.69944838 1.20270621776581 0.0
156.69944838 1.86145389080048 1.0
155.19059899 2.17171716690063 2.0
149.72592819 4.87661075592041 3.0
5.7691653889 4.46536254882812 4.0
5.7691653889 3.30468130111694 5.0
-51.413304008 0.180302560329437 6.0
64.665155062 0.469828397035599 7.0
41.745464572 0.57602059841156 8.0
22.91969049 0.727755963802338 9.0
83.822442966 1.22363185882568 10.0
70.445860439 0.0485061556100845 11.0
80.231912313 0.137787461280823 12.0
80.231912313 0.657888114452362 13.0
80.231912313 0.0548693388700485 14.0
0 0.0267674215137959 15.0
0 0.208887204527855 16.0
0 0.580970764160156 17.0
57.377572794 1.45977461338043 18.0
-57.377572794 2.04579186439514 19.0
-57.377572794 0.246485531330109 20.0
-57.377572794 -0.0392543822526932 21.0
79.166356899 1.57543396949768 22.0
130.79799513 0.883365511894226 23.0
101.5590729 0.701558530330658 24.0
19.129209821 0.0653379037976265 25.0
209.96435203 1.10794520378113 26.0
209.96435203 1.88115108013153 27.0
27.78020935 2.05675792694092 28.0
237.74456138 1.2457879781723 29.0
237.74456138 1.74175870418549 30.0
47.314230105 0.927419722080231 31.0
9.3198798254 0.325811952352524 32.0
21.788784104 1.55186343193054 33.0
21.788784104 0.556589841842651 34.0
21.788784104 0.153731361031532 35.0
10.163793573 0.032437801361084 36.0
9.0516314301 0.368306845426559 37.0
0 0.00824399664998055 38.0
0 0.100722342729568 39.0
-10.644302795 0.973398387432098 40.0
-23.194671744 0.258854597806931 41.0
5.1573142645 0.706231236457825 42.0
-31.122618027 0.830653846263885 43.0
77.566111462 0.414094239473343 44.0
77.566111462 0.0934415310621262 45.0
7.5939321872 0.521215915679932 46.0
2.8141910276 0.388813465833664 47.0
156.69944838 0.158754274249077 48.0
149.72592819 0.265308260917664 49.0
5.7691653889 0.0889333859086037 50.0
64.665155062 0.242980644106865 51.0
83.822442966 0.126318290829659 52.0
80.231912313 0.0193645544350147 53.0
80.231912313 0.0297768451273441 54.0
80.231912313 0.0275344103574753 55.0
0 4.88901233673096 56.0
79.166356899 -0.0131834596395493 57.0
209.96435203 0.552640020847321 58.0
237.74456138 0.587813556194305 59.0
21.788784104 -0.0028194896876812 60.0
0 0.102506168186665 61.0
153.19736492 0.991840720176697 0.0
153.19736492 1.689408659935 1.0
152.12942826 1.57607138156891 2.0
147.34032303 5.54113101959229 3.0
0 5.6233024597168 4.0
0 3.86381435394287 5.0
-45.76890034 0.239689871668816 6.0
54.568060474 0.542906045913696 7.0
47.936901005 0.794052243232727 8.0
6.6311594689 0.8128462433815 9.0
83.43661948 1.05149674415588 10.0
70.630701742 0.1286780834198 11.0
79.622484454 0.713234007358551 12.0
79.622484454 0.295650810003281 13.0
79.622484454 0.106670811772346 14.0
0 0.0761048719286919 15.0
0 0.170615881681442 16.0
0 1.47447514533997 17.0
55.68470901 1.14048445224762 18.0
-55.68470901 1.35825645923615 19.0
-55.68470901 0.762977004051208 20.0
-55.68470901 -0.0443017221987247 21.0
74.636119807 3.40998387336731 22.0
138.80562866 3.28022408485413 23.0
106.40100173 0.451373547315598 24.0
23.360229987 0.0955722108483315 25.0
213.44174847 1.74123382568359 26.0
213.44174847 0.891547441482544 27.0
27.428893712 0.982072234153748 28.0
240.87064218 0.252715140581131 29.0
240.87064218 1.6960529088974 30.0
46.222687014 1.05226409435272 31.0
9.3627242053 0.413077414035797 32.0
18.951410797 1.60716986656189 33.0
18.951410797 0.742201507091522 34.0
18.951410797 0.0849849209189415 35.0
8.5308706587 0.0293433703482151 36.0
7.1686728491 0.387864291667938 37.0
0 0.00758121907711029 38.0
0 0.399291038513184 39.0
-11.634020609 0.38741734623909 40.0
-25.532277132 0.77886027097702 41.0
5.41210008 0.390842586755753 42.0
-33.24289132 0.622426092624664 43.0
102.97060432 0.795323014259338 44.0
102.97060432 0.0846874788403511 45.0
5.7818602718 0.692683339118958 46.0
2.1723530603 0.393377423286438 47.0
153.19736492 0.130395799875259 48.0
147.34032303 0.460931867361069 49.0
0 0.680687248706818 50.0
54.568060474 0.117493666708469 51.0
83.43661948 0.0723919421434402 52.0
79.622484454 0.0669429153203964 53.0
79.622484454 0.0186314154416323 54.0
79.622484454 0.0844980254769325 55.0
0 2.75965809822083 56.0
74.636119807 0.0414806939661503 57.0
213.44174847 0.334680557250977 58.0
240.87064218 0.22383625805378 59.0
18.951410797 -0.00373721122741699 60.0
0 0.108195677399635 61.0
135.53478065 0.879077732563019 0.0
135.53478065 0.82087779045105 1.0
133.98891813 1.86890172958374 2.0
128.5242474 5.02594709396362 3.0
3.4595629724 7.10552835464478 4.0
3.4595629724 6.37851572036743 5.0
-75.038568134 0.799232721328735 6.0
86.374987653 0.816137969493866 7.0
57.916761109 0.741687417030334 8.0
28.458226544 0.736337304115295 9.0
79.660955683 0.826469242572784 10.0
67.972956598 0.0252186730504036 11.0
79.39913303 0.858555138111115 12.0
79.39913303 0.0846026539802551 13.0
79.39913303 0.0664407908916473 14.0
0 0.0464287512004375 15.0
0 0.158667802810669 16.0
0 1.32728064060211 17.0
52.013330033 0.945635676383972 18.0
-52.013330033 0.585260808467865 19.0
-52.013330033 0.360071659088135 20.0
-52.013330033 0.23655116558075 21.0
74.115375867 1.8519275188446 22.0
142.70592982 1.29230415821075 23.0
104.19767201 0.411610752344131 24.0
21.78786802 0.123057886958122 25.0
216.82130568 1.60742664337158 26.0
216.82130568 2.15210318565369 27.0
24.75835394 0.825733780860901 28.0
241.57965962 0.459243535995483 29.0
241.57965962 1.38735139369965 30.0
46.145880983 1.09180164337158 31.0
9.3057813852 0.30061274766922 32.0
22.102045834 0.862963914871216 33.0
22.102045834 0.894075095653534 34.0
22.102045834 0.111992120742798 35.0
9.2759403142 0.0206608455628157 36.0
8.0418122459 0.404089033603668 37.0
0 0.0567766278982162 38.0
0 0.238693103194237 39.0
-12.120279323 0.503977954387665 40.0
-25.110804399 0.619689583778381 41.0
5.4532468447 0.348718106746674 42.0
-33.234294261 0.196945995092392 43.0
83.320583491 0.72331964969635 44.0
83.320583491 0.137528449296951 45.0
7.244043578 0.583119034767151 46.0
2.1977487094 0.33678337931633 47.0
135.53478065 0.0872040838003159 48.0
128.5242474 0.557830631732941 49.0
3.4595629724 0.863173723220825 50.0
86.374987653 0.268636584281921 51.0
79.660955683 0.201844170689583 52.0
79.39913303 0.0219701193273067 53.0
79.39913303 0.0651439279317856 54.0
79.39913303 0.0626271888613701 55.0
0 2.90803456306458 56.0
74.115375867 -0.036797221750021 57.0
216.82130568 1.05193138122559 58.0
241.57965962 0.318890154361725 59.0
22.102045834 0.0149630066007376 60.0
0 0.0901092290878296 61.0
155.64029907 0.673416256904602 0.0
155.64029907 1.63515818119049 1.0
154.44691338 1.70488429069519 2.0
149.65780807 3.87061786651611 3.0
0 4.38959074020386 4.0
0 4.82495450973511 5.0
-64.491665681 0.397174686193466 6.0
77.748000901 0.815243661403656 7.0
61.703957459 0.741107702255249 8.0
16.044043442 0.866950392723083 9.0
96.052557383 0.910583198070526 10.0
83.068808382 0.367013454437256 11.0
89.098216504 0.457894444465637 12.0
89.098216504 0.536719381809235 13.0
89.098216504 0.0701934248209 14.0
0 0.193077981472015 15.0
0 0.220001816749573 16.0
0 1.17107665538788 17.0
54.811764405 1.60198986530304 18.0
-54.811764405 1.29982101917267 19.0
-54.811764405 0.482776999473572 20.0
-54.811764405 0.122482195496559 21.0
75.108661136 1.42811489105225 22.0
159.25461958 0.735682666301727 23.0
123.27399113 0.714862108230591 24.0
24.723462581 0.188112631440163 25.0
234.36328071 1.3079799413681 26.0
234.36328071 1.525066614151 27.0
27.683755017 1.79114949703217 28.0
262.04703573 1.16870176792145 29.0
262.04703573 1.99399495124817 30.0
46.502804776 1.28993284702301 31.0
9.2584469465 0.327033638954163 32.0
20.296896731 1.51958739757538 33.0
20.296896731 0.906082034111023 34.0
20.296896731 0.149801909923553 35.0
9.9952996979 0.0248148273676634 36.0
8.8531289562 0.324476718902588 37.0
0 0.0562066771090031 38.0
0 0.1153873950243 39.0
-11.343929261 0.497729361057281 40.0
-24.941499441 0.721612155437469 41.0
4.9077928235 0.279121488332748 42.0
-32.167553649 0.400172412395477 43.0
77.234956721 0.451698452234268 44.0
77.234956721 0.0889969244599342 45.0
6.4079756541 0.429557204246521 46.0
2.4856220937 0.323604553937912 47.0
155.64029907 0.168250218033791 48.0
149.65780807 0.263381570577621 49.0
0 0.158758014440536 50.0
77.748000901 0.165750324726105 51.0
96.052557383 0.30225670337677 52.0
89.098216504 0.0871649831533432 53.0
89.098216504 0.0877489820122719 54.0
89.098216504 0.196611419320107 55.0
0 1.30495667457581 56.0
75.108661136 -0.056483119726181 57.0
234.36328071 0.387147277593613 58.0
262.04703573 0.404889196157455 59.0
20.296896731 -0.000501645728945732 60.0
0 0.100030973553658 61.0
161.04439202 0.988034546375275 0.0
161.04439202 2.36233949661255 1.0
160.01353871 2.12282824516296 2.0
154.54888971 5.8396053314209 3.0
2.4398206386 7.54658222198486 4.0
2.4398206386 2.56771421432495 5.0
-54.083959471 0.201968878507614 6.0
62.809855052 0.831412672996521 7.0
36.029384459 0.307999461889267 8.0
26.780470593 0.245640128850937 9.0
83.943056805 1.08183181285858 10.0
71.882753561 0.298790663480759 11.0
80.311418203 0.514451324939728 12.0
80.311418203 0.953871965408325 13.0
80.311418203 0.0501895286142826 14.0
0 0.197839170694351 15.0
0 0.194892674684525 16.0
0 0.995380103588104 17.0
53.077778244 1.39697468280792 18.0
-53.077778244 0.891478657722473 19.0
-53.077778244 0.643048465251923 20.0
-53.077778244 -0.037067599594593 21.0
74.446410859 1.62093842029572 22.0
140.51350363 0.672490179538727 23.0
105.10957567 1.17314171791077 24.0
19.695912606 0.0681998431682587 25.0
214.95991449 0.833732068538666 26.0
214.95991449 2.76384472846985 27.0
25.822078415 2.77081394195557 28.0
240.78199291 1.32255625724792 29.0
240.78199291 1.8223397731781 30.0
46.779612445 1.22165715694427 31.0
9.2226912265 0.424356251955032 32.0
21.368632616 1.3520085811615 33.0
21.368632616 0.520094275474548 34.0
21.368632616 0.116938106715679 35.0
8.644945779 0.034071072936058 36.0
7.4818047687 0.418964892625809 37.0
0 0.0225602276623249 38.0
0 0.10821982473135 39.0
-10.952449029 0.582640767097473 40.0
-24.187336077 0.673186659812927 41.0
5.0087953019 0.355800747871399 42.0
-31.793714892 0.385771602392197 43.0
89.834589115 0.647851645946503 44.0
89.834589115 0.0344311743974686 45.0
6.7245182975 0.339340776205063 46.0
2.2749568454 0.291189044713974 47.0
161.04439202 0.0537585020065308 48.0
154.54888971 0.254245102405548 49.0
2.4398206386 0.543400466442108 50.0
62.809855052 0.0238635241985321 51.0
83.943056805 0.354292750358582 52.0
80.311418203 0.0700320899486542 53.0
80.311418203 0.245026424527168 54.0
80.311418203 0.294356763362885 55.0
0 3.98369812965393 56.0
74.446410859 0.10025891661644 57.0
214.95991449 0.547536134719849 58.0
240.78199291 0.828135311603546 59.0
21.368632616 0.00279395654797554 60.0
0 0.0894554257392883 61.0
162.11753559 1.13555085659027 0.0
162.11753559 0.631008267402649 1.0
161.19671938 1.31187224388123 2.0
155.73204863 4.59549951553345 3.0
5.3040512165 3.18210315704346 4.0
5.3040512165 4.7393684387207 5.0
-53.472716004 0.139872461557388 6.0
63.882871662 0.880093574523926 7.0
44.574564012 0.685567021369934 8.0
19.30830765 1.0495845079422 9.0
84.975670369 1.16047775745392 10.0
72.108611264 0.10879247635603 11.0
70.272086404 0.447375506162643 12.0
70.272086404 0.38882103562355 13.0
70.272086404 0.111738190054893 14.0
0 0.0840597674250603 15.0
0 0.181689128279686 16.0
0 1.55154860019684 17.0
51.54680133 1.01353049278259 18.0
-51.54680133 0.625037789344788 19.0
-51.54680133 0.359519869089127 20.0
-51.54680133 -0.0834548696875572 21.0
75.403809113 1.49636578559875 22.0
147.78178998 1.57379293441772 23.0
106.92157847 0.239528447389603 24.0
23.782903821 0.0809517502784729 25.0
223.18559909 1.21710729598999 26.0
223.18559909 1.46806454658508 27.0
28.854299498 0.859334766864777 28.0
252.03989859 0.39740115404129 29.0
252.03989859 1.32434451580048 30.0
45.690569253 1.1918009519577 31.0
8.7648682368 0.317190408706665 32.0
23.857007783 2.05807185173035 33.0
23.857007783 0.70281195640564 34.0
23.857007783 0.109751991927624 35.0
10.679455916 0.0285045839846134 36.0
9.1659405297 0.358004510402679 37.0
0 0.0107675865292549 38.0
0 0.282590746879578 39.0
-10.19752497 0.574873208999634 40.0
-21.952137503 0.670076727867126 41.0
5.7337312342 0.334155261516571 42.0
-30.01866517 0.580505669116974 43.0
75.611578404 0.63892275094986 44.0
75.611578404 0.100482545793056 45.0
5.9674944216 0.592103600502014 46.0
2.3532733637 0.416064977645874 47.0
162.11753559 0.177741229534149 48.0
155.73204863 0.0803966447710991 49.0
5.3040512165 0.0493666939437389 50.0
63.882871662 0.255869776010513 51.0
84.975670369 0.180156409740448 52.0
70.272086404 0.0789062157273293 53.0
70.272086404 0.0595250390470028 54.0
70.272086404 0.0736214071512222 55.0
0 5.78151416778564 56.0
75.403809113 0.179050043225288 57.0
223.18559909 0.207561656832695 58.0
252.03989859 -0.0319612994790077 59.0
23.857007783 -0.00782868638634682 60.0
0 0.0873017236590385 61.0
147.11767849 1.05483973026276 0.0
147.11767849 1.61756885051727 1.0
145.72722618 2.28997611999512 2.0
140.26255557 6.23175430297852 3.0
2.7977596075 8.20210742950439 4.0
2.7977596075 5.65194225311279 5.0
-67.56788514 1.50976920127869 6.0
79.537977711 0.518675625324249 7.0
50.156462904 0.650708973407745 8.0
29.381514807 0.827458322048187 9.0
91.450618298 1.20339834690094 10.0
79.426590199 0.578447282314301 11.0
76.041045742 0.391420066356659 12.0
76.041045742 0.594188928604126 13.0
76.041045742 0.0923257544636726 14.0
0 0.100907012820244 15.0
0 0.206647455692291 16.0
0 0.829667091369629 17.0
54.673962058 1.1094366312027 18.0
-54.673962058 1.17072689533234 19.0
-54.673962058 0.608729481697083 20.0
-54.673962058 -0.0487555265426636 21.0
76.458185806 1.90973043441772 22.0
138.00453855 1.91862547397614 23.0
101.29717311 0.347906947135925 24.0
20.447798932 0.0976291000843048 25.0
214.46272435 1.38243436813354 26.0
214.46272435 2.07430219650269 27.0
28.372340124 1.40608930587769 28.0
242.83506448 0.561827600002289 29.0
242.83506448 1.63384258747101 30.0
46.979934084 1.18091654777527 31.0
9.458588214 0.455215126276016 32.0
21.784223748 1.58096730709076 33.0
21.784223748 0.849456489086151 34.0
21.784223748 0.107739865779877 35.0
9.9949645326 0.0424657166004181 36.0
8.8090418666 0.413187235593796 37.0
0 0.00402135774493217 38.0
0 0.195006683468819 39.0
-10.816608692 0.417705088853836 40.0
-23.556854438 0.718363285064697 41.0
4.6913625514 0.357510417699814 42.0
-30.713867222 0.482723116874695 43.0
92.11058591 0.571430683135986 44.0
92.11058591 0.100956477224827 45.0
6.8214419531 0.546777069568634 46.0
2.8076673658 0.323668748140335 47.0
147.11767849 0.117837585508823 48.0
140.26255557 0.425400495529175 49.0
2.7977596075 0.481996119022369 50.0
79.537977711 0.113236144185066 51.0
91.450618298 0.394838750362396 52.0
76.041045742 0.137771427631378 53.0
76.041045742 0.0885733887553215 54.0
76.041045742 0.136411726474762 55.0
0 5.66307497024536 56.0
76.458185806 -0.0103815924376249 57.0
214.46272435 0.830628991127014 58.0
242.83506448 0.368062555789948 59.0
21.784223748 -0.000106852501630783 60.0
0 0.211344510316849 61.0
132.90894321 0.915570557117462 0.0
132.90894321 1.53690183162689 1.0
131.93839258 1.38970386981964 2.0
126.47372186 3.48726987838745 3.0
5.5628014689 7.64757966995239 4.0
5.5628014689 3.36392712593079 5.0
-67.441433701 1.34716093540192 6.0
79.653366435 0.607314109802246 7.0
47.036989586 0.564800262451172 8.0
32.616376849 0.774090170860291 9.0
98.50691869 1.01055812835693 10.0
85.514754856 0.26048144698143 11.0
80.568821437 0.469226121902466 12.0
80.568821437 0.680523097515106 13.0
80.568821437 0.111755959689617 14.0
0 0.168614879250526 15.0
0 0.193433314561844 16.0
0 1.41519808769226 17.0
59.824038801 1.33594632148743 18.0
-59.824038801 1.00271916389465 19.0
-59.824038801 0.382737725973129 20.0
-59.824038801 0.143408745527267 21.0
79.155232431 2.20767617225647 22.0
137.94374135 1.82306551933289 23.0
103.05893529 0.500958263874054 24.0
20.322690255 0.181700140237808 25.0
217.09897378 1.35904598236084 26.0
217.09897378 1.32834327220917 27.0
23.819686307 1.39816129207611 28.0
240.91866009 1.07610702514648 29.0
240.91866009 1.84315884113312 30.0
47.52426374 1.08524072170258 31.0
9.503908681 0.42773824930191 32.0
19.331193631 1.31795108318329 33.0
19.331193631 0.849763154983521 34.0
19.331193631 0.128144323825836 35.0
9.1195592048 0.0546870417892933 36.0
7.8533182961 0.39951303601265 37.0
0 0.026302807033062 38.0
0 0.128473028540611 39.0
-11.920676823 1.01524233818054 40.0
-24.219156445 0.311748713254929 41.0
5.784079953 0.744141697883606 42.0
-32.758125691 0.693394243717194 43.0
87.71738398 0.660642743110657 44.0
87.71738398 0.10228181630373 45.0
7.7372456276 0.66244500875473 46.0
2.7614758576 0.376778602600098 47.0
132.90894321 0.0874953344464302 48.0
126.47372186 0.156122788786888 49.0
5.5628014689 0.444330155849457 50.0
79.653366435 0.0654537379741669 51.0
98.50691869 0.186149686574936 52.0
80.568821437 0.0789469256997108 53.0
80.568821437 0.124489143490791 54.0
80.568821437 0.0331261232495308 55.0
0 3.52354454994202 56.0
79.155232431 -0.0705814212560654 57.0
217.09897378 0.252325564622879 58.0
240.91866009 1.04510545730591 59.0
19.331193631 -0.00451529026031494 60.0
0 0.104298979043961 61.0
174.93618467 1.07723796367645 0.0
174.93618467 2.16377758979797 1.0
173.62312295 1.78207337856293 2.0
168.71643521 6.58048820495605 3.0
0 7.33277273178101 4.0
0 4.78385496139526 5.0
-74.728933705 0.631788611412048 6.0
87.960702485 0.53848397731781 7.0
55.956465957 1.03055942058563 8.0
32.004236527 1.12864530086517 9.0
84.903757353 1.25144875049591 10.0
70.867262478 0.384004473686218 11.0
84.560431885 0.101840458810329 12.0
84.560431885 0.125265523791313 13.0
84.560431885 0.0874347165226936 14.0
0 0.226966306567192 15.0
0 0.183405295014381 16.0
0 1.49595713615417 17.0
58.44860821 0.881361663341522 18.0
-58.44860821 0.796909153461456 19.0
-58.44860821 0.70191764831543 20.0
-58.44860821 0.020501796156168 21.0
80.222934362 1.60409963130951 22.0
141.03970576 1.08231794834137 23.0
104.6466142 0.552336454391479 24.0
22.007790235 0.117674201726913 25.0
221.26264012 1.00496411323547 26.0
221.26264012 1.21791005134583 27.0
22.882408701 1.97208178043365 28.0
244.14504882 0.88262814283371 29.0
244.14504882 1.69661402702332 30.0
47.048728006 1.0986795425415 31.0
9.2571292178 0.451199442148209 32.0
21.774326151 1.29459738731384 33.0
21.774326151 0.898826837539673 34.0
21.774326151 0.0900700464844704 35.0
9.2740612116 0.0211807899177074 36.0
8.0905121061 0.434244960546494 37.0
0 0.0207433328032494 38.0
0 0.252328276634216 39.0
-12.260118719 0.200704872608185 40.0
-24.897940697 0.790955901145935 41.0
5.3263314586 0.379150420427322 42.0
-33.124752809 0.588204622268677 43.0
85.74799061 0.507070660591125 44.0
85.74799061 0.0743931382894516 45.0
6.3579316802 0.366244703531265 46.0
2.2608878034 0.33020681142807 47.0
174.93618467 0.174645960330963 48.0
168.71643521 0.114894136786461 49.0
0 0.66858047246933 50.0
87.960702485 0.311147809028625 51.0
84.903757353 0.466759532690048 52.0
84.560431885 0.0456778295338154 53.0
84.560431885 0.0783393979072571 54.0
84.560431885 0.11311899125576 55.0
0 2.85373377799988 56.0
80.222934362 0.107466667890549 57.0
221.26264012 0.243107005953789 58.0
244.14504882 0.19790343940258 59.0
21.774326151 -0.016916785389185 60.0
0 0.101336151361465 61.0
168.56263983 0.588784635066986 0.0
168.56263983 1.01213371753693 1.0
167.33637877 1.01864612102509 2.0
166.23988129 2.56110787391663 3.0
0 3.56659603118896 4.0
0 2.806480884552 5.0
-66.527510749 1.24644386768341 6.0
76.688650011 0.571358978748322 7.0
47.463276192 1.4182687997818 8.0
29.225373819 1.37099528312683 9.0
81.171616968 0.662450611591339 10.0
66.796168676 0.187591776251793 11.0
65.187967252 0.173619762063026 12.0
65.187967252 0.110423713922501 13.0
65.187967252 0.088864728808403 14.0
20.064828518 0.60515421628952 15.0
20.064828518 0.188482329249382 16.0
0 1.39998173713684 17.0
64.686839538 0.76952338218689 18.0
-64.686839538 1.61761546134949 19.0
-64.686839538 0.376974284648895 20.0
-64.686839538 0.0672029703855515 21.0
86.489277443 0.698240458965302 22.0
159.53837008 0.588998794555664 23.0
122.33381494 0.224099218845367 24.0
24.542336485 0.642627835273743 25.0
246.02764752 1.01043450832367 26.0
246.02764752 2.19284391403198 27.0
27.614600868 1.7684907913208 28.0
273.64224839 1.14390969276428 29.0
273.64224839 1.8673300743103 30.0
45.781968688 0.947599768638611 31.0
9.1607415645 0.423036485910416 32.0
21.802437905 1.30509793758392 33.0
21.802437905 1.17416775226593 34.0
21.802437905 0.152916595339775 35.0
8.5614241337 0.0540459603071213 36.0
7.24080777 0.407462686300278 37.0
0 -0.0164764728397131 38.0
0 0.203415259718895 39.0
-12.474406538 0.404936552047729 40.0
-25.839097902 0.63136488199234 41.0
5.5408224136 0.613860726356506 42.0
-33.936888202 0.627997100353241 43.0
93.765591893 1.09001350402832 44.0
93.765591893 0.0678415969014168 45.0
6.0975369275 0.723109841346741 46.0
2.1735438288 0.324465990066528 47.0
168.56263983 0.0495617166161537 48.0
166.23988129 0.216699719429016 49.0
0 0.2387665361166 50.0
76.688650011 0.120753608644009 51.0
81.171616968 0.0414320230484009 52.0
65.187967252 0.121212519705296 53.0
65.187967252 0.102178499102592 54.0
65.187967252 -0.00564657524228096 55.0
0 0.378273218870163 56.0
86.489277443 0.20695735514164 57.0
246.02764752 0.639191448688507 58.0
273.64224839 0.740359604358673 59.0
21.802437905 -0.000950867310166359 60.0
0 0.109484247863293 61.0
163.51573788 0.801814198493958 0.0
163.51573788 0.694341480731964 1.0
162.47945649 1.2008216381073 2.0
157.01478579 5.0727424621582 3.0
7.2890363866 3.56863856315613 4.0
7.2890363866 3.96474599838257 5.0
-66.746731358 2.66566514968872 6.0
77.027930865 0.780144393444061 7.0
48.979506299 0.917400062084198 8.0
28.048424565 0.927219867706299 9.0
98.830936011 0.902816772460938 10.0
84.86511771 0.48050919175148 11.0
87.235344863 0.32135546207428 12.0
87.235344863 0.244317978620529 13.0
87.235344863 0.115375071763992 14.0
0 0.252470105886459 15.0
0 0.125786438584328 16.0
0 1.62818109989166 17.0
58.161060307 1.57069432735443 18.0
-58.161060307 1.48721778392792 19.0
-58.161060307 0.796314835548401 20.0
-58.161060307 -0.027810225263238 21.0
78.414752238 1.72818112373352 22.0
156.52520957 1.59509861469269 23.0
120.45277044 0.508890867233276 24.0
23.48284663 0.166296288371086 25.0
234.93996181 1.21476972103119 26.0
234.93996181 1.70785963535309 27.0
29.57353968 1.48534977436066 28.0
264.51350149 0.71890664100647 29.0
264.51350149 1.92221200466156 30.0
46.677977397 1.12207543849945 31.0
9.3873983645 0.461806684732437 32.0
20.25369193 1.58662211894989 33.0
20.25369193 0.948106646537781 34.0
20.25369193 0.141492441296577 35.0
9.3434881075 0.037940114736557 36.0
8.0727318395 0.439963579177856 37.0
0 0.0220717377960682 38.0
0 0.234613627195358 39.0
-12.391386156 0.755549609661102 40.0
-24.449686293 0.617081582546234 41.0
4.9250528352 0.55889880657196 42.0
-31.997428775 0.681659162044525 43.0
93.622763199 0.86497700214386 44.0
93.622763199 0.114799976348877 45.0
6.5506927778 0.662949979305267 46.0
2.3443817682 0.422737121582031 47.0
163.51573788 0.245081543922424 48.0
157.01478579 0.210528716444969 49.0
7.2890363866 0.192791283130646 50.0
77.027930865 0.239933878183365 51.0
98.830936011 0.314771503210068 52.0
87.235344863 0.136918514966965 53.0
87.235344863 0.0827213302254677 54.0
87.235344863 0.0762770026922226 55.0
0 1.13122069835663 56.0
78.414752238 0.0200381930917501 57.0
234.93996181 0.379814296960831 58.0
264.51350149 0.763751149177551 59.0
20.25369193 -0.012920867651701 60.0
0 0.109577938914299 61.0
145.03415847 0.940849483013153 0.0
145.03415847 1.25679075717926 1.0
143.83869909 1.26664304733276 2.0
138.37402848 3.12527561187744 3.0
9.6681949671 7.09576940536499 4.0
9.6681949671 3.62259125709534 5.0
-61.722719285 -0.0143427401781082 6.0
73.910428394 0.772878646850586 7.0
51.714134836 0.659103870391846 8.0
22.196293558 0.997996091842651 9.0
78.029575459 1.01147854328156 10.0
65.256811664 0.171732917428017 11.0
77.130198438 0.233957499265671 12.0
77.130198438 0.444097220897675 13.0
77.130198438 0.0384200066328049 14.0
0 0.011418916285038 15.0
0 0.155459687113762 16.0
0 0.919543087482452 17.0
62.552878487 1.07593107223511 18.0
-62.552878487 0.98271769285202 19.0
-62.552878487 0.757593631744385 20.0
-62.552878487 -0.0291397124528885 21.0
84.551602238 1.61207115650177 22.0
152.43243644 1.45653033256531 23.0
116.55672998 0.584113240242004 24.0
22.130935955 0.115970283746719 25.0
236.98403868 0.5792595744133 26.0
236.98403868 2.02388262748718 27.0
28.844226742 2.00765895843506 28.0
265.82826542 0.873366057872772 29.0
265.82826542 1.47536933422089 30.0
47.303329509 1.09946441650391 31.0
9.3886225119 0.237887099385262 32.0
21.998723752 1.49065732955933 33.0
21.998723752 1.00642609596252 34.0
21.998723752 0.0932600572705269 35.0
9.2347737771 0.0335547812283039 36.0
8.0678316188 0.393574863672256 37.0
0 0.0715538412332535 38.0
0 0.197469711303711 39.0
-12.92785829 0.577595353126526 40.0
-26.025072764 0.617536008358002 41.0
5.0603778408 0.567480981349945 42.0
-33.402525391 0.602366924285889 43.0
114.59975835 0.613393068313599 44.0
114.59975835 0.0926986634731293 45.0
6.5460929149 0.523687362670898 46.0
2.2134489496 0.329530268907547 47.0
145.03415847 0.171532586216927 48.0
138.37402848 0.0853204280138016 49.0
9.6681949671 0.210113525390625 50.0
73.910428394 0.152220919728279 51.0
78.029575459 0.319539070129395 52.0
77.130198438 0.0170555971562862 53.0
77.130198438 0.0809129327535629 54.0
77.130198438 0.0385885760188103 55.0
0 3.37794613838196 56.0
84.551602238 -0.0326487496495247 57.0
236.98403868 0.21870519220829 58.0
265.82826542 0.845498502254486 59.0
21.998723752 0.0108132921159267 60.0
0 0.0757432654500008 61.0
177.94666931 0.924443483352661 0.0
177.94666931 1.09173822402954 1.0
176.51971382 1.28747773170471 2.0
171.73064386 6.53763294219971 3.0
0 6.27849912643433 4.0
0 7.4478120803833 5.0
-76.742448822 9.81291484832764 6.0
85.547941727 0.748135983943939 7.0
60.020585255 0.696958065032959 8.0
25.527356472 0.793246388435364 9.0
80.524629492 0.83624404668808 10.0
67.252909735 0.0577749237418175 11.0
80.526551972 0.611635088920593 12.0
80.526551972 0.516233086585999 13.0
80.526551972 0.074840635061264 14.0
0 0.0237042177468538 15.0
0 0.226005926728249 16.0
0 1.27519500255585 17.0
63.163438313 1.79636776447296 18.0
-63.163438313 1.33065712451935 19.0
-63.163438313 1.1434154510498 20.0
-63.163438313 -1.39698386192322e-05 21.0
83.288874712 1.1362247467041 22.0
156.47528264 1.11474430561066 23.0
119.39818128 0.210408419370651 24.0
22.797250307 0.380225598812103 25.0
239.76415736 1.34287130832672 26.0
239.76415736 1.59065139293671 27.0
25.384552479 1.85599088668823 28.0
265.14870983 0.793776631355286 29.0
265.14870983 1.69388496875763 30.0
46.417641278 1.02526652812958 31.0
9.3053070044 0.588599801063538 32.0
20.1254364 1.0474374294281 33.0
20.1254364 1.05403900146484 34.0
20.1254364 0.0746691897511482 35.0
9.4527316428 0.0212286096066236 36.0
8.1907367851 0.35529613494873 37.0
0 0.138215407729149 38.0
0 0.231981784105301 39.0
-13.584384183 0.952701270580292 40.0
-26.626603605 0.515294194221497 41.0
4.7858426715 0.442982316017151 42.0
-33.864177867 0.247318133711815 43.0
94.205943314 0.761326014995575 44.0
94.205943314 0.0498948618769646 45.0
7.459260418 0.71260792016983 46.0
2.4452692666 0.429767072200775 47.0
177.94666931 0.425418078899384 48.0
171.73064386 0.202967464923859 49.0
0 1.10724031925201 50.0
85.547941727 0.36081999540329 51.0
80.524629492 0.282353699207306 52.0
80.526551972 0.00428229523822665 53.0
80.526551972 0.126334369182587 54.0
80.526551972 0.021164558827877 55.0
0 4.86361360549927 56.0
83.288874712 -0.019269160926342 57.0
239.76415736 0.409198522567749 58.0
265.14870983 0.171065151691437 59.0
20.1254364 0.0172009393572807 60.0
0 0.0879360288381577 61.0
208.64110972 0.902597367763519 0.0
208.64110972 0.439006626605988 1.0
207.11347132 0.866762280464172 2.0
202.3243662 3.68985962867737 3.0
0 3.12813758850098 4.0
0 4.92680072784424 5.0
-51.835431527 3.00720357894897 6.0
61.296756663 0.607963681221008 7.0
38.506293471 0.75465315580368 8.0
22.790463192 0.899966299533844 9.0
94.318748993 0.876637995243073 10.0
79.689706716 0.0724039450287819 11.0
87.287462306 0.126007184386253 12.0
87.287462306 0.0784745663404465 13.0
87.287462306 0.148799523711205 14.0
0 0.0695968791842461 15.0
0 0.171472117304802 16.0
0 1.72506392002106 17.0
58.315626048 1.11584675312042 18.0
-58.315626048 0.66927444934845 19.0
-58.315626048 0.637133657932281 20.0
-58.315626048 0.137293145060539 21.0
81.551092271 1.93212842941284 22.0
141.29034527 1.96190047264099 23.0
108.71893322 0.700615882873535 24.0
20.064406928 0.125948086380959 25.0
222.84143754 0.702260911464691 26.0
222.84143754 2.21304130554199 27.0
21.977233977 3.27431654930115 28.0
244.81867152 2.12635397911072 29.0
244.81867152 2.68040871620178 30.0
45.983551286 0.897756278514862 31.0
9.1710684355 0.338495701551437 32.0
23.235466223 1.76046085357666 33.0
23.235466223 1.48649179935455 34.0
23.235466223 0.121163979172707 35.0
10.087988451 0.0428194850683212 36.0
8.9040912451 0.444531142711639 37.0
0 0.0314652658998966 38.0
0 0.177324444055557 39.0
-12.718126474 0.713527202606201 40.0
-23.538576828 0.641699612140656 41.0
5.8820673498 0.379732012748718 42.0
-32.324957514 0.525401711463928 43.0
90.449743323 0.574058890342712 44.0
90.449743323 0.103810243308544 45.0
7.8524274878 0.519823968410492 46.0
2.7013902894 0.333832859992981 47.0
208.64110972 -0.00480928272008896 48.0
202.3243662 0.00287173804827034 49.0
0 0.262976765632629 50.0
61.296756663 0.281529158353806 51.0
94.318748993 0.00687197875231504 52.0
87.287462306 0.0107903443276882 53.0
87.287462306 0.0928725078701973 54.0
87.287462306 0.071902260184288 55.0
0 3.14212226867676 56.0
81.551092271 0.0310862753540277 57.0
222.84143754 0.114335976541042 58.0
244.81867152 1.65849959850311 59.0
23.235466223 -0.0191175118088722 60.0
0 0.107388041913509 61.0
159.5571771 0.898534834384918 0.0
159.5571771 1.97545516490936 1.0
158.17912131 1.97120547294617 2.0
152.71445077 5.00374412536621 3.0
0.63371035052 5.28942012786865 4.0
0.63371035052 4.27857971191406 5.0
-66.817832944 3.45455527305603 6.0
77.693544719 0.769375383853912 7.0
53.714085617 1.24235320091248 8.0
23.979459103 1.44723510742188 9.0
87.198176239 1.05841970443726 10.0
73.382658175 0.348755061626434 11.0
75.368711745 0.719528198242188 12.0
75.368711745 0.561562716960907 13.0
75.368711745 0.0696544721722603 14.0
0 0.0261573642492294 15.0
0 0.209177479147911 16.0
0 1.31748795509338 17.0
56.9409705 1.27058243751526 18.0
-56.9409705 0.869713187217712 19.0
-56.9409705 0.616771578788757 20.0
-56.9409705 0.00449353829026222 21.0
77.689071996 2.20691561698914 22.0
149.74479778 1.97327101230621 23.0
116.51591008 0.581745147705078 24.0
23.439747989 0.253973603248596 25.0
227.43386977 1.58573865890503 26.0
227.43386977 1.82319724559784 27.0
27.729632201 1.55680060386658 28.0
255.16350198 1.00094413757324 29.0
255.16350198 1.58741569519043 30.0
46.23934191 1.03475022315979 31.0
9.4204013289 0.458431303501129 32.0
20.748101496 1.49178826808929 33.0
20.748101496 1.14441776275635 34.0
20.748101496 0.0947977676987648 35.0
9.1743730672 0.0197076667100191 36.0
7.9387332532 0.447081625461578 37.0
0 0.00261268019676208 38.0
0 0.218181312084198 39.0
-11.120678959 0.552415132522583 40.0
-23.97451958 0.7194544672966 41.0
5.0776705182 0.420027494430542 42.0
-31.733196285 0.561047673225403 43.0
101.89798675 0.68963897228241 44.0
101.89798675 0.0797554776072502 45.0
7.186367618 0.572837054729462 46.0
2.3514943922 0.378225713968277 47.0
159.5571771 0.116640269756317 48.0
152.71445077 0.19564563035965 49.0
0.63371035052 0.185382276773453 50.0
77.693544719 0.352334946393967 51.0
87.198176239 0.433083683252335 52.0
75.368711745 0.0397118479013443 53.0
75.368711745 0.10932432115078 54.0
75.368711745 0.35318461060524 55.0
0 5.41661357879639 56.0
77.689071996 0.11205130815506 57.0
227.43386977 0.637334406375885 58.0
255.16350198 0.666736841201782 59.0
20.748101496 -0.0094806794077158 60.0
0 0.113498032093048 61.0
144.45000355 0.897466242313385 0.0
144.45000355 1.29483807086945 1.0
143.39622228 1.56035578250885 2.0
137.93155161 3.60771465301514 3.0
6.0399106098 5.94262599945068 4.0
6.0399106098 5.88351249694824 5.0
-64.154554744 6.8073673248291 6.0
74.184710702 0.804256975650787 7.0
37.596340436 0.491583377122879 8.0
36.588370266 0.33303302526474 9.0
94.960255152 1.04451477527618 10.0
82.985874469 0.280607014894485 11.0
85.397525822 0.849511861801147 12.0
85.397525822 0.194038212299347 13.0
85.397525822 0.0431154072284698 14.0
0 0.0817589536309242 15.0
0 0.243801385164261 16.0
0 0.983881235122681 17.0
57.301496962 0.947108447551727 18.0
-57.301496962 0.750577449798584 19.0
-57.301496962 0.607902526855469 20.0
-57.301496962 -0.0601217746734619 21.0
79.008977088 1.27952456474304 22.0
142.66537129 0.89721143245697 23.0
109.73479195 0.789824247360229 24.0
23.158256127 0.126485273241997 25.0
221.67434838 1.2192497253418 26.0
221.67434838 2.4068660736084 27.0
23.861385157 2.00415253639221 28.0
245.53573353 0.9861661195755 29.0
245.53573353 1.51165878772736 30.0
46.748533255 1.06704163551331 31.0
9.2307729019 0.307708591222763 32.0
21.707480126 1.49039041996002 33.0
21.707480126 1.11894392967224 34.0
21.707480126 0.11838286370039 35.0
10.355078011 0.0547237209975719 36.0
8.9667878792 0.394062370061874 37.0
0 0.00905479118227959 38.0
0 0.148811385035515 39.0
-12.696598083 0.725773751735687 40.0
-26.207961588 0.457795113325119 41.0
5.4119999308 0.376420587301254 42.0
-34.101493918 0.425393640995026 43.0
97.611569749 0.642637014389038 44.0
97.611569749 0.0872357934713364 45.0
7.2643182189 0.564092636108398 46.0
2.4598043719 0.438913315534592 47.0
144.45000355 0.0680903121829033 48.0
137.93155161 0.148047983646393 49.0
6.0399106098 0.688950777053833 50.0
74.184710702 0.214117437601089 51.0
94.960255152 0.492013067007065 52.0
85.397525822 0.0906243100762367 53.0
85.397525822 0.0903391018509865 54.0
85.397525822 0.0530925542116165 55.0
0 2.77603125572205 56.0
79.008977088 0.266700953245163 57.0
221.67434838 0.957485496997833 58.0
245.53573353 0.577506244182587 59.0
21.707480126 0.0173552483320236 60.0
0 0.10169743001461 61.0
188.42846089 0.880747735500336 0.0
188.42846089 1.27207279205322 1.0
186.9578621 1.40579640865326 2.0
182.16875698 3.26967644691467 3.0
0 4.54068183898926 4.0
0 4.60757303237915 5.0
-52.968424475 1.15791749954224 6.0
65.022100669 0.883547186851501 7.0
47.524627314 0.659705221652985 8.0
17.497473356 0.566202521324158 9.0
91.158382345 0.894168853759766 10.0
78.223009574 0.137123748660088 11.0
80.337619071 0.872877836227417 12.0
80.337619071 0.635036766529083 13.0
80.337619071 0.0646234303712845 14.0
0 0.180047243833542 15.0
0 0.203039020299911 16.0
0 1.11581778526306 17.0
53.751269733 1.64515018463135 18.0
-53.751269733 1.42870497703552 19.0
-53.751269733 0.256935626268387 20.0
-53.751269733 0.0379157066345215 21.0
75.587744178 2.03138637542725 22.0
152.76054086 1.12386155128479 23.0
116.2437233 0.931845247745514 24.0
22.510961781 0.118762657046318 25.0
228.34828504 1.11735832691193 26.0
228.34828504 1.54024302959442 27.0
24.023066936 1.09468829631805 28.0
252.37135198 0.669002592563629 29.0
252.37135198 1.56971442699432 30.0
46.796559708 0.992251396179199 31.0
9.2814983367 0.481607794761658 32.0
21.836474445 1.24177241325378 33.0
21.836474445 1.22357273101807 34.0
21.836474445 0.102688908576965 35.0
9.5275771309 0.0191670097410679 36.0
8.1574305425 0.455590605735779 37.0
0 0.0071660540997982 38.0
0 0.248259276151657 39.0
-10.5153985 0.548901975154877 40.0
-24.046207028 0.455804288387299 41.0
4.8529915403 0.456665933132172 42.0
-31.576198784 0.466987758874893 43.0
98.51475512 0.644085586071014 44.0
98.51475512 0.0721120685338974 45.0
7.6576740542 0.603720784187317 46.0
2.887328792 0.395032107830048 47.0
188.42846089 0.0811167657375336 48.0
182.16875698 0.125070452690125 49.0
0 0.159808710217476 50.0
65.022100669 0.290216863155365 51.0
91.158382345 0.0894153714179993 52.0
80.337619071 0.129625111818314 53.0
80.337619071 0.172791942954063 54.0
80.337619071 0.0832620784640312 55.0
0 1.86365509033203 56.0
75.587744178 0.0780620947480202 57.0
228.34828504 0.552696645259857 58.0
252.37135198 0.810597479343414 59.0
21.836474445 0.0162375532090664 60.0
0 0.0914809629321098 61.0
137.80206464 1.07310402393341 0.0
137.80206464 0.44777125120163 1.0
136.54621278 0.717711627483368 2.0
131.75710752 3.05544710159302 3.0
0 5.22751617431641 4.0
0 4.064293384552 5.0
-68.209592268 0.388223588466644 6.0
80.802653958 0.703163981437683 7.0
63.994325747 1.12789142131805 8.0
16.808328211 1.00021469593048 9.0
87.612652529 1.09271883964539 10.0
74.223598688 0.00776984496042132 11.0
83.381950447 0.666312456130981 12.0
83.381950447 0.815945625305176 13.0
83.381950447 0.104440905153751 14.0
0 0.268453389406204 15.0
0 0.115604355931282 16.0
0 1.36226344108582 17.0
64.436971254 1.18772065639496 18.0
-64.436971254 1.45044112205505 19.0
-64.436971254 0.191150739789009 20.0
-64.436971254 0.0187168605625629 21.0
86.588266729 2.42871999740601 22.0
157.75224384 2.02873253822327 23.0
122.67143731 0.356945753097534 24.0
24.833446655 0.268766343593597 25.0
244.34051057 1.32290983200073 26.0
244.34051057 1.71103203296661 27.0
25.671517382 1.75563597679138 28.0
270.01202795 0.296794414520264 29.0
270.01202795 1.77398931980133 30.0
46.898965769 1.14272379875183 31.0
9.4742710951 0.54225093126297 32.0
22.151295475 1.22525060176849 33.0
22.151295475 1.00662803649902 34.0
22.151295475 0.132533878087997 35.0
9.3519716343 0.0306075606495142 36.0
8.0930530638 0.414606273174286 37.0
0 -0.00405526533722878 38.0
0 0.100624814629555 39.0
-13.078174934 0.712246656417847 40.0
-25.850202009 0.467144876718521 41.0
4.7589930344 0.739028632640839 42.0
-33.070366018 0.761973679065704 43.0
92.037547934 0.797356069087982 44.0
92.037547934 0.0755505189299583 45.0
6.5750048303 0.492580115795135 46.0
2.6872686624 0.346090137958527 47.0
137.80206464 0.150636941194534 48.0
131.75710752 0.127959281206131 49.0
0 0.563689768314362 50.0
80.802653958 0.238200202584267 51.0
87.612652529 0.151021271944046 52.0
83.381950447 0.00309770577587187 53.0
83.381950447 0.181473702192307 54.0
83.381950447 0.0447721742093563 55.0
0 1.60863506793976 56.0
86.588266729 0.0383557453751564 57.0
244.34051057 0.291816860437393 58.0
270.01202795 0.308333873748779 59.0
22.151295475 -0.00823538191616535 60.0
0 0.103038489818573 61.0
125.91858506 0.732330799102783 0.0
125.91858506 1.52285444736481 1.0
124.66263693 1.50905990600586 2.0
119.87356697 4.83890581130981 3.0
0 6.3691611289978 4.0
0 4.6920313835144 5.0
-51.623954806 2.04627084732056 6.0
60.256590068 1.29247796535492 7.0
46.292238404 0.651414453983307 8.0
13.964351664 0.513969898223877 9.0
95.890232927 0.954078733921051 10.0
84.226813844 0.646995842456818 11.0
88.668351635 0.819122135639191 12.0
88.668351635 0.108220525085926 13.0
88.668351635 0.0907400995492935 14.0
0 0.22633720934391 15.0
0 0.19771634042263 16.0
0 1.28110074996948 17.0
52.142339353 1.50942778587341 18.0
-52.142339353 1.08889532089233 19.0
-52.142339353 0.543705701828003 20.0
-52.142339353 -0.00253214314579964 21.0
72.649988248 3.13624715805054 22.0
135.85142511 2.08938217163086 23.0
96.512872902 0.578692972660065 24.0
21.1204338 0.100853711366653 25.0
208.50141336 2.21206736564636 26.0
208.50141336 2.80271887779236 27.0
30.06801763 2.67509508132935 28.0
238.56943099 1.65217065811157 29.0
238.56943099 2.18119072914124 30.0
46.26393064 1.067946434021 31.0
9.3643136538 0.507821381092072 32.0
20.507648895 1.24594557285309 33.0
20.507648895 0.987766265869141 34.0
20.507648895 0.135841727256775 35.0
8.0649762128 0.0537487342953682 36.0
6.8977533455 0.367531925439835 37.0
0 0.084314376115799 38.0
0 0.163808375597 39.0
-11.982853504 0.884680867195129 40.0
-23.849847447 0.481321334838867 41.0
4.9438065815 0.366907060146332 42.0
-30.915257451 0.607205867767334 43.0
88.194088595 0.636641383171082 44.0
88.194088595 0.110421627759933 45.0
7.4217016841 0.466776072978973 46.0
3.0232235849 0.345893740653992 47.0
125.91858506 0.0384608507156372 48.0
119.87356697 0.281417071819305 49.0
0 0.789027631282806 50.0
60.256590068 0.155704528093338 51.0
95.890232927 0.597239315509796 52.0
88.668351635 0.0961932018399239 53.0
88.668351635 0.0394111387431622 54.0
88.668351635 0.0390575602650642 55.0
0 1.36116707324982 56.0
72.649988248 0.123967736959457 57.0
208.50141336 1.06448030471802 58.0
238.56943099 1.30149376392365 59.0
20.507648895 0.0147259347140789 60.0
0 0.0537325441837311 61.0
184.8065798 0.854268491268158 0.0
184.8065798 2.06556081771851 1.0
183.86141257 1.73775458335876 2.0
180.3653041 4.46992063522339 3.0
0 4.69529914855957 4.0
0 2.54654121398926 5.0
-56.011309846 0.241318166255951 6.0
64.647700233 0.337540566921234 7.0
39.705899657 0.676297128200531 8.0
24.941800576 0.906639933586121 9.0
100.56353804 0.993620157241821 10.0
86.846452987 0.290753871202469 11.0
87.789820022 0.11492058634758 12.0
87.789820022 0.370192050933838 13.0
87.789820022 0.0567408874630928 14.0
1.2126831654 0.171874105930328 15.0
1.2126831654 0.153273239731789 16.0
0 1.23566651344299 17.0
59.606637889 0.872062087059021 18.0
-59.606637889 1.18928587436676 19.0
-59.606637889 0.237824738025665 20.0
-59.606637889 0.023006834089756 21.0
80.591070231 2.0399112701416 22.0
131.85304098 1.33866596221924 23.0
100.69406232 0.892574846744537 24.0
19.376838452 0.071937695145607 25.0
212.44411121 1.22832477092743 26.0
212.44411121 1.45469605922699 27.0
27.603060568 1.14944076538086 28.0
240.04717178 0.420328378677368 29.0
240.04717178 1.39185988903046 30.0
47.219821074 0.948290526866913 31.0
9.4762518122 0.472346365451813 32.0
20.984432342 1.26309728622437 33.0
20.984432342 1.14531695842743 34.0
20.984432342 0.114028871059418 35.0
9.5026335649 0.0219049751758575 36.0
8.0490884341 0.337977886199951 37.0
0 0.00607592985033989 38.0
0 0.110339008271694 39.0
-12.85803725 0.817846477031708 40.0
-24.194782702 0.38820481300354 41.0
4.7386085085 0.504680275917053 42.0
-31.623099148 0.727871358394623 43.0
99.626990211 0.681792974472046 44.0
99.626990211 0.0691131502389908 45.0
6.8497174605 0.607743144035339 46.0
2.525972859 0.404682815074921 47.0
184.8065798 0.129943072795868 48.0
180.3653041 0.0595699474215508 49.0
0 0.0996639579534531 50.0
64.647700233 0.154987499117851 51.0
100.56353804 0.349480479955673 52.0
87.789820022 0.0212701875716448 53.0
87.789820022 0.0243695564568043 54.0
87.789820022 0.0442249663174152 55.0
0 0.28902867436409 56.0
80.591070231 0.191484987735748 57.0
212.44411121 0.621080935001373 58.0
240.04717178 0.31989648938179 59.0
20.984432342 0.0166489649564028 60.0
0 0.0851707980036736 61.0
162.73061508 0.846804082393646 0.0
162.73061508 0.783928453922272 1.0
161.29340693 1.21362125873566 2.0
156.50430167 5.74978733062744 3.0
0 4.05180692672729 4.0
0 4.84444236755371 5.0
-71.188202712 3.93226408958435 6.0
81.439048908 0.637967228889465 7.0
58.735345611 0.813396036624908 8.0
22.703703297 0.995651543140411 9.0
98.789788177 1.03286683559418 10.0
87.292588063 0.346845120191574 11.0
89.806066678 0.394270092248917 12.0
89.806066678 0.119486451148987 13.0
89.806066678 0.130174294114113 14.0
0 0.248944476246834 15.0
0 0.134695634245872 16.0
0 1.52868449687958 17.0
54.271709916 1.49198579788208 18.0
-54.271709916 1.81073224544525 19.0
-54.271709916 0.387915551662445 20.0
-54.271709916 -0.0286140292882919 21.0
74.72612697 1.63927793502808 22.0
143.29976482 1.50115740299225 23.0
107.44366475 0.511394858360291 24.0
24.577692684 0.0970668122172356 25.0
218.02589179 0.905242621898651 26.0
218.02589179 1.47517216205597 27.0
25.185211457 1.59555780887604 28.0
243.21110325 0.916685223579407 29.0
243.21110325 1.84264659881592 30.0
46.637498905 1.06513750553131 31.0
9.2586185153 0.477791488170624 32.0
20.454417053 1.52638959884644 33.0
20.454417053 0.893426358699799 34.0
20.454417053 0.115389317274094 35.0
8.9781579683 0.0398354940116405 36.0
7.7209158022 0.401719272136688 37.0
0 0.0153028406202793 38.0
0 0.151883453130722 39.0
-12.21070188 0.786044299602509 40.0
-24.615659716 0.686744093894958 41.0
5.2460679617 0.458003550767899 42.0
-32.235088225 0.541744470596313 43.0
91.816104099 0.64973658323288 44.0
91.816104099 0.15043032169342 45.0
7.3018885515 0.646695137023926 46.0
2.709907605 0.456014424562454 47.0
162.73061508 0.0933566242456436 48.0
156.50430167 0.142038822174072 49.0
0 0.167536467313766 50.0
81.439048908 0.107720479369164 51.0
98.789788177 0.429551064968109 52.0
89.806066678 0.0644099786877632 53.0
89.806066678 0.04335081204772 54.0
89.806066678 0.0544120967388153 55.0
0 0.931071996688843 56.0
74.72612697 0.112246342003345 57.0
218.02589179 0.455036818981171 58.0
243.21110325 0.271025121212006 59.0
20.454417053 0.0109949484467506 60.0
0 0.0982657819986343 61.0
168.33682567 0.83817732334137 0.0
168.33682567 1.10779821872711 1.0
167.03048749 0.870532751083374 2.0
161.56585208 3.93889617919922 3.0
5.7519858086 8.93206787109375 4.0
5.7519858086 3.33021593093872 5.0
-72.342243228 0.0581407248973846 6.0
82.742658882 0.841159999370575 7.0
45.763951978 0.465337514877319 8.0
36.978706904 0.599486529827118 9.0
99.192275941 0.867119610309601 10.0
85.554193834 0.401266753673553 11.0
80.468431339 0.766378164291382 12.0
80.468431339 0.593706071376801 13.0
80.468431339 -0.000299011706374586 14.0
0 0.108683235943317 15.0
0 0.213931277394295 16.0
0 1.17678081989288 17.0
64.733766161 1.26415061950684 18.0
-64.733766161 1.56727886199951 19.0
-64.733766161 0.805091440677643 20.0
-64.733766161 -0.0381038747727871 21.0
84.885416521 1.49606084823608 22.0
141.04947855 0.78472226858139 23.0
111.2487277 1.0078192949295 24.0
21.348237137 0.0812505781650543 25.0
225.93489507 0.589562118053436 26.0
225.93489507 1.01754558086395 27.0
29.735188592 1.47011923789978 28.0
255.67008366 0.770539999008179 29.0
255.67008366 1.91927850246429 30.0
46.977823457 1.06818377971649 31.0
9.2989669636 0.471503347158432 32.0
20.151650361 1.14913654327393 33.0
20.151650361 1.04781937599182 34.0
20.151650361 0.103557243943214 35.0
9.6941164983 0.0467002913355827 36.0
8.416223794 0.3335280418396 37.0
0 0.0320758894085884 38.0
0 0.166989803314209 39.0
-11.662382341 0.772317826747894 40.0
-24.028833165 0.731403529644012 41.0
4.5763331392 0.573225796222687 42.0
-30.90436432 0.699995398521423 43.0
87.441739979 0.644348680973053 44.0
87.441739979 0.0713384449481964 45.0
6.1123298024 0.666091620922089 46.0
1.9688078247 0.439180195331573 47.0
168.33682567 0.138680785894394 48.0
161.56585208 0.087014228105545 49.0
5.7519858086 0.561420798301697 50.0
82.742658882 0.104671776294708 51.0
99.192275941 0.381963580846786 52.0
80.468431339 0.035048570483923 53.0
80.468431339 0.0287452228367329 54.0
80.468431339 0.101668372750282 55.0
0 1.3061774969101 56.0
84.885416521 0.0354802533984184 57.0
225.93489507 0.219086855649948 58.0
255.67008366 0.844297170639038 59.0
20.151650361 0.0380083657801151 60.0
0 0.0887784361839294 61.0
193.03024718 0.931154847145081 0.0
193.03024718 1.0360666513443 1.0
191.58171292 1.46323931217194 2.0
186.11704238 5.22311067581177 3.0
15.408508378 7.81684064865112 4.0
15.408508378 5.84836006164551 5.0
-59.492135461 1.58307933807373 6.0
72.004437705 0.553118765354156 7.0
39.65053868 1.00097239017487 8.0
32.353899024 1.07730185985565 9.0
83.737330921 0.984310805797577 10.0
70.270230304 0.0746861100196838 11.0
74.498736211 0.885760307312012 12.0
74.498736211 0.602230489253998 13.0
74.498736211 0.00671367160975933 14.0
0 0.0353216044604778 15.0
0 0.199733793735504 16.0
0 0.994681715965271 17.0
58.765623478 1.1772096157074 18.0
-58.765623478 1.20160925388336 19.0
-58.765623478 0.818147003650665 20.0
-58.765623478 -0.046873651444912 21.0
80.757072959 1.67115724086761 22.0
141.32112601 1.50660133361816 23.0
107.47347904 0.421558856964111 24.0
21.937323754 0.164516523480415 25.0
222.07819897 1.38131535053253 26.0
222.07819897 2.12574362754822 27.0
24.199086013 1.87528467178345 28.0
246.27728498 0.874271035194397 29.0
246.27728498 1.75953602790833 30.0
46.346519017 1.13254702091217 31.0
9.2501742477 0.412299454212189 32.0
21.991449481 1.38017046451569 33.0
21.991449481 0.580050408840179 34.0
21.991449481 0.10486639291048 35.0
9.5465542408 0.0329035297036171 36.0
8.2037323979 0.391633540391922 37.0
0 0.0808892473578453 38.0
0 0.107499860227108 39.0
-11.74772184 0.541199684143066 40.0
-25.360016158 0.526288628578186 41.0
4.6687778212 0.38781327009201 42.0
-32.594502238 0.370714336633682 43.0
101.97283936 0.769761860370636 44.0
101.97283936 0.106350250542164 45.0
7.0035618944 0.654468297958374 46.0
2.4559445461 0.454097986221313 47.0
193.03024718 0.14845572412014 48.0
186.11704238 0.0365565977990627 49.0
15.408508378 0.50298535823822 50.0
72.004437705 0.296583741903305 51.0
83.737330921 0.270840376615524 52.0
74.498736211 0.0345584340393543 53.0
74.498736211 0.129914253950119 54.0
74.498736211 0.225097417831421 55.0
0 3.39984440803528 56.0
80.757072959 0.030817948281765 57.0
222.07819897 0.945175886154175 58.0
246.27728498 0.598101019859314 59.0
21.991449481 0.0121785532683134 60.0
0 0.0827962160110474 61.0
179.32488726 0.98015832901001 0.0
179.32488726 0.486563891172409 1.0
177.91523185 1.01234078407288 2.0
173.12616139 6.41454362869263 3.0
0 6.7375168800354 4.0
0 3.8483943939209 5.0
-72.304238807 2.03743076324463 6.0
83.343683696 0.756872892379761 7.0
46.70621387 0.66949462890625 8.0
36.637469827 0.760560274124146 9.0
92.549303346 1.11051440238953 10.0
77.971950444 0.571298241615295 11.0
78.100807395 0.136154070496559 12.0
78.100807395 0.1445132791996 13.0
78.100807395 0.11582862585783 14.0
0 0.0686241239309311 15.0
0 0.190094828605652 16.0
0 1.18004667758942 17.0
56.854336539 1.3837274312973 18.0
-56.854336539 1.71756899356842 19.0
-56.854336539 0.227357089519501 20.0
-56.854336539 0.417269438505173 21.0
78.681895318 1.39677500724792 22.0
145.27114677 0.717303454875946 23.0
110.18777954 0.778502285480499 24.0
20.716780328 0.14187128841877 25.0
223.95304209 1.47591042518616 26.0
223.95304209 1.98050165176392 27.0
25.387621271 2.07519245147705 28.0
249.34066336 1.17161083221436 29.0
249.34066336 1.74082696437836 30.0
46.229992992 1.031609416008 31.0
9.3854207474 0.366981625556946 32.0
21.827558779 1.17825889587402 33.0
21.827558779 0.849940180778503 34.0
21.827558779 0.0557102896273136 35.0
8.233935592 0.021085225045681 36.0
6.9871559504 0.460515707731247 37.0
0 0.0493331775069237 38.0
0 0.192225992679596 39.0
-11.044064765 0.908586323261261 40.0
-23.426640044 0.578926384449005 41.0
5.4459390751 0.490130841732025 42.0
-31.318377059 0.709366083145142 43.0
90.453224647 0.578425168991089 44.0
90.453224647 0.0891529694199562 45.0
5.9119627188 0.511337280273438 46.0
2.4331776781 0.414702355861664 47.0
179.32488726 0.161075621843338 48.0
173.12616139 0.245225951075554 49.0
0 0.599749445915222 50.0
83.343683696 0.103668175637722 51.0
92.549303346 0.420299112796783 52.0
78.100807395 0.0967845097184181 53.0
78.100807395 0.0668712183833122 54.0
78.100807395 0.0691538974642754 55.0
0 6.09668445587158 56.0
78.681895318 0.0337763503193855 57.0
223.95304209 0.412196278572083 58.0
249.34066336 0.338840365409851 59.0
21.827558779 0.0193953067064285 60.0
0 0.0612215921282768 61.0
184.96463649 0.974933385848999 0.0
184.96463649 1.68519067764282 1.0
183.59555229 1.393186211586 2.0
182.53957185 2.78693771362305 3.0
0 4.43937301635742 4.0
0 4.81247138977051 5.0
-58.149960739 0.467548072338104 6.0
66.648013549 0.923765778541565 7.0
55.483629028 0.9261274933815 8.0
11.16438452 0.708000540733337 9.0
79.31579168 1.03123950958252 10.0
65.733787857 0.101409241557121 11.0
76.973368414 0.704166412353516 12.0
76.973368414 0.990120887756348 13.0
76.973368414 0.118822157382965 14.0
0 0.0593079179525375 15.0
0 0.185737058520317 16.0
0 1.46577703952789 17.0
55.638019791 1.28732597827911 18.0
-55.638019791 1.53978717327118 19.0
-56.306287533 0.516419649124146 20.0
-56.374849132 -0.0160591006278992 21.0
78.664150711 1.95369791984558 22.0
141.45810027 2.30264139175415 23.0
109.58615186 0.366826295852661 24.0
21.555988132 0.196739301085472 25.0
220.12225098 1.46678674221039 26.0
220.12225098 1.51157486438751 27.0
21.62188815 1.65068757534027 28.0
241.74413913 0.572187781333923 29.0
241.74413913 1.53187274932861 30.0
46.672923783 1.26812243461609 31.0
9.3507452152 0.481862604618073 32.0
22.289301579 1.44511353969574 33.0
22.289301579 1.04760181903839 34.0
22.289301579 0.147107422351837 35.0
10.305747878 0.0187478419393301 36.0
9.1286591061 0.412564873695374 37.0
0 0.0180058255791664 38.0
0 0.0792604237794876 39.0
-12.080297042 0.297878056764603 40.0
-24.000865267 0.727915823459625 41.0
4.553069008 0.336499333381653 42.0
-30.700597562 0.397047728300095 43.0
80.079076575 0.710649192333221 44.0
78.842343562 0.106096342206001 45.0
7.6317197814 0.662703037261963 46.0
2.6937448457 0.352742552757263 47.0
184.96463649 0.215149655938148 48.0
182.53957185 0.0249063801020384 49.0
0 0.0983642488718033 50.0
66.648013549 0.134269267320633 51.0
79.31579168 0.121594950556755 52.0
76.973368414 0.030397979542613 53.0
76.973368414 0.361097753047943 54.0
76.973368414 0.0364114344120026 55.0
0 5.53494739532471 56.0
78.664150711 0.126896321773529 57.0
220.12225098 0.550679206848145 58.0
241.74413913 -0.0781551599502563 59.0
22.289301579 0.0240440107882023 60.0
0 0.0623421110212803 61.0
139.12411079 0.873637676239014 0.0
139.12411079 1.08151817321777 1.0
137.9751448 1.6079238653183 2.0
133.18603954 4.96750354766846 3.0
0 7.03188753128052 4.0
0 5.42750215530396 5.0
-55.864183738 0.429368257522583 6.0
66.44200979 0.479171544313431 7.0
54.283629715 0.682031393051147 8.0
12.158380075 0.834844768047333 9.0
98.549866708 0.97808974981308 10.0
84.876550677 0.282780736684799 11.0
83.887014491 0.230683922767639 12.0
83.887014491 0.484139949083328 13.0
83.887014491 0.0928157716989517 14.0
0 0.162042811512947 15.0
0 0.180479377508163 16.0
0 1.63014197349548 17.0
54.297911345 1.28109812736511 18.0
-54.297911345 1.07898163795471 19.0
-54.297911345 0.881658017635345 20.0
-54.297911345 -0.0580008924007416 21.0
76.497200074 1.44741892814636 22.0
142.88818217 1.65681934356689 23.0
105.04063148 0.428965866565704 24.0
23.904977721 0.100345239043236 25.0
219.38538224 0.523965537548065 26.0
219.38538224 1.11193227767944 27.0
27.211221487 1.17386877536774 28.0
246.59660373 0.660868644714355 29.0
246.59660373 1.5478982925415 30.0
46.971118899 0.979048848152161 31.0
9.4225513607 0.516572713851929 32.0
22.199288729 1.73567974567413 33.0
22.199288729 0.904778242111206 34.0
22.199288729 0.129310682415962 35.0
10.020216408 0.0471296980977058 36.0
8.7893739691 0.398561924695969 37.0
0 0.0292552970349789 38.0
0 0.243775546550751 39.0
-11.579773552 0.632790327072144 40.0
-24.666386011 0.693711459636688 41.0
5.4575232242 0.394417464733124 42.0
-32.564791012 0.429869174957275 43.0
84.341997141 0.577127516269684 44.0
84.341997141 0.111268475651741 45.0
6.5981665502 0.395535081624985 46.0
2.5529860287 0.337444663047791 47.0
139.12411079 0.168014526367188 48.0
133.18603954 0.621005415916443 49.0
0 0.402758061885834 50.0
66.44200979 0.0988813266158104 51.0
98.549866708 0.178722873330116 52.0
83.887014491 0.0955438986420631 53.0
83.887014491 0.0693946108222008 54.0
83.887014491 0.084172211587429 55.0
0 0.529645502567291 56.0
76.497200074 0.0398339331150055 57.0
219.38538224 0.220848828554153 58.0
246.59660373 0.808968961238861 59.0
22.199288729 -0.0062604621052742 60.0
0 0.113546371459961 61.0
170.46516087 0.815700054168701 0.0
170.46516087 0.96945732831955 1.0
169.3726973 0.948817014694214 2.0
164.58359213 4.86259269714355 3.0
0 10.2219142913818 4.0
0 3.88473606109619 5.0
-71.213376508 0.552260398864746 6.0
82.129179099 0.41287288069725 7.0
39.961195196 0.657948076725006 8.0
42.167983904 0.58100688457489 9.0
67.619509622 0.931231737136841 10.0
55.032815118 0.0654620230197906 11.0
72.199760421 1.06742572784424 12.0
72.199760421 0.276553839445114 13.0
72.199760421 0.20446902513504 14.0
0 0.109320476651192 15.0
0 0.157788574695587 16.0
0 1.42616355419159 17.0
55.78468877 0.971682250499725 18.0
-55.78468877 0.4145447909832 19.0
-55.78468877 0.984914243221283 20.0
-55.78468877 0.00324860960245132 21.0
75.27059646 1.04232597351074 22.0
153.34912926 0.331094682216644 23.0
115.8443991 0.798750638961792 24.0
19.945798068 0.142544373869896 25.0
228.61972572 0.82480400800705 26.0
228.61972572 1.22032761573792 27.0
22.05016278 1.10213196277618 28.0
250.6698885 0.500159680843353 29.0
250.6698885 1.49120914936066 30.0
46.787550763 1.00263297557831 31.0
9.4819430592 0.683735370635986 32.0
19.48590769 1.49560701847076 33.0
19.48590769 1.4862105846405 34.0
19.48590769 0.0897237285971642 35.0
9.2366083867 0.0267139375209808 36.0
7.8973767602 0.388699978590012 37.0
0 0.0376167893409729 38.0
0 0.146665811538696 39.0
-10.369458928 0.70977509021759 40.0
-24.867781137 0.336549341678619 41.0
5.1221994611 0.323419481515884 42.0
-32.708595107 0.580185174942017 43.0
86.530920739 0.770148932933807 44.0
86.530920739 0.0656218007206917 45.0
7.0951640307 0.596697688102722 46.0
2.4811417063 0.316011071205139 47.0
170.46516087 0.214780122041702 48.0
164.58359213 0.255067735910416 49.0
0 0.362934082746506 50.0
82.129179099 0.0870069414377213 51.0
67.619509622 0.0841479897499084 52.0
72.199760421 0.0227791368961334 53.0
72.199760421 0.13419309258461 54.0
72.199760421 0.0686277374625206 55.0
0 9.44851684570312 56.0
75.27059646 0.102180823683739 57.0
228.61972572 0.455371737480164 58.0
250.6698885 0.372318238019943 59.0
19.48590769 -0.0356826446950436 60.0
0 0.0891270339488983 61.0
137.45253488 1.12843382358551 0.0
137.45253488 1.97231698036194 1.0
136.07196323 1.83066022396088 2.0
131.28285795 3.5935537815094 3.0
0 3.93045163154602 4.0
0 3.46678042411804 5.0
-70.109988846 0.0321379601955414 6.0
81.994084739 1.08489143848419 7.0
58.402267157 1.12168371677399 8.0
23.591817582 1.13953506946564 9.0
94.238198202 1.23037827014923 10.0
81.733337377 0.216499358415604 11.0
87.070470644 0.403968244791031 12.0
87.070470644 0.148228198289871 13.0
87.070470644 0.0721542462706566 14.0
0 0.179263651371002 15.0
0 0.170518532395363 16.0
0 1.54261374473572 17.0
59.690915468 1.69138300418854 18.0
-59.690915468 1.93104135990143 19.0
-59.690915468 0.209800109267235 20.0
-59.690915468 0.0164104662835598 21.0
81.294303089 1.49728238582611 22.0
152.09435353 1.47119438648224 23.0
118.43381445 0.528082013130188 24.0
23.460647476 0.183857873082161 25.0
233.38865662 0.885174751281738 26.0
233.38865662 1.91448056697845 27.0
23.916452276 1.52283239364624 28.0
257.30510889 0.83568286895752 29.0
257.30510889 1.71690392494202 30.0
47.099540731 1.06265926361084 31.0
9.2880710752 0.528415739536285 32.0
21.603387621 1.47334623336792 33.0
21.603387621 0.871881127357483 34.0
21.603387621 0.139819249510765 35.0
9.6140214729 0.0277354344725609 36.0
8.3990975666 0.348192781209946 37.0
0 0.0544004775583744 38.0
0 0.22108106315136 39.0
-13.115351 0.751643598079681 40.0
-25.437758467 0.631169736385345 41.0
5.2733343001 0.540335953235626 42.0
-33.552566985 0.728326201438904 43.0
81.128491287 0.599133014678955 44.0
81.128491287 0.120124384760857 45.0
6.9589034038 0.537972152233124 46.0
2.7479313368 0.373104840517044 47.0
137.45253488 0.153158694505692 48.0
131.28285795 0.25774160027504 49.0
0 0.0800643637776375 50.0
81.994084739 0.170469909906387 51.0
94.238198202 0.189304068684578 52.0
87.070470644 0.0537810102105141 53.0
87.070470644 0.0700355097651482 54.0
87.070470644 0.0598481297492981 55.0
0 0.302005439996719 56.0
81.294303089 0.164541214704514 57.0
233.38865662 0.613406598567963 58.0
257.30510889 0.252030253410339 59.0
21.603387621 0.0200321935117245 60.0
0 0.0903976038098335 61.0
178.88257624 0.751320958137512 0.0
178.88257624 0.917753338813782 1.0
177.59130817 1.54596447944641 2.0
172.80223822 4.58943796157837 3.0
0 5.44524621963501 4.0
0 4.03782510757446 5.0
-57.634826544 0.335281908512115 6.0
66.726505284 0.505849599838257 7.0
62.939607911 1.34816920757294 8.0
3.7868973734 1.55282437801361 9.0
99.553406608 0.861525297164917 10.0
87.867065851 0.288726806640625 11.0
86.619235189 0.305946469306946 12.0
86.619235189 0.0713485032320023 13.0
86.619235189 0.0541081540286541 14.0
0 0.168106034398079 15.0
0 0.199708864092827 16.0
0 0.589014112949371 17.0
60.665412426 1.32554614543915 18.0
-60.665412426 1.50053751468658 19.0
-60.665412426 0.738242089748383 20.0
-60.665412426 -0.0594115555286407 21.0
80.79284932 1.95699882507324 22.0
146.27498479 1.50699889659882 23.0
108.64730467 0.674422442913055 24.0
19.837339198 0.0773445814847946 25.0
227.06783411 1.1445564031601 26.0
227.06783411 1.82659816741943 27.0
23.737830274 1.78508365154266 28.0
250.80566439 0.941242218017578 29.0
250.80566439 1.75366461277008 30.0
46.496000145 1.1497038602829 31.0
9.3413294868 0.529549717903137 32.0
20.127436894 1.75905919075012 33.0
20.127436894 0.783385276794434 34.0
20.127436894 0.116377234458923 35.0
9.0441614093 0.0334497168660164 36.0
7.902384413 0.350644320249557 37.0
0 0.0250100940465927 38.0
0 0.0763727426528931 39.0
-11.859219815 0.407698214054108 40.0
-25.876849243 0.601775884628296 41.0
5.1341897089 0.39846795797348 42.0
-33.614983801 0.43598997592926 43.0
100.01353907 0.727316439151764 44.0
100.01353907 0.0996879264712334 45.0
6.9635624408 0.526099324226379 46.0
2.4098550053 0.357126772403717 47.0
178.88257624 0.121046118438244 48.0
172.80223822 0.345021963119507 49.0
0 0.265084147453308 50.0
66.726505284 0.233583211898804 51.0
99.553406608 0.369257211685181 52.0
86.619235189 0.0941308736801147 53.0
86.619235189 0.0915354192256927 54.0
86.619235189 0.0598352439701557 55.0
0 2.15453791618347 56.0
80.79284932 0.033743791282177 57.0
227.06783411 0.712723314762115 58.0
250.80566439 0.463341355323792 59.0
20.127436894 -0.0114327147603035 60.0
0 0.130283743143082 61.0
183.50194191 1.00746524333954 0.0
183.50194191 0.662627935409546 1.0
182.44495821 1.14628028869629 2.0
176.9803228 3.21010875701904 3.0
3.9347771982 5.29699945449829 4.0
3.9347771982 7.77082395553589 5.0
-70.478577559 5.24086570739746 6.0
83.46863161 1.34919333457947 7.0
39.635876301 0.493392407894135 8.0
43.832755309 0.308586895465851 9.0
79.785034309 0.949303507804871 10.0
65.742819528 0.00311758974567056 11.0
78.759121823 0.658830165863037 12.0
78.759121823 0.111189320683479 13.0
78.759121823 0.0394781269133091 14.0
0 -0.00870328769087791 15.0
0 0.243096128106117 16.0
0 0.955014526844025 17.0
58.186923061 0.913798272609711 18.0
-58.186923061 1.15552890300751 19.0
-58.186923061 0.50487756729126 20.0
-58.186923061 -0.0385962352156639 21.0
78.677818263 1.17582774162292 22.0
152.33173281 1.16428303718567 23.0
118.63987427 0.324837416410446 24.0
24.436198726 0.144592761993408 25.0
231.00955107 1.28681600093842 26.0
231.00955107 2.36133933067322 27.0
23.935825653 2.38003849983215 28.0
254.94537672 0.632612705230713 29.0
254.94537672 1.73366355895996 30.0
46.583001816 1.04359436035156 31.0
9.1788722165 0.350889772176743 32.0
20.490895202 1.51177310943604 33.0
20.490895202 0.406363040208817 34.0
20.490895202 0.139372006058693 35.0
9.1868756553 0.0339343547821045 36.0
7.8484648229 0.301257193088531 37.0
0 0.0688634067773819 38.0
0 0.096707820892334 39.0
-12.158425293 1.22209823131561 40.0
-25.18258962 0.346101373434067 41.0
5.7816291407 0.528746366500854 42.0
-33.765883961 0.399656057357788 43.0
91.828524748 0.701762139797211 44.0
91.828524748 0.0918647423386574 45.0
7.1935378074 0.636696457862854 46.0
2.9249886623 0.408212244510651 47.0
183.50194191 0.0963153466582298 48.0
176.9803228 0.0599168539047241 49.0
3.9347771982 0.280784547328949 50.0
83.46863161 0.165560558438301 51.0
79.785034309 0.13210654258728 52.0
78.759121823 0.00188236520625651 53.0
78.759121823 0.103533901274204 54.0
78.759121823 0.0882363617420197 55.0
0 5.06395483016968 56.0
78.677818263 0.018245005980134 57.0
231.00955107 0.786473095417023 58.0
254.94537672 0.407153517007828 59.0
20.490895202 0.000190341845154762 60.0
0 0.103030316531658 61.0
158.88439459 0.79080206155777 0.0
158.88439459 2.06514811515808 1.0
157.8045222 1.83234441280365 2.0
153.01541711 3.24974703788757 3.0
0 3.43024754524231 4.0
0 4.33539438247681 5.0
-63.973091784 1.75652897357941 6.0
72.893885295 0.382496803998947 7.0
56.69149516 0.742991268634796 8.0
16.202390135 0.957570552825928 9.0
88.374091766 0.891270041465759 10.0
74.382832456 0.174941971898079 11.0
80.666251306 0.421808183193207 12.0
80.666251306 0.349936962127686 13.0
80.666251306 0.105279795825481 14.0
0 0.184310346841812 15.0
0 0.175939679145813 16.0
0 1.20536375045776 17.0
52.274922202 1.03300511837006 18.0
-52.274922202 0.819904386997223 19.0
-52.274922202 0.343191683292389 20.0
-52.274922202 0.0645703226327896 21.0
72.985663326 1.59306573867798 22.0
150.83602754 1.85037970542908 23.0
116.10478598 0.189448982477188 24.0
22.093544967 0.289295494556427 25.0
223.82169087 0.833393037319183 26.0
223.82169087 2.11422348022461 27.0
25.628329112 1.97035610675812 28.0
249.45001998 0.91418731212616 29.0
249.45001998 2.0055513381958 30.0
46.637941468 0.884432435035706 31.0
9.2672935651 0.512581169605255 32.0
20.710741123 1.37547516822815 33.0
20.710741123 1.10076200962067 34.0
20.710741123 0.161262229084969 35.0
9.3637599931 0.0416520200669765 36.0
8.056809081 0.386390089988708 37.0
0 0.0186292007565498 38.0
0 0.300703883171082 39.0
-11.144869703 0.564981698989868 40.0
-23.864928176 0.853855967521667 41.0
5.5900310202 0.386041074991226 42.0
-32.077750694 0.52333527803421 43.0
101.45619394 0.710873782634735 44.0
101.45619394 0.136147677898407 45.0
6.4337184456 0.632760107517242 46.0
1.7946838339 0.449905425310135 47.0
158.88439459 0.0483594946563244 48.0
153.01541711 0.145444869995117 49.0
0 0.356966286897659 50.0
72.893885295 0.0782046839594841 51.0
88.374091766 0.153064116835594 52.0
80.666251306 0.0712384432554245 53.0
80.666251306 0.11518158018589 54.0
80.666251306 0.133571982383728 55.0
0 5.56671619415283 56.0
72.985663326 0.111018277704716 57.0
223.82169087 0.573281824588776 58.0
249.45001998 1.02847707271576 59.0
20.710741123 -0.0226183291524649 60.0
0 0.0865188390016556 61.0
199.12524731 1.16833221912384 0.0
199.12524731 0.61841207742691 1.0
197.8906384 0.677204370498657 2.0
196.83465796 5.21027565002441 3.0
0 5.18277883529663 4.0
0 2.89084768295288 5.0
-65.711776749 0.197634518146515 6.0
79.702764951 0.896775841712952 7.0
42.861872273 1.65420401096344 8.0
36.840892679 1.7144787311554 9.0
76.232277528 1.09120833873749 10.0
63.42233561 0.0408610366284847 11.0
75.649322177 0.454584777355194 12.0
75.649322177 0.16741681098938 13.0
75.649322177 0.0733268260955811 14.0
0 -0.00675031542778015 15.0
0 0.254908084869385 16.0
0 1.35736525058746 17.0
53.518843641 1.06237626075745 18.0
-53.518843641 0.785921454429626 19.0
-54.187111383 0.85416704416275 20.0
-54.255672982 -0.0702551603317261 21.0
76.260885498 2.69920015335083 22.0
155.44044626 2.54532098770142 23.0
122.39650828 0.672584056854248 24.0
26.57525726 0.12038354575634 25.0
231.70133176 1.71406137943268 26.0
231.70133176 1.93723511695862 27.0
27.770340123 1.84861922264099 28.0
259.47167189 0.620547831058502 29.0
259.47167189 1.84491157531738 30.0
46.285924575 1.27673900127411 31.0
9.4929314519 0.743785917758942 32.0
22.005212516 1.5616043806076 33.0
22.005212516 1.12949788570404 34.0
22.005212516 0.100354194641113 35.0
9.6370644044 0.0323916561901569 36.0
8.2869771395 0.362466633319855 37.0
0 0.107489466667175 38.0
0 0.30365327000618 39.0
-12.814878993 0.238463118672371 40.0
-24.050421379 0.929577767848969 41.0
5.5430450619 0.503574132919312 42.0
-32.48033847 0.671602427959442 43.0
93.082533863 0.766636192798615 44.0
91.84580085 0.0591483414173126 45.0
7.5289389054 0.639345109462738 46.0
2.2673579625 0.379301577806473 47.0
199.12524731 0.100103743374348 48.0
196.83465796 0.0495613366365433 49.0
0 0.233573988080025 50.0
79.702764951 0.333881616592407 51.0
76.232277528 0.112077139317989 52.0
75.649322177 0.0108446031808853 53.0
75.649322177 0.08648881316185 54.0
75.649322177 0.0734529942274094 55.0
0 5.02834892272949 56.0
76.260885498 0.113732941448689 57.0
231.70133176 0.959077775478363 58.0
259.47167189 0.315606772899628 59.0
22.005212516 0.0175221897661686 60.0
0 0.0891422480344772 61.0
147.64510332 0.708131730556488 0.0
147.64510332 1.77965331077576 1.0
146.55060815 1.71937024593353 2.0
141.76150303 3.29348659515381 3.0
0 4.57436561584473 4.0
0 6.41393804550171 5.0
-58.864224825 2.22457361221313 6.0
68.890408837 0.544228255748749 7.0
53.267762399 1.10652244091034 8.0
15.622646438 1.09021019935608 9.0
92.408398428 0.858854532241821 10.0
79.131947024 0.423697799444199 11.0
79.25935887 0.462181568145752 12.0
79.25935887 0.208146840333939 13.0
79.25935887 0.080887958407402 14.0
0 0.0344610251486301 15.0
0 0.20353689789772 16.0
0 1.34629583358765 17.0
62.64504577 1.15103280544281 18.0
-62.64504577 1.4230831861496 19.0
-62.64504577 0.261338621377945 20.0
-62.64504577 0.151854544878006 21.0
82.099628439 2.18597102165222 22.0
162.38499289 2.37264490127563 23.0
123.19002458 0.355874210596085 24.0
24.60156965 0.362518846988678 25.0
244.48462133 1.42542409896851 26.0
244.48462133 2.36697316169739 27.0
23.783892693 3.07009625434875 28.0
268.26851402 0.967801570892334 29.0
268.26851402 1.93437504768372 30.0
46.813954198 1.04641485214233 31.0
9.4658915688 0.68111664056778 32.0
19.454582669 1.32755172252655 33.0
19.454582669 1.24902153015137 34.0
19.454582669 0.150699108839035 35.0
8.886027625 0.0393403917551041 36.0
7.7818008788 0.405156016349792 37.0
0 0.010322529822588 38.0
0 0.0616783574223518 39.0
-12.137715725 0.514677405357361 40.0
-25.441888091 0.501027941703796 41.0
5.301548615 0.409249186515808 42.0
-33.406430156 0.27536952495575 43.0
100.19550888 0.83639669418335 44.0
100.19550888 0.038120724260807 45.0
7.493441007 0.698457181453705 46.0
2.7125885448 0.390946388244629 47.0
147.64510332 0.190178290009499 48.0
141.76150303 0.242192685604095 49.0
0 0.691029250621796 50.0
68.890408837 0.364957064390182 51.0
92.408398428 0.239416673779488 52.0
79.25935887 0.0910840183496475 53.0
79.25935887 0.0314029268920422 54.0
79.25935887 0.0885281413793564 55.0
0 5.45232439041138 56.0
82.099628439 -0.0243554152548313 57.0
244.48462133 0.646613836288452 58.0
268.26851402 0.314445972442627 59.0
19.454582669 -0.015738382935524 60.0
0 0.0980808436870575 61.0
162.82040973 0.99894255399704 0.0
162.82040973 0.774955809116364 1.0
161.67365999 0.874804794788361 2.0
156.20898871 3.4314239025116 3.0
13.409606088 4.94115543365479 4.0
13.409606088 3.59904932975769 5.0
-50.045577565 1.7483628988266 6.0
62.449872396 1.12069976329803 7.0
39.068749107 0.359711408615112 8.0
23.381123288 0.34051126241684 9.0
90.306452882 0.984950065612793 10.0
77.868051105 0.124681428074837 11.0
76.465549532 0.109721764922142 12.0
76.465549532 0.347407162189484 13.0
76.465549532 0.0210201479494572 14.0
0 -0.00240765605121851 15.0
0 0.169105529785156 16.0
0 0.953877031803131 17.0
55.765361793 0.895882308483124 18.0
-55.765361793 0.487537294626236 19.0
-55.765361793 0.780027091503143 20.0
-55.765361793 -0.0326168835163116 21.0
77.999182182 1.87509477138519 22.0
136.97153837 1.57111179828644 23.0
102.84398519 0.354452669620514 24.0
20.645558752 0.14935901761055 25.0
214.97072055 1.07369649410248 26.0
214.97072055 1.4186863899231 27.0
24.266400295 1.78234505653381 28.0
239.23712085 0.858800172805786 29.0
239.23712085 1.77194368839264 30.0
46.322022249 1.09024524688721 31.0
9.4339629471 0.394093751907349 32.0
22.233820389 1.07539546489716 33.0
22.233820389 1.01691353321075 34.0
22.233820389 0.106154665350914 35.0
10.202991064 0.0244341399520636 36.0
8.9589376461 0.389727473258972 37.0
0 0.0373336374759674 38.0
0 0.320530831813812 39.0
-12.783321886 0.61417955160141 40.0
-27.204014881 0.381116539239883 41.0
5.5506581516 0.529331028461456 42.0
-34.965974406 0.290493428707123 43.0
76.712341419 0.847606003284454 44.0
76.712341419 0.0929021015763283 45.0
6.1201878798 0.760817468166351 46.0
2.2802560452 0.407781064510345 47.0
162.82040973 0.126440152525902 48.0
156.20898871 0.276153862476349 49.0
13.409606088 0.338782906532288 50.0
62.449872396 0.0765667259693146 51.0
90.306452882 0.0932100266218185 52.0
76.465549532 0.034052699804306 53.0
76.465549532 0.12043934315443 54.0
76.465549532 0.106351859867573 55.0
0 2.55998706817627 56.0
77.999182182 0.0458986684679985 57.0
214.97072055 0.483524739742279 58.0
239.23712085 0.733847200870514 59.0
22.233820389 -0.00604180060327053 60.0
0 0.108855456113815 61.0
168.09723337 0.885767042636871 0.0
168.09723337 0.956972360610962 1.0
166.97230805 1.11625492572784 2.0
162.18320281 5.82098579406738 3.0
0 5.31840515136719 4.0
0 5.5937614440918 5.0
-61.955446098 1.32058322429657 6.0
70.683542734 0.901358723640442 7.0
41.721407717 0.939769864082336 8.0
28.962135018 0.726278305053711 9.0
75.219422944 0.932693719863892 10.0
63.048129125 0.0060812090523541 11.0
75.81977318 0.12002332508564 12.0
75.81977318 0.339142143726349 13.0
75.81977318 0.136415109038353 14.0
0 0.0873342826962471 15.0
0 0.109715655446053 16.0
0 1.32186841964722 17.0
53.541416149 1.32803690433502 18.0
-53.541416149 1.42634344100952 19.0
-53.541416149 0.240816801786423 20.0
-53.541416149 0.187025532126427 21.0
76.785382887 2.5181884765625 22.0
154.94315424 2.42270398139954 23.0
120.27227311 0.406896322965622 24.0
20.069453433 0.207693353295326 25.0
231.72853713 2.09514546394348 26.0
231.72853713 2.1592013835907 27.0
30.221068046 1.96920609474182 28.0
261.94960518 1.16393077373505 29.0
261.94960518 1.88655614852905 30.0
46.773432039 1.00617790222168 31.0
9.3729665796 0.448413670063019 32.0
23.243966737 0.962125957012177 33.0
23.243966737 0.782724261283875 34.0
23.243966737 0.0985158979892731 35.0
9.7303444059 0.0254130382090807 36.0
8.5919973492 0.383919090032578 37.0
0 0.0220430158078671 38.0
0 0.296476244926453 39.0
-11.583491201 0.473969429731369 40.0
-23.561706517 0.554639399051666 41.0
4.9054618198 0.565072774887085 42.0
-30.976356926 0.709348022937775 43.0
94.71350396 0.577989041805267 44.0
94.71350396 0.0925491228699684 45.0
6.3046294904 0.528083503246307 46.0
2.1381165882 0.433659881353378 47.0
168.09723337 0.154248714447021 48.0
162.18320281 0.0466595999896526 49.0
0 0.515178561210632 50.0
70.683542734 0.414241075515747 51.0
75.219422944 0.132311761379242 52.0
75.81977318 0.0103483740240335 53.0
75.81977318 0.102968193590641 54.0
75.81977318 0.0100139155983925 55.0
0 6.12263965606689 56.0
76.785382887 0.0378909520804882 57.0
231.72853713 1.11774575710297 58.0
261.94960518 0.600914776325226 59.0
23.243966737 0.0140998438000679 60.0
0 0.113571867346764 61.0
147.42280297 0.796269297599792 0.0
147.42280297 2.15931367874146 1.0
146.27362907 2.42618489265442 2.0
140.80895841 5.92823123931885 3.0
4.4698785838 5.73908805847168 4.0
4.4698785838 6.05264043807983 5.0
-71.082941197 2.94980764389038 6.0
80.862304397 0.499594837427139 7.0
55.56066924 0.737301290035248 8.0
25.301635157 0.91112220287323 9.0
93.566981961 0.930428147315979 10.0
79.680570752 0.351521790027618 11.0
77.506286526 0.294193714857101 12.0
77.506286526 0.50338751077652 13.0
77.506286526 0.113792762160301 14.0
0 0.128211244940758 15.0
0 0.148043364286423 16.0
0 1.18657338619232 17.0
56.795902649 1.58174920082092 18.0
-56.795902649 1.62639570236206 19.0
-56.795902649 0.31932544708252 20.0
-56.795902649 0.00882707163691521 21.0
76.487551981 1.47355914115906 22.0
137.11828682 1.25583934783936 23.0
104.82222192 0.457218885421753 24.0
23.758469265 0.107853457331657 25.0
213.6058388 0.780499696731567 26.0
213.6058388 1.48036372661591 27.0
23.029202407 1.57489275932312 28.0
236.63504121 0.712928295135498 29.0
236.63504121 1.69772469997406 30.0
46.722111655 1.13777089118958 31.0
9.3358527973 0.611590147018433 32.0
19.691649332 1.85566079616547 33.0
19.691649332 1.0514794588089 34.0
19.691649332 0.154081419110298 35.0
8.7535829019 0.0382118374109268 36.0
7.5562319937 0.334916204214096 37.0
0 0.00360489636659622 38.0
0 0.163411378860474 39.0
-13.016319271 1.09657514095306 40.0
-26.025367913 0.492027163505554 41.0
4.8668621043 0.530088722705841 42.0
-33.453968037 0.453780263662338 43.0
96.517267085 0.70498138666153 44.0
96.517267085 0.0858306437730789 45.0
6.6882226804 0.691270172595978 46.0
2.4335497707 0.402426868677139 47.0
147.42280297 0.115733653306961 48.0
140.80895841 0.539792954921722 49.0
4.4698785838 0.363057911396027 50.0
80.862304397 0.130037322640419 51.0
93.566981961 0.319660097360611 52.0
77.506286526 0.0310284215956926 53.0
77.506286526 0.061987966299057 54.0
77.506286526 0.127249404788017 55.0
0 4.59804677963257 56.0
76.487551981 -0.0449902340769768 57.0
213.6058388 0.231079429388046 58.0
236.63504121 0.269331097602844 59.0
19.691649332 0.00175613351166248 60.0
0 0.114494368433952 61.0
153.43281263 0.933929443359375 0.0
153.43281263 1.31136500835419 1.0
152.29742346 2.00127983093262 2.0
147.50831836 4.14150285720825 3.0
0 8.73071002960205 4.0
0 3.05264115333557 5.0
-55.33403125 1.49111080169678 6.0
64.680885758 0.364767104387283 7.0
55.90392579 0.826726734638214 8.0
8.7769599678 0.997127115726471 9.0
79.358245842 1.00280940532684 10.0
67.233239128 0.260385721921921 11.0
76.698864089 0.649517953395844 12.0
76.698864089 0.543789505958557 13.0
76.698864089 0.0950520634651184 14.0
0 0.0633267015218735 15.0
0 0.150188773870468 16.0
0 1.60845458507538 17.0
51.693016285 1.04316759109497 18.0
-51.693016285 1.28616714477539 19.0
-51.693016285 0.433970600366592 20.0
-51.693016285 0.0732256323099136 21.0
73.691466029 1.64682078361511 22.0
150.78246863 1.91405701637268 23.0
117.02939498 0.3909991979599 24.0
24.658515063 0.347127795219421 25.0
224.47393466 1.20090425014496 26.0
224.47393466 2.33456444740295 27.0
23.92050135 2.80227899551392 28.0
248.39443601 1.47759008407593 29.0
248.39443601 2.26855993270874 30.0
46.193590316 0.988496601581573 31.0
9.283494359 0.551908671855927 32.0
21.998449744 1.12730526924133 33.0
21.998449744 1.09156894683838 34.0
21.998449744 0.12108951061964 35.0
9.6299717634 0.0343629345297813 36.0
8.3606895512 0.399237513542175 37.0
0 0.0193595923483372 38.0
0 0.278364777565002 39.0
-12.14428464 0.606966316699982 40.0
-23.63109674 0.633509278297424 41.0
4.6354934804 0.553139805793762 42.0
-30.961411487 0.696441173553467 43.0
104.37571961 0.806728541851044 44.0
104.37571961 0.123728089034557 45.0
6.8276362305 0.677291452884674 46.0
2.7726354106 0.387090563774109 47.0
153.43281263 0.0963660478591919 48.0
147.50831836 0.244133114814758 49.0
0 0.221005350351334 50.0
64.680885758 0.113668762147427 51.0
79.358245842 0.366976410150528 52.0
76.698864089 0.0699630081653595 53.0
76.698864089 0.0699231177568436 54.0
76.698864089 0.203985124826431 55.0
0 4.3696460723877 56.0
73.691466029 0.0892743170261383 57.0
224.47393466 0.723080217838287 58.0
248.39443601 1.05803084373474 59.0
21.998449744 -0.00580788217484951 60.0
0 0.109598264098167 61.0
183.36499001 1.24041938781738 0.0
183.36499001 0.482128739356995 1.0
182.20610464 1.29897391796112 2.0
177.4169994 4.52911949157715 3.0
0 7.65555047988892 4.0
0 4.68708610534668 5.0
-57.059025243 3.97314596176147 6.0
69.730678505 0.703238487243652 7.0
48.308109468 1.05420243740082 8.0
21.422569036 0.735486447811127 9.0
90.53103808 1.3003762960434 10.0
76.942783961 0.198000952601433 11.0
80.514720218 0.583062469959259 12.0
80.514720218 0.109441176056862 13.0
80.514720218 0.0589556023478508 14.0
0 0.0363869965076447 15.0
0 0.189450472593307 16.0
0 1.15698778629303 17.0
53.288082054 1.31272041797638 18.0
-53.288082054 1.2042510509491 19.0
-53.288082054 0.681594967842102 20.0
-53.288082054 -0.000646412372589111 21.0
75.392631835 2.65413379669189 22.0
141.21805081 2.48461413383484 23.0
107.78018536 0.229862898588181 24.0
22.320689248 0.206115931272507 25.0
216.61068264 1.43082439899445 26.0
216.61068264 1.84809756278992 27.0
23.74839415 1.40249085426331 28.0
240.35907679 0.663047552108765 29.0
240.35907679 1.62707686424255 30.0
46.651429595 0.952883124351501 31.0
9.303896409 0.327268332242966 32.0
22.104549781 1.3908497095108 33.0
22.104549781 0.602687537670135 34.0
22.104549781 0.0507210716605186 35.0
10.055530399 0.0193463414907455 36.0
8.8171706205 0.440992087125778 37.0
0 0.0547928959131241 38.0
0 0.246985226869583 39.0
-11.91903148 0.425744771957397 40.0
-25.394480768 0.762138426303864 41.0
5.4555379973 0.418513476848602 42.0
-33.16946222 0.644792020320892 43.0
98.722965964 0.888315975666046 44.0
98.722965964 0.109060876071453 45.0
7.240545659 0.538979172706604 46.0
2.5772658285 0.311530709266663 47.0
183.36499001 0.0779465138912201 48.0
177.4169994 0.286693543195724 49.0
0 0.309265434741974 50.0
69.730678505 0.261909067630768 51.0
90.53103808 0.105726383626461 52.0
80.514720218 0.0661923661828041 53.0
80.514720218 0.0496045053005219 54.0
80.514720218 0.0325984954833984 55.0
0 3.23047089576721 56.0
75.392631835 0.0541315227746964 57.0
216.61068264 0.78725528717041 58.0
240.35907679 0.852892816066742 59.0
22.104549781 0.00722856633365154 60.0
0 0.0997292697429657 61.0
185.69709554 1.05524253845215 0.0
185.69709554 0.605674088001251 1.0
184.30768179 1.19045162200928 2.0
179.5185765 5.03708362579346 3.0
0 6.23768424987793 4.0
0 4.41423034667969 5.0
-61.289837524 0.731576025485992 6.0
71.286848605 1.22143638134003 7.0
56.859352462 0.459378272294998 8.0
14.427496144 0.333618849515915 9.0
87.488691141 1.08371078968048 10.0
74.353241882 0.0470233596861362 11.0
82.807554838 0.717571020126343 12.0
82.807554838 0.553512811660767 13.0
82.807554838 0.0949155017733574 14.0
0 0.147041499614716 15.0
0 0.210690587759018 16.0
0 1.41504955291748 17.0
56.444791928 1.42180597782135 18.0
-56.444791928 1.46900749206543 19.0
-56.444791928 0.321096301078796 20.0
-56.444791928 -0.00650669261813164 21.0
77.957533042 1.11193406581879 22.0
154.66238923 0.876318335533142 23.0
118.17685465 0.577330827713013 24.0
21.671318225 0.124920077621937 25.0
232.61992227 1.10834074020386 26.0
232.61992227 1.25489103794098 27.0
28.757723815 1.3404153585434 28.0
261.37764608 0.998532950878143 29.0
261.37764608 1.77959191799164 30.0
46.68706274 0.994426012039185 31.0
9.4051416136 0.395752221345901 32.0
21.512741114 1.61484098434448 33.0
21.512741114 0.956378400325775 34.0
21.512741114 0.157020971179008 35.0
10.014669628 0.0318176373839378 36.0
8.7719386072 0.439269006252289 37.0
0 0.0169552192091942 38.0
0 0.203031569719315 39.0
-11.632672878 0.34798675775528 40.0
-25.542325801 0.675689041614532 41.0
4.812660968 0.312331199645996 42.0
-32.612106724 0.298578768968582 43.0
84.197071553 0.70865535736084 44.0
84.197071553 0.114424273371696 45.0
6.508926969 0.632449388504028 46.0
2.5322661494 0.414701342582703 47.0
185.69709554 0.100958086550236 48.0
179.5185765 0.11371748149395 49.0
0 0.414908796548843 50.0
71.286848605 0.0909160226583481 51.0
87.488691141 0.124661967158318 52.0
82.807554838 0.022884663194418 53.0
82.807554838 0.260253965854645 54.0
82.807554838 0.0741299986839294 55.0
0 3.49878811836243 56.0
77.957533042 0.156805887818336 57.0
232.61992227 0.584029853343964 58.0
261.37764608 0.836855173110962 59.0
21.512741114 -0.0144621767103672 60.0
0 0.0921387746930122 61.0
182.15122468 0.858787536621094 0.0
182.15122468 1.76690638065338 1.0
180.73983126 1.89446663856506 2.0
175.27516072 4.05795431137085 3.0
19.318032617 6.17099189758301 4.0
19.318032617 4.56197500228882 5.0
-55.860986796 1.09743523597717 6.0
66.132970661 0.607142627239227 7.0
39.975908451 0.826774597167969 8.0
26.15706221 0.933592557907104 9.0
84.879677044 0.883353412151337 10.0
72.177244408 0.128685012459755 11.0
77.628064773 0.632736086845398 12.0
77.628064773 0.255060315132141 13.0
77.628064773 0.0375978201627731 14.0
0 0.0112034250050783 15.0
0 0.199800878763199 16.0
0 0.957026183605194 17.0
58.523749548 1.6981109380722 18.0
-58.523749548 1.83870673179626 19.0
-58.523749548 0.662107288837433 20.0
-58.523749548 -0.0165716931223869 21.0
80.186642185 1.35979509353638 22.0
134.62696398 0.986329019069672 23.0
100.99740769 0.659434914588928 24.0
20.480809265 0.0792835354804993 25.0
214.81360616 0.675788700580597 26.0
214.81360616 1.07009851932526 27.0
28.683099954 1.14930808544159 28.0
243.49670612 0.510861873626709 29.0
243.49670612 1.46556901931763 30.0
46.723723671 1.16273832321167 31.0
9.2132061853 0.318861722946167 32.0
21.662892637 1.84616565704346 33.0
21.662892637 0.921013355255127 34.0
21.662892637 0.122446149587631 35.0
9.2760919471 0.0342350155115128 36.0
8.0831128201 0.390449106693268 37.0
0 0.0105191320180893 38.0
0 0.230928346514702 39.0
-11.59301581 0.602786183357239 40.0
-23.957851687 0.595685839653015 41.0
5.2166770354 0.446564078330994 42.0
-32.06138413 0.499984532594681 43.0
99.71958807 0.738499462604523 44.0
99.71958807 0.0786589458584785 45.0
6.5710286264 0.471112668514252 46.0
1.7499330797 0.338409066200256 47.0
182.15122468 0.0680012628436089 48.0
175.27516072 0.257623970508575 49.0
19.318032617 0.162649035453796 50.0
66.132970661 0.313504457473755 51.0
84.879677044 0.0724793672561646 52.0
77.628064773 0.0491499193012714 53.0
77.628064773 0.176039695739746 54.0
77.628064773 0.0866962671279907 55.0
0 3.99510312080383 56.0
80.186642185 0.0689305812120438 57.0
214.81360616 0.291097581386566 58.0
243.49670612 0.553716540336609 59.0
21.662892637 -0.00537311099469662 60.0
0 0.0896536037325859 61.0
172.51114396 0.859100043773651 0.0
172.51114396 0.953552603721619 1.0
171.27931764 1.6426739692688 2.0
166.49021243 4.88148164749146 3.0
0 8.03607845306396 4.0
0 3.51444697380066 5.0
-61.641178044 0.0440505594015121 6.0
72.040995515 0.630763053894043 7.0
41.682932815 1.43247449398041 8.0
30.3580627 1.28828847408295 9.0
81.974761705 0.869587063789368 10.0
68.638218949 0.146017700433731 11.0
77.442057368 0.82225489616394 12.0
77.442057368 0.40839758515358 13.0
77.442057368 0.0604360476136208 14.0
0 0.112532936036587 15.0
0 0.192822188138962 16.0
0 1.10448837280273 17.0
55.696548781 1.46029078960419 18.0
-55.696548781 0.94445413351059 19.0
-55.696548781 0.695158660411835 20.0
-55.696548781 -0.0376901887357235 21.0
77.421154938 1.31617414951324 22.0
142.25052859 1.42852818965912 23.0
106.98702057 0.379229366779327 24.0
23.671706516 0.118822641670704 25.0
219.67168353 1.0605366230011 26.0
219.67168353 1.82550704479218 27.0
26.035619906 2.09755659103394 28.0
245.70730343 1.25882506370544 29.0
245.70730343 1.8855437040329 30.0
46.191302233 1.06137013435364 31.0
9.3669889842 0.52629828453064 32.0
21.724606157 1.35610735416412 33.0
21.724606157 1.17116498947144 34.0
21.724606157 0.141342610120773 35.0
9.4407811129 0.0458224266767502 36.0
8.0693438615 0.359854280948639 37.0
0 -0.0160797014832497 38.0
0 0.155979514122009 39.0
-11.601054442 0.800023972988129 40.0
-23.514216471 0.447962701320648 41.0
4.8436084968 0.470343023538589 42.0
-30.774000387 0.522029399871826 43.0
99.819583465 0.696986794471741 44.0
99.819583465 0.0895337611436844 45.0
6.1339190015 0.505633234977722 46.0
2.3866627823 0.364161342382431 47.0
172.51114396 0.284845799207687 48.0
166.49021243 0.450084805488586 49.0
0 0.417493730783463 50.0
72.040995515 0.212352231144905 51.0
81.974761705 0.278816074132919 52.0
77.442057368 0.104830861091614 53.0
77.442057368 0.0476254485547543 54.0
77.442057368 0.180449441075325 55.0
0 5.45248413085938 56.0
77.421154938 0.181769102811813 57.0
219.67168353 0.621838390827179 58.0
245.70730343 0.713003098964691 59.0
21.724606157 -0.00620096083730459 60.0
0 0.103490807116032 61.0
142.65125621 0.809024751186371 0.0
142.65125621 1.88196241855621 1.0
141.46382505 1.56865072250366 2.0
136.67471977 3.66175746917725 3.0
0 5.72452974319458 4.0
0 3.83826398849487 5.0
-53.047337633 1.07332396507263 6.0
62.868302581 0.419821053743362 7.0
50.454465082 0.934461891651154 8.0
12.413837499 1.00911486148834 9.0
84.054308469 0.953055322170258 10.0
71.676519478 0.236096724867821 11.0
78.119222843 0.717122733592987 12.0
78.119222843 0.573404669761658 13.0
78.119222843 0.0734401792287827 14.0
0 0.117890723049641 15.0
0 0.160091131925583 16.0
0 1.55786979198456 17.0
58.20722455 1.25934731960297 18.0
-58.20722455 1.01465654373169 19.0
-58.20722455 0.408394783735275 20.0
-58.20722455 0.0492700301110744 21.0
80.824150762 1.41147100925446 22.0
150.3563264 0.966493368148804 23.0
117.1218015 0.784079909324646 24.0
23.535244478 0.0762364864349365 25.0
231.18047716 0.765152513980865 26.0
231.18047716 1.24464511871338 27.0
25.92435656 1.52151775360107 28.0
257.10483372 0.987391471862793 29.0
257.10483372 1.98592901229858 30.0
47.225047811 1.06530654430389 31.0
9.3302974235 0.536527156829834 32.0
22.616926213 1.52451753616333 33.0
22.616926213 0.798855543136597 34.0
22.616926213 0.111171677708626 35.0
9.6492236207 0.0204262193292379 36.0
8.3012669969 0.451222747564316 37.0
0 0.0046941265463829 38.0
0 0.114301517605782 39.0
-12.196864239 0.671836078166962 40.0
-25.097144325 0.341687649488449 41.0
5.2977144127 0.450815767049789 42.0
-32.810812721 0.44223091006279 43.0
80.525869481 0.540885627269745 44.0
80.525869481 0.118204303085804 45.0
6.7850950291 0.510601282119751 46.0
2.7961196546 0.438408434391022 47.0
142.65125621 0.175684481859207 48.0
136.67471977 0.23737108707428 49.0
0 0.360812991857529 50.0
62.868302581 0.199080780148506 51.0
84.054308469 0.306487888097763 52.0
78.119222843 0.0851571038365364 53.0
78.119222843 0.0335825309157372 54.0
78.119222843 0.0622634626924992 55.0
0 2.0415096282959 56.0
80.824150762 -0.0570908784866333 57.0
231.18047716 0.223203033208847 58.0
257.10483372 0.701211929321289 59.0
22.616926213 -0.00215143710374832 60.0
0 0.103259228169918 61.0
183.59838832 0.782936751842499 0.0
183.59838832 0.63165670633316 1.0
182.20461209 0.534656882286072 2.0
181.1486269 7.89897298812866 3.0
0 3.44476270675659 4.0
0 5.6012020111084 5.0
-63.222082958 1.26971757411957 6.0
77.116407801 0.975637912750244 7.0
59.356349273 0.540170431137085 8.0
17.760058527 0.700173795223236 9.0
89.291597438 1.05222177505493 10.0
74.965026116 0.369882851839066 11.0
91.489591551 0.714422285556793 12.0
91.489591551 0.447599768638611 13.0
91.489591551 0.0838503390550613 14.0
0 0.0398723408579826 15.0
0 0.17882052063942 16.0
0 1.33709347248077 17.0
51.970655073 0.85222601890564 18.0
-51.970655073 0.513164877891541 19.0
-52.639042039 0.845062673091888 20.0
-52.707630903 -0.0168942678719759 21.0
73.313213694 1.38192355632782 22.0
150.85824886 1.4118914604187 23.0
114.53767404 0.0920073688030243 24.0
18.500473561 0.290348261594772 25.0
224.17146255 0.962707459926605 26.0
224.17146255 2.51125311851501 27.0
27.278734516 2.51701688766479 28.0
251.45019707 1.18275701999664 29.0
251.45019707 1.84388971328735 30.0
46.369056314 1.05812275409698 31.0
9.1644434319 0.631528615951538 32.0
20.605582791 1.26920819282532 33.0
20.605582791 1.39613008499146 34.0
20.605582791 0.0694156065583229 35.0
10.050335667 0.0224615205079317 36.0
8.77745706 0.387272208929062 37.0
0 0.0410671345889568 38.0
0 0.201489076018333 39.0
-11.107968174 0.52907121181488 40.0
-24.516531124 0.628455221652985 41.0
5.7590358269 0.34482604265213 42.0
-32.957133148 0.356743067502975 43.0
86.802073439 0.713005363941193 44.0
85.985049244 0.0789176002144814 45.0
6.5701378735 0.513804018497467 46.0
1.7460566892 0.371039837598801 47.0
183.59838832 0.16063916683197 48.0
181.1486269 0.29152774810791 49.0
0 0.219537138938904 50.0
77.116407801 0.272976100444794 51.0
89.291597438 0.64367413520813 52.0
91.489591551 0.0856630206108093 53.0
91.489591551 0.0217121262103319 54.0
91.489591551 0.0715135708451271 55.0
0 2.9813334941864 56.0
73.313213694 0.308206886053085 57.0
224.17146255 0.739126980304718 58.0
251.45019707 1.28805005550385 59.0
20.605582791 0.0123175233602524 60.0
0 0.0453248396515846 61.0
161.08499141 1.10797309875488 0.0
161.08499141 0.807829916477203 1.0
159.72609247 0.912957906723022 2.0
154.93698733 2.80114769935608 3.0
0 4.17750930786133 4.0
0 3.78791642189026 5.0
-63.962991452 1.73683249950409 6.0
73.75527475 0.626376509666443 7.0
58.986724341 1.54072296619415 8.0
14.768550409 1.54306995868683 9.0
82.662132407 1.09618532657623 10.0
69.24226841 0.223743677139282 11.0
80.291795633 0.160169690847397 12.0
80.291795633 0.198373526334763 13.0
80.291795633 0.083625391125679 14.0
0 0.0459986925125122 15.0
0 0.190593123435974 16.0
0 1.04541742801666 17.0
61.863075664 1.65310609340668 18.0
-61.863075664 1.35721170902252 19.0
-61.863075664 0.475396275520325 20.0
-61.863075664 -0.0385075360536575 21.0
81.565338039 1.85330593585968 22.0
152.22035394 1.37313079833984 23.0
116.79313295 0.686587810516357 24.0
23.118835444 0.168039619922638 25.0
233.78569198 0.985013842582703 26.0
233.78569198 1.79947757720947 27.0
28.523035509 1.52891552448273 28.0
262.30872749 0.41074725985527 29.0
262.30872749 1.57318162918091 30.0
47.03623386 1.19406473636627 31.0
9.2960587639 0.400670558214188 32.0
19.702262375 1.96275293827057 33.0
19.702262375 1.04107081890106 34.0
19.702262375 0.145173251628876 35.0
8.9144329158 0.0292265582829714 36.0
7.6147478742 0.438090145587921 37.0
0 0.0151508152484894 38.0
0 0.169731110334396 39.0
-12.452518571 0.443200767040253 40.0
-26.354479363 0.454201012849808 41.0
4.7547003466 0.385924637317657 42.0
-33.932763153 0.413965284824371 43.0
86.084364241 0.662518441677094 44.0
86.084364241 0.101432181894779 45.0
7.0805268604 0.596099138259888 46.0
2.3255628882 0.398272633552551 47.0
161.08499141 0.198705777525902 48.0
154.93698733 0.14164625108242 49.0
0 0.33950287103653 50.0
73.75527475 0.236962407827377 51.0
82.662132407 0.355425715446472 52.0
80.291795633 0.0507020875811577 53.0
80.291795633 0.0762331187725067 54.0
80.291795633 0.0838591903448105 55.0
0 5.08759498596191 56.0
81.565338039 0.0995342880487442 57.0
233.78569198 0.550109684467316 58.0
262.30872749 0.634141325950623 59.0
19.702262375 -0.0166891105473042 60.0
0 0.110593520104885 61.0
208.19507791 0.837999045848846 0.0
208.19507791 1.2761378288269 1.0
207.07100221 1.25258052349091 2.0
201.60633155 4.00454616546631 3.0
8.334401207 4.22818374633789 4.0
8.334401207 3.90731906890869 5.0
-66.84956132 -0.000803157687187195 6.0
77.680489141 0.795336782932281 7.0
39.643611205 0.744165778160095 8.0
38.036877936 0.545708656311035 9.0
80.355982428 1.06387782096863 10.0
67.830088895 0.232451245188713 11.0
79.821896599 0.738539159297943 12.0
79.821896599 0.602033197879791 13.0
79.821896599 0.16550301015377 14.0
0 0.132635667920113 15.0
0 0.164477840065956 16.0
0 1.87070953845978 17.0
57.382376149 1.39959728717804 18.0
-57.382376149 0.771075487136841 19.0
-57.382376149 1.26997542381287 20.0
-57.382376149 -0.0733785033226013 21.0
77.172956854 2.23806190490723 22.0
138.414463 2.12120318412781 23.0
101.51780044 0.478717654943466 24.0
25.948959276 0.0912805944681168 25.0
215.58741985 1.71602952480316 26.0
215.58741985 2.43099308013916 27.0
24.991933618 2.74292325973511 28.0
240.57935347 1.12424182891846 29.0
240.57935347 1.90789473056793 30.0
46.197126898 1.29951202869415 31.0
9.4118351257 0.381537288427353 32.0
19.790580705 1.42428719997406 33.0
19.790580705 0.587953269481659 34.0
19.790580705 0.0543479882180691 35.0
8.5599699413 0.0262830164283514 36.0
7.3144007844 0.393314480781555 37.0
0 0.0387365631759167 38.0
0 0.155025735497475 39.0
-11.547925793 0.798577904701233 40.0
-24.377747437 0.706475019454956 41.0
4.7726583151 0.461276799440384 42.0
-31.696908377 0.788300216197968 43.0
106.61987305 0.638987183570862 44.0
106.61987305 0.0608417093753815 45.0
6.6476481062 0.613487839698792 46.0
2.2812487108 0.423660963773727 47.0
208.19507791 0.100727498531342 48.0
201.60633155 0.0285817477852106 49.0
8.334401207 0.275322288274765 50.0
77.680489141 0.0664363875985146 51.0
80.355982428 0.246322706341743 52.0
79.821896599 0.0947942510247231 53.0
79.821896599 0.249263197183609 54.0
79.821896599 0.207529306411743 55.0
0 5.67131328582764 56.0
77.172956854 0.0646174252033234 57.0
215.58741985 0.955170512199402 58.0
240.57935347 0.553599238395691 59.0
19.790580705 0.0212780386209488 60.0
0 0.114398553967476 61.0
194.50170171 1.02935516834259 0.0
194.50170171 1.61478269100189 1.0
193.11810205 1.35008096694946 2.0
188.3289968 5.94211530685425 3.0
0 5.11630725860596 4.0
0 2.66228890419006 5.0
-60.475590556 0.234171122312546 6.0
72.861584866 1.05593478679657 7.0
61.586526893 1.2494705915451 8.0
11.275057973 0.959578573703766 9.0
77.612045379 1.06727051734924 10.0
65.088867742 0.0422745160758495 11.0
81.56923471 0.368477463722229 12.0
81.56923471 0.475472182035446 13.0
81.56923471 0.135456085205078 14.0
0 0.305055767297745 15.0
0 0.131678074598312 16.0
0 1.08685851097107 17.0
60.300457332 1.67870211601257 18.0
-60.300457332 1.61969685554504 19.0
-60.300457332 1.06533646583557 20.0
-60.300457332 -0.0710164979100227 21.0
81.60365662 0.983856678009033 22.0
132.76667178 0.478357344865799 23.0
101.85209844 0.51531445980072 24.0
22.75003202 0.139470934867859 25.0
214.3703284 0.894522368907928 26.0
214.3703284 1.90227961540222 27.0
24.55593141 2.11459994316101 28.0
238.92625981 0.922430038452148 29.0
238.92625981 1.84028458595276 30.0
46.97017428 1.05531394481659 31.0
9.3179788218 0.515474438667297 32.0
21.303199288 1.50909960269928 33.0
21.303199288 1.23508977890015 34.0
21.303199288 0.108162075281143 35.0
9.9554163939 0.0314384140074253 36.0
8.697504173 0.415121674537659 37.0
0 0.00591558590531349 38.0
0 0.281405657529831 39.0
-11.6305019 0.372585535049438 40.0
-24.488858547 0.78130042552948 41.0
5.6313806163 0.400716423988342 42.0
-32.848120611 0.481546938419342 43.0
95.984816747 0.595000743865967 44.0
95.984816747 0.122732050716877 45.0
7.0770272002 0.660912096500397 46.0
2.9411798606 0.436002850532532 47.0
194.50170171 0.136096477508545 48.0
188.3289968 0.105466283857822 49.0
0 0.0820357427000999 50.0
72.861584866 0.196529775857925 51.0
77.612045379 0.139943927526474 52.0
81.56923471 0.0153229413554072 53.0
81.56923471 -0.00273627880960703 54.0
81.56923471 0.0957971662282944 55.0
0 4.13109064102173 56.0
81.60365662 0.0913872718811035 57.0
214.3703284 0.747739672660828 58.0
238.92625981 0.626033186912537 59.0
21.303199288 0.00390191562473774 60.0
0 0.0961694791913033 61.0
183.69999789 0.647425472736359 0.0
183.69999789 1.54349648952484 1.0
182.50369561 1.83926427364349 2.0
177.71459037 6.79102325439453 3.0
0 4.51676368713379 4.0
0 3.39124464988708 5.0
-58.71932813 0.307300329208374 6.0
70.635094299 0.705095767974854 7.0
55.880749395 0.873346209526062 8.0
14.754344904 1.09871673583984 9.0
82.242630383 0.802451193332672 10.0
69.037509967 0.28563666343689 11.0
78.433628669 0.725159049034119 12.0
78.433628669 0.174852848052979 13.0
78.433628669 0.107119277119637 14.0
0 0.0823544859886169 15.0
0 0.130053341388702 16.0
0 1.34073102474213 17.0
58.328118355 1.46418571472168 18.0
-58.328118355 1.7772513628006 19.0
-58.328118355 0.277236610651016 20.0
-58.328118355 -0.019220806658268 21.0
79.451400155 2.33192539215088 22.0
146.03824648 2.19456481933594 23.0
112.16846093 0.490680158138275 24.0
21.627058863 0.14919239282608 25.0
225.48964663 1.25133347511292 26.0
225.48964663 1.46640706062317 27.0
24.299494677 1.83786523342133 28.0
249.78914131 0.641206085681915 29.0
249.78914131 1.57640206813812 30.0
46.881112859 0.954581618309021 31.0
9.4773378478 0.329154819250107 32.0
21.1232818 1.46562433242798 33.0
21.1232818 0.996239542961121 34.0
21.1232818 0.130567416548729 35.0
9.473716361 0.0274454038590193 36.0
8.2890415367 0.43854495882988 37.0
0 0.0169695168733597 38.0
0 0.250156939029694 39.0
-11.990210174 0.312401056289673 40.0
-25.418987505 0.662982761859894 41.0
4.8236077824 0.329709768295288 42.0
-32.565902332 0.377004146575928 43.0
90.211376344 0.683645308017731 44.0
90.211376344 0.0797935053706169 45.0
7.0988708479 0.478282988071442 46.0
2.8066024965 0.383196711540222 47.0
183.69999789 0.0455600507557392 48.0
177.71459037 0.283888936042786 49.0
0 0.191308408975601 50.0
70.635094299 0.179605394601822 51.0
82.242630383 0.472678989171982 52.0
78.433628669 0.0955191105604172 53.0
78.433628669 0.0446479916572571 54.0
78.433628669 0.0584967434406281 55.0
0 3.54086184501648 56.0
79.451400155 0.0711545944213867 57.0
225.48964663 0.36067146062851 58.0
249.78914131 0.787602543830872 59.0
21.1232818 0.0154610499739647 60.0
0 0.0920040383934975 61.0
151.79962168 0.881441831588745 0.0
151.79962168 1.3869594335556 1.0
150.56717209 1.72587788105011 2.0
145.77806688 4.92841529846191 3.0
0 4.05098438262939 4.0
0 3.91730713844299 5.0
-62.064986709 0.536410570144653 6.0
74.496285256 0.595803260803223 7.0
41.736262273 1.29367434978485 8.0
32.760022983 1.22640132904053 9.0
90.857234024 1.02489018440247 10.0
78.510104175 0.49253648519516 11.0
79.499510243 0.21582093834877 12.0
79.499510243 0.455286055803299 13.0
79.499510243 0.100075453519821 14.0
0 0.136718705296516 15.0
0 0.201147317886353 16.0
0 1.27059948444366 17.0
55.242450764 1.42101144790649 18.0
-55.242450764 1.22294282913208 19.0
-55.242450764 0.776023507118225 20.0
-55.242450764 -0.0445055402815342 21.0
76.278866658 2.15926432609558 22.0
136.17693125 1.40735340118408 23.0
102.53387984 0.798494577407837 24.0
21.308412306 0.0846073552966118 25.0
212.45579791 0.938416182994843 26.0
212.45579791 1.89283490180969 27.0
25.219481873 2.2628767490387 28.0
237.67527978 0.915550172328949 29.0
237.67527978 1.87185382843018 30.0
46.328888269 1.18300092220306 31.0
9.2348424502 0.615842938423157 32.0
21.036415893 1.56310904026031 33.0
21.036415893 0.941190898418427 34.0
21.036415893 0.127617508172989 35.0
9.4877161828 0.0331641063094139 36.0
8.3233468466 0.417451322078705 37.0
0 0.0227071046829224 38.0
0 0.230773404240608 39.0
-12.591419795 0.495222270488739 40.0
-24.437575144 0.858507990837097 41.0
5.0946089143 0.420621186494827 42.0
-32.131842941 0.561925649642944 43.0
99.319880989 0.803879797458649 44.0
99.319880989 0.0781847313046455 45.0
7.2662508562 0.682705760002136 46.0
2.2573494564 0.414355248212814 47.0
151.79962168 0.151280432939529 48.0
145.77806688 0.417180448770523 49.0
0 0.195320889353752 50.0
74.496285256 0.159462094306946 51.0
90.857234024 0.325343072414398 52.0
79.499510243 0.0420277565717697 53.0
79.499510243 0.0548484846949577 54.0
79.499510243 0.0897667557001114 55.0
0 3.84139323234558 56.0
76.278866658 0.0709204450249672 57.0
212.45579791 0.350229382514954 58.0
237.67527978 0.396919548511505 59.0
21.036415893 -0.0184634327888489 60.0
0 0.0997949540615082 61.0
127.40555257 1.03959047794342 0.0
127.40555257 0.672287166118622 1.0
126.30946788 0.848548829555511 2.0
121.52036259 3.7630398273468 3.0
0 3.62679743766785 4.0
0 2.33307886123657 5.0
-49.313863777 0.143884122371674 6.0
60.575699802 0.754775047302246 7.0
52.285796951 1.52048754692078 8.0
8.2899028507 1.46781349182129 9.0
90.202633047 1.18328297138214 10.0
77.395712645 0.257036834955215 11.0
88.63739864 0.115650489926338 12.0
88.63739864 0.16082452237606 13.0
88.63739864 0.0862445756793022 14.0
0 0.215907782316208 15.0
0 0.103031367063522 16.0
0 1.26938199996948 17.0
56.858113562 1.17099106311798 18.0
-56.858113562 1.07882618904114 19.0
-56.858113562 0.296036273241043 20.0
-56.858113562 0.0210992768406868 21.0
77.464204398 1.07312726974487 22.0
139.68570448 0.993299782276154 23.0
104.29201625 0.575437605381012 24.0
24.319437697 0.175244688987732 25.0
217.14990888 0.878221213817596 26.0
217.14990888 1.25378489494324 27.0
27.104942625 1.50313520431519 28.0
244.2548515 0.901268839836121 29.0
244.2548515 1.60313737392426 30.0
46.68738368 1.08830976486206 31.0
9.4136799298 0.295384168624878 32.0
20.606090836 1.60126805305481 33.0
20.606090836 0.869408428668976 34.0
20.606090836 0.120872974395752 35.0
8.7443964272 0.0262276139110327 36.0
7.5182266887 0.448368459939957 37.0
0 0.0329139232635498 38.0
0 0.17939230799675 39.0
-13.167038089 0.643928408622742 40.0
-27.110257345 0.557962119579315 41.0
5.6344423256 0.3823102414608 42.0
-35.572574766 0.35213440656662 43.0
83.021806802 0.915078282356262 44.0
83.021806802 0.0838768631219864 45.0
6.7606537416 0.696699857711792 46.0
2.2741268014 0.357460796833038 47.0
127.40555257 0.136998027563095 48.0
121.52036259 0.288351505994797 49.0
0 0.251211434602737 50.0
60.575699802 0.285615384578705 51.0
90.202633047 0.102919317781925 52.0
88.63739864 0.04295589402318 53.0
88.63739864 0.0563257150352001 54.0
88.63739864 0.0699488520622253 55.0
0 0.67897891998291 56.0
77.464204398 0.131865352392197 57.0
217.14990888 0.557361423969269 58.0
244.2548515 0.803959369659424 59.0
20.606090836 0.00877100601792336 60.0
0 0.096418708562851 61.0
184.42146557 1.14805066585541 0.0
184.42146557 0.617996513843536 1.0
182.98351994 0.678849816322327 2.0
178.19441473 3.12794637680054 3.0
0 8.11086463928223 4.0
0 4.66737270355225 5.0
-62.169110156 3.48130345344543 6.0
72.33796749 0.735122740268707 7.0
45.425535882 0.773694515228271 8.0
26.912431607 0.778886616230011 9.0
103.63425947 1.1788238286972 10.0
90.890451533 0.425048440694809 11.0
89.221507657 0.166792169213295 12.0
89.221507657 0.220411717891693 13.0
89.221507657 0.104717001318932 14.0
0 0.272728353738785 15.0
0 0.168405696749687 16.0
0 1.65068602561951 17.0
54.125850205 1.16539812088013 18.0
-54.125850205 1.3293628692627 19.0
-54.125850205 0.410729795694351 20.0
-54.125850205 -0.0606355965137482 21.0
76.169748967 1.21778535842896 22.0
149.33812674 0.759897947311401 23.0
114.93852075 0.694982886314392 24.0
21.949187294 0.192475333809853 25.0
225.5078757 0.912805616855621 26.0
225.5078757 1.80109417438507 27.0
24.955388812 1.77161502838135 28.0
250.46326451 0.469569653272629 29.0
250.46326451 1.81238269805908 30.0
47.322019181 0.985197603702545 31.0
9.4492968349 0.434333324432373 32.0
22.043898762 1.7170501947403 33.0
22.043898762 0.553196549415588 34.0
22.043898762 0.0754928812384605 35.0
10.006145241 0.0263432804495096 36.0
8.6465512668 0.36921888589859 37.0
0 0.0518544055521488 38.0
0 0.105813257396221 39.0
-10.336688113 1.00073969364166 40.0
-22.542414579 0.402252942323685 41.0
5.5890419901 0.606426954269409 42.0
-30.327133045 0.546151101589203 43.0
101.11046017 0.73082447052002 44.0
101.11046017 0.0505085811018944 45.0
7.4080975103 0.69850891828537 46.0
2.7125111043 0.41568660736084 47.0
184.42146557 0.149499416351318 48.0
178.19441473 0.054302055388689 49.0
0 0.627146542072296 50.0
72.33796749 0.225495770573616 51.0
103.63425947 0.333685338497162 52.0
89.221507657 0.0758415460586548 53.0
89.221507657 0.0847312808036804 54.0
89.221507657 0.098334863781929 55.0
0 1.1973557472229 56.0
76.169748967 0.206942185759544 57.0
225.5078757 0.676012337207794 58.0
250.46326451 0.834491908550262 59.0
22.043898762 -0.0012014415115118 60.0
0 0.0900094881653786 61.0
151.73637818 0.75657457113266 0.0
151.73637818 2.19406962394714 1.0
150.67984186 1.92336273193359 2.0
145.89075832 3.87366318702698 3.0
0 4.38742971420288 4.0
0 2.73166227340698 5.0
-51.489996521 0.916537225246429 6.0
62.421258879 0.538989126682281 7.0
51.03269579 1.10632193088531 8.0
11.388563089 1.06589388847351 9.0
80.968282411 0.939029037952423 10.0
68.765910513 0.222332388162613 11.0
79.376403315 0.604895114898682 12.0
79.376403315 0.574840545654297 13.0
79.376403315 0.101829573512077 14.0
0 0.246764421463013 15.0
0 0.148471862077713 16.0
0 1.87138438224792 17.0
55.543185432 1.06440949440002 18.0
-55.543185432 0.431217879056931 19.0
-55.543185432 1.10890817642212 20.0
-55.543185432 0.0321369431912899 21.0
78.503372183 1.31418478488922 22.0
152.00045802 1.06519782543182 23.0
117.01079404 0.492550551891327 24.0
20.487338159 0.215351015329361 25.0
230.5038302 0.975489675998688 26.0
230.5038302 1.00461304187775 27.0
25.684511029 0.912663698196411 28.0
256.18834123 0.532827198505402 29.0
256.18834123 1.3309725522995 30.0
46.782375871 1.02111148834229 31.0
9.3447725178 0.420618325471878 32.0
22.960186751 1.12976396083832 33.0
22.960186751 0.941573560237885 34.0
22.960186751 0.156416833400726 35.0
9.3696544112 0.0306444093585014 36.0
8.201695568 0.353472292423248 37.0
0 0.0101839862763882 38.0
0 0.217737376689911 39.0
-13.171011583 0.37894669175148 40.0
-25.573926873 0.718474626541138 41.0
4.8426914241 0.460261017084122 42.0
-33.072411011 0.369816482067108 43.0
82.814217302 0.716109216213226 44.0
82.814217302 0.0409986972808838 45.0
7.0374944446 0.689320683479309 46.0
2.6416836684 0.407725006341934 47.0
151.73637818 0.123038306832314 48.0
145.89075832 0.404251962900162 49.0
0 0.0625228732824326 50.0
62.421258879 0.124162055552006 51.0
80.968282411 0.321109771728516 52.0
79.376403315 0.0123137077316642 53.0
79.376403315 0.232045844197273 54.0
79.376403315 0.10615935921669 55.0
0 2.60315442085266 56.0
78.503372183 0.0337848663330078 57.0
230.5038302 0.360479980707169 58.0
256.18834123 0.412864446640015 59.0
22.960186751 -0.00857758335769176 60.0
0 0.113928698003292 61.0
149.81972226 0.908835530281067 0.0
149.81972226 0.75358510017395 1.0
148.39549992 0.908169746398926 2.0
143.60639482 3.1539204120636 3.0
0 6.39977645874023 4.0
0 3.84166646003723 5.0
-54.290978713 1.72582828998566 6.0
64.808131396 0.766392827033997 7.0
54.525918143 1.25066554546356 8.0
10.282213253 1.36220836639404 9.0
93.27880083 0.937231838703156 10.0
81.628383649 0.0519330687820911 11.0
82.695057484 0.654160141944885 12.0
82.695057484 0.164048552513123 13.0
82.695057484 0.064632810652256 14.0
0 0.0992672517895699 15.0
0 0.231763988733292 16.0
0 0.874421358108521 17.0
61.786055459 1.08030164241791 18.0
-61.786055459 1.17516398429871 19.0
-61.786055459 0.767922520637512 20.0
-61.786055459 -0.0475694686174393 21.0
81.960944692 1.39712691307068 22.0
134.13924812 0.697955310344696 23.0
101.06275776 0.765034377574921 24.0
21.368223938 0.0766186267137527 25.0
216.10019281 0.835239946842194 26.0
216.10019281 1.79768908023834 27.0
23.551427686 2.31998705863953 28.0
239.65162049 1.35133945941925 29.0
239.65162049 2.08186340332031 30.0
46.957048245 1.00153994560242 31.0
9.3980435551 0.45617401599884 32.0
20.174889233 1.6273558139801 33.0
20.174889233 0.869098484516144 34.0
20.174889233 0.129654824733734 35.0
9.190067186 0.0427811741828918 36.0
7.766965458 0.419292867183685 37.0
0 0.00995646789669991 38.0
0 0.200377598404884 39.0
-11.738851155 0.59545224905014 40.0
-25.579153749 0.775126338005066 41.0
5.4706232157 0.414217710494995 42.0
-33.726969433 0.463846415281296 43.0
104.48700111 0.937263965606689 44.0
104.48700111 0.071533590555191 45.0
6.7512769179 0.633075296878815 46.0
2.6964038617 0.319420218467712 47.0
149.81972226 0.07338847219944 48.0
143.60639482 0.219420030713081 49.0
0 0.198494926095009 50.0
64.808131396 0.311305046081543 51.0
93.27880083 0.0740509629249573 52.0
82.695057484 0.0727699100971222 53.0
82.695057484 0.0955030247569084 54.0
82.695057484 0.108730487525463 55.0
0 2.75188684463501 56.0
81.960944692 -0.0253024622797966 57.0
216.10019281 0.274095118045807 58.0
239.65162049 1.04515027999878 59.0
20.174889233 0.0147373415529728 60.0
0 0.0976214408874512 61.0
203.3595739 0.661561191082001 0.0
203.3595739 0.627890110015869 1.0
202.10663442 1.02329802513123 2.0
197.3175291 2.92323613166809 3.0
0 3.45544862747192 4.0
0 3.17867922782898 5.0
-59.217010559 0.255103409290314 6.0
71.043665267 2.02553868293762 7.0
60.652264036 0.19012813270092 8.0
10.391401231 0.353964388370514 9.0
81.785202608 0.884556651115417 10.0
67.989127024 0.3750059902668 11.0
81.680684147 0.740128457546234 12.0
81.680684147 0.306830704212189 13.0
81.680684147 0.0524300411343575 14.0
0 0.109202764928341 15.0
0 0.198155164718628 16.0
0 1.22176921367645 17.0
57.372618451 1.90405094623566 18.0
-57.372618451 2.40807771682739 19.0
-57.372618451 0.302353829145432 20.0
-57.372618451 0.055483303964138 21.0
79.006527069 2.0256712436676 22.0
137.22980194 1.56301033496857 23.0
107.96232972 0.610239624977112 24.0
20.163624296 0.153120890259743 25.0
216.23632901 0.420585542917252 26.0
216.23632901 1.12114989757538 27.0
26.065521282 0.912157893180847 28.0
242.30185029 0.610344707965851 29.0
242.30185029 1.49516677856445 30.0
46.119012747 0.951353967189789 31.0
8.3967777265 0.483103185892105 32.0
21.633908618 1.5645124912262 33.0
21.633908618 0.900327801704407 34.0
21.633908618 0.0752088576555252 35.0
8.7482758976 0.0210397057235241 36.0
7.5417734463 0.453021228313446 37.0
0 0.0544888414442539 38.0
0 0.0371595919132233 39.0
-12.767343304 0.491342335939407 40.0
-25.059438436 0.428637057542801 41.0
4.5756794272 0.396097719669342 42.0
-31.91031445 0.714596748352051 43.0
74.490981606 0.571499407291412 44.0
74.490981606 0.143392115831375 45.0
6.3603475137 0.598351657390594 46.0
2.4934230104 0.430517673492432 47.0
203.3595739 0.0582809299230576 48.0
197.3175291 0.190684571862221 49.0
0 0.0737137496471405 50.0
71.043665267 0.223018020391464 51.0
81.785202608 0.544955968856812 52.0
81.680684147 0.0865272730588913 53.0
81.680684147 0.1270682066679 54.0
81.680684147 0.19573187828064 55.0
0 4.44552373886108 56.0
79.006527069 0.134619399905205 57.0
216.23632901 0.480875432491302 58.0
242.30185029 0.796063601970673 59.0
21.633908618 0.00364955142140388 60.0
0 0.0727926567196846 61.0
186.20606677 0.863807559013367 0.0
186.20606677 1.52864420413971 1.0
184.70264457 1.14666187763214 2.0
179.91353936 3.2813675403595 3.0
0 8.3846263885498 4.0
0 4.87734079360962 5.0
-61.29463067 2.86050939559937 6.0
71.22068778 0.713257014751434 7.0
47.391736187 0.77756941318512 8.0
23.828951593 0.727648317813873 9.0
93.379051852 0.992511093616486 10.0
79.958090473 0.142838969826698 11.0
86.05926258 1.09023928642273 12.0
86.05926258 0.737542629241943 13.0
86.05926258 0.0796494856476784 14.0
0 0.188175022602081 15.0
0 0.221425458788872 16.0
0 1.35733497142792 17.0
57.790502128 1.90142166614532 18.0
-57.790502128 2.13822793960571 19.0
-57.790502128 0.215586751699448 20.0
-57.790502128 0.114363670349121 21.0
80.2416614 2.63528442382812 22.0
136.07764422 2.53782224655151 23.0
102.37895569 0.347064137458801 24.0
21.970825357 0.0826581418514252 25.0
216.31930562 1.15169262886047 26.0
216.31930562 1.64961171150208 27.0
27.262638029 1.7831164598465 28.0
243.58194364 1.08207285404205 29.0
243.58194364 1.81158423423767 30.0
46.533226913 1.02805137634277 31.0
9.3243278685 0.405088543891907 32.0
22.451159272 1.34683918952942 33.0
22.451159272 1.07160305976868 34.0
22.451159272 0.121059887111187 35.0
9.8288884689 0.0346276387572289 36.0
8.5508723738 0.332813203334808 37.0
0 0.000553213059902191 38.0
0 0.101072952151299 39.0
-12.899545223 0.734219372272491 40.0
-26.755286668 0.519943177700043 41.0
5.0389987341 0.519778788089752 42.0
-34.669815372 0.744937896728516 43.0
101.44152999 0.852920293807983 44.0
101.44152999 0.10246055573225 45.0
7.2074835101 0.700040936470032 46.0
2.3905661894 0.442150413990021 47.0
186.20606677 0.122329816222191 48.0
179.91353936 -0.00406454037874937 49.0
0 0.287690043449402 50.0
71.22068778 0.165995866060257 51.0
93.379051852 0.180582642555237 52.0
86.05926258 0.0913067907094955 53.0
86.05926258 0.22039470076561 54.0
86.05926258 0.147693410515785 55.0
0 2.78215527534485 56.0
80.2416614 0.0820717662572861 57.0
216.31930562 0.328149080276489 58.0
243.58194364 0.571888864040375 59.0
22.451159272 -0.00738109927624464 60.0
0 0.0933584719896317 61.0
172.00374129 0.997789323329926 0.0
172.00374129 1.44459545612335 1.0
170.85613742 1.20087265968323 2.0
166.06703228 5.77778482437134 3.0
0 4.7475733757019 4.0
0 3.54012989997864 5.0
-62.660349977 0.268980443477631 6.0
73.167637269 1.23788475990295 7.0
58.109395013 0.495616316795349 8.0
15.058242257 0.41725105047226 9.0
75.747114997 1.07800126075745 10.0
63.803469301 0.0690115168690681 11.0
77.43966839 0.801568746566772 12.0
77.43966839 0.556322574615479 13.0
77.43966839 0.110556751489639 14.0
0 0.0598020553588867 15.0
0 0.166769981384277 16.0
0 1.47458052635193 17.0
58.741888394 1.22447645664215 18.0
-58.741888394 1.55767393112183 19.0
-58.741888394 0.261794984340668 20.0
-58.741888394 0.0539136230945587 21.0
80.178367188 1.63345515727997 22.0
139.72954853 1.5466126203537 23.0
109.94047388 0.308950483798981 24.0
22.184616559 0.234225898981094 25.0
219.90791572 1.00292098522186 26.0
219.90791572 1.63749921321869 27.0
28.621656236 0.953915178775787 28.0
248.52957196 0.395615071058273 29.0
248.52957196 1.37998604774475 30.0
46.557274852 1.20419359207153 31.0
9.3415776271 0.443885385990143 32.0
21.436478794 1.37835395336151 33.0
21.436478794 0.872047543525696 34.0
21.436478794 0.115705877542496 35.0
8.4988522236 0.0206800326704979 36.0
7.1278601339 0.356602877378464 37.0
0 0.0107512287795544 38.0
0 0.127119541168213 39.0
-11.158128043 0.605096817016602 40.0
-24.809443943 0.520123183727264 41.0
4.9655274025 0.342493832111359 42.0
-32.424563914 0.285915970802307 43.0
87.724771361 0.677142083644867 44.0
87.724771361 0.0890596136450768 45.0
7.0246547395 0.548772215843201 46.0
2.2519532749 0.33171558380127 47.0
172.00374129 0.136821582913399 48.0
166.06703228 0.0775111690163612 49.0
0 0.0566046684980392 50.0
73.167637269 0.179641798138618 51.0
75.747114997 0.183203518390656 52.0
77.43966839 0.0287295691668987 53.0
77.43966839 0.0784789770841599 54.0
77.43966839 0.127935469150543 55.0
0 3.27378511428833 56.0
80.178367188 0.143868252635002 57.0
219.90791572 0.443676263093948 58.0
248.52957196 0.0424135439097881 59.0
21.436478794 0.0250845514237881 60.0
0 0.0985542163252831 61.0
181.302314 0.833515584468842 0.0
181.302314 1.50550282001495 1.0
180.08657547 2.03270292282104 2.0
174.62192647 4.63693618774414 3.0
6.5083582603 4.73330116271973 4.0
6.5083582603 5.45924949645996 5.0
-67.475651308 3.33813643455505 6.0
79.526526068 0.402396023273468 7.0
39.584618087 0.781837582588196 8.0
39.941907981 0.838179588317871 9.0
90.135351137 0.926033794879913 10.0
77.880603513 0.174737170338631 11.0
79.103666896 0.63693505525589 12.0
79.103666896 0.490253895521164 13.0
79.103666896 0.0463378131389618 14.0
0 0.0510876998305321 15.0
0 0.20376493036747 16.0
0 1.16893601417542 17.0
54.642769028 0.812692105770111 18.0
-54.642769028 0.627020180225372 19.0
-54.642769028 0.655944645404816 20.0
-54.642769028 -0.0177853479981422 21.0
76.903514751 1.61705076694489 22.0
150.45256946 0.982254207134247 23.0
114.56233216 0.689346313476562 24.0
21.556113862 0.0914801359176636 25.0
227.35608421 1.30149030685425 26.0
227.35608421 1.86339151859283 27.0
27.985636393 1.36701190471649 28.0
255.3417206 1.2136834859848 29.0
255.3417206 1.86879599094391 30.0
46.690545386 1.1870756149292 31.0
9.4158934537 0.363401651382446 32.0
22.260745723 1.53884732723236 33.0
22.260745723 0.732035338878632 34.0
22.260745723 0.170481756329536 35.0
9.7951161783 0.0358495488762856 36.0
8.5739732014 0.307302981615067 37.0
0 0.0127916261553764 38.0
0 0.152344852685928 39.0
-11.338411612 0.466947823762894 40.0
-24.561154909 0.477112293243408 41.0
5.6037152639 0.529812097549438 42.0
-32.827121476 0.560521900653839 43.0
89.847490679 0.727598547935486 44.0
89.847490679 0.0484858304262161 45.0
7.0703280326 0.60611480474472 46.0
2.6981480744 0.436074435710907 47.0
181.302314 0.104998409748077 48.0
174.62192647 0.268339693546295 49.0
6.5083582603 0.291932344436646 50.0
79.526526068 0.140577986836433 51.0
90.135351137 0.188234955072403 52.0
79.103666896 0.04588383436203 53.0
79.103666896 0.0376271568238735 54.0
79.103666896 0.198984757065773 55.0
0 3.19006776809692 56.0
76.903514751 0.127181977033615 57.0
227.35608421 0.542401432991028 58.0
255.3417206 1.000159740448 59.0
22.260745723 -0.00658087991178036 60.0
0 0.123046346008778 61.0
204.53943952 0.79002970457077 0.0
204.53943952 1.80091488361359 1.0
203.12532372 2.21645021438599 2.0
198.33621843 7.07941055297852 3.0
0 5.09593343734741 4.0
0 4.85364484786987 5.0
-48.650990499 2.27145004272461 6.0
60.299239774 0.808810710906982 7.0
62.658836069 0.646297812461853 8.0
-2.3595962951 0.821400761604309 9.0
79.605935167 0.949191391468048 10.0
66.40737184 0.454322874546051 11.0
76.713547747 0.180829584598541 12.0
76.713547747 0.786666870117188 13.0
76.713547747 0.127151846885681 14.0
0 0.132469266653061 15.0
0 0.177385315299034 16.0
0 1.03650879859924 17.0
56.137019797 1.111781001091 18.0
-56.137019797 1.30968737602234 19.0
-56.137019797 0.522931575775146 20.0
-56.137019797 0.131433174014091 21.0
77.611644005 2.11284422874451 22.0
140.53864441 1.57815861701965 23.0
110.08318468 0.28193211555481 24.0
22.205371306 0.264388382434845 25.0
218.15028841 1.78902232646942 26.0
218.15028841 2.58684587478638 27.0
25.734175303 2.78969669342041 28.0
243.88446372 1.66500663757324 29.0
243.88446372 2.20021915435791 30.0
46.190995783 1.24039232730865 31.0
9.3364113865 0.381989032030106 32.0
21.474624207 1.32819104194641 33.0
21.474624207 0.626841723918915 34.0
21.474624207 0.105891942977905 35.0
9.779637669 0.0297880079597235 36.0
8.5945666072 0.304573267698288 37.0
0 0.0472092404961586 38.0
0 0.235967114567757 39.0
-10.455624466 0.502240478992462 40.0
-24.393299752 0.621914386749268 41.0
4.5624835522 0.441123157739639 42.0
-31.571123741 0.757857859134674 43.0
86.95681894 0.562341392040253 44.0
86.95681894 0.119495376944542 45.0
6.8174027992 0.624276101589203 46.0
2.633622076 0.362769782543182 47.0
204.53943952 0.242296636104584 48.0
198.33621843 0.146891102194786 49.0
0 0.63401848077774 50.0
60.299239774 0.165268763899803 51.0
79.605935167 0.693995594978333 52.0
76.713547747 0.0562111549079418 53.0
76.713547747 0.180631518363953 54.0
76.713547747 0.147585287690163 55.0
0 7.62070512771606 56.0
77.611644005 -0.0483868233859539 57.0
218.15028841 0.66582977771759 58.0
243.88446372 0.926715731620789 59.0
21.474624207 -0.00905560329556465 60.0
0 0.105850078165531 61.0
148.67433049 1.05404901504517 0.0
148.67433049 2.49666404724121 1.0
147.72219808 2.05078935623169 2.0
142.93312812 4.78031015396118 3.0
0 5.44116687774658 4.0
0 2.65175342559814 5.0
-60.643017938 0.240596503019333 6.0
71.039708714 0.777727425098419 7.0
46.019361323 0.673011839389801 8.0
25.020347391 0.5697962641716 9.0
85.127329202 1.14075303077698 10.0
72.096669128 0.180843025445938 11.0
78.985289965 0.362368255853653 12.0
78.985289965 0.960625886917114 13.0
78.985289965 0.00420629139989614 14.0
0 0.169412925839424 15.0
0 0.202449962496758 16.0
0 1.15258204936981 17.0
61.658722841 1.35523915290833 18.0
-61.658722841 0.922788679599762 19.0
-61.658722841 0.690721094608307 20.0
-61.658722841 -0.0122111923992634 21.0
83.726702352 1.63686573505402 22.0
151.12445442 1.66584229469299 23.0
118.6635481 0.26314640045166 24.0
22.844869096 0.392409712076187 25.0
234.85115677 1.49037742614746 26.0
234.85115677 1.84610366821289 27.0
26.827859715 0.993287742137909 28.0
261.67901649 0.710776388645172 29.0
261.67901649 1.41305661201477 30.0
46.876475912 1.07864832878113 31.0
9.4593555542 0.542729616165161 32.0
22.067979511 1.35174345970154 33.0
22.067979511 1.04843509197235 34.0
22.067979511 0.153292283415794 35.0
10.748937108 0.031190212816 36.0
9.4218556759 0.368318378925323 37.0
0 0.0627446845173836 38.0
0 0.143823653459549 39.0
-12.975326135 0.886081457138062 40.0
-25.574711791 0.488237679004669 41.0
4.7392420114 0.482770353555679 42.0
-33.019618988 0.395861119031906 43.0
94.398730374 0.712561070919037 44.0
94.398730374 0.0854862853884697 45.0
7.6212665598 0.540560483932495 46.0
2.7221356426 0.360897183418274 47.0
148.67433049 0.141494259238243 48.0
142.93312812 0.154864311218262 49.0
0 0.19404599070549 50.0
71.039708714 0.202612146735191 51.0
85.127329202 0.290585219860077 52.0
78.985289965 0.0471445322036743 53.0
78.985289965 0.0831735655665398 54.0
78.985289965 0.503648638725281 55.0
0 2.61266469955444 56.0
83.726702352 0.009353157132864 57.0
234.85115677 0.598999798297882 58.0
261.67901649 0.465281933546066 59.0
22.067979511 0.00383491441607475 60.0
0 0.108525395393372 61.0
191.55893169 1.0183230638504 0.0
191.55893169 1.38020241260529 1.0
190.01586671 1.48048627376556 2.0
185.2267616 5.0140552520752 3.0
0 6.50680017471313 4.0
0 4.53895950317383 5.0
-62.290749844 1.09841215610504 6.0
75.245656137 0.577126324176788 7.0
45.101350688 0.934569418430328 8.0
30.144305449 0.856066703796387 9.0
76.420565451 1.19837737083435 10.0
62.832444367 0.089250274002552 11.0
80.461885333 1.1202974319458 12.0
80.461885333 0.329821020364761 13.0
80.461885333 0.108954153954983 14.0
0 0.0547990724444389 15.0
0 0.170679658651352 16.0
0 1.46269690990448 17.0
62.372702393 1.17014515399933 18.0
-62.372702393 1.45727872848511 19.0
-62.372702393 0.314326465129852 20.0
-62.372702393 0.0281310491263866 21.0
83.07586285 2.25480914115906 22.0
135.72078864 1.48534870147705 23.0
98.843726704 0.674948513507843 24.0
21.411416353 0.104236721992493 25.0
218.79665149 1.25666797161102 26.0
218.79665149 1.43711829185486 27.0
23.930926837 1.40404558181763 28.0
242.72757832 0.755476534366608 29.0
242.72757832 1.741863489151 30.0
46.815910316 1.11089611053467 31.0
9.4921137495 0.665644824504852 32.0
20.703160457 1.27481746673584 33.0
20.703160457 0.799411356449127 34.0
20.703160457 0.108119077980518 35.0
9.4322822137 0.048400767147541 36.0
8.2002324307 0.360588163137436 37.0
0 0.0446849055588245 38.0
0 0.208439230918884 39.0
-11.139319899 0.741276681423187 40.0
-25.811904333 0.729300975799561 41.0
4.5975385326 0.470784902572632 42.0
-32.911595073 0.673134803771973 43.0
93.344859564 0.638388991355896 44.0
93.344859564 0.0872943699359894 45.0
7.4230060963 0.717321693897247 46.0
2.8377285462 0.422391593456268 47.0
191.55893169 0.136091664433479 48.0
185.2267616 0.0749615058302879 49.0
0 0.196761175990105 50.0
75.245656137 0.132264479994774 51.0
76.420565451 0.088901475071907 52.0
80.461885333 0.0798489972949028 53.0
80.461885333 0.126267611980438 54.0
80.461885333 0.152871042490005 55.0
0 4.1867208480835 56.0
83.07586285 -0.0298535376787186 57.0
218.79665149 0.402811765670776 58.0
242.72757832 0.737087607383728 59.0
20.703160457 -0.00319319218397141 60.0
0 0.0806223824620247 61.0
135.70888622 0.729627668857574 0.0
135.70888622 1.17632949352264 1.0
134.44983889 1.35088074207306 2.0
129.66075535 3.14471554756165 3.0
0 7.47465324401855 4.0
0 6.2383394241333 5.0
-70.634236551 0.545919060707092 6.0
79.304806204 0.779691934585571 7.0
46.408011469 0.491999715566635 8.0
32.896794735 0.586435914039612 9.0
79.490052575 0.809770405292511 10.0
66.231611869 0.14532881975174 11.0
78.079073572 0.304754436016083 12.0
78.079073572 0.600041091442108 13.0
78.079073572 0.0896952152252197 14.0
0 0.130150929093361 15.0
0 0.237868696451187 16.0
0 0.541775345802307 17.0
55.546925223 1.1386045217514 18.0
-55.546925223 1.65504717826843 19.0
-55.546925223 0.277338534593582 20.0
-55.546925223 -0.0275987833738327 21.0
77.240024568 2.63867783546448 22.0
136.86544111 3.07428097724915 23.0
105.38828454 0.338818430900574 24.0
24.102494124 0.0919848084449768 25.0
214.10546568 2.48663902282715 26.0
214.10546568 3.13773918151855 27.0
26.294190631 1.74291169643402 28.0
240.39965631 0.404742300510406 29.0
240.39965631 1.35432577133179 30.0
46.30875454 1.0578430891037 31.0
9.3525269954 0.369037389755249 32.0
21.693099345 2.05480027198792 33.0
21.693099345 1.16011357307434 34.0
21.693099345 0.116092830896378 35.0
10.301231255 0.0232915710657835 36.0
9.1342123587 0.433488875627518 37.0
0 0.0331448093056679 38.0
0 0.197032302618027 39.0
-13.294814064 0.698796451091766 40.0
-25.231141791 0.654915153980255 41.0
5.0763010456 0.436306446790695 42.0
-32.742052832 0.385685503482819 43.0
92.194585314 0.693511307239532 44.0
92.194585314 0.0718592777848244 45.0
6.7495204835 0.553203582763672 46.0
2.2571891683 0.326462954282761 47.0
135.70888622 0.178078383207321 48.0
129.66075535 0.093399852514267 49.0
0 0.80971360206604 50.0
79.304806204 0.0604049377143383 51.0
79.490052575 0.134379893541336 52.0
78.079073572 0.00987581629306078 53.0
78.079073572 0.107149809598923 54.0
78.079073572 0.0458491407334805 55.0
0 6.52532052993774 56.0
77.240024568 0.152079328894615 57.0
214.10546568 1.45044648647308 58.0
240.39965631 0.107486210763454 59.0
21.693099345 0.0161094069480896 60.0
0 0.0917760729789734 61.0
184.34293034 0.835293829441071 0.0
184.34293034 1.3816534280777 1.0
182.9931204 2.19313430786133 2.0
178.83280448 5.51688241958618 3.0
0 5.8948974609375 4.0
0 3.79289436340332 5.0
-72.337028835 1.30793499946594 6.0
85.052757403 1.29097270965576 7.0
42.319260119 0.433859884738922 8.0
42.733497284 0.222032248973846 9.0
91.021467009 0.990351259708405 10.0
77.217220719 0.291279315948486 11.0
83.375961891 1.00617289543152 12.0
83.375961891 0.109051570296288 13.0
83.375961891 0.101895079016685 14.0
7.1634779453 0.240224540233612 15.0
7.1634779453 0.170647352933884 16.0
0 1.40224182605743 17.0
56.131026432 1.70214700698853 18.0
-56.131026432 1.61057472229004 19.0
-56.131026432 0.241687878966331 20.0
-56.131026432 0.150229603052139 21.0
77.361127055 2.09590363502502 22.0
135.62130788 1.21566128730774 23.0
102.19947403 0.72141969203949 24.0
22.981811473 0.121963083744049 25.0
212.98243493 1.70828521251678 26.0
212.98243493 1.82854855060577 27.0
30.784862302 1.71116280555725 28.0
243.76729723 0.743607878684998 29.0
243.76729723 1.58192026615143 30.0
46.186692242 1.23797607421875 31.0
9.2965712672 0.324886649847031 32.0
21.230100623 1.21028673648834 33.0
21.230100623 0.817932605743408 34.0
21.230100623 0.111769966781139 35.0
9.5742508816 0.0367371700704098 36.0
8.2387064005 0.318171679973602 37.0
0 0.0154010243713856 38.0
0 0.152519315481186 39.0
-11.741972832 0.422162055969238 40.0
-25.493723728 0.784716010093689 41.0
5.6841020186 0.34728330373764 42.0
-33.687320149 0.594485342502594 43.0
100.0664493 0.973408579826355 44.0
100.0664493 0.0303746908903122 45.0
6.5348317228 0.702964186668396 46.0
2.4469504849 0.326339542865753 47.0
184.34293034 0.154551297426224 48.0
178.83280448 0.286326795816422 49.0
0 0.352971434593201 50.0
85.052757403 0.134607017040253 51.0
91.021467009 0.316666126251221 52.0
83.375961891 0.129091739654541 53.0
83.375961891 0.0902285203337669 54.0
83.375961891 0.0705169066786766 55.0
0 0.90702360868454 56.0
77.361127055 0.117038160562515 57.0
212.98243493 0.471645385026932 58.0
243.76729723 0.203301697969437 59.0
21.230100623 0.0189357753843069 60.0
0 0.0967977195978165 61.0
162.82386144 0.964021742343903 0.0
162.82386144 0.465629518032074 1.0
161.63488927 1.08000123500824 2.0
156.84578416 4.66072845458984 3.0
0 7.29994297027588 4.0
0 6.31623983383179 5.0
-75.22300543 1.9389636516571 6.0
85.183309764 0.436122953891754 7.0
57.602983797 0.972789168357849 8.0
27.580325967 1.07456588745117 9.0
86.663095648 1.05071687698364 10.0
73.124377172 0.164054974913597 11.0
82.842950995 0.216722100973129 12.0
82.842950995 0.0952488332986832 13.0
82.842950995 0.0999132245779037 14.0
0 0.254843801259995 15.0
0 0.172629296779633 16.0
0 1.64671993255615 17.0
57.539140466 0.766521811485291 18.0
-57.539140466 0.671385705471039 19.0
-57.539140466 0.606646955013275 20.0
-57.539140466 -0.0805596709251404 21.0
78.952745142 2.57320284843445 22.0
137.78692456 2.46291732788086 23.0
103.18617834 0.345612347126007 24.0
22.072329416 0.0836125090718269 25.0
216.7396697 1.71310746669769 26.0
216.7396697 2.51360368728638 27.0
29.326626146 1.74921107292175 28.0
246.06629585 0.445982486009598 29.0
246.06629585 1.66203594207764 30.0
46.309473232 0.955744683742523 31.0
9.331786076 0.419274210929871 32.0
21.413604676 1.75737309455872 33.0
21.413604676 0.597349047660828 34.0
21.413604676 0.1092898696661 35.0
9.5121975452 0.0204330962151289 36.0
8.1237078905 0.43316912651062 37.0
0 0.0160177648067474 38.0
0 0.147933274507523 39.0
-11.720556659 0.68902987241745 40.0
-24.15111285 0.260666161775589 41.0
4.89115332 0.569303154945374 42.0
-31.408335991 0.543049037456512 43.0
100.42087403 0.625022351741791 44.0
100.42087403 0.12546481192112 45.0
7.7738787363 0.486242800951004 46.0
2.8075368102 0.318066537380219 47.0
162.82386144 0.0858632326126099 48.0
156.84578416 0.174786508083344 49.0
0 1.07396292686462 50.0
85.183309764 0.214637622237206 51.0
86.663095648 0.362557381391525 52.0
82.842950995 0.0139682060107589 53.0
82.842950995 0.0986248999834061 54.0
82.842950995 0.188356637954712 55.0
0 2.52265286445618 56.0
78.952745142 0.137218147516251 57.0
216.7396697 0.836446225643158 58.0
246.06629585 0.606201171875 59.0
21.413604676 0.0139043089002371 60.0
0 0.111215084791183 61.0
157.96899011 0.820196032524109 0.0
157.96899011 1.43407893180847 1.0
156.65999482 1.58785009384155 2.0
151.19532424 4.8294472694397 3.0
4.7454631868 3.78655171394348 4.0
4.7454631868 5.65066146850586 5.0
-56.029227429 0.500391066074371 6.0
68.902965313 0.61214280128479 7.0
41.99931586 0.744743764400482 8.0
26.903649453 0.973896145820618 9.0
98.527325498 0.978650867938995 10.0
85.4857284 0.271908491849899 11.0
87.12714854 0.285794019699097 12.0
87.12714854 0.194214016199112 13.0
87.12714854 0.0993937253952026 14.0
0 0.20540626347065 15.0
0 0.17027622461319 16.0
0 1.67292451858521 17.0
55.819922698 1.11876344680786 18.0
-55.819922698 1.10493612289429 19.0
-55.819922698 0.511387407779694 20.0
-55.819922698 0.013502050191164 21.0
78.935808213 1.33907341957092 22.0
151.70405717 1.43529224395752 23.0
116.59382794 0.564263224601746 24.0
21.704369868 0.14391303062439 25.0
230.63986538 0.947758555412292 26.0
230.63986538 1.57628071308136 27.0
24.802627514 1.97058200836182 28.0
255.4424929 0.976764678955078 29.0
255.4424929 1.78877902030945 30.0
46.397079337 1.24479711055756 31.0
9.2643410562 0.406666427850723 32.0
23.115885515 1.09659039974213 33.0
23.115885515 1.01316571235657 34.0
23.115885515 0.139714553952217 35.0
10.061755047 0.029011357575655 36.0
8.7787706235 0.355044156312943 37.0
0 0.00177901238203049 38.0
0 0.18155625462532 39.0
-12.132001931 0.79293817281723 40.0
-24.446655188 0.408768832683563 41.0
5.5677852315 0.5839022397995 42.0
-32.420222388 0.659866690635681 43.0
96.009644132 0.647221207618713 44.0
96.009644132 0.0852943807840347 45.0
7.5031986733 0.536106526851654 46.0
2.5826956076 0.402039051055908 47.0
157.96899011 0.147428274154663 48.0
151.19532424 0.354384422302246 49.0
4.7454631868 0.35408827662468 50.0
68.902965313 0.428489863872528 51.0
98.527325498 0.168934375047684 52.0
87.12714854 0.0778764635324478 53.0
87.12714854 0.0497020073235035 54.0
87.12714854 0.0502081625163555 55.0
0 1.88350701332092 56.0
78.935808213 0.132851034402847 57.0
230.63986538 0.248433247208595 58.0
255.4424929 0.348804801702499 59.0
23.115885515 0.00775973312556744 60.0
0 0.0957470461726189 61.0
172.01848265 1.16889631748199 0.0
172.01848265 1.2732218503952 1.0
170.94936247 1.57892560958862 2.0
166.16029252 4.09493827819824 3.0
0 7.42182445526123 4.0
0 3.65495824813843 5.0
-55.932382015 0.339504301548004 6.0
68.445792062 0.792811274528503 7.0
60.247371391 0.943045139312744 8.0
8.1984206704 0.683180153369904 9.0
94.397972105 1.16001653671265 10.0
82.308782402 0.119986668229103 11.0
87.192292108 0.145962536334991 12.0
87.192292108 0.214174628257751 13.0
87.192292108 0.0848724395036697 14.0
0 0.181765139102936 15.0
0 0.165413558483124 16.0
0 1.50388753414154 17.0
56.220490876 1.46762633323669 18.0
-56.220490876 1.35589528083801 19.0
-56.220490876 0.21173070371151 20.0
-56.220490876 0.0565532892942429 21.0
77.547264993 1.69617676734924 22.0
148.13848508 1.65253698825836 23.0
110.68382898 0.481226921081543 24.0
24.929814723 0.107732884585857 25.0
225.68575007 1.54090619087219 26.0
225.68575007 2.1347451210022 27.0
29.704647869 1.91328454017639 28.0
255.39039794 0.947652876377106 29.0
255.39039794 1.62106561660767 30.0
47.313575479 1.03852915763855 31.0
9.4541713132 0.376030087471008 32.0
21.326774116 1.22916340827942 33.0
21.326774116 0.962195515632629 34.0
21.326774116 0.114663571119308 35.0
9.4982753121 0.0341967716813087 36.0
8.2696446633 0.371487557888031 37.0
0 0.0460925325751305 38.0
0 0.215384259819984 39.0
-11.536979082 0.871450066566467 40.0
-24.824071206 0.431968361139297 41.0
4.6564244124 0.483928084373474 42.0
-32.233397556 0.520629525184631 43.0
95.676162732 0.772492110729218 44.0
95.676162732 0.0705416277050972 45.0
7.4503119902 0.631439089775085 46.0
2.8896631723 0.443556576967239 47.0
172.01848265 0.112469851970673 48.0
166.16029252 0.316591918468475 49.0
0 0.290788233280182 50.0
68.445792062 0.295713067054749 51.0
94.397972105 0.170990526676178 52.0
87.192292108 0.0435661524534225 53.0
87.192292108 0.0607370771467686 54.0
87.192292108 0.0323718562722206 55.0
0 1.06933498382568 56.0
77.547264993 0.0650545656681061 57.0
225.68575007 0.682441771030426 58.0
255.39039794 0.644656300544739 59.0
21.326774116 -0.000498216599225998 60.0
0 0.103245034813881 61.0
182.86272095 1.09740340709686 0.0
182.86272095 1.42683374881744 1.0
181.79245486 1.37997961044312 2.0
177.00334978 4.31452941894531 3.0
0 6.3350076675415 4.0
0 4.00117683410645 5.0
-61.570219217 0.0312931835651398 6.0
71.284587301 0.923808336257935 7.0
51.207160983 1.11060500144958 8.0
20.077426318 1.19004893302917 9.0
79.086660069 1.14669632911682 10.0
65.830243024 0.0594440549612045 11.0
74.673043219 0.124007612466812 12.0
74.673043219 0.759714186191559 13.0
74.673043219 0.0831100046634674 14.0
0 0.169471979141235 15.0
0 0.212058827280998 16.0
0 1.28077399730682 17.0
53.552228785 1.26741302013397 18.0
-53.552228785 1.45203542709351 19.0
-53.552228785 0.611747145652771 20.0
-53.552228785 0.184119373559952 21.0
75.319837657 1.01877391338348 22.0
143.8383462 0.467838823795319 23.0
104.2373445 0.50311005115509 24.0
24.797805394 0.296440660953522 25.0
219.15818386 1.15176570415497 26.0
219.15818386 2.03351783752441 27.0
30.307504045 1.55748760700226 28.0
249.46568791 0.648409307003021 29.0
249.46568791 1.97255599498749 30.0
47.442756021 1.27880561351776 31.0
9.3904185428 0.41264933347702 32.0
21.767608872 0.699085891246796 33.0
21.767608872 1.66285598278046 34.0
21.767608872 0.0771478340029716 35.0
10.099633403 0.025148369371891 36.0
8.8187366803 0.422091245651245 37.0
0 0.00911704450845718 38.0
0 0.267116039991379 39.0
-11.774814379 0.811511635780334 40.0
-23.554784409 0.554696619510651 41.0
4.850828772 0.576973497867584 42.0
-30.665060053 0.683244824409485 43.0
104.96123248 0.711892604827881 44.0
104.96123248 0.123696990311146 45.0
7.5616510973 0.630769670009613 46.0
2.8062337041 0.444521456956863 47.0
182.86272095 0.131589621305466 48.0
177.00334978 0.13471631705761 49.0
0 0.364854156970978 50.0
71.284587301 0.470559418201447 51.0
79.086660069 0.158966362476349 52.0
74.673043219 0.00471965922042727 53.0
74.673043219 0.00751468539237976 54.0
74.673043219 0.420532733201981 55.0
0 5.35207986831665 56.0
75.319837657 -0.0456258505582809 57.0
219.15818386 0.906422853469849 58.0
249.46568791 0.291285127401352 59.0
21.767608872 -0.0228727534413338 60.0
0 0.103310041129589 61.0
185.18489662 0.649707436561584 0.0
185.18489662 0.867270529270172 1.0
183.86334408 1.01924359798431 2.0
179.0742389 4.43315839767456 3.0
0 7.09371995925903 4.0
0 3.85784506797791 5.0
-71.354809429 0.630432724952698 6.0
80.799959895 0.616805374622345 7.0
54.540564923 0.823320686817169 8.0
26.259394972 0.961453080177307 9.0
88.265505966 0.861414611339569 10.0
74.667372948 0.482116371393204 11.0
81.069207392 0.597342312335968 12.0
81.069207392 0.144559115171432 13.0
81.069207392 0.0802957713603973 14.0
0 0.20500785112381 15.0
0 0.157112747430801 16.0
0 1.23259544372559 17.0
57.999549211 0.980306565761566 18.0
-57.999549211 0.782490134239197 19.0
-57.999549211 0.65748792886734 20.0
-57.999549211 -0.0366817526519299 21.0
79.415744906 2.06689929962158 22.0
149.82113309 1.77907729148865 23.0
113.11405249 0.512722611427307 24.0
22.512501728 0.198349148035049 25.0
229.23687799 1.21672677993774 26.0
229.23687799 2.40418577194214 27.0
25.123270456 2.11037802696228 28.0
254.36014845 0.803449988365173 29.0
254.36014845 1.73020732402802 30.0
46.852905944 1.13861382007599 31.0
9.3664323213 0.579615950584412 32.0
21.416195696 1.51837849617004 33.0
21.416195696 1.15880930423737 34.0
21.416195696 0.144106686115265 35.0
9.9566666715 0.0314928069710732 36.0
8.6014592906 0.305878549814224 37.0
0 0.0467532686889172 38.0
0 0.138160884380341 39.0
-12.310255276 0.582882046699524 40.0
-24.763069113 0.57035905122757 41.0
5.3744869826 0.421803802251816 42.0
-33.038255102 0.374931007623672 43.0
104.48093981 0.725183010101318 44.0
104.48093981 0.100902527570724 45.0
7.4685994252 0.620798051357269 46.0
2.4312789478 0.376836329698563 47.0
185.18489662 0.0669593662023544 48.0
179.0742389 0.0533963888883591 49.0
0 0.175576612353325 50.0
80.799959895 0.129009574651718 51.0
88.265505966 0.504715383052826 52.0
81.069207392 0.152862787246704 53.0
81.069207392 0.0666926577687263 54.0
81.069207392 0.0527683570981026 55.0
0 1.3282790184021 56.0
79.415744906 0.167246177792549 57.0
229.23687799 0.819859683513641 58.0
254.36014845 0.469517052173615 59.0
21.416195696 0.0113711729645729 60.0
0 0.0868797898292542 61.0
183.54907433 1.29034757614136 0.0
183.54907433 0.837789952754974 1.0
182.16584062 0.729701995849609 2.0
176.70116988 4.75539350509644 3.0
4.5135199002 4.28293943405151 4.0
4.5135199002 2.5530469417572 5.0
-55.928054032 0.107570588588715 6.0
64.62202611 0.587765455245972 7.0
38.084991553 0.450227975845337 8.0
26.537034557 0.46318656206131 9.0
102.40441757 1.3900089263916 10.0
89.248355333 0.364720970392227 11.0
90.43932101 0.666143178939819 12.0
90.43932101 0.183381527662277 13.0
90.43932101 0.0573304742574692 14.0
0 0.209284439682961 15.0
0 0.194770127534866 16.0
0 0.907053232192993 17.0
55.821497953 1.64074730873108 18.0
-55.821497953 1.97503960132599 19.0
-55.821497953 0.298471629619598 20.0
-55.821497953 0.0392884537577629 21.0
76.838733476 2.39729404449463 22.0
149.92371144 1.73002636432648 23.0
118.02378336 0.738814234733582 24.0
19.314191642 0.157179459929466 25.0
226.76244491 1.68917918205261 26.0
226.76244491 1.79610741138458 27.0
22.783565626 0.76781815290451 28.0
249.54601054 0.566437900066376 29.0
249.54601054 1.52752709388733 30.0
46.930552347 1.19013297557831 31.0
9.2417553532 0.509136199951172 32.0
21.017235523 1.13742470741272 33.0
21.017235523 1.02700734138489 34.0
21.017235523 0.108056746423244 35.0
9.1717239043 0.0516273267567158 36.0
7.9579952077 0.401247620582581 37.0
0 0.0431812517344952 38.0
0 0.308190226554871 39.0
-12.10721343 0.607751786708832 40.0
-24.910560289 0.70085620880127 41.0
5.5191323276 0.647899687290192 42.0
-33.149914185 0.834846973419189 43.0
80.062279197 0.771159052848816 44.0
80.062279197 0.0821557715535164 45.0
7.3018851885 0.709161162376404 46.0
2.3269545061 0.311206489801407 47.0
183.54907433 0.187471032142639 48.0
176.70116988 0.0376967005431652 49.0
4.5135199002 0.135325506329536 50.0
64.62202611 0.112475819885731 51.0
102.40441757 0.355754166841507 52.0
90.43932101 0.121649980545044 53.0
90.43932101 0.079220786690712 54.0
90.43932101 0.0713196694850922 55.0
0 0.576546728610992 56.0
76.838733476 0.106718741357327 57.0
226.76244491 0.511539101600647 58.0
249.54601054 0.363372385501862 59.0
21.017235523 0.0174112655222416 60.0
0 0.100363373756409 61.0
160.98081479 0.988904058933258 0.0
160.98081479 2.1492395401001 1.0
159.70800254 1.77080118656158 2.0
154.24333179 3.60016655921936 3.0
0.15616439635 7.38242149353027 4.0
0.15616439635 4.63619804382324 5.0
-61.409657902 0.656546890735626 6.0
75.085453677 0.558803498744965 7.0
45.984095376 0.580724000930786 8.0
29.101358301 0.674455881118774 9.0
79.100477064 1.08806800842285 10.0
65.768483667 0.0804215967655182 11.0
81.44917042 0.923583388328552 12.0
81.44917042 0.70609837770462 13.0
81.44917042 0.104938037693501 14.0
0 0.242540776729584 15.0
0 0.161678284406662 16.0
0 2.01070022583008 17.0
52.792880751 1.73697626590729 18.0
-52.792880751 1.14659011363983 19.0
-52.792880751 0.394134253263474 20.0
-52.792880751 0.257141292095184 21.0
71.65740104 0.742403030395508 22.0
147.10387619 0.737945079803467 23.0
111.40797939 0.387812048196793 24.0
20.60811044 0.091367207467556 25.0
218.76127723 0.829626679420471 26.0
218.76127723 1.3530261516571 27.0
25.968502277 0.917825102806091 28.0
244.72977951 0.27826201915741 29.0
244.72977951 2.32247376441956 30.0
47.291478811 1.41341519355774 31.0
9.469052978 0.670038640499115 32.0
18.864520289 1.07786881923676 33.0
18.864520289 1.58150839805603 34.0
18.864520289 0.154332518577576 35.0
9.4062192244 0.0332079790532589 36.0
8.2862392604 0.313770592212677 37.0
0 0.0201321169734001 38.0
0 0.271561145782471 39.0
-11.833535791 0.574567854404449 40.0
-24.562882825 0.497345745563507 41.0
4.8832998835 0.588974118232727 42.0
-31.992492408 0.667866110801697 43.0
79.560769687 0.693294584751129 44.0
79.560769687 0.0963463932275772 45.0
7.2499880767 0.547696769237518 46.0
2.266710138 0.388912230730057 47.0
160.98081479 0.117107063531876 48.0
154.24333179 0.185445860028267 49.0
0.15616439635 0.449542760848999 50.0
75.085453677 0.102417804300785 51.0
79.100477064 0.0150715801864862 52.0
81.44917042 0.0948275998234749 53.0
81.44917042 0.0274302456527948 54.0
81.44917042 0.27275025844574 55.0
0 2.96318793296814 56.0
71.65740104 0.0772310197353363 57.0
218.76127723 0.587303936481476 58.0
244.72977951 0.950952112674713 59.0
18.864520289 -0.0167192406952381 60.0
0 0.096225418150425 61.0
159.5456801 0.945791244506836 0.0
159.5456801 1.49752247333527 1.0
158.43230426 1.36408221721649 2.0
153.64319895 2.4132936000824 3.0
0 3.38912105560303 4.0
0 4.95106172561646 5.0
-78.948624087 2.61875009536743 6.0
88.975806875 0.791195750236511 7.0
61.006463529 1.53429841995239 8.0
27.969343346 1.50112950801849 9.0
74.665745472 1.16941356658936 10.0
62.211627484 0.16420978307724 11.0
78.186220506 0.132179036736488 12.0
78.186220506 0.103178694844246 13.0
78.186220506 0.104893244802952 14.0
0 0.179697170853615 15.0
0 0.225078761577606 16.0
0 0.969500601291656 17.0
63.772351875 1.13965773582458 18.0
-63.772351875 0.749337017536163 19.0
-63.772351875 0.665402293205261 20.0
-63.772351875 0.0211919024586678 21.0
84.956402701 2.3089075088501 22.0
150.89310879 1.65262830257416 23.0
115.64860914 0.641820907592773 24.0
23.311076595 0.0915810316801071 25.0
235.84951149 0.892542958259583 26.0
235.84951149 1.02983927726746 27.0
28.742606517 1.31482112407684 28.0
264.59211801 0.810864806175232 29.0
264.59211801 1.79337501525879 30.0
46.384649095 1.05517065525055 31.0
9.3431594773 0.386323660612106 32.0
21.184050826 1.20753014087677 33.0
21.184050826 1.16901409626007 34.0
21.184050826 0.107225351035595 35.0
9.4525172636 0.026439918205142 36.0
8.1187121327 0.387919634580612 37.0
0 0.023416418582201 38.0
0 0.284333378076553 39.0
-12.560424535 0.407768189907074 40.0
-25.860794404 0.608429193496704 41.0
4.6560087943 0.523749470710754 42.0
-32.897056998 0.512910008430481 43.0
81.445490389 0.809927105903625 44.0
81.445490389 0.0887362286448479 45.0
6.3176699934 0.811828851699829 46.0
2.385022712 0.45017209649086 47.0
159.5456801 0.163372293114662 48.0
153.64319895 0.0726049467921257 49.0
0 0.157075792551041 50.0
88.975806875 0.221420660614967 51.0
74.665745472 0.457977682352066 52.0
78.186220506 0.0147142065688968 53.0
78.186220506 0.0928257554769516 54.0
78.186220506 0.116772949695587 55.0
0 6.30941295623779 56.0
84.956402701 0.0262103658169508 57.0
235.84951149 0.286322772502899 58.0
264.59211801 0.826829195022583 59.0
21.184050826 -0.0120754949748516 60.0
0 0.11548912525177 61.0
189.11781904 1.0393682718277 0.0
189.11781904 1.62850034236908 1.0
187.61032932 1.77102959156036 2.0
186.55434888 3.53036212921143 3.0
0 4.44727230072021 4.0
0 3.45118474960327 5.0
-80.52490008 1.05542206764221 6.0
91.458066391 1.07604396343231 7.0
53.329923439 2.00846576690674 8.0
38.128142952 1.74525511264801 9.0
98.566567137 1.03218221664429 10.0
86.83027519 0.188580453395844 11.0
91.771845245 0.890905737876892 12.0
91.771845245 0.117852315306664 13.0
91.771845245 0.122955039143562 14.0
0 0.307259976863861 15.0
0 0.0808373093605042 16.0
0 1.67708802223206 17.0
52.838891973 1.77515161037445 18.0
-52.838891973 1.14770185947418 19.0
-53.507159715 1.46266067028046 20.0
-53.575721314 -0.022055983543396 21.0
74.366649303 1.38446879386902 22.0
129.28582399 1.21204829216003 23.0
98.911284244 0.666329085826874 24.0
20.393336178 0.0738689377903938 25.0
203.65247329 0.893804669380188 26.0
203.65247329 1.63018977642059 27.0
26.477507216 1.60716950893402 28.0
230.12998051 0.162502512335777 29.0
230.12998051 1.7512538433075 30.0
47.396601339 1.02611327171326 31.0
9.4426267344 0.428465694189072 32.0
20.790927989 1.8142021894455 33.0
20.790927989 1.46241414546967 34.0
20.790927989 0.072245180606842 35.0
9.8361977403 0.0314570106565952 36.0
8.6364088891 0.413060069084167 37.0
0 0.0688248872756958 38.0
0 0.195059314370155 39.0
-12.045951953 0.883060812950134 40.0
-26.790810993 0.657715141773224 41.0
4.4167108859 0.506189942359924 42.0
-34.177479994 0.508700788021088 43.0
82.057384325 0.584053874015808 44.0
80.820651312 0.132904440164566 45.0
5.782117539 0.446340978145599 46.0
1.0549937242 0.358381539583206 47.0
189.11781904 0.168358251452446 48.0
186.55434888 0.104686588048935 49.0
0 0.169738754630089 50.0
91.458066391 0.293924897909164 51.0
98.566567137 0.428721696138382 52.0
91.771845245 0.0871972069144249 53.0
91.771845245 0.00337545853108168 54.0
91.771845245 0.0540681853890419 55.0
0 1.18967175483704 56.0
74.366649303 0.0438072867691517 57.0
203.65247329 0.639449179172516 58.0
230.12998051 0.609161674976349 59.0
20.790927989 0.0283536463975906 60.0
0 0.0824321657419205 61.0
166.82653038 0.870546936988831 0.0
166.82653038 1.8127658367157 1.0
165.63553963 2.06582164764404 2.0
160.84643453 5.95934677124023 3.0
0 4.2127251625061 4.0
0 3.47680234909058 5.0
-68.959631459 0.336901187896729 6.0
81.457970989 0.810800790786743 7.0
47.037430268 0.663951694965363 8.0
34.420540721 0.693456292152405 9.0
92.069571442 0.952399373054504 10.0
78.449458642 0.182332277297974 11.0
80.886452012 0.393706411123276 12.0
80.886452012 0.141201168298721 13.0
80.886452012 0.113663375377655 14.0
0 0.242154628038406 15.0
0 0.175166606903076 16.0
0 1.62182176113129 17.0
59.641450948 1.02611422538757 18.0
-59.641450948 0.575872182846069 19.0
-59.641450948 1.06780910491943 20.0
-59.641450948 -0.0940598323941231 21.0
81.246403666 2.79161047935486 22.0
135.93002074 2.37473392486572 23.0
101.93873394 0.369411200284958 24.0
21.793943594 0.110817164182663 25.0
217.17642441 1.37978041172028 26.0
217.17642441 1.77140188217163 27.0
25.184860866 1.93738603591919 28.0
242.36128527 0.788240551948547 29.0
242.36128527 1.98115003108978 30.0
46.490661232 1.1733592748642 31.0
9.4221979101 0.55129212141037 32.0
21.604952718 1.74937808513641 33.0
21.604952718 0.631786525249481 34.0
21.604952718 0.119640789926052 35.0
9.7348949911 0.0271922033280134 36.0
8.4275255746 0.425943315029144 37.0
0 0.00365867093205452 38.0
0 0.212481901049614 39.0
-11.924369277 0.727775156497955 40.0
-25.881102304 0.595011472702026 41.0
5.4300169629 0.441930532455444 42.0
-33.699377136 0.329664170742035 43.0
100.69177696 0.75980418920517 44.0
100.69177696 0.0961272120475769 45.0
6.7686326279 0.651121079921722 46.0
2.1398979868 0.391261667013168 47.0
166.82653038 0.0393229871988297 48.0
160.84643453 0.185864329338074 49.0
0 0.107709631323814 50.0
81.457970989 0.223346054553986 51.0
92.069571442 0.188196569681168 52.0
80.886452012 0.0950445309281349 53.0
80.886452012 0.116450868546963 54.0
80.886452012 0.0568859837949276 55.0
0 3.22789001464844 56.0
81.246403666 0.15209686756134 57.0
217.17642441 0.375335246324539 58.0
242.36128527 0.893391609191895 59.0
21.604952718 0.00793009251356125 60.0
0 0.104997977614403 61.0
191.59821342 0.82877379655838 0.0
191.59821342 0.787597715854645 1.0
190.14592034 0.890944719314575 2.0
184.68124968 5.46086025238037 3.0
8.1863013858 7.30200242996216 4.0
8.1863013858 6.20275592803955 5.0
-64.261496338 2.57002568244934 6.0
76.739021729 0.877437353134155 7.0
46.314354974 0.600789070129395 8.0
30.424666755 0.712228894233704 9.0
84.971650136 0.953277349472046 10.0
71.942063479 0.163920059800148 11.0
76.900207067 0.576936066150665 12.0
76.900207067 0.184338808059692 13.0
76.900207067 0.161221370100975 14.0
0 0.149263069033623 15.0
0 0.143336862325668 16.0
0 1.44060933589935 17.0
57.552587529 0.833484947681427 18.0
-57.552587529 1.09232950210571 19.0
-57.552587529 0.283332288265228 20.0
-57.552587529 -0.0243786964565516 21.0
77.598353697 1.63627398014069 22.0
141.41820547 1.50651478767395 23.0
110.83569201 0.521646797657013 24.0
23.714971631 0.127596318721771 25.0
219.01655917 1.24939572811127 26.0
219.01655917 2.49255537986755 27.0
26.156314714 1.80974638462067 28.0
245.17287388 0.944602489471436 29.0
245.17287388 1.73946070671082 30.0
46.355428285 0.929034113883972 31.0
9.3664059144 0.472872853279114 32.0
20.045766168 1.83785057067871 33.0
20.045766168 0.616816639900208 34.0
20.045766168 0.0934168100357056 35.0
9.1418957739 0.0274593681097031 36.0
7.8036210782 0.388533711433411 37.0
0 0.0477504283189774 38.0
0 0.151988536119461 39.0
-11.328048906 0.738660275936127 40.0
-25.126365886 0.724402785301208 41.0
5.5780876457 0.386447131633759 42.0
-33.124105697 0.379009068012238 43.0
101.45492258 0.869770884513855 44.0
101.45492258 0.109280876815319 45.0
7.1049066408 0.659661650657654 46.0
2.5167647673 0.361880242824554 47.0
191.59821342 0.0976539924740791 48.0
184.68124968 0.156578794121742 49.0
8.1863013858 0.500398337841034 50.0
76.739021729 0.198834761977196 51.0
84.971650136 0.117641419172287 52.0
76.900207067 0.0755290910601616 53.0
76.900207067 0.0507459938526154 54.0
76.900207067 0.0334422253072262 55.0
0 6.11283302307129 56.0
77.598353697 0.167961075901985 57.0
219.01655917 1.05977380275726 58.0
245.17287388 0.462244808673859 59.0
20.045766168 -0.014752927236259 60.0
0 0.0789993330836296 61.0
169.36260149 1.04808926582336 0.0
169.36260149 1.16323387622833 1.0
168.39536175 1.3541100025177 2.0
162.93069096 3.51284313201904 3.0
4.1981720493 7.64410781860352 4.0
4.1981720493 5.17620134353638 5.0
-70.293884628 5.24074649810791 6.0
78.663195668 0.495023280382156 7.0
53.347142431 1.30275690555573 8.0
25.316053237 1.68536841869354 9.0
89.260872812 1.12073004245758 10.0
76.026429174 0.0596818029880524 11.0
81.713401265 0.627042174339294 12.0
81.713401265 0.39391902089119 13.0
81.713401265 0.0322150662541389 14.0
0 -0.00129823386669159 15.0
0 0.240087226033211 16.0
0 0.95799732208252 17.0
53.456998704 0.707645237445831 18.0
-53.456998704 0.746515512466431 19.0
-53.456998704 0.314978450536728 20.0
-53.456998704 0.0862513333559036 21.0
75.488258127 1.38820171356201 22.0
155.18201305 1.10986053943634 23.0
116.80565456 0.197958022356033 24.0
23.362938597 0.276434391736984 25.0
230.67027118 0.969643235206604 26.0
230.67027118 1.66030716896057 27.0
23.284268994 1.3946430683136 28.0
253.95454017 1.2155841588974 29.0
253.95454017 1.94770121574402 30.0
46.938870748 1.28719103336334 31.0
9.1738321201 0.22441266477108 32.0
22.031259423 1.26935958862305 33.0
22.031259423 0.794232070446014 34.0
22.031259423 0.15339957177639 35.0
10.168962585 0.0448011942207813 36.0
8.8220171336 0.364990800619125 37.0
0 0.0321676395833492 38.0
0 0.129236817359924 39.0
-10.486280289 0.730558276176453 40.0
-22.687414316 0.279161661863327 41.0
5.4794737781 0.580391526222229 42.0
-30.784325686 0.65608274936676 43.0
76.257127992 0.646665811538696 44.0
76.257127992 0.0965817272663116 45.0
6.4214401044 0.561853110790253 46.0
2.290025216 0.360885202884674 47.0
169.36260149 0.171853050589561 48.0
162.93069096 0.372760802507401 49.0
4.1981720493 0.336179703474045 50.0
78.663195668 0.19137991964817 51.0
89.260872812 0.0711654424667358 52.0
81.713401265 0.0230134930461645 53.0
81.713401265 0.102829746901989 54.0
81.713401265 0.0960430949926376 55.0
0 3.99163150787354 56.0
75.488258127 0.0780865773558617 57.0
230.67027118 0.234584495425224 58.0
253.95454017 0.647375464439392 59.0
22.031259423 0.000935740768909454 60.0
0 0.0977292060852051 61.0
140.14875598 0.991987884044647 0.0
140.14875598 1.64016461372375 1.0
139.13518214 1.4730269908905 2.0
134.34607703 3.89779925346375 3.0
0 6.1247181892395 4.0
0 4.96717119216919 5.0
-73.982735117 3.01478624343872 6.0
86.038916469 0.492586463689804 7.0
46.343465618 0.992485880851746 8.0
39.69545085 0.924249827861786 9.0
91.767615109 1.06251382827759 10.0
78.686965221 0.119255930185318 11.0
84.160041046 0.688804984092712 12.0
84.160041046 0.477083176374435 13.0
84.160041046 0.128223329782486 14.0
0 0.223083287477493 15.0
0 0.221844643354416 16.0
0 1.56779611110687 17.0
59.596836443 1.60916030406952 18.0
-59.596836443 1.66637992858887 19.0
-59.596836443 0.275426506996155 20.0
-59.596836443 0.0361080691218376 21.0
80.122146209 1.92305028438568 22.0
146.03400822 1.85503911972046 23.0
114.06086027 0.65729832649231 24.0
22.045769712 0.0966308414936066 25.0
226.15615443 1.18475675582886 26.0
226.15615443 1.6128613948822 27.0
29.276029516 1.10176873207092 28.0
255.43218394 0.5716632604599 29.0
255.43218394 1.66631233692169 30.0
46.794326965 0.986433148384094 31.0
9.4386425841 0.5177281498909 32.0
20.525309766 1.28385746479034 33.0
20.525309766 0.604821801185608 34.0
20.525309766 0.118715226650238 35.0
8.8059137511 0.0369659811258316 36.0
7.463653335 0.360085278749466 37.0
0 0.0408347658813 38.0
0 0.0519419685006142 39.0
-11.374684038 0.513721764087677 40.0
-22.854904372 0.568986475467682 41.0
5.1294595168 0.486430615186691 42.0
-30.737366337 0.55800849199295 43.0
102.01728422 0.798182547092438 44.0
102.01728422 0.0784047544002533 45.0
7.2751162656 0.591932773590088 46.0
2.7740873257 0.375708132982254 47.0
140.14875598 0.156699240207672 48.0
134.34607703 0.27142533659935 49.0
0 0.512010514736176 50.0
86.038916469 0.250813543796539 51.0
91.767615109 0.266300320625305 52.0
84.160041046 0.0583412796258926 53.0
84.160041046 0.161149814724922 54.0
84.160041046 0.109663672745228 55.0
0 3.42986130714417 56.0
80.122146209 0.0779799893498421 57.0
226.15615443 0.520189642906189 58.0
255.43218394 0.585730671882629 59.0
20.525309766 0.0107858646661043 60.0
0 0.193552076816559 61.0
163.43342874 1.01004862785339 0.0
163.43342874 1.93163502216339 1.0
162.31277649 1.9406054019928 2.0
157.52367126 5.49610233306885 3.0
0 6.08105039596558 4.0
0 3.72154688835144 5.0
-53.633126321 0.237115740776062 6.0
63.184365347 0.536162555217743 7.0
50.70435729 0.834210753440857 8.0
12.480008057 1.03641486167908 9.0
91.1844223 1.08606648445129 10.0
78.402886488 0.250514328479767 11.0
83.517475888 0.60611766576767 12.0
83.517475888 0.305191993713379 13.0
83.517475888 0.120745100080967 14.0
0 0.158798485994339 15.0
0 0.146754890680313 16.0
0 1.57717418670654 17.0
61.625867095 1.53690481185913 18.0
-61.625867095 1.44398856163025 19.0
-61.625867095 0.455025017261505 20.0
-61.625867095 0.173322796821594 21.0
82.739903865 2.63934087753296 22.0
150.86790554 2.59061002731323 23.0
114.66643048 0.627662420272827 24.0
24.142577928 0.119181022047997 25.0
233.6078094 1.55731475353241 26.0
233.6078094 1.36967349052429 27.0
25.902547019 1.26988959312439 28.0
259.51035642 0.722051739692688 29.0
259.51035642 1.78695714473724 30.0
46.162653429 1.13878130912781 31.0
9.2442176252 0.399702161550522 32.0
21.11403677 1.13962805271149 33.0
21.11403677 0.914537370204926 34.0
21.11403677 0.146196901798248 35.0
8.9996805165 0.0416355729103088 36.0
7.7333028687 0.417122006416321 37.0
0 0.0717224702239037 38.0
0 0.142680123448372 39.0
-11.976805913 0.689197599887848 40.0
-25.629578867 0.562706291675568 41.0
5.1285833863 0.379484117031097 42.0
-33.358935313 0.414118409156799 43.0
92.792834116 0.70710414648056 44.0
92.792834116 0.102183952927589 45.0
6.7240556775 0.652838826179504 46.0
2.3957492695 0.442359387874603 47.0
163.43342874 0.127922087907791 48.0
157.52367126 0.21869008243084 49.0
0 0.416697084903717 50.0
63.184365347 0.293028444051743 51.0
91.1844223 0.198136851191521 52.0
83.517475888 0.0671142339706421 53.0
83.517475888 0.114860534667969 54.0
83.517475888 0.0752901956439018 55.0
0 2.29422998428345 56.0
82.739903865 -0.0519668459892273 57.0
233.6078094 0.399847030639648 58.0
259.51035642 0.492872685194016 59.0
21.11403677 0.00836629793047905 60.0
0 0.101466178894043 61.0
144.33957339 1.03831923007965 0.0
144.33957339 1.10619676113129 1.0
143.00147154 1.32581865787506 2.0
138.21240159 2.98704218864441 3.0
0 5.66341543197632 4.0
0 7.36274862289429 5.0
-75.555957744 0.770099878311157 6.0
85.335858033 1.15198302268982 7.0
45.47011893 0.657206475734711 8.0
39.865739103 0.434622824192047 9.0
92.446328255 1.20504462718964 10.0
79.77081136 0.409880697727203 11.0
81.74685938 0.539471685886383 12.0
81.74685938 0.093963049352169 13.0
81.74685938 0.122577540576458 14.0
0 0.0575685128569603 15.0
0 0.159991264343262 16.0
0 1.25936782360077 17.0
59.676474428 1.52530407905579 18.0
-59.676474428 1.73692440986633 19.0
-59.676474428 0.457399845123291 20.0
-59.676474428 -0.0221233181655407 21.0
81.22108124 1.48616480827332 22.0
153.00745299 1.47381830215454 23.0
119.50114689 0.433694928884506 24.0
22.396308827 0.220988765358925 25.0
234.22853423 0.797718405723572 26.0
234.22853423 1.70579481124878 27.0
29.733678764 1.89999163150787 28.0
263.96221299 0.810044527053833 29.0
263.96221299 1.7676340341568 30.0
47.3192223 0.929695069789886 31.0
9.4004565868 0.335717767477036 32.0
21.544606812 1.57076418399811 33.0
21.544606812 0.672466576099396 34.0
21.544606812 0.154065608978271 35.0
9.4079562021 0.0426576770842075 36.0
8.0904317704 0.385132700204849 37.0
0 0.0563233233988285 38.0
0 0.259426474571228 39.0
-11.601845551 0.471939355134964 40.0
-24.333949198 0.733213186264038 41.0
5.3068409594 0.476791262626648 42.0
-31.999183117 0.41489452123642 43.0
91.946190954 0.732809722423553 44.0
91.946190954 0.0474562048912048 45.0
6.9090054916 0.685328602790833 46.0
2.0967197464 0.399423331022263 47.0
144.33957339 0.0889061316847801 48.0
138.21240159 0.0258705113083124 49.0
0 0.384654492139816 50.0
85.335858033 0.263548374176025 51.0
92.446328255 0.354306697845459 52.0
81.74685938 0.0585005730390549 53.0
81.74685938 0.105289131402969 54.0
81.74685938 0.0733478665351868 55.0
0 3.2397677898407 56.0
81.22108124 0.101683512330055 57.0
234.22853423 0.275660693645477 58.0
263.96221299 0.859048068523407 59.0
21.544606812 0.000482397153973579 60.0
0 0.0816387236118317 61.0
189.92361099 0.79407787322998 0.0
189.92361099 1.05970513820648 1.0
188.47818809 1.16739535331726 2.0
183.68908295 3.7240297794342 3.0
0 5.83140230178833 4.0
0 5.28428745269775 5.0
-52.96859832 0.567160785198212 6.0
61.311120026 0.573509514331818 7.0
59.564288028 1.32457375526428 8.0
1.7468319972 1.41975212097168 9.0
102.3186601 0.882436692714691 10.0
89.498602518 0.256209433078766 11.0
88.040440317 0.633683562278748 12.0
88.040440317 0.726154267787933 13.0
88.040440317 0.0474975854158401 14.0
0 0.0446116104722023 15.0
0 0.171189203858376 16.0
0 1.20102274417877 17.0
63.679914696 1.79351079463959 18.0
-63.679914696 1.39082205295563 19.0
-63.679914696 0.687161087989807 20.0
-63.679914696 -0.0378689989447594 21.0
84.210885326 1.9180269241333 22.0
155.10993287 1.82563316822052 23.0
117.13374748 0.349795252084732 24.0
22.753726315 0.447912395000458 25.0
239.32081819 1.05570340156555 26.0
239.32081819 1.70432639122009 27.0
28.736263632 1.75924611091614 28.0
268.05708182 0.355647832155228 29.0
268.05708182 1.55509841442108 30.0
46.507498407 1.05280137062073 31.0
9.3156129251 0.635520458221436 32.0
20.530970631 0.902952671051025 33.0
20.530970631 1.45927929878235 34.0
20.530970631 0.121244058012962 35.0
8.8567810857 0.0222220420837402 36.0
7.619412903 0.408119261264801 37.0
0 -0.0274111647158861 38.0
0 0.325811326503754 39.0
-12.939050014 0.372673809528351 40.0
-26.033573989 0.883881449699402 41.0
5.1731740024 0.462715238332748 42.0
-33.967920863 0.610534310340881 43.0
95.041383741 0.646358788013458 44.0
95.041383741 0.171345487236977 45.0
7.622062562 0.586119115352631 46.0
2.9189356804 0.384365528821945 47.0
189.92361099 0.128708526492119 48.0
183.68908295 0.23522287607193 49.0
0 0.139501675963402 50.0
61.311120026 0.248646184802055 51.0
102.3186601 0.181920096278191 52.0
88.040440317 0.0611579567193985 53.0
88.040440317 0.165654614567757 54.0
88.040440317 0.211547181010246 55.0
0 1.50043225288391 56.0
84.210885326 0.199050948023796 57.0
239.32081819 0.354512274265289 58.0
268.05708182 -0.0203821286559105 59.0
20.530970631 0.00976277515292168 60.0
0 0.130054399371147 61.0
180.08815068 0.965062201023102 0.0
180.08815068 1.27654695510864 1.0
178.65769592 2.17921686172485 2.0
173.86859071 6.48299646377563 3.0
0 6.18277311325073 4.0
0 2.95310616493225 5.0
-55.597823859 -0.0302772298455238 6.0
67.64836061 0.622839450836182 7.0
47.23006542 1.04068195819855 8.0
20.41829519 1.0943192243576 9.0
84.040347957 0.997364699840546 10.0
71.389840206 0.01937210932374 11.0
85.15274321 0.542810499668121 12.0
85.15274321 0.301564186811447 13.0
85.15274321 0.0820717662572861 14.0
0 0.172507673501968 15.0
0 0.144722759723663 16.0
0 1.18096923828125 17.0
51.720359184 1.34493947029114 18.0
-51.720359184 1.5761399269104 19.0
-51.720359184 0.445231974124908 20.0
-51.720359184 0.000415749847888947 21.0
73.442361258 2.2377667427063 22.0
137.41970107 2.17047333717346 23.0
104.25877057 0.274425595998764 24.0
22.477941354 0.202060967683792 25.0
210.86206233 1.08016002178192 26.0
210.86206233 1.90133547782898 27.0
27.882356685 1.23024225234985 28.0
238.74441901 0.728854119777679 29.0
238.74441901 1.76401162147522 30.0
46.974035493 0.967096745967865 31.0
9.4104164935 0.436493098735809 32.0
21.722002074 1.48438918590546 33.0
21.722002074 0.981684505939484 34.0
21.722002074 0.126522034406662 35.0
9.2261908657 0.0280191469937563 36.0
8.0335159544 0.393515944480896 37.0
0 0.0216186866164207 38.0
0 0.225496560335159 39.0
-11.435962638 0.470172673463821 40.0
-24.320030064 0.683609783649445 41.0
4.6723421319 0.353926569223404 42.0
-31.342303289 0.44476854801178 43.0
98.35603696 0.728790819644928 44.0
98.35603696 0.108726464211941 45.0
7.2060307592 0.463975131511688 46.0
2.2727939527 0.322379291057587 47.0
180.08815068 0.0519653782248497 48.0
173.86859071 0.378897994756699 49.0
0 0.0913796946406364 50.0
67.64836061 0.290694773197174 51.0
84.040347957 0.191476240754128 52.0
85.15274321 0.0367898158729076 53.0
85.15274321 0.0556145943701267 54.0
85.15274321 0.0788419097661972 55.0
0 2.63926386833191 56.0
73.442361258 0.0217796266078949 57.0
210.86206233 0.702091932296753 58.0
238.74441901 0.682590067386627 59.0
21.722002074 -0.0117479357868433 60.0
0 0.0962932482361794 61.0
198.29243017 0.857920050621033 0.0
198.29243017 1.27347528934479 1.0
196.86030653 1.18312251567841 2.0
192.07120143 4.08202171325684 3.0
0 9.39733505249023 4.0
0 5.34139966964722 5.0
-56.407618465 0.554543614387512 6.0
68.352215148 0.413391143083572 7.0
38.328009585 1.2326112985611 8.0
30.024205563 1.0485212802887 9.0
87.537931611 0.908912718296051 10.0
73.780819552 0.146497413516045 11.0
76.830302274 0.540931344032288 12.0
76.830302274 0.580467820167542 13.0
76.830302274 0.0655243396759033 14.0
0 0.00157644227147102 15.0
0 0.181558758020401 16.0
0 0.820370495319366 17.0
53.853188464 0.918757915496826 18.0
-53.853188464 0.989623486995697 19.0
-53.853188464 0.620887160301208 20.0
-53.853188464 0.0664581656455994 21.0
75.388464189 1.59456551074982 22.0
157.05903658 0.989336669445038 23.0
122.0252753 0.609131097793579 24.0
22.549718687 0.149392873048782 25.0
232.44750077 1.2593731880188 26.0
232.44750077 1.44113159179688 27.0
27.017539824 1.35723280906677 28.0
259.46504059 0.72610205411911 29.0
259.46504059 1.47020161151886 30.0
46.181505698 1.21186351776123 31.0
9.2477523719 0.301953256130219 32.0
21.535275725 1.65281057357788 33.0
21.535275725 0.999982893466949 34.0
21.535275725 0.142544969916344 35.0
9.4181999019 0.0380938984453678 36.0
8.0381415953 0.320866644382477 37.0
0 0.0280571952462196 38.0
0 0.229849874973297 39.0
-11.350983477 0.393659561872482 40.0
-25.455581329 0.881487965583801 41.0
4.9146527471 0.428117573261261 42.0
-32.906536183 0.593762993812561 43.0
101.49232133 0.69676810503006 44.0
101.49232133 0.103946574032307 45.0
6.0925266669 0.638597428798676 46.0
2.2221556806 0.454246699810028 47.0
198.29243017 0.195539310574532 48.0
192.07120143 0.0688766986131668 49.0
0 0.38668891787529 50.0
68.352215148 0.165571123361588 51.0
87.537931611 0.232660606503487 52.0
76.830302274 0.0613340213894844 53.0
76.830302274 0.0916405394673347 54.0
76.830302274 -0.0155390677973628 55.0
0 5.55013275146484 56.0
75.388464189 -0.0446321368217468 57.0
232.44750077 0.556904375553131 58.0
259.46504059 0.435262262821198 59.0
21.535275725 -0.0119338883087039 60.0
0 0.0993962064385414 61.0
164.9807995 1.00010979175568 0.0
164.9807995 1.22395837306976 1.0
163.82600318 1.3591148853302 2.0
159.03691964 5.36529541015625 3.0
0 4.55031108856201 4.0
0 4.91919898986816 5.0
-53.671167704 1.13250720500946 6.0
66.465272004 0.443665474653244 7.0
52.782510829 0.904193997383118 8.0
13.682761175 1.03734993934631 9.0
80.83184058 1.10355687141418 10.0
68.069490113 0.139930620789528 11.0
79.679721634 0.633293211460114 12.0
79.679721634 0.293539643287659 13.0
79.679721634 0.0868568494915962 14.0
0 0.197374805808067 15.0
0 0.192181050777435 16.0
0 1.51493096351624 17.0
56.599765935 1.0608731508255 18.0
-56.599765935 0.836500525474548 19.0
-56.599765935 0.801455914974213 20.0
-56.599765935 -0.058076485991478 21.0
78.259894628 1.24169588088989 22.0
143.47617615 0.645170032978058 23.0
106.70508229 0.569855749607086 24.0
23.855849662 0.267764568328857 25.0
221.73607078 0.97319084405899 26.0
221.73607078 1.69001173973083 27.0
23.512014248 1.72022712230682 28.0
245.24808502 0.796381950378418 29.0
245.24808502 1.83682787418365 30.0
46.089912016 0.850470304489136 31.0
9.2414496558 0.503453195095062 32.0
21.660128692 1.53548741340637 33.0
21.660128692 0.889383554458618 34.0
21.660128692 0.126880168914795 35.0
10.171733917 0.0318337976932526 36.0
8.9809095074 0.374353617429733 37.0
0 0.0612190552055836 38.0
0 0.24138368666172 39.0
-12.663570808 0.757604241371155 40.0
-26.65332849 0.829057455062866 41.0
5.5498206539 0.523390114307404 42.0
-35.09016566 0.617635905742645 43.0
91.030387831 0.839246690273285 44.0
91.030387831 0.064355693757534 45.0
6.4865298001 0.670108377933502 46.0
2.6978812347 0.389651507139206 47.0
164.9807995 0.0896426066756248 48.0
159.03691964 0.0514492839574814 49.0
0 0.283927261829376 50.0
66.465272004 0.0693591684103012 51.0
80.83184058 0.245604664087296 52.0
79.679721634 0.0287124179303646 53.0
79.679721634 0.169385999441147 54.0
79.679721634 0.068650022149086 55.0
0 1.6619256734848 56.0
78.259894628 0.143800169229507 57.0
221.73607078 0.628899455070496 58.0
245.24808502 1.113898396492 59.0
21.660128692 0.00916118919849396 60.0
0 0.094145655632019 61.0
148.27404442 0.990883111953735 0.0
148.27404442 1.48746931552887 1.0
146.84166092 1.30546474456787 2.0
142.05255565 4.4938006401062 3.0
0 3.648601770401 4.0
0 4.82186794281006 5.0
-70.565085119 0.24589841067791 6.0
79.197768742 0.910023391246796 7.0
47.367798332 1.12814271450043 8.0
31.82997041 0.895128786563873 9.0
92.930488545 0.918831467628479 10.0
80.604644384 0.0383923649787903 11.0
77.107937185 0.863907098770142 12.0
77.107937185 0.261302053928375 13.0
77.107937185 0.0883819609880447 14.0
0 0.303613811731339 15.0
0 0.182131707668304 16.0
0 0.866127371788025 17.0
60.24170034 1.28380191326141 18.0
-60.24170034 1.73216187953949 19.0
-60.24170034 0.328146517276764 20.0
-60.24170034 0.11280944943428 21.0
80.727355887 3.04311776161194 22.0
147.48081923 3.26800584793091 23.0
109.77567216 0.234364435076714 24.0
20.157172323 0.378898650407791 25.0
228.20817512 1.57447600364685 26.0
228.20817512 2.04097938537598 27.0
24.111931947 1.71057343482971 28.0
252.32010707 0.27705055475235 29.0
252.32010707 1.76785707473755 30.0
46.267055517 1.14439213275909 31.0
9.4456526626 0.347393363714218 32.0
20.485655546 1.42097330093384 33.0
20.485655546 0.51257461309433 34.0
20.485655546 0.14162689447403 35.0
9.1862307547 0.0358883440494537 36.0
8.0791700826 0.430840313434601 37.0
0 0.0327129028737545 38.0
0 0.22610367834568 39.0
-11.280040154 0.420498788356781 40.0
-23.756557535 0.517889082431793 41.0
5.1978335113 0.599337518215179 42.0
-31.699729805 0.650616407394409 43.0
93.675766932 0.723498344421387 44.0
93.675766932 0.171337872743607 45.0
6.8221576501 0.518937885761261 46.0
2.3905878685 0.313235878944397 47.0
148.27404442 0.0337698832154274 48.0
142.05255565 0.0612791180610657 49.0
0 0.122941702604294 50.0
79.197768742 0.331583499908447 51.0
92.930488545 0.204910218715668 52.0
77.107937185 0.094930499792099 53.0
77.107937185 0.185755282640457 54.0
77.107937185 0.255719244480133 55.0
0 4.89483785629272 56.0
80.727355887 -0.0498520880937576 57.0
228.20817512 0.375247091054916 58.0
252.32010707 0.27710634469986 59.0
20.485655546 -0.013713862746954 60.0
0 0.143127158284187 61.0
128.28441391 0.974141716957092 0.0
128.28441391 1.61553752422333 1.0
127.259034 1.55276489257812 2.0
122.4699288 2.89643907546997 3.0
0 6.35170984268188 4.0
0 3.91718649864197 5.0
-61.24167099 0.328722894191742 6.0
69.995254746 0.374923884868622 7.0
58.511452948 0.711164891719818 8.0
11.483801798 0.901525259017944 9.0
95.829744692 1.17263150215149 10.0
82.543534353 0.723695039749146 11.0
86.663604536 0.697975695133209 12.0
86.663604536 0.323942244052887 13.0
86.663604536 0.0471156984567642 14.0
0 0.103947289288044 15.0
0 0.190554156899452 16.0
0 1.09728109836578 17.0
56.266026146 1.43027412891388 18.0
-56.266026146 1.44391763210297 19.0
-56.266026146 0.407204627990723 20.0
-56.266026146 0.14811472594738 21.0
79.26223551 1.21091771125793 22.0
159.5036492 1.05483520030975 23.0
120.93714649 0.298707455396652 24.0
25.070302144 0.387505769729614 25.0
238.76588471 0.931607127189636 26.0
238.76588471 1.77776181697845 27.0
25.145839296 2.27027750015259 28.0
263.91172401 0.822495758533478 29.0
263.91172401 1.63224101066589 30.0
46.945970016 1.06311976909637 31.0
9.481182573 0.391328990459442 32.0
22.996209364 1.11632084846497 33.0
22.996209364 1.07574844360352 34.0
22.996209364 0.0647612065076828 35.0
9.5272043655 0.0224502868950367 36.0
8.3912798957 0.460338354110718 37.0
0 0.0159848630428314 38.0
0 0.117336183786392 39.0
-12.895204402 0.577922642230988 40.0
-24.013689491 0.339145243167877 41.0
5.5385489279 0.553333878517151 42.0
-31.840657777 0.948306202888489 43.0
104.30531017 0.724413096904755 44.0
104.30531017 0.0616610050201416 45.0
6.2643359725 0.628055393695831 46.0
2.3189054933 0.374014675617218 47.0
128.28441391 0.178524196147919 48.0
122.4699288 0.150819420814514 49.0
0 0.436014384031296 50.0
69.995254746 0.149363189935684 51.0
95.829744692 0.781883120536804 52.0
86.663604536 0.151922792196274 53.0
86.663604536 0.0656279101967812 54.0
86.663604536 0.222903832793236 55.0
0 3.53763341903687 56.0
79.26223551 0.0258302614092827 57.0
238.76588471 0.185210391879082 58.0
263.91172401 0.427632480859756 59.0
22.996209364 0.00161612592637539 60.0
0 0.114298366010189 61.0
188.59377275 0.846422612667084 0.0
188.59377275 1.14357554912567 1.0
187.2412238 1.36024069786072 2.0
181.77658839 4.83314847946167 3.0
6.0740361794 8.72816181182861 4.0
6.0740361794 6.62617540359497 5.0
-65.428609679 3.78976130485535 6.0
77.053037787 1.10648286342621 7.0
49.817666381 0.718411982059479 8.0
27.235371406 0.579127013683319 9.0
91.555032239 0.993003189563751 10.0
78.071135239 0.178964242339134 11.0
90.634995241 0.134203791618347 12.0
90.634995241 0.104076378047466 13.0
90.634995241 0.112085491418839 14.0
0 0.0747172757983208 15.0
0 0.208944141864777 16.0
0 0.648752570152283 17.0
49.168197788 1.19963812828064 18.0
-49.168197788 1.02363133430481 19.0
-49.168197788 0.38704726099968 20.0
-49.168197788 0.0634185373783112 21.0
68.37047442 2.36118364334106 22.0
141.41593956 1.77751219272614 23.0
103.28103057 0.542862772941589 24.0
24.382961876 0.0763617604970932 25.0
209.78641398 1.52615082263947 26.0
209.78641398 1.52545070648193 27.0
29.119170806 1.34252047538757 28.0
238.90558479 0.6233069896698 29.0
238.90558479 1.71332740783691 30.0
47.099058349 0.988650023937225 31.0
9.4810609137 0.500255107879639 32.0
19.202276632 1.68741834163666 33.0
19.202276632 0.932959854602814 34.0
19.202276632 0.132583424448967 35.0
9.06412154 0.0560269877314568 36.0
7.8286120613 0.363205879926682 37.0
0 0.011886939406395 38.0
0 0.238891869783401 39.0
-11.568713663 0.630282700061798 40.0
-23.045200991 0.570713818073273 41.0
5.1630528327 0.399249523878098 42.0
-30.912889001 0.418160796165466 43.0
83.856716623 0.598823487758636 44.0
83.856716623 0.114165432751179 45.0
6.6289030449 0.513028264045715 46.0
2.3834446697 0.424008131027222 47.0
188.59377275 0.0751910954713821 48.0
181.77658839 0.161812365055084 49.0
6.0740361794 0.551536500453949 50.0
77.053037787 0.0981079861521721 51.0
91.555032239 0.206526204943657 52.0
90.634995241 0.00995591469109058 53.0
90.634995241 0.0812909603118896 54.0
90.634995241 0.0930455103516579 55.0
0 4.33826351165771 56.0
68.37047442 -0.034634493291378 57.0
209.78641398 0.402389466762543 58.0
238.90558479 1.13021183013916 59.0
19.202276632 -0.0121856834739447 60.0
0 0.10183247923851 61.0
153.46796282 1.04492807388306 0.0
153.46796282 1.22799694538116 1.0
152.1315148 2.54073596000671 2.0
147.34240965 3.9508593082428 3.0
0 6.89969158172607 4.0
0 2.8536388874054 5.0
-52.646361724 0.709184169769287 6.0
66.600605247 0.615804076194763 7.0
53.715868032 1.84428298473358 8.0
12.884737216 1.97845387458801 9.0
71.381249807 0.991551280021667 10.0
59.85062403 0.0226493217051029 11.0
73.436775641 1.16073381900787 12.0
73.436775641 0.381730854511261 13.0
73.436775641 0.182101935148239 14.0
0 0.0755128860473633 15.0
0 0.202101081609726 16.0
0 1.81758081912994 17.0
58.724703845 1.15357911586761 18.0
-58.724703845 1.16813600063324 19.0
-58.724703845 0.711804986000061 20.0
-58.724703845 -0.0119124464690685 21.0
79.20166172 1.76693618297577 22.0
135.64475192 0.938470005989075 23.0
101.01311933 0.907467663288116 24.0
20.039898202 0.0743076801300049 25.0
214.84641364 1.51716923713684 26.0
214.84641364 1.84206128120422 27.0
30.852867523 1.26887381076813 28.0
245.69928116 0.879145741462708 29.0
245.69928116 1.45355188846588 30.0
46.077929399 1.14274108409882 31.0
9.3741085818 0.443322569131851 32.0
20.476957875 1.33109772205353 33.0
20.476957875 0.448548495769501 34.0
20.476957875 0.094309501349926 35.0
8.5244740007 0.02568219602108 36.0
7.2604293052 0.445978611707687 37.0
0 0.0578000396490097 38.0
0 0.27130138874054 39.0
-12.342335757 0.901656031608582 40.0
-23.818254625 0.190789222717285 41.0
5.8480061227 0.519206047058105 42.0
-31.942836908 0.493751168251038 43.0
96.586313964 0.679694831371307 44.0
96.586313964 0.113220371305943 45.0
7.3557977005 0.453576654195786 46.0
3.1441195857 0.350881725549698 47.0
153.46796282 0.268534243106842 48.0
147.34240965 0.420129954814911 49.0
0 0.243176832795143 50.0
66.600605247 0.138290256261826 51.0
71.381249807 -0.0353952310979366 52.0
73.436775641 0.100207097828388 53.0
73.436775641 0.033027920871973 54.0
73.436775641 0.273802757263184 55.0
0 8.25414085388184 56.0
79.20166172 -0.040646918118 57.0
214.84641364 0.390932530164719 58.0
245.69928116 0.0529650412499905 59.0
20.476957875 -0.0132018532603979 60.0
0 0.0864294692873955 61.0
178.48337876 0.930717468261719 0.0
178.48337876 0.789822936058044 1.0
177.3551215 0.830380797386169 2.0
172.56601644 4.01016521453857 3.0
0 4.58850336074829 4.0
0 2.16682267189026 5.0
-55.965447263 0.731874704360962 6.0
68.322588162 0.617492616176605 7.0
65.834555623 0.663746654987335 8.0
2.488032539 0.810812294483185 9.0
90.041145437 1.03755104541779 10.0
76.705149968 0.163466572761536 11.0
82.152014601 0.608622848987579 12.0
82.152014601 0.108428150415421 13.0
82.152014601 0.0593786537647247 14.0
0 0.0247819051146507 15.0
0 0.261887729167938 16.0
0 0.809147834777832 17.0
51.846908037 1.04509890079498 18.0
-51.846908037 1.06126320362091 19.0
-51.846908037 0.673581779003143 20.0
-51.846908037 -0.00389421358704567 21.0
74.114456111 1.52530407905579 22.0
148.90470929 1.84637606143951 23.0
114.40029477 0.461288571357727 24.0
20.809741581 0.254315108060837 25.0
223.0191654 0.517797529697418 26.0
223.0191654 1.02308094501495 27.0
27.810256063 1.16551780700684 28.0
250.82942147 0.546808362007141 29.0
250.82942147 1.46584475040436 30.0
46.467460569 1.11173987388611 31.0
9.3354212382 0.504652261734009 32.0
22.267548074 1.77263510227203 33.0
22.267548074 1.08366763591766 34.0
22.267548074 0.118811883032322 35.0
9.3629380695 0.02471281029284 36.0
8.2176986113 0.414419651031494 37.0
0 -0.0167631134390831 38.0
0 0.267840921878815 39.0
-11.417386901 0.615618586540222 40.0
-24.20813698 0.837622821331024 41.0
5.4837074403 0.35650834441185 42.0
-31.902165715 0.58968859910965 43.0
110.45013728 0.67218554019928 44.0
110.45013728 0.159961193799973 45.0
6.0367118251 0.586106419563293 46.0
2.1291866524 0.393322288990021 47.0
178.48337876 0.186789140105247 48.0
172.56601644 0.273070186376572 49.0
0 0.0692295059561729 50.0
68.322588162 0.126758143305779 51.0
90.041145437 0.173110008239746 52.0
82.152014601 0.00352866249158978 53.0
82.152014601 0.167180329561234 54.0
82.152014601 0.105391204357147 55.0
0 4.05132007598877 56.0
74.114456111 0.0829039514064789 57.0
223.0191654 0.35182511806488 58.0
250.82942147 0.562920153141022 59.0
22.267548074 0.0301361121237278 60.0
0 0.113655269145966 61.0
132.88158741 0.905698895454407 0.0
132.88158741 1.10337471961975 1.0
131.34187612 1.08886194229126 2.0
126.55277086 4.74141454696655 3.0
0 5.7412166595459 4.0
0 3.15809059143066 5.0
-63.741159783 0.883751392364502 6.0
74.851284869 0.441040843725204 7.0
50.094250156 0.858045220375061 8.0
24.757034713 0.898099541664124 9.0
94.247445987 1.01099967956543 10.0
81.554946234 0.117786392569542 11.0
85.698187859 0.488012790679932 12.0
85.698187859 0.348406493663788 13.0
85.698187859 0.0983039438724518 14.0
0 0.17301182448864 15.0
0 0.231083512306213 16.0
0 1.21419668197632 17.0
51.563290615 1.6188223361969 18.0
-51.563290615 1.46830260753632 19.0
-51.563290615 0.531784951686859 20.0
-51.563290615 -0.0463020913302898 21.0
73.218510815 1.31933307647705 22.0
141.84947522 0.776037037372589 23.0
108.77420684 0.874828636646271 24.0
25.871280701 0.105621740221977 25.0
215.06798604 1.29786098003387 26.0
215.06798604 1.35641014575958 27.0
25.592547255 1.79861319065094 28.0
240.66053329 0.77639639377594 29.0
240.66053329 1.59085297584534 30.0
47.230767262 0.832279026508331 31.0
9.222331662 0.47492903470993 32.0
21.6552202 1.37668812274933 33.0
21.6552202 0.878374993801117 34.0
21.6552202 0.108362205326557 35.0
9.7481589843 0.0204884689301252 36.0
8.5800785036 0.380189269781113 37.0
0 0.0721022188663483 38.0
0 0.088191494345665 39.0
-12.076847616 0.640259623527527 40.0
-24.347336693 0.556121468544006 41.0
5.5935400593 0.424163699150085 42.0
-32.287838929 0.40140038728714 43.0
87.216527715 0.678363502025604 44.0
87.216527715 0.122369147837162 45.0
6.5958043192 0.471920251846313 46.0
1.4992626296 0.365728169679642 47.0
132.88158741 0.124112486839294 48.0
126.55277086 0.292731702327728 49.0
0 0.259261667728424 50.0
74.851284869 0.1428542137146 51.0
94.247445987 0.200323447585106 52.0
85.698187859 0.0241627246141434 53.0
85.698187859 0.17615020275116 54.0
85.698187859 0.128084808588028 55.0
0 5.41616725921631 56.0
73.218510815 0.0559850111603737 57.0
215.06798604 0.420709013938904 58.0
240.66053329 0.809826552867889 59.0
21.6552202 -0.00710880197584629 60.0
0 0.113185755908489 61.0
161.71231742 0.813247859477997 0.0
161.71231742 1.02037119865417 1.0
160.73416961 1.48570334911346 2.0
155.94509965 2.63477230072021 3.0
0 4.39097881317139 4.0
0 2.30580043792725 5.0
-56.134326814 0.0759118944406509 6.0
67.659143653 1.09710955619812 7.0
44.792500719 0.595008075237274 8.0
22.866642934 0.460394322872162 9.0
94.316489002 0.917474091053009 10.0
81.633984396 0.70810729265213 11.0
80.248372451 0.542943000793457 12.0
80.248372451 0.0876580774784088 13.0
80.248372451 0.0851746425032616 14.0
0 0.121987506747246 15.0
0 0.215367943048477 16.0
0 0.504692018032074 17.0
57.004911901 1.3802250623703 18.0
-57.004911901 1.95054697990417 19.0
-57.004911901 0.30784147977829 20.0
-57.004911901 0.0674476400017738 21.0
79.86585422 1.43992733955383 22.0
151.75796061 2.08333969116211 23.0
119.78955652 0.217862859368324 24.0
24.244922791 0.334429979324341 25.0
231.62381483 1.21798312664032 26.0
231.62381483 1.53354203701019 27.0
23.218496341 1.58816516399384 28.0
254.84231117 0.360240966081619 29.0
254.84231117 1.39235377311707 30.0
46.74238976 1.11712443828583 31.0
9.2297551487 0.568370521068573 32.0
22.860942319 1.3917281627655 33.0
22.860942319 1.24275946617126 34.0
22.860942319 0.132254853844643 35.0
10.182230855 0.0235210340470076 36.0
8.7767656481 0.37847575545311 37.0
0 0.0117429420351982 38.0
0 0.17042051255703 39.0
-10.750589258 0.722541809082031 40.0
-25.313909549 0.360801577568054 41.0
5.3119251988 0.506929218769073 42.0
-33.31486094 0.525060951709747 43.0
98.029651962 0.732762455940247 44.0
98.029651962 0.143226340413094 45.0
6.5894411779 0.62332546710968 46.0
2.3103931106 0.406192481517792 47.0
161.71231742 0.0885471627116203 48.0
155.94509965 0.0221589636057615 49.0
0 0.0337705537676811 50.0
67.659143653 0.0701517462730408 51.0
94.316489002 0.35266387462616 52.0
80.248372451 0.124405048787594 53.0
80.248372451 0.108757115900517 54.0
80.248372451 0.0856365486979485 55.0
0 4.86871337890625 56.0
79.86585422 -0.0318820253014565 57.0
231.62381483 0.725651502609253 58.0
254.84231117 -0.0613895654678345 59.0
22.860942319 -0.00913230422884226 60.0
0 0.0938863456249237 61.0
148.27059083 0.965267241001129 0.0
148.27059083 2.57286167144775 1.0
146.87654661 2.12431502342224 2.0
142.08744145 3.49791979789734 3.0
0 4.75307655334473 4.0
0 4.37994956970215 5.0
-54.584372313 0.22027525305748 6.0
66.140016292 1.22226238250732 7.0
45.063760697 0.455807864665985 8.0
21.076255595 0.380581319332123 9.0
88.205334122 1.06286585330963 10.0
75.616683435 0.0813410207629204 11.0
88.322749963 0.427824258804321 12.0
88.322749963 0.615513026714325 13.0
88.322749963 0.0701045393943787 14.0
0 0.18861098587513 15.0
0 0.168913885951042 16.0
0 1.13586831092834 17.0
51.782563461 1.07728457450867 18.0
-51.782563461 1.39226269721985 19.0
-51.782563461 0.291927874088287 20.0
-51.782563461 -0.0342605002224445 21.0
74.08706796 1.69586431980133 22.0
151.89704404 1.32807672023773 23.0
114.19198536 0.469335198402405 24.0
22.385629522 0.326786011457443 25.0
225.98411201 1.1256959438324 26.0
225.98411201 1.52229535579681 27.0
25.357623321 1.72800827026367 28.0
251.34173533 0.654705822467804 29.0
251.34173533 1.73057126998901 30.0
46.392518853 1.27666366100311 31.0
9.2479706295 0.47698250412941 32.0
22.304504499 1.2427042722702 33.0
22.304504499 0.903376579284668 34.0
22.304504499 0.0921427309513092 35.0
9.9061870773 0.0204368811100721 36.0
8.5897437195 0.439207673072815 37.0
0 0.00715336576104164 38.0
0 0.112035632133484 39.0
-10.803925543 0.680284798145294 40.0
-23.422741798 0.322760492563248 41.0
5.3250388351 0.506203770637512 42.0
-31.353368905 0.527451395988464 43.0
90.403061147 0.518818438053131 44.0
90.403061147 0.0931604355573654 45.0
6.4679530831 0.475480765104294 46.0
1.7618884916 0.392495691776276 47.0
148.27059083 0.131375446915627 48.0
142.08744145 0.23047262430191 49.0
0 0.215839177370071 50.0
66.140016292 0.0453232601284981 51.0
88.205334122 0.236969247460365 52.0
88.322749963 0.0559829026460648 53.0
88.322749963 0.0643987208604813 54.0
88.322749963 0.206797704100609 55.0
0 1.81389045715332 56.0
74.08706796 0.175697803497314 57.0
225.98411201 0.111831843852997 58.0
251.34173533 0.0467811301350594 59.0
22.304504499 0.0241100005805492 60.0
0 0.176384523510933 61.0
179.77044206 0.925223171710968 0.0
179.77044206 0.944614768028259 1.0
178.73131347 0.815210461616516 2.0
173.94220827 3.6788923740387 3.0
0 3.99947357177734 4.0
0 3.92826700210571 5.0
-73.989051083 0.969535052776337 6.0
86.693049118 1.47695088386536 7.0
60.594526992 0.40828675031662 8.0
26.098522126 0.374911725521088 9.0
81.438692705 1.09179401397705 10.0
69.178323406 0.151864036917686 11.0
77.009694635 0.940282225608826 12.0
77.009694635 0.932674467563629 13.0
77.009694635 0.0432563461363316 14.0
0 0.0109934583306313 15.0
0 0.185938984155655 16.0
0 1.12829446792603 17.0
59.006515738 0.921001195907593 18.0
-59.006515738 0.686425685882568 19.0
-59.006515738 1.06891560554504 20.0
-59.006515738 -0.0927874147891998 21.0
81.754205685 1.94945728778839 22.0
151.41034593 1.7229700088501 23.0
116.34839068 0.537232935428619 24.0
22.568975634 0.115132570266724 25.0
233.16455162 1.1783447265625 26.0
233.16455162 1.2663711309433 27.0
23.390470975 1.47291278839111 28.0
256.55502259 1.05763030052185 29.0
256.55502259 1.60717988014221 30.0
46.331295925 1.15679776668549 31.0
9.3545929913 0.409103721380234 32.0
22.747689946 1.7684770822525 33.0
22.747689946 1.05057978630066 34.0
22.747689946 0.134028226137161 35.0
9.8190081642 0.0258763786405325 36.0
8.5866993734 0.433207958936691 37.0
0 0.0225311405956745 38.0
0 0.190597489476204 39.0
-12.467570944 1.14195096492767 40.0
-25.850352649 0.419042885303497 41.0
5.1019350244 0.481223940849304 42.0
-33.491380845 0.489346385002136 43.0
102.61697429 0.682323038578033 44.0
102.61697429 0.089689664542675 45.0
6.5180635588 0.641810119152069 46.0
2.5388180007 0.429197520017624 47.0
179.77044206 0.0755525529384613 48.0
173.94220827 -0.00019512209109962 49.0
0 0.0954516381025314 50.0
86.693049118 0.115798979997635 51.0
81.438692705 0.336923569440842 52.0
77.009694635 0.0791162550449371 53.0
77.009694635 0.246584296226501 54.0
77.009694635 0.185310542583466 55.0
0 4.61731386184692 56.0
81.754205685 0.108341038227081 57.0
233.16455162 0.280876100063324 58.0
256.55502259 0.686202347278595 59.0
22.747689946 -0.01528031937778 60.0
0 0.101864136755466 61.0
201.34349109 1.1850301027298 0.0
201.34349109 1.4697402715683 1.0
200.14480087 1.68327188491821 2.0
195.35569565 7.07004690170288 3.0
0 3.85466289520264 4.0
0 3.38952612876892 5.0
-60.474644176 0.34652316570282 6.0
74.131981853 0.779213011264801 7.0
42.632332123 1.26708233356476 8.0
31.49964973 1.15498948097229 9.0
90.05650852 1.1486394405365 10.0
77.727546989 0.0103541519492865 11.0
83.982977561 0.533106923103333 12.0
83.982977561 0.811984956264496 13.0
83.982977561 0.110125675797462 14.0
0 0.393162190914154 15.0
0 0.16990864276886 16.0
0 1.35997068881989 17.0
50.766294244 1.68500328063965 18.0
-50.766294244 1.86600530147552 19.0
-50.766294244 0.487743318080902 20.0
-50.766294244 0.023914135992527 21.0
74.08209502 0.76676070690155 22.0
153.25619071 0.560081779956818 23.0
122.94951501 0.769758939743042 24.0
24.879652313 0.225647151470184 25.0
227.33828573 1.06988573074341 26.0
227.33828573 1.47624444961548 27.0
23.160842685 1.8978259563446 28.0
250.49912842 1.15630066394806 29.0
250.49912842 2.0640549659729 30.0
47.167915842 1.15856242179871 31.0
9.3052307618 0.35978689789772 32.0
23.315800776 1.22785174846649 33.0
23.315800776 0.833869934082031 34.0
23.315800776 0.10465169698 35.0
10.431715079 0.0304717849940062 36.0
9.0224492516 0.387763231992722 37.0
0 0.00385166332125664 38.0
0 0.214041858911514 39.0
-11.414407906 0.672836124897003 40.0
-24.423563024 0.593859076499939 41.0
4.6011644356 0.358677089214325 42.0
-31.367984292 0.241847515106201 43.0
97.20108782 0.818418741226196 44.0
97.20108782 0.0684422105550766 45.0
5.8667868113 0.582149684429169 46.0
2.255054556 0.371326833963394 47.0
201.34349109 0.168438836932182 48.0
195.35569565 0.166808933019638 49.0
0 0.125979378819466 50.0
74.131981853 0.271396726369858 51.0
90.05650852 0.0915494337677956 52.0
83.982977561 0.00874674506485462 53.0
83.982977561 0.153428092598915 54.0
83.982977561 0.215889379382133 55.0
0 2.87373042106628 56.0
74.08209502 -0.0416733585298061 57.0
227.33828573 0.374400615692139 58.0
250.49912842 0.656728684902191 59.0
23.315800776 0.00201716460287571 60.0
0 0.101082257926464 61.0
137.67482597 1.17424702644348 0.0
137.67482597 2.52001523971558 1.0
136.21153417 2.60096716880798 2.0
131.42242888 6.37641906738281 3.0
0 4.90900945663452 4.0
0 4.44687509536743 5.0
-64.029989861 0.37433248758316 6.0
73.8677578 0.959635615348816 7.0
60.672232393 0.962643980979919 8.0
13.195525407 1.09254550933838 9.0
86.999278231 1.12547445297241 10.0
73.813548597 0.0580349490046501 11.0
76.736860873 0.210269451141357 12.0
76.736860873 0.260414779186249 13.0
76.736860873 0.0938127562403679 14.0
0 0.0661765709519386 15.0
0 0.22053387761116 16.0
0 0.872383594512939 17.0
63.218026632 1.0775842666626 18.0
-63.218026632 0.98661994934082 19.0
-63.218026632 0.345717638731003 20.0
-63.218026632 0.0359701216220856 21.0
87.011528508 1.58872854709625 22.0
158.71960984 1.34301495552063 23.0
119.2863918 0.477738529443741 24.0
21.28438265 0.253642618656158 25.0
245.73113835 1.26539552211761 26.0
245.73113835 2.50216293334961 27.0
28.685707285 1.73420977592468 28.0
274.41684563 1.25046539306641 29.0
274.41684563 2.26315236091614 30.0
46.599237059 1.23111271858215 31.0
9.4862193699 0.637286365032196 32.0
23.793501876 1.39722776412964 33.0
23.793501876 0.921444892883301 34.0
23.793501876 0.0616907961666584 35.0
10.313464602 0.0205146726220846 36.0
9.0861365856 0.463877201080322 37.0
0 0.00317329913377762 38.0
0 0.218078821897507 39.0
-10.558713842 0.772018611431122 40.0
-22.303418519 0.579532563686371 41.0
4.611295487 0.43871733546257 42.0
-29.084155816 0.78240042924881 43.0
90.226578056 0.676414370536804 44.0
90.226578056 0.185894548892975 45.0
7.1706968143 0.460361421108246 46.0
2.5082320297 0.330432832241058 47.0
137.67482597 0.107926577329636 48.0
131.42242888 0.71601003408432 49.0
0 0.199188202619553 50.0
73.8677578 0.274897933006287 51.0
86.999278231 0.152466401457787 52.0
76.736860873 0.0235127788037062 53.0
76.736860873 0.0856954306364059 54.0
76.736860873 0.0952448546886444 55.0
0 7.0310697555542 56.0
87.011528508 -0.0021691769361496 57.0
245.73113835 0.933561980724335 58.0
274.41684563 0.914990186691284 59.0
23.793501876 -0.00260749086737633 60.0
0 0.0849463939666748 61.0
207.45932653 1.21317601203918 0.0
207.45932653 1.56326174736023 1.0
206.53908934 1.68143951892853 2.0
201.07441854 7.33404111862183 3.0
2.7113595555 5.25481224060059 4.0
2.7113595555 1.62279796600342 5.0
-65.356755834 0.475834101438522 6.0
76.248753823 0.671614766120911 7.0
41.880618601 0.728234231472015 8.0
34.368135222 0.896597325801849 9.0
75.031074251 1.12295508384705 10.0
63.29337321 0.043388195335865 11.0
61.474866434 0.0812566503882408 12.0
61.474866434 0.0934011116623878 13.0
61.474866434 0.0899259522557259 14.0
14.426160808 0.187460720539093 15.0
14.426160808 0.212691634893417 16.0
0 1.66235506534576 17.0
62.862971615 0.408112406730652 18.0
-62.862971615 0.25441175699234 19.0
-62.862971615 1.71920704841614 20.0
-62.862971615 -0.00782468542456627 21.0
84.554825306 2.1615891456604 22.0
133.41497523 1.80341839790344 23.0
100.56338033 0.757009983062744 24.0
10.183304534 0.0829897522926331 25.0
217.96980054 2.34867024421692 26.0
217.96980054 2.84415316581726 27.0
21.982136428 3.21204161643982 28.0
239.95193697 1.85768282413483 29.0
239.95193697 2.21942090988159 30.0
47.341693341 1.00623893737793 31.0
9.404171119 0.399115025997162 32.0
21.691853691 1.26879847049713 33.0
21.691853691 1.04929792881012 34.0
21.691853691 0.0768531933426857 35.0
9.7698705598 0.0343988612294197 36.0
8.389163949 0.331532955169678 37.0
0 0.0280192829668522 38.0
0 0.042431153357029 39.0
-11.682117462 0.726628243923187 40.0
-26.552860336 0.645253777503967 41.0
5.2068084659 0.730738699436188 42.0
-34.816033736 0.703264355659485 43.0
70.777694627 0.588556170463562 44.0
70.777694627 0.0944212079048157 45.0
6.2792507137 0.489041358232498 46.0
2.0202199257 0.383516699075699 47.0
207.45932653 0.0688408613204956 48.0
201.07441854 0.353439569473267 49.0
2.7113595555 0.0764047354459763 50.0
76.248753823 0.0795269981026649 51.0
75.031074251 0.176532492041588 52.0
61.474866434 0.0332070291042328 53.0
61.474866434 0.111341871321201 54.0
61.474866434 0.0429081059992313 55.0
0 3.41363477706909 56.0
84.554825306 0.0897314101457596 57.0
217.96980054 1.04462015628815 58.0
239.95193697 1.19475090503693 59.0
21.691853691 0.0311141647398472 60.0
0 0.119315072894096 61.0
181.49602651 1.01353251934052 0.0
181.49602651 1.74817669391632 1.0
180.3886483 1.55939936637878 2.0
175.59954316 4.90740013122559 3.0
0 6.30363082885742 4.0
0 3.9065854549408 5.0
-54.546413482 1.18348908424377 6.0
67.365071226 0.661837220191956 7.0
62.257346032 0.798753678798676 8.0
5.1077251935 0.921167373657227 9.0
100.12180651 1.09227514266968 10.0
88.185935667 0.260367870330811 11.0
87.987342451 0.141332715749741 12.0
87.987342451 0.119292691349983 13.0
87.987342451 0.121087700128555 14.0
0 0.231679558753967 15.0
0 0.145057648420334 16.0
0 1.80980324745178 17.0
63.640711715 1.23650586605072 18.0
-63.640711715 0.631073415279388 19.0
-63.640711715 0.749592959880829 20.0
-63.640711715 -0.0174516141414642 21.0
83.82712737 2.12239241600037 22.0
147.30849977 2.04437303543091 23.0
115.09205636 0.363917320966721 24.0
21.871035563 0.172961413860321 25.0
231.13562714 1.44043350219727 26.0
231.13562714 1.88868546485901 27.0
25.087742032 2.02356791496277 28.0
256.22336917 1.17294776439667 29.0
256.22336917 1.72360110282898 30.0
46.45617143 0.971220970153809 31.0
9.4583103119 0.300270885229111 32.0
20.186415655 1.90668857097626 33.0
20.186415655 0.977504849433899 34.0
20.186415655 0.13397428393364 35.0
8.9934529272 0.0369073674082756 36.0
7.6948462194 0.396966248750687 37.0
0 0.00832164660096169 38.0
0 0.219246178865433 39.0
-12.801594315 0.459142506122589 40.0
-26.240790486 0.624887943267822 41.0
5.1865228383 0.585416376590729 42.0
-34.027516715 0.679652392864227 43.0
89.101320077 0.763999104499817 44.0
89.101320077 0.0867034196853638 45.0
7.2979463306 0.527150511741638 46.0
2.5957956521 0.352638095617294 47.0
181.49602651 0.129073575139046 48.0
175.59954316 0.133217692375183 49.0
0 0.318317115306854 50.0
67.365071226 0.152030721306801 51.0
100.12180651 0.105987370014191 52.0
87.987342451 0.0873685851693153 53.0
87.987342451 0.0683665126562119 54.0
87.987342451 0.0653073787689209 55.0
0 1.9339964389801 56.0
83.82712737 0.0665892064571381 57.0
231.13562714 0.427485823631287 58.0
256.22336917 0.570990979671478 59.0
20.186415655 0.00604765024036169 60.0
0 0.109206572175026 61.0
204.8978633 1.17141282558441 0.0
204.8978633 2.65381455421448 1.0
203.30149704 1.86222243309021 2.0
197.87674004 4.03490781784058 3.0
10.917770374 8.20114707946777 4.0
10.917770374 6.23907423019409 5.0
-60.48139168 2.62792658805847 6.0
70.152327314 0.564447522163391 7.0
40.621753457 0.45815521478653 8.0
29.530573856 0.51393735408783 9.0
69.888851047 1.28010976314545 10.0
57.524608445 0.101537391543388 11.0
73.2413186 0.608414351940155 12.0
73.2413186 0.775603234767914 13.0
73.2413186 0.0744897425174713 14.0
0 0.0561836138367653 15.0
0 0.170377716422081 16.0
0 1.03019821643829 17.0
56.002103704 1.21928405761719 18.0
-56.002103704 1.38356482982635 19.0
-56.641888529 0.908366501331329 20.0
-56.708851839 -0.0759853646159172 21.0
79.249405044 1.14728307723999 22.0
157.86199551 0.567973434925079 23.0
121.65812903 0.800499498844147 24.0
24.437528011 0.149876743555069 25.0
237.11140056 0.900414526462555 26.0
237.11140056 1.34597849845886 27.0
30.971782594 1.60452795028687 28.0
268.08318315 1.17441213130951 29.0
268.08318315 1.72343635559082 30.0
47.22147371 1.08820569515228 31.0
9.5516636012 0.298429310321808 32.0
22.540553205 1.69124960899353 33.0
22.540553205 0.711879134178162 34.0
22.540553205 0.0758635848760605 35.0
10.257186205 0.0250372420996428 36.0
8.7588005382 0.409338504076004 37.0
0 0.0220269337296486 38.0
0 0.188760086894035 39.0
-13.282016636 0.587925136089325 40.0
-24.843915197 0.730321109294891 41.0
5.4568553752 0.396509766578674 42.0
-32.883455403 0.500887393951416 43.0
98.563647187 0.575064659118652 44.0
97.326919809 0.123174056410789 45.0
7.2769971851 0.637425780296326 46.0
2.6614685414 0.411653637886047 47.0
204.8978633 0.132083863019943 48.0
197.87674004 0.109416015446186 49.0
10.917770374 0.309545814990997 50.0
70.152327314 0.152711644768715 51.0
69.888851047 0.32673653960228 52.0
73.2413186 -0.00118216162081808 53.0
73.2413186 0.0711402595043182 54.0
73.2413186 0.119460344314575 55.0
0 6.13275480270386 56.0
79.249405044 0.12454267591238 57.0
237.11140056 0.609333753585815 58.0
268.08318315 0.552351653575897 59.0
22.540553205 -0.00227602384984493 60.0
0 0.0852414220571518 61.0
126.7780229 0.707726836204529 0.0
126.7780229 0.481438606977463 1.0
125.49446438 0.931508362293243 2.0
120.70535912 4.35455942153931 3.0
0 8.07654190063477 4.0
0 4.61797046661377 5.0
-56.103828004 3.76341676712036 6.0
68.221387082 0.809503555297852 7.0
62.072422164 1.33418667316437 8.0
6.1489649174 1.1026017665863 9.0
83.187955212 0.7638840675354 10.0
69.553175274 0.13524141907692 11.0
75.83878634 0.69280606508255 12.0
75.83878634 0.20551061630249 13.0
75.83878634 0.154500737786293 14.0
0 0.0820368826389313 15.0
0 0.169412612915039 16.0
0 1.66776180267334 17.0
51.042335107 1.45921730995178 18.0
-51.042335107 1.50759160518646 19.0
-51.042335107 0.446270048618317 20.0
-51.042335107 -0.0468510054051876 21.0
73.780788666 1.25377786159515 22.0
148.27645823 1.2225182056427 23.0
113.09847773 0.353742599487305 24.0
23.841701973 0.142135798931122 25.0
222.0572469 0.996738612651825 26.0
222.0572469 1.77724695205688 27.0
26.721046085 1.61912405490875 28.0
248.77829299 1.1681295633316 29.0
248.77829299 1.85414469242096 30.0
47.317214907 1.04696142673492 31.0
9.2691402386 0.443709850311279 32.0
22.73845356 1.30010807514191 33.0
22.73845356 1.17526507377625 34.0
22.73845356 0.104642696678638 35.0
10.315903361 0.0253839902579784 36.0
9.1158671205 0.39606785774231 37.0
0 0.00440913438796997 38.0
0 0.228148281574249 39.0
-11.458701402 0.893943071365356 40.0
-25.060217199 0.500588059425354 41.0
4.611039543 0.396936148405075 42.0
-31.968733698 0.749229192733765 43.0
88.101653189 0.555078506469727 44.0
88.101653189 0.116657339036465 45.0
6.5463749295 0.483908265829086 46.0
2.3173461897 0.412989795207977 47.0
126.7780229 0.0440507605671883 48.0
120.70535912 0.167875677347183 49.0
0 0.706060707569122 50.0
68.221387082 0.111488416790962 51.0
83.187955212 0.0443305410444736 52.0
75.83878634 0.0616084672510624 53.0
75.83878634 0.079012505710125 54.0
75.83878634 0.13498467206955 55.0
0 7.32595920562744 56.0
73.780788666 0.0836263671517372 57.0
222.0572469 0.359289616346359 58.0
248.77829299 0.789969205856323 59.0
22.73845356 0.0174264740198851 60.0
0 0.0789836943149567 61.0
173.31321854 1.13044536113739 0.0
173.31321854 1.41575396060944 1.0
172.10466173 1.73094642162323 2.0
167.31555654 5.13612747192383 3.0
0 6.6256856918335 4.0
0 4.22099828720093 5.0
-67.810349244 0.340411096811295 6.0
76.618983311 1.32059168815613 7.0
67.708106024 0.506949663162231 8.0
8.9108772867 0.519703209400177 9.0
94.025493863 1.03219258785248 10.0
81.789303841 0.0272574126720428 11.0
88.888388406 0.967755198478699 12.0
88.888388406 0.396297842264175 13.0
88.888388406 0.119695872068405 14.0
0 0.258046478033066 15.0
0 0.131442427635193 16.0
0 1.09809219837189 17.0
57.589482564 1.19506287574768 18.0
-57.589482564 0.914921045303345 19.0
-57.589482564 0.500319063663483 20.0
-57.589482564 0.00430859997868538 21.0
78.359902859 1.71948087215424 22.0
145.36526643 1.41243076324463 23.0
111.94629522 0.653083026409149 24.0
24.346379444 0.132092267274857 25.0
223.72516929 1.7946845293045 26.0
223.72516929 2.19492769241333 27.0
28.418841706 1.54530453681946 28.0
252.144011 0.609540522098541 29.0
252.144011 1.56657552719116 30.0
47.375888159 1.25155472755432 31.0
9.4874127778 0.407132804393768 32.0
20.770420295 1.08469617366791 33.0
20.770420295 1.30414927005768 34.0
20.770420295 0.138897776603699 35.0
9.0764441519 0.0304542817175388 36.0
7.9209460286 0.399820268154144 37.0
0 0.0449424386024475 38.0
0 0.182041868567467 39.0
-11.744229517 0.953380942344666 40.0
-24.598575196 0.654167532920837 41.0
4.6934886506 0.333956718444824 42.0
-32.138460358 0.378490775823593 43.0
86.710095476 0.502745032310486 44.0
86.710095476 0.126751527190208 45.0
6.2304075107 0.663033604621887 46.0
2.1971592082 0.418176710605621 47.0
173.31321854 0.241407692432404 48.0
167.31555654 0.162325277924538 49.0
0 0.366290926933289 50.0
76.618983311 0.174973875284195 51.0
94.025493863 0.219851031899452 52.0
88.888388406 0.0891752392053604 53.0
88.888388406 0.188392221927643 54.0
88.888388406 0.0859686583280563 55.0
0 0.657207846641541 56.0
78.359902859 0.080132469534874 57.0
223.72516929 0.763253390789032 58.0
252.144011 0.0570665299892426 59.0
20.770420295 -0.0148108266294003 60.0
0 0.0796838700771332 61.0
174.6171084 0.78731369972229 0.0
174.6171084 1.74037277698517 1.0
173.42771086 2.02923774719238 2.0
168.63860557 3.77450942993164 3.0
0 5.09628820419312 4.0
0 6.27720212936401 5.0
-73.281957524 0.610337257385254 6.0
81.560066446 0.62726366519928 7.0
54.687100945 1.13963508605957 8.0
26.8729655 1.22557163238525 9.0
100.80255242 0.914359629154205 10.0
86.528047496 0.368132442235947 11.0
88.701775882 0.481750011444092 12.0
88.701775882 0.543488740921021 13.0
88.701775882 0.130052208900452 14.0
0 0.0737891718745232 15.0
0 0.197162628173828 16.0
0 1.37835502624512 17.0
56.84001596 1.69823622703552 18.0
-56.84001596 2.44076824188232 19.0
-56.84001596 0.219939962029457 20.0
-56.84001596 0.0497791655361652 21.0
77.213755561 1.81193089485168 22.0
136.21134189 2.337162733078 23.0
106.77886061 0.502691507339478 24.0
21.88597672 0.165976732969284 25.0
213.42509745 2.10975551605225 26.0
213.42509745 2.35741639137268 27.0
25.938747972 1.65999269485474 28.0
239.36384543 0.891500353813171 29.0
239.36384543 2.05582904815674 30.0
47.100714163 1.32124626636505 31.0
9.213894629 0.613599061965942 32.0
20.373739601 1.94026875495911 33.0
20.373739601 0.615383267402649 34.0
20.373739601 0.125798150897026 35.0
8.8719637236 0.0347763448953629 36.0
7.6463373209 0.292830914258957 37.0
0 0.0322183147072792 38.0
0 0.267734438180923 39.0
-11.09576928 0.5179203748703 40.0
-24.648477557 0.810528099536896 41.0
5.5381821367 0.532295346260071 42.0
-32.560295825 0.595097362995148 43.0
82.385490066 0.91808009147644 44.0
82.385490066 0.0526296496391296 45.0
6.332476731 0.625162065029144 46.0
2.3363647366 0.334510862827301 47.0
174.6171084 0.12216380238533 48.0
168.63860557 0.470894247293472 49.0
0 0.224308550357819 50.0
81.560066446 0.278717875480652 51.0
100.80255242 0.301313191652298 52.0
88.701775882 0.0995186567306519 53.0
88.701775882 0.053486805409193 54.0
88.701775882 0.071477048099041 55.0
0 2.42880797386169 56.0
77.213755561 0.0489933714270592 57.0
213.42509745 1.0589075088501 58.0
239.36384543 0.400567322969437 59.0
20.373739601 -0.0117437383159995 60.0
0 0.105983920395374 61.0
192.8888337 0.854077100753784 0.0
192.8888337 1.17097556591034 1.0
191.82223255 1.5324227809906 2.0
186.35756183 4.45321750640869 3.0
15.668870502 7.77662563323975 4.0
15.668870502 4.59551239013672 5.0
-62.732771651 1.31082999706268 6.0
75.682449006 0.387042880058289 7.0
51.806794155 0.635670900344849 8.0
23.87565485 0.749602317810059 9.0
96.158824979 0.983195722103119 10.0
81.305023571 0.370317608118057 11.0
86.572281299 0.139250665903091 12.0
86.572281299 0.414431005716324 13.0
86.572281299 0.123566791415215 14.0
0 0.244938284158707 15.0
0 0.170541122555733 16.0
0 1.55386102199554 17.0
57.099880795 1.2049206495285 18.0
-57.099880795 1.59611308574677 19.0
-57.099880795 0.228422462940216 20.0
-57.099880795 0.0434001348912716 21.0
76.755257939 2.44801878929138 22.0
144.108578 2.32599997520447 23.0
110.70440487 0.458266198635101 24.0
10.694626166 0.118429243564606 25.0
220.86383594 1.6919310092926 26.0
220.86383594 1.38577950000763 27.0
25.207734788 1.48619759082794 28.0
246.07157073 0.306642770767212 29.0
246.07157073 1.5835200548172 30.0
46.476834391 1.18514692783356 31.0
9.4431835918 0.365045040845871 32.0
19.655377144 0.876077353954315 33.0
19.655377144 0.815409421920776 34.0
19.655377144 0.127366498112679 35.0
8.8716850379 0.0254760887473822 36.0
7.6353000899 0.346048414707184 37.0
0 0.0644341558218002 38.0
0 0.0920309275388718 39.0
-12.646290181 0.603021025657654 40.0
-25.997277366 0.579736173152924 41.0
5.0849626214 0.603052854537964 42.0
-33.41326137 0.718975305557251 43.0
89.733012345 0.914440155029297 44.0
89.733012345 0.0446431189775467 45.0
6.7083368611 0.690680503845215 46.0
2.7195446075 0.416627824306488 47.0
192.8888337 0.109578497707844 48.0
186.35756183 0.288386523723602 49.0
15.668870502 0.237235352396965 50.0
75.682449006 0.0535513944923878 51.0
96.158824979 0.225207179784775 52.0
86.572281299 0.0113405548036098 53.0
86.572281299 0.0536915957927704 54.0
86.572281299 0.0436109825968742 55.0
0 2.30272126197815 56.0
76.755257939 0.0596769452095032 57.0
220.86383594 0.535912215709686 58.0
246.07157073 0.467749655246735 59.0
19.655377144 0.0196722261607647 60.0
0 0.0844739228487015 61.0
168.29469661 0.840383768081665 0.0
168.29469661 0.987602531909943 1.0
167.13662988 1.06569719314575 2.0
162.34752479 2.66637754440308 3.0
0 4.72610521316528 4.0
0 5.54223728179932 5.0
-62.154070568 4.07060861587524 6.0
74.85083675 1.39643585681915 7.0
57.128654854 0.455260217189789 8.0
17.722181896 0.418067663908005 9.0
90.25818799 0.86477518081665 10.0
76.210037024 0.0619885250926018 11.0
79.953321897 0.908493280410767 12.0
79.953321897 0.738140225410461 13.0
79.953321897 0.0767079442739487 14.0
0 0.0687789022922516 15.0
0 0.17771877348423 16.0
0 1.57974398136139 17.0
63.363088378 1.22958052158356 18.0
-63.363088378 0.974106252193451 19.0
-63.363088378 0.366759687662125 20.0
-63.363088378 -0.0427040569484234 21.0
83.966254573 2.3736138343811 22.0
151.51674487 2.13444447517395 23.0
119.19172983 0.363512307405472 24.0
24.515515497 0.165048956871033 25.0
235.48299944 1.34316349029541 26.0
235.48299944 1.49595081806183 27.0
25.552528757 1.66893661022186 28.0
261.0355282 0.878310263156891 29.0
261.0355282 1.99772202968597 30.0
46.241192998 1.15525984764099 31.0
9.3980629491 0.574771344661713 32.0
20.603166195 1.28032910823822 33.0
20.603166195 1.00853836536407 34.0
20.603166195 0.137704938650131 35.0
8.7570119861 0.0271677915006876 36.0
7.4579586199 0.364011853933334 37.0
0 0.0271598808467388 38.0
0 0.0631386041641235 39.0
-11.575213299 0.661146461963654 40.0
-25.721110304 0.480005323886871 41.0
4.6470455413 0.419232726097107 42.0
-32.907775058 0.520798921585083 43.0
106.26030312 0.745087027549744 44.0
106.26030312 0.114389352500439 45.0
6.5492410246 0.633363246917725 46.0
2.5382578441 0.419735848903656 47.0
168.29469661 0.0602505803108215 48.0
162.34752479 -0.00873918272554874 49.0
0 0.222016379237175 50.0
74.85083675 0.215310513973236 51.0
90.25818799 0.152730613946915 52.0
79.953321897 0.0525653846561909 53.0
79.953321897 0.226405650377274 54.0
79.953321897 0.220246195793152 55.0
0 3.45621871948242 56.0
83.966254573 0.206515818834305 57.0
235.48299944 0.457643687725067 58.0
261.0355282 0.608737826347351 59.0
20.603166195 0.0424594357609749 60.0
0 0.0631292909383774 61.0
176.46493416 0.932765185832977 0.0
176.46493416 2.23283433914185 1.0
175.3803314 2.09094476699829 2.0
170.59122624 3.48852157592773 3.0
0 3.52604675292969 4.0
0 3.70532464981079 5.0
-60.405181992 0.312813371419907 6.0
69.370927711 1.71598875522614 7.0
55.516252121 0.591894149780273 8.0
13.85467559 0.478679567575455 9.0
89.89868193 1.2396936416626 10.0
77.242812591 0.494332402944565 11.0
80.435794396 0.298287510871887 12.0
80.435794396 0.124413102865219 13.0
80.435794396 0.130846157670021 14.0
0 0.272511601448059 15.0
0 0.171058222651482 16.0
0 1.50960159301758 17.0
59.078768919 0.85422694683075 18.0
-59.078768919 0.425926089286804 19.0
-59.078768919 0.961989462375641 20.0
-59.078768919 -0.0878229662775993 21.0
80.699987376 2.15721321105957 22.0
138.36478569 1.99397456645966 23.0
103.69106675 0.342734903097153 24.0
21.141735456 0.162598133087158 25.0
219.06477307 1.23996889591217 26.0
219.06477307 1.64885294437408 27.0
24.337347592 2.09711527824402 28.0
243.40212066 1.07876992225647 29.0
243.40212066 1.76948523521423 30.0
47.366074264 1.12856256961823 31.0
9.4830367228 0.493105739355087 32.0
21.621218457 1.82845175266266 33.0
21.621218457 0.926698088645935 34.0
21.621218457 0.0925280898809433 35.0
8.8639189342 0.0206168871372938 36.0
7.5126934672 0.435124337673187 37.0
0 0.0222322270274162 38.0
0 0.0751052275300026 39.0
-11.910780969 0.55889368057251 40.0
-24.664185452 0.764375388622284 41.0
5.5250429537 0.512350380420685 42.0
-32.977220864 0.481161564588547 43.0
91.097296125 0.808082103729248 44.0
91.097296125 0.130872711539268 45.0
6.7396251714 0.690840005874634 46.0
1.8479479934 0.390786230564117 47.0
176.46493416 0.136604443192482 48.0
170.59122624 0.387787759304047 49.0
0 0.184593349695206 50.0
69.370927711 0.142918214201927 51.0
89.89868193 0.408466011285782 52.0
80.435794396 0.138947814702988 53.0
80.435794396 0.0624579451978207 54.0
80.435794396 0.0415013656020164 55.0
0 3.72703838348389 56.0
80.699987376 0.137492179870605 57.0
219.06477307 0.351411312818527 58.0
243.40212066 0.41436505317688 59.0
21.621218457 -0.0100589087232947 60.0
0 0.0973941311240196 61.0
197.97432905 0.820210635662079 0.0
197.97432905 1.62065029144287 1.0
196.55427834 2.10107135772705 2.0
191.76520849 4.66751098632812 3.0
0 3.78918552398682 4.0
0 5.85256195068359 5.0
-47.65952205 2.43326640129089 6.0
57.142097894 1.75199866294861 7.0
48.006537628 0.352503806352615 8.0
9.1355602662 0.277166187763214 9.0
92.657908339 1.07935166358948 10.0
79.387790386 0.535236299037933 11.0
82.760693482 0.484961569309235 12.0
82.760693482 0.579808235168457 13.0
82.760693482 0.143491432070732 14.0
0 0.136515125632286 15.0
0 0.137880831956863 16.0
0 1.51423168182373 17.0
55.05379426 1.92826962471008 18.0
-55.05379426 1.90120792388916 19.0
-55.05379426 0.28292578458786 20.0
-55.05379426 -0.0567963160574436 21.0
77.29273142 1.7401202917099 22.0
139.13723697 1.30114710330963 23.0
98.677690026 0.345316886901855 24.0
19.752003135 0.110137313604355 25.0
216.42996839 1.70487678050995 26.0
216.42996839 2.4518346786499 27.0
24.439087897 2.70519924163818 28.0
240.86905629 1.05087971687317 29.0
240.86905629 1.95591735839844 30.0
47.168452151 1.41414225101471 31.0
9.5454699379 0.5533047914505 32.0
22.23893716 1.21100914478302 33.0
22.23893716 1.44562363624573 34.0
22.23893716 0.0680826604366302 35.0
10.335621276 0.0263197496533394 36.0
9.1163652981 0.345019072294235 37.0
0 0.0382276736199856 38.0
0 0.0918785855174065 39.0
-12.412199868 0.982024252414703 40.0
-25.956259738 0.337247729301453 41.0
4.4723426543 0.434137105941772 42.0
-32.845984869 0.636303663253784 43.0
114.53112295 0.793155372142792 44.0
114.53112295 0.077949732542038 45.0
7.1163693445 0.558105409145355 46.0
2.6498556464 0.391291737556458 47.0
197.97432905 0.0575513541698456 48.0
191.76520849 0.173252120614052 49.0
0 0.327758282423019 50.0
57.142097894 0.241122275590897 51.0
92.657908339 0.542657852172852 52.0
82.760693482 0.0221039485186338 53.0
82.760693482 0.185087040066719 54.0
82.760693482 0.104492172598839 55.0
0 6.09665489196777 56.0
77.29273142 0.171721249818802 57.0
216.42996839 0.891907572746277 58.0
240.86905629 0.444366663694382 59.0
22.23893716 0.0371212922036648 60.0
0 0.0726439654827118 61.0
142.80235089 0.956370055675507 0.0
142.80235089 0.589911818504333 1.0
141.36429962 0.949175477027893 2.0
136.57519434 3.9223256111145 3.0
0 4.44769287109375 4.0
0 2.86704254150391 5.0
-60.123597397 1.45964884757996 6.0
71.975548217 0.945838510990143 7.0
60.123296519 1.04745304584503 8.0
11.852251698 1.00367534160614 9.0
81.989656907 1.04989504814148 10.0
70.043900087 0.193892434239388 11.0
86.232822314 0.278460621833801 12.0
86.232822314 0.422462731599808 13.0
86.232822314 0.119978681206703 14.0
0 0.216790825128555 15.0
0 0.190570950508118 16.0
0 0.988786339759827 17.0
50.08157939 0.937541484832764 18.0
-50.08157939 1.14486455917358 19.0
-50.08157939 0.51438981294632 20.0
-50.08157939 -0.0797070115804672 21.0
72.260470743 2.5913610458374 22.0
134.78280818 2.83799481391907 23.0
103.60832016 0.487012147903442 24.0
21.452901241 0.0709649249911308 25.0
207.04327893 2.18777704238892 26.0
207.04327893 2.03924965858459 27.0
26.881122887 1.79716742038727 28.0
233.92440181 1.03772640228271 29.0
233.92440181 2.01469278335571 30.0
47.009026912 1.02456176280975 31.0
9.3279960162 0.415724903345108 32.0
22.178891353 1.61710226535797 33.0
22.178891353 1.1635479927063 34.0
22.178891353 0.106730550527573 35.0
10.576870807 0.0246434416621923 36.0
9.3171770468 0.363526999950409 37.0
0 0.0338522829115391 38.0
0 0.082511879503727 39.0
-13.498629219 0.872325360774994 40.0
-26.209002066 0.581239521503448 41.0
5.5152413026 0.465200901031494 42.0
-34.530786572 0.625325679779053 43.0
89.037649499 0.729169964790344 44.0
89.037649499 0.105499371886253 45.0
6.3842457777 0.647384583950043 46.0
2.7822383022 0.410165518522263 47.0
142.80235089 0.102673031389713 48.0
136.57519434 0.261916935443878 49.0
0 0.295839011669159 50.0
71.975548217 0.247518047690392 51.0
81.989656907 0.100083395838737 52.0
86.232822314 0.0655689612030983 53.0
86.232822314 0.0404812693595886 54.0
86.232822314 0.0888467207551003 55.0
0 4.28162813186646 56.0
72.260470743 0.17945608496666 57.0
207.04327893 1.13678729534149 58.0
233.92440181 1.1048389673233 59.0
22.178891353 -0.0165015142410994 60.0
0 0.085330605506897 61.0
169.20651906 0.883208155632019 0.0
169.20651906 1.82281923294067 1.0
168.08923209 1.64597296714783 2.0
163.30012687 3.94134378433228 3.0
0 4.81373500823975 4.0
0 4.56841421127319 5.0
-53.201423804 0.458570539951324 6.0
63.67332327 0.770740687847137 7.0
55.372697741 0.858095824718475 8.0
8.3006255293 0.737576067447662 9.0
91.263671948 1.00385558605194 10.0
78.038202089 0.205096274614334 11.0
81.377475775 0.767022967338562 12.0
81.377475775 0.296482622623444 13.0
81.377475775 0.114301174879074 14.0
0 0.0994034707546234 15.0
0 0.189568564295769 16.0
0 1.42311203479767 17.0
52.960247902 1.42876410484314 18.0
-52.960247902 1.15659773349762 19.0
-52.960247902 0.62887054681778 20.0
-52.960247902 -0.052284050732851 21.0
73.269720647 1.91391348838806 22.0
145.60750997 1.49023675918579 23.0
110.01583832 0.485621839761734 24.0
22.966346167 0.104595556855202 25.0
218.87723062 1.79538798332214 26.0
218.87723062 2.32323169708252 27.0
24.635947442 1.59975183010101 28.0
243.51317806 0.435781806707382 29.0
243.51317806 1.60507690906525 30.0
46.408214539 0.938988447189331 31.0
9.4423871118 0.516185700893402 32.0
20.309472745 1.91389429569244 33.0
20.309472745 0.808068811893463 34.0
20.309472745 0.103455565869808 35.0
9.2648080913 0.0367224961519241 36.0
7.9564224082 0.43454498052597 37.0
0 0.0750741958618164 38.0
0 0.242182374000549 39.0
-11.462192483 0.650967597961426 40.0
-23.137794104 0.430361300706863 41.0
4.7922383948 0.470781922340393 42.0
-30.747720701 0.312974750995636 43.0
103.63916772 0.824204325675964 44.0
103.63916772 0.128342419862747 45.0
6.3970948558 0.639932215213776 46.0
2.4987811517 0.363145917654037 47.0
169.20651906 0.110536888241768 48.0
163.30012687 0.168751612305641 49.0
0 0.416150242090225 50.0
63.67332327 0.117918893694878 51.0
91.263671948 0.20816770195961 52.0
81.377475775 0.126226648688316 53.0
81.377475775 0.0524349175393581 54.0
81.377475775 0.0685805231332779 55.0
0 4.38994550704956 56.0
73.269720647 0.00939331296831369 57.0
218.87723062 0.888475656509399 58.0
243.51317806 0.616008877754211 59.0
20.309472745 0.014554038643837 60.0
0 0.062253013253212 61.0
129.94874836 0.865103363990784 0.0
129.94874836 0.877371549606323 1.0
128.70873688 1.3255535364151 2.0
123.91963168 5.07382392883301 3.0
0 7.5727276802063 4.0
0 4.91357231140137 5.0
-50.975261114 0.426429629325867 6.0
63.688114388 0.819984555244446 7.0
65.088815752 0.563242137432098 8.0
-1.4007013637 0.529933512210846 9.0
89.246213992 0.86839896440506 10.0
76.322721763 0.121548607945442 11.0
80.832722745 0.630714058876038 12.0
80.832722745 0.84587162733078 13.0
80.832722745 0.0562335550785065 14.0
0 0.0133113637566566 15.0
0 0.19998824596405 16.0
0 1.08833360671997 17.0
62.618790597 1.24670386314392 18.0
-62.618790597 1.21615660190582 19.0
-62.618790597 0.500303387641907 20.0
-62.618790597 -0.0424663051962852 21.0
83.760510812 1.49831664562225 22.0
136.90734865 1.67506039142609 23.0
98.27435455 0.230237975716591 24.0
20.773288599 0.146440908312798 25.0
220.66785946 0.63209593296051 26.0
220.66785946 1.75605940818787 27.0
22.908333952 1.16368496417999 28.0
243.57619341 0.954019904136658 29.0
243.57619341 1.69327199459076 30.0
46.741664245 0.927558839321136 31.0
9.3450416774 0.571664333343506 32.0
21.141720214 1.65716075897217 33.0
21.141720214 0.710169196128845 34.0
21.141720214 0.115870535373688 35.0
9.4358229497 0.0429997183382511 36.0
8.0851054586 0.406069695949554 37.0
0 0.0434815436601639 38.0
0 0.370383471250534 39.0
-13.221940974 0.557654321193695 40.0
-25.69179615 0.534295082092285 41.0
4.8934517395 0.68917316198349 42.0
-33.279078959 0.612369775772095 43.0
108.21087294 0.641044199466705 44.0
108.21087294 0.0495657026767731 45.0
6.6803860199 0.675791263580322 46.0
2.6862304913 0.411186784505844 47.0
129.94874836 0.163002520799637 48.0
123.91963168 0.572950720787048 49.0
0 0.358892410993576 50.0
63.688114388 0.0807858556509018 51.0
89.246213992 0.182931870222092 52.0
80.832722745 0.0111477291211486 53.0
80.832722745 0.350932717323303 54.0
80.832722745 0.0877146497368813 55.0
0 2.23413515090942 56.0
83.760510812 0.0369005240499973 57.0
220.66785946 0.385969310998917 58.0
243.57619341 1.08092820644379 59.0
21.141720214 0.000940809026360512 60.0
0 0.087457075715065 61.0
197.35703181 0.924314558506012 0.0
197.35703181 1.89214289188385 1.0
195.95769933 1.5156717300415 2.0
191.16859409 4.01507759094238 3.0
0 8.42006015777588 4.0
0 3.4720516204834 5.0
-59.644208341 -0.0200568661093712 6.0
68.512276177 1.0226868391037 7.0
53.228761466 0.52956211566925 8.0
15.283514711 0.448241025209427 9.0
91.58018615 1.0700501203537 10.0
77.774327586 0.338833034038544 11.0
82.4244949 0.858751773834229 12.0
82.4244949 0.516866564750671 13.0
82.4244949 0.0674842819571495 14.0
0 0.0578377656638622 15.0
0 0.213843673467636 16.0
0 1.15725135803223 17.0
60.414981175 1.39693129062653 18.0
-60.414981175 1.15356528759003 19.0
-60.414981175 0.658098995685577 20.0
-60.414981175 -0.056304257363081 21.0
82.320971047 1.79433679580688 22.0
151.81000455 1.17826497554779 23.0
115.67976317 0.678194522857666 24.0
24.107621015 0.0991692617535591 25.0
234.1309756 1.20232021808624 26.0
234.1309756 1.8383606672287 27.0
28.390552974 2.36105823516846 28.0
262.52152857 1.24571895599365 29.0
262.52152857 1.70895397663116 30.0
46.491595185 1.03967595100403 31.0
9.4338238114 0.377460241317749 32.0
21.905989872 1.58435249328613 33.0
21.905989872 0.698398232460022 34.0
21.905989872 0.0922364518046379 35.0
9.7622925797 0.0231869500130415 36.0
8.4871249741 0.440059900283813 37.0
0 0.0637474656105042 38.0
0 0.168757677078247 39.0
-11.638180849 0.720985412597656 40.0
-24.358911507 0.676371216773987 41.0
4.6229139177 0.441006153821945 42.0
-31.727954171 0.368576139211655 43.0
94.593477055 0.722484350204468 44.0
94.593477055 0.0883335769176483 45.0
7.2048150916 0.808335959911346 46.0
2.368282304 0.425610542297363 47.0
197.35703181 0.13386133313179 48.0
191.16859409 0.00695787789300084 49.0
0 0.199903249740601 50.0
68.512276177 0.189397260546684 51.0
91.58018615 0.41424486041069 52.0
82.4244949 0.103975050151348 53.0
82.4244949 0.0424610786139965 54.0
82.4244949 0.124345920979977 55.0
0 3.70678210258484 56.0
82.320971047 0.0921865180134773 57.0
234.1309756 0.323547184467316 58.0
262.52152857 0.58479905128479 59.0
21.905989872 0.0183383449912071 60.0
0 0.0806185454130173 61.0
192.94534819 0.886978447437286 0.0
192.94534819 1.14722204208374 1.0
191.56972748 1.29762578010559 2.0
190.51373304 7.37711715698242 3.0
0 4.14153575897217 4.0
0 6.05395269393921 5.0
-76.533379324 5.6963267326355 6.0
88.740512956 0.603429615497589 7.0
67.345943206 1.41266179084778 8.0
21.394569749 1.53338897228241 9.0
97.722429686 1.23702955245972 10.0
84.696295834 0.833122670650482 11.0
84.318538813 1.15998113155365 12.0
84.318538813 0.478897094726562 13.0
84.318538813 0.133665055036545 14.0
0 0.279600381851196 15.0
0 0.0789758414030075 16.0
0 1.69077587127686 17.0
60.654206846 1.03346979618073 18.0
-60.654206846 1.39757823944092 19.0
-61.322825898 0.472271591424942 20.0
-61.391467839 0.0147470347583294 21.0
82.125166959 1.20557034015656 22.0
145.86622625 1.97195017337799 23.0
110.52223827 0.500241875648499 24.0
23.234443892 0.0841031968593597 25.0
227.99139321 0.671237647533417 26.0
227.99139321 1.09796118736267 27.0
30.431156299 1.589888215065 28.0
258.42254951 0.915602028369904 29.0
258.42254951 1.52890074253082 30.0
47.4121207 0.97423130273819 31.0
9.4507409643 0.443251818418503 32.0
20.733699121 1.71970200538635 33.0
20.733699121 0.837188839912415 34.0
20.733699121 0.130783334374428 35.0
9.0603882759 0.0520148202776909 36.0
7.9157489887 0.358330547809601 37.0
0 0.010206013917923 38.0
0 0.0967765152454376 39.0
-12.239784586 0.744505882263184 40.0
-24.830024008 0.640220582485199 41.0
5.081089649 0.472880452871323 42.0
-32.664721565 0.650483667850494 43.0
91.282753155 0.591699659824371 44.0
91.282753155 0.131206482648849 45.0
7.1689302127 0.500494003295898 46.0
2.6302615655 0.398606181144714 47.0
192.94534819 0.149588569998741 48.0
190.51373304 0.101286791265011 49.0
0 0.553184688091278 50.0
88.740512956 0.217218086123466 51.0
97.722429686 0.727247834205627 52.0
84.318538813 0.329796522855759 53.0
84.318538813 0.125566527247429 54.0
84.318538813 0.147631496191025 55.0
0 2.52496767044067 56.0
82.125166959 0.0466422326862812 57.0
227.99139321 0.172992572188377 58.0
258.42254951 0.809618651866913 59.0
20.733699121 -0.00684988917782903 60.0
0 0.118534982204437 61.0
212.22496621 0.855757653713226 0.0
212.22496621 0.472877472639084 1.0
211.31744014 0.84473717212677 2.0
205.85279112 2.46751403808594 3.0
16.156536387 4.16724061965942 4.0
16.156536387 3.45933413505554 5.0
-51.631399946 0.307483315467834 6.0
65.454397238 0.609525084495544 7.0
37.522079411 0.607166171073914 8.0
27.932317827 0.532289981842041 9.0
59.822448085 1.2357314825058 10.0
45.318461261 0.522686541080475 11.0
59.768392148 0.485428184270859 12.0
59.768392148 0.558438122272491 13.0
59.768392148 0.0676339492201805 14.0
29.573486731 0.0302215479314327 15.0
29.573486731 0.187533721327782 16.0
0 0.958128929138184 17.0
53.075876923 1.89321160316467 18.0
-53.075876923 1.79187607765198 19.0
-53.075876923 0.917602479457855 20.0
-53.075876923 -0.0306283216923475 21.0
72.949992577 2.15731763839722 22.0
150.43676989 1.61725282669067 23.0
113.07687145 0.389736294746399 24.0
20.007644154 0.188892751932144 25.0
223.38676247 1.17417359352112 26.0
223.38676247 1.81291711330414 27.0
22.332263666 3.09910082817078 28.0
245.71902613 1.1729199886322 29.0
245.71902613 1.78422069549561 30.0
47.673366062 1.01226389408112 31.0
9.4606100939 0.323295205831528 32.0
19.874115654 1.76085138320923 33.0
19.874115654 0.626538753509521 34.0
19.874115654 0.156061142683029 35.0
8.5104332548 0.0211555957794189 36.0
7.1075615093 0.370914936065674 37.0
0 0.0234735663980246 38.0
0 0.17704176902771 39.0
-11.114255551 0.641517460346222 40.0
-26.292349584 0.454311966896057 41.0
5.5158856109 0.377325862646103 42.0
-34.055064746 0.331004738807678 43.0
92.112666711 0.540736138820648 44.0
92.112666711 0.0654803812503815 45.0
6.8138014386 0.498234868049622 46.0
2.2590592038 0.442180693149567 47.0
212.22496621 0.0571186095476151 48.0
205.85279112 0.135361641645432 49.0
16.156536387 0.0970044061541557 50.0
65.454397238 0.282832622528076 51.0
59.822448085 0.575243055820465 52.0
59.768392148 0.0755648240447044 53.0
59.768392148 0.0674975365400314 54.0
59.768392148 0.0539980083703995 55.0
0 3.11926460266113 56.0
72.949992577 0.0692373961210251 57.0
223.38676247 0.271544337272644 58.0
245.71902613 0.310175031423569 59.0
19.874115654 0.0164065882563591 60.0
0 0.112312398850918 61.0
170.63189639 0.918550908565521 0.0
170.63189639 0.857141733169556 1.0
169.1712713 1.05915856361389 2.0
164.38220145 3.87340235710144 3.0
0 5.43271207809448 4.0
0 4.43589115142822 5.0
-59.591408172 2.80915498733521 6.0
70.381396273 0.468929797410965 7.0
42.006573367 0.753858089447021 8.0
28.374822906 0.762903451919556 9.0
84.090462149 1.05714678764343 10.0
69.98601363 0.0750981345772743 11.0
79.502085228 0.763932049274445 12.0
79.502085228 0.554619073867798 13.0
79.502085228 0.152155727148056 14.0
0 0.153937503695488 15.0
0 0.161996871232986 16.0
0 1.70008683204651 17.0
52.711748661 0.861312866210938 18.0
-52.711748661 0.647504031658173 19.0
-52.711748661 1.05503046512604 20.0
-52.711748661 -0.0513911135494709 21.0
74.407528396 1.12466061115265 22.0
153.4387336 0.794001996517181 23.0
116.75193329 0.502934336662292 24.0
22.82906067 0.169391289353371 25.0
227.84626199 1.01821744441986 26.0
227.84626199 1.3953298330307 27.0
28.617933226 1.8028347492218 28.0
256.46419522 0.95440149307251 29.0
256.46419522 1.78031969070435 30.0
46.942709044 1.12221395969391 31.0
9.3602346597 0.438864827156067 32.0
21.695779735 1.61737418174744 33.0
21.695779735 0.876997828483582 34.0
21.695779735 0.143996462225914 35.0
9.9950037982 0.0414855033159256 36.0
8.8181619378 0.380557954311371 37.0
0 -0.000735830515623093 38.0
0 0.0632914528250694 39.0
-11.937377912 0.568988919258118 40.0
-26.109777166 0.539260029792786 41.0
5.1366549036 0.437745749950409 42.0
-33.80162391 0.35420361161232 43.0
100.7445469 0.804507434368134 44.0
100.7445469 0.0653919875621796 45.0
6.6386158985 0.487800270318985 46.0
2.0297274273 0.337161034345627 47.0
170.63189639 0.130939781665802 48.0
164.38220145 0.125053510069847 49.0
0 0.137703210115433 50.0
70.381396273 0.113287732005119 51.0
84.090462149 0.168952077627182 52.0
79.502085228 0.0470245741307735 53.0
79.502085228 0.177717506885529 54.0
79.502085228 0.0533352196216583 55.0
0 4.99299097061157 56.0
74.407528396 0.0769814699888229 57.0
227.84626199 0.497633099555969 58.0
256.46419522 0.758926212787628 59.0
21.695779735 0.0113426223397255 60.0
0 0.16219899058342 61.0
121.84698054 0.761508345603943 0.0
121.84698054 1.59212744235992 1.0
120.75731924 2.40147042274475 2.0
115.96821411 6.41115760803223 3.0
0 4.02240943908691 4.0
0 4.35092973709106 5.0
-53.254393933 0.642149209976196 6.0
63.218180482 1.06288290023804 7.0
61.613019349 0.536014378070831 8.0
1.6051611323 0.586166381835938 9.0
101.45470047 0.816125333309174 10.0
90.523553555 0.487539768218994 11.0
92.010949828 0.473095178604126 12.0
92.010949828 0.264794945716858 13.0
92.010949828 0.145297974348068 14.0
0 0.249037384986877 15.0
0 0.189289808273315 16.0
0 1.52735674381256 17.0
60.810566074 1.09066915512085 18.0
-60.810566074 1.01324212551117 19.0
-60.810566074 0.413457006216049 20.0
-60.810566074 -0.0668452829122543 21.0
84.15168778 1.66081595420837 22.0
149.87445044 1.76011288166046 23.0
115.48998651 0.68298864364624 24.0
21.164511377 0.248591184616089 25.0
234.02613822 1.31836760044098 26.0
234.02613822 1.95383012294769 27.0
29.227684931 1.00269317626953 28.0
263.25382315 0.289730221033096 29.0
263.25382315 1.30318510532379 30.0
46.481600164 1.08093106746674 31.0
9.1692775009 0.42360907793045 32.0
23.341121705 1.88768649101257 33.0
23.341121705 1.19248592853546 34.0
23.341121705 0.146212667226791 35.0
10.167130606 0.0371878370642662 36.0
9.0087550309 0.314286202192307 37.0
0 0.0162866860628128 38.0
0 0.132019758224487 39.0
-11.20959113 0.962464272975922 40.0
-25.271601055 0.443760097026825 41.0
5.1073270994 0.319417119026184 42.0
-32.594126191 0.183820739388466 43.0
98.242970223 0.945412993431091 44.0
98.242970223 0.0442603453993797 45.0
6.5989886106 0.800043225288391 46.0
1.7900574521 0.393200874328613 47.0
121.84698054 0.0191716253757477 48.0
115.96821411 0.58855801820755 49.0
0 0.530626833438873 50.0
63.218180482 0.165041282773018 51.0
101.45470047 0.579323649406433 52.0
92.010949828 0.035685483366251 53.0
92.010949828 0.0807199031114578 54.0
92.010949828 0.266843944787979 55.0
0 0.623149573802948 56.0
84.15168778 0.272825986146927 57.0
234.02613822 0.96464741230011 58.0
263.25382315 0.0634610205888748 59.0
23.341121705 -0.00628076028078794 60.0
0 0.110575065016747 61.0
173.5299409 0.837865889072418 0.0
173.5299409 0.666527986526489 1.0
172.00437627 0.866349458694458 2.0
166.53970552 3.23454260826111 3.0
9.1024688443 4.14190292358398 4.0
9.1024688443 2.47840404510498 5.0
-47.929862072 0.145275771617889 6.0
62.040281506 0.459845274686813 7.0
37.011744302 0.89612740278244 8.0
25.028537204 1.22907471656799 9.0
86.443592689 1.03419375419617 10.0
75.267977754 0.215512558817863 11.0
88.188516068 0.151227355003357 12.0
88.188516068 0.058938629925251 13.0
88.188516068 0.141237825155258 14.0
0 0.220384940505028 15.0
0 0.172261700034142 16.0
0 1.09752666950226 17.0
57.982534624 1.2920138835907 18.0
-57.982534624 0.829057395458221 19.0
-57.982534624 0.269705504179001 20.0
-57.982534624 0.0561267957091331 21.0
79.644376982 0.649463772773743 22.0
132.13597993 0.394380748271942 23.0
99.024281121 0.404892414808273 24.0
23.114445322 0.260950982570648 25.0
211.78035691 0.663199186325073 26.0
211.78035691 1.19705772399902 27.0
29.111989618 1.08183288574219 28.0
240.89234653 0.824817538261414 29.0
240.89234653 1.27390992641449 30.0
46.605666302 1.16487920284271 31.0
9.4100948361 0.648922920227051 32.0
21.661842358 1.505042552948 33.0
21.661842358 0.894296884536743 34.0
21.661842358 0.130858674645424 35.0
8.6730360699 0.0497046262025833 36.0
7.5086673257 0.380279809236526 37.0
0 0.0273516476154327 38.0
0 0.151975631713867 39.0
-11.445499739 0.345271348953247 40.0
-23.971204883 0.666735231876373 41.0
4.482286165 0.457756638526917 42.0
-30.882578348 0.272530615329742 43.0
74.8318924 0.808226585388184 44.0
74.8318924 0.0961209312081337 45.0
6.7800813628 0.599633812904358 46.0
2.8936325787 0.309463053941727 47.0
173.5299409 0.0487677827477455 48.0
166.53970552 0.409541636705399 49.0
9.1024688443 0.178795620799065 50.0
62.040281506 0.254833430051804 51.0
86.443592689 0.284425586462021 52.0
88.188516068 0.00727126188576221 53.0
88.188516068 0.0397450178861618 54.0
88.188516068 0.0634347051382065 55.0
0 2.59420609474182 56.0
79.644376982 0.242203816771507 57.0
211.78035691 0.420740842819214 58.0
240.89234653 0.601010799407959 59.0
21.661842358 0.0106700025498867 60.0
0 0.0996483191847801 61.0
161.95777325 0.990004479885101 0.0
161.95777325 0.903188169002533 1.0
160.67345767 1.83876097202301 2.0
155.88435249 3.79121971130371 3.0
0 6.98320865631104 4.0
0 4.20515823364258 5.0
-69.605459314 0.346984177827835 6.0
79.699852246 0.757275938987732 7.0
59.980898558 0.961574494838715 8.0
19.718953688 0.84368884563446 9.0
86.947108974 0.959339261054993 10.0
73.542464799 0.0266994759440422 11.0
80.271581933 0.821865677833557 12.0
80.271581933 0.42791336774826 13.0
80.271581933 0.129691675305367 14.0
0 0.0856917351484299 15.0
0 0.201238378882408 16.0
0 1.36420023441315 17.0
56.38982179 1.23340618610382 18.0
-56.38982179 0.857374012470245 19.0
-56.38982179 0.529532670974731 20.0
-56.38982179 0.00456266477704048 21.0
77.863117977 1.77723169326782 22.0
139.37184013 1.02887165546417 23.0
104.47663111 0.812980055809021 24.0
24.314596554 0.0895062536001205 25.0
217.23495811 0.917721927165985 26.0
217.23495811 1.82625842094421 27.0
23.424637353 1.3820503950119 28.0
240.65959546 1.01938784122467 29.0
240.65959546 1.94948124885559 30.0
46.432976116 1.00584077835083 31.0
9.2819975849 0.450208783149719 32.0
21.473296187 1.42459440231323 33.0
21.473296187 1.02780652046204 34.0
21.473296187 0.136290714144707 35.0
9.5586940065 0.0436903908848763 36.0
8.2616305414 0.402430802583694 37.0
0 0.0181851945817471 38.0
0 0.210456728935242 39.0
-11.080904556 0.671578049659729 40.0
-24.446761389 0.582557201385498 41.0
4.6533693599 0.467422127723694 42.0
-31.942766086 0.462436378002167 43.0
86.79053112 0.888347625732422 44.0
86.79053112 0.100274257361889 45.0
5.9602135806 0.811077892780304 46.0
2.1780297095 0.384730458259583 47.0
161.95777325 0.11333079636097 48.0
155.88435249 0.342310637235641 49.0
0 0.313903510570526 50.0
79.699852246 0.266262292861938 51.0
86.947108974 0.176969468593597 52.0
80.271581933 0.0598974116146564 53.0
80.271581933 0.104166738688946 54.0
80.271581933 0.0312541276216507 55.0
0 4.18757104873657 56.0
77.863117977 0.0925769805908203 57.0
217.23495811 0.722632229328156 58.0
240.65959546 1.00565075874329 59.0
21.473296187 -0.00162448920309544 60.0
0 0.102346315979958 61.0
172.0502004 0.896714746952057 0.0
172.0502004 1.74910068511963 1.0
170.95555651 2.12496113777161 2.0
165.49088582 4.71497488021851 3.0
0.26026231599 5.31900691986084 4.0
0.26026231599 4.28802871704102 5.0
-66.55820507 0.410296082496643 6.0
79.516518223 0.960042476654053 7.0
45.95437218 0.434631943702698 8.0
33.562146043 0.327242195606232 9.0
73.008942017 1.18713533878326 10.0
59.487688018 0.265732109546661 11.0
74.036450202 0.372440725564957 12.0
74.036450202 0.239550203084946 13.0
74.036450202 0.115259438753128 14.0
0 0.116943649947643 15.0
0 0.240962281823158 16.0
0 1.1028608083725 17.0
56.106789476 1.12940633296967 18.0
-56.106789476 0.738432586193085 19.0
-56.106789476 0.340883851051331 20.0
-56.106789476 0.0391758382320404 21.0
75.9562213 2.07406306266785 22.0
140.40414017 2.38114190101624 23.0
106.77809147 0.325336635112762 24.0
22.201773628 0.0801801532506943 25.0
216.36036147 1.19146001338959 26.0
216.36036147 1.49239873886108 27.0
28.706809261 1.5468887090683 28.0
245.06717074 0.553552865982056 29.0
245.06717074 1.48774993419647 30.0
46.145296633 1.24125933647156 31.0
9.2834361743 0.481372803449631 32.0
19.849431824 1.62222349643707 33.0
19.849431824 0.775259077548981 34.0
19.849431824 0.102948315441608 35.0
9.0506376442 0.0288947317749262 36.0
7.6721381216 0.400033175945282 37.0
0 0.0423203036189079 38.0
0 0.288184732198715 39.0
-12.17206888 0.500113070011139 40.0
-25.615296716 0.606488108634949 41.0
4.8915926952 0.577840507030487 42.0
-32.982619188 0.854580521583557 43.0
98.393163232 0.6596799492836 44.0
98.393163232 0.131850987672806 45.0
7.3514163856 0.535622000694275 46.0
2.6162848864 0.39701846241951 47.0
172.0502004 0.172234207391739 48.0
165.49088582 0.650321125984192 49.0
0.26026231599 0.301292359828949 50.0
79.516518223 0.0844746753573418 51.0
73.008942017 0.32721284031868 52.0
74.036450202 0.066397026181221 53.0
74.036450202 0.0731214135885239 54.0
74.036450202 0.190362632274628 55.0
0 7.39289569854736 56.0
75.9562213 -0.0788546055555344 57.0
216.36036147 0.597090125083923 58.0
245.06717074 0.436872690916061 59.0
19.849431824 -0.0131879318505526 60.0
0 0.0946756228804588 61.0
196.94770412 1.06473171710968 0.0
196.94770412 0.923701047897339 1.0
195.92677233 0.899134278297424 2.0
191.13770238 5.66865205764771 3.0
0 7.89872980117798 4.0
0 5.42813587188721 5.0
-74.067421024 0.6262167096138 6.0
84.139163137 0.836441397666931 7.0
45.470326377 0.928138852119446 8.0
38.66883676 0.861597061157227 9.0
97.912779043 1.0776275396347 10.0
84.308329217 0.0537457130849361 11.0
87.926792545 0.94891768693924 12.0
87.926792545 0.826962232589722 13.0
87.926792545 0.0730026513338089 14.0
0 0.212749242782593 15.0
0 0.193418145179749 16.0
0 1.12449038028717 17.0
60.416261166 1.32132136821747 18.0
-60.416261166 1.19681966304779 19.0
-60.416261166 0.287102103233337 20.0
-60.416261166 0.151556521654129 21.0
82.039059418 0.790245234966278 22.0
140.15299451 0.52527117729187 23.0
109.79389395 0.656374037265778 24.0
23.883677018 0.161890938878059 25.0
222.19205393 0.940879821777344 26.0
222.19205393 2.09592175483704 27.0
28.089081135 1.57507836818695 28.0
250.28113506 0.542448163032532 29.0
250.28113506 1.65553843975067 30.0
47.010879976 1.03451859951019 31.0
9.4070794515 0.493868350982666 32.0
21.622798252 1.26071190834045 33.0
21.622798252 1.07098865509033 34.0
21.622798252 0.147006794810295 35.0
9.4079344866 0.0323826298117638 36.0
7.9862924356 0.351332902908325 37.0
0 0.0245250202715397 38.0
0 0.205910667777061 39.0
-12.891129672 0.720057249069214 40.0
-26.081811384 0.398013979196548 41.0
4.7991047312 0.57585346698761 42.0
-33.43344994 0.305109143257141 43.0
93.604597384 0.797138571739197 44.0
93.604597384 0.0733971521258354 45.0
7.6262545988 0.688155651092529 46.0
2.8966168184 0.409569889307022 47.0
196.94770412 0.148523241281509 48.0
191.13770238 0.0864017009735107 49.0
0 0.241195306181908 50.0
84.139163137 0.194818317890167 51.0
97.912779043 0.145329266786575 52.0
87.926792545 0.0270646195858717 53.0
87.926792545 0.299110591411591 54.0
87.926792545 0.209395870566368 55.0
0 2.26721882820129 56.0
82.039059418 -0.0273485668003559 57.0
222.19205393 0.408428251743317 58.0
250.28113506 0.475654453039169 59.0
21.622798252 -0.00368141941726208 60.0
0 0.0849780738353729 61.0
173.07440313 0.958040356636047 0.0
173.07440313 1.09096109867096 1.0
172.07505308 1.07260751724243 2.0
166.61038225 7.52047729492188 3.0
7.3181101148 7.32794332504272 4.0
7.3181101148 3.77311611175537 5.0
-49.497891306 0.271254360675812 6.0
60.267721361 0.594651937484741 7.0
36.978719752 0.840175747871399 8.0
23.289001609 0.923368752002716 9.0
96.058426232 1.06699526309967 10.0
83.216834228 0.259673863649368 11.0
79.303458114 0.146533384919167 12.0
79.303458114 0.0736971944570541 13.0
79.303458114 0.173155412077904 14.0
0 0.327657341957092 15.0
0 0.0797230005264282 16.0
0 0.912520229816437 17.0
55.971430706 1.33021819591522 18.0
-55.971430706 1.39694428443909 19.0
-55.971430706 0.704511940479279 20.0
-55.971430706 -0.122234225273132 21.0
77.637159799 1.24495649337769 22.0
135.73372703 1.03227508068085 23.0
102.36654881 0.744195163249969 24.0
23.602759517 0.0653870701789856 25.0
213.37088683 0.692705392837524 26.0
213.37088683 1.01027631759644 27.0
29.255581499 1.22270131111145 28.0
242.62646833 0.606966257095337 29.0
242.62646833 1.99224472045898 30.0
46.469834449 1.15683829784393 31.0
9.4222374291 0.37280085682869 32.0
21.665729093 2.20966029167175 33.0
21.665729093 0.745576977729797 34.0
21.665729093 0.136407181620598 35.0
9.2402495074 0.0576951988041401 36.0
8.0148158746 0.379626035690308 37.0
0 0.0805495828390121 38.0
0 0.203861802816391 39.0
-10.239808512 0.677681267261505 40.0
-24.260723943 0.668148159980774 41.0
5.3430687195 0.478569507598877 42.0
-32.156275555 0.911965370178223 43.0
71.903899396 0.5348219871521 44.0
71.903899396 0.103549115359783 45.0
6.1859010729 0.35831406712532 46.0
2.0185373757 0.296792328357697 47.0
173.07440313 0.139198362827301 48.0
166.61038225 -0.0164702404290438 49.0
7.3181101148 0.202589064836502 50.0
60.267721361 0.269478142261505 51.0
96.058426232 0.160150468349457 52.0
79.303458114 -0.00279300985857844 53.0
79.303458114 0.0501798316836357 54.0
79.303458114 0.0312409978359938 55.0
0 8.34347343444824 56.0
77.637159799 0.247227713465691 57.0
213.37088683 0.623108685016632 58.0
242.62646833 0.744510352611542 59.0
21.665729093 0.0195013582706451 60.0
0 0.077219620347023 61.0
191.24605775 1.0030255317688 0.0
191.24605775 1.17866516113281 1.0
190.0792202 1.7886244058609 2.0
185.29011499 6.8537335395813 3.0
0 4.92870426177979 4.0
0 2.58891439437866 5.0
-64.812063742 1.44768643379211 6.0
78.471280966 0.452909857034683 7.0
44.209591337 0.84079521894455 8.0
34.26168963 0.815204799175262 9.0
86.233023714 0.965590715408325 10.0
73.077564673 0.141513377428055 11.0
81.763104051 0.549161612987518 12.0
81.763104051 0.530432343482971 13.0
81.763104051 0.140868946909904 14.0
0 0.178600296378136 15.0
0 0.211332321166992 16.0
0 1.27838277816772 17.0
50.117064539 1.80970764160156 18.0
-50.117064539 1.65689444541931 19.0
-50.117064539 0.681478083133698 20.0
-50.117064539 0.0608786083757877 21.0
72.538742864 1.84860742092133 22.0
143.9867873 2.80762672424316 23.0
111.48579833 0.115594297647476 24.0
24.675172949 0.357987493276596 25.0
216.52553016 0.865889430046082 26.0
216.52553016 1.10732650756836 27.0
29.683621944 1.48351538181305 28.0
246.20915211 0.560464262962341 29.0
246.20915211 1.65928423404694 30.0
47.212111874 0.847691416740417 31.0
9.3770822574 0.527069449424744 32.0
22.421678325 1.45923137664795 33.0
22.421678325 0.964060068130493 34.0
22.421678325 0.125748515129089 35.0
9.8972307849 0.033838301897049 36.0
8.6371198909 0.373902380466461 37.0
0 0.0171639658510685 38.0
0 0.2944455742836 39.0
-11.628171026 0.484543442726135 40.0
-24.775572891 0.470272570848465 41.0
5.1153660515 0.509331047534943 42.0
-32.08345438 0.530608654022217 43.0
98.625703645 0.501990795135498 44.0
98.625703645 0.0961293280124664 45.0
6.9680656738 0.525736093521118 46.0
2.4157861035 0.461995601654053 47.0
191.24605775 0.109614320099354 48.0
185.29011499 0.338318079710007 49.0
0 0.15095679461956 50.0
78.471280966 0.21063756942749 51.0
86.233023714 0.127659738063812 52.0
81.763104051 0.0674240961670876 53.0
81.763104051 0.163778334856033 54.0
81.763104051 -0.00262938626110554 55.0
0 7.01383972167969 56.0
72.538742864 -0.0186946801841259 57.0
216.52553016 0.0515561290085316 58.0
246.20915211 0.973288536071777 59.0
22.421678325 -0.0132184335961938 60.0
0 0.104558393359184 61.0
149.18550935 0.775726199150085 0.0
149.18550935 1.22133934497833 1.0
148.13123349 1.15016973018646 2.0
143.34212833 3.73480033874512 3.0
0 7.47158622741699 4.0
0 4.13888072967529 5.0
-53.807737451 0.394958078861237 6.0
63.205555267 1.07250106334686 7.0
54.177666072 0.744180500507355 8.0
9.0278891949 0.746284484863281 9.0
81.258572833 1.03071105480194 10.0
69.062453902 0.260603576898575 11.0
75.147224069 0.438536465167999 12.0
75.147224069 0.406582713127136 13.0
75.147224069 0.0722900927066803 14.0
0 0.045241005718708 15.0
0 0.214695334434509 16.0
0 1.3287889957428 17.0
51.621179927 1.04912543296814 18.0
-51.621179927 0.79834657907486 19.0
-51.621179927 0.264554053544998 20.0
-51.621179927 0.188444867730141 21.0
74.295324228 2.26479625701904 22.0
146.3576066 2.3242564201355 23.0
114.30167296 0.427192151546478 24.0
22.863560462 0.341591536998749 25.0
220.65293082 1.4039843082428 26.0
220.65293082 1.34169733524323 27.0
29.751151131 1.67034351825714 28.0
250.40408195 0.97949892282486 29.0
250.40408195 1.78756952285767 30.0
47.228577837 0.986666023731232 31.0
9.4034635666 0.469916641712189 32.0
22.6741443 0.929219424724579 33.0
22.6741443 0.985059380531311 34.0
22.6741443 0.0946780815720558 35.0
9.9394118318 0.0303746275603771 36.0
8.559747664 0.34867262840271 37.0
0 0.0478640757501125 38.0
0 0.276655197143555 39.0
-11.917304186 0.801051378250122 40.0
-24.717465162 0.630370855331421 41.0
5.2230308326 0.558124661445618 42.0
-32.432014656 0.552872359752655 43.0
88.828900294 0.759467482566833 44.0
88.828900294 0.0934467241168022 45.0
7.1519181061 0.58853542804718 46.0
2.8679360415 0.364352524280548 47.0
149.18550935 0.129269674420357 48.0
143.34212833 0.233831167221069 49.0
0 0.772642970085144 50.0
63.205555267 0.200875028967857 51.0
81.258572833 0.387508273124695 52.0
75.147224069 0.0515253394842148 53.0
75.147224069 0.0282471030950546 54.0
75.147224069 0.0855067372322083 55.0
0 4.03229570388794 56.0
74.295324228 -0.0463691651821136 57.0
220.65293082 0.729725241661072 58.0
250.40408195 0.735804617404938 59.0
22.6741443 -0.00940433982759714 60.0
0 0.0907695665955544 61.0
146.866115 0.829577267169952 0.0
146.866115 0.533565878868103 1.0
145.66050235 0.729950368404388 2.0
140.58621836 3.01692771911621 3.0
0 8.0043363571167 4.0
0 6.22620964050293 5.0
-50.16165532 0.471343070268631 6.0
62.417838204 0.901530742645264 7.0
37.236959481 0.593969762325287 8.0
25.180878723 0.442650973796844 9.0
95.523016388 0.930965602397919 10.0
82.110830432 0.3088039457798 11.0
85.686293621 0.917986035346985 12.0
85.686293621 0.589654505252838 13.0
85.686293621 0.0399526208639145 14.0
0 0.102180108428001 15.0
0 0.239339023828506 16.0
0 1.20015943050385 17.0
52.986015164 1.02277970314026 18.0
-52.986015164 0.794994294643402 19.0
-52.986015164 0.305566996335983 20.0
-52.986015164 0.00829737633466721 21.0
77.278286303 1.41802644729614 22.0
152.99672364 1.25000953674316 23.0
115.94441707 0.643940091133118 24.0
22.6710607 0.127593383193016 25.0
230.27500994 1.10158371925354 26.0
230.27500994 2.46486711502075 27.0
23.402335988 2.45353722572327 28.0
253.67734593 1.27630829811096 29.0
253.67734593 1.97118353843689 30.0
46.420524846 1.03880167007446 31.0
9.3039531238 0.690598249435425 32.0
24.292271139 1.18571746349335 33.0
24.292271139 1.01011288166046 34.0
24.292271139 0.10451877117157 35.0
10.933669529 0.0354625284671783 36.0
9.6855277017 0.32028391957283 37.0
0 0.0106939747929573 38.0
0 0.217215716838837 39.0
-11.209721285 0.88175892829895 40.0
-23.347168741 0.648055016994476 41.0
4.7375379564 0.523725509643555 42.0
-30.84741431 0.399261832237244 43.0
107.17818078 0.800570666790009 44.0
107.17818078 0.0952071398496628 45.0
6.9931623324 0.542329251766205 46.0
2.5015182992 0.375218898057938 47.0
146.866115 0.0793366432189941 48.0
140.58621836 0.0186579208821058 49.0
0 0.688696026802063 50.0
62.417838204 0.205609306693077 51.0
95.523016388 0.140166223049164 52.0
85.686293621 0.0971223041415215 53.0
85.686293621 0.0525317378342152 54.0
85.686293621 0.0592274740338326 55.0
0 2.91002941131592 56.0
77.278286303 0.299607485532761 57.0
230.27500994 0.771285891532898 58.0
253.67734593 0.652911424636841 59.0
24.292271139 0.0087104644626379 60.0
0 0.0928195342421532 61.0
183.20773235 1.2393707036972 0.0
183.20773235 1.0040956735611 1.0
181.99721326 0.98102068901062 2.0
177.208108 3.64852070808411 3.0
0 8.00492286682129 4.0
0 3.23637056350708 5.0
-65.352688884 0.300923258066177 6.0
76.649913664 1.06730937957764 7.0
53.217174191 0.552272021770477 8.0
23.432739473 0.65054178237915 9.0
92.012672697 1.35161375999451 10.0
78.163596113 0.24625638127327 11.0
83.657981281 0.396703720092773 12.0
83.657981281 0.926972448825836 13.0
83.657981281 0.0669031143188477 14.0
0 0.23698003590107 15.0
0 0.178849160671234 16.0
0 1.40636825561523 17.0
60.864492055 1.20086276531219 18.0
-60.864492055 1.09102702140808 19.0
-60.864492055 0.326975584030151 20.0
-60.864492055 0.14450965821743 21.0
83.862038708 1.30095660686493 22.0
145.0638221 0.85194194316864 23.0
109.4344719 0.452528804540634 24.0
25.219906385 0.135405510663986 25.0
228.9258608 1.19151175022125 26.0
228.9258608 2.18716859817505 27.0
27.367578212 2.66679406166077 28.0
256.29343902 1.38498544692993 29.0
256.29343902 1.8562353849411 30.0
46.134785412 0.946496069431305 31.0
9.3183430353 0.522197604179382 32.0
22.997546653 1.15648925304413 33.0
22.997546653 0.848349094390869 34.0
22.997546653 0.131018802523613 35.0
9.4223092864 0.0318543650209904 36.0
8.0893228994 0.361429035663605 37.0
0 0.0665309131145477 38.0
0 0.195907384157181 39.0
-11.297720336 0.39564523100853 40.0
-23.782448305 0.759647011756897 41.0
5.2515450138 0.587867498397827 42.0
-31.41594033 0.627884209156036 43.0
84.71388879 0.661263763904572 44.0
84.71388879 0.0873033925890923 45.0
6.9930813729 0.617406189441681 46.0
2.6481082495 0.416379332542419 47.0
183.20773235 0.183388248085976 48.0
177.208108 0.0265063308179379 49.0
0 0.292745292186737 50.0
76.649913664 0.160040676593781 51.0
92.012672697 0.190344601869583 52.0
83.657981281 0.0304603017866611 53.0
83.657981281 0.247853174805641 54.0
83.657981281 0.319447875022888 55.0
0 1.32216680049896 56.0
83.862038708 0.0668329522013664 57.0
228.9258608 0.730193555355072 58.0
256.29343902 0.934711873531342 59.0
22.997546653 0.0107555650174618 60.0
0 0.0937195122241974 61.0
164.93917861 0.924661040306091 0.0
164.93917861 2.33361434936523 1.0
163.99272297 1.84153425693512 2.0
159.20361777 6.6163387298584 3.0
0 2.90840744972229 4.0
0 3.01518964767456 5.0
-57.160144978 3.24712038040161 6.0
66.87358347 0.547463655471802 7.0
44.905540952 1.12617111206055 8.0
21.968042518 1.18135273456573 9.0
88.599572848 0.937946021556854 10.0
77.14469259 0.0303484946489334 11.0
86.176448046 0.0945698618888855 12.0
86.176448046 0.0773419961333275 13.0
86.176448046 0.100614905357361 14.0
0 0.187276750802994 15.0
0 0.225162550806999 16.0
0 1.16731536388397 17.0
63.750017733 1.69207763671875 18.0
-63.750017733 2.08138108253479 19.0
-63.750017733 0.733793079853058 20.0
-63.750017733 0.00520197302103043 21.0
81.534231817 1.16241645812988 22.0
146.43496386 0.629390776157379 23.0
112.73493316 0.275289237499237 24.0
22.8596958 0.555049002170563 25.0
227.96919567 0.616954565048218 26.0
227.96919567 1.45188081264496 27.0
24.717504 0.752670764923096 28.0
252.68669967 0.366934418678284 29.0
252.68669967 1.68578565120697 30.0
47.044270793 0.939309000968933 31.0
9.3188359369 0.363885074853897 32.0
17.784214084 1.5882910490036 33.0
17.784214084 1.19142436981201 34.0
17.784214084 0.126045435667038 35.0
9.8795879917 0.0347406789660454 36.0
8.5695837674 0.368356436491013 37.0
0 0.000273983925580978 38.0
0 0.379192769527435 39.0
-12.716673241 0.543084621429443 40.0
-25.348985282 0.74224328994751 41.0
5.6539954481 0.379866063594818 42.0
-33.956088008 0.50406014919281 43.0
109.83770207 0.431586861610413 44.0
109.83770207 0.104801751673222 45.0
5.9201737427 0.596948146820068 46.0
2.3816303074 0.435220450162888 47.0
164.93917861 0.185884594917297 48.0
159.20361777 0.56054812669754 49.0
0 0.0430376306176186 50.0
66.87358347 0.114172965288162 51.0
88.599572848 0.0701177790760994 52.0
86.176448046 0.23501892387867 53.0
86.176448046 0.0953365340828896 54.0
86.176448046 0.093899205327034 55.0
0 3.82834768295288 56.0
81.534231817 -0.0519296526908875 57.0
227.96919567 0.322343528270721 58.0
252.68669967 0.803395092487335 59.0
17.784214084 0.0303152557462454 60.0
0 0.0833606421947479 61.0
146.80996543 0.79931503534317 0.0
146.80996543 1.84366393089294 1.0
145.33919237 1.66752028465271 2.0
140.5500871 2.94714260101318 3.0
0 7.08407163619995 4.0
0 5.30744934082031 5.0
-56.76635099 3.55945873260498 6.0
68.64411873 0.757597744464874 7.0
49.593024212 0.531149089336395 8.0
19.051094518 0.570396900177002 9.0
85.841695082 0.905221045017242 10.0
72.883299775 0.148127302527428 11.0
80.968854916 0.641974687576294 12.0
80.968854916 0.220641821622849 13.0
80.968854916 0.1299017816782 14.0
0 0.19186694920063 15.0
0 0.162430629134178 16.0
0 1.01612019538879 17.0
53.046049247 1.25298571586609 18.0
-53.046049247 1.11516928672791 19.0
-53.046049247 0.456074386835098 20.0
-53.046049247 0.0260345786809921 21.0
73.672373113 1.38576829433441 22.0
142.62670063 0.861712276935577 23.0
110.14730353 0.510408282279968 24.0
21.60195013 0.172338515520096 25.0
216.29907375 1.12857484817505 26.0
216.29907375 1.38960754871368 27.0
25.329647969 1.2300261259079 28.0
241.62872171 0.81829971075058 29.0
241.62872171 1.70282125473022 30.0
46.889981581 1.0087206363678 31.0
9.4634103677 0.507110178470612 32.0
20.626323866 1.09134042263031 33.0
20.626323866 0.964799225330353 34.0
20.626323866 0.122682929039001 35.0
10.0367329 0.033861979842186 36.0
8.821550244 0.401878595352173 37.0
0 0.00474473088979721 38.0
0 0.0541826114058495 39.0
-11.356829692 0.593056738376617 40.0
-23.355422341 0.489267617464066 41.0
4.7192052738 0.536893129348755 42.0
-30.324103135 0.652093231678009 43.0
83.932257644 0.63927298784256 44.0
83.932257644 0.0861218199133873 45.0
6.1956227295 0.488979160785675 46.0
2.3137825065 0.322245091199875 47.0
146.80996543 0.151202023029327 48.0
140.5500871 0.107526153326035 49.0
0 0.448757082223892 50.0
68.64411873 0.14248289167881 51.0
85.841695082 0.255728244781494 52.0
80.968854916 0.0135794058442116 53.0
80.968854916 0.0739237815141678 54.0
80.968854916 0.0361156538128853 55.0
0 5.17663097381592 56.0
73.672373113 -0.0597160346806049 57.0
216.29907375 0.395416080951691 58.0
241.62872171 0.419381529092789 59.0
20.626323866 -0.0090276999399066 60.0
0 0.0918194130063057 61.0
161.50258213 0.816354274749756 0.0
161.50258213 1.3009387254715 1.0
160.17414968 1.44530296325684 2.0
154.70947913 4.95676183700562 3.0
1.917618995 5.41358041763306 4.0
1.917618995 3.57870507240295 5.0
-59.00006487 0.592887878417969 6.0
70.607150856 0.501710593700409 7.0
46.011851075 0.98356294631958 8.0
24.595299781 1.14335429668427 9.0
96.652730851 1.01577234268188 10.0
83.415304821 0.396378546953201 11.0
87.860975939 0.53568434715271 12.0
87.860975939 0.222640663385391 13.0
87.860975939 0.058722298592329 14.0
0 0.195998892188072 15.0
0 0.189471274614334 16.0
0 1.05656635761261 17.0
59.621572394 1.31732201576233 18.0
-59.621572394 1.46722316741943 19.0
-59.621572394 0.384248644113541 20.0
-59.621572394 -0.028618523851037 21.0
81.107119709 1.89552676677704 22.0
141.04994292 1.89701950550079 23.0
107.42302482 0.365649223327637 24.0
23.735641156 0.146281331777573 25.0
222.15706262 1.31023037433624 26.0
222.15706262 1.85321462154388 27.0
26.334447159 0.866000413894653 28.0
248.49150978 0.797579646110535 29.0
248.49150978 1.51325082778931 30.0
46.610697172 0.980688154697418 31.0
9.2447353757 0.592989921569824 32.0
21.485547315 1.4258896112442 33.0
21.485547315 1.05904006958008 34.0
21.485547315 0.117116466164589 35.0
10.689851727 0.0265866443514824 36.0
9.4512911086 0.361020117998123 37.0
0 0.0144056789577007 38.0
0 0.150079429149628 39.0
-11.703571221 0.403712511062622 40.0
-24.028059449 0.845158815383911 41.0
4.6270455588 0.36632314324379 42.0
-31.123163928 0.424976229667664 43.0
99.21278526 0.691817462444305 44.0
99.21278526 0.0779736414551735 45.0
6.8883336148 0.687254250049591 46.0
1.9523445926 0.42029732465744 47.0
161.50258213 0.105163678526878 48.0
154.70947913 0.156449243426323 49.0
1.917618995 0.132237881422043 50.0
70.607150856 0.164030969142914 51.0
96.652730851 0.177474796772003 52.0
87.860975939 0.088807687163353 53.0
87.860975939 0.0771612972021103 54.0
87.860975939 0.0747105628252029 55.0
0 2.00957345962524 56.0
81.107119709 0.122751578688622 57.0
222.15706262 0.428644001483917 58.0
248.49150978 0.719872951507568 59.0
21.485547315 0.0219761878252029 60.0
0 0.0716926828026772 61.0
187.2556273 0.950508117675781 0.0
187.2556273 0.760014295578003 1.0
185.902954 0.965668797492981 2.0
181.11384887 3.17564129829407 3.0
0 5.79083776473999 4.0
0 4.08963012695312 5.0
-54.459241872 1.27029669284821 6.0
64.897161008 0.645142674446106 7.0
46.334874895 1.33571076393127 8.0
18.562286113 1.24844658374786 9.0
85.733365041 0.959511339664459 10.0
73.212023232 0.153488978743553 11.0
82.618329424 0.647493839263916 12.0
82.618329424 0.159669205546379 13.0
82.618329424 0.0986774414777756 14.0
0 0.110575199127197 15.0
0 0.13219042122364 16.0
0 1.62141311168671 17.0
58.642257365 1.5954577922821 18.0
-58.642257365 2.00085282325745 19.0
-58.642257365 0.200967818498611 20.0
-58.642257365 0.0278985090553761 21.0
79.963747233 1.2612167596817 22.0
153.38973597 1.20186293125153 23.0
117.32266769 0.230532705783844 24.0
25.229371279 0.275082319974899 25.0
233.35348321 0.890599727630615 26.0
233.35348321 1.70193767547607 27.0
26.071094669 1.96208488941193 28.0
259.42457788 0.697835981845856 29.0
259.42457788 1.60705256462097 30.0
47.012627992 1.14543724060059 31.0
9.2140776792 0.640588879585266 32.0
21.321489869 1.48997521400452 33.0
21.321489869 0.947872221469879 34.0
21.321489869 0.140738919377327 35.0
9.7494117079 0.0291210133582354 36.0
8.3748786933 0.406533151865005 37.0
0 0.0138879790902138 38.0
0 0.0977326855063438 39.0
-11.858146484 0.746561646461487 40.0
-24.365062512 0.589872062206268 41.0
4.8450916071 0.537838459014893 42.0
-31.874641363 0.65723842382431 43.0
91.274215193 0.677925884723663 44.0
91.274215193 0.104731351137161 45.0
6.8547651219 0.475636780261993 46.0
2.3556050611 0.351441472768784 47.0
187.2556273 0.133356884121895 48.0
181.11384887 0.145221337676048 49.0
0 0.545781075954437 50.0
64.897161008 0.264883577823639 51.0
85.733365041 0.182663977146149 52.0
82.618329424 0.0532440021634102 53.0
82.618329424 0.154136747121811 54.0
82.618329424 0.089787520468235 55.0
0 1.74351513385773 56.0
79.963747233 0.0171650275588036 57.0
233.35348321 0.291209548711777 58.0
259.42457788 0.420549035072327 59.0
21.321489869 0.032517172396183 60.0
0 0.116623066365719 61.0
176.33237477 0.786037504673004 0.0
176.33237477 1.73319411277771 1.0
174.84901999 1.52131807804108 2.0
169.38434949 2.86349868774414 3.0
21.136763247 5.27187061309814 4.0
21.136763247 4.30501794815063 5.0
-58.085603595 2.85487461090088 6.0
67.39383281 0.434738606214523 7.0
35.80429991 0.485801815986633 8.0
31.5895329 0.555547773838043 9.0
77.174990976 0.861803710460663 10.0
65.608472857 0.0904656872153282 11.0
78.197460008 0.166338711977005 12.0
78.197460008 0.702665984630585 13.0
78.197460008 0.0874365046620369 14.0
0 0.186231061816216 15.0
0 0.204606413841248 16.0
0 1.48696434497833 17.0
62.121515273 2.01733374595642 18.0
-62.121515273 2.66920590400696 19.0
-62.121515273 0.149461090564728 20.0
-62.121515273 0.0315717235207558 21.0
82.804442849 0.852407336235046 22.0
149.65493597 1.24314057826996 23.0
118.2279754 0.551963150501251 24.0
23.182145135 0.145564928650856 25.0
232.45937882 0.873964071273804 26.0
232.45937882 1.73918950557709 27.0
24.540449426 1.87011241912842 28.0
256.99982825 0.428281784057617 29.0
256.99982825 1.84181642532349 30.0
46.943203737 1.18517208099365 31.0
9.2564556592 0.428449809551239 32.0
20.682927576 1.89535820484161 33.0
20.682927576 0.873720467090607 34.0
20.682927576 0.0975782424211502 35.0
8.661114673 0.0250035878270864 36.0
7.2084356789 0.288152158260345 37.0
0 0.0197223015129566 38.0
0 0.041143286973238 39.0
-12.028851605 0.59420508146286 40.0
-25.70264448 0.754894971847534 41.0
4.8490105839 0.458999902009964 42.0
-33.470959097 0.432232618331909 43.0
112.13137155 0.716254234313965 44.0
112.13137155 0.0724470987915993 45.0
6.4339989681 0.704724371433258 46.0
2.3381918132 0.450262665748596 47.0
176.33237477 0.0420073792338371 48.0
169.38434949 0.118105672299862 49.0
21.136763247 0.449698597192764 50.0
67.39383281 0.13636477291584 51.0
77.174990976 0.0922080427408218 52.0
78.197460008 0.0365346781909466 53.0
78.197460008 0.0369001887738705 54.0
78.197460008 0.451661765575409 55.0
0 6.16369295120239 56.0
82.804442849 0.0394270941615105 57.0
232.45937882 0.336991876363754 58.0
256.99982825 0.371821671724319 59.0
20.682927576 -0.0100597627460957 60.0
0 0.285725146532059 61.0
136.87436379 1.06348621845245 0.0
136.87436379 1.21731746196747 1.0
135.86045754 1.48742401599884 2.0
130.3957868 4.23229694366455 3.0
1.7546858954 5.53991842269897 4.0
1.7546858954 7.27910184860229 5.0
-74.541302957 5.16958093643188 6.0
84.3411203 0.620539486408234 7.0
50.432201046 0.685988664627075 8.0
33.908919254 0.906139492988586 9.0
76.934908732 1.02569842338562 10.0
62.700203632 0.0733363479375839 11.0
77.409329352 0.128376305103302 12.0
77.409329352 0.742609202861786 13.0
77.409329352 0.0452968813478947 14.0
0 0.0654292926192284 15.0
0 0.222758531570435 16.0
0 1.07066798210144 17.0
57.147720797 1.43724608421326 18.0
-57.147720797 1.43491852283478 19.0
-57.147720797 0.253634750843048 20.0
-57.147720797 0.0733043253421783 21.0
79.883118729 1.64911711215973 22.0
140.16223564 1.49677217006683 23.0
105.32860444 0.427403181791306 24.0
20.529216228 0.149352952837944 25.0
220.04535437 1.39088594913483 26.0
220.04535437 1.68232786655426 27.0
27.765603624 1.18779730796814 28.0
247.81095799 0.660467803478241 29.0
247.81095799 1.57246494293213 30.0
46.681457512 1.01142454147339 31.0
9.3823928166 0.360282003879547 32.0
22.735397933 1.10533058643341 33.0
22.735397933 0.842456042766571 34.0
22.735397933 0.134898647665977 35.0
9.4583600473 0.0387162864208221 36.0
8.1155872085 0.404134124517441 37.0
0 0.00892126187682152 38.0
0 0.163754776120186 39.0
-11.418317195 0.601690948009491 40.0
-24.675976415 0.546604156494141 41.0
4.9434784511 0.420494258403778 42.0
-31.922285994 0.503786444664001 43.0
88.060241438 0.811689019203186 44.0
88.060241438 0.0965119451284409 45.0
6.8557742772 0.622359752655029 46.0
2.4772310278 0.336113721132278 47.0
136.87436379 0.186543166637421 48.0
130.3957868 0.102182865142822 49.0
1.7546858954 0.374954581260681 50.0
84.3411203 0.113781064748764 51.0
76.934908732 0.0619094893336296 52.0
77.409329352 0.0141907911747694 53.0
77.409329352 0.0434375815093517 54.0
77.409329352 0.376340180635452 55.0
0 5.09837818145752 56.0
79.883118729 0.00933921709656715 57.0
220.04535437 0.450509488582611 58.0
247.81095799 0.318865478038788 59.0
22.735397933 0.00962863489985466 60.0
0 0.0838714987039566 61.0
125.49898512 0.891663134098053 0.0
125.49898512 2.3943498134613 1.0
124.02083539 2.09655165672302 2.0
118.55616424 3.2297465801239 3.0
0.47355099045 4.55257415771484 4.0
0.47355099045 3.70044183731079 5.0
-69.419526721 0.0043758749961853 6.0
83.5973988 1.00360643863678 7.0
60.234796226 1.38170886039734 8.0
23.362602574 2.01185917854309 9.0
107.89754707 0.924428761005402 10.0
95.974955001 0.260532706975937 11.0
91.526596554 1.09276163578033 12.0
91.526596554 0.776971399784088 13.0
91.526596554 0.0848660469055176 14.0
0 0.468792796134949 15.0
0 0.177099198102951 16.0
0 1.36303627490997 17.0
60.612211733 1.33628726005554 18.0
-60.612211733 1.02868735790253 19.0
-60.612211733 0.54406201839447 20.0
-60.612211733 -0.04721499979496 21.0
81.028854327 1.91828596591949 22.0
146.184721 1.47172594070435 23.0
114.29253258 0.0892255753278732 24.0
21.941801048 0.305097222328186 25.0
227.21357533 1.25448930263519 26.0
227.21357533 1.12838459014893 27.0
29.668877728 1.5216566324234 28.0
256.88245305 0.604483723640442 29.0
256.88245305 1.56294548511505 30.0
47.437283036 0.946810841560364 31.0
9.4675636665 0.46554708480835 32.0
20.416642594 1.60852658748627 33.0
20.416642594 0.588538408279419 34.0
20.416642594 0.150144055485725 35.0
8.4144757952 0.0486920326948166 36.0
7.2394229479 0.318693876266479 37.0
0 0.00951813906431198 38.0
0 0.203997552394867 39.0
-13.861503891 0.442640513181686 40.0
-27.140449057 1.05595266819 41.0
4.4812879402 0.346019089221954 42.0
-33.803833239 0.521631717681885 43.0
101.97965713 0.674091815948486 44.0
101.97965713 0.0517861023545265 45.0
7.314005764 0.453328132629395 46.0
2.2251543042 0.306264728307724 47.0
125.49898512 0.156749665737152 48.0
118.55616424 0.469870686531067 49.0
0.47355099045 0.362215131521225 50.0
83.5973988 0.58235639333725 51.0
107.89754707 0.0850170254707336 52.0
91.526596554 0.091780312359333 53.0
91.526596554 0.34065917134285 54.0
91.526596554 0.226124912500381 55.0
0 0.618505239486694 56.0
81.028854327 0.0143859479576349 57.0
227.21357533 0.200517326593399 58.0
256.88245305 0.594960272312164 59.0
20.416642594 0.00801169686019421 60.0
0 0.256356298923492 61.0
177.82524012 0.869212031364441 0.0
177.82524012 0.88832688331604 1.0
176.60185823 0.839572608470917 2.0
171.13722283 4.31741333007812 3.0
0.06572514228 7.21044158935547 4.0
0.06572514228 6.57489538192749 5.0
-67.057292993 5.55903291702271 6.0
76.338322172 0.646174311637878 7.0
51.992582519 0.685184299945831 8.0
24.345739653 0.585516571998596 9.0
83.262126117 1.04853618144989 10.0
70.761691507 0.0933544933795929 11.0
75.447073933 0.614361226558685 12.0
75.447073933 0.43555223941803 13.0
75.447073933 0.0875518321990967 14.0
0 0.132931977510452 15.0
0 0.212311029434204 16.0
0 1.53212606906891 17.0
57.095123021 1.53943002223969 18.0
-57.095123021 1.82520341873169 19.0
-57.095123021 0.259181588888168 20.0
-57.095123021 0.0299528501927853 21.0
78.675997579 2.15543389320374 22.0
141.17473979 2.59433078765869 23.0
104.23433561 0.278122097253799 24.0
23.292846104 0.235020741820335 25.0
219.85073737 1.3034348487854 26.0
219.85073737 2.37280011177063 27.0
25.373937425 2.52307724952698 28.0
245.22467479 1.09276962280273 29.0
245.22467479 1.67786467075348 30.0
47.037903464 1.09167730808258 31.0
9.315944408 0.3905169069767 32.0
21.580874558 1.51633787155151 33.0
21.580874558 0.772617816925049 34.0
21.580874558 0.0953413769602776 35.0
9.5901287357 0.0244623236358166 36.0
8.2925886068 0.411919742822647 37.0
0 0.0178893283009529 38.0
0 0.214379012584686 39.0
-12.115084054 0.54402083158493 40.0
-24.939875654 0.60597026348114 41.0
5.3603858203 0.383182674646378 42.0
-32.721451654 0.489872872829437 43.0
82.612571412 0.45663133263588 44.0
82.612571412 0.0793308466672897 45.0
6.4736556493 0.494640409946442 46.0
2.6146860387 0.398189723491669 47.0
177.82524012 0.192919045686722 48.0
171.13722283 0.184471696615219 49.0
0.06572514228 0.743488550186157 50.0
76.338322172 0.146757155656815 51.0
83.262126117 0.270625859498978 52.0
75.447073933 0.0356239378452301 53.0
75.447073933 0.165855810046196 54.0
75.447073933 0.100758619606495 55.0
0 3.57893013954163 56.0
78.675997579 0.000754653476178646 57.0
219.85073737 0.786987125873566 58.0
245.22467479 0.546471476554871 59.0
21.580874558 0.00883774273097515 60.0
0 0.0792337656021118 61.0
176.51183478 1.02351450920105 0.0
176.51183478 0.752897322177887 1.0
175.28684661 1.28308486938477 2.0
170.49774131 5.5442795753479 3.0
0 4.35063886642456 4.0
0 3.53658175468445 5.0
-69.688242207 1.37176060676575 6.0
82.798260784 0.441399931907654 7.0
53.581290043 1.10246539115906 8.0
29.216970741 1.16311860084534 9.0
92.914190867 1.17954099178314 10.0
79.933222894 0.316899865865707 11.0
87.98548101 0.804145932197571 12.0
87.98548101 0.528660833835602 13.0
87.98548101 0.0501018241047859 14.0
0 0.118420608341694 15.0
0 0.194327026605606 16.0
0 0.971645832061768 17.0
54.585409444 1.22459828853607 18.0
-54.585409444 0.930916011333466 19.0
-54.585409444 0.615139782428741 20.0
-54.585409444 -0.0662316083908081 21.0
75.961935219 2.49180006980896 22.0
149.7876677 1.601646900177 23.0
112.08264815 0.408787429332733 24.0
24.521758883 0.140460059046745 25.0
225.74960292 1.56422531604767 26.0
225.74960292 1.73109173774719 27.0
25.996450255 1.56646621227264 28.0
251.74605317 1.27260971069336 29.0
251.74605317 1.78854501247406 30.0
46.338175399 0.952160120010376 31.0
9.3546836743 0.402952671051025 32.0
21.376525775 1.65128493309021 33.0
21.376525775 1.20498144626617 34.0
21.376525775 0.116876013576984 35.0
9.4270782772 0.0228109788149595 36.0
8.2606826798 0.378915250301361 37.0
0 0.00398280099034309 38.0
0 0.129522114992142 39.0
-11.674937149 0.600403130054474 40.0
-23.364324845 0.45594522356987 41.0
5.5273254883 0.49603208899498 42.0
-31.22943148 0.530707597732544 43.0
83.935863823 0.69174063205719 44.0
83.935863823 0.0990466997027397 45.0
6.2062535686 0.670403957366943 46.0
2.5077765131 0.431522250175476 47.0
176.51183478 0.106416925787926 48.0
170.49774131 0.11038726568222 49.0
0 0.195467084646225 50.0
82.798260784 0.162715211510658 51.0
92.914190867 0.352254688739777 52.0
87.98548101 0.10284161567688 53.0
87.98548101 0.16677850484848 54.0
87.98548101 0.00662450306117535 55.0
0 1.32384645938873 56.0
75.961935219 0.168883338570595 57.0
225.74960292 0.587603569030762 58.0
251.74605317 0.995086312294006 59.0
21.376525775 -0.0137411495670676 60.0
0 0.108711376786232 61.0
181.16601595 0.962726831436157 0.0
181.16601595 0.565891146659851 1.0
179.7702386 1.22758507728577 2.0
174.3055679 5.12783288955688 3.0
12.350928627 7.47272396087646 4.0
12.350928627 5.39462232589722 5.0
-58.063749211 2.5951554775238 6.0
69.464451258 0.436682939529419 7.0
44.756865332 0.976381301879883 8.0
24.707585926 1.12924420833588 9.0
88.944440243 0.956118583679199 10.0
77.233365457 0.173739224672318 11.0
82.758248855 0.223882436752319 12.0
82.758248855 0.600446105003357 13.0
82.758248855 0.0957177206873894 14.0
0 0.230549871921539 15.0
0 0.179157763719559 16.0
0 0.967643678188324 17.0
63.133152207 1.23983216285706 18.0
-63.133152207 1.33422935009003 19.0
-63.133152207 0.443994909524918 20.0
-63.133152207 0.0263458043336868 21.0
84.693325111 1.15287137031555 22.0
146.62307594 0.834370791912079 23.0
111.13727741 0.477902233600616 24.0
24.053943274 0.146948844194412 25.0
231.31640105 0.81435364484787 26.0
231.31640105 1.23795700073242 27.0
27.427337784 1.14330363273621 28.0
258.74373884 0.589763820171356 29.0
258.74373884 1.80349731445312 30.0
46.947993974 1.06081569194794 31.0
9.2669922847 0.386115282773972 32.0
21.560172904 1.47356617450714 33.0
21.560172904 0.923171520233154 34.0
21.560172904 0.140955910086632 35.0
9.8633330065 0.0407558679580688 36.0
8.5523255625 0.410810142755508 37.0
0 0.0277705118060112 38.0
0 0.0956166312098503 39.0
-10.853448271 0.526358067989349 40.0
-23.119083083 0.617408812046051 41.0
5.4544966328 0.482576996088028 42.0
-30.874520537 0.403640657663345 43.0
88.075328875 0.622191548347473 44.0
88.075328875 0.081306204199791 45.0
7.5558096126 0.70652449131012 46.0
2.8420325062 0.402708679437637 47.0
181.16601595 0.135888010263443 48.0
174.3055679 0.347563117742538 49.0
12.350928627 0.961242020130157 50.0
69.464451258 0.0945957973599434 51.0
88.944440243 0.330014139413834 52.0
82.758248855 0.0345624387264252 53.0
82.758248855 0.0625158697366714 54.0
82.758248855 0.169635310769081 55.0
0 4.5097827911377 56.0
84.693325111 0.142241805791855 57.0
231.31640105 0.461495488882065 58.0
258.74373884 0.597839057445526 59.0
21.560172904 -0.0131626389920712 60.0
0 0.0978670790791512 61.0
189.59029805 0.917535543441772 0.0
189.59029805 1.30099010467529 1.0
188.57257488 1.62333154678345 2.0
187.47604261 3.48758888244629 3.0
0 5.53705215454102 4.0
0 4.1501293182373 5.0
-67.993910727 2.47727537155151 6.0
78.572215501 0.887585341930389 7.0
50.867258574 0.797705411911011 8.0
27.704956927 0.921505391597748 9.0
87.200935502 1.04923593997955 10.0
72.75655906 0.247065454721451 11.0
79.827001903 0.873238205909729 12.0
79.827001903 0.234611183404922 13.0
79.827001903 0.120516985654831 14.0
1.5421738041 0.0696571990847588 15.0
1.5421738041 0.148981392383575 16.0
0 1.25277137756348 17.0
55.030249157 1.00979125499725 18.0
-55.030249157 1.16781556606293 19.0
-55.030249157 0.333398371934891 20.0
-55.030249157 0.00293701514601707 21.0
75.942618462 1.73141419887543 22.0
145.39403536 0.809673666954041 23.0
108.55644329 0.743848443031311 24.0
24.334701312 0.0743977278470993 25.0
221.33665382 1.07596111297607 26.0
221.33665382 2.01517486572266 27.0
28.839063304 1.77597773075104 28.0
250.17571712 0.707675576210022 29.0
250.17571712 1.69561719894409 30.0
46.560805274 1.27054405212402 31.0
9.3630029578 0.362515687942505 32.0
20.912369305 1.22045814990997 33.0
20.912369305 1.14261209964752 34.0
20.912369305 0.09241683781147 35.0
9.3477232566 0.022732013836503 36.0
8.0122641525 0.387358486652374 37.0
0 0.0195529758930206 38.0
0 0.142325520515442 39.0
-12.793539042 0.943823516368866 40.0
-24.817442831 0.388851016759872 41.0
5.5163864248 0.594509184360504 42.0
-32.892794551 0.801614999771118 43.0
98.026586068 0.640276968479156 44.0
98.026586068 0.104031838476658 45.0
6.8147155322 0.543704450130463 46.0
2.5942802637 0.418009966611862 47.0
189.59029805 0.0372423455119133 48.0
187.47604261 0.285157054662704 49.0
0 0.123369216918945 50.0
78.572215501 0.142264053225517 51.0
87.200935502 0.248478680849075 52.0
79.827001903 0.0837281420826912 53.0
79.827001903 0.124968267977238 54.0
79.827001903 0.076308399438858 55.0
0 4.98221874237061 56.0
75.942618462 0.0483120679855347 57.0
221.33665382 0.76673549413681 58.0
250.17571712 0.420201867818832 59.0
20.912369305 -0.00136340409517288 60.0
0 0.0889027416706085 61.0
157.1061922 1.1776956319809 0.0
157.1061922 0.593016147613525 1.0
155.51038857 0.612279415130615 2.0
150.72128326 4.42020559310913 3.0
0 6.13989210128784 4.0
0 6.70325708389282 5.0
-78.329139515 4.00687122344971 6.0
86.144867883 0.684349775314331 7.0
48.275709595 1.43737137317657 8.0
37.869158288 1.37442529201508 9.0
91.70837967 1.26544141769409 10.0
78.338038284 0.430639058351517 11.0
85.956091449 0.433024197816849 12.0
85.956091449 0.619925975799561 13.0
85.956091449 0.133200258016586 14.0
0 0.228320404887199 15.0
0 0.120615005493164 16.0
0 1.26922953128815 17.0
58.905650103 1.06531953811646 18.0
-58.905650103 1.16946518421173 19.0
-58.905650103 1.00659561157227 20.0
-58.905650103 -0.050816185772419 21.0
79.932004295 1.50097644329071 22.0
156.37724687 0.603566229343414 23.0
121.81757001 0.684689581394196 24.0
27.102135655 0.16897976398468 25.0
236.30925116 1.44007730484009 26.0
236.30925116 2.15510845184326 27.0
31.159757422 1.4599324464798 28.0
267.46900858 0.734855711460114 29.0
267.46900858 1.64445745944977 30.0
45.821737251 0.968654930591583 31.0
9.2240878862 0.453849107027054 32.0
21.026354192 1.15197861194611 33.0
21.026354192 1.42841100692749 34.0
21.026354192 0.0719383582472801 35.0
9.2401168475 0.0267804656177759 36.0
8.0228445175 0.370560556650162 37.0
0 0.00348889827728271 38.0
0 0.309027194976807 39.0
-13.412483359 0.463797181844711 40.0
-26.62351884 0.524389564990997 41.0
5.5368556546 0.469237506389618 42.0
-34.440843085 0.582315683364868 43.0
75.02388304 0.505503952503204 44.0
75.02388304 0.118731118738651 45.0
7.0429625723 0.258295953273773 46.0
2.7762359354 0.310137331485748 47.0
157.1061922 0.0220071505755186 48.0
150.72128326 0.00843669287860394 49.0
0 0.309974491596222 50.0
86.144867883 0.227220699191093 51.0
91.70837967 0.452232509851456 52.0
85.956091449 0.0801925659179688 53.0
85.956091449 0.0408838093280792 54.0
85.956091449 0.0081631476059556 55.0
0 4.72510051727295 56.0
79.932004295 0.145221471786499 57.0
236.30925116 0.795534491539001 58.0
267.46900858 0.662839889526367 59.0
21.026354192 -0.012735235504806 60.0
0 0.105537213385105 61.0
142.00657213 0.841080009937286 0.0
142.00657213 1.57432961463928 1.0
140.6353596 2.37674355506897 2.0
135.84625442 4.55938291549683 3.0
0 5.01800537109375 4.0
0 5.84719181060791 5.0
-52.649546889 3.86111736297607 6.0
61.453640657 0.661005854606628 7.0
38.95976453 0.712436079978943 8.0
22.493876128 0.586470305919647 9.0
90.550424131 1.16384315490723 10.0
77.678694914 0.936001896858215 11.0
78.58578384 0.63524055480957 12.0
78.58578384 0.259435027837753 13.0
78.58578384 0.132096827030182 14.0
0 0.2259541451931 15.0
0 0.153754279017448 16.0
0 1.56147086620331 17.0
52.329024593 0.731553375720978 18.0
-52.329024593 1.130202293396 19.0
-52.329024593 0.300193727016449 20.0
-52.329024593 0.114555038511753 21.0
72.14130943 1.02604103088379 22.0
155.1228373 0.54356461763382 23.0
118.92227581 0.702270686626434 24.0
20.772913062 0.132351458072662 25.0
227.26414673 0.755316376686096 26.0
227.26414673 1.9791431427002 27.0
24.313009542 2.2722601890564 28.0
251.57715627 1.13487827777863 29.0
251.57715627 2.22935771942139 30.0
46.876052376 0.929310917854309 31.0
9.3154269282 0.542308449745178 32.0
19.812284837 1.00634801387787 33.0
19.812284837 1.33292376995087 34.0
19.812284837 0.113543942570686 35.0
9.0953847888 0.026412608101964 36.0
7.9560285557 0.341704726219177 37.0
0 0.0142892859876156 38.0
0 0.354139983654022 39.0
-12.317804047 0.302621781826019 40.0
-26.241224244 0.762632966041565 41.0
5.6104063378 0.520521998405457 42.0
-34.243381953 0.774438440799713 43.0
84.353888876 0.719440519809723 44.0
84.353888876 0.0881310105323792 45.0
7.1781946445 0.695316731929779 46.0
2.2499335919 0.325129985809326 47.0
142.00657213 0.229753121733665 48.0
135.84625442 0.434526681900024 49.0
0 0.688248574733734 50.0
61.453640657 0.347043663263321 51.0
90.550424131 0.985650420188904 52.0
78.58578384 0.129292458295822 53.0
78.58578384 0.0668247565627098 54.0
78.58578384 0.198926135897636 55.0
0 3.43927502632141 56.0
72.14130943 0.0174126252532005 57.0
227.26414673 0.516574442386627 58.0
251.57715627 1.06011700630188 59.0
19.812284837 -0.0122020933777094 60.0
0 0.0918604284524918 61.0
150.74380829 0.729436337947845 0.0
150.74380829 1.11156415939331 1.0
149.55636392 1.0682178735733 2.0
144.09169337 3.17926025390625 3.0
4.0298780501 6.4466495513916 4.0
4.0298780501 5.02541255950928 5.0
-64.746467178 0.438639372587204 6.0
77.163270557 0.630462050437927 7.0
44.657884003 0.548582911491394 8.0
32.505386554 0.650224506855011 9.0
82.914184044 0.804730951786041 10.0
70.415369527 0.299584448337555 11.0
77.542662693 0.597449839115143 12.0
77.542662693 0.720410227775574 13.0
77.542662693 0.122548326849937 14.0
0 0.169474825263023 15.0
0 0.219971984624863 16.0
0 1.41832363605499 17.0
57.736118575 1.52686679363251 18.0
-57.736118575 0.829683780670166 19.0
-57.736118575 0.98797732591629 20.0
-57.736118575 -0.0350199341773987 21.0
78.170285296 1.55063986778259 22.0
152.33544257 1.22841858863831 23.0
117.6911069 0.326856017112732 24.0
24.973289358 0.290624856948853 25.0
230.50572787 1.68800127506256 26.0
230.50572787 2.65120434761047 27.0
24.53759966 2.02448964118958 28.0
255.04332753 1.0408684015274 29.0
255.04332753 2.15864777565002 30.0
46.680699933 1.13474082946777 31.0
9.4521482203 0.437170565128326 32.0
20.434166721 1.60657584667206 33.0
20.434166721 1.19057321548462 34.0
20.434166721 0.126758128404617 35.0
9.2452231575 0.0204738602042198 36.0
7.9564773637 0.355240195989609 37.0
0 0.0199721045792103 38.0
0 0.203692212700844 39.0
-10.949435549 0.32744163274765 40.0
-24.183499349 0.637962639331818 41.0
5.4292701827 0.409570902585983 42.0
-32.354329776 0.524568021297455 43.0
100.82471258 0.638473331928253 44.0
100.82471258 0.14456832408905 45.0
7.1664037764 0.518082857131958 46.0
2.7028781884 0.416270703077316 47.0
150.74380829 0.146658003330231 48.0
144.09169337 0.181735008955002 49.0
4.0298780501 0.277821063995361 50.0
77.163270557 0.153724685311317 51.0
82.914184044 0.383583575487137 52.0
77.542662693 0.0918340310454369 53.0
77.542662693 0.0464498475193977 54.0
77.542662693 0.202438816428185 55.0
0 5.7004919052124 56.0
78.170285296 0.0918566435575485 57.0
230.50572787 1.06087696552277 58.0
255.04332753 0.898094236850739 59.0
20.434166721 -0.0144063467159867 60.0
0 0.0945939943194389 61.0
137.69758631 1.10320127010345 0.0
137.69758631 0.919937789440155 1.0
136.41598444 1.21062874794006 2.0
131.62687916 3.20485734939575 3.0
0 3.42469573020935 4.0
0 3.19908165931702 5.0
-52.224518651 0.161271616816521 6.0
63.112776978 0.601678371429443 7.0
59.552749198 1.43345260620117 8.0
3.5600277792 1.39914512634277 9.0
88.278473926 1.14137446880341 10.0
75.967442776 0.16563293337822 11.0
77.22220226 0.351158529520035 12.0
77.22220226 0.254525303840637 13.0
77.22220226 0.0479296743869781 14.0
0 0.0024866433814168 15.0
0 0.217790797352791 16.0
0 1.18814587593079 17.0
51.349955579 1.11120331287384 18.0
-51.349955579 0.838525056838989 19.0
-51.349955579 0.745203912258148 20.0
-51.349955579 -0.0427626483142376 21.0
73.105047296 2.74539422988892 22.0
145.00983075 1.92210626602173 23.0
108.46194054 0.583741128444672 24.0
21.476335853 0.085273414850235 25.0
218.11487804 1.60034549236298 26.0
218.11487804 1.64052426815033 27.0
30.039935897 1.99766433238983 28.0
248.15481394 0.983800947666168 29.0
248.15481394 1.88162982463837 30.0
46.701647481 0.96783310174942 31.0
9.3123267678 0.417482048273087 32.0
21.755091717 1.64163506031036 33.0
21.755091717 1.18329298496246 34.0
21.755091717 0.128551423549652 35.0
9.3181594506 0.0275026746094227 36.0
8.0864003463 0.447290778160095 37.0
0 0.0160388350486755 38.0
0 0.144788235425949 39.0
-12.089211172 0.467604070901871 40.0
-24.199776116 0.736060440540314 41.0
4.7318520922 0.576031267642975 42.0
-31.621029779 0.502570927143097 43.0
84.48878126 0.727093696594238 44.0
84.48878126 0.129323586821556 45.0
6.6383426552 0.517558217048645 46.0
2.1631512063 0.356127619743347 47.0
137.69758631 0.106010571122169 48.0
131.62687916 0.246528074145317 49.0
0 0.280972510576248 50.0
63.112776978 0.212148919701576 51.0
88.278473926 0.153802499175072 52.0
77.22220226 0.0670359507203102 53.0
77.22220226 0.0515520013868809 54.0
77.22220226 0.0630527958273888 55.0
0 4.04990577697754 56.0
73.105047296 0.114171288907528 57.0
218.11487804 0.614819407463074 58.0
248.15481394 0.730964422225952 59.0
21.755091717 -0.0176736954599619 60.0
0 0.095367044210434 61.0
184.69081909 0.781020402908325 0.0
184.69081909 1.31253600120544 1.0
183.37872929 1.66177570819855 2.0
177.91405859 4.04092502593994 3.0
9.4171244202 8.44843292236328 4.0
9.4171244202 5.49739551544189 5.0
-65.531459077 4.89294576644897 6.0
77.828221015 0.630986154079437 7.0
53.709016733 0.580866992473602 8.0
24.119204282 0.593270659446716 9.0
84.155601463 0.733302175998688 10.0
71.383931315 0.129167035222054 11.0
76.379035613 0.168632239103317 12.0
76.379035613 0.448630839586258 13.0
76.379035613 0.0201781764626503 14.0
0 0.00163597147911787 15.0
0 0.208133488893509 16.0
0 1.12229681015015 17.0
56.324522058 1.36722373962402 18.0
-56.324522058 1.94032573699951 19.0
-56.324522058 0.237867310643196 20.0
-56.324522058 0.0167188718914986 21.0
77.576191507 2.01644682884216 22.0
140.27436408 1.20410585403442 23.0
106.5513088 0.604682683944702 24.0
25.382438745 0.151481881737709 25.0
217.85055559 1.07800304889679 26.0
217.85055559 1.9626727104187 27.0
29.64726426 2.23697137832642 28.0
247.49781985 1.15075206756592 29.0
247.49781985 1.93523216247559 30.0
46.449045804 1.11438262462616 31.0
9.2145896789 0.624729990959167 32.0
21.251669448 1.37669038772583 33.0
21.251669448 1.08576679229736 34.0
21.251669448 0.13374325633049 35.0
9.7482715376 0.0330910161137581 36.0
8.5295223134 0.36107411980629 37.0
0 0.00602574273943901 38.0
0 0.127130284905434 39.0
-11.787639757 1.00207602977753 40.0
-25.299066571 0.380290299654007 41.0
5.6174960152 0.432951867580414 42.0
-33.28980074 0.442827343940735 43.0
90.162872475 0.659074902534485 44.0
90.162872475 0.0875487700104713 45.0
6.4602226418 0.71447741985321 46.0
2.7171776916 0.406574815511703 47.0
184.69081909 0.212529271841049 48.0
177.91405859 0.0906668528914452 49.0
9.4171244202 0.333713114261627 50.0
77.828221015 0.18364205956459 51.0
84.155601463 0.189292877912521 52.0
76.379035613 0.0327560901641846 53.0
76.379035613 0.0807713270187378 54.0
76.379035613 0.109987452626228 55.0
0 2.83376932144165 56.0
77.576191507 0.197110503911972 57.0
217.85055559 0.325884103775024 58.0
247.49781985 0.506276249885559 59.0
21.251669448 0.0120092742145061 60.0
0 0.089957669377327 61.0
127.12427025 0.903272211551666 0.0
127.12427025 2.33907556533813 1.0
126.03145916 2.50679850578308 2.0
121.2423892 4.16037607192993 3.0
0 5.57183885574341 4.0
0 2.2454981803894 5.0
-49.988588368 0.0424934029579163 6.0
61.527908474 0.733081758022308 7.0
67.564931234 0.920583665370941 8.0
-6.0370227596 0.79457300901413 9.0
98.594161161 1.13282227516174 10.0
85.780401536 0.287823677062988 11.0
88.421855532 1.20442485809326 12.0
88.421855532 1.04256284236908 13.0
88.421855532 0.0568311512470245 14.0
0 0.155464038252831 15.0
0 0.207335889339447 16.0
0 0.978149712085724 17.0
53.327543994 0.963340044021606 18.0
-53.327543994 1.01082468032837 19.0
-53.327543994 0.257206797599792 20.0
-53.327543994 -0.0694550126791 21.0
75.327179755 1.58690917491913 22.0
145.80080642 1.55424439907074 23.0
115.49990452 0.587164044380188 24.0
20.535769254 0.11642187088728 25.0
221.12798618 0.732907056808472 26.0
221.12798618 1.50084888935089 27.0
22.898208012 1.21347784996033 28.0
244.02619419 0.845894873142242 29.0
244.02619419 1.77179777622223 30.0
47.036326696 1.42118048667908 31.0
9.1845106356 0.485580056905746 32.0
21.999635761 2.42930006980896 33.0
21.999635761 1.20455098152161 34.0
21.999635761 0.156970083713531 35.0
8.917768712 0.0327878966927528 36.0
7.7749657327 0.435302555561066 37.0
0 0.0658043473958969 38.0
0 0.260520905256271 39.0
-11.108507635 0.614839375019073 40.0
-24.625864673 0.787329971790314 41.0
4.7669772977 0.540125131607056 42.0
-32.138599687 0.639996349811554 43.0
85.9008187 0.664898455142975 44.0
85.9008187 0.107036784291267 45.0
6.7200356433 0.457549929618835 46.0
2.2134832474 0.354424089193344 47.0
127.12427025 0.116229467093945 48.0
121.2423892 0.625791370868683 49.0
0 0.138514310121536 50.0
61.527908474 0.006091283634305 51.0
98.594161161 0.330784469842911 52.0
88.421855532 0.14339679479599 53.0
88.421855532 0.392798990011215 54.0
88.421855532 0.146424889564514 55.0
0 2.46155261993408 56.0
75.327179755 0.113527677953243 57.0
221.12798618 0.567626059055328 58.0
244.02619419 0.484679520130157 59.0
21.999635761 -0.0184525735676289 60.0
0 0.0907271653413773 61.0
197.18001354 0.71313613653183 0.0
197.18001354 0.424742221832275 1.0
196.25999519 1.15910506248474 2.0
191.47088991 5.54737710952759 3.0
0 7.40214538574219 4.0
0 4.88460779190063 5.0
-45.499056939 5.63727426528931 6.0
54.688444345 1.03313517570496 7.0
44.397849928 0.445395886898041 8.0
10.290594417 0.334136545658112 9.0
88.161448605 0.761952579021454 10.0
74.647415285 0.148837521672249 11.0
89.372543174 0.231974110007286 12.0
89.372543174 0.181717336177826 13.0
89.372543174 0.0954114571213722 14.0
0 0.183373034000397 15.0
0 0.195954367518425 16.0
0 1.3623856306076 17.0
53.720100368 0.915305554866791 18.0
-53.720100368 0.685105264186859 19.0
-53.720100368 1.27985525131226 20.0
-53.720100368 0.0401765294373035 21.0
76.293206367 1.33940267562866 22.0
153.85225962 1.98117768764496 23.0
115.89459539 0.214421451091766 24.0
22.548692677 0.398246198892593 25.0
230.14546599 0.392828494310379 26.0
230.14546599 1.47699224948883 27.0
29.428992307 1.59079384803772 28.0
259.5744583 0.388305962085724 29.0
259.5744583 1.97672486305237 30.0
47.587748425 1.23108279705048 31.0
9.503173052 0.652814388275146 32.0
22.573105999 0.917838394641876 33.0
22.573105999 1.08449804782867 34.0
22.573105999 0.123416475951672 35.0
10.179070389 0.0195901710540056 36.0
9.0955732236 0.348681211471558 37.0
0 0.0457072481513023 38.0
0 0.322296500205994 39.0
-11.574161363 0.476886838674545 40.0
-25.553668798 0.723163187503815 41.0
5.6684703779 0.334174454212189 42.0
-33.778273215 0.444093465805054 43.0
88.144836728 0.443875551223755 44.0
88.144836728 0.0828854665160179 45.0
7.2384582899 0.558113157749176 46.0
2.2491568282 0.431128233671188 47.0
197.18001354 0.0744840502738953 48.0
191.47088991 0.085473045706749 49.0
0 0.65683913230896 50.0
54.688444345 0.235783487558365 51.0
88.161448605 0.0735238417983055 52.0
89.372543174 0.0841820389032364 53.0
89.372543174 0.0848102867603302 54.0
89.372543174 0.103340990841389 55.0
0 1.84251010417938 56.0
76.293206367 0.00317073054611683 57.0
230.14546599 0.329308778047562 58.0
259.5744583 0.759880423545837 59.0
22.573105999 -0.00135903432965279 60.0
0 0.0904197543859482 61.0
189.02461311 1.09112632274628 0.0
189.02461311 2.58180069923401 1.0
187.78945792 2.44493818283081 2.0
182.32478734 6.70059013366699 3.0
5.1471378649 3.70345616340637 4.0
5.1471378649 3.77563500404358 5.0
-77.109181575 1.24612748622894 6.0
89.067904019 1.0467015504837 7.0
59.823138617 0.551591575145721 8.0
29.244765402 0.828415632247925 9.0
94.013167199 1.13331270217896 10.0
79.676227225 0.154989212751389 11.0
77.817691433 1.34185385704041 12.0
77.817691433 0.772550523281097 13.0
77.817691433 0.148107528686523 14.0
0 0.0820765942335129 15.0
0 0.188134983181953 16.0
0 1.20138823986053 17.0
61.105894706 0.769375205039978 18.0
-61.105894706 0.778560638427734 19.0
-61.105894706 0.33517998456955 20.0
-61.105894706 0.454461991786957 21.0
81.46010082 1.51355230808258 22.0
145.36106558 1.91310775279999 23.0
115.41776707 0.126129105687141 24.0
24.989547574 0.498959362506866 25.0
226.8211664 0.681028306484222 26.0
226.8211664 2.83439111709595 27.0
28.659573525 2.79013133049011 28.0
255.48073993 1.11528146266937 29.0
255.48073993 1.65376019477844 30.0
46.530607578 1.10121440887451 31.0
9.5289461939 0.464415609836578 32.0
20.354206115 1.30952978134155 33.0
20.354206115 1.21163105964661 34.0
20.354206115 0.0837209150195122 35.0
8.5549279802 0.0447156764566898 36.0
7.3669801598 0.374183058738708 37.0
0 0.0770683661103249 38.0
0 0.0783583968877792 39.0
-13.115575665 1.11062002182007 40.0
-25.061839167 0.273196041584015 41.0
5.3497651799 0.604857206344604 42.0
-33.418852686 0.534764230251312 43.0
101.06030298 0.587952494621277 44.0
101.06030298 0.0354913026094437 45.0
6.5788147448 0.633478760719299 46.0
2.0126880934 0.411661237478256 47.0
189.02461311 0.170226842164993 48.0
182.32478734 0.329373091459274 49.0
5.1471378649 0.208075374364853 50.0
89.067904019 0.121166601777077 51.0
94.013167199 0.175089791417122 52.0
77.817691433 0.181995749473572 53.0
77.817691433 0.454062819480896 54.0
77.817691433 0.0404994636774063 55.0
0 6.24566984176636 56.0
81.46010082 0.0634010955691338 57.0
226.8211664 0.101068660616875 58.0
255.48073993 0.496983587741852 59.0
20.354206115 -0.0150562236085534 60.0
0 0.0745715200901031 61.0
161.14191039 0.95779150724411 0.0
161.14191039 0.869253993034363 1.0
159.79703305 1.03235638141632 2.0
154.33239765 3.28177905082703 3.0
8.8297386641 6.96745872497559 4.0
8.8297386641 3.51533198356628 5.0
-64.102868072 0.56539511680603 6.0
74.337995765 0.834116399288177 7.0
46.960557549 0.538514256477356 8.0
27.377438216 0.471354633569717 9.0
91.803377594 1.14155793190002 10.0
80.129799789 0.209053814411163 11.0
89.16267373 0.408098936080933 12.0
89.16267373 0.113205805420876 13.0
89.16267373 0.0495828837156296 14.0
0 0.214073687791824 15.0
0 0.158869922161102 16.0
0 0.631596148014069 17.0
61.030810786 1.352179646492 18.0
-61.030810786 1.34536468982697 19.0
-61.030810786 0.390250086784363 20.0
-61.030810786 -0.0374880954623222 21.0
81.171265396 2.1191246509552 22.0
138.97239811 1.84544408321381 23.0
102.55223275 0.444872617721558 24.0
23.35372007 0.0839878767728806 25.0
220.14366351 0.838371336460114 26.0
220.14366351 1.54495072364807 27.0
30.118259722 1.7382116317749 28.0
250.26192323 0.656472146511078 29.0
250.26192323 1.49969565868378 30.0
46.59211346 0.942698180675507 31.0
9.4688759485 0.429467231035233 32.0
20.14045461 1.41318535804749 33.0
20.14045461 1.00992941856384 34.0
20.14045461 0.107338570058346 35.0
9.2478547441 0.0397288762032986 36.0
8.0724055039 0.394168496131897 37.0
0 0.0367842018604279 38.0
0 0.183446437120438 39.0
-10.872679249 0.734297037124634 40.0
-23.511705313 0.34416002035141 41.0
4.8765653702 0.461494535207748 42.0
-30.841585015 0.503908753395081 43.0
92.941635776 0.578775763511658 44.0
92.941635776 0.128878951072693 45.0
7.58897931 0.480123728513718 46.0
2.9053738149 0.408728361129761 47.0
161.14191039 0.12945856153965 48.0
154.33239765 0.140853956341743 49.0
8.8297386641 0.275202333927155 50.0
74.337995765 0.19737309217453 51.0
91.803377594 0.328967720270157 52.0
89.16267373 0.0271955337375402 53.0
89.16267373 0.0384654328227043 54.0
89.16267373 0.100316852331161 55.0
0 2.55045056343079 56.0
81.171265396 0.197784453630447 57.0
220.14366351 0.474075049161911 58.0
250.26192323 0.285581082105637 59.0
20.14045461 0.0215972363948822 60.0
0 0.0992221087217331 61.0
150.89780876 0.86728310585022 0.0
150.89780876 1.20582032203674 1.0
149.77025506 1.6055554151535 2.0
144.30561966 3.08265352249146 3.0
5.1952262893 5.8505392074585 4.0
5.1952262893 7.73745679855347 5.0
-69.574055606 1.46758437156677 6.0
80.570659475 0.763031244277954 7.0
45.88951236 0.919525623321533 8.0
34.681147115 0.988069593906403 9.0
93.483370299 0.96796989440918 10.0
80.269178716 0.28920504450798 11.0
85.565072797 0.520323157310486 12.0
85.565072797 0.513362407684326 13.0
85.565072797 0.0484669022262096 14.0
0 0.180674776434898 15.0
0 0.218331053853035 16.0
0 1.2808518409729 17.0
56.475661835 1.62468039989471 18.0
-56.475661835 1.35386049747467 19.0
-56.475661835 0.299979388713837 20.0
-56.475661835 -0.0224403161555529 21.0
79.245851442 1.58871245384216 22.0
155.81228965 1.29834568500519 23.0
118.88881849 0.329315721988678 24.0
21.644155466 0.403161555528641 25.0
235.0581411 1.02815353870392 26.0
235.0581411 1.25393116474152 27.0
26.987816279 1.05735468864441 28.0
262.04595738 0.796491682529449 29.0
262.04595738 1.52353763580322 30.0
46.131299078 1.0016815662384 31.0
9.2636786025 0.333206832408905 32.0
22.770189608 1.42271602153778 33.0
22.770189608 1.01376056671143 34.0
22.770189608 0.130660638213158 35.0
9.7840986639 0.0254765786230564 36.0
8.5847611336 0.381342202425003 37.0
0 0.0415538549423218 38.0
0 0.155762270092964 39.0
-12.768059495 0.763933658599854 40.0
-26.12429349 0.575073421001434 41.0
4.8699332454 0.440277367830276 42.0
-33.60571889 0.594670236110687 43.0
93.442882727 0.647043526172638 44.0
93.442882727 0.0692112073302269 45.0
6.4343352345 0.420750170946121 46.0
2.1277854352 0.33173879981041 47.0
150.89780876 0.0728660747408867 48.0
144.30561966 0.251601815223694 49.0
5.1952262893 0.459078669548035 50.0
80.570659475 0.21327155828476 51.0
93.483370299 0.299646079540253 52.0
85.565072797 0.0468870960175991 53.0
85.565072797 0.0557647608220577 54.0
85.565072797 0.271953761577606 55.0
0 2.32923269271851 56.0
79.245851442 0.059185765683651 57.0
235.0581411 0.271643072366714 58.0
262.04595738 0.314111948013306 59.0
22.770189608 0.00367157161235809 60.0
0 0.0934552326798439 61.0
179.83102741 0.976465940475464 0.0
179.83102741 1.07511734962463 1.0
178.43445481 1.76210224628448 2.0
173.64534956 3.95944547653198 3.0
0 6.76751899719238 4.0
0 3.11552095413208 5.0
-54.916032661 0.315248608589172 6.0
64.682574512 0.777942657470703 7.0
54.924348213 0.623579263687134 8.0
9.7582262988 0.73276960849762 9.0
86.835069753 1.08118939399719 10.0
74.134666356 0.183839544653893 11.0
75.63578369 0.459049224853516 12.0
75.63578369 0.202561646699905 13.0
75.63578369 0.0635452046990395 14.0
0 0.0225120112299919 15.0
0 0.183048516511917 16.0
0 0.675414741039276 17.0
54.829610628 1.37003397941589 18.0
-54.829610628 1.65280425548553 19.0
-54.829610628 0.35101905465126 20.0
-54.829610628 0.0264303237199783 21.0
76.131229431 1.63756573200226 22.0
156.003631 1.51038491725922 23.0
120.64107195 0.388658434152603 24.0
24.868720882 0.177983731031418 25.0
232.13486043 0.890231728553772 26.0
232.13486043 1.72812974452972 27.0
23.641418929 2.1692361831665 28.0
255.77627936 1.01564502716064 29.0
255.77627936 1.79584980010986 30.0
46.477676918 1.04080486297607 31.0
9.4353048889 0.389566093683243 32.0
21.301618803 1.24403464794159 33.0
21.301618803 1.15514481067657 34.0
21.301618803 0.120211474597454 35.0
9.3219016213 0.0399438105523586 36.0
7.9186862468 0.36881810426712 37.0
0 0.0207860618829727 38.0
0 0.11916683614254 39.0
-11.438567485 0.611993491649628 40.0
-24.785431384 0.471917539834976 41.0
5.5823865453 0.457898795604706 42.0
-32.837830138 0.687511026859283 43.0
96.354792721 0.655223846435547 44.0
96.354792721 0.156880632042885 45.0
7.3117595752 0.420449554920197 46.0
2.5827726193 0.338732063770294 47.0
179.83102741 0.0734351873397827 48.0
173.64534956 0.368264347314835 49.0
0 0.155250921845436 50.0
64.682574512 0.145125031471252 51.0
86.835069753 0.315674215555191 52.0
75.63578369 0.092334620654583 53.0
75.63578369 0.0744332894682884 54.0
75.63578369 0.0566236115992069 55.0
0 3.70932221412659 56.0
76.131229431 0.165001481771469 57.0
232.13486043 0.295663297176361 58.0
255.77627936 0.481124579906464 59.0
21.301618803 -0.0075448714196682 60.0
0 0.0832955539226532 61.0
188.87506322 0.703169941902161 0.0
188.87506322 1.89526009559631 1.0
187.63228152 1.89466524124146 2.0
182.84317628 6.28420829772949 3.0
0 4.73500967025757 4.0
0 3.43871784210205 5.0
-65.119870676 1.41821444034576 6.0
74.061898598 0.476107001304626 7.0
53.620308045 0.953177690505981 8.0
20.441590553 1.09915053844452 9.0
82.92176804 0.876085937023163 10.0
70.546367607 0.192156597971916 11.0
76.141201572 0.780668258666992 12.0
76.141201572 0.30728143453598 13.0
76.141201572 0.0495510995388031 14.0
0 0.0361360721290112 15.0
0 0.180852815508842 16.0
0 1.09210431575775 17.0
56.183117892 1.33484423160553 18.0
-56.183117892 1.26120185852051 19.0
-56.183117892 0.56216299533844 20.0
-56.183117892 -0.056750975549221 21.0
77.233956382 1.53650510311127 22.0
138.98852642 0.973200798034668 23.0
101.57614931 0.754515647888184 24.0
20.554379168 0.103557199239731 25.0
216.2224828 0.980111122131348 26.0
216.2224828 1.83261716365814 27.0
24.946193492 2.57585191726685 28.0
241.1686763 1.69187462329865 29.0
241.1686763 1.98087584972382 30.0
46.955596109 0.957006514072418 31.0
9.4387536371 0.34601816534996 32.0
21.05083849 1.43954062461853 33.0
21.05083849 0.675963580608368 34.0
21.05083849 0.140933334827423 35.0
9.6475413602 0.0361094065010548 36.0
8.2683050649 0.397995233535767 37.0
0 0.0495295189321041 38.0
0 0.20982326567173 39.0
-11.571417821 0.40215852856636 40.0
-25.096634032 0.893627464771271 41.0
4.6794178542 0.385892748832703 42.0
-32.210670424 0.544162154197693 43.0
93.562627634 0.778522551059723 44.0
93.562627634 0.0669541582465172 45.0
7.0288112525 0.521315455436707 46.0
2.5360535118 0.339049875736237 47.0
188.87506322 0.14760285615921 48.0
182.84317628 0.104440800845623 49.0
0 0.0812728404998779 50.0
74.061898598 0.169223248958588 51.0
82.92176804 0.407978743314743 52.0
76.141201572 0.0688837245106697 53.0
76.141201572 0.143795445561409 54.0
76.141201572 0.0947035104036331 55.0
0 4.64217662811279 56.0
77.233956382 0.160495340824127 57.0
216.2224828 0.202070504426956 58.0
241.1686763 1.4232531785965 59.0
21.05083849 -0.00412504933774471 60.0
0 0.0920460522174835 61.0
183.599801 0.828835248947144 0.0
183.599801 1.1153005361557 1.0
182.69281499 1.31287276744843 2.0
177.2680438 5.78918886184692 3.0
0 9.05545043945312 4.0
0 3.68498706817627 5.0
-51.346869789 0.241933375597 6.0
62.067474328 0.662870466709137 7.0
38.155675748 0.408450782299042 8.0
23.91179858 0.564939022064209 9.0
92.030765893 0.855791926383972 10.0
79.683410738 0.0707109719514847 11.0
74.647508849 0.558465898036957 12.0
74.647508849 0.23669271171093 13.0
74.647508849 0.081530824303627 14.0
0 0.0659677386283875 15.0
0 0.192497134208679 16.0
0 1.57410311698914 17.0
59.471525461 1.18479013442993 18.0
-59.471525461 0.647932052612305 19.0
-60.111661594 0.516136586666107 20.0
-60.178705246 0.00752686709165573 21.0
82.908021373 2.29946088790894 22.0
153.85493244 2.04791307449341 23.0
117.64050465 0.532733976840973 24.0
20.944007795 0.0732476413249969 25.0
236.76295381 0.679670035839081 26.0
236.76295381 1.16925263404846 27.0
28.476043719 1.32620561122894 28.0
265.23899753 0.783841729164124 29.0
265.23899753 1.70522677898407 30.0
47.007805648 1.16683638095856 31.0
9.4720738608 0.404882848262787 32.0
22.729316128 1.60668170452118 33.0
22.729316128 0.806661546230316 34.0
22.729316128 0.0997215956449509 35.0
9.5049363692 0.0273416638374329 36.0
8.169344109 0.312934100627899 37.0
0 0.0597212538123131 38.0
0 0.357403188943863 39.0
-11.380389967 0.580843269824982 40.0
-24.395925419 0.598570168018341 41.0
5.1801952835 0.433798670768738 42.0
-31.824345909 0.357793658971786 43.0
84.579806409 0.763542294502258 44.0
84.579806409 0.116211257874966 45.0
7.1962661842 0.553125500679016 46.0
2.1744031669 0.36443042755127 47.0
183.599801 0.126520603895187 48.0
177.2680438 0.22774413228035 49.0
0 0.691395699977875 50.0
62.067474328 0.0747224688529968 51.0
92.030765893 0.0699035376310349 52.0
74.647508849 0.037280660122633 53.0
74.647508849 0.144278049468994 54.0
74.647508849 0.0351504161953926 55.0
0 5.23740148544312 56.0
82.908021373 0.197541356086731 57.0
236.76295381 0.531882226467133 58.0
265.23899753 0.449224531650543 59.0
22.729316128 0.00962529331445694 60.0
0 0.105543091893196 61.0
196.07354072 0.869911730289459 0.0
196.07354072 0.764820277690887 1.0
195.02723149 1.44651818275452 2.0
190.23812622 4.50678634643555 3.0
0 3.61638784408569 4.0
0 2.50373435020447 5.0
-66.597458366 0.301026821136475 6.0
76.774369082 0.524801433086395 7.0
63.547271611 0.853726148605347 8.0
13.227097471 1.0912446975708 9.0
80.354288348 1.0747321844101 10.0
66.916445479 0.286953330039978 11.0
80.982865245 0.491379410028458 12.0
80.982865245 0.452972531318665 13.0
80.982865245 0.11823620647192 14.0
0 0.0873489528894424 15.0
0 0.129491537809372 16.0
0 1.43955135345459 17.0
64.151479407 0.756116926670074 18.0
-64.151479407 0.433558702468872 19.0
-64.151479407 0.917169511318207 20.0
-64.151479407 0.0780338197946548 21.0
85.867739137 1.03860449790955 22.0
151.47229701 0.724398612976074 23.0
113.41935844 0.514189422130585 24.0
22.690709363 0.35020649433136 25.0
237.34003615 0.757354855537415 26.0
237.34003615 2.42694783210754 27.0
26.232059969 2.35700821876526 28.0
263.57209612 0.82317328453064 29.0
263.57209612 1.63118994235992 30.0
46.297537063 1.17678999900818 31.0
9.4103637321 0.377962917089462 32.0
21.71625973 0.934963881969452 33.0
21.71625973 1.62302613258362 34.0
21.71625973 0.120103642344475 35.0
8.8735011357 0.0324453450739384 36.0
7.7840258981 0.34164297580719 37.0
0 -0.00126349925994873 38.0
0 0.200097486376762 39.0
-10.849335021 0.539505839347839 40.0
-23.872379271 0.784615159034729 41.0
4.7050989684 0.369346380233765 42.0
-30.686953078 0.339019030332565 43.0
87.975364028 0.708071887493134 44.0
87.975364028 0.088579997420311 45.0
6.25782212 0.635868847370148 46.0
2.2484677008 0.422927588224411 47.0
196.07354072 0.12453755736351 48.0
190.23812622 0.188718065619469 49.0
0 0.0490951798856258 50.0
76.774369082 0.09551902115345 51.0
80.354288348 0.282772183418274 52.0
80.982865245 0.0639293342828751 53.0
80.982865245 0.124243505299091 54.0
80.982865245 0.0632199943065643 55.0
0 2.63219285011292 56.0
85.867739137 -0.0341306775808334 57.0
237.34003615 0.550527215003967 58.0
263.57209612 0.0955172926187515 59.0
21.71625973 -0.00792138651013374 60.0
0 0.084989532828331 61.0
139.46439154 0.821917414665222 0.0
139.46439154 2.07544136047363 1.0
138.35274148 1.67363321781158 2.0
133.56363638 3.70540904998779 3.0
0 9.21518898010254 4.0
0 5.45186901092529 5.0
-53.900238456 1.32018029689789 6.0
64.782480605 1.43512034416199 7.0
34.802993953 0.465297341346741 8.0
29.979486652 0.281267791986465 9.0
100.10107957 0.955962121486664 10.0
87.413917198 0.311968415975571 11.0
88.625932489 0.434021592140198 12.0
88.625932489 0.0780649706721306 13.0
88.625932489 0.222855925559998 14.0
0 0.492953509092331 15.0
0 0.109600439667702 16.0
0 1.30247521400452 17.0
56.357415993 1.20509684085846 18.0
-56.357415993 1.23752689361572 19.0
-56.357415993 0.710854768753052 20.0
-56.357415993 -0.0588981099426746 21.0
78.682421028 1.31993854045868 22.0
147.07262646 0.856633365154266 23.0
111.4874583 0.633288025856018 24.0
21.147818881 0.164784207940102 25.0
225.75504749 1.10465884208679 26.0
225.75504749 1.42270410060883 27.0
28.44497264 1.58925604820251 28.0
254.20002013 0.592666327953339 29.0
254.20002013 1.4925571680069 30.0
47.084157205 1.05635857582092 31.0
9.1862408391 0.432987600564957 32.0
22.325005035 1.4187114238739 33.0
22.325005035 1.41216349601746 34.0
22.325005035 0.0874509960412979 35.0
9.41497378 0.0231153629720211 36.0
8.1061249364 0.387341260910034 37.0
0 0.0243514329195023 38.0
0 0.17696550488472 39.0
-12.457967823 0.682379424571991 40.0
-25.846971626 0.592749714851379 41.0
5.4412671025 0.459930181503296 42.0
-33.651480301 0.589634478092194 43.0
100.45626612 0.655949652194977 44.0
100.45626612 0.0791252329945564 45.0
7.1575188313 0.728054583072662 46.0
2.2175748677 0.374700039625168 47.0
139.46439154 0.113300576806068 48.0
133.56363638 0.130740493535995 49.0
0 0.705446302890778 50.0
64.782480605 0.154882550239563 51.0
100.10107957 0.504695653915405 52.0
88.625932489 0.0171865951269865 53.0
88.625932489 0.0729084461927414 54.0
88.625932489 0.0871302410960197 55.0
0 0.438397973775864 56.0
78.682421028 0.107923820614815 57.0
225.75504749 0.301680743694305 58.0
254.20002013 0.592355132102966 59.0
22.325005035 -0.0127373710274696 60.0
0 0.0846060365438461 61.0
190.8585275 1.0005019903183 0.0
190.8585275 1.19795346260071 1.0
189.75931857 1.45924985408783 2.0
184.97021336 4.00539064407349 3.0
0 7.07184886932373 4.0
0 3.03756141662598 5.0
-53.94490302 0.963401556015015 6.0
63.7109191 0.427873849868774 7.0
42.192608111 1.18428456783295 8.0
21.518310989 1.01411640644073 9.0
89.148664939 0.910091578960419 10.0
76.221579969 0.0750087201595306 11.0
81.679654974 0.815099537372589 12.0
81.679654974 0.254145443439484 13.0
81.679654974 0.0801299959421158 14.0
0 0.0952318757772446 15.0
0 0.179602399468422 16.0
0 1.34726905822754 17.0
53.104131249 1.02069485187531 18.0
-53.104131249 0.823024034500122 19.0
-53.104131249 0.4169782102108 20.0
-53.104131249 -0.0258212871849537 21.0
74.277987348 2.64313173294067 22.0
142.73809387 2.81012368202209 23.0
107.15732567 0.306297987699509 24.0
21.611592032 0.153870925307274 25.0
217.01608122 1.34226739406586 26.0
217.01608122 1.94052624702454 27.0
27.085556367 1.67534422874451 28.0
244.10163758 0.952917397022247 29.0
244.10163758 1.60614907741547 30.0
46.239666747 0.950376808643341 31.0
9.2327918844 0.416693896055222 32.0
21.173856099 1.49378800392151 33.0
21.173856099 0.936330080032349 34.0
21.173856099 0.0983858704566956 35.0
10.172530957 0.0289370734244585 36.0
8.7924787994 0.398350805044174 37.0
0 0.0176047682762146 38.0
0 0.0849298760294914 39.0
-11.712973924 1.08220458030701 40.0
-25.249174759 0.357579082250595 41.0
4.6866270086 0.571584463119507 42.0
-32.545620328 0.669906973838806 43.0
102.03014019 0.733125865459442 44.0
102.03014019 0.155404910445213 45.0
6.9389560573 0.62269914150238 46.0
2.2914827065 0.405826270580292 47.0
190.8585275 0.229760348796844 48.0
184.97021336 0.037313424050808 49.0
0 0.276827901601791 50.0
63.7109191 0.094112329185009 51.0
89.148664939 0.105899892747402 52.0
81.679654974 0.0944505929946899 53.0
81.679654974 0.164923146367073 54.0
81.679654974 0.0835696458816528 55.0
0 1.71689307689667 56.0
74.277987348 0.0621507875621319 57.0
217.01608122 0.54833847284317 58.0
244.10163758 0.422760367393494 59.0
21.173856099 -0.00694520026445389 60.0
0 0.097477950155735 61.0
142.74604951 0.68713903427124 0.0
142.74604951 0.3538898229599 1.0
141.55866532 1.31185781955719 2.0
136.7695602 5.07539510726929 3.0
0 5.93600273132324 4.0
0 5.82545280456543 5.0
-49.311553465 0.590756893157959 6.0
63.002086048 1.19496238231659 7.0
65.631830367 0.334919571876526 8.0
-2.6297443186 0.365982681512833 9.0
81.245427385 0.853032648563385 10.0
67.036109934 0.304901540279388 11.0
79.958000699 0.110757872462273 12.0
79.958000699 0.749452233314514 13.0
79.958000699 0.0982026308774948 14.0
0 0.231242656707764 15.0
0 0.134355157613754 16.0
0 1.48297357559204 17.0
60.703964604 1.55294632911682 18.0
-60.703964604 0.936048567295074 19.0
-60.703964604 1.03283441066742 20.0
-60.703964604 -0.0864406377077103 21.0
83.033484234 1.4497903585434 22.0
146.09139199 1.52269911766052 23.0
117.23119705 0.515294134616852 24.0
23.653949303 0.373265385627747 25.0
229.12487622 0.952019214630127 26.0
229.12487622 3.12759017944336 27.0
27.349261669 2.74948000907898 28.0
256.47413789 0.935590505599976 29.0
256.47413789 1.98055708408356 30.0
47.271170523 1.46053242683411 31.0
9.2235099867 0.472444474697113 32.0
22.32951963 1.28346073627472 33.0
22.32951963 1.03036260604858 34.0
22.32951963 0.124705165624619 35.0
9.1828001075 0.0291625428944826 36.0
8.0403714396 0.311949551105499 37.0
0 0.0248873755335808 38.0
0 0.415577739477158 39.0
-12.567941981 0.143715053796768 40.0
-26.350837236 0.759322941303253 41.0
5.7948197296 0.326654076576233 42.0
-34.725137157 0.14908641576767 43.0
97.436089876 0.840394794940948 44.0
97.436089876 0.0425473600625992 45.0
6.8539634656 0.596436977386475 46.0
2.6517124136 0.356694430112839 47.0
142.74604951 0.0577560216188431 48.0
136.7695602 0.444506168365479 49.0
0 1.02191591262817 50.0
63.002086048 0.0302639007568359 51.0
81.245427385 0.195648148655891 52.0
79.958000699 0.032800480723381 53.0
79.958000699 0.143166586756706 54.0
79.958000699 0.0982942879199982 55.0
0 2.99967932701111 56.0
83.033484234 0.0992583483457565 57.0
229.12487622 1.00655853748322 58.0
256.47413789 0.214518740773201 59.0
22.32951963 -0.0131614990532398 60.0
0 0.102208591997623 61.0
166.42419926 0.75499826669693 0.0
166.42419926 1.91173803806305 1.0
164.94799061 2.23576974868774 2.0
160.15888535 4.60946846008301 3.0
0 5.74107646942139 4.0
0 4.00381803512573 5.0
-76.174497139 0.500128090381622 6.0
85.47838914 0.452719271183014 7.0
61.313973163 0.789379477500916 8.0
24.164415977 1.05643248558044 9.0
91.0512978 0.852710843086243 10.0
78.033797094 0.135746359825134 11.0
88.159680923 0.10474256426096 12.0
88.159680923 0.0871911495923996 13.0
88.159680923 0.180654600262642 14.0
0 0.232104003429413 15.0
0 0.125714495778084 16.0
0 1.69825851917267 17.0
49.269340366 1.37055218219757 18.0
-49.269340366 0.639573276042938 19.0
-49.269340366 1.35354208946228 20.0
-49.269340366 -0.037708442658186 21.0
68.588435665 1.41933619976044 22.0
150.78620558 1.34553170204163 23.0
120.36162028 0.645498216152191 24.0
24.808997969 0.201083064079285 25.0
219.37464124 1.05355858802795 26.0
219.37464124 2.53926181793213 27.0
27.086950151 2.75331544876099 28.0
246.46159139 1.04516386985779 29.0
246.46159139 1.82669377326965 30.0
46.645018734 1.04536485671997 31.0
9.3442621207 0.529061913490295 32.0
19.319095298 1.82581806182861 33.0
19.319095298 1.21593940258026 34.0
19.319095298 0.128362730145454 35.0
9.3447467622 0.033019982278347 36.0
8.0705277871 0.333913326263428 37.0
0 0.0512009002268314 38.0
0 0.111725032329559 39.0
-11.331664907 0.553694903850555 40.0
-25.105023294 0.449083507061005 41.0
5.3579932695 0.525624632835388 42.0
-32.972486439 0.830201268196106 43.0
90.824998913 0.433116763830185 44.0
90.824998913 0.0915957689285278 45.0
7.0577661818 0.506538808345795 46.0
2.3835067776 0.388944774866104 47.0
166.42419926 0.129950791597366 48.0
160.15888535 0.493453085422516 49.0
0 0.135967925190926 50.0
85.47838914 0.0183766297996044 51.0
91.0512978 0.354726403951645 52.0
88.159680923 0.00821901578456163 53.0
88.159680923 0.0668197274208069 54.0
88.159680923 0.0653550624847412 55.0
0 1.0675436258316 56.0
68.588435665 0.0475205928087234 57.0
219.37464124 0.230949059128761 58.0
246.46159139 0.328949749469757 59.0
19.319095298 -0.0153250563889742 60.0
0 0.0972310155630112 61.0
173.21866956 0.912407577037811 0.0
173.21866956 1.52905333042145 1.0
172.12572008 1.42294347286224 2.0
166.66104948 3.53152799606323 3.0
4.2027303981 6.5576171875 4.0
4.2027303981 4.65009498596191 5.0
-68.019427243 1.47085392475128 6.0
80.207610859 1.08173537254333 7.0
41.953435913 0.524902880191803 8.0
38.254174946 0.335376977920532 9.0
91.722934843 0.949978947639465 10.0
77.799504396 0.210432752966881 11.0
85.467399562 0.267574310302734 12.0
85.467399562 0.451202183961868 13.0
85.467399562 0.0990901663899422 14.0
0 0.212548688054085 15.0
0 0.185726553201675 16.0
0 0.85372120141983 17.0
53.80660323 1.44712340831757 18.0
-53.80660323 1.7136595249176 19.0
-53.80660323 0.442668676376343 20.0
-53.80660323 0.0854079872369766 21.0
75.721415591 1.8749293088913 22.0
147.1005976 1.86466789245605 23.0
115.07118212 0.398691803216934 24.0
21.520352903 0.232755944132805 25.0
222.82201319 1.30696749687195 26.0
222.82201319 1.64186275005341 27.0
27.215312311 1.59492754936218 28.0
250.0373255 1.06444644927979 29.0
250.0373255 1.63429307937622 30.0
46.946254092 1.07968056201935 31.0
9.4645721068 0.30881404876709 32.0
21.914812361 0.993964374065399 33.0
21.914812361 0.990124046802521 34.0
21.914812361 0.122881568968296 35.0
9.6864399737 0.0283771101385355 36.0
8.4578431951 0.398631393909454 37.0
0 0.0125485807657242 38.0
0 0.245563879609108 39.0
-12.188389927 0.693805038928986 40.0
-26.010787768 0.399028956890106 41.0
4.6668853521 0.551266133785248 42.0
-33.211288397 0.476232767105103 43.0
91.597911028 0.782275438308716 44.0
91.597911028 0.0769986435770988 45.0
7.1495326394 0.689687848091125 46.0
2.8272700953 0.416535198688507 47.0
173.21866956 0.058256134390831 48.0
166.66104948 0.130899295210838 49.0
4.2027303981 0.448600441217422 50.0
80.207610859 0.163822278380394 51.0
91.722934843 0.25865250825882 52.0
85.467399562 0.0603558979928493 53.0
85.467399562 0.0326527021825314 54.0
85.467399562 0.101594664156437 55.0
0 4.26458024978638 56.0
75.721415591 0.0562261566519737 57.0
222.82201319 0.457850128412247 58.0
250.0373255 0.575981497764587 59.0
21.914812361 -0.0135955754667521 60.0
0 0.118760302662849 61.0
152.80661014 1.1574319601059 0.0
152.80661014 1.53420495986938 1.0
151.52005181 1.68383169174194 2.0
146.0554164 6.8034815788269 3.0
0.042117307104 7.65908527374268 4.0
0.042117307104 3.01325225830078 5.0
-62.236038146 0.218774139881134 6.0
75.338465707 0.450925230979919 7.0
35.292306228 1.45360863208771 8.0
40.046159479 1.26777243614197 9.0
81.957754525 1.12483203411102 10.0
68.865691774 0.102764815092087 11.0
78.94112399 0.459630727767944 12.0
78.94112399 0.885651648044586 13.0
78.94112399 0.0762523412704468 14.0
0 0.12498639523983 15.0
0 0.180213540792465 16.0
0 1.90804553031921 17.0
56.092969884 1.31162309646606 18.0
-56.092969884 1.3930698633194 19.0
-56.092969884 0.331257820129395 20.0
-56.092969884 0.213887304067612 21.0
75.175939183 2.2175817489624 22.0
148.86058834 2.61493396759033 23.0
113.63641074 0.387292504310608 24.0
23.11353803 0.184859246015549 25.0
224.03652752 1.11442124843597 26.0
224.03652752 1.19085597991943 27.0
29.191875434 0.884713292121887 28.0
253.22840296 0.515538811683655 29.0
253.22840296 1.30746698379517 30.0
46.684188432 1.10523617267609 31.0
9.2253850079 0.414678454399109 32.0
19.082969299 1.40394151210785 33.0
19.082969299 1.00027358531952 34.0
19.082969299 0.150735214352608 35.0
8.7905097066 0.0345863401889801 36.0
7.4281106342 0.378772795200348 37.0
0 0.0284460224211216 38.0
0 0.202307969331741 39.0
-11.208933044 0.66521418094635 40.0
-25.001210748 0.471381098031998 41.0
5.4669387141 0.395033001899719 42.0
-33.297393114 0.459524065256119 43.0
95.10452003 0.876147508621216 44.0
95.10452003 0.0689732059836388 45.0
7.5457998013 0.638002634048462 46.0
2.8138329325 0.414629429578781 47.0
152.80661014 0.156229197978973 48.0
146.0554164 0.222677335143089 49.0
0.042117307104 0.663412749767303 50.0
75.338465707 0.128974169492722 51.0
81.957754525 0.246856391429901 52.0
78.94112399 0.0362556762993336 53.0
78.94112399 0.0499402061104774 54.0
78.94112399 0.356980800628662 55.0
0 4.25725603103638 56.0
75.175939183 0.109552539885044 57.0
224.03652752 0.521842241287231 58.0
253.22840296 0.326162815093994 59.0
19.082969299 -0.000339560210704803 60.0
0 0.0813726708292961 61.0
183.87647201 0.724262118339539 0.0
183.87647201 0.746528744697571 1.0
182.5511627 1.15788412094116 2.0
177.76205749 5.73163604736328 3.0
0 8.23821449279785 4.0
0 4.47265434265137 5.0
-70.714749822 0.577858090400696 6.0
82.115998346 0.614256501197815 7.0
52.95531968 0.810751974582672 8.0
29.160678666 0.918853878974915 9.0
84.400607897 0.759489774703979 10.0
70.400109894 0.0208443365991116 11.0
82.928300568 0.894464075565338 12.0
82.928300568 0.496457636356354 13.0
82.928300568 0.131393238902092 14.0
0 0.275168597698212 15.0
0 0.203541234135628 16.0
0 1.43505823612213 17.0
59.543728838 1.04095065593719 18.0
-59.543728838 0.880423843860626 19.0
-59.543728838 0.755787491798401 20.0
-59.543728838 -0.0524636879563332 21.0
79.78716443 1.18752479553223 22.0
132.37919677 0.861376345157623 23.0
101.27480821 0.473398417234421 24.0
21.259985398 0.097861036658287 25.0
212.1663612 1.12722313404083 26.0
212.1663612 2.18889975547791 27.0
24.418641089 1.75149750709534 28.0
236.58500229 0.566290438175201 29.0
236.58500229 1.43969583511353 30.0
46.431463669 1.03617131710052 31.0
9.2741504373 0.38399937748909 32.0
20.243435593 1.56927180290222 33.0
20.243435593 0.818757116794586 34.0
20.243435593 0.112282007932663 35.0
8.9591845316 0.021194115281105 36.0
7.5861003641 0.355176091194153 37.0
0 0.0791712552309036 38.0
0 0.146055549383163 39.0
-12.383907707 0.955996930599213 40.0
-24.688522991 0.471506536006927 41.0
5.090569243 0.532995879650116 42.0
-32.374844728 0.345671027898788 43.0
100.1827157 0.785700738430023 44.0
100.1827157 0.0702250152826309 45.0
6.6968092732 0.661438524723053 46.0
2.3369685322 0.381128370761871 47.0
183.87647201 0.071659579873085 48.0
177.76205749 0.181152537465096 49.0
0 0.44220033288002 50.0
82.115998346 0.268254578113556 51.0
84.400607897 0.153606474399567 52.0
82.928300568 0.0103325387462974 53.0
82.928300568 0.168869256973267 54.0
82.928300568 0.122906967997551 55.0
0 3.63437581062317 56.0
79.78716443 0.0798635110259056 57.0
212.1663612 0.56335723400116 58.0
236.58500229 0.388346612453461 59.0
20.243435593 0.00996252149343491 60.0
0 0.0883665308356285 61.0
195.67472878 0.813340961933136 0.0
195.67472878 1.20504474639893 1.0
194.22078269 1.71238875389099 2.0
189.43167747 4.0718879699707 3.0
0 4.58668994903564 4.0
0 4.13373374938965 5.0
-64.892285566 1.84705829620361 6.0
75.508974715 1.45049452781677 7.0
63.818842993 0.580954790115356 8.0
11.690131722 0.507058084011078 9.0
85.326405068 0.997808277606964 10.0
71.888216331 0.302419751882553 11.0
77.817088564 0.442983657121658 12.0
77.817088564 0.306890279054642 13.0
77.817088564 0.128170847892761 14.0
0 0.0896884500980377 15.0
0 0.140974938869476 16.0
0 1.14367723464966 17.0
57.746188752 1.58691298961639 18.0
-57.746188752 1.82106125354767 19.0
-57.746188752 0.629290580749512 20.0
-57.746188752 0.0308217816054821 21.0
81.079862566 2.20060348510742 22.0
151.79377049 1.88442254066467 23.0
117.85850085 0.529588878154755 24.0
21.075663728 0.123052403330803 25.0
232.87363306 1.42586708068848 26.0
232.87363306 1.38634479045868 27.0
23.907438113 1.26756823062897 28.0
256.78107117 0.718605041503906 29.0
256.78107117 2.00858044624329 30.0
46.453613364 1.19369292259216 31.0
9.3331831638 0.564865708351135 32.0
23.333673814 1.15061569213867 33.0
23.333673814 0.777038812637329 34.0
23.333673814 0.0816832855343819 35.0
9.9757547374 0.0245640631765127 36.0
8.6792211595 0.405501335859299 37.0
0 0.0214800499379635 38.0
0 0.149923026561737 39.0
-12.011803058 0.830051839351654 40.0
-25.567112953 0.602732062339783 41.0
5.036650855 0.469303041696548 42.0
-33.128773394 0.365351617336273 43.0
99.191092334 0.744701981544495 44.0
99.191092334 0.0722789168357849 45.0
7.1446727336 0.583222508430481 46.0
2.6018639254 0.369330108165741 47.0
195.67472878 0.0647255033254623 48.0
189.43167747 0.271059304475784 49.0
0 0.148783966898918 50.0
75.508974715 0.145295292139053 51.0
85.326405068 0.177596196532249 52.0
77.817088564 0.043307363986969 53.0
77.817088564 0.156196475028992 54.0
77.817088564 0.0345944538712502 55.0
0 4.92736482620239 56.0
81.079862566 -0.00278545590117574 57.0
232.87363306 0.744322896003723 58.0
256.78107117 0.723100185394287 59.0
23.333673814 0.0121690463274717 60.0
0 0.0948276519775391 61.0
192.28867817 1.13427889347076 0.0
192.28867817 0.898150026798248 1.0
191.10062484 1.22590863704681 2.0
185.63595405 3.23514556884766 3.0
9.7559531357 3.91610598564148 4.0
9.7559531357 4.87962627410889 5.0
-50.694797717 0.746023952960968 6.0
63.739225439 0.500381410121918 7.0
36.69422996 0.771762847900391 8.0
27.044995479 0.805275559425354 9.0
98.757445886 1.2426837682724 10.0
85.975071636 0.184908747673035 11.0
87.602313917 0.128387689590454 12.0
87.602313917 0.389976322650909 13.0
87.602313917 0.0468078404664993 14.0
0 0.224746838212013 15.0
0 0.187970548868179 16.0
0 1.15356731414795 17.0
64.726463988 1.46970474720001 18.0
-64.726463988 1.43875432014465 19.0
-64.726463988 0.342710018157959 20.0
-64.726463988 -0.0316304601728916 21.0
87.210894681 1.35466063022614 22.0
144.55046639 0.767607390880585 23.0
111.82196814 0.571006953716278 24.0
20.893082796 0.103298671543598 25.0
231.76136107 1.0329065322876 26.0
231.76136107 1.03885388374329 27.0
27.024125607 1.58340537548065 28.0
258.78548668 0.645426869392395 29.0
258.78548668 1.72048997879028 30.0
46.731107402 1.08054578304291 31.0
9.3545407714 0.549875378608704 32.0
22.484430693 1.65994822978973 33.0
22.484430693 0.793001115322113 34.0
22.484430693 0.155017778277397 35.0
9.6704322151 0.0293239112943411 36.0
8.4310718005 0.38081306219101 37.0
0 0.0445606522262096 38.0
0 0.0339599661529064 39.0
-12.24505794 0.779786050319672 40.0
-26.083992289 0.368925303220749 41.0
5.4952323368 0.513725757598877 42.0
-33.803817456 0.496305167675018 43.0
75.776307173 0.564333140850067 44.0
75.776307173 0.0946266278624535 45.0
6.3292207936 0.633773267269135 46.0
1.373805183 0.442561626434326 47.0
192.28867817 0.186049073934555 48.0
185.63595405 0.138919323682785 49.0
9.7559531357 0.242709919810295 50.0
63.739225439 0.0933412909507751 51.0
98.757445886 0.103689536452293 52.0
87.602313917 0.0228639058768749 53.0
87.602313917 0.0785622298717499 54.0
87.602313917 0.080122746527195 55.0
0 0.286335289478302 56.0
87.210894681 0.0704411193728447 57.0
231.76136107 0.229506775736809 58.0
258.78548668 0.752267837524414 59.0
22.484430693 -0.0290928110480309 60.0
0 0.11421237885952 61.0
170.10180393 0.960724472999573 0.0
170.10180393 1.30219793319702 1.0
168.59540652 1.28898394107819 2.0
163.80630136 3.68016290664673 3.0
0 8.60775375366211 4.0
0 7.237229347229 5.0
-75.993910668 4.52328681945801 6.0
85.296253782 0.475162982940674 7.0
45.592333074 1.32261121273041 8.0
39.703920709 1.25948667526245 9.0
81.017628636 0.979635953903198 10.0
68.933191404 0.154873788356781 11.0
80.467099004 0.382612824440002 12.0
80.467099004 0.342232704162598 13.0
80.467099004 0.0526855885982513 14.0
0 0.114177472889423 15.0
0 0.176098391413689 16.0
0 1.18915069103241 17.0
57.446357347 1.10486316680908 18.0
-57.446357347 0.769800543785095 19.0
-57.446357347 0.511732697486877 20.0
-57.446357347 0.0183108448982239 21.0
78.787642768 1.50973165035248 22.0
136.66360072 1.06217336654663 23.0
104.33199887 0.477751076221466 24.0
21.591771204 0.208862021565437 25.0
215.45124349 1.47735023498535 26.0
215.45124349 1.9786125421524 27.0
25.948347768 1.53400433063507 28.0
241.39959126 0.947940587997437 29.0
241.39959126 1.84259831905365 30.0
46.35587349 0.941032707691193 31.0
9.4460554194 0.502442240715027 32.0
21.341285421 1.36726570129395 33.0
21.341285421 1.17934346199036 34.0
21.341285421 0.123751565814018 35.0
9.7898339181 0.0413623340427876 36.0
8.5565345154 0.399879485368729 37.0
0 0.0150809623301029 38.0
0 0.146850228309631 39.0
-12.030080944 0.797837018966675 40.0
-23.482182342 0.43800014257431 41.0
4.7389071615 0.497301608324051 42.0
-30.645622724 0.735414981842041 43.0
90.181480492 0.534179091453552 44.0
90.181480492 0.098177582025528 45.0
6.4878036499 0.578291475772858 46.0
2.6639004223 0.421228557825089 47.0
170.10180393 0.167831301689148 48.0
163.80630136 0.185780301690102 49.0
0 0.423189073801041 50.0
85.296253782 0.257330358028412 51.0
81.017628636 0.260146796703339 52.0
80.467099004 0.0317088887095451 53.0
80.467099004 0.0556887350976467 54.0
80.467099004 0.132878229022026 55.0
0 3.14028549194336 56.0
78.787642768 -0.0154770771041512 57.0
215.45124349 0.575783789157867 58.0
241.39959126 0.473880618810654 59.0
21.341285421 -0.00903134699910879 60.0
0 0.098930299282074 61.0
146.67767097 0.576620399951935 0.0
146.67767097 0.598128020763397 1.0
145.52939458 1.04930436611176 2.0
140.7402893 2.67339491844177 3.0
0 4.54265928268433 4.0
0 3.63373041152954 5.0
-53.809868416 0.0288953185081482 6.0
67.608243861 0.843236088752747 7.0
50.63609335 0.79200541973114 8.0
16.972150512 0.539347052574158 9.0
95.603913935 0.821245014667511 10.0
81.748268462 0.468297123908997 11.0
76.485854912 0.115633003413677 12.0
76.485854912 0.176337122917175 13.0
76.485854912 0.0214585307985544 14.0
0 0.0125343948602676 15.0
0 0.228552103042603 16.0
0 0.93834388256073 17.0
53.572498565 0.825685083866119 18.0
-53.572498565 0.446372598409653 19.0
-53.572498565 0.958040773868561 20.0
-53.572498565 -0.072147972881794 21.0
73.010685594 2.36211323738098 22.0
132.75004287 1.35930001735687 23.0
100.58171308 0.369223058223724 24.0
19.965361254 0.101204842329025 25.0
205.76072847 0.626868426799774 26.0
205.76072847 1.77039563655853 27.0
26.02935168 1.63376152515411 28.0
231.79008015 0.762920200824738 29.0
231.79008015 1.63045430183411 30.0
46.295782642 1.42948305606842 31.0
9.4123492983 0.304036766290665 32.0
19.438187028 1.52311217784882 33.0
19.438187028 1.41151762008667 34.0
19.438187028 0.0749920979142189 35.0
10.354199312 0.0242207795381546 36.0
9.1454806497 0.347582042217255 37.0
0 0.0316654816269875 38.0
0 0.168430745601654 39.0
-11.911875944 0.645765781402588 40.0
-25.589823335 0.827216565608978 41.0
5.4976819224 0.538897633552551 42.0
-33.345804166 0.644768416881561 43.0
89.704127999 0.698005080223083 44.0
89.704127999 0.1261046230793 45.0
6.4783411312 0.565424084663391 46.0
2.6495904387 0.407943785190582 47.0
146.67767097 0.0942981243133545 48.0
140.7402893 0.0827504396438599 49.0
0 0.194096460938454 50.0
67.608243861 0.248959109187126 51.0
95.603913935 0.635048985481262 52.0
76.485854912 0.0201394893229008 53.0
76.485854912 0.0695087164640427 54.0
76.485854912 0.0751060247421265 55.0
0 2.51979327201843 56.0
73.010685594 0.187336161732674 57.0
205.76072847 0.461408942937851 58.0
231.79008015 0.0765946209430695 59.0
19.438187028 0.00651071593165398 60.0
0 0.0595848597586155 61.0
213.85488752 0.850777983665466 0.0
213.85488752 1.45977795124054 1.0
212.43915062 1.43793451786041 2.0
206.97448008 3.85407400131226 3.0
19.60191405 3.92552900314331 4.0
19.60191405 4.66167306900024 5.0
-52.708616027 1.24252283573151 6.0
63.164775212 0.71654486656189 7.0
37.844364933 0.9754279255867 8.0
25.320410279 1.09589552879333 9.0
98.022827169 1.11597633361816 10.0
84.094504413 0.812816321849823 11.0
87.480188325 0.717076599597931 12.0
87.480188325 0.623382031917572 13.0
87.480188325 0.120004370808601 14.0
0 0.248503789305687 15.0
0 0.0979900881648064 16.0
0 1.8169082403183 17.0
59.761197724 1.37753546237946 18.0
-59.761197724 1.57003581523895 19.0
-59.761197724 0.213470727205276 20.0
-59.761197724 0.218026682734489 21.0
81.570558986 1.58292520046234 22.0
137.93948102 0.985786616802216 23.0
106.41495686 0.480402827262878 24.0
24.926152679 0.159593373537064 25.0
219.51004 1.01917386054993 26.0
219.51004 1.88431882858276 27.0
28.86491318 1.64790117740631 28.0
248.37495318 0.680101931095123 29.0
248.37495318 1.71905755996704 30.0
46.674570153 0.997226893901825 31.0
9.5046754297 0.611616849899292 32.0
21.809361263 1.00001156330109 33.0
21.809361263 0.746498167514801 34.0
21.809361263 0.100077785551548 35.0
8.3968995831 0.0244577378034592 36.0
7.0931649288 0.351789921522141 37.0
0 0.0313177816569805 38.0
0 0.164818048477173 39.0
-12.601215276 0.447944104671478 40.0
-23.910136667 0.509290456771851 41.0
4.7031466069 0.374268442392349 42.0
-31.078353678 0.410178303718567 43.0
100.68738708 0.677571654319763 44.0
100.68738708 0.110831446945667 45.0
7.0705780592 0.522520422935486 46.0
3.1131638707 0.366348773241043 47.0
213.85488752 0.085432231426239 48.0
206.97448008 0.10725475102663 49.0
19.60191405 0.112719051539898 50.0
63.164775212 0.353134006261826 51.0
98.022827169 0.572458267211914 52.0
87.480188325 0.21811580657959 53.0
87.480188325 0.0613125413656235 54.0
87.480188325 0.300947993993759 55.0
0 2.28575229644775 56.0
81.570558986 -0.0083417184650898 57.0
219.51004 0.741655647754669 58.0
248.37495318 0.393538892269135 59.0
21.809361263 -0.00200636126101017 60.0
0 0.0897916704416275 61.0
144.2376365 1.05886507034302 0.0
144.2376365 1.74253225326538 1.0
142.9363782 1.70661079883575 2.0
138.14727292 3.52041125297546 3.0
0 3.78934621810913 4.0
0 5.01417207717896 5.0
-70.427278915 2.24300861358643 6.0
78.758025347 1.18602991104126 7.0
58.929114606 0.353847175836563 8.0
19.828910741 0.349697530269623 9.0
75.078033552 1.04273056983948 10.0
61.326469486 0.034776471555233 11.0
71.371202418 0.359487116336823 12.0
71.371202418 0.113836519420147 13.0
71.371202418 0.0831944718956947 14.0
0 0.0963269546627998 15.0
0 0.206071749329567 16.0
0 1.60104632377625 17.0
63.778407216 1.23104119300842 18.0
-63.778407216 1.08023357391357 19.0
-63.778407216 0.325132727622986 20.0
-63.778407216 -0.00283562764525414 21.0
83.549250838 1.22016310691833 22.0
139.4133214 1.23278868198395 23.0
102.52240187 0.502338171005249 24.0
22.232820772 0.071165457367897 25.0
222.96257224 0.700666904449463 26.0
222.96257224 1.93306684494019 27.0
26.176322026 2.17234134674072 28.0
249.13889427 0.971705555915833 29.0
249.13889427 1.80358052253723 30.0
46.743958249 1.36456990242004 31.0
9.2229714802 0.429989993572235 32.0
19.770843622 1.62763750553131 33.0
19.770843622 0.906900703907013 34.0
19.770843622 0.106985181570053 35.0
9.0793469498 0.0262647140771151 36.0
7.9215724319 0.38176640868187 37.0
0 0.0017007514834404 38.0
0 0.306390285491943 39.0
-11.78083066 0.330262035131454 40.0
-25.614878632 0.603588402271271 41.0
5.1300138267 0.494087725877762 42.0
-33.581326597 0.606301248073578 43.0
80.372110489 0.681068181991577 44.0
80.372110489 0.0954227671027184 45.0
6.1039149004 0.598820149898529 46.0
2.5483054971 0.305608451366425 47.0
144.2376365 0.131041169166565 48.0
138.14727292 0.115496642887592 49.0
0 0.267780750989914 50.0
78.758025347 0.0753505229949951 51.0
75.078033552 0.0941787511110306 52.0
71.371202418 0.0335584990680218 53.0
71.371202418 0.0544376820325851 54.0
71.371202418 0.136299982666969 55.0
0 3.97916269302368 56.0
83.549250838 0.0624392256140709 57.0
222.96257224 0.53520393371582 58.0
249.13889427 0.248286575078964 59.0
19.770843622 0.00852761417627335 60.0
0 0.133813917636871 61.0
135.45277362 0.951301038265228 0.0
135.45277362 1.2920013666153 1.0
134.17271103 2.18017554283142 2.0
129.38360578 5.86622142791748 3.0
0 4.90592432022095 4.0
0 5.27095174789429 5.0
-61.772052707 1.19712662696838 6.0
71.986854625 0.663314282894135 7.0
61.589185442 1.19291841983795 8.0
10.397669183 1.16038060188293 9.0
85.218796436 0.921801626682281 10.0
72.384286735 0.0278372056782246 11.0
84.158331328 0.733103692531586 12.0
84.158331328 0.45914152264595 13.0
84.158331328 0.147222384810448 14.0
0 0.245332419872284 15.0
0 0.152590736746788 16.0
0 1.13468503952026 17.0
63.494625944 1.39774060249329 18.0
-63.494625944 1.38793015480042 19.0
-63.494625944 0.496881276369095 20.0
-63.494625944 0.0117331556975842 21.0
86.367743605 2.38687682151794 22.0
143.86318743 1.92975926399231 23.0
111.70899015 0.539411544799805 24.0
23.543232034 0.0780128538608551 25.0
230.23093103 1.61297583580017 26.0
230.23093103 2.12645936012268 27.0
29.57086341 2.73594760894775 28.0
259.80179444 1.45272862911224 29.0
259.80179444 1.91215670108795 30.0
46.852462346 0.995152354240417 31.0
9.3901094561 0.481944710016251 32.0
22.873117661 1.10430467128754 33.0
22.873117661 0.832703590393066 34.0
22.873117661 0.0913119316101074 35.0
9.7760068727 0.0280313901603222 36.0
8.5206061684 0.372374892234802 37.0
0 0.0669036060571671 38.0
0 0.290018320083618 39.0
-12.965213435 0.395396113395691 40.0
-24.77431772 0.906972408294678 41.0
4.7562314401 0.29140517115593 42.0
-32.07517073 0.201575011014938 43.0
87.226480087 0.728987634181976 44.0
87.226480087 0.0875437557697296 45.0
7.3819293544 0.633892953395844 46.0
2.7010181203 0.403622567653656 47.0
135.45277362 0.171452164649963 48.0
129.38360578 0.652988851070404 49.0
0 0.621566414833069 50.0
71.986854625 0.177768036723137 51.0
85.218796436 0.091813325881958 52.0
84.158331328 0.0673092231154442 53.0
84.158331328 0.122435078024864 54.0
84.158331328 0.0679895579814911 55.0
0 6.10984802246094 56.0
86.367743605 0.0497997477650642 57.0
230.23093103 0.710217297077179 58.0
259.80179444 0.674598276615143 59.0
22.873117661 0.0115387439727783 60.0
0 0.0814564824104309 61.0
152.72476885 0.908173978328705 0.0
152.72476885 0.677014648914337 1.0
151.48183113 0.81968879699707 2.0
146.01716038 2.60669565200806 3.0
4.626607659 2.67999792098999 4.0
4.626607659 6.06546115875244 5.0
-70.03280559 1.77851247787476 6.0
80.307895812 0.530851125717163 7.0
44.030622381 0.632921576499939 8.0
36.277273432 0.825955986976624 9.0
85.316294887 1.0663822889328 10.0
70.591150973 0.229122892022133 11.0
89.446446608 0.184389054775238 12.0
89.446446608 0.984617710113525 13.0
89.446446608 0.0742439031600952 14.0
0 0.240956857800484 15.0
0 0.175567954778671 16.0
0 1.81539559364319 17.0
58.11519006 1.15699660778046 18.0
-58.11519006 0.863621652126312 19.0
-58.11519006 0.369847685098648 20.0
-58.11519006 0.203236788511276 21.0
79.142852112 2.5022439956665 22.0
143.72729637 2.68735814094543 23.0
105.04849551 0.412498295307159 24.0
24.239308284 0.0824574381113052 25.0
222.87014848 0.474497079849243 26.0
222.87014848 2.00556969642639 27.0
26.958741443 2.9362416267395 28.0
249.82888992 1.40968751907349 29.0
249.82888992 2.18612408638 30.0
46.161559715 1.29347026348114 31.0
9.2421840312 0.300597995519638 32.0
21.027662052 1.10105419158936 33.0
21.027662052 0.847812294960022 34.0
21.027662052 0.139137789607048 35.0
9.3499135762 0.054196435958147 36.0
8.0987806515 0.388721466064453 37.0
0 0.00297875329852104 38.0
0 0.144259318709373 39.0
-11.557095413 0.653836488723755 40.0
-23.578107153 0.596498727798462 41.0
5.7636372016 0.469309091567993 42.0
-32.228077932 0.577065348625183 43.0
77.115234135 0.539081275463104 44.0
77.115234135 0.106191374361515 45.0
6.9752872668 0.485793203115463 46.0
2.2767519295 0.410486251115799 47.0
152.72476885 0.274233847856522 48.0
146.01716038 0.264191567897797 49.0
4.626607659 0.553966879844666 50.0
80.307895812 0.246653035283089 51.0
85.316294887 0.265105932950974 52.0
89.446446608 0.00507142534479499 53.0
89.446446608 0.158411413431168 54.0
89.446446608 0.267370611429214 55.0
0 1.61575889587402 56.0
79.142852112 0.116471320390701 57.0
222.87014848 0.20649042725563 58.0
249.82888992 0.547534763813019 59.0
21.027662052 -0.00133208557963371 60.0
0 0.0857737809419632 61.0
202.48143369 0.674546301364899 0.0
202.48143369 0.643237948417664 1.0
200.9323118 0.994328320026398 2.0
198.28770152 3.61268734931946 3.0
0 8.19001388549805 4.0
0 3.94801878929138 5.0
-69.358354886 1.20370280742645 6.0
78.38007054 0.822245299816132 7.0
38.281469635 1.08015787601471 8.0
40.098600905 0.843787372112274 9.0
57.443401354 0.860128700733185 10.0
43.456299037 0.207290396094322 11.0
46.096388694 0.876705646514893 12.0
46.096388694 0.388153463602066 13.0
46.096388694 0.125843584537506 14.0
36.875099932 0.30816850066185 15.0
36.875099932 0.188748255372047 16.0
0 1.68348801136017 17.0
53.228431247 1.18566799163818 18.0
-53.228431247 1.51500475406647 19.0
-53.990225685 0.410829097032547 20.0
-54.064096065 -0.0568938553333282 21.0
72.458373247 3.02729868888855 22.0
137.23593083 2.88110327720642 23.0
104.05104372 0.330171555280685 24.0
23.395147513 0.0873218774795532 25.0
209.69430407 1.47823083400726 26.0
209.69430407 2.59751272201538 27.0
24.628251997 2.28266930580139 28.0
234.32255607 1.39724326133728 29.0
234.32255607 1.88269424438477 30.0
46.833020465 1.13306665420532 31.0
9.4962919806 0.510018885135651 32.0
18.394277182 1.72392332553864 33.0
18.394277182 0.925778746604919 34.0
18.394277182 0.0913151875138283 35.0
8.021057789 0.0284746140241623 36.0
6.7934921789 0.372555702924728 37.0
0 0.064272403717041 38.0
0 0.118170239031315 39.0
-11.918743686 0.775801181793213 40.0
-25.60724617 0.305796027183533 41.0
4.7648515773 0.499043226242065 42.0
-32.663458934 0.640704452991486 43.0
86.941617242 0.469677686691284 44.0
86.941617242 0.147451162338257 45.0
6.6617107294 0.488188952207565 46.0
2.8732080847 0.41224279999733 47.0
202.48143369 0.14680415391922 48.0
198.28770152 0.195651397109032 49.0
0 0.255123764276505 50.0
78.38007054 0.305771917104721 51.0
57.443401354 0.351109772920609 52.0
46.096388694 0.101826399564743 53.0
46.096388694 0.192323908209801 54.0
46.096388694 0.12692429125309 55.0
0 2.98149728775024 56.0
72.458373247 0.105483271181583 57.0
209.69430407 0.54912668466568 58.0
234.32255607 0.793932497501373 59.0
18.394277182 -0.00521586276590824 60.0
0 0.083604097366333 61.0
200.0725195 0.907171487808228 0.0
200.0725195 2.3005805015564 1.0
198.69657104 2.35413908958435 2.0
193.23190043 5.61173152923584 3.0
7.8366716725 5.22491550445557 4.0
7.8366716725 6.10086965560913 5.0
-68.312620614 1.45610082149506 6.0
80.39083943 0.893464922904968 7.0
40.982737513 0.882671654224396 8.0
39.408101917 0.777283012866974 9.0
102.0610436 0.884020209312439 10.0
88.442439844 0.131838738918304 11.0
89.431086152 0.18641072511673 12.0
89.431086152 0.258471518754959 13.0
89.431086152 0.10158447176218 14.0
0 0.222567409276962 15.0
0 0.166115701198578 16.0
0 1.48748898506165 17.0
59.73839721 1.1065833568573 18.0
-59.73839721 0.758223652839661 19.0
-59.73839721 0.953824043273926 20.0
-59.73839721 0.00523072108626366 21.0
80.759547559 1.89455878734589 22.0
144.20490456 1.78272140026093 23.0
109.14643662 0.603245198726654 24.0
22.722711543 0.0954771786928177 25.0
224.96445212 1.08876430988312 26.0
224.96445212 1.97495806217194 27.0
28.407858228 2.24955224990845 28.0
253.37231035 1.34413087368011 29.0
253.37231035 1.85658013820648 30.0
47.268489213 0.930475115776062 31.0
9.4663895417 0.460622817277908 32.0
21.02115035 1.1734174489975 33.0
21.02115035 0.771868288516998 34.0
21.02115035 0.142973676323891 35.0
9.5932103191 0.0220260303467512 36.0
8.4216384647 0.350739479064941 37.0
0 0.00897777825593948 38.0
0 0.154291048645973 39.0
-11.874467471 0.384994119405746 40.0
-23.5861651 0.847790479660034 41.0
4.9792286443 0.424289494752884 42.0
-31.226686974 0.513462781906128 43.0
90.62619239 0.876102864742279 44.0
90.62619239 0.0950663685798645 45.0
6.4773915814 0.716585040092468 46.0
1.8057102677 0.409658223390579 47.0
200.0725195 0.120546147227287 48.0
193.23190043 0.31601557135582 49.0
7.8366716725 0.368744134902954 50.0
80.39083943 0.276102960109711 51.0
102.0610436 0.182025134563446 52.0
89.431086152 0.04895880818367 53.0
89.431086152 0.0847225040197372 54.0
89.431086152 0.115522168576717 55.0
0 0.358427226543427 56.0
80.759547559 0.109443068504333 57.0
224.96445212 0.337395548820496 58.0
253.37231035 1.00213801860809 59.0
21.02115035 0.00101112946867943 60.0
0 0.0854707881808281 61.0
151.2869403 1.10993564128876 0.0
151.2869403 1.43053615093231 1.0
150.10595743 1.84367871284485 2.0
145.31685234 3.59432697296143 3.0
0 7.06945133209229 4.0
0 5.6040678024292 5.0
-71.892608461 1.31329071521759 6.0
84.376370682 0.453519314527512 7.0
58.760984297 1.11670243740082 8.0
25.615386385 1.11880779266357 9.0
88.02831565 1.19345080852509 10.0
75.276623806 0.103543907403946 11.0
78.089038295 1.00607347488403 12.0
78.089038295 0.609348893165588 13.0
78.089038295 0.123370118439198 14.0
0 0.254006236791611 15.0
0 0.179834678769112 16.0
0 1.36131024360657 17.0
61.393609588 1.14621806144714 18.0
-61.393609588 0.957790911197662 19.0
-61.393609588 0.99996680021286 20.0
-61.393609588 -0.0756048262119293 21.0
82.72842585 1.37729704380035 22.0
145.0363981 0.865760385990143 23.0
111.53812388 0.733464181423187 24.0
25.116870102 0.0852637365460396 25.0
227.76482395 1.4157407283783 26.0
227.76482395 1.23821043968201 27.0
28.005626734 1.44699025154114 28.0
255.77045068 0.993500828742981 29.0
255.77045068 1.83928346633911 30.0
46.315648745 1.25101816654205 31.0
9.3604937085 0.400637596845627 32.0
21.334816262 1.5238481760025 33.0
21.334816262 1.05352389812469 34.0
21.334816262 0.113321125507355 35.0
9.792995097 0.031517032533884 36.0
8.6009118288 0.345331132411957 37.0
0 0.0540508255362511 38.0
0 0.222695589065552 39.0
-13.070330259 0.502865731716156 40.0
-26.347396165 0.665011644363403 41.0
4.6686771342 0.36829674243927 42.0
-33.639102484 0.511554896831512 43.0
101.31155383 0.65877765417099 44.0
101.31155383 0.136146888136864 45.0
6.4401921191 0.566049993038177 46.0
2.5057586189 0.383344709873199 47.0
151.2869403 0.061452854424715 48.0
145.31685234 0.292692422866821 49.0
0 0.713498711585999 50.0
84.376370682 0.153053224086761 51.0
88.02831565 0.0642553418874741 52.0
78.089038295 0.107439175248146 53.0
78.089038295 0.193458721041679 54.0
78.089038295 0.225446075201035 55.0
0 5.50266408920288 56.0
82.72842585 0.151406973600388 57.0
227.76482395 0.375773131847382 58.0
255.77045068 0.793678462505341 59.0
21.334816262 0.00902784615755081 60.0
0 0.0973255038261414 61.0
165.02399592 1.04641592502594 0.0
165.02399592 1.59526062011719 1.0
163.69908693 2.07742309570312 2.0
158.90998167 6.47482872009277 3.0
0 3.2975218296051 4.0
0 3.71954154968262 5.0
-55.161432409 0.977321565151215 6.0
63.634329469 0.682416915893555 7.0
59.715340415 0.549256205558777 8.0
3.9189890544 0.737496197223663 9.0
81.374753088 1.17882823944092 10.0
69.224410545 0.0786473453044891 11.0
77.014222748 0.168027698993683 12.0
77.014222748 0.279940783977509 13.0
77.014222748 0.0924377590417862 14.0
0 0.132790014147758 15.0
0 0.198014289140701 16.0
0 1.15795707702637 17.0
56.397836684 1.57020580768585 18.0
-56.397836684 0.879941642284393 19.0
-56.397836684 0.60513311624527 20.0
-56.397836684 -0.0740280300378799 21.0
78.771538351 1.76138496398926 22.0
138.92908443 1.30956411361694 23.0
105.52645911 0.407479971647263 24.0
22.940099852 0.133734345436096 25.0
217.70062278 1.22669219970703 26.0
217.70062278 1.44195592403412 27.0
22.219212273 1.31325387954712 28.0
239.91983505 0.801801145076752 29.0
239.91983505 1.59086108207703 30.0
46.375304217 0.894517600536346 31.0
9.3058266358 0.331557303667068 32.0
22.373701667 1.65570712089539 33.0
22.373701667 0.764796257019043 34.0
22.373701667 0.137916535139084 35.0
9.5810923237 0.0314173586666584 36.0
8.3195910338 0.383161664009094 37.0
0 0.0444311276078224 38.0
0 0.0647865980863571 39.0
-13.060078402 0.618421316146851 40.0
-25.731138837 0.559240162372589 41.0
4.7082080007 0.410358369350433 42.0
-32.805049393 0.542676210403442 43.0
86.076069194 0.7117879986763 44.0
86.076069194 0.109878391027451 45.0
6.0805214093 0.51598995923996 46.0
2.3054797633 0.351792067289352 47.0
165.02399592 0.172159165143967 48.0
158.90998167 0.428288578987122 49.0
0 0.138584688305855 50.0
63.634329469 0.138227447867393 51.0
81.374753088 0.0986698418855667 52.0
77.014222748 0.0305598415434361 53.0
77.014222748 0.0868792682886124 54.0
77.014222748 0.173074319958687 55.0
0 6.51055431365967 56.0
78.771538351 0.133827403187752 57.0
217.70062278 0.850595057010651 58.0
239.91983505 1.0343906879425 59.0
22.373701667 -0.0130335595458746 60.0
0 0.0863974019885063 61.0
164.67037943 0.734589099884033 0.0
164.67037943 1.84023022651672 1.0
163.59019255 2.00245356559753 2.0
158.80108726 4.28983306884766 3.0
0 6.25509214401245 4.0
0 3.25421953201294 5.0
-70.835058575 0.140872299671173 6.0
81.436272685 0.375814765691757 7.0
55.336983582 1.16588509082794 8.0
26.099289104 1.20781683921814 9.0
89.034228014 0.913439273834229 10.0
75.800649026 0.240648344159126 11.0
81.370411243 0.613736271858215 12.0
81.370411243 0.302787691354752 13.0
81.370411243 0.0568669438362122 14.0
0 0.187440618872643 15.0
0 0.201147139072418 16.0
0 1.24178600311279 17.0
60.129165517 1.6915168762207 18.0
-60.129165517 1.92412543296814 19.0
-60.129165517 0.328381478786469 20.0
-60.129165517 -0.00830434262752533 21.0
81.57986475 1.37606048583984 22.0
146.23144877 0.953486323356628 23.0
110.85398848 0.561113953590393 24.0
22.651783246 0.102021872997284 25.0
227.81131352 0.804354965686798 26.0
227.81131352 1.62042081356049 27.0
28.89548796 1.45573031902313 28.0
256.70680148 0.463358551263809 29.0
256.70680148 1.85623848438263 30.0
46.646999831 1.25703811645508 31.0
9.42612787 0.569796741008759 32.0
21.450699233 1.61204838752747 33.0
21.450699233 0.38968226313591 34.0
21.450699233 0.131984621286392 35.0
9.7405344968 0.0304691828787327 36.0
8.3876865515 0.399222761392593 37.0
0 0.0897453874349594 38.0
0 0.129912376403809 39.0
-12.047803943 0.845976412296295 40.0
-24.9667495 0.433176398277283 41.0
5.0164583813 0.482201278209686 42.0
-32.296483737 0.490770250558853 43.0
81.476171401 0.604317188262939 44.0
81.476171401 0.113395735621452 45.0
6.3474086085 0.552577614784241 46.0
2.3291739048 0.425777703523636 47.0
164.67037943 0.0952681601047516 48.0
158.80108726 0.527288734912872 49.0
0 0.2711281478405 50.0
81.436272685 0.135951042175293 51.0
89.034228014 0.311447978019714 52.0
81.370411243 0.03945666924119 53.0
81.370411243 0.111697934567928 54.0
81.370411243 0.20298658311367 55.0
0 3.02660250663757 56.0
81.57986475 0.0646452456712723 57.0
227.81131352 0.420932471752167 58.0
256.70680148 0.309450954198837 59.0
21.450699233 0.0109148323535919 60.0
0 0.108130060136318 61.0
160.16443681 0.84353119134903 0.0
160.16443681 1.58300375938416 1.0
158.70992203 1.10070180892944 2.0
153.92083778 3.85799860954285 3.0
0 6.43715667724609 4.0
0 5.31757354736328 5.0
-49.254395312 1.10929906368256 6.0
57.069949162 0.409325003623962 7.0
54.677326329 0.816874504089355 8.0
2.3926228337 1.04652082920074 9.0
72.397747333 1.1435889005661 10.0
59.431411323 0.389029443264008 11.0
73.316448758 0.347495704889297 12.0
73.316448758 1.08223235607147 13.0
73.316448758 0.111364133656025 14.0
0 0.110549420118332 15.0
0 0.198362097144127 16.0
0 1.43618321418762 17.0
49.194792434 1.17842972278595 18.0
-49.194792434 0.592638313770294 19.0
-49.194792434 0.750235199928284 20.0
-49.194792434 -0.0162690728902817 21.0
68.801144711 1.52308666706085 22.0
132.39380641 0.78677624464035 23.0
95.163375313 0.829860746860504 24.0
20.930955428 0.0814507156610489 25.0
201.19495112 0.944974303245544 26.0
201.19495112 1.50569975376129 27.0
31.287753606 1.19061994552612 28.0
232.48270473 0.400154620409012 29.0
232.48270473 1.62807500362396 30.0
45.86768047 1.21338701248169 31.0
9.1624158698 0.677899956703186 32.0
19.606352277 1.80167889595032 33.0
19.606352277 1.09788906574249 34.0
19.606352277 0.101576797664165 35.0
9.8396443249 0.0212647542357445 36.0
8.5142601681 0.31696030497551 37.0
0 0.0069713294506073 38.0
0 0.238512128591537 39.0
-10.666227081 0.314388185739517 40.0
-24.001841092 0.542192339897156 41.0
4.6298451548 0.33439177274704 42.0
-31.468023066 0.398588955402374 43.0
84.226835027 0.52048659324646 44.0
84.226835027 0.0940490663051605 45.0
6.7642534012 0.544448792934418 46.0
2.0269475105 0.427402257919312 47.0
160.16443681 0.108606584370136 48.0
153.92083778 0.195415675640106 49.0
0 0.763352394104004 50.0
57.069949162 0.134447574615479 51.0
72.397747333 0.364453732967377 52.0
73.316448758 0.0724939182400703 53.0
73.316448758 0.289331346750259 54.0
73.316448758 0.174340039491653 55.0
0 5.45999765396118 56.0
68.801144711 0.0832798779010773 57.0
201.19495112 0.304140686988831 58.0
232.48270473 0.176186040043831 59.0
19.606352277 0.00588393956422806 60.0
0 0.0828757211565971 61.0
168.14189489 1.09627068042755 0.0
168.14189489 1.63445854187012 1.0
166.87794958 1.76025319099426 2.0
162.08884433 3.83998680114746 3.0
0 7.65759134292603 4.0
0 3.85322618484497 5.0
-56.121857522 0.691454827785492 6.0
65.490787198 0.962257325649261 7.0
38.95850365 0.819794714450836 8.0
26.532283548 0.697991967201233 9.0
86.960979644 1.10308027267456 10.0
72.661102848 0.108863294124603 11.0
76.17729116 0.757358014583588 12.0
76.17729116 0.294727951288223 13.0
76.17729116 0.123039826750755 14.0
0 0.0524717345833778 15.0
0 0.164375603199005 16.0
0 1.15488910675049 17.0
52.065695581 1.4424079656601 18.0
-52.065695581 1.45672488212585 19.0
-52.065695581 0.601998746395111 20.0
-52.065695581 0.0758143961429596 21.0
72.943236367 2.33703255653381 22.0
139.15751921 1.97554528713226 23.0
106.96212527 0.356270253658295 24.0
21.788974306 0.0825168937444687 25.0
212.10075557 1.51563823223114 26.0
212.10075557 2.51746368408203 27.0
29.523726255 2.6073899269104 28.0
241.62448183 1.41401410102844 29.0
241.62448183 1.84821808338165 30.0
46.620084186 1.19689059257507 31.0
9.3723094281 0.358225613832474 32.0
20.877540786 1.10629081726074 33.0
20.877540786 0.888921439647675 34.0
20.877540786 0.125019401311874 35.0
9.2535648757 0.0330762416124344 36.0
8.041908898 0.33985099196434 37.0
0 0.0312405861914158 38.0
0 0.187885999679565 39.0
-12.2911258 0.651669263839722 40.0
-26.210604767 0.559778094291687 41.0
5.0352866811 0.514102280139923 42.0
-33.527440818 0.793464004993439 43.0
84.942004881 0.613679826259613 44.0
84.942004881 0.0933410599827766 45.0
7.2207271525 0.509115099906921 46.0
2.4266829079 0.390203475952148 47.0
168.14189489 0.134993657469749 48.0
162.08884433 0.165037155151367 49.0
0 0.504098951816559 50.0
65.490787198 0.137772381305695 51.0
86.960979644 0.119703128933907 52.0
76.17729116 0.0801394134759903 53.0
76.17729116 0.103000581264496 54.0
76.17729116 0.0253582578152418 55.0
0 6.89837169647217 56.0
72.943236367 -0.0136437015607953 57.0
212.10075557 0.654833137989044 58.0
241.62448183 0.813043773174286 59.0
20.877540786 -0.00361902825534344 60.0
0 0.0972982943058014 61.0
155.48908802 0.732495307922363 0.0
155.48908802 1.58645558357239 1.0
154.01321658 1.30631482601166 2.0
149.22411128 5.90685749053955 3.0
0 4.38992357254028 4.0
0 3.23862767219543 5.0
-48.607819902 2.01014113426208 6.0
57.037911183 0.884769558906555 7.0
63.61444153 0.83014041185379 8.0
-6.5765303473 1.03477954864502 9.0
93.974199934 0.822367608547211 10.0
79.982791749 0.108205124735832 11.0
87.77703242 0.181308627128601 12.0
87.77703242 0.0864982306957245 13.0
87.77703242 0.0868214517831802 14.0
0 0.17514805495739 15.0
0 0.191253215074539 16.0
0 1.39585721492767 17.0
55.9186848 1.48517155647278 18.0
-55.9186848 1.74579250812531 19.0
-55.9186848 0.764750897884369 20.0
-55.9186848 -0.044071439653635 21.0
78.768711007 1.23991787433624 22.0
142.62631242 1.24320983886719 23.0
113.56410463 0.456943273544312 24.0
23.170714388 0.253278493881226 25.0
221.39502342 0.662962794303894 26.0
221.39502342 1.61027467250824 27.0
22.915332966 1.89603590965271 28.0
244.31035639 1.01705980300903 29.0
244.31035639 1.77938735485077 30.0
47.538471977 1.22896695137024 31.0
9.4754439277 0.421923160552979 32.0
22.850026207 1.21449279785156 33.0
22.850026207 1.68141746520996 34.0
22.850026207 0.10175634175539 35.0
10.113096094 0.0221519600600004 36.0
8.9394056208 0.378596007823944 37.0
0 0.0161087363958359 38.0
0 0.311403125524521 39.0
-10.672137253 0.551502227783203 40.0
-23.568099575 0.570545852184296 41.0
4.7892394063 0.425386756658554 42.0
-31.102074721 0.423664450645447 43.0
85.586776591 0.588395535945892 44.0
85.586776591 0.0985732600092888 45.0
6.9537816751 0.612452924251556 46.0
2.6303614158 0.409365653991699 47.0
155.48908802 0.0893229246139526 48.0
149.22411128 0.40213879942894 49.0
0 0.111090265214443 50.0
57.037911183 0.0436237417161465 51.0
93.974199934 0.126441195607185 52.0
87.77703242 0.072412371635437 53.0
87.77703242 0.0703662782907486 54.0
87.77703242 0.0898350030183792 55.0
0 0.330860793590546 56.0
78.768711007 0.087665818631649 57.0
221.39502342 0.232313171029091 58.0
244.31035639 0.433556258678436 59.0
22.850026207 -0.0133107993751764 60.0
0 0.078608937561512 61.0
162.00578586 0.861819565296173 0.0
162.00578586 2.11094903945923 1.0
160.95564274 2.71563005447388 2.0
155.49097214 5.98623943328857 3.0
14.56541059 5.23046398162842 4.0
14.56541059 3.7790355682373 5.0
-58.49060831 0.304795354604721 6.0
69.277575621 0.435543984174728 7.0
43.930886211 0.64012348651886 8.0
25.34668941 0.823476076126099 9.0
94.920122486 0.999449014663696 10.0
82.098559832 0.293137311935425 11.0
78.761270147 0.666913330554962 12.0
78.761270147 0.736612319946289 13.0
78.761270147 0.0838225781917572 14.0
0 0.11651599407196 15.0
0 0.209494456648827 16.0
0 1.12240028381348 17.0
57.182920797 1.93684256076813 18.0
-57.182920797 2.38995361328125 19.0
-57.182920797 0.22463247179985 20.0
-57.182920797 0.0207796208560467 21.0
80.007833193 1.93353712558746 22.0
148.99653627 1.27176284790039 23.0
113.99605937 0.567377984523773 24.0
23.209474331 0.171528682112694 25.0
229.00436946 1.54406785964966 26.0
229.00436946 1.52562713623047 27.0
27.096334824 1.3811023235321 28.0
256.10070428 0.615936100482941 29.0
256.10070428 1.77141511440277 30.0
47.228601008 1.12266254425049 31.0
9.4758091468 0.490876466035843 32.0
22.824912396 1.49523341655731 33.0
22.824912396 0.855060458183289 34.0
22.824912396 0.143358811736107 35.0
9.7122313788 0.0401880331337452 36.0
8.3232864458 0.412852942943573 37.0
0 0.0610355213284492 38.0
0 0.303297698497772 39.0
-11.365477068 0.516706347465515 40.0
-24.600411206 0.84550952911377 41.0
5.1716867 0.456364214420319 42.0
-32.447130533 0.570482850074768 43.0
85.452816557 0.717186987400055 44.0
85.452816557 0.0999433398246765 45.0
7.7951504229 0.654180943965912 46.0
2.7804177762 0.439446181058884 47.0
162.00578586 0.146972224116325 48.0
155.49097214 0.81275737285614 49.0
14.56541059 0.448072910308838 50.0
69.277575621 0.19403164088726 51.0
94.920122486 0.374756276607513 52.0
78.761270147 0.115947216749191 53.0
78.761270147 0.178782626986504 54.0
78.761270147 0.0995406433939934 55.0
0 4.94096660614014 56.0
80.007833193 -0.069061167538166 57.0
229.00436946 0.812861561775208 58.0
256.10070428 0.287636041641235 59.0
22.824912396 -0.0108695980161428 60.0
0 0.0883212313055992 61.0
163.27592062 0.731040596961975 0.0
163.27592062 1.06154882907867 1.0
162.1197661 1.4209371805191 2.0
157.33069614 2.60840916633606 3.0
0 4.30550384521484 4.0
0 5.67945528030396 5.0
-59.840758501 0.4918053150177 6.0
69.02707983 1.38214826583862 7.0
59.455236372 0.933353900909424 8.0
9.571843458 0.64097672700882 9.0
92.918059348 0.762240171432495 10.0
80.244023183 0.11313770711422 11.0
86.734352592 0.975073575973511 12.0
86.734352592 0.744554460048676 13.0
86.734352592 0.0772246122360229 14.0
0 0.184197247028351 15.0
0 0.152592346072197 16.0
0 1.06419456005096 17.0
58.94768768 1.35264110565186 18.0
-58.94768768 1.73458194732666 19.0
-58.94768768 0.28910106420517 20.0
-58.94768768 -0.00630896911025047 21.0
80.090631456 1.08039236068726 22.0
148.02905701 0.64686119556427 23.0
114.95208316 0.507041275501251 24.0
23.89976275 0.211878314614296 25.0
228.11968846 0.966200888156891 26.0
228.11968846 1.57384729385376 27.0
23.390250959 1.88992536067963 28.0
251.50993942 0.761149525642395 29.0
251.50993942 1.60867118835449 30.0
46.725644086 1.09852039813995 31.0
9.2606639341 0.556684672832489 32.0
21.142943776 2.13098645210266 33.0
21.142943776 0.733471810817719 34.0
21.142943776 0.115799874067307 35.0
8.9111231474 0.0330888703465462 36.0
7.551726817 0.40471339225769 37.0
0 0.0101525969803333 38.0
0 0.18135903775692 39.0
-12.15064454 0.587448239326477 40.0
-25.377000965 0.589455485343933 41.0
4.8526867661 0.419095993041992 42.0
-32.657617486 0.352575749158859 43.0
93.634159911 0.810527741909027 44.0
93.634159911 0.0684969946742058 45.0
6.4798193089 0.579254925251007 46.0
2.5113066022 0.330770462751389 47.0
163.27592062 0.114976890385151 48.0
157.33069614 0.001619016751647 49.0
0 0.408389925956726 50.0
69.02707983 0.330453336238861 51.0
92.918059348 0.147336930036545 52.0
86.734352592 0.0700056850910187 53.0
86.734352592 0.331024765968323 54.0
86.734352592 0.0772029608488083 55.0
0 2.14196157455444 56.0
80.090631456 0.167001202702522 57.0
228.11968846 0.204252362251282 58.0
251.50993942 0.308753222227097 59.0
21.142943776 -0.00390255451202393 60.0
0 0.0947553813457489 61.0
180.37186108 0.814742505550385 0.0
180.37186108 2.05994629859924 1.0
178.91540921 2.19115686416626 2.0
173.45073862 4.64368391036987 3.0
3.0381857225 4.36329746246338 4.0
3.0381857225 3.82247018814087 5.0
-69.367139451 0.619854927062988 6.0
82.380314515 1.0788848400116 7.0
40.829781585 0.481488168239594 8.0
41.55053293 0.319410562515259 9.0
88.929989571 1.01417076587677 10.0
76.119451117 0.674118340015411 11.0
75.212476623 0.482750684022903 12.0
75.212476623 0.356263697147369 13.0
75.212476623 0.107459977269173 14.0
0 0.08231021463871 15.0
0 0.178290128707886 16.0
0 1.25557494163513 17.0
55.291415835 1.33892381191254 18.0
-55.291415835 1.45256173610687 19.0
-55.291415835 0.643994092941284 20.0
-55.291415835 -0.0720750465989113 21.0
75.514241736 1.75732600688934 22.0
134.06951955 0.944057583808899 23.0
101.50540772 0.48308527469635 24.0
21.59506469 0.11401629447937 25.0
209.58376129 1.0677604675293 26.0
209.58376129 1.57205176353455 27.0
27.184008895 1.33872973918915 28.0
236.76777018 0.587583720684052 29.0
236.76777018 1.47794282436371 30.0
46.541653901 1.1757470369339 31.0
9.3211119983 0.325357794761658 32.0
20.222825901 1.57099068164825 33.0
20.222825901 0.962727844715118 34.0
20.222825901 0.0913839116692543 35.0
8.7068315919 0.0249243341386318 36.0
7.4151013314 0.363541066646576 37.0
0 0.0595705509185791 38.0
0 0.164035424590111 39.0
-11.432352293 0.67527824640274 40.0
-23.106078758 0.544025242328644 41.0
5.3646280481 0.35433304309845 42.0
-31.123098143 0.308442413806915 43.0
89.25964392 0.784158527851105 44.0
89.25964392 0.0683693215250969 45.0
6.7746458601 0.474872678518295 46.0
1.9533223471 0.356542944908142 47.0
180.37186108 0.208447679877281 48.0
173.45073862 0.408689618110657 49.0
3.0381857225 0.0652124583721161 50.0
82.380314515 0.0563125684857368 51.0
88.929989571 0.434241652488708 52.0
75.212476623 0.15223528444767 53.0
75.212476623 0.0349165834486485 54.0
75.212476623 0.106182716786861 55.0
0 5.03940391540527 56.0
75.514241736 0.151510581374168 57.0
209.58376129 0.305941551923752 58.0
236.76777018 0.623893916606903 59.0
20.222825901 -0.00583059154450893 60.0
0 0.056724701076746 61.0
165.41474659 1.01021313667297 0.0
165.41474659 0.785517811775208 1.0
164.12925609 1.40062594413757 2.0
158.66462068 3.72918653488159 3.0
0.99667084507 8.04678058624268 4.0
0.99667084507 3.10430955886841 5.0
-64.585627192 0.410812377929688 6.0
74.673574776 0.544117033481598 7.0
38.313180498 1.02722823619843 8.0
36.360394278 0.83983302116394 9.0
100.39435388 1.19098925590515 10.0
87.871115345 1.0942131280899 11.0
81.909599915 1.60399508476257 12.0
81.909599915 0.154274389147758 13.0
81.909599915 0.159447893500328 14.0
0 0.341860711574554 15.0
0 0.144023165106773 16.0
0 2.51940202713013 17.0
62.222564295 1.02866888046265 18.0
-62.222564295 1.12439560890198 19.0
-62.222564295 0.242442458868027 20.0
-62.222564295 -0.0507434606552124 21.0
81.690239544 1.07720458507538 22.0
144.69738163 0.479576826095581 23.0
110.4636252 1.22994983196259 24.0
15.005320126 0.059427548199892 25.0
226.38762118 0.65241926908493 26.0
226.38762118 1.44050550460815 27.0
27.139349435 0.718838512897491 28.0
253.52697061 0.6596440076828 29.0
253.52697061 1.2339391708374 30.0
45.932393674 1.07484209537506 31.0
9.169905447 0.352830320596695 32.0
19.467675248 1.8035374879837 33.0
19.467675248 0.764264702796936 34.0
19.467675248 0.0856290608644485 35.0
9.5781543209 0.02374910376966 36.0
8.1501443517 0.363442927598953 37.0
0 0.0557601153850555 38.0
0 0.0662604868412018 39.0
-12.561577978 0.579134106636047 40.0
-24.056942825 0.593361258506775 41.0
4.8572446431 0.436507999897003 42.0
-31.118924573 0.565396130084991 43.0
99.17331213 0.963752508163452 44.0
99.17331213 0.130495682358742 45.0
6.6019083167 0.708602666854858 46.0
2.8969652981 0.288386523723602 47.0
165.41474659 -0.00446796324104071 48.0
158.66462068 0.1844841837883 49.0
0.99667084507 0.22368024289608 50.0
74.673574776 0.086434930562973 51.0
100.39435388 0.536487102508545 52.0
81.909599915 0.300834149122238 53.0
81.909599915 0.252523630857468 54.0
81.909599915 0.137383833527565 55.0
0 2.11430191993713 56.0
81.690239544 0.169840320944786 57.0
226.38762118 0.375499546527863 58.0
253.52697061 0.131611898541451 59.0
19.467675248 -0.00986090116202831 60.0
0 0.11295609921217 61.0
154.09817101 1.04961633682251 0.0
154.09817101 0.847492456436157 1.0
152.97693298 1.73550260066986 2.0
148.18786303 3.49815273284912 3.0
0 8.19198036193848 4.0
0 6.45203495025635 5.0
-51.764756816 3.38078665733337 6.0
60.387335655 0.870636105537415 7.0
57.488285546 0.802621245384216 8.0
2.8990501089 0.937479972839355 9.0
86.131388926 1.01294159889221 10.0
73.454827345 0.134545490145683 11.0
85.315135162 0.640767633914948 12.0
85.315135162 0.289135545492172 13.0
85.315135162 0.13895283639431 14.0
0 0.193141296505928 15.0
0 0.0955294072628021 16.0
0 1.56557762622833 17.0
55.113245538 1.36876559257507 18.0
-55.113245538 1.08879542350769 19.0
-55.113245538 0.566350638866425 20.0
-55.113245538 0.0149463936686516 21.0
76.694204798 1.22924470901489 22.0
155.10640599 0.709787130355835 23.0
117.71593952 0.765903294086456 24.0
22.752942755 0.115065306425095 25.0
231.80061079 0.645267248153687 26.0
231.80061079 1.95488560199738 27.0
30.358891035 2.01879930496216 28.0
262.15950183 0.731594026088715 29.0
262.15950183 1.82199048995972 30.0
46.495244752 1.08317995071411 31.0
9.2353507558 0.478549361228943 32.0
21.58095926 1.44656026363373 33.0
21.58095926 1.23430156707764 34.0
21.58095926 0.148258849978447 35.0
9.4915231995 0.0359575040638447 36.0
8.1220085061 0.321940332651138 37.0
0 0.0406947322189808 38.0
0 0.146472990512848 39.0
-10.540657304 0.598457038402557 40.0
-24.525413477 0.796373248100281 41.0
5.1430917818 0.335792034864426 42.0
-32.584135527 0.4579216837883 43.0
88.565474355 0.529765367507935 44.0
88.565474355 0.090078666806221 45.0
7.7988923978 0.616661727428436 46.0
2.6327853979 0.414009213447571 47.0
154.09817101 0.180225819349289 48.0
148.18786303 0.214868992567062 49.0
0 0.507661104202271 50.0
60.387335655 0.176480025053024 51.0
86.131388926 0.017649345099926 52.0
85.315135162 0.0235361531376839 53.0
85.315135162 0.0188472960144281 54.0
85.315135162 0.0459821298718452 55.0
0 0.780924141407013 56.0
76.694204798 0.0222366414964199 57.0
231.80061079 0.346594125032425 58.0
262.15950183 0.577973067760468 59.0
21.58095926 -0.0181559566408396 60.0
0 0.0789956375956535 61.0
172.23609189 0.931859254837036 0.0
172.23609189 1.52081429958344 1.0
170.76513715 1.7180700302124 2.0
165.97603185 4.38315296173096 3.0
0 7.53723669052124 4.0
0 3.97055506706238 5.0
-53.795121459 3.65819692611694 6.0
64.87147767 0.98918080329895 7.0
38.285842378 0.758430600166321 8.0
26.585635292 0.552263796329498 9.0
88.121139277 1.08690512180328 10.0
74.666835695 0.341531664133072 11.0
78.640389642 0.273554623126984 12.0
78.640389642 0.247426867485046 13.0
78.640389642 0.0726872608065605 14.0
0 0.176803022623062 15.0
0 0.136400550603867 16.0
0 1.19464588165283 17.0
51.810227686 1.88321828842163 18.0
-51.810227686 2.12211155891418 19.0
-51.810227686 0.23846323788166 20.0
-51.810227686 0.0302748531103134 21.0
73.920304173 1.4293384552002 22.0
153.08328813 1.38722765445709 23.0
115.71151889 0.315884858369827 24.0
24.528651099 0.243339091539383 25.0
227.0035923 0.951256573200226 26.0
227.0035923 1.72951424121857 27.0
23.883279739 1.60377073287964 28.0
250.88687204 0.551256477832794 29.0
250.88687204 1.65994572639465 30.0
47.053081193 0.911920666694641 31.0
9.1886829911 0.441324889659882 32.0
22.110076487 1.54736161231995 33.0
22.110076487 0.751318573951721 34.0
22.110076487 0.162170603871346 35.0
8.8741986673 0.0265697576105595 36.0
7.7169196994 0.304469048976898 37.0
0 0.0459738299250603 38.0
0 0.158303707838058 39.0
-11.85002246 0.706856191158295 40.0
-24.21576565 0.499348998069763 41.0
4.562308162 0.406705647706985 42.0
-31.36020885 0.397162944078445 43.0
83.010726185 0.603443622589111 44.0
83.010726185 0.102554768323898 45.0
7.5398048194 0.681367576122284 46.0
2.7543324142 0.407385259866714 47.0
172.23609189 0.100286975502968 48.0
165.97603185 0.130524501204491 49.0
0 0.732662200927734 50.0
64.87147767 0.129206091165543 51.0
88.121139277 0.375996381044388 52.0
78.640389642 0.030343571677804 53.0
78.640389642 0.0612048879265785 54.0
78.640389642 0.0867344588041306 55.0
0 3.31152081489563 56.0
73.920304173 0.0696112662553787 57.0
227.0035923 0.427469044923782 58.0
250.88687204 0.960962176322937 59.0
22.110076487 -0.021593801677227 60.0
0 0.107424348592758 61.0
159.37250151 0.839582026004791 0.0
159.37250151 1.46615135669708 1.0
157.81493083 1.87568044662476 2.0
153.02582551 3.07950448989868 3.0
0 7.95094156265259 4.0
0 4.75705146789551 5.0
-72.767519397 0.400185346603394 6.0
83.566156596 0.844440996646881 7.0
64.804063669 0.717897176742554 8.0
18.762092927 0.981837928295135 9.0
88.084808736 0.792896389961243 10.0
74.442867551 0.006910081487149 11.0
93.645430048 0.201533257961273 12.0
93.645430048 0.480185657739639 13.0
93.645430048 0.157133385539055 14.0
0 0.548684895038605 15.0
0 0.165081441402435 16.0
0 1.48570156097412 17.0
49.056546704 1.80088043212891 18.0
-49.056546704 2.31976318359375 19.0
-49.056546704 0.389515310525894 20.0
-49.056546704 -0.0489615947008133 21.0
73.286658992 2.28900361061096 22.0
150.2715467 1.79417037963867 23.0
112.75654789 0.301293194293976 24.0
21.892675949 0.258926898241043 25.0
223.55820569 1.16780352592468 26.0
223.55820569 1.88268852233887 27.0
29.916648065 1.81641185283661 28.0
253.47485376 1.29988670349121 29.0
253.47485376 1.87002503871918 30.0
46.405570889 1.16424596309662 31.0
9.2702531572 0.308952271938324 32.0
24.230112288 1.86364543437958 33.0
24.230112288 0.877067506313324 34.0
24.230112288 0.132745176553726 35.0
10.077627168 0.0273444149643183 36.0
8.8127446584 0.393694370985031 37.0
0 0.0469465889036655 38.0
0 0.291606485843658 39.0
-11.870124225 0.305571526288986 40.0
-24.510117145 0.819729804992676 41.0
4.7768618882 0.341016680002213 42.0
-32.008520265 0.622151851654053 43.0
82.60879149 0.567400217056274 44.0
82.60879149 0.0823933035135269 45.0
7.7347164339 0.624293863773346 46.0
2.9935629954 0.442262530326843 47.0
159.37250151 0.123159050941467 48.0
153.02582551 0.19603767991066 49.0
0 0.29819855093956 50.0
83.566156596 0.0740742459893227 51.0
88.084808736 0.170661330223083 52.0
93.645430048 0.175213038921356 53.0
93.645430048 0.0518803596496582 54.0
93.645430048 0.265874028205872 55.0
0 0.3592489361763 56.0
73.286658992 0.234894618391991 57.0
223.55820569 0.745056331157684 58.0
253.47485376 0.659489274024963 59.0
24.230112288 -0.00997922755777836 60.0
0 0.120414182543755 61.0
152.34076267 0.884125113487244 0.0
152.34076267 0.632935583591461 1.0
151.23439982 1.17843055725098 2.0
146.44529449 5.12044334411621 3.0
0 5.43359327316284 4.0
0 5.43515014648438 5.0
-62.572640344 0.561229646205902 6.0
76.249790475 0.866939723491669 7.0
40.68681384 1.44802367687225 8.0
35.562976634 1.47711563110352 9.0
94.349904854 0.964570760726929 10.0
80.24898041 0.325742602348328 11.0
82.624089539 1.00925588607788 12.0
82.624089539 0.877941489219666 13.0
82.624089539 0.0491267666220665 14.0
0 0.0389410108327866 15.0
0 0.17788103222847 16.0
0 1.02161300182343 17.0
56.617030857 0.853666543960571 18.0
-56.617030857 0.839703261852264 19.0
-56.617030857 0.341039299964905 20.0
-56.617030857 0.0496357195079327 21.0
76.363279812 2.94377017021179 22.0
139.51006887 3.77863478660583 23.0
106.66280004 0.204270198941231 24.0
21.866426296 0.235840857028961 25.0
215.87334868 1.79051232337952 26.0
215.87334868 1.65672850608826 27.0
30.024327455 1.07978129386902 28.0
245.89767613 0.67221999168396 29.0
245.89767613 1.85007333755493 30.0
46.372354938 1.07423722743988 31.0
9.2618964304 0.509677708148956 32.0
19.746248956 1.35955059528351 33.0
19.746248956 1.42100882530212 34.0
19.746248956 0.126325711607933 35.0
8.0425091115 0.0266281832009554 36.0
6.7590395936 0.383153080940247 37.0
0 -0.023496275767684 38.0
0 0.0716590061783791 39.0
-10.831037071 0.58838164806366 40.0
-23.706509057 0.4745112657547 41.0
5.509883023 0.665842890739441 42.0
-31.509556538 0.628832340240479 43.0
79.686786401 0.717623889446259 44.0
79.686786401 0.0784360319375992 45.0
7.2613285745 0.693543791770935 46.0
2.4770816691 0.424603670835495 47.0
152.34076267 0.119134724140167 48.0
146.44529449 0.187241494655609 49.0
0 1.03724348545074 50.0
76.249790475 0.520875930786133 51.0
94.349904854 0.377795308828354 52.0
82.624089539 0.151121377944946 53.0
82.624089539 0.146834164857864 54.0
82.624089539 0.314207762479782 55.0
0 3.32430219650269 56.0
76.363279812 0.000933400355279446 57.0
215.87334868 0.438403010368347 58.0
245.89767613 0.758168756961823 59.0
19.746248956 -0.00117987114936113 60.0
0 0.101736932992935 61.0
184.93990915 1.16038489341736 0.0
184.93990915 0.808645367622375 1.0
183.56369619 1.28490233421326 2.0
178.7745909 4.95638179779053 3.0
0 7.21027135848999 4.0
0 5.55129194259644 5.0
-58.686880093 2.07801675796509 6.0
67.462761311 0.626404166221619 7.0
46.055519853 0.42389965057373 8.0
21.407241458 0.568115711212158 9.0
82.052473133 1.20891487598419 10.0
67.320640814 0.088661253452301 11.0
77.414158023 0.291195571422577 12.0
77.414158023 0.588209509849548 13.0
77.414158023 0.131397604942322 14.0
0 0.0551998987793922 15.0
0 0.13639834523201 16.0
0 1.52322220802307 17.0
57.721255976 1.04113173484802 18.0
-57.721255976 1.27028369903564 19.0
-57.721255976 0.570946395397186 20.0
-57.721255976 0.0757378116250038 21.0
80.028725631 1.65381276607513 22.0
144.9730388 0.700936257839203 23.0
109.31422676 0.824600696563721 24.0
21.293175646 0.0849525406956673 25.0
225.00176443 1.52493357658386 26.0
225.00176443 2.35197567939758 27.0
26.255409731 2.07707166671753 28.0
251.25717416 0.811187267303467 29.0
251.25717416 1.59250843524933 30.0
46.673539894 1.04631662368774 31.0
9.2027064841 0.485946834087372 32.0
22.307469655 0.924720883369446 33.0
22.307469655 0.965495824813843 34.0
22.307469655 0.111162468791008 35.0
9.7711292883 0.025017648935318 36.0
8.5369413396 0.36414623260498 37.0
0 -0.0054969061166048 38.0
0 0.0424979254603386 39.0
-11.339610624 0.761074900627136 40.0
-23.268674852 0.260689169168472 41.0
4.8825125728 0.522523045539856 42.0
-30.906848857 0.559505105018616 43.0
88.679291109 0.796184539794922 44.0
88.679291109 0.149556040763855 45.0
6.7001132113 0.565980553627014 46.0
2.6710934398 0.353664666414261 47.0
184.93990915 0.131446599960327 48.0
178.7745909 0.0347302705049515 49.0
0 0.713307917118073 50.0
67.462761311 0.0723595470190048 51.0
82.052473133 0.324119389057159 52.0
77.414158023 0.0194108206778765 53.0
77.414158023 0.0308972112834454 54.0
77.414158023 0.155917510390282 55.0
0 3.95994472503662 56.0
80.028725631 -0.0622060708701611 57.0
225.00176443 0.806397795677185 58.0
251.25717416 0.651411890983582 59.0
22.307469655 -4.92651015520096e-05 60.0
0 0.118941351771355 61.0
153.66752344 1.02521300315857 0.0
153.66752344 1.42105007171631 1.0
152.32792031 1.47587192058563 2.0
147.53881495 5.12530279159546 3.0
0 6.04861545562744 4.0
0 2.96488571166992 5.0
-61.841380393 0.20459109544754 6.0
71.195474434 0.997673273086548 7.0
46.215862336 0.477646052837372 8.0
24.979612098 0.408347487449646 9.0
80.361898015 1.01364314556122 10.0
67.726847488 0.00325573072768748 11.0
81.121076788 0.515411078929901 12.0
81.121076788 0.110372498631477 13.0
81.121076788 0.0981716439127922 14.0
0 0.176332011818886 15.0
0 0.184258729219437 16.0
0 1.21280610561371 17.0
57.041060509 2.05554723739624 18.0
-57.041060509 1.71976339817047 19.0
-57.041060509 0.888795793056488 20.0
-57.041060509 -0.0510952770709991 21.0
76.351758832 0.985420644283295 22.0
141.76983716 0.890960693359375 23.0
112.12654544 0.61380809545517 24.0
21.034495949 0.119496829807758 25.0
218.12159599 0.932582437992096 26.0
218.12159599 1.92690634727478 27.0
25.566856847 1.03758490085602 28.0
243.68845284 0.793129503726959 29.0
243.68845284 1.83313071727753 30.0
47.009535408 1.19396162033081 31.0
9.1761516087 0.423483490943909 32.0
19.310698323 1.69085884094238 33.0
19.310698323 0.647455990314484 34.0
19.310698323 0.122963219881058 35.0
10.736034731 0.0191686432808638 36.0
9.6236643631 0.381206095218658 37.0
0 0.0297099985182285 38.0
0 0.0563868060708046 39.0
-11.120857566 0.682507812976837 40.0
-23.166809592 0.44564414024353 41.0
5.7026553959 0.571365058422089 42.0
-31.702260937 0.783918261528015 43.0
75.754266768 0.779458403587341 44.0
75.754266768 0.0871675238013268 45.0
6.1696236028 0.616889476776123 46.0
2.3350082485 0.344520807266235 47.0
153.66752344 0.176128044724464 48.0
147.53881495 0.356967389583588 49.0
0 0.37046006321907 50.0
71.195474434 0.0123494565486908 51.0
80.361898015 0.177315399050713 52.0
81.121076788 0.0072775948792696 53.0
81.121076788 0.107337273657322 54.0
81.121076788 0.07927206158638 55.0
0 3.22001457214355 56.0
76.351758832 0.0819305926561356 57.0
218.12159599 0.832821428775787 58.0
243.68845284 0.465370029211044 59.0
19.310698323 -0.00214922428131104 60.0
0 0.233413025736809 61.0
153.55246 0.942111492156982 0.0
153.55246 0.971037685871124 1.0
152.43497198 1.26204085350037 2.0
147.64586669 5.02000856399536 3.0
0 7.40194177627563 4.0
0 3.82964324951172 5.0
-61.809650267 0.34829780459404 6.0
71.401842699 1.49644708633423 7.0
62.07967229 0.610497713088989 8.0
9.3221704095 0.450038701295853 9.0
85.082298654 1.1420921087265 10.0
71.796755822 0.294448167085648 11.0
81.352453141 0.762073040008545 12.0
81.352453141 0.212558180093765 13.0
81.352453141 0.0764666423201561 14.0
0 0.0571213439106941 15.0
0 0.184468954801559 16.0
0 0.868714690208435 17.0
56.12143567 1.36137938499451 18.0
-56.12143567 1.47016072273254 19.0
-56.12143567 0.604485511779785 20.0
-56.12143567 0.118359506130219 21.0
78.213710262 1.77765607833862 22.0
142.27790703 1.42907047271729 23.0
106.03417061 0.660806596279144 24.0
21.428772436 0.151061356067657 25.0
220.49161729 1.22501516342163 26.0
220.49161729 1.37760448455811 27.0
25.986056728 1.83032655715942 28.0
246.47767402 0.633512675762177 29.0
246.47767402 1.63269686698914 30.0
46.711705574 1.14457488059998 31.0
9.4495755455 0.428453058004379 32.0
22.092274592 1.02360117435455 33.0
22.092274592 0.855964422225952 34.0
22.092274592 0.162870034575462 35.0
10.26260548 0.0393317192792892 36.0
8.9177195157 0.41469070315361 37.0
0 0.0233143270015717 38.0
0 0.141657456755638 39.0
-11.826183247 0.678178489208221 40.0
-25.549059336 0.509552597999573 41.0
4.7486399943 0.538668930530548 42.0
-32.663849482 0.67864453792572 43.0
81.963899252 0.647874295711517 44.0
81.963899252 0.0903716236352921 45.0
7.3244466443 0.620626211166382 46.0
2.7962086805 0.455352574586868 47.0
153.55246 0.0938217639923096 48.0
147.64586669 0.304218530654907 49.0
0 0.281253695487976 50.0
71.401842699 0.0506208315491676 51.0
85.082298654 0.189968973398209 52.0
81.352453141 0.100492529571056 53.0
81.352453141 0.0100978594273329 54.0
81.352453141 0.0619659796357155 55.0
0 4.6430459022522 56.0
78.213710262 0.0233524944633245 57.0
220.49161729 0.349728912115097 58.0
246.47767402 0.359833151102066 59.0
22.092274592 0.00277400761842728 60.0
0 0.0919436365365982 61.0
139.25734397 0.973803877830505 0.0
139.25734397 1.90723359584808 1.0
138.1163718 1.71307122707367 2.0
133.32726669 3.25661015510559 3.0
0 5.14664268493652 4.0
0 3.78711748123169 5.0
-58.242380011 1.38262736797333 6.0
70.444158984 0.762445867061615 7.0
53.718689016 0.639734923839569 8.0
16.725469968 0.740916788578033 9.0
82.413287542 0.962233483791351 10.0
70.447910266 0.0206770710647106 11.0
77.321303871 0.398714065551758 12.0
77.321303871 0.631666004657745 13.0
77.321303871 0.100882925093174 14.0
0 0.154058009386063 15.0
0 0.160066187381744 16.0
0 1.24766087532043 17.0
57.484026902 1.34955680370331 18.0
-57.484026902 1.49368894100189 19.0
-57.484026902 0.300664007663727 20.0
-57.484026902 0.0177452713251114 21.0
80.942211784 2.24347591400146 22.0
136.86537282 1.64491534233093 23.0
99.447288986 0.391834378242493 24.0
20.932184331 0.0921628102660179 25.0
217.80758461 1.20557761192322 26.0
217.80758461 1.53680193424225 27.0
24.96433499 1.70620667934418 28.0
242.7719196 0.729580998420715 29.0
242.7719196 1.77636110782623 30.0
47.278862686 1.063352227211 31.0
9.4630569306 0.594246327877045 32.0
23.458184881 1.44453930854797 33.0
23.458184881 0.91712349653244 34.0
23.458184881 0.10981759428978 35.0
10.5480721 0.0257299244403839 36.0
9.2431247327 0.356766521930695 37.0
0 0.0333508886396885 38.0
0 0.13959139585495 39.0
-12.252783403 0.456599712371826 40.0
-25.853770406 0.647544264793396 41.0
5.5040840968 0.498504161834717 42.0
-34.114184895 0.673084020614624 43.0
94.231123562 0.70100075006485 44.0
94.231123562 0.069887638092041 45.0
6.8186925829 0.458204507827759 46.0
2.5786699068 0.337282001972198 47.0
139.25734397 0.0700577720999718 48.0
133.32726669 0.190709695219994 49.0
0 0.545513272285461 50.0
70.444158984 0.164806917309761 51.0
82.413287542 0.117699645459652 52.0
77.321303871 0.0129645308479667 53.0
77.321303871 0.0835408642888069 54.0
77.321303871 0.201088130474091 55.0
0 4.86205816268921 56.0
80.942211784 0.13150230050087 57.0
217.80758461 0.435775309801102 58.0
242.7719196 0.337565660476685 59.0
23.458184881 -0.0149531159549952 60.0
0 0.0807435661554337 61.0
178.89142964 0.902951955795288 0.0
178.89142964 2.30848908424377 1.0
177.79683904 1.94436454772949 2.0
173.00773372 5.17138624191284 3.0
0 4.98015832901001 4.0
0 5.83167791366577 5.0
-63.387858472 4.27020597457886 6.0
75.221729998 1.45662713050842 7.0
52.042167052 0.650950491428375 8.0
23.179562946 0.549438893795013 9.0
90.710752718 1.00241816043854 10.0
76.349074104 0.214174389839172 11.0
80.033149738 0.724849462509155 12.0
80.033149738 0.385669380426407 13.0
80.033149738 0.134368419647217 14.0
0 0.0865800008177757 15.0
0 0.137649223208427 16.0
0 1.50539207458496 17.0
53.657997722 1.70773077011108 18.0
-53.657997722 1.79293620586395 19.0
-53.657997722 0.556766748428345 20.0
-53.657997722 -0.0466715693473816 21.0
74.928555954 2.27238035202026 22.0
147.24276692 1.9542897939682 23.0
112.18529345 0.625715136528015 24.0
19.443772864 0.0934060290455818 25.0
222.17132288 1.79765260219574 26.0
222.17132288 2.22500467300415 27.0
27.389749444 1.82690358161926 28.0
249.56107232 1.22052931785583 29.0
249.56107232 1.86816453933716 30.0
46.350602877 1.04646480083466 31.0
9.4623589309 0.48005810379982 32.0
21.270558232 1.53919839859009 33.0
21.270558232 1.20044231414795 34.0
21.270558232 0.0968834683299065 35.0
9.7838252736 0.0215738993138075 36.0
8.4241055753 0.396613776683807 37.0
0 0.0420163162052631 38.0
0 0.112962134182453 39.0
-10.628703496 0.549442529678345 40.0
-22.911910982 0.726311504840851 41.0
4.8044492175 0.396180421113968 42.0
-30.391645251 0.579566657543182 43.0
81.503907733 0.626922607421875 44.0
81.503907733 0.110240958631039 45.0
7.1178035777 0.658317685127258 46.0
2.2425467799 0.391824096441269 47.0
178.89142964 0.0795232132077217 48.0
173.00773372 0.177866011857986 49.0
0 0.532888114452362 50.0
75.221729998 0.14750911295414 51.0
90.710752718 0.352593839168549 52.0
80.033149738 0.055746216326952 53.0
80.033149738 0.142709121108055 54.0
80.033149738 0.0231661777943373 55.0
0 3.81109404563904 56.0
74.928555954 0.163098081946373 57.0
222.17132288 0.878947198390961 58.0
249.56107232 1.10721659660339 59.0
21.270558232 -0.00183001533150673 60.0
0 0.0976335480809212 61.0
176.31025288 0.934133231639862 0.0
176.31025288 1.04225659370422 1.0
174.88411443 1.07049775123596 2.0
170.09504447 5.37457704544067 3.0
0 6.91890811920166 4.0
0 5.10859441757202 5.0
-74.136998444 0.378350883722305 6.0
83.405994436 0.686222195625305 7.0
58.886222226 0.708004474639893 8.0
24.51977221 0.886830806732178 9.0
78.214487302 1.17552483081818 10.0
65.022898118 0.434702157974243 11.0
74.065305418 0.908297717571259 12.0
74.065305418 0.266193062067032 13.0
74.065305418 0.0870367139577866 14.0
0 0.159470394253731 15.0
0 0.180123403668404 16.0
0 1.01348114013672 17.0
59.110795181 1.54627048969269 18.0
-59.110795181 1.93912756443024 19.0
-59.110795181 0.36899346113205 20.0
-59.110795181 -0.0523791834712029 21.0
79.632099115 1.26516509056091 22.0
136.2858588 0.853641927242279 23.0
103.55200479 0.597681522369385 24.0
22.443265554 0.182728931307793 25.0
215.91795792 1.01663053035736 26.0
215.91795792 1.40179419517517 27.0
23.341085236 1.65158426761627 28.0
239.25904315 1.14115309715271 29.0
239.25904315 1.68446588516235 30.0
46.636666531 1.02335476875305 31.0
9.2281000222 0.483787477016449 32.0
20.521303935 1.65147984027863 33.0
20.521303935 1.07669162750244 34.0
20.521303935 0.109582468867302 35.0
9.7769309956 0.0415159091353416 36.0
8.469827617 0.416571527719498 37.0
0 0.0272380672395229 38.0
0 0.168055355548859 39.0
-11.815138393 0.681099712848663 40.0
-25.539788339 0.447941184043884 41.0
4.6452808809 0.355045735836029 42.0
-32.904407589 0.425995379686356 43.0
94.754523018 0.737812519073486 44.0
94.754523018 0.0492869913578033 45.0
6.7926585263 0.517768263816833 46.0
2.462936574 0.387941747903824 47.0
176.31025288 0.157403975725174 48.0
170.09504447 0.144259199500084 49.0
0 0.46559950709343 50.0
83.405994436 0.234536737203598 51.0
78.214487302 0.580095112323761 52.0
74.065305418 0.112647145986557 53.0
74.065305418 0.142833068966866 54.0
74.065305418 0.068355605006218 55.0
0 5.17730760574341 56.0
79.632099115 0.288984596729279 57.0
215.91795792 0.485570251941681 58.0
239.25904315 0.593192756175995 59.0
20.521303935 -0.00224841013550758 60.0
0 0.0772219598293304 61.0
155.8107474 1.02242434024811 0.0
155.8107474 1.73795413970947 1.0
154.22378197 2.49346828460693 2.0
149.43471202 5.20356178283691 3.0
0 5.1889533996582 4.0
0 3.76267099380493 5.0
-78.029867818 0.994824767112732 6.0
87.897523784 0.487384587526321 7.0
68.975293937 1.09486997127533 8.0
18.922229848 0.880384087562561 9.0
97.389254453 1.06499981880188 10.0
83.076108068 0.391989976167679 11.0
87.387369457 0.196245551109314 12.0
87.387369457 0.374227315187454 13.0
87.387369457 0.106477700173855 14.0
0 0.343320578336716 15.0
0 0.200814187526703 16.0
0 1.30535340309143 17.0
58.530778697 0.820550799369812 18.0
-58.530778697 1.36179280281067 19.0
-58.530778697 0.355302810668945 20.0
-58.530778697 0.0394549667835236 21.0
77.769405657 2.15896201133728 22.0
134.37889281 2.04662036895752 23.0
103.5904099 0.557107090950012 24.0
18.308574818 0.0975833535194397 25.0
212.14829847 0.853839933872223 26.0
212.14829847 1.85903358459473 27.0
22.955624479 1.59425437450409 28.0
235.10392294 1.55716192722321 29.0
235.10392294 2.33733892440796 30.0
45.809673139 1.11366713047028 31.0
9.1398395414 0.505242884159088 32.0
19.23862696 1.22354638576508 33.0
19.23862696 0.822699546813965 34.0
19.23862696 0.0838463753461838 35.0
8.359804298 0.0204831957817078 36.0
7.0926656604 0.330628395080566 37.0
0 -0.00218266248703003 38.0
0 0.130650371313095 39.0
-12.99697937 0.433686912059784 40.0
-25.748573268 0.749959230422974 41.0
5.9049480073 0.41619884967804 42.0
-34.051123147 0.465879768133163 43.0
79.36476323 0.582070887088776 44.0
79.36476323 0.104273177683353 45.0
5.6389736882 0.404667258262634 46.0
1.9057605761 0.347614586353302 47.0
155.8107474 0.154022514820099 48.0
149.43471202 0.621745944023132 49.0
0 0.354503452777863 50.0
87.897523784 0.0244276393204927 51.0
97.389254453 0.058942873030901 52.0
87.387369457 0.070536382496357 53.0
87.387369457 0.126711145043373 54.0
87.387369457 0.0621679350733757 55.0
0 0.370950907468796 56.0
77.769405657 0.0942680761218071 57.0
212.14829847 0.443268269300461 58.0
235.10392294 1.37330663204193 59.0
19.23862696 -0.000816302374005318 60.0
0 0.228604793548584 61.0
167.58476687 0.968797922134399 0.0
167.58476687 1.84183192253113 1.0
166.45886125 1.87064385414124 2.0
161.66975596 4.4442024230957 3.0
0 4.51036357879639 4.0
0 2.84955191612244 5.0
-58.670790816 1.2027028799057 6.0
67.78533863 0.728314280509949 7.0
50.438008451 0.942678391933441 8.0
17.347330179 0.987624108791351 9.0
96.895131904 1.07428431510925 10.0
83.034315638 0.209745362401009 11.0
85.607941579 0.871069073677063 12.0
85.607941579 0.243926465511322 13.0
85.607941579 0.155510067939758 14.0
0 0.25121733546257 15.0
0 0.150704637169838 16.0
0 1.69859743118286 17.0
56.902192565 1.19150769710541 18.0
-56.902192565 1.44798946380615 19.0
-56.902192565 0.286358386278152 20.0
-56.902192565 0.0526123382151127 21.0
78.767885738 1.24736130237579 22.0
140.18781487 1.19867408275604 23.0
108.71376094 0.366418570280075 24.0
25.740868769 0.123536713421345 25.0
218.95570061 0.950046360492706 26.0
218.95570061 1.82471561431885 27.0
26.917202792 1.3729499578476 28.0
245.8729034 1.15468299388885 29.0
245.8729034 1.70749831199646 30.0
47.159394166 0.932928681373596 31.0
9.3038510552 0.305076420307159 32.0
21.865693172 1.32072556018829 33.0
21.865693172 0.84685480594635 34.0
21.865693172 0.125967398285866 35.0
9.2912503939 0.0389614626765251 36.0
8.060465871 0.380560010671616 37.0
0 0.0467011071741581 38.0
0 0.168155416846275 39.0
-12.119596314 0.681820750236511 40.0
-24.240829246 0.402358919382095 41.0
4.8375322539 0.399440199136734 42.0
-31.43185156 0.45293065905571 43.0
84.624232609 0.775050342082977 44.0
84.624232609 0.0895495489239693 45.0
7.3620429333 0.728615283966064 46.0
2.5783254546 0.411185681819916 47.0
167.58476687 0.093631312251091 48.0
161.66975596 0.112087212502956 49.0
0 0.215786889195442 50.0
67.78533863 0.206976801156998 51.0
96.895131904 0.0924663990736008 52.0
85.607941579 0.133687883615494 53.0
85.607941579 0.125021532177925 54.0
85.607941579 0.0576784424483776 55.0
0 3.05086088180542 56.0
78.767885738 0.111325547099113 57.0
218.95570061 0.468698412179947 58.0
245.8729034 0.789149582386017 59.0
21.865693172 0.00617345236241817 60.0
0 0.103159785270691 61.0
156.08789713 0.974113047122955 0.0
156.08789713 1.77532732486725 1.0
154.95316032 1.72184884548187 2.0
150.16405506 3.96872544288635 3.0
0 7.37822151184082 4.0
0 5.31990337371826 5.0
-53.33229543 0.484250068664551 6.0
63.966655915 0.766461491584778 7.0
51.284947676 1.06861460208893 8.0
12.681708239 1.00416266918182 9.0
92.533194047 1.05390202999115 10.0
79.215615293 0.153181210160255 11.0
87.242604736 0.555795848369598 12.0
87.242604736 0.31621253490448 13.0
87.242604736 0.0829692482948303 14.0
0 0.172308161854744 15.0
0 0.192475557327271 16.0
0 1.16114783287048 17.0
59.915327384 1.08865654468536 18.0
-59.915327384 1.34574246406555 19.0
-59.915327384 0.268329799175262 20.0
-59.915327384 0.0171011611819267 21.0
81.305298416 1.33171856403351 22.0
137.92840948 0.911332011222839 23.0
102.7066557 0.553419172763824 24.0
19.936334489 0.157280489802361 25.0
219.2337079 0.885992288589478 26.0
219.2337079 1.52242588996887 27.0
26.608522288 2.3044319152832 28.0
245.84223019 0.848998606204987 29.0
245.84223019 1.62799882888794 30.0
47.164631988 1.13544857501984 31.0
9.2935625192 0.385152012109756 32.0
21.389971031 1.52344608306885 33.0
21.389971031 1.07337629795074 34.0
21.389971031 0.102921940386295 35.0
8.7921738174 0.0480792112648487 36.0
7.5451704455 0.372902482748032 37.0
0 0.0617901682853699 38.0
0 0.286932110786438 39.0
-12.297373571 0.458943635225296 40.0
-25.602396415 0.729335486888885 41.0
5.5378582942 0.38625380396843 42.0
-33.421401006 0.501193702220917 43.0
83.796678898 0.615040123462677 44.0
83.796678898 0.105707578361034 45.0
6.8790822101 0.607161998748779 46.0
2.7882881722 0.422707259654999 47.0
156.08789713 0.150203078985214 48.0
150.16405506 0.277400493621826 49.0
0 0.366330325603485 50.0
63.966655915 0.446292966604233 51.0
92.533194047 0.142440348863602 52.0
87.242604736 0.0448075085878372 53.0
87.242604736 0.0263365730643272 54.0
87.242604736 0.0352559387683868 55.0
0 1.5916143655777 56.0
81.305298416 0.110433287918568 57.0
219.2337079 0.35664901137352 58.0
245.84223019 0.125255897641182 59.0
21.389971031 0.000693997368216515 60.0
0 0.0919663533568382 61.0
193.94514301 0.932916104793549 0.0
193.94514301 1.25796127319336 1.0
192.82392381 1.84193301200867 2.0
188.03481853 4.69516324996948 3.0
0 3.71304798126221 4.0
0 3.28091621398926 5.0
-53.417039651 0.82874584197998 6.0
64.053798282 0.603202939033508 7.0
45.725427373 1.21734941005707 8.0
18.328370909 1.27391731739044 9.0
87.665727528 0.929432332515717 10.0
73.824888343 0.0934225767850876 11.0
80.510469113 0.551371037960052 12.0
80.510469113 0.818304479122162 13.0
80.510469113 0.106597453355789 14.0
0 0.0334607735276222 15.0
0 0.201014369726181 16.0
0 1.62842905521393 17.0
64.82248417 1.36659014225006 18.0
-64.82248417 1.17597389221191 19.0
-64.82248417 0.498847484588623 20.0
-64.82248417 -0.00526305288076401 21.0
87.675841742 2.15736985206604 22.0
145.1576121 2.54028558731079 23.0
111.8920979 0.370064854621887 24.0
21.93051647 0.157303646206856 25.0
232.83345384 1.46582698822021 26.0
232.83345384 1.75560450553894 27.0
26.608410385 1.45321619510651 28.0
259.44186422 0.718832969665527 29.0
259.44186422 1.74907231330872 30.0
47.259126461 0.965582847595215 31.0
9.418167516 0.403832495212555 32.0
22.853357571 1.41327607631683 33.0
22.853357571 0.894143044948578 34.0
22.853357571 0.10872957110405 35.0
9.8965897066 0.0279416963458061 36.0
8.7185484467 0.342476844787598 37.0
0 0.00836833193898201 38.0
0 0.258416712284088 39.0
-12.52212192 0.488780826330185 40.0
-25.730844207 0.643869698047638 41.0
4.8566033288 0.451883107423782 42.0
-33.268296353 0.622081518173218 43.0
82.524278175 0.681254923343658 44.0
82.524278175 0.094186432659626 45.0
7.4737993716 0.530130326747894 46.0
2.5564461683 0.362710893154144 47.0
193.94514301 0.209790885448456 48.0
188.03481853 0.271700710058212 49.0
0 0.26296192407608 50.0
64.053798282 0.278696298599243 51.0
87.665727528 0.177883476018906 52.0
80.510469113 0.0187056045979261 53.0
80.510469113 0.268337488174438 54.0
80.510469113 0.152512297034264 55.0
0 3.85624742507935 56.0
87.675841742 0.133609816431999 57.0
232.83345384 0.389410048723221 58.0
259.44186422 1.06845796108246 59.0
22.853357571 0.0120870172977448 60.0
0 0.0990921854972839 61.0
152.51906857 0.728897154331207 0.0
152.51906857 1.94788861274719 1.0
151.4952481 2.18226838111877 2.0
146.706143 3.14912986755371 3.0
0 5.23494911193848 4.0
0 5.05392551422119 5.0
-59.378028872 3.8516902923584 6.0
70.939122478 0.690621376037598 7.0
45.498046444 0.758503317832947 8.0
25.441076034 0.481404215097427 9.0
92.910248474 0.924944519996643 10.0
79.226034317 0.486007243394852 11.0
84.712190841 0.338523477315903 12.0
84.712190841 0.153641328215599 13.0
84.712190841 0.115236043930054 14.0
0 0.18294183909893 15.0
0 0.17485299706459 16.0
0 1.24809515476227 17.0
60.134538067 1.80847406387329 18.0
-60.134538067 1.63864195346832 19.0
-60.134538067 0.396015584468842 20.0
-60.134538067 0.00115881487727165 21.0
81.927754293 2.03784608840942 22.0
144.89265236 1.24205946922302 23.0
112.91457734 0.53236997127533 24.0
22.287696903 0.164650082588196 25.0
226.82040665 1.47638845443726 26.0
226.82040665 1.89695763587952 27.0
27.266969145 1.48389780521393 28.0
254.0873758 0.915495097637177 29.0
254.0873758 1.6498407125473 30.0
47.332098928 1.15166056156158 31.0
9.3981889821 0.451551854610443 32.0
21.793216226 1.4652065038681 33.0
21.793216226 0.979728698730469 34.0
21.793216226 0.15882371366024 35.0
10.012704417 0.031006246805191 36.0
8.7213780271 0.380558788776398 37.0
0 0.0214940384030342 38.0
0 0.107160612940788 39.0
-12.094575553 0.756503701210022 40.0
-24.520879184 0.571249783039093 41.0
4.6893449527 0.469927847385406 42.0
-31.494862517 0.73529326915741 43.0
99.462969307 0.659740090370178 44.0
99.462969307 0.0809668004512787 45.0
7.2194891048 0.599521577358246 46.0
2.5367770203 0.452977240085602 47.0
152.51906857 0.073230043053627 48.0
146.706143 0.303758144378662 49.0
0 0.496217429637909 50.0
70.939122478 0.141599521040916 51.0
92.910248474 0.324802130460739 52.0
84.712190841 0.058175154030323 53.0
84.712190841 0.0692225769162178 54.0
84.712190841 0.0674589723348618 55.0
0 2.64450669288635 56.0
81.927754293 -0.00727850710973144 57.0
226.82040665 0.734779834747314 58.0
254.0873758 0.552979350090027 59.0
21.793216226 -0.0179258156567812 60.0
0 0.108260437846184 61.0
158.59835796 0.938708782196045 0.0
158.59835796 0.914584338665009 1.0
157.33945014 1.30216467380524 2.0
152.55034503 4.79687309265137 3.0
0 6.80721569061279 4.0
0 4.23274803161621 5.0
-53.458539513 0.325808584690094 6.0
64.600895079 1.03558468818665 7.0
40.584689978 0.493040144443512 8.0
24.016205101 0.385973751544952 9.0
88.70886412 1.09485650062561 10.0
75.817145814 0.129537552595139 11.0
83.809627188 0.446422904729843 12.0
83.809627188 0.311502516269684 13.0
83.809627188 0.0875946879386902 14.0
0 0.233401983976364 15.0
0 0.162806630134583 16.0
0 1.48735177516937 17.0
63.331376541 1.86731278896332 18.0
-63.331376541 1.88323962688446 19.0
-63.331376541 0.551416516304016 20.0
-63.331376541 -0.0485236868262291 21.0
85.469766216 1.69500494003296 22.0
140.84248392 1.2219614982605 23.0
108.2261755 0.407916277647018 24.0
23.305171283 0.260763734579086 25.0
226.31225014 1.13658690452576 26.0
226.31225014 2.00361323356628 27.0
27.702766654 1.14214408397675 28.0
254.01501679 0.685345590114594 29.0
254.01501679 1.55689346790314 30.0
46.245794507 1.04514062404633 31.0
9.3550512565 0.331210613250732 32.0
22.138389675 1.51861715316772 33.0
22.138389675 0.825138866901398 34.0
22.138389675 0.129409343004227 35.0
9.3298365605 0.0263879559934139 36.0
8.0042960024 0.365474015474319 37.0
0 0.0529863461852074 38.0
0 0.192342504858971 39.0
-11.125482227 0.648508012294769 40.0
-24.281146309 0.7025426030159 41.0
5.0106791581 0.508852243423462 42.0
-32.093847493 0.562854886054993 43.0
101.06691207 0.690610289573669 44.0
101.06691207 0.139868170022964 45.0
6.9745626713 0.657049000263214 46.0
2.8181284221 0.40499809384346 47.0
158.59835796 0.117585197091103 48.0
152.55034503 0.19016620516777 49.0
0 0.251427054405212 50.0
64.600895079 0.213451579213142 51.0
88.70886412 0.141888797283173 52.0
83.809627188 0.102346852421761 53.0
83.809627188 0.0709299370646477 54.0
83.809627188 0.05665672570467 55.0
0 1.20600783824921 56.0
85.469766216 0.0443110018968582 57.0
226.31225014 0.77274215221405 58.0
254.01501679 0.593366861343384 59.0
22.138389675 0.011764595285058 60.0
0 0.0795009285211563 61.0
180.60397398 0.773415625095367 0.0
180.60397398 1.6791797876358 1.0
179.15825757 1.75514912605286 2.0
174.3691523 4.58112382888794 3.0
0 6.4782338142395 4.0
0 3.5198290348053 5.0
-65.961621177 1.70955419540405 6.0
76.712407775 0.982133746147156 7.0
50.318603201 0.876531302928925 8.0
26.393804574 0.858929514884949 9.0
95.594170953 0.808447480201721 10.0
83.119620861 0.0598427131772041 11.0
87.17507401 0.779391288757324 12.0
87.17507401 0.619606375694275 13.0
87.17507401 0.0904008895158768 14.0
0 0.175914272665977 15.0
0 0.238080888986588 16.0
0 1.22288680076599 17.0
55.028947908 0.926543474197388 18.0
-55.028947908 0.587005198001862 19.0
-55.028947908 0.479786843061447 20.0
-55.028947908 0.014452587813139 21.0
76.895326099 1.07343327999115 22.0
142.5123643 0.675884783267975 23.0
107.839879 0.404888540506363 24.0
25.253698034 0.229578226804733 25.0
219.4076904 0.920438170433044 26.0
219.4076904 2.01776218414307 27.0
26.811718688 1.93215954303741 28.0
246.21940908 0.907247245311737 29.0
246.21940908 1.64279675483704 30.0
47.054958676 1.06596899032593 31.0
9.3784137338 0.462070435285568 32.0
21.866378191 1.56149315834045 33.0
21.866378191 1.05159187316895 34.0
21.866378191 0.0986416786909103 35.0
9.5116024415 0.0191980600357056 36.0
8.3418316618 0.388302803039551 37.0
0 0.0332828834652901 38.0
0 0.283239603042603 39.0
-11.289064138 0.379164487123489 40.0
-25.192553458 0.814645886421204 41.0
4.9241193617 0.440272182226181 42.0
-32.820824007 0.428862690925598 43.0
88.564777724 0.839222609996796 44.0
88.564777724 0.0474341064691544 45.0
7.6085467444 0.667673349380493 46.0
2.8399342833 0.333551108837128 47.0
180.60397398 0.168672919273376 48.0
174.3691523 0.455792039632797 49.0
0 0.365931749343872 50.0
76.712407775 0.142546340823174 51.0
95.594170953 0.139310017228127 52.0
87.17507401 0.044169507920742 53.0
87.17507401 0.259002596139908 54.0
87.17507401 0.07037103921175 55.0
0 2.49397110939026 56.0
76.895326099 0.255000948905945 57.0
219.4076904 0.467748463153839 58.0
246.21940908 0.990867674350739 59.0
21.866378191 -0.00941211171448231 60.0
0 0.107528366148472 61.0
127.180679 0.749062240123749 0.0
127.180679 2.7391369342804 1.0
125.58578297 2.53487467765808 2.0
120.79667777 2.99513578414917 3.0
0 4.26979970932007 4.0
0 5.06571960449219 5.0
-70.413304397 5.56256246566772 6.0
78.24615527 1.35908794403076 7.0
60.529441952 0.441686749458313 8.0
17.716713318 0.370275139808655 9.0
82.903645957 0.789711475372314 10.0
70.701085713 0.0750414878129959 11.0
71.243794013 0.277390152215958 12.0
71.243794013 0.202068299055099 13.0
71.243794013 0.0819668993353844 14.0
0 0.0921089798212051 15.0
0 0.308565199375153 16.0
0 1.30931401252747 17.0
66.640029121 2.30836200714111 18.0
-66.640029121 1.82000386714935 19.0
-66.640029121 0.584513545036316 20.0
-66.640029121 -0.0469973720610142 21.0
86.96605969 1.77240359783173 22.0
141.75640182 1.64696848392487 23.0
107.98047003 0.510650277137756 24.0
11.103412742 0.0686446279287338 25.0
228.72246151 0.975058376789093 26.0
228.72246151 1.04160559177399 27.0
23.39623603 1.58383417129517 28.0
252.11869754 0.760948419570923 29.0
252.11869754 1.3491063117981 30.0
46.26612198 1.01979076862335 31.0
9.1686963647 0.502631843090057 32.0
20.326030569 1.22971963882446 33.0
20.326030569 0.898577213287354 34.0
20.326030569 0.145616665482521 35.0
9.7017669077 0.0348492115736008 36.0
8.532989078 0.37519109249115 37.0
0 0.0214014034718275 38.0
0 0.122348427772522 39.0
-14.005801964 0.732178390026093 40.0
-25.861788039 0.389613687992096 41.0
5.7401857571 0.715308725833893 42.0
-34.190447116 0.795021116733551 43.0
104.36049256 0.591799795627594 44.0
104.36049256 0.0712939351797104 45.0
6.8567290075 0.559761226177216 46.0
2.3854900507 0.354342013597488 47.0
127.180679 0.155576765537262 48.0
120.79667777 0.444889158010483 49.0
0 0.501926243305206 50.0
78.24615527 0.176195114850998 51.0
82.903645957 0.100213617086411 52.0
71.243794013 0.0502329021692276 53.0
71.243794013 0.0370591729879379 54.0
71.243794013 0.15302100777626 55.0
0 7.33933782577515 56.0
86.96605969 0.219504922628403 57.0
228.72246151 0.107056885957718 58.0
252.11869754 0.867597043514252 59.0
20.326030569 -0.00453308876603842 60.0
0 0.104240447282791 61.0
173.93792385 0.89235919713974 0.0
173.93792385 0.850219964981079 1.0
172.46945066 1.2144056558609 2.0
167.01626118 4.59403228759766 3.0
0 3.90005946159363 4.0
0 4.37148094177246 5.0
-65.107834758 0.59557831287384 6.0
75.641075381 0.566247582435608 7.0
41.371102879 0.844977140426636 8.0
34.269972502 0.715158224105835 9.0
90.69642859 0.844036996364594 10.0
77.575362446 0.11094406247139 11.0
85.65875839 0.892620146274567 12.0
85.65875839 0.843087255954742 13.0
85.65875839 0.0995373278856277 14.0
0 0.204510152339935 15.0
0 0.163638681173325 16.0
0 1.14811909198761 17.0
56.424836521 1.49224102497101 18.0
-56.424836521 1.39195466041565 19.0
-56.424836521 0.503290593624115 20.0
-56.424836521 0.00661348551511765 21.0
79.147563353 2.02186465263367 22.0
147.96877914 1.39821863174438 23.0
110.87958171 0.584644019603729 24.0
24.841228324 0.174803018569946 25.0
227.11634249 1.47721362113953 26.0
227.11634249 1.87678194046021 27.0
24.406257906 1.5619181394577 28.0
251.52260039 0.809309124946594 29.0
251.52260039 1.93714833259583 30.0
46.223366616 0.984576463699341 31.0
9.3301905712 0.436596900224686 32.0
22.722726833 1.27790462970734 33.0
22.722726833 0.997171223163605 34.0
22.722726833 0.12878255546093 35.0
9.8272840449 0.0343481786549091 36.0
8.4809750319 0.377813696861267 37.0
0 0.01039033010602 38.0
0 0.0521742701530457 39.0
-11.20829884 0.612146198749542 40.0
-22.750849984 0.673234522342682 41.0
4.6527084944 0.421085596084595 42.0
-29.980870202 0.353848576545715 43.0
94.063519646 0.776563584804535 44.0
94.063519646 0.0863417387008667 45.0
7.1891994591 0.713403761386871 46.0
2.496303848 0.423412382602692 47.0
173.93792385 0.249827668070793 48.0
167.01626118 -0.00857461523264647 49.0
0 0.214208513498306 50.0
75.641075381 0.122331157326698 51.0
90.69642859 0.211086273193359 52.0
85.65875839 0.0170088075101376 53.0
85.65875839 0.198897361755371 54.0
85.65875839 0.282437831163406 55.0
0 3.83818745613098 56.0
79.147563353 0.0995512902736664 57.0
227.11634249 0.493632316589355 58.0
251.52260039 0.931183576583862 59.0
22.722726833 -0.00551546923816204 60.0
0 0.0962932556867599 61.0
149.67088946 0.746440052986145 0.0
149.67088946 2.28305125236511 1.0
148.69252459 1.93071746826172 2.0
143.90341931 4.61247873306274 3.0
0 4.09856653213501 4.0
0 4.2315502166748 5.0
-65.093240364 0.365220248699188 6.0
74.215048866 0.758724689483643 7.0
64.261097531 1.26860904693604 8.0
9.9539513355 1.4240220785141 9.0
88.103818123 0.909116864204407 10.0
77.313378911 0.542916834354401 11.0
76.321898925 1.13076508045197 12.0
76.321898925 0.448456257581711 13.0
76.321898925 0.099871538579464 14.0
0 0.257013499736786 15.0
0 0.102865561842918 16.0
0 1.64091312885284 17.0
54.074815799 1.09279346466064 18.0
-54.074815799 1.82813370227814 19.0
-54.074815799 0.26557269692421 20.0
-54.074815799 0.0542236156761646 21.0
75.094919951 1.0076390504837 22.0
131.56417937 0.772089958190918 23.0
98.963449061 0.803999781608582 24.0
11.289456379 0.0806578770279884 25.0
206.65909932 0.756150305271149 26.0
206.65909932 2.22370553016663 27.0
23.614012197 2.67562747001648 28.0
230.27311152 0.885013282299042 29.0
230.27311152 1.52884232997894 30.0
47.510733401 1.07468330860138 31.0
9.3496446732 0.507940649986267 32.0
21.020104152 1.84059083461761 33.0
21.020104152 0.32375380396843 34.0
21.020104152 0.141133382916451 35.0
9.6263425571 0.0361296162009239 36.0
8.2889530371 0.453685939311981 37.0
0 0.0794814005494118 38.0
0 0.266529113054276 39.0
-10.561840765 0.258192449808121 40.0
-22.967558184 0.86712509393692 41.0
5.8858232065 0.492247760295868 42.0
-31.654023277 0.831627249717712 43.0
85.408499905 0.852964997291565 44.0
85.408499905 0.0866098999977112 45.0
6.9381910974 0.719549417495728 46.0
2.5411370005 0.351379901170731 47.0
149.67088946 0.0896399691700935 48.0
143.90341931 0.115279860794544 49.0
0 0.306916773319244 50.0
74.215048866 0.338965713977814 51.0
88.103818123 0.0796594396233559 52.0
76.321898925 0.219022333621979 53.0
76.321898925 -0.0278245601803064 54.0
76.321898925 0.198954671621323 55.0
0 2.61449694633484 56.0
75.094919951 0.0776474326848984 57.0
206.65909932 0.460396289825439 58.0
230.27311152 0.10784126073122 59.0
21.020104152 -0.0133614391088486 60.0
0 0.0781195983290672 61.0
164.01816274 1.04684925079346 0.0
164.01816274 1.02931153774261 1.0
162.86926654 1.42310106754303 2.0
158.08016131 4.57212591171265 3.0
0 4.75159883499146 4.0
0 3.69443941116333 5.0
-55.62235735 0.829688668251038 6.0
68.221144015 1.06743931770325 7.0
66.100553178 0.836564660072327 8.0
2.1205908368 1.00042736530304 9.0
97.677001083 1.12372171878815 10.0
85.270865341 0.245772451162338 11.0
89.096036806 0.677231013774872 12.0
89.096036806 0.157022744417191 13.0
89.096036806 0.101527146995068 14.0
0 0.129266828298569 15.0
0 0.169135972857475 16.0
0 1.50290560722351 17.0
59.682707468 1.08249092102051 18.0
-59.682707468 0.893071889877319 19.0
-59.682707468 1.07368731498718 20.0
-59.682707468 -0.0195853151381016 21.0
81.409799031 1.49674069881439 22.0
139.49078302 1.93955707550049 23.0
104.85796362 0.339016437530518 24.0
24.153070493 0.150437325239182 25.0
220.90058205 1.15053534507751 26.0
220.90058205 1.80984842777252 27.0
24.965978452 1.6895238161087 28.0
245.8665605 0.682919204235077 29.0
245.8665605 1.66721272468567 30.0
46.550753569 1.21247410774231 31.0
9.4592395804 0.379583925008774 32.0
21.727091563 1.55397748947144 33.0
21.727091563 1.08063769340515 34.0
21.727091563 0.108796469867229 35.0
9.8869517721 0.0381510183215141 36.0
8.6772340563 0.352915793657303 37.0
0 0.02911177277565 38.0
0 0.202090725302696 39.0
-12.411761253 0.453614771366119 40.0
-25.835374203 0.658250510692596 41.0
5.3049073253 0.473633915185928 42.0
-33.613246958 0.425116956233978 43.0
88.935573833 0.716306269168854 44.0
88.935573833 0.0838581547141075 45.0
6.4824864256 0.590247809886932 46.0
2.7985627485 0.349633783102036 47.0
164.01816274 0.143859446048737 48.0
158.08016131 0.304457604885101 49.0
0 0.241260454058647 50.0
68.221144015 0.0648697838187218 51.0
97.677001083 0.242162391543388 52.0
89.096036806 0.140967100858688 53.0
89.096036806 0.0743238776922226 54.0
89.096036806 0.0669883638620377 55.0
0 0.988121449947357 56.0
81.409799031 -0.040532611310482 57.0
220.90058205 0.663220882415771 58.0
245.8665605 0.308161348104477 59.0
21.727091563 -0.0107783786952496 60.0
0 0.0977928638458252 61.0
169.70431856 1.11992216110229 0.0
169.70431856 1.63292217254639 1.0
168.6141255 1.25753164291382 2.0
163.82502028 3.81693935394287 3.0
0 5.94592332839966 4.0
0 5.39562082290649 5.0
-68.533865757 0.686688542366028 6.0
78.975225766 1.01625871658325 7.0
61.375968763 0.891045689582825 8.0
17.599257002 0.64980685710907 9.0
82.255299467 1.11445844173431 10.0
70.788586759 0.302688479423523 11.0
71.329271595 0.236151859164238 12.0
71.329271595 0.946238279342651 13.0
71.329271595 0.046182818710804 14.0
0 0.063129149377346 15.0
0 0.180567026138306 16.0
0 1.4332070350647 17.0
56.549662459 1.46436429023743 18.0
-56.549662459 0.821803867816925 19.0
-56.549662459 0.721475422382355 20.0
-56.549662459 -0.0722120255231857 21.0
78.965570719 2.07222461700439 22.0
145.42076472 1.81835520267487 23.0
111.35344771 0.112119257450104 24.0
24.358459417 0.194707438349724 25.0
224.38633544 1.3540096282959 26.0
224.38633544 1.70337724685669 27.0
23.057515288 2.32764148712158 28.0
247.44385072 1.48776149749756 29.0
247.44385072 2.19381213188171 30.0
47.067235662 1.03985929489136 31.0
9.4916512131 0.536261141300201 32.0
22.415908259 2.02801156044006 33.0
22.415908259 0.762780129909515 34.0
22.415908259 0.115347795188427 35.0
10.410332191 0.02672372572124 36.0
9.0323139749 0.36367604136467 37.0
0 0.0396559499204159 38.0
0 0.342231839895248 39.0
-10.538716491 0.549051880836487 40.0
-23.885084661 0.694280743598938 41.0
4.8173070918 0.53065824508667 42.0
-31.322331291 0.789683282375336 43.0
97.136455395 0.560333430767059 44.0
97.136455395 0.147814333438873 45.0
7.4018520587 0.580250382423401 46.0
2.35138937 0.399888932704926 47.0
169.70431856 0.145825326442719 48.0
163.82502028 0.0556047931313515 49.0
0 0.264610171318054 50.0
78.975225766 0.15956574678421 51.0
82.255299467 0.346019089221954 52.0
71.329271595 0.052895825356245 53.0
71.329271595 0.303936183452606 54.0
71.329271595 0.171270966529846 55.0
0 4.89491653442383 56.0
78.965570719 0.0258775278925896 57.0
224.38633544 0.245093867182732 58.0
247.44385072 1.21142065525055 59.0
22.415908259 -0.0220861583948135 60.0
0 0.0779752880334854 61.0
141.47758763 0.778614580631256 0.0
141.47758763 0.499496728181839 1.0
140.2183693 0.727067232131958 2.0
134.75369856 4.40848827362061 3.0
4.054769559 6.35576248168945 4.0
4.054769559 6.02398538589478 5.0
-71.239077348 6.21525096893311 6.0
81.518758795 0.308448135852814 7.0
52.602264112 0.664051115512848 8.0
28.916494683 0.824024379253387 9.0
85.180381752 1.05311715602875 10.0
73.417582217 0.794004678726196 11.0
78.08585359 1.39998137950897 12.0
78.08585359 0.827953636646271 13.0
78.08585359 0.0197831019759178 14.0
0 -0.0148347551003098 15.0
0 0.223803922533989 16.0
0 1.03657197952271 17.0
53.122791149 2.03728127479553 18.0
-53.122791149 1.81877946853638 19.0
-53.122791149 0.241274386644363 20.0
-53.122791149 0.120533317327499 21.0
74.272664508 2.57396960258484 22.0
142.02825532 2.67823553085327 23.0
108.33489201 0.442614763975143 24.0
18.720276829 0.156598508358002 25.0
216.30091982 1.88888454437256 26.0
216.30091982 1.97012877464294 27.0
25.962655088 1.11749649047852 28.0
242.26357491 0.76105797290802 29.0
242.26357491 1.51092207431793 30.0
46.25824961 0.929167091846466 31.0
9.399781992 0.452336311340332 32.0
21.149873359 1.20134150981903 33.0
21.149873359 1.36696410179138 34.0
21.149873359 0.140220254659653 35.0
9.2793870099 0.0270556434988976 36.0
7.9174347001 0.377859860658646 37.0
0 -0.0102338343858719 38.0
0 0.10456008464098 39.0
-11.85914753 0.684298634529114 40.0
-25.099438019 0.461516499519348 41.0
4.8953983892 0.339477896690369 42.0
-32.714908976 0.52878212928772 43.0
83.463907951 0.575260162353516 44.0
83.463907951 0.104712255299091 45.0
6.5661482577 0.524162590503693 46.0
2.672545718 0.407368510961533 47.0
141.47758763 0.167616739869118 48.0
134.75369856 0.088379368185997 49.0
4.054769559 0.612865269184113 50.0
81.518758795 0.190635532140732 51.0
85.180381752 0.714950621128082 52.0
78.08585359 0.245173573493958 53.0
78.08585359 0.319370567798615 54.0
78.08585359 0.0811482295393944 55.0
0 2.96658277511597 56.0
74.272664508 -0.0607784390449524 57.0
216.30091982 0.774251401424408 58.0
242.26357491 1.01680016517639 59.0
21.149873359 0.0361649692058563 60.0
0 0.0960872918367386 61.0
140.70939037 0.801798939704895 0.0
140.70939037 1.80450701713562 1.0
139.21788816 2.05708026885986 2.0
133.75325276 4.74685001373291 3.0
1.2597677528 8.95980262756348 4.0
1.2597677528 6.51952409744263 5.0
-67.211027668 0.845490992069244 6.0
75.664382849 1.12970614433289 7.0
53.511019087 0.782010495662689 8.0
22.153363762 0.956093490123749 9.0
96.808441977 0.80060613155365 10.0
82.313075781 0.272689670324326 11.0
77.471984819 1.10708749294281 12.0
77.471984819 0.391500532627106 13.0
77.471984819 0.0548424385488033 14.0
0 0.0167082548141479 15.0
0 0.170359551906586 16.0
0 1.19300556182861 17.0
52.482014533 1.25609540939331 18.0
-52.482014533 1.60603010654449 19.0
-52.482014533 0.342394590377808 20.0
-52.482014533 0.013392873108387 21.0
72.985649869 1.55423784255981 22.0
148.39189188 0.725847005844116 23.0
118.99093525 0.629138946533203 24.0
24.352320698 0.14192721247673 25.0
221.37754175 1.38581430912018 26.0
221.37754175 2.72216176986694 27.0
26.467613723 2.71200370788574 28.0
247.84515547 0.839910328388214 29.0
247.84515547 1.66676914691925 30.0
47.527029764 1.10174238681793 31.0
9.5220019424 0.513395845890045 32.0
20.503635336 1.14967310428619 33.0
20.503635336 1.16633927822113 34.0
20.503635336 0.091340184211731 35.0
10.017291351 0.020884582772851 36.0
8.8995715592 0.420567154884338 37.0
0 0.061737135052681 38.0
0 0.212489575147629 39.0
-10.969748703 0.382411330938339 40.0
-23.664727755 0.968661069869995 41.0
4.7337053842 0.359171032905579 42.0
-30.459635165 0.260336875915527 43.0
84.601319187 0.85645192861557 44.0
84.601319187 0.0276453644037247 45.0
8.1834264958 0.617937326431274 46.0
2.9318521221 0.398272603750229 47.0
140.70939037 0.120301447808743 48.0
133.75325276 0.37105405330658 49.0
1.2597677528 0.819765448570251 50.0
75.664382849 0.327832251787186 51.0
96.808441977 0.072473905980587 52.0
77.471984819 0.179951339960098 53.0
77.471984819 0.137829259037971 54.0
77.471984819 0.155042290687561 55.0
0 3.28992700576782 56.0
72.985649869 0.26233047246933 57.0
221.37754175 0.95178633928299 58.0
247.84515547 0.06693035364151 59.0
20.503635336 -0.00068042054772377 60.0
0 0.0960425585508347 61.0
145.33847562 0.816981673240662 0.0
145.33847562 2.04870581626892 1.0
143.92724998 1.89742207527161 2.0
139.13814471 4.1843729019165 3.0
0 4.22861385345459 4.0
0 3.42574024200439 5.0
-56.701034302 0.240014553070068 6.0
67.882393474 0.860779285430908 7.0
53.821804983 0.90195369720459 8.0
14.060588491 1.00170826911926 9.0
87.592201246 1.06986618041992 10.0
75.113100616 0.45414400100708 11.0
75.890744291 0.472328156232834 12.0
75.890744291 0.566316962242126 13.0
75.890744291 0.150350838899612 14.0
0 0.111185193061829 15.0
0 0.142799004912376 16.0
0 1.36634409427643 17.0
59.516526739 1.16773462295532 18.0
-59.516526739 1.02204370498657 19.0
-59.516526739 0.277786672115326 20.0
-59.516526739 0.0415200889110565 21.0
81.784354786 2.09523820877075 22.0
140.71013139 1.58560383319855 23.0
107.74880538 0.552335500717163 24.0
25.538685574 0.163082376122475 25.0
222.49448617 0.840231597423553 26.0
222.49448617 1.56010317802429 27.0
23.602278144 1.66973543167114 28.0
246.09676432 1.10029006004333 29.0
246.09676432 1.75662183761597 30.0
47.1263038 1.12489116191864 31.0
9.3807188356 0.455442011356354 32.0
22.267828047 1.55234956741333 33.0
22.267828047 1.03985631465912 34.0
22.267828047 0.128130197525024 35.0
9.7942093011 0.0268810745328665 36.0
8.3760142065 0.379208952188492 37.0
0 0.0104343257844448 38.0
0 0.128116577863693 39.0
-11.657861798 0.600336253643036 40.0
-23.341668152 0.602225542068481 41.0
4.760537955 0.593904137611389 42.0
-30.630695803 0.767793476581573 43.0
89.218231372 0.781799614429474 44.0
89.218231372 0.0937781929969788 45.0
6.9791008771 0.657372355461121 46.0
2.5850211943 0.373846650123596 47.0
145.33847562 0.100558273494244 48.0
139.13814471 0.338267177343369 49.0
0 0.185127332806587 50.0
67.882393474 0.294496089220047 51.0
87.592201246 0.427413314580917 52.0
75.890744291 0.062452144920826 53.0
75.890744291 0.0708401575684547 54.0
75.890744291 0.10663229227066 55.0
0 5.94572925567627 56.0
81.784354786 0.036654818803072 57.0
222.49448617 0.239036813378334 58.0
246.09676432 1.10099291801453 59.0
22.267828047 -0.0126877892762423 60.0
0 0.0940820649266243 61.0
169.92499607 0.989831984043121 0.0
169.92499607 1.35506224632263 1.0
168.91400019 1.51328992843628 2.0
164.12491665 4.05334758758545 3.0
0 8.97686100006104 4.0
0 3.78590679168701 5.0
-52.877859821 2.10850477218628 6.0
63.061598806 0.821117520332336 7.0
62.554724304 0.612490475177765 8.0
0.50687450216 0.796794950962067 9.0
92.745406167 0.913215219974518 10.0
80.644911587 0.00896568596363068 11.0
87.156357995 0.379171013832092 12.0
87.156357995 0.363159567117691 13.0
87.156357995 0.143732041120529 14.0
0 0.313402861356735 15.0
0 0.147333815693855 16.0
0 1.48876941204071 17.0
50.267519904 0.877823531627655 18.0
-50.267519904 0.484540969133377 19.0
-50.267519904 0.710407376289368 20.0
-50.267519904 -0.0631736516952515 21.0
72.298866596 0.640070855617523 22.0
140.58954472 0.457287520170212 23.0
107.59070108 0.470393627882004 24.0
21.937366852 0.294465929269791 25.0
212.88841132 0.847785234451294 26.0
212.88841132 1.38655233383179 27.0
25.285042347 1.47650754451752 28.0
238.17345367 0.498218894004822 29.0
238.17345367 1.53791832923889 30.0
46.364973894 1.3232227563858 31.0
9.2172570461 0.296719580888748 32.0
22.031346692 0.961329817771912 33.0
22.031346692 1.33488988876343 34.0
22.031346692 0.0982741042971611 35.0
9.9871282768 0.0239274855703115 36.0
8.7679051322 0.406859815120697 37.0
0 0.0349758416414261 38.0
0 0.269409090280533 39.0
-10.330832217 0.387101471424103 40.0
-22.810770556 0.752013802528381 41.0
5.3561223474 0.601856470108032 42.0
-30.771704455 0.636171936988831 43.0
85.157477093 0.663252592086792 44.0
85.157477093 0.0815039798617363 45.0
6.9893235804 0.638820767402649 46.0
2.6659979393 0.416665196418762 47.0
169.92499607 0.184608474373817 48.0
164.12491665 0.19124598801136 49.0
0 1.23537147045135 50.0
63.061598806 0.116129547357559 51.0
92.745406167 0.184748664498329 52.0
87.156357995 0.0410706885159016 53.0
87.156357995 0.0370861254632473 54.0
87.156357995 0.00489569455385208 55.0
0 1.62954652309418 56.0
72.298866596 0.245779901742935 57.0
212.88841132 0.635441601276398 58.0
238.17345367 0.121191784739494 59.0
22.031346692 -0.0108015015721321 60.0
0 0.0876148566603661 61.0
195.56007582 1.12952041625977 0.0
195.56007582 1.21890568733215 1.0
194.55775627 1.47212135791779 2.0
189.09308552 4.95594120025635 3.0
13.824906671 6.0577335357666 4.0
13.824906671 6.75724411010742 5.0
-54.308878836 4.25876760482788 6.0
66.965070032 0.438470423221588 7.0
40.244572138 0.481199979782104 8.0
26.720497895 0.660236120223999 9.0
69.195427002 1.14511799812317 10.0
56.295472326 0.0177251771092415 11.0
72.931255285 0.925024509429932 12.0
72.931255285 0.814217746257782 13.0
72.931255285 0.160649418830872 14.0
0 0.177799910306931 15.0
0 0.198497802019119 16.0
0 1.55308938026428 17.0
59.600097115 1.35943210124969 18.0
-59.600097115 1.0166609287262 19.0
-59.600097115 0.285414904356003 20.0
-59.600097115 0.274930655956268 21.0
80.671946426 1.10861051082611 22.0
143.83289899 0.997003018856049 23.0
108.84223244 0.797512292861938 24.0
21.756983483 0.166768878698349 25.0
224.50484542 1.17938077449799 26.0
224.50484542 1.56584346294403 27.0
22.787673889 2.42889785766602 28.0
247.2925193 0.767736554145813 29.0
247.2925193 1.60006773471832 30.0
46.618213595 1.16084313392639 31.0
9.2853366316 0.615816414356232 32.0
21.071849311 1.52325820922852 33.0
21.071849311 0.917657256126404 34.0
21.071849311 0.129581570625305 35.0
8.7509198518 0.0250429920852184 36.0
7.4833217157 0.340721189975739 37.0
0 0.0718502104282379 38.0
0 0.289733201265335 39.0
-11.272585315 0.449352353811264 40.0
-23.346695982 0.60426139831543 41.0
5.5855467463 0.449273765087128 42.0
-31.778948576 0.594947636127472 43.0
81.136163517 0.709294021129608 44.0
81.136163517 0.100224301218987 45.0
6.7139376693 0.609954357147217 46.0
2.0952352346 0.417434304952621 47.0
195.56007582 0.105142295360565 48.0
189.09308552 0.176550626754761 49.0
13.824906671 0.65046501159668 50.0
66.965070032 0.0781786367297173 51.0
69.195427002 0.0872409492731094 52.0
72.931255285 0.0394497402012348 53.0
72.931255285 0.134975507855415 54.0
72.931255285 0.0591237060725689 55.0
0 6.39119100570679 56.0
80.671946426 0.0597647950053215 57.0
224.50484542 0.324213773012161 58.0
247.2925193 0.163360953330994 59.0
21.071849311 0.00397186726331711 60.0
0 0.0867924839258194 61.0
196.62754221 0.914658606052399 0.0
196.62754221 1.49226975440979 1.0
195.55373236 1.95870566368103 2.0
190.08906177 3.88719820976257 3.0
19.285150223 7.82661056518555 4.0
19.285150223 3.5152702331543 5.0
-56.663044678 0.87945568561554 6.0
68.312012447 0.951918482780457 7.0
38.210536663 0.595150709152222 8.0
30.101475785 0.528703808784485 9.0
83.657303842 0.942676067352295 10.0
71.865355248 0.220926180481911 11.0
78.537376528 0.128927513957024 12.0
78.537376528 0.465893030166626 13.0
78.537376528 0.0726580396294594 14.0
0 0.0401646830141544 15.0
0 0.211715146899223 16.0
0 1.30891168117523 17.0
53.820093188 0.999176025390625 18.0
-53.820093188 0.947526574134827 19.0
-53.820093188 0.557441771030426 20.0
-53.820093188 0.0327421203255653 21.0
73.55833736 1.53134453296661 22.0
150.3552748 1.42723226547241 23.0
113.30786982 0.448957353830338 24.0
21.378501789 0.187224999070168 25.0
223.91361216 1.18244683742523 26.0
223.91361216 2.21851015090942 27.0
28.777796346 2.52662134170532 28.0
252.6914085 1.11063718795776 29.0
252.6914085 1.70151662826538 30.0
46.958926197 1.12976443767548 31.0
9.4654154081 0.502961993217468 32.0
19.738244172 0.99612283706665 33.0
19.738244172 1.04269516468048 34.0
19.738244172 0.100045561790466 35.0
8.7873093224 0.0215999837964773 36.0
7.4089133137 0.403053164482117 37.0
0 0.0336881168186665 38.0
0 0.124446719884872 39.0
-12.152682061 0.648639917373657 40.0
-25.170035012 0.596660852432251 41.0
5.2975704318 0.384213209152222 42.0
-32.820701219 0.696335315704346 43.0
95.246503424 0.705743908882141 44.0
95.246503424 0.0887703895568848 45.0
7.2796958254 0.626347959041595 46.0
2.7348114115 0.411921083927155 47.0
196.62754221 0.159784287214279 48.0
190.08906177 0.271130412817001 49.0
19.285150223 0.496803790330887 50.0
68.312012447 0.137298703193665 51.0
83.657303842 0.216356694698334 52.0
78.537376528 0.0489468090236187 53.0
78.537376528 0.0488253124058247 54.0
78.537376528 0.141182109713554 55.0
0 4.52076482772827 56.0
73.55833736 0.0620643943548203 57.0
223.91361216 0.804028868675232 58.0
252.6914085 0.377580434083939 59.0
19.738244172 -0.0104864947497845 60.0
0 0.09085463732481 61.0
122.55406229 0.849186837673187 0.0
122.55406229 1.30016732215881 1.0
121.469698 1.92224144935608 2.0
116.68059273 3.04071021080017 3.0
0 4.69502019882202 4.0
0 2.81698894500732 5.0
-62.138047006 0.2180435359478 6.0
70.667299108 0.405896544456482 7.0
53.951159789 0.770420432090759 8.0
16.716139318 1.07823967933655 9.0
95.014329025 0.987693667411804 10.0
80.400203699 0.461147278547287 11.0
81.445609321 0.130500316619873 12.0
81.445609321 0.591915190219879 13.0
81.445609321 0.0257377978414297 14.0
0 0.0278334934264421 15.0
0 0.206488087773323 16.0
0 0.996777951717377 17.0
57.557254896 1.53256380558014 18.0
-57.557254896 1.19104623794556 19.0
-57.557254896 1.14552652835846 20.0
-57.557254896 -0.0607312843203545 21.0
79.162565015 0.852371513843536 22.0
152.41823057 1.5490642786026 23.0
114.6878733 0.324971973896027 24.0
18.802337869 0.129682958126068 25.0
231.58079558 1.34428286552429 26.0
231.58079558 2.08821296691895 27.0
24.027616737 2.16253304481506 28.0
255.60841232 0.697780609130859 29.0
255.60841232 1.50388395786285 30.0
46.647079924 1.0947242975235 31.0
9.3357005085 0.456152230501175 32.0
21.605310119 1.70324659347534 33.0
21.605310119 0.865020930767059 34.0
21.605310119 0.113863252103329 35.0
8.7996270174 0.0587753653526306 36.0
7.6346333946 0.383225560188293 37.0
0 0.0863194316625595 38.0
0 0.0727606862783432 39.0
-13.422730317 0.579452037811279 40.0
-26.137398387 0.580781042575836 41.0
4.8282847184 0.439620018005371 42.0
-33.69188016 0.17379567027092 43.0
86.874495597 0.593947768211365 44.0
86.874495597 0.0896635204553604 45.0
7.3423044485 0.774078369140625 46.0
2.2806164287 0.41869455575943 47.0
122.55406229 0.116376608610153 48.0
116.68059273 0.542014896869659 49.0
0 0.335027456283569 50.0
70.667299108 0.10519877076149 51.0
95.014329025 0.450311124324799 52.0
81.445609321 0.109950423240662 53.0
81.445609321 0.0140689667314291 54.0
81.445609321 0.125385165214539 55.0
0 1.53123939037323 56.0
79.162565015 -0.0433643274009228 57.0
231.58079558 0.267517238855362 58.0
255.60841232 0.0511101111769676 59.0
21.605310119 0.0155153274536133 60.0
0 0.0819604620337486 61.0
166.23839015 0.950781583786011 0.0
166.23839015 1.40397119522095 1.0
164.85099552 1.97871029376984 2.0
160.06189024 3.75984048843384 3.0
0 3.758629322052 4.0
0 3.30260610580444 5.0
-59.3856539 0.699019014835358 6.0
68.425117201 1.35868072509766 7.0
53.73666498 0.635447263717651 8.0
14.688452221 0.46616667509079 9.0
93.105131594 1.03201699256897 10.0
80.479180503 0.212681844830513 11.0
84.594984286 0.265997707843781 12.0
84.594984286 0.202582269906998 13.0
84.594984286 0.0973488017916679 14.0
0 0.239779263734818 15.0
0 0.162663027644157 16.0
0 1.40705370903015 17.0
61.252300221 1.77319967746735 18.0
-61.252300221 2.10518550872803 19.0
-61.252300221 0.636904180049896 20.0
-61.252300221 -0.0574212074279785 21.0
82.608969732 1.45417463779449 22.0
149.10293155 1.32580578327179 23.0
117.24613198 0.40519705414772 24.0
24.860427897 0.21948429942131 25.0
231.71190128 1.14747655391693 26.0
231.71190128 2.22030401229858 27.0
28.673638234 2.37693929672241 28.0
260.38553952 1.17960238456726 29.0
260.38553952 1.93092799186707 30.0
46.72995385 0.963322758674622 31.0
9.2091171008 0.550303041934967 32.0
21.356669511 1.58300304412842 33.0
21.356669511 0.84810334444046 34.0
21.356669511 0.0994155183434486 35.0
8.9501111083 0.0456627681851387 36.0
7.6066234296 0.38636189699173 37.0
0 0.0485272593796253 38.0
0 0.260570228099823 39.0
-12.226084926 0.502489745616913 40.0
-24.690952506 0.685933172702789 41.0
4.9114620766 0.481413394212723 42.0
-32.201696357 0.781811475753784 43.0
80.103408481 0.451392620801926 44.0
80.103408481 0.0946704745292664 45.0
6.371129099 0.438435018062592 46.0
2.3392827963 0.346606343984604 47.0
166.23839015 0.138210117816925 48.0
160.06189024 0.138055399060249 49.0
0 0.173745721578598 50.0
68.425117201 0.0763022154569626 51.0
93.105131594 0.0957197174429893 52.0
84.594984286 0.0801963657140732 53.0
84.594984286 0.0762624889612198 54.0
84.594984286 0.15841668844223 55.0
0 3.2228901386261 56.0
82.608969732 0.0573107562959194 57.0
231.71190128 0.758667528629303 58.0
260.38553952 0.807592809200287 59.0
21.356669511 -0.0167735628783703 60.0
0 0.11079853028059 61.0
149.27231714 0.799007713794708 0.0
149.27231714 2.04725027084351 1.0
148.00599744 1.89310193061829 2.0
143.21692698 3.68498516082764 3.0
0 5.08701848983765 4.0
0 5.17559862136841 5.0
-67.988215029 1.02885007858276 6.0
76.805015048 0.906356155872345 7.0
47.308956609 0.575598955154419 8.0
29.496058439 0.560709118843079 9.0
90.515006687 0.824019312858582 10.0
77.752885891 0.081503689289093 11.0
85.948620997 0.519890248775482 12.0
85.948620997 0.478984832763672 13.0
85.948620997 0.0646601170301437 14.0
0 0.153139382600784 15.0
0 0.19141611456871 16.0
0 1.04580640792847 17.0
52.276290298 1.2061927318573 18.0
-52.276290298 0.917367815971375 19.0
-52.276290298 0.552559435367584 20.0
-52.276290298 -0.0425689369440079 21.0
73.344005493 1.77649247646332 22.0
143.37915342 1.37708389759064 23.0
107.30612181 0.389683663845062 24.0
21.077507932 0.114127442240715 25.0
216.72315892 1.61881136894226 26.0
216.72315892 2.13370013237 27.0
26.527326507 1.02897918224335 28.0
243.25048542 0.484541296958923 29.0
243.25048542 1.58152842521667 30.0
46.803392776 1.01645040512085 31.0
9.2759993819 0.492188543081284 32.0
21.067715195 1.6119749546051 33.0
21.067715195 1.03492069244385 34.0
21.067715195 0.116222195327282 35.0
9.2528652688 0.0390720181167126 36.0
7.9514533614 0.451814144849777 37.0
0 0.0296778306365013 38.0
0 0.206785887479782 39.0
-10.831759261 0.536099970340729 40.0
-24.960898624 0.694745600223541 41.0
4.8927007 0.436473220586777 42.0
-32.375890504 0.59074854850769 43.0
93.928465337 0.617517471313477 44.0
93.928465337 0.0830790996551514 45.0
6.8926790287 0.648733079433441 46.0
2.5594977717 0.402958989143372 47.0
149.27231714 0.181254237890244 48.0
143.21692698 0.216258227825165 49.0
0 0.210762605071068 50.0
76.805015048 0.0477159544825554 51.0
90.515006687 0.165278360247612 52.0
85.948620997 0.0233123507350683 53.0
85.948620997 0.1894900649786 54.0
85.948620997 0.0741520524024963 55.0
0 0.821762382984161 56.0
73.344005493 0.0718544721603394 57.0
216.72315892 0.871963143348694 58.0
243.25048542 0.169961884617805 59.0
21.067715195 -0.0180313363671303 60.0
0 0.10345932841301 61.0
138.92465096 0.729943573474884 0.0
138.92465096 1.49343156814575 1.0
137.73729054 1.38341569900513 2.0
132.94822058 3.01134777069092 3.0
0 4.97264289855957 4.0
0 5.701416015625 5.0
-53.840341231 2.76946878433228 6.0
65.001865219 0.65961742401123 7.0
51.169368487 1.40469098091125 8.0
13.832496732 1.40715789794922 9.0
91.55162573 0.732432961463928 10.0
79.332174552 0.0357360988855362 11.0
88.913975335 0.346454441547394 12.0
88.913975335 0.368043124675751 13.0
88.913975335 0.0410362035036087 14.0
0 0.0527793802320957 15.0
0 0.212793692946434 16.0
0 0.937512099742889 17.0
52.053944892 1.05657184123993 18.0
-52.053944892 1.29479348659515 19.0
-52.053944892 0.550292670726776 20.0
-52.053944892 -0.095646932721138 21.0
72.752420004 1.28926455974579 22.0
143.59745453 1.25792849063873 23.0
107.59685714 0.377355694770813 24.0
24.984742511 0.136414423584938 25.0
216.34987454 1.19294571876526 26.0
216.34987454 1.72380423545837 27.0
28.998098196 1.65643680095673 28.0
245.34797273 0.850610196590424 29.0
245.34797273 1.87298274040222 30.0
46.942490812 1.13507676124573 31.0
9.2907863714 0.395226359367371 32.0
20.698475112 1.87400734424591 33.0
20.698475112 0.837187170982361 34.0
20.698475112 0.141824647784233 35.0
9.9167808981 0.0317502394318581 36.0
8.6265597351 0.333555370569229 37.0
0 0.00703350082039833 38.0
0 0.301185190677643 39.0
-12.401506234 0.48747056722641 40.0
-24.758160973 0.777743279933929 41.0
4.7527991178 0.326783835887909 42.0
-32.398027946 0.479276239871979 43.0
88.741114939 0.631365835666656 44.0
88.741114939 0.122835040092468 45.0
6.5407915011 0.572534918785095 46.0
2.6417364922 0.424743890762329 47.0
138.92465096 0.105841770768166 48.0
132.94822058 0.0632313564419746 49.0
0 0.533763945102692 50.0
65.001865219 0.379703998565674 51.0
91.55162573 0.103240743279457 52.0
88.913975335 0.0116471890360117 53.0
88.913975335 0.133590281009674 54.0
88.913975335 0.0456616841256618 55.0
0 1.71719789505005 56.0
72.752420004 0.292347222566605 57.0
216.34987454 0.340486168861389 58.0
245.34797273 0.572860479354858 59.0
20.698475112 0.000773642212152481 60.0
0 0.0943178981542587 61.0
151.51647527 0.808449327945709 0.0
151.51647527 1.35725951194763 1.0
150.09298717 1.30897235870361 2.0
144.62831663 2.74262499809265 3.0
12.659075962 4.91902828216553 4.0
12.659075962 6.2475414276123 5.0
-63.237121161 1.66443836688995 6.0
75.752593718 0.741462528705597 7.0
48.480479072 0.877911746501923 8.0
27.272114647 0.820325672626495 9.0
97.865756266 0.830624520778656 10.0
85.067034802 0.148554116487503 11.0
81.684111069 0.149522662162781 12.0
81.684111069 0.408810406923294 13.0
81.684111069 0.0645589604973793 14.0
0 0.166371554136276 15.0
0 0.214532494544983 16.0
0 0.832500219345093 17.0
62.718470112 1.27578675746918 18.0
-62.718470112 1.52209615707397 19.0
-62.718470112 0.334273427724838 20.0
-62.718470112 -0.0259216967970133 21.0
83.52737989 1.43301749229431 22.0
147.69597115 0.950520813465118 23.0
114.58312433 0.547446429729462 24.0
21.257976449 0.0992770567536354 25.0
231.22335104 0.864807367324829 26.0
231.22335104 1.66747832298279 27.0
27.381484914 1.75255644321442 28.0
258.60483595 0.784878075122833 29.0
258.60483595 1.90515458583832 30.0
47.084838893 0.966256082057953 31.0
9.2830387656 0.526606321334839 32.0
20.808909778 1.52293956279755 33.0
20.808909778 1.37808561325073 34.0
20.808909778 0.150110989809036 35.0
9.5399093385 0.0287633817642927 36.0
8.2893602287 0.434114038944244 37.0
0 0.0157189108431339 38.0
0 0.261100232601166 39.0
-11.032521161 0.69765043258667 40.0
-24.684753321 0.40869352221489 41.0
4.8708539998 0.390915811061859 42.0
-32.128741588 0.457089632749557 43.0
108.90787442 0.70535933971405 44.0
108.90787442 0.132129192352295 45.0
6.5074749842 0.553205251693726 46.0
1.4795352345 0.363647788763046 47.0
151.51647527 0.0315912142395973 48.0
144.62831663 0.149237155914307 49.0
12.659075962 0.343717694282532 50.0
75.752593718 0.286918044090271 51.0
97.865756266 0.166668802499771 52.0
81.684111069 0.0496694296598434 53.0
81.684111069 0.129718199372292 54.0
81.684111069 0.0731401219964027 55.0
0 2.37344360351562 56.0
83.52737989 0.183234751224518 57.0
231.22335104 0.369754582643509 58.0
258.60483595 0.877121686935425 59.0
20.808909778 -0.023712744936347 60.0
0 0.116232842206955 61.0
138.49207636 0.722691595554352 0.0
138.49207636 2.04559659957886 1.0
137.34227165 1.89227378368378 2.0
132.55316653 5.48132419586182 3.0
0 4.66554641723633 4.0
0 3.3798291683197 5.0
-55.37796877 -0.0245196297764778 6.0
68.006278157 0.688230872154236 7.0
55.400168528 0.850076198577881 8.0
12.606109629 0.946069598197937 9.0
93.818851033 1.02763724327087 10.0
81.676323387 0.721681773662567 11.0
81.958683621 1.20119738578796 12.0
81.958683621 0.690559923648834 13.0
81.958683621 0.0686648115515709 14.0
0 0.166809976100922 15.0
0 0.210181102156639 16.0
0 0.697215437889099 17.0
59.432843504 1.17156612873077 18.0
-59.432843504 1.42506170272827 19.0
-59.432843504 0.301140695810318 20.0
-59.432843504 -0.0437451153993607 21.0
81.173554138 1.93111765384674 22.0
145.50162006 1.63151943683624 23.0
112.55480839 0.627673208713531 24.0
23.758049386 0.115032717585564 25.0
226.6751742 1.46189248561859 26.0
226.6751742 2.00814938545227 27.0
25.190731975 2.31526327133179 28.0
251.86590617 1.24520528316498 29.0
251.86590617 2.09865307807922 30.0
46.762730767 0.917447447776794 31.0
9.3912391254 0.365594863891602 32.0
21.740710635 1.3733731508255 33.0
21.740710635 1.37558424472809 34.0
21.740710635 0.100485667586327 35.0
10.253326281 0.0267807487398386 36.0
8.9348480143 0.372093498706818 37.0
0 -0.00148907676339149 38.0
0 0.265440553426743 39.0
-11.83411089 0.442020535469055 40.0
-24.821008261 0.552665650844574 41.0
5.0128305311 0.362828284502029 42.0
-32.324985117 0.585735201835632 43.0
97.112751964 0.750759184360504 44.0
97.112751964 0.0755738914012909 45.0
6.599028913 0.57948112487793 46.0
1.9629272385 0.429703623056412 47.0
138.49207636 0.102408319711685 48.0
132.55316653 0.261490762233734 49.0
0 0.163642555475235 50.0
68.006278157 0.36655929684639 51.0
93.818851033 0.723397672176361 52.0
81.958683621 0.217086315155029 53.0
81.958683621 0.17636539041996 54.0
81.958683621 0.148832783102989 55.0
0 4.93761968612671 56.0
81.173554138 0.179524511098862 57.0
226.6751742 0.67250519990921 58.0
251.86590617 1.09880185127258 59.0
21.740710635 -0.0123735498636961 60.0
0 0.117730900645256 61.0
194.95709628 0.839015185832977 0.0
194.95709628 1.60424292087555 1.0
193.77694392 1.89929783344269 2.0
188.98783863 3.81078553199768 3.0
0 5.87001609802246 4.0
0 5.56201314926147 5.0
-76.873292208 1.12653148174286 6.0
86.007425327 0.726049721240997 7.0
55.647566402 0.796664714813232 8.0
30.359858924 0.851506471633911 9.0
77.174797487 1.00228655338287 10.0
65.27210667 0.183894038200378 11.0
78.245302285 0.787126660346985 12.0
78.245302285 0.418785393238068 13.0
78.245302285 0.0415104776620865 14.0
0 0.00151068903505802 15.0
0 0.183871433138847 16.0
0 1.09454262256622 17.0
57.127440499 0.786359071731567 18.0
-57.127440499 0.820721805095673 19.0
-57.127440499 0.52113664150238 20.0
-57.127440499 0.0684445202350616 21.0
79.119928718 1.74225234985352 22.0
152.07865974 1.36711609363556 23.0
119.13945439 0.660071134567261 24.0
23.293026375 0.148002684116364 25.0
231.19858846 1.20568561553955 26.0
231.19858846 1.83102250099182 27.0
28.45531396 1.27327477931976 28.0
259.65390242 0.615012764930725 29.0
259.65390242 1.75518894195557 30.0
46.200863332 1.1125568151474 31.0
9.3111260598 0.393078655004501 32.0
21.99248822 1.70552277565002 33.0
21.99248822 0.893701314926147 34.0
21.99248822 0.11972850561142 35.0
9.8273592441 0.0236692689359188 36.0
8.5676019226 0.361069411039352 37.0
0 0.0132829546928406 38.0
0 0.155229270458221 39.0
-11.253052102 0.378558456897736 40.0
-25.249861627 0.787373304367065 41.0
4.9224726828 0.363869428634644 42.0
-32.845459308 0.55977338552475 43.0
80.94641181 0.562333583831787 44.0
80.94641181 0.0857474803924561 45.0
7.1093020647 0.578064024448395 46.0
2.6107789281 0.367022395133972 47.0
194.95709628 0.0253824330866337 48.0
188.98783863 0.285326033830643 49.0
0 0.550575911998749 50.0
86.007425327 0.10148261487484 51.0
77.174797487 0.413649201393127 52.0
78.245302285 0.038505595177412 53.0
78.245302285 0.125705942511559 54.0
78.245302285 0.0608177930116653 55.0
0 3.59756827354431 56.0
79.119928718 0.0668482482433319 57.0
231.19858846 0.273395925760269 58.0
259.65390242 0.440733194351196 59.0
21.99248822 -0.0169051513075829 60.0
0 0.121677599847317 61.0
156.33965971 0.904168725013733 0.0
156.33965971 0.811529576778412 1.0
155.03757515 0.931659340858459 2.0
150.24846983 4.56089353561401 3.0
0 5.52802467346191 4.0
0 4.96432733535767 5.0
-68.143260153 0.474486589431763 6.0
78.792718668 0.421942591667175 7.0
47.199878891 0.750193059444427 8.0
31.592839777 0.852893590927124 9.0
81.187559837 0.999897718429565 10.0
68.38657592 0.184183523058891 11.0
81.698550115 0.129658386111259 12.0
81.698550115 0.136453181505203 13.0
81.698550115 0.131026178598404 14.0
0 0.158371269702911 15.0
0 0.142117008566856 16.0
0 1.58438968658447 17.0
60.309447101 1.23371231555939 18.0
-60.309447101 1.00445985794067 19.0
-60.309447101 0.582709610462189 20.0
-60.309447101 -0.0164170786738396 21.0
82.469865971 1.70813620090485 22.0
134.88491879 1.31578195095062 23.0
102.10999893 0.711316466331482 24.0
20.830693379 0.086676798760891 25.0
217.35478476 1.00467395782471 26.0
217.35478476 2.08363461494446 27.0
24.460638757 2.34511280059814 28.0
241.81542352 1.16737699508667 29.0
241.81542352 1.93812954425812 30.0
46.318984441 0.973815143108368 31.0
9.393629492 0.574949860572815 32.0
22.16041887 1.35770452022552 33.0
22.16041887 0.901052534580231 34.0
22.16041887 0.120805814862251 35.0
9.7540654822 0.0423928648233414 36.0
8.4614226778 0.377646505832672 37.0
0 0.0187331400811672 38.0
0 0.24143348634243 39.0
-11.716724948 0.379089385271072 40.0
-25.112344944 0.810479044914246 41.0
5.4401113873 0.379996001720428 42.0
-33.160386072 0.461165726184845 43.0
80.369485835 0.859931886196136 44.0
80.369485835 0.0876808539032936 45.0
6.9312814051 0.719017446041107 46.0
2.1506547956 0.342219293117523 47.0
156.33965971 0.145192742347717 48.0
150.24846983 0.046617217361927 49.0
0 0.53010630607605 50.0
78.792718668 0.169373840093613 51.0
81.187559837 0.318283706903458 52.0
81.698550115 0.0324533171951771 53.0
81.698550115 0.0527631342411041 54.0
81.698550115 0.0663935765624046 55.0
0 3.18813467025757 56.0
82.469865971 0.0496131479740143 57.0
217.35478476 0.667098641395569 58.0
241.81542352 0.573022961616516 59.0
22.16041887 0.00288157537579536 60.0
0 0.100506007671356 61.0
160.10390321 0.90221107006073 0.0
160.10390321 1.57350218296051 1.0
158.90015225 1.56957495212555 2.0
154.11104698 5.47198438644409 3.0
0 6.95037841796875 4.0
0 4.64686918258667 5.0
-55.534785228 4.56131649017334 6.0
66.769210108 1.11899793148041 7.0
50.22816608 0.739425659179688 8.0
16.541044028 0.545699775218964 9.0
85.955449228 0.971886873245239 10.0
73.65993637 0.0145252253860235 11.0
84.114899102 0.811692833900452 12.0
84.114899102 0.77503776550293 13.0
84.114899102 0.0840804800391197 14.0
0 0.190505832433701 15.0
0 0.200816482305527 16.0
0 1.409672498703 17.0
58.614957959 1.39662325382233 18.0
-58.614957959 1.18367183208466 19.0
-58.614957959 0.450478732585907 20.0
-58.614957959 0.0660034939646721 21.0
80.468225094 1.15613865852356 22.0
142.00626249 1.1462789773941 23.0
107.33175311 0.511053383350372 24.0
24.179684866 0.152224630117416 25.0
222.47448758 1.04956746101379 26.0
222.47448758 1.49747252464294 27.0
24.837423005 1.05275976657867 28.0
247.31191059 0.63200718164444 29.0
247.31191059 1.90850210189819 30.0
46.410157307 1.16786205768585 31.0
9.4605281614 0.423207759857178 32.0
21.853267135 0.985135197639465 33.0
21.853267135 1.24568152427673 34.0
21.853267135 0.126471504569054 35.0
9.0847342014 0.0403828658163548 36.0
7.7590396524 0.352912664413452 37.0
0 -0.00728319212794304 38.0
0 0.103149421513081 39.0
-12.463644596 0.697423815727234 40.0
-25.006821509 0.473403662443161 41.0
5.643040497 0.60883504152298 42.0
-33.522428292 0.740454077720642 43.0
92.309293778 0.53580629825592 44.0
92.309293778 0.127723127603531 45.0
6.882811094 0.594708204269409 46.0
2.4759778753 0.40636882185936 47.0
160.10390321 0.0964876636862755 48.0
154.11104698 0.116242647171021 49.0
0 0.887122690677643 50.0
66.769210108 0.223420158028603 51.0
85.955449228 0.10852474719286 52.0
84.114899102 0.0330899991095066 53.0
84.114899102 0.282858788967133 54.0
84.114899102 0.0894020274281502 55.0
0 3.63077282905579 56.0
80.468225094 0.0255225468426943 57.0
222.47448758 0.635907113552094 58.0
247.31191059 0.52319997549057 59.0
21.853267135 -0.00950809940695763 60.0
0 0.112595207989216 61.0
135.2137287 0.969121515750885 0.0
135.2137287 0.984527826309204 1.0
133.67708246 1.39650619029999 2.0
128.21241175 5.02573013305664 3.0
6.6844697341 7.11558532714844 4.0
6.6844697341 3.73983478546143 5.0
-68.790136565 2.19644927978516 6.0
80.678638915 0.635691702365875 7.0
38.653421485 0.922750115394592 8.0
42.02521743 0.767170965671539 9.0
85.108540191 1.01212430000305 10.0
71.85014423 0.0021266785915941 11.0
76.848865631 0.899752497673035 12.0
76.848865631 0.334231853485107 13.0
76.848865631 0.119678109884262 14.0
0 0.0503564104437828 15.0
0 0.207428961992264 16.0
0 1.7821581363678 17.0
51.376134658 1.47600018978119 18.0
-51.376134658 2.08234715461731 19.0
-51.376134658 0.315468192100525 20.0
-51.376134658 -0.0161349847912788 21.0
73.719317827 1.28948140144348 22.0
135.87222029 0.881137371063232 23.0
98.586722371 0.887819528579712 24.0
22.936456796 0.0790199860930443 25.0
209.59153812 1.34636926651001 26.0
209.59153812 2.47206377983093 27.0
25.35232745 2.7374746799469 28.0
234.94386557 2.01425337791443 29.0
234.94386557 2.09423494338989 30.0
47.087910882 1.17512238025665 31.0
9.3380252933 0.381937175989151 32.0
22.343183169 1.50628709793091 33.0
22.343183169 0.928760826587677 34.0
22.343183169 0.172650456428528 35.0
9.6957203347 0.027596166357398 36.0
8.271867357 0.306581258773804 37.0
0 0.0252809599041939 38.0
0 0.130291566252708 39.0
-11.816690281 0.739216387271881 40.0
-23.075138877 0.621529817581177 41.0
4.6872192426 0.44269648194313 42.0
-30.098407494 0.636187374591827 43.0
96.570983814 0.670121490955353 44.0
96.570983814 0.14018839597702 45.0
7.3765508947 0.614016711711884 46.0
2.2448458513 0.405775874853134 47.0
135.2137287 0.1593376994133 48.0
128.21241175 0.498770892620087 49.0
6.6844697341 0.265093564987183 50.0
80.678638915 0.327183157205582 51.0
85.108540191 0.215892091393471 52.0
76.848865631 0.0130040356889367 53.0
76.848865631 0.0748454332351685 54.0
76.848865631 0.0437830686569214 55.0
0 4.86692476272583 56.0
73.719317827 0.0183027237653732 57.0
209.59153812 0.451187163591385 58.0
234.94386557 1.35100710391998 59.0
22.343183169 -0.0199584439396858 60.0
0 0.0999768152832985 61.0
189.99328023 1.02515888214111 0.0
189.99328023 1.03375005722046 1.0
188.57083898 0.927441000938416 2.0
187.4743416 5.13261890411377 3.0
0 7.29489612579346 4.0
0 5.49592065811157 5.0
-61.699270471 2.08538198471069 6.0
74.605997377 0.469059407711029 7.0
63.556565012 1.54263663291931 8.0
11.049432366 1.65540683269501 9.0
97.787686625 1.07060289382935 10.0
84.982496284 0.485839575529099 11.0
82.159417939 0.956022977828979 12.0
82.159417939 0.764377653598785 13.0
82.159417939 0.106370657682419 14.0
6.1047708807 0.251016855239868 15.0
6.1047708807 0.171709164977074 16.0
0 1.59967863559723 17.0
53.821277883 1.08075094223022 18.0
-53.821277883 0.895040333271027 19.0
-53.821277883 0.620010435581207 20.0
-53.821277883 -0.0234127193689346 21.0
76.335746028 1.54951047897339 22.0
143.28016938 1.59002578258514 23.0
107.82755161 0.359259456396103 24.0
22.494005745 0.0943541079759598 25.0
219.61591541 1.12412738800049 26.0
219.61591541 1.76888334751129 27.0
25.165578716 2.16885852813721 28.0
244.78149412 0.845878899097443 29.0
244.78149412 1.58659541606903 30.0
46.937262502 0.977363288402557 31.0
9.4620408032 0.300052165985107 32.0
22.514468145 1.60940957069397 33.0
22.514468145 0.837188839912415 34.0
22.514468145 0.107878834009171 35.0
9.5554369454 0.025514118373394 36.0
8.3511412287 0.354765385389328 37.0
0 0.0377007536590099 38.0
0 0.145019829273224 39.0
-11.138434692 0.648852944374084 40.0
-23.203856655 0.504208326339722 41.0
5.0885114952 0.421325951814651 42.0
-30.563978564 0.54096394777298 43.0
108.9239996 0.773543477058411 44.0
108.9239996 0.068126954138279 45.0
7.1468962388 0.504692494869232 46.0
2.9078329432 0.366316080093384 47.0
189.99328023 0.112699903547764 48.0
187.4743416 0.0990186631679535 49.0
0 0.440009027719498 50.0
74.605997377 0.0831995084881783 51.0
97.787686625 0.343527406454086 52.0
82.159417939 0.163596749305725 53.0
82.159417939 0.205484464764595 54.0
82.159417939 0.164052680134773 55.0
0 1.93725383281708 56.0
76.335746028 -0.0616424158215523 57.0
219.61591541 0.367776542901993 58.0
244.78149412 0.367133527994156 59.0
22.514468145 -0.00689293630421162 60.0
0 0.102007329463959 61.0
167.0605791 0.911963641643524 0.0
167.0605791 0.974955499172211 1.0
165.54337202 1.74950957298279 2.0
160.75426684 6.19389486312866 3.0
0 8.78692531585693 4.0
0 3.96269750595093 5.0
-55.255090868 2.57928991317749 6.0
67.833105565 0.75499027967453 7.0
68.405370857 0.931102752685547 8.0
-0.57226529178 1.24237263202667 9.0
84.80708522 1.01699650287628 10.0
72.139134138 0.294546455144882 11.0
84.969039246 0.128594741225243 12.0
84.969039246 0.840574026107788 13.0
84.969039246 0.103320762515068 14.0
0 0.228344738483429 15.0
0 0.198391258716583 16.0
0 2.01916885375977 17.0
57.561203375 1.28624880313873 18.0
-57.561203375 0.755096733570099 19.0
-57.561203375 0.306060165166855 20.0
-57.561203375 0.0167047455906868 21.0
76.429722979 1.24357974529266 22.0
142.43317231 1.04616451263428 23.0
104.99538466 0.721593141555786 24.0
25.30698063 0.092455267906189 25.0
218.86289529 1.58004856109619 26.0
218.86289529 1.30717170238495 27.0
26.15081074 0.680510222911835 28.0
245.01370603 0.545470118522644 29.0
245.01370603 1.880774974823 30.0
46.175949645 1.22147357463837 31.0
9.2263855146 0.33743742108345 32.0
18.868519604 1.87992656230927 33.0
18.868519604 0.984077930450439 34.0
18.868519604 0.106589555740356 35.0
10.997976336 0.0348261557519436 36.0
9.673112647 0.41293129324913 37.0
0 0.0178156904876232 38.0
0 0.0338536314666271 39.0
-12.672578544 0.616822242736816 40.0
-24.347403182 0.227558463811874 41.0
5.2084914773 0.616504430770874 42.0
-31.817831648 1.1105729341507 43.0
88.193837523 0.693899989128113 44.0
88.193837523 0.133654966950417 45.0
6.0221896883 0.517849504947662 46.0
1.8998439852 0.391751945018768 47.0
167.0605791 0.150999695062637 48.0
160.75426684 0.658299267292023 49.0
0 0.27812922000885 50.0
67.833105565 0.140237063169479 51.0
84.80708522 0.371067196130753 52.0
84.969039246 0.0297567620873451 53.0
84.969039246 0.0801469162106514 54.0
84.969039246 0.298135459423065 55.0
0 2.60262823104858 56.0
76.429722979 0.0954782217741013 57.0
218.86289529 0.428917199373245 58.0
245.01370603 0.435447007417679 59.0
18.868519604 0.0147708803415298 60.0
0 0.127027779817581 61.0
181.88034508 0.869521737098694 0.0
181.88034508 0.964857578277588 1.0
180.63305338 1.36151897907257 2.0
178.59839568 4.41276264190674 3.0
0 9.21860218048096 4.0
0 7.07366371154785 5.0
-84.049678304 1.69195353984833 6.0
95.646021226 0.366238296031952 7.0
66.982887981 1.34406518936157 8.0
28.663133245 1.77552175521851 9.0
89.796244793 0.839561700820923 10.0
77.157904458 0.0324049517512321 11.0
83.265967162 0.783465385437012 12.0
83.265967162 0.878297865390778 13.0
83.265967162 0.0694530606269836 14.0
1.0647807124 0.237947076559067 15.0
1.0647807124 0.166764557361603 16.0
0 1.27882254123688 17.0
58.452273265 1.18124377727509 18.0
-58.452273265 1.28594958782196 19.0
-58.452273265 0.628856897354126 20.0
-58.452273265 -0.0400914289057255 21.0
80.455900704 1.22220027446747 22.0
139.56878574 1.30734300613403 23.0
103.73816167 0.414872586727142 24.0
23.197193842 0.0899237096309662 25.0
220.02468644 0.952449858188629 26.0
220.02468644 1.3266419172287 27.0
23.240455725 1.61376297473907 28.0
243.26514217 1.27669239044189 29.0
243.26514217 1.56324863433838 30.0
46.678497567 0.861381649971008 31.0
9.42628753 0.408207356929779 32.0
22.003627439 1.51274871826172 33.0
22.003627439 1.24873328208923 34.0
22.003627439 0.108477905392647 35.0
8.8185505482 0.0375267826020718 36.0
7.4923357542 0.312784880399704 37.0
0 0.0497953966259956 38.0
0 0.0789642035961151 39.0
-10.779540552 0.451966047286987 40.0
-24.281422004 0.648734569549561 41.0
5.6667201598 0.358048796653748 42.0
-32.404817507 0.289598882198334 43.0
86.770433484 0.700306117534637 44.0
86.770433484 0.0439991429448128 45.0
6.8298842157 0.697502315044403 46.0
2.4861049723 0.426466315984726 47.0
181.88034508 0.149105578660965 48.0
178.59839568 0.272600680589676 49.0
0 1.46617758274078 50.0
95.646021226 0.0346402153372765 51.0
89.796244793 0.110356256365776 52.0
83.265967162 0.0431777127087116 53.0
83.265967162 0.21828219294548 54.0
83.265967162 0.254090785980225 55.0
0 0.52817440032959 56.0
80.455900704 0.0786302834749222 57.0
220.02468644 0.258661359548569 58.0
243.26514217 0.96652215719223 59.0
22.003627439 -0.0194185227155685 60.0
0 0.0846998393535614 61.0
160.56330383 0.792542517185211 0.0
160.56330383 0.700278878211975 1.0
159.23357231 1.16386747360229 2.0
153.76890155 5.95832395553589 3.0
17.081212985 5.30199670791626 4.0
17.081212985 3.63199377059937 5.0
-60.915266927 0.0419046431779861 6.0
74.608253129 0.659838080406189 7.0
47.077071457 1.29395127296448 8.0
27.531181671 1.28525531291962 9.0
88.062038989 0.869499742984772 10.0
74.743653644 0.0682205632328987 11.0
86.11511294 0.661486923694611 12.0
86.11511294 0.42466402053833 13.0
86.11511294 0.0842838734388351 14.0
0 0.133941531181335 15.0
0 0.190352499485016 16.0
0 1.65273344516754 17.0
61.344927678 1.06909954547882 18.0
-61.344927678 0.866348266601562 19.0
-61.344927678 0.312223225831985 20.0
-61.344927678 0.0153690427541733 21.0
82.172472093 2.15785503387451 22.0
136.53959414 1.14821529388428 23.0
99.793421972 1.06768798828125 24.0
20.342849671 0.0696635842323303 25.0
218.71206623 1.0622181892395 26.0
218.71206623 1.17669582366943 27.0
23.29168928 1.21135866641998 28.0
242.00375551 0.579604685306549 29.0
242.00375551 1.38081765174866 30.0
47.040780968 1.05845475196838 31.0
9.3603606917 0.377239376306534 32.0
20.827544416 2.02296137809753 33.0
20.827544416 1.09136378765106 34.0
20.827544416 0.082307443022728 35.0
9.9192272091 0.0234470181167126 36.0
8.6185247047 0.39347243309021 37.0
0 0.08787602186203 38.0
0 0.308687031269073 39.0
-11.123852935 0.72800213098526 40.0
-25.00452777 0.437309116125107 41.0
5.7352558518 0.534098207950592 42.0
-33.083467466 0.259250849485397 43.0
75.872293052 0.784538447856903 44.0
75.872293052 0.0853320732712746 45.0
7.177612927 0.744121670722961 46.0
2.3285680252 0.402486205101013 47.0
160.56330383 0.0594466775655746 48.0
153.76890155 0.0948417484760284 49.0
17.081212985 0.18437571823597 50.0
74.608253129 0.162419080734253 51.0
88.062038989 0.0557605102658272 52.0
86.11511294 0.0130409970879555 53.0
86.11511294 0.122153721749783 54.0
86.11511294 0.169239699840546 55.0
0 2.0889949798584 56.0
82.172472093 0.0785520970821381 57.0
218.71206623 0.428474873304367 58.0
242.00375551 0.625409722328186 59.0
20.827544416 -0.0335844829678535 60.0
0 0.102315321564674 61.0
187.08048411 1.13407444953918 0.0
187.08048411 1.40495336055756 1.0
185.92618625 1.78903317451477 2.0
181.13708096 4.02423810958862 3.0
0 7.15128040313721 4.0
0 3.85721826553345 5.0
-60.806174521 0.448765009641647 6.0
70.557380302 0.502437055110931 7.0
60.125554127 1.30475258827209 8.0
10.431826176 1.33382666110992 9.0
99.549815114 1.18914413452148 10.0
85.493442095 0.251126617193222 11.0
90.55173188 1.26622664928436 12.0
90.55173188 0.548817455768585 13.0
90.55173188 0.0343488529324532 14.0
0 0.136680603027344 15.0
0 0.171010956168175 16.0
0 0.956958770751953 17.0
58.393142926 1.37094974517822 18.0
-58.393142926 1.95440483093262 19.0
-58.393142926 0.265382170677185 20.0
-58.393142926 0.0682768747210503 21.0
79.69528605 1.03673577308655 22.0
151.64484147 0.543890833854675 23.0
114.95934475 0.643518209457397 24.0
22.651424832 0.104739665985107 25.0
231.34012752 0.636284112930298 26.0
231.34012752 1.07957291603088 27.0
25.337751137 1.00822269916534 28.0
256.67787865 0.423607170581818 29.0
256.67787865 1.71811854839325 30.0
46.516059089 1.28395247459412 31.0
9.396192187 0.60151606798172 32.0
21.302143124 1.03776860237122 33.0
21.302143124 1.21186709403992 34.0
21.302143124 0.11752463132143 35.0
10.522068337 0.0210094805806875 36.0
9.2630460344 0.420502334833145 37.0
0 0.0885858535766602 38.0
0 0.225928410887718 39.0
-12.888053902 0.506438434123993 40.0
-26.148310959 0.592852592468262 41.0
4.7908624609 0.678872406482697 42.0
-33.624406264 0.788549542427063 43.0
76.924014643 0.38346004486084 44.0
76.924014643 0.0916874781250954 45.0
7.3168902547 0.496001571416855 46.0
2.7878601779 0.432921320199966 47.0
187.08048411 0.271353363990784 48.0
181.13708096 0.19748666882515 49.0
0 0.203279465436935 50.0
70.557380302 0.167300969362259 51.0
99.549815114 0.247316926717758 52.0
90.55173188 0.0634355098009109 53.0
90.55173188 0.295853585004807 54.0
90.55173188 0.0777724459767342 55.0
0 0.319767743349075 56.0
79.69528605 -0.0412099249660969 57.0
231.34012752 0.31178292632103 58.0
256.67787865 0.128638550639153 59.0
21.302143124 0.0194212235510349 60.0
0 0.091634713113308 61.0
149.16294493 1.32592177391052 0.0
149.16294493 0.71733021736145 1.0
147.94859882 1.28720843791962 2.0
143.15952887 5.54885482788086 3.0
0 3.63712501525879 4.0
0 3.57894659042358 5.0
-53.430098044 1.22331166267395 6.0
65.951004955 0.646819949150085 7.0
61.987771516 1.34701800346375 8.0
3.9632334391 1.52630889415741 9.0
93.861854672 1.3962277173996 10.0
80.797523361 0.224727541208267 11.0
86.303344966 0.761923432350159 12.0
86.303344966 0.309416532516479 13.0
86.303344966 0.102665819227695 14.0
0 0.0588673017919064 15.0
0 0.164456427097321 16.0
0 1.28140640258789 17.0
54.082393176 1.42218887805939 18.0
-54.082393176 1.47249269485474 19.0
-54.082393176 0.852987706661224 20.0
-54.082393176 0.110273122787476 21.0
74.695631539 1.29673540592194 22.0
158.55181038 1.0396454334259 23.0
120.21986267 0.422124415636063 24.0
26.085189148 0.117215856909752 25.0
233.24744192 1.49876999855042 26.0
233.24744192 2.22576355934143 27.0
29.148904091 1.95637273788452 28.0
262.39634601 0.422399997711182 29.0
262.39634601 1.86309170722961 30.0
46.505961411 1.20620119571686 31.0
9.4569572238 0.601905286312103 32.0
20.613238364 0.9931880235672 33.0
20.613238364 1.19480741024017 34.0
20.613238364 0.102893106639385 35.0
9.4944945642 0.0220928117632866 36.0
8.1698238306 0.408407598733902 37.0
0 -0.000368870794773102 38.0
0 0.186019450426102 39.0
-11.962083145 0.60649311542511 40.0
-24.811748408 0.505904853343964 41.0
5.2825849343 0.608083307743073 42.0
-32.956606764 0.700700163841248 43.0
86.501071062 0.649720907211304 44.0
86.501071062 0.0942669659852982 45.0
6.6369276149 0.442423135042191 46.0
2.5669588692 0.384042322635651 47.0
149.16294493 0.180408358573914 48.0
143.15952887 0.0991281569004059 49.0
0 0.125966250896454 50.0
65.951004955 0.299528121948242 51.0
93.861854672 0.260759681463242 52.0
86.303344966 0.0456883311271667 53.0
86.303344966 0.0545867681503296 54.0
86.303344966 0.030075840651989 55.0
0 2.84719085693359 56.0
74.695631539 -0.0748617425560951 57.0
233.24744192 0.273785471916199 58.0
262.39634601 0.3278968334198 59.0
20.613238364 0.0024139191955328 60.0
0 0.0756614729762077 61.0
164.65362966 0.824695348739624 0.0
164.65362966 1.86048626899719 1.0
163.45812203 2.33468723297119 2.0
158.66905208 5.33033418655396 3.0
0 8.85829544067383 4.0
0 3.04333090782166 5.0
-74.573481347 0.282923847436905 6.0
86.505382853 0.706338286399841 7.0
64.904996873 0.96759706735611 8.0
21.600385979 0.772911906242371 9.0
92.546381493 1.17278838157654 10.0
78.294410645 0.605278849601746 11.0
79.845668855 0.463663697242737 12.0
79.845668855 0.414685398340225 13.0
79.845668855 0.141172587871552 14.0
0 0.18684646487236 15.0
0 0.123781561851501 16.0
0 0.864139318466187 17.0
61.103675624 1.636798620224 18.0
-61.103675624 1.82777380943298 19.0
-61.103675624 0.473361432552338 20.0
-61.103675624 0.21953871846199 21.0
82.613384621 1.48994433879852 22.0
142.6693436 1.15259718894958 23.0
110.59181358 0.404535830020905 24.0
22.773375903 0.229768827557564 25.0
225.28272823 1.054936170578 26.0
225.28272823 1.75022184848785 27.0
25.66327141 2.46411967277527 28.0
250.94599964 1.58933997154236 29.0
250.94599964 2.12214922904968 30.0
46.731595525 1.08764624595642 31.0
9.3883650885 0.538493573665619 32.0
21.509708997 0.882515490055084 33.0
21.509708997 1.00327956676483 34.0
21.509708997 0.179448172450066 35.0
9.7333347757 0.0458367764949799 36.0
8.562815685 0.396449595689774 37.0
0 -0.00150772742927074 38.0
0 0.0381467230618 39.0
-11.183988785 0.729027450084686 40.0
-23.342466188 0.406270503997803 41.0
5.1435701596 0.484257102012634 42.0
-30.737762492 0.527503848075867 43.0
97.01200101 0.740313649177551 44.0
97.01200101 0.0716598480939865 45.0
7.1737741183 0.422925114631653 46.0
2.6117550864 0.309950679540634 47.0
164.65362966 0.0301867723464966 48.0
158.66905208 0.439815193414688 49.0
0 1.22004497051239 50.0
86.505382853 0.168054312467575 51.0
92.546381493 0.65261310338974 52.0
79.845668855 0.126796245574951 53.0
79.845668855 0.161998882889748 54.0
79.845668855 0.0388002321124077 55.0
0 5.40267467498779 56.0
82.613384621 0.0360437631607056 57.0
225.28272823 0.225192219018936 58.0
250.94599964 0.960947275161743 59.0
21.509708997 -0.00975837372243404 60.0
0 0.122481875121593 61.0
175.93421517 1.01848435401917 0.0
175.93421517 0.860501825809479 1.0
174.67059987 0.871656596660614 2.0
169.88149459 5.09014368057251 3.0
0 4.85501956939697 4.0
0 3.41097593307495 5.0
-74.339690062 0.299133807420731 6.0
84.392361348 0.428876101970673 7.0
43.993094168 1.04167795181274 8.0
40.39926718 0.971093237400055 9.0
82.064246905 1.1969518661499 10.0
69.222090834 0.260889530181885 11.0
76.035591361 0.395262956619263 12.0
76.035591361 0.493977278470993 13.0
76.035591361 0.142189800739288 14.0
0 0.0958771705627441 15.0
0 0.118580564856529 16.0
0 0.922622442245483 17.0
53.959275376 1.17022514343262 18.0
-53.959275376 1.59375238418579 19.0
-53.959275376 0.495265185832977 20.0
-53.959275376 0.0114330463111401 21.0
75.238330837 1.64696788787842 22.0
155.86672277 1.50437653064728 23.0
118.83276114 0.460527926683426 24.0
25.66910359 0.104802995920181 25.0
231.1050536 0.948979735374451 26.0
231.1050536 1.32574772834778 27.0
25.70013885 1.27950060367584 28.0
256.80519245 0.888066530227661 29.0
256.80519245 1.49251759052277 30.0
46.328429699 0.991070449352264 31.0
9.4344196493 0.369860231876373 32.0
21.279055461 1.29874289035797 33.0
21.279055461 0.656447410583496 34.0
21.279055461 0.107470102608204 35.0
9.5355631783 0.0315682515501976 36.0
8.3565681692 0.334029853343964 37.0
0 0.0284563675522804 38.0
0 0.117326386272907 39.0
-11.305865795 0.907974481582642 40.0
-25.309538823 0.580901086330414 41.0
5.4026466342 0.400688081979752 42.0
-33.516866149 0.503548085689545 43.0
80.117883602 0.665388524532318 44.0
80.117883602 0.100707463920116 45.0
6.9650849828 0.495201259851456 46.0
2.8237776236 0.355762541294098 47.0
175.93421517 0.123870097100735 48.0
169.88149459 0.0149026289582253 49.0
0 0.0677484720945358 50.0
84.392361348 0.223484724760056 51.0
82.064246905 0.377692341804504 52.0
76.035591361 0.064502015709877 53.0
76.035591361 0.057286199182272 54.0
76.035591361 0.0552757531404495 55.0
0 6.44996547698975 56.0
75.238330837 -0.0213032253086567 57.0
231.1050536 0.419934958219528 58.0
256.80519245 0.272658616304398 59.0
21.279055461 0.00203008763492107 60.0
0 0.106738284230232 61.0
141.91611971 0.909484684467316 0.0
141.91611971 1.61751651763916 1.0
140.79446146 1.29654145240784 2.0
136.00535632 5.1955099105835 3.0
0 7.67741966247559 4.0
0 5.19094610214233 5.0
-65.378796522 3.74875926971436 6.0
74.688217257 0.502933084964752 7.0
62.039364427 0.904977381229401 8.0
12.64885283 0.941255569458008 9.0
79.273514938 0.92030531167984 10.0
67.401342954 0.0475896820425987 11.0
75.160788995 0.243006139993668 12.0
75.160788995 0.251842677593231 13.0
75.160788995 0.152224719524384 14.0
0 0.134858936071396 15.0
0 0.184170693159103 16.0
0 1.3148341178894 17.0
59.987299829 1.49168193340302 18.0
-59.987299829 1.4051525592804 19.0
-59.987299829 0.345134139060974 20.0
-59.987299829 0.0548813082277775 21.0
80.647611927 1.97921979427338 22.0
142.01675949 1.54462814331055 23.0
107.00326747 0.615537583827972 24.0
22.588631621 0.112891755998135 25.0
222.66437142 1.14485478401184 26.0
222.66437142 1.5725257396698 27.0
24.19488623 1.85386216640472 28.0
246.85925765 0.886565566062927 29.0
246.85925765 1.73886442184448 30.0
46.508724614 1.18508744239807 31.0
9.3322698972 0.544590890407562 32.0
20.660312098 1.00329911708832 33.0
20.660312098 1.17213976383209 34.0
20.660312098 0.0781862735748291 35.0
8.4053021385 0.0237153396010399 36.0
7.2465470867 0.388897448778152 37.0
0 0.0194379948079586 38.0
0 0.124766707420349 39.0
-11.784229802 0.65553879737854 40.0
-24.340024161 0.612427294254303 41.0
5.0496022022 0.51995313167572 42.0
-32.073295664 0.514214634895325 43.0
91.513577477 0.639267921447754 44.0
91.513577477 0.0823633521795273 45.0
6.9776744324 0.573822200298309 46.0
2.3526936644 0.354926466941833 47.0
141.91611971 0.138902321457863 48.0
136.00535632 0.0754435285925865 49.0
0 0.512912511825562 50.0
74.688217257 0.127807766199112 51.0
79.273514938 0.10303595662117 52.0
75.160788995 0.00836762972176075 53.0
75.160788995 0.0966800525784492 54.0
75.160788995 0.0968662649393082 55.0
0 7.41329336166382 56.0
80.647611927 -0.00219076266512275 57.0
222.66437142 0.558283388614655 58.0
246.85925765 0.325287610292435 59.0
20.660312098 -0.00446489080786705 60.0
0 0.083035334944725 61.0
187.00380857 0.943860828876495 0.0
187.00380857 0.82089376449585 1.0
185.6123635 1.05744218826294 2.0
180.82325843 3.76528859138489 3.0
0 6.91913890838623 4.0
0 6.18502330780029 5.0
-65.373360708 0.616014122962952 6.0
75.764315395 0.641056656837463 7.0
63.995356639 1.11542737483978 8.0
11.768958757 1.18557906150818 9.0
92.172801382 1.16363537311554 10.0
78.792383989 0.331425696611404 11.0
83.456171593 0.725791335105896 12.0
83.456171593 0.530689895153046 13.0
83.456171593 0.0516225248575211 14.0
0 0.159117951989174 15.0
0 0.219791799783707 16.0
0 1.21160435676575 17.0
57.542065831 1.32589769363403 18.0
-57.542065831 1.10685992240906 19.0
-57.542065831 0.330182999372482 20.0
-57.542065831 0.00987933576107025 21.0
79.772496604 2.02407717704773 22.0
148.30627883 2.13942360877991 23.0
112.81954164 0.366489768028259 24.0
24.02737484 0.166987419128418 25.0
228.07877543 0.725068986415863 26.0
228.07877543 1.56702637672424 27.0
26.634275048 1.61599564552307 28.0
254.71305048 1.00641083717346 29.0
254.71305048 1.80245637893677 30.0
46.797681538 0.968118190765381 31.0
9.2318507097 0.377535313367844 32.0
22.230430773 1.59569871425629 33.0
22.230430773 1.23948168754578 34.0
22.230430773 0.136051341891289 35.0
9.1564039409 0.0384638234972954 36.0
7.8190261294 0.385418057441711 37.0
0 0.0452830530703068 38.0
0 0.0975200086832047 39.0
-13.214798275 0.67118638753891 40.0
-26.093338024 0.423230737447739 41.0
5.6372763001 0.617430210113525 42.0
-34.525452858 0.699330747127533 43.0
107.9888621 0.611500263214111 44.0
107.9888621 0.0755286440253258 45.0
6.6113245268 0.612308919429779 46.0
2.8163239538 0.401928961277008 47.0
187.00380857 0.0832860618829727 48.0
180.82325843 0.331388205289841 49.0
0 0.778513610363007 50.0
75.764315395 0.38943013548851 51.0
92.172801382 0.523362100124359 52.0
83.456171593 0.110674910247326 53.0
83.456171593 0.0816229283809662 54.0
83.456171593 0.120070926845074 55.0
0 3.09071636199951 56.0
79.772496604 0.0057477904483676 57.0
228.07877543 0.503438413143158 58.0
254.71305048 0.641328871250153 59.0
22.230430773 -0.0193255860358477 60.0
0 0.0883837416768074 61.0
198.93890907 0.979153335094452 0.0
198.93890907 1.45784413814545 1.0
198.05389261 2.23513507843018 2.0
193.26478748 2.93001747131348 3.0
0 4.71960020065308 4.0
0 3.90840101242065 5.0
-75.179198951 1.79726278781891 6.0
86.054729641 0.30045148730278 7.0
59.36825757 1.14796280860901 8.0
26.686472071 1.29669392108917 9.0
90.196491291 1.20913207530975 10.0
76.341238123 0.468146055936813 11.0
83.519580076 1.33365523815155 12.0
83.519580076 0.954235315322876 13.0
83.519580076 0.0406236238777637 14.0
0 0.193550437688828 15.0
0 0.158503726124763 16.0
0 1.38730072975159 17.0
58.588226298 0.621069610118866 18.0
-58.588226298 0.422121196985245 19.0
-58.588226298 1.2822939157486 20.0
-58.588226298 0.0959305018186569 21.0
81.21363966 2.98385787010193 22.0
149.71948946 2.06996870040894 23.0
114.54953798 0.466364562511444 24.0
22.713505905 0.0853463485836983 25.0
230.93312912 1.57193803787231 26.0
230.93312912 1.60642504692078 27.0
22.511663052 1.67162942886353 28.0
253.44479217 0.547137260437012 29.0
253.44479217 1.631920337677 30.0
45.974826581 1.08834612369537 31.0
9.3564412115 0.298906177282333 32.0
22.625413362 0.645958423614502 33.0
22.625413362 1.1114673614502 34.0
22.625413362 0.10010389983654 35.0
8.8331020039 0.0292845983058214 36.0
7.6010469659 0.380636960268021 37.0
0 0.0172569043934345 38.0
0 0.145623236894608 39.0
-13.239391014 0.675944626331329 40.0
-25.583506541 0.889942824840546 41.0
5.6175526284 0.413301676511765 42.0
-33.379892146 0.290603309869766 43.0
96.939697994 0.697478592395782 44.0
96.939697994 0.0436060950160027 45.0
5.984575305 0.559679508209229 46.0
2.2845688024 0.394280642271042 47.0
198.93890907 0.264528065919876 48.0
193.26478748 0.297999620437622 49.0
0 0.23004138469696 50.0
86.054729641 0.0945698320865631 51.0
90.196491291 0.769595146179199 52.0
83.519580076 0.0642166137695312 53.0
83.519580076 0.216480389237404 54.0
83.519580076 0.20965076982975 55.0
0 2.05796265602112 56.0
81.21363966 0.0108865164220333 57.0
230.93312912 0.57779985666275 58.0
253.44479217 0.559026479721069 59.0
22.625413362 0.0253149010241032 60.0
0 0.0791018605232239 61.0
133.27577072 1.07198965549469 0.0
133.27577072 0.942000091075897 1.0
132.12088317 1.39537811279297 2.0
127.33177783 3.64760804176331 3.0
0 7.37359714508057 4.0
0 2.79257559776306 5.0
-55.075522902 0.259450644254684 6.0
66.646889395 1.0656019449234 7.0
60.454143483 1.69999492168427 8.0
6.1927459125 1.8606538772583 9.0
87.443626562 1.07366108894348 10.0
74.289656103 0.391485512256622 11.0
86.297071852 0.576671779155731 12.0
86.297071852 0.928900003433228 13.0
86.297071852 0.00454979855567217 14.0
0 0.123442374169827 15.0
0 0.171921193599701 16.0
0 0.991921007633209 17.0
62.228366859 0.977122485637665 18.0
-62.228366859 0.71674120426178 19.0
-62.228366859 0.465417325496674 20.0
-62.228366859 0.10195979475975 21.0
84.060065407 1.46080029010773 22.0
144.92870082 0.930639207363129 23.0
115.94569301 0.385427832603455 24.0
24.728433469 0.266953974962234 25.0
228.98876623 0.778865396976471 26.0
228.98876623 1.83156621456146 27.0
25.562233158 2.14106392860413 28.0
254.55099939 1.011514544487 29.0
254.55099939 1.93234312534332 30.0
46.251183023 0.997263789176941 31.0
9.4770734018 0.431725442409515 32.0
21.831698547 0.669947683811188 33.0
21.831698547 0.903949081897736 34.0
21.831698547 0.0622736997902393 35.0
9.0773646284 0.0239274352788925 36.0
7.7280338503 0.384445279836655 37.0
0 0.0157804600894451 38.0
0 0.0459315441548824 39.0
-10.945776008 0.611246347427368 40.0
-23.738715003 0.741474509239197 41.0
5.3881126758 0.402774035930634 42.0
-31.452017525 0.490662515163422 43.0
75.923007938 0.725297510623932 44.0
75.923007938 0.0965234339237213 45.0
6.754683352 0.503202855587006 46.0
2.8348320459 0.351115971803665 47.0
133.27577072 0.620721697807312 48.0
127.33177783 0.250775665044785 49.0
0 0.77946925163269 50.0
66.646889395 0.329007983207703 51.0
87.443626562 0.466405153274536 52.0
86.297071852 0.0636215135455132 53.0
86.297071852 0.143710106611252 54.0
86.297071852 0.280877947807312 55.0
0 0.423353761434555 56.0
84.060065407 -0.0356411635875702 57.0
228.98876623 0.330513924360275 58.0
254.55099939 0.744134664535522 59.0
21.831698547 0.0225010309368372 60.0
0 0.137546136975288 61.0
133.55227972 0.72459864616394 0.0
133.55227972 1.39143145084381 1.0
132.20148219 1.65076267719269 2.0
127.41239865 2.68567514419556 3.0
0 5.21022415161133 4.0
0 6.26975965499878 5.0
-49.162528412 0.974367916584015 6.0
59.711153935 0.707631945610046 7.0
45.406588254 1.27073574066162 8.0
14.304565681 1.33167135715485 9.0
73.866279456 0.725525438785553 10.0
62.009685321 0.0355861894786358 11.0
77.426600848 0.576971888542175 12.0
77.426600848 0.475761711597443 13.0
77.426600848 0.0762139484286308 14.0
0 0.00268222391605377 15.0
0 0.205282270908356 16.0
0 1.27108895778656 17.0
51.558006383 0.380965143442154 18.0
-51.558006383 0.899583518505096 19.0
-51.558006383 0.527772426605225 20.0
-51.558006383 -0.0430373512208462 21.0
71.514404516 2.31065654754639 22.0
149.48224607 2.28857731819153 23.0
113.59169727 0.596936523914337 24.0
23.575956751 0.113325998187065 25.0
220.99665058 1.14590239524841 26.0
220.99665058 1.32172322273254 27.0
25.608957827 1.77647817134857 28.0
246.60560841 1.72142279148102 29.0
246.60560841 2.16017532348633 30.0
47.13372898 0.940283179283142 31.0
9.3035830806 0.490346163511276 32.0
19.956398133 1.1524965763092 33.0
19.956398133 1.15601241588593 34.0
19.956398133 0.141290292143822 35.0
8.9389394561 0.0429577231407166 36.0
7.6847333474 0.272549748420715 37.0
0 0.0511997193098068 38.0
0 0.290617346763611 39.0
-11.178962276 1.16353225708008 40.0
-24.027932673 0.37933674454689 41.0
5.3150968453 0.697086930274963 42.0
-31.558067677 0.437218964099884 43.0
87.456157169 0.659734845161438 44.0
87.456157169 0.0875911340117455 45.0
6.0833221662 0.58732533454895 46.0
2.4044379614 0.435955256223679 47.0
133.55227972 0.175344839692116 48.0
127.41239865 0.243970438838005 49.0
0 0.49126523733139 50.0
59.711153935 0.279964655637741 51.0
73.866279456 0.104899317026138 52.0
77.426600848 0.0385223627090454 53.0
77.426600848 0.283790618181229 54.0
77.426600848 0.00215221755206585 55.0
0 5.24474143981934 56.0
71.514404516 0.0635357201099396 57.0
220.99665058 0.239200234413147 58.0
246.60560841 1.78994953632355 59.0
19.956398133 0.00501252710819244 60.0
0 0.0444793403148651 61.0
167.30548768 1.22574317455292 0.0
167.30548768 0.776448369026184 1.0
165.74756617 1.75807642936707 2.0
160.95846089 3.92079949378967 3.0
0 9.26605319976807 4.0
0 6.42453193664551 5.0
-53.206510837 1.03261613845825 6.0
61.316805634 1.35650682449341 7.0
39.107697549 0.291098088026047 8.0
22.209108086 0.170515418052673 9.0
82.776241602 1.21347141265869 10.0
70.240216967 0.0633357241749763 11.0
77.098495323 0.110120892524719 12.0
77.098495323 0.0585241913795471 13.0
77.098495323 0.145171746611595 14.0
0 0.0755535513162613 15.0
0 0.12667840719223 16.0
0 1.06109082698822 17.0
62.545809875 0.884688794612885 18.0
-62.545809875 0.667946517467499 19.0
-62.545809875 0.435899913311005 20.0
-62.545809875 -0.0720576196908951 21.0
82.20522665 3.20113730430603 22.0
135.76292262 2.9602906703949 23.0
103.35691881 0.57914000749588 24.0
23.883306935 0.0756841078400612 25.0
217.96814927 1.83367490768433 26.0
217.96814927 3.14598679542542 27.0
23.08751029 3.85508584976196 28.0
241.05565956 1.55876815319061 29.0
241.05565956 1.95514345169067 30.0
45.976795729 0.976077973842621 31.0
9.1599674239 0.501287877559662 32.0
19.659416775 2.94315981864929 33.0
19.659416775 0.97889232635498 34.0
19.659416775 0.186472520232201 35.0
8.1644278897 0.0336069539189339 36.0
6.8141966323 0.382867246866226 37.0
0 0.0195577666163445 38.0
0 0.0836569368839264 39.0
-11.057704719 0.44610208272934 40.0
-25.221078239 0.76703804731369 41.0
5.1578802842 0.478007376194 42.0
-32.679699669 0.617655754089355 43.0
76.077782033 0.66285103559494 44.0
76.077782033 0.0803478211164474 45.0
6.2308248236 0.611993849277496 46.0
2.773177145 0.257545858621597 47.0
167.30548768 0.0549202375113964 48.0
160.95846089 0.00636648898944259 49.0
0 0.606302618980408 50.0
61.316805634 0.129266709089279 51.0
82.776241602 0.134344533085823 52.0
77.098495323 0.00861174706369638 53.0
77.098495323 0.0613975003361702 54.0
77.098495323 0.08469158411026 55.0
0 7.11303043365479 56.0
82.20522665 0.210297033190727 57.0
217.96814927 1.0527435541153 58.0
241.05565956 0.610578954219818 59.0
19.659416775 0.0331680476665497 60.0
0 0.118492320179939 61.0
161.05795939 1.07161808013916 0.0
161.05795939 1.96837162971497 1.0
159.5044617 1.98484015464783 2.0
154.03979108 3.03506541252136 3.0
22.078014352 5.36688423156738 4.0
22.078014352 6.27740144729614 5.0
-58.152847417 7.34493923187256 6.0
66.833840575 0.360318779945374 7.0
36.184523905 0.453371852636337 8.0
30.649316669 0.52751213312149 9.0
97.62667247 1.06113517284393 10.0
84.155612057 0.00760483276098967 11.0
84.733603953 0.803469181060791 12.0
84.733603953 0.100633665919304 13.0
84.733603953 0.106482177972794 14.0
0 0.087666854262352 15.0
0 0.199462339282036 16.0
0 0.679885447025299 17.0
50.995583168 1.40347921848297 18.0
-50.995583168 1.24279737472534 19.0
-50.995583168 1.1132572889328 20.0
-50.995583168 -0.0796605348587036 21.0
69.106424556 1.68752455711365 22.0
145.45363305 2.57984900474548 23.0
115.31878393 0.179752245545387 24.0
20.869672764 0.377876818180084 25.0
214.56005761 1.07116889953613 26.0
214.56005761 1.67424201965332 27.0
22.228134928 1.25550639629364 28.0
236.78819254 0.680566966533661 29.0
236.78819254 1.64583659172058 30.0
46.198139108 0.924075245857239 31.0
9.2720509576 0.35387122631073 32.0
18.110841388 1.80901074409485 33.0
18.110841388 1.27612221240997 34.0
18.110841388 0.0948680862784386 35.0
10.148178879 0.0279202684760094 36.0
8.9240245461 0.347552299499512 37.0
0 0.0273618437349796 38.0
0 0.311546415090561 39.0
-12.734777597 0.169537305831909 40.0
-25.272070303 0.586042463779449 41.0
4.843893683 0.512451946735382 42.0
-32.997796338 0.44935268163681 43.0
85.469156643 0.754434049129486 44.0
85.469156643 0.113954961299896 45.0
7.0036159212 0.558485984802246 46.0
2.209973424 0.361469358205795 47.0
161.05795939 0.128688454627991 48.0
154.03979108 0.225845843553543 49.0
22.078014352 0.325884401798248 50.0
66.833840575 0.279820710420609 51.0
97.62667247 0.103390634059906 52.0
84.733603953 0.0162695273756981 53.0
84.733603953 0.121044866740704 54.0
84.733603953 0.0845259875059128 55.0
0 5.5994553565979 56.0
69.106424556 0.0656747221946716 57.0
214.56005761 0.654313802719116 58.0
236.78819254 0.814258337020874 59.0
18.110841388 0.00288476049900055 60.0
0 0.0956558138132095 61.0
190.8672771 0.689196944236755 0.0
190.8672771 1.96094071865082 1.0
189.76098848 1.38577353954315 2.0
184.29631788 5.0446605682373 3.0
3.5615309947 7.58207702636719 4.0
3.5615309947 5.18455123901367 5.0
-65.042357111 0.429368674755096 6.0
75.324706003 0.583326458930969 7.0
45.723795461 0.84812319278717 8.0
29.600910542 0.951536893844604 9.0
92.633634205 0.961383104324341 10.0
79.110093569 0.432062655687332 11.0
85.396177132 0.487302422523499 12.0
85.396177132 0.0876929461956024 13.0
85.396177132 0.0752599239349365 14.0
0 0.15484382212162 15.0
0 0.200209140777588 16.0
0 1.0988701581955 17.0
55.145164786 0.948229551315308 18.0
-55.145164786 0.660322666168213 19.0
-55.145164786 0.622218132019043 20.0
-55.145164786 -0.0254656914621592 21.0
77.71338079 1.98609042167664 22.0
144.46232149 1.83081150054932 23.0
111.14564034 0.484128713607788 24.0
23.065148313 0.108904480934143 25.0
222.17570228 1.45719182491302 26.0
222.17570228 1.74764502048492 27.0
28.424340821 1.28325045108795 28.0
250.6000431 0.620050013065338 29.0
250.6000431 1.80281734466553 30.0
46.513756764 1.09060096740723 31.0
9.4052650917 0.588297724723816 32.0
22.568216004 1.77971506118774 33.0
22.568216004 1.01254498958588 34.0
22.568216004 0.12414313107729 35.0
9.6211166457 0.0281281974166632 36.0
8.2274732434 0.352873802185059 37.0
0 0.00326899066567421 38.0
0 0.124085307121277 39.0
-11.832749584 0.679014563560486 40.0
-24.746576267 0.476376801729202 41.0
5.3821631285 0.617152452468872 42.0
-32.781740258 0.706102013587952 43.0
88.032937865 0.68851900100708 44.0
88.032937865 0.109241269528866 45.0
7.5513568499 0.704877257347107 46.0
2.8145699452 0.421972811222076 47.0
190.8672771 0.0998421087861061 48.0
184.29631788 0.203063294291496 49.0
3.5615309947 0.218917950987816 50.0
75.324706003 0.323250621557236 51.0
92.633634205 0.533659517765045 52.0
85.396177132 0.106073297560215 53.0
85.396177132 0.0938044786453247 54.0
85.396177132 0.0877901986241341 55.0
0 1.26831638813019 56.0
77.71338079 0.0419815480709076 57.0
222.17570228 0.98418140411377 58.0
250.6000431 0.478456795215607 59.0
22.568216004 -0.00839872006326914 60.0
0 0.117406509816647 61.0
161.74844261 0.851931512355804 0.0
161.74844261 2.22759056091309 1.0
160.50672483 1.98863768577576 2.0
155.71761958 5.34797954559326 3.0
0 5.26486253738403 4.0
0 3.01373386383057 5.0
-59.950178693 1.53249847888947 6.0
71.750230443 0.452355563640594 7.0
48.220093284 1.00905680656433 8.0
23.530137159 1.00332760810852 9.0
90.631985904 0.939231216907501 10.0
76.464001592 0.207767471671104 11.0
88.216302207 0.779142260551453 12.0
88.216302207 0.50480705499649 13.0
88.216302207 0.11477056145668 14.0
0 0.252301424741745 15.0
0 0.141950964927673 16.0
0 2.01038789749146 17.0
59.618240814 1.33584201335907 18.0
-59.618240814 1.03743016719818 19.0
-59.618240814 0.326605021953583 20.0
-59.618240814 0.236183136701584 21.0
80.854387945 1.22808086872101 22.0
145.64522971 0.788993299007416 23.0
112.2636749 0.429794430732727 24.0
20.593765735 0.297311782836914 25.0
226.49961765 0.746600091457367 26.0
226.49961765 1.97461831569672 27.0
26.011416951 2.34953451156616 28.0
252.5110346 1.23149728775024 29.0
252.5110346 1.81890320777893 30.0
46.679914917 1.15554928779602 31.0
9.2434772317 0.416747748851776 32.0
21.23614713 1.27849960327148 33.0
21.23614713 1.06007218360901 34.0
21.23614713 0.124167039990425 35.0
9.8859374736 0.0225074719637632 36.0
8.6040942825 0.400393545627594 37.0
0 0.0273255482316017 38.0
0 0.104762181639671 39.0
-12.026372073 0.752952218055725 40.0
-24.785151777 0.545372545719147 41.0
5.4499668663 0.513249576091766 42.0
-32.821746322 0.679102778434753 43.0
89.36608736 0.645014047622681 44.0
89.36608736 0.0905745700001717 45.0
7.308578314 0.453947305679321 46.0
2.4792286838 0.348519057035446 47.0
161.74844261 0.0993135124444962 48.0
155.71761958 0.419683039188385 49.0
0 0.175728008151054 50.0
71.750230443 0.187094360589981 51.0
90.631985904 0.0487899966537952 52.0
88.216302207 0.0295435879379511 53.0
88.216302207 0.173654526472092 54.0
88.216302207 0.0909774601459503 55.0
0 0.488434046506882 56.0
80.854387945 -0.0601505264639854 57.0
226.49961765 0.467630118131638 58.0
252.5110346 0.86751252412796 59.0
21.23614713 -0.0185113921761513 60.0
0 0.103829450905323 61.0
159.4625599 1.06700170040131 0.0
159.4625599 1.2294362783432 1.0
158.38282764 0.898896455764771 2.0
153.59372253 3.26982307434082 3.0
0 6.49976348876953 4.0
0 4.88762903213501 5.0
-58.128479564 0.503046631813049 6.0
68.584585704 0.539519548416138 7.0
53.914386307 1.2765029668808 8.0
14.670199397 1.29431521892548 9.0
86.417542918 1.17014169692993 10.0
73.248129137 0.12700192630291 11.0
79.158572755 0.167538285255432 12.0
79.158572755 0.158141061663628 13.0
79.158572755 0.0826630517840385 14.0
0 0.0506662279367447 15.0
0 0.169032141566277 16.0
0 1.31583249568939 17.0
61.348735949 1.51669335365295 18.0
-61.348735949 1.24286198616028 19.0
-61.348735949 1.38243126869202 20.0
-61.348735949 -0.0811099335551262 21.0
82.960070242 1.45145499706268 22.0
148.19036428 0.732650816440582 23.0
109.95483582 0.754071652889252 24.0
25.008328437 0.14802286028862 25.0
231.15043452 0.841192483901978 26.0
231.15043452 1.28591680526733 27.0
25.372721104 1.57265448570251 28.0
256.52315562 0.60929149389267 29.0
256.52315562 1.83550882339478 30.0
46.262503623 1.08558821678162 31.0
9.3104191869 0.550309896469116 32.0
21.611334293 1.83257389068604 33.0
21.611334293 0.900573253631592 34.0
21.611334293 0.132135957479477 35.0
9.5740935448 0.0363218858838081 36.0
8.3030793026 0.384256571531296 37.0
0 0.0320270769298077 38.0
0 0.159547433257103 39.0
-10.9009422 0.563563585281372 40.0
-23.66955116 0.629762947559357 41.0
5.3461402126 0.56255316734314 42.0
-31.353595014 0.692979514598846 43.0
98.220817907 0.743132412433624 44.0
98.220817907 0.0896980613470078 45.0
6.8602844389 0.674961924552917 46.0
2.7124755567 0.408707410097122 47.0
159.4625599 0.16949138045311 48.0
153.59372253 0.0601471848785877 49.0
0 0.23107448220253 50.0
68.584585704 0.239489480853081 51.0
86.417542918 0.0871228128671646 52.0
79.158572755 0.0295050367712975 53.0
79.158572755 0.101559571921825 54.0
79.158572755 0.093755692243576 55.0
0 4.7127833366394 56.0
82.960070242 0.124938301742077 57.0
231.15043452 0.295647323131561 58.0
256.52315562 0.606746971607208 59.0
21.611334293 -0.0140890795737505 60.0
0 0.0877156853675842 61.0
157.83262283 0.723164439201355 0.0
157.83262283 1.20636260509491 1.0
156.73943912 1.67131972312927 2.0
151.95033379 4.19232034683228 3.0
0 4.4726243019104 4.0
0 3.1956958770752 5.0
-48.488767256 0.337958455085754 6.0
58.249776742 0.911917567253113 7.0
60.903947329 1.43422710895538 8.0
-2.6541705862 1.59653770923615 9.0
84.361225862 0.751395344734192 10.0
72.319685791 0.043609157204628 11.0
75.467528718 0.763648927211761 12.0
75.467528718 0.107287324965 13.0
75.467528718 0.1572635024786 14.0
0 0.275897890329361 15.0
0 0.173510149121284 16.0
0 1.66496610641479 17.0
54.303331413 1.03080940246582 18.0
-54.303331413 0.699999868869781 19.0
-54.303331413 0.446349173784256 20.0
-54.303331413 -0.0062575601041317 21.0
74.593024822 1.59794914722443 22.0
155.84345573 1.43452632427216 23.0
117.64534474 0.255271524190903 24.0
24.082002281 0.272069007158279 25.0
230.43648055 0.746564984321594 26.0
230.43648055 1.58076775074005 27.0
28.453014624 1.96905112266541 28.0
258.88949518 1.30664992332458 29.0
258.88949518 2.20416760444641 30.0
46.372836231 1.22296965122223 31.0
9.497829098 0.541154026985168 32.0
20.289693409 1.45894289016724 33.0
20.289693409 0.810920596122742 34.0
20.289693409 0.0928287655115128 35.0
9.1329277441 0.0273426417261362 36.0
7.8674225462 0.361564576625824 37.0
0 0.00573841854929924 38.0
0 0.287533760070801 39.0
-11.209850006 0.573233544826508 40.0
-24.143926025 0.506546378135681 41.0
4.8044276655 0.522156357765198 42.0
-31.642107821 0.36114290356636 43.0
78.490529121 0.818745255470276 44.0
78.490529121 0.0761281996965408 45.0
7.2944715451 0.577675640583038 46.0
2.4651630167 0.373693406581879 47.0
157.83262283 0.0832094699144363 48.0
151.95033379 0.516117572784424 49.0
0 0.135467395186424 50.0
58.249776742 0.319101572036743 51.0
84.361225862 0.17452235519886 52.0
75.467528718 0.0353393740952015 53.0
75.467528718 0.112995848059654 54.0
75.467528718 0.0353878811001778 55.0
0 5.48187875747681 56.0
74.593024822 0.11478291451931 57.0
230.43648055 0.555361807346344 58.0
258.88949518 0.855026423931122 59.0
20.289693409 0.00192937813699245 60.0
0 0.0968105792999268 61.0
150.14365773 0.987328410148621 0.0
150.14365773 0.777155935764313 1.0
148.64845679 1.38375401496887 2.0
143.85935146 4.2054295539856 3.0
0 5.20081758499146 4.0
0 3.76227450370789 5.0
-62.01000027 2.6701455116272 6.0
72.400928632 0.588169157505035 7.0
59.833253188 0.659011363983154 8.0
12.567675445 0.917600274085999 9.0
72.391100415 0.906359314918518 10.0
59.685135174 0.0147639438509941 11.0
77.088806191 0.369699478149414 12.0
77.088806191 0.19767153263092 13.0
77.088806191 0.0592724159359932 14.0
0 0.12458910048008 15.0
0 0.184620827436447 16.0
0 1.00090289115906 17.0
62.540987001 0.835657715797424 18.0
-62.540987001 1.01639187335968 19.0
-62.540987001 0.801585555076599 20.0
-62.540987001 -0.0714277923107147 21.0
83.990486315 2.63085436820984 22.0
137.16944543 2.49286103248596 23.0
101.01047398 0.275089174509048 24.0
22.400186501 0.142389565706253 25.0
221.15993174 0.982751488685608 26.0
221.15993174 1.91449308395386 27.0
26.200564475 1.394047498703 28.0
247.36049622 0.949979305267334 29.0
247.36049622 1.37870001792908 30.0
46.84179847 1.11719918251038 31.0
9.3173033824 0.355804145336151 32.0
21.449499314 1.63987171649933 33.0
21.449499314 0.950821399688721 34.0
21.449499314 0.0731787532567978 35.0
9.0342973999 0.0324213653802872 36.0
7.714724995 0.384069979190826 37.0
0 0.0410283431410789 38.0
0 0.0849469006061554 39.0
-11.748483895 0.767171263694763 40.0
-26.560481674 0.415912210941315 41.0
4.9424375697 0.512188136577606 42.0
-33.819962159 0.725173354148865 43.0
78.73652031 0.516389966011047 44.0
78.73652031 0.120925441384315 45.0
6.8382321501 0.479266077280045 46.0
2.7203063259 0.342499524354935 47.0
150.14365773 0.297973334789276 48.0
143.85935146 0.352494060993195 49.0
0 0.427061706781387 50.0
72.400928632 0.104791432619095 51.0
72.391100415 0.20387390255928 52.0
77.088806191 0.0255076084285975 53.0
77.088806191 0.112662568688393 54.0
77.088806191 0.0759538263082504 55.0
0 3.38096046447754 56.0
83.990486315 0.117733865976334 57.0
221.15993174 0.181533247232437 58.0
247.36049622 0.253348797559738 59.0
21.449499314 0.023120004683733 60.0
0 0.09493438154459 61.0
172.62134752 0.813189387321472 0.0
172.62134752 1.49768912792206 1.0
171.56113356 2.06395530700684 2.0
166.77202834 6.24463510513306 3.0
0 4.23896932601929 4.0
0 4.52924013137817 5.0
-73.133413199 0.457376569509506 6.0
81.54599972 1.40915250778198 7.0
62.643378215 0.404481142759323 8.0
18.902621506 0.361876130104065 9.0
82.539490087 0.860004663467407 10.0
70.00055089 0.0845433995127678 11.0
81.213832661 0.687082231044769 12.0
81.213832661 0.542753636837006 13.0
81.213832661 0.0746564418077469 14.0
0 0.1852787733078 15.0
0 0.198587670922279 16.0
0 1.07125377655029 17.0
57.806803591 1.65494751930237 18.0
-57.806803591 1.94034719467163 19.0
-57.806803591 0.349065661430359 20.0
-57.806803591 0.130161166191101 21.0
78.322356213 1.28615021705627 22.0
154.54240245 0.792489588260651 23.0
116.26784037 0.843281686306 24.0
16.239558804 0.0992827191948891 25.0
232.86475866 0.728772222995758 26.0
232.86475866 1.61491096019745 27.0
22.941659887 2.22717785835266 28.0
255.80641855 1.20939719676971 29.0
255.80641855 2.12066864967346 30.0
47.031295114 1.18082857131958 31.0
9.2422126361 0.590341806411743 32.0
20.515552622 0.852224886417389 33.0
20.515552622 1.21813178062439 34.0
20.515552622 0.117057144641876 35.0
9.2078586133 0.0283538121730089 36.0
8.0323707042 0.37660551071167 37.0
0 0.00587134435772896 38.0
0 0.25909960269928 39.0
-10.543627998 0.491652011871338 40.0
-22.509205968 0.782673239707947 41.0
5.0439454008 0.482096284627914 42.0
-29.703990206 0.811789453029633 43.0
99.299124382 0.756628751754761 44.0
99.299124382 0.0794521197676659 45.0
7.1714387509 0.662941992282867 46.0
2.2626205623 0.422738432884216 47.0
172.62134752 0.190121337771416 48.0
166.77202834 0.378191649913788 49.0
0 0.346054583787918 50.0
81.54599972 0.0889019668102264 51.0
82.539490087 0.0761663690209389 52.0
81.213832661 0.046022854745388 53.0
81.213832661 0.129686713218689 54.0
81.213832661 0.169607773423195 55.0
0 3.07225823402405 56.0
78.322356213 0.0670654624700546 57.0
232.86475866 0.172369092702866 58.0
255.80641855 0.601523697376251 59.0
20.515552622 -0.0039070975035429 60.0
0 0.0809505879878998 61.0
144.25614352 0.94814521074295 0.0
144.25614352 1.85985338687897 1.0
142.91533818 2.00989079475403 2.0
138.12623241 5.85707950592041 3.0
0 6.58772897720337 4.0
0 5.35553359985352 5.0
-56.397430226 4.17776870727539 6.0
66.030805048 0.467075109481812 7.0
54.438266007 0.584386825561523 8.0
11.592539041 0.801828861236572 9.0
93.616666644 0.931921184062958 10.0
81.233723314 0.074065312743187 11.0
85.529676044 0.605405926704407 12.0
85.529676044 0.617387533187866 13.0
85.529676044 0.0586739480495453 14.0
0 0.154116898775101 15.0
0 0.211692705750465 16.0
0 1.17153322696686 17.0
61.708187415 1.4590448141098 18.0
-61.708187415 1.31836271286011 19.0
-61.708187415 0.60041731595993 20.0
-61.708187415 -0.0394443646073341 21.0
83.249383924 1.66305553913116 22.0
143.45112441 1.23402643203735 23.0
111.92704378 0.468885004520416 24.0
23.955302169 0.125984191894531 25.0
226.70050833 1.47626912593842 26.0
226.70050833 2.10844588279724 27.0
23.981442424 2.36181688308716 28.0
250.68195075 1.14143395423889 29.0
250.68195075 1.78532552719116 30.0
46.705934601 1.09999251365662 31.0
9.4631610509 0.376857936382294 32.0
21.541196508 1.31092691421509 33.0
21.541196508 1.01520490646362 34.0
21.541196508 0.0812582746148109 35.0
9.6068323196 0.02048066444695 36.0
8.2575018291 0.414620399475098 37.0
0 0.0313890166580677 38.0
0 0.228652536869049 39.0
-10.590832697 0.375356495380402 40.0
-22.598139961 0.797067821025848 41.0
4.7052051101 0.426432758569717 42.0
-29.889738866 0.704243898391724 43.0
85.203388842 0.501592874526978 44.0
85.203388842 0.108302898705006 45.0
7.5202322728 0.483235895633698 46.0
2.4819164244 0.42286428809166 47.0
144.25614352 0.104970723390579 48.0
138.12623241 0.239652678370476 49.0
0 0.395235508680344 50.0
66.030805048 0.118921503424644 51.0
93.616666644 0.105560548603535 52.0
85.529676044 0.0852398127317429 53.0
85.529676044 0.107965379953384 54.0
85.529676044 0.17586675286293 55.0
0 3.64042139053345 56.0
83.249383924 0.0629168599843979 57.0
226.70050833 0.66698831319809 58.0
250.68195075 0.711898505687714 59.0
21.541196508 0.00380693376064301 60.0
0 0.0956846922636032 61.0
185.50946859 0.695237278938293 0.0
185.50946859 1.86199593544006 1.0
184.55153388 2.49581241607666 2.0
179.76242864 3.73163318634033 3.0
0 3.44579315185547 4.0
0 6.03647232055664 5.0
-52.978709544 0.546611905097961 6.0
63.188984759 1.02897238731384 7.0
52.91181697 0.892256140708923 8.0
10.277167789 1.10743021965027 9.0
78.825238719 0.80346691608429 10.0
66.867921984 0.303405672311783 11.0
71.366206727 0.480972021818161 12.0
71.366206727 0.109322488307953 13.0
71.366206727 0.04997294023633 14.0
0 0.00666831806302071 15.0
0 0.188007652759552 16.0
0 1.09521019458771 17.0
50.917756727 1.87462985515594 18.0
-50.917756727 2.81994247436523 19.0
-50.917756727 0.324942260980606 20.0
-50.917756727 0.0259743295609951 21.0
72.372813388 2.60992884635925 22.0
153.31401766 1.83736765384674 23.0
119.64240193 0.35904198884964 24.0
24.114540503 0.139282271265984 25.0
225.68683105 1.72044098377228 26.0
225.68683105 1.49045312404633 27.0
23.271398702 1.18187499046326 28.0
248.95822975 0.317972391843796 29.0
248.95822975 1.96861410140991 30.0
47.509957449 1.17207002639771 31.0
9.4423844192 0.616530299186707 32.0
21.455056661 1.24723923206329 33.0
21.455056661 1.15536034107208 34.0
21.455056661 0.0998825430870056 35.0
9.8261366104 0.0259277727454901 36.0
8.5963702886 0.396241664886475 37.0
0 0.010781317949295 38.0
0 0.0579868219792843 39.0
-12.131439735 0.712408363819122 40.0
-25.460501136 0.490806728601456 41.0
4.6858662486 0.455689698457718 42.0
-32.93173625 0.20260052382946 43.0
89.012480878 0.868250846862793 44.0
89.012480878 0.0680765137076378 45.0
6.2043201579 0.717870473861694 46.0
2.5055429896 0.394755721092224 47.0
185.50946859 0.21083091199398 48.0
179.76242864 0.673185288906097 49.0
0 0.216017320752144 50.0
63.188984759 0.183776840567589 51.0
78.825238719 0.536699593067169 52.0
71.366206727 0.0413896627724171 53.0
71.366206727 0.101766839623451 54.0
71.366206727 0.0841755867004395 55.0
0 7.23207712173462 56.0
72.372813388 0.0814327076077461 57.0
225.68683105 0.71598619222641 58.0
248.95822975 0.605663001537323 59.0
21.455056661 0.00197960808873177 60.0
0 0.0860675945878029 61.0
155.48959034 0.949722170829773 0.0
155.48959034 1.59751844406128 1.0
154.07048748 1.84808945655823 2.0
149.28138228 4.09425497055054 3.0
0 8.65810203552246 4.0
0 5.20760202407837 5.0
-58.58517971 0.470996916294098 6.0
69.449031242 0.719248652458191 7.0
55.000229388 1.02537679672241 8.0
14.448801854 1.10982406139374 9.0
83.625631664 1.21845042705536 10.0
70.039984637 0.362508684396744 11.0
84.723939169 0.408062875270844 12.0
84.723939169 0.324761837720871 13.0
84.723939169 0.0457837209105492 14.0
0 0.0879696011543274 15.0
0 0.196405500173569 16.0
0 0.745917856693268 17.0
52.740039311 1.32048416137695 18.0
-52.740039311 1.40618777275085 19.0
-52.740039311 0.445136904716492 20.0
-52.740039311 -0.000527869910001755 21.0
74.20997158 2.0764148235321 22.0
149.03716666 1.59044861793518 23.0
113.52132986 0.35360324382782 24.0
22.94721838 0.0879698991775513 25.0
223.24713824 1.46974062919617 26.0
223.24713824 1.64238095283508 27.0
25.297643026 1.62901782989502 28.0
248.54478126 0.949707269668579 29.0
248.54478126 1.86577701568604 30.0
46.439587138 1.15721559524536 31.0
9.2641047681 0.512881517410278 32.0
21.469932269 1.36591005325317 33.0
21.469932269 1.10584998130798 34.0
21.469932269 0.0934811234474182 35.0
9.6231699146 0.0212522093206644 36.0
8.3279894663 0.385056376457214 37.0
0 0.0805384963750839 38.0
0 0.254807710647583 39.0
-12.175147628 0.409832954406738 40.0
-24.285248154 0.630153357982635 41.0
5.1733107502 0.339656233787537 42.0
-31.836681148 0.428229987621307 43.0
102.49222243 0.700760304927826 44.0
102.49222243 0.103686198592186 45.0
7.0712013143 0.63144850730896 46.0
2.4290164875 0.379914581775665 47.0
155.48959034 0.105979666113853 48.0
149.28138228 0.189806997776031 49.0
0 0.319208145141602 50.0
69.449031242 0.276791512966156 51.0
83.625631664 0.338982820510864 52.0
84.723939169 0.0538782030344009 53.0
84.723939169 0.143026441335678 54.0
84.723939169 0.0553115643560886 55.0
0 3.04310369491577 56.0
74.20997158 0.108262218534946 57.0
223.24713824 0.665384411811829 58.0
248.54478126 0.667069017887115 59.0
21.469932269 0.00620612502098083 60.0
0 0.0728220492601395 61.0
120.60586326 0.764551937580109 0.0
120.60586326 0.67436295747757 1.0
119.61170855 1.39468657970428 2.0
114.14703778 7.01169204711914 3.0
2.3643269733 2.9114716053009 4.0
2.3643269733 3.24904656410217 5.0
-72.518018327 0.00658516585826874 6.0
84.560804519 0.895364463329315 7.0
56.697762754 0.965374112129211 8.0
27.863041765 0.92511522769928 9.0
96.710137035 1.16670119762421 10.0
85.575306017 0.556443095207214 11.0
86.620738739 0.72026914358139 12.0
86.620738739 0.174566954374313 13.0
86.620738739 0.119536183774471 14.0
0 0.259117156267166 15.0
0 0.232681259512901 16.0
0 1.33055460453033 17.0
62.960374559 0.612307548522949 18.0
-62.960374559 0.52311909198761 19.0
-62.960374559 0.393229275941849 20.0
-62.960374559 0.447477847337723 21.0
85.377236716 0.451980590820312 22.0
135.09614312 0.609497129917145 23.0
101.095075 0.642285406589508 24.0
20.090166743 0.0745798945426941 25.0
220.47337983 0.829106450080872 26.0
220.47337983 1.42620313167572 27.0
30.222547463 1.1812903881073 28.0
250.6959273 1.16885781288147 29.0
250.6959273 2.04450583457947 30.0
46.835945757 1.18414199352264 31.0
9.1860282176 0.36787810921669 32.0
22.416862157 1.04938209056854 33.0
22.416862157 0.837179958820343 34.0
22.416862157 0.0779774188995361 35.0
9.7059070136 0.035501804202795 36.0
8.2802526734 0.368919610977173 37.0
0 0.0174052156507969 38.0
0 0.06465744972229 39.0
-13.29094846 1.12307441234589 40.0
-26.747697339 0.325028270483017 41.0
5.673418305 0.591369271278381 42.0
-34.97737166 0.771076858043671 43.0
72.884012268 0.608557403087616 44.0
72.884012268 0.0994693115353584 45.0
7.0690351449 0.322719126939774 46.0
2.2996401232 0.284683734178543 47.0
120.60586326 0.132183283567429 48.0
114.14703778 0.481847614049911 49.0
2.3643269733 -0.0307158399373293 50.0
84.560804519 0.235765933990479 51.0
96.710137035 0.761503398418427 52.0
86.620738739 0.107919774949551 53.0
86.620738739 0.00302517600357533 54.0
86.620738739 0.0324688106775284 55.0
0 2.96367526054382 56.0
85.377236716 -0.0113368639722466 57.0
220.47337983 0.642609715461731 58.0
250.6959273 0.732288956642151 59.0
22.416862157 0.0418614074587822 60.0
0 0.0869180634617805 61.0
168.29662169 0.817944526672363 0.0
168.29662169 0.996867418289185 1.0
166.9339465 1.62197136878967 2.0
161.46927526 4.60463857650757 3.0
10.467402326 3.63934755325317 4.0
10.467402326 3.65964722633362 5.0
-53.277100868 0.264210194349289 6.0
66.340657756 0.786105632781982 7.0
41.118725644 0.820609331130981 8.0
25.221932113 0.989912271499634 9.0
95.37903172 0.835938811302185 10.0
81.444725889 0.204495593905449 11.0
81.614376089 0.179400682449341 12.0
81.614376089 0.432757347822189 13.0
81.614376089 0.063494510948658 14.0
0 0.0220537558197975 15.0
0 0.232611089944839 16.0
0 1.43504476547241 17.0
52.474226981 1.21961379051208 18.0
-52.474226981 0.968955814838409 19.0
-52.474226981 0.400252252817154 20.0
-52.474226981 0.0108827464282513 21.0
74.889792096 1.51811385154724 22.0
151.82576015 1.96807718276978 23.0
115.98071045 0.130389392375946 24.0
22.496223233 0.320058435201645 25.0
226.71555224 0.971990823745728 26.0
226.71555224 1.38398671150208 27.0
29.381682139 1.37479591369629 28.0
256.09723438 0.603611588478088 29.0
256.09723438 1.73292219638824 30.0
46.469284019 1.05479121208191 31.0
9.4627346421 0.601389944553375 32.0
22.415565116 1.30150485038757 33.0
22.415565116 0.909669101238251 34.0
22.415565116 0.100108429789543 35.0
9.9668361361 0.0309555567800999 36.0
8.7327961612 0.348376542329788 37.0
0 0.0323585048317909 38.0
0 0.176541656255722 39.0
-11.562383597 0.388622552156448 40.0
-24.785138644 0.509353697299957 41.0
4.929097967 0.600573539733887 42.0
-32.05401104 0.618143141269684 43.0
89.970882882 0.732185661792755 44.0
89.970882882 0.103320106863976 45.0
7.3596563842 0.706039130687714 46.0
2.9162321577 0.330140143632889 47.0
168.29662169 0.0222337655723095 48.0
161.46927526 0.259080946445465 49.0
10.467402326 0.278682440519333 50.0
66.340657756 0.257770150899887 51.0
95.37903172 0.19437301158905 52.0
81.614376089 0.0667522251605988 53.0
81.614376089 0.0465983264148235 54.0
81.614376089 0.0607212781906128 55.0
0 4.17100095748901 56.0
74.889792096 0.0710866823792458 57.0
226.71555224 0.567206740379333 58.0
256.09723438 0.521949529647827 59.0
22.415565116 0.0227374136447906 60.0
0 0.117963045835495 61.0
144.41263541 1.02961659431458 0.0
144.41263541 0.743367314338684 1.0
143.18388008 1.39571619033813 2.0
138.39477482 7.55828666687012 3.0
0 9.45283985137939 4.0
0 2.95382928848267 5.0
-69.271004649 0.362803965806961 6.0
81.619072021 0.37087869644165 7.0
37.632276512 1.26780366897583 8.0
43.986795509 1.03631174564362 9.0
87.918132689 0.964756011962891 10.0
74.566347316 0.0477202534675598 11.0
88.128366667 0.490102410316467 12.0
88.128366667 0.274110972881317 13.0
88.128366667 0.0548179596662521 14.0
0 0.0992236286401749 15.0
0 0.130816042423248 16.0
0 1.23775768280029 17.0
54.031540706 1.58413672447205 18.0
-54.031540706 1.58709049224854 19.0
-54.031540706 0.62078058719635 20.0
-54.031540706 0.00361935049295425 21.0
73.974049617 0.99669623374939 22.0
148.2997402 0.404742449522018 23.0
110.91529689 1.02280724048615 24.0
17.543401987 0.0987009331583977 25.0
222.27378982 0.569385409355164 26.0
222.27378982 1.33625948429108 27.0
21.771844228 1.77201271057129 28.0
244.04563405 0.235377177596092 29.0
244.04563405 1.7031809091568 30.0
47.167519538 1.0744366645813 31.0
9.465183639 0.454770177602768 32.0
19.94250891 1.10869789123535 33.0
19.94250891 1.08281111717224 34.0
19.94250891 0.146982058882713 35.0
8.9841078315 0.0303637571632862 36.0
7.8060363896 0.366181492805481 37.0
0 0.0121873803436756 38.0
0 0.0436737686395645 39.0
-13.073794309 1.03396356105804 40.0
-25.512469111 0.300897419452667 41.0
4.8147032541 0.523469150066376 42.0
-32.822460705 0.583488285541534 43.0
86.302508924 0.727921664714813 44.0
86.302508924 0.0703486874699593 45.0
7.5763647403 0.745194673538208 46.0
3.1293380817 0.395986795425415 47.0
144.41263541 0.0726868137717247 48.0
138.39477482 0.0209576021879911 49.0
0 0.961851239204407 50.0
81.619072021 0.127781122922897 51.0
87.918132689 0.260845363140106 52.0
88.128366667 0.00738455960527062 53.0
88.128366667 0.0584429576992989 54.0
88.128366667 0.00204037129878998 55.0
0 0.411133229732513 56.0
73.974049617 -0.0155410207808018 57.0
222.27378982 0.21316684782505 58.0
244.04563405 0.324122428894043 59.0
19.94250891 -0.00406445190310478 60.0
0 0.090334951877594 61.0
126.83128159 0.865339875221252 0.0
126.83128159 1.73662841320038 1.0
125.26107328 2.20849418640137 2.0
120.47196803 4.94644260406494 3.0
0 4.49516773223877 4.0
0 6.38736915588379 5.0
-61.798871972 2.51920580863953 6.0
73.026698318 0.56569516658783 7.0
63.34008349 1.44709396362305 8.0
9.6866148281 1.59708309173584 9.0
93.554675125 0.923231601715088 10.0
78.808116916 0.0824327170848846 11.0
87.926901944 1.0086874961853 12.0
87.926901944 0.128948360681534 13.0
87.926901944 0.0885980278253555 14.0
0 0.142675533890724 15.0
0 0.199058204889297 16.0
0 1.40428793430328 17.0
52.776771846 0.503482818603516 18.0
-52.776771846 0.471140652894974 19.0
-52.776771846 1.07165133953094 20.0
-52.776771846 0.0113545656204224 21.0
73.897110895 1.840824842453 22.0
141.25091032 0.909659922122955 23.0
106.99214306 0.975748121738434 24.0
24.199198064 0.0809732377529144 25.0
215.14802122 1.42662513256073 26.0
215.14802122 1.65269303321838 27.0
25.686256509 1.7181704044342 28.0
240.83427773 0.513269662857056 29.0
240.83427773 1.82156598567963 30.0
46.903259174 0.939391136169434 31.0
9.3631709189 0.343609571456909 32.0
21.12033905 0.833908319473267 33.0
21.12033905 0.826507806777954 34.0
21.12033905 0.129708051681519 35.0
8.4086222409 0.0492429360747337 36.0
7.0700698031 0.412944376468658 37.0
0 0.0154997073113918 38.0
0 0.114379778504372 39.0
-11.430029971 0.569973587989807 40.0
-24.57051397 0.749015867710114 41.0
5.0044904806 0.389959692955017 42.0
-31.801998994 0.616581499576569 43.0
90.849656531 0.855029940605164 44.0
90.849656531 0.075954407453537 45.0
6.8497272533 0.784755945205688 46.0
2.2075737231 0.408540368080139 47.0
126.83128159 0.123252913355827 48.0
120.47196803 0.456575155258179 49.0
0 0.949566721916199 50.0
73.026698318 0.196872502565384 51.0
93.554675125 0.204879924654961 52.0
87.926901944 0.0613418854773045 53.0
87.926901944 0.0774004608392715 54.0
87.926901944 0.0641988664865494 55.0
0 2.74035334587097 56.0
73.897110895 0.0566367991268635 57.0
215.14802122 0.710570514202118 58.0
240.83427773 1.18738639354706 59.0
21.12033905 0.00948349572718143 60.0
0 0.0998087376356125 61.0
154.13687962 0.722038984298706 0.0
154.13687962 1.4495302438736 1.0
153.00781961 1.38346910476685 2.0
148.21871434 2.67057657241821 3.0
0 3.65291213989258 4.0
0 4.10161399841309 5.0
-61.421100799 0.399193078279495 6.0
73.84472737 1.2927987575531 7.0
48.25273717 0.523645222187042 8.0
25.5919902 0.379512190818787 9.0
81.124853509 0.875572085380554 10.0
69.202773871 0.30084216594696 11.0
79.756486446 0.679095506668091 12.0
79.756486446 0.142636194825172 13.0
79.756486446 0.0549032650887966 14.0
0 0.0230962373316288 15.0
0 0.203838214278221 16.0
0 0.96781849861145 17.0
63.208863753 1.38015711307526 18.0
-63.208863753 1.37507355213165 19.0
-63.208863753 0.289941668510437 20.0
-63.208863753 0.016043484210968 21.0
84.194030483 1.46167826652527 22.0
146.38429718 1.17982125282288 23.0
112.32251102 0.544870913028717 24.0
25.062071364 0.0946426391601562 25.0
230.57832767 0.871717512607574 26.0
230.57832767 1.752406001091 27.0
24.723948847 1.78962469100952 28.0
255.30227652 0.703451693058014 29.0
255.30227652 1.75171303749084 30.0
47.221667292 1.19491720199585 31.0
9.4026835839 0.326652109622955 32.0
20.98516673 1.31544840335846 33.0
20.98516673 0.963256061077118 34.0
20.98516673 0.11051145195961 35.0
8.8458024804 0.0372473299503326 36.0
7.6368393769 0.378811627626419 37.0
0 0.0437662228941917 38.0
0 0.113577835261822 39.0
-12.945610337 0.71539169549942 40.0
-25.716232777 0.514467060565948 41.0
4.6962958086 0.449773281812668 42.0
-32.812074376 0.349335849285126 43.0
90.517542375 0.812745809555054 44.0
90.517542375 0.0898854210972786 45.0
7.2987561918 0.715055704116821 46.0
2.7125384862 0.407576978206635 47.0
154.13687962 0.0985607355833054 48.0
148.21871434 0.144482389092445 49.0
0 0.238637551665306 50.0
73.84472737 0.000962391495704651 51.0
81.124853509 0.347868978977203 52.0
79.756486446 0.0743017569184303 53.0
79.756486446 0.0861208289861679 54.0
79.756486446 0.0805556178092957 55.0
0 3.6854522228241 56.0
84.194030483 0.138723388314247 57.0
230.57832767 0.444612592458725 58.0
255.30227652 0.614088952541351 59.0
20.98516673 0.00563818402588367 60.0
0 0.0877688452601433 61.0
208.70200743 0.865278303623199 0.0
208.70200743 0.990231871604919 1.0
207.37512405 1.02912890911102 2.0
201.91045351 4.42659473419189 3.0
2.992479991 7.77618885040283 4.0
2.992479991 3.35878419876099 5.0
-64.441439505 0.384982526302338 6.0
73.324653026 0.452680051326752 7.0
41.337514483 1.02358686923981 8.0
31.987138543 1.14596676826477 9.0
89.356159841 1.00857615470886 10.0
75.516807956 0.218739688396454 11.0
79.757433223 0.244484543800354 12.0
79.757433223 0.56255704164505 13.0
79.757433223 0.11010367423296 14.0
0 0.0521812736988068 15.0
0 0.205177128314972 16.0
0 1.28034675121307 17.0
59.065809555 1.38281202316284 18.0
-59.065809555 1.35862183570862 19.0
-59.065809555 0.615580081939697 20.0
-59.065809555 -0.047431793063879 21.0
78.017342131 2.4708514213562 22.0
144.31679276 1.99238216876984 23.0
110.78672928 0.667416036128998 24.0
22.354725185 0.0950068831443787 25.0
222.33413489 1.79229927062988 26.0
222.33413489 2.05528354644775 27.0
27.374143202 1.71256613731384 28.0
249.70827809 0.856892585754395 29.0
249.70827809 1.60629558563232 30.0
46.859939668 1.17258882522583 31.0
9.2965508046 0.510215997695923 32.0
18.951532576 1.82291781902313 33.0
18.951532576 0.626713156700134 34.0
18.951532576 0.169778242707253 35.0
8.772734566 0.0260564554482698 36.0
7.4858055258 0.300714164972305 37.0
0 0.019687183201313 38.0
0 0.158876955509186 39.0
-13.27846618 0.327259927988052 40.0
-25.71033205 0.834814548492432 41.0
4.9541956521 0.335226595401764 42.0
-33.302695337 0.426915556192398 43.0
104.35869102 0.715579569339752 44.0
104.35869102 0.0647791847586632 45.0
7.0222829326 0.661782622337341 46.0
2.3706374113 0.451127618551254 47.0
208.70200743 0.117782466113567 48.0
201.91045351 0.0689314380288124 49.0
2.992479991 0.488433957099915 50.0
73.324653026 0.22629776597023 51.0
89.356159841 0.411140859127045 52.0
79.757433223 0.0260369777679443 53.0
79.757433223 0.0806572809815407 54.0
79.757433223 0.0469164662063122 55.0
0 4.75438261032104 56.0
78.017342131 0.112345144152641 57.0
222.33413489 0.47714626789093 58.0
249.70827809 0.158410012722015 59.0
18.951532576 0.00837588682770729 60.0
0 0.124621875584126 61.0
181.62515241 0.77395224571228 0.0
181.62515241 1.27448189258575 1.0
180.15455471 1.56487786769867 2.0
174.6899193 4.89447593688965 3.0
4.2564202358 4.12511825561523 4.0
4.2564202358 5.21644496917725 5.0
-72.098795726 5.05665016174316 6.0
83.483097521 0.821495294570923 7.0
44.196627271 0.844960689544678 8.0
39.28647025 0.588896870613098 9.0
88.614125945 0.870840728282928 10.0
75.720556835 0.0538153350353241 11.0
84.215141666 0.434432953596115 12.0
84.215141666 0.102218724787235 13.0
84.215141666 0.0977029651403427 14.0
0 0.204241573810577 15.0
0 0.199656277894974 16.0
0 1.10894298553467 17.0
55.600708123 1.24509751796722 18.0
-55.600708123 1.61771774291992 19.0
-55.600708123 0.539692461490631 20.0
-55.600708123 0.0130514800548553 21.0
76.040437674 1.8791252374649 22.0
149.19915747 1.33565711975098 23.0
112.54209859 0.535442411899567 24.0
17.994948387 0.0933840796351433 25.0
225.23959515 1.38072681427002 26.0
225.23959515 2.63139891624451 27.0
26.747557331 2.35709095001221 28.0
251.98715248 0.908278942108154 29.0
251.98715248 1.62195932865143 30.0
46.40963008 0.975223124027252 31.0
9.3703610219 0.49119159579277 32.0
20.439729551 0.956153512001038 33.0
20.439729551 1.08602714538574 34.0
20.439729551 0.128188356757164 35.0
9.318360727 0.037059310823679 36.0
7.9501512172 0.393954306840897 37.0
0 0.0250542834401131 38.0
0 0.115595176815987 39.0
-10.856119697 0.585587859153748 40.0
-24.189111458 0.758440673351288 41.0
4.7077414443 0.364520817995071 42.0
-31.359917142 0.48941507935524 43.0
93.397348051 0.685194492340088 44.0
93.397348051 0.0815336406230927 45.0
7.7040900788 0.61883944272995 46.0
2.9367042804 0.420782208442688 47.0
181.62515241 0.141347214579582 48.0
174.6899193 0.335437119007111 49.0
4.2564202358 0.293581962585449 50.0
83.483097521 0.12456089258194 51.0
88.614125945 0.0972898676991463 52.0
84.215141666 0.046744741499424 53.0
84.215141666 0.0659556910395622 54.0
84.215141666 0.117332242429256 55.0
0 3.53287363052368 56.0
76.040437674 0.031311396509409 57.0
225.23959515 0.929195582866669 58.0
251.98715248 0.421241372823715 59.0
20.439729551 -0.0127614727243781 60.0
0 0.112396255135536 61.0
149.07264479 1.26879906654358 0.0
149.07264479 1.01861417293549 1.0
148.03508876 1.88680636882782 2.0
143.24598357 5.40679454803467 3.0
0 4.63999843597412 4.0
0 4.51323652267456 5.0
-53.593741838 0.439511358737946 6.0
66.914477695 0.968517780303955 7.0
64.651230919 1.98112893104553 8.0
2.2632467768 1.9595559835434 9.0
91.520457925 1.22732055187225 10.0
78.70784005 0.086358554661274 11.0
81.887164027 0.508620798587799 12.0
81.887164027 0.528809368610382 13.0
81.887164027 0.102325960993767 14.0
0 0.0437159165740013 15.0
0 0.133175700902939 16.0
0 1.3726270198822 17.0
58.338314998 1.76224398612976 18.0
-58.338314998 2.02895617485046 19.0
-58.338314998 0.528564929962158 20.0
-58.338314998 0.0190793946385384 21.0
77.352507509 1.30455684661865 22.0
129.81382164 0.901462912559509 23.0
96.730522234 0.730312824249268 24.0
21.880474473 0.0633650869131088 25.0
207.16632915 0.81663054227829 26.0
207.16632915 1.94771480560303 27.0
26.218402311 2.02582716941833 28.0
233.38473146 0.748258888721466 29.0
233.38473146 2.08506774902344 30.0
47.471453549 1.22544252872467 31.0
9.4582182937 0.664818525314331 32.0
19.014192512 1.01630067825317 33.0
19.014192512 1.58447659015656 34.0
19.014192512 0.0696292445063591 35.0
7.9484746859 0.0254577845335007 36.0
6.706596863 0.385986983776093 37.0
0 0.00537882000207901 38.0
0 0.248455494642258 39.0
-11.92109063 0.477990031242371 40.0
-26.043160659 0.278300791978836 41.0
4.72587342 0.504801213741302 42.0
-33.67198599 0.473577827215195 43.0
105.98775386 0.508140683174133 44.0
105.98775386 0.0803900361061096 45.0
6.0359485108 0.361323416233063 46.0
1.0947542526 0.365516066551208 47.0
149.07264479 0.13765823841095 48.0
143.24598357 0.24198779463768 49.0
0 0.236373767256737 50.0
66.914477695 0.304880410432816 51.0
91.520457925 0.079376332461834 52.0
81.887164027 0.0328628532588482 53.0
81.887164027 0.0698317959904671 54.0
81.887164027 0.0840614661574364 55.0
0 2.90994143486023 56.0
77.352507509 0.137611612677574 57.0
207.16632915 0.448469430208206 58.0
233.38473146 0.712320923805237 59.0
19.014192512 0.00835515558719635 60.0
0 0.0868747979402542 61.0
190.01757336 1.19314956665039 0.0
190.01757336 1.36165034770966 1.0
188.66885637 1.71612513065338 2.0
183.8797513 5.57063341140747 3.0
0 6.67977857589722 4.0
0 4.66009855270386 5.0
-62.224949146 4.79847526550293 6.0
72.252696205 0.813793778419495 7.0
42.225108301 0.535607576370239 8.0
30.027587904 0.394716560840607 9.0
98.665536108 1.16714751720428 10.0
86.022583608 0.0851399600505829 11.0
87.10126157 0.348487824201584 12.0
87.10126157 0.266006171703339 13.0
87.10126157 0.0731261894106865 14.0
0 0.199253514409065 15.0
0 0.172797337174416 16.0
0 1.44315409660339 17.0
59.519964427 1.22509264945984 18.0
-59.519964427 1.54762542247772 19.0
-59.519964427 0.242423564195633 20.0
-59.519964427 0.0708224326372147 21.0
80.061809254 1.61802732944489 22.0
143.47833961 1.0800644159317 23.0
112.21057079 0.590951502323151 24.0
21.192106155 0.27567845582962 25.0
223.54014887 1.58004069328308 26.0
223.54014887 1.92745804786682 27.0
24.76826497 1.92783081531525 28.0
248.30841384 1.01859164237976 29.0
248.30841384 1.80757653713226 30.0
46.58597942 0.947660684585571 31.0
9.3534796094 0.492993116378784 32.0
20.541844827 1.37176764011383 33.0
20.541844827 0.940848767757416 34.0
20.541844827 0.111075654625893 35.0
9.1433347029 0.0231475885957479 36.0
7.8684789051 0.393179416656494 37.0
0 0.0371768847107887 38.0
0 0.345610529184341 39.0
-12.108944006 0.579564690589905 40.0
-26.246809287 0.693377673625946 41.0
5.4084105018 0.463393867015839 42.0
-34.001402868 0.428085267543793 43.0
104.69358593 0.699505746364594 44.0
104.69358593 0.0728006586432457 45.0
6.9357531978 0.729182183742523 46.0
2.4432741465 0.411013692617416 47.0
190.01757336 0.127402871847153 48.0
183.8797513 0.296944588422775 49.0
0 0.373631238937378 50.0
72.252696205 0.172848477959633 51.0
98.665536108 0.138857707381248 52.0
87.10126157 0.0395825691521168 53.0
87.10126157 0.0788434445858002 54.0
87.10126157 0.155312016606331 55.0
0 2.38655304908752 56.0
80.061809254 -0.033378753811121 57.0
223.54014887 0.684181272983551 58.0
248.30841384 0.746520817279816 59.0
20.541844827 -0.0118873491883278 60.0
0 0.0810307785868645 61.0
152.74185211 0.701635718345642 0.0
152.74185211 1.61339843273163 1.0
151.3136501 1.8395859003067 2.0
146.52454485 3.72221875190735 3.0
0 7.09278392791748 4.0
0 4.57159423828125 5.0
-53.058856393 2.8312976360321 6.0
65.815672699 0.750014007091522 7.0
59.684716309 0.686477243900299 8.0
6.13095639 0.731305241584778 9.0
88.272600992 0.840838611125946 10.0
75.606332991 0.366126924753189 11.0
78.824543299 0.333306670188904 12.0
78.824543299 0.468341529369354 13.0
78.824543299 0.0744504779577255 14.0
0 0.0558766089379787 15.0
0 0.19895476102829 16.0
0 0.703804433345795 17.0
61.644158432 1.33203995227814 18.0
-61.644158432 1.10433614253998 19.0
-61.644158432 0.324507415294647 20.0
-61.644158432 0.0432500876486301 21.0
82.203990772 1.40845715999603 22.0
144.04309677 1.08802020549774 23.0
109.08354117 0.625823736190796 24.0
25.242376103 0.0878151208162308 25.0
226.24708754 0.600692749023438 26.0
226.24708754 1.39155101776123 27.0
24.578545731 1.94386875629425 28.0
250.82563327 0.985382974147797 29.0
250.82563327 1.95374596118927 30.0
46.483145703 1.1470959186554 31.0
9.4247891361 0.494189411401749 32.0
20.55983234 1.40939033031464 33.0
20.55983234 1.0290002822876 34.0
20.55983234 0.10460689663887 35.0
9.3291238408 0.0185669120401144 36.0
8.0531536111 0.379356682300568 37.0
0 0.0254945941269398 38.0
0 0.0656513720750809 39.0
-11.273727066 0.531384229660034 40.0
-23.742589389 0.302229434251785 41.0
5.6386418392 0.425309866666794 42.0
-31.862133103 0.733618319034576 43.0
96.553437636 0.712056457996368 44.0
96.553437636 0.0887379497289658 45.0
6.8411366213 0.573697328567505 46.0
2.8980359277 0.369579583406448 47.0
152.74185211 0.132716879248619 48.0
146.52454485 0.291942358016968 49.0
0 0.623414635658264 50.0
65.815672699 0.0903513133525848 51.0
88.272600992 0.15154641866684 52.0
78.824543299 0.0848023667931557 53.0
78.824543299 0.126977071166039 54.0
78.824543299 0.01818473264575 55.0
0 5.31753873825073 56.0
82.203990772 0.0846741572022438 57.0
226.24708754 0.178713068366051 58.0
250.82563327 0.794159889221191 59.0
20.55983234 -0.0229376088827848 60.0
0 0.0959920734167099 61.0
136.25361613 1.00009155273438 0.0
136.25361613 0.899943709373474 1.0
134.75939214 1.57688748836517 2.0
129.97028681 4.53774261474609 3.0
0 6.29325294494629 4.0
0 4.11978960037231 5.0
-67.852109221 0.518535673618317 6.0
76.10669552 0.503001987934113 7.0
42.191487825 0.749316096305847 8.0
33.915207695 0.778334498405457 9.0
83.60485899 0.926687002182007 10.0
70.104996639 0.0214209984987974 11.0
76.693596989 0.975265741348267 12.0
76.693596989 0.712641894817352 13.0
76.693596989 0.0529224723577499 14.0
0 0.00797851011157036 15.0
0 0.244575276970863 16.0
0 1.25079369544983 17.0
60.179770257 1.79614055156708 18.0
-60.179770257 1.35160005092621 19.0
-60.179770257 0.632064461708069 20.0
-60.179770257 -0.0325477123260498 21.0
79.559582881 1.35470259189606 22.0
151.60728787 1.06062924861908 23.0
119.44171363 0.538218259811401 24.0
24.307269965 0.164422929286957 25.0
231.16687075 1.23362469673157 26.0
231.16687075 2.10091137886047 27.0
24.37208719 1.90236961841583 28.0
255.53895794 0.737056493759155 29.0
255.53895794 1.85805296897888 30.0
46.871204431 1.07564890384674 31.0
9.2494074966 0.716241478919983 32.0
19.379812624 1.74436867237091 33.0
19.379812624 1.02464187145233 34.0
19.379812624 0.0952519774436951 35.0
9.3888666675 0.0223682038486004 36.0
8.0907354596 0.383057117462158 37.0
0 0.0270274318754673 38.0
0 0.27951192855835 39.0
-10.841184909 0.631132006645203 40.0
-25.177110446 0.603476345539093 41.0
5.1473037014 0.582646489143372 42.0
-32.567391753 0.528544127941132 43.0
78.453281568 0.74794065952301 44.0
78.453281568 0.0849883928894997 45.0
6.8001310737 0.714322626590729 46.0
2.6965019402 0.409977495670319 47.0
136.25361613 0.379426330327988 48.0
129.97028681 0.520866692066193 49.0
0 0.298989236354828 50.0
76.10669552 0.0744566172361374 51.0
83.60485899 0.131999105215073 52.0
76.693596989 0.0764592364430428 53.0
76.693596989 0.0898473039269447 54.0
76.693596989 0.121152594685555 55.0
0 5.71122884750366 56.0
79.559582881 0.166545063257217 57.0
231.16687075 0.686722099781036 58.0
255.53895794 0.384964227676392 59.0
19.379812624 -0.00869473442435265 60.0
0 0.0886383503675461 61.0
128.20001158 0.862100124359131 0.0
128.20001158 1.05526578426361 1.0
126.84002918 1.12829911708832 2.0
121.37539377 5.89237689971924 3.0
1.6466500011 3.96922588348389 4.0
1.6466500011 4.453444480896 5.0
-68.187320945 0.0464214533567429 6.0
77.259096858 0.766847968101501 7.0
54.920982011 0.403925359249115 8.0
22.338114847 0.370109379291534 9.0
84.067800019 0.794265806674957 10.0
71.081821153 0.169589251279831 11.0
80.632790553 0.630309522151947 12.0
80.632790553 0.182491645216942 13.0
80.632790553 0.109960541129112 14.0
0 0.150102227926254 15.0
0 0.134079724550247 16.0
0 1.15619325637817 17.0
55.58310554 1.21718037128448 18.0
-55.58310554 1.49511861801147 19.0
-55.58310554 0.53266167640686 20.0
-55.58310554 0.0242566540837288 21.0
76.153678805 1.17993319034576 22.0
146.47193818 0.950662672519684 23.0
115.48090519 0.475268363952637 24.0
22.649269719 0.386644423007965 25.0
222.62561699 0.901921272277832 26.0
222.62561699 1.52518594264984 27.0
25.896068997 1.12321996688843 28.0
248.52168599 0.552332043647766 29.0
248.52168599 1.70084750652313 30.0
46.682608366 1.24568927288055 31.0
9.4789957948 0.48745334148407 32.0
20.570573265 1.21616005897522 33.0
20.570573265 0.876435160636902 34.0
20.570573265 0.101833529770374 35.0
9.8869165723 0.0344606079161167 36.0
8.647936582 0.412443399429321 37.0
0 0.0331370159983635 38.0
0 0.238442063331604 39.0
-12.290368646 0.48407781124115 40.0
-25.845548047 0.643240928649902 41.0
5.5246199433 0.407539278268814 42.0
-33.736420292 0.415044248104095 43.0
103.05748979 0.915187239646912 44.0
103.05748979 0.0472690835595131 45.0
6.3483320677 0.632656633853912 46.0
2.1040989432 0.340175569057465 47.0
128.20001158 0.395661115646362 48.0
121.37539377 0.388452649116516 49.0
1.6466500011 0.283186376094818 50.0
77.259096858 0.129579335451126 51.0
84.067800019 0.236569434404373 52.0
80.632790553 0.0731820464134216 53.0
80.632790553 0.0516294501721859 54.0
80.632790553 0.0351716056466103 55.0
0 4.13470029830933 56.0
76.153678805 0.0626825839281082 57.0
222.62561699 0.679116904735565 58.0
248.52168599 0.422657459974289 59.0
20.570573265 -0.0141661707311869 60.0
0 0.107092335820198 61.0
204.56220038 0.759300529956818 0.0
204.56220038 1.04117119312286 1.0
203.61065689 1.44046032428741 2.0
198.14598614 4.57262182235718 3.0
3.8651869801 5.31572341918945 4.0
3.8651869801 3.33708739280701 5.0
-52.830544304 0.142384275794029 6.0
66.660966419 0.444663465023041 7.0
33.169125843 0.712317168712616 8.0
33.491840576 0.692140996456146 9.0
82.308168289 0.79463130235672 10.0
68.836995196 0.0147204417735338 11.0
78.570014423 0.187498599290848 12.0
78.570014423 0.101840026676655 13.0
78.570014423 0.113560937345028 14.0
6.0105823476 0.401863873004913 15.0
6.0105823476 0.147958815097809 16.0
0 1.21580290794373 17.0
49.000948978 1.10709929466248 18.0
-49.000948978 1.13192462921143 19.0
-49.000948978 0.930731296539307 20.0
-49.000948978 -0.0618722215294838 21.0
68.088424927 1.24507331848145 22.0
126.71407341 1.15330219268799 23.0
98.271755322 0.246233388781548 24.0
21.398824728 0.185842037200928 25.0
194.80249834 1.04080271720886 26.0
194.80249834 1.13297390937805 27.0
27.848450313 0.811146080493927 28.0
222.65094865 0.57962566614151 29.0
222.65094865 1.42697715759277 30.0
46.023111031 1.18116533756256 31.0
9.2427518925 0.491599410772324 32.0
19.087475949 1.26546359062195 33.0
19.087475949 0.939704298973083 34.0
19.087475949 0.0502979829907417 35.0
9.1636210689 0.0229762867093086 36.0
7.9694005139 0.393036454916 37.0
0 0.0256089083850384 38.0
0 0.121042594313622 39.0
-11.962833369 0.411619961261749 40.0
-25.616129431 0.777637600898743 41.0
5.621987391 0.318734616041183 42.0
-33.799103574 0.634722113609314 43.0
87.004087066 0.64202880859375 44.0
87.004087066 0.100973188877106 45.0
7.036088106 0.589753150939941 46.0
2.3785219692 0.407362997531891 47.0
204.56220038 0.0569316223263741 48.0
198.14598614 0.164657324552536 49.0
3.8651869801 0.254405558109283 50.0
66.660966419 0.306437075138092 51.0
82.308168289 0.0935262143611908 52.0
78.570014423 0.0536581836640835 53.0
78.570014423 0.0843000486493111 54.0
78.570014423 -0.0017767921090126 55.0
0 3.04433012008667 56.0
68.088424927 0.223822757601738 57.0
194.80249834 0.134749948978424 58.0
222.65094865 0.312169224023819 59.0
19.087475949 -0.00777195952832699 60.0
0 0.089310884475708 61.0
174.48200453 1.19869208335876 0.0
174.48200453 0.933844149112701 1.0
172.96735448 1.53465116024017 2.0
168.17824933 3.46803545951843 3.0
0 3.40008401870728 4.0
0 1.98616230487823 5.0
-50.82719255 0.841063022613525 6.0
61.022641272 0.466282784938812 7.0
41.779490182 1.23820495605469 8.0
19.24315109 1.09619569778442 9.0
85.847974644 1.32035613059998 10.0
71.288098999 0.479087889194489 11.0
82.828656494 0.194361120462418 12.0
82.828656494 0.259696662425995 13.0
82.828656494 0.11864498257637 14.0
0 0.291484147310257 15.0
0 0.148068755865097 16.0
0 1.71802163124084 17.0
61.894179861 1.55971467494965 18.0
-61.894179861 1.65500092506409 19.0
-61.894179861 0.366552382707596 20.0
-61.894179861 -0.0534004978835583 21.0
82.024743155 0.606770634651184 22.0
138.71602212 0.42010173201561 23.0
99.610198078 0.698723316192627 24.0
20.499340607 0.0687107592821121 25.0
220.74076528 1.09333026409149 26.0
220.74076528 2.40977478027344 27.0
31.295013294 2.43307542800903 28.0
252.03577857 2.02938985824585 29.0
252.03577857 2.1731870174408 30.0
46.221995264 0.993498086929321 31.0
9.3956880025 0.494311690330505 32.0
20.130563295 1.83919501304626 33.0
20.130563295 0.419060587882996 34.0
20.130563295 0.109546720981598 35.0
8.7725384614 0.036929938942194 36.0
7.4713598055 0.438819229602814 37.0
0 0.0520410910248756 38.0
0 0.076917365193367 39.0
-12.976678274 0.583637297153473 40.0
-27.443974273 0.615168333053589 41.0
5.4515567405 0.324457883834839 42.0
-35.64183884 0.644923567771912 43.0
92.524200347 0.631641030311584 44.0
92.524200347 0.0957462266087532 45.0
7.1470551281 0.640522181987762 46.0
2.6622659411 0.367139488458633 47.0
174.48200453 0.186378479003906 48.0
168.17824933 0.102614693343639 49.0
0 0.108600541949272 50.0
61.022641272 0.0705549269914627 51.0
85.847974644 0.702826857566833 52.0
82.828656494 0.0425395257771015 53.0
82.828656494 0.106167316436768 54.0
82.828656494 0.0666494816541672 55.0
0 1.45101523399353 56.0
82.024743155 0.0971386358141899 57.0
220.74076528 1.0230256319046 58.0
252.03577857 1.86496293544769 59.0
20.130563295 0.0118459314107895 60.0
0 0.0763135850429535 61.0
187.81654296 0.866145849227905 0.0
187.81654296 0.920844316482544 1.0
186.48606613 0.853694438934326 2.0
181.02139543 4.97477436065674 3.0
15.345284605 5.14609241485596 4.0
15.345284605 4.48681831359863 5.0
-51.977232776 2.82973122596741 6.0
63.962826921 0.84958291053772 7.0
38.693185357 0.623959124088287 8.0
25.269641564 0.734932601451874 9.0
88.632447572 1.16537749767303 10.0
76.518717957 0.546152710914612 11.0
88.620093384 0.481897592544556 12.0
88.620093384 0.448009967803955 13.0
88.620093384 0.0689113810658455 14.0
0 0.180257394909859 15.0
0 0.192229360342026 16.0
0 1.08524751663208 17.0
51.637770145 1.25436329841614 18.0
-51.637770145 0.844735443592072 19.0
-51.637770145 0.754131376743317 20.0
-51.637770145 0.00485607236623764 21.0
72.146235953 1.91591787338257 22.0
141.31440828 2.0003068447113 23.0
107.70456328 0.321994096040726 24.0
21.048351762 0.113609872758389 25.0
213.46064423 1.23362874984741 26.0
213.46064423 1.29331207275391 27.0
28.667152278 1.79216647148132 28.0
242.12779651 0.919958591461182 29.0
242.12779651 1.75048089027405 30.0
46.270581594 1.20957541465759 31.0
9.2975411303 0.464576929807663 32.0
20.508465808 1.15914618968964 33.0
20.508465808 0.771256148815155 34.0
20.508465808 0.136031001806259 35.0
10.14317879 0.0287567153573036 36.0
8.9761668829 0.388382434844971 37.0
0 0.00776365399360657 38.0
0 0.129573568701744 39.0
-11.83576856 0.807787954807281 40.0
-25.506349666 0.606835961341858 41.0
5.5767332167 0.448092877864838 42.0
-33.461439074 0.394072681665421 43.0
88.218467389 0.749425053596497 44.0
88.218467389 0.088446244597435 45.0
6.5384229395 0.501442074775696 46.0
2.5326519879 0.327724575996399 47.0
187.81654296 0.123218134045601 48.0
181.02139543 0.0254178829491138 49.0
15.345284605 0.365925043821335 50.0
63.962826921 0.13656097650528 51.0
88.632447572 0.784020125865936 52.0
88.620093384 0.0721911862492561 53.0
88.620093384 0.196931749582291 54.0
88.620093384 0.00439848471432924 55.0
0 1.70857286453247 56.0
72.146235953 -0.0255230646580458 57.0
213.46064423 0.453170478343964 58.0
242.12779651 0.622172653675079 59.0
20.508465808 0.0191562604159117 60.0
0 0.104886397719383 61.0
190.55766073 1.00125551223755 0.0
190.55766073 0.579770803451538 1.0
189.29791954 0.556822538375854 2.0
184.50884959 5.58889198303223 3.0
0 6.597984790802 4.0
0 5.0024585723877 5.0
-51.299277527 1.60178112983704 6.0
61.309949461 0.975409984588623 7.0
52.862413838 1.11298799514771 8.0
8.4475356226 0.809273421764374 9.0
95.483229314 1.13162159919739 10.0
83.518178595 0.46481728553772 11.0
84.64222469 0.554642975330353 12.0
84.64222469 0.205910891294479 13.0
84.64222469 0.101856231689453 14.0
0 0.239229038357735 15.0
0 0.179518699645996 16.0
0 1.62858307361603 17.0
54.852458955 1.74802350997925 18.0
-54.852458955 2.02408123016357 19.0
-54.852458955 0.264652162790298 20.0
-54.852458955 0.0415476597845554 21.0
75.788286842 2.61541414260864 22.0
152.93528759 3.56503033638 23.0
116.68628014 0.135069370269775 24.0
22.314286748 0.389801442623138 25.0
228.72357443 1.49190378189087 26.0
228.72357443 2.12004709243774 27.0
22.948197039 2.70873498916626 28.0
251.67177147 1.65186285972595 29.0
251.67177147 1.99736785888672 30.0
47.156666519 0.912701547145844 31.0
9.1957565751 0.465330362319946 32.0
20.935827887 1.98406910896301 33.0
20.935827887 0.649462401866913 34.0
20.935827887 0.136784598231316 35.0
9.2687886756 0.0274934507906437 36.0
7.8623317612 0.352924108505249 37.0
0 0.0391868427395821 38.0
0 0.0812445729970932 39.0
-12.270257024 0.559684753417969 40.0
-25.11961485 0.316101461648941 41.0
4.7659920015 0.489701330661774 42.0
-32.164030036 0.441363334655762 43.0
101.59337924 0.681019604206085 44.0
101.59337924 0.028687410056591 45.0
6.6929391315 0.483588546514511 46.0
2.346476354 0.334955155849457 47.0
190.55766073 0.104569748044014 48.0
184.50884959 0.147438257932663 49.0
0 0.359386026859283 50.0
61.309949461 0.354322850704193 51.0
95.483229314 0.160590887069702 52.0
84.64222469 0.0945315062999725 53.0
84.64222469 0.153607100248337 54.0
84.64222469 0.051067940890789 55.0
0 1.14339733123779 56.0
75.788286842 0.0661228969693184 57.0
228.72357443 0.243054866790771 58.0
251.67177147 1.09581136703491 59.0
20.935827887 0.00733036547899246 60.0
0 0.0852999538183212 61.0
169.68724982 0.876257240772247 0.0
169.68724982 1.48048365116119 1.0
168.56187988 2.13034248352051 2.0
163.77277457 5.9509129524231 3.0
0 7.9781608581543 4.0
0 5.49156761169434 5.0
-61.35810797 2.74159979820251 6.0
73.893222687 0.945495426654816 7.0
42.672463121 0.851732730865479 8.0
31.220759566 0.829060137271881 9.0
92.374730155 0.740662395954132 10.0
79.667439624 0.0475265383720398 11.0
84.614097885 0.573707401752472 12.0
84.614097885 1.00942802429199 13.0
84.614097885 0.0974004939198494 14.0
0 0.246801540255547 15.0
0 0.136680692434311 16.0
0 1.30680918693542 17.0
54.718906931 1.26114463806152 18.0
-54.718906931 0.966992020606995 19.0
-54.718906931 1.09550786018372 20.0
-54.718906931 -0.0865451693534851 21.0
76.802661769 1.26294529438019 22.0
154.16435229 1.62810623645782 23.0
122.02854974 0.49349319934845 24.0
22.503906652 0.181227996945381 25.0
230.96701406 1.35499453544617 26.0
230.96701406 2.0594425201416 27.0
28.952144085 1.44551134109497 28.0
259.91915814 1.06799125671387 29.0
259.91915814 2.071040391922 30.0
46.393426855 1.00606715679169 31.0
9.4226693453 0.405500948429108 32.0
22.083754838 1.93822014331818 33.0
22.083754838 0.980170905590057 34.0
22.083754838 0.139569237828255 35.0
9.9072548809 0.0303202848881483 36.0
8.5173728445 0.32197704911232 37.0
0 0.148421227931976 38.0
0 0.287101536989212 39.0
-11.719733111 0.735181450843811 40.0
-24.407464761 0.5677689909935 41.0
5.5537444643 0.503570854663849 42.0
-32.777858781 0.483480930328369 43.0
78.009905996 0.850559413433075 44.0
78.009905996 0.0791805237531662 45.0
6.4837352881 0.723866045475006 46.0
2.4895695356 0.428081154823303 47.0
169.68724982 0.348300606012344 48.0
163.77277457 0.2752745449543 49.0
0 0.439323872327805 50.0
73.893222687 0.350180774927139 51.0
92.374730155 0.195695921778679 52.0
84.614097885 0.0177522245794535 53.0
84.614097885 0.0769605860114098 54.0
84.614097885 0.397857517004013 55.0
0 4.47227764129639 56.0
76.802661769 0.141912743449211 57.0
230.96701406 0.931433498859406 58.0
259.91915814 1.34722781181335 59.0
22.083754838 -0.00576429255306721 60.0
0 0.07445128262043 61.0
192.23968465 0.878396511077881 0.0
192.23968465 1.6061794757843 1.0
190.79528618 1.63340818881989 2.0
186.00618094 6.09701919555664 3.0
0 7.49949932098389 4.0
0 5.73423767089844 5.0
-59.160014219 2.91923570632935 6.0
68.594493765 0.783634781837463 7.0
51.900780698 1.04016101360321 8.0
16.693713068 1.00726938247681 9.0
90.058321878 0.972484767436981 10.0
77.730720321 0.34744194149971 11.0
76.449980946 0.862942039966583 12.0
76.449980946 0.79547131061554 13.0
76.449980946 0.0795834213495255 14.0
0 0.0983958691358566 15.0
0 0.173433691263199 16.0
0 1.23027467727661 17.0
62.895426489 1.74382877349854 18.0
-62.895426489 2.06236219406128 19.0
-62.895426489 0.529320538043976 20.0
-62.895426489 -0.02683917991817 21.0
84.966956855 1.64879083633423 22.0
138.93287225 1.12755250930786 23.0
107.39309201 0.806399285793304 24.0
20.529843753 0.122469499707222 25.0
223.89982911 1.03832161426544 26.0
223.89982911 1.72656905651093 27.0
25.139583726 2.12252855300903 28.0
249.03941283 1.1064156293869 29.0
249.03941283 1.80669379234314 30.0
46.185879918 1.0408821105957 31.0
9.2628171483 0.515705347061157 32.0
22.071530366 1.54705905914307 33.0
22.071530366 0.627830147743225 34.0
22.071530366 0.115192838013172 35.0
10.369555062 0.0470835268497467 36.0
9.0802375422 0.360621362924576 37.0
0 0.0437635704874992 38.0
0 0.109075270593166 39.0
-11.967059398 0.561777889728546 40.0
-24.152728061 0.56938111782074 41.0
5.3382499486 0.406658262014389 42.0
-31.841806355 0.691309750080109 43.0
96.670164845 0.551152288913727 44.0
96.670164845 0.103651002049446 45.0
6.798131413 0.540435433387756 46.0
2.611301853 0.379953145980835 47.0
192.23968465 0.154035121202469 48.0
186.00618094 0.0689379796385765 49.0
0 0.552197098731995 50.0
68.594493765 0.192233949899673 51.0
90.058321878 0.334281206130981 52.0
76.449980946 0.102572277188301 53.0
76.449980946 0.183916479349136 54.0
76.449980946 0.197454407811165 55.0
0 3.90446639060974 56.0
84.966956855 0.120183721184731 57.0
223.89982911 0.321356564760208 58.0
249.03941283 0.55709582567215 59.0
22.071530366 0.00460361689329147 60.0
0 0.0803254693746567 61.0
193.02527893 0.841996908187866 0.0
193.02527893 0.568236947059631 1.0
191.5180648 0.95768940448761 2.0
186.05342939 3.20896148681641 3.0
5.7246548114 6.74352741241455 4.0
5.7246548114 2.87200570106506 5.0
-57.257263093 0.340268313884735 6.0
67.06368777 0.838436484336853 7.0
43.42448526 0.543184995651245 8.0
23.63920251 0.585694968700409 9.0
88.033331271 0.972951769828796 10.0
74.562036346 0.372237265110016 11.0
76.69518956 0.414314776659012 12.0
76.69518956 0.297206163406372 13.0
76.69518956 0.0543385818600655 14.0
0 0.085417814552784 15.0
0 0.190766856074333 16.0
0 1.11242985725403 17.0
53.436514807 1.70477867126465 18.0
-53.436514807 1.54604053497314 19.0
-53.436514807 0.225403547286987 20.0
-53.436514807 0.246197864413261 21.0
75.221423417 1.58047676086426 22.0
146.74417796 1.27527105808258 23.0
112.54936797 0.375793248414993 24.0
20.574831604 0.0893174186348915 25.0
221.96560137 1.64791882038116 26.0
221.96560137 1.81928706169128 27.0
29.311697191 1.61193072795868 28.0
251.27729856 0.851017653942108 29.0
251.27729856 1.97587430477142 30.0
47.196092055 1.07392978668213 31.0
9.4303944078 0.600846409797668 32.0
21.78490861 0.97481095790863 33.0
21.78490861 0.996410250663757 34.0
21.78490861 0.0840874314308167 35.0
9.845591153 0.026517940685153 36.0
8.555670841 0.380238860845566 37.0
0 0.00887485221028328 38.0
0 0.218660533428192 39.0
-12.100522604 0.509086608886719 40.0
-24.554303358 0.53804349899292 41.0
5.1521932111 0.459092617034912 42.0
-32.507052732 0.534988641738892 43.0
88.233764753 0.529208719730377 44.0
88.233764753 0.0762200579047203 45.0
6.2500811361 0.416485100984573 46.0
2.2980609116 0.339435070753098 47.0
193.02527893 0.127801060676575 48.0
186.05342939 0.188886448740959 49.0
5.7246548114 0.372510969638824 50.0
67.06368777 0.132494240999222 51.0
88.033331271 0.389288187026978 52.0
76.69518956 0.0959994345903397 53.0
76.69518956 0.137835383415222 54.0
76.69518956 0.0746236741542816 55.0
0 4.69435358047485 56.0
75.221423417 -0.0782298371195793 57.0
221.96560137 0.995060682296753 58.0
251.27729856 0.812002420425415 59.0
21.78490861 0.0138335563242435 60.0
0 0.0902324020862579 61.0
171.24719214 0.938813924789429 0.0
171.24719214 1.69387924671173 1.0
170.04774409 1.58197045326233 2.0
165.25863882 6.07439374923706 3.0
0 5.3001184463501 4.0
0 4.96874046325684 5.0
-55.053496427 1.99286472797394 6.0
68.190402646 0.536986827850342 7.0
47.134195225 1.53671646118164 8.0
21.056207421 1.55186545848846 9.0
96.118974942 0.999618530273438 10.0
82.455276048 0.138381361961365 11.0
87.896839546 0.189981251955032 12.0
87.896839546 0.128603339195251 13.0
87.896839546 0.0951489508152008 14.0
0 0.192242786288261 15.0
0 0.19212606549263 16.0
0 1.34245312213898 17.0
53.272767689 1.14235556125641 18.0
-53.272767689 0.865686595439911 19.0
-53.272767689 0.542109131813049 20.0
-53.272767689 0.0146407559514046 21.0
73.838738013 1.23908996582031 22.0
148.00034546 0.964865446090698 23.0
116.80413749 0.523708045482635 24.0
22.234338655 0.317017436027527 25.0
221.83908348 1.01731932163239 26.0
221.83908348 1.37920653820038 27.0
26.161354702 1.42572951316833 28.0
248.00043818 0.563335299491882 29.0
248.00043818 1.53649997711182 30.0
47.090253799 0.979219734668732 31.0
9.4101158305 0.448295950889587 32.0
20.565970324 1.21123456954956 33.0
20.565970324 1.0623551607132 34.0
20.565970324 0.126612514257431 35.0
9.1838181188 0.044560968875885 36.0
7.8253415608 0.369061768054962 37.0
0 0.0312690138816833 38.0
0 0.110082507133484 39.0
-10.523630198 0.383775442838669 40.0
-22.232130674 0.782505810260773 41.0
4.727264687 0.371926516294479 42.0
-29.710361088 0.543046951293945 43.0
88.062635051 0.778298556804657 44.0
88.062635051 0.100100420415401 45.0
7.0851940461 0.586597621440887 46.0
2.6067824934 0.401010811328888 47.0
171.24719214 0.0464725345373154 48.0
165.25863882 0.21247486770153 49.0
0 0.494876950979233 50.0
68.190402646 0.164340704679489 51.0
96.118974942 0.235506325960159 52.0
87.896839546 0.0311409346759319 53.0
87.896839546 0.0635132342576981 54.0
87.896839546 0.0480959862470627 55.0
0 1.54685854911804 56.0
73.838738013 0.137499555945396 57.0
221.83908348 0.571420133113861 58.0
248.00043818 0.319619953632355 59.0
20.565970324 -0.00655167736113071 60.0
0 0.0994587913155556 61.0
176.40877081 1.1616770029068 0.0
176.40877081 1.16861701011658 1.0
174.92202762 0.96705424785614 2.0
170.13292236 3.38797640800476 3.0
0 2.98856616020203 4.0
0 5.01600027084351 5.0
-68.430752019 1.65143394470215 6.0
80.597081809 0.278485178947449 7.0
43.298602045 1.02934396266937 8.0
37.298479764 1.01006674766541 9.0
103.2898637 1.35966360569 10.0
91.265877589 0.226339921355247 11.0
91.014067281 0.914428472518921 12.0
91.014067281 0.100183516740799 13.0
91.014067281 0.0766604393720627 14.0
0 0.562368273735046 15.0
0 0.196873441338539 16.0
0 1.10721516609192 17.0
57.021021264 1.02246117591858 18.0
-57.021021264 1.06039953231812 19.0
-57.021021264 0.222951576113701 20.0
-57.021021264 0.28517758846283 21.0
79.899388271 1.24571716785431 22.0
127.01400932 0.743741631507874 23.0
95.289468473 0.331297278404236 24.0
19.328022181 0.319411903619766 25.0
206.91339759 1.09377837181091 26.0
206.91339759 1.4752950668335 27.0
29.288553841 0.811049818992615 28.0
236.20195143 0.558444619178772 29.0
236.20195143 1.47759985923767 30.0
45.929567394 0.980206370353699 31.0
9.224461739 0.362950474023819 32.0
22.878367008 0.977277159690857 33.0
22.878367008 1.16209769248962 34.0
22.878367008 0.136938199400902 35.0
9.7165787534 0.0539287477731705 36.0
8.2061429826 0.381025284528732 37.0
0 0.0179296247661114 38.0
0 0.299154877662659 39.0
-11.877719525 0.260341554880142 40.0
-23.415356757 0.598852574825287 41.0
5.3831757668 0.322387367486954 42.0
-31.170050491 0.519880533218384 43.0
85.233700043 0.573719441890717 44.0
85.233700043 0.0960480868816376 45.0
6.6582539263 0.560938000679016 46.0
1.5305907357 0.354402899742126 47.0
176.40877081 0.163838267326355 48.0
170.13292236 -0.0199772007763386 49.0
0 0.250962793827057 50.0
80.597081809 0.0869506374001503 51.0
103.2898637 0.0631723403930664 52.0
91.014067281 -0.00526761217042804 53.0
91.014067281 0.261036157608032 54.0
91.014067281 0.0748873129487038 55.0
0 0.310978472232819 56.0
79.899388271 0.0608485341072083 57.0
206.91339759 0.341275304555893 58.0
236.20195143 0.259943217039108 59.0
22.878367008 -0.0122292432934046 60.0
0 0.0855804607272148 61.0
145.49087547 1.01498854160309 0.0
145.49087547 0.886719524860382 1.0
144.09169656 1.13966679573059 2.0
139.30259145 3.65639042854309 3.0
0 6.91458415985107 4.0
0 7.73227310180664 5.0
-62.2015273 0.722572386264801 6.0
75.9242971 1.04885601997375 7.0
55.991606537 0.954056799411774 8.0
19.932690564 0.955875515937805 9.0
102.41342441 1.14074850082397 10.0
88.489949586 0.0679114460945129 11.0
88.002167797 0.435049384832382 12.0
88.002167797 0.361735105514526 13.0
88.002167797 0.0499511361122131 14.0
0 0.137003496289253 15.0
0 0.222074434161186 16.0
0 1.36139047145844 17.0
62.582202439 1.44893550872803 18.0
-62.582202439 1.53509294986725 19.0
-62.582202439 0.319487273693085 20.0
-62.582202439 0.0259523503482342 21.0
85.300164661 0.644415497779846 22.0
150.62161552 1.01653349399567 23.0
116.32634554 0.189583361148834 24.0
22.890036247 0.251108735799789 25.0
235.92178019 1.05880177021027 26.0
235.92178019 2.54246997833252 27.0
23.284156188 1.80817651748657 28.0
259.20593637 1.4566251039505 29.0
259.20593637 1.83133387565613 30.0
46.367176325 1.22730720043182 31.0
9.5126699108 0.351575136184692 32.0
22.717962222 1.57302701473236 33.0
22.717962222 1.16682243347168 34.0
22.717962222 0.145526990294456 35.0
9.5162271647 0.0359234884381294 36.0
8.3255384689 0.321552395820618 37.0
0 0.115962475538254 38.0
0 0.241818934679031 39.0
-11.475777927 0.533049464225769 40.0
-25.404833077 0.299458384513855 41.0
5.1648857607 0.569383442401886 42.0
-33.464006936 0.615804612636566 43.0
102.50544954 0.796000421047211 44.0
102.50544954 0.0442963838577271 45.0
6.4938620307 0.654902279376984 46.0
2.6855047717 0.39445286989212 47.0
145.49087547 0.177673771977425 48.0
139.30259145 0.196686059236526 49.0
0 0.780059456825256 50.0
75.9242971 0.370414614677429 51.0
102.41342441 0.0243894942104816 52.0
88.002167797 0.0631318911910057 53.0
88.002167797 0.0545595474541187 54.0
88.002167797 0.0503544472157955 55.0
0 1.44042932987213 56.0
85.300164661 0.00375494640320539 57.0
235.92178019 1.05042397975922 58.0
259.20593637 0.637419760227203 59.0
22.717962222 0.0160928964614868 60.0
0 0.0825267434120178 61.0
194.02926817 1.03602349758148 0.0
194.02926817 0.575345158576965 1.0
192.64844627 1.23016595840454 2.0
187.22368928 4.02322149276733 3.0
0 6.19446134567261 4.0
0 5.43282890319824 5.0
-56.925578101 4.65843296051025 6.0
65.573220122 0.925679624080658 7.0
41.450819979 0.264418989419937 8.0
24.122400143 0.291067004203796 9.0
84.603330338 1.03516602516174 10.0
71.11733725 0.103351831436157 11.0
79.485988318 1.2033087015152 12.0
79.485988318 0.679476261138916 13.0
79.485988318 0.0508567914366722 14.0
0 0.0322247520089149 15.0
0 0.215192973613739 16.0
0 1.07123231887817 17.0
52.913424228 0.855587840080261 18.0
-52.913424228 0.665664792060852 19.0
-53.553209053 0.570669949054718 20.0
-53.620172363 -0.0173787325620651 21.0
73.354514442 2.28722548484802 22.0
131.57627462 1.84715974330902 23.0
100.88451532 0.450513303279877 24.0
21.824360772 0.111987181007862 25.0
204.93078906 0.986654758453369 26.0
204.93078906 1.34076797962189 27.0
24.100376761 1.17789399623871 28.0
229.03116582 0.590003192424774 29.0
229.03116582 1.70982491970062 30.0
46.095821207 1.13117146492004 31.0
9.1723954761 0.708464384078979 32.0
19.734342079 1.84564805030823 33.0
19.734342079 0.793371677398682 34.0
19.734342079 0.122805953025818 35.0
8.5174054793 0.0211562588810921 36.0
7.2823548109 0.384950578212738 37.0
0 0.0262242592871189 38.0
0 0.237708911299706 39.0
-11.891658867 0.391243129968643 40.0
-25.688848698 0.900266468524933 41.0
4.7199337406 0.558009266853333 42.0
-32.839294431 0.9292933344841 43.0
90.65530567 0.525470435619354 44.0
89.418578292 0.0969847142696381 45.0
6.2860013989 0.467225730419159 46.0
2.2400265151 0.394178301095963 47.0
194.02926817 0.086220495402813 48.0
187.22368928 0.00618107104673982 49.0
0 0.386634141206741 50.0
65.573220122 0.220212638378143 51.0
84.603330338 0.0226186513900757 52.0
79.485988318 0.0697065591812134 53.0
79.485988318 0.363035261631012 54.0
79.485988318 0.0861956477165222 55.0
0 2.54713988304138 56.0
73.354514442 0.0974587053060532 57.0
204.93078906 0.557393550872803 58.0
229.03116582 0.200816839933395 59.0
19.734342079 -0.0205469019711018 60.0
0 0.105088889598846 61.0
204.29410172 0.971299409866333 0.0
204.29410172 2.82595992088318 1.0
202.80926699 3.08714938163757 2.0
198.02016165 8.55262851715088 3.0
0 11.7102870941162 4.0
0 5.41492414474487 5.0
-48.862454801 0.0738078653812408 6.0
58.022223212 0.740823149681091 7.0
66.385244536 0.860015511512756 8.0
-8.3630213248 1.28323900699615 9.0
78.949466743 1.0483113527298 10.0
66.680089798 0.117587402462959 11.0
80.951138141 0.125583827495575 12.0
80.951138141 0.0925240740180016 13.0
80.951138141 0.155569180846214 14.0
0 0.164048224687576 15.0
0 0.163133442401886 16.0
0 1.6417224407196 17.0
54.765964967 1.32328569889069 18.0
-54.765964967 0.623284518718719 19.0
-54.765964967 0.854744851589203 20.0
-54.765964967 0.0721680596470833 21.0
75.924295673 0.89042866230011 22.0
142.14186286 0.797163784503937 23.0
113.12250323 0.384396374225616 24.0
22.536987215 0.210189327597618 25.0
218.06615853 0.43430233001709 26.0
218.06615853 1.23146772384644 27.0
22.386748036 1.71919274330139 28.0
240.45290657 0.755913078784943 29.0
240.45290657 1.8056298494339 30.0
46.642839502 1.23714220523834 31.0
9.1422332773 0.465883582830429 32.0
21.158330706 1.19130265712738 33.0
21.158330706 1.30102968215942 34.0
21.158330706 0.0586286708712578 35.0
9.9308048779 0.022910313680768 36.0
8.5206813372 0.386561810970306 37.0
0 -0.00433273985981941 38.0
0 0.142423897981644 39.0
-12.514955484 0.536767363548279 40.0
-24.860248527 0.745546519756317 41.0
5.6688222084 0.274861305952072 42.0
-33.528555601 0.290923118591309 43.0
73.080061375 0.715622901916504 44.0
73.080061375 0.0837411507964134 45.0
6.7468551204 0.693854451179504 46.0
2.5730355129 0.402412056922913 47.0
204.29410172 0.137941211462021 48.0
198.02016165 0.44150897860527 49.0
0 0.333572626113892 50.0
58.022223212 0.381281614303589 51.0
78.949466743 -0.00727987475693226 52.0
80.951138141 0.0238699242472649 53.0
80.951138141 0.0761861503124237 54.0
80.951138141 0.0549687370657921 55.0
0 5.43508625030518 56.0
75.924295673 -0.0216608345508575 57.0
218.06615853 0.114890091121197 58.0
240.45290657 0.256498605012894 59.0
21.158330706 1.1758878827095e-05 60.0
0 0.143178671598434 61.0
198.44166966 1.2850683927536 0.0
198.44166966 0.540923535823822 1.0
197.08481875 0.899198234081268 2.0
192.29571366 4.92610883712769 3.0
0 9.97281742095947 4.0
0 5.32566022872925 5.0
-77.874667395 0.550374329090118 6.0
91.101415014 0.837695598602295 7.0
65.196008292 2.21427607536316 8.0
25.905406722 2.2049617767334 9.0
88.597452296 1.22962081432343 10.0
76.315175007 0.0252896714955568 11.0
83.722706349 1.44426941871643 12.0
83.722706349 1.05383586883545 13.0
83.722706349 0.0476194843649864 14.0
0 0.033959798514843 15.0
0 0.17430879175663 16.0
0 1.37034702301025 17.0
59.115788846 1.46126174926758 18.0
-59.115788846 1.56422448158264 19.0
-59.115788846 0.65084981918335 20.0
-59.115788846 0.00832023471593857 21.0
78.960766285 1.72286903858185 22.0
135.57026732 2.25516104698181 23.0
104.98958588 0.414569944143295 24.0
22.930136984 0.0881580337882042 25.0
214.53103361 1.22787833213806 26.0
214.53103361 1.77897191047668 27.0
28.532783725 0.906042993068695 28.0
243.06381733 0.70746910572052 29.0
243.06381733 1.81351983547211 30.0
46.955036213 1.05148911476135 31.0
9.2344558411 0.426715165376663 32.0
19.844977439 1.82284688949585 33.0
19.844977439 1.25176954269409 34.0
19.844977439 0.116120219230652 35.0
9.3896542989 0.0252050496637821 36.0
8.0528675964 0.326966881752014 37.0
0 0.0302827470004559 38.0
0 0.144188597798347 39.0
-11.247281583 0.370710402727127 40.0
-23.524814112 0.706621468067169 41.0
5.2046703387 0.3515465259552 42.0
-31.150277642 0.462074339389801 43.0
108.16426123 0.5866978764534 44.0
108.16426123 0.0923806577920914 45.0
6.474244194 0.540322244167328 46.0
2.3459469603 0.326685070991516 47.0
198.44166966 0.152483642101288 48.0
192.29571366 0.213877722620964 49.0
0 0.388079136610031 50.0
91.101415014 0.303642332553864 51.0
88.597452296 0.179122403264046 52.0
83.722706349 0.0893999338150024 53.0
83.722706349 0.398396074771881 54.0
83.722706349 0.351780146360397 55.0
0 1.09773194789886 56.0
78.960766285 -0.00349306128919125 57.0
214.53103361 0.879264414310455 58.0
243.06381733 0.910735964775085 59.0
19.844977439 -0.0345632024109364 60.0
0 0.108654282987118 61.0
118.57360974 0.809613347053528 0.0
118.57360974 1.26536905765533 1.0
117.67047646 1.67500281333923 2.0
112.88140651 5.24393224716187 3.0
0 3.43827438354492 4.0
0 4.59075546264648 5.0
-45.796011368 2.4907341003418 6.0
58.496287345 1.4237060546875 7.0
57.750181905 0.685149431228638 8.0
0.74610543916 0.440696358680725 9.0
91.799603859 1.01572120189667 10.0
77.232676037 0.631007015705109 11.0
84.873785938 1.35944366455078 12.0
84.873785938 1.10653173923492 13.0
84.873785938 0.100947290658951 14.0
0 0.374528884887695 15.0
0 0.154090955853462 16.0
0 1.66167974472046 17.0
53.480146807 0.730894684791565 18.0
-53.480146807 0.56026691198349 19.0
-53.480146807 0.273435324430466 20.0
-53.480146807 0.0841882675886154 21.0
73.692145795 1.06639933586121 22.0
146.20587087 1.02181458473206 23.0
108.21128809 0.870460212230682 24.0
21.822213089 0.0690702199935913 25.0
219.89801666 0.532647728919983 26.0
219.89801666 1.41494333744049 27.0
24.993559393 1.68100357055664 28.0
244.89157605 0.399007976055145 29.0
244.89157605 1.81457102298737 30.0
46.832866547 1.33618295192719 31.0
9.4857479204 0.503068685531616 32.0
20.211998988 1.59237468242645 33.0
20.211998988 1.64208877086639 34.0
20.211998988 0.0807831734418869 35.0
8.5075811786 0.0322939567267895 36.0
7.3981083278 0.395664811134338 37.0
0 -0.00616645440459251 38.0
0 0.114570662379265 39.0
-12.479984121 0.753484666347504 40.0
-25.056770773 0.693061590194702 41.0
4.3996063817 0.31461027264595 42.0
-31.930723558 0.389348208904266 43.0
94.054280987 0.743187189102173 44.0
94.054280987 0.0821309685707092 45.0
7.0641751371 0.512160360813141 46.0
2.6971625389 0.442755699157715 47.0
118.57360974 0.0984978973865509 48.0
112.88140651 0.318072736263275 49.0
0 0.99415922164917 50.0
58.496287345 0.401958614587784 51.0
91.799603859 0.409696727991104 52.0
84.873785938 0.139281913638115 53.0
84.873785938 0.443205863237381 54.0
84.873785938 0.0300183687359095 55.0
0 0.855513334274292 56.0
73.692145795 0.0395483523607254 57.0
219.89801666 0.259453982114792 58.0
244.89157605 0.0881274938583374 59.0
20.211998988 0.0292409025132656 60.0
0 0.0589010268449783 61.0
187.77453258 0.99933135509491 0.0
187.77453258 0.880231857299805 1.0
186.81551868 1.13695025444031 2.0
182.02644873 3.46708178520203 3.0
0 7.0222225189209 4.0
0 3.20531058311462 5.0
-53.212326552 1.36321246623993 6.0
64.265947856 0.910586535930634 7.0
67.270071733 1.00527334213257 8.0
-3.0041238777 0.827426254749298 9.0
92.365308043 1.05702686309814 10.0
79.312645718 0.0578134059906006 11.0
85.553760765 0.988432824611664 12.0
85.553760765 0.240755990147591 13.0
85.553760765 0.0252790115773678 14.0
0 0.102579399943352 15.0
0 0.221988365054131 16.0
0 1.299684882164 17.0
54.884844661 1.17784380912781 18.0
-54.884844661 0.797622382640839 19.0
-54.884844661 0.879202127456665 20.0
-54.884844661 -0.0391678288578987 21.0
74.003334326 0.773214340209961 22.0
132.0035095 0.838935554027557 23.0
102.51218031 0.355883002281189 24.0
22.485684747 0.231144055724144 25.0
206.00684382 0.754628539085388 26.0
206.00684382 1.64109742641449 27.0
28.59980615 1.95630240440369 28.0
234.60664997 1.41740429401398 29.0
234.60664997 1.95261740684509 30.0
47.347690189 1.06907200813293 31.0
9.319619603 0.406160831451416 32.0
19.118489665 1.36745941638947 33.0
19.118489665 0.851295053958893 34.0
19.118489665 0.133609056472778 35.0
8.5956956059 0.0462649688124657 36.0
7.2966245311 0.313526958227158 37.0
0 0.0501395165920258 38.0
0 0.179139256477356 39.0
-10.770282237 0.499231696128845 40.0
-23.99367635 0.58909684419632 41.0
4.962430168 0.37121719121933 42.0
-31.205424656 0.363960027694702 43.0
85.700209185 0.65764331817627 44.0
85.700209185 0.0568981692194939 45.0
6.1973308096 0.61331707239151 46.0
2.2333029486 0.347634106874466 47.0
187.77453258 0.059639435261488 48.0
182.02644873 0.0856696888804436 49.0
0 0.430965632200241 50.0
64.265947856 0.248519986867905 51.0
92.365308043 0.158620983362198 52.0
85.553760765 0.10308950394392 53.0
85.553760765 0.224755421280861 54.0
85.553760765 0.102521762251854 55.0
0 2.00396251678467 56.0
74.003334326 0.121494919061661 57.0
206.00684382 0.0847211703658104 58.0
234.60664997 1.5097074508667 59.0
19.118489665 0.0116920005530119 60.0
0 0.102294035255909 61.0
164.60098408 0.835897266864777 0.0
164.60098408 1.53018760681152 1.0
163.26387047 2.29944777488708 2.0
158.47476519 4.12860298156738 3.0
0 8.17317390441895 4.0
0 4.74157762527466 5.0
-58.580730478 0.441233932971954 6.0
68.301738639 0.760686635971069 7.0
62.189251143 1.47126686573029 8.0
6.1124874966 1.5618976354599 9.0
77.97345271 0.978700220584869 10.0
64.640498101 0.048868652433157 11.0
75.601277141 0.870285153388977 12.0
75.601277141 0.374996095895767 13.0
75.601277141 0.101946115493774 14.0
0 0.058516401797533 15.0
0 0.148192137479782 16.0
0 1.10031819343567 17.0
53.389230863 1.19076418876648 18.0
-53.389230863 1.1769859790802 19.0
-53.389230863 0.673039376735687 20.0
-53.389230863 0.0446748547255993 21.0
74.097144913 1.31508672237396 22.0
141.58153865 0.906804382801056 23.0
106.64300686 0.68423718214035 24.0
21.42133641 0.0946491658687592 25.0
215.67868356 0.976992011070251 26.0
215.67868356 1.41617000102997 27.0
25.401898257 1.67462909221649 28.0
241.08058182 0.89798378944397 29.0
241.08058182 1.80353045463562 30.0
46.62940836 0.952098488807678 31.0
9.311219318 0.594239234924316 32.0
20.707914049 1.19451284408569 33.0
20.707914049 1.04410886764526 34.0
20.707914049 0.146848380565643 35.0
9.0758207603 0.0324466377496719 36.0
7.6700551349 0.418888241052628 37.0
0 0.020274568349123 38.0
0 0.0912868827581406 39.0
-13.193087233 0.69516795873642 40.0
-26.93588909 0.385038256645203 41.0
4.9776170981 0.502312421798706 42.0
-34.75900372 0.522648394107819 43.0
84.93895025 0.631244242191315 44.0
84.93895025 0.104193225502968 45.0
7.6250429663 0.634261250495911 46.0
3.0074020316 0.398280620574951 47.0
164.60098408 0.157923713326454 48.0
158.47476519 0.579572379589081 49.0
0 0.436789512634277 50.0
68.301738639 0.275958895683289 51.0
77.97345271 0.164640605449677 52.0
75.601277141 0.0548799075186253 53.0
75.601277141 0.0894621983170509 54.0
75.601277141 0.0422334596514702 55.0
0 5.28856134414673 56.0
74.097144913 0.0937989056110382 57.0
215.67868356 0.166039198637009 58.0
241.08058182 0.976640462875366 59.0
20.707914049 0.00123396702110767 60.0
0 0.0917754471302032 61.0
140.35427877 0.903278052806854 0.0
140.35427877 1.10790061950684 1.0
138.93540894 1.26833236217499 2.0
134.14630366 4.44099187850952 3.0
0 5.45168972015381 4.0
0 5.56707859039307 5.0
-67.96512411 0.606346905231476 6.0
77.072843435 0.53513365983963 7.0
43.072291049 1.00746870040894 8.0
34.000552386 0.943637847900391 9.0
79.993701856 1.01674592494965 10.0
67.239866767 0.154037609696388 11.0
76.725061039 0.605939924716949 12.0
76.725061039 0.831341505050659 13.0
76.725061039 0.126289486885071 14.0
0 0.104308523237705 15.0
0 0.172143921256065 16.0
0 1.32260680198669 17.0
52.713742783 1.8138382434845 18.0
-52.713742783 1.64818513393402 19.0
-52.713742783 0.282924503087997 20.0
-52.713742783 0.0317104086279869 21.0
74.383771122 1.44197309017181 22.0
144.2704997 0.762336015701294 23.0
113.40364389 0.728536188602448 24.0
21.139473111 0.103773102164268 25.0
218.65427083 0.787230789661407 26.0
218.65427083 2.0956130027771 27.0
26.362738805 1.75650000572205 28.0
245.01700963 0.550842881202698 29.0
245.01700963 1.81875371932983 30.0
46.358758638 1.15716135501862 31.0
9.3383368662 0.58096569776535 32.0
21.670028339 0.925815999507904 33.0
21.670028339 1.20004653930664 34.0
21.670028339 0.102333799004555 35.0
9.1610333582 0.0252029318362474 36.0
7.982954739 0.421025156974792 37.0
0 0.0451139099895954 38.0
0 0.330222606658936 39.0
-11.217366766 0.475833833217621 40.0
-24.446082249 0.596183896064758 41.0
5.34380363 0.417348623275757 42.0
-32.433364427 0.315201133489609 43.0
82.097130361 0.647480547428131 44.0
82.097130361 0.0810292586684227 45.0
6.9156456982 0.657452464103699 46.0
2.2286764068 0.457634925842285 47.0
140.35427877 0.0271754786372185 48.0
134.14630366 0.0366142094135284 49.0
0 0.596418619155884 50.0
77.072843435 0.197465509176254 51.0
79.993701856 0.345263689756393 52.0
76.725061039 0.0311940219253302 53.0
76.725061039 0.21054071187973 54.0
76.725061039 0.263838589191437 55.0
0 7.90524625778198 56.0
74.383771122 0.211410850286484 57.0
218.65427083 0.278087347745895 58.0
245.01700963 0.406170547008514 59.0
21.670028339 0.0245203301310539 60.0
0 0.0860894098877907 61.0
179.63037792 0.821803629398346 0.0
179.63037792 1.6010764837265 1.0
178.31365076 1.2167888879776 2.0
173.52454554 3.05760407447815 3.0
0 5.84911298751831 4.0
0 4.28312015533447 5.0
-61.620260897 0.995944619178772 6.0
72.63144856 1.15093755722046 7.0
53.854227074 0.703023135662079 8.0
18.777221487 0.419683218002319 9.0
101.83409404 0.865503430366516 10.0
88.256973916 0.213467046618462 11.0
88.716438407 0.306009918451309 12.0
88.716438407 0.458096623420715 13.0
88.716438407 0.0804405063390732 14.0
0 0.271194398403168 15.0
0 0.182432815432549 16.0
0 1.26911818981171 17.0
59.444370655 1.70320522785187 18.0
-59.444370655 1.734046459198 19.0
-59.444370655 0.258241891860962 20.0
-59.444370655 0.060020700097084 21.0
80.487274099 1.67309033870697 22.0
146.41482043 1.48215365409851 23.0
109.15437588 0.255849629640579 24.0
25.35643013 0.215854316949844 25.0
226.90209453 1.24803614616394 26.0
226.90209453 1.43019795417786 27.0
23.782888744 1.40801858901978 28.0
250.68498327 0.804340064525604 29.0
250.68498327 1.39805018901825 30.0
46.856580589 1.09801530838013 31.0
9.2243034265 0.399451613426208 32.0
21.042903444 1.45006597042084 33.0
21.042903444 1.1821756362915 34.0
21.042903444 0.133591040968895 35.0
9.3468154162 0.0287857968360186 36.0
7.9642404921 0.382123470306396 37.0
0 0.00928300991654396 38.0
0 0.196223646402359 39.0
-11.894283302 0.809072196483612 40.0
-25.219880069 0.479184180498123 41.0
5.4689062238 0.479484260082245 42.0
-33.463752083 0.532835006713867 43.0
98.419310672 0.896374225616455 44.0
98.419310672 0.10762570053339 45.0
6.6402196316 0.760541439056396 46.0
2.4231892975 0.397682100534439 47.0
179.63037792 0.161607503890991 48.0
173.52454554 0.0156493820250034 49.0
0 0.205811753869057 50.0
72.63144856 0.0999206304550171 51.0
101.83409404 0.278762638568878 52.0
88.716438407 0.0649038255214691 53.0
88.716438407 0.117385655641556 54.0
88.716438407 0.0561921447515488 55.0
0 1.62079691886902 56.0
80.487274099 0.146144807338715 57.0
226.90209453 0.213948324322701 58.0
250.68498327 0.251047492027283 59.0
21.042903444 -0.00853942986577749 60.0
0 0.11049523204565 61.0
165.29840002 0.981226444244385 0.0
165.29840002 0.584851145744324 1.0
164.16514492 1.15667831897736 2.0
159.37603965 3.792396068573 3.0
0 4.45791149139404 4.0
0 4.71502304077148 5.0
-69.626281213 2.1375138759613 6.0
78.616832725 1.17339730262756 7.0
62.63086703 0.686948418617249 8.0
15.985965695 0.620879530906677 9.0
97.605549904 1.00102150440216 10.0
83.781282697 0.216437369585037 11.0
81.106733252 0.685723900794983 12.0
81.106733252 0.485307455062866 13.0
81.106733252 0.138441026210785 14.0
0 0.255347430706024 15.0
0 0.144893273711205 16.0
0 1.7044163942337 17.0
59.415660808 1.33962178230286 18.0
-59.415660808 0.851988852024078 19.0
-59.415660808 0.970426440238953 20.0
-59.415660808 -0.0664493590593338 21.0
80.326297192 1.36691641807556 22.0
153.72713414 1.13631546497345 23.0
118.20961852 0.500331282615662 24.0
22.649740086 0.172036871314049 25.0
234.05343134 0.842461824417114 26.0
234.05343134 1.2497763633728 27.0
26.829395291 1.28095698356628 28.0
260.88282663 0.59331738948822 29.0
260.88282663 1.53649210929871 30.0
47.309330478 1.1034140586853 31.0
9.4145777291 0.327040433883667 32.0
20.910636384 1.7325975894928 33.0
20.910636384 1.01803183555603 34.0
20.910636384 0.121428668498993 35.0
9.45509927 0.02485067024827 36.0
8.2412799594 0.419801712036133 37.0
0 0.0212864316999912 38.0
0 0.0848695188760757 39.0
-12.796363143 0.980319142341614 40.0
-26.511752444 0.226821348071098 41.0
5.2350720491 0.572335541248322 42.0
-34.44524044 0.63521671295166 43.0
89.244649728 0.628761827945709 44.0
89.244649728 0.112368777394295 45.0
6.6305171329 0.621883809566498 46.0
2.6360153017 0.412397533655167 47.0
165.29840002 0.0859877020120621 48.0
159.37603965 0.245036631822586 49.0
0 0.186558991670609 50.0
78.616832725 0.109840929508209 51.0
97.605549904 0.26028123497963 52.0
81.106733252 0.0739360824227333 53.0
81.106733252 0.0863127037882805 54.0
81.106733252 0.089451476931572 55.0
0 3.69670629501343 56.0
80.326297192 0.0845642015337944 57.0
234.05343134 0.61957848072052 58.0
260.88282663 0.699842095375061 59.0
20.910636384 -0.0100787039846182 60.0
0 0.103819750249386 61.0
135.90130816 0.771541655063629 0.0
135.90130816 1.38621509075165 1.0
134.43440591 1.66305100917816 2.0
129.64530065 5.15166568756104 3.0
0 6.16683959960938 4.0
0 4.96156120300293 5.0
-60.203403667 0.688403487205505 6.0
73.155982052 0.706435859203339 7.0
57.190266017 0.820548474788666 8.0
15.965716035 0.96759819984436 9.0
94.000981868 0.743685722351074 10.0
80.661161902 0.0780563652515411 11.0
84.132048838 0.675823092460632 12.0
84.132048838 0.200527906417847 13.0
84.132048838 0.0602047219872475 14.0
0 0.00765444710850716 15.0
0 0.224456787109375 16.0
0 1.0004495382309 17.0
57.917340376 0.929426252841949 18.0
-57.917340376 0.735349416732788 19.0
-57.917340376 0.357698500156403 20.0
-57.917340376 0.0127234570682049 21.0
79.112992823 1.65264296531677 22.0
154.6653547 1.18179535865784 23.0
119.26828592 0.335104554891586 24.0
24.498178685 0.363573342561722 25.0
233.77834752 0.747646272182465 26.0
233.77834752 1.57592976093292 27.0
25.329268416 1.35974812507629 28.0
259.10761594 0.965920805931091 29.0
259.10761594 1.97229647636414 30.0
46.261705761 1.00099587440491 31.0
9.3235534975 0.59908002614975 32.0
21.195652447 1.58980786800385 33.0
21.195652447 0.976607799530029 34.0
21.195652447 0.133644327521324 35.0
9.1889157778 0.0268598571419716 36.0
7.9307769197 0.420724391937256 37.0
0 0.090956874191761 38.0
0 0.172725051641464 39.0
-12.26904313 0.786906003952026 40.0
-24.914938809 0.430922836065292 41.0
4.9150291762 0.41853392124176 42.0
-32.726734069 0.310226947069168 43.0
88.040405822 0.742443084716797 44.0
88.040405822 0.0997964441776276 45.0
7.6179288945 0.553469777107239 46.0
2.7745259855 0.327838927507401 47.0
135.90130816 0.14216648042202 48.0
129.64530065 0.166567206382751 49.0
0 0.627801835536957 50.0
73.155982052 0.0483630523085594 51.0
94.000981868 0.117047019302845 52.0
84.132048838 0.0531421303749084 53.0
84.132048838 0.0395468361675739 54.0
84.132048838 0.0753303915262222 55.0
0 2.77985692024231 56.0
79.112992823 0.208928346633911 57.0
233.77834752 0.463070452213287 58.0
259.10761594 0.908502042293549 59.0
21.195652447 -0.00472063384950161 60.0
0 0.0999726057052612 61.0
176.45685117 0.791757345199585 0.0
176.45685117 2.47074103355408 1.0
175.10003061 2.05024719238281 2.0
170.3109253 2.60925936698914 3.0
0 4.32603025436401 4.0
0 3.20701241493225 5.0
-64.038533849 0.223820328712463 6.0
72.666940171 0.593824505805969 7.0
53.023385588 1.70203971862793 8.0
19.643554583 1.77119445800781 9.0
86.397716365 0.845947265625 10.0
73.859437273 0.0194748491048813 11.0
82.789737102 0.8500856757164 12.0
82.789737102 0.854746460914612 13.0
82.789737102 0.0873807147145271 14.0
0 0.225684747099876 15.0
0 0.145405635237694 16.0
0 1.82760262489319 17.0
49.498040347 1.65784120559692 18.0
-49.498040347 1.76184022426605 19.0
-49.498040347 0.383139789104462 20.0
-49.498040347 0.0182258263230324 21.0
71.335372639 1.18985664844513 22.0
136.4475324 0.889687120914459 23.0
107.47935996 0.766523361206055 24.0
23.174069511 0.153747230768204 25.0
207.78290504 1.06325256824493 26.0
207.78290504 1.13887703418732 27.0
22.559050631 1.36822152137756 28.0
230.34195567 0.357453405857086 29.0
230.34195567 2.16791462898254 30.0
47.362962095 1.29163241386414 31.0
9.2603803927 0.630560696125031 32.0
21.837332292 1.48367631435394 33.0
21.837332292 0.724687159061432 34.0
21.837332292 0.127011716365814 35.0
8.91341321 0.0221822615712881 36.0
7.6953384523 0.318080306053162 37.0
0 0.0577031075954437 38.0
0 0.169833987951279 39.0
-11.864889073 0.882078468799591 40.0
-24.948990913 0.675820589065552 41.0
4.7128168882 0.295025110244751 42.0
-31.961432775 0.432668387889862 43.0
85.605693422 0.693169414997101 44.0
85.605693422 0.0445259809494019 45.0
6.0394708524 0.659225821495056 46.0
2.2494933886 0.312690407037735 47.0
176.45685117 0.0747990012168884 48.0
170.3109253 0.132590994238853 49.0
0 0.157813847064972 50.0
72.666940171 0.18650908768177 51.0
86.397716365 0.169692546129227 52.0
82.789737102 0.0264201629906893 53.0
82.789737102 0.103577867150307 54.0
82.789737102 0.148762226104736 55.0
0 0.358385860919952 56.0
71.335372639 0.0687379091978073 57.0
207.78290504 0.524628281593323 58.0
230.34195567 0.634078085422516 59.0
21.837332292 -0.0300682596862316 60.0
0 0.0921911150217056 61.0
168.23779491 0.898356914520264 0.0
168.23779491 0.740956902503967 1.0
166.6976292 1.40908992290497 2.0
161.90852385 3.48301863670349 3.0
0 3.66253614425659 4.0
0 4.6092095375061 5.0
-52.269879824 0.264643341302872 6.0
62.419723139 0.567323386669159 7.0
35.171138476 0.507766246795654 8.0
27.248584663 0.531283020973206 9.0
79.039086758 0.9665407538414 10.0
64.879648223 0.174069911241531 11.0
75.803978085 1.03029918670654 12.0
75.803978085 0.308807700872421 13.0
75.803978085 0.034973967820406 14.0
0 0.029871454462409 15.0
0 0.209801271557808 16.0
0 1.19765949249268 17.0
52.506821002 1.9100044965744 18.0
-52.506821002 1.89035475254059 19.0
-52.506821002 0.361531376838684 20.0
-52.506821002 0.0861392021179199 21.0
73.048110614 1.271448969841 22.0
142.1534915 1.25575375556946 23.0
107.7951143 0.614226996898651 24.0
11.236247314 0.0831014364957809 25.0
215.20160211 1.06280434131622 26.0
215.20160211 1.66810536384583 27.0
25.261487453 1.25784182548523 28.0
240.46308957 0.510793745517731 29.0
240.46308957 1.63594603538513 30.0
46.886002416 0.991611480712891 31.0
9.3265410969 0.37267941236496 32.0
20.541289612 1.23245203495026 33.0
20.541289612 0.752328336238861 34.0
20.541289612 0.125350430607796 35.0
8.3450079065 0.0495154559612274 36.0
7.2162904214 0.367287755012512 37.0
0 0.0508030503988266 38.0
0 0.275329291820526 39.0
-11.118715824 0.532828629016876 40.0
-24.09979193 0.515767991542816 41.0
4.741293205 0.467130452394485 42.0
-31.813959747 0.745773077011108 43.0
75.69210124 0.452618777751923 44.0
75.69210124 0.121769055724144 45.0
6.5600787515 0.520955801010132 46.0
1.4856450193 0.418206423521042 47.0
168.23779491 0.186291456222534 48.0
161.90852385 0.326814144849777 49.0
0 0.186012774705887 50.0
62.419723139 0.115737497806549 51.0
79.039086758 0.389787942171097 52.0
75.803978085 0.104665018618107 53.0
75.803978085 0.0942714139819145 54.0
75.803978085 0.0553511828184128 55.0
0 5.50338697433472 56.0
73.048110614 -0.0295584741979837 57.0
215.20160211 0.534345507621765 58.0
240.46308957 0.687387704849243 59.0
20.541289612 0.0160717107355595 60.0
0 0.100543431937695 61.0
118.23909838 0.943190455436707 0.0
118.23909838 0.670798540115356 1.0
117.26849355 1.23463010787964 2.0
112.4794236 4.81660079956055 3.0
0 9.02137851715088 4.0
0 6.56545829772949 5.0
-67.221541603 4.80739164352417 6.0
76.169389621 0.895134329795837 7.0
64.994406746 0.833913862705231 8.0
11.174982875 1.09183073043823 9.0
81.329440325 1.01333844661713 10.0
68.588116517 0.0356085300445557 11.0
73.681734744 0.173865899443626 12.0
73.681734744 0.583444178104401 13.0
73.681734744 0.247085973620415 14.0
0 0.296787679195404 15.0
0 0.210491567850113 16.0
0 1.16794192790985 17.0
48.45367305 1.24547958374023 18.0
-48.45367305 1.27197456359863 19.0
-48.45367305 0.403713583946228 20.0
-48.45367305 0.0570294670760632 21.0
68.563463256 1.13482701778412 22.0
138.8101624 1.1272931098938 23.0
101.20643856 0.73911190032959 24.0
19.714570992 0.0907885134220123 25.0
207.37362566 0.485042661428452 26.0
207.37362566 2.65994811058044 27.0
22.592965037 2.51448535919189 28.0
229.96659069 1.41836512088776 29.0
229.96659069 2.14349961280823 30.0
46.753495129 1.22806394100189 31.0
9.4906979359 0.501304447650909 32.0
20.109790206 1.47665679454803 33.0
20.109790206 0.709418714046478 34.0
20.109790206 0.13006754219532 35.0
9.2952248749 0.0446606948971748 36.0
8.1062511848 0.42531543970108 37.0
0 -0.000613130629062653 38.0
0 0.185251638293266 39.0
-13.582375199 0.357671052217484 40.0
-26.483369163 0.781112909317017 41.0
4.9266095299 0.413396298885345 42.0
-33.587623594 0.455693066120148 43.0
85.534893918 0.825749814510345 44.0
85.534893918 0.0178180187940598 45.0
6.0411818124 0.626736342906952 46.0
1.271088499 0.416208267211914 47.0
118.23909838 0.144557222723961 48.0
112.4794236 0.556095361709595 49.0
0 0.890501976013184 50.0
76.169389621 0.42512646317482 51.0
81.329440325 0.185888841748238 52.0
73.681734744 0.011598733253777 53.0
73.681734744 0.0670409500598907 54.0
73.681734744 0.387677103281021 55.0
0 9.86588382720947 56.0
68.563463256 0.119587182998657 57.0
207.37362566 0.214388102293015 58.0
229.96659069 1.03119206428528 59.0
20.109790206 0.0245658438652754 60.0
0 0.10166322439909 61.0
128.65979821 0.929358243942261 0.0
128.65979821 1.13891267776489 1.0
127.61211513 2.36241936683655 2.0
122.82300979 7.21961307525635 3.0
0 2.76726627349854 4.0
0 3.12969183921814 5.0
-67.18638484 1.93441307544708 6.0
80.383679511 0.987610161304474 7.0
56.046226867 0.56942480802536 8.0
24.337452644 0.691852807998657 9.0
79.663961857 1.0626984834671 10.0
67.166952448 0.344281733036041 11.0
73.874609823 0.597693622112274 12.0
73.874609823 0.10824117064476 13.0
73.874609823 0.084740474820137 14.0
0 0.0324106179177761 15.0
0 0.207318097352982 16.0
0 0.672987878322601 17.0
53.034754571 1.29297971725464 18.0
-53.034754571 1.35690498352051 19.0
-53.034754571 0.873576045036316 20.0
-53.034754571 -0.0817871391773224 21.0
75.680329112 2.07665300369263 22.0
144.11048889 2.4506995677948 23.0
111.93504592 0.350740402936935 24.0
24.456933939 0.149386644363403 25.0
219.790818 1.27359306812286 26.0
219.790818 1.94757497310638 27.0
27.733836362 2.97467803955078 28.0
247.52465436 1.58176684379578 29.0
247.52465436 2.22464561462402 30.0
46.0761385 1.07097899913788 31.0
9.2737456825 0.317622691392899 32.0
22.645574542 2.1585967540741 33.0
22.645574542 0.846202194690704 34.0
22.645574542 0.0799504667520523 35.0
9.9362240095 0.0326114483177662 36.0
8.5966806983 0.400434255599976 37.0
0 0.0298514664173126 38.0
0 0.0700926631689072 39.0
-10.872866581 0.726774990558624 40.0
-24.589191176 0.690560758113861 41.0
5.387094143 0.564809143543243 42.0
-32.231313424 0.544716954231262 43.0
77.960732335 0.542076289653778 44.0
77.960732335 0.11917332559824 45.0
5.8664558263 0.718608856201172 46.0
2.2500536567 0.39984866976738 47.0
128.65979821 0.0601775906980038 48.0
122.82300979 0.786224782466888 49.0
0 0.0720777288079262 50.0
80.383679511 0.104747220873833 51.0
79.663961857 0.552993297576904 52.0
73.874609823 0.123419165611267 53.0
73.874609823 0.0805125534534454 54.0
73.874609823 0.0536581054329872 55.0
0 7.3273138999939 56.0
75.680329112 0.0916526466608047 57.0
219.790818 0.403662949800491 58.0
247.52465436 1.18466758728027 59.0
22.645574542 -0.00205858610570431 60.0
0 0.0910667851567268 61.0
163.05578897 0.598612248897552 0.0
163.05578897 1.78767049312592 1.0
162.00935308 2.00510334968567 2.0
156.54468228 5.94028902053833 3.0
4.5430354835 4.80419683456421 4.0
4.5430354835 5.61553716659546 5.0
-72.365643262 7.57576179504395 6.0
84.711286984 1.29693055152893 7.0
50.232728806 0.359578102827072 8.0
34.478558179 0.333569377660751 9.0
85.441475359 1.05067908763885 10.0
72.298514334 0.47451177239418 11.0
79.61408865 0.183045864105225 12.0
79.61408865 0.777715504169464 13.0
79.61408865 0.130309954285622 14.0
0 0.134114950895309 15.0
0 0.151759549975395 16.0
0 1.60649585723877 17.0
54.839221179 1.21591794490814 18.0
-54.839221179 1.1472772359848 19.0
-54.839221179 0.223880603909492 20.0
-54.839221179 -0.0123584046959877 21.0
73.826488529 1.59014022350311 22.0
149.77973355 1.43008124828339 23.0
115.93150242 0.482380241155624 24.0
24.015063047 0.224959760904312 25.0
223.60622208 1.92933821678162 26.0
223.60622208 2.8929226398468 27.0
24.806022343 2.36870694160461 28.0
248.41224442 1.03683924674988 29.0
248.41224442 1.88155877590179 30.0
47.102642479 1.28707265853882 31.0
9.4832793205 0.446846216917038 32.0
18.987267351 1.5738879442215 33.0
18.987267351 1.16379106044769 34.0
18.987267351 0.0829069241881371 35.0
9.0270028345 0.0254383850842714 36.0
7.7349408043 0.405495166778564 37.0
0 0.0212500356137753 38.0
0 0.335911691188812 39.0
-11.630814718 0.604577302932739 40.0
-23.506468292 0.800402581691742 41.0
5.6228012919 0.481677204370499 42.0
-31.883779562 0.451160848140717 43.0
77.425923687 0.559809327125549 44.0
77.425923687 0.121794939041138 45.0
6.6509465875 0.558631420135498 46.0
2.6770306263 0.357751697301865 47.0
163.05578897 0.17540916800499 48.0
156.54468228 0.262838125228882 49.0
4.5430354835 0.244608223438263 50.0
84.711286984 0.267558068037033 51.0
85.441475359 0.660711765289307 52.0
79.61408865 0.0302613116800785 53.0
79.61408865 0.161284267902374 54.0
79.61408865 0.149017184972763 55.0
0 3.68217730522156 56.0
73.826488529 0.0668428018689156 57.0
223.60622208 1.41195344924927 58.0
248.41224442 0.253329157829285 59.0
18.987267351 0.00408292375504971 60.0
0 0.100297704339027 61.0
186.89013254 0.908641219139099 0.0
186.89013254 0.588847637176514 1.0
185.43019329 0.873651564121246 2.0
180.64112334 4.86755323410034 3.0
0 8.76339435577393 4.0
0 5.89687204360962 5.0
-58.156717052 2.79966402053833 6.0
70.189693955 1.46249067783356 7.0
60.614788615 0.446949601173401 8.0
9.5749053401 0.409181207418442 9.0
88.283762948 0.932735860347748 10.0
75.210763944 0.129578620195389 11.0
80.739365451 0.762908935546875 12.0
80.739365451 0.332509398460388 13.0
80.739365451 0.0808480456471443 14.0
0 0.0799199044704437 15.0
0 0.166354417800903 16.0
0 0.925786793231964 17.0
52.900823808 1.04434382915497 18.0
-52.900823808 0.968463957309723 19.0
-52.900823808 0.286746561527252 20.0
-52.900823808 0.0455225668847561 21.0
73.767767692 1.72853195667267 22.0
141.46189332 1.59684610366821 23.0
110.93964821 0.727852642536163 24.0
23.948346173 0.160391286015511 25.0
215.22966101 1.02261102199554 26.0
215.22966101 1.29273927211761 27.0
24.687641801 0.949247360229492 28.0
239.91730281 0.543665885925293 29.0
239.91730281 1.40938651561737 30.0
47.147975317 1.04109179973602 31.0
9.2434522172 0.520728886127472 32.0
20.866943884 1.37998855113983 33.0
20.866943884 0.632419466972351 34.0
20.866943884 0.143806248903275 35.0
9.5968826512 0.0440340414643288 36.0
8.2474499839 0.384724497795105 37.0
0 0.0415036976337433 38.0
0 0.193384230136871 39.0
-11.320206007 0.322510719299316 40.0
-23.407613905 0.849933803081512 41.0
4.7503282587 0.495350301265717 42.0
-31.00136054 0.576795160770416 43.0
91.30700198 0.751016497612 44.0
91.30700198 0.0922968611121178 45.0
7.2643458478 0.488371968269348 46.0
2.5654632591 0.346669524908066 47.0
186.89013254 0.244573935866356 48.0
180.64112334 0.0332685522735119 49.0
0 0.38572970032692 50.0
70.189693955 0.139970451593399 51.0
88.283762948 0.258222073316574 52.0
80.739365451 0.0527104958891869 53.0
80.739365451 0.0719882622361183 54.0
80.739365451 0.0625214502215385 55.0
0 5.02042961120605 56.0
73.767767692 0.0728137195110321 57.0
215.22966101 0.312237054109573 58.0
239.91730281 0.5855792760849 59.0
20.866943884 -0.000735672190785408 60.0
0 0.0980293452739716 61.0
183.8269926 0.874637603759766 0.0
183.8269926 1.98260986804962 1.0
182.37484071 2.19461226463318 2.0
177.58575716 5.65724277496338 3.0
0 5.68055438995361 4.0
0 5.76396369934082 5.0
-63.268148117 5.56699562072754 6.0
73.213654569 0.764061570167542 7.0
44.67437762 0.740052402019501 8.0
28.539276949 0.709704697132111 9.0
87.776594584 0.945366442203522 10.0
76.508050455 0.0560503117740154 11.0
80.117380204 0.669725835323334 12.0
80.117380204 0.496246099472046 13.0
80.117380204 0.165886610746384 14.0
0 0.293485283851624 15.0
0 0.170734971761703 16.0
0 1.77386045455933 17.0
55.446379828 0.700224697589874 18.0
-55.446379828 0.996877431869507 19.0
-55.446379828 0.260491132736206 20.0
-55.446379828 0.013305589556694 21.0
77.953955833 1.77507066726685 22.0
141.90704869 1.57824003696442 23.0
103.99659818 0.242710158228874 24.0
22.693423536 0.188088640570641 25.0
219.86100452 0.900658786296844 26.0
219.86100452 2.08454608917236 27.0
24.950030881 2.68376755714417 28.0
244.8110354 1.46073615550995 29.0
244.8110354 2.12337183952332 30.0
46.262387572 1.18067622184753 31.0
9.4667188813 0.320851027965546 32.0
22.507576004 1.35524821281433 33.0
22.507576004 1.20651364326477 34.0
22.507576004 0.123217895627022 35.0
9.4519266479 0.0275257844477892 36.0
8.3386050255 0.361825197935104 37.0
0 0.0937138348817825 38.0
0 0.215048760175705 39.0
-12.673645297 0.394138693809509 40.0
-25.704572931 0.892532527446747 41.0
5.1475157319 0.582948625087738 42.0
-33.415434197 0.420026510953903 43.0
83.182385685 0.761208891868591 44.0
83.182385685 0.0856381058692932 45.0
7.4560297119 0.549048781394958 46.0
2.9441272771 0.339862734079361 47.0
183.8269926 0.15569880604744 48.0
177.58575716 0.309767454862595 49.0
0 0.444298952817917 50.0
73.213654569 0.161004096269608 51.0
87.776594584 0.205540418624878 52.0
80.117380204 0.133604630827904 53.0
80.117380204 -0.00164831569418311 54.0
80.117380204 0.214635193347931 55.0
0 4.03308534622192 56.0
77.953955833 0.255380809307098 57.0
219.86100452 0.529725134372711 58.0
244.8110354 0.877484023571014 59.0
22.507576004 -0.0080078262835741 60.0
0 0.0773448050022125 61.0
141.75814029 1.19479835033417 0.0
141.75814029 0.975217223167419 1.0
140.63794789 1.44356441497803 2.0
135.84884263 3.33001089096069 3.0
0 3.61800312995911 4.0
0 2.86055827140808 5.0
-52.256503044 0.511073768138885 6.0
65.403896501 0.77064973115921 7.0
61.41678781 0.97379595041275 8.0
3.9871086904 0.896058320999146 9.0
90.442925999 1.26622211933136 10.0
76.090831498 0.247417420148849 11.0
85.389756398 0.360987693071365 12.0
85.389756398 0.750221312046051 13.0
85.389756398 0.0699901208281517 14.0
0 0.163227662444115 15.0
0 0.208889409899712 16.0
0 0.826062679290771 17.0
60.959286373 1.78502488136292 18.0
-60.959286373 1.61312210559845 19.0
-60.959286373 0.370966643095016 20.0
-60.959286373 -0.069329708814621 21.0
82.38113064 1.77482223510742 22.0
149.25207721 1.29963457584381 23.0
116.38013368 0.554380238056183 24.0
25.016194025 0.276861041784286 25.0
231.63320785 1.30516874790192 26.0
231.63320785 1.94194269180298 27.0
27.528252206 2.00029921531677 28.0
259.16146005 0.768633782863617 29.0
259.16146005 1.40296149253845 30.0
46.449431424 0.962997019290924 31.0
9.3068308797 0.340747863054276 32.0
21.421844267 1.47947037220001 33.0
21.421844267 1.06856191158295 34.0
21.421844267 0.122707948088646 35.0
9.5911212874 0.0345563478767872 36.0
8.2752905875 0.33558201789856 37.0
0 0.022600632160902 38.0
0 0.144339263439178 39.0
-11.884533869 0.377081722021103 40.0
-25.464666927 0.862760782241821 41.0
5.034840478 0.44279733300209 42.0
-33.202086351 0.583019196987152 43.0
89.771498349 0.898597598075867 44.0
89.771498349 0.0770185887813568 45.0
6.8296552943 0.618265569210052 46.0
2.7547315658 0.309154033660889 47.0
141.75814029 0.133084014058113 48.0
135.84884263 0.112472027540207 49.0
0 0.0528609417378902 50.0
65.403896501 0.287096589803696 51.0
90.442925999 0.0226330067962408 52.0
85.389756398 0.0304119624197483 53.0
85.389756398 0.153693646192551 54.0
85.389756398 0.227815553545952 55.0
0 5.1152811050415 56.0
82.38113064 0.296229779720306 57.0
231.63320785 0.712341248989105 58.0
259.16146005 0.280058592557907 59.0
21.421844267 0.0328988470137119 60.0
0 0.100961908698082 61.0
188.44159689 1.04665768146515 0.0
188.44159689 1.11408579349518 1.0
186.94436338 1.24752247333527 2.0
182.15525821 3.90138149261475 3.0
0 4.93653011322021 4.0
0 6.19353866577148 5.0
-68.701412333 1.53162813186646 6.0
78.606711742 1.21155619621277 7.0
64.575467893 0.650838136672974 8.0
14.031243849 0.574476957321167 9.0
87.338979257 1.1874463558197 10.0
73.383462414 0.263034403324127 11.0
79.045091377 0.272354274988174 12.0
79.045091377 0.179841458797455 13.0
79.045091377 0.0957639738917351 14.0
0 0.183805286884308 15.0
0 0.204385697841644 16.0
0 1.29696524143219 17.0
61.037236686 1.51239585876465 18.0
-61.037236686 1.84257805347443 19.0
-61.037236686 0.383890122175217 20.0
-61.037236686 0.000742953270673752 21.0
83.475504307 1.75202083587646 22.0
135.76964462 1.53601861000061 23.0
103.59299734 0.265742272138596 24.0
23.461368676 0.215361297130585 25.0
219.24514892 1.19439435005188 26.0
219.24514892 1.89436614513397 27.0
25.538708191 2.38094234466553 28.0
244.78385711 1.14858615398407 29.0
244.78385711 1.7042533159256 30.0
47.283145424 1.08425259590149 31.0
9.3957472395 0.471983253955841 32.0
22.438267621 1.72974252700806 33.0
22.438267621 0.913781046867371 34.0
22.438267621 0.118357561528683 35.0
10.421489587 0.0353190824389458 36.0
9.1295423585 0.384316235780716 37.0
0 0.01924092695117 38.0
0 0.289036810398102 39.0
-11.67841395 0.496616065502167 40.0
-23.820633922 0.636253893375397 41.0
4.88716298 0.440186023712158 42.0
-31.188792453 0.585600972175598 43.0
86.073969807 0.63800585269928 44.0
86.073969807 0.101056635379791 45.0
7.168880119 0.515611886978149 46.0
2.9346231629 0.3672776222229 47.0
188.44159689 0.163203686475754 48.0
182.15525821 0.118900820612907 49.0
0 0.349854648113251 50.0
78.606711742 0.21434585750103 51.0
87.338979257 0.326061576604843 52.0
79.045091377 0.0500399693846703 53.0
79.045091377 0.0673065111041069 54.0
79.045091377 0.135207042098045 55.0
0 5.39008951187134 56.0
83.475504307 0.0714407712221146 57.0
219.24514892 0.425298005342484 58.0
244.78385711 0.290713310241699 59.0
22.438267621 0.0118869319558144 60.0
0 0.0828690677881241 61.0
199.56215579 0.93052464723587 0.0
199.56215579 0.915569067001343 1.0
198.00891336 1.29007160663605 2.0
196.94387668 3.9595959186554 3.0
0 3.72639179229736 4.0
0 2.49496817588806 5.0
-68.89156837 0.332302331924438 6.0
82.286336847 1.05807673931122 7.0
45.94597129 1.04597854614258 8.0
36.340365557 0.639999508857727 9.0
58.017539208 1.05488216876984 10.0
43.725187362 0.337592214345932 11.0
42.171278611 0.118672400712967 12.0
42.171278611 0.451941788196564 13.0
42.171278611 0.0139268804341555 14.0
29.516487592 0.00227757822722197 15.0
29.516487592 0.192662805318832 16.0
0 0.969853579998016 17.0
54.271211778 0.925116360187531 18.0
-54.271211778 0.455765098333359 19.0
-55.04346689 0.755075871944427 20.0
-55.11792426 0.00573800131678581 21.0
76.32151188 1.61796081066132 22.0
143.87467206 2.26615691184998 23.0
108.85066269 0.267772823572159 24.0
21.454179716 0.154344230890274 25.0
220.19618394 1.51297926902771 26.0
220.19618394 1.36616694927216 27.0
30.096619042 1.39720118045807 28.0
250.29280299 0.911562561988831 29.0
250.29280299 1.40510165691376 30.0
46.893641659 1.00960218906403 31.0
9.491577563 0.337657153606415 32.0
21.20358762 1.51336467266083 33.0
21.20358762 1.07839465141296 34.0
21.20358762 0.124374888837337 35.0
10.350729363 0.0235064439475536 36.0
8.9597317337 0.386377513408661 37.0
0 0.0252577662467957 38.0
0 0.192367687821388 39.0
-11.353137729 0.41901233792305 40.0
-23.492640442 0.53886878490448 41.0
5.1975613381 0.35769522190094 42.0
-31.148640851 0.415101259946823 43.0
97.940043367 0.538909792900085 44.0
97.940043367 0.0730180665850639 45.0
6.8218413416 0.631565630435944 46.0
2.4977998871 0.399479895830154 47.0
199.56215579 0.0704242363572121 48.0
196.94387668 0.101609662175179 49.0
0 0.0617361851036549 50.0
82.286336847 -0.00449502468109131 51.0
58.017539208 0.278500437736511 52.0
42.171278611 0.0465585589408875 53.0
42.171278611 0.0716883689165115 54.0
42.171278611 0.205385103821754 55.0
0 1.76456785202026 56.0
76.32151188 0.0169038251042366 57.0
220.19618394 0.653791189193726 58.0
250.29280299 0.181545987725258 59.0
21.20358762 -0.0213528480380774 60.0
0 0.0959408953785896 61.0
207.35679288 0.728789269924164 0.0
207.35679288 1.08002972602844 1.0
205.94043086 2.10847210884094 2.0
200.47576012 5.90120553970337 3.0
31.933599144 3.52045202255249 4.0
31.933599144 2.86157298088074 5.0
-46.584783993 0.254152178764343 6.0
57.748521661 0.879788935184479 7.0
33.516335849 0.408757299184799 8.0
24.232185813 0.434107780456543 9.0
63.639466031 0.954875409603119 10.0
51.318815172 0.468346655368805 11.0
59.450942311 0.694977343082428 12.0
59.450942311 0.154312714934349 13.0
59.450942311 0.121360272169113 14.0
18.564394511 0.132201686501503 15.0
18.564394511 0.207220286130905 16.0
0 1.27363967895508 17.0
54.75185584 1.96547269821167 18.0
-54.75185584 2.1228506565094 19.0
-54.75185584 0.568917334079742 20.0
-54.75185584 -0.0617712065577507 21.0
75.671575155 2.86042237281799 22.0
143.96904672 2.81468462944031 23.0
107.352331 0.228912368416786 24.0
23.985604022 0.29803603887558 25.0
219.64062187 1.62043714523315 26.0
219.64062187 2.51112008094788 27.0
23.649410995 2.57952690124512 28.0
243.29003287 1.31060838699341 29.0
243.29003287 1.78289687633514 30.0
46.083311888 0.968645870685577 31.0
9.306016149 0.388102889060974 32.0
20.919719316 1.19605267047882 33.0
20.919719316 0.909942209720612 34.0
20.919719316 0.125796139240265 35.0
9.1829737342 0.0256720539182425 36.0
7.927038785 0.422115683555603 37.0
0 0.0202152132987976 38.0
0 0.0749368369579315 39.0
-11.415485823 0.493741512298584 40.0
-25.178056612 0.550728678703308 41.0
4.6534874258 0.591512441635132 42.0
-32.239588695 0.666638970375061 43.0
90.768637766 0.716998159885406 44.0
90.768637766 0.069824256002903 45.0
7.0473758339 0.407466679811478 46.0
2.6577055754 0.334940910339355 47.0
207.35679288 0.164420515298843 48.0
200.47576012 0.179547816514969 49.0
31.933599144 0.063069224357605 50.0
57.748521661 0.0664232596755028 51.0
63.639466031 0.493578910827637 52.0
59.450942311 0.0973641872406006 53.0
59.450942311 0.166120335459709 54.0
59.450942311 0.171029031276703 55.0
0 6.96039247512817 56.0
75.671575155 0.169657409191132 57.0
219.64062187 0.830576956272125 58.0
243.29003287 0.52629142999649 59.0
20.919719316 -0.00557347387075424 60.0
0 0.212823063135147 61.0
148.274306 1.2169144153595 0.0
148.274306 0.671589076519012 1.0
147.10573182 1.00024700164795 2.0
142.31662669 4.15242576599121 3.0
0 3.45315861701965 4.0
0 5.08364343643188 5.0
-65.384104516 0.756888031959534 6.0
74.947070234 0.713752150535583 7.0
57.756042199 1.0198655128479 8.0
17.191028035 1.00861382484436 9.0
93.642497004 1.29460191726685 10.0
80.300651611 0.148045897483826 11.0
88.699096302 0.882547318935394 12.0
88.699096302 0.103331804275513 13.0
88.699096302 0.164444983005524 14.0
0 0.337510913610458 15.0
0 0.141381725668907 16.0
0 1.41018235683441 17.0
54.283559367 1.8523815870285 18.0
-54.283559367 1.78810846805573 19.0
-54.283559367 0.616525292396545 20.0
-54.283559367 0.0255076996982098 21.0
76.749095208 1.25039601325989 22.0
140.98383621 0.751704692840576 23.0
110.00202461 0.647718548774719 24.0
21.90797708 0.239660397171974 25.0
217.73293142 0.779855310916901 26.0
217.73293142 1.5141167640686 27.0
25.717683733 1.70674574375153 28.0
243.45061515 0.708690464496613 29.0
243.45061515 1.85654890537262 30.0
46.53426563 0.916698098182678 31.0
9.4176457329 0.478524088859558 32.0
22.465535841 1.27309894561768 33.0
22.465535841 0.688737809658051 34.0
22.465535841 0.148496553301811 35.0
9.977246845 0.0325443670153618 36.0
8.7067929791 0.31368362903595 37.0
0 0.0393725968897343 38.0
0 0.200532972812653 39.0
-12.056810999 0.522347211837769 40.0
-25.318899386 0.440131038427353 41.0
5.4979440764 0.511092603206635 42.0
-33.317015545 0.494205504655838 43.0
87.995434101 0.636287927627563 44.0
87.995434101 0.091354139149189 45.0
6.9942070045 0.591147005558014 46.0
2.3742861972 0.4005506336689 47.0
148.274306 0.157426059246063 48.0
142.31662669 0.270973801612854 49.0
0 0.204029053449631 50.0
74.947070234 0.142253771424294 51.0
93.642497004 0.160979628562927 52.0
88.699096302 0.067049004137516 53.0
88.699096302 0.0797875449061394 54.0
88.699096302 0.0479793287813663 55.0
0 4.26897954940796 56.0
76.749095208 0.124178618192673 57.0
217.73293142 0.450775593519211 58.0
243.45061515 0.88920921087265 59.0
22.465535841 0.0159705523401499 60.0
0 0.118533067405224 61.0
142.99877025 0.931610703468323 0.0
142.99877025 1.68337774276733 1.0
141.56910711 2.36449933052063 2.0
136.780002 4.54863739013672 3.0
0 4.34207391738892 4.0
0 4.48727369308472 5.0
-53.035170971 1.30554342269897 6.0
61.886389598 1.3147679567337 7.0
59.76255184 0.555890440940857 8.0
2.1238377583 0.500365138053894 9.0
91.927451869 0.980482459068298 10.0
77.581392968 0.0797640308737755 11.0
83.694789501 0.340809345245361 12.0
83.694789501 0.534827351570129 13.0
83.694789501 0.0671687796711922 14.0
0 0.212455376982689 15.0
0 0.195932477712631 16.0
0 1.24426376819611 17.0
58.256252824 1.34467828273773 18.0
-58.256252824 1.71227586269379 19.0
-58.256252824 0.274878770112991 20.0
-58.256252824 0.281588762998581 21.0
81.881312388 1.17989480495453 22.0
141.70493926 0.689059257507324 23.0
106.02293839 0.753574013710022 24.0
23.950689862 0.0906625315546989 25.0
223.58625165 0.727403879165649 26.0
223.58625165 0.97253006696701 27.0
29.356181068 1.23409533500671 28.0
252.94243271 0.634210586547852 29.0
252.94243271 1.67299509048462 30.0
46.50842358 0.969349145889282 31.0
9.3768554227 0.47178453207016 32.0
23.625059564 1.09095966815948 33.0
23.625059564 0.950224339962006 34.0
23.625059564 0.0921621397137642 35.0
10.722848048 0.0227604508399963 36.0
9.3462025992 0.439746022224426 37.0
0 0.0739015713334084 38.0
0 0.186507821083069 39.0
-13.087674545 0.919386744499207 40.0
-25.945092162 0.528402984142303 41.0
5.0705705861 0.421673715114594 42.0
-33.429518331 0.277106761932373 43.0
100.92626744 0.824785947799683 44.0
100.92626744 0.0845705419778824 45.0
6.3931094215 0.748146653175354 46.0
2.3323312013 0.401015222072601 47.0
142.99877025 0.0910489112138748 48.0
136.780002 0.640592634677887 49.0
0 0.396098583936691 50.0
61.886389598 0.0678316801786423 51.0
91.927451869 0.188140034675598 52.0
83.694789501 0.0253818556666374 53.0
83.694789501 0.100542545318604 54.0
83.694789501 0.118169248104095 55.0
0 1.1792163848877 56.0
81.881312388 -0.045380562543869 57.0
223.58625165 0.0843629539012909 58.0
252.94243271 0.864556908607483 59.0
23.625059564 0.0029592290520668 60.0
0 0.0689599514007568 61.0
179.92665298 1.14051699638367 0.0
179.92665298 1.70709848403931 1.0
178.57936712 1.83641278743744 2.0
173.79026188 2.9385826587677 3.0
0 4.03368759155273 4.0
0 4.64383506774902 5.0
-66.673678836 0.424589097499847 6.0
78.575620982 0.471123814582825 7.0
48.170538739 1.14830505847931 8.0
30.405082242 1.02443587779999 9.0
85.606700874 1.12415897846222 10.0
72.32519244 0.0870788618922234 11.0
77.297358514 0.201247155666351 12.0
77.297358514 0.145865023136139 13.0
77.297358514 0.0619290322065353 14.0
0 0.0713644772768021 15.0
0 0.192736372351646 16.0
0 1.45592474937439 17.0
55.689766986 1.11711478233337 18.0
-55.689766986 0.975951552391052 19.0
-55.689766986 0.355907738208771 20.0
-55.689766986 0.0607171505689621 21.0
76.826419803 0.890713155269623 22.0
135.01409642 0.806322395801544 23.0
101.81902662 0.480539202690125 24.0
20.939280775 0.0883618593215942 25.0
211.84051622 1.03476452827454 26.0
211.84051622 2.09365200996399 27.0
25.515764254 2.46345329284668 28.0
237.35628047 1.43783009052277 29.0
237.35628047 1.90553712844849 30.0
46.70968034 0.997921884059906 31.0
9.4160424657 0.560425877571106 32.0
21.136652818 1.58723711967468 33.0
21.136652818 0.901798963546753 34.0
21.136652818 0.147700875997543 35.0
9.4879158559 0.0402962565422058 36.0
8.2082912651 0.385591894388199 37.0
0 0.0358911678195 38.0
0 0.0870920568704605 39.0
-12.377670644 0.746275782585144 40.0
-26.604495766 0.335209369659424 41.0
4.6790789501 0.645935833454132 42.0
-33.725237041 0.645741999149323 43.0
89.034348996 0.67908787727356 44.0
89.034348996 0.0748378112912178 45.0
6.9595549355 0.619073808193207 46.0
2.5328665232 0.415260255336761 47.0
179.92665298 0.12892684340477 48.0
173.79026188 0.109233550727367 49.0
0 0.112211547791958 50.0
78.575620982 0.234187513589859 51.0
85.606700874 0.15516023337841 52.0
77.297358514 0.0379860401153564 53.0
77.297358514 0.0709702894091606 54.0
77.297358514 0.172688141465187 55.0
0 4.48967838287354 56.0
76.826419803 0.0448487810790539 57.0
211.84051622 0.807672142982483 58.0
237.35628047 1.01108002662659 59.0
21.136652818 -0.00108965113759041 60.0
0 0.0975241735577583 61.0
147.61710906 0.799995720386505 0.0
147.61710906 2.03463411331177 1.0
146.2652921 2.06727504730225 2.0
141.47618699 4.57546520233154 3.0
0 6.28012847900391 4.0
0 3.31519460678101 5.0
-60.312883825 1.54528784751892 6.0
71.780324685 0.411824733018875 7.0
61.388682859 0.748917281627655 8.0
10.391641826 0.969464659690857 9.0
83.75579312 0.939943075180054 10.0
71.253024927 0.23958232998848 11.0
77.479545096 0.53856498003006 12.0
77.479545096 0.220651626586914 13.0
77.479545096 0.0683064684271812 14.0
0 0.110801421105862 15.0
0 0.20737236738205 16.0
0 1.39644145965576 17.0
56.250869478 1.65430009365082 18.0
-56.250869478 1.56428587436676 19.0
-56.250869478 0.890877962112427 20.0
-56.250869478 -0.0491928905248642 21.0
77.648984182 1.847163438797 22.0
134.74389827 0.966003119945526 23.0
103.07910813 0.803529024124146 24.0
21.843755599 0.0651481002569199 25.0
212.39288245 1.55897235870361 26.0
212.39288245 2.30536103248596 27.0
25.665928316 2.17976212501526 28.0
238.05881077 0.654134809970856 29.0
238.05881077 1.54378807544708 30.0
46.250173788 1.1478967666626 31.0
9.3045714097 0.527260780334473 32.0
21.398114704 1.41508138179779 33.0
21.398114704 0.826943039894104 34.0
21.398114704 0.115015171468258 35.0
9.6672269358 0.0270460974425077 36.0
8.3436984619 0.36467719078064 37.0
0 0.0168403200805187 38.0
0 0.0845330804586411 39.0
-12.300807824 0.536837160587311 40.0
-26.746799693 0.666283369064331 41.0
4.6230245366 0.491345793008804 42.0
-33.990613428 0.458502173423767 43.0
100.25352853 0.863680064678192 44.0
100.25352853 0.0485359281301498 45.0
6.5179863803 0.683857083320618 46.0
2.2919560778 0.392679393291473 47.0
147.61710906 0.159501016139984 48.0
141.47618699 0.368887931108475 49.0
0 0.448528289794922 50.0
71.780324685 0.138014376163483 51.0
83.75579312 0.338633209466934 52.0
77.479545096 0.0291315335780382 53.0
77.479545096 0.0508402138948441 54.0
77.479545096 0.0799635350704193 55.0
0 2.73112511634827 56.0
77.648984182 0.0959355384111404 57.0
212.39288245 0.557554125785828 58.0
238.05881077 0.190015554428101 59.0
21.398114704 0.00457649677991867 60.0
0 0.0953208357095718 61.0
193.00296329 0.87044632434845 0.0
193.00296329 1.2475723028183 1.0
191.75443731 1.35214376449585 2.0
186.96533201 4.25003147125244 3.0
0 8.8120288848877 4.0
0 4.49085092544556 5.0
-71.506026535 1.31366848945618 6.0
82.757369373 1.14980268478394 7.0
65.108990569 2.00245213508606 8.0
17.648378804 1.92278218269348 9.0
101.95577778 1.09900343418121 10.0
89.370846836 0.601342618465424 11.0
89.229571928 0.168795242905617 12.0
89.229571928 0.453746825456619 13.0
89.229571928 0.102983802556992 14.0
0 0.203033104538918 15.0
0 0.194590657949448 16.0
0 1.41157484054565 17.0
63.997278519 1.08071482181549 18.0
-63.997278519 1.00281405448914 19.0
-63.997278519 0.483739405870438 20.0
-63.997278519 -0.0530712641775608 21.0
86.830120105 2.10258483886719 22.0
133.14471472 2.04695415496826 23.0
102.80391849 0.225782319903374 24.0
19.996737015 0.228918880224228 25.0
219.97483482 1.52189075946808 26.0
219.97483482 3.09240388870239 27.0
28.832837974 3.28505420684814 28.0
248.8076728 1.33651220798492 29.0
248.8076728 2.12918519973755 30.0
46.823313078 1.18422949314117 31.0
9.1692277033 0.580465078353882 32.0
22.832841586 1.46772885322571 33.0
22.832841586 0.892881393432617 34.0
22.832841586 0.0638746544718742 35.0
10.233457376 0.0293877404183149 36.0
9.0206996824 0.354782819747925 37.0
0 0.0242438055574894 38.0
0 0.107324689626694 39.0
-12.982269593 0.465197503566742 40.0
-26.321324463 0.71252703666687 41.0
4.7866036819 0.398924946784973 42.0
-33.332556644 0.559420108795166 43.0
77.827336903 0.565766572952271 44.0
77.827336903 0.0975780338048935 45.0
7.4701280361 0.58644425868988 46.0
2.1356739969 0.38094887137413 47.0
193.00296329 0.133048385381699 48.0
186.96533201 0.0093726608902216 49.0
0 0.461445271968842 50.0
82.757369373 0.334852457046509 51.0
101.95577778 0.244630351662636 52.0
89.229571928 0.0627624541521072 53.0
89.229571928 0.0876235887408257 54.0
89.229571928 0.0416692644357681 55.0
0 1.17833995819092 56.0
86.830120105 0.0134897939860821 57.0
219.97483482 0.46027597784996 58.0
248.8076728 0.540816247463226 59.0
22.832841586 0.0366955772042274 60.0
0 0.134236454963684 61.0
137.48916608 0.967748403549194 0.0
137.48916608 0.895710408687592 1.0
136.30950473 1.62155103683472 2.0
131.28858801 3.20040059089661 3.0
0 4.32362461090088 4.0
0 4.47349452972412 5.0
-66.446925913 0.703740179538727 6.0
78.295875111 0.524543881416321 7.0
48.23936551 0.635946214199066 8.0
30.056509601 0.803605735301971 9.0
98.494238121 1.04706060886383 10.0
85.79597493 0.257034778594971 11.0
87.90114052 0.338145852088928 12.0
87.90114052 0.132572323083878 13.0
87.90114052 0.105022452771664 14.0
0 0.199895590543747 15.0
0 0.196911945939064 16.0
0 1.69491875171661 17.0
59.52254863 1.12505197525024 18.0
-59.52254863 0.799322962760925 19.0
-59.52254863 0.727068781852722 20.0
-59.52254863 -0.0335600003600121 21.0
81.205636016 2.62652349472046 22.0
150.90112731 2.40199899673462 23.0
113.75140362 0.311609715223312 24.0
20.8113073 0.186860769987106 25.0
232.10676332 1.47730004787445 26.0
232.10676332 1.91166937351227 27.0
24.037094881 2.2020525932312 28.0
256.1438582 0.993248105049133 29.0
256.1438582 1.86600971221924 30.0
46.386564786 0.991315603256226 31.0
9.4324181014 0.468283265829086 32.0
21.683087386 1.35413873195648 33.0
21.683087386 1.05587780475616 34.0
21.683087386 0.125657379627228 35.0
9.8964544621 0.0247850362211466 36.0
8.5937204187 0.433616727590561 37.0
0 0.0541197657585144 38.0
0 0.159558311104774 39.0
-10.960059547 0.797141194343567 40.0
-24.164125917 0.401699274778366 41.0
5.3237942758 0.40917757153511 42.0
-31.889955191 0.402597606182098 43.0
89.918701993 0.69642162322998 44.0
89.918701993 0.107398614287376 45.0
6.9511702863 0.579391717910767 46.0
2.9057793458 0.430200934410095 47.0
137.48916608 0.132282972335815 48.0
131.28858801 0.221369028091431 49.0
0 0.520544469356537 50.0
78.295875111 0.250978380441666 51.0
98.494238121 0.445235133171082 52.0
87.90114052 0.0266452245414257 53.0
87.90114052 0.0449633896350861 54.0
87.90114052 0.0448108538985252 55.0
0 0.97239100933075 56.0
81.205636016 0.067220576107502 57.0
232.10676332 0.693676114082336 58.0
256.1438582 0.580259621143341 59.0
21.683087386 0.00193390995264053 60.0
0 0.085355743765831 61.0
202.38954559 0.817289233207703 0.0
202.38954559 1.13345277309418 1.0
201.24650467 1.18265855312347 2.0
195.78183397 3.41843843460083 3.0
11.229806471 6.5381498336792 4.0
11.229806471 4.03664636611938 5.0
-64.863901711 0.205746069550514 6.0
74.244599562 0.64225435256958 7.0
36.912863908 1.29463696479797 8.0
37.331735655 1.30727648735046 9.0
76.565729304 0.800018548965454 10.0
64.035255218 0.0127277886494994 11.0
69.242160237 0.790717780590057 12.0
69.242160237 0.111069977283478 13.0
69.242160237 0.104051977396011 14.0
9.1277749745 0.201475620269775 15.0
9.1277749745 0.234302043914795 16.0
0 1.63130390644073 17.0
49.068920023 1.15948414802551 18.0
-49.068920023 0.942884802818298 19.0
-49.068920023 0.290050566196442 20.0
-49.068920023 0.0335111729800701 21.0
72.747375394 1.79958474636078 22.0
143.77481724 1.27389001846313 23.0
106.89829233 0.473666042089462 24.0
25.571156148 0.0947148203849792 25.0
216.52219264 1.5457239151001 26.0
216.52219264 1.55342292785645 27.0
22.666562066 1.46007466316223 28.0
239.1887547 0.788859486579895 29.0
239.1887547 1.78407728672028 30.0
46.128032658 1.10705816745758 31.0
9.4126998573 0.478574424982071 32.0
23.67845537 1.53696274757385 33.0
23.67845537 0.81255429983139 34.0
23.67845537 0.0821674019098282 35.0
10.483861212 0.036041084676981 36.0
9.3341849784 0.370334148406982 37.0
0 0.0240673273801804 38.0
0 0.187500163912773 39.0
-11.595828584 0.342893958091736 40.0
-23.081031621 0.776696383953094 41.0
5.1338673132 0.503711938858032 42.0
-30.665865162 0.858167886734009 43.0
92.110428522 0.66619086265564 44.0
92.110428522 0.0866677910089493 45.0
6.4246693319 0.516157686710358 46.0
2.4067794768 0.362885385751724 47.0
202.38954559 0.262586265802383 48.0
195.78183397 0.239685863256454 49.0
11.229806471 0.128197550773621 50.0
74.244599562 0.344011217355728 51.0
76.565729304 0.112088412046432 52.0
69.242160237 0.0289744026958942 53.0
69.242160237 0.0209731534123421 54.0
69.242160237 0.0456770658493042 55.0
0 3.24515056610107 56.0
72.747375394 0.155027821660042 57.0
216.52219264 0.764676451683044 58.0
239.1887547 0.526545405387878 59.0
23.67845537 0.0301810950040817 60.0
0 0.13362780213356 61.0
148.1866219 0.907368659973145 0.0
148.1866219 1.02624547481537 1.0
147.10057998 1.92543697357178 2.0
142.31147482 4.27187347412109 3.0
0 5.93563652038574 4.0
0 5.83188009262085 5.0
-53.86096717 0.504061102867126 6.0
63.655202737 0.8773432970047 7.0
40.67824508 0.783710777759552 8.0
22.976957657 0.536402821540833 9.0
88.60380846 1.17746436595917 10.0
74.419702839 0.408272355794907 11.0
79.982083747 0.173748195171356 12.0
79.982083747 0.156291171908379 13.0
79.982083747 0.0589582100510597 14.0
0 0.0906992182135582 15.0
0 0.224711492657661 16.0
0 1.46486234664917 17.0
59.959216453 1.10426044464111 18.0
-59.959216453 0.756531238555908 19.0
-59.959216453 0.397724032402039 20.0
-59.959216453 0.104078851640224 21.0
81.37172811 1.27368807792664 22.0
145.40227383 1.54482364654541 23.0
111.00308635 0.325112521648407 24.0
24.155521002 0.0833766311407089 25.0
226.77400194 0.809041142463684 26.0
226.77400194 1.61480152606964 27.0
28.157319668 1.54491329193115 28.0
254.93132161 0.853574693202972 29.0
254.93132161 2.10693049430847 30.0
47.085277871 1.17035591602325 31.0
9.413625788 0.67607980966568 32.0
21.412511657 1.10153388977051 33.0
21.412511657 0.975685119628906 34.0
21.412511657 0.105126895010471 35.0
9.1742142661 0.0351825878024101 36.0
8.0706838694 0.379450976848602 37.0
0 0.0345263592898846 38.0
0 0.223012298345566 39.0
-11.835283101 0.522929847240448 40.0
-23.966780393 0.725175201892853 41.0
4.6821267095 0.396197408437729 42.0
-31.008367919 0.447782427072525 43.0
91.645823698 0.611116409301758 44.0
91.645823698 0.0854941830039024 45.0
7.325747257 0.655272483825684 46.0
2.6544612408 0.431145995855331 47.0
148.1866219 0.0957593247294426 48.0
142.31147482 0.542217552661896 49.0
0 0.390602588653564 50.0
63.655202737 0.30080583691597 51.0
88.60380846 0.668339848518372 52.0
79.982083747 0.0487101003527641 53.0
79.982083747 0.0604526065289974 54.0
79.982083747 0.0778583437204361 55.0
0 2.55964636802673 56.0
81.37172811 0.0775828957557678 57.0
226.77400194 0.624768972396851 58.0
254.93132161 0.895954251289368 59.0
21.412511657 -0.00274436734616756 60.0
0 0.0980272814631462 61.0
184.43950514 0.836469411849976 0.0
184.43950514 1.7448490858078 1.0
183.00874356 2.11567664146423 2.0
178.21966001 4.68889236450195 3.0
0 7.11474418640137 4.0
0 4.52276086807251 5.0
-61.163926963 0.988570630550385 6.0
74.019271278 0.75443708896637 7.0
57.246758975 0.995728135108948 8.0
16.772512304 0.98934805393219 9.0
91.08860488 0.978359997272491 10.0
78.739462738 0.365867972373962 11.0
79.938083179 0.63006067276001 12.0
79.938083179 0.356127828359604 13.0
79.938083179 0.123979814350605 14.0
0 0.217019587755203 15.0
0 0.165186479687691 16.0
0 1.46658575534821 17.0
57.437724342 1.92435872554779 18.0
-57.437724342 1.57168555259705 19.0
-57.437724342 0.757690787315369 20.0
-57.437724342 -0.0309772789478302 21.0
78.416131409 1.3205178976059 22.0
149.14139087 0.877375662326813 23.0
112.87101275 0.88563597202301 24.0
21.629941893 0.0723415166139603 25.0
227.55752228 1.19730818271637 26.0
227.55752228 2.03292369842529 27.0
23.994862962 1.88040721416473 28.0
251.55238524 1.09775722026825 29.0
251.55238524 1.83016192913055 30.0
47.302146013 1.17422461509705 31.0
9.3964543614 0.455627411603928 32.0
20.978407066 1.33141314983368 33.0
20.978407066 1.44785821437836 34.0
20.978407066 0.133672744035721 35.0
8.789263838 0.0446613542735577 36.0
7.4906004389 0.373291164636612 37.0
0 0.0155496373772621 38.0
0 0.325354099273682 39.0
-11.483123038 0.334437131881714 40.0
-25.616476777 0.733004987239838 41.0
5.0328315286 0.400111049413681 42.0
-33.27975066 0.485908478498459 43.0
89.08432113 0.671420276165009 44.0
89.08432113 0.0721813663840294 45.0
6.4263751888 0.531575500965118 46.0
2.2150087179 0.386410385370255 47.0
184.43950514 0.107062458992004 48.0
178.21966001 0.138332188129425 49.0
0 0.300404399633408 50.0
74.019271278 0.12482962012291 51.0
91.08860488 0.307660609483719 52.0
79.938083179 0.0743743181228638 53.0
79.938083179 0.0854164659976959 54.0
79.938083179 0.0661465898156166 55.0
0 5.35726070404053 56.0
78.416131409 0.0803681463003159 57.0
227.55752228 0.881181359291077 58.0
251.55238524 0.472726464271545 59.0
20.978407066 -0.0158702991902828 60.0
0 0.106518559157848 61.0
150.82653501 0.818205237388611 0.0
150.82653501 1.17324817180634 1.0
149.78138422 1.64225876331329 2.0
144.99227912 5.95299434661865 3.0
0 4.79503059387207 4.0
0 2.90375423431396 5.0
-53.351079346 -0.0210302025079727 6.0
64.247156221 0.552368223667145 7.0
46.450355941 1.07697772979736 8.0
17.79680028 1.28368771076202 9.0
107.26332329 0.954685688018799 10.0
94.232591422 0.346830993890762 11.0
91.364417622 0.858422219753265 12.0
91.364417622 0.347508490085602 13.0
91.364417622 0.0679540410637856 14.0
0 0.121541522443295 15.0
0 0.207624152302742 16.0
0 1.10769462585449 17.0
58.077587111 1.41622817516327 18.0
-58.077587111 1.58423495292664 19.0
-58.077587111 0.68862372636795 20.0
-58.077587111 0.166063547134399 21.0
81.500934589 1.86303317546844 22.0
152.00289661 1.25638902187347 23.0
115.64862739 0.490266859531403 24.0
20.848115366 0.460403621196747 25.0
233.5038312 1.23019349575043 26.0
233.5038312 1.79244410991669 27.0
23.726146058 2.11855149269104 28.0
257.22997726 0.692052006721497 29.0
257.22997726 1.80886137485504 30.0
46.188274298 1.20185732841492 31.0
9.189193986 0.571941018104553 32.0
23.423347478 1.27824115753174 33.0
23.423347478 1.16056132316589 34.0
23.423347478 0.15203183889389 35.0
10.561582998 0.0360996648669243 36.0
9.2067895485 0.322322338819504 37.0
0 0.0467805787920952 38.0
0 0.274002343416214 39.0
-11.876078682 0.370312005281448 40.0
-22.913476455 0.771699249744415 41.0
5.4037306496 0.344012349843979 42.0
-30.987433719 0.269969969987869 43.0
106.75347873 0.610085308551788 44.0
106.75347873 0.0132202208042145 45.0
6.5077996063 0.589665949344635 46.0
1.2836174785 0.345740467309952 47.0
150.82653501 0.111660122871399 48.0
144.99227912 0.206670880317688 49.0
0 0.283603072166443 50.0
64.247156221 0.230347350239754 51.0
107.26332329 0.296254724264145 52.0
91.364417622 0.116692267358303 53.0
91.364417622 0.125858724117279 54.0
91.364417622 0.0706081241369247 55.0
0 1.3659805059433 56.0
81.500934589 0.132003173232079 57.0
233.5038312 0.411072015762329 58.0
257.22997726 0.471883714199066 59.0
23.423347478 -0.0320074781775475 60.0
0 0.105691500008106 61.0
168.06736724 0.737634122371674 0.0
168.06736724 0.941940665245056 1.0
166.83311442 1.19626331329346 2.0
162.04400916 3.67015743255615 3.0
0 8.05707740783691 4.0
0 4.43875360488892 5.0
-56.7167052 1.75186741352081 6.0
69.45897224 1.31354105472565 7.0
61.286120657 0.415605634450912 8.0
8.1728515831 0.403540432453156 9.0
91.175597156 0.823957204818726 10.0
77.218571995 0.17401622235775 11.0
85.262957107 0.463404327630997 12.0
85.262957107 0.258415341377258 13.0
85.262957107 0.0728358030319214 14.0
0 0.141721367835999 15.0
0 0.216016218066216 16.0
0 1.18374490737915 17.0
55.554327298 1.49315941333771 18.0
-55.554327298 1.1268767118454 19.0
-55.554327298 0.737646162509918 20.0
-55.554327298 -0.044781856238842 21.0
77.297179973 2.16120314598083 22.0
133.43202101 1.75121629238129 23.0
99.623640187 0.439150899648666 24.0
22.22820147 0.079489953815937 25.0
210.72920098 1.7889244556427 26.0
210.72920098 1.89052081108093 27.0
24.839981383 1.45071423053741 28.0
235.56918237 0.833098173141479 29.0
235.56918237 1.7690235376358 30.0
46.933910773 0.944701552391052 31.0
9.3181829379 0.410819172859192 32.0
21.742852675 1.28536367416382 33.0
21.742852675 1.06176090240479 34.0
21.742852675 0.11163979023695 35.0
9.861554295 0.0290655624121428 36.0
8.6092526103 0.367920488119125 37.0
0 0.0591601431369781 38.0
0 0.263440608978271 39.0
-12.177708159 0.834162533283234 40.0
-25.066983321 0.39609831571579 41.0
4.8436933944 0.530169725418091 42.0
-32.627999423 0.582794964313507 43.0
90.173871845 0.812219679355621 44.0
90.173871845 0.0870437771081924 45.0
7.0555957533 0.492929488420486 46.0
2.6103485392 0.324863106012344 47.0
168.06736724 0.0289268363267183 48.0
162.04400916 0.172710463404655 49.0
0 0.303639382123947 50.0
69.45897224 0.0837949141860008 51.0
91.175597156 0.188500076532364 52.0
85.262957107 0.0288201607763767 53.0
85.262957107 0.0602942816913128 54.0
85.262957107 0.0299525707960129 55.0
0 2.46089863777161 56.0
77.297179973 -0.0178223624825478 57.0
210.72920098 0.855304777622223 58.0
235.56918237 0.855167984962463 59.0
21.742852675 0.0135995596647263 60.0
0 0.0729754418134689 61.0
148.48733102 0.953011393547058 0.0
148.48733102 0.766536593437195 1.0
147.52794158 1.02210581302643 2.0
142.73883621 4.08231401443481 3.0
0 7.85867500305176 4.0
0 5.78699445724487 5.0
-66.85382803 1.20357000827789 6.0
75.579747895 0.571776628494263 7.0
72.435824834 0.712647199630737 8.0
3.1439230608 0.991221725940704 9.0
80.979812667 1.21691679954529 10.0
66.628992116 0.319494932889938 11.0
86.452151722 0.212307825684547 12.0
86.452151722 0.137890636920929 13.0
86.452151722 0.0265904515981674 14.0
0 0.141267046332359 15.0
0 0.257715553045273 16.0
0 0.690498828887939 17.0
54.272911016 1.41810083389282 18.0
-54.272911016 1.61517977714539 19.0
-54.272911016 0.194277867674828 20.0
-54.272911016 0.449890553951263 21.0
75.544962042 1.86632680892944 22.0
136.37680949 1.45184910297394 23.0
104.86652489 0.576569437980652 24.0
10.740358276 0.0652145817875862 25.0
211.92177153 1.37355375289917 26.0
211.92177153 1.14625608921051 27.0
30.118893363 1.44074881076813 28.0
242.04066489 0.551451563835144 29.0
242.04066489 1.74897480010986 30.0
47.354293467 1.04263377189636 31.0
9.4292157666 0.631872355937958 32.0
21.272051026 1.0080304145813 33.0
21.272051026 0.927722930908203 34.0
21.272051026 0.100575938820839 35.0
9.4541934713 0.0324417054653168 36.0
8.3769919556 0.465198457241058 37.0
0 0.018825601786375 38.0
0 0.15372309088707 39.0
-11.660351574 0.528599143028259 40.0
-24.848923962 0.623276054859161 41.0
4.5353783178 0.525971829891205 42.0
-32.088556401 0.695177376270294 43.0
73.715549897 0.660919725894928 44.0
73.715549897 0.135426759719849 45.0
6.9780748056 0.654864132404327 46.0
2.5334710609 0.367587149143219 47.0
148.48733102 0.175872728228569 48.0
142.73883621 0.343226283788681 49.0
0 0.648999750614166 50.0
75.579747895 0.105916894972324 51.0
80.979812667 0.766108393669128 52.0
86.452151722 0.0359385907649994 53.0
86.452151722 0.0713690295815468 54.0
86.452151722 0.118887186050415 55.0
0 0.65073424577713 56.0
75.544962042 -0.0662220418453217 57.0
211.92177153 0.407510578632355 58.0
242.04066489 0.685413956642151 59.0
21.272051026 0.00431602075695992 60.0
0 0.106627993285656 61.0
184.93208518 0.959546566009521 0.0
184.93208518 0.644363224506378 1.0
183.42823009 0.90541797876358 2.0
177.96359468 4.32340955734253 3.0
8.7353116781 5.89359474182129 4.0
8.7353116781 4.50108337402344 5.0
-67.185574645 1.74586606025696 6.0
80.152119334 0.975276827812195 7.0
49.258763883 0.619870603084564 8.0
30.893355451 0.780769765377045 9.0
93.824348508 1.07255101203918 10.0
81.729678732 0.259526997804642 11.0
87.73793038 0.810733556747437 12.0
87.73793038 0.79220974445343 13.0
87.73793038 0.0916353166103363 14.0
0 0.190115168690681 15.0
0 0.156099796295166 16.0
0 1.24337208271027 17.0
53.21694855 1.16448044776917 18.0
-53.21694855 0.57576847076416 19.0
-53.21694855 1.137486577034 20.0
-53.21694855 -0.0577025227248669 21.0
73.757218178 2.31247901916504 22.0
143.85748575 2.27411389350891 23.0
110.47095779 0.369240641593933 24.0
25.864880933 0.133485361933708 25.0
217.61470393 0.941612362861633 26.0
217.61470393 1.51291835308075 27.0
28.781096925 1.31940460205078 28.0
246.39580085 0.840899884700775 29.0
246.39580085 1.83764934539795 30.0
46.865556003 0.861616969108582 31.0
9.2212436272 0.426193356513977 32.0
20.540269627 1.61890923976898 33.0
20.540269627 0.983857333660126 34.0
20.540269627 0.128682479262352 35.0
9.0981452727 0.0345124155282974 36.0
7.9009100048 0.38147297501564 37.0
0 0.0366502739489079 38.0
0 0.189254432916641 39.0
-11.324012252 0.803396821022034 40.0
-24.561646862 0.542329847812653 41.0
5.5353589903 0.492504060268402 42.0
-32.530013633 0.595722913742065 43.0
101.71643208 0.72759222984314 44.0
101.71643208 0.045491948723793 45.0
6.5146317451 0.729351878166199 46.0
2.2436142103 0.400200963020325 47.0
184.93208518 0.147868081927299 48.0
177.96359468 0.0847986713051796 49.0
8.7353116781 0.336233705282211 50.0
80.152119334 0.0827145501971245 51.0
93.824348508 0.164570569992065 52.0
87.73793038 0.0617871731519699 53.0
87.73793038 0.200742110610008 54.0
87.73793038 0.10968454927206 55.0
0 1.57846462726593 56.0
73.757218178 -0.0378546416759491 57.0
217.61470393 0.195125162601471 58.0
246.39580085 0.770729541778564 59.0
20.540269627 -0.0167215019464493 60.0
0 0.0805115178227425 61.0
196.7623493 0.740217924118042 0.0
196.7623493 1.66631436347961 1.0
195.80477325 1.70955729484558 2.0
191.77782662 5.23139381408691 3.0
0 6.73324966430664 4.0
0 2.65758299827576 5.0
-74.150151332 1.35381257534027 6.0
86.562358475 0.774834096431732 7.0
53.130173601 1.22249662876129 8.0
33.432184874 1.2106716632843 9.0
82.93853169 0.846915304660797 10.0
69.852395107 0.511823177337646 11.0
75.046523192 0.877440094947815 12.0
75.046523192 0.273482024669647 13.0
75.046523192 0.130674704909325 14.0
0.22530124412 0.173289462924004 15.0
0.22530124412 0.184216305613518 16.0
0 1.20852994918823 17.0
50.809450027 1.27743327617645 18.0
-50.809450027 1.32561266422272 19.0
-50.809450027 0.792879283428192 20.0
-50.809450027 -0.087145134806633 21.0
72.584749813 2.61806559562683 22.0
148.37926716 2.74293398857117 23.0
113.19577404 0.48154816031456 24.0
21.308544879 0.0961534827947617 25.0
220.96401697 1.07839465141296 26.0
220.96401697 1.87873661518097 27.0
26.109087029 2.10462284088135 28.0
247.073104 1.00999367237091 29.0
247.073104 1.68420696258545 30.0
46.986998705 1.12900733947754 31.0
9.3151717677 0.375390082597733 32.0
21.775299785 2.41647362709045 33.0
21.775299785 0.89227706193924 34.0
21.775299785 0.12399460375309 35.0
9.4427480489 0.0232738330960274 36.0
8.2355623235 0.432200700044632 37.0
0 0.0464014671742916 38.0
0 0.170781224966049 39.0
-11.976636272 0.746789991855621 40.0
-24.42947319 0.669582426548004 41.0
4.9623528987 0.460473716259003 42.0
-32.148006085 0.678133606910706 43.0
109.84906933 0.644443094730377 44.0
109.84906933 0.0535126477479935 45.0
7.1127517447 0.690335869789124 46.0
2.6536538352 0.397308617830276 47.0
196.7623493 0.113998629152775 48.0
191.77782662 0.15019316971302 49.0
0 0.292789787054062 50.0
86.562358475 0.345887988805771 51.0
82.93853169 0.278326809406281 52.0
75.046523192 0.138193115592003 53.0
75.046523192 0.0608853176236153 54.0
75.046523192 0.0858322829008102 55.0
0 5.34429216384888 56.0
72.584749813 0.114511325955391 57.0
220.96401697 0.313085913658142 58.0
247.073104 0.511016368865967 59.0
21.775299785 -0.0161077678203583 60.0
0 0.115235522389412 61.0
197.70394508 0.847224533557892 0.0
197.70394508 0.783146440982819 1.0
196.53756678 0.954470872879028 2.0
191.74846145 3.32436895370483 3.0
0 9.1660270690918 4.0
0 5.61236572265625 5.0
-59.622582762 0.756870925426483 6.0
70.091549501 1.48861396312714 7.0
58.683360431 0.402174741029739 8.0
11.408189069 0.284847885370255 9.0
88.172825057 0.902608156204224 10.0
74.511891605 0.270921915769577 11.0
83.723585832 0.53897225856781 12.0
83.723585832 0.322388380765915 13.0
83.723585832 0.020008722320199 14.0
0 0.106955662369728 15.0
0 0.239426791667938 16.0
0 0.978670001029968 17.0
53.082051632 1.67275035381317 18.0
-53.082051632 1.7501060962677 19.0
-53.082051632 0.245173260569572 20.0
-53.082051632 0.165745139122009 21.0
74.166848611 0.824921727180481 22.0
149.12985792 0.413907408714294 23.0
110.45487994 0.376119405031204 24.0
23.670559498 0.0967563912272453 25.0
223.29670653 1.09346115589142 26.0
223.29670653 2.351717710495 27.0
24.272574381 2.61430048942566 28.0
247.56928091 1.06147623062134 29.0
247.56928091 1.68323302268982 30.0
47.408168 1.18461084365845 31.0
9.4994370674 0.420816093683243 32.0
21.084796979 1.27654385566711 33.0
21.084796979 1.13218069076538 34.0
21.084796979 0.0915301740169525 35.0
9.3824763008 0.0351596549153328 36.0
8.136856323 0.400211125612259 37.0
0 -0.00672570429742336 38.0
0 0.189170181751251 39.0
-11.576011187 1.01933097839355 40.0
-25.100932968 0.681860566139221 41.0
5.8356909345 0.373787701129913 42.0
-33.639108819 0.993560671806335 43.0
77.132623395 0.744630932807922 44.0
77.132623395 0.104293525218964 45.0
7.1840335032 0.434185326099396 46.0
2.6499721779 0.279665440320969 47.0
197.70394508 0.235719978809357 48.0
191.74846145 0.118335589766502 49.0
0 0.433899313211441 50.0
70.091549501 0.216148123145103 51.0
88.172825057 0.130161389708519 52.0
83.723585832 0.0429479330778122 53.0
83.723585832 0.0888412296772003 54.0
83.723585832 0.198226973414421 55.0
0 3.29095506668091 56.0
74.166848611 -0.0598048195242882 57.0
223.29670653 0.65135270357132 58.0
247.56928091 0.212422072887421 59.0
21.084796979 -0.00410613045096397 60.0
0 0.0981395244598389 61.0
138.14843862 0.744961559772491 0.0
138.14843862 1.36863851547241 1.0
136.79049456 1.6293203830719 2.0
131.32582382 3.01135540008545 3.0
1.8588493873 5.61554574966431 4.0
1.8588493873 3.66397356987 5.0
-73.45239637 0.421870172023773 6.0
83.701986594 1.28909277915955 7.0
53.889544352 0.427699714899063 8.0
29.812442242 0.433048576116562 9.0
95.596475296 0.814107596874237 10.0
82.134970097 0.0751993134617805 11.0
87.110370086 0.825725257396698 12.0
87.110370086 0.36157688498497 13.0
87.110370086 0.155557304620743 14.0
0 0.31001740694046 15.0
0 0.131431490182877 16.0
0 1.489990234375 17.0
55.04003612 1.74368166923523 18.0
-55.04003612 1.7992205619812 19.0
-55.04003612 0.311341464519501 20.0
-55.04003612 -0.053180918097496 21.0
76.50895732 0.985655605792999 22.0
140.65921949 0.529203534126282 23.0
106.11015443 0.566615104675293 24.0
25.059729386 0.078324094414711 25.0
217.16817682 0.908498764038086 26.0
217.16817682 1.59954810142517 27.0
27.218488957 1.48274290561676 28.0
244.38666577 0.609034657478333 29.0
244.38666577 1.86431896686554 30.0
46.469917031 1.01273477077484 31.0
9.4579001793 0.699961006641388 32.0
21.468921201 1.86535799503326 33.0
21.468921201 1.18546605110168 34.0
21.468921201 0.160394236445427 35.0
9.5666018322 0.040629655122757 36.0
8.2190993143 0.369097411632538 37.0
0 0.0403261445462704 38.0
0 0.192639395594597 39.0
-11.726516228 0.486817359924316 40.0
-25.049110285 0.560156524181366 41.0
5.4077494423 0.673789143562317 42.0
-33.210692328 0.858248054981232 43.0
87.719443768 0.505722284317017 44.0
87.719443768 0.151017338037491 45.0
6.101697111 0.408686339855194 46.0
2.0001686855 0.350255399942398 47.0
138.14843862 0.064962811768055 48.0
131.32582382 0.0945904180407524 49.0
1.8588493873 0.551186561584473 50.0
83.701986594 0.174631297588348 51.0
95.596475296 0.209516257047653 52.0
87.110370086 0.0735905468463898 53.0
87.110370086 0.160726875066757 54.0
87.110370086 0.128625556826591 55.0
0 4.1836051940918 56.0
76.50895732 0.188244313001633 57.0
217.16817682 0.285960286855698 58.0
244.38666577 0.636325776576996 59.0
21.468921201 -0.0197143144905567 60.0
0 0.100712835788727 61.0
191.39906633 0.819351315498352 0.0
191.39906633 1.05433976650238 1.0
190.36171537 1.13006174564362 2.0
185.57261021 3.32971668243408 3.0
0 4.5256142616272 4.0
0 5.36243295669556 5.0
-58.784110272 1.78411853313446 6.0
69.673463795 0.771186411380768 7.0
55.489122234 0.816197633743286 8.0
14.184341561 0.675935566425323 9.0
100.9268776 0.988182187080383 10.0
87.976852482 0.421799570322037 11.0
85.602038528 0.549903929233551 12.0
85.602038528 0.0868107005953789 13.0
85.602038528 0.0429742410778999 14.0
0 0.0728842094540596 15.0
0 0.184546992182732 16.0
0 0.632756233215332 17.0
59.548153419 1.50130999088287 18.0
-59.548153419 1.30446696281433 19.0
-59.548153419 0.460993051528931 20.0
-59.548153419 0.346462190151215 21.0
80.865145325 2.64396286010742 22.0
142.20019774 2.32102847099304 23.0
108.37223106 0.369523555040359 24.0
25.48697783 0.10950793325901 25.0
223.06534307 1.63338303565979 26.0
223.06534307 1.77460312843323 27.0
23.224944447 1.7287540435791 28.0
246.29028751 0.669938504695892 29.0
246.29028751 1.59956610202789 30.0
46.673166587 1.2146999835968 31.0
9.2439008457 0.466365963220596 32.0
21.316991906 0.992926597595215 33.0
21.316991906 0.881810665130615 34.0
21.316991906 0.0833793506026268 35.0
9.6483787693 0.0187755469232798 36.0
8.4865119024 0.453261345624924 37.0
0 0.0245074853301048 38.0
0 0.169717684388161 39.0
-11.780893335 0.673895537853241 40.0
-24.002020673 0.559894859790802 41.0
5.401667806 0.383204370737076 42.0
-32.181533964 0.43018639087677 43.0
91.607959556 0.7027468085289 44.0
91.607959556 0.0776729881763458 45.0
6.4310643299 0.67461633682251 46.0
2.4431116248 0.408032983541489 47.0
191.39906633 0.13830278813839 48.0
185.57261021 0.121564105153084 49.0
0 0.21086885035038 50.0
69.673463795 0.221937492489815 51.0
100.9268776 0.301972895860672 52.0
85.602038528 0.126502707600594 53.0
85.602038528 0.0730889737606049 54.0
85.602038528 0.0583895221352577 55.0
0 2.37397456169128 56.0
80.865145325 -0.0529642403125763 57.0
223.06534307 0.411707818508148 58.0
246.29028751 0.433146834373474 59.0
21.316991906 0.0081530287861824 60.0
0 0.114859387278557 61.0
151.14743098 0.79568213224411 0.0
151.14743098 1.16525161266327 1.0
149.67131554 2.62125706672668 2.0
144.88224558 5.65489530563354 3.0
0 8.1214714050293 4.0
0 7.66508674621582 5.0
-65.479144505 5.41510200500488 6.0
74.261390522 0.685717046260834 7.0
55.117829033 0.628814041614532 8.0
19.143561489 0.853992521762848 9.0
87.960449123 0.908138930797577 10.0
74.297749223 0.166133061051369 11.0
86.693859179 0.173665940761566 12.0
86.693859179 0.530586302280426 13.0
86.693859179 0.0521536804735661 14.0
0 0.164734229445457 15.0
0 0.211402386426926 16.0
0 1.26444447040558 17.0
60.024188611 0.98239678144455 18.0
-60.024188611 1.04710674285889 19.0
-60.024188611 0.823962330818176 20.0
-60.024188611 -0.0170562826097012 21.0
82.390576653 1.36877906322479 22.0
142.63817785 0.983274102210999 23.0
107.75260742 0.548855006694794 24.0
19.967780844 0.0748018026351929 25.0
225.02875451 0.913084447383881 26.0
225.02875451 1.69133830070496 27.0
29.304438576 1.10912501811981 28.0
254.33319308 0.383596748113632 29.0
254.33319308 1.31552159786224 30.0
47.277491796 1.20274221897125 31.0
9.2646646879 0.615661084651947 32.0
22.366388042 1.91765570640564 33.0
22.366388042 0.934247314929962 34.0
22.366388042 0.120155982673168 35.0
9.248796401 0.031343899667263 36.0
7.898229999 0.310561418533325 37.0
0 0.0525937937200069 38.0
0 0.206542283296585 39.0
-12.541310604 0.420566320419312 40.0
-25.903708665 0.774763107299805 41.0
5.4067072052 0.485757529735565 42.0
-34.084947183 0.533812642097473 43.0
91.388132575 0.712813198566437 44.0
91.388132575 0.0428954213857651 45.0
6.0004343152 0.657215356826782 46.0
1.1898173093 0.433951675891876 47.0
151.14743098 -0.00456101167947054 48.0
144.88224558 0.544241189956665 49.0
0 0.613366186618805 50.0
74.261390522 0.113362289965153 51.0
87.960449123 0.234895527362823 52.0
86.693859179 0.0162106622010469 53.0
86.693859179 0.0663301348686218 54.0
86.693859179 0.233197897672653 55.0
0 2.03493928909302 56.0
82.390576653 0.103359572589397 57.0
225.02875451 0.414594858884811 58.0
254.33319308 -0.0612681545317173 59.0
22.366388042 0.00413504615426064 60.0
0 0.12391109764576 61.0
181.57234711 0.946764767169952 0.0
181.57234711 1.07476210594177 1.0
180.13194437 1.17914867401123 2.0
175.34283909 5.76828002929688 3.0
0 8.60364151000977 4.0
0 4.209716796875 5.0
-64.566143008 0.294256448745728 6.0
77.763314344 1.06403994560242 7.0
51.098115379 0.643240571022034 8.0
26.665198965 0.497062921524048 9.0
88.082750922 1.02713167667389 10.0
75.771102368 0.326813936233521 11.0
80.83455248 0.539644300937653 12.0
80.83455248 0.421113431453705 13.0
80.83455248 0.0892768502235413 14.0
0 0.0408049002289772 15.0
0 0.184694975614548 16.0
0 1.17025649547577 17.0
54.716012068 1.17963254451752 18.0
-54.716012068 1.03929400444031 19.0
-54.716012068 0.575767338275909 20.0
-54.716012068 0.0233541242778301 21.0
74.897975466 1.38532209396362 22.0
141.50548416 0.830060422420502 23.0
104.91212623 0.385057955980301 24.0
22.301453952 0.169255256652832 25.0
216.40345963 0.975223064422607 26.0
216.40345963 1.98640215396881 27.0
27.320021183 2.39871644973755 28.0
243.72348081 1.00908291339874 29.0
243.72348081 1.74642193317413 30.0
46.470838928 1.01773738861084 31.0
9.3810067906 0.41335791349411 32.0
20.181963398 1.75322294235229 33.0
20.181963398 0.912394642829895 34.0
20.181963398 0.168754130601883 35.0
9.0522064061 0.0333963930606842 36.0
7.8106485425 0.337080538272858 37.0
0 0.0270029678940773 38.0
0 0.214927762746811 39.0
-12.205244696 0.734098136425018 40.0
-24.375681924 0.616682648658752 41.0
5.6088320697 0.465529263019562 42.0
-32.448072774 0.338759958744049 43.0
86.193548387 0.725851893424988 44.0
86.193548387 0.0830208137631416 45.0
6.897119405 0.591408789157867 46.0
2.3688579911 0.337630689144135 47.0
181.57234711 0.247400611639023 48.0
175.34283909 0.0164670348167419 49.0
0 0.625406205654144 50.0
77.763314344 0.203878179192543 51.0
88.082750922 0.32860204577446 52.0
80.83455248 0.0745175629854202 53.0
80.83455248 0.0954621881246567 54.0
80.83455248 0.0287239588797092 55.0
0 3.70141363143921 56.0
74.897975466 0.0471007898449898 57.0
216.40345963 0.217917516827583 58.0
243.72348081 0.259620159864426 59.0
20.181963398 -0.0224853307008743 60.0
0 0.118550501763821 61.0
173.14184151 0.750752389431 0.0
173.14184151 1.14204585552216 1.0
172.02657774 0.976183414459229 2.0
167.23750779 5.14056539535522 3.0
0 3.55146455764771 4.0
0 4.95862722396851 5.0
-74.346492076 3.55631279945374 6.0
84.63620233 0.879620254039764 7.0
60.078380297 1.25851154327393 8.0
24.557822033 1.11603343486786 9.0
78.825185938 0.85718435049057 10.0
65.426764632 0.0969191789627075 11.0
79.709432894 0.562263011932373 12.0
79.709432894 0.107293143868446 13.0
79.709432894 0.154542118310928 14.0
0 0.224159359931946 15.0
0 0.189352244138718 16.0
0 1.07882380485535 17.0
55.070175494 1.25913858413696 18.0
-55.070175494 1.36166369915009 19.0
-55.070175494 0.602223694324493 20.0
-55.070175494 0.0101569518446922 21.0
76.060323739 1.59991419315338 22.0
136.31068239 1.24247825145721 23.0
105.75801572 0.499489665031433 24.0
23.661094904 0.0771852135658264 25.0
212.37100613 0.918910026550293 26.0
212.37100613 1.57752013206482 27.0
30.728271201 1.56372153759003 28.0
243.09927733 0.43018639087677 29.0
243.09927733 1.47845578193665 30.0
47.135213975 1.0287561416626 31.0
9.3783829595 0.338982969522476 32.0
20.990148244 1.31015336513519 33.0
20.990148244 1.07609963417053 34.0
20.990148244 0.119435235857964 35.0
8.2510666405 0.0297275464981794 36.0
7.0259593316 0.404578983783722 37.0
0 0.0149108804762363 38.0
0 0.129708603024483 39.0
-10.732793992 0.46972793340683 40.0
-23.790281497 0.515702605247498 41.0
5.5392726327 0.4962098300457 42.0
-31.616725651 0.61776077747345 43.0
101.16843275 0.510415136814117 44.0
101.16843275 0.0647879987955093 45.0
6.6336847127 0.503295063972473 46.0
2.7113684671 0.34790563583374 47.0
173.14184151 0.153939783573151 48.0
167.23750779 0.0539849549531937 49.0
0 0.083172969520092 50.0
84.63620233 0.371057748794556 51.0
78.825185938 0.30082967877388 52.0
79.709432894 0.0567935332655907 53.0
79.709432894 0.0354086086153984 54.0
79.709432894 0.230363935232162 55.0
0 6.06407499313354 56.0
76.060323739 0.0201004631817341 57.0
212.37100613 0.501167714595795 58.0
243.09927733 0.247740417718887 59.0
20.990148244 -0.00658147968351841 60.0
0 0.117208048701286 61.0
131.1745711 0.758676469326019 0.0
131.1745711 1.14535188674927 1.0
130.18980603 1.08260834217072 2.0
125.40070075 3.08026099205017 3.0
0 4.61312198638916 4.0
0 5.70177316665649 5.0
-52.869065084 0.708238959312439 6.0
63.654888241 0.575607597827911 7.0
46.197049929 0.620251357555389 8.0
17.457838313 0.689486801624298 9.0
90.699045327 1.02000284194946 10.0
77.969469706 0.433384120464325 11.0
78.177060091 0.330224573612213 12.0
78.177060091 0.610749840736389 13.0
78.177060091 0.122103363275528 14.0
0 0.17576590180397 15.0
0 0.18758325278759 16.0
0 1.26951599121094 17.0
53.557277526 0.888078331947327 18.0
-53.557277526 0.654220342636108 19.0
-53.557277526 0.689680576324463 20.0
-53.557277526 0.0283989906311035 21.0
76.37958456 1.89730072021484 22.0
144.67087904 1.53464090824127 23.0
111.99184894 0.215592533349991 24.0
20.084430136 0.50005841255188 25.0
221.0504636 0.770593166351318 26.0
221.0504636 1.89328193664551 27.0
27.136164941 1.19606363773346 28.0
248.18662854 1.24765336513519 29.0
248.18662854 1.77076923847198 30.0
47.104935845 1.01635468006134 31.0
9.380734391 0.506580054759979 32.0
22.822307035 1.38729393482208 33.0
22.822307035 1.06179106235504 34.0
22.822307035 0.054574478417635 35.0
9.8957494266 0.0282543990761042 36.0
8.5976783145 0.436919361352921 37.0
0 0.0150350816547871 38.0
0 0.0687438547611237 39.0
-11.687111997 0.940641283988953 40.0
-25.904430476 0.364566534757614 41.0
5.4216861449 0.699329257011414 42.0
-33.669487903 0.804045379161835 43.0
84.289914489 0.592266619205475 44.0
84.289914489 0.101351976394653 45.0
7.0207666912 0.473173350095749 46.0
2.2661515391 0.407832443714142 47.0
131.1745711 0.034427173435688 48.0
125.40070075 0.0156303010880947 49.0
0 0.708274841308594 50.0
63.654888241 0.28456449508667 51.0
90.699045327 0.547743678092957 52.0
78.177060091 0.0649954304099083 53.0
78.177060091 0.0217390209436417 54.0
78.177060091 0.355581223964691 55.0
0 7.69819831848145 56.0
76.37958456 0.0307562202215195 57.0
221.0504636 0.840050160884857 58.0
248.18662854 0.700666487216949 59.0
22.822307035 0.0271575972437859 60.0
0 0.15908545255661 61.0
131.85835995 0.945657253265381 0.0
131.85835995 0.554909646511078 1.0
130.29996198 1.12580871582031 2.0
125.51089203 5.62604188919067 3.0
0 6.02821254730225 4.0
0 4.42379522323608 5.0
-69.408023193 4.56695604324341 6.0
82.307606663 0.733927965164185 7.0
60.519772711 1.49420964717865 8.0
21.787833952 1.34753692150116 9.0
83.124911606 1.05399990081787 10.0
71.447838432 0.192697077989578 11.0
77.166040663 0.834413409233093 12.0
77.166040663 0.104352235794067 13.0
77.166040663 0.0400456115603447 14.0
0 0.00868521071970463 15.0
0 0.194999054074287 16.0
0 1.03437435626984 17.0
50.310088024 1.01625084877014 18.0
-50.310088024 0.684109330177307 19.0
-50.310088024 0.820992946624756 20.0
-50.310088024 0.0670762062072754 21.0
71.635099814 1.40235579013824 22.0
152.64433683 1.33259153366089 23.0
117.73969685 0.685846149921417 24.0
24.437330322 0.0981414318084717 25.0
224.27943665 1.2676796913147 26.0
224.27943665 1.81147980690002 27.0
27.074741153 1.9911071062088 28.0
251.3541778 1.05212509632111 29.0
251.3541778 1.96647429466248 30.0
47.181004219 1.14038228988647 31.0
9.4935076526 0.526440680027008 32.0
21.325011789 1.27337718009949 33.0
21.325011789 0.944661140441895 34.0
21.325011789 0.118653379380703 35.0
9.7910971749 0.036387674510479 36.0
8.4790808297 0.304971933364868 37.0
0 0.0369558744132519 38.0
0 0.0449009910225868 39.0
-11.465025326 0.932609438896179 40.0
-26.111967312 0.241552412509918 41.0
4.7295038686 0.578278064727783 42.0
-33.093849748 0.501690924167633 43.0
82.851075064 0.57993358373642 44.0
82.851075064 0.105433389544487 45.0
7.3113509715 0.521145880222321 46.0
2.5950520304 0.392544060945511 47.0
131.85835995 0.11400992423296 48.0
125.51089203 0.319111734628677 49.0
0 0.427175372838974 50.0
82.307606663 0.141919896006584 51.0
83.124911606 0.275352865457535 52.0
77.166040663 0.0700554922223091 53.0
77.166040663 0.122302405536175 54.0
77.166040663 0.101589471101761 55.0
0 3.8910551071167 56.0
71.635099814 0.0729790329933167 57.0
224.27943665 0.648776650428772 58.0
251.3541778 0.746319949626923 59.0
21.325011789 -0.0276654828339815 60.0
0 0.0904686376452446 61.0
140.30028978 1.00159537792206 0.0
140.30028978 1.59729635715485 1.0
139.20243377 1.80667388439178 2.0
133.7377626 2.94015789031982 3.0
3.9620819615 4.94695901870728 4.0
3.9620819615 6.8141303062439 5.0
-70.101831076 3.33117866516113 6.0
83.190247525 0.811610341072083 7.0
43.056499358 0.613907516002655 8.0
40.133748167 0.617054522037506 9.0
97.985929887 1.13034951686859 10.0
84.796589316 0.250099003314972 11.0
89.469456434 0.269041121006012 12.0
89.469456434 0.578799724578857 13.0
89.469456434 0.0707344859838486 14.0
0 0.22143067419529 15.0
0 0.168757647275925 16.0
0 0.826549470424652 17.0
54.708453257 1.42824494838715 18.0
-54.708453257 1.24255537986755 19.0
-54.708453257 0.70226776599884 20.0
-54.708453257 -0.0235871635377407 21.0
75.554102773 1.34595799446106 22.0
148.66434363 0.764562427997589 23.0
113.07434235 0.664197504520416 24.0
25.151520585 0.128225207328796 25.0
224.2184464 0.866857290267944 26.0
224.2184464 1.69773721694946 27.0
26.170285337 1.6338700056076 28.0
250.38873174 0.753095805644989 29.0
250.38873174 1.92129957675934 30.0
47.17752884 1.17634510993958 31.0
9.3584459308 0.542063593864441 32.0
20.845649516 1.13593447208405 33.0
20.845649516 1.18453073501587 34.0
20.845649516 0.130994379520416 35.0
10.488348221 0.0326739773154259 36.0
9.0973841233 0.350317418575287 37.0
0 0.0323568880558014 38.0
0 0.250634789466858 39.0
-12.040422036 0.557042598724365 40.0
-24.755573477 0.857841968536377 41.0
4.9202400811 0.373685777187347 42.0
-32.040421924 0.607205867767334 43.0
101.99709752 0.784053146839142 44.0
101.99709752 0.0949172899127007 45.0
6.9592403961 0.499616682529449 46.0
2.6638755211 0.322543919086456 47.0
140.30028978 0.12091837823391 48.0
133.7377626 0.355129808187485 49.0
3.9620819615 0.388644099235535 50.0
83.190247525 0.275308340787888 51.0
97.985929887 0.076201818883419 52.0
89.469456434 0.0759422853589058 53.0
89.469456434 0.0552894175052643 54.0
89.469456434 0.215546116232872 55.0
0 3.25332880020142 56.0
75.554102773 0.222136840224266 57.0
224.2184464 0.669027149677277 58.0
250.38873174 0.620394110679626 59.0
20.845649516 0.00656664744019508 60.0
0 0.0928671211004257 61.0
182.90203464 0.847197771072388 0.0
182.90203464 0.924566566944122 1.0
181.81154714 1.24528884887695 2.0
176.3468764 3.84437990188599 3.0
8.4715588892 4.55019474029541 4.0
8.4715588892 3.71822023391724 5.0
-66.016469096 0.448728829622269 6.0
76.260715465 0.450833588838577 7.0
47.786145503 1.02998828887939 8.0
28.474569962 1.19185817241669 9.0
85.834800708 1.03945517539978 10.0
73.482098555 0.211714640259743 11.0
81.582045369 0.566912412643433 12.0
81.582045369 0.236770957708359 13.0
81.582045369 0.0824932232499123 14.0
0 0.13063833117485 15.0
0 0.239970177412033 16.0
0 1.28949189186096 17.0
55.109842226 1.61409366130829 18.0
-55.109842226 1.68836224079132 19.0
-55.109842226 0.279658168554306 20.0
-55.109842226 0.0299070477485657 21.0
74.701088049 1.30299687385559 22.0
137.8337799 1.1361163854599 23.0
101.45060379 0.359279125928879 24.0
20.505936181 0.124597035348415 25.0
212.53486795 0.695471227169037 26.0
212.53486795 2.03699707984924 27.0
27.616558717 2.21569895744324 28.0
240.15142666 1.18264555931091 29.0
240.15142666 1.80905699729919 30.0
46.954558637 0.953103005886078 31.0
9.3615193596 0.51999044418335 32.0
19.591245823 1.22342872619629 33.0
19.591245823 1.02287912368774 34.0
19.591245823 0.142892673611641 35.0
8.014909238 0.0221016891300678 36.0
6.8776769506 0.391548603773117 37.0
0 0.019455261528492 38.0
0 0.17794194817543 39.0
-12.569881335 0.776163816452026 40.0
-25.365664789 0.434785962104797 41.0
5.3852267135 0.593196213245392 42.0
-33.452560638 0.493652045726776 43.0
83.496242549 0.721484780311584 44.0
83.496242549 0.0978720337152481 45.0
6.707045444 0.491716533899307 46.0
2.3765604885 0.35449892282486 47.0
182.90203464 0.144023582339287 48.0
176.3468764 0.104659475386143 49.0
8.4715588892 0.241222128272057 50.0
76.260715465 0.141377568244934 51.0
85.834800708 0.496100753545761 52.0
81.582045369 0.0418563783168793 53.0
81.582045369 0.0263694506138563 54.0
81.582045369 0.077358178794384 55.0
0 4.86886501312256 56.0
74.701088049 0.141395062208176 57.0
212.53486795 0.572321951389313 58.0
240.15142666 1.12628948688507 59.0
19.591245823 -0.00724457763135433 60.0
0 0.112759545445442 61.0
161.2869866 0.906634926795959 0.0
161.2869866 1.05005276203156 1.0
159.88819322 1.20476353168488 2.0
154.42355782 3.00400710105896 3.0
20.469501097 6.82030630111694 4.0
20.469501097 7.04210901260376 5.0
-45.133540285 5.27382850646973 6.0
55.268542043 0.876728057861328 7.0
36.867377523 1.17502593994141 8.0
18.40116452 1.29789733886719 9.0
94.247736881 0.906531572341919 10.0
81.655697371 0.082622118294239 11.0
80.30921307 0.859784424304962 12.0
80.30921307 0.95373147726059 13.0
80.30921307 0.126770690083504 14.0
0 0.293058842420578 15.0
0 0.197223678231239 16.0
0 1.55071020126343 17.0
54.846088471 1.10168969631195 18.0
-54.846088471 1.43354892730713 19.0
-54.846088471 0.736591339111328 20.0
-54.846088471 -0.111603751778603 21.0
76.825265434 1.86093544960022 22.0
151.34031774 2.16717720031738 23.0
119.35798967 0.795600652694702 24.0
22.460051186 0.139720678329468 25.0
228.16558317 1.21255218982697 26.0
228.16558317 2.6541314125061 27.0
29.98559772 2.59719967842102 28.0
258.15118089 1.47681331634521 29.0
258.15118089 2.06452751159668 30.0
46.499528046 0.932788074016571 31.0
9.5393201807 0.569084167480469 32.0
21.979176964 2.00349378585815 33.0
21.979176964 0.696891486644745 34.0
21.979176964 0.139584869146347 35.0
10.453657997 0.0483055673539639 36.0
9.1802875729 0.350208073854446 37.0
0 0.0371603667736053 38.0
0 0.0578051656484604 39.0
-12.686334797 0.525694489479065 40.0
-26.441202474 0.730887889862061 41.0
5.6643057656 0.315625697374344 42.0
-34.672134135 0.235298752784729 43.0
92.862632959 0.902595460414886 44.0
92.862632959 0.0770036354660988 45.0
7.846165307 0.668014228343964 46.0
2.7949140178 0.417083203792572 47.0
161.2869866 0.0776653066277504 48.0
154.42355782 0.0642123594880104 49.0
20.469501097 0.647704720497131 50.0
55.268542043 0.528245329856873 51.0
94.247736881 0.0483779758214951 52.0
80.30921307 0.13781550526619 53.0
80.30921307 0.278505146503448 54.0
80.30921307 0.275727093219757 55.0
0 4.92901849746704 56.0
76.825265434 0.342646181583405 57.0
228.16558317 0.69716215133667 58.0
258.15118089 0.598486363887787 59.0
21.979176964 0.00169607438147068 60.0
0 0.132342502474785 61.0
165.84326542 0.781323313713074 0.0
165.84326542 1.07338798046112 1.0
164.80476871 1.45134377479553 2.0
159.34009797 5.88585329055786 3.0
0.87574008468 5.62201929092407 4.0
0.87574008468 5.78354454040527 5.0
-66.598299373 1.86205291748047 6.0
77.262594178 0.508937358856201 7.0
44.841360359 0.715593993663788 8.0
32.421233819 0.850389063358307 9.0
85.472987722 0.946906924247742 10.0
72.265886473 0.136351630091667 11.0
82.938562668 1.08616542816162 12.0
82.938562668 0.622835755348206 13.0
82.938562668 0.122306168079376 14.0
0 0.243193313479424 15.0
0 0.164306282997131 16.0
0 1.39084470272064 17.0
54.738708666 1.2197173833847 18.0
-54.738708666 1.52804982662201 19.0
-54.738708666 0.459493070840836 20.0
-54.738708666 0.0454011224210262 21.0
77.834141589 1.75365924835205 22.0
147.62930966 1.61463010311127 23.0
113.81067267 0.41884458065033 24.0
22.332620393 0.0910594910383224 25.0
225.46345125 0.773357808589935 26.0
225.46345125 1.65966761112213 27.0
28.530444941 2.02646088600159 28.0
253.99389619 1.09089350700378 29.0
253.99389619 1.88139450550079 30.0
46.400661754 0.957505881786346 31.0
9.3509216544 0.352362155914307 32.0
23.095432923 1.33756816387177 33.0
23.095432923 1.14709496498108 34.0
23.095432923 0.133815824985504 35.0
10.466583801 0.0323125869035721 36.0
9.1089409265 0.339475512504578 37.0
0 0.0187030993402004 38.0
0 0.235171929001808 39.0
-12.262784198 0.493833929300308 40.0
-24.405458806 0.641516804695129 41.0
5.3839792218 0.505289793014526 42.0
-32.265540716 0.48615887761116 43.0
84.882651157 0.800077974796295 44.0
84.882651157 0.0975278988480568 45.0
6.6239115123 0.675122261047363 46.0
2.3578462882 0.364879280328751 47.0
165.84326542 0.0476034507155418 48.0
159.34009797 0.21718367934227 49.0
0.87574008468 0.28727063536644 50.0
77.262594178 0.20073327422142 51.0
85.472987722 0.193304225802422 52.0
82.938562668 0.0677031204104424 53.0
82.938562668 0.177913144230843 54.0
82.938562668 0.085725836455822 55.0
0 3.77537155151367 56.0
77.834141589 0.0475500971078873 57.0
225.46345125 0.204742550849915 58.0
253.99389619 0.807002305984497 59.0
23.095432923 -0.00876359175890684 60.0
0 0.111105248332024 61.0
194.34052746 0.761185586452484 0.0
194.34052746 1.3798531293869 1.0
193.252508 1.05931270122528 2.0
192.15601062 5.14484596252441 3.0
0 6.57300281524658 4.0
0 5.94134902954102 5.0
-67.685024885 0.452302008867264 6.0
80.527180408 1.32929599285126 7.0
59.815617345 0.478050291538239 8.0
20.711563063 0.431487262248993 9.0
69.469818022 0.794248282909393 10.0
57.639566252 0.0722789615392685 11.0
67.185585095 0.63268905878067 12.0
67.185585095 0.435655295848846 13.0
67.185585095 0.0653378143906593 14.0
20.552521613 0.246414452791214 15.0
20.552521613 0.18652968108654 16.0
0 1.4290292263031 17.0
55.086664725 0.806843161582947 18.0
-55.086664725 0.605705976486206 19.0
-55.086664725 1.15607976913452 20.0
-55.086664725 -0.0512073412537575 21.0
76.821403734 1.05593633651733 22.0
144.42840509 1.14723014831543 23.0
112.08117699 0.48763245344162 24.0
19.634122797 0.101042702794075 25.0
221.24980882 1.01305043697357 26.0
221.24980882 1.37590086460114 27.0
30.825414664 1.22212362289429 28.0
252.07522348 0.610109150409698 29.0
252.07522348 1.45550692081451 30.0
46.489507678 1.18962979316711 31.0
9.3780962248 0.343586325645447 32.0
21.734739009 1.52971196174622 33.0
21.734739009 0.779254257678986 34.0
21.734739009 0.0883415564894676 35.0
9.6420421482 0.0233556032180786 36.0
8.2683457115 0.392414599657059 37.0
0 0.0880862548947334 38.0
0 0.18139123916626 39.0
-13.303174551 0.862480700016022 40.0
-27.496499629 0.69936865568161 41.0
5.3451291971 0.290455996990204 42.0
-35.091221678 0.49828690290451 43.0
93.88631236 0.594528317451477 44.0
93.88631236 0.065801702439785 45.0
6.3853361933 0.665908575057983 46.0
2.3315644971 0.40072375535965 47.0
194.34052746 0.124421782791615 48.0
192.15601062 0.0836865603923798 49.0
0 0.342918813228607 50.0
80.527180408 0.358112215995789 51.0
69.469818022 0.108340598642826 52.0
67.185585095 0.051696103066206 53.0
67.185585095 0.14995488524437 54.0
67.185585095 0.0719606876373291 55.0
0 0.495564252138138 56.0
76.821403734 0.000243996270000935 57.0
221.24980882 0.401590377092361 58.0
252.07522348 0.25409197807312 59.0
21.734739009 -0.0103092156350613 60.0
0 0.0938137024641037 61.0
144.43560787 0.894717037677765 0.0
144.43560787 1.01273107528687 1.0
142.95493412 1.89775228500366 2.0
138.16582903 3.26865124702454 3.0
0 5.44690799713135 4.0
0 4.71548652648926 5.0
-60.140665632 0.486290097236633 6.0
73.321668393 0.521738588809967 7.0
55.104172517 1.07550454139709 8.0
18.217495877 1.04706227779388 9.0
84.11879186 0.856296718120575 10.0
71.598999477 0.0261367373168468 11.0
78.040119021 0.329548478126526 12.0
78.040119021 0.100388526916504 13.0
78.040119021 0.107052534818649 14.0
0 0.123780861496925 15.0
0 0.191175758838654 16.0
0 1.20076763629913 17.0
59.429838235 1.90669131278992 18.0
-59.429838235 2.23780703544617 19.0
-59.429838235 0.322953671216965 20.0
-59.429838235 -0.00706754997372627 21.0
80.987269054 2.18407344818115 22.0
142.46931718 2.01910996437073 23.0
108.75233455 0.218332514166832 24.0
24.14729464 0.236103147268295 25.0
223.45658623 1.54000246524811 26.0
223.45658623 2.1206533908844 27.0
28.437965598 1.50281655788422 28.0
251.89455183 1.21171820163727 29.0
251.89455183 1.84961950778961 30.0
46.28869026 1.22880363464355 31.0
9.3617299349 0.316073954105377 32.0
21.557430819 1.61115729808807 33.0
21.557430819 0.93638277053833 34.0
21.557430819 0.129272952675819 35.0
9.0184374292 0.0299768578261137 36.0
7.727940605 0.379694998264313 37.0
0 0.0377111025154591 38.0
0 0.204715803265572 39.0
-12.56633907 0.445080041885376 40.0
-26.603321117 0.699040055274963 41.0
5.4151583614 0.511940538883209 42.0
-34.371534147 0.5956791639328 43.0
100.74583251 0.76312130689621 44.0
100.74583251 0.0967294052243233 45.0
6.8047590509 0.563535153865814 46.0
2.838804184 0.339968711137772 47.0
144.43560787 0.105710610747337 48.0
138.16582903 0.420194029808044 49.0
0 0.28211897611618 50.0
73.321668393 0.244058385491371 51.0
84.11879186 0.118440791964531 52.0
78.040119021 0.0248661264777184 53.0
78.040119021 0.0439162366092205 54.0
78.040119021 0.0633271783590317 55.0
0 5.01200246810913 56.0
80.987269054 0.0970091596245766 57.0
223.45658623 0.775679171085358 58.0
251.89455183 0.993179440498352 59.0
21.557430819 0.0156341250985861 60.0
0 0.0927159935235977 61.0
143.21630231 0.965808987617493 0.0
143.21630231 1.01271152496338 1.0
141.89317635 1.26916754245758 2.0
137.10407121 5.22333574295044 3.0
0 6.90927028656006 4.0
0 5.58212566375732 5.0
-76.574642219 1.12778341770172 6.0
86.981625332 0.691325545310974 7.0
57.898138319 0.722372233867645 8.0
29.083487014 0.846495449542999 9.0
79.124450999 1.01644420623779 10.0
66.941361752 0.151142820715904 11.0
78.40744407 0.389160454273224 12.0
78.40744407 0.113212883472443 13.0
78.40744407 0.0420248098671436 14.0
0 0.126407146453857 15.0
0 0.197304219007492 16.0
0 1.05273103713989 17.0
61.306152682 1.53409004211426 18.0
-61.306152682 1.12049281597137 19.0
-61.306152682 0.389587044715881 20.0
-61.306152682 -0.0742210298776627 21.0
82.730000871 1.30658221244812 22.0
147.48674607 0.658249318599701 23.0
110.51438039 0.761428892612457 24.0
24.491842901 0.0870746448636055 25.0
230.21674694 1.08994650840759 26.0
230.21674694 1.93293082714081 27.0
27.760741456 2.50550413131714 28.0
257.97748839 1.18939590454102 29.0
257.97748839 1.7573835849762 30.0
46.750308967 0.991289556026459 31.0
9.2858269016 0.458668649196625 32.0
21.423848189 2.45871925354004 33.0
21.423848189 1.11568009853363 34.0
21.423848189 0.128151372075081 35.0
9.3009193365 0.0369922257959843 36.0
8.0664162123 0.402464985847473 37.0
0 0.0355864837765694 38.0
0 0.259031057357788 39.0
-12.9697041 0.521072745323181 40.0
-24.976590218 0.727519273757935 41.0
5.0274523815 0.396930515766144 42.0
-32.463802578 0.639397621154785 43.0
94.167552892 0.626680552959442 44.0
94.167552892 0.086899146437645 45.0
7.4937021282 0.622768104076385 46.0
2.6395367575 0.423579335212708 47.0
143.21630231 0.101441115140915 48.0
137.10407121 0.0881232246756554 49.0
0 0.7409747838974 50.0
86.981625332 0.125358775258064 51.0
79.124450999 0.277405470609665 52.0
78.40744407 0.0505876839160919 53.0
78.40744407 0.102312542498112 54.0
78.40744407 0.0856856852769852 55.0
0 4.09775066375732 56.0
82.730000871 0.193446159362793 57.0
230.21674694 0.704267621040344 58.0
257.97748839 0.77967381477356 59.0
21.423848189 -0.0140598770231009 60.0
0 0.08720912784338 61.0
176.85325592 1.04235315322876 0.0
176.85325592 1.77983272075653 1.0
175.91648528 2.09648108482361 2.0
171.12740174 2.46824955940247 3.0
0 5.21900224685669 4.0
0 4.50620698928833 5.0
-45.990724428 1.15310704708099 6.0
56.007713256 0.663498103618622 7.0
59.583410624 0.755551993846893 8.0
-3.575697368 0.967554211616516 9.0
87.970576228 1.14484417438507 10.0
74.761876424 0.0616725832223892 11.0
90.397067076 0.394799590110779 12.0
90.397067076 0.0678518116474152 13.0
90.397067076 0.207248762249947 14.0
0 0.399231553077698 15.0
0 0.126829296350479 16.0
0 2.13037395477295 17.0
55.082657739 1.88696455955505 18.0
-55.082657739 1.87481451034546 19.0
-55.082657739 0.505836069583893 20.0
-55.082657739 0.00253478065133095 21.0
75.798181801 1.44508576393127 22.0
141.14129213 1.20976686477661 23.0
109.61819006 0.99113667011261 24.0
20.67890352 0.183866858482361 25.0
216.93947393 1.854660987854 26.0
216.93947393 2.09457802772522 27.0
30.204940659 1.99734497070312 28.0
247.14441459 0.468423783779144 29.0
247.14441459 1.90793585777283 30.0
46.559250559 1.13802015781403 31.0
9.3687725372 0.520727872848511 32.0
20.715524062 1.36462378501892 33.0
20.715524062 1.07923126220703 34.0
20.715524062 0.110034123063087 35.0
8.2456867597 0.0192099697887897 36.0
7.2173289948 0.403599560260773 37.0
0 0.0361881926655769 38.0
0 0.182598173618317 39.0
-12.981561134 0.862290024757385 40.0
-23.816352754 0.850486159324646 41.0
4.3966117144 0.389092057943344 42.0
-30.735042835 0.494398057460785 43.0
77.833111936 0.501672625541687 44.0
77.833111936 0.116604723036289 45.0
7.6474397507 0.523315906524658 46.0
2.652943753 0.426422327756882 47.0
176.85325592 0.0409147627651691 48.0
171.12740174 0.256828397512436 49.0
0 0.273237526416779 50.0
56.007713256 0.052410826086998 51.0
87.970576228 0.0387804619967937 52.0
90.397067076 0.00979736261069775 53.0
90.397067076 0.0870038717985153 54.0
90.397067076 0.0818584263324738 55.0
0 4.82207584381104 56.0
75.798181801 -0.000445364974439144 57.0
216.93947393 1.23508048057556 58.0
247.14441459 0.775281608104706 59.0
20.715524062 -0.0178663413971663 60.0
0 0.0840506255626678 61.0
149.18052322 0.841620326042175 0.0
149.18052322 1.1962890625 1.0
147.97110856 1.30057346820831 2.0
143.18200347 4.30950450897217 3.0
0 5.78167247772217 4.0
0 4.47885036468506 5.0
-74.773807991 0.529173851013184 6.0
84.910602432 0.472301185131073 7.0
63.859952733 1.31601977348328 8.0
21.0506497 1.46146273612976 9.0
91.813790141 0.980816066265106 10.0
77.615325411 0.293456375598907 11.0
78.527038468 0.87518322467804 12.0
78.527038468 0.544931054115295 13.0
78.527038468 0.109824553132057 14.0
0 0.0483336821198463 15.0
0 0.245590835809708 16.0
0 1.1684045791626 17.0
56.879320994 1.23237454891205 18.0
-56.879320994 1.37918567657471 19.0
-56.879320994 0.580763339996338 20.0
-56.879320994 -0.0694181174039841 21.0
78.640827411 1.0710654258728 22.0
153.94615385 0.946940422058105 23.0
120.47720819 0.395108014345169 24.0
23.644096034 0.389421880245209 25.0
232.58698127 0.7884441614151 26.0
232.58698127 1.68479883670807 27.0
23.50598265 2.45771646499634 28.0
256.09296392 0.973151206970215 29.0
256.09296392 1.66548252105713 30.0
46.764704937 1.01438546180725 31.0
9.3498190061 0.445250928401947 32.0
21.761506417 1.7077978849411 33.0
21.761506417 0.913278758525848 34.0
21.761506417 0.115741111338139 35.0
9.7605077334 0.0355382934212685 36.0
8.4854269595 0.420527577400208 37.0
0 0.0305926278233528 38.0
0 0.187667369842529 39.0
-11.576263405 0.88667494058609 40.0
-23.503182636 0.598153650760651 41.0
4.6596094508 0.35718435049057 42.0
-30.671324217 0.354856997728348 43.0
104.44730732 0.729674220085144 44.0
104.44730732 0.0939816907048225 45.0
7.1128502707 0.575926423072815 46.0
2.6468372562 0.398884415626526 47.0
149.18052322 0.118587791919708 48.0
143.18200347 0.373399078845978 49.0
0 0.20618724822998 50.0
84.910602432 0.148440837860107 51.0
91.813790141 0.130984947085381 52.0
78.527038468 0.0986789464950562 53.0
78.527038468 0.127662822604179 54.0
78.527038468 0.241172164678574 55.0
0 6.53330183029175 56.0
78.640827411 0.190323352813721 57.0
232.58698127 0.418785333633423 58.0
256.09296392 0.288910567760468 59.0
21.761506417 -0.0157266557216644 60.0
0 0.104913800954819 61.0
137.56043147 0.904108464717865 0.0
137.56043147 2.06723213195801 1.0
136.44430481 1.92251348495483 2.0
131.65523485 4.8733344078064 3.0
0 4.3642578125 4.0
0 4.52913570404053 5.0
-56.193513877 1.38340091705322 6.0
67.54532865 0.951561570167542 7.0
46.95329248 0.756254017353058 8.0
20.59203617 0.863876819610596 9.0
90.171924605 0.908676981925964 10.0
77.070752959 0.349200755357742 11.0
75.754159544 0.233654469251633 12.0
75.754159544 0.431835681200027 13.0
75.754159544 0.136262208223343 14.0
0 0.125022917985916 15.0
0 0.166553914546967 16.0
0 1.37885975837708 17.0
60.610036531 0.926784992218018 18.0
-60.610036531 0.647256672382355 19.0
-60.610036531 0.967427968978882 20.0
-60.610036531 -0.0902962535619736 21.0
82.374704051 1.98298120498657 22.0
140.15929194 1.61626982688904 23.0
107.07139029 0.500727474689484 24.0
21.91458185 0.0705893039703369 25.0
222.53399599 1.41040861606598 26.0
222.53399599 1.31307470798492 27.0
29.023775877 1.74923396110535 28.0
251.55777186 1.02105343341827 29.0
251.55777186 1.79544317722321 30.0
46.283224934 1.09865891933441 31.0
9.2494271786 0.374184519052505 32.0
21.76466752 1.52548253536224 33.0
21.76466752 1.14561724662781 34.0
21.76466752 0.0940260961651802 35.0
9.1349352902 0.0209016352891922 36.0
7.9909450997 0.389445394277573 37.0
0 0.014657411724329 38.0
0 0.112744569778442 39.0
-10.615595153 1.01173162460327 40.0
-22.922902166 0.422024548053741 41.0
5.5227563374 0.475604027509689 42.0
-30.799555843 0.5222207903862 43.0
86.3021903 0.636192083358765 44.0
86.3021903 0.108538024127483 45.0
6.0063811814 0.604609608650208 46.0
2.3320763561 0.404065310955048 47.0
137.56043147 0.25351083278656 48.0
131.65523485 0.480378478765488 49.0
0 0.338818788528442 50.0
67.54532865 0.355746626853943 51.0
90.171924605 0.0409551225602627 52.0
75.754159544 0.100392132997513 53.0
75.754159544 0.0312614813446999 54.0
75.754159544 0.156649872660637 55.0
0 8.23819923400879 56.0
82.374704051 0.168398857116699 57.0
222.53399599 0.542163729667664 58.0
251.55777186 0.374362021684647 59.0
21.76466752 0.0274665281176567 60.0
0 0.0927803516387939 61.0
163.38575052 0.827582955360413 0.0
163.38575052 0.728432714939117 1.0
161.92623051 1.2765885591507 2.0
157.13712527 5.79088401794434 3.0
0 5.22148418426514 4.0
0 4.91958618164062 5.0
-65.211645717 3.27881050109863 6.0
75.686714109 0.520151853561401 7.0
41.497475428 1.09350574016571 8.0
34.189238681 1.06405889987946 9.0
88.63589219 0.981908202171326 10.0
75.652927705 0.209633648395538 11.0
84.719064816 0.353706926107407 12.0
84.719064816 0.535862445831299 13.0
84.719064816 0.0510285049676895 14.0
0 0.129496842622757 15.0
0 0.193750858306885 16.0
0 0.958986520767212 17.0
54.654886348 1.30002474784851 18.0
-54.654886348 1.41824400424957 19.0
-54.654886348 0.25850909948349 20.0
-54.654886348 0.0380379483103752 21.0
76.734751357 1.22253704071045 22.0
140.92172622 0.52126556634903 23.0
108.79286927 0.551508605480194 24.0
21.584570504 0.10327422618866 25.0
217.65647758 0.955248296260834 26.0
217.65647758 2.08959364891052 27.0
26.601581784 2.46023082733154 28.0
244.25805936 1.18229830265045 29.0
244.25805936 1.94009184837341 30.0
46.299871492 0.969262719154358 31.0
9.4097577574 0.409972339868546 32.0
22.079865009 1.20994591712952 33.0
22.079865009 1.25210726261139 34.0
22.079865009 0.0967244952917099 35.0
10.128961149 0.0231865625828505 36.0
8.8923543526 0.37094458937645 37.0
0 0.00643806532025337 38.0
0 0.120787732303143 39.0
-11.823407979 0.769196689128876 40.0
-24.93393226 0.433295071125031 41.0
4.6910028316 0.50788676738739 42.0
-32.370550992 0.583837568759918 43.0
89.095605281 0.663005828857422 44.0
89.095605281 0.116068877279758 45.0
6.5483952145 0.505173325538635 46.0
2.3466560888 0.360002964735031 47.0
163.38575052 0.0525677725672722 48.0
157.13712527 0.246758937835693 49.0
0 0.515691220760345 50.0
75.686714109 0.29684442281723 51.0
88.63589219 0.315359771251678 52.0
84.719064816 0.0662712678313255 53.0
84.719064816 0.0377474576234818 54.0
84.719064816 0.0998777449131012 55.0
0 0.616901457309723 56.0
76.734751357 0.110389716923237 57.0
217.65647758 0.519420802593231 58.0
244.25805936 1.02796149253845 59.0
22.079865009 -0.0156004372984171 60.0
0 0.0936206206679344 61.0
173.91001931 1.15772294998169 0.0
173.91001931 1.73064649105072 1.0
172.59241463 2.11953067779541 2.0
167.8033094 4.16673183441162 3.0
0 7.1259822845459 4.0
0 3.66405725479126 5.0
-72.497030234 0.610064446926117 6.0
84.497685108 0.587072789669037 7.0
48.815409621 1.42732393741608 8.0
35.682275488 1.34219777584076 9.0
96.891205468 1.38219237327576 10.0
83.879020908 0.473712086677551 11.0
86.170825641 0.879441559314728 12.0
86.170825641 0.651417851448059 13.0
86.170825641 0.0977456271648407 14.0
0 0.235196381807327 15.0
0 0.152822583913803 16.0
0 1.47883152961731 17.0
54.499936454 1.50303983688354 18.0
-54.499936454 1.00854241847992 19.0
-54.499936454 0.631685853004456 20.0
-54.499936454 -0.0531449727714062 21.0
76.990925105 1.69277703762054 22.0
142.25380253 0.845031380653381 23.0
109.00045113 0.851466655731201 24.0
22.088838759 0.0652391016483307 25.0
219.24472764 0.633745670318604 26.0
219.24472764 0.975595653057098 27.0
26.347585711 1.31431531906128 28.0
245.59231335 0.730932474136353 29.0
245.59231335 1.81340491771698 30.0
46.728206379 1.10413718223572 31.0
9.2311319023 0.445535719394684 32.0
22.490988651 1.56178665161133 33.0
22.490988651 0.982765316963196 34.0
22.490988651 0.135831937193871 35.0
9.3731077064 0.0235105976462364 36.0
7.9798702496 0.388113617897034 37.0
0 0.0680063813924789 38.0
0 0.12972317636013 39.0
-11.709069229 0.691423058509827 40.0
-25.59090651 0.578798174858093 41.0
5.4243369778 0.569079458713531 42.0
-33.285426608 0.527937412261963 43.0
93.432776531 0.696666121482849 44.0
93.432776531 0.103252157568932 45.0
6.574376034 0.652121722698212 46.0
2.5382211718 0.410839200019836 47.0
173.91001931 0.140243336558342 48.0
167.8033094 0.519272744655609 49.0
0 0.168116122484207 50.0
84.497685108 0.210132077336311 51.0
96.891205468 0.418892323970795 52.0
86.170825641 0.0988698601722717 53.0
86.170825641 0.156703308224678 54.0
86.170825641 0.197616502642632 55.0
0 3.38128280639648 56.0
76.990925105 0.0465599521994591 57.0
219.24472764 0.178513377904892 58.0
245.59231335 0.564245223999023 59.0
22.490988651 0.00797596201300621 60.0
0 0.0938449129462242 61.0
141.00442144 0.990773260593414 0.0
141.00442144 0.538806200027466 1.0
139.97439003 1.03548645973206 2.0
135.18528471 2.63751149177551 3.0
0 5.58022737503052 4.0
0 5.34977054595947 5.0
-58.493258839 6.35739707946777 6.0
68.121293435 0.492199420928955 7.0
55.389659656 1.543177485466 8.0
12.731633779 1.48571383953094 9.0
90.868432951 0.978732526302338 10.0
77.410644275 0.121585004031658 11.0
82.747251097 0.292737662792206 12.0
82.747251097 0.369154810905457 13.0
82.747251097 0.0746416226029396 14.0
0 0.160627022385597 15.0
0 0.189820691943169 16.0
0 1.27152609825134 17.0
60.169901758 0.679430603981018 18.0
-60.169901758 0.763729214668274 19.0
-60.169901758 0.415321350097656 20.0
-60.169901758 0.182445541024208 21.0
80.476079553 1.52921175956726 22.0
154.20008576 1.42128717899323 23.0
117.58955719 0.620153844356537 24.0
24.670008207 0.177594065666199 25.0
234.67616531 0.915195286273956 26.0
234.67616531 1.73889338970184 27.0
25.328565395 1.0239474773407 28.0
260.0047307 0.565818548202515 29.0
260.0047307 1.59438335895538 30.0
45.995091173 1.08391463756561 31.0
9.1945182233 0.774565637111664 32.0
20.306177794 0.757842600345612 33.0
20.306177794 1.09599339962006 34.0
20.306177794 0.103597566485405 35.0
10.326495283 0.0265918932855129 36.0
9.1101985662 0.377424687147141 37.0
0 0.0186282731592655 38.0
0 0.160861298441887 39.0
-11.829643341 0.663420379161835 40.0
-24.785746923 0.586789429187775 41.0
4.7910244999 0.511893749237061 42.0
-31.85994565 0.580662190914154 43.0
75.764021082 0.699444890022278 44.0
75.764021082 0.0902950316667557 45.0
6.8734038886 0.628467738628387 46.0
2.8175399447 0.455810725688934 47.0
141.00442144 0.135351285338402 48.0
135.18528471 0.104131326079369 49.0
0 0.57092273235321 50.0
68.121293435 0.238141000270844 51.0
90.868432951 0.0695768147706985 52.0
82.747251097 0.0581979379057884 53.0
82.747251097 0.159809768199921 54.0
82.747251097 0.0606410428881645 55.0
0 1.77166056632996 56.0
80.476079553 0.0194464847445488 57.0
234.67616531 0.231686979532242 58.0
260.0047307 0.487064093351364 59.0
20.306177794 -0.0144273918122053 60.0
0 0.0995539128780365 61.0
159.62273392 0.894117832183838 0.0
159.62273392 1.67215502262115 1.0
158.26794924 2.01823997497559 2.0
153.47884396 4.20364332199097 3.0
0 3.88762545585632 4.0
0 5.15535926818848 5.0
-62.140553457 1.11869430541992 6.0
73.497834444 0.44405198097229 7.0
48.430304965 1.21036720275879 8.0
25.067529478 1.26033306121826 9.0
89.556653846 0.97769570350647 10.0
76.22519728 0.458783864974976 11.0
80.135660708 1.11247098445892 12.0
80.135660708 0.728425264358521 13.0
80.135660708 0.194767206907272 14.0
0 0.335300266742706 15.0
0 0.174559563398361 16.0
0 1.7468626499176 17.0
52.282505982 1.30522418022156 18.0
-52.282505982 1.14387559890747 19.0
-52.282505982 0.293693035840988 20.0
-52.282505982 -0.0340676121413708 21.0
74.726947873 1.67090392112732 22.0
140.02306618 1.29113948345184 23.0
104.53120831 0.68634444475174 24.0
22.282083862 0.0830674916505814 25.0
214.75001406 0.733366787433624 26.0
214.75001406 1.41994142532349 27.0
27.469389764 1.83326721191406 28.0
242.21940382 1.01978766918182 29.0
242.21940382 1.90318083763123 30.0
46.380936858 0.974884867668152 31.0
9.2823928296 0.516964852809906 32.0
22.444441891 1.74370455741882 33.0
22.444441891 0.857721567153931 34.0
22.444441891 0.113863065838814 35.0
9.7568871205 0.0241906642913818 36.0
8.4037683939 0.359003812074661 37.0
0 0.00401638820767403 38.0
0 0.132218554615974 39.0
-12.715371589 0.589220345020294 40.0
-25.117034735 0.670976102352142 41.0
5.3957828528 0.438043177127838 42.0
-32.874404399 0.576683700084686 43.0
88.392567318 0.755380511283875 44.0
88.392567318 0.100035116076469 45.0
6.2140224337 0.683128237724304 46.0
2.202036917 0.408464163541794 47.0
159.62273392 0.105323396623135 48.0
153.47884396 0.160353139042854 49.0
0 0.362878382205963 50.0
73.497834444 0.1366957873106 51.0
89.556653846 0.343631237745285 52.0
80.135660708 0.184491112828255 53.0
80.135660708 0.155702441930771 54.0
80.135660708 0.105602994561195 55.0
0 5.56813192367554 56.0
74.726947873 0.0408392325043678 57.0
214.75001406 0.501348793506622 58.0
242.21940382 0.826191246509552 59.0
22.444441891 -0.000192150473594666 60.0
0 0.0990442261099815 61.0
170.74940841 0.953086316585541 0.0
170.74940841 0.610057950019836 1.0
169.60912814 1.54941153526306 2.0
164.82002292 3.51663184165955 3.0
0 5.82502555847168 4.0
0 2.10453271865845 5.0
-51.317267491 0.795784831047058 6.0
63.617865507 0.499444454908371 7.0
54.571120162 1.34563612937927 8.0
9.0467453451 1.41618549823761 9.0
82.473451168 0.956189036369324 10.0
70.261004987 0.00416558235883713 11.0
78.516426532 1.0510665178299 12.0
78.516426532 0.870483815670013 13.0
78.516426532 0.0937910452485085 14.0
0 0.0160609744489193 15.0
0 0.196579158306122 16.0
0 1.26263618469238 17.0
58.582272607 0.979717195034027 18.0
-58.582272607 1.12418556213379 19.0
-58.582272607 0.314383447170258 20.0
-58.582272607 0.0902600064873695 21.0
79.407062515 1.8688440322876 22.0
150.88543358 1.56098902225494 23.0
114.80434632 0.499214231967926 24.0
25.796225902 0.152487874031067 25.0
230.2924961 0.845035374164581 26.0
230.2924961 1.82720506191254 27.0
30.433403994 1.85124373435974 28.0
260.72590009 0.960375428199768 29.0
260.72590009 1.70460629463196 30.0
46.365731601 1.02362489700317 31.0
9.4489715588 0.503254652023315 32.0
20.824789908 0.808700501918793 33.0
20.824789908 0.961238086223602 34.0
20.824789908 0.0738470703363419 35.0
9.4504477201 0.0184555016458035 36.0
8.0452462207 0.413222551345825 37.0
0 0.0384225323796272 38.0
0 0.314801394939423 39.0
-13.085178144 0.448435246944427 40.0
-25.944050049 0.492268979549408 41.0
4.5922566721 0.388338953256607 42.0
-33.481920882 0.604401290416718 43.0
103.29439126 0.836163938045502 44.0
103.29439126 0.0795269757509232 45.0
7.4842189635 0.59428882598877 46.0
2.2402877255 0.387551605701447 47.0
170.74940841 0.0499822162091732 48.0
164.82002292 0.297966480255127 49.0
0 0.113898433744907 50.0
63.617865507 0.11105889081955 51.0
82.473451168 0.277929931879044 52.0
78.516426532 0.00483025517314672 53.0
78.516426532 0.40127244591713 54.0
78.516426532 0.0671935677528381 55.0
0 4.39813089370728 56.0
79.407062515 -0.0453195944428444 57.0
230.2924961 0.214903756976128 58.0
260.72590009 0.850694000720978 59.0
20.824789908 -0.0126935057342052 60.0
0 0.118502333760262 61.0
175.49338028 1.17088294029236 0.0
175.49338028 1.50177586078644 1.0
174.15919264 1.97589826583862 2.0
169.37008746 5.29410314559937 3.0
0 5.369300365448 4.0
0 5.44490098953247 5.0
-74.404685655 3.81628394126892 6.0
86.274275568 0.875296831130981 7.0
53.401617017 0.826933801174164 8.0
32.872658551 0.570888340473175 9.0
93.151840683 1.24269247055054 10.0
80.311330442 0.203786194324493 11.0
86.174446131 0.996284008026123 12.0
86.174446131 0.418695211410522 13.0
86.174446131 0.118660181760788 14.0
0 0.274832636117935 15.0
0 0.141569152474403 16.0
0 1.41173267364502 17.0
53.202747677 1.0445613861084 18.0
-53.202747677 0.923795461654663 19.0
-53.202747677 0.698735535144806 20.0
-53.202747677 -0.0330849774181843 21.0
75.796048508 2.03443241119385 22.0
147.48562868 1.96001052856445 23.0
110.99250144 0.191248834133148 24.0
20.846918293 0.259311378002167 25.0
223.28167719 1.55273914337158 26.0
223.28167719 2.60256242752075 27.0
24.179109377 2.21462345123291 28.0
247.46078656 0.954282879829407 29.0
247.46078656 1.86111783981323 30.0
47.098827299 1.16947841644287 31.0
9.2854003826 0.575474560260773 32.0
22.593300831 1.46758079528809 33.0
22.593300831 1.06273400783539 34.0
22.593300831 0.117551781237125 35.0
9.3069963998 0.030177790671587 36.0
8.0405944568 0.346565634012222 37.0
0 0.00831856951117516 38.0
0 0.196202456951141 39.0
-11.93872842 0.320705950260162 40.0
-24.341199685 0.763886868953705 41.0
5.458983829 0.3841712474823 42.0
-32.168364658 0.271949827671051 43.0
107.98392786 0.765826165676117 44.0
107.98392786 0.0578613355755806 45.0
7.3669624978 0.61606103181839 46.0
2.7898409701 0.355890810489655 47.0
175.49338028 0.128694161772728 48.0
169.37008746 0.324763149023056 49.0
0 0.156905114650726 50.0
86.274275568 0.291563600301743 51.0
93.151840683 0.101783163845539 52.0
86.174446131 0.108989305794239 53.0
86.174446131 0.151001378893852 54.0
86.174446131 0.220603182911873 55.0
0 3.44061899185181 56.0
75.796048508 -0.0123254880309105 57.0
223.28167719 0.819224715232849 58.0
247.46078656 0.550559103488922 59.0
22.593300831 0.00687629729509354 60.0
0 0.0793994069099426 61.0
185.99124384 0.667260825634003 0.0
185.99124384 0.977214634418488 1.0
184.78481823 0.958480834960938 2.0
179.99571292 5.02513837814331 3.0
0 9.43603324890137 4.0
0 3.07518911361694 5.0
-48.516444782 0.212722659111023 6.0
59.368878279 1.10712444782257 7.0
59.679234058 1.04735684394836 8.0
-0.31035577905 0.750004470348358 9.0
80.795343298 0.644975125789642 10.0
68.174551419 0.048416294157505 11.0
79.629689834 0.835074484348297 12.0
79.629689834 0.749376714229584 13.0
79.629689834 0.0693120062351227 14.0
0 0.0333823226392269 15.0
0 0.205321550369263 16.0
0 1.13505864143372 17.0
53.436770828 1.89634585380554 18.0
-53.436770828 1.95628547668457 19.0
-53.436770828 0.482694268226624 20.0
-53.436770828 0.0235792323946953 21.0
73.90058853 2.01847696304321 22.0
149.47671212 1.07083356380463 23.0
115.6983543 0.783762872219086 24.0
24.861984527 0.0640078037977219 25.0
223.37730065 1.47827589511871 26.0
223.37730065 1.48531579971313 27.0
22.920316685 1.42745649814606 28.0
246.29761733 0.461533188819885 29.0
246.29761733 1.50591862201691 30.0
46.624680146 0.939948856830597 31.0
9.4908995215 0.47372442483902 32.0
20.463817702 1.53108394145966 33.0
20.463817702 0.898874700069427 34.0
20.463817702 0.141656160354614 35.0
9.211259762 0.0511355549097061 36.0
7.9350085025 0.392215609550476 37.0
0 0.0194985195994377 38.0
0 0.228194296360016 39.0
-10.691825317 0.73799991607666 40.0
-24.753545037 0.503899812698364 41.0
4.9183613459 0.609240829944611 42.0
-31.999322056 0.728417217731476 43.0
78.206877916 0.876846730709076 44.0
78.206877916 0.127399206161499 45.0
5.9807063751 0.629883110523224 46.0
2.5180878327 0.339991718530655 47.0
185.99124384 0.116159863770008 48.0
179.99571292 0.0107549540698528 49.0
0 0.258109331130981 50.0
59.368878279 0.255396753549576 51.0
80.795343298 0.279978334903717 52.0
79.629689834 0.0169962514191866 53.0
79.629689834 0.194831341505051 54.0
79.629689834 0.0695157945156097 55.0
0 4.214195728302 56.0
73.90058853 0.181002140045166 57.0
223.37730065 0.246648758649826 58.0
246.29761733 0.733415126800537 59.0
20.463817702 -0.00389900803565979 60.0
0 0.0783452242612839 61.0
173.46104658 0.987646818161011 0.0
173.46104658 1.44612264633179 1.0
172.40983156 1.47874820232391 2.0
167.62072633 2.94020652770996 3.0
0 4.38762807846069 4.0
0 4.50739812850952 5.0
-79.848985049 1.40816521644592 6.0
91.108311605 1.25211191177368 7.0
61.944098703 0.61496913433075 8.0
29.164212902 0.631117105484009 9.0
88.217316653 1.14025890827179 10.0
74.30303592 0.390773504972458 11.0
75.353776291 0.310398578643799 12.0
75.353776291 0.723273932933807 13.0
75.353776291 0.0727229937911034 14.0
0 0.0584698878228664 15.0
0 0.13292221724987 16.0
0 1.59130954742432 17.0
55.308102422 1.17224872112274 18.0
-55.308102422 0.747384786605835 19.0
-55.308102422 0.889432907104492 20.0
-55.308102422 0.00171872973442078 21.0
75.614933087 1.26208543777466 22.0
139.62734189 0.691491425037384 23.0
106.37451512 0.743619561195374 24.0
21.239255573 0.161044210195541 25.0
215.24227497 1.26713061332703 26.0
215.24227497 1.23841834068298 27.0
27.985915296 1.55616426467896 28.0
243.22819027 1.00375413894653 29.0
243.22819027 1.93981313705444 30.0
45.967231337 0.974608957767487 31.0
9.2282491638 0.479217439889908 32.0
20.306830665 1.15216958522797 33.0
20.306830665 1.244833111763 34.0
20.306830665 0.123910427093506 35.0
8.8507634146 0.0304885357618332 36.0
7.5325158763 0.378205388784409 37.0
0 -0.0137184895575047 38.0
0 0.0602740608155727 39.0
-11.396701753 0.623996257781982 40.0
-23.931834589 0.420555263757706 41.0
5.6876141621 0.349181056022644 42.0
-31.838921324 0.526149332523346 43.0
90.818701988 0.524414002895355 44.0
90.818701988 0.11482772231102 45.0
6.7617709924 0.356438130140305 46.0
2.4465873795 0.293821901082993 47.0
173.46104658 0.123312279582024 48.0
167.62072633 0.0874537527561188 49.0
0 0.240289866924286 50.0
91.108311605 0.0658267438411713 51.0
88.217316653 0.185759723186493 52.0
75.353776291 0.0255811996757984 53.0
75.353776291 0.0295326225459576 54.0
75.353776291 0.286203563213348 55.0
0 3.10241484642029 56.0
75.614933087 0.105247780680656 57.0
215.24227497 0.682408452033997 58.0
243.22819027 0.78926283121109 59.0
20.306830665 -0.00101699121296406 60.0
0 0.0924783572554588 61.0
142.6153501 1.10347902774811 0.0
142.6153501 1.08707308769226 1.0
141.14534079 0.908381700515747 2.0
136.35623551 3.54982233047485 3.0
0 6.18104124069214 4.0
0 3.68560695648193 5.0
-51.61446643 -0.00613243877887726 6.0
64.193234412 0.462498933076859 7.0
32.710713361 1.02170443534851 8.0
31.48252105 0.874534964561462 9.0
86.157095601 1.13055765628815 10.0
73.200968921 0.081917330622673 11.0
81.30217923 0.507086336612701 12.0
81.30217923 0.145529091358185 13.0
81.30217923 0.117045655846596 14.0
0 0.0783475339412689 15.0
0 0.152821779251099 16.0
0 0.749635457992554 17.0
57.106254328 1.42504131793976 18.0
-57.106254328 1.70488405227661 19.0
-57.106254328 0.770524382591248 20.0
-57.106254328 0.0395707674324512 21.0
76.849908249 0.852553009986877 22.0
136.13233109 0.642770409584045 23.0
107.54064198 0.57165664434433 24.0
21.536631717 0.149891927838326 25.0
212.98223934 0.670220732688904 26.0
212.98223934 1.20463287830353 27.0
23.948579054 1.54055595397949 28.0
236.9308184 0.375833034515381 29.0
236.9308184 1.32927846908569 30.0
46.582412551 1.00751602649689 31.0
9.2798044154 0.395698189735413 32.0
19.743653921 1.65837407112122 33.0
19.743653921 1.46131539344788 34.0
19.743653921 0.100544035434723 35.0
8.6563373866 0.023782055824995 36.0
7.2394028685 0.34140282869339 37.0
0 0.0345428586006165 38.0
0 0.13264225423336 39.0
-11.808640404 0.534455716609955 40.0
-23.512258708 0.809911966323853 41.0
5.655477083 0.39656201004982 42.0
-31.886974266 0.326568961143494 43.0
89.944157975 0.853226900100708 44.0
89.944157975 0.127075910568237 45.0
6.115074872 0.757819771766663 46.0
2.3502173855 0.406048983335495 47.0
142.6153501 0.189627185463905 48.0
136.35623551 -0.00398124661296606 49.0
0 0.323758810758591 50.0
64.193234412 0.282288461923599 51.0
86.157095601 -0.0389646515250206 52.0
81.30217923 0.0690058544278145 53.0
81.30217923 0.0320607014000416 54.0
81.30217923 0.074930876493454 55.0
0 6.12640523910522 56.0
76.849908249 0.0333301089704037 57.0
212.98223934 -0.0276555456221104 58.0
236.9308184 0.068798303604126 59.0
19.743653921 -0.0368827320635319 60.0
0 0.0909933149814606 61.0
191.59743091 1.04686915874481 0.0
191.59743091 0.905251145362854 1.0
190.48608494 0.706035077571869 2.0
185.69697966 3.53977632522583 3.0
0 4.78407144546509 4.0
0 4.15283250808716 5.0
-54.61192855 2.31357908248901 6.0
63.418745893 1.20181608200073 7.0
59.515880901 1.35501492023468 8.0
3.9028649918 1.3001434803009 9.0
88.935105659 1.15652704238892 10.0
76.841343584 0.690854430198669 11.0
79.245668956 0.644391179084778 12.0
79.245668956 0.581809282302856 13.0
79.245668956 0.0967617258429527 14.0
0 0.0528937578201294 15.0
0 0.210715621709824 16.0
0 1.32114517688751 17.0
53.066004365 0.739382565021515 18.0
-53.066004365 0.674341559410095 19.0
-53.066004365 0.344176113605499 20.0
-53.066004365 0.0435381270945072 21.0
74.231285236 1.99729704856873 22.0
144.93922029 1.07240569591522 23.0
113.24559668 0.74548202753067 24.0
23.677387429 0.109120525419712 25.0
219.17050552 1.35210120677948 26.0
219.17050552 1.26701307296753 27.0
23.066439625 0.9928218126297 28.0
242.23694515 0.393751174211502 29.0
242.23694515 1.27616667747498 30.0
46.786010277 1.03389692306519 31.0
9.2687545623 0.461169987916946 32.0
21.165280872 1.57498347759247 33.0
21.165280872 1.05072891712189 34.0
21.165280872 0.0992912352085114 35.0
9.3596008773 0.0246742870658636 36.0
8.2207717107 0.340082854032516 37.0
0 0.01888208091259 38.0
0 0.141154155135155 39.0
-12.081781272 0.519170939922333 40.0
-24.156622151 0.594066083431244 41.0
5.5551557048 0.555753827095032 42.0
-32.36830543 0.685272037982941 43.0
88.046195101 0.632988274097443 44.0
88.046195101 0.13073593378067 45.0
5.9377011165 0.625735402107239 46.0
2.1532219566 0.411496490240097 47.0
191.59743091 0.130900859832764 48.0
185.69697966 -0.0212063696235418 49.0
0 0.159944176673889 50.0
63.418745893 0.481322884559631 51.0
88.935105659 0.300757706165314 52.0
79.245668956 0.199193701148033 53.0
79.245668956 0.125499680638313 54.0
79.245668956 0.0626905411481857 55.0
0 6.03382253646851 56.0
74.231285236 0.137003794312477 57.0
219.17050552 0.490591883659363 58.0
242.23694515 -0.0318921133875847 59.0
21.165280872 0.00608652830123901 60.0
0 0.099497377872467 61.0
151.53052471 0.885693669319153 0.0
151.53052471 1.14558362960815 1.0
150.14722819 1.58417248725891 2.0
145.35812291 5.41663646697998 3.0
0 6.27334642410278 4.0
0 4.06710386276245 5.0
-53.990936638 1.31675481796265 6.0
66.111984117 1.24458718299866 7.0
58.662566091 0.622080981731415 8.0
7.4494180257 0.539926469326019 9.0
93.103458186 0.95982950925827 10.0
78.82049806 0.170554459095001 11.0
88.009692169 0.227263405919075 12.0
88.009692169 0.133932918310165 13.0
88.009692169 0.0558657199144363 14.0
0 0.155589938163757 15.0
0 0.197009116411209 16.0
0 1.21108520030975 17.0
59.341610229 1.11346280574799 18.0
-59.341610229 0.545884490013123 19.0
-59.341610229 0.885667979717255 20.0
-59.341610229 -0.0643464773893356 21.0
81.553164138 1.80065190792084 22.0
136.03193825 1.298224568367 23.0
104.61527782 0.735620379447937 24.0
22.205649722 0.079417921602726 25.0
217.58510239 1.15813910961151 26.0
217.58510239 1.57974147796631 27.0
23.563663583 1.61925411224365 28.0
241.14876597 0.750994682312012 29.0
241.14876597 1.76426529884338 30.0
47.309820325 1.01638114452362 31.0
9.4361905408 0.481956869363785 32.0
22.211553908 1.83788561820984 33.0
22.211553908 0.661019504070282 34.0
22.211553908 0.0875989571213722 35.0
10.219718866 0.0225408338010311 36.0
8.8844852164 0.399383753538132 37.0
0 0.050758995115757 38.0
0 0.200397223234177 39.0
-11.46898772 0.50811767578125 40.0
-25.082210101 0.938471257686615 41.0
4.7573970333 0.50469833612442 42.0
-32.149025218 0.44216376543045 43.0
82.654447329 0.719072341918945 44.0
82.654447329 0.0896633267402649 45.0
6.3426538129 0.693106591701508 46.0
2.3457693384 0.433667063713074 47.0
151.53052471 0.138828694820404 48.0
145.35812291 0.518092691898346 49.0
0 0.453904271125793 50.0
66.111984117 0.185608848929405 51.0
93.103458186 0.118024520576 52.0
88.009692169 0.0278152599930763 53.0
88.009692169 0.111731640994549 54.0
88.009692169 0.0837904289364815 55.0
0 0.291524976491928 56.0
81.553164138 0.116809844970703 57.0
217.58510239 0.674700975418091 58.0
241.14876597 0.407731145620346 59.0
22.211553908 -0.00584646686911583 60.0
0 0.0860885828733444 61.0
149.40154789 1.15642154216766 0.0
149.40154789 1.3758100271225 1.0
148.26354772 2.02584981918335 2.0
143.47444246 3.45778727531433 3.0
0 4.85509300231934 4.0
0 3.28418111801147 5.0
-72.137797024 -0.036456897854805 6.0
84.064600859 1.1509735584259 7.0
42.133656387 1.67295074462891 8.0
41.930944472 1.64202082157135 9.0
93.109330801 1.3580014705658 10.0
78.143823658 0.476182043552399 11.0
74.559892923 0.151187062263489 12.0
74.559892923 0.0640727430582047 13.0
74.559892923 0.033547218888998 14.0
0 -0.0157901830971241 15.0
0 0.229137137532234 16.0
0 1.07455003261566 17.0
57.802095058 1.19586992263794 18.0
-57.802095058 0.988015711307526 19.0
-57.802095058 1.10442996025085 20.0
-57.802095058 -0.0628872364759445 21.0
79.903707132 1.46089684963226 22.0
135.20015781 0.782958269119263 23.0
98.462811025 0.776517152786255 24.0
23.562867226 0.0864752531051636 25.0
215.10386494 1.87714695930481 26.0
215.10386494 2.61660385131836 27.0
30.796592497 1.76371765136719 28.0
245.90045744 1.13309526443481 29.0
245.90045744 1.98025619983673 30.0
46.320752068 1.13994634151459 31.0
9.4856805107 0.590888738632202 32.0
22.101612074 1.65329802036285 33.0
22.101612074 1.22274744510651 34.0
22.101612074 0.0919422060251236 35.0
8.295384445 0.0290513131767511 36.0
7.1338191809 0.436043441295624 37.0
0 -0.00560120493173599 38.0
0 0.124357014894485 39.0
-10.393254738 0.600402176380157 40.0
-22.126014812 0.510194838047028 41.0
5.3277137299 0.464243501424789 42.0
-30.185906049 0.541731655597687 43.0
86.729139982 0.599952280521393 44.0
86.729139982 0.119064621627331 45.0
7.312727729 0.536156058311462 46.0
2.3162893119 0.404853165149689 47.0
149.40154789 0.163881421089172 48.0
143.47444246 0.331012845039368 49.0
0 0.51611602306366 50.0
84.064600859 0.528824746608734 51.0
93.109330801 0.255295127630234 52.0
74.559892923 0.122050099074841 53.0
74.559892923 0.120831489562988 54.0
74.559892923 0.0879160612821579 55.0
0 3.93699407577515 56.0
79.903707132 0.0857602059841156 57.0
215.10386494 1.35163640975952 58.0
245.90045744 0.36760801076889 59.0
22.101612074 -0.00899862218648195 60.0
0 0.0738655924797058 61.0
162.85166479 0.917908310890198 0.0
162.85166479 1.30844640731812 1.0
161.60924583 1.37930643558502 2.0
156.82014068 4.06210517883301 3.0
0 7.83902835845947 4.0
0 4.06002330780029 5.0
-61.719585822 0.273793637752533 6.0
72.129016842 1.20503914356232 7.0
53.22630633 0.41959735751152 8.0
18.902710512 0.37030428647995 9.0
76.468945719 1.03626036643982 10.0
62.60115083 0.138341560959816 11.0
77.431002803 0.633173704147339 12.0
77.431002803 0.466393202543259 13.0
77.431002803 0.045590665191412 14.0
0 0.0179910585284233 15.0
0 0.182180792093277 16.0
0 0.971507012844086 17.0
54.437547354 0.943780243396759 18.0
-54.437547354 0.997910380363464 19.0
-54.437547354 0.549302816390991 20.0
-54.437547354 -0.0654100775718689 21.0
76.571749123 2.25058889389038 22.0
141.01159699 1.96062052249908 23.0
109.22231222 0.623643159866333 24.0
24.742687661 0.112239964306355 25.0
217.58334612 1.06828010082245 26.0
217.58334612 1.57602214813232 27.0
22.98151392 1.94290661811829 28.0
240.56486004 1.13539934158325 29.0
240.56486004 1.85153162479401 30.0
47.1847955 0.934639811515808 31.0
9.3162406816 0.560024380683899 32.0
22.13420177 1.79938673973083 33.0
22.13420177 0.979853630065918 34.0
22.13420177 0.101616337895393 35.0
9.9805888837 0.0312637016177177 36.0
8.7550928309 0.380798280239105 37.0
0 0.026020709425211 38.0
0 0.252315908670425 39.0
-12.273415974 0.617136180400848 40.0
-25.514968449 0.633954763412476 41.0
4.886823914 0.4854816198349 42.0
-33.169600568 0.650374472141266 43.0
88.408968499 0.606418907642365 44.0
88.408968499 0.101492486894131 45.0
7.6925428825 0.477007478475571 46.0
2.7547805011 0.335815966129303 47.0
162.85166479 0.125782012939453 48.0
156.82014068 0.287292182445526 49.0
0 0.421093940734863 50.0
72.129016842 0.272097080945969 51.0
76.468945719 0.413849532604218 52.0
77.431002803 0.0120147429406643 53.0
77.431002803 0.136022299528122 54.0
77.431002803 0.0968859568238258 55.0
0 2.30071473121643 56.0
76.571749123 0.0553706809878349 57.0
217.58334612 0.297951698303223 58.0
240.56486004 0.819614171981812 59.0
22.13420177 0.0444775260984898 60.0
0 0.0843134373426437 61.0
155.54601184 0.896695196628571 0.0
155.54601184 1.07480442523956 1.0
154.40887142 1.34122359752655 2.0
148.94422242 2.82802510261536 3.0
5.1990450941 4.01788949966431 4.0
5.1990450941 3.99844479560852 5.0
-68.845491608 1.78282916545868 6.0
81.912099248 0.581000566482544 7.0
48.188807259 0.661263763904572 8.0
33.723291989 0.590037286281586 9.0
87.294208754 1.17901575565338 10.0
75.075150522 0.553330302238464 11.0
77.412502301 0.528836190700531 12.0
77.412502301 0.27190899848938 13.0
77.412502301 0.0637731552124023 14.0
0 0.0367007479071617 15.0
0 0.163583591580391 16.0
0 1.20100617408752 17.0
59.403349659 1.4073748588562 18.0
-59.403349659 1.02145624160767 19.0
-59.403349659 0.401438444852829 20.0
-59.403349659 0.0294384732842445 21.0
80.926763816 1.58714246749878 22.0
145.87717747 1.57617664337158 23.0
113.6851496 0.333839356899261 24.0
24.445439013 0.273483127355576 25.0
226.80394129 1.20158088207245 26.0
226.80394129 2.01294183731079 27.0
24.314083335 1.43216037750244 28.0
251.11802462 0.920596599578857 29.0
251.11802462 1.90349161624908 30.0
47.050197347 1.11252868175507 31.0
9.4613249791 0.527355194091797 32.0
21.523414157 1.3928849697113 33.0
21.523414157 1.03036749362946 34.0
21.523414157 0.136202335357666 35.0
9.5380427079 0.0436416119337082 36.0
8.3584538059 0.349090486764908 37.0
0 0.0229913182556629 38.0
0 0.114022858440876 39.0
-11.615603613 0.445603460073471 40.0
-24.528733738 0.630997776985168 41.0
5.0479736661 0.38672661781311 42.0
-32.136496049 0.355701863765717 43.0
88.989860394 0.697574079036713 44.0
88.989860394 0.0903785154223442 45.0
7.3605957383 0.707718789577484 46.0
2.4893488285 0.414903879165649 47.0
155.54601184 0.109346948564053 48.0
148.94422242 0.0498318746685982 49.0
5.1990450941 0.318382322788239 50.0
81.912099248 0.109437882900238 51.0
87.294208754 0.614493608474731 52.0
77.412502301 0.0459318831562996 53.0
77.412502301 0.119215667247772 54.0
77.412502301 0.084180474281311 55.0
0 4.34272909164429 56.0
80.926763816 0.0630815252661705 57.0
226.80394129 0.654511392116547 58.0
251.11802462 0.472525328397751 59.0
21.523414157 -0.0175910759717226 60.0
0 0.113568365573883 61.0
176.40951044 0.823261260986328 0.0
176.40951044 1.68313407897949 1.0
175.28889144 1.64366483688354 2.0
169.82425604 3.68944239616394 3.0
5.0553702274 6.40245628356934 4.0
5.0553702274 3.57910108566284 5.0
-49.095404047 0.313933223485947 6.0
62.941670371 0.336897253990173 7.0
36.193135056 0.550321698188782 8.0
26.748535315 0.61093133687973 9.0
72.115658669 0.80004358291626 10.0
58.427570907 0.153814047574997 11.0
67.098628128 0.212490141391754 12.0
67.098628128 0.493563205003738 13.0
67.098628128 0.101019352674484 14.0
18.383009797 0.201377466320992 15.0
18.383009797 0.178268924355507 16.0
0 1.57951390743256 17.0
53.60554984 2.09998726844788 18.0
-53.60554984 1.82829535007477 19.0
-53.60554984 0.402272701263428 20.0
-53.60554984 0.0397138334810734 21.0
73.811958817 1.00534462928772 22.0
160.60911316 2.37288451194763 23.0
119.67690678 0.216203659772873 24.0
20.845701198 0.234436184167862 25.0
234.42107197 0.668097138404846 26.0
234.42107197 1.49539160728455 27.0
21.577401813 1.32640326023102 28.0
255.99847379 0.736165046691895 29.0
255.99847379 1.67984569072723 30.0
46.593712191 0.874233484268188 31.0
9.2409333828 0.359103441238403 32.0
20.206408976 1.31965303421021 33.0
20.206408976 1.37195062637329 34.0
20.206408976 0.107319489121437 35.0
8.5644512156 0.0188988875597715 36.0
7.1708571145 0.432125478982925 37.0
0 0.0628044232726097 38.0
0 0.261903196573257 39.0
-10.494930387 0.97279554605484 40.0
-24.830663261 0.455988466739655 41.0
5.4015421744 0.584787428379059 42.0
-32.527272263 0.609579026699066 43.0
106.28912851 0.725422859191895 44.0
106.28912851 0.000538498163223267 45.0
7.7626674007 0.66362589597702 46.0
2.3034453566 0.429341614246368 47.0
176.40951044 0.0961288958787918 48.0
169.82425604 0.376427948474884 49.0
5.0553702274 0.255173623561859 50.0
62.941670371 0.104742847383022 51.0
72.115658669 0.293367683887482 52.0
67.098628128 0.0326648317277431 53.0
67.098628128 0.0230400376021862 54.0
67.098628128 0.0920015648007393 55.0
0 2.2946572303772 56.0
73.811958817 0.0392159968614578 57.0
234.42107197 0.281122773885727 58.0
255.99847379 0.772485375404358 59.0
20.206408976 -0.0298086442053318 60.0
0 0.107365220785141 61.0
160.55427464 0.913893401622772 0.0
160.55427464 1.11500191688538 1.0
159.42357638 1.12756967544556 2.0
153.95892739 4.15751218795776 3.0
3.5984364195 3.65807151794434 4.0
3.5984364195 5.32539939880371 5.0
-67.895670989 3.88308238983154 6.0
80.825727003 0.538268327713013 7.0
45.766575249 0.588619232177734 8.0
35.059151754 0.599672198295593 9.0
81.211411072 1.03770184516907 10.0
67.591938727 0.121584661304951 11.0
79.109273438 0.391784280538559 12.0
79.109273438 0.338714897632599 13.0
79.109273438 0.0343381837010384 14.0
0 0.000609877519309521 15.0
0 0.233047261834145 16.0
0 1.17989826202393 17.0
54.988886697 0.770411193370819 18.0
-54.988886697 0.37825819849968 19.0
-54.988886697 1.32075715065002 20.0
-54.988886697 -0.0742188319563866 21.0
75.068689998 2.3233847618103 22.0
146.50039811 2.67036533355713 23.0
114.12471657 0.259619891643524 24.0
24.292763811 0.262045592069626 25.0
221.56908811 1.25063252449036 26.0
221.56908811 1.79011249542236 27.0
25.772248874 2.05581045150757 28.0
247.34133699 1.34971356391907 29.0
247.34133699 1.97464573383331 30.0
46.767635626 0.935161232948303 31.0
9.3613113141 0.418438881635666 32.0
20.079803301 1.25510215759277 33.0
20.079803301 0.888709366321564 34.0
20.079803301 0.0706121772527695 35.0
8.8914302557 0.0225633624941111 36.0
7.6549074394 0.40450194478035 37.0
0 0.0181680880486965 38.0
0 0.269940465688705 39.0
-11.63438109 0.480247229337692 40.0
-23.110406884 0.608395874500275 41.0
4.9098856157 0.519712090492249 42.0
-30.408093881 0.562471389770508 43.0
93.046852182 0.653860032558441 44.0
93.046852182 0.112661227583885 45.0
6.8145786655 0.559860348701477 46.0
2.6989869818 0.403244435787201 47.0
160.55427464 0.118988797068596 48.0
153.95892739 0.155777931213379 49.0
3.5984364195 0.248495012521744 50.0
80.825727003 0.135756924748421 51.0
81.211411072 0.139253512024879 52.0
79.109273438 0.0474553629755974 53.0
79.109273438 0.0652644783258438 54.0
79.109273438 0.08034847676754 55.0
0 3.79151105880737 56.0
75.068689998 0.0864637792110443 57.0
221.56908811 0.332651823759079 58.0
247.34133699 0.99417382478714 59.0
20.079803301 0.0117909871041775 60.0
0 0.095392182469368 61.0
140.2092591 1.19203794002533 0.0
140.2092591 2.36007809638977 1.0
139.11826211 2.94474411010742 2.0
134.32919215 6.16521072387695 3.0
0 7.15724897384644 4.0
0 4.06592559814453 5.0
-65.803457424 0.755045294761658 6.0
76.616347993 1.0327912569046 7.0
59.863080982 0.88303804397583 8.0
16.75326701 1.0585755109787 9.0
82.96786272 1.16882991790771 10.0
69.191075997 0.0546266548335552 11.0
79.005126858 0.402736485004425 12.0
79.005126858 0.678666412830353 13.0
79.005126858 0.0698936060070992 14.0
0 0.12105019390583 15.0
0 0.238116756081581 16.0
0 0.69797420501709 17.0
55.0727395 1.10590696334839 18.0
-55.0727395 1.09708988666534 19.0
-55.0727395 0.33776792883873 20.0
-55.0727395 -0.00426129251718521 21.0
75.700421671 1.52838814258575 22.0
151.41764864 0.975136399269104 23.0
118.26556775 0.627014517784119 24.0
20.10424817 0.166996583342552 25.0
227.11807031 1.40513265132904 26.0
227.11807031 1.20915222167969 27.0
27.374393454 1.42183649539948 28.0
254.49246376 0.962447762489319 29.0
254.49246376 1.51716637611389 30.0
46.546813558 1.01847171783447 31.0
9.4003090941 0.313517510890961 32.0
20.62768217 1.38378238677979 33.0
20.62768217 1.09351301193237 34.0
20.62768217 0.135121271014214 35.0
8.8886558823 0.0344334878027439 36.0
7.5265507551 0.377511590719223 37.0
0 0.050764087587595 38.0
0 0.182018637657166 39.0
-10.634324093 0.551000356674194 40.0
-25.20851379 0.537963628768921 41.0
5.6355650637 0.603258311748505 42.0
-32.989504374 0.740577459335327 43.0
105.78639096 0.91025310754776 44.0
105.78639096 0.109452068805695 45.0
7.6288346698 0.637183785438538 46.0
2.7080583402 0.344728738069534 47.0
140.2092591 0.0916664898395538 48.0
134.32919215 0.918482780456543 49.0
0 0.881667196750641 50.0
76.616347993 0.245871663093567 51.0
82.96786272 0.190924182534218 52.0
79.005126858 0.0151399960741401 53.0
79.005126858 0.148049086332321 54.0
79.005126858 0.0395067222416401 55.0
0 3.64819693565369 56.0
75.700421671 0.0027965996414423 57.0
227.11807031 0.624311923980713 58.0
254.49246376 1.09059107303619 59.0
20.62768217 -0.0158358681946993 60.0
0 0.0750500932335854 61.0
157.26950346 0.811347186565399 0.0
157.26950346 1.01229083538055 1.0
156.10603714 1.66106402873993 2.0
151.31693189 4.5485258102417 3.0
0 5.29473257064819 4.0
0 5.10884666442871 5.0
-65.339796797 3.83871412277222 6.0
77.031202939 0.842387199401855 7.0
48.187901866 1.36257088184357 8.0
28.843301073 1.31778347492218 9.0
88.648495325 0.844620883464813 10.0
75.489418414 0.151928544044495 11.0
78.693554997 0.568590819835663 12.0
78.693554997 0.105093590915203 13.0
78.693554997 0.130670875310898 14.0
0 0.103620044887066 15.0
0 0.14650870859623 16.0
0 1.38560879230499 17.0
54.671336481 1.05852234363556 18.0
-54.671336481 1.34321415424347 19.0
-54.671336481 0.625900506973267 20.0
-54.671336481 0.0135402120649815 21.0
74.789483629 2.16868305206299 22.0
146.09355915 1.26812827587128 23.0
112.10406662 0.810951471328735 24.0
20.147895924 0.0759987831115723 25.0
220.88304278 1.64926874637604 26.0
220.88304278 2.20901417732239 27.0
24.101851953 2.01421189308167 28.0
244.98489473 0.884043097496033 29.0
244.98489473 1.83126068115234 30.0
47.192799645 1.05562055110931 31.0
9.2532758362 0.545456409454346 32.0
20.118147148 1.22104406356812 33.0
20.118147148 1.29977703094482 34.0
20.118147148 0.117664664983749 35.0
8.9236381775 0.0475912913680077 36.0
7.5857081007 0.391327917575836 37.0
0 0.0225191973149776 38.0
0 0.213875785470009 39.0
-11.628510257 0.664310336112976 40.0
-23.955383462 0.571786284446716 41.0
5.4276955667 0.340873926877975 42.0
-31.747888306 0.413295209407806 43.0
93.101137104 0.892232894897461 44.0
93.101137104 0.0727145746350288 45.0
6.6445601574 0.730365395545959 46.0
2.2642964621 0.411806762218475 47.0
157.26950346 0.170180484652519 48.0
151.31693189 0.205415278673172 49.0
0 0.337573766708374 50.0
77.031202939 0.382989078760147 51.0
88.648495325 0.142912119626999 52.0
78.693554997 0.0605073496699333 53.0
78.693554997 0.0302705448120832 54.0
78.693554997 0.0732762888073921 55.0
0 5.14173650741577 56.0
74.789483629 0.00195283815264702 57.0
220.88304278 0.835300922393799 58.0
244.98489473 0.62675279378891 59.0
20.118147148 -0.0159280635416508 60.0
0 0.0970615297555923 61.0
138.05956582 1.00968432426453 0.0
138.05956582 0.443578898906708 1.0
136.84742061 1.01491367816925 2.0
131.38274993 3.9784848690033 3.0
0.093271390944 8.12871265411377 4.0
0.093271390944 4.87839603424072 5.0
-76.375865462 3.49419498443604 6.0
88.926975824 0.372870594263077 7.0
61.045152385 1.19847583770752 8.0
27.88182344 1.42166948318481 9.0
77.613230782 0.97285395860672 10.0
64.368373773 0.0522505342960358 11.0
74.519628457 0.472447514533997 12.0
74.519628457 0.185301303863525 13.0
74.519628457 0.0895596295595169 14.0
0 0.0238280296325684 15.0
0 0.246605232357979 16.0
0 1.18281126022339 17.0
51.911935511 1.4449657201767 18.0
-51.911935511 0.828679203987122 19.0
-51.911935511 0.635722398757935 20.0
-51.911935511 0.0458614192903042 21.0
74.297949661 1.80988895893097 22.0
138.0834514 1.62581098079681 23.0
101.64827892 0.213134482502937 24.0
21.353156127 0.229670599102974 25.0
212.38140107 1.21657383441925 26.0
212.38140107 1.4562486410141 27.0
29.696372774 1.77354753017426 28.0
242.07777384 0.754425168037415 29.0
242.07777384 1.59878396987915 30.0
46.771682563 1.21122491359711 31.0
9.2451829328 0.563221991062164 32.0
22.38601415 1.13776755332947 33.0
22.38601415 1.16510677337646 34.0
22.38601415 0.0749814957380295 35.0
10.095616208 0.0216762870550156 36.0
8.9942130319 0.44732066988945 37.0
0 0.0256530903279781 38.0
0 0.204120546579361 39.0
-11.443151783 0.814136445522308 40.0
-25.442175948 0.262722820043564 41.0
5.1225558611 0.566293597221375 42.0
-33.282981079 0.82473361492157 43.0
98.685519528 0.758764028549194 44.0
98.685519528 0.144527614116669 45.0
7.3966205993 0.508687496185303 46.0
2.1491584645 0.352071732282639 47.0
138.05956582 0.0461971573531628 48.0
131.38274993 0.189378261566162 49.0
0.093271390944 0.369444578886032 50.0
88.926975824 0.0542653053998947 51.0
77.613230782 0.238512322306633 52.0
74.519628457 0.0253767911344767 53.0
74.519628457 0.066459059715271 54.0
74.519628457 0.0646418631076813 55.0
0 6.04848098754883 56.0
74.297949661 0.0721580684185028 57.0
212.38140107 0.348478019237518 58.0
242.07777384 0.349458545446396 59.0
22.38601415 -0.000931581482291222 60.0
0 0.0813951790332794 61.0
188.29620924 1.00855422019958 0.0
188.29620924 2.05067253112793 1.0
187.08295788 1.57630121707916 2.0
181.61828713 3.46413016319275 3.0
3.9529967627 7.69604539871216 4.0
3.9529967627 5.2295503616333 5.0
-49.193896386 5.73126029968262 6.0
57.760063546 0.341787308454514 7.0
35.154610918 0.705342233181 8.0
22.605452628 0.790182709693909 9.0
94.332019374 0.989954650402069 10.0
80.461027984 0.0489104837179184 11.0
87.533529506 0.173387229442596 12.0
87.533529506 0.320840090513229 13.0
87.533529506 0.106392174959183 14.0
0 0.204888164997101 15.0
0 0.173418775200844 16.0
0 1.5102322101593 17.0
56.253945253 0.94476318359375 18.0
-56.253945253 1.08705592155457 19.0
-56.253945253 0.380650281906128 20.0
-56.253945253 0.0719437152147293 21.0
78.351014623 2.25456285476685 22.0
144.31231126 0.909802436828613 23.0
111.91062065 0.940122008323669 24.0
20.296017297 0.061879750341177 25.0
222.66332588 1.37592554092407 26.0
222.66332588 1.67565417289734 27.0
27.005261793 1.25564420223236 28.0
249.66858768 0.740617454051971 29.0
249.66858768 1.72890377044678 30.0
46.506666783 1.16926145553589 31.0
9.3858072101 0.587213039398193 32.0
22.09706937 1.13468241691589 33.0
22.09706937 1.02177965641022 34.0
22.09706937 0.121615037322044 35.0
9.4367318354 0.0218451824039221 36.0
8.1904413682 0.412051647901535 37.0
0 0.0153196975588799 38.0
0 0.0645163357257843 39.0
-10.780209388 0.892303586006165 40.0
-23.531732002 0.61448073387146 41.0
5.7184594864 0.6386758685112 42.0
-31.681497466 0.596695482730865 43.0
76.203043298 0.656388461589813 44.0
76.203043298 0.0905233770608902 45.0
6.6783627073 0.655371725559235 46.0
2.6976305599 0.418248951435089 47.0
188.29620924 0.107346698641777 48.0
181.61828713 0.0592140853404999 49.0
3.9529967627 0.836025297641754 50.0
57.760063546 0.275895178318024 51.0
94.332019374 0.239249721169472 52.0
87.533529506 0.0469642505049706 53.0
87.533529506 0.041071955114603 54.0
87.533529506 0.0115699861198664 55.0
0 2.55785036087036 56.0
78.351014623 0.0173895210027695 57.0
222.66332588 0.276564180850983 58.0
249.66858768 0.4283347427845 59.0
22.09706937 -0.00458175875246525 60.0
0 0.0910898074507713 61.0
162.19928555 0.913906812667847 0.0
162.19928555 0.958791077136993 1.0
160.85946888 1.23597323894501 2.0
156.07036362 5.75299978256226 3.0
0 5.143639087677 4.0
0 5.23172950744629 5.0
-69.561825024 4.61464834213257 6.0
79.855960386 1.03265380859375 7.0
53.223525305 0.518334329128265 8.0
26.632435081 0.39191859960556 9.0
85.914090692 0.926062285900116 10.0
73.746964322 0.106655091047287 11.0
87.249064689 0.38258495926857 12.0
87.249064689 0.500069737434387 13.0
87.249064689 0.0472701415419579 14.0
0 0.161699414253235 15.0
0 0.167092949151993 16.0
0 0.537193715572357 17.0
54.606660943 1.29071927070618 18.0
-54.606660943 1.57063329219818 19.0
-54.606660943 0.457278549671173 20.0
-54.606660943 -0.0173888206481934 21.0
75.728983984 1.95546770095825 22.0
154.09155003 1.25513422489166 23.0
115.85435522 0.688328385353088 24.0
23.520736851 0.153956741094589 25.0
229.82053401 0.9153151512146 26.0
229.82053401 1.64539313316345 27.0
23.463781271 1.1420419216156 28.0
253.28431529 0.432400643825531 29.0
253.28431529 1.45304441452026 30.0
46.795656853 1.18997466564178 31.0
9.4593510034 0.36322033405304 32.0
21.122323041 1.56660997867584 33.0
21.122323041 1.11084687709808 34.0
21.122323041 0.139217585325241 35.0
9.9781041801 0.0280140098184347 36.0
8.706695241 0.343932926654816 37.0
0 0.0228227823972702 38.0
0 0.213283255696297 39.0
-12.046932573 0.668388664722443 40.0
-24.181162191 0.676170527935028 41.0
5.3299109806 0.506526052951813 42.0
-32.041520973 0.605558931827545 43.0
93.211521161 0.607492566108704 44.0
93.211521161 0.10476665943861 45.0
6.6051238157 0.670080184936523 46.0
2.3735123779 0.415386825799942 47.0
162.19928555 0.0986713543534279 48.0
156.07036362 0.332853049039841 49.0
0 0.456466883420944 50.0
79.855960386 0.143609225749969 51.0
85.914090692 0.257759243249893 52.0
87.249064689 0.0454121679067612 53.0
87.249064689 0.0313907600939274 54.0
87.249064689 0.13860896229744 55.0
0 2.58524584770203 56.0
75.728983984 0.0800808444619179 57.0
229.82053401 0.628550231456757 58.0
253.28431529 0.199318632483482 59.0
21.122323041 -0.0202337484806776 60.0
0 0.0943526104092598 61.0
186.10475382 0.858292996883392 0.0
186.10475382 1.70729005336761 1.0
185.04668228 1.68571257591248 2.0
179.58204688 6.08416986465454 3.0
4.0962972007 5.00754117965698 4.0
4.0962972007 5.35542392730713 5.0
-67.390943902 3.09095644950867 6.0
80.329052147 0.610696971416473 7.0
46.579700621 1.34193873405457 8.0
33.749351526 1.36585795879364 9.0
81.230282927 0.962106049060822 10.0
68.794904756 0.111577987670898 11.0
80.794344593 0.4802266061306 12.0
80.794344593 0.105551645159721 13.0
80.794344593 0.021413354203105 14.0
0 0.110794104635715 15.0
0 0.224736928939819 16.0
0 1.13226103782654 17.0
52.483228897 1.06281113624573 18.0
-52.483228897 0.65470427274704 19.0
-52.483228897 0.637682020664215 20.0
-52.483228897 0.0452976450324059 21.0
72.83051276 1.09261107444763 22.0
150.58650994 0.95123016834259 23.0
118.69548481 0.452109128236771 24.0
24.786504889 0.146077752113342 25.0
223.4170227 1.10943174362183 26.0
223.4170227 1.66624677181244 27.0
27.50827134 1.9420131444931 28.0
250.92529404 0.739064991474152 29.0
250.92529404 1.74946260452271 30.0
47.055719644 1.23000514507294 31.0
9.3071332854 0.381036967039108 32.0
20.347283863 1.32935309410095 33.0
20.347283863 1.01209247112274 34.0
20.347283863 0.0714698284864426 35.0
8.5924156408 0.0267585124820471 36.0
7.2223789188 0.385822236537933 37.0
0 0.0519098937511444 38.0
0 0.238910436630249 39.0
-11.911566112 0.386758118867874 40.0
-23.876555691 0.824873924255371 41.0
5.4674227036 0.403578162193298 42.0
-31.925930511 0.55050528049469 43.0
89.536626921 0.774211883544922 44.0
89.536626921 0.0559018179774284 45.0
6.6748507481 0.520551860332489 46.0
2.4046882129 0.326644897460938 47.0
186.10475382 0.192435041069984 48.0
179.58204688 0.180576100945473 49.0
4.0962972007 0.427248328924179 50.0
80.329052147 0.334727257490158 51.0
81.230282927 0.2069923132658 52.0
80.794344593 0.01649134978652 53.0
80.794344593 0.0351749062538147 54.0
80.794344593 0.066395103931427 55.0
0 1.81471145153046 56.0
72.83051276 -0.0188840106129646 57.0
223.4170227 0.617994248867035 58.0
250.92529404 0.181950926780701 59.0
20.347283863 0.00510895252227783 60.0
0 0.0858523994684219 61.0
189.28314948 0.844103634357452 0.0
189.28314948 1.83713483810425 1.0
188.13581586 1.78364753723145 2.0
182.67114529 3.67650151252747 3.0
7.8331706244 6.66209697723389 4.0
7.8331706244 5.36553430557251 5.0
-63.189746237 0.963452458381653 6.0
73.140025957 0.482768028974533 7.0
45.611958817 0.894233703613281 8.0
27.52806714 1.02135276794434 9.0
92.830292818 1.09344077110291 10.0
79.66551119 0.457534581422806 11.0
81.368138841 0.371039301156998 12.0
81.368138841 0.258892953395844 13.0
81.368138841 0.0495134554803371 14.0
0 0.118238553404808 15.0
0 0.190624818205833 16.0
0 1.33150243759155 17.0
58.655690234 1.84219813346863 18.0
-58.655690234 1.70341825485229 19.0
-58.655690234 0.924841821193695 20.0
-58.655690234 -0.0613920129835606 21.0
80.047138613 2.5839638710022 22.0
135.79897661 1.74681973457336 23.0
102.49468045 0.555835545063019 24.0
20.888740159 0.0737932771444321 25.0
215.84611522 1.77752506732941 26.0
215.84611522 1.81320357322693 27.0
25.32512667 1.81083023548126 28.0
241.17124189 0.389695644378662 29.0
241.17124189 1.67649447917938 30.0
47.306742373 1.09603238105774 31.0
9.4378932156 0.430407911539078 32.0
21.391448379 1.65923476219177 33.0
21.391448379 0.782840490341187 34.0
21.391448379 0.117586113512516 35.0
10.054108385 0.0359005928039551 36.0
8.8591085027 0.416739106178284 37.0
0 0.0426312834024429 38.0
0 0.257035106420517 39.0
-10.674129348 0.52452540397644 40.0
-24.042976919 0.663913488388062 41.0
4.7551397298 0.46038231253624 42.0
-31.354864327 0.764147937297821 43.0
97.858726392 0.553358376026154 44.0
97.858726392 0.137419328093529 45.0
6.8378457271 0.545582473278046 46.0
2.1600766 0.408623874187469 47.0
189.28314948 0.125835359096527 48.0
182.67114529 0.0946671441197395 49.0
7.8331706244 0.247354850172997 50.0
73.140025957 0.167247116565704 51.0
92.830292818 0.447109580039978 52.0
81.368138841 0.0906776040792465 53.0
81.368138841 0.120304599404335 54.0
81.368138841 0.0604477748274803 55.0
0 3.50675988197327 56.0
80.047138613 0.0217323489487171 57.0
215.84611522 0.728731632232666 58.0
241.17124189 0.113340809941292 59.0
21.391448379 0.0106748919934034 60.0
0 0.09111587703228 61.0
159.47443134 0.867692112922668 0.0
159.47443134 1.30705904960632 1.0
158.18648244 1.5922886133194 2.0
153.39737718 3.91944122314453 3.0
0 5.59962940216064 4.0
0 3.07421612739563 5.0
-64.566383997 1.0307993888855 6.0
74.764042495 0.503303050994873 7.0
64.917157611 0.854707062244415 8.0
9.8468848831 1.13691294193268 9.0
81.312380927 0.947797536849976 10.0
68.96917552 0.106688991189003 11.0
75.721798462 0.131976693868637 12.0
75.721798462 0.185954988002777 13.0
75.721798462 0.0823268890380859 14.0
0 0.0374026596546173 15.0
0 0.23406794667244 16.0
0 1.1157523393631 17.0
57.260951428 1.25502300262451 18.0
-57.260951428 1.21856355667114 19.0
-57.260951428 0.505194067955017 20.0
-57.260951428 0.0224274881184101 21.0
77.735601648 1.85165143013 22.0
135.56013112 1.55939471721649 23.0
104.12885288 0.35170903801918 24.0
22.725520897 0.157007157802582 25.0
213.29573277 1.54861044883728 26.0
213.29573277 2.28573060035706 27.0
29.168826848 1.93893992900848 28.0
242.46455961 0.824556708335876 29.0
242.46455961 1.74501061439514 30.0
46.67894551 1.20972084999084 31.0
9.3294654533 0.337257653474808 32.0
20.47465022 1.66145861148834 33.0
20.47465022 0.746854662895203 34.0
20.47465022 0.0827865302562714 35.0
8.4261351323 0.0205846093595028 36.0
7.0415115616 0.441844612360001 37.0
0 0.0577780269086361 38.0
0 0.31809863448143 39.0
-11.418219024 0.448307305574417 40.0
-24.367491099 0.820701360702515 41.0
5.5816270242 0.375756770372391 42.0
-32.633331897 0.389474034309387 43.0
87.252535425 0.851473391056061 44.0
87.252535425 0.0821733251214027 45.0
7.042023597 0.573126316070557 46.0
2.8065242672 0.329306751489639 47.0
159.47443134 0.107563324272633 48.0
153.39737718 0.431538045406342 49.0
0 0.0982330441474915 50.0
74.764042495 0.0629031360149384 51.0
81.312380927 0.18996188044548 52.0
75.721798462 0.0132998079061508 53.0
75.721798462 0.0604345202445984 54.0
75.721798462 0.0622915476560593 55.0
0 4.63653612136841 56.0
77.735601648 0.0131489764899015 57.0
213.29573277 0.99046003818512 58.0
242.46455961 0.43138325214386 59.0
20.47465022 -0.00024881400167942 60.0
0 0.0870950222015381 61.0
186.40459118 0.989734947681427 0.0
186.40459118 1.76659262180328 1.0
185.12517493 1.89633560180664 2.0
180.33606975 4.67660188674927 3.0
0 9.00272655487061 4.0
0 6.28521060943604 5.0
-71.512183621 2.12202501296997 6.0
83.812843124 0.447496056556702 7.0
61.854836934 1.02278637886047 8.0
21.95800619 1.30519413948059 9.0
91.940644559 1.27327311038971 10.0
79.46224619 0.436336666345596 11.0
84.743139269 0.359761416912079 12.0
84.743139269 0.75879293680191 13.0
84.743139269 0.0322016477584839 14.0
0 0.113180965185165 15.0
0 0.20529717206955 16.0
0 1.19670927524567 17.0
58.72735176 1.48602890968323 18.0
-58.72735176 1.35790276527405 19.0
-58.72735176 0.3181112408638 20.0
-58.72735176 0.312865972518921 21.0
79.633810667 2.0381646156311 22.0
155.59405561 2.15386080741882 23.0
119.32159701 0.541021585464478 24.0
23.190442046 0.222862079739571 25.0
235.22786627 1.15107142925262 26.0
235.22786627 1.6569721698761 27.0
23.531983705 1.45021390914917 28.0
258.75984998 0.6668741106987 29.0
258.75984998 1.83141231536865 30.0
47.070644358 1.01680779457092 31.0
9.3898599723 0.590841889381409 32.0
20.906458907 0.868745148181915 33.0
20.906458907 1.12466061115265 34.0
20.906458907 0.11850244551897 35.0
8.9233867751 0.0313074849545956 36.0
7.6889645145 0.372450560331345 37.0
0 0.0138310194015503 38.0
0 0.150572553277016 39.0
-11.373630175 0.764349341392517 40.0
-23.736546235 0.649874031543732 41.0
5.1621851894 0.434284269809723 42.0
-31.640722073 0.519753575325012 43.0
89.248476284 0.89065545797348 44.0
89.248476284 0.0696616098284721 45.0
6.6060125136 0.547132313251495 46.0
2.4566185779 0.325717836618423 47.0
186.40459118 0.268917471170425 48.0
180.33606975 0.151881948113441 49.0
0 0.312608122825623 50.0
83.812843124 0.117582604289055 51.0
91.940644559 0.699228167533875 52.0
84.743139269 0.0493702851235867 53.0
84.743139269 0.171472370624542 54.0
84.743139269 0.244254782795906 55.0
0 3.08509802818298 56.0
79.633810667 -0.0204219818115234 57.0
235.22786627 0.392091184854507 58.0
258.75984998 0.402323007583618 59.0
20.906458907 0.0107487067580223 60.0
0 0.0991297140717506 61.0
171.36771588 0.949854135513306 0.0
171.36771588 0.912864744663239 1.0
169.95842443 1.36104273796082 2.0
165.16931911 6.75192403793335 3.0
0 3.46071100234985 4.0
0 6.21910190582275 5.0
-64.726510565 0.922379612922668 6.0
73.827572851 1.22111296653748 7.0
46.60653004 0.805382013320923 8.0
27.221042811 0.575822114944458 9.0
79.114460587 1.03846263885498 10.0
66.025539518 0.0687182769179344 11.0
73.827467163 0.251592099666595 12.0
73.827467163 0.440382182598114 13.0
73.827467163 0.0406500212848186 14.0
0 0.032733179628849 15.0
0 0.223775073885918 16.0
0 1.23665082454681 17.0
49.131802314 1.91637194156647 18.0
-49.131802314 2.30835509300232 19.0
-49.131802314 0.188903003931046 20.0
-49.131802314 0.0384881161153316 21.0
71.055338292 1.81698346138 22.0
141.10006694 1.06950175762177 23.0
109.02119559 0.686786890029907 24.0
20.628814395 0.0899547785520554 25.0
212.15540523 1.76482081413269 26.0
212.15540523 1.76168966293335 27.0
25.493041922 1.88838911056519 28.0
237.64844716 1.29551196098328 29.0
237.64844716 1.66371035575867 30.0
46.382160408 1.10498642921448 31.0
9.4837352825 0.396429806947708 32.0
21.923535978 1.54138898849487 33.0
21.923535978 1.16880285739899 34.0
21.923535978 0.155942857265472 35.0
11.133510592 0.0288887061178684 36.0
9.8913542972 0.38442650437355 37.0
0 0.00947175174951553 38.0
0 0.249270781874657 39.0
-10.816036536 0.629761755466461 40.0
-23.318697035 0.511732995510101 41.0
4.6536746005 0.512769281864166 42.0
-30.249917511 0.555540263652802 43.0
79.245497494 0.554192721843719 44.0
79.245497494 0.0926147550344467 45.0
7.2223717101 0.605691373348236 46.0
2.498305106 0.407668918371201 47.0
171.36771588 0.15853913128376 48.0
165.16931911 0.143416181206703 49.0
0 0.536427140235901 50.0
73.827572851 0.396910786628723 51.0
79.114460587 0.145169168710709 52.0
73.827467163 0.0258300565183163 53.0
73.827467163 0.0912626162171364 54.0
73.827467163 0.189091563224792 55.0
0 5.1945276260376 56.0
71.055338292 -0.0583547875285149 57.0
212.15540523 0.54577112197876 58.0
237.64844716 0.655665993690491 59.0
21.923535978 -0.012287893332541 60.0
0 0.0689070224761963 61.0
175.89223632 0.675942003726959 0.0
175.89223632 1.2967209815979 1.0
174.84133717 1.24969244003296 2.0
170.052232 3.18831586837769 3.0
0 3.64386248588562 4.0
0 3.26371455192566 5.0
-69.274919002 -0.0063069760799408 6.0
82.21214305 0.634836077690125 7.0
46.025721922 1.67839765548706 8.0
36.186421128 1.55681824684143 9.0
95.796376484 0.838241279125214 10.0
83.170719869 0.246189638972282 11.0
91.480749786 0.940012693405151 12.0
91.480749786 0.687642991542816 13.0
91.480749786 0.10113912075758 14.0
0 0.147766962647438 15.0
0 0.22397643327713 16.0
0 1.30840587615967 17.0
53.793764564 1.29742288589478 18.0
-53.793764564 1.20377111434937 19.0
-53.793764564 0.718633830547333 20.0
-53.793764564 0.0354205034673214 21.0
76.35103465 2.52645254135132 22.0
138.56804196 2.70328807830811 23.0
106.13479275 0.288680672645569 24.0
22.621420253 0.221013531088829 25.0
214.91907661 1.53506541252136 26.0
214.91907661 1.48947644233704 27.0
27.015907153 2.20811676979065 28.0
241.93498376 1.13854658603668 29.0
241.93498376 1.9653993844986 30.0
46.616049709 1.12380754947662 31.0
9.3318161235 0.307892382144928 32.0
22.557270086 0.800986230373383 33.0
22.557270086 1.19832098484039 34.0
22.557270086 0.0732826963067055 35.0
9.5962121821 0.0239035319536924 36.0
8.2910958494 0.459402650594711 37.0
0 0.0526567026972771 38.0
0 0.27102318406105 39.0
-11.94766413 0.459908008575439 40.0
-25.88185316 0.795011043548584 41.0
5.6557289463 0.379445672035217 42.0
-34.478761529 0.766504943370819 43.0
109.90298963 0.752717912197113 44.0
109.90298963 0.10048758238554 45.0
6.2878953827 0.615711390972137 46.0
2.3457957548 0.370242059230804 47.0
175.89223632 0.155950620770454 48.0
170.052232 0.027283389121294 49.0
0 0.0670924410223961 50.0
82.21214305 0.239630699157715 51.0
95.796376484 0.348692446947098 52.0
91.480749786 0.0482639484107494 53.0
91.480749786 0.340025842189789 54.0
91.480749786 0.0521592944860458 55.0
0 2.69802665710449 56.0
76.35103465 -0.0543405413627625 57.0
214.91907661 0.572322070598602 58.0
241.93498376 1.16949653625488 59.0
22.557270086 0.0232077278196812 60.0
0 0.0897967368364334 61.0
189.36608383 0.838030517101288 0.0
189.36608383 0.956773579120636 1.0
188.08013462 0.724129199981689 2.0
183.29106466 3.49861431121826 3.0
0 6.34245109558105 4.0
0 4.30512237548828 5.0
-74.450218319 0.396753013134003 6.0
86.694990385 0.7676682472229 7.0
59.878177159 1.1282879114151 8.0
26.816813225 1.17024075984955 9.0
84.215520524 0.92149406671524 10.0
71.567776542 0.0548607148230076 11.0
79.244251443 0.190188869833946 12.0
79.244251443 0.540488719940186 13.0
79.244251443 0.103903464972973 14.0
0 0.157466173171997 15.0
0 0.153899729251862 16.0
0 1.53159117698669 17.0
57.833727814 1.11439454555511 18.0
-57.833727814 0.833557784557343 19.0
-57.833727814 0.559919655323029 20.0
-57.833727814 0.0175836160778999 21.0
78.409084814 2.52197694778442 22.0
144.29276811 2.46985864639282 23.0
108.4272569 0.54288387298584 24.0
23.664205459 0.0827183425426483 25.0
222.70185292 1.37095415592194 26.0
222.70185292 1.34745121002197 27.0
24.355180577 1.57639706134796 28.0
247.0570335 1.36059474945068 29.0
247.0570335 1.78734850883484 30.0
47.069837776 1.07345569133759 31.0
9.3589043687 0.442775696516037 32.0
20.575357 1.28147542476654 33.0
20.575357 0.913348257541656 34.0
20.575357 0.153989866375923 35.0
8.8899653519 0.0286379139870405 36.0
7.6437953445 0.342629283666611 37.0
0 0.0260608047246933 38.0
0 0.117551751434803 39.0
-11.536293121 0.797135531902313 40.0
-24.937591666 0.36509981751442 41.0
5.4506686122 0.507036685943604 42.0
-32.979458992 0.721011638641357 43.0
98.779214435 0.637348175048828 44.0
98.779214435 0.113562226295471 45.0
7.312187368 0.522331416606903 46.0
2.8939448951 0.391389489173889 47.0
189.36608383 0.108495146036148 48.0
183.29106466 0.0314492620527744 49.0
0 0.136566922068596 50.0
86.694990385 0.145559966564178 51.0
84.215520524 0.123462036252022 52.0
79.244251443 0.024148341268301 53.0
79.244251443 0.0729960352182388 54.0
79.244251443 0.159246638417244 55.0
0 5.05722570419312 56.0
78.409084814 -0.0121523011475801 57.0
222.70185292 0.329314410686493 58.0
247.0570335 1.15186190605164 59.0
20.575357 -0.00182719901204109 60.0
0 0.0904725715517998 61.0
156.79172529 1.03182232379913 0.0
156.79172529 1.73130071163177 1.0
155.2295763 2.52776741981506 2.0
149.76490557 3.68235039710999 3.0
4.2937617424 6.39559268951416 4.0
4.2937617424 6.76739931106567 5.0
-68.820688193 1.97574901580811 6.0
81.035652249 0.352859139442444 7.0
55.653605988 0.589073300361633 8.0
25.382046261 0.870074391365051 9.0
85.226634036 1.19174754619598 10.0
70.497300106 0.349950283765793 11.0
74.655934959 0.215269446372986 12.0
74.655934959 0.110589735209942 13.0
74.655934959 0.0238722320646048 14.0
0 -0.0608397498726845 15.0
0 0.222899630665779 16.0
0 1.18528914451599 17.0
55.307771832 1.76180815696716 18.0
-55.307771832 1.79230391979218 19.0
-55.307771832 0.260064959526062 20.0
-55.307771832 0.355018377304077 21.0
77.721143576 2.25664448738098 22.0
144.96290683 2.96190309524536 23.0
112.98454028 0.125585466623306 24.0
25.200431303 0.294784843921661 25.0
222.68405041 1.50247704982758 26.0
222.68405041 1.44743406772614 27.0
31.329525303 1.07291877269745 28.0
254.01357571 0.649676442146301 29.0
254.01357571 1.6086345911026 30.0
46.960261806 1.15268695354462 31.0
9.1928053624 0.644026041030884 32.0
22.413371744 1.39108657836914 33.0
22.413371744 1.18134903907776 34.0
22.413371744 0.0711018294095993 35.0
10.146152134 0.0245275851339102 36.0
9.0170490002 0.390843868255615 37.0
0 0.0354237630963326 38.0
0 0.181340575218201 39.0
-11.120472671 0.338215321302414 40.0
-24.284688441 1.02365863323212 41.0
5.4055512608 0.320979654788971 42.0
-32.291299375 0.394686967134476 43.0
79.591364471 0.673094868659973 44.0
79.591364471 0.0996429771184921 45.0
6.3831658038 0.515594840049744 46.0
2.6304742083 0.310433954000473 47.0
156.79172529 0.123989582061768 48.0
149.76490557 0.609649240970612 49.0
4.2937617424 0.657104730606079 50.0
81.035652249 0.0440085157752037 51.0
85.226634036 0.438679397106171 52.0
74.655934959 0.00961015559732914 53.0
74.655934959 0.0667998641729355 54.0
74.655934959 0.234988406300545 55.0
0 5.32104539871216 56.0
77.721143576 0.0736401826143265 57.0
222.68405041 0.530565142631531 58.0
254.01357571 0.144938841462135 59.0
22.413371744 0.0161709897220135 60.0
0 0.0596052929759026 61.0
142.22359954 1.31998407840729 0.0
142.22359954 2.13871693611145 1.0
140.80381906 1.8232262134552 2.0
136.01471397 5.42125749588013 3.0
0 8.79107475280762 4.0
0 4.73690032958984 5.0
-56.902170142 1.90841019153595 6.0
66.056966224 0.56501841545105 7.0
39.465884478 0.638681828975677 8.0
26.591081746 0.647191405296326 9.0
97.758085125 1.54822278022766 10.0
85.966977958 0.462800681591034 11.0
88.706999619 0.66969907283783 12.0
88.706999619 0.639096677303314 13.0
88.706999619 0.108024731278419 14.0
0 0.402001321315765 15.0
0 0.177211061120033 16.0
0 1.57423579692841 17.0
53.889886298 1.75945329666138 18.0
-53.889886298 1.67387712001801 19.0
-53.889886298 0.525702595710754 20.0
-53.889886298 0.0232527107000351 21.0
74.579719173 1.41530871391296 22.0
132.06317907 1.73436439037323 23.0
101.36704801 0.30540657043457 24.0
22.465536706 0.182369738817215 25.0
206.64289825 0.558379352092743 26.0
206.64289825 1.84161007404327 27.0
24.543991539 1.45189452171326 28.0
231.18688978 1.4048877954483 29.0
231.18688978 1.96475756168365 30.0
46.30325049 0.898152112960815 31.0
9.3789317818 0.571681141853333 32.0
20.689832875 1.36354053020477 33.0
20.689832875 1.24672067165375 34.0
20.689832875 0.134106799960136 35.0
8.8924863812 0.034596748650074 36.0
7.5124642697 0.277571171522141 37.0
0 -0.0106798224151134 38.0
0 0.0618329234421253 39.0
-11.785348588 1.03030741214752 40.0
-26.234289938 0.394841194152832 41.0
4.5203459351 0.560461401939392 42.0
-33.34037427 0.452790319919586 43.0
102.1600118 0.705447554588318 44.0
102.1600118 0.15763783454895 45.0
8.1856720457 0.606548011302948 46.0
2.8784199502 0.361216336488724 47.0
142.22359954 0.0878894478082657 48.0
136.01471397 0.232680901885033 49.0
0 0.706404268741608 50.0
66.056966224 0.158403843641281 51.0
97.758085125 0.521438777446747 52.0
88.706999619 0.0501082390546799 53.0
88.706999619 0.0370392873883247 54.0
88.706999619 0.337303102016449 55.0
0 1.98251795768738 56.0
74.579719173 -0.0226755738258362 57.0
206.64289825 0.647803246974945 58.0
231.18688978 1.04512751102448 59.0
20.689832875 -0.00997819006443024 60.0
0 0.086290031671524 61.0
141.47288138 0.96555107831955 0.0
141.47288138 1.30795812606812 1.0
140.15464927 1.69862961769104 2.0
135.36554402 3.29603004455566 3.0
0 4.42811727523804 4.0
0 5.52977085113525 5.0
-52.208744277 2.23180413246155 6.0
62.325660466 0.512007713317871 7.0
44.250469821 1.21086657047272 8.0
18.075190646 1.34775340557098 9.0
74.052078011 1.22214162349701 10.0
60.613185826 0.285435914993286 11.0
74.585368145 0.69984096288681 12.0
74.585368145 0.536537349224091 13.0
74.585368145 0.0489622093737125 14.0
0 0.125980317592621 15.0
0 0.193967878818512 16.0
0 1.11296582221985 17.0
58.026048658 1.68142879009247 18.0
-58.026048658 1.87609767913818 19.0
-58.026048658 0.242463022470474 20.0
-58.026048658 -0.00230229645967484 21.0
78.462129947 1.31292843818665 22.0
137.48127784 0.586310565471649 23.0
102.31178763 1.08577275276184 24.0
22.981284351 0.0669434815645218 25.0
215.94340778 0.653005301952362 26.0
215.94340778 1.4793872833252 27.0
26.075719633 1.64178216457367 28.0
242.01912742 0.690603256225586 29.0
242.01912742 1.85886538028717 30.0
46.432113092 1.03436553478241 31.0
9.423841774 0.514979481697083 32.0
20.436081289 1.12395548820496 33.0
20.436081289 1.24938333034515 34.0
20.436081289 0.107434712350368 35.0
9.7359334136 0.0299260783940554 36.0
8.5156453618 0.388367712497711 37.0
0 0.0331512205302715 38.0
0 0.205975264310837 39.0
-12.063358588 0.765858054161072 40.0
-24.688516078 0.366058230400085 41.0
4.9496962397 0.507342994213104 42.0
-32.015663043 0.458489775657654 43.0
87.827770448 0.658999919891357 44.0
87.827770448 0.0995681956410408 45.0
6.8663821513 0.584680676460266 46.0
2.2967769204 0.416044563055038 47.0
141.47288138 0.0649806559085846 48.0
135.36554402 0.305831432342529 49.0
0 0.387456566095352 50.0
62.325660466 0.273096114397049 51.0
74.052078011 0.469060778617859 52.0
74.585368145 0.104394070804119 53.0
74.585368145 0.086647167801857 54.0
74.585368145 0.086009681224823 55.0
0 3.39164590835571 56.0
78.462129947 0.118359334766865 57.0
215.94340778 0.210544943809509 58.0
242.01912742 0.924376130104065 59.0
20.436081289 -0.00379956886172295 60.0
0 0.0896961688995361 61.0
149.53802641 1.00420784950256 0.0
149.53802641 0.673682808876038 1.0
147.99166425 0.853869736194611 2.0
143.20255911 4.52971410751343 3.0
0 4.03635120391846 4.0
0 3.91660618782043 5.0
-64.120410681 0.94030100107193 6.0
76.000496159 1.60560142993927 7.0
66.244152676 0.601316511631012 8.0
9.7563434838 0.527871370315552 9.0
96.668718355 1.00334775447845 10.0
82.94371564 0.222811818122864 11.0
87.49671786 0.422551929950714 12.0
87.49671786 0.178386360406876 13.0
87.49671786 0.0604087337851524 14.0
0 0.0733951330184937 15.0
0 0.210661202669144 16.0
0 0.994471490383148 17.0
52.620023965 0.692110896110535 18.0
-52.620023965 0.757199347019196 19.0
-52.620023965 0.470952033996582 20.0
-52.620023965 0.0230115950107574 21.0
71.38200837 1.70400667190552 22.0
160.10458658 1.87831854820251 23.0
121.38226482 0.468402087688446 24.0
25.877582861 0.198276951909065 25.0
231.48659495 0.928802669048309 26.0
231.48659495 1.62546229362488 27.0
24.517137528 2.03421306610107 28.0
256.00373248 1.53581869602203 29.0
256.00373248 2.07501077651978 30.0
46.781540647 1.01769697666168 31.0
9.459139685 0.542394399642944 32.0
18.761984405 1.02390778064728 33.0
18.761984405 0.645650804042816 34.0
18.761984405 0.0882755219936371 35.0
10.203704609 0.0294602140784264 36.0
9.1073900224 0.390339076519012 37.0
0 0.0396806821227074 38.0
0 0.276555240154266 39.0
-11.691494814 0.280054420232773 40.0
-23.196809693 1.05480027198792 41.0
5.6114493876 0.378964424133301 42.0
-31.107019463 0.300104558467865 43.0
89.761476804 0.748386442661285 44.0
89.761476804 0.0851963981986046 45.0
6.8772509798 0.619791924953461 46.0
2.4759335417 0.415754109621048 47.0
149.53802641 0.167545408010483 48.0
143.20255911 0.00732662668451667 49.0
0 0.266266912221909 50.0
76.000496159 0.070506177842617 51.0
96.668718355 0.352450132369995 52.0
87.49671786 0.0717134401202202 53.0
87.49671786 0.110885046422482 54.0
87.49671786 0.091050460934639 55.0
0 2.55599784851074 56.0
71.38200837 0.0181545130908489 57.0
231.48659495 0.456457018852234 58.0
256.00373248 1.19963502883911 59.0
18.761984405 0.00799896009266376 60.0
0 0.0931708365678787 61.0
162.29195213 0.938902378082275 0.0
162.29195213 0.574175834655762 1.0
160.97571777 0.945971846580505 2.0
156.18661257 6.64707469940186 3.0
0 5.13038349151611 4.0
0 5.33939981460571 5.0
-70.591465441 4.31309175491333 6.0
79.70071536 1.29350781440735 7.0
57.083952026 0.268391132354736 8.0
22.616763334 0.251588761806488 9.0
87.679401123 0.910412549972534 10.0
76.177330748 -0.00115124112926424 11.0
89.039208189 0.370187371969223 12.0
89.039208189 0.204642787575722 13.0
89.039208189 0.154000118374825 14.0
0 0.37750717997551 15.0
0 0.0885665714740753 16.0
0 1.13422501087189 17.0
49.858836026 1.347776055336 18.0
-49.858836026 1.18700587749481 19.0
-49.858836026 0.393235087394714 20.0
-49.858836026 0.122121229767799 21.0
71.205147963 1.22180318832397 22.0
136.05079522 0.750046670436859 23.0
99.921027771 1.1714940071106 24.0
21.097239401 0.0608668327331543 25.0
207.25594318 1.25036668777466 26.0
207.25594318 2.7063353061676 27.0
26.643728323 2.58518218994141 28.0
233.8996715 1.53363656997681 29.0
233.8996715 2.53698968887329 30.0
47.421990481 0.843797922134399 31.0
9.3572596766 0.387675285339355 32.0
21.346311938 0.526302695274353 33.0
21.346311938 1.20928418636322 34.0
21.346311938 0.079580157995224 35.0
9.47525323 0.0210804361850023 36.0
8.0410430552 0.412684440612793 37.0
0 0.0508901923894882 38.0
0 0.166291028261185 39.0
-13.208342043 1.00883972644806 40.0
-26.226544378 0.327712535858154 41.0
5.3802805813 0.357973515987396 42.0
-34.478777094 0.638558089733124 43.0
103.5710375 0.908435106277466 44.0
103.5710375 0.139205932617188 45.0
5.835086706 0.677684426307678 46.0
1.2581510909 0.282752215862274 47.0
162.29195213 0.12192364782095 48.0
156.18661257 0.283206701278687 49.0
0 0.858013689517975 50.0
79.70071536 0.185624986886978 51.0
87.679401123 0.172777369618416 52.0
89.039208189 0.00429768301546574 53.0
89.039208189 0.010142844170332 54.0
89.039208189 0.0680980235338211 55.0
0 3.74325656890869 56.0
71.205147963 -0.0198637582361698 57.0
207.25594318 0.78571081161499 58.0
233.8996715 1.40325689315796 59.0
21.346311938 0.0310035534203053 60.0
0 0.0992845147848129 61.0
187.16445754 0.804886758327484 0.0
187.16445754 0.945904076099396 1.0
186.24924882 1.79740631580353 2.0
180.78457822 5.35743856430054 3.0
37.373282753 7.33499145507812 4.0
37.373282753 4.49774885177612 5.0
-42.175820352 0.320258319377899 6.0
52.1369808 0.300518393516541 7.0
33.522757401 0.67768120765686 8.0
18.614223399 0.907698452472687 9.0
85.338588506 0.889149248600006 10.0
73.342999099 0.485907971858978 11.0
82.870567431 1.23078572750092 12.0
82.870567431 1.16290903091431 13.0
82.870567431 0.06143793836236 14.0
0 0.150091916322708 15.0
0 0.209405720233917 16.0
0 1.20657026767731 17.0
62.054348985 0.993713617324829 18.0
-62.054348985 0.996527791023254 19.0
-62.054348985 0.274816155433655 20.0
-62.054348985 -0.0585925579071045 21.0
83.572088715 2.17128229141235 22.0
143.73869884 3.19369530677795 23.0
106.8424173 0.1566132158041 24.0
21.074582151 0.262730032205582 25.0
227.31078755 1.82654273509979 26.0
227.31078755 3.18364834785461 27.0
26.477472823 2.26617908477783 28.0
253.78826038 1.45125126838684 29.0
253.78826038 1.67620456218719 30.0
47.000749572 1.02146685123444 31.0
9.4058028007 0.509315669536591 32.0
21.51773973 2.03806352615356 33.0
21.51773973 1.6441490650177 34.0
21.51773973 0.12976536154747 35.0
10.322590817 0.028536532074213 36.0
9.0650239652 0.380376279354095 37.0
0 -0.0433174110949039 38.0
0 0.150942772626877 39.0
-13.308046683 0.523341536521912 40.0
-25.510311168 0.941088199615479 41.0
5.867302282 0.398718595504761 42.0
-34.189919283 0.723178505897522 43.0
86.660757108 0.887406527996063 44.0
86.660757108 0.103922627866268 45.0
7.0564764154 0.647447645664215 46.0
2.2798927439 0.274160325527191 47.0
187.16445754 0.355193436145782 48.0
180.78457822 0.412701636552811 49.0
37.373282753 0.389248311519623 50.0
52.1369808 0.00758172944188118 51.0
85.338588506 0.739322066307068 52.0
82.870567431 0.156198993325233 53.0
82.870567431 0.387225419282913 54.0
82.870567431 0.0416096337139606 55.0
0 2.17347359657288 56.0
83.572088715 0.142727240920067 57.0
227.31078755 1.26680886745453 58.0
253.78826038 0.838292360305786 59.0
21.51773973 0.0571253933012486 60.0
0 0.12038379907608 61.0
148.44318286 0.883876502513885 0.0
148.44318286 1.20266532897949 1.0
147.08127841 1.56538486480713 2.0
141.61660769 3.77811622619629 3.0
4.0734926628 5.19208812713623 4.0
4.0734926628 4.73903036117554 5.0
-60.732978256 0.344198971986771 6.0
73.896260485 0.895570695400238 7.0
49.050086675 0.783317863941193 8.0
24.846173809 0.596620202064514 9.0
95.92056965 1.04894948005676 10.0
81.726683275 0.262307167053223 11.0
85.171481263 0.766980707645416 12.0
85.171481263 0.206568092107773 13.0
85.171481263 0.127412557601929 14.0
0 0.276161521673203 15.0
0 0.165370941162109 16.0
0 1.41653299331665 17.0
58.013166269 1.8390725851059 18.0
-58.013166269 1.74509024620056 19.0
-58.013166269 0.640756905078888 20.0
-58.013166269 -0.0349460877478123 21.0
79.534635786 1.21936166286469 22.0
141.85994287 1.1482310295105 23.0
103.16379753 0.607307732105255 24.0
21.488096843 0.156284123659134 25.0
221.39457866 0.869481146335602 26.0
221.39457866 1.50896155834198 27.0
26.936353456 1.92083251476288 28.0
248.33093211 0.856212079524994 29.0
248.33093211 1.96810507774353 30.0
46.513438587 1.11463725566864 31.0
9.2232698741 0.478847354650497 32.0
21.521469516 1.7213921546936 33.0
21.521469516 0.975116372108459 34.0
21.521469516 0.0961166843771935 35.0
9.6071669553 0.0349160358309746 36.0
8.2232655122 0.386381328105927 37.0
0 0.0695229023694992 38.0
0 0.231369346380234 39.0
-11.141974028 0.732779502868652 40.0
-24.318612235 0.541802704334259 41.0
5.5177543802 0.391299188137054 42.0
-32.347591216 0.599729776382446 43.0
84.026562425 0.583755850791931 44.0
84.026562425 0.128916800022125 45.0
7.1973694326 0.448184251785278 46.0
2.2694114308 0.343086153268814 47.0
148.44318286 0.157298073172569 48.0
141.61660769 0.253096699714661 49.0
4.0734926628 0.353779852390289 50.0
73.896260485 0.269540131092072 51.0
95.92056965 0.410888016223907 52.0
85.171481263 0.10386373847723 53.0
85.171481263 0.0437798500061035 54.0
85.171481263 0.0383658930659294 55.0
0 3.29688906669617 56.0
79.534635786 0.165142640471458 57.0
221.39457866 0.455485314130783 58.0
248.33093211 0.676536440849304 59.0
21.521469516 -0.0112052466720343 60.0
0 0.083508774638176 61.0
134.45568 1.01254677772522 0.0
134.45568 1.16072368621826 1.0
132.88930357 1.39117455482483 2.0
128.10019829 4.36787462234497 3.0
0 5.35901308059692 4.0
0 6.48858880996704 5.0
-60.893006589 3.86567902565002 6.0
70.570191957 0.76644092798233 7.0
46.789132624 1.30918192863464 8.0
23.781059333 1.0469536781311 9.0
91.532044818 1.02025532722473 10.0
77.551602615 0.0796131491661072 11.0
77.774795038 0.730268180370331 12.0
77.774795038 0.565888643264771 13.0
77.774795038 0.101724378764629 14.0
0 0.263174086809158 15.0
0 0.194843515753746 16.0
0 1.47741115093231 17.0
55.906759464 2.01889085769653 18.0
-55.906759464 1.54590797424316 19.0
-55.906759464 0.968496799468994 20.0
-55.906759464 -0.0514481589198112 21.0
75.285307473 1.27398443222046 22.0
127.89029336 0.733912706375122 23.0
98.199415758 0.438053011894226 24.0
20.791916166 0.0938285440206528 25.0
203.17560083 1.09651505947113 26.0
203.17560083 1.4708137512207 27.0
22.226218022 0.765828728675842 28.0
225.40181885 0.580049335956573 29.0
225.40181885 1.73993158340454 30.0
46.543115647 1.05517089366913 31.0
9.3773937728 0.528276801109314 32.0
19.378548009 1.38603007793427 33.0
19.378548009 0.664665937423706 34.0
19.378548009 0.0737621262669563 35.0
8.9738093021 0.0207638684660196 36.0
7.7506289787 0.43762069940567 37.0
0 0.0326641723513603 38.0
0 0.0568509921431541 39.0
-11.011561503 0.523882627487183 40.0
-23.828731079 0.604956388473511 41.0
5.6850190496 0.481156587600708 42.0
-32.193666092 0.447439640760422 43.0
86.90399783 0.826016962528229 44.0
86.90399783 0.0547068491578102 45.0
6.8183393378 0.662903070449829 46.0
2.470649002 0.416354060173035 47.0
134.45568 0.163474753499031 48.0
128.10019829 0.181248188018799 49.0
0 0.52591335773468 50.0
70.570191957 0.228951424360275 51.0
91.532044818 0.118617855012417 52.0
77.774795038 0.0387605614960194 53.0
77.774795038 0.0483179315924644 54.0
77.774795038 0.266275554895401 55.0
0 5.65216064453125 56.0
75.285307473 0.0036324467509985 57.0
203.17560083 0.207002058625221 58.0
225.40181885 0.316114723682404 59.0
19.378548009 -0.0114111891016364 60.0
0 0.0960958525538445 61.0
174.23553928 0.739185035228729 0.0
174.23553928 1.46224820613861 1.0
172.87038343 1.41743433475494 2.0
167.40571292 3.15812754631042 3.0
20.026764744 6.83105993270874 4.0
20.026764744 3.37393021583557 5.0
-56.817482229 0.252271473407745 6.0
67.486626938 0.567979216575623 7.0
41.280723092 0.513426303863525 8.0
26.205903846 0.600036442279816 9.0
84.791843699 0.901128172874451 10.0
71.85280483 0.448538780212402 11.0
72.529951167 0.986134052276611 12.0
72.529951167 0.417451500892639 13.0
72.529951167 0.0518784001469612 14.0
0 0.0258677136152983 15.0
0 0.190167814493179 16.0
0 1.35212123394012 17.0
53.309884324 1.66438686847687 18.0
-53.309884324 1.42830085754395 19.0
-53.309884324 0.552765607833862 20.0
-53.309884324 -0.0216770209372044 21.0
73.350752899 0.992293477058411 22.0
142.79952362 0.666194438934326 23.0
109.78252772 0.726329624652863 24.0
18.984860494 0.186087772250175 25.0
216.15027652 0.874839961528778 26.0
216.15027652 1.74355697631836 27.0
22.397190666 2.65818047523499 28.0
238.54746719 1.92935347557068 29.0
238.54746719 2.12141299247742 30.0
47.003289393 0.999005436897278 31.0
9.4125863917 0.448402971029282 32.0
20.040868575 1.54591298103333 33.0
20.040868575 1.1364278793335 34.0
20.040868575 0.133678779006004 35.0
9.3003768706 0.0305823385715485 36.0
7.9229481103 0.373881697654724 37.0
0 -0.00386092439293861 38.0
0 0.291173964738846 39.0
-11.736615706 0.708613157272339 40.0
-24.808051562 0.735584855079651 41.0
4.8035165061 0.439708650112152 42.0
-32.138519868 0.56216949224472 43.0
110.63885847 0.716500341892242 44.0
110.63885847 0.0542277321219444 45.0
5.9417201309 0.643040359020233 46.0
2.3147838076 0.444126904010773 47.0
174.23553928 0.161808967590332 48.0
167.40571292 0.0503315255045891 49.0
20.026764744 0.609910011291504 50.0
67.486626938 0.27971076965332 51.0
84.791843699 0.0651375353336334 52.0
72.529951167 0.119856245815754 53.0
72.529951167 0.262068122625351 54.0
72.529951167 0.0824882984161377 55.0
0 5.27949953079224 56.0
73.350752899 -0.0142530044540763 57.0
216.15027652 0.309034079313278 58.0
238.54746719 1.44705259799957 59.0
20.040868575 0.00220710411667824 60.0
0 0.104561626911163 61.0
205.68431563 1.00860965251923 0.0
205.68431563 0.666595876216888 1.0
204.14381087 1.42418646812439 2.0
199.35470562 5.49476146697998 3.0
0 7.83516216278076 4.0
0 3.46660947799683 5.0
-51.8655256 2.37551712989807 6.0
62.294710392 0.461019814014435 7.0
52.013343645 0.5986687541008 8.0
10.281366748 0.839747130870819 9.0
82.15245044 1.22836685180664 10.0
70.848211937 0.323231965303421 11.0
85.139855919 0.394465416669846 12.0
85.139855919 0.550842940807343 13.0
85.139855919 0.0574054941534996 14.0
0 0.0261661782860756 15.0
0 0.208289787173271 16.0
0 0.666898131370544 17.0
51.075416356 1.1498281955719 18.0
-51.075416356 0.697976648807526 19.0
-51.075416356 1.29943883419037 20.0
-51.075416356 -0.116064995527267 21.0
72.37363016 1.52953994274139 22.0
146.93263446 1.57385766506195 23.0
112.93653626 0.664166688919067 24.0
24.309485824 0.0890201330184937 25.0
219.30626462 0.716525673866272 26.0
219.30626462 0.812538981437683 27.0
27.604269858 0.574180901050568 28.0
246.91053448 0.518308937549591 29.0
246.91053448 1.34339880943298 30.0
46.42012217 1.1081417798996 31.0
9.1697627169 0.458167195320129 32.0
21.298213804 1.82227313518524 33.0
21.298213804 1.1101039648056 34.0
21.298213804 0.0915945395827293 35.0
9.7639305725 0.0246775485575199 36.0
8.4348117277 0.290429145097733 37.0
0 0.046940490603447 38.0
0 0.216982245445251 39.0
-11.851936192 0.543303787708282 40.0
-24.288472509 0.590518414974213 41.0
4.5540045273 0.673065781593323 42.0
-31.448819557 0.719964206218719 43.0
92.351618425 0.629690289497375 44.0
92.351618425 0.0957650989294052 45.0
7.3780339492 0.430199801921844 46.0
2.7780386022 0.327584058046341 47.0
205.68431563 0.10216261446476 48.0
199.35470562 0.292905658483505 49.0
0 0.538851082324982 50.0
62.294710392 0.0827770605683327 51.0
82.15245044 0.577775597572327 52.0
85.139855919 0.0483157373964787 53.0
85.139855919 0.192944496870041 54.0
85.139855919 0.0720482468605042 55.0
0 5.35010957717896 56.0
72.37363016 0.248434156179428 57.0
219.30626462 0.178485870361328 58.0
246.91053448 0.459668070077896 59.0
21.298213804 0.0143673121929169 60.0
0 0.0797900930047035 61.0
142.5414676 0.896335303783417 0.0
142.5414676 0.66567325592041 1.0
141.29898283 1.74240100383759 2.0
136.50991287 4.51105785369873 3.0
0 3.85506463050842 4.0
0 6.85995292663574 5.0
-69.791105713 5.81985759735107 6.0
80.736240696 1.18135333061218 7.0
52.789319119 1.04896724224091 8.0
27.946921577 1.1751868724823 9.0
101.32589231 0.963786840438843 10.0
87.712803682 0.228270962834358 11.0
91.30200341 0.597597181797028 12.0
91.30200341 0.0991689264774323 13.0
91.30200341 0.0612475238740444 14.0
0 0.187347799539566 15.0
0 0.171947211027145 16.0
0 1.31854891777039 17.0
55.047651779 1.39401340484619 18.0
-55.047651779 1.37534236907959 19.0
-55.047651779 0.353575795888901 20.0
-55.047651779 0.00576651468873024 21.0
76.315147157 1.63658404350281 22.0
151.63648346 1.69938433170319 23.0
112.65685602 0.435383170843124 24.0
18.691788938 0.1060681194067 25.0
227.95163061 1.01011061668396 26.0
227.95163061 1.33206379413605 27.0
21.966068595 1.54970479011536 28.0
249.91769921 0.689214289188385 29.0
249.91769921 1.74164974689484 30.0
46.071892912 0.951567053794861 31.0
9.2554181345 0.393087238073349 32.0
21.267495378 1.76579236984253 33.0
21.267495378 0.803799390792847 34.0
21.267495378 0.125123143196106 35.0
9.1489852997 0.0268064104020596 36.0
8.0092271173 0.38080969452858 37.0
0 0.0501157343387604 38.0
0 0.183906823396683 39.0
-11.302024285 0.610644042491913 40.0
-24.178508841 0.47208046913147 41.0
5.4340131595 0.577025234699249 42.0
-32.058448239 0.782218098640442 43.0
90.460367404 0.731582283973694 44.0
90.460367404 0.0618868358433247 45.0
6.8146775536 0.669390320777893 46.0
2.7152763057 0.35558158159256 47.0
142.5414676 0.172002568840981 48.0
136.50991287 0.220508739352226 49.0
0 0.379509180784225 50.0
80.736240696 0.175653547048569 51.0
101.32589231 0.373361110687256 52.0
91.30200341 0.0280339196324348 53.0
91.30200341 0.154214903712273 54.0
91.30200341 0.108312368392944 55.0
0 0.247506931424141 56.0
76.315147157 0.0275934170931578 57.0
227.95163061 0.594892919063568 58.0
249.91769921 0.851777315139771 59.0
21.267495378 0.00302357785403728 60.0
0 0.101322710514069 61.0
126.45720862 0.783340096473694 0.0
126.45720862 0.744838535785675 1.0
125.5012725 1.161989569664 2.0
120.71216732 3.24360680580139 3.0
0 3.95812463760376 4.0
0 5.59907197952271 5.0
-51.134697005 4.78711271286011 6.0
63.263378656 0.427790284156799 7.0
50.994675448 0.777585029602051 8.0
12.268703208 0.938355624675751 9.0
77.819180739 0.958688616752625 10.0
64.831956813 0.352955162525177 11.0
75.370932761 0.93589574098587 12.0
75.370932761 0.77994704246521 13.0
75.370932761 0.0156706031411886 14.0
0 -0.00328864995390177 15.0
0 0.248152792453766 16.0
0 0.877309799194336 17.0
64.434599665 1.43806123733521 18.0
-64.434599665 1.05715215206146 19.0
-64.434599665 0.669489145278931 20.0
-64.434599665 0.0796843618154526 21.0
85.521545206 1.78198027610779 22.0
146.12602312 1.1308981180191 23.0
113.36243993 0.506041169166565 24.0
20.850536129 0.171693503856659 25.0
231.64756832 0.88313752412796 26.0
231.64756832 1.49117314815521 27.0
27.552838971 1.90694189071655 28.0
259.20040729 0.709677040576935 29.0
259.20040729 1.77244770526886 30.0
46.649670579 1.12346422672272 31.0
9.5069450284 0.704993367195129 32.0
21.086945541 1.2151905298233 33.0
21.086945541 1.27665328979492 34.0
21.086945541 0.147429198026657 35.0
10.278056575 0.0316382125020027 36.0
9.0908010915 0.414962619543076 37.0
0 0.00598512589931488 38.0
0 0.268368721008301 39.0
-10.725395967 0.709573149681091 40.0
-23.768725172 0.412201702594757 41.0
4.7140759776 0.498434811830521 42.0
-31.322817864 0.762837290763855 43.0
108.62631246 0.691627323627472 44.0
108.62631246 0.150770574808121 45.0
7.1063242657 0.520706534385681 46.0
1.9986173883 0.346340149641037 47.0
126.45720862 0.0762152150273323 48.0
120.71216732 0.142586842179298 49.0
0 0.410516709089279 50.0
63.263378656 0.0859480798244476 51.0
77.819180739 0.428435206413269 52.0
75.370932761 0.0420986413955688 53.0
75.370932761 0.297565400600433 54.0
75.370932761 0.15199801325798 55.0
0 5.1666407585144 56.0
85.521545206 0.123581290245056 57.0
231.64756832 0.262900471687317 58.0
259.20040729 0.394604742527008 59.0
21.086945541 -0.0141333248466253 60.0
0 0.116840109229088 61.0
188.88492301 0.896037340164185 0.0
188.88492301 0.821733057498932 1.0
187.70732542 1.13882339000702 2.0
182.91822013 4.25247240066528 3.0
0 4.54739189147949 4.0
0 2.98749423027039 5.0
-62.417313731 0.148374050855637 6.0
74.365900359 0.720742166042328 7.0
62.827880476 1.45785439014435 8.0
11.538019883 1.76001262664795 9.0
89.355863787 0.898636698722839 10.0
76.310323622 0.0987300723791122 11.0
78.702042331 0.853452563285828 12.0
78.702042331 0.703805208206177 13.0
78.702042331 0.104798756539822 14.0
0 0.0732143148779869 15.0
0 0.234175711870193 16.0
0 1.29801821708679 17.0
59.334675125 1.74184727668762 18.0
-59.334675125 1.7046537399292 19.0
-59.334675125 0.205801159143448 20.0
-59.334675125 0.0609842464327812 21.0
79.354833042 2.20381116867065 22.0
148.5938632 2.10063624382019 23.0
113.37814468 0.455469906330109 24.0
22.908951917 0.168466627597809 25.0
227.94869624 0.927442252635956 26.0
227.94869624 1.65325117111206 27.0
27.307259645 1.94352376461029 28.0
255.25595589 1.08593487739563 29.0
255.25595589 1.66618120670319 30.0
46.90991487 0.8764368891716 31.0
9.2980290422 0.508247196674347 32.0
20.020157917 1.11749792098999 33.0
20.020157917 0.892766416072845 34.0
20.020157917 0.135545313358307 35.0
8.9792273746 0.0366876907646656 36.0
7.7514168708 0.425873935222626 37.0
0 0.0684886127710342 38.0
0 0.113661572337151 39.0
-11.359805578 0.821952879428864 40.0
-24.416593195 0.451545745134354 41.0
5.2387106005 0.454574465751648 42.0
-32.477185007 0.688581824302673 43.0
85.737641742 0.69802862405777 44.0
85.737641742 0.11371973156929 45.0
6.9373317214 0.596454501152039 46.0
1.7869796644 0.390366733074188 47.0
188.88492301 0.14720493555069 48.0
182.91822013 0.0720366090536118 49.0
0 0.0546736121177673 50.0
74.365900359 0.284009337425232 51.0
89.355863787 0.197432726621628 52.0
78.702042331 0.091759204864502 53.0
78.702042331 0.202932894229889 54.0
78.702042331 0.106277748942375 55.0
0 6.40106725692749 56.0
79.354833042 -0.0337448492646217 57.0
227.94869624 0.296437203884125 58.0
255.25595589 1.01068103313446 59.0
20.020157917 0.00172186270356178 60.0
0 0.0999620854854584 61.0
190.10892855 0.877425730228424 0.0
190.10892855 1.16860556602478 1.0
188.90693851 1.31719756126404 2.0
184.11783324 3.89320158958435 3.0
0 9.50790882110596 4.0
0 3.88055396080017 5.0
-57.724123077 0.530786395072937 6.0
67.701396411 0.518294990062714 7.0
50.421193351 1.30299711227417 8.0
17.28020306 1.33986377716064 9.0
90.07208072 1.06821691989899 10.0
76.818078341 0.37662798166275 11.0
83.695799577 0.847671031951904 12.0
83.695799577 0.764571249485016 13.0
83.695799577 0.055750135332346 14.0
0 0.148133590817451 15.0
0 0.198480427265167 16.0
0 1.50945007801056 17.0
60.748587059 1.34657382965088 18.0
-60.748587059 1.284099817276 19.0
-60.748587059 0.57858681678772 20.0
-60.748587059 -0.0659148842096329 21.0
81.734303514 2.07133007049561 22.0
139.92845336 1.39750647544861 23.0
107.22758835 0.653128325939178 24.0
22.258690171 0.0807948261499405 25.0
221.66275687 1.13506090641022 26.0
221.66275687 1.82425999641418 27.0
29.016828546 1.22833979129791 28.0
250.67958542 0.577105283737183 29.0
250.67958542 1.56776523590088 30.0
46.468860165 1.17612552642822 31.0
9.2961182977 0.639299929141998 32.0
20.985716455 1.57595634460449 33.0
20.985716455 1.05762040615082 34.0
20.985716455 0.0801571756601334 35.0
9.4427999088 0.0225427523255348 36.0
8.2277874908 0.408896267414093 37.0
0 0.0399548336863518 38.0
0 0.180766806006432 39.0
-12.608876743 0.618034839630127 40.0
-25.944759952 0.423881113529205 41.0
5.0057477525 0.512594282627106 42.0
-33.694726775 0.595147967338562 43.0
84.227755584 0.63560152053833 44.0
84.227755584 0.0906277075409889 45.0
7.2892813175 0.57773220539093 46.0
2.7518445458 0.389414578676224 47.0
190.10892855 0.121721595525742 48.0
184.11783324 0.278703540563583 49.0
0 0.315731853246689 50.0
67.701396411 0.212983831763268 51.0
90.07208072 0.171557068824768 52.0
83.695799577 0.157029271125793 53.0
83.695799577 0.235855430364609 54.0
83.695799577 0.290970653295517 55.0
0 1.19786846637726 56.0
81.734303514 0.0922393649816513 57.0
221.66275687 0.365109622478485 58.0
250.67958542 0.612196266651154 59.0
20.985716455 0.0277629420161247 60.0
0 0.0858553573489189 61.0
144.94870121 1.18644988536835 0.0
144.94870121 0.97168892621994 1.0
143.74927966 1.26682710647583 2.0
138.96017456 4.74141359329224 3.0
0 4.56449890136719 4.0
0 2.80225586891174 5.0
-45.997545419 -0.0326422974467278 6.0
56.293222608 0.564949154853821 7.0
44.943707334 0.521281003952026 8.0
11.349515274 0.645879864692688 9.0
94.042301879 1.25355100631714 10.0
79.957973388 0.00276172789745033 11.0
90.124374948 0.625028848648071 12.0
90.124374948 0.70599901676178 13.0
90.124374948 0.104931548237801 14.0
0 0.19978266954422 15.0
0 0.171276062726974 16.0
0 1.66568160057068 17.0
66.659958892 2.07816219329834 18.0
-66.659958892 1.39884090423584 19.0
-66.659958892 0.433647185564041 20.0
-66.659958892 -0.0642709508538246 21.0
88.885635141 2.66325950622559 22.0
143.03828603 1.62259602546692 23.0
107.19780804 0.490407973527908 24.0
20.537126554 0.253327786922455 25.0
231.92392117 1.82125568389893 26.0
231.92392117 2.75820779800415 27.0
30.648114514 2.8956573009491 28.0
262.57203568 1.77581262588501 29.0
262.57203568 2.76605772972107 30.0
47.533905178 1.17527103424072 31.0
9.466654687 0.539670288562775 32.0
22.225676249 1.82463824748993 33.0
22.225676249 0.73317289352417 34.0
22.225676249 0.0776803195476532 35.0
9.4462733533 0.0245538782328367 36.0
8.2548780743 0.375462144613266 37.0
0 0.014310821890831 38.0
0 0.290889233350754 39.0
-10.529085226 0.691691935062408 40.0
-23.247756264 0.609915673732758 41.0
4.9918288859 0.46388766169548 42.0
-31.135407911 0.517056703567505 43.0
101.83127413 0.84874302148819 44.0
101.83127413 0.0929524824023247 45.0
6.9498142467 0.513636469841003 46.0
3.0735791918 0.373415321111679 47.0
144.94870121 0.174746870994568 48.0
138.96017456 0.0833521783351898 49.0
0 0.384049117565155 50.0
56.293222608 0.225720629096031 51.0
94.042301879 0.0987352132797241 52.0
90.124374948 0.0285004433244467 53.0
90.124374948 0.138922572135925 54.0
90.124374948 0.0536306649446487 55.0
0 0.863305926322937 56.0
88.885635141 0.0177848152816296 57.0
231.92392117 0.578917503356934 58.0
262.57203568 1.58801865577698 59.0
22.225676249 0.0336896516382694 60.0
0 0.0890274941921234 61.0
195.80830855 1.04445254802704 0.0
195.80830855 0.892417013645172 1.0
194.38278538 1.38680565357208 2.0
189.59368029 3.20905447006226 3.0
0 5.10536432266235 4.0
0 4.59056758880615 5.0
-73.936601256 1.34196448326111 6.0
85.171228013 0.544631004333496 7.0
64.946248735 1.25288105010986 8.0
20.224979278 1.2705717086792 9.0
94.665457244 1.11918091773987 10.0
81.583561892 0.33837679028511 11.0
80.249153537 0.410970747470856 12.0
80.249153537 0.549891710281372 13.0
80.249153537 0.0452404990792274 14.0
0 0.182683318853378 15.0
0 0.188622877001762 16.0
0 1.19120371341705 17.0
54.790473126 1.12415814399719 18.0
-54.790473126 0.897156536579132 19.0
-54.790473126 0.317417919635773 20.0
-54.790473126 0.0194554142653942 21.0
77.058954503 1.77503728866577 22.0
135.80478387 1.95418763160706 23.0
104.84769919 0.148373663425446 24.0
19.639914913 0.213829219341278 25.0
212.86373837 1.18899703025818 26.0
212.86373837 1.97384285926819 27.0
29.609505182 2.20503211021423 28.0
242.47324355 1.10868656635284 29.0
242.47324355 1.76056253910065 30.0
46.917363001 0.998308956623077 31.0
9.3000903128 0.574305593967438 32.0
22.268481377 1.48446154594421 33.0
22.268481377 0.672118604183197 34.0
22.268481377 0.0789095759391785 35.0
10.36531417 0.0298432819545269 36.0
9.0405062318 0.382853925228119 37.0
0 0.0591454915702343 38.0
0 0.233650729060173 39.0
-12.618556312 0.593489587306976 40.0
-25.336605962 0.787976145744324 41.0
4.7554236871 0.495997369289398 42.0
-32.560798126 0.516856372356415 43.0
103.31055626 0.8302983045578 44.0
103.31055626 0.118111923336983 45.0
6.4381596996 0.546782493591309 46.0
2.2958756603 0.343316704034805 47.0
195.80830855 0.112467400729656 48.0
189.59368029 0.185375481843948 49.0
0 0.193884089589119 50.0
85.171228013 0.115860037505627 51.0
94.665457244 0.16893894970417 52.0
80.249153537 0.0531044900417328 53.0
80.249153537 0.0493238009512424 54.0
80.249153537 0.249002143740654 55.0
0 3.72500014305115 56.0
77.058954503 -0.0187746807932854 57.0
212.86373837 0.404874891042709 58.0
242.47324355 0.550926268100739 59.0
22.268481377 -0.000759173184633255 60.0
0 0.0961634963750839 61.0
174.01707332 0.977249264717102 0.0
174.01707332 1.5596182346344 1.0
173.00499201 1.66463470458984 2.0
168.21588677 4.06873512268066 3.0
0 6.82217121124268 4.0
0 4.02469253540039 5.0
-53.272752836 1.10364615917206 6.0
62.907736476 0.622594714164734 7.0
51.618833522 0.920745491981506 8.0
11.288902953 0.973511099815369 9.0
74.422512699 1.00247263908386 10.0
61.552394364 0.104563228785992 11.0
76.417045621 0.19783441722393 12.0
76.417045621 0.145076334476471 13.0
76.417045621 0.0707774609327316 14.0
0 0.0617310591042042 15.0
0 0.165124803781509 16.0
0 1.22607111930847 17.0
55.309198287 1.43634343147278 18.0
-55.309198287 1.06167984008789 19.0
-55.309198287 0.466257095336914 20.0
-55.309198287 0.0491742677986622 21.0
76.757530173 2.57935953140259 22.0
143.05044154 2.2373218536377 23.0
110.59923467 0.407272666692734 24.0
23.342709887 0.208111733198166 25.0
219.80797171 1.17838990688324 26.0
219.80797171 1.58175480365753 27.0
25.848900858 1.24190175533295 28.0
245.65687257 0.777312099933624 29.0
245.65687257 1.59374046325684 30.0
47.150262724 0.980101525783539 31.0
9.447115369 0.427665501832962 32.0
21.448331886 1.19701433181763 33.0
21.448331886 0.804029405117035 34.0
21.448331886 0.126603573560715 35.0
9.361900834 0.0214599315077066 36.0
8.1467178543 0.384818106889725 37.0
0 0.0243343114852905 38.0
0 0.305038154125214 39.0
-11.367518673 0.556504487991333 40.0
-23.479429922 0.55333536863327 41.0
5.1103967099 0.566518664360046 42.0
-31.377391121 0.358521610498428 43.0
95.260211729 0.807386338710785 44.0
95.260211729 0.0666332766413689 45.0
7.063358935 0.525204420089722 46.0
2.4871626868 0.367117196321487 47.0
174.01707332 0.150633603334427 48.0
168.21588677 0.134532555937767 49.0
0 0.531970918178558 50.0
62.907736476 0.292852222919464 51.0
74.422512699 0.245069146156311 52.0
76.417045621 0.0108024738729 53.0
76.417045621 0.0902168303728104 54.0
76.417045621 0.078898087143898 55.0
0 4.90438413619995 56.0
76.757530173 0.0338196717202663 57.0
219.80797171 0.305971026420593 58.0
245.65687257 0.301369488239288 59.0
21.448331886 0.0122229587286711 60.0
0 0.108052626252174 61.0
181.5183248 0.705926060676575 0.0
181.5183248 1.57190871238708 1.0
180.23273034 1.19564998149872 2.0
175.36874665 5.12022924423218 3.0
0 8.67783451080322 4.0
0 3.24150371551514 5.0
-59.072182936 -0.0254942029714584 6.0
70.070362618 0.680279076099396 7.0
37.99571088 1.13456249237061 8.0
32.074651738 1.08738029003143 9.0
96.94794065 0.782699525356293 10.0
84.628156619 0.19232514500618 11.0
85.879035142 0.59362518787384 12.0
85.879035142 0.5167236328125 13.0
85.879035142 0.0744045376777649 14.0
0 0.181920096278191 15.0
0 0.215655222535133 16.0
0 0.696107268333435 17.0
52.99060809 1.5162079334259 18.0
-52.99060809 1.82922649383545 19.0
-52.99060809 0.457482665777206 20.0
-52.99060809 -0.0561605431139469 21.0
73.069203644 1.87944555282593 22.0
147.64354683 1.63223874568939 23.0
115.81545302 0.477463781833649 24.0
19.246918837 0.11399157345295 25.0
220.71275048 1.55928993225098 26.0
220.71275048 1.32140958309174 27.0
23.019072533 1.60509777069092 28.0
243.73182301 0.796985745429993 29.0
243.73182301 1.57317423820496 30.0
46.343014205 1.23623657226562 31.0
9.3966069339 0.425382137298584 32.0
20.078595553 1.69101524353027 33.0
20.078595553 0.762485325336456 34.0
20.078595553 0.143507093191147 35.0
9.7397432116 0.0356718376278877 36.0
8.4209145176 0.388854682445526 37.0
0 0.0191627703607082 38.0
0 0.284840553998947 39.0
-12.514405426 0.558290839195251 40.0
-25.508959778 0.70572566986084 41.0
5.5700185072 0.577008068561554 42.0
-33.865694536 0.602469146251678 43.0
83.547635903 0.741849184036255 44.0
83.547635903 0.093963049352169 45.0
6.885872485 0.682054400444031 46.0
2.2304981919 0.419632941484451 47.0
181.5183248 0.117509081959724 48.0
175.36874665 0.0201405584812164 49.0
0 0.229582652449608 50.0
70.070362618 0.309085845947266 51.0
96.94794065 0.243177458643913 52.0
85.879035142 0.0426028557121754 53.0
85.879035142 0.169029265642166 54.0
85.879035142 0.0803831219673157 55.0
0 2.89253664016724 56.0
73.069203644 0.0912391990423203 57.0
220.71275048 0.582872092723846 58.0
243.73182301 0.103889890015125 59.0
20.078595553 0.000818585976958275 60.0
0 0.105693012475967 61.0
128.38115941 0.767703294754028 0.0
128.38115941 1.21651232242584 1.0
127.07861303 1.59376871585846 2.0
122.28954307 6.47645616531372 3.0
0 3.74037098884583 4.0
0 4.20861434936523 5.0
-56.976615201 2.48267555236816 6.0
69.729508567 0.419838339090347 7.0
44.89515062 0.716274380683899 8.0
24.834357948 0.765544176101685 9.0
83.47489562 0.793497383594513 10.0
70.09076985 0.0429087840020657 11.0
73.389544594 0.991071939468384 12.0
73.389544594 0.626594364643097 13.0
73.389544594 0.0264783427119255 14.0
0 0.000290278345346451 15.0
0 0.22709909081459 16.0
0 1.26185858249664 17.0
65.172951833 1.32940268516541 18.0
-65.172951833 1.39349961280823 19.0
-65.172951833 0.676231741905212 20.0
-65.172951833 -0.0659674406051636 21.0
84.672743473 2.63375949859619 22.0
146.0220289 2.178706407547 23.0
107.22959122 0.492682814598083 24.0
24.336947367 0.0730364173650742 25.0
230.69477237 1.36573016643524 26.0
230.69477237 1.54855227470398 27.0
25.686430591 2.19543647766113 28.0
256.38120297 1.32610905170441 29.0
256.38120297 1.64048898220062 30.0
46.024579455 0.945054292678833 31.0
9.2427937103 0.350473642349243 32.0
19.49979164 2.48002600669861 33.0
19.49979164 0.641038000583649 34.0
19.49979164 0.0795600414276123 35.0
9.0953814449 0.0121160866692662 36.0
7.8112197247 0.371262013912201 37.0
0 0.0129103437066078 38.0
0 0.164832338690758 39.0
-12.277387566 0.863594889640808 40.0
-25.425412783 0.582134187221527 41.0
4.8896190178 0.274066746234894 42.0
-32.498187876 0.636922478675842 43.0
90.197370135 0.685608267784119 44.0
90.197370135 0.0909993648529053 45.0
6.7083314219 0.571544528007507 46.0
2.2264394125 0.372440427541733 47.0
128.38115941 0.187148049473763 48.0
122.28954307 0.360163509845734 49.0
0 0.346660137176514 50.0
69.729508567 0.235846996307373 51.0
83.47489562 0.153595089912415 52.0
73.389544594 0.125438362360001 53.0
73.389544594 0.0689295828342438 54.0
73.389544594 0.0995515733957291 55.0
0 4.03459930419922 56.0
84.672743473 0.122416295111179 57.0
230.69477237 0.382218778133392 58.0
256.38120297 0.813690483570099 59.0
19.49979164 -0.00654120463877916 60.0
0 0.102640584111214 61.0
205.55374133 0.782604455947876 0.0
205.55374133 1.35348415374756 1.0
204.39581554 1.41181421279907 2.0
198.93114489 6.38553476333618 3.0
2.9535595117 3.90684819221497 4.0
2.9535595117 6.34157085418701 5.0
-70.081065316 0.959128379821777 6.0
82.748117402 0.459414333105087 7.0
43.17002982 0.745476484298706 8.0
39.578087583 0.821025133132935 9.0
96.717729055 0.982579290866852 10.0
82.128965463 0.34571248292923 11.0
87.514570997 0.324617266654968 12.0
87.514570997 1.01470530033112 13.0
87.514570997 0.00769606418907642 14.0
0 0.133406832814217 15.0
0 0.211158066987991 16.0
0 0.820601582527161 17.0
67.832266459 1.31361663341522 18.0
-67.832266459 1.65058267116547 19.0
-67.832266459 0.922731041908264 20.0
-67.832266459 -0.0800326764583588 21.0
87.734278584 1.33573186397552 22.0
143.75971518 1.13532638549805 23.0
110.11938533 0.677571594715118 24.0
13.749873946 0.0884086191654205 25.0
231.49399376 0.57706093788147 26.0
231.49399376 1.94570207595825 27.0
26.611859082 2.67456960678101 28.0
258.10585285 1.71492779254913 29.0
258.10585285 2.06374621391296 30.0
46.107475153 0.997657179832458 31.0
9.1950515851 0.515791535377502 32.0
19.902012125 1.77152609825134 33.0
19.902012125 1.19277226924896 34.0
19.902012125 0.14659708738327 35.0
9.444235798 0.0395999699831009 36.0
8.0987148655 0.407576978206635 37.0
0 0.00431523472070694 38.0
0 0.12911219894886 39.0
-11.952146742 0.914715766906738 40.0
-25.845248898 0.264747351408005 41.0
4.5914424019 0.492821007966995 42.0
-33.357879101 0.421675384044647 43.0
103.46640595 1.04116821289062 44.0
103.46640595 0.0414653941988945 45.0
6.4425737082 0.518132627010345 46.0
2.2767471451 0.288420915603638 47.0
205.55374133 0.153998732566833 48.0
198.93114489 0.236820936203003 49.0
2.9535595117 0.552895426750183 50.0
82.748117402 0.165839925408363 51.0
96.717729055 0.342742919921875 52.0
87.514570997 0.0253401957452297 53.0
87.514570997 0.0966826230287552 54.0
87.514570997 0.32918056845665 55.0
0 1.60165667533875 56.0
87.734278584 0.15624675154686 57.0
231.49399376 0.259792774915695 58.0
258.10585285 1.00350844860077 59.0
19.902012125 -0.0136350058019161 60.0
0 0.104029394686222 61.0
131.51631006 0.972471117973328 0.0
131.51631006 2.27806782722473 1.0
130.16050561 1.81661796569824 2.0
125.37140037 3.80059742927551 3.0
0 9.94368839263916 4.0
0 5.41748046875 5.0
-52.959980337 3.54368615150452 6.0
65.103490799 0.82939475774765 7.0
66.987921504 0.614375472068787 8.0
-1.8844307053 0.796977043151855 9.0
92.609464233 1.13806772232056 10.0
81.097383267 0.59762579202652 11.0
75.458073581 0.652961671352386 12.0
75.458073581 0.659829616546631 13.0
75.458073581 0.0655377730727196 14.0
0 0.100820504128933 15.0
0 0.155760556459427 16.0
0 1.08940923213959 17.0
57.907758467 0.786636590957642 18.0
-57.907758467 0.678508758544922 19.0
-57.907758467 0.54109388589859 20.0
-57.907758467 0.0150196179747581 21.0
79.230688706 2.69818615913391 22.0
134.78193314 2.52501249313354 23.0
105.15204748 0.250785022974014 24.0
25.995061249 0.0912260711193085 25.0
214.01262185 1.63708913326263 26.0
214.01262185 1.62577652931213 27.0
21.557828467 2.05219578742981 28.0
235.57045032 1.41939663887024 29.0
235.57045032 2.15317440032959 30.0
46.720123196 1.15101027488708 31.0
9.3484088364 0.595070660114288 32.0
21.322930239 1.69475924968719 33.0
21.322930239 0.961715996265411 34.0
21.322930239 0.126875773072243 35.0
9.7769056756 0.0416458398103714 36.0
8.6145044563 0.292556017637253 37.0
0 0.0722504556179047 38.0
0 0.134600698947906 39.0
-12.164574614 0.649082005023956 40.0
-25.891603799 0.699630439281464 41.0
5.0467208451 0.336758941411972 42.0
-33.03965251 0.620454370975494 43.0
91.804817254 0.594118475914001 44.0
91.804817254 0.095123864710331 45.0
7.7705116274 0.581817746162415 46.0
2.6378095275 0.440045863389969 47.0
131.51631006 0.10670530796051 48.0
125.37140037 0.0865665376186371 49.0
0 0.739861249923706 50.0
65.103490799 0.0695579871535301 51.0
92.609464233 0.348962306976318 52.0
75.458073581 0.0463664121925831 53.0
75.458073581 0.0895042717456818 54.0
75.458073581 0.339274048805237 55.0
0 6.8391170501709 56.0
79.230688706 0.0747132822871208 57.0
214.01262185 0.442241668701172 58.0
235.57045032 0.940972745418549 59.0
21.322930239 -0.0205496810376644 60.0
0 0.0814211666584015 61.0
213.73414265 0.716765403747559 0.0
213.73414265 1.5194833278656 1.0
212.34614246 2.11141967773438 2.0
206.88147174 3.44147777557373 3.0
13.992260217 3.86752557754517 4.0
13.992260217 3.37649369239807 5.0
-60.095387616 0.452031493186951 6.0
69.626616649 0.420350164175034 7.0
45.705624702 0.924821138381958 8.0
23.920991948 1.09842109680176 9.0
71.955244778 0.763766527175903 10.0
58.27753525 0.401137232780457 11.0
58.046888372 0.255504250526428 12.0
58.046888372 0.356959521770477 13.0
58.046888372 0.0596298463642597 14.0
26.303015044 0.217933028936386 15.0
26.303015044 0.118281528353691 16.0
0 1.42709839344025 17.0
57.341311034 0.950267136096954 18.0
-57.341311034 0.763904750347137 19.0
-57.341311034 0.61798369884491 20.0
-57.341311034 0.0254749171435833 21.0
77.490715432 1.44577193260193 22.0
153.03677616 1.36914277076721 23.0
115.07712339 0.330089628696442 24.0
22.623082928 0.228198900818825 25.0
230.5274916 0.897609174251556 26.0
230.5274916 1.75785112380981 27.0
21.931113466 2.08212375640869 28.0
252.45860506 0.968222558498383 29.0
252.45860506 1.96829915046692 30.0
47.024352324 1.04434263706207 31.0
9.3809535092 0.362718671560287 32.0
20.149404398 1.32449686527252 33.0
20.149404398 0.941496014595032 34.0
20.149404398 0.120180152356625 35.0
8.501639726 0.0237142778933048 36.0
7.3571148903 0.416395306587219 37.0
0 0.0605416484177113 38.0
0 0.0650568455457687 39.0
-12.558926173 0.58366996049881 40.0
-24.628954098 0.474856644868851 41.0
4.7989937537 0.498637467622757 42.0
-32.052974629 0.606308162212372 43.0
86.008420086 0.674239873886108 44.0
86.008420086 0.0961870104074478 45.0
6.37627291 0.515899121761322 46.0
2.2262265017 0.374828577041626 47.0
213.73414265 0.0105080185458064 48.0
206.88147174 0.313928097486496 49.0
13.992260217 0.145023092627525 50.0
69.626616649 0.249672204256058 51.0
71.955244778 0.403792321681976 52.0
58.046888372 0.0195101983845234 53.0
58.046888372 0.0662086606025696 54.0
58.046888372 0.173362150788307 55.0
0 1.4741712808609 56.0
77.490715432 0.0883204191923141 57.0
230.5274916 0.219038397073746 58.0
252.45860506 0.889186263084412 59.0
20.149404398 0.0149388685822487 60.0
0 0.086146742105484 61.0
181.54108195 0.648681879043579 0.0
181.54108195 1.34495222568512 1.0
180.61766042 1.32911884784698 2.0
178.2457227 3.40842437744141 3.0
0 7.01718473434448 4.0
0 3.43171525001526 5.0
-84.871822721 0.188446551561356 6.0
96.655020759 0.832747459411621 7.0
49.734956822 1.71302378177643 8.0
46.920063936 1.63011646270752 9.0
61.931156338 0.647929728031158 10.0
49.099127946 0.295512050390244 11.0
47.880811211 0.343344151973724 12.0
47.880811211 0.332628965377808 13.0
47.880811211 0.0386841334402561 14.0
33.024345954 0.208492994308472 15.0
33.024345954 0.167649239301682 16.0
0 0.994302451610565 17.0
54.269210979 0.667755484580994 18.0
-54.269210979 0.833570241928101 19.0
-55.032811181 0.484145104885101 20.0
-55.10678289 -0.083210676908493 21.0
76.377174991 1.41021966934204 22.0
142.56450223 0.742747843265533 23.0
110.31999126 0.430578798055649 24.0
20.804801219 0.391839921474457 25.0
218.94167722 0.889513194561005 26.0
218.94167722 1.27699303627014 27.0
25.528765747 1.04862058162689 28.0
244.47044297 0.43886861205101 29.0
244.47044297 1.24554443359375 30.0
46.631351303 0.95978194475174 31.0
9.4569627896 0.356364339590073 32.0
21.270392101 1.45777976512909 33.0
21.270392101 0.847825765609741 34.0
21.270392101 0.12144061923027 35.0
8.6969542686 0.0248724911361933 36.0
7.5545738002 0.296832323074341 37.0
0 0.0107448771595955 38.0
0 0.104016534984112 39.0
-11.311451563 1.21604895591736 40.0
-23.817011302 0.346602112054825 41.0
5.2750760907 0.711237251758575 42.0
-31.471484548 0.704731464385986 43.0
87.317358567 0.610683023929596 44.0
87.317358567 0.0686808377504349 45.0
6.0526567901 0.612630903720856 46.0
1.6824588834 0.44447922706604 47.0
181.54108195 0.190583258867264 48.0
178.2457227 0.107883296906948 49.0
0 0.648294270038605 50.0
96.655020759 0.150493413209915 51.0
61.931156338 0.391960710287094 52.0
47.880811211 0.0221473835408688 53.0
47.880811211 0.0556365251541138 54.0
47.880811211 0.0589659661054611 55.0
0 0.274462431669235 56.0
76.377174991 0.194783359766006 57.0
218.94167722 0.373217076063156 58.0
244.47044297 0.0298228561878204 59.0
21.270392101 -0.000205181539058685 60.0
0 0.114756748080254 61.0
150.84793787 0.95849609375 0.0
150.84793787 1.04773783683777 1.0
149.34483161 0.621400654315948 2.0
144.55572647 3.21311831474304 3.0
0 5.33908414840698 4.0
0 3.88870453834534 5.0
-64.548284449 0.451034307479858 6.0
75.840176935 0.611863195896149 7.0
53.03204179 1.60469436645508 8.0
22.808135145 1.7536940574646 9.0
80.593789481 1.15904188156128 10.0
69.694515646 0.11867842823267 11.0
73.73453194 0.255199790000916 12.0
73.73453194 0.325239360332489 13.0
73.73453194 0.110126614570618 14.0
0 0.138705968856812 15.0
0 0.205170810222626 16.0
0 1.03770244121552 17.0
58.756133507 1.11885273456573 18.0
-58.756133507 1.16706693172455 19.0
-58.756133507 0.281039714813232 20.0
-58.756133507 0.510937631130219 21.0
81.439798906 1.41921722888947 22.0
132.42379779 1.43475103378296 23.0
98.861655873 0.338620364665985 24.0
21.369503764 0.185656875371933 25.0
213.8635967 1.06665849685669 26.0
213.8635967 1.51930785179138 27.0
22.367125158 1.73351514339447 28.0
236.23072186 0.900348722934723 29.0
236.23072186 2.43407845497131 30.0
45.936591452 1.35296452045441 31.0
9.3548687047 0.695847570896149 32.0
22.683665399 0.890358090400696 33.0
22.683665399 1.16502451896667 34.0
22.683665399 0.110793769359589 35.0
9.1393339082 0.041003480553627 36.0
7.8760310359 0.357164770364761 37.0
0 0.0175417512655258 38.0
0 0.320958018302917 39.0
-11.952503202 0.717121481895447 40.0
-25.346370893 0.531672358512878 41.0
5.3851451653 0.664246916770935 42.0
-33.676442064 0.413084208965302 43.0
97.934213229 0.79822039604187 44.0
97.934213229 0.145852118730545 45.0
7.2252676557 0.68408066034317 46.0
2.8157826842 0.406925201416016 47.0
150.84793787 0.163461834192276 48.0
144.55572647 0.0153550039976835 49.0
0 0.179590284824371 50.0
75.840176935 0.173560008406639 51.0
80.593789481 0.155646741390228 52.0
73.73453194 0.0835927501320839 53.0
73.73453194 0.148169204592705 54.0
73.73453194 0.0451902560889721 55.0
0 5.45166206359863 56.0
81.439798906 -0.0254999846220016 57.0
213.8635967 0.155677556991577 58.0
236.23072186 0.781635165214539 59.0
22.683665399 0.0285823158919811 60.0
0 0.0718695372343063 61.0
175.03868236 1.28042769432068 0.0
175.03868236 0.573209881782532 1.0
173.75695871 0.863357186317444 2.0
168.29228801 5.31539392471313 3.0
3.7242144716 6.7122220993042 4.0
3.7242144716 4.42378330230713 5.0
-69.493860806 0.408629149198532 6.0
81.290575539 0.500812590122223 7.0
39.987496965 0.888455867767334 8.0
41.303078574 0.847782492637634 9.0
81.004669893 1.20458519458771 10.0
67.99544406 0.0177624002099037 11.0
81.179414414 0.894338726997375 12.0
81.179414414 0.444291025400162 13.0
81.179414414 0.0923582091927528 14.0
0 0.132720395922661 15.0
0 0.192321836948395 16.0
0 1.36954939365387 17.0
50.485126379 1.1206978559494 18.0
-50.485126379 1.11617207527161 19.0
-50.485126379 0.673361301422119 20.0
-50.485126379 -0.0678516775369644 21.0
70.490303231 1.5671284198761 22.0
138.95946809 0.713700294494629 23.0
108.47351469 0.85395747423172 24.0
17.144658839 0.0705118626356125 25.0
209.44977132 0.624466359615326 26.0
209.44977132 1.70011913776398 27.0
30.505244918 2.41364598274231 28.0
239.95501624 1.62898993492126 29.0
239.95501624 2.05126428604126 30.0
45.987897615 1.02645075321198 31.0
9.2056399339 0.409269541501999 32.0
20.005176852 1.57929587364197 33.0
20.005176852 0.966335535049438 34.0
20.005176852 0.0637095645070076 35.0
8.6600915844 0.0203526895493269 36.0
7.5858455983 0.432327389717102 37.0
0 0.027655016630888 38.0
0 0.150860726833344 39.0
-11.784038567 0.513260364532471 40.0
-25.52557116 1.01993644237518 41.0
4.6197623389 0.387934148311615 42.0
-32.9557986 0.400896966457367 43.0
91.891392037 0.891483664512634 44.0
91.891392037 0.128886610269547 45.0
6.755838127 0.822543740272522 46.0
2.2769711106 0.394812703132629 47.0
175.03868236 0.148099035024643 48.0
168.29228801 0.149834871292114 49.0
3.7242144716 0.259655624628067 50.0
81.290575539 0.23066633939743 51.0
81.004669893 0.0208445973694324 52.0
81.179414414 0.0119242733344436 53.0
81.179414414 0.22206974029541 54.0
81.179414414 0.0966554507613182 55.0
0 4.09066581726074 56.0
70.490303231 0.16000509262085 57.0
209.44977132 0.135462045669556 58.0
239.95501624 1.13475024700165 59.0
20.005176852 0.028347909450531 60.0
0 0.0976415425539017 61.0
153.79444329 0.900936007499695 0.0
153.79444329 0.693982779979706 1.0
152.70181459 1.84685182571411 2.0
147.9127093 3.76733922958374 3.0
0 5.52072048187256 4.0
0 6.25816822052002 5.0
-57.328166673 0.759138464927673 6.0
66.382409468 0.734878897666931 7.0
43.320285871 0.890909731388092 8.0
23.062123597 0.818728029727936 9.0
96.199867498 0.970693051815033 10.0
82.578818711 0.261388689279556 11.0
80.989558599 0.198059946298599 12.0
80.989558599 0.633349478244781 13.0
80.989558599 0.0772267654538155 14.0
0 0.161666363477707 15.0
0 0.156903952360153 16.0
0 1.06449961662292 17.0
55.913728811 1.850705742836 18.0
-55.913728811 2.21274709701538 19.0
-55.913728811 0.191514521837234 20.0
-55.913728811 0.0181458070874214 21.0
76.983852732 1.39756894111633 22.0
147.80435896 0.628717362880707 23.0
109.17367484 0.837895452976227 24.0
22.790789672 0.0965299755334854 25.0
224.78821169 0.815800070762634 26.0
224.78821169 1.51258957386017 27.0
23.134368835 1.39552426338196 28.0
247.92258053 0.714313566684723 29.0
247.92258053 1.69611728191376 30.0
46.224429595 1.01026999950409 31.0
9.2303841584 0.495188742876053 32.0
21.070123921 1.4479386806488 33.0
21.070123921 0.803364872932434 34.0
21.070123921 0.0864724665880203 35.0
9.3386140898 0.0213455706834793 36.0
8.0746573963 0.397305876016617 37.0
0 0.0495899356901646 38.0
0 0.148936629295349 39.0
-10.826422597 1.00935065746307 40.0
-23.736831341 0.439013361930847 41.0
4.7539746653 0.490762948989868 42.0
-31.057244828 0.431022495031357 43.0
84.44550114 0.780839204788208 44.0
84.44550114 0.0891837626695633 45.0
6.750723469 0.532156527042389 46.0
2.3879574473 0.30449452996254 47.0
153.79444329 0.0958215072751045 48.0
147.9127093 0.555813193321228 49.0
0 0.339088290929794 50.0
66.382409468 0.155474856495857 51.0
96.199867498 0.138837352395058 52.0
80.989558599 0.0715387463569641 53.0
80.989558599 0.17553336918354 54.0
80.989558599 0.0534268468618393 55.0
0 3.27157139778137 56.0
76.983852732 0.0440438538789749 57.0
224.78821169 0.355054646730423 58.0
247.92258053 0.691790401935577 59.0
21.070123921 -0.00259530358016491 60.0
0 0.0810278207063675 61.0
135.79302475 0.736331284046173 0.0
135.79302475 1.98282659053802 1.0
134.6830194 2.23289036750793 2.0
129.89391411 3.36699795722961 3.0
0 4.9158182144165 4.0
0 4.34664964675903 5.0
-69.792166611 0.411317110061646 6.0
78.648761607 1.39449465274811 7.0
63.674516954 0.4242262840271 8.0
14.974244654 0.408301413059235 9.0
93.763983582 0.88914155960083 10.0
80.547887359 0.314029783010483 11.0
82.461772551 0.750853896141052 12.0
82.461772551 0.362423956394196 13.0
82.461772551 0.0602912977337837 14.0
0 0.199825435876846 15.0
0 0.1573805809021 16.0
0 1.09179866313934 17.0
60.434704456 1.56661176681519 18.0
-60.434704456 2.25060701370239 19.0
-60.434704456 0.260759174823761 20.0
-60.434704456 0.0151870809495449 21.0
83.259390929 1.39905607700348 22.0
138.07855642 1.60486197471619 23.0
103.63984138 0.513525247573853 24.0
21.24852851 0.152483806014061 25.0
221.33794735 1.2290518283844 26.0
221.33794735 2.1386251449585 27.0
28.89849955 1.94962930679321 28.0
250.2364469 1.04619801044464 29.0
250.2364469 1.88471496105194 30.0
47.300649217 1.14790117740631 31.0
9.4641781894 0.3913254737854 32.0
22.824686473 1.42522835731506 33.0
22.824686473 0.981374502182007 34.0
22.824686473 0.133279800415039 35.0
10.328177597 0.0389675572514534 36.0
9.1865862064 0.413984000682831 37.0
0 0.0491133593022823 38.0
0 0.203249871730804 39.0
-10.691771358 0.413899004459381 40.0
-24.153967369 0.735659837722778 41.0
5.4651650132 0.374468326568604 42.0
-32.124025796 0.606842994689941 43.0
89.032091995 0.622619807720184 44.0
89.032091995 0.0981169492006302 45.0
6.734939768 0.475564777851105 46.0
2.7497774124 0.351554363965988 47.0
135.79302475 0.112482860684395 48.0
129.89391411 0.476892173290253 49.0
0 0.213568106293678 50.0
78.648761607 0.225390821695328 51.0
93.763983582 0.323768138885498 52.0
82.461772551 0.0453956201672554 53.0
82.461772551 0.151654571294785 54.0
82.461772551 0.0828992277383804 55.0
0 2.42108273506165 56.0
83.259390929 0.102443486452103 57.0
221.33794735 0.867318093776703 58.0
250.2364469 0.442908257246017 59.0
22.824686473 -0.00864014029502869 60.0
0 0.0724048539996147 61.0
212.40221691 1.2499932050705 0.0
212.40221691 0.806941628456116 1.0
211.32612528 0.915704786777496 2.0
206.53705533 5.84765911102295 3.0
0 10.9776782989502 4.0
0 4.21693468093872 5.0
-58.567902943 4.88471174240112 6.0
67.326843536 0.486687123775482 7.0
52.031727702 0.689019680023193 8.0
15.295115834 0.871861577033997 9.0
77.918864729 1.21355450153351 10.0
64.916005863 0.111762523651123 11.0
76.692054563 1.21419656276703 12.0
76.692054563 0.7842817902565 13.0
76.692054563 0.0232406947761774 14.0
0 0.00982673466205597 15.0
0 0.205470234155655 16.0
0 0.750458061695099 17.0
52.483180916 1.0627418756485 18.0
-52.483180916 1.21285629272461 19.0
-52.483180916 0.703308284282684 20.0
-52.483180916 -0.040522363036871 21.0
72.847012496 2.89284372329712 22.0
148.07337034 1.87971448898315 23.0
116.67542219 0.167066976428032 24.0
20.800366418 0.63647335767746 25.0
220.92038283 1.8199028968811 26.0
220.92038283 1.89320063591003 27.0
25.724093116 1.879434466362 28.0
246.64447595 0.891323208808899 29.0
246.64447595 1.72327554225922 30.0
46.305243579 1.09515833854675 31.0
9.1476898309 0.297691851854324 32.0
20.36383158 1.399418592453 33.0
20.36383158 1.14207994937897 34.0
20.36383158 0.149138167500496 35.0
9.2147449645 0.0430670008063316 36.0
7.8591465739 0.417148500680923 37.0
0 0.0197821259498596 38.0
0 0.186139345169067 39.0
-12.989825702 0.26950865983963 40.0
-25.046957575 0.91602635383606 41.0
4.8939308315 0.270669519901276 42.0
-32.165570849 0.631866097450256 43.0
90.432400473 0.634531080722809 44.0
90.432400473 0.104519538581371 45.0
6.0243860308 0.473727464675903 46.0
2.7167578452 0.318688303232193 47.0
212.40221691 0.0884707495570183 48.0
206.53705533 0.15140700340271 49.0
0 0.718832790851593 50.0
67.326843536 0.207557424902916 51.0
77.918864729 0.206609696149826 52.0
76.692054563 0.120475597679615 53.0
76.692054563 0.0970123633742332 54.0
76.692054563 0.286541312932968 55.0
0 5.86717653274536 56.0
72.847012496 -0.0225193090736866 57.0
220.92038283 0.49699655175209 58.0
246.64447595 0.783746421337128 59.0
20.36383158 -0.0312303975224495 60.0
0 0.0948054641485214 61.0
170.41548941 0.735308945178986 0.0
170.41548941 1.02635884284973 1.0
168.99692145 1.38349568843842 2.0
163.53225071 4.75192308425903 3.0
5.8515514041 8.36015892028809 4.0
5.8515514041 4.76777648925781 5.0
-69.413716442 1.71351253986359 6.0
80.753028959 0.780767679214478 7.0
47.915558485 0.782079219818115 8.0
32.837470474 0.906887471675873 9.0
100.288099 0.923222482204437 10.0
86.415895214 0.556784093379974 11.0
85.095233367 0.898817777633667 12.0
85.095233367 0.697752237319946 13.0
85.095233367 0.0366153307259083 14.0
0 0.151369497179985 15.0
0 0.221744999289513 16.0
0 1.20915687084198 17.0
52.865316117 0.949003338813782 18.0
-52.865316117 0.485321789979935 19.0
-52.865316117 0.883510410785675 20.0
-52.865316117 -0.0498809739947319 21.0
73.705532591 1.28895890712738 22.0
144.0722346 1.27500712871552 23.0
109.90313803 0.44358766078949 24.0
20.618057058 0.104561164975166 25.0
217.77776719 1.38292419910431 26.0
217.77776719 1.5799400806427 27.0
25.846618213 1.88035774230957 28.0
243.6243854 0.666867613792419 29.0
243.6243854 1.75904154777527 30.0
47.211204539 1.09902036190033 31.0
9.3656838119 0.540342450141907 32.0
20.840216474 1.5842570066452 33.0
20.840216474 0.853544592857361 34.0
20.840216474 0.109251521527767 35.0
9.2117472995 0.0370771177113056 36.0
7.941450213 0.357161313295364 37.0
0 0.0207180008292198 38.0
0 0.108802244067192 39.0
-11.556200275 0.706679940223694 40.0
-23.509835792 0.536169230937958 41.0
4.7287816915 0.389454782009125 42.0
-30.712501049 0.382966279983521 43.0
82.019909884 0.650322496891022 44.0
82.019909884 0.085733637213707 45.0
6.5470353846 0.653522193431854 46.0
2.6549123953 0.402924031019211 47.0
170.41548941 0.118210762739182 48.0
163.53225071 0.221999228000641 49.0
5.8515514041 0.423573732376099 50.0
80.753028959 0.234817624092102 51.0
100.288099 0.385029733181 52.0
85.095233367 0.153723314404488 53.0
85.095233367 0.13516104221344 54.0
85.095233367 0.224090233445168 55.0
0 3.01409840583801 56.0
73.705532591 0.19233812391758 57.0
217.77776719 0.695382475852966 58.0
243.6243854 0.3941730260849 59.0
20.840216474 -0.00383767671883106 60.0
0 0.0867666304111481 61.0
138.88949076 0.764767050743103 0.0
138.88949076 1.67738282680511 1.0
137.3036866 1.97413730621338 2.0
132.51458134 5.06622743606567 3.0
0 7.90588283538818 4.0
0 3.88117837905884 5.0
-55.04903721 1.22259104251862 6.0
68.834561901 1.28595197200775 7.0
60.265189164 0.601294636726379 8.0
8.5693727376 0.871151566505432 9.0
90.787736447 0.857428252696991 10.0
76.821686369 0.326180458068848 11.0
82.669244216 0.885235726833344 12.0
82.669244216 0.705343127250671 13.0
82.669244216 0.0557050593197346 14.0
0 0.26945436000824 15.0
0 0.188205286860466 16.0
0 1.76195526123047 17.0
50.130612991 0.967366099357605 18.0
-50.130612991 0.294740349054337 19.0
-50.130612991 1.35892724990845 20.0
-50.130612991 -0.0536944605410099 21.0
72.066172075 1.55661797523499 22.0
132.53980148 1.98839473724365 23.0
102.17803623 0.172981217503548 24.0
25.786179319 0.118947707116604 25.0
204.60597355 1.0487562417984 26.0
204.60597355 1.5451180934906 27.0
23.229616059 1.89953911304474 28.0
227.83558961 1.12947952747345 29.0
227.83558961 2.11408543586731 30.0
46.318125332 1.16470694541931 31.0
9.2803890897 0.552125632762909 32.0
21.935559084 2.06290030479431 33.0
21.935559084 0.875463128089905 34.0
21.935559084 0.123024761676788 35.0
9.2300789769 0.0407742261886597 36.0
7.9931560584 0.451735019683838 37.0
0 0.0141362845897675 38.0
0 0.2072983533144 39.0
-11.190084723 0.699554741382599 40.0
-25.147984739 0.549871623516083 41.0
5.3700692872 0.666977405548096 42.0
-33.305644622 0.926372408866882 43.0
92.719519616 0.498037993907928 44.0
92.719519616 0.078934520483017 45.0
6.0940007396 0.511198163032532 46.0
2.400791835 0.377121180295944 47.0
138.88949076 0.124570846557617 48.0
132.51458134 0.206055030226707 49.0
0 0.915018439292908 50.0
68.834561901 0.195005521178246 51.0
90.787736447 0.000499233603477478 52.0
82.669244216 0.112369388341904 53.0
82.669244216 0.0671800971031189 54.0
82.669244216 0.261429220438004 55.0
0 0.73395699262619 56.0
72.066172075 0.0878981500864029 57.0
204.60597355 0.347695142030716 58.0
227.83558961 0.884596168994904 59.0
21.935559084 0.038924690335989 60.0
0 0.0888963639736176 61.0
195.56142797 0.882277250289917 0.0
195.56142797 1.34837448596954 1.0
194.40744428 1.47230076789856 2.0
188.94277363 5.26169204711914 3.0
6.9786360283 6.74861574172974 4.0
6.9786360283 6.27402973175049 5.0
-71.801529577 7.36725330352783 6.0
83.311722151 0.575565755367279 7.0
53.019300648 0.615915238857269 8.0
30.292421503 0.816143035888672 9.0
88.991457946 0.935623466968536 10.0
76.109275561 0.0892801284790039 11.0
84.734794128 0.6266930103302 12.0
84.734794128 0.196704894304276 13.0
84.734794128 0.119382947683334 14.0
2.4481295822 0.413878381252289 15.0
2.4481295822 0.134641602635384 16.0
0 1.8155345916748 17.0
64.592611099 0.930388331413269 18.0
-64.592611099 0.684388041496277 19.0
-64.592611099 0.598212718963623 20.0
-64.592611099 0.0166925899684429 21.0
88.270044556 2.01402735710144 22.0
142.78885642 2.14487338066101 23.0
113.23725341 0.619858145713806 24.0
23.169772597 0.133989915251732 25.0
231.05890097 1.88608825206757 26.0
231.05890097 1.61641705036163 27.0
27.047148444 2.01069736480713 28.0
258.10604942 0.676984429359436 29.0
258.10604942 1.44470369815826 30.0
46.363576746 1.2388858795166 31.0
9.4297482709 0.398237586021423 32.0
23.677433457 1.49040389060974 33.0
23.677433457 0.95698481798172 34.0
23.677433457 0.0894492864608765 35.0
10.22411942 0.0301143378019333 36.0
8.8360720474 0.339503675699234 37.0
0 0.0101909302175045 38.0
0 0.276100993156433 39.0
-11.989707509 0.61468642950058 40.0
-24.679587512 0.418206483125687 41.0
5.3342553019 0.493418037891388 42.0
-32.685446472 0.771724283695221 43.0
101.56385708 0.503242373466492 44.0
101.56385708 0.0976650789380074 45.0
6.5645491542 0.507381618022919 46.0
2.6736878824 0.36241015791893 47.0
195.56142797 0.114014610648155 48.0
188.94277363 0.236550688743591 49.0
6.9786360283 0.498487651348114 50.0
83.311722151 0.237194210290909 51.0
88.991457946 0.0786779075860977 52.0
84.734794128 0.0727171823382378 53.0
84.734794128 0.124545492231846 54.0
84.734794128 0.112820737063885 55.0
0 0.864163219928741 56.0
88.270044556 0.121477775275707 57.0
231.05890097 0.766783893108368 58.0
258.10604942 0.0590970553457737 59.0
23.677433457 0.00596725568175316 60.0
0 0.103321440517902 61.0
161.32095466 0.95427930355072 0.0
161.32095466 1.59816157817841 1.0
160.17358178 2.07482218742371 2.0
154.70891109 5.33038139343262 3.0
2.2100889033 5.74018049240112 4.0
2.2100889033 4.7681713104248 5.0
-60.892708514 0.610802352428436 6.0
72.373572322 0.258345574140549 7.0
42.932236918 0.872215032577515 8.0
29.441335403 1.06480312347412 9.0
93.350616037 1.03017163276672 10.0
80.851772663 0.224049255251884 11.0
80.007269827 0.562401831150055 12.0
80.007269827 0.633283078670502 13.0
80.007269827 0.0186095517128706 14.0
0 0.00559646915644407 15.0
0 0.23189751803875 16.0
0 0.793949007987976 17.0
57.129735173 1.42591822147369 18.0
-57.129735173 1.83097076416016 19.0
-57.129735173 0.29780986905098 20.0
-57.129735173 0.0358702056109905 21.0
78.204319412 1.88874673843384 22.0
142.59154525 1.62623274326324 23.0
106.7572482 0.389773815870285 24.0
19.772400637 0.185904055833817 25.0
220.79586466 0.682532787322998 26.0
220.79586466 1.70044183731079 27.0
25.538859846 1.25168919563293 28.0
246.3347245 0.613466620445251 29.0
246.3347245 1.5036518573761 30.0
46.805540583 0.976130068302155 31.0
9.4214703107 0.427414804697037 32.0
21.074584239 1.29374217987061 33.0
21.074584239 1.11468636989594 34.0
21.074584239 0.0828547030687332 35.0
9.9851327834 0.0208495315164328 36.0
8.7858463415 0.448349565267563 37.0
0 0.048467829823494 38.0
0 0.189941436052322 39.0
-11.462185799 0.381065219640732 40.0
-24.718076738 0.861702799797058 41.0
4.8828755628 0.350633710622787 42.0
-32.31848492 0.414008796215057 43.0
100.33855704 0.627919554710388 44.0
100.33855704 0.137817114591599 45.0
6.3377194046 0.621678054332733 46.0
2.3417998499 0.409503072500229 47.0
161.32095466 0.158174693584442 48.0
154.70891109 0.197875514626503 49.0
2.2100889033 0.553482174873352 50.0
72.373572322 0.0740498900413513 51.0
93.350616037 0.1177728921175 52.0
80.007269827 0.120395235717297 53.0
80.007269827 0.127156347036362 54.0
80.007269827 0.266192764043808 55.0
0 3.93335056304932 56.0
78.204319412 0.0616625770926476 57.0
220.79586466 0.287728369235992 58.0
246.3347245 0.364997088909149 59.0
21.074584239 0.00930929929018021 60.0
0 0.0799846053123474 61.0
126.68128716 1.04836475849152 0.0
126.68128716 1.81409811973572 1.0
125.16201234 2.49940204620361 2.0
120.37290703 5.7380690574646 3.0
0 6.45740079879761 4.0
0 3.35044622421265 5.0
-76.811283844 -0.00468048453330994 6.0
87.125020362 0.702763497829437 7.0
59.00824986 0.948050260543823 8.0
28.116770502 0.892619729042053 9.0
85.522644486 1.08124756813049 10.0
72.569286989 0.0788071602582932 11.0
76.714701575 0.586871266365051 12.0
76.714701575 0.508751392364502 13.0
76.714701575 0.0625627040863037 14.0
0 0.0337870270013809 15.0
0 0.193980827927589 16.0
0 1.34752869606018 17.0
59.578505673 1.61649286746979 18.0
-59.578505673 0.996399581432343 19.0
-59.578505673 1.07386779785156 20.0
-59.578505673 -0.0593948401510715 21.0
80.763713013 1.88990473747253 22.0
143.44029231 1.40628039836884 23.0
109.8258654 0.60156112909317 24.0
23.472370371 0.0798478275537491 25.0
224.20400532 0.311233580112457 26.0
224.20400532 1.80193638801575 27.0
26.901605459 2.50177526473999 28.0
251.10561078 0.758814692497253 29.0
251.10561078 1.81349325180054 30.0
47.204151779 1.16282618045807 31.0
9.28010623 0.716469347476959 32.0
21.18520734 1.47682869434357 33.0
21.18520734 0.947286903858185 34.0
21.18520734 0.125156506896019 35.0
8.6579126197 0.0456397086381912 36.0
7.4465067225 0.412790685892105 37.0
0 0.0268787406384945 38.0
0 0.0697527229785919 39.0
-11.426601878 0.471257627010345 40.0
-24.764639918 0.736723661422729 41.0
4.5299265436 0.46785631775856 42.0
-31.62500974 0.497241258621216 43.0
75.287780085 0.753168642520905 44.0
75.287780085 0.0910759493708611 45.0
6.0500615188 0.674245953559875 46.0
2.3212776028 0.439350992441177 47.0
126.68128716 0.056415531784296 48.0
120.37290703 0.499837934970856 49.0
0 0.276998192071915 50.0
87.125020362 0.281837642192841 51.0
85.522644486 0.131199240684509 52.0
76.714701575 0.0751495882868767 53.0
76.714701575 0.0636875033378601 54.0
76.714701575 0.107220150530338 55.0
0 3.39599561691284 56.0
80.763713013 0.0946084260940552 57.0
224.20400532 0.170638844370842 58.0
251.10561078 0.0944257453083992 59.0
21.18520734 -0.0230058412998915 60.0
0 0.100872404873371 61.0
167.60423927 1.01754653453827 0.0
167.60423927 0.402144223451614 1.0
166.18000705 0.948628306388855 2.0
160.71533634 5.71584415435791 3.0
15.739675618 7.64720392227173 4.0
15.739675618 4.22936296463013 5.0
-53.941704205 1.42901957035065 6.0
65.668402435 0.804495692253113 7.0
39.180502977 0.354521155357361 8.0
26.487899458 0.397319585084915 9.0
80.284260146 1.23098814487457 10.0
68.085716306 0.262368410825729 11.0
71.60118367 0.133185639977455 12.0
71.60118367 0.514866769313812 13.0
71.60118367 0.110975205898285 14.0
0 0.062167901545763 15.0
0 0.222983509302139 16.0
0 0.945860147476196 17.0
48.586119659 1.07041704654694 18.0
-48.586119659 1.34964561462402 19.0
-48.586119659 0.251691043376923 20.0
-48.586119659 0.0621148198843002 21.0
71.591832926 1.46631574630737 22.0
154.45233944 1.1149570941925 23.0
118.76696731 0.7900151014328 24.0
22.159099846 0.106931328773499 25.0
226.04417236 1.3587554693222 26.0
226.04417236 1.92802441120148 27.0
29.404859806 1.7279132604599 28.0
255.44903217 0.266227811574936 29.0
255.44903217 1.72704875469208 30.0
46.526973554 1.04155313968658 31.0
9.2482431102 0.483981788158417 32.0
23.005713266 1.6457062959671 33.0
23.005713266 0.769952654838562 34.0
23.005713266 0.104295656085014 35.0
9.9491357046 0.0210492759943008 36.0
8.6483981134 0.42710092663765 37.0
0 0.0873632282018661 38.0
0 0.155515521764755 39.0
-10.697952212 0.931737720966339 40.0
-23.087533729 0.382114052772522 41.0
4.576893052 0.513096570968628 42.0
-30.271156916 0.650017857551575 43.0
82.426233053 0.652481317520142 44.0
82.426233053 0.0851472392678261 45.0
7.1879424942 0.714383244514465 46.0
2.2079451955 0.424108922481537 47.0
167.60423927 0.0745128095149994 48.0
160.71533634 0.223793879151344 49.0
15.739675618 0.752408921718597 50.0
65.668402435 0.173510134220123 51.0
80.284260146 0.483217865228653 52.0
71.60118367 0.0399913415312767 53.0
71.60118367 0.0395391844213009 54.0
71.60118367 0.339407235383987 55.0
0 7.8823447227478 56.0
71.591832926 0.0146608110517263 57.0
226.04417236 0.452055543661118 58.0
255.44903217 0.416236817836761 59.0
23.005713266 -0.0188860166817904 60.0
0 0.0800934955477715 61.0
186.060442 1.29556918144226 0.0
186.060442 1.08000493049622 1.0
185.12015456 1.53765952587128 2.0
180.19968333 2.58400988578796 3.0
0 5.15838432312012 4.0
0 6.04805660247803 5.0
-54.930189798 5.04269409179688 6.0
66.714508386 0.78540825843811 7.0
43.459256517 0.657708525657654 8.0
23.255251869 0.620933890342712 9.0
74.832412626 1.2849338054657 10.0
62.330385663 0.168597355484962 11.0
72.924849084 0.823290646076202 12.0
72.924849084 0.539342820644379 13.0
72.924849084 0.147531136870384 14.0
0 0.236134469509125 15.0
0 0.134537190198898 16.0
0 1.48422336578369 17.0
61.176254522 1.86138367652893 18.0
-61.176254522 2.36809587478638 19.0
-61.176254522 0.330234110355377 20.0
-61.176254522 -0.0110022351145744 21.0
84.713715286 3.52960062026978 22.0
137.08907192 1.80940091609955 23.0
100.8925565 0.977704524993896 24.0
10.040596393 0.098330482840538 25.0
221.80278721 2.90122699737549 26.0
221.80278721 3.65075302124023 27.0
26.027191657 3.28261280059814 28.0
247.82997886 1.34266948699951 29.0
247.82997886 1.71643352508545 30.0
47.008105568 1.08584463596344 31.0
9.2271603004 0.483178377151489 32.0
23.537460764 2.31558227539062 33.0
23.537460764 1.48732364177704 34.0
23.537460764 0.130877509713173 35.0
10.185355204 0.0300844367593527 36.0
8.6955592914 0.359425842761993 37.0
0 0.00107087939977646 38.0
0 0.0542868487536907 39.0
-10.330287348 0.60741513967514 40.0
-21.569655553 0.392319083213806 41.0
5.8124746173 0.267899245023727 42.0
-30.319726941 0.359760463237762 43.0
71.326444668 0.911054134368896 44.0
71.326444668 0.0956019312143326 45.0
6.482061837 0.791121423244476 46.0
2.4977324901 0.438835561275482 47.0
186.060442 0.0441991128027439 48.0
180.19968333 0.0332497134804726 49.0
0 0.846857786178589 50.0
66.714508386 0.364326238632202 51.0
74.832412626 0.197770774364471 52.0
72.924849084 0.0435439050197601 53.0
72.924849084 0.0413183383643627 54.0
72.924849084 0.0347488150000572 55.0
0 5.58611631393433 56.0
84.713715286 0.0815139412879944 57.0
221.80278721 1.50740814208984 58.0
247.82997886 0.302000313997269 59.0
23.537460764 -0.0218476988375187 60.0
0 0.114524304866791 61.0
143.60164308 0.898443222045898 0.0
143.60164308 0.588836669921875 1.0
142.46212932 1.36952292919159 2.0
136.99745859 5.20952129364014 3.0
12.669558759 5.44013166427612 4.0
12.669558759 6.41914701461792 5.0
-56.41594283 5.59757375717163 6.0
70.023498098 0.513554275035858 7.0
43.871022803 1.18863868713379 8.0
26.152475295 1.30675172805786 9.0
70.969110344 0.994486570358276 10.0
58.358881586 0.0442294590175152 11.0
75.21723331 0.867319107055664 12.0
75.21723331 0.367950826883316 13.0
75.21723331 0.18176406621933 14.0
0 0.115194238722324 15.0
0 0.200445488095284 16.0
0 1.24398589134216 17.0
55.289713588 0.996777534484863 18.0
-55.289713588 0.604595899581909 19.0
-55.289713588 0.895862579345703 20.0
-55.289713588 -0.0654098019003868 21.0
75.850666528 1.94511044025421 22.0
150.51523349 1.53825509548187 23.0
114.69222111 0.276519775390625 24.0
26.051553436 0.355439126491547 25.0
226.36590001 1.61708319187164 26.0
226.36590001 1.72124683856964 27.0
23.617056258 1.83838284015656 28.0
249.98295627 0.90664130449295 29.0
249.98295627 1.53386723995209 30.0
46.013439987 1.01906979084015 31.0
9.1981238917 0.496062576770782 32.0
20.560952939 1.78375506401062 33.0
20.560952939 1.01764965057373 34.0
20.560952939 0.11370937526226 35.0
9.3694427819 0.0345391258597374 36.0
8.0842908116 0.378102540969849 37.0
0 0.0233819186687469 38.0
0 0.153458029031754 39.0
-12.110906829 0.771996080875397 40.0
-23.326421276 0.493800789117813 41.0
4.9863031374 0.388133347034454 42.0
-30.719909857 0.227113291621208 43.0
87.523278081 0.81134819984436 44.0
87.523278081 0.0759233087301254 45.0
6.1931760395 0.700756013393402 46.0
1.2353173828 0.394327193498611 47.0
143.60164308 0.0279634222388268 48.0
136.99745859 0.253217548131943 49.0
12.669558759 0.987422168254852 50.0
70.023498098 0.198300093412399 51.0
70.969110344 0.22003535926342 52.0
75.21723331 0.0667786300182343 53.0
75.21723331 0.176337793469429 54.0
75.21723331 0.0979244112968445 55.0
0 7.28204154968262 56.0
75.850666528 0.185136809945107 57.0
226.36590001 0.828026413917542 58.0
249.98295627 0.28660860657692 59.0
20.560952939 -0.0108980126678944 60.0
0 0.107308588922024 61.0
131.47300151 0.719732344150543 0.0
131.47300151 0.816398918628693 1.0
130.49969616 1.25471127033234 2.0
125.71059109 3.96680855751038 3.0
0 3.6194052696228 4.0
0 5.27027654647827 5.0
-72.675944622 0.590486407279968 6.0
81.006034619 0.825615346431732 7.0
48.899579201 1.34355306625366 8.0
32.106455417 1.30622625350952 9.0
86.446462997 0.883335947990417 10.0
73.366050304 0.284874826669693 11.0
77.337675585 0.712847888469696 12.0
77.337675585 0.0752561092376709 13.0
77.337675585 0.118533827364445 14.0
0 0.130982711911201 15.0
0 0.160022765398026 16.0
0 0.849345326423645 17.0
56.149808624 1.17187082767487 18.0
-56.149808624 0.809367656707764 19.0
-56.149808624 1.01849699020386 20.0
-56.149808624 0.0569290593266487 21.0
76.179467392 1.77839410305023 22.0
145.79082302 1.72377288341522 23.0
108.09904829 0.245455265045166 24.0
21.699905038 0.318255633115768 25.0
221.97029041 1.62901425361633 26.0
221.97029041 1.41485261917114 27.0
22.63624724 1.54849004745483 28.0
244.60653765 0.464406192302704 29.0
244.60653765 1.63888931274414 30.0
46.62667419 1.11420094966888 31.0
9.4660992178 0.472482442855835 32.0
20.029658769 1.07965791225433 33.0
20.029658769 1.36878895759583 34.0
20.029658769 0.0950415804982185 35.0
8.7216238438 0.0242409911006689 36.0
7.3572175697 0.357698380947113 37.0
0 0.0076494924724102 38.0
0 0.0746038779616356 39.0
-10.71052662 0.485272228717804 40.0
-24.004826975 0.458067655563354 41.0
4.8251023572 0.331570744514465 42.0
-31.457035419 0.616452515125275 43.0
107.45647055 0.656760334968567 44.0
107.45647055 0.117628440260887 45.0
6.1608164867 0.482710659503937 46.0
2.5080128127 0.363681465387344 47.0
131.47300151 0.0855394750833511 48.0
125.71059109 0.383862882852554 49.0
0 0.447068899869919 50.0
81.006034619 0.275593847036362 51.0
86.446462997 0.12547442317009 52.0
77.337675585 0.0383791849017143 53.0
77.337675585 0.152868986129761 54.0
77.337675585 0.0732644721865654 55.0
0 6.77419328689575 56.0
76.179467392 0.00143088353797793 57.0
221.97029041 0.768574297428131 58.0
244.60653765 0.511202335357666 59.0
20.029658769 -0.015783753246069 60.0
0 0.103535763919353 61.0
174.65036534 0.964578807353973 0.0
174.65036534 1.8608546257019 1.0
173.36304793 1.60888993740082 2.0
168.57397797 2.92036771774292 3.0
0 3.94988441467285 4.0
0 4.86557292938232 5.0
-66.679017384 0.59646064043045 6.0
79.480933257 1.23763585090637 7.0
44.448917143 0.474162369966507 8.0
35.032016114 0.323959618806839 9.0
88.116416915 1.1011209487915 10.0
75.0198731 0.0497950650751591 11.0
83.736284893 0.747706294059753 12.0
83.736284893 0.16852793097496 13.0
83.736284893 0.0812644362449646 14.0
0 0.165791258215904 15.0
0 0.190528690814972 16.0
0 1.38418531417847 17.0
55.435437614 1.19517636299133 18.0
-55.435437614 0.757297515869141 19.0
-55.435437614 0.593742787837982 20.0
-55.435437614 0.0427449233829975 21.0
76.14308076 1.60183453559875 22.0
155.94278626 1.35580730438232 23.0
119.24151141 0.361016571521759 24.0
24.288198892 0.18300786614418 25.0
232.08586702 1.26244211196899 26.0
232.08586702 2.0988564491272 27.0
25.029977245 2.32009291648865 28.0
257.11584427 1.19352293014526 29.0
257.11584427 1.79480063915253 30.0
47.188648354 1.08178305625916 31.0
9.4754901609 0.395085513591766 32.0
20.707643145 1.12220394611359 33.0
20.707643145 1.00998449325562 34.0
20.707643145 0.135958194732666 35.0
9.0844035594 0.0304800029844046 36.0
7.7209752465 0.381997168064117 37.0
0 0.0587851293385029 38.0
0 0.0527559332549572 39.0
-12.518535056 0.683025717735291 40.0
-24.690767004 0.465523481369019 41.0
5.5554798221 0.418373763561249 42.0
-33.00649667 0.637614965438843 43.0
93.442700378 0.568645417690277 44.0
93.442700378 0.125892117619514 45.0
7.0095585341 0.465372264385223 46.0
2.7965661876 0.374897301197052 47.0
174.65036534 0.169136002659798 48.0
168.57397797 0.0489592105150223 49.0
0 0.165310233831406 50.0
79.480933257 0.0878178328275681 51.0
88.116416915 0.133967712521553 52.0
83.736284893 0.0207256115972996 53.0
83.736284893 0.0317587479948997 54.0
83.736284893 0.0980503782629967 55.0
0 2.8053936958313 56.0
76.14308076 -0.0475873127579689 57.0
232.08586702 0.658840894699097 58.0
257.11584427 0.755779325962067 59.0
20.707643145 -0.00129170902073383 60.0
0 0.087800569832325 61.0
138.71142171 0.888124763965607 0.0
138.71142171 2.19465279579163 1.0
137.20321838 1.96929121017456 2.0
132.41411318 3.44704914093018 3.0
0 3.35772776603699 4.0
0 3.4983913898468 5.0
-73.956600873 0.109339788556099 6.0
83.055112417 0.6934974193573 7.0
66.674189034 0.579296767711639 8.0
16.380923383 0.780222654342651 9.0
79.69945449 0.861272275447845 10.0
67.934790711 0.0402486696839333 11.0
78.902566348 0.230684787034988 12.0
78.902566348 0.827412664890289 13.0
78.902566348 0.0157837495207787 14.0
0 -0.0200798250734806 15.0
0 0.22034364938736 16.0
0 1.05708801746368 17.0
58.518734887 1.69107842445374 18.0
-58.518734887 1.62881219387054 19.0
-58.518734887 0.194002732634544 20.0
-58.518734887 0.102016538381577 21.0
78.680993539 2.30081391334534 22.0
142.0509721 2.07543540000916 23.0
110.99732883 0.328512996435165 24.0
22.21331113 0.110707089304924 25.0
220.73196564 1.04890811443329 26.0
220.73196564 1.33234035968781 27.0
26.518980466 1.49215614795685 28.0
247.25094611 0.578960061073303 29.0
247.25094611 1.7591450214386 30.0
46.23566186 1.05112957954407 31.0
9.5021819208 0.658403337001801 32.0
20.162258652 0.924151241779327 33.0
20.162258652 1.73802185058594 34.0
20.162258652 0.118053659796715 35.0
9.5526135833 0.0417151860892773 36.0
8.3870233644 0.383200228214264 37.0
0 0.0142385177314281 38.0
0 0.178302556276321 39.0
-12.870050168 0.772208392620087 40.0
-26.303759998 0.741757750511169 41.0
5.6307834574 0.485143542289734 42.0
-34.563056237 0.739459097385406 43.0
97.894844129 0.728745460510254 44.0
97.894844129 0.0714453831315041 45.0
6.7748454629 0.537826120853424 46.0
2.8753748416 0.364467471837997 47.0
138.71142171 0.243822425603867 48.0
132.41411318 0.280412167310715 49.0
0 0.289818942546844 50.0
83.055112417 0.107403784990311 51.0
79.69945449 0.152214348316193 52.0
78.902566348 0.0162996556609869 53.0
78.902566348 0.225265637040138 54.0
78.902566348 0.118079446256161 55.0
0 3.46333503723145 56.0
78.680993539 0.0207887701690197 57.0
220.73196564 0.21914878487587 58.0
247.25094611 0.812558829784393 59.0
20.162258652 -0.0228212457150221 60.0
0 0.0829594284296036 61.0
134.11316248 0.735683560371399 0.0
134.11316248 1.88822972774506 1.0
132.65978682 2.20488834381104 2.0
127.87070327 6.15601921081543 3.0
0 7.52577018737793 4.0
0 4.70826482772827 5.0
-82.474234523 0.842522263526917 6.0
94.455105937 1.31264805793762 7.0
64.163522813 1.91199958324432 8.0
30.291583124 1.84342849254608 9.0
91.237517845 0.941950857639313 10.0
78.810255729 0.246789082884789 11.0
90.938911954 0.800918459892273 12.0
90.938911954 0.176959455013275 13.0
90.938911954 0.101424172520638 14.0
0 0.274987190961838 15.0
0 0.20220585167408 16.0
0 1.08557164669037 17.0
44.94447977 1.34733772277832 18.0
-44.94447977 1.25222766399384 19.0
-44.94447977 0.999703645706177 20.0
-44.94447977 -0.0421523861587048 21.0
66.450409365 1.97001779079437 22.0
156.06325408 1.48939311504364 23.0
117.4919213 0.198379069566727 24.0
21.745384932 0.452633768320084 25.0
222.51366344 0.781805038452148 26.0
222.51366344 1.7414585351944 27.0
23.209842068 1.97119176387787 28.0
245.72350551 0.587616443634033 29.0
245.72350551 1.67868328094482 30.0
46.159124527 1.32731461524963 31.0
9.1936892297 0.466991454362869 32.0
21.505929595 1.58237946033478 33.0
21.505929595 1.89210438728333 34.0
21.505929595 0.130655318498611 35.0
10.567652562 0.0493952929973602 36.0
9.2146140669 0.371574074029922 37.0
0 -0.0187117420136929 38.0
0 0.263112097978592 39.0
-13.894432954 0.716256618499756 40.0
-28.030955934 0.638879597187042 41.0
4.3959428333 0.45158851146698 42.0
-35.339547548 0.396271497011185 43.0
89.127606666 0.679218590259552 44.0
89.127606666 0.119974315166473 45.0
7.4366831083 0.61080002784729 46.0
2.8486304635 0.445277333259583 47.0
134.11316248 0.144595175981522 48.0
127.87070327 0.547180950641632 49.0
0 0.393359154462814 50.0
94.455105937 0.389675348997116 51.0
91.237517845 0.375151246786118 52.0
90.938911954 0.0731087848544121 53.0
90.938911954 0.0367153659462929 54.0
90.938911954 0.0700367465615273 55.0
0 1.06496393680573 56.0
66.450409365 0.0269680581986904 57.0
222.51366344 0.474334448575974 58.0
245.72350551 0.25378155708313 59.0
21.505929595 0.0292919967323542 60.0
0 0.0652605667710304 61.0
157.06387008 0.966103732585907 0.0
157.06387008 0.386257648468018 1.0
155.86086677 0.941824913024902 2.0
151.07176155 3.19553971290588 3.0
0 3.68849968910217 4.0
0 2.97680377960205 5.0
-55.453525013 1.2977899312973 6.0
65.72077209 0.436570942401886 7.0
53.602998522 0.885306835174561 8.0
12.117773569 0.987761557102203 9.0
81.619390857 1.208092212677 10.0
68.845132809 0.203707069158554 11.0
76.566977903 0.530915021896362 12.0
76.566977903 0.189728245139122 13.0
76.566977903 0.177343621850014 14.0
0 0.140364393591881 15.0
0 0.191318690776825 16.0
0 1.87153363227844 17.0
54.780299692 2.06446480751038 18.0
-54.780299692 1.82983255386353 19.0
-54.780299692 0.387064307928085 20.0
-54.780299692 -0.0675554797053337 21.0
73.762605549 1.39846229553223 22.0
139.36049549 1.14543759822845 23.0
106.79965381 0.675645053386688 24.0
19.050758478 0.0639908835291862 25.0
213.12310104 1.03133988380432 26.0
213.12310104 1.8761990070343 27.0
25.658149788 2.25978207588196 28.0
238.78125083 0.854027807712555 29.0
238.78125083 1.77832520008087 30.0
46.386792751 1.10874438285828 31.0
9.4803627149 0.54757171869278 32.0
18.982305857 1.46952950954437 33.0
18.982305857 1.32097434997559 34.0
18.982305857 0.105761215090752 35.0
9.1924362874 0.039205513894558 36.0
8.0244520574 0.367843449115753 37.0
0 0.00518803298473358 38.0
0 0.0966385304927826 39.0
-11.594199692 0.642274916172028 40.0
-23.959038832 0.628957629203796 41.0
5.5561420081 0.607221126556396 42.0
-31.878375481 0.843056321144104 43.0
97.803900841 0.646385908126831 44.0
97.803900841 0.0980690121650696 45.0
7.4552010572 0.565638065338135 46.0
2.4596567083 0.426998317241669 47.0
157.06387008 0.0106092169880867 48.0
151.07176155 0.208062022924423 49.0
0 0.307370781898499 50.0
65.72077209 0.0624536350369453 51.0
81.619390857 0.602153718471527 52.0
76.566977903 0.0540511049330235 53.0
76.566977903 0.0277925375849009 54.0
76.566977903 0.200543358922005 55.0
0 7.1138858795166 56.0
73.762605549 0.203173369169235 57.0
213.12310104 0.481400668621063 58.0
238.78125083 0.191226899623871 59.0
18.982305857 0.0133093930780888 60.0
0 0.110249347984791 61.0
194.77521648 1.36420404911041 0.0
194.77521648 0.661883592605591 1.0
193.69106694 1.75527191162109 2.0
190.47005092 5.01358032226562 3.0
0 9.84722709655762 4.0
0 3.09538316726685 5.0
-60.977535821 1.16241466999054 6.0
73.815588259 1.69619297981262 7.0
60.765801087 0.368724822998047 8.0
13.049787172 0.407964169979095 9.0
85.792303707 1.41064310073853 10.0
73.08406895 0.118597723543644 11.0
78.238232719 0.817174017429352 12.0
78.238232719 0.842373967170715 13.0
78.238232719 0.0577647760510445 14.0
0 0.021920382976532 15.0
0 0.230043947696686 16.0
0 1.0955525636673 17.0
56.247236885 1.35157477855682 18.0
-56.247236885 0.962109565734863 19.0
-56.247236885 0.341693729162216 20.0
-56.247236885 -0.00930379331111908 21.0
77.628230144 2.45842981338501 22.0
146.82364831 1.88706862926483 23.0
114.94858818 0.504900574684143 24.0
16.213029089 0.101122736930847 25.0
224.45187845 1.57518827915192 26.0
224.45187845 2.23809313774109 27.0
27.997627597 1.72566413879395 28.0
252.44950605 0.711164832115173 29.0
252.44950605 1.59868764877319 30.0
47.243304216 1.03777468204498 31.0
9.3247837379 0.556661903858185 32.0
21.380993259 1.52596950531006 33.0
21.380993259 1.41246962547302 34.0
21.380993259 0.131600439548492 35.0
9.6562917229 0.0269759763032198 36.0
8.4813785452 0.458643853664398 37.0
0 0.0345201082527637 38.0
0 0.14443451166153 39.0
-11.559026749 1.0631799697876 40.0
-24.917133485 0.677971720695496 41.0
5.237990289 0.462754309177399 42.0
-32.63047123 0.423940479755402 43.0
91.190758866 0.928771317005157 44.0
91.190758866 0.087470680475235 45.0
6.8107357647 0.795455396175385 46.0
2.768084473 0.392301380634308 47.0
194.77521648 0.165020257234573 48.0
190.47005092 0.438262969255447 49.0
0 0.268439173698425 50.0
73.815588259 0.0670554786920547 51.0
85.792303707 0.0863779336214066 52.0
78.238232719 0.109062142670155 53.0
78.238232719 0.157479047775269 54.0
78.238232719 0.21564781665802 55.0
0 5.28852224349976 56.0
77.628230144 0.150874316692352 57.0
224.45187845 0.781818091869354 58.0
252.44950605 0.392219334840775 59.0
21.380993259 -0.0186565890908241 60.0
0 0.095920592546463 61.0
160.7323558 0.935526609420776 0.0
160.7323558 1.26019382476807 1.0
159.25023949 1.86534535884857 2.0
154.46113424 3.54419684410095 3.0
0 7.27804374694824 4.0
0 4.38387250900269 5.0
-61.433800055 1.88840126991272 6.0
73.770295978 0.6706463098526 7.0
56.060933302 0.691947281360626 8.0
17.709362676 0.812249302864075 9.0
93.812987141 0.971393525600433 10.0
81.409353675 0.0760999917984009 11.0
86.820498517 0.379279732704163 12.0
86.820498517 0.242316901683807 13.0
86.820498517 0.0811208561062813 14.0
0 0.1671122610569 15.0
0 0.190381824970245 16.0
0 1.03573882579803 17.0
62.983465757 0.987978518009186 18.0
-62.983465757 0.993841409683228 19.0
-62.983465757 0.488170087337494 20.0
-62.983465757 0.00395533442497253 21.0
85.219667714 0.911582231521606 22.0
141.50230135 0.766403615474701 23.0
104.768184 0.53231018781662 24.0
21.766242387 0.0949283093214035 25.0
226.72196906 0.846765637397766 26.0
226.72196906 1.71247565746307 27.0
28.822494583 1.19747090339661 28.0
255.54446364 0.771563351154327 29.0
255.54446364 1.43296098709106 30.0
46.120516872 1.09937381744385 31.0
9.2467006256 0.39178529381752 32.0
22.236201957 1.75754332542419 33.0
22.236201957 1.00380682945251 34.0
22.236201957 0.11315244436264 35.0
9.332378351 0.023502167314291 36.0
8.0900206081 0.406798303127289 37.0
0 0.0142820551991463 38.0
0 0.114394374191761 39.0
-11.234514982 1.02731132507324 40.0
-25.502614856 0.42363429069519 41.0
5.1710206474 0.572777807712555 42.0
-33.047737166 0.713504135608673 43.0
88.13603962 0.569860875606537 44.0
88.13603962 0.101716250181198 45.0
6.5323372556 0.419027805328369 46.0
2.3455492481 0.34635517001152 47.0
160.7323558 0.038371779024601 48.0
154.46113424 0.165333300828934 49.0
0 0.453470021486282 50.0
73.770295978 0.107130087912083 51.0
93.812987141 0.0345919542014599 52.0
86.820498517 0.0384770706295967 53.0
86.820498517 0.117330558598042 54.0
86.820498517 0.0339749716222286 55.0
0 2.98045587539673 56.0
85.219667714 0.0984765142202377 57.0
226.72196906 0.432340383529663 58.0
255.54446364 0.355337023735046 59.0
22.236201957 0.00537171121686697 60.0
0 0.105774879455566 61.0
128.17445815 0.769881188869476 0.0
128.17445815 0.81488835811615 1.0
127.03323101 0.598811328411102 2.0
122.24412575 3.68090295791626 3.0
0 8.85958957672119 4.0
0 6.92450952529907 5.0
-61.958628048 0.668789803981781 6.0
73.044785757 0.813955783843994 7.0
59.222137058 0.813607037067413 8.0
13.822648699 0.838652729988098 9.0
89.448836934 0.779959559440613 10.0
76.823147362 0.196748211979866 11.0
84.543069832 0.993895769119263 12.0
84.543069832 0.528376996517181 13.0
84.543069832 0.0305971708148718 14.0
0 0.133156850934029 15.0
0 0.220760688185692 16.0
0 1.20254898071289 17.0
56.637700669 0.833987891674042 18.0
-56.637700669 0.81494152545929 19.0
-56.637700669 1.11998808383942 20.0
-56.637700669 0.0159116797149181 21.0
76.431494039 1.63747620582581 22.0
141.95179398 1.08349871635437 23.0
106.81958156 0.549422442913055 24.0
21.830193058 0.119924664497375 25.0
218.38328802 1.23579514026642 26.0
218.38328802 1.80135357379913 27.0
24.405286314 1.70118081569672 28.0
242.78857433 0.754915535449982 29.0
242.78857433 1.74452936649323 30.0
47.299452072 1.12350451946259 31.0
9.239207542 0.613495647907257 32.0
19.793793369 1.17599856853485 33.0
19.793793369 0.899795055389404 34.0
19.793793369 0.107059895992279 35.0
8.5831718436 0.0278260558843613 36.0
7.2159556796 0.378053486347198 37.0
0 0.0259980335831642 38.0
0 0.181798115372658 39.0
-11.109563371 0.674526572227478 40.0
-23.406524519 0.751087605953217 41.0
4.7524350679 0.500829875469208 42.0
-30.77376228 0.755326151847839 43.0
84.693089201 0.588057219982147 44.0
84.693089201 0.103697501122952 45.0
6.0508171976 0.587904810905457 46.0
1.0790617721 0.430189192295074 47.0
128.17445815 0.495535671710968 48.0
122.24412575 0.0477575547993183 49.0
0 0.535209834575653 50.0
73.044785757 0.160027593374252 51.0
89.448836934 0.39934229850769 52.0
84.543069832 0.0647291839122772 53.0
84.543069832 0.282675296068192 54.0
84.543069832 0.191049709916115 55.0
0 3.58364176750183 56.0
76.431494039 -0.00199270434677601 57.0
218.38328802 0.201211303472519 58.0
242.78857433 0.362272560596466 59.0
19.793793369 0.0284903198480606 60.0
0 0.0987988114356995 61.0
170.52193119 0.752055883407593 0.0
170.52193119 0.978945136070251 1.0
169.46125414 0.91872102022171 2.0
164.67217059 4.72503757476807 3.0
0 7.73516273498535 4.0
0 3.76178884506226 5.0
-56.329336357 1.24170398712158 6.0
66.282442522 0.60160231590271 7.0
42.956334877 0.894980072975159 8.0
23.326107645 0.8069167137146 9.0
82.355735406 0.870902121067047 10.0
70.75134467 0.145061612129211 11.0
80.295509344 0.428598135709763 12.0
80.295509344 0.528189361095428 13.0
80.295509344 0.133871078491211 14.0
0 0.157673835754395 15.0
0 0.139561980962753 16.0
0 1.5367990732193 17.0
57.347042118 1.27913844585419 18.0
-57.347042118 0.95626300573349 19.0
-57.347042118 0.583953320980072 20.0
-57.347042118 0.0258134715259075 21.0
79.81000192 1.96810257434845 22.0
141.97099414 1.59814429283142 23.0
110.08067386 0.445310264825821 24.0
25.420138534 0.189668208360672 25.0
221.78099606 1.12877249717712 26.0
221.78099606 2.06435394287109 27.0
27.019145966 2.52818512916565 28.0
248.80014203 1.12523174285889 29.0
248.80014203 2.02194857597351 30.0
46.936434678 1.13243496417999 31.0
9.4663089553 0.590764462947845 32.0
22.462959802 1.42892253398895 33.0
22.462959802 0.872210085391998 34.0
22.462959802 0.116586588323116 35.0
9.9148696514 0.0402644313871861 36.0
8.7326804563 0.33586785197258 37.0
0 0.02510105073452 38.0
0 0.165422081947327 39.0
-12.609439097 0.626289188861847 40.0
-25.651380706 0.618711471557617 41.0
5.4022462135 0.413366198539734 42.0
-33.390507453 0.542815208435059 43.0
90.692556825 0.638897180557251 44.0
90.692556825 0.0806038677692413 45.0
6.8581566058 0.575757205486298 46.0
2.5016695486 0.36170020699501 47.0
170.52193119 0.087245911359787 48.0
164.67217059 0.0524817034602165 49.0
0 0.402119845151901 50.0
66.282442522 0.168648585677147 51.0
82.355735406 0.179557144641876 52.0
80.295509344 0.0480086095631123 53.0
80.295509344 0.113852508366108 54.0
80.295509344 0.143898442387581 55.0
0 5.63694763183594 56.0
79.81000192 0.00286369118839502 57.0
221.78099606 0.373565852642059 58.0
248.80014203 0.472299486398697 59.0
22.462959802 -0.00329559110105038 60.0
0 0.0974060297012329 61.0
184.12337027 0.704884886741638 0.0
184.12337027 1.34296548366547 1.0
182.62077442 1.26392006874084 2.0
177.8316693 4.93905401229858 3.0
0 5.09075164794922 4.0
0 2.48998069763184 5.0
-54.030635948 0.184971943497658 6.0
64.759672621 1.39987254142761 7.0
64.438570682 0.580062627792358 8.0
0.32110193809 0.682188749313354 9.0
90.068567713 0.848805129528046 10.0
77.158368032 0.383686929941177 11.0
88.104330228 0.583741247653961 12.0
88.104330228 0.678058743476868 13.0
88.104330228 0.100890718400478 14.0
0 0.254388689994812 15.0
0 0.177191913127899 16.0
0 1.07369554042816 17.0
62.130800766 1.14925754070282 18.0
-62.130800766 1.22135853767395 19.0
-62.130800766 0.329938113689423 20.0
-62.130800766 -0.028817193582654 21.0
83.962391563 1.11305594444275 22.0
137.22026214 0.714196026325226 23.0
103.23073055 0.644584774971008 24.0
22.458380092 0.14150258898735 25.0
221.1826537 0.802840113639832 26.0
221.1826537 2.13582110404968 27.0
29.627068716 2.20697784423828 28.0
250.80972242 1.16023766994476 29.0
250.80972242 1.72039926052094 30.0
46.7589209 0.991787731647491 31.0
9.3560381993 0.400832921266556 32.0
21.831590797 1.54926156997681 33.0
21.831590797 0.903191745281219 34.0
21.831590797 0.135678097605705 35.0
9.5302243809 0.0245998110622168 36.0
8.2730574962 0.343805342912674 37.0
0 0.0368010327219963 38.0
0 0.145757257938385 39.0
-11.745827088 0.913735389709473 40.0
-25.397834136 0.334754467010498 41.0
4.8413300676 0.563492238521576 42.0
-32.977178862 0.637685775756836 43.0
93.241109918 0.624478995800018 44.0
93.241109918 0.0860269740223885 45.0
7.3171912572 0.560351192951202 46.0
2.4024643522 0.39272677898407 47.0
184.12337027 0.133597373962402 48.0
177.8316693 0.0288035981357098 49.0
0 0.0877744257450104 50.0
64.759672621 0.109070427715778 51.0
90.068567713 0.328368425369263 52.0
88.104330228 0.0533998571336269 53.0
88.104330228 0.254334092140198 54.0
88.104330228 0.0296121425926685 55.0
0 1.39838349819183 56.0
83.962391563 0.054668553173542 57.0
221.1826537 0.288652151823044 58.0
250.80972242 0.657981336116791 59.0
21.831590797 -0.0168797448277473 60.0
0 0.117957964539528 61.0
145.38500316 1.09373354911804 0.0
145.38500316 0.695472538471222 1.0
144.02158606 1.0369873046875 2.0
139.23248083 4.40646457672119 3.0
0 3.96889472007751 4.0
0 6.43852853775024 5.0
-66.844834415 4.82787847518921 6.0
76.210141714 0.60230278968811 7.0
37.936700351 1.30649387836456 8.0
38.273441363 1.11409378051758 9.0
87.524734174 1.16882026195526 10.0
74.124204568 0.273544669151306 11.0
82.567132253 0.309658348560333 12.0
82.567132253 0.360295414924622 13.0
82.567132253 0.117595285177231 14.0
0 0.125048488378525 15.0
0 0.18345545232296 16.0
0 1.02933669090271 17.0
57.466154279 1.23113942146301 18.0
-57.466154279 1.35660946369171 19.0
-57.466154279 0.53617924451828 20.0
-57.466154279 -0.0417554043233395 21.0
79.776827696 1.67565393447876 22.0
147.49214044 1.60539257526398 23.0
110.34123559 0.263023376464844 24.0
25.069329422 0.217635482549667 25.0
227.26896813 0.963995575904846 26.0
227.26896813 1.2695631980896 27.0
26.335016439 1.14802575111389 28.0
253.60398457 0.515885531902313 29.0
253.60398457 1.52714240550995 30.0
47.143059046 1.01944947242737 31.0
9.2445758426 0.397059917449951 32.0
22.310673417 1.34478425979614 33.0
22.310673417 0.635384619235992 34.0
22.310673417 0.0824702456593513 35.0
10.247027967 0.0226084310561419 36.0
9.0310406786 0.415587604045868 37.0
0 0.0420675687491894 38.0
0 0.250518381595612 39.0
-11.698212545 0.521759867668152 40.0
-24.521965306 0.683087825775146 41.0
5.0957344095 0.365091502666473 42.0
-32.221525522 0.450934022665024 43.0
93.535874132 0.804555416107178 44.0
93.535874132 0.108458243310452 45.0
6.590405439 0.557744264602661 46.0
2.5846725263 0.320042192935944 47.0
145.38500316 0.180449962615967 48.0
139.23248083 0.180956572294235 49.0
0 0.371540784835815 50.0
76.210141714 0.270880281925201 51.0
87.524734174 0.185900941491127 52.0
82.567132253 0.0823176652193069 53.0
82.567132253 0.0994924530386925 54.0
82.567132253 0.167124778032303 55.0
0 6.09248828887939 56.0
79.776827696 0.177939131855965 57.0
227.26896813 0.124327436089516 58.0
253.60398457 0.473625034093857 59.0
22.310673417 -0.00203347578644753 60.0
0 0.112479604780674 61.0
202.85123174 0.885942041873932 0.0
202.85123174 1.93361401557922 1.0
201.8424909 2.3498203754425 2.0
197.05340735 6.53042507171631 3.0
0 6.12844848632812 4.0
0 4.17030000686646 5.0
-69.740305693 1.13893663883209 6.0
79.250104155 1.49825859069824 7.0
69.791933671 0.183101087808609 8.0
9.4581704844 0.367509067058563 9.0
91.356163386 1.11975264549255 10.0
77.784054321 0.458590626716614 11.0
76.441453106 0.660891771316528 12.0
76.441453106 0.845836460590363 13.0
76.441453106 -0.0329162590205669 14.0
0 0.0555808991193771 15.0
0 0.210276424884796 16.0
0 1.147500872612 17.0
48.294093329 1.73551571369171 18.0
-48.294093329 1.70325326919556 19.0
-48.294093329 0.26811671257019 20.0
-48.294093329 0.100858822464943 21.0
68.5284755 1.62915110588074 22.0
133.84026423 0.913841009140015 23.0
100.69062387 0.82448798418045 24.0
10.485492434 0.0898899510502815 25.0
202.36873973 1.56982231140137 26.0
202.36873973 1.43843746185303 27.0
26.643646969 1.47711062431335 28.0
229.0123867 0.808963358402252 29.0
229.0123867 2.03876781463623 30.0
46.117789559 1.09433162212372 31.0
9.259032604 0.520574569702148 32.0
20.234382171 1.34614944458008 33.0
20.234382171 1.72431635856628 34.0
20.234382171 0.043295331299305 35.0
8.4076304665 0.0261972323060036 36.0
7.1399060388 0.477997839450836 37.0
0 0.0691195204854012 38.0
0 0.509258449077606 39.0
-13.381191363 0.353227287530899 40.0
-27.052934898 0.794469833374023 41.0
5.5767439712 0.400484681129456 42.0
-35.359344395 0.457263946533203 43.0
78.045373006 0.5417640209198 44.0
78.045373006 0.126194134354591 45.0
7.1582553886 0.264830887317657 46.0
2.5000410983 0.330820381641388 47.0
202.85123174 0.263811260461807 48.0
197.05340735 0.246755495667458 49.0
0 0.191517859697342 50.0
79.250104155 0.0287856012582779 51.0
91.356163386 0.521248757839203 52.0
76.441453106 0.0575379095971584 53.0
76.441453106 0.124205105006695 54.0
76.441453106 0.326906770467758 55.0
0 4.80609178543091 56.0
68.5284755 0.0335662290453911 57.0
202.36873973 0.862102091312408 58.0
229.0123867 1.04063951969147 59.0
20.234382171 -0.0240023694932461 60.0
0 0.133715137839317 61.0
155.92654059 1.14929485321045 0.0
155.92654059 1.01846826076508 1.0
154.90281679 2.05400919914246 2.0
150.11371154 6.06507015228271 3.0
0 4.62544059753418 4.0
0 3.90588212013245 5.0
-53.217507828 0.391493529081345 6.0
63.403559874 0.565430223941803 7.0
54.148669416 1.36138486862183 8.0
9.2548904579 1.3101419210434 9.0
88.95512151 1.1117616891861 10.0
76.200109617 0.153907120227814 11.0
76.859318449 0.173972889780998 12.0
76.859318449 0.111546136438847 13.0
76.859318449 0.104562141001225 14.0
0 0.081577442586422 15.0
0 0.202876284718513 16.0
0 1.18235969543457 17.0
57.711583257 1.41272079944611 18.0
-57.711583257 1.61389446258545 19.0
-57.711583257 0.314879089593887 20.0
-57.711583257 -0.0521360598504543 21.0
79.827613944 2.73658728599548 22.0
153.83843995 2.2309877872467 23.0
117.88236333 0.373288959264755 24.0
23.765158656 0.127083837985992 25.0
233.6660539 1.68608283996582 26.0
233.6660539 2.25164270401001 27.0
24.946779224 2.52826118469238 28.0
258.61283312 1.14596056938171 29.0
258.61283312 1.78907859325409 30.0
46.711542067 1.05353093147278 31.0
9.3799637947 0.637744605541229 32.0
22.116030687 1.6299135684967 33.0
22.116030687 0.914623856544495 34.0
22.116030687 0.131030067801476 35.0
9.4970630829 0.0214056875556707 36.0
8.314538722 0.340628117322922 37.0
0 0.0529131032526493 38.0
0 0.139930009841919 39.0
-11.806839709 0.542155086994171 40.0
-24.63563382 0.630261182785034 41.0
5.171258856 0.476373642683029 42.0
-32.56996582 0.703051924705505 43.0
93.942990125 0.611093044281006 44.0
93.942990125 0.110536627471447 45.0
6.4320378889 0.584476411342621 46.0
1.7394833722 0.322333425283432 47.0
155.92654059 0.0994582623243332 48.0
150.11371154 0.211208298802376 49.0
0 0.210548043251038 50.0
63.403559874 0.13571360707283 51.0
88.95512151 0.0868154019117355 52.0
76.859318449 0.0866069793701172 53.0
76.859318449 0.0853802189230919 54.0
76.859318449 0.0563549548387527 55.0
0 4.00222778320312 56.0
79.827613944 0.0990335196256638 57.0
233.6660539 0.763896226882935 58.0
258.61283312 0.560582756996155 59.0
22.116030687 0.018985453993082 60.0
0 0.0803634375333786 61.0
144.47127012 0.786598563194275 0.0
144.47127012 0.830491542816162 1.0
143.4349517 1.09771168231964 2.0
137.97028099 3.4281895160675 3.0
9.3790080667 3.72633481025696 4.0
9.3790080667 3.28432869911194 5.0
-64.926250319 0.120011150836945 6.0
74.316490327 1.0537725687027 7.0
40.391655623 0.597162246704102 8.0
33.924834704 0.677530646324158 9.0
92.658791734 0.903778195381165 10.0
79.87947372 0.152510643005371 11.0
86.295158918 0.906100213527679 12.0
86.295158918 0.463421106338501 13.0
86.295158918 0.0791395008563995 14.0
0 0.180655106902122 15.0
0 0.165626093745232 16.0
0 1.45060980319977 17.0
59.00579515 0.65252822637558 18.0
-59.00579515 0.522420227527618 19.0
-59.00579515 0.632297217845917 20.0
-59.00579515 0.120882734656334 21.0
80.492033221 2.55970907211304 22.0
147.31687088 2.23539996147156 23.0
110.98101846 0.337277472019196 24.0
24.78850129 0.117186762392521 25.0
227.8089041 1.00806665420532 26.0
227.8089041 1.18508911132812 27.0
23.860371716 1.40495538711548 28.0
251.66927581 1.27899169921875 29.0
251.66927581 1.95876610279083 30.0
46.411492226 1.17047190666199 31.0
9.2844054014 0.405275404453278 32.0
21.48623807 0.908675253391266 33.0
21.48623807 1.31420993804932 34.0
21.48623807 0.144417688250542 35.0
9.465203117 0.0272721629589796 36.0
8.0961058758 0.329070121049881 37.0
0 0.0238771960139275 38.0
0 0.175019502639771 39.0
-12.027632108 0.971386194229126 40.0
-25.684653191 0.323810815811157 41.0
5.2564542924 0.640648484230042 42.0
-33.826249703 0.585645258426666 43.0
86.442219937 0.772071480751038 44.0
86.442219937 0.0983662903308868 45.0
8.0089620516 0.681502163410187 46.0
2.8856136414 0.396570414304733 47.0
144.47127012 0.213187292218208 48.0
137.97028099 0.148459032177925 49.0
9.3790080667 0.0831369981169701 50.0
74.316490327 0.153506115078926 51.0
92.658791734 0.533140778541565 52.0
86.295158918 -0.000550851749721915 53.0
86.295158918 0.140837803483009 54.0
86.295158918 0.0374533236026764 55.0
0 0.482477337121964 56.0
80.492033221 0.0432926118373871 57.0
227.8089041 0.0691784918308258 58.0
251.66927581 1.40510714054108 59.0
21.48623807 -0.00551047548651695 60.0
0 0.0789525881409645 61.0
147.93046669 0.547354459762573 0.0
147.93046669 1.10726022720337 1.0
146.34476405 0.714688837528229 2.0
140.88012864 4.6272759437561 3.0
5.2747373398 8.20054149627686 4.0
5.2747373398 6.14330101013184 5.0
-66.558961561 6.22268199920654 6.0
74.94503152 1.34269535541534 7.0
36.183079955 0.600874602794647 8.0
38.761951565 0.298057854175568 9.0
86.726702138 0.94463175535202 10.0
74.492852687 0.376802384853363 11.0
81.172389035 0.177973836660385 12.0
81.172389035 0.960527002811432 13.0
81.172389035 0.0713954046368599 14.0
0 0.116138182580471 15.0
0 0.21668928861618 16.0
0 1.29787683486938 17.0
59.606333313 1.34171080589294 18.0
-59.606333313 1.95429623126984 19.0
-59.606333313 0.278189957141876 20.0
-59.606333313 0.10116508603096 21.0
80.181633078 2.40066647529602 22.0
154.48242211 2.31216549873352 23.0
119.01938905 0.321331560611725 24.0
22.900554214 0.167346477508545 25.0
234.66405519 0.705812811851501 26.0
234.66405519 2.23007607460022 27.0
27.237088168 1.74008655548096 28.0
261.90114335 1.49234294891357 29.0
261.90114335 2.32677221298218 30.0
46.065922389 1.11647391319275 31.0
9.3199929604 0.530386507511139 32.0
20.575299765 1.39227521419525 33.0
20.575299765 1.11624157428741 34.0
20.575299765 0.162241861224174 35.0
8.8934718876 0.036821786314249 36.0
7.5960359148 0.371726125478745 37.0
0 0.0253812968730927 38.0
0 0.176505133509636 39.0
-10.727419299 0.861635029315948 40.0
-23.53388398 0.182182490825653 41.0
5.1304691021 0.368274658918381 42.0
-30.888822679 0.424986779689789 43.0
87.353706371 0.689398527145386 44.0
87.353706371 0.0826994776725769 45.0
7.884494929 0.534206390380859 46.0
2.9532253582 0.441489785909653 47.0
147.93046669 0.163726091384888 48.0
140.88012864 -0.0364263504743576 49.0
5.2747373398 1.1700667142868 50.0
74.94503152 0.24442782998085 51.0
86.726702138 0.668182075023651 52.0
81.172389035 0.026772353798151 53.0
81.172389035 0.0585552677512169 54.0
81.172389035 0.166724234819412 55.0
0 2.0120849609375 56.0
80.181633078 -0.0752245038747787 57.0
234.66405519 0.182569533586502 58.0
261.90114335 1.43928265571594 59.0
20.575299765 -0.0224282648414373 60.0
0 0.129160329699516 61.0
177.03944419 0.764315068721771 0.0
177.03944419 0.763028502464294 1.0
176.10893275 0.906548917293549 2.0
170.64429745 3.00646352767944 3.0
8.2645991435 3.47688484191895 4.0
8.2645991435 3.15027952194214 5.0
-61.01271853 0.107933051884174 6.0
74.078121692 0.795230984687805 7.0
39.24121421 1.26114320755005 8.0
34.836907483 1.34284687042236 9.0
81.842930231 0.803925335407257 10.0
68.794074369 0.0982939451932907 11.0
73.49185523 0.573989748954773 12.0
73.49185523 0.648437201976776 13.0
73.49185523 0.0759177207946777 14.0
5.4426552722 0.223113417625427 15.0
5.4426552722 0.262266576290131 16.0
0 1.22914481163025 17.0
50.393158306 1.56870651245117 18.0
-50.393158306 1.59537959098816 19.0
-50.393158306 0.762351453304291 20.0
-50.393158306 0.0157117694616318 21.0
71.954523806 1.34537577629089 22.0
149.67504035 1.70716595649719 23.0
120.48741789 0.21911858022213 24.0
19.963998027 0.322021216154099 25.0
221.62956416 0.684163451194763 26.0
221.62956416 1.51619505882263 27.0
27.609830848 0.928033351898193 28.0
249.23939501 0.561724305152893 29.0
249.23939501 1.3652628660202 30.0
46.197289182 0.83262687921524 31.0
9.4558054713 0.498792439699173 32.0
21.5613655 1.02184534072876 33.0
21.5613655 1.33563530445099 34.0
21.5613655 0.0925085842609406 35.0
8.6257706404 0.0274424459785223 36.0
7.5180493065 0.406947463750839 37.0
0 0.0195761509239674 38.0
0 0.0901463106274605 39.0
-13.262707265 0.509727358818054 40.0
-24.899432659 0.565067291259766 41.0
5.1682213407 0.347351670265198 42.0
-32.596475646 0.419656276702881 43.0
90.499627785 0.604130327701569 44.0
90.499627785 0.0614293590188026 45.0
6.0461756449 0.355551779270172 46.0
1.099071799 0.310718983411789 47.0
177.03944419 0.144710168242455 48.0
170.64429745 0.057827290147543 49.0
8.2645991435 0.0789452865719795 50.0
74.078121692 0.221714749932289 51.0
81.842930231 0.0966230407357216 52.0
73.49185523 -0.00445941090583801 53.0
73.49185523 0.101158887147903 54.0
73.49185523 0.235042318701744 55.0
0 3.61108565330505 56.0
71.954523806 0.190194189548492 57.0
221.62956416 0.720133066177368 58.0
249.23939501 0.464450240135193 59.0
21.5613655 -0.0264152213931084 60.0
0 0.128790929913521 61.0
231.50967489 1.17307269573212 0.0
231.50967489 0.454520493745804 1.0
229.96922794 0.729912281036377 2.0
224.50455742 4.19512271881104 3.0
28.950399434 4.56202411651611 4.0
28.950399434 4.5510516166687 5.0
-52.557899769 5.90867280960083 6.0
63.130942561 0.303117334842682 7.0
38.927561588 1.33264255523682 8.0
24.203380973 1.68804252147675 9.0
93.945003055 1.20699167251587 10.0
82.289657079 0.503416419029236 11.0
77.057391435 0.59639585018158 12.0
77.057391435 0.187991946935654 13.0
77.057391435 0.141872674226761 14.0
0 0.0753543078899384 15.0
0 0.224922686815262 16.0
0 1.83218836784363 17.0
56.75766505 1.0544159412384 18.0
-56.75766505 0.224430829286575 19.0
-56.75766505 1.29735910892487 20.0
-56.75766505 0.0106231234967709 21.0
77.612285336 1.24923658370972 22.0
148.73741789 0.432306736707687 23.0
116.28173153 0.584697067737579 24.0
20.277262655 0.170991584658623 25.0
226.34970323 0.904309332370758 26.0
226.34970323 0.932719647884369 27.0
30.560293162 1.05621910095215 28.0
256.90999639 0.367313593626022 29.0
256.90999639 1.25526762008667 30.0
47.328079483 1.076131939888 31.0
9.2304486969 0.30228590965271 32.0
20.854620286 1.2567743062973 33.0
20.854620286 1.50787127017975 34.0
20.854620286 0.11888100206852 35.0
8.185865909 0.0279081854969263 36.0
6.7494229848 0.448111563920975 37.0
0 -0.00141526758670807 38.0
0 0.0415617562830448 39.0
-13.000069431 0.840541958808899 40.0
-26.860258454 0.565222978591919 41.0
5.5843811405 0.641249656677246 42.0
-35.422196377 0.718398153781891 43.0
111.1997351 0.946044504642487 44.0
111.1997351 0.0987049490213394 45.0
7.196456908 0.54434061050415 46.0
2.7492581128 0.320081233978271 47.0
231.50967489 0.023655004799366 48.0
224.50455742 0.01837801001966 49.0
28.950399434 0.282471179962158 50.0
63.130942561 0.134064987301826 51.0
93.945003055 0.0486469119787216 52.0
77.057391435 0.12564080953598 53.0
77.057391435 0.131238892674446 54.0
77.057391435 0.0341919884085655 55.0
0 5.56152296066284 56.0
77.612285336 0.102561876177788 57.0
226.34970323 0.113402634859085 58.0
256.90999639 0.146832659840584 59.0
20.854620286 0.0180117730051279 60.0
0 0.108941167593002 61.0
172.28195477 0.797178626060486 0.0
172.28195477 0.726316809654236 1.0
171.09472898 0.980405688285828 2.0
165.63005842 4.7279167175293 3.0
1.6353215826 5.98219203948975 4.0
1.6353215826 4.7252311706543 5.0
-68.613816596 0.92333048582077 6.0
78.077576551 0.63589996099472 7.0
42.026963985 0.750103831291199 8.0
36.050612565 0.538933277130127 9.0
81.753003303 0.914946317672729 10.0
69.663617324 0.0990900993347168 11.0
78.401119981 0.563141703605652 12.0
78.401119981 0.391701936721802 13.0
78.401119981 0.0355137772858143 14.0
0 0.0805162712931633 15.0
0 0.206458315253258 16.0
0 1.12616050243378 17.0
56.030487746 1.36041629314423 18.0
-56.030487746 1.5213714838028 19.0
-56.030487746 0.245541214942932 20.0
-56.030487746 0.212671205401421 21.0
78.168074636 1.26691734790802 22.0
137.249274 1.13915288448334 23.0
104.56893578 0.280509412288666 24.0
24.563492958 0.273993611335754 25.0
215.41734863 1.25246667861938 26.0
215.41734863 1.85923862457275 27.0
29.44768233 2.68520307540894 28.0
244.86503096 1.50415933132172 29.0
244.86503096 1.88517355918884 30.0
46.884482071 0.976297438144684 31.0
9.4060009455 0.473509013652802 32.0
22.137586891 1.0457535982132 33.0
22.137586891 0.741277873516083 34.0
22.137586891 0.0817484483122826 35.0
9.7290350096 0.0272895488888025 36.0
8.4004162084 0.399255037307739 37.0
0 0.0430469885468483 38.0
0 0.192708283662796 39.0
-10.923839521 0.690408825874329 40.0
-23.429371912 0.712290644645691 41.0
4.9580435782 0.330023586750031 42.0
-30.652963358 0.571640431880951 43.0
100.26617439 0.600811362266541 44.0
100.26617439 0.0649794489145279 45.0
6.2320820322 0.542585372924805 46.0
2.3115688322 0.385078251361847 47.0
172.28195477 0.096139557659626 48.0
165.63005842 0.117745198309422 49.0
1.6353215826 0.215463995933533 50.0
78.077576551 0.148064151406288 51.0
81.753003303 0.226885497570038 52.0
78.401119981 0.0169723108410835 53.0
78.401119981 0.117726348340511 54.0
78.401119981 0.00567335076630116 55.0
0 2.64867663383484 56.0
78.168074636 -0.0395596027374268 57.0
215.41734863 0.67682546377182 58.0
244.86503096 1.26383471488953 59.0
22.137586891 0.00896843709051609 60.0
0 0.0787276104092598 61.0
145.56267684 1.16784250736237 0.0
145.56267684 0.580924391746521 1.0
144.16755157 1.38626050949097 2.0
139.3784463 4.06399726867676 3.0
0 5.80206346511841 4.0
0 5.77563333511353 5.0
-59.396384803 3.71552133560181 6.0
71.548139273 0.869879841804504 7.0
49.522455121 0.751047194004059 8.0
22.025684152 0.882158160209656 9.0
90.481502129 1.17981493473053 10.0
76.27268523 0.224702075123787 11.0
79.593043685 0.842637538909912 12.0
79.593043685 0.277426838874817 13.0
79.593043685 0.104356452822685 14.0
0 0.0488338246941566 15.0
0 0.152583837509155 16.0
0 1.16974568367004 17.0
57.892961402 1.01911020278931 18.0
-57.892961402 1.02947771549225 19.0
-57.892961402 0.310676485300064 20.0
-57.892961402 0.00777428597211838 21.0
80.63898829 1.13978338241577 22.0
145.48214403 0.850925266742706 23.0
109.02233774 0.834744870662689 24.0
21.28018831 0.0684969574213028 25.0
226.12113232 1.16169011592865 26.0
226.12113232 1.54203820228577 27.0
23.221672472 1.45437753200531 28.0
249.34280479 0.437988221645355 29.0
249.34280479 1.54199814796448 30.0
46.437966943 1.04036474227905 31.0
9.3136921602 0.416483998298645 32.0
22.746026888 1.58991599082947 33.0
22.746026888 0.767455697059631 34.0
22.746026888 0.127951338887215 35.0
10.010492551 0.0232164412736893 36.0
8.783555806 0.352585077285767 37.0
0 0.0346853956580162 38.0
0 0.155448749661446 39.0
-12.022448055 0.45253974199295 40.0
-24.900678829 0.784736514091492 41.0
5.6366614611 0.364065617322922 42.0
-33.166576772 0.391492545604706 43.0
90.646377097 0.754151940345764 44.0
90.646377097 0.0818454176187515 45.0
6.4786973242 0.658949851989746 46.0
2.0683436417 0.364095062017441 47.0
145.56267684 0.0797132849693298 48.0
139.3784463 0.463600158691406 49.0
0 0.780563950538635 50.0
71.548139273 0.161531955003738 51.0
90.481502129 0.119146123528481 52.0
79.593043685 0.123373709619045 53.0
79.593043685 0.183127552270889 54.0
79.593043685 0.0455716624855995 55.0
0 3.16716480255127 56.0
80.63898829 0.0881581604480743 57.0
226.12113232 0.477680683135986 58.0
249.34280479 0.301769703626633 59.0
22.746026888 0.0136232674121857 60.0
0 0.108646705746651 61.0
172.93268288 0.768465459346771 0.0
172.93268288 1.80088543891907 1.0
171.43834711 1.73605954647064 2.0
166.64927716 5.48766422271729 3.0
0 6.18229866027832 4.0
0 4.05684852600098 5.0
-74.27971848 1.13895320892334 6.0
85.052210489 0.761842906475067 7.0
64.002864029 1.37300801277161 8.0
21.04934646 1.5576730966568 9.0
84.102004883 0.809220910072327 10.0
69.997304244 0.0729665160179138 11.0
80.218739662 0.787833750247955 12.0
80.218739662 0.393895119428635 13.0
80.218739662 0.125421449542046 14.0
0 0.223325818777084 15.0
0 0.170749947428703 16.0
0 1.38546907901764 17.0
62.486372213 1.50548386573792 18.0
-62.486372213 1.55522489547729 19.0
-62.486372213 0.501146793365479 20.0
-62.486372213 -0.0548379868268967 21.0
85.788652358 1.53031516075134 22.0
136.69056614 0.919554889202118 23.0
103.38620254 0.730599284172058 24.0
25.021146327 0.096102848649025 25.0
222.4792185 1.09518313407898 26.0
222.4792185 1.68955552577972 27.0
22.995818193 2.25674104690552 28.0
245.47503669 1.14104783535004 29.0
245.47503669 1.81332635879517 30.0
46.202619889 0.938594579696655 31.0
9.2800974493 0.411397069692612 32.0
23.302280145 1.55316090583801 33.0
23.302280145 0.852112293243408 34.0
23.302280145 0.157989114522934 35.0
10.380691646 0.0202167946845293 36.0
8.9975793609 0.38823664188385 37.0
0 0.0120049938559532 38.0
0 0.0964741259813309 39.0
-11.821208596 0.87935870885849 40.0
-25.305930495 0.525899171829224 41.0
4.9129498592 0.509950697422028 42.0
-32.871030813 0.535440325737 43.0
98.635467036 0.75601601600647 44.0
98.635467036 0.0442516803741455 45.0
7.1226941775 0.695441722869873 46.0
2.2190234624 0.424040496349335 47.0
172.93268288 0.171101033687592 48.0
166.64927716 0.140208452939987 49.0
0 0.150974527001381 50.0
85.052210489 0.318141281604767 51.0
84.102004883 0.0266154855489731 52.0
80.218739662 0.063026063144207 53.0
80.218739662 0.129632085561752 54.0
80.218739662 0.0414019823074341 55.0
0 3.35058569908142 56.0
85.788652358 0.151500791311264 57.0
222.4792185 0.299757838249207 58.0
245.47503669 0.467159658670425 59.0
23.302280145 -0.0138420322909951 60.0
0 0.0975340753793716 61.0
163.90056975 0.930785477161407 0.0
163.90056975 0.632906317710876 1.0
162.43094044 1.01949310302734 2.0
157.64183513 4.67631959915161 3.0
0 7.30055904388428 4.0
0 3.33938384056091 5.0
-59.553757697 0.0358271300792694 6.0
71.849131599 0.585556268692017 7.0
57.107883988 1.06005489826202 8.0
14.741247611 1.09178483486176 9.0
83.249219544 1.01457166671753 10.0
70.213553486 0.257070749998093 11.0
79.303144637 1.01202726364136 12.0
79.303144637 0.906045913696289 13.0
79.303144637 0.0645485669374466 14.0
0 0.122142925858498 15.0
0 0.216782048344612 16.0
0 1.40965223312378 17.0
59.299484974 0.914451658725739 18.0
-59.299484974 0.485339164733887 19.0
-59.299484974 0.909603953361511 20.0
-59.299484974 -0.0480820573866367 21.0
80.829584161 1.89880406856537 22.0
152.3507231 1.99039697647095 23.0
117.44446375 0.640867948532104 24.0
24.052232191 0.107161484658718 25.0
233.18030726 0.804120302200317 26.0
233.18030726 1.23005986213684 27.0
24.942777281 1.27198696136475 28.0
258.12308455 0.49626836180687 29.0
258.12308455 1.45919322967529 30.0
46.369057021 1.00386917591095 31.0
9.3089862072 0.326018184423447 32.0
21.530099187 1.68351984024048 33.0
21.530099187 0.898273646831512 34.0
21.530099187 0.129714295268059 35.0
9.8472933161 0.0430265218019485 36.0
8.6350080706 0.391039907932281 37.0
0 -0.00120209157466888 38.0
0 0.270252734422684 39.0
-12.345117437 0.423513293266296 40.0
-24.820458583 0.771794974803925 41.0
4.6761358183 0.334700554609299 42.0
-32.122741499 0.420228898525238 43.0
80.378629677 0.677203714847565 44.0
80.378629677 0.102701395750046 45.0
6.6866782172 0.476516932249069 46.0
2.1531594916 0.376645773649216 47.0
163.90056975 0.155753284692764 48.0
157.64183513 0.00321264285594225 49.0
0 0.324872612953186 50.0
71.849131599 0.13614721596241 51.0
83.249219544 0.268029063940048 52.0
79.303144637 0.124869890511036 53.0
79.303144637 0.251228153705597 54.0
79.303144637 0.231068834662437 55.0
0 2.78303933143616 56.0
80.829584161 0.101228453218937 57.0
233.18030726 0.0733210891485214 58.0
258.12308455 0.636294364929199 59.0
21.530099187 0.0300814472138882 60.0
0 0.106033563613892 61.0
151.39227718 0.892824530601501 0.0
151.39227718 1.26031708717346 1.0
150.0028495 2.05830025672913 2.0
145.21374425 5.67778253555298 3.0
0 6.86319923400879 4.0
0 2.97769784927368 5.0
-52.985716721 1.17563343048096 6.0
65.951739555 0.658568978309631 7.0
59.094586348 1.05169701576233 8.0
6.8571532078 1.18038558959961 9.0
97.009164615 0.92722088098526 10.0
84.421232776 0.2853022813797 11.0
85.474583896 0.3440942466259 12.0
85.474583896 0.574621915817261 13.0
85.474583896 0.0900231748819351 14.0
0 0.283118158578873 15.0
0 0.172517031431198 16.0
0 1.63895153999329 17.0
55.11913594 1.34512317180634 18.0
-55.11913594 1.36616885662079 19.0
-55.11913594 0.381222367286682 20.0
-55.11913594 0.0268198475241661 21.0
75.498416588 0.877121806144714 22.0
142.0153541 0.660919845104218 23.0
109.40126562 0.672707378864288 24.0
21.482277249 0.140909224748611 25.0
217.51377069 0.668847739696503 26.0
217.51377069 1.40393924713135 27.0
29.442624021 1.27648913860321 28.0
246.95639471 0.707621335983276 29.0
246.95639471 1.89367842674255 30.0
47.128713803 1.2874983549118 31.0
9.2381755208 0.666470050811768 32.0
20.379280648 1.21289312839508 33.0
20.379280648 1.26212394237518 34.0
20.379280648 0.109222039580345 35.0
9.6009357564 0.0279515981674194 36.0
8.3777797691 0.432807207107544 37.0
0 0.00419037416577339 38.0
0 0.128558278083801 39.0
-10.911123588 0.777738928794861 40.0
-24.588496228 0.463690012693405 41.0
5.0187061021 0.471159398555756 42.0
-31.962786227 0.538955569267273 43.0
88.821755911 0.602348744869232 44.0
88.821755911 0.108442723751068 45.0
7.1456534936 0.45209014415741 46.0
2.5056885922 0.387483894824982 47.0
151.39227718 0.145037680864334 48.0
145.21374425 0.579831540584564 49.0
0 0.377092987298965 50.0
65.951739555 0.169149518013 51.0
97.009164615 0.258285015821457 52.0
85.474583896 0.0976106822490692 53.0
85.474583896 0.093253031373024 54.0
85.474583896 0.171555370092392 55.0
0 1.23988139629364 56.0
75.498416588 0.119006261229515 57.0
217.51377069 0.149459958076477 58.0
246.95639471 0.241253554821014 59.0
20.379280648 0.0243509374558926 60.0
0 0.137344643473625 61.0
184.31542661 0.757505238056183 0.0
184.31542661 1.69879174232483 1.0
183.10405325 1.80656087398529 2.0
178.31494804 4.44926166534424 3.0
0 6.12483024597168 4.0
0 4.63871097564697 5.0
-62.392897522 0.799358069896698 6.0
72.583068506 0.83223682641983 7.0
50.914220057 0.723093390464783 8.0
21.668848449 0.757907092571259 9.0
87.990300539 0.977709114551544 10.0
74.385947491 0.402781307697296 11.0
78.461657764 0.857921779155731 12.0
78.461657764 0.318681210279465 13.0
78.461657764 0.0890041664242744 14.0
0 0.0720786154270172 15.0
0 0.212065741419792 16.0
0 1.46533524990082 17.0
59.265271075 1.06728672981262 18.0
-59.265271075 1.0022599697113 19.0
-59.265271075 0.350003898143768 20.0
-59.265271075 -0.00509278103709221 21.0
80.02513781 1.89106523990631 22.0
142.62710345 1.17492640018463 23.0
109.81692416 0.832231283187866 24.0
24.133816608 0.119822770357132 25.0
222.65224126 1.2983570098877 26.0
222.65224126 1.59029304981232 27.0
27.117099324 1.67403340339661 28.0
249.76934058 1.12880957126617 29.0
249.76934058 1.80622732639313 30.0
47.24984596 1.14453029632568 31.0
9.3706994253 0.572527885437012 32.0
20.759866735 1.35079514980316 33.0
20.759866735 0.700758814811707 34.0
20.759866735 0.142111301422119 35.0
9.5306188867 0.041414812207222 36.0
8.2730602312 0.359642595052719 37.0
0 0.0210547670722008 38.0
0 0.141318768262863 39.0
-11.26247511 0.883395373821259 40.0
-23.456889453 0.332260638475418 41.0
5.500011211 0.504862427711487 42.0
-31.510262791 0.472461402416229 43.0
96.927001833 0.726060211658478 44.0
96.927001833 0.0816301107406616 45.0
6.3310506797 0.636447906494141 46.0
2.4835470545 0.394596874713898 47.0
184.31542661 0.12913005053997 48.0
178.31494804 0.300391763448715 49.0
0 0.176258608698845 50.0
72.583068506 0.146414875984192 51.0
87.990300539 0.570262849330902 52.0
78.461657764 0.101380228996277 53.0
78.461657764 0.175498321652412 54.0
78.461657764 0.116264671087265 55.0
0 4.4029712677002 56.0
80.02513781 0.0334665812551975 57.0
222.65224126 0.525064289569855 58.0
249.76934058 0.682334363460541 59.0
20.759866735 0.00655818544328213 60.0
0 0.100177884101868 61.0
198.2074152 1.15824282169342 0.0
198.2074152 1.92503416538239 1.0
197.03823949 1.70315384864807 2.0
192.24913428 7.57342004776001 3.0
0 4.92180442810059 4.0
0 2.20423626899719 5.0
-72.367755978 0.536665320396423 6.0
83.908434301 0.856674075126648 7.0
59.069661847 0.680047273635864 8.0
24.838772454 0.886018455028534 9.0
68.044681156 1.22217345237732 10.0
54.698953789 0.100067839026451 11.0
70.422358742 0.122822239995003 12.0
70.422358742 0.679010689258575 13.0
70.422358742 0.15791355073452 14.0
0 0.117378778755665 15.0
0 0.160920202732086 16.0
0 1.09590542316437 17.0
53.099076087 1.09601676464081 18.0
-53.099076087 1.35301518440247 19.0
-53.099076087 0.239514842629433 20.0
-53.099076087 0.477106362581253 21.0
71.534135059 0.673070669174194 22.0
138.11698554 0.446625113487244 23.0
104.49849883 0.525974094867706 24.0
24.263860969 0.073225811123848 25.0
209.6511206 0.733933508396149 26.0
209.6511206 2.35144329071045 27.0
27.023632711 2.4974524974823 28.0
236.67475331 0.977954149246216 29.0
236.67475331 1.86864602565765 30.0
46.117636885 1.21464264392853 31.0
9.4473289252 0.66824609041214 32.0
18.435058972 0.857275605201721 33.0
18.435058972 1.35541939735413 34.0
18.435058972 0.131303578615189 35.0
8.1592199155 0.0385790318250656 36.0
6.7955507496 0.414582431316376 37.0
0 -0.0228487271815538 38.0
0 0.101034015417099 39.0
-12.649485234 0.597566246986389 40.0
-24.549044693 0.399131536483765 41.0
5.1940664091 0.468253970146179 42.0
-32.390567289 0.440551787614822 43.0
84.908281171 0.691802620887756 44.0
84.908281171 0.103480018675327 45.0
6.3604965152 0.663681030273438 46.0
2.4884631536 0.427673935890198 47.0
198.2074152 0.141007632017136 48.0
192.24913428 0.103858418762684 49.0
0 0.0569290854036808 50.0
83.908434301 0.140995174646378 51.0
68.044681156 0.0255467556416988 52.0
70.422358742 0.0268611907958984 53.0
70.422358742 0.00600172113627195 54.0
70.422358742 0.0833489447832108 55.0
0 7.85073518753052 56.0
71.534135059 -0.0728750973939896 57.0
209.6511206 0.157397210597992 58.0
236.67475331 0.14640548825264 59.0
18.435058972 -0.00138906575739384 60.0
0 0.113245628774166 61.0
178.16262983 1.03230953216553 0.0
178.16262983 1.13253366947174 1.0
177.12811894 1.02260148525238 2.0
171.66348353 4.50193309783936 3.0
2.1564568758 7.92483139038086 4.0
2.1564568758 4.06815624237061 5.0
-69.75332244 0.655368804931641 6.0
79.36324797 0.700481414794922 7.0
55.58551042 0.763739287853241 8.0
23.777737549 1.01868438720703 9.0
95.110741782 1.1243109703064 10.0
82.997886837 0.14965246617794 11.0
88.558427145 0.150136202573776 12.0
88.558427145 0.213620975613594 13.0
88.558427145 0.0772545784711838 14.0
0 0.195626735687256 15.0
0 0.20286451280117 16.0
0 1.28629779815674 17.0
52.958683467 1.19968962669373 18.0
-52.958683467 0.979847490787506 19.0
-52.958683467 0.636566042900085 20.0
-52.958683467 -0.097225233912468 21.0
73.072424658 1.52732157707214 22.0
137.52295013 1.13846921920776 23.0
105.84405416 0.59516841173172 24.0
21.768849376 0.187049552798271 25.0
210.59537478 1.20369482040405 26.0
210.59537478 1.95913600921631 27.0
26.416144842 1.04787993431091 28.0
237.01151963 0.635861754417419 29.0
237.01151963 1.67917048931122 30.0
46.211993097 0.940362811088562 31.0
9.2838959219 0.411139816045761 32.0
20.11374119 2.00781154632568 33.0
20.11374119 0.732977271080017 34.0
20.11374119 0.135749146342278 35.0
8.5093672193 0.0290977731347084 36.0
7.1697203069 0.383121073246002 37.0
0 0.0798320174217224 38.0
0 0.145474061369896 39.0
-11.868485072 0.69950807094574 40.0
-23.850110928 0.311499238014221 41.0
5.0381050295 0.489397257566452 42.0
-31.572631369 0.467724561691284 43.0
93.006813283 0.544654428958893 44.0
93.006813283 0.106408908963203 45.0
6.4047781814 0.613624155521393 46.0
1.9583948767 0.428015887737274 47.0
178.16262983 0.191386997699738 48.0
171.66348353 0.0710709914565086 49.0
2.1564568758 0.202368661761284 50.0
79.36324797 0.21191494166851 51.0
95.110741782 0.281880617141724 52.0
88.558427145 0.039844386279583 53.0
88.558427145 0.111742272973061 54.0
88.558427145 0.0645989328622818 55.0
0 0.484989404678345 56.0
73.072424658 0.298940986394882 57.0
210.59537478 0.517302870750427 58.0
237.01151963 0.675654768943787 59.0
20.11374119 0.00179585814476013 60.0
0 0.0844373106956482 61.0
160.4687021 1.02405631542206 0.0
160.4687021 1.76345670223236 1.0
159.50300866 2.63903641700745 2.0
154.71390334 5.1156210899353 3.0
0 9.30986499786377 4.0
0 4.78988075256348 5.0
-57.566860572 2.2721107006073 6.0
69.976085765 0.764728248119354 7.0
59.42698485 0.656384825706482 8.0
10.549100915 0.509627759456635 9.0
79.72325974 1.0361111164093 10.0
67.157019961 0.0346980765461922 11.0
80.095432736 0.626970887184143 12.0
80.095432736 0.395457923412323 13.0
80.095432736 0.0856264755129814 14.0
0 0.20441946387291 15.0
0 0.135522723197937 16.0
0 1.42897772789001 17.0
63.132647618 1.98085021972656 18.0
-63.132647618 2.48603749275208 19.0
-63.132647618 0.349529266357422 20.0
-63.132647618 0.0455910600721836 21.0
85.090233301 0.811901152133942 22.0
131.3021621 0.846183240413666 23.0
96.063328512 0.447316348552704 24.0
22.293114678 0.177847146987915 25.0
216.3923954 1.37418234348297 26.0
216.3923954 1.5197948217392 27.0
21.721824156 1.40574359893799 28.0
238.11421956 0.935917973518372 29.0
238.11421956 1.91583502292633 30.0
45.957560251 0.945016860961914 31.0
9.1494401389 0.514727234840393 32.0
21.957585684 1.30307197570801 33.0
21.957585684 0.931557774543762 34.0
21.957585684 0.182973548769951 35.0
9.3654044659 0.0290047749876976 36.0
8.0881655075 0.320167928934097 37.0
0 0.0427253246307373 38.0
0 0.144443497061729 39.0
-13.395010947 0.764162600040436 40.0
-26.518827676 0.186218976974487 41.0
5.3384036015 0.373985588550568 42.0
-34.643579108 0.32518059015274 43.0
77.385441482 0.806390345096588 44.0
77.385441482 0.13234980404377 45.0
6.8435993326 0.594193756580353 46.0
2.5265094483 0.346272051334381 47.0
160.4687021 0.040232241153717 48.0
154.71390334 0.688727915287018 49.0
0 0.407366424798965 50.0
69.976085765 0.0938728973269463 51.0
79.72325974 0.101873636245728 52.0
80.095432736 0.0321503579616547 53.0
80.095432736 0.0683413967490196 54.0
80.095432736 0.117861859500408 55.0
0 5.76189947128296 56.0
85.090233301 -0.0137542923912406 57.0
216.3923954 0.280784428119659 58.0
238.11421956 1.406986951828 59.0
21.957585684 -0.0205320250242949 60.0
0 0.0817576721310616 61.0
199.22958499 0.972162246704102 0.0
199.22958499 1.87002193927765 1.0
198.08050226 1.6103036403656 2.0
192.6158315 4.83332300186157 3.0
4.6909012764 8.48583698272705 4.0
4.6909012764 6.64903402328491 5.0
-71.188920357 0.576735973358154 6.0
82.357204696 0.83420729637146 7.0
49.965359177 0.631149351596832 8.0
32.391845519 0.779897093772888 9.0
80.685672138 1.12312531471252 10.0
67.408712635 0.291980475187302 11.0
68.759140487 0.911099970340729 12.0
68.759140487 0.37834158539772 13.0
68.759140487 0.116767950356007 14.0
16.127765537 0.236741989850998 15.0
16.127765537 0.174463599920273 16.0
0 1.5118864774704 17.0
64.037737955 1.06937718391418 18.0
-64.037737955 0.952021777629852 19.0
-64.037737955 0.911305129528046 20.0
-64.037737955 -0.0893053635954857 21.0
85.375540629 2.43420600891113 22.0
142.60980845 2.38613939285278 23.0
110.5500535 0.428244411945343 24.0
21.597281482 0.117318034172058 25.0
227.98534908 1.25894641876221 26.0
227.98534908 1.81117475032806 27.0
23.182844855 2.28050398826599 28.0
251.16819394 1.056964635849 29.0
251.16819394 1.81986117362976 30.0
46.202994235 1.03280532360077 31.0
9.3663748395 0.472582757472992 32.0
21.337802675 1.66930282115936 33.0
21.337802675 0.787043452262878 34.0
21.337802675 0.109330452978611 35.0
8.9185305273 0.0205484218895435 36.0
7.6552340101 0.404240667819977 37.0
0 0.0258771702647209 38.0
0 0.210850477218628 39.0
-11.257931081 0.46855428814888 40.0
-24.370090968 0.528030335903168 41.0
4.7489981205 0.576311349868774 42.0
-31.895528821 0.444278210401535 43.0
85.197989969 0.740017533302307 44.0
85.197989969 0.121475383639336 45.0
6.5582125596 0.703365623950958 46.0
2.4377940578 0.424401193857193 47.0
199.22958499 0.145911931991577 48.0
192.6158315 0.091863289475441 49.0
4.6909012764 0.273252218961716 50.0
82.357204696 0.190711244940758 51.0
80.685672138 0.262123107910156 52.0
68.759140487 0.0950773730874062 53.0
68.759140487 0.156354203820229 54.0
68.759140487 0.0201640445739031 55.0
0 2.52240586280823 56.0
85.375540629 0.13150629401207 57.0
227.98534908 0.208290979266167 58.0
251.16819394 0.572669088840485 59.0
21.337802675 0.000753792002797127 60.0
0 0.0925105288624763 61.0
189.99576757 0.67229700088501 0.0
189.99576757 1.79091238975525 1.0
188.43699435 2.28875494003296 2.0
183.6479244 8.29050636291504 3.0
0 4.34924173355103 4.0
0 7.13169860839844 5.0
-76.049806202 5.60251426696777 6.0
83.809297695 1.15023803710938 7.0
58.787875966 0.511139810085297 8.0
25.02142173 0.536412835121155 9.0
109.8479954 0.945812582969666 10.0
96.681907157 0.803392946720123 11.0
89.042362076 0.291591614484787 12.0
89.042362076 0.602508544921875 13.0
89.042362076 0.0924475714564323 14.0
0 0.212216719985008 15.0
0 0.119613908231258 16.0
0 1.37022066116333 17.0
60.190079109 1.04541718959808 18.0
-60.190079109 1.21548557281494 19.0
-60.190079109 0.417983144521713 20.0
-60.190079109 0.0276252403855324 21.0
82.742204211 1.77274310588837 22.0
144.42354644 1.25692868232727 23.0
105.53824458 0.21894171833992 24.0
18.338084709 0.206269830465317 25.0
227.16575066 0.942069947719574 26.0
227.16575066 2.61143350601196 27.0
28.844987281 1.82233929634094 28.0
256.01073794 1.27227509021759 29.0
256.01073794 1.88989913463593 30.0
46.280213898 1.17751395702362 31.0
9.3122703236 0.43701508641243 32.0
22.552125102 1.95825433731079 33.0
22.552125102 1.23075211048126 34.0
22.552125102 0.134744688868523 35.0
9.5772963013 0.0237736757844687 36.0
8.5123584265 0.343328833580017 37.0
0 -0.0078577809035778 38.0
0 0.220079869031906 39.0
-12.250580315 0.277798563241959 40.0
-26.732616812 0.605537891387939 41.0
5.2931506604 0.551478147506714 42.0
-34.644157393 0.393250226974487 43.0
79.832311356 0.914897918701172 44.0
79.832311356 0.0813141614198685 45.0
8.253210219 0.701939284801483 46.0
2.8634104141 0.318025410175323 47.0
189.99576757 0.13370643556118 48.0
183.6479244 0.621183276176453 49.0
0 0.628997266292572 50.0
83.809297695 0.264441251754761 51.0
109.8479954 0.706434845924377 52.0
89.042362076 0.143257215619087 53.0
89.042362076 0.196561232209206 54.0
89.042362076 0.0734285861253738 55.0
0 1.56881034374237 56.0
82.742204211 0.0838596522808075 57.0
227.16575066 0.697691857814789 58.0
256.01073794 0.723644137382507 59.0
22.552125102 0.0278883185237646 60.0
0 0.0997521728277206 61.0
166.24324715 0.984480261802673 0.0
166.24324715 1.00160276889801 1.0
164.893054 1.24696469306946 2.0
159.42840501 4.54739379882812 3.0
4.0462364349 6.60242557525635 4.0
4.0462364349 4.05337333679199 5.0
-60.888446416 0.501057803630829 6.0
73.445144898 0.614936292171478 7.0
46.962227174 0.657155692577362 8.0
26.482917725 0.840161621570587 9.0
90.386938164 1.04871129989624 10.0
75.981866605 0.234161198139191 11.0
79.752147036 0.691820025444031 12.0
79.752147036 0.388148367404938 13.0
79.752147036 0.111904218792915 14.0
0 0.211183696985245 15.0
0 0.161303117871284 16.0
0 0.929572284221649 17.0
52.226576873 1.09502804279327 18.0
-52.226576873 1.00923466682434 19.0
-52.226576873 0.406143844127655 20.0
-52.226576873 0.0208893157541752 21.0
73.983361037 2.5128448009491 22.0
146.57497499 1.73814153671265 23.0
111.97921922 0.484595060348511 24.0
23.806614173 0.214351639151573 25.0
220.55833603 1.06215858459473 26.0
220.55833603 1.41649186611176 27.0
23.451998729 1.96178340911865 28.0
244.01033476 0.752878546714783 29.0
244.01033476 1.93407392501831 30.0
46.436156897 1.15890312194824 31.0
9.2230913929 0.459579080343246 32.0
21.756784165 1.54688107967377 33.0
21.756784165 0.566298007965088 34.0
21.756784165 0.144993409514427 35.0
9.6207741704 0.0377239286899567 36.0
8.4305938465 0.391800314188004 37.0
0 0.04884884506464 38.0
0 0.100397571921349 39.0
-11.555697537 0.630704283714294 40.0
-23.405357549 0.463314175605774 41.0
4.7772475847 0.493800342082977 42.0
-30.975726578 0.307511985301971 43.0
88.679568128 0.704810619354248 44.0
88.679568128 0.064351312816143 45.0
6.4923056088 0.636191070079803 46.0
2.826060295 0.405633181333542 47.0
166.24324715 0.0880999118089676 48.0
159.42840501 0.0975717976689339 49.0
4.0462364349 0.376083076000214 50.0
73.445144898 0.187038108706474 51.0
90.386938164 0.0442543886601925 52.0
79.752147036 0.0836879760026932 53.0
79.752147036 0.0771209001541138 54.0
79.752147036 0.051981583237648 55.0
0 4.79954242706299 56.0
73.983361037 0.0669237300753593 57.0
220.55833603 0.300304800271988 58.0
244.01033476 0.406527280807495 59.0
21.756784165 -0.0113034807145596 60.0
0 0.0948040932416916 61.0
164.32220868 0.83659964799881 0.0
164.32220868 1.43115556240082 1.0
162.97561491 1.99573457241058 2.0
158.18650969 4.4455041885376 3.0
0 7.7259316444397 4.0
0 3.65657997131348 5.0
-54.353208189 0.26994776725769 6.0
67.443984334 0.70349907875061 7.0
58.182674568 0.710549712181091 8.0
9.2613097668 0.722119748592377 9.0
79.304256377 1.29858565330505 10.0
67.494247485 0.513142049312592 11.0
74.611856466 0.429179906845093 12.0
74.611856466 0.715842425823212 13.0
74.611856466 0.0623133890330791 14.0
0 0.014512057416141 15.0
0 0.212738692760468 16.0
0 1.09561395645142 17.0
54.639462484 1.35010981559753 18.0
-54.639462484 1.42144441604614 19.0
-54.639462484 0.201615482568741 20.0
-54.639462484 0.0223250538110733 21.0
75.479133953 1.97607982158661 22.0
141.43654298 1.29446578025818 23.0
107.54793369 0.440612554550171 24.0
22.262160021 0.1453537940979 25.0
216.91567693 1.42197120189667 26.0
216.91567693 1.73490357398987 27.0
27.808428171 2.16538858413696 28.0
244.7241051 0.569926619529724 29.0
244.7241051 1.70812475681305 30.0
46.905389931 1.07010233402252 31.0
9.4383114622 0.421221315860748 32.0
20.839671469 1.20862483978271 33.0
20.839671469 0.766362905502319 34.0
20.839671469 0.0748980641365051 35.0
8.8979067908 0.0231503266841173 36.0
7.6634495064 0.438899755477905 37.0
0 0.0695295482873917 38.0
0 0.245120450854301 39.0
-11.627837593 0.576005578041077 40.0
-24.199739315 0.499427765607834 41.0
4.9369905843 0.493371725082397 42.0
-31.877028638 0.560839772224426 43.0
95.599866399 0.757385969161987 44.0
95.599866399 0.0787518918514252 45.0
6.7773805624 0.501870572566986 46.0
2.4127463002 0.338050663471222 47.0
164.32220868 0.0983489453792572 48.0
158.18650969 0.302481591701508 49.0
0 0.234040170907974 50.0
67.443984334 0.0932353287935257 51.0
79.304256377 0.751410067081451 52.0
74.611856466 0.0234372243285179 53.0
74.611856466 0.194693133234978 54.0
74.611856466 0.154563322663307 55.0
0 5.67695426940918 56.0
75.479133953 0.0702480375766754 57.0
216.91567693 0.558220386505127 58.0
244.7241051 0.233619451522827 59.0
20.839671469 -0.0263766646385193 60.0
0 0.11383892595768 61.0
203.12555414 1.02284896373749 0.0
203.12555414 0.880350232124329 1.0
201.62646872 1.47286319732666 2.0
196.83736347 5.24633836746216 3.0
0 6.21862506866455 4.0
0 5.86796045303345 5.0
-52.510473227 1.0331974029541 6.0
64.268985048 0.458585977554321 7.0
51.78625798 0.745729625225067 8.0
12.482727068 0.82767653465271 9.0
86.850302036 1.01866436004639 10.0
74.232232821 0.359750807285309 11.0
75.524751887 0.563301026821136 12.0
75.524751887 0.280451625585556 13.0
75.524751887 0.0831069126725197 14.0
0 0.0416993722319603 15.0
0 0.210924535989761 16.0
0 1.15383541584015 17.0
56.965668417 2.36941599845886 18.0
-56.965668417 2.66768002510071 19.0
-56.965668417 0.233443677425385 20.0
-56.965668417 -0.0619425997138023 21.0
75.988376096 0.901127099990845 22.0
149.12650618 0.628690123558044 23.0
112.20395097 0.610410094261169 24.0
15.390258186 0.10386299341917 25.0
225.11488228 1.03852677345276 26.0
225.11488228 1.9449428319931 27.0
22.468616293 1.21112108230591 28.0
247.58349857 0.885351777076721 29.0
247.58349857 1.79166889190674 30.0
46.84485141 1.07595932483673 31.0
9.4341339123 0.518084645271301 32.0
19.022707678 1.80535352230072 33.0
19.022707678 1.30696976184845 34.0
19.022707678 0.127186447381973 35.0
9.592686895 0.0405464768409729 36.0
8.4446103689 0.41085085272789 37.0
0 0.044102992862463 38.0
0 0.265470892190933 39.0
-12.861756936 0.600840449333191 40.0
-25.930049654 0.674660086631775 41.0
5.1838761912 0.465981125831604 42.0
-33.907886302 0.394151717424393 43.0
90.041243487 0.707432508468628 44.0
90.041243487 0.138285487890244 45.0
6.1140006815 0.619450509548187 46.0
2.4057619364 0.475221216678619 47.0
203.12555414 0.380102515220642 48.0
196.83736347 0.144949391484261 49.0
0 0.368784040212631 50.0
64.268985048 0.00849087163805962 51.0
86.850302036 0.327951073646545 52.0
75.524751887 0.174907624721527 53.0
75.524751887 0.109763786196709 54.0
75.524751887 0.0555023550987244 55.0
0 6.87580537796021 56.0
75.988376096 0.201636359095573 57.0
225.11488228 0.664317309856415 58.0
247.58349857 0.752327084541321 59.0
19.022707678 -0.00604846514761448 60.0
0 0.0813007205724716 61.0
201.18844422 0.854216933250427 0.0
201.18844422 0.942341148853302 1.0
199.99936354 1.61232733726501 2.0
194.53469293 4.2375955581665 3.0
3.7035276288 5.21953773498535 4.0
3.7035276288 3.73307871818542 5.0
-60.492149243 0.296392560005188 6.0
73.482711427 1.39245462417603 7.0
46.384540487 0.372574061155319 8.0
27.09817094 0.353978455066681 9.0
91.933221189 1.03939568996429 10.0
79.129275992 0.388408929109573 11.0
88.685561805 0.60796993970871 12.0
88.685561805 0.556253612041473 13.0
88.685561805 0.0482900477945805 14.0
0 0.15131451189518 15.0
0 0.209196716547012 16.0
0 0.979758083820343 17.0
63.039184218 1.40365707874298 18.0
-63.039184218 1.64547789096832 19.0
-63.039184218 0.260430186986923 20.0
-63.039184218 0.28956401348114 21.0
83.390010612 1.563889503479 22.0
153.90254999 1.62612545490265 23.0
118.82469201 0.382176399230957 24.0
24.113174071 0.188539445400238 25.0
237.29256061 1.46016240119934 26.0
237.29256061 1.81069624423981 27.0
29.919504755 1.61600995063782 28.0
267.21206536 0.813298225402832 29.0
267.21206536 1.79947400093079 30.0
46.273115427 1.01287865638733 31.0
9.2335204557 0.672246932983398 32.0
20.350826394 1.10347414016724 33.0
20.350826394 1.08845031261444 34.0
20.350826394 0.104515239596367 35.0
9.229520655 0.033129096031189 36.0
7.9675011569 0.359786093235016 37.0
0 0.0217455290257931 38.0
0 0.19988551735878 39.0
-11.266657177 0.576406955718994 40.0
-24.364006574 0.593943357467651 41.0
5.0737495236 0.614747047424316 42.0
-32.183191776 0.806676864624023 43.0
90.502283635 0.795750856399536 44.0
90.502283635 0.0880474224686623 45.0
7.4609178203 0.709840714931488 46.0
2.7949770709 0.400079041719437 47.0
201.18844422 0.104390189051628 48.0
194.53469293 0.258953154087067 49.0
3.7035276288 0.227345302700996 50.0
73.482711427 0.185688972473145 51.0
91.933221189 0.623034179210663 52.0
88.685561805 0.0436648651957512 53.0
88.685561805 0.0931824967265129 54.0
88.685561805 0.117846272885799 55.0
0 1.80800843238831 56.0
83.390010612 0.00981267169117928 57.0
237.29256061 0.602054476737976 58.0
267.21206536 0.49449610710144 59.0
20.350826394 -0.0010027214884758 60.0
0 0.0985710322856903 61.0
133.31485351 1.04382383823395 0.0
133.31485351 1.37804841995239 1.0
132.22350811 1.7207578420639 2.0
127.43440281 5.45570755004883 3.0
0 5.36204528808594 4.0
0 3.42817091941833 5.0
-51.897730107 1.07970678806305 6.0
63.693140933 0.732671618461609 7.0
43.723765839 1.04250431060791 8.0
19.969375094 1.06036972999573 9.0
93.113575231 1.09989106655121 10.0
80.70051183 0.157333254814148 11.0
85.371340622 0.654427230358124 12.0
85.371340622 0.569767475128174 13.0
85.371340622 0.0391854047775269 14.0
0 0.136912643909454 15.0
0 0.172967731952667 16.0
0 1.56283640861511 17.0
60.664308043 1.10551810264587 18.0
-60.664308043 0.797227799892426 19.0
-60.664308043 0.307755291461945 20.0
-60.664308043 0.207942917943001 21.0
82.198925039 1.37085914611816 22.0
152.39329546 0.848332107067108 23.0
116.1574664 0.904774904251099 24.0
23.687312459 0.0636935383081436 25.0
234.5922205 1.12019443511963 26.0
234.5922205 1.63670992851257 27.0
24.863949765 2.28299140930176 28.0
259.45617027 0.756661772727966 29.0
259.45617027 1.77664220333099 30.0
46.971188625 0.968929469585419 31.0
9.2253542091 0.491912484169006 32.0
21.534616996 0.837719202041626 33.0
21.534616996 0.950163602828979 34.0
21.534616996 0.0515363812446594 35.0
9.2120466435 0.0196318365633488 36.0
8.0101367096 0.404475539922714 37.0
0 0.0304740257561207 38.0
0 0.163546115159988 39.0
-11.427547327 0.782339036464691 40.0
-24.858884881 0.408262819051743 41.0
5.2333934532 0.694078266620636 42.0
-32.320825775 0.817378222942352 43.0
79.26324863 0.785864889621735 44.0
79.26324863 0.10011400282383 45.0
6.5898072503 0.449842661619186 46.0
2.2872539224 0.293696522712708 47.0
133.31485351 0.1615050137043 48.0
127.43440281 0.320861726999283 49.0
0 0.364425837993622 50.0
63.693140933 0.217969015240669 51.0
93.113575231 0.0713609904050827 52.0
85.371340622 0.110480718314648 53.0
85.371340622 0.106501020491123 54.0
85.371340622 0.223508983850479 55.0
0 0.255323976278305 56.0
82.198925039 -0.0461624674499035 57.0
234.5922205 0.610770106315613 58.0
259.45617027 0.655059516429901 59.0
21.534616996 0.0374397039413452 60.0
0 0.0955646112561226 61.0
197.75586093 0.876681566238403 0.0
197.75586093 1.45479393005371 1.0
196.68725827 1.77275204658508 2.0
191.26250128 3.75335121154785 3.0
0.7378914899 3.86118531227112 4.0
0.7378914899 3.46718978881836 5.0
-52.883510722 2.11734008789062 6.0
64.883165531 0.465965777635574 7.0
40.438068923 0.824095129966736 8.0
24.445096607 1.17747390270233 9.0
87.160467679 0.879521727561951 10.0
73.482138325 0.0576704107224941 11.0
77.791347927 0.689623355865479 12.0
77.791347927 0.638774991035461 13.0
77.791347927 0.0626691207289696 14.0
0 0.180015861988068 15.0
0 0.176036044955254 16.0
0 1.44636988639832 17.0
55.621655601 1.68954336643219 18.0
-55.621655601 1.12048602104187 19.0
-56.261440426 1.12405848503113 20.0
-56.328403736 -0.0625380501151085 21.0
77.917761014 2.77674102783203 22.0
136.03685519 3.17891025543213 23.0
101.61063067 0.219266772270203 24.0
24.990887168 0.281847536563873 25.0
213.9546162 1.3104612827301 26.0
213.9546162 1.90952217578888 27.0
23.310924609 2.74434733390808 28.0
237.26554081 1.21910488605499 29.0
237.26554081 1.75314688682556 30.0
46.544724818 1.08378267288208 31.0
9.3892206917 0.554797887802124 32.0
21.589357278 1.67204034328461 33.0
21.589357278 0.581547260284424 34.0
21.589357278 0.156314000487328 35.0
9.1287968367 0.0271866451948881 36.0
7.8794960956 0.304385125637054 37.0
0 0.0766205936670303 38.0
0 0.224054098129272 39.0
-12.695538764 0.4518081843853 40.0
-26.82326841 0.734961092472076 41.0
5.4509717935 0.313809782266617 42.0
-34.587844268 0.455388128757477 43.0
88.930622467 0.573727011680603 44.0
87.693895089 0.0774635076522827 45.0
5.9777480402 0.631686747074127 46.0
2.0949513256 0.430959939956665 47.0
197.75586093 0.0783628672361374 48.0
191.26250128 0.17019884288311 49.0
0.7378914899 0.175194725394249 50.0
64.883165531 0.209660306572914 51.0
87.160467679 0.184440791606903 52.0
77.791347927 0.0170583259314299 53.0
77.791347927 0.0748127549886703 54.0
77.791347927 0.266696572303772 55.0
0 4.26128721237183 56.0
77.917761014 0.0203971154987812 57.0
213.9546162 0.223401501774788 58.0
237.26554081 0.504610955715179 59.0
21.589357278 -0.0178170893341303 60.0
0 0.119643643498421 61.0
173.53725159 0.891111493110657 0.0
173.53725159 0.957284033298492 1.0
172.0974945 1.04956293106079 2.0
167.30842455 3.42239904403687 3.0
0 3.95200109481812 4.0
0 2.60470938682556 5.0
-51.826247003 1.29049122333527 6.0
63.729632945 0.696352481842041 7.0
61.374596021 1.00901770591736 8.0
2.3550369232 1.16742861270905 9.0
93.62916115 1.0630692243576 10.0
80.968471622 0.505671918392181 11.0
80.160403892 0.303199052810669 12.0
80.160403892 0.331105947494507 13.0
80.160403892 0.0689327865839005 14.0
0 0.160209223628044 15.0
0 0.226111769676208 16.0
0 0.939070343971252 17.0
58.293350577 1.30434560775757 18.0
-58.293350577 1.28594923019409 19.0
-58.293350577 0.662954568862915 20.0
-58.293350577 0.0672871619462967 21.0
79.791712831 2.8978123664856 22.0
144.22366226 2.68664622306824 23.0
111.85606805 0.335928648710251 24.0
20.864482623 0.19382318854332 25.0
224.01537509 1.57819700241089 26.0
224.01537509 1.65073609352112 27.0
29.674985747 1.9950909614563 28.0
253.69036084 1.11669158935547 29.0
253.69036084 1.88541913032532 30.0
46.034009736 0.844608962535858 31.0
9.2295392251 0.459217727184296 32.0
21.498362255 1.27229869365692 33.0
21.498362255 1.3092839717865 34.0
21.498362255 0.125839173793793 35.0
8.2900560672 0.0235178824514151 36.0
7.1177076123 0.366738379001617 37.0
0 -0.0122542418539524 38.0
0 0.148448780179024 39.0
-11.204114159 0.433162838220596 40.0
-24.869299681 0.726376175880432 41.0
4.7195362228 0.461702764034271 42.0
-31.901608655 0.538964748382568 43.0
103.59005644 0.87761402130127 44.0
103.59005644 0.0942959412932396 45.0
6.6588586334 0.577451407909393 46.0
2.4350625974 0.339223027229309 47.0
173.53725159 0.164439529180527 48.0
167.30842455 0.106267057359219 49.0
0 0.194425195455551 50.0
63.729632945 0.130727842450142 51.0
93.62916115 0.357234746217728 52.0
80.160403892 0.139066457748413 53.0
80.160403892 0.0877413973212242 54.0
80.160403892 0.236939191818237 55.0
0 5.77652454376221 56.0
79.791712831 0.00260150618851185 57.0
224.01537509 0.366858690977097 58.0
253.69036084 1.29766070842743 59.0
21.498362255 0.00276223197579384 60.0
0 0.0975765436887741 61.0
162.03431659 1.00011813640594 0.0
162.03431659 1.27234065532684 1.0
160.90604486 2.09683203697205 2.0
156.11693958 3.97225546836853 3.0
0 4.92256307601929 4.0
0 4.8578782081604 5.0
-57.087866786 0.428600907325745 6.0
66.037095647 0.608030736446381 7.0
43.210350751 0.604496359825134 8.0
22.826744896 0.632043480873108 9.0
93.532754825 0.943430781364441 10.0
80.056752554 0.0261127986013889 11.0
82.374279138 0.844731152057648 12.0
82.374279138 0.361664235591888 13.0
82.374279138 0.0571454055607319 14.0
0 0.0244597662240267 15.0
0 0.209060952067375 16.0
0 1.05490946769714 17.0
60.900697493 1.89262855052948 18.0
-60.900697493 1.95917129516602 19.0
-60.900697493 0.436511248350143 20.0
-60.900697493 0.0349050685763359 21.0
82.932197396 2.47662997245789 22.0
148.15392295 2.49745154380798 23.0
116.86387902 0.157852470874786 24.0
21.276509274 0.425379037857056 25.0
231.08612035 1.24096930027008 26.0
231.08612035 1.2373594045639 27.0
25.93708644 1.62038171291351 28.0
257.02320679 0.739803671836853 29.0
257.02320679 1.63437485694885 30.0
46.178810242 1.02416050434113 31.0
9.2990824132 0.46159029006958 32.0
22.031499903 1.14392161369324 33.0
22.031499903 1.2450966835022 34.0
22.031499903 0.105668991804123 35.0
9.0649409244 0.023317739367485 36.0
7.7837242544 0.41246286034584 37.0
0 -0.0031118355691433 38.0
0 0.149758741259575 39.0
-12.443935516 0.572304785251617 40.0
-25.122971883 0.696576297283173 41.0
5.3022354742 0.524660110473633 42.0
-32.708062861 0.582583427429199 43.0
81.275215187 0.820569038391113 44.0
81.275215187 0.0984836891293526 45.0
7.3502425479 0.547379553318024 46.0
2.8161181409 0.349628299474716 47.0
162.03431659 0.232299447059631 48.0
156.11693958 0.502605438232422 49.0
0 0.168518349528313 50.0
66.037095647 0.162061646580696 51.0
93.532754825 0.112857446074486 52.0
82.374279138 0.0679478943347931 53.0
82.374279138 0.0285511277616024 54.0
82.374279138 0.0936159193515778 55.0
0 4.42890453338623 56.0
82.932197396 0.0671788156032562 57.0
231.08612035 0.241781800985336 58.0
257.02320679 0.258250415325165 59.0
22.031499903 0.0126494541764259 60.0
0 0.093986988067627 61.0
166.16500439 0.966662287712097 0.0
166.16500439 0.698987364768982 1.0
165.15071249 1.0016086101532 2.0
160.36160724 4.53755331039429 3.0
0 4.59699773788452 4.0
0 3.36212420463562 5.0
-59.155224545 0.310724020004272 6.0
70.404649842 0.639944493770599 7.0
42.567925754 0.590841889381409 8.0
27.836724087 0.647809684276581 9.0
83.737932758 1.09957873821259 10.0
70.505905756 0.16260589659214 11.0
78.790251826 0.649796903133392 12.0
78.790251826 0.35979351401329 13.0
78.790251826 0.111402913928032 14.0
0 0.0685497224330902 15.0
0 0.172881677746773 16.0
0 1.43680882453918 17.0
57.920638324 0.84674745798111 18.0
-57.920638324 0.577280521392822 19.0
-57.920638324 0.696015477180481 20.0
-57.920638324 -0.0784590020775795 21.0
78.92802453 1.79683721065521 22.0
139.30222541 1.25407588481903 23.0
105.91859782 0.516023457050323 24.0
20.931912772 0.139163330197334 25.0
218.23024994 1.40063714981079 26.0
218.23024994 1.21164345741272 27.0
26.504377019 1.569291472435 28.0
244.73462696 0.902204036712646 29.0
244.73462696 1.58196747303009 30.0
46.849420047 0.92007577419281 31.0
9.2166050504 0.501075744628906 32.0
21.007386206 1.45140099525452 33.0
21.007386206 1.00498974323273 34.0
21.007386206 0.155651807785034 35.0
9.6018387164 0.0420137867331505 36.0
8.192428436 0.370884895324707 37.0
0 0.0135376378893852 38.0
0 0.109787002205849 39.0
-11.539129722 0.55822479724884 40.0
-23.949360643 0.502229452133179 41.0
4.8799591695 0.411861568689346 42.0
-31.139914201 0.297782599925995 43.0
91.522822448 0.74747622013092 44.0
91.522822448 0.0966284200549126 45.0
6.9532755659 0.670135796070099 46.0
2.6304350609 0.407969146966934 47.0
166.16500439 0.107835203409195 48.0
160.36160724 0.0803695619106293 49.0
0 0.333964854478836 50.0
70.404649842 0.0632568076252937 51.0
83.737932758 0.400909662246704 52.0
78.790251826 0.0322573184967041 53.0
78.790251826 0.103845238685608 54.0
78.790251826 0.106924407184124 55.0
0 5.82008457183838 56.0
78.92802453 0.187749922275543 57.0
218.23024994 0.422817438840866 58.0
244.73462696 0.690605223178864 59.0
21.007386206 0.00614747777581215 60.0
0 0.0871102064847946 61.0
159.86698592 1.00515377521515 0.0
159.86698592 1.52724254131317 1.0
158.46815749 1.87003529071808 2.0
153.00348689 3.82253813743591 3.0
6.3383426047 6.38581514358521 4.0
6.3383426047 3.37935423851013 5.0
-57.300993676 1.05228233337402 6.0
68.742983284 0.323678821325302 7.0
42.532545962 0.671987116336823 8.0
26.210437322 0.835246205329895 9.0
86.611552927 1.0455197095871 10.0
74.056324211 0.26796567440033 11.0
79.537720801 0.672593057155609 12.0
79.537720801 0.145799547433853 13.0
79.537720801 0.0326527655124664 14.0
0 0.132657200098038 15.0
0 0.206658512353897 16.0
0 1.01776397228241 17.0
61.8741384 1.13302636146545 18.0
-61.8741384 1.33999860286713 19.0
-61.8741384 0.564380049705505 20.0
-61.8741384 -0.00114652514457703 21.0
82.521460199 1.64274489879608 22.0
151.60042244 1.30509781837463 23.0
117.67683987 0.258974403142929 24.0
23.51165394 0.280162125825882 25.0
234.12188264 1.08295702934265 26.0
234.12188264 1.49373054504395 27.0
27.924964849 1.61067318916321 28.0
262.04684749 0.986900627613068 29.0
262.04684749 1.61392843723297 30.0
46.915659651 0.957903563976288 31.0
9.3917174515 0.394634515047073 32.0
20.647321799 1.63105630874634 33.0
20.647321799 0.977491676807404 34.0
20.647321799 0.149417623877525 35.0
8.8502682695 0.0368726998567581 36.0
7.5735228227 0.404390037059784 37.0
0 0.0338524915277958 38.0
0 0.235241442918777 39.0
-12.40485667 0.688949167728424 40.0
-25.366488074 0.750395655632019 41.0
5.2650920289 0.376237601041794 42.0
-33.028256935 0.633081316947937 43.0
87.539566678 0.722440421581268 44.0
87.539566678 0.0560198947787285 45.0
6.8796343746 0.5361248254776 46.0
2.2127570483 0.403453856706619 47.0
159.86698592 0.139516845345497 48.0
153.00348689 0.292646735906601 49.0
6.3383426047 0.159566625952721 50.0
68.742983284 0.0931102782487869 51.0
86.611552927 0.0653122216463089 52.0
79.537720801 0.102081432938576 53.0
79.537720801 0.0520593151450157 54.0
79.537720801 0.0861467123031616 55.0
0 2.25407433509827 56.0
82.521460199 0.014723320491612 57.0
234.12188264 0.312180817127228 58.0
262.04684749 0.756900787353516 59.0
20.647321799 -0.0157131068408489 60.0
0 0.0963563174009323 61.0
188.56160617 0.869336068630219 0.0
188.56160617 1.98187792301178 1.0
187.12845876 1.92417466640472 2.0
186.06342207 5.22348213195801 3.0
0 10.5755023956299 4.0
0 4.21617078781128 5.0
-63.306010967 0.193888336420059 6.0
76.959486477 1.30049920082092 7.0
55.823198513 0.315162301063538 8.0
21.136287964 0.305659860372543 9.0
57.712243193 1.02268624305725 10.0
42.989452994 0.541048347949982 11.0
42.895418731 1.27259862422943 12.0
42.895418731 0.210319101810455 13.0
42.895418731 0.0531443506479263 14.0
29.904366056 0.0316894799470901 15.0
29.904366056 0.20005676150322 16.0
0 1.84745991230011 17.0
49.330270261 1.30871021747589 18.0
-49.330270261 1.77662348747253 19.0
-50.102525373 0.593233823776245 20.0
-50.176982742 -0.0615139976143837 21.0
70.166036553 3.05445408821106 22.0
139.77369233 3.06177234649658 23.0
104.52605681 0.291240781545639 24.0
22.536013325 0.251072138547897 25.0
209.93972889 1.68971681594849 26.0
209.93972889 1.31936013698578 27.0
22.812856437 1.48276710510254 28.0
232.75258532 0.900018095970154 29.0
232.75258532 1.81320595741272 30.0
46.747184611 1.15710210800171 31.0
9.3250272879 0.504184067249298 32.0
19.989053811 1.92821836471558 33.0
19.989053811 0.816482424736023 34.0
19.989053811 0.126609638333321 35.0
7.9436371588 0.0286589097231627 36.0
6.8165351521 0.331409871578217 37.0
0 0.044246718287468 38.0
0 0.183855593204498 39.0
-12.907753263 0.752164602279663 40.0
-25.681551671 0.439724743366241 41.0
5.8837002106 0.663411855697632 42.0
-34.249365019 0.625494360923767 43.0
84.58277079 0.892469584941864 44.0
84.58277079 0.0614408627152443 45.0
6.8831746603 0.73509156703949 46.0
2.847788272 0.448864847421646 47.0
188.56160617 0.010637529194355 48.0
186.06342207 0.162980884313583 49.0
0 0.406238973140717 50.0
76.959486477 0.0729058235883713 51.0
57.712243193 0.385398000478745 52.0
42.895418731 0.241328492760658 53.0
42.895418731 0.200813055038452 54.0
42.895418731 0.0851689577102661 55.0
0 1.74197864532471 56.0
70.166036553 0.0755572766065598 57.0
209.93972889 0.374254316091537 58.0
232.75258532 0.727411270141602 59.0
19.989053811 -0.0131686329841614 60.0
0 0.0934764966368675 61.0
136.83559207 0.899505794048309 0.0
136.83559207 1.7447521686554 1.0
135.53512108 1.2263251543045 2.0
130.07045046 3.26563096046448 3.0
2.2300972636 7.82591342926025 4.0
2.2300972636 8.27406597137451 5.0
-76.591612659 4.74924802780151 6.0
87.838671654 0.852903008460999 7.0
46.128075915 0.489217728376389 8.0
41.710595739 0.286204487085342 9.0
86.430499716 1.02049446105957 10.0
73.958266105 0.0626691877841949 11.0
84.866526477 0.980925858020782 12.0
84.866526477 0.137222021818161 13.0
84.866526477 0.110368497669697 14.0
0 0.0669181421399117 15.0
0 0.146729558706284 16.0
0 1.13994193077087 17.0
61.45387258 0.547051966190338 18.0
-61.45387258 0.981691837310791 19.0
-61.45387258 0.30863144993782 20.0
-61.45387258 -0.00622246414422989 21.0
83.759103227 1.42177093029022 22.0
149.1531379 1.94873332977295 23.0
117.45399573 0.277121692895889 24.0
19.346224499 0.529109179973602 25.0
232.91224113 1.16029191017151 26.0
232.91224113 1.2616800069809 27.0
24.849304329 0.985267579555511 28.0
257.76154546 0.443063676357269 29.0
257.76154546 1.51508092880249 30.0
47.090371715 1.30731415748596 31.0
9.3032237419 0.36506375670433 32.0
22.305230647 2.35647869110107 33.0
22.305230647 0.621783316135406 34.0
22.305230647 0.077483169734478 35.0
10.234811905 0.0273132920265198 36.0
8.8489876709 0.401817828416824 37.0
0 0.0322208218276501 38.0
0 0.408542096614838 39.0
-13.129787385 0.806845963001251 40.0
-27.34737389 0.536002218723297 41.0
4.5169501857 0.480349868535995 42.0
-34.280301025 0.284436047077179 43.0
89.268118061 0.953364968299866 44.0
89.268118061 0.0233578085899353 45.0
6.9371473899 0.761211812496185 46.0
3.1383647854 0.378645032644272 47.0
136.83559207 0.0636543035507202 48.0
130.07045046 0.283712029457092 49.0
2.2300972636 0.77999609708786 50.0
87.838671654 0.108605064451694 51.0
86.430499716 0.222366094589233 52.0
84.866526477 0.10721205919981 53.0
84.866526477 0.196492269635201 54.0
84.866526477 0.080400139093399 55.0
0 4.47074270248413 56.0
83.759103227 0.201895415782928 57.0
232.91224113 0.607712864875793 58.0
257.76154546 -0.0471039898693562 59.0
22.305230647 -0.0116759855300188 60.0
0 0.102818533778191 61.0
154.63860481 0.800135016441345 0.0
154.63860481 1.11063981056213 1.0
153.33182585 2.18847751617432 2.0
147.86719044 5.99821186065674 3.0
4.9714022419 6.48160362243652 4.0
4.9714022419 6.39717960357666 5.0
-70.474007802 2.84328889846802 6.0
83.178352775 0.790102005004883 7.0
47.874563273 1.28487360477448 8.0
35.303789501 1.17290198802948 9.0
99.149964381 0.790054440498352 10.0
86.285257319 0.18883304297924 11.0
86.414944871 0.204438149929047 12.0
86.414944871 0.63499516248703 13.0
86.414944871 0.0789517760276794 14.0
0 0.137155994772911 15.0
0 0.170369133353233 16.0
0 1.18523561954498 17.0
54.898556613 1.04903495311737 18.0
-54.898556613 0.531528234481812 19.0
-54.898556613 0.710177659988403 20.0
-54.898556613 -0.0553914457559586 21.0
75.08412529 2.48927140235901 22.0
138.89670229 2.53111004829407 23.0
105.44339317 0.401690006256104 24.0
22.511928778 0.111563310027122 25.0
213.98082758 1.39219129085541 26.0
213.98082758 1.72222709655762 27.0
25.27776715 1.89113068580627 28.0
239.25859473 0.929672420024872 29.0
239.25859473 1.96121418476105 30.0
46.912701697 1.06564593315125 31.0
9.2292377774 0.664635539054871 32.0
20.185568677 1.57923150062561 33.0
20.185568677 1.40323877334595 34.0
20.185568677 0.0794492214918137 35.0
9.0361364995 0.0238934606313705 36.0
7.7868389021 0.363325566053391 37.0
0 0.00721916928887367 38.0
0 0.20387676358223 39.0
-11.719148202 0.675615787506104 40.0
-25.166229369 0.675530254840851 41.0
5.5452243927 0.433419108390808 42.0
-33.465016504 0.609621047973633 43.0
104.91112224 0.679471015930176 44.0
104.91112224 0.0721985697746277 45.0
6.6921418198 0.578454256057739 46.0
2.4225074769 0.410593837499619 47.0
154.63860481 0.164101243019104 48.0
147.86719044 0.508860111236572 49.0
4.9714022419 0.850470304489136 50.0
83.178352775 0.478030532598495 51.0
99.149964381 0.238465696573257 52.0
86.414944871 0.0387210510671139 53.0
86.414944871 0.160447627305984 54.0
86.414944871 0.0517171695828438 55.0
0 2.50424861907959 56.0
75.08412529 0.151358544826508 57.0
213.98082758 0.615904450416565 58.0
239.25859473 1.14726173877716 59.0
20.185568677 0.018154788762331 60.0
0 0.0823418349027634 61.0
152.60651078 0.875927627086639 0.0
152.60651078 0.668419718742371 1.0
151.26475525 0.930499315261841 2.0
145.80008472 4.94291400909424 3.0
0.84508603911 4.7017126083374 4.0
0.84508603911 7.08752298355103 5.0
-75.975896978 7.84820556640625 6.0
85.47635262 0.525385200977325 7.0
61.789316985 0.587786316871643 8.0
23.687035635 0.831973791122437 9.0
86.499729035 0.775604367256165 10.0
72.168350265 0.0834522396326065 11.0
77.722043066 0.109908133745193 12.0
77.722043066 0.244371384382248 13.0
77.722043066 0.0917264968156815 14.0
0 0.0765830799937248 15.0
0 0.164735898375511 16.0
0 0.806994318962097 17.0
58.268917818 1.09763884544373 18.0
-58.268917818 1.26852750778198 19.0
-58.268917818 0.319128185510635 20.0
-58.268917818 0.0718485713005066 21.0
79.069848996 1.8128377199173 22.0
151.55041861 1.93944096565247 23.0
116.3228755 0.791717112064362 24.0
23.190864194 0.170285701751709 25.0
230.62026761 1.75442385673523 26.0
230.62026761 2.30320429801941 27.0
29.193679733 2.099196434021 28.0
259.81394734 0.77951991558075 29.0
259.81394734 1.83613324165344 30.0
47.426038019 1.02640795707703 31.0
9.4456959717 0.396111845970154 32.0
20.800931178 1.73719334602356 33.0
20.800931178 0.650148451328278 34.0
20.800931178 0.147740557789803 35.0
9.1924232532 0.0333826839923859 36.0
8.08983455 0.395949333906174 37.0
0 0.00444620475172997 38.0
0 0.0847976505756378 39.0
-12.682416119 1.06472039222717 40.0
-25.360504999 0.284772008657455 41.0
5.5689398711 0.544046998023987 42.0
-33.354922343 0.65759688615799 43.0
106.99593354 0.727737784385681 44.0
106.99593354 0.0649707764387131 45.0
6.6973304601 0.603980541229248 46.0
2.8414160068 0.400593966245651 47.0
152.60651078 0.321610301733017 48.0
145.80008472 0.207378938794136 49.0
0.84508603911 0.781108438968658 50.0
85.47635262 0.263253390789032 51.0
86.499729035 0.0736755728721619 52.0
77.722043066 0.0376004204154015 53.0
77.722043066 0.0878014639019966 54.0
77.722043066 0.0649615824222565 55.0
0 4.27660799026489 56.0
79.069848996 -0.00581464217975736 57.0
230.62026761 0.480826318264008 58.0
259.81394734 0.857542872428894 59.0
20.800931178 -0.00350010581314564 60.0
0 0.0816058963537216 61.0
158.37607079 1.04246139526367 0.0
158.37607079 1.57549285888672 1.0
156.76687293 1.8528003692627 2.0
151.97776771 5.84328842163086 3.0
0 9.27954006195068 4.0
0 4.5383038520813 5.0
-50.817307321 1.71113300323486 6.0
60.654842471 0.450899183750153 7.0
33.643520201 1.04500997066498 8.0
27.01132227 0.897192776203156 9.0
91.096171908 1.25586819648743 10.0
76.463489298 0.310559064149857 11.0
77.634940567 0.568077921867371 12.0
77.634940567 0.304448068141937 13.0
77.634940567 0.0767370760440826 14.0
0 0.0761691629886627 15.0
0 0.16381524503231 16.0
0 0.711039066314697 17.0
48.25370189 1.24419784545898 18.0
-48.25370189 1.57947969436646 19.0
-48.25370189 0.279219031333923 20.0
-48.25370189 0.189125567674637 21.0
68.498825484 1.79082083702087 22.0
146.56263932 1.31627833843231 23.0
110.18353978 0.850071907043457 24.0
16.883638237 0.0680346190929413 25.0
215.0614648 2.11027312278748 26.0
215.0614648 2.10119700431824 27.0
29.754982704 1.61964404582977 28.0
244.81644751 0.664447069168091 29.0
244.81644751 1.64182806015015 30.0
47.609527879 1.261390209198 31.0
9.4874015374 0.433590918779373 32.0
20.245123594 1.17243349552155 33.0
20.245123594 0.401192963123322 34.0
20.245123594 0.0787502825260162 35.0
8.4536723331 0.0205058883875608 36.0
7.3303195567 0.436295330524445 37.0
0 0.0972321629524231 38.0
0 0.142678499221802 39.0
-13.770594517 0.951343238353729 40.0
-26.174144809 0.484902381896973 41.0
4.9600932677 0.374222695827484 42.0
-33.399188783 0.323929041624069 43.0
101.53557974 0.616848170757294 44.0
101.53557974 0.0287315472960472 45.0
6.9290247564 0.653287589550018 46.0
2.7160967786 0.430927604436874 47.0
158.37607079 0.113404013216496 48.0
151.97776771 0.271099358797073 49.0
0 0.529523134231567 50.0
60.654842471 0.285483837127686 51.0
91.096171908 0.218803703784943 52.0
77.634940567 0.0272075794637203 53.0
77.634940567 0.213710680603981 54.0
77.634940567 0.0648274570703506 55.0
0 6.11377429962158 56.0
68.498825484 -0.0543195530772209 57.0
215.0614648 0.933797061443329 58.0
244.81644751 0.553996801376343 59.0
20.245123594 -0.0208813734352589 60.0
0 0.0705582350492477 61.0
157.35780962 1.00335991382599 0.0
157.35780962 0.36806583404541 1.0
155.76747728 1.27301168441772 2.0
150.97840682 6.1662712097168 3.0
0 3.5131504535675 4.0
0 5.08856153488159 5.0
-68.869242303 7.39281034469604 6.0
78.692998906 0.990456283092499 7.0
52.488731446 0.634548664093018 8.0
26.20426746 0.492759883403778 9.0
98.641448632 1.07674133777618 10.0
87.270146734 0.311904668807983 11.0
82.564146272 1.03083288669586 12.0
82.564146272 0.160772860050201 13.0
82.564146272 0.146289691329002 14.0
0 0.157302707433701 15.0
0 0.191751450300217 16.0
0 1.81771004199982 17.0
49.298363874 1.99325716495514 18.0
-49.298363874 1.80431687831879 19.0
-49.298363874 0.657882153987885 20.0
-49.298363874 -0.0252792052924633 21.0
67.251345325 0.796733796596527 22.0
141.49607034 0.329675704240799 23.0
110.9704912 0.774617254734039 24.0
25.765536643 0.0791632980108261 25.0
208.74741566 0.707918524742126 26.0
208.74741566 1.1567929983139 27.0
26.309043234 1.62339615821838 28.0
235.0564589 0.775515735149384 29.0
235.0564589 1.53042590618134 30.0
46.368584646 1.15014886856079 31.0
9.5176492559 0.429907739162445 32.0
17.952981451 1.65432620048523 33.0
17.952981451 0.698935747146606 34.0
17.952981451 0.143616423010826 35.0
9.1338183036 0.0265838988125324 36.0
7.9092352241 0.402691036462784 37.0
0 0.0423364788293839 38.0
0 0.263715475797653 39.0
-12.589716023 0.409193336963654 40.0
-25.298234814 1.24995744228363 41.0
5.856382821 0.348376542329788 42.0
-33.86135135 0.635459244251251 43.0
94.955219132 0.829630851745605 44.0
94.955219132 0.0551175400614738 45.0
6.5180111439 0.646028995513916 46.0
2.7122761965 0.33715757727623 47.0
157.35780962 0.0264929439872503 48.0
150.97840682 0.22771891951561 49.0
0 0.219199046492577 50.0
78.692998906 0.243735581636429 51.0
98.641448632 0.142607718706131 52.0
82.564146272 0.0924138352274895 53.0
82.564146272 0.130469307303429 54.0
82.564146272 0.146698400378227 55.0
0 4.07575607299805 56.0
67.251345325 0.11095779389143 57.0
208.74741566 0.368800163269043 58.0
235.0564589 0.363081514835358 59.0
17.952981451 0.0449803024530411 60.0
0 0.0781730487942696 61.0
181.53056129 0.727911591529846 0.0
181.53056129 1.35728907585144 1.0
180.25831184 1.40474081039429 2.0
174.83355484 4.22790002822876 3.0
0 9.85721492767334 4.0
0 6.2736382484436 5.0
-59.28442493 0.939644038677216 6.0
71.068484591 0.924140512943268 7.0
37.197141178 0.378861427307129 8.0
33.871343414 0.258574455976486 9.0
96.347952116 0.874313414096832 10.0
83.096437043 0.407732158899307 11.0
81.855430558 0.483324468135834 12.0
81.855430558 0.121768549084663 13.0
81.855430558 0.114977337419987 14.0
0 0.0406751334667206 15.0
0 0.139703750610352 16.0
0 1.6732292175293 17.0
52.043746925 0.701552987098694 18.0
-52.043746925 0.26434662938118 19.0
-52.68353175 0.960887312889099 20.0
-52.75049506 -0.088003970682621 21.0
73.595320676 1.5938868522644 22.0
130.67326751 1.77028071880341 23.0
96.153919513 0.387142509222031 24.0
20.090759253 0.0893927738070488 25.0
204.26858818 1.8446182012558 26.0
204.26858818 2.83888053894043 27.0
29.066406952 1.77310431003571 28.0
233.33499513 0.56825315952301 29.0
233.33499513 1.79027223587036 30.0
46.700343461 0.939468681812286 31.0
9.3001389827 0.563674926757812 32.0
20.844825616 2.03730607032776 33.0
20.844825616 0.870600521564484 34.0
20.844825616 0.107048586010933 35.0
9.5009810669 0.0513057932257652 36.0
8.3635006752 0.376906007528305 37.0
0 -0.00328184477984905 38.0
0 0.179468065500259 39.0
-10.891245067 0.519042372703552 40.0
-25.399376607 1.25706458091736 41.0
4.7975358544 0.314666986465454 42.0
-32.80636368 0.409719973802567 43.0
79.842961712 0.760800302028656 44.0
78.606234334 0.0867694690823555 45.0
7.3133304784 0.59522145986557 46.0
2.5105672472 0.374824583530426 47.0
181.53056129 0.18918052315712 48.0
174.83355484 0.352929890155792 49.0
0 1.17177271842957 50.0
71.068484591 0.0958113819360733 51.0
96.347952116 0.422516286373138 52.0
81.855430558 0.137434408068657 53.0
81.855430558 0.0264418534934521 54.0
81.855430558 0.0716211199760437 55.0
0 3.80561494827271 56.0
73.595320676 0.195063501596451 57.0
204.26858818 1.18393266201019 58.0
233.33499513 1.31131970882416 59.0
20.844825616 0.00443541631102562 60.0
0 0.0803106352686882 61.0
136.30325848 0.827660262584686 0.0
136.30325848 1.09055113792419 1.0
135.24871974 1.12875533103943 2.0
130.4596145 4.55482339859009 3.0
0 4.2291784286499 4.0
0 3.83753538131714 5.0
-53.043540893 0.214477956295013 6.0
64.067320885 0.539173007011414 7.0
42.58603993 0.758549809455872 8.0
21.481280955 0.953257203102112 9.0
87.962975349 0.859651267528534 10.0
75.631660746 0.129082843661308 11.0
73.795447733 0.841392517089844 12.0
73.795447733 0.664983630180359 13.0
73.795447733 0.0418791808187962 14.0
0 0.0158401131629944 15.0
0 0.25185215473175 16.0
0 1.21613383293152 17.0
51.958677822 1.08178460597992 18.0
-51.958677822 0.794518709182739 19.0
-51.958677822 1.0897558927536 20.0
-51.958677822 -0.0939989537000656 21.0
73.588833246 1.22565889358521 22.0
150.72641556 0.863812923431396 23.0
112.7792547 0.701536893844604 24.0
23.776818097 0.0860835760831833 25.0
224.3152488 1.13539123535156 26.0
224.3152488 1.24849891662598 27.0
23.156407954 0.848068714141846 28.0
247.47165676 0.45753800868988 29.0
247.47165676 1.72700917720795 30.0
46.478516816 1.03500950336456 31.0
9.245282389 0.562293767929077 32.0
21.630155424 2.09610891342163 33.0
21.630155424 0.971787989139557 34.0
21.630155424 0.150334179401398 35.0
9.1272586659 0.025754502043128 36.0
8.007517146 0.321751683950424 37.0
0 0.0290166549384594 38.0
0 0.246104329824448 39.0
-11.60177791 0.492823004722595 40.0
-24.090917259 0.725452423095703 41.0
4.7813399671 0.454873383045197 42.0
-31.382485272 0.465015023946762 43.0
98.413664153 0.613313734531403 44.0
98.413664153 0.0817236453294754 45.0
6.6459451034 0.611730813980103 46.0
2.635303721 0.397834599018097 47.0
136.30325848 0.144257426261902 48.0
130.4596145 0.265196293592453 49.0
0 0.65742290019989 50.0
64.067320885 0.268957763910294 51.0
87.962975349 0.0481784343719482 52.0
73.795447733 0.135545864701271 53.0
73.795447733 0.136829525232315 54.0
73.795447733 0.2122453302145 55.0
0 4.06923961639404 56.0
73.588833246 0.109588734805584 57.0
224.3152488 0.622079193592072 58.0
247.47165676 0.581742644309998 59.0
21.630155424 -0.00813673064112663 60.0
0 0.10090521723032 61.0
150.32936301 0.783574223518372 0.0
150.32936301 1.78136467933655 1.0
148.89300417 2.04806208610535 2.0
144.10393421 4.23167753219604 3.0
0 3.91587495803833 4.0
0 5.88028526306152 5.0
-61.133260256 7.42784404754639 6.0
71.803703165 0.981924116611481 7.0
40.581785207 0.516011893749237 8.0
31.221917958 0.340253114700317 9.0
87.645576286 0.779898703098297 10.0
73.545770381 0.00416025146842003 11.0
79.87787502 0.340824693441391 12.0
79.87787502 0.438989728689194 13.0
79.87787502 0.0795236825942993 14.0
0 0.0878348276019096 15.0
0 0.157124251127243 16.0
0 1.49737894535065 17.0
58.774254046 1.63637053966522 18.0
-58.774254046 1.5816605091095 19.0
-58.774254046 0.239512234926224 20.0
-58.774254046 0.0493158474564552 21.0
79.120302185 1.62447226047516 22.0
150.65238628 1.37317490577698 23.0
113.69925725 0.258888423442841 24.0
21.442458106 0.254436820745468 25.0
229.77268847 1.00107443332672 26.0
229.77268847 2.25048518180847 27.0
29.745293552 2.2801353931427 28.0
259.51798202 0.867554128170013 29.0
259.51798202 1.69063913822174 30.0
46.461171036 1.13766241073608 31.0
9.1991467872 0.356363981962204 32.0
20.346048139 1.60263979434967 33.0
20.346048139 0.677609622478485 34.0
20.346048139 0.0680569782853127 35.0
8.7915944002 0.0223491527140141 36.0
7.6647702324 0.436416298151016 37.0
0 0.0269764773547649 38.0
0 0.17290897667408 39.0
-12.094899545 0.899059116840363 40.0
-25.28130343 0.694867193698883 41.0
5.6073251672 0.592631697654724 42.0
-33.697545138 0.653542459011078 43.0
94.204221955 0.947911262512207 44.0
94.204221955 0.027292288839817 45.0
6.9352782511 0.676558256149292 46.0
1.8225684516 0.343513011932373 47.0
150.32936301 0.0911309793591499 48.0
144.10393421 0.623958349227905 49.0
0 0.388256877660751 50.0
71.803703165 0.301550328731537 51.0
87.645576286 0.220923811197281 52.0
79.87787502 0.0163617376238108 53.0
79.87787502 0.0949111878871918 54.0
79.87787502 0.0807251930236816 55.0
0 3.48235464096069 56.0
79.120302185 0.0202823374420404 57.0
229.77268847 0.649313867092133 58.0
259.51798202 0.939969718456268 59.0
20.346048139 -0.000159375369548798 60.0
0 0.0920093357563019 61.0
156.6031822 0.965868294239044 0.0
156.6031822 0.467374503612518 1.0
155.18762966 1.12057054042816 2.0
150.39852443 4.87086725234985 3.0
0 7.44492816925049 4.0
0 4.35135126113892 5.0
-69.704541908 0.0245716273784637 6.0
79.600915596 0.726635932922363 7.0
46.376995208 1.30374181270599 8.0
33.223920388 1.17482662200928 9.0
86.902211505 0.981801211833954 10.0
72.595836453 0.0181074365973473 11.0
81.687311218 0.261601716279984 12.0
81.687311218 0.231448635458946 13.0
81.687311218 0.0456079840660095 14.0
0 0.137576282024384 15.0
0 0.195942908525467 16.0
0 1.18481135368347 17.0
58.07099101 0.985937654972076 18.0
-58.07099101 0.6771479845047 19.0
-58.07099101 0.566753089427948 20.0
-58.07099101 0.0105414241552353 21.0
78.440618809 1.48170840740204 22.0
149.24163607 1.21089673042297 23.0
113.36548023 0.447408348321915 24.0
20.95695127 0.246497064828873 25.0
227.68225488 1.69954395294189 26.0
227.68225488 2.50762820243835 27.0
27.735918983 1.48549830913544 28.0
255.41817386 0.502519428730011 29.0
255.41817386 1.89461779594421 30.0
46.812148987 1.04343783855438 31.0
9.2836258648 0.510762095451355 32.0
20.3696278 1.45208871364594 33.0
20.3696278 1.15044033527374 34.0
20.3696278 0.101268477737904 35.0
8.8254238177 0.0260388310998678 36.0
7.5902300251 0.431265592575073 37.0
0 0.0213985592126846 38.0
0 0.109238348901272 39.0
-12.34424532 0.532565474510193 40.0
-25.578081971 0.399202018976212 41.0
4.850988565 0.600580811500549 42.0
-33.025564041 0.660223126411438 43.0
92.435399216 0.582746028900146 44.0
92.435399216 0.103521175682545 45.0
7.2143620273 0.572896063327789 46.0
2.5639870102 0.389921844005585 47.0
156.6031822 0.0899602249264717 48.0
150.39852443 0.262852996587753 49.0
0 0.305022418498993 50.0
79.600915596 0.358204364776611 51.0
86.902211505 0.109945565462112 52.0
81.687311218 0.0288204271346331 53.0
81.687311218 0.141565382480621 54.0
81.687311218 0.0458964966237545 55.0
0 1.32203388214111 56.0
78.440618809 0.117446459829807 57.0
227.68225488 1.11467158794403 58.0
255.41817386 0.722339332103729 59.0
20.3696278 -0.0162464678287506 60.0
0 0.0818571299314499 61.0
166.31115184 1.28242075443268 0.0
166.31115184 1.09613013267517 1.0
165.10069737 1.08930420875549 2.0
160.31159208 4.1415843963623 3.0
0 3.85357642173767 4.0
0 4.78682994842529 5.0
-74.809724574 2.32468438148499 6.0
87.911551171 0.553504049777985 7.0
53.560910902 0.688736796379089 8.0
34.350640269 0.860032081604004 9.0
85.987636971 1.25800633430481 10.0
73.374503205 0.133939832448959 11.0
75.940727711 0.802960813045502 12.0
75.940727711 0.143078774213791 13.0
75.940727711 0.17662750184536 14.0
0 0.165344148874283 15.0
0 0.162037461996078 16.0
0 1.421382188797 17.0
52.220295952 1.24425053596497 18.0
-52.220295952 1.37519311904907 19.0
-52.220295952 0.313342720270157 20.0
-52.220295952 0.0365187376737595 21.0
72.499578991 2.19542860984802 22.0
144.4020068 1.90576839447021 23.0
113.98146035 0.489886611700058 24.0
23.910422469 0.290312647819519 25.0
216.90158579 1.23969554901123 26.0
216.90158579 2.04079627990723 27.0
27.132496892 1.58845782279968 28.0
244.03408268 0.954674780368805 29.0
244.03408268 1.92128503322601 30.0
46.49183666 1.28095734119415 31.0
9.4819179261 0.454921185970306 32.0
20.279283039 1.36081147193909 33.0
20.279283039 0.826894164085388 34.0
20.279283039 0.116401627659798 35.0
9.1209596274 0.0254507809877396 36.0
7.910849863 0.352474629878998 37.0
0 0.00809575989842415 38.0
0 0.169645935297012 39.0
-11.442046802 0.495151966810226 40.0
-23.956866845 0.445826172828674 41.0
5.0664367121 0.568329691886902 42.0
-31.348008093 0.526988744735718 43.0
86.560410506 0.938202261924744 44.0
86.560410506 0.0762190446257591 45.0
6.9553765767 0.666512966156006 46.0
2.5391686221 0.322210192680359 47.0
166.31115184 0.143468290567398 48.0
160.31159208 0.0247990526258945 49.0
0 0.359721720218658 50.0
87.911551171 0.0608726069331169 51.0
85.987636971 0.115211494266987 52.0
75.940727711 0.110645204782486 53.0
75.940727711 0.062144186347723 54.0
75.940727711 0.0676189810037613 55.0
0 6.5471715927124 56.0
72.499578991 -0.00555163295939565 57.0
216.90158579 0.706181526184082 58.0
244.03408268 0.596807241439819 59.0
20.279283039 0.0243007894605398 60.0
0 0.111365124583244 61.0
139.35317881 1.06543254852295 0.0
139.35317881 1.36595499515533 1.0
137.89802699 1.26438271999359 2.0
133.10892172 3.9929141998291 3.0
0 4.69948101043701 4.0
0 5.45850038528442 5.0
-67.707806305 0.890090942382812 6.0
78.29633507 0.546086132526398 7.0
45.764348086 1.20109248161316 8.0
32.531986984 1.12586665153503 9.0
88.841035679 1.17992424964905 10.0
75.965203068 0.16498401761055 11.0
82.398395198 0.450678735971451 12.0
82.398395198 0.117849409580231 13.0
82.398395198 0.0829429849982262 14.0
0 0.0708006545901299 15.0
0 0.17344719171524 16.0
0 0.63121098279953 17.0
58.242567977 1.27147889137268 18.0
-58.242567977 1.19527244567871 19.0
-58.242567977 0.956547737121582 20.0
-58.242567977 0.00259483605623245 21.0
79.778117336 1.34107434749603 22.0
154.86420692 1.03833973407745 23.0
118.56731093 0.422138035297394 24.0
21.635246364 0.264948010444641 25.0
234.64232426 1.47561454772949 26.0
234.64232426 1.82833850383759 27.0
25.144011485 1.75079095363617 28.0
259.78633575 1.187863945961 29.0
259.78633575 1.80833506584167 30.0
46.348199452 1.23391997814178 31.0
9.3401352866 0.370049595832825 32.0
21.535549359 1.00420045852661 33.0
21.535549359 1.2324538230896 34.0
21.535549359 0.127572610974312 35.0
9.9818526514 0.0410703122615814 36.0
8.7997804348 0.384661078453064 37.0
0 0.0384082868695259 38.0
0 0.153105452656746 39.0
-13.056591704 0.601949691772461 40.0
-26.956683401 0.498362064361572 41.0
4.7595665147 0.377262443304062 42.0
-34.063001993 0.492044538259506 43.0
84.433638434 0.621274650096893 44.0
84.433638434 0.0846449062228203 45.0
7.2581720916 0.624636054039001 46.0
2.4840258875 0.409332484006882 47.0
139.35317881 0.167555257678032 48.0
133.10892172 0.1037243232131 49.0
0 0.366482883691788 50.0
78.29633507 0.263498485088348 51.0
88.841035679 0.214536905288696 52.0
82.398395198 0.0456788539886475 53.0
82.398395198 0.0825609564781189 54.0
82.398395198 0.0849539339542389 55.0
0 5.41775465011597 56.0
79.778117336 0.183783695101738 57.0
234.64232426 0.511039316654205 58.0
259.78633575 0.676720380783081 59.0
21.535549359 0.0217677168548107 60.0
0 0.110243111848831 61.0
153.12897822 0.757755696773529 0.0
153.12897822 1.42335915565491 1.0
152.02405248 1.82020604610443 2.0
147.2349472 5.96466779708862 3.0
0 6.95313119888306 4.0
0 3.47219610214233 5.0
-67.617594354 0.110715135931969 6.0
76.494287179 0.662695050239563 7.0
46.625894452 1.08525705337524 8.0
29.868392727 1.06300783157349 9.0
92.971389529 0.951833307743073 10.0
79.465254755 0.33895406126976 11.0
84.772377265 0.677900731563568 12.0
84.772377265 0.120579428970814 13.0
84.772377265 0.0759338960051537 14.0
0 0.191538617014885 15.0
0 0.179151102900505 16.0
0 1.10658276081085 17.0
54.402526052 1.09581828117371 18.0
-54.402526052 1.19096291065216 19.0
-54.402526052 0.319036573171616 20.0
-54.402526052 0.00561114028096199 21.0
75.708329599 1.69025838375092 22.0
151.08726469 1.96916282176971 23.0
119.79073945 0.278798431158066 24.0
24.764616941 0.426377177238464 25.0
226.79559429 0.670682489871979 26.0
226.79559429 1.43166363239288 27.0
25.597537358 1.53043496608734 28.0
252.39313165 1.03056144714355 29.0
252.39313165 1.98083579540253 30.0
46.843536121 1.00716483592987 31.0
9.3628566483 0.647162199020386 32.0
21.305803546 1.51046168804169 33.0
21.305803546 1.12600588798523 34.0
21.305803546 0.122915834188461 35.0
9.4447529667 0.0354517139494419 36.0
8.1789621457 0.372225224971771 37.0
0 0.0111607946455479 38.0
0 0.0764662325382233 39.0
-11.636953553 0.86277174949646 40.0
-24.822842055 0.374242752790451 41.0
5.1141654853 0.504463851451874 42.0
-32.38508461 0.575267255306244 43.0
89.728710838 0.611087083816528 44.0
89.728710838 0.141064167022705 45.0
6.5765706658 0.536498010158539 46.0
2.0495223591 0.397584855556488 47.0
153.12897822 0.100628077983856 48.0
147.2349472 0.284366697072983 49.0
0 0.229962334036827 50.0
76.494287179 0.342757016420364 51.0
92.971389529 0.3358353972435 52.0
84.772377265 0.131410494446754 53.0
84.772377265 0.0588420704007149 54.0
84.772377265 0.0479189157485962 55.0
0 2.3459849357605 56.0
75.708329599 0.0628227889537811 57.0
226.79559429 0.492947310209274 58.0
252.39313165 0.817783653736115 59.0
21.305803546 -0.01322634331882 60.0
0 0.0931835770606995 61.0
159.66337128 1.01568520069122 0.0
159.66337128 1.72918665409088 1.0
158.2534109 2.10830450057983 2.0
153.46430564 4.61299848556519 3.0
0 6.32301616668701 4.0
0 4.47552919387817 5.0
-64.443632068 3.26253581047058 6.0
76.652516104 0.423245251178741 7.0
41.621878918 0.944264471530914 8.0
35.030637187 0.947934746742249 9.0
87.291833079 1.26962113380432 10.0
73.969635886 0.414868712425232 11.0
79.721650275 0.738493144512177 12.0
79.721650275 0.107368640601635 13.0
79.721650275 0.0436468869447708 14.0
0 0.0169635508209467 15.0
0 0.198484525084496 16.0
0 0.860863447189331 17.0
55.894215796 1.04653382301331 18.0
-55.894215796 1.15374517440796 19.0
-55.894215796 0.471460908651352 20.0
-55.894215796 -0.0642763748764992 21.0
77.685428546 1.76688027381897 22.0
147.05129592 1.34153199195862 23.0
112.40802497 0.542765498161316 24.0
23.580795244 0.166638970375061 25.0
224.73672446 1.21863281726837 26.0
224.73672446 2.29644060134888 27.0
23.77630917 1.45474290847778 28.0
248.51303363 0.547783672809601 29.0
248.51303363 1.72665297985077 30.0
46.662348509 1.0390739440918 31.0
9.4115834335 0.500152766704559 32.0
21.79121275 1.46158194541931 33.0
21.79121275 0.849710464477539 34.0
21.79121275 0.115390479564667 35.0
10.244569953 0.0248522385954857 36.0
8.9678862542 0.354042381048203 37.0
0 0.0234164036810398 38.0
0 0.162560671567917 39.0
-12.151548297 0.872446358203888 40.0
-24.697052016 0.368199914693832 41.0
5.3198069325 0.456626236438751 42.0
-32.912628856 0.623874247074127 43.0
89.226102247 0.588428497314453 44.0
89.226102247 0.0935093015432358 45.0
6.6972957805 0.442509055137634 46.0
2.5242035356 0.338359773159027 47.0
159.66337128 0.134531706571579 48.0
153.46430564 0.311800092458725 49.0
0 0.292461931705475 50.0
76.652516104 0.198789611458778 51.0
87.291833079 0.671112239360809 52.0
79.721650275 0.0667547732591629 53.0
79.721650275 0.113994024693966 54.0
79.721650275 0.078945055603981 55.0
0 3.467937707901 56.0
77.685428546 0.0765755176544189 57.0
224.73672446 0.810203492641449 58.0
248.51303363 0.576662719249725 59.0
21.79121275 0.0211108811199665 60.0
0 0.0934314429759979 61.0
129.37515651 1.03483200073242 0.0
129.37515651 0.729995369911194 1.0
128.3586582 0.824026346206665 2.0
123.56955293 3.09452795982361 3.0
0 3.7263560295105 4.0
0 2.94752979278564 5.0
-54.84779765 0.141663953661919 6.0
64.42416639 1.58162081241608 7.0
51.37098602 0.304829597473145 8.0
13.05318037 0.222865417599678 9.0
85.403760471 1.14539611339569 10.0
71.093082007 0.12366146594286 11.0
80.089375368 1.21546673774719 12.0
80.089375368 0.907564878463745 13.0
80.089375368 0.0873677358031273 14.0
0 0.0852725207805634 15.0
0 0.158242851495743 16.0
0 1.90030932426453 17.0
64.064841207 1.49379646778107 18.0
-64.064841207 1.73592567443848 19.0
-64.064841207 0.562632441520691 20.0
-64.064841207 -0.0517851263284683 21.0
83.854841629 1.73884832859039 22.0
137.85166109 1.51354789733887 23.0
103.41081027 0.287420213222504 24.0
24.666248423 0.158171653747559 25.0
221.70650272 1.4014937877655 26.0
221.70650272 1.62335753440857 27.0
24.191354682 2.22798991203308 28.0
245.8978574 1.13228297233582 29.0
245.8978574 2.1876437664032 30.0
47.341060836 1.30202221870422 31.0
9.4640627764 0.349317789077759 32.0
19.790000423 1.36216104030609 33.0
19.790000423 1.09638726711273 34.0
19.790000423 0.127402693033218 35.0
9.4733160698 0.0273392740637064 36.0
8.3201040436 0.358098059892654 37.0
0 0.0341436974704266 38.0
0 0.0577165707945824 39.0
-11.870624611 0.64793187379837 40.0
-25.413591247 0.625296652317047 41.0
5.4029373468 0.516362190246582 42.0
-33.558110933 0.594544231891632 43.0
87.591633044 0.84169214963913 44.0
87.591633044 0.0880121290683746 45.0
6.2763084607 0.513442873954773 46.0
2.3194071757 0.295731484889984 47.0
129.37515651 0.0610937774181366 48.0
123.56955293 0.0975737944245338 49.0
0 0.185626491904259 50.0
64.42416639 0.179635629057884 51.0
85.403760471 0.019660409539938 52.0
80.089375368 0.0447533391416073 53.0
80.089375368 0.30630350112915 54.0
80.089375368 0.193050712347031 55.0
0 4.28670406341553 56.0
83.854841629 0.138344421982765 57.0
221.70650272 0.398883461952209 58.0
245.8978574 0.516048789024353 59.0
19.790000423 -0.00733925774693489 60.0
0 0.0809513404965401 61.0
145.30172525 1.07793021202087 0.0
145.30172525 1.03318297863007 1.0
143.96001695 2.33531856536865 2.0
139.17091173 5.30055475234985 3.0
0 5.23991537094116 4.0
0 3.83742928504944 5.0
-48.629265349 2.04322195053101 6.0
57.360697095 0.779547512531281 7.0
40.096625207 0.387067139148712 8.0
17.264071889 0.365173876285553 9.0
81.848795851 1.08987879753113 10.0
69.251753648 0.146474063396454 11.0
78.370710985 1.02348291873932 12.0
78.370710985 0.235992670059204 13.0
78.370710985 0.165134191513062 14.0
0 0.0679913908243179 15.0
0 0.170904964208603 16.0
0 1.86229860782623 17.0
53.217752233 1.25348234176636 18.0
-53.217752233 1.26734602451324 19.0
-53.217752233 0.698866724967957 20.0
-53.217752233 -0.0552405826747417 21.0
75.600854304 1.7759473323822 22.0
141.05588269 1.9123318195343 23.0
101.26420866 0.313435137271881 24.0
22.257687249 0.293956100940704 25.0
216.65673699 0.52080774307251 26.0
216.65673699 1.79197180271149 27.0
25.478810119 2.26729846000671 28.0
242.13554711 1.59745931625366 29.0
242.13554711 2.28706336021423 30.0
46.301826162 1.01639556884766 31.0
9.3078834089 0.538473308086395 32.0
22.383102071 1.63825249671936 33.0
22.383102071 1.05870175361633 34.0
22.383102071 0.0815555676817894 35.0
9.3074277894 0.0265703871846199 36.0
8.0759546403 0.400754988193512 37.0
0 0.0645821988582611 38.0
0 0.212963476777077 39.0
-12.216384804 0.538779497146606 40.0
-26.266176812 0.572384238243103 41.0
5.2601747299 0.37183153629303 42.0
-34.12127769 0.321803450584412 43.0
96.245220061 0.814880192279816 44.0
96.245220061 0.0932083800435066 45.0
7.4805145732 0.597199976444244 46.0
2.7017745999 0.365285336971283 47.0
145.30172525 0.13338029384613 48.0
139.17091173 0.345826864242554 49.0
0 0.412426769733429 50.0
57.360697095 0.197346374392509 51.0
81.848795851 0.108418047428131 52.0
78.370710985 0.115341439843178 53.0
78.370710985 0.0988345742225647 54.0
78.370710985 0.0501150451600552 55.0
0 5.25132465362549 56.0
75.600854304 0.0253869127482176 57.0
216.65673699 0.216314658522606 58.0
242.13554711 1.40460908412933 59.0
22.383102071 0.0207026712596416 60.0
0 0.105742551386356 61.0
154.72367992 0.923991739749908 0.0
154.72367992 1.38456845283508 1.0
153.20623582 1.12084567546844 2.0
148.41713066 2.14781045913696 3.0
0 3.21332406997681 4.0
0 2.9510223865509 5.0
-58.873530658 0.18265937268734 6.0
70.591780414 0.961539328098297 7.0
54.030708138 1.24510979652405 8.0
16.561072276 1.2803680896759 9.0
98.647872705 0.989011287689209 10.0
84.37559656 0.172030583024025 11.0
82.706671529 0.131004601716995 12.0
82.706671529 0.503962218761444 13.0
82.706671529 0.0724959969520569 14.0
0 0.24237547814846 15.0
0 0.131366834044456 16.0
0 1.11923432350159 17.0
55.753214423 0.956521809101105 18.0
-55.753214423 0.580682218074799 19.0
-55.753214423 1.06578481197357 20.0
-55.753214423 -0.0369618907570839 21.0
76.52442133 2.24867868423462 22.0
152.29435255 1.83834028244019 23.0
119.05814956 0.607760787010193 24.0
22.60332275 0.348541289567947 25.0
228.81877388 1.20983576774597 26.0
228.81877388 2.07278251647949 27.0
30.036576871 1.03648483753204 28.0
258.85535075 0.572958946228027 29.0
258.85535075 1.45077621936798 30.0
46.355163639 1.09435892105103 31.0
9.4473593746 0.472661286592484 32.0
20.771206907 1.57316064834595 33.0
20.771206907 0.613574206829071 34.0
20.771206907 0.136186674237251 35.0
9.2841781388 0.0353446155786514 36.0
7.9676033247 0.452052593231201 37.0
0 0.0326604470610619 38.0
0 0.0837136059999466 39.0
-11.73087816 0.63742995262146 40.0
-25.800945979 0.337836563587189 41.0
5.115290667 0.484036535024643 42.0
-33.676266619 0.363671898841858 43.0
88.601920979 0.904976665973663 44.0
88.601920979 0.11814621090889 45.0
6.2796045219 0.714824080467224 46.0
2.1626953111 0.400978595018387 47.0
154.72367992 0.0887829512357712 48.0
148.41713066 -0.00912921130657196 49.0
0 0.071128211915493 50.0
70.591780414 0.21973067522049 51.0
98.647872705 0.183318704366684 52.0
82.706671529 0.0293103102594614 53.0
82.706671529 0.0573689863085747 54.0
82.706671529 0.210215330123901 55.0
0 2.3699893951416 56.0
76.52442133 -0.00162580702453852 57.0
228.81877388 0.704649746417999 58.0
258.85535075 0.280907392501831 59.0
20.771206907 -0.0125994998961687 60.0
0 0.0823585987091064 61.0
146.20078791 0.971047639846802 0.0
146.20078791 0.531530559062958 1.0
144.75093624 0.850467443466187 2.0
139.96183045 4.25003957748413 3.0
0 6.7499361038208 4.0
0 4.91231060028076 5.0
-67.286951397 0.54486471414566 6.0
79.506187542 0.56690502166748 7.0
68.290582078 0.89493203163147 8.0
11.215605465 1.0955605506897 9.0
102.34955355 1.02181529998779 10.0
90.613076853 0.22820708155632 11.0
86.368589559 0.10295695066452 12.0
86.368589559 0.0536308698356152 13.0
86.368589559 0.12181418389082 14.0
0 0.0509263128042221 15.0
0 0.156779482960701 16.0
0 1.52217614650726 17.0
45.105623695 0.749440252780914 18.0
-45.105623695 0.925461649894714 19.0
-45.105623695 0.294211119413376 20.0
-45.105623695 0.0493918508291245 21.0
66.407531196 1.38144171237946 22.0
146.2332385 1.71365487575531 23.0
118.55800582 0.378518730401993 24.0
22.043736875 0.169283166527748 25.0
212.6407697 1.17274856567383 26.0
212.6407697 1.07674527168274 27.0
22.606179338 0.989201307296753 28.0
235.24694903 0.773603320121765 29.0
235.24694903 1.7999986410141 30.0
46.588280362 0.896766483783722 31.0
9.4845222371 0.326940834522247 32.0
21.301907501 1.41568243503571 33.0
21.301907501 0.84222137928009 34.0
21.301907501 0.0593588277697563 35.0
9.1439747133 0.0459924638271332 36.0
7.8018751202 0.467824846506119 37.0
0 0.00851092115044594 38.0
0 0.291896611452103 39.0
-12.594101682 0.336085319519043 40.0
-26.04055826 0.948141157627106 41.0
4.878732484 0.316150963306427 42.0
-33.871724907 0.325718909502029 43.0
88.920919763 0.850048959255219 44.0
88.920919763 0.101218834519386 45.0
6.4252714397 0.561611354351044 46.0
2.67368647 0.312449961900711 47.0
146.20078791 0.200968787074089 48.0
139.96183045 0.139833375811577 49.0
0 0.439529627561569 50.0
79.506187542 0.190071523189545 51.0
102.34955355 0.163828790187836 52.0
86.368589559 0.0627065449953079 53.0
86.368589559 0.0790934860706329 54.0
86.368589559 0.0786735713481903 55.0
0 2.32830429077148 56.0
66.407531196 0.053366407752037 57.0
212.6407697 0.155221447348595 58.0
235.24694903 1.46685874462128 59.0
21.301907501 0.0233692191541195 60.0
0 0.0985875576734543 61.0
188.01682404 0.977211534976959 0.0
188.01682404 1.3861483335495 1.0
186.5864898 2.27517223358154 2.0
181.31055227 4.38608074188232 3.0
0 9.21974182128906 4.0
0 3.52730560302734 5.0
-62.735026298 0.627279043197632 6.0
74.605344373 0.387094348669052 7.0
37.052624464 0.842244386672974 8.0
37.552719908 0.683976471424103 9.0
86.69655672 1.08338308334351 10.0
74.022740736 0.318748295307159 11.0
79.330528428 0.463507056236267 12.0
79.330528428 0.164579004049301 13.0
79.330528428 0.0630912929773331 14.0
0 0.016891723498702 15.0
0 0.228924199938774 16.0
0 0.921271502971649 17.0
55.26308359 0.790798127651215 18.0
-55.26308359 0.952984571456909 19.0
-55.26308359 0.31810674071312 20.0
-55.26308359 0.10444924980402 21.0
76.547404888 1.79138696193695 22.0
149.47743956 0.839905381202698 23.0
115.67411978 0.237698912620544 24.0
22.022084455 0.259559333324432 25.0
226.02484445 1.46356225013733 26.0
226.02484445 1.50087141990662 27.0
27.546538409 1.6942036151886 28.0
253.57138286 0.525683581829071 29.0
253.57138286 1.43452596664429 30.0
46.213366402 1.23534214496613 31.0
9.3232986829 0.314898490905762 32.0
21.284321298 1.35265874862671 33.0
21.284321298 1.07931530475616 34.0
21.284321298 0.17400535941124 35.0
8.745605448 0.0274596679955721 36.0
7.4704361895 0.377347350120544 37.0
0 0.0207655690610409 38.0
0 0.279184699058533 39.0
-11.242897821 0.51598584651947 40.0
-24.725347449 0.653749167919159 41.0
5.3689605752 0.36117959022522 42.0
-32.790324402 0.258328497409821 43.0
90.584536461 0.802106618881226 44.0
90.584536461 0.0680763944983482 45.0
6.4799609125 0.658666372299194 46.0
2.2413044365 0.400875180959702 47.0
188.01682404 -0.00666074454784393 48.0
181.31055227 0.280391931533813 49.0
0 0.382473051548004 50.0
74.605344373 0.0383819900453091 51.0
86.69655672 0.0670260712504387 52.0
79.330528428 0.0633890926837921 53.0
79.330528428 0.156512767076492 54.0
79.330528428 0.0978879183530807 55.0
0 4.0169243812561 56.0
76.547404888 0.11862601339817 57.0
226.02484445 0.166891649365425 58.0
253.57138286 0.0458319261670113 59.0
21.284321298 -0.0171841457486153 60.0
0 0.0789695978164673 61.0
157.98666768 0.903721213340759 0.0
157.98666768 1.62755024433136 1.0
156.5643194 1.6724259853363 2.0
151.09964864 6.21442604064941 3.0
1.2545217658 5.5767388343811 4.0
1.2545217658 3.6635844707489 5.0
-74.727203112 0.026674821972847 6.0
85.476898543 0.571258783340454 7.0
50.773674844 1.19144022464752 8.0
34.703223699 1.39784955978394 9.0
87.935689834 1.02702593803406 10.0
74.980860566 0.288843899965286 11.0
79.001005365 0.774208664894104 12.0
79.001005365 0.985741138458252 13.0
79.001005365 0.0303997024893761 14.0
0 0.133540004491806 15.0
0 0.184720128774643 16.0
0 1.22911846637726 17.0
60.240625433 1.2186963558197 18.0
-60.240625433 1.09011483192444 19.0
-60.240625433 0.337427765130997 20.0
-60.240625433 -0.0707803890109062 21.0
80.29456977 2.01185631752014 22.0
132.87257177 1.93147134780884 23.0
100.74032317 0.0805321335792542 24.0
23.240971281 0.262807667255402 25.0
213.16714154 1.22042298316956 26.0
213.16714154 1.22390985488892 27.0
24.246751956 1.28406238555908 28.0
237.41389349 0.467208832502365 29.0
237.41389349 1.55640876293182 30.0
46.035332634 1.10598421096802 31.0
9.2556226252 0.326779752969742 32.0
20.053944337 1.85418546199799 33.0
20.053944337 0.931054353713989 34.0
20.053944337 0.116089925169945 35.0
9.5811672328 0.0251050107181072 36.0
8.3589216345 0.315784841775894 37.0
0 0.0680072829127312 38.0
0 0.185818642377853 39.0
-12.276566761 0.266129642724991 40.0
-24.741775454 0.629112958908081 41.0
5.0206336702 0.445103168487549 42.0
-32.134726996 0.345815420150757 43.0
79.344746155 0.767143189907074 44.0
79.344746155 0.0809581875801086 45.0
7.5735361228 0.795669257640839 46.0
2.515564993 0.412366420030594 47.0
157.98666768 0.14785161614418 48.0
151.09964864 0.385000228881836 49.0
1.2545217658 0.462950855493546 50.0
85.476898543 0.135711520910263 51.0
87.935689834 0.263908237218857 52.0
79.001005365 0.0116283008828759 53.0
79.001005365 0.198263213038445 54.0
79.001005365 0.293289333581924 55.0
0 2.84414076805115 56.0
80.29456977 0.270089089870453 57.0
213.16714154 0.149357423186302 58.0
237.41389349 0.251388758420944 59.0
20.053944337 0.0295568592846394 60.0
0 0.080778993666172 61.0
147.07255833 1.09968769550323 0.0
147.07255833 0.615971088409424 1.0
146.04659821 0.657529473304749 2.0
141.257493 4.13384342193604 3.0
0 7.90033912658691 4.0
0 4.77160930633545 5.0
-64.705117111 0.589268028736115 6.0
74.487232356 0.885724067687988 7.0
50.601532787 0.904852271080017 8.0
23.885699569 1.03747832775116 9.0
89.167216629 1.09014284610748 10.0
76.909791131 0.0703593194484711 11.0
88.454049567 0.115050457417965 12.0
88.454049567 0.74154794216156 13.0
88.454049567 0.00796637870371342 14.0
0 0.166269451379776 15.0
0 0.201633229851723 16.0
0 0.911346793174744 17.0
55.248436379 1.04426860809326 18.0
-55.248436379 0.866336643695831 19.0
-55.248436379 0.484282433986664 20.0
-55.248436379 -0.0477801524102688 21.0
77.88423712 1.79268229007721 22.0
152.74660231 2.14814162254333 23.0
121.1415355 0.183792799711227 24.0
20.611998808 0.364333838224411 25.0
230.63083943 1.07270181179047 26.0
230.63083943 1.77381885051727 27.0
22.908394919 1.94540190696716 28.0
253.53923435 0.896539509296417 29.0
253.53923435 1.55128383636475 30.0
47.321031421 1.02408409118652 31.0
9.4346446295 0.500554919242859 32.0
22.635800741 1.84869027137756 33.0
22.635800741 0.693035542964935 34.0
22.635800741 0.169037312269211 35.0
10.069031641 0.0411865971982479 36.0
8.7922644123 0.416808575391769 37.0
0 0.00304887071251869 38.0
0 0.0985755175352097 39.0
-11.634213311 0.459563732147217 40.0
-25.539438766 0.571896553039551 41.0
5.3660307132 0.329188048839569 42.0
-33.512663275 0.459907978773117 43.0
94.38506299 0.425853371620178 44.0
94.38506299 0.105861738324165 45.0
7.2073360434 0.516767382621765 46.0
3.0607405487 0.450953364372253 47.0
147.07255833 0.170717492699623 48.0
141.257493 0.239245995879173 49.0
0 0.727179527282715 50.0
74.487232356 0.157996252179146 51.0
89.167216629 0.182889521121979 52.0
88.454049567 0.01006945874542 53.0
88.454049567 0.0199852250516415 54.0
88.454049567 0.185421764850616 55.0
0 0.778827846050262 56.0
77.88423712 0.129318431019783 57.0
230.63083943 0.386090666055679 58.0
253.53923435 0.368943780660629 59.0
22.635800741 0.00359104014933109 60.0
0 0.217447429895401 61.0
162.73876188 0.954793989658356 0.0
162.73876188 0.732456922531128 1.0
161.64744213 0.753589630126953 2.0
156.85837218 2.85478329658508 3.0
0 4.38649559020996 4.0
0 4.62109327316284 5.0
-59.629646376 4.34140777587891 6.0
68.731541148 0.706918179988861 7.0
55.309530275 1.2163724899292 8.0
13.422010872 1.35271871089935 9.0
88.894778621 0.861406564712524 10.0
76.349768979 0.249268397688866 11.0
81.415575773 0.291395902633667 12.0
81.415575773 0.145921170711517 13.0
81.415575773 0.0868716835975647 14.0
0 0.0608373060822487 15.0
0 0.2140983492136 16.0
0 1.3053834438324 17.0
57.352066501 1.44461667537689 18.0
-57.352066501 1.05063426494598 19.0
-57.352066501 0.713408052921295 20.0
-57.352066501 0.0465549007058144 21.0
77.196105247 2.11858201026917 22.0
138.46523594 1.7225239276886 23.0
103.76759395 0.28650438785553 24.0
22.150530744 0.227384269237518 25.0
215.66134119 1.48766374588013 26.0
215.66134119 1.87316608428955 27.0
25.129149767 1.4044646024704 28.0
240.79049096 0.793042659759521 29.0
240.79049096 2.08172535896301 30.0
46.936543794 1.20957887172699 31.0
9.4630818153 0.614378392696381 32.0
19.844038746 1.4329024553299 33.0
19.844038746 0.962492942810059 34.0
19.844038746 0.162794053554535 35.0
7.9506227498 0.0372305065393448 36.0
6.7665708143 0.295504987239838 37.0
0 0.0191146917641163 38.0
0 0.203177139163017 39.0
-13.147792096 0.893591642379761 40.0
-26.028572871 0.540366947650909 41.0
4.9720921743 0.505894124507904 42.0
-33.572861448 0.612834930419922 43.0
97.733841142 0.794334888458252 44.0
97.733841142 0.0436649844050407 45.0
6.4250463549 0.669843554496765 46.0
2.3525547438 0.403526276350021 47.0
162.73876188 0.396207273006439 48.0
156.85837218 0.00368771445937455 49.0
0 0.243655741214752 50.0
68.731541148 0.277308344841003 51.0
88.894778621 0.161502838134766 52.0
81.415575773 0.0631808117032051 53.0
81.415575773 0.11733765155077 54.0
81.415575773 0.112652823328972 55.0
0 5.03973531723022 56.0
77.196105247 -0.00535175297409296 57.0
215.66134119 0.334390908479691 58.0
240.79049096 0.681867599487305 59.0
19.844038746 0.0315507464110851 60.0
0 0.0862877666950226 61.0
184.04652619 0.838249325752258 0.0
184.04652619 1.2961517572403 1.0
182.74819937 1.70999252796173 2.0
177.95909412 4.34380865097046 3.0
0 7.57254219055176 4.0
0 4.26969718933105 5.0
-72.857934259 -0.0332365781068802 6.0
84.847847877 0.677641272544861 7.0
60.019390296 1.0065279006958 8.0
24.828457581 0.964657604694366 9.0
91.838916783 1.00923562049866 10.0
78.941917345 0.214960366487503 11.0
85.622831433 0.224030703306198 12.0
85.622831433 0.0902804732322693 13.0
85.622831433 0.088347002863884 14.0
0 0.180687919259071 15.0
0 0.212703719735146 16.0
0 1.13179790973663 17.0
61.916836269 1.39501047134399 18.0
-61.916836269 1.06033933162689 19.0
-61.916836269 0.599187672138214 20.0
-61.916836269 0.0421371124684811 21.0
84.593553732 1.60489928722382 22.0
146.27912612 1.01086115837097 23.0
109.6031757 0.505407691001892 24.0
23.229609132 0.0998954251408577 25.0
230.87267985 0.937348365783691 26.0
230.87267985 1.82135665416718 27.0
28.549634089 1.71793901920319 28.0
259.42231394 0.490870147943497 29.0
259.42231394 2.02719259262085 30.0
46.343146978 1.26984941959381 31.0
9.2584718687 0.504497408866882 32.0
22.676717463 1.2809225320816 33.0
22.676717463 0.785879492759705 34.0
22.676717463 0.149443745613098 35.0
9.9030859785 0.0197079125791788 36.0
8.5091598133 0.404815077781677 37.0
0 0.00403010472655296 38.0
0 0.22462172806263 39.0
-13.166435533 0.801739811897278 40.0
-24.748184622 0.398813545703888 41.0
4.9280373963 0.488685876131058 42.0
-32.50803567 0.349802494049072 43.0
90.351641696 0.681439280509949 44.0
90.351641696 0.0934465453028679 45.0
7.2300449809 0.761050403118134 46.0
2.3970802954 0.398700267076492 47.0
184.04652619 0.166180863976479 48.0
177.95909412 0.318061470985413 49.0
0 0.145230859518051 50.0
84.847847877 0.28825780749321 51.0
91.838916783 0.188112944364548 52.0
85.622831433 0.062068872153759 53.0
85.622831433 0.076998345553875 54.0
85.622831433 0.0879240036010742 55.0
0 2.34543681144714 56.0
84.593553732 0.0607687383890152 57.0
230.87267985 0.29786691069603 58.0
259.42231394 0.702432751655579 59.0
22.676717463 0.00506271421909332 60.0
0 0.111551910638809 61.0
171.18159723 1.18121016025543 0.0
171.18159723 1.77671611309052 1.0
169.79575148 1.99191987514496 2.0
165.00664632 3.88254928588867 3.0
0 7.63420963287354 4.0
0 3.40929508209229 5.0
-57.306143789 -0.0319356396794319 6.0
67.6809321 0.545974969863892 7.0
48.835134962 1.04965198040009 8.0
18.845797138 1.073725938797 9.0
84.936993 1.31902194023132 10.0
71.316816186 0.404072046279907 11.0
76.909329389 0.76469498872757 12.0
76.909329389 0.596316397190094 13.0
76.909329389 0.123252905905247 14.0
0 0.056793924421072 15.0
0 0.203278288245201 16.0
0 1.10560250282288 17.0
56.090067177 1.22557890415192 18.0
-56.090067177 1.17126560211182 19.0
-56.090067177 0.409614205360413 20.0
-56.090067177 -0.0390191599726677 21.0
76.286496841 1.40841734409332 22.0
142.69650593 0.680566847324371 23.0
107.1250296 0.734799683094025 24.0
22.641197326 0.11329872906208 25.0
218.98300277 1.04579186439514 26.0
218.98300277 1.61985659599304 27.0
23.933492436 1.11773419380188 28.0
242.9164952 0.437797635793686 29.0
242.9164952 1.34353625774384 30.0
46.844393808 1.03815352916718 31.0
9.467788464 0.413313329219818 32.0
20.196429664 1.89766442775726 33.0
20.196429664 1.15087711811066 34.0
20.196429664 0.105028681457043 35.0
8.8835251334 0.0257213488221169 36.0
7.6721992122 0.36141911149025 37.0
0 0.000397346913814545 38.0
0 0.143187955021858 39.0
-10.658716597 0.655833780765533 40.0
-23.590203673 0.447314918041229 41.0
5.4763315032 0.556427240371704 42.0
-31.593876543 0.475015223026276 43.0
91.346830936 0.708728492259979 44.0
91.346830936 0.097040943801403 45.0
7.2100738032 0.649062931537628 46.0
2.2372344661 0.395445287227631 47.0
171.18159723 0.135148093104362 48.0
165.00664632 0.293989300727844 49.0
0 0.143023476004601 50.0
67.6809321 0.231407403945923 51.0
84.936993 0.370369732379913 52.0
76.909329389 0.131615519523621 53.0
76.909329389 0.212699383497238 54.0
76.909329389 0.0450490042567253 55.0
0 6.88038158416748 56.0
76.286496841 0.182927057147026 57.0
218.98300277 0.670696318149567 58.0
242.9164952 0.293747276067734 59.0
20.196429664 -0.00589911360293627 60.0
0 0.0875221788883209 61.0
126.38059665 0.80861884355545 0.0
126.38059665 1.16868495941162 1.0
125.14197259 0.799885809421539 2.0
120.35286735 2.25620579719543 3.0
0 4.68198537826538 4.0
0 3.25133538246155 5.0
-51.480561037 2.34970760345459 6.0
63.61156955 1.02933835983276 7.0
65.149279347 0.544848799705505 8.0
-1.537709797 0.44221904873848 9.0
98.977321528 1.00384974479675 10.0
87.005606314 0.345985740423203 11.0
82.40749234 0.161515921354294 12.0
82.40749234 0.852166712284088 13.0
82.40749234 0.0405059792101383 14.0
0 0.0409716032445431 15.0
0 0.159306287765503 16.0
0 0.889758348464966 17.0
50.232938123 1.0347843170166 18.0
-50.232938123 1.37337636947632 19.0
-50.232938123 0.198630660772324 20.0
-50.232938123 0.00638208538293839 21.0
72.40965772 1.14019501209259 22.0
153.5108847 1.33160090446472 23.0
118.30169429 0.182738766074181 24.0
22.752988305 0.356943756341934 25.0
225.92054242 1.06146681308746 26.0
225.92054242 2.11586689949036 27.0
29.502523337 2.02471590042114 28.0
255.42306576 1.43807423114777 29.0
255.42306576 1.73489594459534 30.0
46.358789057 1.18223559856415 31.0
9.3445040566 0.23226709663868 32.0
22.176719597 1.27185690402985 33.0
22.176719597 1.17461812496185 34.0
22.176719597 0.0800581052899361 35.0
9.9384149102 0.03183239325881 36.0
8.6593008782 0.392997741699219 37.0
0 0.00949677079916 38.0
0 0.122364535927773 39.0
-11.699808632 0.645561456680298 40.0
-25.584720354 0.405724823474884 41.0
4.7576733146 0.383735179901123 42.0
-32.990481664 0.848527133464813 43.0
98.341605751 0.703892529010773 44.0
98.341605751 0.0945661962032318 45.0
6.2584743687 0.479213267564774 46.0
2.4294883478 0.315384447574615 47.0
126.38059665 0.172873109579086 48.0
120.35286735 0.0549039766192436 49.0
0 0.531801819801331 50.0
63.61156955 0.180944964289665 51.0
98.977321528 0.217379286885262 52.0
82.40749234 0.0247797816991806 53.0
82.40749234 0.107795186340809 54.0
82.40749234 0.328524619340897 55.0
0 2.23409533500671 56.0
72.40965772 0.0512691400945187 57.0
225.92054242 0.5292848944664 58.0
255.42306576 0.982254326343536 59.0
22.176719597 0.00931641273200512 60.0
0 0.0800940841436386 61.0
139.53333034 1.26844704151154 0.0
139.53333034 1.89609026908875 1.0
137.96976991 2.18107151985168 2.0
133.18066459 4.25192165374756 3.0
0 4.63413143157959 4.0
0 3.3923544883728 5.0
-58.916067821 1.64275908470154 6.0
68.508029584 0.864118218421936 7.0
65.536914216 0.77525669336319 8.0
2.9711153673 0.658203482627869 9.0
82.758632263 1.28989171981812 10.0
70.111836253 0.0698155611753464 11.0
81.478899325 0.606676638126373 12.0
81.478899325 0.723089039325714 13.0
81.478899325 0.0701712742447853 14.0
0 0.134610697627068 15.0
0 0.194533482193947 16.0
0 1.09771120548248 17.0
48.271047417 1.45835030078888 18.0
-48.271047417 1.05836796760559 19.0
-48.271047417 0.801039695739746 20.0
-48.271047417 -0.0414507016539574 21.0
68.204639672 1.45278084278107 22.0
138.86654142 1.40457534790039 23.0
109.91834719 0.322461783885956 24.0
21.390618491 0.306198418140411 25.0
207.0711811 1.14557600021362 26.0
207.0711811 2.23071193695068 27.0
30.111865825 2.59298062324524 28.0
237.18304692 1.42550730705261 29.0
237.18304692 2.01137781143188 30.0
46.993903772 1.10029256343842 31.0
9.337293673 0.568255245685577 32.0
19.933592255 1.47440099716187 33.0
19.933592255 1.17726421356201 34.0
19.933592255 0.132122904062271 35.0
9.2824780383 0.0357880219817162 36.0
7.9293005313 0.354433596134186 37.0
0 0.0301446355879307 38.0
0 0.0938477069139481 39.0
-12.764905043 0.873355686664581 40.0
-25.634678385 0.588110744953156 41.0
4.7092718488 0.498049020767212 42.0
-32.83376888 0.621581375598907 43.0
78.332624732 0.727616369724274 44.0
78.332624732 0.0759802162647247 45.0
6.4995408807 0.626839697360992 46.0
2.5551310271 0.330802112817764 47.0
139.53333034 0.153290539979935 48.0
133.18066459 0.263688802719116 49.0
0 0.248094066977501 50.0
68.508029584 0.0399279743432999 51.0
82.758632263 0.188194096088409 52.0
81.478899325 0.0606190264225006 53.0
81.478899325 0.202579855918884 54.0
81.478899325 0.00922912359237671 55.0
0 3.07162165641785 56.0
68.204639672 0.0195265486836433 57.0
207.0711811 0.633924841880798 58.0
237.18304692 0.869172871112823 59.0
19.933592255 0.00948737934231758 60.0
0 0.0941159278154373 61.0
178.61191588 0.696962773799896 0.0
178.61191588 0.612185657024384 1.0
177.50820782 2.03843402862549 2.0
173.55557339 4.30164098739624 3.0
0 8.12799549102783 4.0
0 3.4228367805481 5.0
-50.547629541 0.282999336719513 6.0
62.062798513 0.617802560329437 7.0
36.103422821 0.311474800109863 8.0
25.959375693 0.346610069274902 9.0
71.796770272 0.901913225650787 10.0
58.904360923 0.734416007995605 11.0
63.10843264 0.948863625526428 12.0
63.10843264 0.0766686946153641 13.0
63.10843264 0.0903057083487511 14.0
20.612312119 0.145725592970848 15.0
20.612312119 0.219197019934654 16.0
0 1.06416022777557 17.0
55.856598191 1.31192874908447 18.0
-55.856598191 1.94867587089539 19.0
-55.856598191 0.447573781013489 20.0
-55.856598191 -0.0109503790736198 21.0
77.546243346 1.34331881999969 22.0
158.2006747 1.55935728549957 23.0
116.67778772 0.282901346683502 24.0
23.630475167 0.125781625509262 25.0
235.74691804 1.31500709056854 26.0
235.74691804 1.14671337604523 27.0
26.299878571 1.88176286220551 28.0
262.04679661 0.503733932971954 29.0
262.04679661 1.77682411670685 30.0
47.072001333 1.06540703773499 31.0
9.2061089831 0.28863388299942 32.0
21.689645154 1.55708360671997 33.0
21.689645154 0.546514332294464 34.0
21.689645154 0.09243493527174 35.0
8.8068469746 0.0299777798354626 36.0
7.677109126 0.346095740795135 37.0
0 0.0776199400424957 38.0
0 0.157070755958557 39.0
-11.959530471 0.845228552818298 40.0
-24.952671964 0.366674453020096 41.0
5.7476943623 0.329348474740982 42.0
-33.50375319 0.508323013782501 43.0
99.592546077 0.57343190908432 44.0
99.592546077 0.0385914817452431 45.0
6.3637774823 0.576239287853241 46.0
2.6609915701 0.428827971220016 47.0
178.61191588 0.142121523618698 48.0
173.55557339 0.620226085186005 49.0
0 0.675111651420593 50.0
62.062798513 0.190802544355392 51.0
71.796770272 0.490962117910385 52.0
63.10843264 0.201027423143387 53.0
63.10843264 0.161335870623589 54.0
63.10843264 0.0936925113201141 55.0
0 5.49316024780273 56.0
77.546243346 0.0228319335728884 57.0
235.74691804 0.073302760720253 58.0
262.04679661 0.740372836589813 59.0
21.689645154 0.0509993731975555 60.0
0 0.0758767127990723 61.0
168.6035028 1.04842948913574 0.0
168.6035028 1.17874717712402 1.0
167.52178477 1.96352910995483 2.0
162.73267951 3.93055319786072 3.0
0 6.96736860275269 4.0
0 3.82923436164856 5.0
-65.176129934 2.33155703544617 6.0
77.813739566 0.771212339401245 7.0
51.188240217 0.797607719898224 8.0
26.625499349 0.893035352230072 9.0
81.887593461 1.12807476520538 10.0
69.090865906 0.200184628367424 11.0
77.630540013 0.513116419315338 12.0
77.630540013 0.184723258018494 13.0
77.630540013 0.131410598754883 14.0
0 0.056363046169281 15.0
0 0.136285215616226 16.0
0 1.65237855911255 17.0
54.715088017 1.73488914966583 18.0
-54.715088017 1.80110132694244 19.0
-54.715088017 0.462619036436081 20.0
-54.715088017 0.0573115535080433 21.0
75.901460384 2.26544880867004 22.0
152.11602441 2.03760933876038 23.0
113.57592738 0.520768463611603 24.0
22.330883628 0.15925432741642 25.0
228.01748479 1.1067533493042 26.0
228.01748479 1.95854473114014 27.0
28.953195422 2.15299105644226 28.0
256.97068021 1.17548310756683 29.0
256.97068021 1.76037132740021 30.0
46.758222198 0.92267382144928 31.0
9.2226334292 0.415034025907516 32.0
21.186372366 1.1030797958374 33.0
21.186372366 1.12047004699707 34.0
21.186372366 0.0854466706514359 35.0
9.4020966316 0.0213426202535629 36.0
8.2067206493 0.434348851442337 37.0
0 0.0101870745420456 38.0
0 0.121454462409019 39.0
-12.003415637 0.744188487529755 40.0
-25.779186194 0.491254687309265 41.0
5.103490607 0.532405436038971 42.0
-33.714166854 0.72730028629303 43.0
86.626356863 0.536063194274902 44.0
86.626356863 0.106740817427635 45.0
6.9983914548 0.457553088665009 46.0
2.721575429 0.350163817405701 47.0
168.6035028 0.121310353279114 48.0
162.73267951 0.340178966522217 49.0
0 0.643506348133087 50.0
77.813739566 0.146840155124664 51.0
81.887593461 0.210148692131042 52.0
77.630540013 0.072763130068779 53.0
77.630540013 0.0339845567941666 54.0
77.630540013 0.0678274482488632 55.0
0 3.58786296844482 56.0
75.901460384 0.0346595123410225 57.0
228.01748479 0.689789116382599 58.0
256.97068021 0.79685765504837 59.0
21.186372366 0.0191578939557076 60.0
0 0.0911727398633957 61.0
133.02307863 0.843040287494659 0.0
133.02307863 0.877126276493073 1.0
131.85391726 0.768005788326263 2.0
127.06481195 3.2988076210022 3.0
0 4.89708042144775 4.0
0 4.0404052734375 5.0
-75.673913865 0.178522989153862 6.0
84.579136714 1.13252854347229 7.0
65.790719787 2.15756297111511 8.0
18.788416926 2.54915690422058 9.0
96.245814229 0.888971745967865 10.0
84.885904181 0.0539652779698372 11.0
83.829975494 0.619076311588287 12.0
83.829975494 0.718951344490051 13.0
83.829975494 0.146726578474045 14.0
0 0.305696904659271 15.0
0 0.116697266697884 16.0
0 1.36665844917297 17.0
50.135179986 1.72519934177399 18.0
-50.135179986 1.96505832672119 19.0
-50.135179986 0.204780042171478 20.0
-50.135179986 0.0549841225147247 21.0
70.495589473 2.53787231445312 22.0
149.27382428 2.72533082962036 23.0
114.91695992 0.386103630065918 24.0
21.933346679 0.251648873090744 25.0
219.76941375 1.44802498817444 26.0
219.76941375 1.8247926235199 27.0
27.012008834 1.5168114900589 28.0
246.78142258 0.512747943401337 29.0
246.78142258 1.64966094493866 30.0
47.035350541 1.18990004062653 31.0
9.4876192754 0.614830613136292 32.0
20.360409487 1.97876214981079 33.0
20.360409487 0.991495847702026 34.0
20.360409487 0.0628059059381485 35.0
8.855942208 0.0289212223142385 36.0
7.6273457245 0.42558616399765 37.0
0 0.0449410788714886 38.0
0 0.0485329180955887 39.0
-11.227382679 0.631892621517181 40.0
-23.117558574 0.458932876586914 41.0
5.352383688 0.74957811832428 42.0
-31.152484319 0.866379916667938 43.0
79.037764841 0.611039102077484 44.0
79.037764841 0.100167326629162 45.0
6.5006574133 0.63508939743042 46.0
2.7935508873 0.389452576637268 47.0
133.02307863 0.168710500001907 48.0
127.06481195 0.0129872225224972 49.0
0 0.51899927854538 50.0
84.579136714 0.264609694480896 51.0
96.245814229 0.201891198754311 52.0
83.829975494 0.029052872210741 53.0
83.829975494 0.115937143564224 54.0
83.829975494 0.102990880608559 55.0
0 5.69312429428101 56.0
70.495589473 0.103802159428596 57.0
219.76941375 0.551417350769043 58.0
246.78142258 0.148995891213417 59.0
20.360409487 -0.0068485364317894 60.0
0 0.091304250061512 61.0
179.5641717 0.769364178180695 0.0
179.5641717 1.18245792388916 1.0
178.24096023 1.07081866264343 2.0
172.7762895 5.65097808837891 3.0
9.9088464172 6.03221654891968 4.0
9.9088464172 5.78238677978516 5.0
-62.748130697 3.75651001930237 6.0
72.978016063 0.466426432132721 7.0
43.812316654 0.529017269611359 8.0
29.165699408 0.650498151779175 9.0
90.117242078 1.12642800807953 10.0
76.487717385 0.417428493499756 11.0
79.235346316 0.720936059951782 12.0
79.235346316 0.75826096534729 13.0
79.235346316 0.123698011040688 14.0
0 0.224422976374626 15.0
0 0.181674808263779 16.0
0 1.57427203655243 17.0
63.050306488 1.25086808204651 18.0
-63.050306488 0.799935817718506 19.0
-63.050306488 0.553435146808624 20.0
-63.050306488 0.0303043089807034 21.0
83.715023031 1.51859331130981 22.0
139.18012212 1.3121645450592 23.0
102.60883892 0.467549860477448 24.0
20.696277957 0.192446753382683 25.0
222.89514516 0.972204983234406 26.0
222.89514516 1.42384541034698 27.0
25.106484336 1.89990293979645 28.0
248.00162949 0.840679883956909 29.0
248.00162949 2.00272560119629 30.0
46.638635782 1.15377044677734 31.0
9.484884044 0.620048463344574 32.0
20.664716543 1.69597804546356 33.0
20.664716543 0.470732867717743 34.0
20.664716543 0.106768257915974 35.0
9.6436288304 0.023011876270175 36.0
8.2857440679 0.407365053892136 37.0
0 0.0238729156553745 38.0
0 0.134282112121582 39.0
-11.751676692 0.447926700115204 40.0
-24.285999402 0.74004727602005 41.0
4.7167717512 0.408632040023804 42.0
-31.77512598 0.429344356060028 43.0
80.681362181 0.781128704547882 44.0
80.681362181 0.0959164798259735 45.0
6.9853007229 0.719549775123596 46.0
2.9370406812 0.408790111541748 47.0
179.5641717 0.115695759654045 48.0
172.7762895 0.0317541509866714 49.0
9.9088464172 0.413820117712021 50.0
72.978016063 0.0967237800359726 51.0
90.117242078 0.543584525585175 52.0
79.235346316 0.110412567853928 53.0
79.235346316 0.166332855820656 54.0
79.235346316 0.0551397353410721 55.0
0 3.98566961288452 56.0
83.715023031 -0.0535163953900337 57.0
222.89514516 0.483738213777542 58.0
248.00162949 0.645029544830322 59.0
20.664716543 0.00162068568170071 60.0
0 0.127151593565941 61.0
189.36690605 0.928334474563599 0.0
189.36690605 2.09018993377686 1.0
188.14192796 2.3386549949646 2.0
183.35282269 3.80683565139771 3.0
0 6.45583248138428 4.0
0 3.92359256744385 5.0
-51.55595448 1.89955234527588 6.0
59.807717003 0.639673948287964 7.0
44.252940058 0.428384989500046 8.0
15.554776945 0.429184705018997 9.0
101.7148049 1.08145415782928 10.0
90.041966283 0.360349655151367 11.0
89.537501722 0.147005274891853 12.0
89.537501722 0.362429261207581 13.0
89.537501722 0.0397437214851379 14.0
0 0.171119973063469 15.0
0 0.177965715527534 16.0
0 0.989876687526703 17.0
63.217874548 0.862585186958313 18.0
-63.217874548 1.01490461826324 19.0
-63.217874548 0.372194796800613 20.0
-63.217874548 0.0231632590293884 21.0
85.608026202 1.27115643024445 22.0
150.30479526 1.10319316387177 23.0
113.48078849 0.490675896406174 24.0
23.482920409 0.364705294370651 25.0
235.91282146 1.06041836738586 26.0
235.91282146 1.29123425483704 27.0
23.320967202 1.761594414711 28.0
259.23378867 1.02056562900543 29.0
259.23378867 1.73950481414795 30.0
46.643895576 1.09738636016846 31.0
9.4883663594 0.412212759256363 32.0
22.390151654 2.02545523643494 33.0
22.390151654 0.483376830816269 34.0
22.390151654 0.153122618794441 35.0
9.6106367521 0.04331886023283 36.0
8.4648328658 0.309330463409424 37.0
0 0.0263690054416656 38.0
0 0.121151953935623 39.0
-12.81117604 0.732013404369354 40.0
-25.218995157 0.385627061128616 41.0
5.4846489348 0.547556579113007 42.0
-32.939069985 0.493233680725098 43.0
91.720647532 0.80151891708374 44.0
91.720647532 0.139996573328972 45.0
7.3158745022 0.632489621639252 46.0
2.5520694292 0.355627357959747 47.0
189.36690605 0.0284256003797054 48.0
183.35282269 0.440297871828079 49.0
0 0.162614449858665 50.0
59.807717003 0.103787891566753 51.0
101.7148049 0.210392892360687 52.0
89.537501722 0.0420645847916603 53.0
89.537501722 0.126704946160316 54.0
89.537501722 0.0149036217480898 55.0
0 1.40746128559113 56.0
85.608026202 0.0564688295125961 57.0
235.91282146 0.514619529247284 58.0
259.23378867 0.586264729499817 59.0
22.390151654 -0.00275817513465881 60.0
0 0.175600275397301 61.0
212.29200619 1.32735741138458 0.0
212.29200619 2.30277609825134 1.0
210.75864982 2.49691581726074 2.0
205.33389283 5.89799547195435 3.0
0 3.15410804748535 4.0
0 4.81162786483765 5.0
-44.790946125 4.07505893707275 6.0
57.072532711 0.417617410421371 7.0
38.707202578 0.566379427909851 8.0
18.365330133 0.776631295681 9.0
95.463926018 1.39309144020081 10.0
82.102794756 0.00635294243693352 11.0
84.883963105 0.735463798046112 12.0
84.883963105 0.0992484092712402 13.0
84.883963105 0.0746658220887184 14.0
0 0.313505381345749 15.0
0 0.172541663050652 16.0
0 1.3737565279007 17.0
55.84794584 2.08486366271973 18.0
-55.84794584 1.7954238653183 19.0
-56.487730665 0.38300284743309 20.0
-56.554693975 -0.0869709700345993 21.0
75.893248404 1.99358355998993 22.0
161.38806463 2.61201190948486 23.0
125.7217888 0.260919660329819 24.0
27.081694154 0.236332982778549 25.0
237.28131303 1.04514122009277 26.0
237.28131303 1.96533536911011 27.0
23.862711426 2.95605421066284 28.0
261.14402446 1.64634203910828 29.0
261.14402446 1.98282957077026 30.0
45.983588194 0.87250542640686 31.0
9.2390308933 0.564670562744141 32.0
19.338554429 2.27588963508606 33.0
19.338554429 0.82848185300827 34.0
19.338554429 0.157135471701622 35.0
9.2706741677 0.0394298210740089 36.0
8.079611314 0.412948161363602 37.0
0 0.0145388506352901 38.0
0 0.15590637922287 39.0
-12.314438103 0.843073010444641 40.0
-24.80836886 0.250170707702637 41.0
5.118926901 0.596624612808228 42.0
-32.73925211 0.550483822822571 43.0
113.40206491 0.912821471691132 44.0
112.16533754 0.00855206698179245 45.0
6.196198341 0.823987722396851 46.0
2.1258133337 0.375066310167313 47.0
212.29200619 0.17972406744957 48.0
205.33389283 0.378583580255508 49.0
0 0.151526972651482 50.0
57.072532711 0.321289360523224 51.0
95.463926018 0.163234174251556 52.0
84.883963105 0.0323137082159519 53.0
84.883963105 0.0749754682183266 54.0
84.883963105 0.0137129528447986 55.0
0 0.332576960325241 56.0
75.893248404 0.141935929656029 57.0
237.28131303 0.368321508169174 58.0
261.14402446 1.35139727592468 59.0
19.338554429 -0.0147422924637794 60.0
0 0.115505866706371 61.0
194.02941892 1.07056796550751 0.0
194.02941892 2.18250393867493 1.0
192.59054059 2.13266658782959 2.0
187.80143535 5.09390449523926 3.0
0 8.53155422210693 4.0
0 3.44763660430908 5.0
-54.778740701 1.23161447048187 6.0
64.852627425 0.822513580322266 7.0
45.898104094 0.994789958000183 8.0
18.954523331 0.877502501010895 9.0
85.735243773 1.10037112236023 10.0
71.870179039 0.0922388806939125 11.0
78.597753615 0.124507248401642 12.0
78.597753615 0.313585340976715 13.0
78.597753615 0.0480764321982861 14.0
0 0.0138884335756302 15.0
0 0.241816982626915 16.0
0 0.959752798080444 17.0
59.025898613 1.1368213891983 18.0
-59.025898613 1.17615079879761 19.0
-59.025898613 0.336828291416168 20.0
-59.025898613 0.0199579410254955 21.0
79.165502164 1.52912759780884 22.0
135.88922171 1.04547619819641 23.0
104.32323273 0.637331247329712 24.0
20.720836172 0.0987446308135986 25.0
215.05472388 0.815717041492462 26.0
215.05472388 1.58246624469757 27.0
25.633099478 1.68933963775635 28.0
240.68782335 0.632560849189758 29.0
240.68782335 1.74592399597168 30.0
47.150462095 1.05845642089844 31.0
9.4739083082 0.429950743913651 32.0
20.139603551 1.46143269538879 33.0
20.139603551 0.92679488658905 34.0
20.139603551 0.162591829895973 35.0
9.1826945799 0.0432843901216984 36.0
7.9195478183 0.328330010175705 37.0
0 0.0267223678529263 38.0
0 0.210833534598351 39.0
-10.465473483 0.63829094171524 40.0
-23.494639891 0.633107721805573 41.0
5.3933372382 0.57181054353714 42.0
-31.425734657 0.725159883499146 43.0
94.107717426 0.615371823310852 44.0
94.107717426 0.0693129599094391 45.0
6.0205951571 0.552461385726929 46.0
2.2351759189 0.384766727685928 47.0
194.02941892 0.108903132379055 48.0
187.80143535 0.268761426210403 49.0
0 0.567865014076233 50.0
64.852627425 0.272462546825409 51.0
85.735243773 0.0872936174273491 52.0
78.597753615 0.0190851185470819 53.0
78.597753615 0.0590277761220932 54.0
78.597753615 0.0877549424767494 55.0
0 4.49096345901489 56.0
79.165502164 0.0976528823375702 57.0
215.05472388 0.495909094810486 58.0
240.68782335 0.85987651348114 59.0
20.139603551 -0.0142007973045111 60.0
0 0.0864612087607384 61.0
192.10935787 0.814829647541046 0.0
192.10935787 0.714042246341705 1.0
190.62421845 0.968928039073944 2.0
185.83511325 5.07367706298828 3.0
0 6.99528884887695 4.0
0 4.02649879455566 5.0
-55.998623139 1.09964370727539 6.0
66.102733923 1.02694153785706 7.0
39.208076926 0.775022387504578 8.0
26.894656998 0.5870201587677 9.0
82.279953891 1.00374710559845 10.0
69.406413956 0.217118129134178 11.0
80.925213235 0.927864015102386 12.0
80.925213235 0.173256278038025 13.0
80.925213235 0.104018308222294 14.0
0 0.289365917444229 15.0
0 0.162379413843155 16.0
0 1.38616049289703 17.0
57.416910412 1.67892813682556 18.0
-57.416910412 1.91751623153687 19.0
-57.416910412 0.259833931922913 20.0
-57.416910412 0.0350218191742897 21.0
78.202223244 1.90257406234741 22.0
153.5764793 1.71887505054474 23.0
115.8561192 0.383279949426651 24.0
20.800530249 0.126292496919632 25.0
231.77870254 1.43841767311096 26.0
231.77870254 1.23891496658325 27.0
29.048029984 1.30758130550385 28.0
260.82673253 0.416601032018661 29.0
260.82673253 1.50158619880676 30.0
47.016812046 1.12608325481415 31.0
9.2972822877 0.622797548770905 32.0
20.785312832 1.15033197402954 33.0
20.785312832 0.813095211982727 34.0
20.785312832 0.140432134270668 35.0
9.5001930682 0.0297599397599697 36.0
8.3042573489 0.369682878255844 37.0
0 0.0509462058544159 38.0
0 0.230266213417053 39.0
-12.555264244 0.52866405248642 40.0
-25.796360742 0.380846381187439 41.0
4.99530091 0.567586719989777 42.0
-33.47536425 0.538105845451355 43.0
98.975181669 0.610834121704102 44.0
98.975181669 0.120180539786816 45.0
6.6942013378 0.566185712814331 46.0
2.6643112187 0.412974685430527 47.0
192.10935787 0.166310161352158 48.0
185.83511325 0.072988472878933 49.0
0 0.559442758560181 50.0
66.102733923 0.268050879240036 51.0
82.279953891 0.314266085624695 52.0
80.925213235 0.151246890425682 53.0
80.925213235 0.0383671931922436 54.0
80.925213235 0.151727482676506 55.0
0 4.49736309051514 56.0
78.202223244 0.0163132548332214 57.0
231.77870254 0.646578073501587 58.0
260.82673253 0.479830265045166 59.0
20.785312832 0.00824998691678047 60.0
0 0.0961699336767197 61.0
165.47959735 0.782674431800842 0.0
165.47959735 1.52066826820374 1.0
164.06979214 1.38826584815979 2.0
158.60512147 2.86720395088196 3.0
2.1509363972 4.77984857559204 4.0
2.1509363972 5.73634958267212 5.0
-67.053302588 5.19055891036987 6.0
78.293517061 0.668163239955902 7.0
41.463893721 0.867847442626953 8.0
36.82962334 0.833780646324158 9.0
90.448200706 0.878364682197571 10.0
77.608580094 0.192237064242363 11.0
86.885826737 0.40797147154808 12.0
86.885826737 0.558374345302582 13.0
86.885826737 0.117986142635345 14.0
0 0.265266090631485 15.0
0 0.173471912741661 16.0
0 1.54971218109131 17.0
56.37916982 1.07619655132294 18.0
-56.37916982 1.20989644527435 19.0
-56.37916982 0.305271446704865 20.0
-56.37916982 -0.00293923169374466 21.0
77.443023017 1.33031988143921 22.0
147.25661912 0.631009340286255 23.0
110.06149049 0.706573009490967 24.0
23.801341397 0.0893106162548065 25.0
224.69964213 0.628822505474091 26.0
224.69964213 0.999083518981934 27.0
24.802862013 1.4001350402832 28.0
249.50250414 1.16842770576477 29.0
249.50250414 1.79313123226166 30.0
46.657482714 0.948408722877502 31.0
9.3883997997 0.398133635520935 32.0
21.063853197 1.5392861366272 33.0
21.063853197 0.833839535713196 34.0
21.063853197 0.0920860469341278 35.0
9.3527499001 0.0198870748281479 36.0
8.1536341086 0.403212547302246 37.0
0 0.0260279513895512 38.0
0 0.258466243743896 39.0
-12.404669274 0.509852886199951 40.0
-25.75748544 0.568787038326263 41.0
5.1758471707 0.575555741786957 42.0
-33.457214689 0.734558761119843 43.0
98.422258095 0.63263064622879 44.0
98.422258095 0.10472359508276 45.0
7.5373937492 0.633237957954407 46.0
2.7827975548 0.441049218177795 47.0
165.47959735 0.168536767363548 48.0
158.60512147 0.0853830799460411 49.0
2.1509363972 0.235179290175438 50.0
78.293517061 0.304440230131149 51.0
90.448200706 0.187727048993111 52.0
86.885826737 0.0345956459641457 53.0
86.885826737 0.154427900910378 54.0
86.885826737 0.0632362887263298 55.0
0 1.30098164081573 56.0
77.443023017 0.0922229588031769 57.0
224.69964213 0.152870342135429 58.0
249.50250414 1.08170092105865 59.0
21.063853197 -0.00407263450324535 60.0
0 0.0961828976869583 61.0
174.15048236 0.946833968162537 0.0
174.15048236 0.912889063358307 1.0
172.56259595 1.31092309951782 2.0
167.09792518 2.36931753158569 3.0
24.327416725 4.69493007659912 4.0
24.327416725 4.31918096542358 5.0
-46.220745246 -0.021659329533577 6.0
57.946446597 0.826930642127991 7.0
38.122916533 0.590494990348816 8.0
19.823530064 0.665824294090271 9.0
78.19919687 1.1177921295166 10.0
65.821378051 0.206763654947281 11.0
72.910772536 0.859110355377197 12.0
72.910772536 0.155677139759064 13.0
72.910772536 -0.0196339935064316 14.0
0 -0.0288951247930527 15.0
0 0.219599604606628 16.0
0 0.858868002891541 17.0
55.433115184 0.718499362468719 18.0
-55.433115184 0.9261274933815 19.0
-55.433115184 0.419239163398743 20.0
-55.433115184 0.220213800668716 21.0
77.697770144 1.47942566871643 22.0
137.64804266 1.24548363685608 23.0
105.22829144 0.200213670730591 24.0
26.300582613 0.07273980230093 25.0
215.3458128 1.66949641704559 26.0
215.3458128 2.4714777469635 27.0
29.864803476 2.89603614807129 28.0
245.21061628 1.64145028591156 29.0
245.21061628 2.18724489212036 30.0
47.612722424 0.996203780174255 31.0
9.4748095219 0.520192921161652 32.0
22.264654961 0.91404390335083 33.0
22.264654961 1.28189194202423 34.0
22.264654961 0.124962605535984 35.0
8.5137264613 0.038286417722702 36.0
7.2424136264 0.358681678771973 37.0
0 0.006570965051651 38.0
0 0.139193117618561 39.0
-13.369269937 0.522052407264709 40.0
-27.403684591 0.622535884380341 41.0
5.7707125663 0.277651369571686 42.0
-35.930799498 0.574403047561646 43.0
77.222023495 0.844280064105988 44.0
77.222023495 0.113432735204697 45.0
6.9361000329 0.811484754085541 46.0
2.3480243012 0.333536475896835 47.0
174.15048236 0.0203345455229282 48.0
167.09792518 0.0505514889955521 49.0
24.327416725 0.323392152786255 50.0
57.946446597 0.267766892910004 51.0
78.19919687 0.560107529163361 52.0
72.910772536 0.082853652536869 53.0
72.910772536 0.0507522411644459 54.0
72.910772536 0.0896787345409393 55.0
0 2.2509651184082 56.0
77.697770144 -0.0704160779714584 57.0
215.3458128 0.714995503425598 58.0
245.21061628 1.07446503639221 59.0
22.264654961 0.00640247017145157 60.0
0 0.102741740643978 61.0
186.69257267 0.803958356380463 0.0
186.69257267 1.35800647735596 1.0
185.70494893 2.32266855239868 2.0
180.24027818 4.64177799224854 3.0
2.524990874 8.99922370910645 4.0
2.524990874 3.45272517204285 5.0
-69.028591785 -0.00474384427070618 6.0
79.018495854 1.21301209926605 7.0
55.485678004 0.446793347597122 8.0
23.532817851 0.396776705980301 9.0
88.611795033 0.979494333267212 10.0
74.985244019 0.38843709230423 11.0
75.479032273 0.577104985713959 12.0
75.479032273 0.244533255696297 13.0
75.479032273 0.141519501805305 14.0
5.4103287499 0.234333112835884 15.0
5.4103287499 0.176077485084534 16.0
0 1.52078461647034 17.0
61.145757379 1.34031367301941 18.0
-61.145757379 1.08891069889069 19.0
-61.145757379 0.665411412715912 20.0
-61.145757379 -0.0450356006622314 21.0
80.223179592 1.4914813041687 22.0
149.69179341 0.856684505939484 23.0
111.81461686 0.363068610429764 24.0
22.620712888 0.133428335189819 25.0
229.914973 1.73617386817932 26.0
229.914973 2.26135492324829 27.0
23.787239007 1.75454044342041 28.0
253.70221201 1.06200957298279 29.0
253.70221201 2.05024313926697 30.0
46.745135488 1.06663346290588 31.0
9.3881710056 0.640174388885498 32.0
19.077422213 1.35464298725128 33.0
19.077422213 0.911587953567505 34.0
19.077422213 0.0892960652709007 35.0
10.047174399 0.0206348747014999 36.0
8.762957805 0.426925510168076 37.0
0 0.00663194805383682 38.0
0 0.289422869682312 39.0
-11.297321337 0.612697064876556 40.0
-23.795358757 0.627838015556335 41.0
4.7072499227 0.556083202362061 42.0
-30.760681596 0.664124071598053 43.0
76.703147218 0.739350199699402 44.0
76.703147218 0.0960333347320557 45.0
6.7981855023 0.504622757434845 46.0
2.3366032062 0.318507790565491 47.0
186.69257267 0.227163225412369 48.0
180.24027818 0.505837023258209 49.0
2.524990874 0.296733379364014 50.0
79.018495854 0.0703993439674377 51.0
88.611795033 0.669627785682678 52.0
75.479032273 0.104445353150368 53.0
75.479032273 0.0615317076444626 54.0
75.479032273 0.0560996979475021 55.0
0 4.36839199066162 56.0
80.223179592 0.0685893669724464 57.0
229.914973 0.623861968517303 58.0
253.70221201 0.648536145687103 59.0
19.077422213 0.0292388014495373 60.0
0 0.0795472413301468 61.0
195.31953958 0.940544962882996 0.0
195.31953958 1.03123450279236 1.0
194.27917189 1.71425986289978 2.0
188.81450118 5.05295515060425 3.0
1.2179598014 4.72395801544189 4.0
1.2179598014 4.90999126434326 5.0
-68.965013896 0.480157494544983 6.0
77.751383163 0.680214643478394 7.0
51.860571493 0.735079705715179 8.0
25.89081167 0.603084027767181 9.0
89.361478567 0.979093432426453 10.0
75.680571447 0.0994596630334854 11.0
81.788241516 0.287791669368744 12.0
81.788241516 0.118634805083275 13.0
81.788241516 0.0958249345421791 14.0
0 0.0605518370866776 15.0
0 0.20882511138916 16.0
0 1.3980278968811 17.0
54.210182982 1.19023036956787 18.0
-54.210182982 1.19681668281555 19.0
-54.210182982 0.394293129444122 20.0
-54.210182982 0.222500503063202 21.0
74.018933663 2.68121147155762 22.0
141.07967112 1.81457722187042 23.0
107.01938142 0.719293594360352 24.0
25.900492052 0.062526136636734 25.0
215.09860478 1.91572320461273 26.0
215.09860478 3.03375697135925 27.0
27.071822658 2.77840137481689 28.0
242.17042744 1.33060228824615 29.0
242.17042744 1.85283505916595 30.0
46.582071302 0.94778311252594 31.0
9.4969752955 0.479353576898575 32.0
19.808750681 0.916918456554413 33.0
19.808750681 1.27161002159119 34.0
19.808750681 0.129441022872925 35.0
9.010096637 0.0267053972929716 36.0
7.7101834916 0.385790556669235 37.0
0 0.0246878191828728 38.0
0 0.251099765300751 39.0
-12.399512853 0.718083560466766 40.0
-25.478637884 0.902905583381653 41.0
4.5350293447 0.374787539243698 42.0
-32.877537447 0.311245024204254 43.0
91.024120452 0.640921592712402 44.0
91.024120452 0.0745090469717979 45.0
6.375809671 0.585412859916687 46.0
2.1178318099 0.405969709157944 47.0
195.31953958 0.0265673492103815 48.0
188.81450118 0.0799446925520897 49.0
1.2179598014 0.29123255610466 50.0
77.751383163 0.228587493300438 51.0
89.361478567 0.0124491062015295 52.0
81.788241516 0.0400737561285496 53.0
81.788241516 0.0756785273551941 54.0
81.788241516 0.0428612120449543 55.0
0 5.02887058258057 56.0
74.018933663 0.0986891984939575 57.0
215.09860478 1.01427149772644 58.0
242.17042744 0.544285178184509 59.0
19.808750681 -0.0165715366601944 60.0
0 0.0764218866825104 61.0
193.14718483 0.783024907112122 0.0
193.14718483 1.72598934173584 1.0
192.08332723 1.45601320266724 2.0
187.29425727 6.32593584060669 3.0
0 4.76578235626221 4.0
0 3.98849844932556 5.0
-50.211137408 0.348433375358582 6.0
59.912384342 0.595538377761841 7.0
55.946791641 1.00749588012695 8.0
3.9655927003 1.06931746006012 9.0
94.133190693 0.978513062000275 10.0
80.128882451 0.218165352940559 11.0
87.060211261 0.797956764698029 12.0
87.060211261 0.796933352947235 13.0
87.060211261 0.0650246143341064 14.0
0 0.178295627236366 15.0
0 0.217452764511108 16.0
0 1.26624858379364 17.0
49.823661078 0.830162227153778 18.0
-49.823661078 0.796838998794556 19.0
-49.823661078 0.354008287191391 20.0
-49.823661078 -0.0832628533244133 21.0
68.628491369 1.78204751014709 22.0
138.62159563 2.09455013275146 23.0
104.55643212 0.539377331733704 24.0
22.777928081 0.162173122167587 25.0
207.250087 0.872008323669434 26.0
207.250087 1.4834520816803 27.0
26.412633652 1.43610727787018 28.0
233.66272066 0.67841374874115 29.0
233.66272066 1.73098409175873 30.0
47.207034863 1.01226997375488 31.0
9.4462145562 0.547782242298126 32.0
18.804830291 2.27855944633484 33.0
18.804830291 1.04744493961334 34.0
18.804830291 0.126581251621246 35.0
8.3952562361 0.0340658277273178 36.0
7.2861761935 0.397944748401642 37.0
0 -0.00564434379339218 38.0
0 0.0720871835947037 39.0
-12.327611624 0.683910965919495 40.0
-23.982955808 0.708022475242615 41.0
4.5911628167 0.318877249956131 42.0
-30.965736753 0.329142361879349 43.0
86.212519074 0.826562523841858 44.0
86.212519074 0.1251600086689 45.0
6.6700883013 0.683707296848297 46.0
1.8604601761 0.403459906578064 47.0
193.14718483 0.101003892719746 48.0
187.29425727 0.360751152038574 49.0
0 0.236928075551987 50.0
59.912384342 0.308108031749725 51.0
94.133190693 0.512913286685944 52.0
87.060211261 0.0116727510467172 53.0
87.060211261 0.219607815146446 54.0
87.060211261 0.209355667233467 55.0
0 3.52529096603394 56.0
68.628491369 0.0895197689533234 57.0
207.250087 0.329517066478729 58.0
233.66272066 0.486092895269394 59.0
18.804830291 0.0157792065292597 60.0
0 0.113029010593891 61.0
156.81929262 0.965441346168518 0.0
156.81929262 0.709591448307037 1.0
155.52513543 1.28225553035736 2.0
150.73603023 4.99066972732544 3.0
0 6.84990358352661 4.0
0 4.85643720626831 5.0
-67.924529922 3.02876281738281 6.0
77.331033667 0.569087386131287 7.0
46.507781837 0.790038704872131 8.0
30.82325183 0.704250872135162 9.0
80.721765067 1.12916362285614 10.0
68.078386221 0.143207162618637 11.0
78.16577212 0.663641631603241 12.0
78.16577212 0.757367730140686 13.0
78.16577212 0.0758127868175507 14.0
0 0.219957455992699 15.0
0 0.163492470979691 16.0
0 1.48178601264954 17.0
57.27254384 1.03948736190796 18.0
-57.27254384 0.869146764278412 19.0
-57.27254384 0.539531588554382 20.0
-57.27254384 -0.0334808677434921 21.0
79.251686091 1.45736455917358 22.0
136.92703621 0.758731603622437 23.0
100.44229889 0.653126835823059 24.0
22.083040794 0.0805917084217072 25.0
216.1787223 1.16359460353851 26.0
216.1787223 1.90036821365356 27.0
29.36873717 1.89102268218994 28.0
245.54745947 1.38118624687195 29.0
245.54745947 1.77346765995026 30.0
47.018148976 0.966794729232788 31.0
9.2884593608 0.45322647690773 32.0
21.979142251 1.2117840051651 33.0
21.979142251 1.07716953754425 34.0
21.979142251 0.161409750580788 35.0
9.7046846753 0.0356924906373024 36.0
8.3548121364 0.376420497894287 37.0
0 -0.0131385996937752 38.0
0 0.222893103957176 39.0
-11.890988666 0.489853948354721 40.0
-24.690055469 0.383187144994736 41.0
5.0066108392 0.644938468933105 42.0
-32.215905428 0.582117438316345 43.0
99.807585005 0.664717137813568 44.0
99.807585005 0.0591956973075867 45.0
7.1990207178 0.478781491518021 46.0
2.6768460191 0.359677791595459 47.0
156.81929262 0.126118496060371 48.0
150.73603023 0.071179136633873 49.0
0 0.433015644550323 50.0
77.331033667 0.102313354611397 51.0
80.721765067 0.358478546142578 52.0
78.16577212 0.0555048137903214 53.0
78.16577212 0.103376850485802 54.0
78.16577212 0.305311501026154 55.0
0 3.26436686515808 56.0
79.251686091 0.0843085646629333 57.0
216.1787223 0.70440000295639 58.0
245.54745947 1.17421436309814 59.0
21.979142251 -0.0055554136633873 60.0
0 0.136212408542633 61.0
183.94584872 0.979091346263885 0.0
183.94584872 2.29657077789307 1.0
182.74095161 1.73253047466278 2.0
177.27628095 4.07621240615845 3.0
15.840025131 5.00799655914307 4.0
15.840025131 4.61182308197021 5.0
-63.463095413 2.49529218673706 6.0
76.92157549 0.654873549938202 7.0
40.931742525 0.568741858005524 8.0
35.989832965 0.573474228382111 9.0
92.039329114 0.993153214454651 10.0
79.671321804 0.122463054955006 11.0
73.76487421 0.979827225208282 12.0
73.76487421 0.787233948707581 13.0
73.76487421 0.00203343550674617 14.0
0 0.0216024722903967 15.0
0 0.259370744228363 16.0
0 1.3544385433197 17.0
55.299964486 1.07434356212616 18.0
-55.299964486 1.57609796524048 19.0
-55.299964486 0.489881455898285 20.0
-55.299964486 0.0488696694374084 21.0
76.310384143 2.13144397735596 22.0
133.78756304 2.0073094367981 23.0
101.58274984 0.352943331003189 24.0
19.716147267 0.155344396829605 25.0
210.09794719 1.31073224544525 26.0
210.09794719 2.01885080337524 27.0
22.126690969 1.47739458084106 28.0
232.22463816 0.487186074256897 29.0
232.22463816 1.67467534542084 30.0
46.51724215 0.937833964824677 31.0
9.376548924 0.395271956920624 32.0
21.010419657 1.40689337253571 33.0
21.010419657 0.936508238315582 34.0
21.010419657 0.0789148658514023 35.0
9.643830595 0.0296354983001947 36.0
8.2752457839 0.361979722976685 37.0
0 0.018603190779686 38.0
0 0.175662010908127 39.0
-13.29727277 0.626204252243042 40.0
-26.505191228 0.830344080924988 41.0
5.2036554404 0.604876935482025 42.0
-34.347722838 0.890324234962463 43.0
97.231089806 0.653448939323425 44.0
97.231089806 0.0816736817359924 45.0
7.447950654 0.595730602741241 46.0
2.7478137532 0.381365239620209 47.0
183.94584872 0.0904674232006073 48.0
177.27628095 0.244941681623459 49.0
15.840025131 0.110894061625004 50.0
76.92157549 0.0754299834370613 51.0
92.039329114 0.161900922656059 52.0
73.76487421 0.0787764042615891 53.0
73.76487421 0.189316913485527 54.0
73.76487421 0.220474243164062 55.0
0 3.06908392906189 56.0
76.310384143 0.0704544186592102 57.0
210.09794719 0.357424944639206 58.0
232.22463816 0.729350447654724 59.0
21.010419657 -0.00233024172484875 60.0
0 0.0956708788871765 61.0
196.85508239 0.82664942741394 0.0
196.85508239 1.56942796707153 1.0
195.48574071 1.57652866840363 2.0
190.69667075 4.4567608833313 3.0
0 6.05758714675903 4.0
0 4.60667133331299 5.0
-66.813629247 0.429809212684631 6.0
79.468256358 0.954522132873535 7.0
56.598274917 0.407438904047012 8.0
22.869981442 0.365233451128006 9.0
88.371096638 1.10475718975067 10.0
74.824391099 0.229070380330086 11.0
84.872622236 0.68928849697113 12.0
84.872622236 0.900225579738617 13.0
84.872622236 0.0179159082472324 14.0
0 0.114069543778896 15.0
0 0.261189818382263 16.0
0 1.18410086631775 17.0
64.832869008 1.45306992530823 18.0
-64.832869008 0.987077713012695 19.0
-64.832869008 0.938958644866943 20.0
-64.832869008 -0.0291204694658518 21.0
85.368828116 2.4117488861084 22.0
151.17162187 3.19773864746094 23.0
113.04545121 0.25885221362114 24.0
23.359341056 0.137092500925064 25.0
236.54044998 2.00381088256836 26.0
236.54044998 2.51323366165161 27.0
24.140611496 1.85455322265625 28.0
260.68106148 0.791944980621338 29.0
260.68106148 1.49378907680511 30.0
45.938698442 1.10494220256805 31.0
9.1812254906 0.292584180831909 32.0
20.535959109 1.2649177312851 33.0
20.535959109 0.822381794452667 34.0
20.535959109 0.182202860713005 35.0
9.3031246856 0.0225858762860298 36.0
8.0275753404 0.351514250040054 37.0
0 0.0226664021611214 38.0
0 0.124376252293587 39.0
-12.546843124 0.983687400817871 40.0
-26.100767684 0.702895998954773 41.0
5.6799227451 0.340583354234695 42.0
-34.744675181 0.316616505384445 43.0
99.029457516 0.808649241924286 44.0
99.029457516 0.0433443039655685 45.0
6.187447527 0.638031959533691 46.0
1.8428222201 0.369861781597137 47.0
196.85508239 0.130673795938492 48.0
190.69667075 0.0156612768769264 49.0
0 0.19208200275898 50.0
79.468256358 0.217328831553459 51.0
88.371096638 0.570582330226898 52.0
84.872622236 0.067843958735466 53.0
84.872622236 0.236941456794739 54.0
84.872622236 0.20010669529438 55.0
0 1.72067391872406 56.0
85.368828116 0.0844174847006798 57.0
236.54044998 1.09856915473938 58.0
260.68106148 0.0775343179702759 59.0
20.535959109 -0.0206314958631992 60.0
0 0.130993813276291 61.0
160.06486877 0.818397641181946 0.0
160.06486877 0.677271366119385 1.0
158.89750065 1.42666840553284 2.0
153.43282994 4.20539617538452 3.0
4.5495481027 3.60127091407776 4.0
4.5495481027 3.51463913917542 5.0
-69.794918813 2.2003116607666 6.0
78.721117789 0.395384281873703 7.0
51.216801101 1.02253246307373 8.0
27.504316688 1.0541558265686 9.0
89.779000471 0.87784880399704 10.0
76.262314858 0.123528532683849 11.0
83.402737111 0.63584703207016 12.0
83.402737111 0.698414742946625 13.0
83.402737111 0.0819641351699829 14.0
0 0.188976913690567 15.0
0 0.205155581235886 16.0
0 1.17723643779755 17.0
60.692889343 2.10224437713623 18.0
-60.692889343 2.6156280040741 19.0
-60.692889343 0.264815598726273 20.0
-60.692889343 -0.0113986376672983 21.0
80.384952393 1.46028554439545 22.0
137.71222829 0.657477915287018 23.0
105.24511328 0.587619125843048 24.0
20.253356076 0.170437425374985 25.0
218.09718068 1.15265965461731 26.0
218.09718068 1.45637953281403 27.0
30.061697477 2.02181792259216 28.0
248.15887816 1.30844140052795 29.0
248.15887816 1.78070378303528 30.0
46.437116323 0.90589964389801 31.0
9.4055535229 0.452057272195816 32.0
19.69206305 1.62009513378143 33.0
19.69206305 1.36913669109344 34.0
19.69206305 0.160417497158051 35.0
8.4107284873 0.0337747633457184 36.0
7.235343924 0.36382731795311 37.0
0 0.0157148651778698 38.0
0 0.13726469874382 39.0
-11.185913047 0.606065809726715 40.0
-25.244543632 0.554683804512024 41.0
5.3815271043 0.630921959877014 42.0
-32.867830311 0.591485142707825 43.0
83.83167096 0.771915078163147 44.0
83.83167096 0.0900904983282089 45.0
6.9267707917 0.682212233543396 46.0
2.6577075269 0.407685309648514 47.0
160.06486877 0.143332049250603 48.0
153.43282994 0.248073980212212 49.0
4.5495481027 0.0509442947804928 50.0
78.721117789 0.166130512952805 51.0
89.779000471 0.267821907997131 52.0
83.402737111 0.0250589977949858 53.0
83.402737111 0.0918169990181923 54.0
83.402737111 0.187820971012115 55.0
0 4.4493579864502 56.0
80.384952393 0.10180327296257 57.0
218.09718068 0.440356642007828 58.0
248.15887816 1.13406372070312 59.0
19.69206305 -0.0236944761127234 60.0
0 0.105111598968506 61.0
194.16539108 1.0294075012207 0.0
194.16539108 0.55212014913559 1.0
192.71746487 0.778858661651611 2.0
187.25282873 3.35406136512756 3.0
0.61049808893 4.72774410247803 4.0
0.61049808893 2.87109589576721 5.0
-67.525035965 1.16565811634064 6.0
75.806096812 0.671655893325806 7.0
45.013322686 0.915393233299255 8.0
30.792774125 0.759207129478455 9.0
74.892802292 1.08434355258942 10.0
64.015296555 0.218587070703506 11.0
69.272791002 0.0933145880699158 12.0
69.272791002 0.0774841755628586 13.0
69.272791002 0.173575565218925 14.0
24.947118847 0.400311887264252 15.0
24.947118847 0.188053041696548 16.0
0 1.89865374565125 17.0
45.595263345 1.31545352935791 18.0
-45.595263345 1.40581095218658 19.0
-45.595263345 0.461874961853027 20.0
-45.595263345 0.0351917073130608 21.0
66.203634595 1.0316082239151 22.0
137.24467902 1.20214664936066 23.0
100.45630621 0.640825092792511 24.0
25.192263622 0.0788407027721405 25.0
203.44831361 1.25948715209961 26.0
203.44831361 2.1999785900116 27.0
26.076909608 2.18370890617371 28.0
229.52522322 1.5007655620575 29.0
229.52522322 1.99472236633301 30.0
46.810327166 1.15027797222137 31.0
9.316695914 0.253407418727875 32.0
20.60837125 1.45169270038605 33.0
20.60837125 0.642615556716919 34.0
20.60837125 0.11875806003809 35.0
8.8711038366 0.0194126758724451 36.0
7.4439686894 0.393112242221832 37.0
0 0.0515421256422997 38.0
0 0.132868126034737 39.0
-12.381130096 0.873818397521973 40.0
-26.945500357 0.56307065486908 41.0
5.2832853707 0.348817467689514 42.0
-35.178982168 0.307660937309265 43.0
83.688386862 0.585406839847565 44.0
83.688386862 0.056502178311348 45.0
6.4910328791 0.682780802249908 46.0
2.8888494258 0.457554638385773 47.0
194.16539108 0.111648947000504 48.0
187.25282873 0.259502470493317 49.0
0.61049808893 0.207106664776802 50.0
75.806096812 0.0947743803262711 51.0
74.892802292 0.213389873504639 52.0
69.272791002 0.0115415630862117 53.0
69.272791002 0.076768696308136 54.0
69.272791002 0.0755179822444916 55.0
0 0.548521399497986 56.0
66.203634595 -0.0080920048058033 57.0
203.44831361 0.873373508453369 58.0
229.52522322 0.841899037361145 59.0
20.60837125 0.0112603381276131 60.0
0 0.0764650404453278 61.0
138.27773802 0.949220299720764 0.0
138.27773802 0.777035057544708 1.0
137.21961409 1.1214839220047 2.0
132.43050896 3.30002975463867 3.0
0 4.72163581848145 4.0
0 3.94188594818115 5.0
-67.394034895 1.82355380058289 6.0
80.043896146 0.84665060043335 7.0
54.627439134 0.633765041828156 8.0
25.416457011 0.72624397277832 9.0
83.639890749 1.03120386600494 10.0
70.350997599 0.114973552525043 11.0
83.659712154 1.04894864559174 12.0
83.659712154 0.0941628515720367 13.0
83.659712154 0.100666284561157 14.0
0 0.202373266220093 15.0
0 0.176114931702614 16.0
0 1.46514499187469 17.0
54.862655385 1.03742742538452 18.0
-54.862655385 0.8143190741539 19.0
-54.862655385 0.696144044399261 20.0
-54.862655385 -0.0688199922442436 21.0
74.845268792 2.0737726688385 22.0
134.2201432 1.93569111824036 23.0
103.70848778 0.653670251369476 24.0
25.168218491 0.0904339700937271 25.0
209.06541199 1.66036689281464 26.0
209.06541199 2.10537004470825 27.0
28.939196253 2.42145466804504 28.0
238.00460825 1.16282033920288 29.0
238.00460825 1.90577971935272 30.0
46.775634475 1.15484344959259 31.0
9.182999079 0.501729965209961 32.0
19.982613407 1.43481862545013 33.0
19.982613407 1.11305916309357 34.0
19.982613407 0.131642654538155 35.0
10.658391829 0.029894107952714 36.0
9.4442862449 0.381972521543503 37.0
0 0.0204356163740158 38.0
0 0.191774815320969 39.0
-11.454690471 0.603633165359497 40.0
-23.979837398 0.372231185436249 41.0
5.4209611469 0.546867370605469 42.0
-31.733062738 0.570149421691895 43.0
91.323186657 0.741871118545532 44.0
91.323186657 0.103476293385029 45.0
6.0427469352 0.62446653842926 46.0
2.536311808 0.370509207248688 47.0
138.27773802 0.251091063022614 48.0
132.43050896 0.256448000669479 49.0
0 0.283044278621674 50.0
80.043896146 0.0652010664343834 51.0
83.639890749 0.251257658004761 52.0
83.659712154 0.022483766078949 53.0
83.659712154 0.13218928873539 54.0
83.659712154 0.0584627464413643 55.0
0 3.21721315383911 56.0
74.845268792 0.142439395189285 57.0
209.06541199 0.454435020685196 58.0
238.00460825 0.535098612308502 59.0
19.982613407 0.0357333198189735 60.0
0 0.0689748972654343 61.0
150.28386013 1.05285429954529 0.0
150.28386013 0.904049873352051 1.0
148.79188019 0.8902907371521 2.0
144.00277489 2.5270721912384 3.0
0 3.59429454803467 4.0
0 3.03583192825317 5.0
-73.331567891 2.62381482124329 6.0
84.807830303 0.535209655761719 7.0
48.572931938 0.594899415969849 8.0
36.234898365 0.646476924419403 9.0
99.19959749 1.18455326557159 10.0
87.58623953 0.328201025724411 11.0
84.799752397 0.305357336997986 12.0
84.799752397 0.132271990180016 13.0
84.799752397 0.0478183999657631 14.0
0 0.120203964412212 15.0
0 0.186860650777817 16.0
0 1.25230884552002 17.0
51.020297372 1.26012909412384 18.0
-51.020297372 1.00780892372131 19.0
-51.020297372 0.555905699729919 20.0
-51.020297372 0.0535172820091248 21.0
72.03962753 1.28021574020386 22.0
140.40809747 0.814284443855286 23.0
100.72074461 0.730737864971161 24.0
21.648780269 0.0768751502037048 25.0
212.447725 0.693473517894745 26.0
212.447725 1.62388825416565 27.0
25.029954293 1.79289317131042 28.0
237.47767929 0.781592726707458 29.0
237.47767929 1.72839283943176 30.0
46.223457319 1.19629383087158 31.0
9.4672225573 0.683892250061035 32.0
21.019330158 1.3890985250473 33.0
21.019330158 1.25817847251892 34.0
21.019330158 0.137630000710487 35.0
9.3982771505 0.0310275554656982 36.0
8.1836596058 0.388675004243851 37.0
0 0.0263395570218563 38.0
0 0.0878064706921577 39.0
-11.571451982 0.503502190113068 40.0
-23.942210567 0.716919541358948 41.0
4.9664558406 0.43704280257225 42.0
-31.400465715 0.312663406133652 43.0
79.372219231 0.817238390445709 44.0
79.372219231 0.0937608331441879 45.0
6.4704063693 0.56857818365097 46.0
2.6371703832 0.340993672609329 47.0
150.28386013 0.171078398823738 48.0
144.00277489 -0.0282108914107084 49.0
0 0.19941346347332 50.0
84.807830303 0.159833639860153 51.0
99.19959749 0.43956246972084 52.0
84.799752397 0.0313625372946262 53.0
84.799752397 0.124768972396851 54.0
84.799752397 0.0825932249426842 55.0
0 3.11141538619995 56.0
72.03962753 0.0819134265184402 57.0
212.447725 0.244590073823929 58.0
237.47767929 0.203565895557404 59.0
21.019330158 -0.0102325584739447 60.0
0 0.104308441281319 61.0
199.52828214 1.17939746379852 0.0
199.52828214 0.82218861579895 1.0
198.18841947 0.956317782402039 2.0
192.72374877 5.72727060317993 3.0
5.4249141519 5.44846725463867 4.0
5.4249141519 3.06812620162964 5.0
-57.39356261 0.270477116107941 6.0
67.305857058 0.510063111782074 7.0
43.497616934 1.00931119918823 8.0
23.808240124 1.09202897548676 9.0
87.07131265 1.27476322650909 10.0
74.04531923 0.320827692747116 11.0
78.514425189 0.725132346153259 12.0
78.514425189 0.101801373064518 13.0
78.514425189 0.139450445771217 14.0
0 0.0837793797254562 15.0
0 0.151393428444862 16.0
0 1.55921113491058 17.0
61.785444634 1.58661496639252 18.0
-61.785444634 2.28426742553711 19.0
-61.785444634 0.274357855319977 20.0
-61.785444634 -0.0512201637029648 21.0
84.789259351 2.12817239761353 22.0
149.57218954 1.59872448444366 23.0
116.53037574 0.645840883255005 24.0
22.10057223 0.138597205281258 25.0
234.36144889 1.68386507034302 26.0
234.36144889 2.19789528846741 27.0
23.217723415 1.49284088611603 28.0
257.57917231 0.78520405292511 29.0
257.57917231 1.74536693096161 30.0
46.6790957 1.02121686935425 31.0
9.3153617141 0.603486716747284 32.0
23.003814717 1.59835720062256 33.0
23.003814717 1.01657569408417 34.0
23.003814717 0.133647605776787 35.0
10.663244027 0.0303923077881336 36.0
9.3371690856 0.389439582824707 37.0
0 -0.000824496150016785 38.0
0 0.131435126066208 39.0
-10.597251546 0.606515645980835 40.0
-22.729777546 0.434205621480942 41.0
4.9269494458 0.35075369477272 42.0
-30.267295635 0.534178733825684 43.0
93.084877185 0.58466762304306 44.0
93.084877185 0.107093833386898 45.0
7.0748729692 0.390214562416077 46.0
2.6873455054 0.357179284095764 47.0
199.52828214 0.102146945893764 48.0
192.72374877 0.113115310668945 49.0
5.4249141519 0.148736238479614 50.0
67.305857058 0.224087536334991 51.0
87.07131265 0.328031539916992 52.0
78.514425189 0.0817227736115456 53.0
78.514425189 0.0487117618322372 54.0
78.514425189 0.0352613292634487 55.0
0 5.26909589767456 56.0
84.789259351 0.172187328338623 57.0
234.36144889 0.822195768356323 58.0
257.57917231 0.490424036979675 59.0
23.003814717 -0.00654192268848419 60.0
0 0.0992669016122818 61.0
128.96126508 1.09739279747009 0.0
128.96126508 2.68183922767639 1.0
127.70281088 2.13878798484802 2.0
122.23816187 2.50990295410156 3.0
9.1868240653 2.26843214035034 4.0
9.1868240653 6.17772150039673 5.0
-70.376807237 1.37546122074127 6.0
81.97277551 0.848000049591064 7.0
56.664720204 0.917315006256104 8.0
25.308055306 1.09591817855835 9.0
102.48562789 1.21645700931549 10.0
88.813446577 0.225848332047462 11.0
83.354679403 0.430656760931015 12.0
83.354679403 1.03842377662659 13.0
83.354679403 0.0382899940013885 14.0
0 0.212083578109741 15.0
0 0.154062807559967 16.0
0 1.4541552066803 17.0
48.274056026 1.79000627994537 18.0
-48.274056026 1.17764174938202 19.0
-48.274056026 0.495701789855957 20.0
-48.274056026 0.179372847080231 21.0
69.891683574 0.629010319709778 22.0
149.21640738 0.784119427204132 23.0
108.86618469 0.584524154663086 24.0
24.123735242 0.202816188335419 25.0
219.10809095 1.25133275985718 26.0
219.10809095 0.899641275405884 27.0
31.316669337 1.3851226568222 28.0
250.42476029 0.949287056922913 29.0
250.42476029 1.57275080680847 30.0
46.372028859 1.00840675830841 31.0
9.3413537193 0.494495183229446 32.0
21.617627548 1.03902685642242 33.0
21.617627548 0.947120189666748 34.0
21.617627548 0.0848431959748268 35.0
9.6473515227 0.0301856007426977 36.0
8.2234926528 0.326122462749481 37.0
0 0.0461778156459332 38.0
0 0.222303360700607 39.0
-10.343385199 0.506421983242035 40.0
-24.260144678 0.769343912601471 41.0
4.661497492 0.356225490570068 42.0
-31.871626417 0.352290719747543 43.0
80.008150938 0.661894798278809 44.0
80.008150938 0.0879573673009872 45.0
8.0285841687 0.637698650360107 46.0
2.870465416 0.397054612636566 47.0
128.96126508 0.09004145860672 48.0
122.23816187 0.258747696876526 49.0
9.1868240653 0.595488369464874 50.0
81.97277551 0.169710144400597 51.0
102.48562789 0.113493233919144 52.0
83.354679403 0.0570915564894676 53.0
83.354679403 0.0356492511928082 54.0
83.354679403 0.462773203849792 55.0
0 1.08782804012299 56.0
69.891683574 0.020122867077589 57.0
219.10809095 0.360700219869614 58.0
250.42476029 0.55275297164917 59.0
21.617627548 0.0249439999461174 60.0
0 0.0791199281811714 61.0
200.50511319 0.946156322956085 0.0
200.50511319 0.802234530448914 1.0
199.3565896 0.542282700538635 2.0
194.56748432 3.08126926422119 3.0
0 7.25405263900757 4.0
0 5.78304290771484 5.0
-71.129620123 6.17838191986084 6.0
81.933519922 0.930708825588226 7.0
43.943700551 0.413114964962006 8.0
37.989819371 0.305204391479492 9.0
80.648935228 0.903057813644409 10.0
68.087828508 0.0601286217570305 11.0
80.743662816 0.524738907814026 12.0
80.743662816 0.823196589946747 13.0
80.743662816 0.0889862701296806 14.0
0 0.111194536089897 15.0
0 0.18835324048996 16.0
0 0.897337436676025 17.0
49.543419705 1.07347643375397 18.0
-49.543419705 0.725887954235077 19.0
-49.543419705 0.766730964183807 20.0
-49.543419705 -0.0529446452856064 21.0
69.731457294 1.78132033348083 22.0
143.94333787 1.24942684173584 23.0
110.43613168 0.723378717899323 24.0
16.529696599 0.0895949900150299 25.0
213.67479517 1.33925139904022 26.0
213.67479517 2.06102895736694 27.0
25.384091389 1.10892820358276 28.0
239.05888656 0.922905743122101 29.0
239.05888656 1.6035829782486 30.0
46.391321646 1.06485652923584 31.0
9.4608138341 0.565470993518829 32.0
20.188037589 1.5539014339447 33.0
20.188037589 1.10416841506958 34.0
20.188037589 0.141812920570374 35.0
9.3199385897 0.055096410214901 36.0
8.1281406233 0.386618405580521 37.0
0 0.0137543715536594 38.0
0 0.161838918924332 39.0
-11.358748821 0.347039610147476 40.0
-26.017057934 0.477725207805634 41.0
5.2657339699 0.465701937675476 42.0
-33.833211624 0.421853750944138 43.0
85.462418241 0.702116310596466 44.0
85.462418241 0.084495447576046 45.0
6.5664453678 0.614159166812897 46.0
2.5828963497 0.470938742160797 47.0
200.50511319 0.26042777299881 48.0
194.56748432 0.114750102162361 49.0
0 0.24446514248848 50.0
81.933519922 0.173039689660072 51.0
80.648935228 0.0850826650857925 52.0
80.743662816 0.0145014831796288 53.0
80.743662816 0.371549010276794 54.0
80.743662816 0.0119533780962229 55.0
0 4.02394104003906 56.0
69.731457294 -0.0368688181042671 57.0
213.67479517 0.458570778369904 58.0
239.05888656 0.149097293615341 59.0
20.188037589 0.0333386808633804 60.0
0 0.142969131469727 61.0
190.03688864 1.04181778430939 0.0
190.03688864 2.22935819625854 1.0
188.4666753 2.39167857170105 2.0
183.0020047 5.17791128158569 3.0
20.838047752 9.80800151824951 4.0
20.838047752 7.24748992919922 5.0
-61.708595949 4.03084659576416 6.0
71.302999697 0.455475211143494 7.0
43.106641484 0.708271384239197 8.0
28.196358213 0.863152384757996 9.0
97.205071136 1.10607290267944 10.0
84.240219862 0.0760601386427879 11.0
84.224284308 0.215013146400452 12.0
84.224284308 0.0803133547306061 13.0
84.224284308 0.106899380683899 14.0
0 0.119397796690464 15.0
0 0.178296521306038 16.0
0 1.67038559913635 17.0
60.445720626 0.888702690601349 18.0
-60.445720626 0.691141068935394 19.0
-60.445720626 0.674779295921326 20.0
-60.445720626 0.016943134367466 21.0
80.256632549 2.54683828353882 22.0
146.76376851 1.63974440097809 23.0
111.49046573 0.627425312995911 24.0
21.939301867 0.182003796100616 25.0
227.02040106 0.922132849693298 26.0
227.02040106 1.55447196960449 27.0
23.533984036 1.65485632419586 28.0
250.5543851 1.15755176544189 29.0
250.5543851 1.60324561595917 30.0
47.303416591 0.897712469100952 31.0
9.5033473803 0.378542542457581 32.0
19.810911924 0.953210294246674 33.0
19.810911924 1.41652846336365 34.0
19.810911924 0.0282786376774311 35.0
8.6332412258 0.0227292254567146 36.0
7.1631773211 0.403130352497101 37.0
0 0.0462836623191833 38.0
0 0.165745630860329 39.0
-12.509063731 0.534508049488068 40.0
-27.012300554 0.617147088050842 41.0
5.1854800554 0.42763677239418 42.0
-34.41269547 0.222141742706299 43.0
86.922151403 0.8194260597229 44.0
86.922151403 0.0577439740300179 45.0
6.9456846816 0.630073070526123 46.0
3.2087886729 0.426999241113663 47.0
190.03688864 0.0890879333019257 48.0
183.0020047 0.316477328538895 49.0
20.838047752 1.00852620601654 50.0
71.302999697 0.176687553524971 51.0
97.205071136 0.288579523563385 52.0
84.224284308 0.0167182609438896 53.0
84.224284308 0.079895056784153 54.0
84.224284308 0.0834599435329437 55.0
0 5.19246387481689 56.0
80.256632549 0.0266863070428371 57.0
227.02040106 0.374475866556168 58.0
250.5543851 0.720804870128632 59.0
19.810911924 -0.00526565313339233 60.0
0 0.0670428872108459 61.0
137.19636913 0.845970511436462 0.0
137.19636913 0.474892139434814 1.0
136.09404456 1.1270580291748 2.0
131.30493936 3.09391140937805 3.0
0 4.8679370880127 4.0
0 2.13396072387695 5.0
-51.570763056 0.0107938796281815 6.0
65.264444408 0.936794817447662 7.0
54.892673617 0.600591838359833 8.0
10.371770791 0.830837666988373 9.0
92.231825944 0.987132012844086 10.0
79.372372687 0.620524525642395 11.0
83.094358673 0.887661635875702 12.0
83.094358673 0.119211830198765 13.0
83.094358673 0.12121195346117 14.0
0 0.276651114225388 15.0
0 0.127735912799835 16.0
0 1.55732822418213 17.0
53.37360292 1.72629058361053 18.0
-53.37360292 1.42592692375183 19.0
-53.37360292 0.361057013273239 20.0
-53.37360292 0.0223911739885807 21.0
73.915408037 0.966659784317017 22.0
146.29016526 0.89051753282547 23.0
114.38710602 0.266703397035599 24.0
20.039170238 0.212690755724907 25.0
220.2055733 0.909189939498901 26.0
220.2055733 1.28267276287079 27.0
28.267052015 1.27331709861755 28.0
248.47262532 0.455386102199554 29.0
248.47262532 1.35044419765472 30.0
46.352262284 1.09810864925385 31.0
9.4192632365 0.43897071480751 32.0
20.541805117 1.33305478096008 33.0
20.541805117 0.692033886909485 34.0
20.541805117 0.0631650239229202 35.0
9.1280589922 0.0233836285769939 36.0
7.9152946896 0.393987864255905 37.0
0 0.0402316749095917 38.0
0 0.237958520650864 39.0
-11.514401886 0.465618431568146 40.0
-23.644990398 0.962510883808136 41.0
5.2883746871 0.3463134765625 42.0
-31.197538025 0.489504635334015 43.0
103.52004751 0.767549812793732 44.0
103.52004751 0.0348183736205101 45.0
6.6649704288 0.769891321659088 46.0
2.1570753131 0.386530458927155 47.0
137.19636913 0.11803212761879 48.0
131.30493936 0.334840536117554 49.0
0 0.172389462590218 50.0
65.264444408 0.283990651369095 51.0
92.231825944 0.594978868961334 52.0
83.094358673 0.17237251996994 53.0
83.094358673 0.101175114512444 54.0
83.094358673 0.0592706054449081 55.0
0 2.78820180892944 56.0
73.915408037 0.185514494776726 57.0
220.2055733 0.689423084259033 58.0
248.47262532 0.371272593736649 59.0
20.541805117 0.00395771861076355 60.0
0 0.0835714265704155 61.0
157.66906819 0.939101934432983 0.0
157.66906819 1.01951730251312 1.0
156.76363162 1.46505212783813 2.0
151.9745264 3.37781691551208 3.0
0 7.01192235946655 4.0
0 2.81579113006592 5.0
-56.763110347 0.433375686407089 6.0
69.046395134 0.664990484714508 7.0
66.749505613 0.672814726829529 8.0
2.2968895211 0.891721069812775 9.0
77.743293011 0.971079766750336 10.0
64.805225639 0.0733703523874283 11.0
72.515903619 0.137997671961784 12.0
72.515903619 0.30330616235733 13.0
72.515903619 0.117999687790871 14.0
0 0.0638168901205063 15.0
0 0.144611582159996 16.0
0 1.16610324382782 17.0
62.23132548 0.921266198158264 18.0
-62.23132548 0.503857254981995 19.0
-62.23132548 0.914037585258484 20.0
-62.23132548 -0.0768212080001831 21.0
85.114426558 1.72010767459869 22.0
150.3706634 0.851625502109528 23.0
113.18965493 0.463449835777283 24.0
24.15664065 0.337796479463577 25.0
235.48508995 1.42175078392029 26.0
235.48508995 1.20931696891785 27.0
30.319457748 1.30651187896729 28.0
265.8045477 1.06240177154541 29.0
265.8045477 2.01823306083679 30.0
46.018969247 1.04534780979156 31.0
9.3382120896 0.455519616603851 32.0
22.883101078 1.41965270042419 33.0
22.883101078 0.85104513168335 34.0
22.883101078 0.108881048858166 35.0
9.8475455264 0.0388539060950279 36.0
8.4790705414 0.347183555364609 37.0
0 0.0380052253603935 38.0
0 0.256100416183472 39.0
-10.191478494 0.673469603061676 40.0
-23.456448401 0.721529304981232 41.0
4.6837084113 0.293011784553528 42.0
-30.794587465 0.304401040077209 43.0
98.735028231 0.747712969779968 44.0
98.735028231 0.0877273306250572 45.0
5.8092428963 0.460669130086899 46.0
1.7778651368 0.334365397691727 47.0
157.66906819 0.10738268494606 48.0
151.9745264 0.205884054303169 49.0
0 0.127093866467476 50.0
69.046395134 0.0351893045008183 51.0
77.743293011 0.0305693596601486 52.0
72.515903619 0.0101770050823689 53.0
72.515903619 0.179906204342842 54.0
72.515903619 0.0471243672072887 55.0
0 5.92944478988647 56.0
85.114426558 0.0709283500909805 57.0
235.48508995 0.314549624919891 58.0
265.8045477 1.36936283111572 59.0
22.883101078 -0.0071795922704041 60.0
0 0.0946623757481575 61.0
128.18387392 0.913895666599274 0.0
128.18387392 1.33326852321625 1.0
126.99544174 1.42953634262085 2.0
121.53077095 3.23265314102173 3.0
1.2738691641 7.49942970275879 4.0
1.2738691641 3.6697826385498 5.0
-73.088761252 -0.0198322981595993 6.0
86.237653801 0.743093371391296 7.0
42.13830218 1.20382118225098 8.0
44.099351621 1.03228664398193 9.0
89.375526986 1.01705551147461 10.0
77.632263423 0.350213050842285 11.0
81.984575678 0.301141232252121 12.0
81.984575678 0.124748893082142 13.0
81.984575678 0.0987832546234131 14.0
0 0.179944798350334 15.0
0 0.219079792499542 16.0
0 1.33364176750183 17.0
59.534118935 1.71014821529388 18.0
-59.534118935 1.77545654773712 19.0
-59.534118935 0.69940984249115 20.0
-59.534118935 -0.0362292006611824 21.0
80.811518139 2.74421215057373 22.0
144.32867971 2.38825559616089 23.0
107.65171823 0.4799624979496 24.0
22.376636389 0.11288894712925 25.0
225.14019785 1.68900167942047 26.0
225.14019785 1.69338309764862 27.0
24.464445145 1.63997280597687 28.0
249.60464299 0.542927742004395 29.0
249.60464299 1.61237049102783 30.0
46.821766789 1.13018620014191 31.0
9.3302719087 0.527182579040527 32.0
21.277399204 1.3804634809494 33.0
21.277399204 0.736553251743317 34.0
21.277399204 0.0854860618710518 35.0
8.6385693854 0.0202639643102884 36.0
7.4771028173 0.433916211128235 37.0
0 0.0622471198439598 38.0
0 0.136494353413582 39.0
-12.230205318 0.998168230056763 40.0
-24.97095309 0.531702399253845 41.0
5.0008760192 0.623425722122192 42.0
-32.719624319 0.713759779930115 43.0
76.682680164 0.471721112728119 44.0
76.682680164 0.0986219793558121 45.0
6.3574613838 0.600400149822235 46.0
2.6649901354 0.363327831029892 47.0
128.18387392 0.123736977577209 48.0
121.53077095 0.150753930211067 49.0
1.2738691641 0.208880320191383 50.0
86.237653801 0.372957348823547 51.0
89.375526986 0.10514285415411 52.0
81.984575678 0.090088851749897 53.0
81.984575678 0.0935482531785965 54.0
81.984575678 0.0299347154796124 55.0
0 3.82553720474243 56.0
80.811518139 -0.0316525362432003 57.0
225.14019785 0.708714723587036 58.0
249.60464299 0.100648678839207 59.0
21.277399204 -0.00416187196969986 60.0
0 0.11661285161972 61.0
141.23596047 0.848983407020569 0.0
141.23596047 0.823598265647888 1.0
139.83876225 1.66209399700165 2.0
135.04965708 4.62677955627441 3.0
0 6.97501850128174 4.0
0 5.70720386505127 5.0
-59.503672113 0.530926704406738 6.0
68.39283117 0.442650318145752 7.0
40.138893468 1.24074280261993 8.0
28.253937701 1.30912280082703 9.0
82.065589957 0.958489716053009 10.0
70.177271356 0.221063822507858 11.0
80.871170663 0.997352659702301 12.0
80.871170663 1.01959896087646 13.0
80.871170663 0.0991364642977715 14.0
0 0.226562008261681 15.0
0 0.154303133487701 16.0
0 1.68726575374603 17.0
62.598231657 1.20606410503387 18.0
-62.598231657 1.48118841648102 19.0
-62.598231657 0.45232093334198 20.0
-62.598231657 -0.0625781789422035 21.0
84.713537271 2.046222448349 22.0
142.72255073 1.20606625080109 23.0
108.63275458 0.329988151788712 24.0
21.596388365 0.231147095561028 25.0
227.436088 0.934530675411224 26.0
227.436088 1.27510237693787 27.0
26.837793816 1.46188712120056 28.0
254.27388181 0.43543267250061 29.0
254.27388181 1.51226532459259 30.0
46.314383014 0.985736608505249 31.0
9.4183138503 0.436170220375061 32.0
22.115305615 1.86031925678253 33.0
22.115305615 0.954313397407532 34.0
22.115305615 0.127594724297523 35.0
9.7784771363 0.0336060114204884 36.0
8.6307434043 0.331515967845917 37.0
0 0.0343941189348698 38.0
0 0.0640448033809662 39.0
-12.114583193 0.596387803554535 40.0
-25.52437579 0.435237556695938 41.0
4.8197479856 0.541233658790588 42.0
-32.610675664 0.465561807155609 43.0
87.401516855 0.890793979167938 44.0
87.401516855 0.0828240588307381 45.0
6.2880027127 0.666170597076416 46.0
2.496154252 0.374043166637421 47.0
141.23596047 0.14313668012619 48.0
135.04965708 0.496027678251266 49.0
0 0.378429681062698 50.0
68.39283117 0.197808474302292 51.0
82.065589957 0.187223643064499 52.0
80.871170663 0.060249324887991 53.0
80.871170663 0.359453290700912 54.0
80.871170663 0.282400995492935 55.0
0 2.63758945465088 56.0
84.713537271 0.133681654930115 57.0
227.436088 0.0956409722566605 58.0
254.27388181 0.103534087538719 59.0
22.115305615 -0.0132038220763206 60.0
0 0.0988567620515823 61.0
169.28280634 0.912599325180054 0.0
169.28280634 0.868115603923798 1.0
168.23513378 0.850864291191101 2.0
163.44606383 5.63928842544556 3.0
0 4.42576217651367 4.0
0 6.65064144134521 5.0
-75.106681581 4.13395309448242 6.0
87.63566979 0.771632373332977 7.0
62.515543298 0.807051777839661 8.0
25.120126492 0.930757999420166 9.0
93.759521801 1.02306175231934 10.0
81.449577209 0.302846431732178 11.0
86.77037343 0.175510391592979 12.0
86.77037343 0.0983591750264168 13.0
86.77037343 0.0402094721794128 14.0
0 0.134971186518669 15.0
0 0.2415821403265 16.0
0 0.889471888542175 17.0
60.564739202 0.625434041023254 18.0
-60.564739202 0.780023634433746 19.0
-60.564739202 0.324700117111206 20.0
-60.564739202 -0.0220949351787567 21.0
79.865111445 1.24550771713257 22.0
154.34492053 0.576768100261688 23.0
116.12367856 0.790408372879028 24.0
20.286297852 0.116942286491394 25.0
234.21003198 1.30689823627472 26.0
234.21003198 2.16511511802673 27.0
22.661630609 2.15949296951294 28.0
256.87166258 0.937775194644928 29.0
256.87166258 1.94414675235748 30.0
47.320902793 1.2246013879776 31.0
9.3323777782 0.487979054450989 32.0
19.300372243 1.5640743970871 33.0
19.300372243 0.591787040233612 34.0
19.300372243 0.076923280954361 35.0
9.5294664628 0.0205731596797705 36.0
8.2216388221 0.409541636705399 37.0
0 0.0350624583661556 38.0
0 0.0941254794597626 39.0
-11.404363339 0.67687064409256 40.0
-23.942604164 0.398970425128937 41.0
5.370868344 0.591608226299286 42.0
-32.06256814 0.66400420665741 43.0
82.633388458 0.562151432037354 44.0
82.633388458 0.0710279196500778 45.0
6.0503550697 0.562469661235809 46.0
2.6502358655 0.419421970844269 47.0
169.28280634 0.250194191932678 48.0
163.44606383 0.0412892065942287 49.0
0 0.298912763595581 50.0
87.63566979 0.186206698417664 51.0
93.759521801 0.400625765323639 52.0
86.77037343 0.0913313329219818 53.0
86.77037343 0.0579616725444794 54.0
86.77037343 0.0987913832068443 55.0
0 1.45849609375 56.0
79.865111445 0.0874070227146149 57.0
234.21003198 0.665114343166351 58.0
256.87166258 0.412892550230026 59.0
19.300372243 0.0480659380555153 60.0
0 0.0866220816969872 61.0
147.41645494 0.755227863788605 0.0
147.41645494 1.18172204494476 1.0
146.10323637 2.25527310371399 2.0
140.63856581 3.85930991172791 3.0
2.2224103242 3.63562273979187 4.0
2.2224103242 4.87143087387085 5.0
-70.926088482 0.705854713916779 6.0
81.660016385 0.527733623981476 7.0
41.887769268 1.14585399627686 8.0
39.772247116 1.06425964832306 9.0
92.141200039 0.735853970050812 10.0
79.118653499 0.0787635520100594 11.0
86.076725017 0.255670040845871 12.0
86.076725017 0.634496688842773 13.0
86.076725017 0.0874684080481529 14.0
0 0.138354390859604 15.0
0 0.200969025492668 16.0
0 0.82780796289444 17.0
59.829556866 1.43182241916656 18.0
-59.829556866 1.61421430110931 19.0
-59.829556866 0.480348318815231 20.0
-59.829556866 0.010166771709919 21.0
80.509100213 1.52831053733826 22.0
150.46366509 1.50265753269196 23.0
116.51041116 0.196961596608162 24.0
21.807929036 0.402468383312225 25.0
230.97276531 1.38088476657867 26.0
230.97276531 2.06990146636963 27.0
23.672067514 1.41621828079224 28.0
254.64483282 0.691338837146759 29.0
254.64483282 1.71085464954376 30.0
46.45115302 0.95773446559906 31.0
9.4029035731 0.567252337932587 32.0
20.679543347 1.15846574306488 33.0
20.679543347 1.0403870344162 34.0
20.679543347 0.117884784936905 35.0
9.5200571655 0.0426865369081497 36.0
8.2835788695 0.386834442615509 37.0
0 0.0141891501843929 38.0
0 0.271177858114243 39.0
-12.000265231 0.603214919567108 40.0
-24.172440832 0.663583695888519 41.0
5.2504618769 0.425439924001694 42.0
-31.961432687 0.620932936668396 43.0
102.38099329 0.649792671203613 44.0
102.38099329 0.10653455555439 45.0
6.3348404441 0.632381677627563 46.0
2.6085582061 0.398772120475769 47.0
147.41645494 0.0515895709395409 48.0
140.63856581 0.689285755157471 49.0
2.2224103242 0.283603876829147 50.0
81.660016385 0.171801820397377 51.0
92.141200039 0.0901412069797516 52.0
86.076725017 0.0566527768969536 53.0
86.076725017 0.155031159520149 54.0
86.076725017 0.115978635847569 55.0
0 3.56529378890991 56.0
80.509100213 0.156527653336525 57.0
230.97276531 0.429013103246689 58.0
254.64483282 0.75631844997406 59.0
20.679543347 0.0227142162621021 60.0
0 0.0927117615938187 61.0
164.31674261 0.887584745883942 0.0
164.31674261 1.95472967624664 1.0
162.9544872 2.18689823150635 2.0
158.16538208 6.04660654067993 3.0
0 7.65239000320435 4.0
0 6.23508071899414 5.0
-68.565749001 2.22677850723267 6.0
77.78931272 0.639743208885193 7.0
51.385387818 0.535200178623199 8.0
26.403924903 0.658410847187042 9.0
92.649723394 0.923675775527954 10.0
79.631258683 0.202773839235306 11.0
82.1995217 0.900466322898865 12.0
82.1995217 0.536517441272736 13.0
82.1995217 0.0753170847892761 14.0
0 0.210111483931541 15.0
0 0.212606772780418 16.0
0 0.9830162525177 17.0
53.92871744 0.896790683269501 18.0
-53.92871744 0.833941042423248 19.0
-53.92871744 0.285620331764221 20.0
-53.92871744 0.0387805961072445 21.0
74.3423093 1.76996088027954 22.0
136.78826848 1.63235437870026 23.0
98.77339726 0.411053746938705 24.0
21.29319951 0.166360288858414 25.0
211.13057778 1.1607506275177 26.0
211.13057778 1.78979051113129 27.0
25.253159997 2.0583930015564 28.0
236.38373777 0.677505135536194 29.0
236.38373777 1.56754970550537 30.0
46.191002711 1.1883819103241 31.0
9.3331413437 0.393504202365875 32.0
20.41359186 1.0813148021698 33.0
20.41359186 1.11477875709534 34.0
20.41359186 0.130874261260033 35.0
8.5244267322 0.0265024118125439 36.0
7.3332655534 0.346778601408005 37.0
0 0.017244927585125 38.0
0 0.0575629509985447 39.0
-11.804020802 0.593728244304657 40.0
-24.311019476 0.521733105182648 41.0
5.2631674774 0.61050683259964 42.0
-32.085112533 0.637859880924225 43.0
90.233165953 0.684516072273254 44.0
90.233165953 0.0754569843411446 45.0
7.1626956017 0.540320634841919 46.0
2.7383888278 0.369676411151886 47.0
164.31674261 0.125580087304115 48.0
158.16538208 0.483346909284592 49.0
0 0.52889621257782 50.0
77.78931272 0.167527347803116 51.0
92.649723394 0.254132211208344 52.0
82.1995217 0.12333907186985 53.0
82.1995217 0.157614693045616 54.0
82.1995217 0.143830135464668 55.0
0 3.24869847297668 56.0
74.3423093 0.0565247759222984 57.0
211.13057778 0.489170044660568 58.0
236.38373777 0.45070493221283 59.0
20.41359186 -0.0153653435409069 60.0
0 0.100422240793705 61.0
179.46144483 0.975566387176514 0.0
179.46144483 1.43828499317169 1.0
178.43466834 1.08206248283386 2.0
173.64556306 3.70479965209961 3.0
0 8.16022682189941 4.0
0 3.55174422264099 5.0
-70.298977325 1.63345742225647 6.0
82.763527322 1.1381653547287 7.0
62.470481477 1.02823436260223 8.0
20.293045845 0.964777886867523 9.0
93.860006438 1.1138162612915 10.0
79.602269819 0.436942458152771 11.0
84.63227534 0.329402387142181 12.0
84.63227534 0.379221111536026 13.0
84.63227534 0.112734325230122 14.0
0 0.210479348897934 15.0
0 0.211485847830772 16.0
0 1.14623415470123 17.0
56.413952129 1.18843913078308 18.0
-56.413952129 0.669744074344635 19.0
-56.413952129 1.18398201465607 20.0
-56.413952129 -0.0279383715242147 21.0
76.352846528 1.97109401226044 22.0
152.97106928 1.35547816753387 23.0
118.78223888 0.254698634147644 24.0
21.208639934 0.445261210203171 25.0
229.3239158 0.854781270027161 26.0
229.3239158 2.53555989265442 27.0
27.856392831 2.61337399482727 28.0
257.18030864 1.12200272083282 29.0
257.18030864 1.70281803607941 30.0
47.121979438 1.02489578723907 31.0
9.284016472 0.335914045572281 32.0
19.938894399 1.27329480648041 33.0
19.938894399 0.967757403850555 34.0
19.938894399 0.14757576584816 35.0
9.6919806556 0.0525770708918571 36.0
8.5150270479 0.371541410684586 37.0
0 0.0183342173695564 38.0
0 0.0479967072606087 39.0
-10.239964195 0.719763100147247 40.0
-24.12270379 0.339038133621216 41.0
5.5774404943 0.614449262619019 42.0
-32.593463752 0.826377868652344 43.0
83.747076454 0.594809532165527 44.0
83.747076454 0.081242524087429 45.0
7.3656435563 0.400258809328079 46.0
2.6514602321 0.270304352045059 47.0
179.46144483 0.17323711514473 48.0
173.64556306 0.0560270063579082 49.0
0 0.211790949106216 50.0
82.763527322 0.256287693977356 51.0
93.860006438 -0.00813169032335281 52.0
84.63227534 0.0565128363668919 53.0
84.63227534 0.0335389077663422 54.0
84.63227534 0.0854519605636597 55.0
0 4.2663722038269 56.0
76.352846528 -0.0502331219613552 57.0
229.3239158 0.524557709693909 58.0
257.18030864 0.542036354541779 59.0
19.938894399 0.000423116609454155 60.0
0 0.120824784040451 61.0
139.69419558 0.611161589622498 0.0
139.69419558 1.36360967159271 1.0
138.45282263 1.64094889163971 2.0
133.66373908 4.3279333114624 3.0
0 4.95878458023071 4.0
0 4.7400598526001 5.0
-69.799084676 1.35549235343933 6.0
79.83745273 0.485253930091858 7.0
57.885849997 1.19170641899109 8.0
21.951602734 1.28879535198212 9.0
89.52192247 0.817710161209106 10.0
77.03220818 0.404637545347214 11.0
88.516748148 0.681445062160492 12.0
88.516748148 0.482710897922516 13.0
88.516748148 0.0814454182982445 14.0
0 0.276622891426086 15.0
0 0.155079632997513 16.0
0 1.18105792999268 17.0
58.974106754 1.74088478088379 18.0
-58.974106754 1.65288007259369 19.0
-58.974106754 0.595358312129974 20.0
-58.974106754 0.00983722880482674 21.0
81.644440683 1.90740573406219 22.0
151.81293437 0.939768493175507 23.0
114.79889649 0.720291197299957 24.0
22.147675449 0.0741223618388176 25.0
233.45737505 0.950157165527344 26.0
233.45737505 1.74373114109039 27.0
28.290795615 2.22074007987976 28.0
261.74817067 0.700749635696411 29.0
261.74817067 1.86823928356171 30.0
46.267408569 1.13802742958069 31.0
9.3928177119 0.62244713306427 32.0
22.67033393 1.39274501800537 33.0
22.67033393 0.850337505340576 34.0
22.67033393 0.131674095988274 35.0
9.9172406935 0.0220898054540157 36.0
8.6800207024 0.388709425926208 37.0
0 0.0123522542417049 38.0
0 0.259725898504257 39.0
-12.614851884 0.440470695495605 40.0
-26.04218084 0.720940470695496 41.0
4.7363934201 0.353404402732849 42.0
-33.188698758 0.374194473028183 43.0
91.479822881 0.804971039295197 44.0
91.479822881 0.0381726250052452 45.0
7.2003601759 0.585130453109741 46.0
2.4154129099 0.354619592428207 47.0
139.69419558 0.0438710935413837 48.0
133.66373908 0.128801122307777 49.0
0 0.258530795574188 50.0
79.83745273 0.107730537652969 51.0
89.52192247 0.645006716251373 52.0
88.516748148 0.0292973071336746 53.0
88.516748148 0.13456030189991 54.0
88.516748148 0.175069898366928 55.0
0 2.56739401817322 56.0
81.644440683 0.0124043691903353 57.0
233.45737505 0.385338187217712 58.0
261.74817067 0.419854879379272 59.0
22.67033393 -0.00491936504840851 60.0
0 0.0955654233694077 61.0
148.75622293 0.876203656196594 0.0
148.75622293 1.19620394706726 1.0
147.45136565 1.84047663211823 2.0
142.66226041 3.67848110198975 3.0
0 5.02819156646729 4.0
0 5.03500604629517 5.0
-64.4086584 2.71925711631775 6.0
74.22425649 0.90860652923584 7.0
59.392160982 0.798211395740509 8.0
14.832095508 0.731339395046234 9.0
85.044571693 0.787548542022705 10.0
71.951755012 0.0110736396163702 11.0
84.779055243 0.915188848972321 12.0
84.779055243 0.541322112083435 13.0
84.779055243 0.134635791182518 14.0
0 0.355887204408646 15.0
0 0.143646508455276 16.0
0 1.50676155090332 17.0
56.559563704 1.24050354957581 18.0
-56.559563704 0.840222537517548 19.0
-56.559563704 0.977082192897797 20.0
-56.559563704 -0.0798175260424614 21.0
79.491380251 1.36942040920258 22.0
144.59428537 0.518086612224579 23.0
112.04194118 0.900907337665558 24.0
20.859189098 0.135938599705696 25.0
224.08566562 0.928812623023987 26.0
224.08566562 1.80737042427063 27.0
23.398917872 2.0029149055481 28.0
247.48458349 1.17292416095734 29.0
247.48458349 1.97870492935181 30.0
47.027869188 1.12691390514374 31.0
9.4379740785 0.581648468971252 32.0
22.931816547 1.65482342243195 33.0
22.931816547 1.34187281131744 34.0
22.931816547 0.106727920472622 35.0
10.336981585 0.0283927787095308 36.0
9.0515347659 0.331636160612106 37.0
0 0.00607884675264359 38.0
0 0.157079845666885 39.0
-12.025082582 0.709743499755859 40.0
-24.944422245 0.696602582931519 41.0
5.6297924446 0.421959698200226 42.0
-32.931606634 0.600868463516235 43.0
93.92656708 0.821179747581482 44.0
93.92656708 0.0527767390012741 45.0
6.862585222 0.54301655292511 46.0
2.8066938136 0.308218210935593 47.0
148.75622293 0.176286101341248 48.0
142.66226041 0.17321103811264 49.0
0 0.31738018989563 50.0
74.22425649 0.186737582087517 51.0
85.044571693 0.0368080586194992 52.0
84.779055243 0.0273077338933945 53.0
84.779055243 0.231550052762032 54.0
84.779055243 0.0460981018841267 55.0
0 2.27557587623596 56.0
79.491380251 0.157164752483368 57.0
224.08566562 0.539258658885956 58.0
247.48458349 0.573727011680603 59.0
22.931816547 -0.016185600310564 60.0
0 0.114263281226158 61.0
170.30839767 0.999564647674561 0.0
170.30839767 0.603300809860229 1.0
169.00232691 0.892920315265656 2.0
163.53765613 2.94766974449158 3.0
15.149228027 7.20551252365112 4.0
15.149228027 8.22354125976562 5.0
-61.168282947 1.90181124210358 6.0
71.686880422 0.954420566558838 7.0
45.981728086 1.08479011058807 8.0
25.705152336 1.18599963188171 9.0
80.478305002 1.08277177810669 10.0
69.044453933 0.0198877677321434 11.0
84.52996236 0.75978422164917 12.0
84.52996236 0.348864912986755 13.0
84.52996236 0.0898339003324509 14.0
0 0.233245968818665 15.0
0 0.197761714458466 16.0
0 1.56994926929474 17.0
55.446883569 1.73856103420258 18.0
-55.446883569 1.99848783016205 19.0
-55.446883569 0.194768756628036 20.0
-55.446883569 0.0289889536798 21.0
75.954137748 2.88031458854675 22.0
157.33619877 3.13945055007935 23.0
117.40583821 0.203682363033295 24.0
26.847953873 0.526420593261719 25.0
233.29033652 0.917598366737366 26.0
233.29033652 1.05876874923706 27.0
24.839491918 1.06816327571869 28.0
258.12982844 0.367261528968811 29.0
258.12982844 1.37727093696594 30.0
47.185960019 1.06817126274109 31.0
9.3927707279 0.667375385761261 32.0
20.507254179 1.48465096950531 33.0
20.507254179 0.77190375328064 34.0
20.507254179 0.117093749344349 35.0
9.817391047 0.0191814452409744 36.0
8.4538744813 0.466438144445419 37.0
0 0.00202008709311485 38.0
0 0.2049680352211 39.0
-11.062687319 0.723349630832672 40.0
-22.664231739 0.498244047164917 41.0
5.4027201047 0.421391904354095 42.0
-30.946551084 0.388703882694244 43.0
80.308909637 0.501316905021667 44.0
80.308909637 0.152543902397156 45.0
6.7205993035 0.516231596469879 46.0
2.272139438 0.381511688232422 47.0
170.30839767 0.20800419151783 48.0
163.53765613 0.0177380852401257 49.0
15.149228027 0.680789530277252 50.0
71.686880422 0.526742458343506 51.0
80.478305002 0.363960713148117 52.0
84.52996236 0.0538577809929848 53.0
84.52996236 0.00409278832376003 54.0
84.52996236 0.185794278979301 55.0
0 1.97903501987457 56.0
75.954137748 0.0106308925896883 57.0
233.29033652 0.136786714196205 58.0
258.12982844 0.000613314099609852 59.0
20.507254179 0.0173289068043232 60.0
0 0.0762483477592468 61.0
157.09103325 0.88914567232132 0.0
157.09103325 1.24615228176117 1.0
156.01436376 2.09175872802734 2.0
151.22525858 5.56899356842041 3.0
0 6.74923133850098 4.0
0 5.51157283782959 5.0
-62.532688968 0.402747064828873 6.0
72.436975735 0.784993946552277 7.0
64.315502209 1.09729766845703 8.0
8.1214735253 1.20869815349579 9.0
86.046494624 0.927338480949402 10.0
74.355002322 0.304029524326324 11.0
77.986974098 0.869836688041687 12.0
77.986974098 0.598876595497131 13.0
77.986974098 0.112069740891457 14.0
0 0.118377588689327 15.0
0 0.192716047167778 16.0
0 1.61727356910706 17.0
55.495355183 1.36481368541718 18.0
-55.495355183 1.36603009700775 19.0
-55.495355183 0.19654768705368 20.0
-55.495355183 0.0646601840853691 21.0
77.569919037 2.58856296539307 22.0
148.64999705 2.68537569046021 23.0
114.56533541 0.319257616996765 24.0
21.183746502 0.262215554714203 25.0
226.21991609 1.33017301559448 26.0
226.21991609 2.25788521766663 27.0
27.285337728 1.85834419727325 28.0
253.50525382 0.760126292705536 29.0
253.50525382 1.71692585945129 30.0
46.43123089 1.14031481742859 31.0
9.3753308689 0.576356828212738 32.0
22.074563854 1.62252235412598 33.0
22.074563854 0.601566731929779 34.0
22.074563854 0.13927036523819 35.0
9.5388683775 0.0401938781142235 36.0
8.2671571433 0.424273014068604 37.0
0 0.0550777353346348 38.0
0 0.162566959857941 39.0
-11.973771046 0.794207751750946 40.0
-24.683601214 0.445182293653488 41.0
5.3453264526 0.574005722999573 42.0
-32.352907225 0.769299507141113 43.0
88.879218426 0.748142182826996 44.0
88.879218426 0.0886452198028564 45.0
6.0421441584 0.679987788200378 46.0
2.2161484039 0.422206670045853 47.0
157.09103325 0.176789849996567 48.0
151.22525858 0.23217910528183 49.0
0 0.733017206192017 50.0
72.436975735 0.331838846206665 51.0
86.046494624 0.250991523265839 52.0
77.986974098 0.138413548469543 53.0
77.986974098 0.101569697260857 54.0
77.986974098 0.1048179641366 55.0
0 3.93249106407166 56.0
77.569919037 0.0466395244002342 57.0
226.21991609 0.423656046390533 58.0
253.50525382 0.631157696247101 59.0
22.074563854 -0.00122436136007309 60.0
0 0.0935570001602173 61.0
169.2880497 0.881158828735352 0.0
169.2880497 0.828561842441559 1.0
167.85821742 1.07441771030426 2.0
162.73378516 5.34774160385132 3.0
0 4.15580749511719 4.0
0 5.15636920928955 5.0
-61.408994779 4.11085367202759 6.0
70.644365373 0.535417079925537 7.0
40.91928926 0.549202680587769 8.0
29.725076113 0.73784077167511 9.0
90.927915489 1.13300824165344 10.0
78.905058711 0.48656490445137 11.0
85.298713149 0.479360610246658 12.0
85.298713149 0.523863852024078 13.0
85.298713149 0.108218148350716 14.0
0 0.110712788999081 15.0
0 0.194745540618896 16.0
0 1.13782107830048 17.0
60.573799892 1.2316746711731 18.0
-60.573799892 1.36712491512299 19.0
-60.573799892 0.457007557153702 20.0
-60.573799892 -0.0351125784218311 21.0
80.914294957 1.03984117507935 22.0
137.24039585 1.08113551139832 23.0
103.34694887 0.437644928693771 24.0
20.72462494 0.1029417142272 25.0
218.15469081 0.941763818264008 26.0
218.15469081 1.75799667835236 27.0
23.93785018 2.33007025718689 28.0
242.09254099 1.21081161499023 29.0
242.09254099 1.89433860778809 30.0
46.605024112 1.06636738777161 31.0
9.259739097 0.515130639076233 32.0
20.340495065 1.48965752124786 33.0
20.340495065 0.847217500209808 34.0
20.340495065 0.134544387459755 35.0
8.9294762193 0.0313141867518425 36.0
7.67643567 0.355542659759521 37.0
0 0.00554994493722916 38.0
0 0.306395977735519 39.0
-11.924035059 0.443645656108856 40.0
-25.255959003 0.844025194644928 41.0
5.0602606968 0.416935801506042 42.0
-32.905335769 0.492134034633636 43.0
90.438253371 0.744509220123291 44.0
90.438253371 0.0876097977161407 45.0
6.0326217419 0.590979397296906 46.0
2.3697449317 0.374893635511398 47.0
169.2880497 0.15746945142746 48.0
162.73378516 0.131328567862511 49.0
0 0.26136389374733 50.0
70.644365373 0.115562856197357 51.0
90.927915489 0.594514727592468 52.0
85.298713149 0.0769393816590309 53.0
85.298713149 0.0875835716724396 54.0
85.298713149 0.126242056488991 55.0
0 4.54633712768555 56.0
80.914294957 0.102989621460438 57.0
218.15469081 0.353495270013809 58.0
242.09254099 0.578827977180481 59.0
20.340495065 0.0103832855820656 60.0
0 0.0953113734722137 61.0
178.57567253 0.734817922115326 0.0
178.57567253 0.8409224152565 1.0
177.54227839 0.997977018356323 2.0
172.75317318 4.3836669921875 3.0
0 7.45109224319458 4.0
0 3.1536660194397 5.0
-59.238622436 0.370606511831284 6.0
69.543807795 0.736213505268097 7.0
59.821549339 0.693075358867645 8.0
9.7222584555 0.88636040687561 9.0
80.272677723 0.728297472000122 10.0
67.149250454 0.0821990370750427 11.0
78.411850821 0.544427871704102 12.0
78.411850821 0.4110187292099 13.0
78.411850821 0.0905794203281403 14.0
0 0.11244985461235 15.0
0 0.157087773084641 16.0
0 0.881061553955078 17.0
56.325889874 0.951623141765594 18.0
-56.325889874 1.30579197406769 19.0
-56.325889874 0.357231795787811 20.0
-56.325889874 -0.000485584139823914 21.0
77.263759302 2.35866546630859 22.0
144.83491877 1.94497692584991 23.0
112.76962871 0.453565746545792 24.0
22.03089971 0.213020130991936 25.0
222.09867807 1.64665055274963 26.0
222.09867807 2.14628100395203 27.0
25.365957847 1.30331718921661 28.0
247.46463591 0.866475462913513 29.0
247.46463591 1.97793531417847 30.0
46.57516531 0.965851187705994 31.0
9.3245746521 0.56913036108017 32.0
20.937869428 1.62234473228455 33.0
20.937869428 0.994637131690979 34.0
20.937869428 0.135073691606522 35.0
8.8073442096 0.028154693543911 36.0
7.6569766732 0.3622807264328 37.0
0 0.0293941758573055 38.0
0 0.139830514788628 39.0
-11.4230916 0.737585186958313 40.0
-25.824029273 0.503007173538208 41.0
4.9105912492 0.441470503807068 42.0
-33.017253922 0.46743056178093 43.0
102.09479203 0.732993304729462 44.0
102.09479203 0.110519699752331 45.0
7.013536235 0.503390848636627 46.0
2.476760553 0.34593003988266 47.0
178.57567253 0.14970774948597 48.0
172.75317318 0.225790083408356 49.0
0 0.636832475662231 50.0
69.543807795 0.089827798306942 51.0
80.272677723 0.0893193185329437 52.0
78.411850821 0.0187177807092667 53.0
78.411850821 0.204380810260773 54.0
78.411850821 0.062188059091568 55.0
0 4.2359414100647 56.0
77.263759302 0.168824255466461 57.0
222.09867807 1.00518941879272 58.0
247.46463591 0.883535861968994 59.0
20.937869428 0.0114082619547844 60.0
0 0.0867363288998604 61.0
190.24147516 0.874382674694061 0.0
190.24147516 0.870975494384766 1.0
188.99098275 1.14922869205475 2.0
183.58744432 5.78963947296143 3.0
0 8.41087245941162 4.0
0 4.28099584579468 5.0
-55.924977368 0.700179696083069 6.0
67.847345683 1.15247714519501 7.0
41.646335742 0.397118479013443 8.0
26.201009941 0.34394770860672 9.0
87.900111901 0.877931773662567 10.0
74.858221533 0.162069335579872 11.0
84.850588179 0.556349277496338 12.0
84.850588179 0.399566352367401 13.0
84.850588179 0.0783847868442535 14.0
0 0.084589920938015 15.0
0 0.202730149030685 16.0
0 1.365593791008 17.0
57.507826407 1.02284419536591 18.0
-57.507826407 0.994307994842529 19.0
-57.507826407 0.547891139984131 20.0
-57.507826407 0.0135607123374939 21.0
78.483281277 2.52261018753052 22.0
151.132537 2.45713472366333 23.0
113.04081936 0.393237501382828 24.0
22.093544992 0.181920632719994 25.0
229.61581827 1.22874367237091 26.0
229.61581827 2.0803210735321 27.0
24.624291085 1.61026215553284 28.0
254.24010936 0.63197249174118 29.0
254.24010936 1.49124825000763 30.0
47.165925521 1.05254769325256 31.0
9.3810614391 0.498234808444977 32.0
20.97545487 1.12377405166626 33.0
20.97545487 1.13531792163849 34.0
20.97545487 0.154458284378052 35.0
9.8777921424 0.0489886924624443 36.0
8.5339223528 0.376490116119385 37.0
0 0.00892902165651321 38.0
0 0.0664699226617813 39.0
-11.974040638 0.634211897850037 40.0
-24.94727576 0.706637561321259 41.0
5.2285315308 0.407013088464737 42.0
-32.623433795 0.501045107841492 43.0
91.204345729 0.659239292144775 44.0
91.204345729 0.107253365218639 45.0
6.7043904095 0.602446973323822 46.0
1.7782528869 0.444740384817123 47.0
190.24147516 0.227940872311592 48.0
183.58744432 0.0339647345244884 49.0
0 0.221713706851006 50.0
67.847345683 0.0409938171505928 51.0
87.900111901 0.32860192656517 52.0
84.850588179 0.0460790395736694 53.0
84.850588179 0.130087450146675 54.0
84.850588179 0.0598043687641621 55.0
0 0.450208246707916 56.0
78.483281277 0.067510262131691 57.0
229.61581827 0.459402412176132 58.0
254.24010936 0.329187542200089 59.0
20.97545487 -0.0212807580828667 60.0
0 0.0980824679136276 61.0
137.34903662 1.02698612213135 0.0
137.34903662 0.720321476459503 1.0
136.09655911 1.13308680057526 2.0
131.30748916 2.85140013694763 3.0
0 5.34853029251099 4.0
0 5.12342071533203 5.0
-56.668918834 1.30678629875183 6.0
66.59279165 0.787775576114655 7.0
48.403525548 0.767610549926758 8.0
18.189266102 1.01394188404083 9.0
85.187907631 1.04098451137543 10.0
71.919626296 0.102652788162231 11.0
77.575953599 0.6131591796875 12.0
77.575953599 0.123199418187141 13.0
77.575953599 0.0733581483364105 14.0
0 0.0285979080945253 15.0
0 0.227877169847488 16.0
0 1.08968496322632 17.0
59.815683141 0.831780195236206 18.0
-59.815683141 0.78752326965332 19.0
-59.815683141 0.490849196910858 20.0
-59.815683141 0.0216157957911491 21.0
81.381751096 1.76268887519836 22.0
147.58172318 1.7440128326416 23.0
114.65793693 0.579293668270111 24.0
24.483570875 0.166405260562897 25.0
228.96347427 1.32754755020142 26.0
228.96347427 2.29165697097778 27.0
24.442001636 1.4550644159317 28.0
253.40547591 0.443432658910751 29.0
253.40547591 1.57415866851807 30.0
47.075501655 1.25838696956635 31.0
9.4093378709 0.43077689409256 32.0
21.566067955 1.31457316875458 33.0
21.566067955 0.942812085151672 34.0
21.566067955 0.0967291742563248 35.0
9.6808646629 0.0266858972609043 36.0
8.4539101287 0.379326969385147 37.0
0 0.0267601534724236 38.0
0 0.148098021745682 39.0
-12.108353889 0.836449265480042 40.0
-26.121836567 0.469508022069931 41.0
5.1144033406 0.553718209266663 42.0
-33.833490036 0.483338057994843 43.0
89.462280133 0.691986858844757 44.0
89.462280133 0.0868731215596199 45.0
7.1688619184 0.648728430271149 46.0
2.8443341688 0.373790979385376 47.0
137.34903662 0.066410094499588 48.0
131.30748916 0.0451239123940468 49.0
0 0.444690883159637 50.0
66.59279165 0.231004804372787 51.0
85.187907631 0.135098293423653 52.0
77.575953599 0.0447024405002594 53.0
77.575953599 0.0309780761599541 54.0
77.575953599 0.0816493332386017 55.0
0 4.85337591171265 56.0
81.381751096 0.00793346855789423 57.0
228.96347427 1.01766300201416 58.0
253.40547591 0.520269274711609 59.0
21.566067955 0.00435818359255791 60.0
0 0.10255204886198 61.0
128.46602349 0.844164609909058 0.0
128.46602349 1.3314756155014 1.0
127.11815456 1.56211686134338 2.0
122.32904927 6.01467990875244 3.0
0 3.72501873970032 4.0
0 5.37905836105347 5.0
-51.60029031 3.16546058654785 6.0
61.39882554 0.5049929022789 7.0
52.454834977 1.09895265102386 8.0
8.9439905627 1.11150813102722 9.0
83.603151289 0.939637064933777 10.0
72.010490096 0.0475992523133755 11.0
79.144002065 0.355584979057312 12.0
79.144002065 0.140244469046593 13.0
79.144002065 0.201470464468002 14.0
0 0.116078406572342 15.0
0 0.155903831124306 16.0
0 1.56179869174957 17.0
56.926119375 1.65106773376465 18.0
-56.926119375 1.34087252616882 19.0
-56.926119375 0.605912923812866 20.0
-56.926119375 -0.042331051081419 21.0
78.078884257 1.00444316864014 22.0
154.47960458 1.27859270572662 23.0
123.38306663 0.213192880153656 24.0
23.501980421 0.439460128545761 25.0
232.55848884 1.05020332336426 26.0
232.55848884 2.30460405349731 27.0
29.167666622 2.426753282547 28.0
261.72615546 1.30175685882568 29.0
261.72615546 1.78938090801239 30.0
47.343285346 1.09692049026489 31.0
9.5075716982 0.356087028980255 32.0
21.152764882 1.5885204076767 33.0
21.152764882 0.562959969043732 34.0
21.152764882 0.142082035541534 35.0
8.7957079078 0.0225047152489424 36.0
7.3290073489 0.373178035020828 37.0
0 0.0133721902966499 38.0
0 0.131665006279945 39.0
-11.835361378 0.909843564033508 40.0
-24.514483267 0.343982666730881 41.0
5.30775834 0.55759984254837 42.0
-32.303980382 0.582070350646973 43.0
75.200590293 0.675183653831482 44.0
75.200590293 0.0805011242628098 45.0
6.6179575979 0.612415313720703 46.0
2.5945629358 0.291201800107956 47.0
128.46602349 0.186028495430946 48.0
122.32904927 0.265341103076935 49.0
0 0.411330431699753 50.0
61.39882554 0.229680001735687 51.0
83.603151289 0.115467399358749 52.0
79.144002065 0.00480621447786689 53.0
79.144002065 0.106009028851986 54.0
79.144002065 0.000772577710449696 55.0
0 6.68492126464844 56.0
78.078884257 0.116192437708378 57.0
232.55848884 0.613715887069702 58.0
261.72615546 0.720376789569855 59.0
21.152764882 -0.00971914827823639 60.0
0 0.114918038249016 61.0
200.74901964 0.925798296928406 0.0
200.74901964 1.22828567028046 1.0
199.72269495 1.21845638751984 2.0
194.25802426 4.3109188079834 3.0
8.6747533725 6.18125915527344 4.0
8.6747533725 5.57254409790039 5.0
-58.845139128 1.561119556427 6.0
71.789100969 0.891448080539703 7.0
38.433734657 0.638416230678558 8.0
33.355366313 0.471933752298355 9.0
76.588196255 0.938333868980408 10.0
65.019007721 0.0930794700980186 11.0
72.552303001 1.00910580158234 12.0
72.552303001 0.278043657541275 13.0
72.552303001 0.0670111328363419 14.0
16.213087482 0.0666242018342018 15.0
16.213087482 0.195068269968033 16.0
0 1.02180695533752 17.0
55.278905676 1.24353158473969 18.0
-55.278905676 1.53542935848236 19.0
-55.278905676 0.267576932907104 20.0
-55.278905676 0.029087595641613 21.0
76.576021013 1.5558854341507 22.0
144.58981007 0.881402790546417 23.0
109.96961612 0.61536717414856 24.0
24.296129477 0.0932708829641342 25.0
221.16583109 1.37263023853302 26.0
221.16583109 1.51442182064056 27.0
23.721107812 1.86035871505737 28.0
244.8869389 0.939563453197479 29.0
244.8869389 1.82647705078125 30.0
47.265622099 1.08791649341583 31.0
9.3385642707 0.532697200775146 32.0
21.297115337 1.1788717508316 33.0
21.297115337 0.635989487171173 34.0
21.297115337 0.0894518494606018 35.0
9.6304941944 0.0264327060431242 36.0
8.2694492378 0.405680179595947 37.0
0 0.0440863817930222 38.0
0 0.153052031993866 39.0
-12.210137921 0.578475832939148 40.0
-25.498739925 0.638358056545258 41.0
5.2950768744 0.432156145572662 42.0
-33.096051348 0.627657532691956 43.0
94.368582316 0.760745286941528 44.0
94.368582316 0.083640992641449 45.0
6.6446899638 0.555002093315125 46.0
2.5174131628 0.319127410650253 47.0
200.74901964 0.11847659945488 48.0
194.25802426 0.0219056978821754 49.0
8.6747533725 0.209432661533356 50.0
71.789100969 0.0953939408063889 51.0
76.588196255 0.0998498499393463 52.0
72.552303001 0.114708296954632 53.0
72.552303001 0.208323791623116 54.0
72.552303001 0.082679845392704 55.0
0 1.54512596130371 56.0
76.576021013 0.061290480196476 57.0
221.16583109 0.465039491653442 58.0
244.8869389 0.364608466625214 59.0
21.297115337 -0.0168783590197563 60.0
0 0.101671144366264 61.0
144.40211627 1.09560084342957 0.0
144.40211627 0.870100080966949 1.0
143.21655753 1.15582489967346 2.0
138.42745223 3.27908515930176 3.0
0 7.51416492462158 4.0
0 4.77405595779419 5.0
-64.229477456 1.95101225376129 6.0
75.547095243 0.903655111789703 7.0
60.080000412 0.753821134567261 8.0
15.467094831 0.536451816558838 9.0
95.951853336 1.12197661399841 10.0
83.69845924 0.244926765561104 11.0
80.696840047 0.251921415328979 12.0
80.696840047 0.275687575340271 13.0
80.696840047 0.125348970293999 14.0
0 0.200673475861549 15.0
0 0.168239235877991 16.0
0 1.45940387248993 17.0
58.280709159 1.43882191181183 18.0
-58.280709159 1.62182092666626 19.0
-58.280709159 0.409338593482971 20.0
-58.280709159 -0.0283407215029001 21.0
79.120393747 1.85091638565063 22.0
147.62537226 1.64364778995514 23.0
108.9598441 0.491349637508392 24.0
23.656717872 0.0911144465208054 25.0
226.74576601 0.857947409152985 26.0
226.74576601 1.34476482868195 27.0
29.621523106 1.00615155696869 28.0
256.36728911 0.513455629348755 29.0
256.36728911 1.63910531997681 30.0
47.176593333 0.975549161434174 31.0
9.4522782104 0.466874837875366 32.0
20.839684588 1.6641960144043 33.0
20.839684588 0.980484008789062 34.0
20.839684588 0.128832817077637 35.0
9.2444301864 0.028222544118762 36.0
7.8835342249 0.407079696655273 37.0
0 0.0327922441065311 38.0
0 0.159397274255753 39.0
-11.494460331 0.784578323364258 40.0
-24.064786087 0.664018213748932 41.0
5.6103685205 0.431383639574051 42.0
-32.462525447 0.568804800510406 43.0
80.215842244 0.773176550865173 44.0
80.215842244 0.0980049520730972 45.0
7.7810276928 0.673703014850616 46.0
2.870571112 0.395273119211197 47.0
144.40211627 0.133452415466309 48.0
138.42745223 0.172821089625359 49.0
0 0.357307314872742 50.0
75.547095243 0.139020174741745 51.0
95.951853336 0.304799735546112 52.0
80.696840047 0.0473508089780807 53.0
80.696840047 0.0545111037790775 54.0
80.696840047 0.128788679838181 55.0
0 5.6968240737915 56.0
79.120393747 0.067688025534153 57.0
226.74576601 0.225631356239319 58.0
256.36728911 0.49859943985939 59.0
20.839684588 0.0106880180537701 60.0
0 0.11117085814476 61.0
171.61103859 1.19786834716797 0.0
171.61103859 1.01634955406189 1.0
170.13011295 1.17103815078735 2.0
164.66544225 2.73544931411743 3.0
12.785605885 4.57976722717285 4.0
12.785605885 7.19412708282471 5.0
-56.427624098 4.37461423873901 6.0
65.462578907 0.47400227189064 7.0
38.880232149 0.906350672245026 8.0
26.582346758 0.813505232334137 9.0
91.638368713 1.42837357521057 10.0
79.484444249 0.719004988670349 11.0
79.090587648 1.00425326824188 12.0
79.090587648 0.813973724842072 13.0
79.090587648 0.131531924009323 14.0
0 0.246213480830193 15.0
0 0.139760240912437 16.0
0 1.76252710819244 17.0
63.455061381 0.851249933242798 18.0
-63.455061381 1.02928423881531 19.0
-63.455061381 0.646643400192261 20.0
-63.455061381 0.0156884044408798 21.0
85.563528367 1.58960247039795 22.0
144.85221743 2.16995811462402 23.0
111.27312781 0.120596908032894 24.0
25.386019457 0.244126379489899 25.0
230.4157458 0.8500856757164 26.0
230.4157458 1.24159288406372 27.0
30.234777536 1.373783826828 28.0
260.65052334 1.14710938930511 29.0
260.65052334 1.69855904579163 30.0
46.53066531 0.877528786659241 31.0
9.2615572207 0.38083752989769 32.0
22.108466986 0.726568043231964 33.0
22.108466986 1.13784849643707 34.0
22.108466986 0.122536532580853 35.0
10.010613785 0.0244616884738207 36.0
8.6759925462 0.397222727537155 37.0
0 -0.0154851097613573 38.0
0 0.231330931186676 39.0
-12.131839973 0.720791757106781 40.0
-23.593469316 0.903031885623932 41.0
4.6393777548 0.364923894405365 42.0
-30.796985344 0.464296847581863 43.0
93.129313196 0.811504065990448 44.0
93.129313196 0.156737297773361 45.0
5.9812846296 0.62252151966095 46.0
2.2800408557 0.279456496238708 47.0
171.61103859 0.173620015382767 48.0
164.66544225 0.158843547105789 49.0
12.785605885 0.4050572514534 50.0
65.462578907 0.213336005806923 51.0
91.638368713 0.200382724404335 52.0
79.090587648 0.23838846385479 53.0
79.090587648 0.09609754383564 54.0
79.090587648 0.301592171192169 55.0
0 5.88219356536865 56.0
85.563528367 -0.00614868011325598 57.0
230.4157458 0.685757040977478 58.0
260.65052334 1.49832737445831 59.0
22.108466986 -0.0052987951785326 60.0
0 0.107144758105278 61.0
137.47724482 1.05651307106018 0.0
137.47724482 0.78199577331543 1.0
136.46656847 1.00132715702057 2.0
131.67746327 2.89598989486694 3.0
0 4.2097339630127 4.0
0 3.53706860542297 5.0
-53.19212432 0.822082340717316 6.0
64.792369058 0.821341633796692 7.0
57.937059336 0.736493766307831 8.0
6.8553097218 0.816888809204102 9.0
85.120738365 1.16583037376404 10.0
72.443644388 0.137227684259415 11.0
81.499063451 0.665321886539459 12.0
81.499063451 0.238975584506989 13.0
81.499063451 0.131112366914749 14.0
0 0.209910899400711 15.0
0 0.218185275793076 16.0
0 1.44976437091827 17.0
57.869805878 1.62103235721588 18.0
-57.869805878 1.7672655582428 19.0
-57.869805878 0.655760884284973 20.0
-57.869805878 0.00243838876485825 21.0
80.487701758 1.79065108299255 22.0
139.34164596 1.55132508277893 23.0
106.08762501 0.324464112520218 24.0
21.337921829 0.184534817934036 25.0
219.82934772 1.37526023387909 26.0
219.82934772 1.82118034362793 27.0
28.641834001 1.78696739673615 28.0
248.47118172 0.610500514507294 29.0
248.47118172 1.59942662715912 30.0
46.826683123 0.946072638034821 31.0
9.3510727346 0.342633247375488 32.0
22.617895881 1.72752261161804 33.0
22.617895881 0.811224102973938 34.0
22.617895881 0.0899379849433899 35.0
9.9497890214 0.0305779464542866 36.0
8.6553634709 0.351306796073914 37.0
0 0.0478838495910168 38.0
0 0.125431254506111 39.0
-11.429899612 0.616860628128052 40.0
-23.826850062 0.479917615652084 41.0
5.531646293 0.530166327953339 42.0
-31.571654981 0.569482088088989 43.0
102.40325063 0.590383887290955 44.0
102.40325063 0.0908988267183304 45.0
6.3996701422 0.590778052806854 46.0
2.2193761653 0.381454199552536 47.0
137.47724482 0.117647469043732 48.0
131.67746327 0.0131300333887339 49.0
0 0.246527165174484 50.0
64.792369058 0.228713795542717 51.0
85.120738365 0.286716490983963 52.0
81.499063451 0.0534677244722843 53.0
81.499063451 0.0645275861024857 54.0
81.499063451 0.0373289696872234 55.0
0 4.44132280349731 56.0
80.487701758 0.0472217090427876 57.0
219.82934772 0.349184483289719 58.0
248.47118172 0.530120968818665 59.0
22.617895881 -0.00580319203436375 60.0
0 0.130020678043365 61.0
171.61970981 0.886334478855133 0.0
171.61970981 1.59429526329041 1.0
170.70929062 1.78410601615906 2.0
165.92018532 4.68028879165649 3.0
0 7.28374671936035 4.0
0 4.16641139984131 5.0
-56.521235191 2.35586762428284 6.0
65.328626503 0.614701867103577 7.0
63.18054133 1.21564793586731 8.0
2.1480851732 1.52816951274872 9.0
75.069387796 1.02835643291473 10.0
62.770119793 0.390303820371628 11.0
72.854630202 1.19896829128265 12.0
72.854630202 0.486760258674622 13.0
72.854630202 0.264037251472473 14.0
0 0.110737778246403 15.0
0 0.19175885617733 16.0
0 1.29263782501221 17.0
65.635431171 1.61876845359802 18.0
-65.635431171 1.86782240867615 19.0
-65.635431171 0.340264409780502 20.0
-65.635431171 -0.0835796073079109 21.0
86.938067566 2.04569530487061 22.0
138.91472571 2.17110323905945 23.0
104.10450812 0.491018563508987 24.0
25.51811941 0.0826187282800674 25.0
225.85279328 1.75198793411255 26.0
225.85279328 2.83434104919434 27.0
26.001532621 2.76785397529602 28.0
251.8543259 1.43832004070282 29.0
251.8543259 2.04846572875977 30.0
46.092045068 0.952930629253387 31.0
9.3954695706 0.320478439331055 32.0
21.302636396 2.12806534767151 33.0
21.302636396 1.06542026996613 34.0
21.302636396 0.127667605876923 35.0
9.5959487996 0.0552892312407494 36.0
8.3059898364 0.344621032476425 37.0
0 0.00301702693104744 38.0
0 0.122415013611317 39.0
-12.754305115 0.659944593906403 40.0
-23.671355749 0.270726233720779 41.0
4.7121160311 0.437621116638184 42.0
-30.685781987 0.44396111369133 43.0
79.455578644 0.740257561206818 44.0
79.455578644 0.051639586687088 45.0
7.6191569442 0.412869513034821 46.0
2.4626219959 0.309064388275146 47.0
171.61970981 0.0953646451234818 48.0
165.92018532 0.376355081796646 49.0
0 0.27943354845047 50.0
65.328626503 0.275186538696289 51.0
75.069387796 0.23903326690197 52.0
72.854630202 0.164510935544968 53.0
72.854630202 0.231269434094429 54.0
72.854630202 0.0727712884545326 55.0
0 12.3234701156616 56.0
86.938067566 0.265253067016602 57.0
225.85279328 0.867893099784851 58.0
251.8543259 1.32584381103516 59.0
21.302636396 0.0278653483837843 60.0
0 0.110397584736347 61.0
175.77561997 0.78844279050827 0.0
175.77561997 0.978088438510895 1.0
174.63124802 1.37468445301056 2.0
169.84214271 5.00500011444092 3.0
0 4.24935531616211 4.0
0 3.19252300262451 5.0
-60.296890072 0.290353983640671 6.0
69.624833673 0.939434766769409 7.0
59.425336197 0.589766979217529 8.0
10.199497477 0.478890925645828 9.0
91.391528356 0.91468870639801 10.0
78.4836413 0.276404082775116 11.0
82.934743326 0.792915344238281 12.0
82.934743326 0.363103806972504 13.0
82.934743326 0.111072421073914 14.0
0 0.216570422053337 15.0
0 0.156520739197731 16.0
0 1.64430975914001 17.0
56.896454339 1.16474175453186 18.0
-56.896454339 0.72829282283783 19.0
-56.896454339 0.518095314502716 20.0
-56.896454339 -0.0348541736602783 21.0
77.401056069 1.30393946170807 22.0
135.88054085 1.67867457866669 23.0
103.83036245 0.405289351940155 24.0
22.112617265 0.105968028306961 25.0
213.28159692 1.31564092636108 26.0
213.28159692 1.33229494094849 27.0
29.162114464 1.15256214141846 28.0
242.44371138 0.399451464414597 29.0
242.44371138 1.49611186981201 30.0
46.407211686 0.989788234233856 31.0
9.326603215 0.36522176861763 32.0
20.50460173 1.53210294246674 33.0
20.50460173 0.743725895881653 34.0
20.50460173 0.139415189623833 35.0
9.5142141217 0.0339469946920872 36.0
8.3354481207 0.376000940799713 37.0
0 0.0155514292418957 38.0
0 0.157272398471832 39.0
-12.233798549 0.75375360250473 40.0
-25.242453895 0.432760179042816 41.0
4.6419099621 0.458038926124573 42.0
-32.245845691 0.415922880172729 43.0
80.287854757 0.681950211524963 44.0
80.287854757 0.0954819843173027 45.0
6.8546785137 0.633487164974213 46.0
2.1323755499 0.400858998298645 47.0
175.77561997 0.0820450112223625 48.0
169.84214271 0.281271904706955 49.0
0 0.064276397228241 50.0
69.624833673 0.0826113820075989 51.0
91.391528356 0.139152765274048 52.0
82.934743326 0.117204383015633 53.0
82.934743326 0.173491165041924 54.0
82.934743326 0.0633737370371819 55.0
0 2.46532845497131 56.0
77.401056069 0.194203943014145 57.0
213.28159692 0.579081475734711 58.0
242.44371138 0.20654222369194 59.0
20.50460173 0.0227228533476591 60.0
0 0.125791683793068 61.0
149.30739128 1.06935834884644 0.0
149.30739128 2.0472571849823 1.0
147.92201394 2.82075023651123 2.0
143.13290865 3.95190477371216 3.0
0 6.58706092834473 4.0
0 4.24236536026001 5.0
-66.107243589 0.367661386728287 6.0
76.693287033 0.431901812553406 7.0
54.846257182 0.979725301265717 8.0
21.847029852 1.32755136489868 9.0
81.008281361 1.03885579109192 10.0
68.655400911 0.0247869677841663 11.0
83.266445692 0.421879023313522 12.0
83.266445692 0.188743963837624 13.0
83.266445692 0.135593459010124 14.0
0 0.221237421035767 15.0
0 0.159337773919106 16.0
0 1.60867261886597 17.0
64.142610883 1.74462020397186 18.0
-64.142610883 1.56601345539093 19.0
-64.142610883 0.348467797040939 20.0
-64.142610883 -0.0248643420636654 21.0
85.345270995 1.28423130512238 22.0
157.17401321 0.840076148509979 23.0
120.15739553 0.628280401229858 24.0
21.33477715 0.120696812868118 25.0
242.5192842 0.525572180747986 26.0
242.5192842 1.00649666786194 27.0
24.016373747 0.853371679782867 28.0
266.53565795 0.529592096805573 29.0
266.53565795 1.46267652511597 30.0
46.838270682 0.947537302970886 31.0
9.2870585209 0.414490401744843 32.0
21.202660113 2.26411294937134 33.0
21.202660113 0.99790632724762 34.0
21.202660113 0.0854785144329071 35.0
8.8077115406 0.0226784106343985 36.0
7.4563542707 0.410068333148956 37.0
0 0.0208248682320118 38.0
0 0.0704025253653526 39.0
-12.094114862 0.885316908359528 40.0
-24.358988041 0.342449367046356 41.0
5.7624371962 0.66470742225647 42.0
-32.833569314 0.81069815158844 43.0
75.381451427 0.562925159931183 44.0
75.381451427 0.1051120236516 45.0
6.6311012355 0.542636275291443 46.0
2.6470739473 0.391325205564499 47.0
149.30739128 0.177299916744232 48.0
143.13290865 0.723894119262695 49.0
0 0.305868685245514 50.0
76.693287033 0.22097909450531 51.0
81.008281361 0.112882733345032 52.0
83.266445692 0.00330935372039676 53.0
83.266445692 0.0290219597518444 54.0
83.266445692 0.121975988149643 55.0
0 3.85676431655884 56.0
85.345270995 0.177121877670288 57.0
242.5192842 0.093670628964901 58.0
266.53565795 0.372194588184357 59.0
21.202660113 -0.0127982646226883 60.0
0 0.106396585702896 61.0
192.40154412 0.633825182914734 0.0
192.40154412 1.36731660366058 1.0
191.07715321 2.26119828224182 2.0
185.61248247 5.61328601837158 3.0
14.524398538 5.55925369262695 4.0
14.524398538 4.08162832260132 5.0
-57.891468975 0.455970019102097 6.0
69.742366944 0.999572038650513 7.0
44.045319191 0.369039475917816 8.0
25.697047752 0.345156192779541 9.0
83.042099899 0.715018570423126 10.0
68.77384799 0.211884096264839 11.0
79.277908973 0.0927984267473221 12.0
79.277908973 0.0816615521907806 13.0
79.277908973 0.0311691239476204 14.0
0 -0.013785395771265 15.0
0 0.216142669320107 16.0
0 0.930666267871857 17.0
49.623625855 1.60109972953796 18.0
-49.623625855 1.07454872131348 19.0
-49.623625855 0.546429991722107 20.0
-49.623625855 -0.0092465803027153 21.0
69.138181583 1.56302952766418 22.0
141.45719524 1.37043142318726 23.0
103.86030853 0.2715123295784 24.0
25.191154066 0.27753221988678 25.0
210.59537682 0.446256756782532 26.0
210.59537682 2.50599193572998 27.0
25.766205494 2.41350245475769 28.0
236.36158231 1.57869136333466 29.0
236.36158231 2.21113634109497 30.0
46.35003276 1.05802130699158 31.0
9.273175002 0.488194793462753 32.0
19.514555728 1.41927087306976 33.0
19.514555728 1.00495862960815 34.0
19.514555728 0.116736993193626 35.0
8.8440523949 0.0517993941903114 36.0
7.5440198098 0.400254458189011 37.0
0 0.0159790143370628 38.0
0 0.189722090959549 39.0
-13.227739405 0.333851933479309 40.0
-26.057638844 1.10128366947174 41.0
5.5913532483 0.454647928476334 42.0
-33.970582782 0.80708110332489 43.0
85.911087929 0.597309947013855 44.0
85.911087929 0.106751285493374 45.0
6.7835681196 0.385783404111862 46.0
1.8734048225 0.338383555412292 47.0
192.40154412 0.00132502615451813 48.0
185.61248247 0.319592297077179 49.0
14.524398538 0.109626702964306 50.0
69.742366944 0.225640386343002 51.0
83.042099899 0.32666677236557 52.0
79.277908973 0.0315050669014454 53.0
79.277908973 0.113599076867104 54.0
79.277908973 0.0820549726486206 55.0
0 2.50163531303406 56.0
69.138181583 -0.00620261719450355 57.0
210.59537682 0.210113346576691 58.0
236.36158231 1.33271217346191 59.0
19.514555728 -0.00906001497060061 60.0
0 0.0908299088478088 61.0
156.33697625 1.44430780410767 0.0
156.33697625 1.28631818294525 1.0
155.11723055 1.69720160961151 2.0
150.32812531 5.10391712188721 3.0
0 5.91794776916504 4.0
0 4.57400274276733 5.0
-50.649104941 2.85907888412476 6.0
59.183054289 0.571996748447418 7.0
65.104790723 0.792838096618652 8.0
-5.9217364348 0.927321314811707 9.0
88.420775992 1.43637883663177 10.0
75.737906505 0.334349989891052 11.0
75.974553915 0.73004424571991 12.0
75.974553915 0.18159407377243 13.0
75.974553915 0.0805686935782433 14.0
0 0.115228146314621 15.0
0 0.158089622855186 16.0
0 1.29944467544556 17.0
52.666492931 1.05261361598969 18.0
-52.666492931 0.904484570026398 19.0
-52.666492931 0.512997806072235 20.0
-52.666492931 0.00395908206701279 21.0
71.847363446 1.68485331535339 22.0
147.00541704 1.89429724216461 23.0
111.53329167 0.410886913537979 24.0
21.144350205 0.286211788654327 25.0
218.85278049 1.61268579959869 26.0
218.85278049 2.16077542304993 27.0
25.675397538 2.09192991256714 28.0
244.52817802 0.544061124324799 29.0
244.52817802 1.61664485931396 30.0
47.082102031 1.04462349414825 31.0
9.1765809687 0.569045603275299 32.0
19.180870515 1.45403873920441 33.0
19.180870515 0.891989469528198 34.0
19.180870515 0.125668153166771 35.0
7.6984468868 0.0339829847216606 36.0
6.6191197525 0.371844500303268 37.0
0 0.0341425314545631 38.0
0 0.214631080627441 39.0
-11.062149188 0.71376770734787 40.0
-23.825471713 0.663611233234406 41.0
5.2832473424 0.532236576080322 42.0
-31.680203818 0.638584494590759 43.0
94.421728054 0.688309073448181 44.0
94.421728054 0.101448610424995 45.0
5.9894840747 0.663843750953674 46.0
2.5343782588 0.438165247440338 47.0
156.33697625 0.132948592305183 48.0
150.32812531 0.392936378717422 49.0
0 0.747683644294739 50.0
59.183054289 0.0919998809695244 51.0
88.420775992 0.148069620132446 52.0
75.974553915 0.143947750329971 53.0
75.974553915 0.108197070658207 54.0
75.974553915 0.198236286640167 55.0
0 4.02566146850586 56.0
71.847363446 0.0928318351507187 57.0
218.85278049 0.468805074691772 58.0
244.52817802 0.240012779831886 59.0
19.180870515 -0.0157427862286568 60.0
0 0.11416482925415 61.0
187.21580087 1.23894548416138 0.0
187.21580087 0.962850987911224 1.0
186.13869245 1.10861194133759 2.0
181.34958718 5.82964134216309 3.0
0 8.47416305541992 4.0
0 3.30298399925232 5.0
-66.072577011 0.325074404478073 6.0
77.939879325 0.542482972145081 7.0
51.985963969 1.25511884689331 8.0
25.953915356 1.25722324848175 9.0
100.89047399 1.3566460609436 10.0
88.502338158 0.352653533220291 11.0
88.028681156 0.443672299385071 12.0
88.028681156 0.287601351737976 13.0
88.028681156 0.0984303504228592 14.0
0 0.205343931913376 15.0
0 0.202929019927979 16.0
0 1.1627037525177 17.0
54.708646323 1.58442747592926 18.0
-54.708646323 1.71304297447205 19.0
-54.708646323 0.372688263654709 20.0
-54.708646323 0.0988291949033737 21.0
73.89297189 1.5599867105484 22.0
147.02706696 1.15019416809082 23.0
110.9797281 0.349151045084 24.0
21.020086357 0.209754452109337 25.0
220.92003885 0.859481334686279 26.0
220.92003885 2.04055976867676 27.0
26.095709393 2.73427796363831 28.0
247.01574825 1.49621081352234 29.0
247.01574825 1.80355930328369 30.0
46.537792832 1.03967714309692 31.0
9.4619910899 0.352378934621811 32.0
19.184325567 1.47524785995483 33.0
19.184325567 1.0493448972702 34.0
19.184325567 0.10959081351757 35.0
8.4807761269 0.0256003420799971 36.0
7.2125464484 0.388948887586594 37.0
0 0.0132273100316525 38.0
0 0.142635658383369 39.0
-12.083964959 0.666424453258514 40.0
-23.792222574 0.664516389369965 41.0
5.2913867064 0.439151495695114 42.0
-31.789561013 0.436035603284836 43.0
88.244216915 0.753354966640472 44.0
88.244216915 0.104641385376453 45.0
6.6201301448 0.641812801361084 46.0
2.5447499493 0.3630550801754 47.0
187.21580087 0.159598752856255 48.0
181.34958718 0.276711344718933 49.0
0 0.510214686393738 50.0
77.939879325 0.210275962948799 51.0
100.89047399 0.101717904210091 52.0
88.028681156 0.106364123523235 53.0
88.028681156 0.0378798060119152 54.0
88.028681156 0.13829006254673 55.0
0 3.33299255371094 56.0
73.89297189 0.0687216594815254 57.0
220.92003885 0.433738023042679 58.0
247.01574825 0.760939419269562 59.0
19.184325567 -0.013349998742342 60.0
0 0.105189919471741 61.0
190.56733155 0.659993350505829 0.0
190.56733155 0.947929561138153 1.0
189.59979291 1.05683088302612 2.0
184.81068772 5.10272598266602 3.0
0 4.19564867019653 4.0
0 5.87220907211304 5.0
-60.425107967 0.52681291103363 6.0
73.833187723 1.02987122535706 7.0
49.091321647 0.948317646980286 8.0
24.741866076 0.702871739864349 9.0
78.792011564 0.741147339344025 10.0
66.289766039 0.278468698263168 11.0
82.582996608 0.486283391714096 12.0
82.582996608 0.538377463817596 13.0
82.582996608 0.103182926774025 14.0
0 0.157072201371193 15.0
0 0.202953487634659 16.0
0 0.937372803688049 17.0
55.175687231 1.81200063228607 18.0
-55.175687231 1.1409900188446 19.0
-55.175687231 0.876776099205017 20.0
-55.175687231 -0.0383462384343147 21.0
76.14502644 2.71119141578674 22.0
145.11986283 2.6179518699646 23.0
107.78744952 0.202470242977142 24.0
22.859194461 0.246926307678223 25.0
221.26488927 2.29953026771545 26.0
221.26488927 3.01021647453308 27.0
23.193373045 2.82168292999268 28.0
244.45826231 1.19010198116302 29.0
244.45826231 1.84229147434235 30.0
46.75433781 1.13705289363861 31.0
9.4961722648 0.569385707378387 32.0
20.969339209 1.74818444252014 33.0
20.969339209 0.871524691581726 34.0
20.969339209 0.15158811211586 35.0
8.4629649467 0.0421713814139366 36.0
7.248561436 0.392793029546738 37.0
0 0.0674833655357361 38.0
0 0.090386375784874 39.0
-12.685566388 1.01671993732452 40.0
-24.051735573 0.183241009712219 41.0
5.323915145 0.649981200695038 42.0
-31.841582484 0.555696308612823 43.0
87.024564669 0.591084659099579 44.0
87.024564669 0.124015249311924 45.0
6.0854261246 0.602876901626587 46.0
2.4893852038 0.460744380950928 47.0
190.56733155 0.1096580773592 48.0
184.81068772 0.0948114544153214 49.0
0 0.407020211219788 50.0
73.833187723 0.383701264858246 51.0
78.792011564 0.454533159732819 52.0
82.582996608 0.0158411152660847 53.0
82.582996608 0.153159871697426 54.0
82.582996608 0.126129508018494 55.0
0 6.39588737487793 56.0
76.14502644 0.00513147935271263 57.0
221.26488927 1.31436574459076 58.0
244.45826231 0.32704022526741 59.0
20.969339209 -0.0315116234123707 60.0
0 0.074796125292778 61.0
146.18167613 0.943065762519836 0.0
146.18167613 0.69063663482666 1.0
145.14376726 1.10579228401184 2.0
140.35469731 3.82346677780151 3.0
0 8.24181175231934 4.0
0 5.824387550354 5.0
-63.969769863 3.32460260391235 6.0
73.199229452 0.618656277656555 7.0
51.757401329 1.03276395797729 8.0
21.441828123 1.08191204071045 9.0
94.847683366 1.07183218002319 10.0
81.729757378 0.2625772356987 11.0
83.890699305 0.491758167743683 12.0
83.890699305 0.279251277446747 13.0
83.890699305 0.128352075815201 14.0
0 0.176095098257065 15.0
0 0.185612544417381 16.0
0 1.32436072826385 17.0
55.150935821 1.5371390581131 18.0
-55.150935821 1.72517359256744 19.0
-55.150935821 0.475171327590942 20.0
-55.150935821 -0.0108969509601593 21.0
75.82266063 1.78856420516968 22.0
143.0894867 2.15038919448853 23.0
107.62344958 0.369661957025528 24.0
20.584603221 0.209335803985596 25.0
218.91214733 0.749114513397217 26.0
218.91214733 1.82691359519958 27.0
27.187648684 1.18618273735046 28.0
246.09979602 0.386545240879059 29.0
246.09979602 1.75484430789948 30.0
46.413840068 1.18591046333313 31.0
9.2562745212 0.565544843673706 32.0
20.671724809 1.38003861904144 33.0
20.671724809 0.864426076412201 34.0
20.671724809 0.12335779517889 35.0
9.2303671397 0.0280971694737673 36.0
8.01659749 0.3506760597229 37.0
0 0.0390961654484272 38.0
0 0.114428177475929 39.0
-12.224818744 0.451735019683838 40.0
-25.641444611 0.698753178119659 41.0
5.2705334257 0.406944841146469 42.0
-33.32510514 0.404626280069351 43.0
91.401874739 0.733353614807129 44.0
91.401874739 0.0997651889920235 45.0
7.5230110873 0.689894676208496 46.0
2.8854187217 0.441372036933899 47.0
146.18167613 0.176964104175568 48.0
140.35469731 0.176490932703018 49.0
0 0.396775811910629 50.0
73.199229452 0.186180308461189 51.0
94.847683366 0.346553921699524 52.0
83.890699305 0.0692255049943924 53.0
83.890699305 0.0837933644652367 54.0
83.890699305 0.0164509247988462 55.0
0 3.98810410499573 56.0
75.82266063 0.0660250261425972 57.0
218.91214733 0.536460161209106 58.0
246.09979602 0.278542369604111 59.0
20.671724809 0.0130005069077015 60.0
0 0.0813503265380859 61.0
152.04746836 0.78553968667984 0.0
152.04746836 1.2537043094635 1.0
150.76117696 1.64010989665985 2.0
145.97207181 4.53208065032959 3.0
0 8.05647945404053 4.0
0 2.82986092567444 5.0
-75.888526128 0.231943219900131 6.0
89.676863985 1.28977751731873 7.0
70.890248349 0.692308068275452 8.0
18.786615636 0.665494024753571 9.0
105.55595818 1.04942965507507 10.0
92.597335069 0.529710352420807 11.0
91.420061756 0.149805396795273 12.0
91.420061756 0.809073686599731 13.0
91.420061756 0.0434307530522346 14.0
0 0.189721047878265 15.0
0 0.142632633447647 16.0
0 1.26598834991455 17.0
49.264385004 1.39623641967773 18.0
-49.264385004 2.13822889328003 19.0
-49.264385004 0.463914930820465 20.0
-49.264385004 -0.0572742968797684 21.0
71.887299653 2.83222508430481 22.0
139.63296703 3.32409191131592 23.0
108.10503504 0.329893857240677 24.0
23.101938804 0.127243623137474 25.0
211.52026668 2.01034212112427 26.0
211.52026668 2.51598072052002 27.0
23.220555873 2.79314756393433 28.0
234.74082255 1.12315118312836 29.0
234.74082255 1.84951102733612 30.0
47.13176573 0.989024102687836 31.0
9.4806411853 0.535536885261536 32.0
22.622914649 1.56950008869171 33.0
22.622914649 1.05466723442078 34.0
22.622914649 0.126933351159096 35.0
9.7475341485 0.0322002097964287 36.0
8.4876932469 0.396292388439178 37.0
0 -0.000304955989122391 38.0
0 0.160162404179573 39.0
-10.361301738 0.819519937038422 40.0
-23.220658029 0.558845520019531 41.0
4.5985885781 0.383919775485992 42.0
-30.056799708 0.293490916490555 43.0
90.263476361 0.777271270751953 44.0
90.263476361 0.0972217619419098 45.0
7.5142874145 0.700733542442322 46.0
2.6184215394 0.376800030469894 47.0
152.04746836 0.355071008205414 48.0
145.97207181 0.511846542358398 49.0
0 0.201467901468277 50.0
89.676863985 0.202926695346832 51.0
105.55595818 0.786080777645111 52.0
91.420061756 0.0257230214774609 53.0
91.420061756 0.139035493135452 54.0
91.420061756 0.533367156982422 55.0
0 0.27712270617485 56.0
71.887299653 0.283204287290573 57.0
211.52026668 0.863911986351013 58.0
234.74082255 0.924062967300415 59.0
22.622914649 0.00568766705691814 60.0
0 0.0661701261997223 61.0
193.13582927 0.765448093414307 0.0
193.13582927 0.77509468793869 1.0
191.61698586 1.3685690164566 2.0
186.82788056 4.82923126220703 3.0
0 7.09267616271973 4.0
0 5.47910451889038 5.0
-64.43184496 3.09077024459839 6.0
76.031171091 0.695244252681732 7.0
63.207120052 1.20923185348511 8.0
12.824051038 1.20653593540192 9.0
80.980287328 1.10865569114685 10.0
68.050146135 0.520185351371765 11.0
75.933141882 0.161861896514893 12.0
75.933141882 0.809505581855774 13.0
75.933141882 0.135625690221786 14.0
0 0.147785753011703 15.0
0 0.206806182861328 16.0
0 1.53186678886414 17.0
53.964217771 2.12104558944702 18.0
-53.964217771 2.12689924240112 19.0
-53.964217771 1.10508215427399 20.0
-53.964217771 -0.118463553488255 21.0
73.977548607 1.70951581001282 22.0
152.79044501 1.84947633743286 23.0
123.00151757 0.590809941291809 24.0
23.353768516 0.103241115808487 25.0
226.76799361 0.952255845069885 26.0
226.76799361 2.34056711196899 27.0
29.848865185 2.89235949516296 28.0
256.6168588 1.6137216091156 29.0
256.6168588 2.05999493598938 30.0
47.440286785 0.927515745162964 31.0
9.4633977909 0.511233925819397 32.0
20.013330836 1.70018362998962 33.0
20.013330836 1.54276967048645 34.0
20.013330836 0.0756664052605629 35.0
8.6364456948 0.0199468545615673 36.0
7.3796599803 0.463315546512604 37.0
0 0.00891546159982681 38.0
0 0.245405554771423 39.0
-11.838661981 0.697496891021729 40.0
-23.126067255 0.957287609577179 41.0
5.0987296435 0.436872839927673 42.0
-30.590228392 0.538561284542084 43.0
77.47410063 0.692957282066345 44.0
77.47410063 0.0878803357481956 45.0
6.5632395226 0.638959944248199 46.0
1.7301184712 0.392594695091248 47.0
193.13582927 -0.035238616168499 48.0
186.82788056 0.176939100027084 49.0
0 0.324284046888351 50.0
76.031171091 0.345165014266968 51.0
80.980287328 0.614014565944672 52.0
75.933141882 0.0390745811164379 53.0
75.933141882 0.0695979073643684 54.0
75.933141882 0.375667333602905 55.0
0 8.12034034729004 56.0
73.977548607 0.31637305021286 57.0
226.76799361 0.580830812454224 58.0
256.6168588 1.1531126499176 59.0
20.013330836 -0.0219635292887688 60.0
0 0.0646670609712601 61.0
156.77814044 0.825352787971497 0.0
156.77814044 0.406342595815659 1.0
155.36471157 0.659972846508026 2.0
150.57560629 2.84235954284668 3.0
0 3.62012100219727 4.0
0 5.56410074234009 5.0
-73.969874053 1.33643472194672 6.0
87.417795796 1.35735976696014 7.0
62.765828366 1.11439120769501 8.0
24.651967429 0.739258885383606 9.0
98.09156016 1.03188705444336 10.0
84.626833685 0.389682769775391 11.0
82.216836679 0.600007474422455 12.0
82.216836679 0.199736297130585 13.0
82.216836679 0.137106001377106 14.0
0 0.16976110637188 15.0
0 0.215394884347916 16.0
0 1.47101521492004 17.0
58.576284253 1.60071325302124 18.0
-58.576284253 1.47459995746613 19.0
-58.576284253 0.745198845863342 20.0
-58.576284253 0.000927329063415527 21.0
77.872551677 1.80103409290314 22.0
147.24411399 1.79049623012543 23.0
115.36893275 0.357554972171783 24.0
18.692945277 0.090848296880722 25.0
225.11666567 1.50243055820465 26.0
225.11666567 2.1064190864563 27.0
23.736082012 2.25949668884277 28.0
248.85274768 0.99822610616684 29.0
248.85274768 1.62490379810333 30.0
46.389442727 0.944639265537262 31.0
9.3174559279 0.487324923276901 32.0
19.296267425 1.28935146331787 33.0
19.296267425 0.747938275337219 34.0
19.296267425 0.0648623332381248 35.0
8.2871943366 0.0223596673458815 36.0
7.1525371665 0.393074810504913 37.0
0 0.0120021030306816 38.0
0 0.127617314457893 39.0
-10.725404661 1.03206467628479 40.0
-21.932645239 0.486118525266647 41.0
5.556219979 0.375809222459793 42.0
-29.715745848 0.546796083450317 43.0
82.1286533 0.815770030021667 44.0
82.1286533 0.10186617821455 45.0
6.4377928649 0.535950899124146 46.0
2.0279520291 0.294730871915817 47.0
156.77814044 0.126531094312668 48.0
150.57560629 0.184315010905266 49.0
0 0.137916907668114 50.0
87.417795796 0.41127148270607 51.0
98.09156016 0.489990830421448 52.0
82.216836679 0.0769225060939789 53.0
82.216836679 0.048394788056612 54.0
82.216836679 0.0207030698657036 55.0
0 5.92188358306885 56.0
77.872551677 0.0975917503237724 57.0
225.11666567 0.770876884460449 58.0
248.85274768 0.906397104263306 59.0
19.296267425 0.018336484208703 60.0
0 0.104304350912571 61.0
158.77659834 0.719385743141174 0.0
158.77659834 0.865807056427002 1.0
157.41645869 1.17397403717041 2.0
151.9517879 3.44379329681396 3.0
22.910528625 2.76120352745056 4.0
22.910528625 3.45845508575439 5.0
-55.960930484 0.14995801448822 6.0
67.216493919 0.652918577194214 7.0
45.887171904 1.31786787509918 8.0
21.329322015 1.424973487854 9.0
92.198736334 0.765783429145813 10.0
79.599663846 0.189956426620483 11.0
82.546144766 0.812070369720459 12.0
82.546144766 0.0764884501695633 13.0
82.546144766 0.0585231333971024 14.0
6.3752186108 0.113013245165348 15.0
6.3752186108 0.174244791269302 16.0
0 1.14224290847778 17.0
49.970389576 1.40404677391052 18.0
-49.970389576 1.91431522369385 19.0
-49.970389576 0.748423457145691 20.0
-49.970389576 0.0473354049026966 21.0
71.277695769 1.4090256690979 22.0
139.7253343 1.77752876281738 23.0
102.56577808 0.313841789960861 24.0
20.669501101 0.127354323863983 25.0
211.00303006 1.00669646263123 26.0
211.00303006 1.58484768867493 27.0
31.125454981 1.00472915172577 28.0
242.12848505 0.526699066162109 29.0
242.12848505 1.08419942855835 30.0
44.893854982 0.821465253829956 31.0
8.1232786264 0.612260818481445 32.0
21.307306193 1.25698006153107 33.0
21.307306193 0.423175603151321 34.0
21.307306193 0.0403564237058163 35.0
8.8224481933 0.0312260799109936 36.0
7.4873948547 0.39890331029892 37.0
0 0.0940230637788773 38.0
0 0.135506749153137 39.0
-11.053549966 0.642994403839111 40.0
-24.790417817 0.348263263702393 41.0
4.5274675718 0.39531672000885 42.0
-31.959014436 0.299663215875626 43.0
73.757331559 0.565266072750092 44.0
73.757331559 0.102668486535549 45.0
6.3396342795 0.515802264213562 46.0
2.2544214096 0.423370599746704 47.0
158.77659834 0.27675724029541 48.0
151.9517879 0.329088538885117 49.0
22.910528625 0.3033686876297 50.0
67.216493919 0.264267951250076 51.0
92.198736334 0.108104862272739 52.0
82.546144766 0.0794160813093185 53.0
82.546144766 0.078883059322834 54.0
82.546144766 0.112877666950226 55.0
0 1.67039477825165 56.0
71.277695769 -0.0376061238348484 57.0
211.00303006 0.317658603191376 58.0
242.12848505 -0.0639015734195709 59.0
21.307306193 -0.0104117933660746 60.0
0 0.103497684001923 61.0
154.24940602 1.01726841926575 0.0
154.24940602 1.37685430049896 1.0
153.11288882 1.35535275936127 2.0
148.32378358 5.33640146255493 3.0
0 4.19674158096313 4.0
0 3.59234356880188 5.0
-58.037958988 0.720639646053314 6.0
67.406279221 0.717401325702667 7.0
44.375205548 0.7767413854599 8.0
23.031073673 0.594107210636139 9.0
76.884713579 1.00858426094055 10.0
63.303156341 0.0783364847302437 11.0
76.383728987 0.418450802564621 12.0
76.383728987 0.321256041526794 13.0
76.383728987 0.147231161594391 14.0
0 0.0761847421526909 15.0
0 0.194056168198586 16.0
0 1.04989016056061 17.0
57.959043739 1.38396036624908 18.0
-57.959043739 1.18840670585632 19.0
-57.959043739 0.292939484119415 20.0
-57.959043739 0.0813859403133392 21.0
80.062991474 2.33231258392334 22.0
147.9708381 2.58481431007385 23.0
116.8504469 0.647418975830078 24.0
22.667469509 0.123550973832607 25.0
228.03382958 1.41049540042877 26.0
228.03382958 1.56643509864807 27.0
28.807971717 1.7655622959137 28.0
256.84180129 0.402636259794235 29.0
256.84180129 1.67381227016449 30.0
47.139667893 0.899209558963776 31.0
9.2685117854 0.444208443164825 32.0
22.103947735 1.14413928985596 33.0
22.103947735 1.13856649398804 34.0
22.103947735 0.116445630788803 35.0
10.339112774 0.0474767610430717 36.0
9.0146835851 0.38953310251236 37.0
0 0.00899260491132736 38.0
0 0.289597511291504 39.0
-13.031784458 0.298338174819946 40.0
-25.623584483 1.0215003490448 41.0
5.0408177106 0.383051246404648 42.0
-33.310487568 0.698563814163208 43.0
90.838147963 0.772656559944153 44.0
90.838147963 0.0815867632627487 45.0
7.170824114 0.625388264656067 46.0
2.2360486823 0.438754945993423 47.0
154.24940602 0.105373710393906 48.0
148.32378358 0.29781100153923 49.0
0 0.221313923597336 50.0
67.406279221 0.202643424272537 51.0
76.884713579 0.151605620980263 52.0
76.383728987 0.0307778529822826 53.0
76.383728987 0.149239748716354 54.0
76.383728987 0.0693042948842049 55.0
0 8.12052726745605 56.0
80.062991474 0.00123188458383083 57.0
228.03382958 0.192777290940285 58.0
256.84180129 0.501330018043518 59.0
22.103947735 0.0103071089833975 60.0
0 0.0967583730816841 61.0
128.41650053 0.814762651920319 0.0
128.41650053 0.751485526561737 1.0
126.97403074 0.688214838504791 2.0
122.18492557 3.21231198310852 3.0
0 6.99061012268066 4.0
0 5.10165309906006 5.0
-65.413627496 0.344646990299225 6.0
77.535316172 0.550200998783112 7.0
48.683402705 0.525691032409668 8.0
28.851913467 0.616833627223969 9.0
85.577324402 1.0128790140152 10.0
73.224190489 0.566504240036011 11.0
78.360049024 0.364693313837051 12.0
78.360049024 0.0943331494927406 13.0
78.360049024 0.0499518997967243 14.0
0 -0.0036637494340539 15.0
0 0.213773474097252 16.0
0 1.13498866558075 17.0
55.585983015 2.13950443267822 18.0
-55.585983015 1.80472230911255 19.0
-55.585983015 0.371984630823135 20.0
-55.585983015 -0.0266429949551821 21.0
77.697868564 2.18280529975891 22.0
146.90829786 3.02425980567932 23.0
107.3687315 0.301497131586075 24.0
23.586996111 0.11393791437149 25.0
224.60616643 0.966265618801117 26.0
224.60616643 1.68780875205994 27.0
30.113709491 2.11276078224182 28.0
254.71987592 1.38863110542297 29.0
254.71987592 1.92024898529053 30.0
46.419081195 0.995157361030579 31.0
9.3989556981 0.386914789676666 32.0
22.111885549 1.6711186170578 33.0
22.111885549 0.98372346162796 34.0
22.111885549 0.108624018728733 35.0
9.4709131191 0.0542391911149025 36.0
8.1649209791 0.373038619756699 37.0
0 0.0122045651078224 38.0
0 0.311964392662048 39.0
-12.629902792 0.149011045694351 40.0
-25.760987781 0.581201016902924 41.0
5.375541835 0.420535892248154 42.0
-33.475045269 0.587937891483307 43.0
89.080175173 0.65890496969223 44.0
89.080175173 0.136024713516235 45.0
6.5158472706 0.374943107366562 46.0
2.1796529411 0.317200124263763 47.0
128.41650053 0.0317501686513424 48.0
122.18492557 -0.0107035618275404 49.0
0 0.568713665008545 50.0
77.535316172 0.170185089111328 51.0
85.577324402 0.395520627498627 52.0
78.360049024 0.0835542306303978 53.0
78.360049024 0.0873661637306213 54.0
78.360049024 0.114812687039375 55.0
0 5.88558483123779 56.0
77.697868564 0.0718897581100464 57.0
224.60616643 0.060061864554882 58.0
254.71987592 0.931742787361145 59.0
22.111885549 -0.00693004578351974 60.0
0 0.0849583446979523 61.0
164.93541688 0.985576510429382 0.0
164.93541688 1.49016225337982 1.0
163.60059343 2.15884590148926 2.0
158.81152348 3.84950256347656 3.0
0 5.73015403747559 4.0
0 5.26268672943115 5.0
-54.846852411 0.43523907661438 6.0
64.29095743 0.878010809421539 7.0
49.556993238 1.29896926879883 8.0
14.733964193 1.28480887413025 9.0
91.303789657 1.09035432338715 10.0
77.688269089 0.307675957679749 11.0
75.91844752 0.308470398187637 12.0
75.91844752 0.293342083692551 13.0
75.91844752 0.0952313393354416 14.0
0 0.0338317453861237 15.0
0 0.202694252133369 16.0
0 1.37776148319244 17.0
58.035250548 1.38994169235229 18.0
-58.035250548 1.43322563171387 19.0
-58.035250548 0.295140236616135 20.0
-58.035250548 0.0102636143565178 21.0
79.649433441 1.66688752174377 22.0
148.96428919 1.12795507907867 23.0
116.32823384 0.691778242588043 24.0
24.174841373 0.103859350085258 25.0
228.61372263 0.776602566242218 26.0
228.61372263 1.23534464836121 27.0
25.889326765 1.10349535942078 28.0
254.5030494 0.50034511089325 29.0
254.5030494 1.55496752262115 30.0
46.878851887 1.08924412727356 31.0
9.4657053198 0.461561799049377 32.0
21.614182893 1.24818229675293 33.0
21.614182893 0.799088418483734 34.0
21.614182893 0.139602944254875 35.0
9.6044926577 0.0389818213880062 36.0
8.3268979502 0.426609873771667 37.0
0 0.0142562836408615 38.0
0 0.168147146701813 39.0
-10.744309522 0.797293484210968 40.0
-24.181939512 0.820793569087982 41.0
5.1228309592 0.484992504119873 42.0
-31.97799134 0.418152391910553 43.0
91.243177428 0.757810115814209 44.0
91.243177428 0.0970860049128532 45.0
7.2949854638 0.692965507507324 46.0
2.7884935019 0.398349314928055 47.0
164.93541688 0.119362488389015 48.0
158.81152348 0.563076555728912 49.0
0 0.277632206678391 50.0
64.29095743 0.353722840547562 51.0
91.303789657 0.150856018066406 52.0
75.91844752 0.0571877621114254 53.0
75.91844752 0.131325259804726 54.0
75.91844752 0.0966861397027969 55.0
0 5.80315065383911 56.0
79.649433441 0.144212737679482 57.0
228.61372263 0.47183695435524 58.0
254.5030494 0.575648128986359 59.0
21.614182893 0.0228290017694235 60.0
0 0.0978146642446518 61.0
190.82999394 0.938874304294586 0.0
190.82999394 0.908664643764496 1.0
189.35565701 1.3777391910553 2.0
183.8910216 5.32123041152954 3.0
5.8672113403 8.21095848083496 4.0
5.8672113403 5.19716787338257 5.0
-63.188514737 3.64035367965698 6.0
73.630155177 0.646826088428497 7.0
39.646590457 0.894699335098267 8.0
33.98356472 0.648958265781403 9.0
92.321724789 1.00535595417023 10.0
77.974536287 0.0929572135210037 11.0
85.086042372 0.38564720749855 12.0
85.086042372 0.115489691495895 13.0
85.086042372 0.123483963310719 14.0
0 0.233069062232971 15.0
0 0.210867047309875 16.0
0 1.476811170578 17.0
59.797384946 1.55276370048523 18.0
-59.797384946 1.71268653869629 19.0
-59.797384946 0.310575693845749 20.0
-59.797384946 0.0836793333292007 21.0
79.99488792 1.47209692001343 22.0
135.89105095 0.76111251115799 23.0
103.244835 0.86740255355835 24.0
22.555853268 0.0862910076975822 25.0
215.88593887 1.25815093517303 26.0
215.88593887 2.22023940086365 27.0
23.847173168 2.46680879592896 28.0
239.73311204 1.44176650047302 29.0
239.73311204 1.78569149971008 30.0
46.463135419 1.17428398132324 31.0
9.2465808427 0.480489104986191 32.0
20.197502974 1.06688380241394 33.0
20.197502974 1.27785992622375 34.0
20.197502974 0.0975145250558853 35.0
9.0175086062 0.028348421677947 36.0
7.7184581652 0.427855312824249 37.0
0 0.0217104069888592 38.0
0 0.119318269193172 39.0
-11.014221201 0.693493127822876 40.0
-24.34965796 0.522555768489838 41.0
5.5726350208 0.378645181655884 42.0
-32.828857569 0.439019620418549 43.0
89.607063455 0.584267377853394 44.0
89.607063455 0.101039625704288 45.0
6.6242236627 0.508131802082062 46.0
2.7089404503 0.42983016371727 47.0
190.82999394 0.13495109975338 48.0
183.8910216 0.331151902675629 49.0
5.8672113403 0.303519695997238 50.0
73.630155177 0.168415680527687 51.0
92.321724789 0.0738664194941521 52.0
85.086042372 0.0166132654994726 53.0
85.086042372 0.0738271325826645 54.0
85.086042372 0.0478030815720558 55.0
0 3.6760048866272 56.0
79.99488792 0.0541807822883129 57.0
215.88593887 0.58183878660202 58.0
239.73311204 0.66940712928772 59.0
20.197502974 -0.0154742635786533 60.0
0 0.0910154581069946 61.0
183.40150083 0.788000822067261 0.0
183.40150083 0.734572231769562 1.0
181.92308236 1.44652652740479 2.0
177.13397724 3.34763312339783 3.0
0 4.89793348312378 4.0
0 5.11883211135864 5.0
-63.821436457 0.532033562660217 6.0
73.565067267 0.64551717042923 7.0
56.77094987 0.668699860572815 8.0
16.794117397 0.91100800037384 9.0
87.248419316 0.806345343589783 10.0
73.803692976 0.123606070876122 11.0
82.981459984 0.402198791503906 12.0
82.981459984 0.667751431465149 13.0
82.981459984 0.073241263628006 14.0
0 0.141507759690285 15.0
0 0.240446239709854 16.0
0 1.28200149536133 17.0
52.812759054 1.2633798122406 18.0
-52.812759054 0.879667937755585 19.0
-52.812759054 0.534929931163788 20.0
-52.812759054 0.014880396425724 21.0
74.533057707 1.56210851669312 22.0
148.83771792 1.13091540336609 23.0
114.38595001 0.518465578556061 24.0
24.278036016 0.200242668390274 25.0
223.37077562 0.613329887390137 26.0
223.37077562 1.38499808311462 27.0
26.927464535 1.07857966423035 28.0
250.29824016 0.642053008079529 29.0
250.29824016 1.58526659011841 30.0
47.01603211 1.01852297782898 31.0
9.432390595 0.39117705821991 32.0
21.720298654 1.63847863674164 33.0
21.720298654 1.0314816236496 34.0
21.720298654 0.121526226401329 35.0
9.6361805363 0.0220211837440729 36.0
8.3244270791 0.362062752246857 37.0
0 0.0282732956111431 38.0
0 0.17073529958725 39.0
-12.936936111 0.611647963523865 40.0
-25.231805419 0.762640058994293 41.0
5.5281433592 0.405013471841812 42.0
-33.124959507 0.482915997505188 43.0
91.620946389 0.723491907119751 44.0
91.620946389 0.0999489575624466 45.0
7.415659618 0.59820830821991 46.0
2.7468808044 0.34638911485672 47.0
183.40150083 0.110425688326359 48.0
177.13397724 0.35923832654953 49.0
0 0.277655392885208 50.0
73.565067267 0.282485604286194 51.0
87.248419316 0.14672914147377 52.0
82.981459984 0.0438729524612427 53.0
82.981459984 0.172820419073105 54.0
82.981459984 0.208572968840599 55.0
0 3.76084184646606 56.0
74.533057707 0.0380099713802338 57.0
223.37077562 0.462443947792053 58.0
250.29824016 0.150174841284752 59.0
21.720298654 -0.0201561376452446 60.0
0 0.0942502021789551 61.0
164.49980156 1.07273149490356 0.0
164.49980156 0.700789570808411 1.0
163.25465115 1.51496613025665 2.0
158.46554586 5.4420371055603 3.0
0 7.54931592941284 4.0
0 3.33749461174011 5.0
-53.989704431 1.96151554584503 6.0
64.690914388 0.558347463607788 7.0
62.672067972 0.680562496185303 8.0
2.0188464151 0.931827366352081 9.0
88.945709805 1.07221841812134 10.0
75.525412498 0.243416368961334 11.0
80.198758388 0.690714001655579 12.0
80.198758388 0.132760316133499 13.0
80.198758388 0.103275142610073 14.0
0 0.141966119408607 15.0
0 0.16340896487236 16.0
0 1.59837591648102 17.0
58.033383681 1.20638608932495 18.0
-58.033383681 0.857377886772156 19.0
-58.033383681 0.661931157112122 20.0
-58.033383681 -0.0227987300604582 21.0
80.101099769 3.01854825019836 22.0
155.30409082 3.09464836120605 23.0
117.51830875 0.185501217842102 24.0
21.799454298 0.314105540513992 25.0
235.40519059 1.22194290161133 26.0
235.40519059 1.99943828582764 27.0
25.701362728 2.47882604598999 28.0
261.10655332 1.38070952892303 29.0
261.10655332 2.05106616020203 30.0
46.393511985 1.04947447776794 31.0
9.4256823102 0.58279949426651 32.0
22.067716088 1.3371444940567 33.0
22.067716088 1.181969165802 34.0
22.067716088 0.143504053354263 35.0
9.4761926867 0.0459675379097462 36.0
8.1307702262 0.388314872980118 37.0
0 0.0454662665724754 38.0
0 0.372994631528854 39.0
-11.464906054 0.437465399503708 40.0
-24.20993056 0.481083512306213 41.0
5.3656101112 0.563865959644318 42.0
-32.248205442 0.600115299224854 43.0
83.090765792 0.627508282661438 44.0
83.090765792 0.1085414737463 45.0
7.2626859435 0.619017004966736 46.0
2.759279317 0.418785661458969 47.0
164.49980156 0.121225371956825 48.0
158.46554586 0.314429014921188 49.0
0 0.637736856937408 50.0
64.690914388 0.117806322872639 51.0
88.945709805 0.311892807483673 52.0
80.198758388 0.0821921229362488 53.0
80.198758388 0.132040247321129 54.0
80.198758388 0.0759207457304001 55.0
0 3.99149012565613 56.0
80.101099769 0.152105793356895 57.0
235.40519059 0.384066879749298 58.0
261.10655332 0.663991093635559 59.0
22.067716088 -0.00202016159892082 60.0
0 0.0937452018260956 61.0
150.81235424 1.07102453708649 0.0
150.81235424 1.85835313796997 1.0
149.72739688 1.98560810089111 2.0
144.9382918 3.95730042457581 3.0
0 7.63133001327515 4.0
0 4.86726093292236 5.0
-74.743636895 3.09320497512817 6.0
83.55478215 1.26972150802612 7.0
61.466626932 0.562057912349701 8.0
22.088155218 0.711700618267059 9.0
89.846827947 1.13412773609161 10.0
76.363850909 0.125067502260208 11.0
78.374080282 0.903843998908997 12.0
78.374080282 0.530636787414551 13.0
78.374080282 0.0917881205677986 14.0
0 0.0474363565444946 15.0
0 0.153276830911636 16.0
0 1.06836354732513 17.0
60.024907912 1.2061585187912 18.0
-60.024907912 1.13449776172638 19.0
-60.024907912 0.535467803478241 20.0
-60.024907912 0.0228685066103935 21.0
80.87495739 1.11265277862549 22.0
142.76569763 0.84730726480484 23.0
108.85942865 0.598598718643188 24.0
21.991252966 0.107418715953827 25.0
223.64065502 0.691053867340088 26.0
223.64065502 1.56259167194366 27.0
26.093130348 1.77270197868347 28.0
249.73378537 0.897925198078156 29.0
249.73378537 1.65807044506073 30.0
46.057794243 1.163538813591 31.0
9.2214154417 0.504581153392792 32.0
20.850049478 1.12949061393738 33.0
20.850049478 1.0528336763382 34.0
20.850049478 0.118491888046265 35.0
9.6711415102 0.0269910786300898 36.0
8.3779875982 0.404002785682678 37.0
0 0.0259696245193481 38.0
0 0.123337537050247 39.0
-12.515637463 1.00343072414398 40.0
-24.681987278 0.439475387334824 41.0
4.8570394375 0.603164315223694 42.0
-32.151027182 0.824267446994781 43.0
106.90909136 0.752127110958099 44.0
106.90909136 0.0693613737821579 45.0
6.605486855 0.646925628185272 46.0
2.3745248654 0.426933318376541 47.0
150.81235424 0.0967616066336632 48.0
144.9382918 0.199405908584595 49.0
0 0.627049565315247 50.0
83.55478215 0.160971164703369 51.0
89.846827947 0.198242723941803 52.0
78.374080282 0.0358937457203865 53.0
78.374080282 0.275446832180023 54.0
78.374080282 0.0489550307393074 55.0
0 3.63374900817871 56.0
80.87495739 0.0831416770815849 57.0
223.64065502 0.423114717006683 58.0
249.73378537 0.447610259056091 59.0
20.850049478 -0.000652402639389038 60.0
0 0.124685168266296 61.0
154.65597763 1.00032484531403 0.0
154.65597763 2.45155620574951 1.0
153.59213239 2.05413269996643 2.0
148.80302718 4.34281396865845 3.0
0 7.02847003936768 4.0
0 4.09591341018677 5.0
-63.087900589 1.73803925514221 6.0
71.991119052 0.909118473529816 7.0
54.01457709 0.643246173858643 8.0
17.976541962 0.442420542240143 9.0
81.559569994 1.2084755897522 10.0
69.810652781 0.3062724173069 11.0
76.101708506 0.923308253288269 12.0
76.101708506 0.482516020536423 13.0
76.101708506 0.127745971083641 14.0
0 0.0777715891599655 15.0
0 0.181327432394028 16.0
0 1.31874072551727 17.0
55.713820776 0.940588891506195 18.0
-55.713820776 0.596906185150146 19.0
-55.713820776 1.13238787651062 20.0
-55.713820776 -0.0943040549755096 21.0
78.017855036 2.14655470848083 22.0
145.59744137 1.82357728481293 23.0
111.90756013 0.30969363451004 24.0
24.139323073 0.228985279798508 25.0
223.61529641 1.31739866733551 26.0
223.61529641 1.30976402759552 27.0
29.067846918 1.32380414009094 28.0
252.68314332 0.817528486251831 29.0
252.68314332 1.56756317615509 30.0
46.24005952 1.10209143161774 31.0
9.2802977502 0.475199580192566 32.0
22.30403426 1.77546143531799 33.0
22.30403426 1.17111396789551 34.0
22.30403426 0.122986979782581 35.0
9.7398878555 0.0279401559382677 36.0
8.447586385 0.356579542160034 37.0
0 0.0326548963785172 38.0
0 0.17951212823391 39.0
-11.500884348 0.685409009456635 40.0
-23.8825039 0.582744896411896 41.0
5.4197222831 0.440452396869659 42.0
-31.99647009 0.738190829753876 43.0
102.87843125 0.721116602420807 44.0
102.87843125 0.141469165682793 45.0
7.2459042855 0.584481656551361 46.0
2.9393925523 0.376090884208679 47.0
154.65597763 0.105988971889019 48.0
148.80302718 0.234015047550201 49.0
0 0.259605288505554 50.0
71.991119052 0.075818844139576 51.0
81.559569994 0.272438794374466 52.0
76.101708506 0.130094632506371 53.0
76.101708506 0.100473709404469 54.0
76.101708506 0.0723965167999268 55.0
0 5.88417053222656 56.0
78.017855036 0.148052781820297 57.0
223.61529641 0.506936132907867 58.0
252.68314332 0.712168991565704 59.0
22.30403426 -0.0162248779088259 60.0
0 0.125207424163818 61.0
145.41412611 0.698932468891144 0.0
145.41412611 2.18508052825928 1.0
144.12046506 2.26291942596436 2.0
139.33139522 4.83305501937866 3.0
0 5.95507431030273 4.0
0 4.53607559204102 5.0
-61.116165897 0.372843950986862 6.0
73.744731752 1.0741138458252 7.0
51.965223096 0.565468907356262 8.0
21.779508655 0.602224588394165 9.0
80.804948425 0.90339720249176 10.0
68.310606984 0.186187207698822 11.0
78.335330681 0.363287389278412 12.0
78.335330681 0.280011117458344 13.0
78.335330681 0.170268982648849 14.0
0 0.189234629273415 15.0
0 0.176809310913086 16.0
0 1.42580759525299 17.0
58.064516649 1.34549582004547 18.0
-58.064516649 1.53913223743439 19.0
-58.064516649 0.369194149971008 20.0
-58.064516649 0.048793088644743 21.0
79.059131531 2.25976586341858 22.0
153.73324517 2.04889345169067 23.0
117.94662485 0.469829946756363 24.0
24.326714539 0.244491592049599 25.0
232.7923767 1.04494071006775 26.0
232.7923767 1.41651701927185 27.0
27.956686127 1.48883903026581 28.0
260.74906283 1.38518559932709 29.0
260.74906283 1.99062955379486 30.0
46.672618318 1.06326842308044 31.0
9.3805319216 0.524892866611481 32.0
20.994614882 1.5676497220993 33.0
20.994614882 1.11212205886841 34.0
20.994614882 0.132072225213051 35.0
9.4189328939 0.0399584993720055 36.0
8.2068275447 0.404403686523438 37.0
0 0.0351444631814957 38.0
0 0.196833223104477 39.0
-11.652258186 0.848542809486389 40.0
-25.595924207 0.570402264595032 41.0
5.4929447903 0.458661109209061 42.0
-33.621149075 0.653256237506866 43.0
95.822375111 0.67692893743515 44.0
95.822375111 0.0890148729085922 45.0
6.4016119291 0.587572753429413 46.0
2.4859186792 0.457231402397156 47.0
145.41412611 0.108118027448654 48.0
139.33139522 0.317175984382629 49.0
0 0.393134504556656 50.0
73.744731752 0.17005817592144 51.0
80.804948425 0.219125971198082 52.0
78.335330681 0.0401832684874535 53.0
78.335330681 0.0511866845190525 54.0
78.335330681 0.0825420543551445 55.0
0 5.73572731018066 56.0
79.059131531 0.0911089405417442 57.0
232.7923767 0.686455249786377 58.0
260.74906283 1.19441449642181 59.0
20.994614882 0.0135090984404087 60.0
0 0.104081496596336 61.0
202.58235083 0.933139383792877 0.0
202.58235083 1.5639386177063 1.0
201.49791021 1.95516014099121 2.0
196.70884025 4.48743534088135 3.0
0 4.6797251701355 4.0
0 5.28226613998413 5.0
-72.155974029 4.7717981338501 6.0
83.565148397 0.859715223312378 7.0
50.184725949 0.838363647460938 8.0
33.380422447 0.552107810974121 9.0
90.843273632 1.05637955665588 10.0
78.020225562 0.327596873044968 11.0
77.515175956 0.128598988056183 12.0
77.515175956 0.225160747766495 13.0
77.515175956 0.135280981659889 14.0
0 0.0971178859472275 15.0
0 0.165377303957939 16.0
0 1.56791925430298 17.0
48.34798212 1.67835783958435 18.0
-48.34798212 1.47741651535034 19.0
-48.34798212 0.59611439704895 20.0
-48.34798212 -0.0510411337018013 21.0
69.596897353 1.45976865291595 22.0
139.67919905 1.23206186294556 23.0
101.70333711 0.188513934612274 24.0
18.022085548 0.28632789850235 25.0
209.2760964 0.852185845375061 26.0
209.2760964 3.27184653282166 27.0
29.904991152 3.19152188301086 28.0
239.18108755 1.57936310768127 29.0
239.18108755 2.18902111053467 30.0
46.865120904 1.20124912261963 31.0
9.5125382272 0.433261543512344 32.0
21.248915233 1.2128769159317 33.0
21.248915233 1.21939432621002 34.0
21.248915233 0.13228651881218 35.0
9.5661380252 0.0446342900395393 36.0
8.1952352501 0.390800386667252 37.0
0 0.019839096814394 38.0
0 0.022561140358448 39.0
-12.221124401 0.862137854099274 40.0
-23.842499343 0.336696267127991 41.0
4.4992350738 0.773187935352325 42.0
-31.100378116 1.16093027591705 43.0
84.004800371 0.5789635181427 44.0
84.004800371 0.0834333822131157 45.0
6.5943903649 0.416661024093628 46.0
2.1426728711 0.416314601898193 47.0
202.58235083 0.116194665431976 48.0
196.70884025 0.312186747789383 49.0
0 0.268404453992844 50.0
83.565148397 0.241370871663094 51.0
90.843273632 0.0554753467440605 52.0
77.515175956 0.0694123432040215 53.0
77.515175956 0.0682131797075272 54.0
77.515175956 0.0632346048951149 55.0
0 8.12201404571533 56.0
69.596897353 0.132843792438507 57.0
209.2760964 0.256556242704391 58.0
239.18108755 1.05876886844635 59.0
21.248915233 -0.0235542319715023 60.0
0 0.114481642842293 61.0
134.24641655 0.832987308502197 0.0
134.24641655 1.54173147678375 1.0
133.05602707 1.78472876548767 2.0
128.26692181 4.16882705688477 3.0
0 4.63146781921387 4.0
0 4.36918210983276 5.0
-63.956841783 0.370714634656906 6.0
73.721729156 0.998454213142395 7.0
48.053895513 0.504417061805725 8.0
25.667833642 0.608784854412079 9.0
89.809178092 0.872387826442719 10.0
77.171451454 0.124167822301388 11.0
82.895709203 0.552663266658783 12.0
82.895709203 0.161888107657433 13.0
82.895709203 0.136484146118164 14.0
0 0.138180047273636 15.0
0 0.150969535112381 16.0
0 1.46758377552032 17.0
56.976196205 1.52892076969147 18.0
-56.976196205 1.19390654563904 19.0
-56.976196205 0.717115342617035 20.0
-56.976196205 -0.0137194730341434 21.0
78.430305775 1.5175085067749 22.0
147.55495442 1.3348468542099 23.0
110.29989837 0.383622229099274 24.0
21.336519253 0.179549440741539 25.0
225.98526019 1.03322494029999 26.0
225.98526019 1.75620472431183 27.0
29.7233967 1.66088330745697 28.0
255.70865689 0.961697220802307 29.0
255.70865689 2.12778949737549 30.0
46.745519543 1.20867526531219 31.0
9.2506330001 0.493614912033081 32.0
21.45410957 1.27853584289551 33.0
21.45410957 0.658487558364868 34.0
21.45410957 0.112358137965202 35.0
8.948835231 0.0261529758572578 36.0
7.7069675132 0.404350310564041 37.0
0 0.0163095965981483 38.0
0 0.145711123943329 39.0
-11.602357467 0.75224632024765 40.0
-24.346768023 0.586340129375458 41.0
5.1715254858 0.412475883960724 42.0
-32.252275521 0.479856669902802 43.0
91.021730333 0.829971313476562 44.0
91.021730333 0.098416730761528 45.0
7.4527752103 0.645909070968628 46.0
2.6349969119 0.380235493183136 47.0
134.24641655 0.126262336969376 48.0
128.26692181 0.50505930185318 49.0
0 0.350378274917603 50.0
73.721729156 0.20947602391243 51.0
89.809178092 0.149856626987457 52.0
82.895709203 0.031306266784668 53.0
82.895709203 0.158649429678917 54.0
82.895709203 0.0516800284385681 55.0
0 3.83252024650574 56.0
78.430305775 0.0578451603651047 57.0
225.98526019 0.510566234588623 58.0
255.70865689 0.679513096809387 59.0
21.45410957 0.00100665912032127 60.0
0 0.111894391477108 61.0
156.68419807 1.01179754734039 0.0
156.68419807 0.460503935813904 1.0
155.35111458 0.97082257270813 2.0
150.56200926 2.96427202224731 3.0
0 6.31318092346191 4.0
0 5.3071346282959 5.0
-61.437016818 4.18528747558594 6.0
72.939605512 0.598715543746948 7.0
52.253322156 1.1723085641861 8.0
20.686283356 0.89716112613678 9.0
91.712681544 1.01703751087189 10.0
78.599452635 0.242427706718445 11.0
81.021975522 0.542181134223938 12.0
81.021975522 0.685887694358826 13.0
81.021975522 0.0779980346560478 14.0
0 0.0611892938613892 15.0
0 0.187961876392365 16.0
0 1.2465273141861 17.0
53.218418507 1.06847310066223 18.0
-53.218418507 0.795942008495331 19.0
-53.218418507 0.592357516288757 20.0
-53.218418507 -0.0241041891276836 21.0
75.320786143 1.59003877639771 22.0
147.03427999 1.72376525402069 23.0
114.25712794 0.396303832530975 24.0
23.066760165 0.12408335506916 25.0
222.35506613 1.01524519920349 26.0
222.35506613 1.65524411201477 27.0
25.031463236 1.91889667510986 28.0
247.38652937 0.924154460430145 29.0
247.38652937 2.09067058563232 30.0
47.059395444 1.21911990642548 31.0
9.4059901614 0.587612807750702 32.0
22.102367637 1.2847775220871 33.0
22.102367637 0.976039826869965 34.0
22.102367637 0.109546519815922 35.0
9.5922355226 0.0265327468514442 36.0
8.2749521418 0.343246102333069 37.0
0 0.0219593718647957 38.0
0 0.170045047998428 39.0
-11.886341531 0.719165861606598 40.0
-23.966189927 0.53447151184082 41.0
4.9675229847 0.523830771446228 42.0
-31.37110925 0.667330980300903 43.0
82.330135259 0.539460420608521 44.0
82.330135259 0.0874156653881073 45.0
6.8570510596 0.616808593273163 46.0
2.8826409021 0.378653407096863 47.0
156.68419807 0.0442338474094868 48.0
150.56200926 0.157035410404205 49.0
0 0.895166635513306 50.0
72.939605512 0.19892954826355 51.0
91.712681544 0.0810416787862778 52.0
81.021975522 0.0986561626195908 53.0
81.021975522 0.0527518093585968 54.0
81.021975522 0.213168203830719 55.0
0 3.88831377029419 56.0
75.320786143 0.0890065580606461 57.0
222.35506613 0.390921294689178 58.0
247.38652937 0.472796946763992 59.0
22.102367637 -0.0039963610470295 60.0
0 0.107700027525425 61.0
181.84591128 1.16160273551941 0.0
181.84591128 0.398734569549561 1.0
180.4375574 0.625164449214935 2.0
176.25806416 4.84077119827271 3.0
0 10.4249134063721 4.0
0 5.45861673355103 5.0
-75.021522496 2.9120717048645 6.0
84.638774372 0.907850980758667 7.0
56.631625844 1.37533640861511 8.0
28.007148528 1.42980456352234 9.0
83.654348483 1.34874844551086 10.0
68.966406589 0.252556830644608 11.0
84.232842475 0.778203248977661 12.0
84.232842475 0.376303195953369 13.0
84.232842475 0.0984333157539368 14.0
0 0.159951820969582 15.0
0 0.211181163787842 16.0
0 1.30771827697754 17.0
64.355990006 0.739289164543152 18.0
-64.355990006 0.430018424987793 19.0
-65.004173864 0.994478106498718 20.0
-65.071669105 0.0051460973918438 21.0
85.376757101 1.75476896762848 22.0
147.31139112 1.96762311458588 23.0
109.3084336 0.650185406208038 24.0
23.517164022 0.0861218571662903 25.0
232.68814823 1.5198210477829 26.0
232.68814823 1.89263939857483 27.0
27.884356466 1.59656929969788 28.0
260.57250469 0.761644303798676 29.0
260.57250469 1.68897533416748 30.0
46.238195771 1.01426327228546 31.0
9.4905071109 0.587058901786804 32.0
20.305087995 1.45153558254242 33.0
20.305087995 1.1476491689682 34.0
20.305087995 0.11545267701149 35.0
9.4571198679 0.0252158287912607 36.0
8.1883165344 0.384907305240631 37.0
0 0.0161443315446377 38.0
0 0.146601095795631 39.0
-12.730853124 0.767201840877533 40.0
-26.917778229 0.749727189540863 41.0
4.660071354 0.354722797870636 42.0
-33.949308716 0.619209885597229 43.0
83.576950893 0.763314306735992 44.0
83.576950893 0.0863459855318069 45.0
6.9768821021 0.743935585021973 46.0
2.2578923246 0.356103658676147 47.0
181.84591128 0.185783803462982 48.0
176.25806416 0.163193464279175 49.0
0 1.82813286781311 50.0
84.638774372 0.451850980520248 51.0
83.654348483 0.414281994104385 52.0
84.232842475 0.0699983537197113 53.0
84.232842475 0.1459139585495 54.0
84.232842475 0.0821249336004257 55.0
0 1.0131471157074 56.0
85.376757101 0.125176921486855 57.0
232.68814823 0.536905407905579 58.0
260.57250469 0.92507129907608 59.0
20.305087995 0.00387781113386154 60.0
0 0.0710973739624023 61.0
152.55260916 0.934232354164124 0.0
152.55260916 1.54321050643921 1.0
151.17126619 2.16633939743042 2.0
145.70659567 4.40632820129395 3.0
16.769756789 4.38057518005371 4.0
16.769756789 4.18896532058716 5.0
-58.578149518 0.399149149656296 6.0
67.670557272 0.665806829929352 7.0
41.700402899 0.465758860111237 8.0
25.970154373 0.568419456481934 9.0
89.036679084 0.886207222938538 10.0
76.928831402 0.189044043421745 11.0
74.479284031 0.409969091415405 12.0
74.479284031 0.57741516828537 13.0
74.479284031 0.136624529957771 14.0
0 0.109133459627628 15.0
0 0.183387205004692 16.0
0 1.45610547065735 17.0
54.588080141 1.17036533355713 18.0
-54.588080141 0.759023785591125 19.0
-54.588080141 0.488544076681137 20.0
-54.588080141 0.0185616910457611 21.0
76.500777291 1.26781666278839 22.0
142.00509971 1.49123740196228 23.0
107.30773773 0.358912438154221 24.0
23.275796448 0.0820421427488327 25.0
218.505877 1.102379322052 26.0
218.505877 1.22364294528961 27.0
24.660033858 1.85385024547577 28.0
243.16591086 0.855289101600647 29.0
243.16591086 1.58352112770081 30.0
46.556061643 1.14419949054718 31.0
9.4120160795 0.356372445821762 32.0
21.91269715 1.15445852279663 33.0
21.91269715 0.895627021789551 34.0
21.91269715 0.125111654400826 35.0
9.837655915 0.0216481853276491 36.0
8.5236913388 0.438463866710663 37.0
0 0.0324757806956768 38.0
0 0.229016527533531 39.0
-12.377232271 0.580659151077271 40.0
-25.636914763 0.700820684432983 41.0
5.4682793643 0.448225528001785 42.0
-33.502563413 0.533842742443085 43.0
94.67282209 0.706074953079224 44.0
94.67282209 0.0939218327403069 45.0
6.3659347288 0.466491013765335 46.0
2.5325640294 0.36383044719696 47.0
152.55260916 0.0438808314502239 48.0
145.70659567 0.330266714096069 49.0
16.769756789 0.464507639408112 50.0
67.670557272 0.104315288364887 51.0
89.036679084 0.0544742941856384 52.0
74.479284031 0.0752904862165451 53.0
74.479284031 0.252745985984802 54.0
74.479284031 -0.00362412072718143 55.0
0 5.61069297790527 56.0
76.500777291 0.0788838714361191 57.0
218.505877 0.336449950933456 58.0
243.16591086 0.190518021583557 59.0
21.91269715 0.015373220667243 60.0
0 0.10365928709507 61.0
193.74452147 1.29294395446777 0.0
193.74452147 2.07453155517578 1.0
192.45436699 1.97859537601471 2.0
191.39838655 4.64389896392822 3.0
0 6.90922164916992 4.0
0 4.54296684265137 5.0
-83.708274178 0.500440776348114 6.0
94.148109518 0.565251588821411 7.0
64.003469173 1.00448286533356 8.0
30.144640345 1.08575212955475 9.0
88.864530414 1.63721024990082 10.0
75.648811188 0.301854938268661 11.0
83.154830619 1.22241818904877 12.0
83.154830619 0.115875698626041 13.0
83.154830619 0.0905372723937035 14.0
0 0.199952900409698 15.0
0 0.202885851264 16.0
0 1.1641092300415 17.0
65.473858738 1.40232169628143 18.0
-65.473858738 1.35263848304749 19.0
-66.14212648 0.509729266166687 20.0
-66.210688079 -0.0183523260056973 21.0
85.985105095 1.85268688201904 22.0
147.91133403 2.15044903755188 23.0
113.347986 0.347092717885971 24.0
21.078513532 0.203217208385468 25.0
233.89643912 1.41152811050415 26.0
233.89643912 2.12314343452454 27.0
24.218940789 1.57135057449341 28.0
258.11537991 0.90205591917038 29.0
258.11537991 1.88627696037292 30.0
46.451290447 1.09418332576752 31.0
9.4602908507 0.547684133052826 32.0
19.774417017 2.20334529876709 33.0
19.774417017 0.504321753978729 34.0
19.774417017 0.141542211174965 35.0
9.4672341843 0.0256024934351444 36.0
8.3045149352 0.369268178939819 37.0
0 0.0721996873617172 38.0
0 0.271811842918396 39.0
-12.560850216 0.263907134532928 40.0
-24.515754737 0.759567975997925 41.0
5.5652834109 0.362890124320984 42.0
-32.530621884 0.331426501274109 43.0
113.03840435 0.964200735092163 44.0
111.80167134 0.0448369979858398 45.0
6.5005127764 0.725055992603302 46.0
2.0365593941 0.332836866378784 47.0
193.74452147 0.128800243139267 48.0
191.39838655 0.109802201390266 49.0
0 0.220853760838509 50.0
94.148109518 0.284732401371002 51.0
88.864530414 0.503647327423096 52.0
83.154830619 0.114085577428341 53.0
83.154830619 0.220101147890091 54.0
83.154830619 0.0667601972818375 55.0
0 2.86047744750977 56.0
85.985105095 -0.0621340461075306 57.0
233.89643912 0.405774176120758 58.0
258.11537991 0.759085536003113 59.0
19.774417017 -0.0190994590520859 60.0
0 0.149395242333412 61.0
147.31239673 0.871568083763123 0.0
147.31239673 0.666943967342377 1.0
145.89314175 0.811727106571198 2.0
141.10403646 4.88939619064331 3.0
0 5.67605018615723 4.0
0 5.95006942749023 5.0
-54.093908737 0.624786019325256 6.0
62.416927133 1.14516127109528 7.0
60.161404291 1.50865542888641 8.0
2.2555228419 1.30338859558105 9.0
76.092312451 0.868423342704773 10.0
64.725710661 -0.00537145603448153 11.0
70.777171578 0.653984844684601 12.0
70.777171578 0.100270643830299 13.0
70.777171578 0.212819218635559 14.0
0 0.0937203913927078 15.0
0 0.243831753730774 16.0
0 1.31652462482452 17.0
45.410365793 1.52851927280426 18.0
-45.410365793 1.37275087833405 19.0
-45.410365793 0.417401671409607 20.0
-45.410365793 -0.0418826788663864 21.0
66.165877812 2.4697380065918 22.0
131.81918802 2.1763870716095 23.0
99.376393976 0.448812931776047 24.0
25.687347685 0.0818744152784348 25.0
197.98506584 1.10485744476318 26.0
197.98506584 2.12124419212341 27.0
25.834001521 2.64330005645752 28.0
223.81906736 1.733962059021 29.0
223.81906736 1.83511185646057 30.0
46.700114835 1.00310587882996 31.0
9.4374202199 0.380270659923553 32.0
20.755512019 0.882941901683807 33.0
20.755512019 1.44924795627594 34.0
20.755512019 0.102947644889355 35.0
10.399655975 0.024939626455307 36.0
9.1556749006 0.344239920377731 37.0
0 -0.000670220702886581 38.0
0 0.187822997570038 39.0
-14.082874669 0.259382486343384 40.0
-26.119418236 0.801411390304565 41.0
5.3460390133 0.268571048974991 42.0
-33.737687271 0.312961459159851 43.0
79.115578509 0.696880400180817 44.0
79.115578509 0.100396238267422 45.0
6.252208081 0.656566143035889 46.0
2.0051102425 0.400883913040161 47.0
147.31239673 -0.00146050192415714 48.0
141.10403646 0.401953339576721 49.0
0 0.701266765594482 50.0
62.416927133 0.464027583599091 51.0
76.092312451 0.178245142102242 52.0
70.777171578 0.00859960447996855 53.0
70.777171578 0.041618462651968 54.0
70.777171578 0.0759937465190887 55.0
0 11.7246618270874 56.0
66.165877812 0.19232140481472 57.0
197.98506584 0.279775977134705 58.0
223.81906736 0.607921779155731 59.0
20.755512019 -0.0219760090112686 60.0
0 0.101925678551197 61.0
187.68349625 0.808481991291046 0.0
187.68349625 0.747288644313812 1.0
186.6452691 0.887272477149963 2.0
181.85616394 3.51279902458191 3.0
0 8.55575084686279 4.0
0 3.16761565208435 5.0
-75.855656742 0.326837360858917 6.0
89.00682781 0.790011465549469 7.0
53.938018166 0.663551032543182 8.0
35.068809644 0.525829911231995 9.0
97.801493632 0.876161754131317 10.0
84.689257381 0.20146706700325 11.0
83.295930287 0.624177753925323 12.0
83.295930287 0.165024667978287 13.0
83.295930287 0.0604065135121346 14.0
0 0.0175580643117428 15.0
0 0.18250846862793 16.0
0 0.970390260219574 17.0
63.232618709 1.22456359863281 18.0
-63.232618709 0.837165892124176 19.0
-63.232618709 0.33735853433609 20.0
-63.232618709 0.00462202355265617 21.0
84.756415518 2.54191517829895 22.0
140.21490849 2.57257461547852 23.0
109.65059255 0.358401924371719 24.0
24.130045713 0.101965844631195 25.0
224.97132401 1.33978748321533 26.0
224.97132401 1.92998814582825 27.0
24.016684008 1.36975705623627 28.0
248.98800802 0.793258190155029 29.0
248.98800802 1.78875398635864 30.0
47.264917877 1.02974534034729 31.0
9.3588517853 0.60925555229187 32.0
21.523796809 1.51301217079163 33.0
21.523796809 0.901291012763977 34.0
21.523796809 0.0907574668526649 35.0
9.9183173671 0.024463189765811 36.0
8.590365507 0.398232847452164 37.0
0 0.0108818151056767 38.0
0 0.0871398746967316 39.0
-11.746607272 0.708762586116791 40.0
-23.758801782 0.265378087759018 41.0
4.7814863596 0.47603714466095 42.0
-31.167177321 0.660469114780426 43.0
86.377252766 0.781805992126465 44.0
86.377252766 0.0823140516877174 45.0
7.7400572348 0.578417241573334 46.0
2.8522257325 0.333063393831253 47.0
187.68349625 0.104909241199493 48.0
181.85616394 0.141087472438812 49.0
0 0.357124656438828 50.0
89.00682781 0.0340732410550117 51.0
97.801493632 0.0931102335453033 52.0
83.295930287 0.129539608955383 53.0
83.295930287 0.0321544148027897 54.0
83.295930287 0.0909767746925354 55.0
0 2.72830724716187 56.0
84.756415518 0.045019805431366 57.0
224.97132401 0.802410185337067 58.0
248.98800802 0.573449015617371 59.0
21.523796809 -0.000460315495729446 60.0
0 0.0908416658639908 61.0
188.29759342 0.697325527667999 0.0
188.29759342 1.42735886573792 1.0
187.11445973 1.6049211025238 2.0
181.64978901 6.71056985855103 3.0
4.1624851492 5.61585569381714 4.0
4.1624851492 5.50005435943604 5.0
-60.337619313 0.620691776275635 6.0
72.179957477 0.443857878446579 7.0
41.078333286 0.69526606798172 8.0
31.101624191 0.839633584022522 9.0
82.703847412 0.810856878757477 10.0
68.969791406 0.240467190742493 11.0
74.477283298 0.929369628429413 12.0
74.477283298 0.708372175693512 13.0
74.477283298 0.0509275496006012 14.0
13.919405726 0.101049900054932 15.0
13.919405726 0.147025257349014 16.0
0 1.25886428356171 17.0
52.529846828 1.12157964706421 18.0
-52.529846828 1.04164659976959 19.0
-52.529846828 1.18344604969025 20.0
-52.529846828 -0.0130447559058666 21.0
73.652558323 1.4266037940979 22.0
145.22226813 1.50149667263031 23.0
113.42053697 0.317232578992844 24.0
26.335745074 0.175169169902802 25.0
218.87482645 1.04948449134827 26.0
218.87482645 1.99457049369812 27.0
26.804761056 1.71042990684509 28.0
245.67958751 0.29804190993309 29.0
245.67958751 1.66558921337128 30.0
46.356576487 1.04533350467682 31.0
9.3596566312 0.558856904506683 32.0
21.122711495 1.59154510498047 33.0
21.122711495 1.47555840015411 34.0
21.122711495 0.120693802833557 35.0
9.0280051119 0.018643157556653 36.0
7.7729463391 0.336322635412216 37.0
0 -0.0156224071979523 38.0
0 0.232533916831017 39.0
-12.435230314 0.486252874135971 40.0
-25.112694683 0.597836077213287 41.0
4.5422461661 0.446813702583313 42.0
-32.431135389 0.718193233013153 43.0
80.236882253 0.478210061788559 44.0
80.236882253 0.107916846871376 45.0
6.874461332 0.351767927408218 46.0
2.0822054428 0.347241640090942 47.0
188.29759342 0.158746346831322 48.0
181.64978901 0.287262320518494 49.0
4.1624851492 0.326973646879196 50.0
72.179957477 0.250471115112305 51.0
82.703847412 0.192208006978035 52.0
74.477283298 0.0745158046483994 53.0
74.477283298 0.277863621711731 54.0
74.477283298 0.205270901322365 55.0
0 0.19593271613121 56.0
73.652558323 -0.0305659472942352 57.0
218.87482645 0.632931292057037 58.0
245.67958751 0.211214140057564 59.0
21.122711495 -0.00261072441935539 60.0
0 0.109790503978729 61.0
127.4916113 0.958369076251984 0.0
127.4916113 0.629799962043762 1.0
126.52838225 1.04975426197052 2.0
121.7392765 3.81734609603882 3.0
0 3.41736459732056 4.0
0 5.52242374420166 5.0
-63.097147267 0.571940898895264 6.0
72.874377847 1.19659280776978 7.0
46.022940366 0.770034670829773 8.0
26.851437481 0.686027526855469 9.0
87.036058425 1.06366109848022 10.0
75.097335903 0.0663071423768997 11.0
83.126312792 0.145950898528099 12.0
83.126312792 0.629225730895996 13.0
83.126312792 0.0800570994615555 14.0
0 0.19052329659462 15.0
0 0.186988532543182 16.0
0 1.14112603664398 17.0
56.825246185 1.42886328697205 18.0
-56.825246185 1.51194131374359 19.0
-56.825246185 0.767138123512268 20.0
-56.825246185 -0.0455877855420113 21.0
76.964175684 1.93782389163971 22.0
148.77542482 0.942617118358612 23.0
109.38717082 0.440265834331512 24.0
22.377913525 0.139547973871231 25.0
225.7396005 1.26081192493439 26.0
225.7396005 1.32726919651031 27.0
29.785755768 1.58474230766296 28.0
255.52535627 0.967039465904236 29.0
255.52535627 1.75165605545044 30.0
46.158130899 1.00744044780731 31.0
9.4528731617 0.563015282154083 32.0
20.138929499 1.30765283107758 33.0
20.138929499 1.08691453933716 34.0
20.138929499 0.122672528028488 35.0
9.6249668229 0.0367083251476288 36.0
8.269788499 0.365840524435043 37.0
0 0.00958218425512314 38.0
0 0.0819207355380058 39.0
-10.358963872 0.365673840045929 40.0
-23.097464629 0.695532262325287 41.0
5.0464059739 0.431182771921158 42.0
-30.946316389 0.583572924137115 43.0
93.549873285 0.793605864048004 44.0
93.549873285 0.076169528067112 45.0
6.676171415 0.663071036338806 46.0
2.5656142879 0.435931354761124 47.0
127.4916113 0.0714109465479851 48.0
121.7392765 0.246562734246254 49.0
0 0.446945518255234 50.0
72.874377847 0.220766499638557 51.0
87.036058425 0.277556627988815 52.0
83.126312792 0.020507600158453 53.0
83.126312792 0.0931515991687775 54.0
83.126312792 0.0245086010545492 55.0
0 1.69210815429688 56.0
76.964175684 0.0266825333237648 57.0
225.7396005 0.685669481754303 58.0
255.52535627 0.919163942337036 59.0
20.138929499 0.0256497822701931 60.0
0 0.169872298836708 61.0
138.44899983 1.01729953289032 0.0
138.44899983 0.369041979312897 1.0
137.1076398 0.775238454341888 2.0
132.31856985 3.02255082130432 3.0
0 5.76577425003052 4.0
0 3.61005973815918 5.0
-49.112778331 2.66251468658447 6.0
61.838781402 1.18190360069275 7.0
47.60953264 0.632976353168488 8.0
14.229248762 0.480105847120285 9.0
90.181261421 1.06473827362061 10.0
77.534808871 0.100724220275879 11.0
83.608301836 0.185162216424942 12.0
83.608301836 0.0896704494953156 13.0
83.608301836 0.183953642845154 14.0
0 0.305242955684662 15.0
0 0.14318622648716 16.0
0 2.38113880157471 17.0
51.448064591 1.01566433906555 18.0
-51.448064591 0.550202608108521 19.0
-51.448064591 0.501964151859283 20.0
-51.448064591 0.102497287094593 21.0
74.363783406 1.62297856807709 22.0
146.69875953 1.28114068508148 23.0
112.38109842 0.65344226360321 24.0
21.937864378 0.130961805582047 25.0
221.06254294 0.774340331554413 26.0
221.06254294 1.16046643257141 27.0
25.230103872 0.95647656917572 28.0
246.29264681 0.43238827586174 29.0
246.29264681 1.5348402261734 30.0
46.3246311 0.943600356578827 31.0
9.2983799427 0.548770487308502 32.0
22.915718814 1.14265489578247 33.0
22.915718814 0.850806355476379 34.0
22.915718814 0.155245408415794 35.0
9.1461464896 0.0329259410500526 36.0
7.977295133 0.357095122337341 37.0
0 0.0154830515384674 38.0
0 0.177997648715973 39.0
-12.287149926 0.675751149654388 40.0
-24.843572016 0.773726224899292 41.0
4.8834761274 0.41037780046463 42.0
-32.521698354 0.566385269165039 43.0
90.720397043 0.484089016914368 44.0
90.720397043 0.0685718059539795 45.0
6.7085691983 0.369687378406525 46.0
2.8003282584 0.297579348087311 47.0
138.44899983 0.102539926767349 48.0
132.31856985 0.218104392290115 49.0
0 0.522173225879669 50.0
61.838781402 0.137063950300217 51.0
90.181261421 0.0631463304162025 52.0
83.608301836 0.0455732941627502 53.0
83.608301836 0.0849043130874634 54.0
83.608301836 0.0441530868411064 55.0
0 3.29888105392456 56.0
74.363783406 -0.0471055060625076 57.0
221.06254294 0.24922239780426 58.0
246.29264681 0.400054395198822 59.0
22.915718814 0.0111877731978893 60.0
0 0.117569386959076 61.0
173.838775 0.918067336082458 0.0
173.838775 0.896126925945282 1.0
172.66153324 1.27027595043182 2.0
167.87242791 4.64421319961548 3.0
0 6.13390350341797 4.0
0 4.17979955673218 5.0
-47.458094566 2.92348766326904 6.0
58.136277571 1.18032944202423 7.0
36.426654256 1.11610174179077 8.0
21.709623314 0.777496516704559 9.0
94.021862162 1.04437077045441 10.0
80.893498967 0.205538406968117 11.0
93.369748316 0.250353246927261 12.0
93.369748316 0.455589056015015 13.0
93.369748316 0.16698032617569 14.0
0 0.170159861445427 15.0
0 0.102807328104973 16.0
0 1.01064145565033 17.0
55.886781468 1.32454061508179 18.0
-55.886781468 1.6887845993042 19.0
-55.886781468 0.599684834480286 20.0
-55.886781468 0.179352447390556 21.0
76.696451501 2.14614844322205 22.0
152.57696846 2.42645835876465 23.0
114.25407436 0.293348461389542 24.0
24.016166779 0.340856492519379 25.0
229.27341996 0.902436017990112 26.0
229.27341996 1.8876348733902 27.0
26.45902329 1.44156801700592 28.0
255.73244325 0.76216322183609 29.0
255.73244325 2.14048528671265 30.0
47.657592287 1.15224969387054 31.0
9.4922159513 0.561199426651001 32.0
20.809670033 1.15572452545166 33.0
20.809670033 1.16804814338684 34.0
20.809670033 0.157807752490044 35.0
9.3046142055 0.0441363900899887 36.0
7.9657477873 0.382224917411804 37.0
0 0.00969903171062469 38.0
0 0.196072369813919 39.0
-13.368777061 0.770963966846466 40.0
-27.423028876 0.532726466655731 41.0
4.7583656909 0.57720673084259 42.0
-34.954182833 0.648158550262451 43.0
74.917428587 0.644970595836639 44.0
74.917428587 0.101199567317963 45.0
7.5791977574 0.641603469848633 46.0
2.2707864338 0.412755936384201 47.0
173.838775 0.207284137606621 48.0
167.87242791 0.332228988409042 49.0
0 0.661112785339355 50.0
58.136277571 0.557117700576782 51.0
94.021862162 0.455577731132507 52.0
93.369748316 0.0384210348129272 53.0
93.369748316 0.113480053842068 54.0
93.369748316 0.157740190625191 55.0
0 0.221137002110481 56.0
76.696451501 -0.0160077195614576 57.0
229.27341996 0.613829553127289 58.0
255.73244325 1.05645906925201 59.0
20.809670033 0.0196447297930717 60.0
0 0.106227390468121 61.0
146.4037254 0.826582908630371 0.0
146.4037254 1.30222880840302 1.0
145.16932945 2.22317671775818 2.0
139.70465882 4.36261606216431 3.0
0.50597267115 4.47880601882935 4.0
0.50597267115 6.64191341400146 5.0
-73.683585497 4.2469654083252 6.0
86.271194116 0.449371427297592 7.0
44.539350689 0.682938575744629 8.0
41.731843427 0.749267995357513 9.0
87.262021213 0.772320866584778 10.0
75.121264576 0.048182912170887 11.0
84.592019311 0.377824902534485 12.0
84.592019311 0.291904747486115 13.0
84.592019311 0.126457557082176 14.0
0 0.223571076989174 15.0
0 0.14090883731842 16.0
0 1.34909343719482 17.0
52.305432374 1.50090038776398 18.0
-52.305432374 1.54490113258362 19.0
-52.305432374 0.560105204582214 20.0
-52.305432374 -0.0749591812491417 21.0
74.010609989 1.50839340686798 22.0
148.09598956 1.28863191604614 23.0
116.68590088 0.342101186513901 24.0
21.58249539 0.205629244446754 25.0
222.10659955 0.927638590335846 26.0
222.10659955 1.77057445049286 27.0
24.664991192 0.918805122375488 28.0
246.77159074 0.692277908325195 29.0
246.77159074 1.63249707221985 30.0
46.799370209 0.988644957542419 31.0
9.2166360455 0.57924348115921 32.0
21.705177615 1.7191469669342 33.0
21.705177615 0.959568083286285 34.0
21.705177615 0.119142778217793 35.0
9.5251931382 0.0287621710449457 36.0
8.1496056292 0.306510716676712 37.0
0 0.0570428445935249 38.0
0 0.138021111488342 39.0
-11.773240986 0.775962889194489 40.0
-23.86978804 0.592466115951538 41.0
5.2699223464 0.407269209623337 42.0
-31.659940663 0.508207678794861 43.0
105.62295827 0.847413837909698 44.0
105.62295827 0.0996577218174934 45.0
7.0774380879 0.582772254943848 46.0
2.573593059 0.340159088373184 47.0
146.4037254 0.0905453860759735 48.0
139.70465882 0.223509237170219 49.0
0.50597267115 0.335552185773849 50.0
86.271194116 0.209041714668274 51.0
87.262021213 0.144945278763771 52.0
84.592019311 0.00984116084873676 53.0
84.592019311 0.114827930927277 54.0
84.592019311 0.044596254825592 55.0
0 4.14082527160645 56.0
74.010609989 0.140609189867973 57.0
222.10659955 0.414325535297394 58.0
246.77159074 0.484112590551376 59.0
21.705177615 -0.00515452399849892 60.0
0 0.0824187695980072 61.0
190.32242274 0.715112328529358 0.0
190.32242274 1.4668220281601 1.0
189.29367714 1.69138157367706 2.0
183.82904173 7.66780471801758 3.0
8.064687472 4.05087566375732 4.0
8.064687472 4.54559421539307 5.0
-63.806073745 0.418885171413422 6.0
74.554976625 0.445150554180145 7.0
32.46423668 1.38771963119507 8.0
42.090739945 1.10051369667053 9.0
89.56993802 0.741160035133362 10.0
77.264886016 0.0465504191815853 11.0
79.094709806 0.701699018478394 12.0
79.094709806 0.358869135379791 13.0
79.094709806 0.0556423664093018 14.0
2.7247727146 0.089986652135849 15.0
2.7247727146 0.18060040473938 16.0
0 1.48575818538666 17.0
46.059573104 1.59441483020782 18.0
-46.059573104 2.13418292999268 19.0
-46.059573104 0.424869120121002 20.0
-46.059573104 -0.02676060795784 21.0
67.295219035 2.60916042327881 22.0
156.44606065 2.69774889945984 23.0
121.99039744 0.281882613897324 24.0
23.676768964 0.476471185684204 25.0
223.74127969 1.33777034282684 26.0
223.74127969 1.78103959560394 27.0
25.368428136 0.72022408246994 28.0
249.10970783 0.429794132709503 29.0
249.10970783 1.38270032405853 30.0
46.612654862 1.12339794635773 31.0
9.1620053226 0.410318672657013 32.0
21.235645931 1.13435649871826 33.0
21.235645931 0.815032482147217 34.0
21.235645931 0.145314514636993 35.0
9.1591843889 0.0331089571118355 36.0
8.0168041117 0.25035485625267 37.0
0 0.0148498453199863 38.0
0 0.28281956911087 39.0
-11.313299402 0.156808525323868 40.0
-24.748577794 0.881169617176056 41.0
5.9146486824 0.483293533325195 42.0
-32.924598278 0.440673410892487 43.0
93.611145207 0.980163633823395 44.0
93.611145207 0.0972991287708282 45.0
7.1575612675 0.60217958688736 46.0
2.6965465279 0.392896145582199 47.0
190.32242274 0.0897810384631157 48.0
183.82904173 0.301121771335602 49.0
8.064687472 0.323673009872437 50.0
74.554976625 0.0641635805368423 51.0
89.56993802 0.00787229649722576 52.0
79.094709806 0.0533177778124809 53.0
79.094709806 0.020986819639802 54.0
79.094709806 0.0680322498083115 55.0
0 2.07320189476013 56.0
67.295219035 0.0783030539751053 57.0
223.74127969 0.649392902851105 58.0
249.10970783 -0.0686037465929985 59.0
21.235645931 0.0135414022952318 60.0
0 0.131488367915154 61.0
168.32209966 0.830341279506683 0.0
168.32209966 0.530393123626709 1.0
166.97140963 1.21013283729553 2.0
161.50677423 2.42811918258667 3.0
0.083898902066 3.32636880874634 4.0
0.083898902066 6.86327791213989 5.0
-66.491165834 0.828922927379608 6.0
74.30069182 0.848299026489258 7.0
49.54939451 0.874911189079285 8.0
24.75129731 1.07794439792633 9.0
100.21696443 0.963586390018463 10.0
87.599770556 0.645306408405304 11.0
85.927066999 0.439434766769409 12.0
85.927066999 0.496589064598083 13.0
85.927066999 0.156051352620125 14.0
0 0.0749086290597916 15.0
0 0.134963124990463 16.0
0 1.81805896759033 17.0
60.160267892 1.54048407077789 18.0
-60.160267892 1.08242726325989 19.0
-60.160267892 0.867560803890228 20.0
-60.160267892 -0.0742328390479088 21.0
81.461064216 1.90329372882843 22.0
134.33422692 1.80581867694855 23.0
102.0068816 0.409645587205887 24.0
23.021072935 0.0706063359975815 25.0
215.79529114 0.369644165039062 26.0
215.79529114 1.33714544773102 27.0
21.776327932 1.29908430576324 28.0
237.57161907 1.00742185115814 29.0
237.57161907 2.03244352340698 30.0
47.197762289 0.926695168018341 31.0
9.3380138401 0.327256143093109 32.0
21.300796324 1.30133271217346 33.0
21.300796324 1.35607385635376 34.0
21.300796324 0.129169806838036 35.0
7.9857192225 0.037283331155777 36.0
6.8850343918 0.344921857118607 37.0
0 -0.0144319105893373 38.0
0 0.074849933385849 39.0
-10.106070968 0.443507701158524 40.0
-21.466268835 0.675698935985565 41.0
4.795572669 0.4244504570961 42.0
-29.056781118 0.592955827713013 43.0
84.876265819 0.859764873981476 44.0
84.876265819 0.107929356396198 45.0
7.1273519994 0.565093994140625 46.0
2.7501155413 0.361229091882706 47.0
168.32209966 0.0901674181222916 48.0
161.50677423 0.122521720826626 49.0
0.083898902066 0.402716755867004 50.0
74.30069182 0.289712965488434 51.0
100.21696443 0.447548866271973 52.0
85.927066999 0.207529947161674 53.0
85.927066999 0.058056678622961 54.0
85.927066999 0.013559197075665 55.0
0 2.96512055397034 56.0
81.461064216 0.124844990670681 57.0
215.79529114 0.496892601251602 58.0
237.57161907 1.15798628330231 59.0
21.300796324 -0.00685009453445673 60.0
0 0.121749624609947 61.0
154.67558141 0.782840847969055 0.0
154.67558141 1.6240394115448 1.0
153.78115565 2.40684151649475 2.0
148.99205046 2.62969398498535 3.0
0 3.57617044448853 4.0
0 5.60965156555176 5.0
-55.814207065 4.64912557601929 6.0
69.213222159 0.583378195762634 7.0
45.871794079 1.08250164985657 8.0
23.34142808 1.26438343524933 9.0
99.338491111 0.832952797412872 10.0
86.176319565 0.182796448469162 11.0
81.9905506 1.098219871521 12.0
81.9905506 0.446142464876175 13.0
81.9905506 0.0824681669473648 14.0
0 0.0510260723531246 15.0
0 0.151318579912186 16.0
0 1.46362459659576 17.0
52.893726064 2.10271191596985 18.0
-52.893726064 2.19881224632263 19.0
-52.893726064 0.566754341125488 20.0
-52.893726064 0.00347860530018806 21.0
73.735103534 1.90073084831238 22.0
138.95245885 2.15821123123169 23.0
108.44631902 0.311002880334854 24.0
22.220710699 0.271963208913803 25.0
212.68756238 0.543596029281616 26.0
212.68756238 1.4680153131485 27.0
26.354116971 1.59838271141052 28.0
239.04167935 0.822807967662811 29.0
239.04167935 1.76428377628326 30.0
47.182958244 0.979826390743256 31.0
9.4728083366 0.352153480052948 32.0
20.84137747 1.68695724010468 33.0
20.84137747 0.761462152004242 34.0
20.84137747 0.135025098919868 35.0
9.6617548141 0.0501480996608734 36.0
8.4295000417 0.408134877681732 37.0
0 -0.00732077844440937 38.0
0 0.067715585231781 39.0
-9.7979495973 0.660785496234894 40.0
-22.319875353 0.531450927257538 41.0
5.3311538087 0.464548826217651 42.0
-30.198490127 0.704658985137939 43.0
89.284505269 0.558144211769104 44.0
89.284505269 0.105649225413799 45.0
6.2583230165 0.580013155937195 46.0
1.4204858927 0.443585842847824 47.0
154.67558141 0.0978657901287079 48.0
148.99205046 0.457285076379776 49.0
0 0.381377935409546 50.0
69.213222159 0.324403494596481 51.0
99.338491111 0.171493515372276 52.0
81.9905506 0.193802326917648 53.0
81.9905506 0.27825191617012 54.0
81.9905506 0.0663733556866646 55.0
0 1.57779276371002 56.0
73.735103534 0.081327311694622 57.0
212.68756238 0.380828738212585 58.0
239.04167935 0.419705301523209 59.0
20.84137747 -0.000533961690962315 60.0
0 0.167213648557663 61.0
143.90202988 0.94081836938858 0.0
143.90202988 1.41167092323303 1.0
142.57048211 1.64395260810852 2.0
137.7813768 5.45940637588501 3.0
0 8.03249359130859 4.0
0 6.00611448287964 5.0
-78.290853603 2.60972356796265 6.0
88.535425834 0.788264811038971 7.0
66.573537661 0.586064517498016 8.0
21.961888173 0.741803348064423 9.0
97.940604824 1.00186884403229 10.0
85.489024817 0.19791966676712 11.0
84.18763914 0.482130914926529 12.0
84.18763914 0.388311952352524 13.0
84.18763914 0.0781644135713577 14.0
0 0.0547421388328075 15.0
0 0.179977521300316 16.0
0 1.41491532325745 17.0
60.524177866 0.628896534442902 18.0
-60.524177866 0.821331977844238 19.0
-60.524177866 0.491943538188934 20.0
-60.524177866 -0.0330457575619221 21.0
81.198888269 1.79188919067383 22.0
146.04463621 1.9261782169342 23.0
111.60689474 0.381041347980499 24.0
23.119018773 0.22567418217659 25.0
227.24352448 1.7866804599762 26.0
227.24352448 1.8988584280014 27.0
25.874279817 2.59658241271973 28.0
253.1178043 1.33857011795044 29.0
253.1178043 1.85078704357147 30.0
47.026005036 0.851531565189362 31.0
9.4305301758 0.483792215585709 32.0
20.674710403 1.65812504291534 33.0
20.674710403 0.74146568775177 34.0
20.674710403 0.0730672553181648 35.0
9.237309844 0.0215308126062155 36.0
7.8984410375 0.416918128728867 37.0
0 0.0391829572618008 38.0
0 0.264337480068207 39.0
-10.632565454 0.799927413463593 40.0
-23.774304057 0.376724928617477 41.0
5.2402358901 0.643478989601135 42.0
-31.756880392 0.478748440742493 43.0
77.637011937 0.702901184558868 44.0
77.637011937 0.0963053405284882 45.0
7.1470068081 0.484434902667999 46.0
3.0216372349 0.337520927190781 47.0
143.90202988 0.118516594171524 48.0
137.7813768 0.320181041955948 49.0
0 0.448797404766083 50.0
88.535425834 0.0737288147211075 51.0
97.940604824 0.19850580394268 52.0
84.18763914 0.0576208382844925 53.0
84.18763914 0.137806415557861 54.0
84.18763914 0.0839770659804344 55.0
0 3.73339605331421 56.0
81.198888269 0.0562359429895878 57.0
227.24352448 0.809509992599487 58.0
253.1178043 1.24464893341064 59.0
20.674710403 -0.00845814682543278 60.0
0 0.0753567665815353 61.0
151.42544809 0.902950882911682 0.0
151.42544809 1.70734715461731 1.0
149.91346828 1.72653043270111 2.0
145.12438474 4.94020462036133 3.0
0 8.50820350646973 4.0
0 5.41500520706177 5.0
-55.554436377 2.47387361526489 6.0
64.238328566 1.34780430793762 7.0
65.799611257 1.57584655284882 8.0
-1.5612826908 1.45969367027283 9.0
82.613093221 0.881466090679169 10.0
70.058800506 0.0867889523506165 11.0
77.867092396 0.523670375347137 12.0
77.867092396 0.461680978536606 13.0
77.867092396 0.153393253684044 14.0
0 0.217117428779602 15.0
0 0.149835079908371 16.0
0 1.65558838844299 17.0
57.682775904 1.59689664840698 18.0
-57.682775904 1.30329251289368 19.0
-57.682775904 1.07036399841309 20.0
-57.682775904 -0.0383128337562084 21.0
80.030397242 0.756463348865509 22.0
147.29387846 0.462780922651291 23.0
110.0935849 0.870048701763153 24.0
22.48660971 0.0847301557660103 25.0
227.32427571 0.972285687923431 26.0
227.32427571 1.80249309539795 27.0
30.185527704 2.67178726196289 28.0
257.50980341 1.62210154533386 29.0
257.50980341 2.48405957221985 30.0
46.305603668 1.01485073566437 31.0
9.1707978219 0.607654392719269 32.0
22.347621338 1.45431530475616 33.0
22.347621338 1.33592283725739 34.0
22.347621338 0.11633912473917 35.0
9.9699801631 0.0349084697663784 36.0
8.6175826564 0.414824396371841 37.0
0 -0.00405557081103325 38.0
0 0.122373938560486 39.0
-11.977073173 0.599190652370453 40.0
-24.698173824 0.450769990682602 41.0
5.7407971986 0.429197132587433 42.0
-33.14083308 0.412124335765839 43.0
83.079038575 0.886966705322266 44.0
83.079038575 0.110078386962414 45.0
6.2872310326 0.633854746818542 46.0
2.4575396083 0.330080509185791 47.0
151.42544809 0.121503926813602 48.0
145.12438474 0.395614385604858 49.0
0 0.51439106464386 50.0
64.238328566 0.4438316822052 51.0
82.613093221 0.0587682314217091 52.0
77.867092396 0.0427324362099171 53.0
77.867092396 0.0667771548032761 54.0
77.867092396 0.0975815653800964 55.0
0 6.06598854064941 56.0
80.030397242 0.272881895303726 57.0
227.32427571 0.501200497150421 58.0
257.50980341 1.49930667877197 59.0
22.347621338 -0.00854687765240669 60.0
0 0.0924921706318855 61.0
144.08368682 1.2376149892807 0.0
144.08368682 0.654755055904388 1.0
142.80623573 1.01680040359497 2.0
138.01713042 3.59654664993286 3.0
0 8.75580406188965 4.0
0 4.98281240463257 5.0
-69.582318221 1.37224280834198 6.0
81.688543999 0.558846712112427 7.0
59.677440718 0.901903331279755 8.0
22.01110328 0.922332406044006 9.0
77.384243836 1.26702570915222 10.0
66.335393372 0.0719897672533989 11.0
77.932303449 0.10832567512989 12.0
77.932303449 0.228486835956573 13.0
77.932303449 0.122476682066917 14.0
0 0.0737593397498131 15.0
0 0.118831321597099 16.0
0 1.03406524658203 17.0
56.030910525 1.71264123916626 18.0
-56.030910525 1.59097242355347 19.0
-56.030910525 0.34522020816803 20.0
-56.030910525 0.0797507241368294 21.0
79.182155276 1.02699542045593 22.0
141.78702134 0.493684649467468 23.0
105.23951289 0.923009514808655 24.0
24.548674861 0.0988465622067451 25.0
220.96917661 1.58492934703827 26.0
220.96917661 1.67820596694946 27.0
24.899727552 2.25171065330505 28.0
245.86890417 0.905107736587524 29.0
245.86890417 1.75379776954651 30.0
45.9488964 1.31998360157013 31.0
9.1963354144 0.460127770900726 32.0
23.151244752 1.50546598434448 33.0
23.151244752 0.805232107639313 34.0
23.151244752 0.124820835888386 35.0
10.165139735 0.0486017614603043 36.0
8.9331070488 0.371195495128632 37.0
0 0.0159912556409836 38.0
0 0.186995208263397 39.0
-11.744838107 0.809510886669159 40.0
-23.304442529 0.552285254001617 41.0
5.767758709 0.364198625087738 42.0
-31.268945547 0.285140186548233 43.0
80.718726377 0.661885321140289 44.0
80.718726377 0.096824087202549 45.0
5.9288359659 0.630016028881073 46.0
2.0782696651 0.406092762947083 47.0
144.08368682 0.0783942043781281 48.0
138.01713042 0.097525343298912 49.0
0 0.325507968664169 50.0
81.688543999 0.20319689810276 51.0
77.384243836 0.179641991853714 52.0
77.932303449 0.0132150715216994 53.0
77.932303449 0.0734295397996902 54.0
77.932303449 0.0643292218446732 55.0
0 6.55746746063232 56.0
79.182155276 0.103798493742943 57.0
220.96917661 0.710837602615356 58.0
245.86890417 0.228699997067451 59.0
23.151244752 -0.0125537980347872 60.0
0 0.101126566529274 61.0
152.47408946 0.723750412464142 0.0
152.47408946 2.05092120170593 1.0
151.18163708 1.83180272579193 2.0
146.39253197 3.07546591758728 3.0
0 6.66957092285156 4.0
0 3.63617300987244 5.0
-69.256988289 3.79060745239258 6.0
79.950459547 1.58186292648315 7.0
64.897843225 0.506387114524841 8.0
15.052616322 0.481247842311859 9.0
95.13902794 0.772128760814667 10.0
81.84762393 0.0988279208540916 11.0
80.473090831 0.650382578372955 12.0
80.473090831 0.398423463106155 13.0
80.473090831 0.123462431132793 14.0
0 0.0788761824369431 15.0
0 0.19987815618515 16.0
0 1.07810068130493 17.0
62.108067132 1.43815457820892 18.0
-62.108067132 1.54515635967255 19.0
-62.108067132 0.325870335102081 20.0
-62.108067132 -0.0724761188030243 21.0
84.429143744 2.46036791801453 22.0
156.47719868 2.64777827262878 23.0
118.56224982 0.145137995481491 24.0
22.781748111 0.299613267183304 25.0
240.90634242 1.52330136299133 26.0
240.90634242 1.61537456512451 27.0
27.033522042 1.81309854984283 28.0
267.93986446 0.949687659740448 29.0
267.93986446 2.11554884910583 30.0
46.34958025 1.19447541236877 31.0
9.4819859507 0.546267986297607 32.0
22.321076612 2.14598059654236 33.0
22.321076612 0.889387726783752 34.0
22.321076612 0.135809332132339 35.0
9.6124504136 0.0311420187354088 36.0
8.506943485 0.337264090776443 37.0
0 0.0114388205111027 38.0
0 0.229304909706116 39.0
-12.107565002 0.596295237541199 40.0
-23.737569134 0.683251261711121 41.0
5.5084153983 0.382608234882355 42.0
-31.783609866 0.413362890481949 43.0
100.72903997 0.69002890586853 44.0
100.72903997 0.0494915619492531 45.0
6.2577082282 0.550880670547485 46.0
2.4303815632 0.350642472505569 47.0
152.47408946 0.0919475108385086 48.0
146.39253197 0.123656824231148 49.0
0 0.593790769577026 50.0
79.950459547 0.167975157499313 51.0
95.13902794 0.0749439895153046 52.0
80.473090831 0.0736009329557419 53.0
80.473090831 0.125680178403854 54.0
80.473090831 0.0548245869576931 55.0
0 7.04294872283936 56.0
84.429143744 0.180818185210228 57.0
240.90634242 0.279996246099472 58.0
267.93986446 0.490973263978958 59.0
22.321076612 -0.00358960963785648 60.0
0 0.103350222110748 61.0
170.02654924 0.884567856788635 0.0
170.02654924 1.4028377532959 1.0
168.94550434 1.58404064178467 2.0
164.1563991 5.17106485366821 3.0
0 6.39028835296631 4.0
0 6.05573701858521 5.0
-69.527089249 3.87921500205994 6.0
82.187436467 1.25658404827118 7.0
57.39639899 0.527302265167236 8.0
24.791037477 0.517043232917786 9.0
91.719560249 1.06130802631378 10.0
78.058662738 0.349513381719589 11.0
79.18825531 0.914061844348907 12.0
79.18825531 0.435098230838776 13.0
79.18825531 0.10412336140871 14.0
0 0.170734912157059 15.0
0 0.181720703840256 16.0
0 1.25327169895172 17.0
56.566278005 1.52421832084656 18.0
-56.566278005 0.867080926895142 19.0
-56.566278005 0.507377684116364 20.0
-56.566278005 -0.0348963029682636 21.0
77.221848967 2.12368679046631 22.0
147.59113097 1.55626606941223 23.0
109.27673315 0.536803066730499 24.0
22.049744341 0.0794757902622223 25.0
224.81297993 0.756305813789368 26.0
224.81297993 1.80846667289734 27.0
23.381313758 2.0478503704071 28.0
248.19429369 0.879025340080261 29.0
248.19429369 1.65410459041595 30.0
46.617458634 0.999547600746155 31.0
9.4333706562 0.488242387771606 32.0
20.655570962 1.54234302043915 33.0
20.655570962 1.2325873374939 34.0
20.655570962 0.11550160497427 35.0
9.4302812286 0.0266775321215391 36.0
8.2154562389 0.403149574995041 37.0
0 0.0141088925302029 38.0
0 0.170013278722763 39.0
-12.630698741 0.72867339849472 40.0
-24.95716136 0.475203454494476 41.0
5.008126567 0.585563540458679 42.0
-32.827739272 0.519719779491425 43.0
93.251762048 0.699830651283264 44.0
93.251762048 0.104911431670189 45.0
6.2672619832 0.565559923648834 46.0
2.4113160029 0.370413482189178 47.0
170.02654924 0.0690378770232201 48.0
164.1563991 0.331445008516312 49.0
0 0.385991156101227 50.0
82.187436467 0.19433382153511 51.0
91.719560249 0.295583248138428 52.0
79.18825531 0.173297017812729 53.0
79.18825531 0.0723878964781761 54.0
79.18825531 0.0243979413062334 55.0
0 3.67691850662231 56.0
77.221848967 0.122839286923409 57.0
224.81297993 0.253144353628159 58.0
248.19429369 0.240748703479767 59.0
20.655570962 -0.0182712264358997 60.0
0 0.109909415245056 61.0
174.68490085 0.957260429859161 0.0
174.68490085 1.69826829433441 1.0
173.65656114 1.89409971237183 2.0
168.19192573 6.730553150177 3.0
20.101261513 4.31157064437866 4.0
20.101261513 5.36170291900635 5.0
-48.605560283 0.604402244091034 6.0
61.312556849 0.645886659622192 7.0
39.00094957 0.584595918655396 8.0
22.31160728 0.562398195266724 9.0
102.40800784 0.996234178543091 10.0
88.142091283 0.0799190998077393 11.0
86.250841891 0.30481880903244 12.0
86.250841891 0.727890729904175 13.0
86.250841891 0.0532770082354546 14.0
0 0.151589766144753 15.0
0 0.219038397073746 16.0
0 1.19654738903046 17.0
60.157541174 1.36958515644073 18.0
-60.157541174 1.28109931945801 19.0
-60.157541174 0.752067506313324 20.0
-60.157541174 -0.0180744528770447 21.0
81.942709263 1.04059636592865 22.0
133.54930195 1.04452478885651 23.0
98.840301274 0.734198570251465 24.0
23.258167981 0.08265121281147 25.0
215.49201121 1.10596215724945 26.0
215.49201121 1.66964054107666 27.0
28.599535994 1.22128796577454 28.0
244.09154721 0.971864938735962 29.0
244.09154721 1.73632156848907 30.0
46.865511127 1.19160056114197 31.0
9.2566604605 0.41362601518631 32.0
21.785168089 1.64325487613678 33.0
21.785168089 0.935323715209961 34.0
21.785168089 0.0938892364501953 35.0
10.337302816 0.029893783852458 36.0
8.9792001126 0.417397558689117 37.0
0 0.0121687166392803 38.0
0 0.132934167981148 39.0
-12.708013536 0.364496558904648 40.0
-24.184146652 0.671836793422699 41.0
4.665225063 0.471351087093353 42.0
-31.317547354 0.328876107931137 43.0
93.762327885 0.831067502498627 44.0
93.762327885 0.0702368393540382 45.0
6.2047901917 0.540111243724823 46.0
2.228065783 0.36798033118248 47.0
174.68490085 0.112552314996719 48.0
168.19192573 0.195285767316818 49.0
20.101261513 0.240987718105316 50.0
61.312556849 0.321294546127319 51.0
102.40800784 0.0582071542739868 52.0
86.250841891 0.0252162348479033 53.0
86.250841891 0.110988646745682 54.0
86.250841891 0.0973776057362556 55.0
0 1.88664579391479 56.0
81.942709263 0.118368476629257 57.0
215.49201121 0.554709255695343 58.0
244.09154721 0.339522838592529 59.0
21.785168089 -0.0133559908717871 60.0
0 0.112401001155376 61.0
184.9826402 1.00768852233887 0.0
184.9826402 0.625448167324066 1.0
183.86747364 1.03592610359192 2.0
181.29095747 4.0418815612793 3.0
0 5.17376232147217 4.0
0 5.56293439865112 5.0
-70.795272588 3.35166954994202 6.0
81.633600104 0.996441125869751 7.0
41.768943966 0.360099732875824 8.0
39.864656138 0.246508240699768 9.0
64.47228175 1.29503035545349 10.0
51.539222216 0.345619589090347 11.0
65.374993639 0.722236156463623 12.0
65.374993639 0.448124825954437 13.0
65.374993639 0.130186453461647 14.0
25.222853634 0.310406804084778 15.0
25.222853634 0.0928156077861786 16.0
0 1.87643122673035 17.0
61.585100389 1.94327890872955 18.0
-61.585100389 2.32401061058044 19.0
-61.585100389 0.194961667060852 20.0
-61.585100389 0.0318075381219387 21.0
82.153163591 1.99685180187225 22.0
146.81033639 2.19876313209534 23.0
117.04333675 0.0931820273399353 24.0
22.259178114 0.444745481014252 25.0
228.96349999 0.84019136428833 26.0
228.96349999 1.77434229850769 27.0
25.060977266 1.38983535766602 28.0
254.02447725 1.30970454216003 29.0
254.02447725 1.73237657546997 30.0
47.267143648 1.1237086057663 31.0
9.414588504 0.447164267301559 32.0
20.568063202 1.44105625152588 33.0
20.568063202 0.804747581481934 34.0
20.568063202 0.0570450723171234 35.0
8.5171159621 0.035479586571455 36.0
7.1608913708 0.353524446487427 37.0
0 0.00768431276082993 38.0
0 0.122246742248535 39.0
-12.54371005 1.1489942073822 40.0
-23.822384057 0.702863991260529 41.0
4.8338433433 0.387420237064362 42.0
-31.231094906 0.395132482051849 43.0
102.37090932 1.00781059265137 44.0
102.37090932 0.0635479167103767 45.0
7.7080985841 0.825832784175873 46.0
3.0171637124 0.395360499620438 47.0
184.9826402 0.00707016699016094 48.0
181.29095747 -0.0119818057864904 49.0
0 0.428412437438965 50.0
81.633600104 0.076763741672039 51.0
64.47228175 0.658706545829773 52.0
65.374993639 0.0921762511134148 53.0
65.374993639 0.0642477348446846 54.0
65.374993639 0.106494061648846 55.0
0 0.542553961277008 56.0
82.153163591 0.141485989093781 57.0
228.96349999 0.439659953117371 58.0
254.02447725 0.750176906585693 59.0
20.568063202 -0.00253143720328808 60.0
0 0.112742975354195 61.0
138.5895143 0.679464221000671 0.0
138.5895143 1.44798457622528 1.0
137.17962095 1.75167918205261 2.0
132.39051588 5.15032196044922 3.0
0 6.56229829788208 4.0
0 3.33893775939941 5.0
-53.668812869 0.25782835483551 6.0
67.136780403 0.675161182880402 7.0
68.034178256 1.16127526760101 8.0
-0.8973978531 1.28058135509491 9.0
104.62611754 0.988464891910553 10.0
90.600425603 0.570974051952362 11.0
90.149634165 0.638969898223877 12.0
90.149634165 0.648298859596252 13.0
90.149634165 0.085456520318985 14.0
0 0.224063664674759 15.0
0 0.163927495479584 16.0
0 0.634469211101532 17.0
59.127234379 1.57469606399536 18.0
-59.127234379 1.52510201931 19.0
-59.127234379 0.748864650726318 20.0
-59.127234379 -0.0532412081956863 21.0
78.398490907 2.47941255569458 22.0
149.97642708 2.41069078445435 23.0
109.77544667 0.229772314429283 24.0
22.176776694 0.284188330173492 25.0
228.37491798 1.00868964195251 26.0
228.37491798 1.20924317836761 27.0
24.41487276 1.43607187271118 28.0
252.78979074 0.746339440345764 29.0
252.78979074 1.98642337322235 30.0
47.17154021 1.21368181705475 31.0
9.3851180132 0.676889538764954 32.0
19.271256528 1.57218778133392 33.0
19.271256528 1.0743396282196 34.0
19.271256528 0.0723002627491951 35.0
10.080532451 0.0204676184803247 36.0
8.7378487637 0.432840287685394 37.0
0 0.0403299368917942 38.0
0 0.166161343455315 39.0
-11.678033291 0.587225019931793 40.0
-25.160393475 0.637227654457092 41.0
5.1655558135 0.443066656589508 42.0
-32.832312306 0.569164037704468 43.0
107.15372833 0.721958935260773 44.0
107.15372833 0.0651781782507896 45.0
7.6267568903 0.467524230480194 46.0
2.5728099707 0.280099481344223 47.0
138.5895143 0.104741185903549 48.0
132.39051588 0.546387672424316 49.0
0 0.211998715996742 50.0
67.136780403 0.0274143610149622 51.0
104.62611754 0.698490560054779 52.0
90.149634165 0.171193778514862 53.0
90.149634165 0.107100382447243 54.0
90.149634165 0.112939171493053 55.0
0 3.99624419212341 56.0
78.398490907 0.175767466425896 57.0
228.37491798 0.223107501864433 58.0
252.78979074 0.625250935554504 59.0
19.271256528 0.00756491348147392 60.0
0 0.107932642102242 61.0
161.9508797 0.850454390048981 0.0
161.9508797 0.696060001850128 1.0
160.5591384 0.733887255191803 2.0
155.7700333 4.96986198425293 3.0
0 6.63870286941528 4.0
0 3.70794939994812 5.0
-75.464338212 0.797273099422455 6.0
84.004320464 0.874190151691437 7.0
68.471590358 0.938571929931641 8.0
15.532730106 0.752995431423187 9.0
88.340877358 0.828873753547668 10.0
76.004955771 0.191473931074142 11.0
80.588432644 0.725312113761902 12.0
80.588432644 0.232171028852463 13.0
80.588432644 0.114513695240021 14.0
0 0.164289861917496 15.0
0 0.205387011170387 16.0
0 1.46532106399536 17.0
55.028173943 1.57366561889648 18.0
-55.028173943 1.80231559276581 19.0
-55.028173943 0.297099411487579 20.0
-55.028173943 0.174162939190865 21.0
75.22076436 0.835100114345551 22.0
146.97706408 0.395213395357132 23.0
110.90643174 0.694706737995148 24.0
22.777240292 0.153723984956741 25.0
222.19782844 0.661884009838104 26.0
222.19782844 1.28389716148376 27.0
22.07882737 1.64026951789856 28.0
244.27665581 0.346006512641907 29.0
244.27665581 1.64976227283478 30.0
46.856864755 1.09408712387085 31.0
9.5174788826 0.347986727952957 32.0
20.192590417 1.5271463394165 33.0
20.192590417 0.779050886631012 34.0
20.192590417 0.109323665499687 35.0
8.5897866594 0.0323045551776886 36.0
7.2427685964 0.400421261787415 37.0
0 0.077909380197525 38.0
0 0.156619161367416 39.0
-12.52245393 0.67120373249054 40.0
-23.912563428 0.440315753221512 41.0
4.6678064023 0.385741680860519 42.0
-31.259910104 0.560442447662354 43.0
100.56506183 0.761763155460358 44.0
100.56506183 0.0910888537764549 45.0
6.3078375195 0.542008757591248 46.0
2.3596529716 0.34140682220459 47.0
161.9508797 0.272679775953293 48.0
155.7700333 0.0346787944436073 49.0
0 0.207177817821503 50.0
84.004320464 0.208062276244164 51.0
88.340877358 0.199978142976761 52.0
80.588432644 0.063053585588932 53.0
80.588432644 0.155919283628464 54.0
80.588432644 0.0314350612461567 55.0
0 4.32532644271851 56.0
75.22076436 0.130941852927208 57.0
222.19782844 0.254666745662689 58.0
244.27665581 0.293959379196167 59.0
20.192590417 0.00324137508869171 60.0
0 0.101206794381142 61.0
146.70400979 0.860409021377563 0.0
146.70400979 1.00063169002533 1.0
145.52861206 1.47039783000946 2.0
140.73950675 5.53142833709717 3.0
0 5.09494018554688 4.0
0 3.86649990081787 5.0
-70.77371669 0.665746212005615 6.0
80.552502689 0.986912906169891 7.0
50.91840228 1.06049692630768 8.0
29.634100408 0.7378830909729 9.0
92.807269999 1.02195847034454 10.0
79.561874457 0.335354387760162 11.0
82.982430897 0.793308079242706 12.0
82.982430897 0.497578948736191 13.0
82.982430897 0.068117268383503 14.0
0 0.163146942853928 15.0
0 0.217345133423805 16.0
0 0.884051024913788 17.0
55.337089338 1.69826328754425 18.0
-55.337089338 1.48068559169769 19.0
-55.337089338 0.575837373733521 20.0
-55.337089338 -0.0519198291003704 21.0
76.995688972 2.13277959823608 22.0
141.37093829 1.94618439674377 23.0
109.67384779 0.364750683307648 24.0
22.689682713 0.088126614689827 25.0
218.36662726 1.46198868751526 26.0
218.36662726 1.76500058174133 27.0
25.734242359 1.02930736541748 28.0
244.10086962 0.518794417381287 29.0
244.10086962 1.4859105348587 30.0
47.054625355 1.0629802942276 31.0
9.3930972078 0.303731918334961 32.0
21.658599634 1.75114440917969 33.0
21.658599634 0.996160387992859 34.0
21.658599634 0.0886287465691566 35.0
10.270933902 0.020990964025259 36.0
9.1327819119 0.429279834032059 37.0
0 0.00764168053865433 38.0
0 0.163216412067413 39.0
-11.686695404 0.698573887348175 40.0
-25.181792746 0.406202375888824 41.0
5.5268649005 0.423061430454254 42.0
-33.036271263 0.681333422660828 43.0
81.491265812 0.612808227539062 44.0
81.491265812 0.0926032513380051 45.0
7.3012905142 0.519366800785065 46.0
2.4238493042 0.315218567848206 47.0
146.70400979 0.099547415971756 48.0
140.73950675 0.490407675504684 49.0
0 0.17974491417408 50.0
80.552502689 0.229441404342651 51.0
92.807269999 0.350104451179504 52.0
82.982430897 0.0663865655660629 53.0
82.982430897 0.230299353599548 54.0
82.982430897 0.044180616736412 55.0
0 2.4104700088501 56.0
76.995688972 0.114187061786652 57.0
218.36662726 0.925137341022491 58.0
244.10086962 0.490954548120499 59.0
21.658599634 0.00981530174612999 60.0
0 0.0911146998405457 61.0
193.98701252 0.877480745315552 0.0
193.98701252 0.722706139087677 1.0
192.59627956 1.03958010673523 2.0
187.80717446 5.83548641204834 3.0
0 7.72178745269775 4.0
0 3.937903881073 5.0
-52.720785767 2.66711974143982 6.0
61.182796579 1.58725476264954 7.0
60.047576244 0.187010183930397 8.0
1.1352203348 0.307276546955109 9.0
90.201950953 0.989534199237823 10.0
76.533810277 0.257735729217529 11.0
79.548600848 0.407447099685669 12.0
79.548600848 0.308815479278564 13.0
79.548600848 0.0414821058511734 14.0
0 0.167839869856834 15.0
0 0.207540363073349 16.0
0 0.908982753753662 17.0
58.389946631 1.57061922550201 18.0
-58.389946631 1.40629625320435 19.0
-58.389946631 0.400858789682388 20.0
-58.389946631 0.270566672086716 21.0
81.936952693 1.53995609283447 22.0
141.83846928 0.926047563552856 23.0
108.83512944 0.9711993932724 24.0
25.227011073 0.0761899575591087 25.0
223.77542197 1.895911693573 26.0
223.77542197 1.759446144104 27.0
25.078331419 0.953766763210297 28.0
248.85375339 0.542452692985535 29.0
248.85375339 1.26893043518066 30.0
47.003288952 1.04484689235687 31.0
9.423157829 0.379197597503662 32.0
23.547006061 1.05738627910614 33.0
23.547006061 0.521285474300385 34.0
23.547006061 0.121915057301521 35.0
10.184730587 0.0280981976538897 36.0
8.9943949365 0.471923321485519 37.0
0 0.0153367090970278 38.0
0 0.0495124869048595 39.0
-10.564656632 0.922003448009491 40.0
-21.8875035 0.314302086830139 41.0
5.126270948 0.678757548332214 42.0
-29.597173502 0.572764933109283 43.0
109.02940547 0.857209265232086 44.0
109.02940547 0.0878639295697212 45.0
6.4651113315 0.711640954017639 46.0
2.2565127066 0.421644508838654 47.0
193.98701252 0.138306364417076 48.0
187.80717446 0.250355422496796 49.0
0 0.232436865568161 50.0
61.182796579 0.157466769218445 51.0
90.201950953 0.18003636598587 52.0
79.548600848 0.0445891581475735 53.0
79.548600848 0.166489735245705 54.0
79.548600848 0.052157323807478 55.0
0 3.5565619468689 56.0
81.936952693 0.0379672944545746 57.0
223.77542197 1.17317950725555 58.0
248.85375339 0.271371573209763 59.0
23.547006061 0.0276270918548107 60.0
0 0.139965981245041 61.0
175.0314728 0.761222124099731 0.0
175.0314728 2.23944878578186 1.0
173.51675848 2.00016498565674 2.0
168.72765323 5.35791397094727 3.0
0 3.7646951675415 4.0
0 4.43569469451904 5.0
-48.651899333 1.00014364719391 6.0
61.783479488 0.384669214487076 7.0
36.368788278 0.631984353065491 8.0
25.41469121 0.607887983322144 9.0
86.835523091 0.872257173061371 10.0
75.74458427 0.158028900623322 11.0
83.276169073 0.982548236846924 12.0
83.276169073 0.517708718776703 13.0
83.276169073 0.165535449981689 14.0
0 0.284715563058853 15.0
0 0.129675820469856 16.0
0 1.04632103443146 17.0
52.015312843 1.18032562732697 18.0
-52.015312843 0.791593611240387 19.0
-52.015312843 0.359521418809891 20.0
-52.015312843 -0.0391977690160275 21.0
72.836966324 1.84622371196747 22.0
138.31954892 1.75237321853638 23.0
103.87095273 0.475236237049103 24.0
23.45530304 0.123140513896942 25.0
211.15651525 0.402856051921844 26.0
211.15651525 1.30117547512054 27.0
25.6463441 1.58772683143616 28.0
236.80285935 1.64938592910767 29.0
236.80285935 2.14171814918518 30.0
46.206704998 0.899169445037842 31.0
9.4585015109 0.396272122859955 32.0
20.821653481 1.60867083072662 33.0
20.821653481 1.32890713214874 34.0
20.821653481 0.11364334821701 35.0
8.7229095342 0.0359301418066025 36.0
7.5647868 0.342084050178528 37.0
0 0.012894194573164 38.0
0 0.202845543622971 39.0
-11.665901409 0.510495066642761 40.0
-25.071036962 0.647534430027008 41.0
5.6490857714 0.396331936120987 42.0
-32.968744381 0.504510760307312 43.0
90.519025855 0.637604415416718 44.0
90.519025855 0.0397967845201492 45.0
5.7041735865 0.558068871498108 46.0
1.7657108799 0.366581499576569 47.0
175.0314728 0.0491795316338539 48.0
168.72765323 0.38603800535202 49.0
0 0.42214834690094 50.0
61.783479488 0.121213592588902 51.0
86.835523091 0.312424182891846 52.0
83.276169073 0.11714793741703 53.0
83.276169073 0.20463202893734 54.0
83.276169073 0.119994200766087 55.0
0 5.7957820892334 56.0
72.836966324 0.311830669641495 57.0
211.15651525 0.0687659755349159 58.0
236.80285935 2.04323554039001 59.0
20.821653481 -0.00126254931092262 60.0
0 0.0961512252688408 61.0
172.89743615 1.09158611297607 0.0
172.89743615 1.05210912227631 1.0
171.97800763 2.02586936950684 2.0
167.18893768 3.87157726287842 3.0
0 8.87905502319336 4.0
0 2.86418175697327 5.0
-46.741140796 0.0757606029510498 6.0
59.541407076 0.854313373565674 7.0
63.0400574 0.409950196743011 8.0
-3.4986503239 0.672182083129883 9.0
92.774571429 1.02302753925323 10.0
80.189377665 0.406181365251541 11.0
82.530322975 0.468882262706757 12.0
82.530322975 0.246004462242126 13.0
82.530322975 0.228520855307579 14.0
0 0.315300732851028 15.0
0 0.201612219214439 16.0
0 2.06459331512451 17.0
54.008886903 1.23839557170868 18.0
-54.008886903 0.841093957424164 19.0
-54.008886903 0.983454465866089 20.0
-54.008886903 -0.0376538224518299 21.0
75.707487803 1.21356964111328 22.0
136.06725135 0.579878211021423 23.0
103.12537154 1.2030086517334 24.0
21.9246839 0.0625148043036461 25.0
211.77473915 0.987887680530548 26.0
211.77473915 0.962252795696259 27.0
24.448152128 1.5978592634201 28.0
236.22289128 0.680358290672302 29.0
236.22289128 1.64967143535614 30.0
47.364408976 0.956661820411682 31.0
9.2194362825 0.388328462839127 32.0
21.698600901 1.71211230754852 33.0
21.698600901 1.65845704078674 34.0
21.698600901 0.112466096878052 35.0
9.6948089265 0.0380981490015984 36.0
8.4159242439 0.243665233254433 37.0
0 -0.0102383680641651 38.0
0 0.117581635713577 39.0
-13.485866154 0.463861256837845 40.0
-25.773710148 0.369180053472519 41.0
5.3654044766 0.302830338478088 42.0
-33.437148531 0.428720772266388 43.0
95.673131474 0.716383099555969 44.0
95.673131474 0.152175098657608 45.0
5.6620420246 0.550289213657379 46.0
1.9859707974 0.435416847467422 47.0
172.89743615 0.322602242231369 48.0
167.18893768 0.082369476556778 49.0
0 0.328982651233673 50.0
59.541407076 0.0521684139966965 51.0
92.774571429 0.144461616873741 52.0
82.530322975 0.0779277086257935 53.0
82.530322975 0.109622746706009 54.0
82.530322975 0.0442497283220291 55.0
0 4.87186622619629 56.0
75.707487803 0.178651988506317 57.0
211.77473915 0.0563711486756802 58.0
236.22289128 0.527146577835083 59.0
21.698600901 -0.0152305327355862 60.0
0 0.0774592757225037 61.0
167.65790184 0.841184318065643 0.0
167.65790184 0.893616437911987 1.0
166.24773332 1.45857238769531 2.0
161.45862804 5.65230703353882 3.0
0 7.91779708862305 4.0
0 5.01867008209229 5.0
-63.839591226 2.78330445289612 6.0
74.527308544 0.490395247936249 7.0
47.743654409 0.653502821922302 8.0
26.783654135 0.718727707862854 9.0
99.822645713 1.02019441127777 10.0
86.564086207 0.375026136636734 11.0
85.582135002 0.275509387254715 12.0
85.582135002 0.564802527427673 13.0
85.582135002 0.088087871670723 14.0
0 0.265658736228943 15.0
0 0.152453497052193 16.0
0 1.56428813934326 17.0
54.170882936 1.3699107170105 18.0
-54.170882936 1.08274161815643 19.0
-54.170882936 0.518208086490631 20.0
-54.170882936 -0.0555751249194145 21.0
74.510978213 1.83490967750549 22.0
135.75794991 1.31275677680969 23.0
101.42709706 0.415693521499634 24.0
21.351052442 0.192363917827606 25.0
210.26892812 1.29488897323608 26.0
210.26892812 1.71120452880859 27.0
29.288135422 1.78174912929535 28.0
239.55706354 0.705022990703583 29.0
239.55706354 1.86968672275543 30.0
46.456190808 1.08740019798279 31.0
9.4527989809 0.525276184082031 32.0
20.340095276 1.60965824127197 33.0
20.340095276 0.808547079563141 34.0
20.340095276 0.0898550227284431 35.0
9.4393646839 0.0230480674654245 36.0
8.2870248754 0.435761332511902 37.0
0 0.0278349034488201 38.0
0 0.273120105266571 39.0
-12.059359114 0.312221705913544 40.0
-26.060071287 0.760502994060516 41.0
4.7018171466 0.437603831291199 42.0
-33.297633254 0.525996565818787 43.0
88.830952902 0.756882727146149 44.0
88.830952902 0.127917766571045 45.0
7.3798261116 0.632199287414551 46.0
2.5089864892 0.348671913146973 47.0
167.65790184 0.0920446664094925 48.0
161.45862804 0.363820433616638 49.0
0 0.292465090751648 50.0
74.527308544 0.055236142128706 51.0
99.822645713 0.368589520454407 52.0
85.582135002 0.0462435781955719 53.0
85.582135002 0.107591010630131 54.0
85.582135002 0.146424889564514 55.0
0 1.80439710617065 56.0
74.510978213 0.150510683655739 57.0
210.26892812 0.526148676872253 58.0
239.55706354 0.401861846446991 59.0
20.340095276 -0.00565797835588455 60.0
0 0.0937685519456863 61.0
142.92216094 0.886323511600494 0.0
142.92216094 1.31924676895142 1.0
141.46450184 1.72707808017731 2.0
136.67539666 3.37166833877563 3.0
0 5.03277540206909 4.0
0 4.25639533996582 5.0
-63.865755383 0.456044554710388 6.0
76.613825591 0.51237565279007 7.0
59.761151022 1.29391038417816 8.0
16.85267457 1.45425689220428 9.0
84.148517198 1.0186619758606 10.0
71.085561938 0.195517480373383 11.0
83.082609556 0.330951631069183 12.0
83.082609556 0.784165859222412 13.0
83.082609556 0.0717553496360779 14.0
0 0.2061737626791 15.0
0 0.206153273582458 16.0
0 0.701880872249603 17.0
51.640395424 1.3415983915329 18.0
-51.640395424 1.55008447170258 19.0
-51.640395424 0.590178906917572 20.0
-51.640395424 -0.0352457091212273 21.0
72.673155981 2.34977388381958 22.0
143.37802634 1.72300386428833 23.0
108.65728599 0.642019867897034 24.0
24.140829862 0.180546224117279 25.0
216.05118232 1.01367139816284 26.0
216.05118232 1.32279312610626 27.0
24.304079933 1.93564605712891 28.0
240.35526226 0.942983508110046 29.0
240.35526226 1.81491804122925 30.0
46.345294954 0.960529565811157 31.0
9.4192770138 0.505438029766083 32.0
21.032760557 1.38778352737427 33.0
21.032760557 1.01443147659302 34.0
21.032760557 0.0974330082535744 35.0
8.9907519653 0.0221480615437031 36.0
7.7165694394 0.407818377017975 37.0
0 0.00193924829363823 38.0
0 0.108964920043945 39.0
-11.330835923 0.717869222164154 40.0
-23.629891524 0.666421830654144 41.0
5.3857209725 0.367012560367584 42.0
-31.743757385 0.241454035043716 43.0
105.07114166 0.846461355686188 44.0
105.07114166 0.118347018957138 45.0
6.7787796975 0.577378034591675 46.0
2.8420431899 0.350321620702744 47.0
142.92216094 0.0826456323266029 48.0
136.67539666 0.213515743613243 49.0
0 0.216312095522881 50.0
76.613825591 0.217844694852829 51.0
84.148517198 0.336380779743195 52.0
83.082609556 0.0540851727128029 53.0
83.082609556 0.139090940356255 54.0
83.082609556 0.173002392053604 55.0
0 3.17074251174927 56.0
72.673155981 0.0990020036697388 57.0
216.05118232 0.190357238054276 58.0
240.35526226 0.743927836418152 59.0
21.032760557 -0.00605230312794447 60.0
0 0.0903408229351044 61.0
191.00323243 0.866100490093231 0.0
191.00323243 0.760236322879791 1.0
189.71817062 0.627661347389221 2.0
186.69707679 2.48343110084534 3.0
0 4.25805616378784 4.0
0 4.11450099945068 5.0
-80.517189422 0.907509505748749 6.0
92.925939432 1.12228417396545 7.0
56.586277228 0.432620853185654 8.0
36.339662204 0.35527715086937 9.0
70.137191247 0.867769658565521 10.0
55.684184457 0.0453303754329681 11.0
59.676703965 0.877163529396057 12.0
59.676703965 0.668336629867554 13.0
59.676703965 0.0432359538972378 14.0
26.743126621 0.137936875224113 15.0
26.743126621 0.234706163406372 16.0
0 1.31322360038757 17.0
63.889745625 1.2785097360611 18.0
-63.889745625 1.09750878810883 19.0
-63.889745625 1.38666498661041 20.0
-63.889745625 -0.0264777187258005 21.0
84.221729146 2.29658818244934 22.0
150.14922455 2.63957166671753 23.0
115.11811625 0.277983069419861 24.0
19.736326811 0.553874254226685 25.0
234.3709537 1.45589196681976 26.0
234.3709537 1.96098804473877 27.0
30.374125496 1.55408215522766 28.0
264.7450792 1.08766222000122 29.0
264.7450792 1.59952282905579 30.0
46.36650678 0.975654244422913 31.0
9.431322508 0.39434489607811 32.0
20.331983521 1.20093595981598 33.0
20.331983521 0.908712029457092 34.0
20.331983521 0.130828663706779 35.0
8.4715859957 0.0285016540437937 36.0
7.2468156799 0.354872018098831 37.0
0 0.0156876109540462 38.0
0 0.238808333873749 39.0
-11.393417808 0.52103453874588 40.0
-24.961525965 0.552852272987366 41.0
5.1484221191 0.462121814489365 42.0
-32.69739392 0.39452537894249 43.0
86.308571153 0.705558478832245 44.0
86.308571153 0.10223200917244 45.0
6.9792862122 0.632495105266571 46.0
1.909535392 0.388445526361465 47.0
191.00323243 0.184939369559288 48.0
186.69707679 0.0543515644967556 49.0
0 0.103718779981136 50.0
92.925939432 0.0751827210187912 51.0
70.137191247 0.0894131660461426 52.0
59.676703965 0.0724223852157593 53.0
59.676703965 0.326096624135971 54.0
59.676703965 0.055077999830246 55.0
0 1.97048103809357 56.0
84.221729146 -0.0252574495971203 57.0
234.3709537 0.91577273607254 58.0
264.7450792 0.437578499317169 59.0
20.331983521 0.0104050431400537 60.0
0 0.0868582129478455 61.0
154.85163725 1.08305311203003 0.0
154.85163725 0.971098661422729 1.0
153.36517276 1.4507247209549 2.0
148.57606751 3.73352026939392 3.0
0 8.30858039855957 4.0
0 4.16202640533447 5.0
-68.100368114 1.79594707489014 6.0
81.068827495 0.800500512123108 7.0
52.756091257 1.17076587677002 8.0
28.312736239 1.18064022064209 9.0
90.269851304 1.08778548240662 10.0
77.870409471 0.206893667578697 11.0
80.440470295 0.428399384021759 12.0
80.440470295 0.66983163356781 13.0
80.440470295 0.0645508617162704 14.0
0 0.0402315035462379 15.0
0 0.227361530065536 16.0
0 1.05869174003601 17.0
54.11461538 1.01180827617645 18.0
-54.11461538 0.866385221481323 19.0
-54.11461538 0.643723011016846 20.0
-54.11461538 -0.0787794515490532 21.0
75.542713828 2.79402470588684 22.0
144.28070155 2.64139151573181 23.0
112.90866657 0.308422774076462 24.0
24.546527618 0.0978928655385971 25.0
219.82341538 1.37873876094818 26.0
219.82341538 1.64470934867859 27.0
24.513871139 2.08869457244873 28.0
244.33728652 1.14034938812256 29.0
244.33728652 1.90109801292419 30.0
46.417993202 1.04525887966156 31.0
9.3451410889 0.389091968536377 32.0
21.428098448 1.80127370357513 33.0
21.428098448 0.844874560832977 34.0
21.428098448 0.122433573007584 35.0
9.2511136223 0.032694648951292 36.0
7.9821693915 0.378368586301804 37.0
0 0.0276069976389408 38.0
0 0.300361484289169 39.0
-12.053293887 0.507663488388062 40.0
-24.950305261 0.784373641014099 41.0
5.0433950079 0.354237526655197 42.0
-32.644849029 0.556477069854736 43.0
86.825092048 0.517015516757965 44.0
86.825092048 0.0989219546318054 45.0
7.3334576153 0.551400423049927 46.0
2.8641846372 0.400038301944733 47.0
154.85163725 0.141598969697952 48.0
148.57606751 0.141692906618118 49.0
0 0.5372314453125 50.0
81.068827495 0.318125993013382 51.0
90.269851304 0.272725850343704 52.0
80.440470295 0.0538859702646732 53.0
80.440470295 0.127056479454041 54.0
80.440470295 0.0733583346009254 55.0
0 3.89702582359314 56.0
75.542713828 0.194182932376862 57.0
219.82341538 0.332419604063034 58.0
244.33728652 0.548340618610382 59.0
21.428098448 -0.0102614993229508 60.0
0 0.104660257697105 61.0
161.35690124 0.658553719520569 0.0
161.35690124 1.59416234493256 1.0
160.21010556 2.02891707420349 2.0
155.4210356 5.09754753112793 3.0
0 6.73509502410889 4.0
0 3.2614426612854 5.0
-52.869915023 0.352345913648605 6.0
62.702814969 0.559452533721924 7.0
59.99802327 0.995765924453735 8.0
2.7047916991 1.19687676429749 9.0
92.164160542 0.836512744426727 10.0
79.340699624 0.430926948785782 11.0
75.438499931 0.481866389513016 12.0
75.438499931 0.141585424542427 13.0
75.438499931 0.0503137521445751 14.0
0 0.0729213207960129 15.0
0 0.193029493093491 16.0
0 1.17651307582855 17.0
53.702783282 1.48620176315308 18.0
-53.702783282 1.43395686149597 19.0
-53.702783282 0.773915767669678 20.0
-53.702783282 -0.0413477197289467 21.0
76.537164685 2.00424337387085 22.0
145.63220549 1.25874698162079 23.0
114.9029359 0.508969187736511 24.0
24.655333851 0.106508366763592 25.0
222.16937017 1.04585361480713 26.0
222.16937017 1.26397657394409 27.0
29.723348635 1.59207272529602 28.0
251.89271881 0.556409120559692 29.0
251.89271881 1.73384463787079 30.0
46.866106673 0.992725551128387 31.0
9.3050138724 0.468130618333817 32.0
22.834381403 1.14552867412567 33.0
22.834381403 1.09163129329681 34.0
22.834381403 0.139467716217041 35.0
10.01209209 0.0414751023054123 36.0
8.7529006689 0.424024552106857 37.0
0 0.015363946557045 38.0
0 0.0441228002309799 39.0
-11.985264093 0.734483003616333 40.0
-24.77070127 0.512568056583405 41.0
5.3533762497 0.407766342163086 42.0
-32.483444277 0.56630927324295 43.0
98.932892238 0.771159052848816 44.0
98.932892238 0.0569088533520699 45.0
6.9199219046 0.457576483488083 46.0
2.2929415342 0.303667426109314 47.0
161.35690124 0.0936094373464584 48.0
155.4210356 0.317372411489487 49.0
0 0.472622573375702 50.0
62.702814969 0.0982887893915176 51.0
92.164160542 0.174113884568214 52.0
75.438499931 0.0951813533902168 53.0
75.438499931 0.0525829270482063 54.0
75.438499931 0.123665310442448 55.0
0 5.45170927047729 56.0
76.537164685 0.153349161148071 57.0
222.16937017 0.136241137981415 58.0
251.89271881 0.339490711688995 59.0
22.834381403 0.0195584632456303 60.0
0 0.107806742191315 61.0
180.48820831 0.965656399726868 0.0
180.48820831 1.86149096488953 1.0
179.19361808 1.32493078708649 2.0
173.76886109 5.7679591178894 3.0
0 4.4637451171875 4.0
0 4.20378732681274 5.0
-52.162541306 2.89346766471863 6.0
60.01574268 0.498147785663605 7.0
39.531036531 0.509529769420624 8.0
20.484706149 0.588711857795715 9.0
86.78210159 1.09822249412537 10.0
73.542207202 0.557964742183685 11.0
76.800217967 0.908789217472076 12.0
76.800217967 0.713657736778259 13.0
76.800217967 -0.011388904415071 14.0
0 0.0691804885864258 15.0
0 0.204857334494591 16.0
0 0.914751529693604 17.0
54.627928984 0.913784563541412 18.0
-54.627928984 0.945167243480682 19.0
-55.267713809 0.631198763847351 20.0
-55.334677119 -0.0919677168130875 21.0
77.204032752 2.10751533508301 22.0
138.40103208 2.62307739257812 23.0
109.35549104 0.586225867271423 24.0
21.604901443 0.065657414495945 25.0
215.60506484 1.16311383247375 26.0
215.60506484 2.21406531333923 27.0
24.76967101 1.88812780380249 28.0
240.37473585 0.984070897102356 29.0
240.37473585 1.78210890293121 30.0
47.020840759 1.005744099617 31.0
9.2409313665 0.381338238716125 32.0
21.869355633 1.9123797416687 33.0
21.869355633 1.09179925918579 34.0
21.869355633 0.120425194501877 35.0
10.018968833 0.0331013798713684 36.0
8.6392390658 0.346802771091461 37.0
0 0.0406179614365101 38.0
0 0.160582289099693 39.0
-11.883575997 0.69756555557251 40.0
-25.928489621 0.203711196780205 41.0
5.7533183536 0.541380405426025 42.0
-33.833279062 0.446411162614822 43.0
111.91065639 0.759692788124084 44.0
110.67392902 0.130827277898788 45.0
6.3838384313 0.529935359954834 46.0
1.8432270081 0.316065937280655 47.0
180.48820831 0.164382562041283 48.0
173.76886109 0.195515707135201 49.0
0 0.118533104658127 50.0
60.01574268 0.318377673625946 51.0
86.78210159 0.525331556797028 52.0
76.800217967 0.115215502679348 53.0
76.800217967 0.158436238765717 54.0
76.800217967 0.00419483007863164 55.0
0 2.47160673141479 56.0
77.204032752 0.144790932536125 57.0
215.60506484 0.764776468276978 58.0
240.37473585 0.766097784042358 59.0
21.869355633 0.0024910531938076 60.0
0 0.105797491967678 61.0
168.15741008 0.854045689105988 0.0
168.15741008 0.95329624414444 1.0
167.25598816 0.943697094917297 2.0
161.79131739 2.79131269454956 3.0
21.30424522 4.60180950164795 4.0
21.30424522 4.5518627166748 5.0
-58.928482249 0.0740011632442474 6.0
67.894018317 1.20820498466492 7.0
38.091437033 0.390852302312851 8.0
29.802581284 0.274882048368454 9.0
83.00569868 0.987471461296082 10.0
68.435899366 0.227953732013702 11.0
77.166824582 0.479923337697983 12.0
77.166824582 0.42263001203537 13.0
77.166824582 0.0192457363009453 14.0
2.1279160358 0.0128116644918919 15.0
2.1279160358 0.204555720090866 16.0
0 1.28353273868561 17.0
53.731562293 0.639059543609619 18.0
-53.731562293 1.08527326583862 19.0
-53.731562293 0.196685791015625 20.0
-53.731562293 0.250637829303741 21.0
71.889878193 1.23185479640961 22.0
146.93594599 1.17844152450562 23.0
113.48540655 0.119833126664162 24.0
23.877787159 0.427797317504883 25.0
218.82582418 1.09523439407349 26.0
218.82582418 1.20280730724335 27.0
29.073493074 1.27517712116241 28.0
247.89931725 0.452456325292587 29.0
247.89931725 1.22720122337341 30.0
46.993841635 0.94404000043869 31.0
9.3757922859 0.461803197860718 32.0
18.1583159 1.45922255516052 33.0
18.1583159 0.890130221843719 34.0
18.1583159 0.136813849210739 35.0
8.6862044818 0.0266061276197433 36.0
7.5134179853 0.432097911834717 37.0
0 0.0210198685526848 38.0
0 0.326260983943939 39.0
-11.672225543 0.62298047542572 40.0
-24.106489414 0.95429915189743 41.0
4.9020167432 0.524057924747467 42.0
-31.244394319 0.931318700313568 43.0
71.837722103 0.328353136777878 44.0
71.837722103 0.106103897094727 45.0
7.888258533 0.354106724262238 46.0
3.0429212952 0.304246008396149 47.0
168.15741008 0.139499723911285 48.0
161.79131739 0.276462197303772 49.0
21.30424522 0.312457054853439 50.0
67.894018317 0.245556086301804 51.0
83.00569868 0.44549822807312 52.0
77.166824582 0.0626878142356873 53.0
77.166824582 0.0171934701502323 54.0
77.166824582 0.10386960208416 55.0
0 4.33148431777954 56.0
71.889878193 0.0237481128424406 57.0
218.82582418 0.602699458599091 58.0
247.89931725 0.319915026426315 59.0
18.1583159 0.00590801984071732 60.0
0 0.102369464933872 61.0
161.56808421 0.84864753484726 0.0
161.56808421 1.60560190677643 1.0
160.46061593 1.78884267807007 2.0
154.99598052 3.21040225028992 3.0
7.6242250654 6.01764154434204 4.0
7.6242250654 4.63559341430664 5.0
-65.381494948 3.00675511360168 6.0
75.124125802 0.34632220864296 7.0
39.666618139 0.618138432502747 8.0
35.457507663 0.733553171157837 9.0
84.612004521 0.784494519233704 10.0
71.348026742 0.0462839901447296 11.0
76.893591861 0.0918282046914101 12.0
76.893591861 0.0990859270095825 13.0
76.893591861 0.113248832523823 14.0
0 0.0571438446640968 15.0
0 0.184922784566879 16.0
0 1.06366229057312 17.0
59.457580655 1.6987396478653 18.0
-59.457580655 1.60373151302338 19.0
-59.457580655 0.607834458351135 20.0
-59.457580655 -0.0536336675286293 21.0
81.381430863 1.22708439826965 22.0
150.78428607 1.12702322006226 23.0
117.20817456 0.537273645401001 24.0
23.732195823 0.141536265611649 25.0
232.16571693 0.582894623279572 26.0
232.16571693 1.40147423744202 27.0
24.979735841 1.52141213417053 28.0
257.14545277 0.983938336372375 29.0
257.14545277 1.74056804180145 30.0
46.007280465 1.02399957180023 31.0
9.2119552426 0.413377523422241 32.0
21.923850208 1.22876238822937 33.0
21.923850208 1.01422345638275 34.0
21.923850208 0.142826497554779 35.0
10.35560301 0.0294704549014568 36.0
9.1002623359 0.423014253377914 37.0
0 0.0171507224440575 38.0
0 0.0824742987751961 39.0
-11.574133853 0.448945701122284 40.0
-25.416391479 0.401026338338852 41.0
4.6761107437 0.353966176509857 42.0
-32.544441595 0.55681586265564 43.0
96.507140542 0.862703502178192 44.0
96.507140542 0.0404167547821999 45.0
7.0729198436 0.521631956100464 46.0
2.750788754 0.273168653249741 47.0
161.56808421 0.186396107077599 48.0
154.99598052 0.339845836162567 49.0
7.6242250654 0.427625685930252 50.0
75.124125802 0.144137263298035 51.0
84.612004521 0.1794144064188 52.0
76.893591861 0.0480550974607468 53.0
76.893591861 0.0897247865796089 54.0
76.893591861 0.086908258497715 55.0
0 8.44205379486084 56.0
81.381430863 0.236693054437637 57.0
232.16571693 0.230678573250771 58.0
257.14545277 0.442114859819412 59.0
21.923850208 -0.00279194489121437 60.0
0 0.098735973238945 61.0
193.72083452 0.935678780078888 0.0
193.72083452 1.84496557712555 1.0
192.24741378 1.86911344528198 2.0
186.78276478 3.75687646865845 3.0
2.4691746117 4.71400690078735 4.0
2.4691746117 4.66204166412354 5.0
-69.140226565 2.21049237251282 6.0
81.256079727 0.820269346237183 7.0
51.425224982 0.815524935722351 8.0
29.830854745 0.851186990737915 9.0
86.840837456 0.97397243976593 10.0
73.486999055 0.0569858476519585 11.0
84.701362592 0.762857139110565 12.0
84.701362592 0.425399124622345 13.0
84.701362592 0.0717581510543823 14.0
0 0.136865183711052 15.0
0 0.176172390580177 16.0
0 0.913699984550476 17.0
57.889267832 1.74340295791626 18.0
-57.889267832 1.45401179790497 19.0
-57.889267832 0.750672340393066 20.0
-57.889267832 0.00621366873383522 21.0
78.347850792 2.85488343238831 22.0
136.47495042 2.12015652656555 23.0
100.33607699 0.546976089477539 24.0
21.392607535 0.110319420695305 25.0
214.82280121 1.241290807724 26.0
214.82280121 1.55489611625671 27.0
24.167949361 2.18088245391846 28.0
238.99075057 1.07054352760315 29.0
238.99075057 1.6967465877533 30.0
46.939706998 0.964807152748108 31.0
9.4763460741 0.330597072839737 32.0
20.45858296 1.26486933231354 33.0
20.45858296 1.18161594867706 34.0
20.45858296 0.120137616991997 35.0
9.1982786887 0.028632365167141 36.0
7.8386597746 0.381074756383896 37.0
0 0.0406023226678371 38.0
0 0.160135447978973 39.0
-13.188188914 0.499724358320236 40.0
-27.583459341 0.710593700408936 41.0
4.9864362017 0.451511204242706 42.0
-35.21929809 0.506504416465759 43.0
88.654100133 0.743479073047638 44.0
88.654100133 0.0739511922001839 45.0
6.8929019156 0.676586985588074 46.0
2.3501442197 0.428362667560577 47.0
193.72083452 0.0620441660284996 48.0
186.78276478 0.0626598000526428 49.0
2.4691746117 0.128438428044319 50.0
81.256079727 0.29165643453598 51.0
86.840837456 0.161448448896408 52.0
84.701362592 0.0360854379832745 53.0
84.701362592 0.238092079758644 54.0
84.701362592 0.0745624750852585 55.0
0 3.73551988601685 56.0
78.347850792 0.102068334817886 57.0
214.82280121 0.340814977884293 58.0
238.99075057 0.421031266450882 59.0
20.45858296 -0.0115966750308871 60.0
0 0.0933811292052269 61.0
184.52645839 0.836770355701447 0.0
184.52645839 1.44889616966248 1.0
183.12960159 2.04377937316895 2.0
178.34049632 4.20881509780884 3.0
0 9.12124538421631 4.0
0 6.29649591445923 5.0
-73.629023739 5.82530498504639 6.0
84.94783681 0.840631783008575 7.0
61.521324085 1.14939141273499 8.0
23.426512725 0.807205200195312 9.0
84.31948026 0.942682325839996 10.0
70.11541242 0.272130370140076 11.0
80.174282035 0.660898089408875 12.0
80.174282035 0.69587767124176 13.0
80.174282035 0.0462561808526516 14.0
0 0.0437327101826668 15.0
0 0.196361124515533 16.0
0 1.09769189357758 17.0
52.615358262 1.38708794116974 18.0
-52.615358262 1.8882976770401 19.0
-52.615358262 0.386700689792633 20.0
-52.615358262 0.00190260633826256 21.0
72.816016715 1.27546417713165 22.0
141.01444259 0.821883976459503 23.0
103.74449647 0.910113632678986 24.0
21.57661486 0.0521594770252705 25.0
213.8304593 1.30897998809814 26.0
213.8304593 2.09719944000244 27.0
26.146791145 2.35013318061829 28.0
239.97725045 0.938171148300171 29.0
239.97725045 1.91825091838837 30.0
46.69838396 1.06921124458313 31.0
9.4641202954 0.437971323728561 32.0
20.200658453 1.2008900642395 33.0
20.200658453 0.963880717754364 34.0
20.200658453 0.107808589935303 35.0
9.7457295303 0.0305376034229994 36.0
8.485414955 0.380574554204941 37.0
0 0.0538777634501457 38.0
0 0.19118943810463 39.0
-12.615053953 0.596499919891357 40.0
-25.100408789 0.770692586898804 41.0
5.5180908764 0.419911205768585 42.0
-33.264712008 0.698311805725098 43.0
84.692285028 0.653653562068939 44.0
84.692285028 0.0871255546808243 45.0
7.1134642172 0.529550731182098 46.0
2.4262701902 0.361263513565063 47.0
184.52645839 0.0919668972492218 48.0
178.34049632 0.381654411554337 49.0
0 0.553959250450134 50.0
84.94783681 0.187770158052444 51.0
84.31948026 0.420185834169388 52.0
80.174282035 0.0495606064796448 53.0
80.174282035 0.171100154519081 54.0
80.174282035 0.210742458701134 55.0
0 3.16108417510986 56.0
72.816016715 0.0519561544060707 57.0
213.8304593 0.616854071617126 58.0
239.97725045 0.482427507638931 59.0
20.200658453 -0.0025180671364069 60.0
0 0.101567521691322 61.0
176.56837479 0.784772753715515 0.0
176.56837479 1.27213561534882 1.0
175.16302105 1.4368953704834 2.0
170.3739375 3.38533973693848 3.0
0 8.46764278411865 4.0
0 4.32608842849731 5.0
-68.967079942 0.38925439119339 6.0
78.190691129 0.860342383384705 7.0
49.472395336 1.21470022201538 8.0
28.718295792 1.02689599990845 9.0
83.194741495 1.03557670116425 10.0
70.097347946 0.317063421010971 11.0
80.801204438 0.705350577831268 12.0
80.801204438 0.595338821411133 13.0
80.801204438 0.117365159094334 14.0
0 0.254194110631943 15.0
0 0.17134153842926 16.0
0 1.36844217777252 17.0
59.14798884 1.04606568813324 18.0
-59.14798884 1.09545004367828 19.0
-59.14798884 0.547482013702393 20.0
-59.14798884 -0.0454649738967419 21.0
81.09638154 1.45983529090881 22.0
152.51514538 1.22551894187927 23.0
118.51586475 0.365825951099396 24.0
24.630385762 0.226491957902908 25.0
233.61152692 0.944984078407288 26.0
233.61152692 1.61012578010559 27.0
26.355936635 1.51318109035492 28.0
259.96746355 0.74369889497757 29.0
259.96746355 1.49828267097473 30.0
46.147807074 1.16168296337128 31.0
9.2736270941 0.378619641065598 32.0
21.9483927 1.48940920829773 33.0
21.9483927 0.963412404060364 34.0
21.9483927 0.116539932787418 35.0
9.772407862 0.030020497739315 36.0
8.5364075255 0.353958159685135 37.0
0 0.0304745770990849 38.0
0 0.180624738335609 39.0
-11.419962849 0.723857223987579 40.0
-23.681223645 0.554637372493744 41.0
5.5884815317 0.545547664165497 42.0
-32.026723413 0.559193074703217 43.0
106.18909429 0.855133235454559 44.0
106.18909429 0.0805413126945496 45.0
6.6467911262 0.745050132274628 46.0
2.7331438281 0.391510128974915 47.0
176.56837479 0.0730572044849396 48.0
170.3739375 0.143617376685143 49.0
0 0.237122640013695 50.0
78.190691129 0.376577436923981 51.0
83.194741495 0.388826310634613 52.0
80.801204438 0.0300429239869118 53.0
80.801204438 0.226870030164719 54.0
80.801204438 0.0358106195926666 55.0
0 3.27838277816772 56.0
81.09638154 0.0241824686527252 57.0
233.61152692 0.425659477710724 58.0
259.96746355 0.404353439807892 59.0
21.9483927 0.011855024844408 60.0
0 0.0977435708045959 61.0
193.68906328 0.741457641124725 0.0
193.68906328 1.66688978672028 1.0
192.56935083 1.80178618431091 2.0
187.78024571 3.84762287139893 3.0
0 8.06265354156494 4.0
0 6.40716505050659 5.0
-74.784358578 1.0988450050354 6.0
87.242437161 1.23854196071625 7.0
60.531234635 0.9305419921875 8.0
26.711202526 0.894926309585571 9.0
79.895433 0.876228153705597 10.0
66.41996762 0.135670736432076 11.0
79.620983406 0.658938765525818 12.0
79.620983406 0.347450017929077 13.0
79.620983406 0.076024055480957 14.0
0 0.178849548101425 15.0
0 0.160510390996933 16.0
0 1.44221031665802 17.0
55.998348001 1.06923162937164 18.0
-55.998348001 0.705757200717926 19.0
-55.998348001 1.02462589740753 20.0
-55.998348001 -0.0921014696359634 21.0
77.932458275 1.57707333564758 22.0
141.69579806 0.913780808448792 23.0
105.80235798 0.603614151477814 24.0
22.836016599 0.0808003693819046 25.0
219.62825634 1.07539772987366 26.0
219.62825634 1.67725586891174 27.0
23.649524457 2.32783341407776 28.0
243.27778079 0.914121150970459 29.0
243.27778079 1.75392472743988 30.0
46.581205032 1.14746117591858 31.0
9.361555298 0.566838383674622 32.0
21.934110274 1.64185631275177 33.0
21.934110274 0.778769791126251 34.0
21.934110274 0.139228165149689 35.0
9.2573615965 0.0376473665237427 36.0
8.1126749498 0.310145109891891 37.0
0 0.024888601154089 38.0
0 0.143775105476379 39.0
-12.910480761 0.451146334409714 40.0
-24.721280974 0.722852826118469 41.0
5.3224357229 0.344598710536957 42.0
-32.424346199 0.508157849311829 43.0
95.227010125 0.604404151439667 44.0
95.227010125 0.0715937614440918 45.0
7.0088251319 0.640965282917023 46.0
2.430607617 0.409988462924957 47.0
193.68906328 0.0987102389335632 48.0
187.78024571 0.100674860179424 49.0
0 0.401959896087646 50.0
87.242437161 0.177689626812935 51.0
79.895433 0.26181498169899 52.0
79.620983406 0.0199487283825874 53.0
79.620983406 0.111947044730186 54.0
79.620983406 0.185196921229362 55.0
0 2.96841478347778 56.0
77.932458275 0.143558919429779 57.0
219.62825634 0.470657020807266 58.0
243.27778079 0.305319368839264 59.0
21.934110274 0.0347552746534348 60.0
0 0.0862783342599869 61.0
176.62577222 0.852194726467133 0.0
176.62577222 1.05115330219269 1.0
175.48186715 1.23252487182617 2.0
170.69276188 4.24201774597168 3.0
0 5.17670726776123 4.0
0 2.74234294891357 5.0
-66.275389592 1.39244341850281 6.0
75.160024087 0.512164711952209 7.0
56.074231025 0.498373717069626 8.0
19.085793063 0.81313943862915 9.0
78.70530008 0.874955058097839 10.0
64.286290725 0.199174836277962 11.0
75.356469148 0.544417858123779 12.0
75.356469148 0.0808096677064896 13.0
75.356469148 0.101946048438549 14.0
0 0.0667447596788406 15.0
0 0.215823784470558 16.0
0 1.36254191398621 17.0
49.695966198 1.34448432922363 18.0
-49.695966198 0.993174195289612 19.0
-49.695966198 0.742922902107239 20.0
-49.695966198 -0.0432698540389538 21.0
70.041458639 1.25715112686157 22.0
151.23048088 1.85948967933655 23.0
114.9137013 0.16627536714077 24.0
24.849151317 0.410584956407547 25.0
221.27193952 1.21195673942566 26.0
221.27193952 1.47782981395721 27.0
29.486827227 1.30914044380188 28.0
250.75876674 0.550646305084229 29.0
250.75876674 1.49495685100555 30.0
46.714293134 1.13473808765411 31.0
9.1816726814 0.474903851747513 32.0
20.345492441 1.16409921646118 33.0
20.345492441 1.51565992832184 34.0
20.345492441 0.130001693964005 35.0
9.3965376349 0.0313073098659515 36.0
8.1475667957 0.355870693922043 37.0
0 0.0279409512877464 38.0
0 0.308909386396408 39.0
-11.549378016 0.473083555698395 40.0
-24.032548669 0.63011634349823 41.0
4.6364633235 0.611263394355774 42.0
-30.85877064 0.666521728038788 43.0
92.977292805 0.655004501342773 44.0
92.977292805 0.13552987575531 45.0
6.5366749276 0.495952159166336 46.0
2.7497352111 0.365736812353134 47.0
176.62577222 0.152008652687073 48.0
170.69276188 0.146513000130653 49.0
0 0.0874541476368904 50.0
75.160024087 0.106381371617317 51.0
78.70530008 0.303130507469177 52.0
75.356469148 0.0258056987076998 53.0
75.356469148 0.0964686423540115 54.0
75.356469148 0.0826058834791183 55.0
0 4.72510480880737 56.0
70.041458639 0.128824204206467 57.0
221.27193952 0.231733739376068 58.0
250.75876674 0.623123943805695 59.0
20.345492441 -0.0158270709216595 60.0
0 0.100101590156555 61.0
151.44070371 1.06269836425781 0.0
151.44070371 2.23000121116638 1.0
149.9355002 2.70210289955139 2.0
144.4708512 4.79070234298706 3.0
20.648404785 7.51273250579834 4.0
20.648404785 4.91418600082397 5.0
-58.683665911 6.58207416534424 6.0
68.412630429 0.509082674980164 7.0
40.497481066 0.926263809204102 8.0
27.915149362 1.00142979621887 9.0
79.218207491 1.21027565002441 10.0
67.303469935 0.157917857170105 11.0
74.010836646 0.624238729476929 12.0
74.010836646 0.504051089286804 13.0
74.010836646 0.17197872698307 14.0
0 0.220460996031761 15.0
0 0.133884951472282 16.0
0 1.59375286102295 17.0
53.46404414 1.01426577568054 18.0
-53.46404414 1.02427577972412 19.0
-53.46404414 0.289573043584824 20.0
-53.46404414 -0.00857829675078392 21.0
76.773402278 1.98330068588257 22.0
142.13207685 2.04825186729431 23.0
103.9125555 0.276609897613525 24.0
22.626440266 0.214532151818275 25.0
218.90547912 1.07695043087006 26.0
218.90547912 1.58557045459747 27.0
26.528782665 1.56019282341003 28.0
245.43426179 0.418744713068008 29.0
245.43426179 1.64951074123383 30.0
46.581123424 1.18803513050079 31.0
9.3317632892 0.494829922914505 32.0
23.309358138 1.17655980587006 33.0
23.309358138 1.11625051498413 34.0
23.309358138 0.127692997455597 35.0
10.65668083 0.0416243523359299 36.0
9.4313337285 0.32417619228363 37.0
0 0.0165532417595387 38.0
0 0.301893562078476 39.0
-12.299853146 0.631340324878693 40.0
-26.496527952 1.10240149497986 41.0
4.7958293044 0.343912154436111 42.0
-33.771132056 0.292517244815826 43.0
84.629442133 0.787740707397461 44.0
84.629442133 0.0676214247941971 45.0
7.2721478761 0.598351716995239 46.0
2.6647718917 0.401732891798019 47.0
151.44070371 0.139237582683563 48.0
144.4708512 0.671603441238403 49.0
20.648404785 0.524158656597137 50.0
68.412630429 0.248652443289757 51.0
79.218207491 0.348215728998184 52.0
74.010836646 0.0570704489946365 53.0
74.010836646 0.147575333714485 54.0
74.010836646 0.0634512081742287 55.0
0 5.46938705444336 56.0
76.773402278 0.0162498541176319 57.0
218.90547912 0.246470361948013 58.0
245.43426179 0.216971725225449 59.0
23.309358138 -0.0154051193967462 60.0
0 0.06395423412323 61.0
168.95350197 0.717783808708191 0.0
168.95350197 0.938222110271454 1.0
167.55335374 0.925001621246338 2.0
162.76424848 4.49725723266602 3.0
0 6.17495489120483 4.0
0 3.52477717399597 5.0
-55.723204526 1.00656223297119 6.0
68.520753378 0.97538834810257 7.0
45.001583811 0.723132848739624 8.0
23.519169567 0.693767488002777 9.0
87.962931551 0.868617832660675 10.0
76.064915038 0.144531339406967 11.0
79.726060652 0.79510635137558 12.0
79.726060652 0.961065232753754 13.0
79.726060652 0.0756762772798538 14.0
0 0.128628894686699 15.0
0 0.169230937957764 16.0
0 1.11908900737762 17.0
56.961613425 1.25148320198059 18.0
-56.961613425 1.38045275211334 19.0
-56.961613425 0.437850892543793 20.0
-56.961613425 0.0223422534763813 21.0
78.471320661 1.37176823616028 22.0
145.89773349 1.19021713733673 23.0
110.13321001 0.633119642734528 24.0
21.961044811 0.072913259267807 25.0
224.36905415 1.27352285385132 26.0
224.36905415 1.41310346126556 27.0
28.983893383 1.00566744804382 28.0
253.35294754 0.71471893787384 29.0
253.35294754 1.58053660392761 30.0
46.815347548 1.07138729095459 31.0
9.2115906188 0.425501793622971 32.0
21.509707236 1.04949998855591 33.0
21.509707236 1.2721985578537 34.0
21.509707236 0.131455317139626 35.0
9.3517900206 0.027889508754015 36.0
8.1793029163 0.358565241098404 37.0
0 0.0121179632842541 38.0
0 0.160376265645027 39.0
-11.361140511 0.31013685464859 40.0
-25.659249907 0.851583182811737 41.0
4.9667490783 0.36988890171051 42.0
-33.248611826 0.577365815639496 43.0
84.159893707 0.56186705827713 44.0
84.159893707 0.0834569334983826 45.0
7.2854643332 0.534929513931274 46.0
2.2761000185 0.390516698360443 47.0
168.95350197 0.057414922863245 48.0
162.76424848 0.00203636661171913 49.0
0 0.206565856933594 50.0
68.520753378 0.177446842193604 51.0
87.962931551 0.313846290111542 52.0
79.726060652 0.0294367205351591 53.0
79.726060652 0.295272022485733 54.0
79.726060652 0.219917699694633 55.0
0 3.84913897514343 56.0
78.471320661 0.0181174762547016 57.0
224.36905415 0.249512121081352 58.0
253.35294754 0.756105661392212 59.0
21.509707236 -0.0237331222742796 60.0
0 0.102543212473392 61.0
160.51375379 0.967484295368195 0.0
160.51375379 0.691177606582642 1.0
159.32142308 0.649332463741302 2.0
154.53233953 3.60580635070801 3.0
0 6.31833934783936 4.0
0 4.23467969894409 5.0
-63.243508488 0.754779040813446 6.0
72.852342542 1.4304563999176 7.0
47.582604012 0.628939867019653 8.0
25.269738529 0.409285247325897 9.0
83.637463181 1.15378022193909 10.0
71.028808156 0.3342564702034 11.0
82.307150511 0.861450552940369 12.0
82.307150511 0.748194575309753 13.0
82.307150511 0.108504742383957 14.0
0 0.218799531459808 15.0
0 0.173629447817802 16.0
0 1.52114868164062 17.0
58.374378215 1.26086723804474 18.0
-58.374378215 1.1256548166275 19.0
-58.374378215 0.695191919803619 20.0
-58.374378215 -0.0126569606363773 21.0
79.419870625 1.22841799259186 22.0
138.92283868 0.924781203269958 23.0
101.82270129 0.311095774173737 24.0
20.834261982 0.0830819010734558 25.0
218.3427093 1.03514528274536 26.0
218.3427093 1.30641961097717 27.0
24.573107299 1.34734559059143 28.0
242.9158166 0.762845695018768 29.0
242.9158166 1.92200994491577 30.0
47.18393738 0.88035786151886 31.0
9.2554900709 0.402740269899368 32.0
21.04549241 1.55606985092163 33.0
21.04549241 0.846823394298553 34.0
21.04549241 0.126965180039406 35.0
9.7017134142 0.0254589971154928 36.0
8.3531042963 0.388211965560913 37.0
0 0.0396985113620758 38.0
0 0.170177638530731 39.0
-12.609931164 0.586246907711029 40.0
-25.171137786 0.673103630542755 41.0
5.3284527266 0.410817265510559 42.0
-33.296025955 0.545380890369415 43.0
88.847905539 0.709686100482941 44.0
88.847905539 0.0862757563591003 45.0
7.2211099533 0.614360332489014 46.0
2.5890291919 0.447364419698715 47.0
160.51375379 0.178753778338432 48.0
154.53233953 -0.0241238754242659 49.0
0 0.376768201589584 50.0
72.852342542 0.198561623692513 51.0
83.637463181 0.608192145824432 52.0
82.307150511 0.0519561395049095 53.0
82.307150511 0.105402171611786 54.0
82.307150511 0.32175475358963 55.0
0 3.03305387496948 56.0
79.419870625 0.116741582751274 57.0
218.3427093 0.64079350233078 58.0
242.9158166 1.17981243133545 59.0
21.04549241 -0.00430332683026791 60.0
0 0.0899490565061569 61.0
171.05843258 1.03264284133911 0.0
171.05843258 2.22357726097107 1.0
170.02949329 2.26755213737488 2.0
164.56482203 5.74589204788208 3.0
12.400044397 8.3762378692627 4.0
12.400044397 5.90288352966309 5.0
-65.342316715 0.680156946182251 6.0
73.943798895 0.810093879699707 7.0
44.753899682 0.828051805496216 8.0
29.189899213 0.646762371063232 9.0
96.784755741 1.00416100025177 10.0
82.91451382 0.0961961671710014 11.0
90.196572785 0.854943215847015 12.0
90.196572785 0.488737553358078 13.0
90.196572785 0.111130014061928 14.0
0 0.293571621179581 15.0
0 0.175436764955521 16.0
0 2.0410487651825 17.0
57.641012227 0.987376391887665 18.0
-57.641012227 0.640051543712616 19.0
-57.641012227 0.587160348892212 20.0
-57.641012227 -0.0870022475719452 21.0
79.496568937 1.2773904800415 22.0
163.44812651 1.32473790645599 23.0
124.02043094 0.211773052811623 24.0
24.451735808 0.389663964509964 25.0
242.94469545 0.766023755073547 26.0
242.94469545 1.58311486244202 27.0
24.915760406 1.26285910606384 28.0
267.86045586 0.530008435249329 29.0
267.86045586 1.71293354034424 30.0
46.894805834 1.08046245574951 31.0
9.2496323274 0.543375074863434 32.0
21.85555671 2.03236317634583 33.0
21.85555671 0.675092339515686 34.0
21.85555671 0.0826669558882713 35.0
9.1577463892 0.0306989978998899 36.0
7.9146651812 0.332345247268677 37.0
0 0.0598904117941856 38.0
0 0.119463093578815 39.0
-10.328809041 1.19029951095581 40.0
-21.762903916 0.504272818565369 41.0
4.567660073 0.51074492931366 42.0
-28.537465347 0.590414881706238 43.0
85.779940835 0.799370527267456 44.0
85.779940835 0.0852646455168724 45.0
7.4232573627 0.720420718193054 46.0
2.6107391777 0.383608222007751 47.0
171.05843258 0.101708024740219 48.0
164.56482203 0.39455372095108 49.0
12.400044397 0.441899955272675 50.0
73.943798895 0.31296893954277 51.0
96.784755741 0.0753546729683876 52.0
90.196572785 0.0978796631097794 53.0
90.196572785 0.0643600448966026 54.0
90.196572785 0.233207121491432 55.0
0 0.843404173851013 56.0
79.496568937 0.27363532781601 57.0
242.94469545 0.441907376050949 58.0
267.86045586 0.407837241888046 59.0
21.85555671 0.0112649369984865 60.0
0 0.0967526212334633 61.0
190.27224874 0.808080673217773 0.0
190.27224874 0.883690655231476 1.0
189.03910267 1.38113808631897 2.0
187.45212546 3.52659463882446 3.0
0 7.79131031036377 4.0
0 3.35952258110046 5.0
-59.948735326 1.44096267223358 6.0
72.645341964 0.549978494644165 7.0
43.130276884 1.01983463764191 8.0
29.51506508 0.91478443145752 9.0
98.995588479 0.965951085090637 10.0
86.272434746 0.415747314691544 11.0
88.617577996 0.970175445079803 12.0
88.617577996 0.610152840614319 13.0
88.617577996 0.0583766624331474 14.0
0 0.19416181743145 15.0
0 0.203173607587814 16.0
0 0.884813785552979 17.0
58.8728769 1.52060496807098 18.0
-58.8728769 0.960515260696411 19.0
-58.8728769 1.11681985855103 20.0
-58.8728769 -0.033379640430212 21.0
80.648190601 1.60720348358154 22.0
144.59566836 0.821071147918701 23.0
109.03693974 0.718963503837585 24.0
20.190475608 0.0816176980733871 25.0
225.24385896 1.25480163097382 26.0
225.24385896 1.53763329982758 27.0
25.581207363 2.0923171043396 28.0
250.82506632 1.18212783336639 29.0
250.82506632 2.00511455535889 30.0
46.360627272 0.980172574520111 31.0
9.3055415729 0.528112173080444 32.0
21.775313701 1.45603501796722 33.0
21.775313701 1.14156913757324 34.0
21.775313701 0.12879578769207 35.0
9.6675692437 0.0311112012714148 36.0
8.3352560945 0.35026490688324 37.0
0 0.0289650112390518 38.0
0 0.335737973451614 39.0
-11.423942043 0.50372576713562 40.0
-24.416891853 0.578173458576202 41.0
4.9463947517 0.491724282503128 42.0
-31.914439508 0.462014257907867 43.0
100.05595002 0.742744922637939 44.0
100.05595002 0.0429123789072037 45.0
6.5709251929 0.655675232410431 46.0
2.4428093473 0.405832320451736 47.0
190.27224874 0.193285867571831 48.0
187.45212546 0.112642660737038 49.0
0 0.227146774530411 50.0
72.645341964 0.233450770378113 51.0
98.995588479 0.452127009630203 52.0
88.617577996 0.151030912995338 53.0
88.617577996 0.116549283266068 54.0
88.617577996 0.183933258056641 55.0
0 3.67476916313171 56.0
80.648190601 0.127894535660744 57.0
225.24385896 0.314950376749039 58.0
250.82506632 1.13328742980957 59.0
21.775313701 -0.0173823200166225 60.0
0 0.105299428105354 61.0
143.67679143 1.07416331768036 0.0
143.67679143 1.71550476551056 1.0
142.75611065 1.83021473884583 2.0
137.9670407 3.84902954101562 3.0
0 4.63747882843018 4.0
0 3.16288042068481 5.0
-62.304751924 0.269747048616409 6.0
70.14095354 0.940596461296082 7.0
39.461610657 1.50847041606903 8.0
30.679342883 1.69230926036835 9.0
90.294915622 1.21449685096741 10.0
76.911207962 0.109678350389004 11.0
91.733729935 0.13362167775631 12.0
91.733729935 0.125329911708832 13.0
91.733729935 0.0472676604986191 14.0
0 0.147625610232353 15.0
0 0.235860839486122 16.0
0 1.11564421653748 17.0
57.794897878 0.70349109172821 18.0
-57.794897878 0.657849550247192 19.0
-57.794897878 0.482953488826752 20.0
-57.794897878 0.16931939125061 21.0
81.299112293 0.71537834405899 22.0
130.54496998 0.324198126792908 23.0
99.429972734 0.87889701128006 24.0
10.886942234 0.0941599458456039 25.0
211.84408228 0.8665372133255 26.0
211.84408228 2.19380068778992 27.0
31.516475281 1.06215512752533 28.0
243.36055756 0.744395136833191 29.0
243.36055756 1.80174827575684 30.0
47.401393335 1.13368046283722 31.0
9.1938903663 0.331625819206238 32.0
23.504214415 1.1730443239212 33.0
23.504214415 0.776828825473785 34.0
23.504214415 0.152302011847496 35.0
10.083059827 0.0343685820698738 36.0
8.9693057099 0.348156958818436 37.0
0 0.00333840772509575 38.0
0 0.0638150125741959 39.0
-11.332241131 0.654829502105713 40.0
-23.793939384 0.65568333864212 41.0
4.470492759 0.529044032096863 42.0
-30.547157932 0.692372262477875 43.0
83.624531015 0.725542604923248 44.0
83.624531015 0.125361651182175 45.0
6.4279530608 0.49726203083992 46.0
2.0921249708 0.293963372707367 47.0
143.67679143 0.12459921091795 48.0
137.9670407 0.311717957258224 49.0
0 0.179006353020668 50.0
70.14095354 0.374916791915894 51.0
90.294915622 0.251069039106369 52.0
91.733729935 0.00162955140694976 53.0
91.733729935 0.0899099186062813 54.0
91.733729935 0.0284887216985226 55.0
0 1.95598077774048 56.0
81.299112293 -0.078167051076889 57.0
211.84408228 0.780007064342499 58.0
243.36055756 0.506192624568939 59.0
23.504214415 0.00832484848797321 60.0
0 0.215621337294579 61.0
176.89146066 1.10668385028839 0.0
176.89146066 1.54210579395294 1.0
175.62628432 1.70754039287567 2.0
170.83717904 5.70145988464355 3.0
0 4.16456460952759 4.0
0 5.87655639648438 5.0
-66.474806172 0.95354425907135 6.0
77.491000913 0.48801651597023 7.0
46.03289627 1.47875928878784 8.0
31.458104643 1.37172544002533 9.0
90.703016916 1.27164804935455 10.0
78.054231509 0.220530673861504 11.0
83.69143535 0.26091206073761 12.0
83.69143535 0.271494388580322 13.0
83.69143535 0.119821906089783 14.0
0 0.116052769124508 15.0
0 0.198791921138763 16.0
0 1.21535360813141 17.0
52.012480041 1.54472291469574 18.0
-52.012480041 1.85219049453735 19.0
-52.012480041 0.311418950557709 20.0
-52.012480041 0.0807036235928535 21.0
73.714403932 1.82508516311646 22.0
155.78848795 1.70624363422394 23.0
122.6628693 0.201804503798485 24.0
25.176776819 0.282142400741577 25.0
229.50289188 1.02039694786072 26.0
229.50289188 1.67067742347717 27.0
27.397464999 1.50401091575623 28.0
256.90035688 1.13828611373901 29.0
256.90035688 2.1419882774353 30.0
46.654640681 1.1150975227356 31.0
9.3958176741 0.667444825172424 32.0
21.701923891 1.26121413707733 33.0
21.701923891 0.827897608280182 34.0
21.701923891 0.141778945922852 35.0
8.3444180863 0.0325500667095184 36.0
7.0801581193 0.393959045410156 37.0
0 0.0303605496883392 38.0
0 0.13791960477829 39.0
-12.944066732 0.780859410762787 40.0
-26.304364612 0.665501713752747 41.0
5.4926672867 0.4584139585495 42.0
-34.159089142 0.436835944652557 43.0
85.713143051 0.72631847858429 44.0
85.713143051 0.086881123483181 45.0
6.7237453708 0.625023722648621 46.0
2.2175227862 0.398683995008469 47.0
176.89146066 0.136618748307228 48.0
170.83717904 0.101091027259827 49.0
0 0.238842368125916 50.0
77.491000913 0.202122583985329 51.0
90.703016916 0.132822304964066 52.0
83.69143535 0.0276764798909426 53.0
83.69143535 0.086705707013607 54.0
83.69143535 0.0434729754924774 55.0
0 6.32249212265015 56.0
73.714403932 -0.0381823107600212 57.0
229.50289188 0.338102847337723 58.0
256.90035688 1.25425231456757 59.0
21.701923891 0.00343999080359936 60.0
0 0.106338188052177 61.0
186.07272118 0.989260017871857 0.0
186.07272118 1.04313671588898 1.0
184.80978207 1.84361636638641 2.0
180.02067681 5.16186237335205 3.0
0 5.81402635574341 4.0
0 4.39888858795166 5.0
-70.760833538 2.20722913742065 6.0
81.245776294 0.537681460380554 7.0
54.940331828 1.31100463867188 8.0
26.305444466 1.26840591430664 9.0
82.086958061 1.08019065856934 10.0
68.706492789 0.230097845196724 11.0
80.04972047 0.329882860183716 12.0
80.04972047 0.620805621147156 13.0
80.04972047 0.0718314200639725 14.0
0 0.070590503513813 15.0
0 0.250675827264786 16.0
0 1.06969881057739 17.0
57.649196408 1.4387229681015 18.0
-57.649196408 1.45928049087524 19.0
-57.649196408 0.91613095998764 20.0
-57.649196408 -0.0933009684085846 21.0
79.747844275 1.90445268154144 22.0
148.1752735 1.4722501039505 23.0
112.12545451 0.361439913511276 24.0
22.403204322 0.101458519697189 25.0
227.92311778 1.29413962364197 26.0
227.92311778 1.89094829559326 27.0
29.754947484 1.01674258708954 28.0
257.67806526 0.440654367208481 29.0
257.67806526 1.67335379123688 30.0
46.965490296 1.08468246459961 31.0
9.313830399 0.567296624183655 32.0
22.098647867 2.03470611572266 33.0
22.098647867 1.01811265945435 34.0
22.098647867 0.0848339796066284 35.0
9.457998101 0.0268894769251347 36.0
8.1592721225 0.397608727216721 37.0
0 0.0701582729816437 38.0
0 0.270652115345001 39.0
-11.917057876 0.525919556617737 40.0
-24.635222198 0.654587149620056 41.0
5.0058306743 0.532816648483276 42.0
-32.531513669 0.633786141872406 43.0
86.012745905 0.658144414424896 44.0
86.012745905 0.0981107652187347 45.0
6.8025532448 0.580909252166748 46.0
2.7297276856 0.365556538105011 47.0
186.07272118 0.28405898809433 48.0
180.02067681 0.320847570896149 49.0
0 0.34517514705658 50.0
81.245776294 0.192106008529663 51.0
82.086958061 0.235409408807755 52.0
80.04972047 0.0779127553105354 53.0
80.04972047 0.0787945464253426 54.0
80.04972047 0.247235387563705 55.0
0 5.0813512802124 56.0
79.747844275 0.198524594306946 57.0
227.92311778 0.596968412399292 58.0
257.67806526 0.634101569652557 59.0
22.098647867 -0.00175851583480835 60.0
0 0.0621297806501389 61.0
153.81594618 0.790739774703979 0.0
153.81594618 1.37401330471039 1.0
152.59055335 1.26963984966278 2.0
147.80144809 4.92365550994873 3.0
0 7.12081432342529 4.0
0 3.03104066848755 5.0
-54.871311323 0.640169978141785 6.0
65.355245845 0.527829170227051 7.0
54.354496185 0.807035148143768 8.0
11.00074966 0.786545217037201 9.0
97.103064841 0.850446164608002 10.0
84.153781818 0.207226812839508 11.0
89.767387911 0.606109738349915 12.0
89.767387911 0.278165638446808 13.0
89.767387911 0.124247260391712 14.0
0 0.283115267753601 15.0
0 0.15468268096447 16.0
0 1.86091279983521 17.0
51.570805514 1.45955288410187 18.0
-51.570805514 1.47490322589874 19.0
-51.570805514 0.210995897650719 20.0
-51.570805514 0.115856990218163 21.0
72.164968 1.84526896476746 22.0
141.66811458 1.52494597434998 23.0
109.63767552 0.436408698558807 24.0
24.045197586 0.115710988640785 25.0
213.83308258 1.04765152931213 26.0
213.83308258 2.21567106246948 27.0
26.936552518 1.28671085834503 28.0
240.7696351 0.759757995605469 29.0
240.7696351 1.54665207862854 30.0
47.115580382 1.1231507062912 31.0
9.3033985268 0.498399168252945 32.0
20.594162486 0.903400778770447 33.0
20.594162486 1.09822416305542 34.0
20.594162486 0.0716290175914764 35.0
9.3834558009 0.0249484982341528 36.0
8.1214490483 0.38965430855751 37.0
0 0.0482464283704758 38.0
0 0.298979580402374 39.0
-12.408567835 0.567409932613373 40.0
-24.880370844 0.603651285171509 41.0
4.9535973368 0.512734591960907 42.0
-32.42814805 0.560178458690643 43.0
89.118523978 0.684543490409851 44.0
89.118523978 0.111617311835289 45.0
6.8162489186 0.44929438829422 46.0
2.5923858473 0.340853005647659 47.0
153.81594618 0.145884469151497 48.0
147.80144809 0.163015693426132 49.0
0 0.144060358405113 50.0
65.355245845 0.0446045212447643 51.0
97.103064841 0.320479989051819 52.0
89.767387911 0.0664985850453377 53.0
89.767387911 0.138615548610687 54.0
89.767387911 0.075188435614109 55.0
0 0.678700804710388 56.0
72.164968 0.0016336040571332 57.0
213.83308258 0.88109302520752 58.0
240.7696351 0.384357839822769 59.0
20.594162486 0.00431889295578003 60.0
0 0.0779132694005966 61.0
145.63045174 0.87889039516449 0.0
145.63045174 2.00507497787476 1.0
144.18338385 1.90539574623108 2.0
139.39427859 5.70664834976196 3.0
0 4.01519823074341 4.0
0 3.43107795715332 5.0
-55.335900893 1.4044349193573 6.0
65.470707953 0.979190111160278 7.0
50.748275771 0.781384170055389 8.0
14.722432182 0.851176500320435 9.0
89.010658222 0.935429692268372 10.0
76.272383331 0.130007326602936 11.0
82.54396644 0.526082098484039 12.0
82.54396644 0.734342575073242 13.0
82.54396644 0.0242360550910234 14.0
0 0.130768865346909 15.0
0 0.241908177733421 16.0
0 0.855347573757172 17.0
60.544874714 1.30625545978546 18.0
-60.544874714 1.00834906101227 19.0
-60.544874714 1.09285092353821 20.0
-60.544874714 -0.0678591877222061 21.0
81.141355824 1.4028856754303 22.0
133.66776704 0.937908351421356 23.0
101.330306 0.601788103580475 24.0
19.904083592 0.0991556942462921 25.0
214.80912286 0.906598269939423 26.0
214.80912286 1.11853456497192 27.0
27.457802811 1.12107968330383 28.0
242.26692567 0.708053350448608 29.0
242.26692567 1.7743912935257 30.0
47.087052804 1.02811670303345 31.0
9.4139430172 0.616612136363983 32.0
20.59648111 1.50826370716095 33.0
20.59648111 0.648353278636932 34.0
20.59648111 0.120079316198826 35.0
8.9784041439 0.0327565632760525 36.0
7.6441778468 0.390313982963562 37.0
0 0.0205785185098648 38.0
0 0.18674436211586 39.0
-11.236946706 0.659311234951019 40.0
-23.863764875 0.518165707588196 41.0
5.2276979027 0.432069033384323 42.0
-31.427557712 0.579752326011658 43.0
88.885975483 0.643274188041687 44.0
88.885975483 0.0937589406967163 45.0
6.8741160333 0.532012939453125 46.0
2.3500880768 0.365675240755081 47.0
145.63045174 0.10157173871994 48.0
139.39427859 0.366679906845093 49.0
0 0.1286641061306 50.0
65.470707953 0.205892950296402 51.0
89.010658222 0.22278556227684 52.0
82.54396644 0.0576404146850109 53.0
82.54396644 0.142284765839577 54.0
82.54396644 0.143018364906311 55.0
0 1.65473306179047 56.0
81.141355824 0.153482288122177 57.0
214.80912286 0.547013878822327 58.0
242.26692567 0.690899312496185 59.0
20.59648111 0.0105139873921871 60.0
0 0.11228820681572 61.0
153.55045842 0.770257592201233 0.0
153.55045842 1.50225186347961 1.0
152.28475162 1.92538940906525 2.0
146.82008101 5.48481273651123 3.0
4.6050688072 5.76116895675659 4.0
4.6050688072 3.46634912490845 5.0
-61.718081966 0.0381054282188416 6.0
72.85196309 0.97274112701416 7.0
38.713647693 0.611671388149261 8.0
34.138315396 0.455772161483765 9.0
86.354690511 0.823955714702606 10.0
73.528203513 0.0686260461807251 11.0
85.781129159 0.729730248451233 12.0
85.781129159 0.642377734184265 13.0
85.781129159 0.0948000252246857 14.0
0 0.19205929338932 15.0
0 0.193545952439308 16.0
0 1.07091593742371 17.0
53.779591292 1.24416208267212 18.0
-53.779591292 1.27583479881287 19.0
-53.779591292 0.3856480717659 20.0
-53.779591292 -0.0738606303930283 21.0
73.435340759 2.87782549858093 22.0
141.45029133 2.65002608299255 23.0
109.2734842 0.399721533060074 24.0
20.606277189 0.102066248655319 25.0
214.88563209 1.76779437065125 26.0
214.88563209 1.86705195903778 27.0
29.631827564 2.32972931861877 28.0
244.51745965 0.942920804023743 29.0
244.51745965 1.68144464492798 30.0
46.44250145 0.962441802024841 31.0
9.3521107698 0.465549468994141 32.0
19.655749466 1.95811498165131 33.0
19.655749466 0.942514777183533 34.0
19.655749466 0.128372758626938 35.0
8.8683218456 0.0410927347838879 36.0
7.6048688255 0.395906537771225 37.0
0 0.0241393223404884 38.0
0 0.298092007637024 39.0
-12.976535059 0.214347437024117 40.0
-25.972352965 0.869778335094452 41.0
4.8666496451 0.353543996810913 42.0
-33.433022147 0.43656188249588 43.0
110.78287824 0.873773694038391 44.0
110.78287824 0.0579771995544434 45.0
7.0404209553 0.707940995693207 46.0
2.1954164657 0.383879542350769 47.0
153.55045842 0.176456421613693 48.0
146.82008101 0.482568770647049 49.0
4.6050688072 0.260030567646027 50.0
72.85196309 0.0835612937808037 51.0
86.354690511 0.143534228205681 52.0
85.781129159 0.0474706441164017 53.0
85.781129159 0.0824768990278244 54.0
85.781129159 0.0868940278887749 55.0
0 2.59734010696411 56.0
73.435340759 0.0868335664272308 57.0
214.88563209 0.666410803794861 58.0
244.51745965 0.453622817993164 59.0
19.655749466 -0.00299707055091858 60.0
0 0.0838049128651619 61.0
171.36176291 0.949825882911682 0.0
171.36176291 1.80779850482941 1.0
169.98041043 2.05388760566711 2.0
164.51573983 5.54489040374756 3.0
7.3955257637 5.71949577331543 4.0
7.3955257637 3.09809398651123 5.0
-66.755335276 0.702775001525879 6.0
76.322270073 0.663221180438995 7.0
53.613557781 1.11089730262756 8.0
22.708712292 1.24949502944946 9.0
94.904838812 1.00318145751953 10.0
83.590482865 0.422616064548492 11.0
83.585150207 0.1164790391922 12.0
83.585150207 0.14026015996933 13.0
83.585150207 0.0521436631679535 14.0
0 0.120764218270779 15.0
0 0.249628201127052 16.0
0 1.02518928050995 17.0
60.320211842 1.57776975631714 18.0
-60.320211842 1.66053664684296 19.0
-60.320211842 0.745934128761292 20.0
-60.320211842 -0.0692797154188156 21.0
81.044185647 2.86639499664307 22.0
136.84211586 1.76341784000397 23.0
104.68463298 0.322081953287125 24.0
22.380445016 0.270363956689835 25.0
217.88630151 1.44423818588257 26.0
217.88630151 1.83362174034119 27.0
26.783136107 2.06057214736938 28.0
244.66943762 1.28720414638519 29.0
244.66943762 2.13235068321228 30.0
46.071271952 0.971523880958557 31.0
9.2845993444 0.500184535980225 32.0
20.723973805 1.90508151054382 33.0
20.723973805 1.00770962238312 34.0
20.723973805 0.131999999284744 35.0
9.5722065779 0.0353196002542973 36.0
8.3327040526 0.389713525772095 37.0
0 0.000109519809484482 38.0
0 0.122287735342979 39.0
-13.148375027 0.703203022480011 40.0
-25.565111545 0.359380900859833 41.0
4.6994085794 0.443105846643448 42.0
-32.957228699 0.493931740522385 43.0
92.416047654 0.537776350975037 44.0
92.416047654 0.120404183864594 45.0
6.2352861677 0.554483711719513 46.0
2.3922674491 0.374717593193054 47.0
171.36176291 0.102841377258301 48.0
164.51573983 0.217705190181732 49.0
7.3955257637 0.2189661860466 50.0
76.322270073 0.155478566884995 51.0
94.904838812 0.329329997301102 52.0
83.585150207 0.0546330735087395 53.0
83.585150207 0.0788227990269661 54.0
83.585150207 0.0184897966682911 55.0
0 3.31878542900085 56.0
81.044185647 0.143167659640312 57.0
217.88630151 0.309876561164856 58.0
244.66943762 1.38218653202057 59.0
20.723973805 -0.0016088429838419 60.0
0 0.0905134379863739 61.0
145.93491707 0.772922277450562 0.0
145.93491707 1.11898982524872 1.0
144.43767542 1.79262959957123 2.0
138.97300473 5.46912336349487 3.0
0.48870892336 5.95657825469971 4.0
0.48870892336 3.23170495033264 5.0
-60.782561835 0.158021330833435 6.0
73.072393264 0.464204430580139 7.0
41.820536257 0.642321288585663 8.0
31.251857007 0.752036690711975 9.0
82.87446039 0.908468544483185 10.0
69.641232239 0.261772155761719 11.0
75.541817955 0.447130799293518 12.0
75.541817955 0.243315726518631 13.0
75.541817955 0.0426960662007332 14.0
0 0.0112550128251314 15.0
0 0.198016241192818 16.0
0 1.17389297485352 17.0
60.160620102 1.88595688343048 18.0
-60.160620102 1.66371047496796 19.0
-60.160620102 0.476720929145813 20.0
-60.160620102 0.0111430063843727 21.0
80.517194373 1.65107977390289 22.0
148.18801156 1.69030976295471 23.0
114.12159154 0.216345980763435 24.0
21.436365646 0.223705664277077 25.0
228.70520593 1.3016300201416 26.0
228.70520593 1.66840279102325 27.0
23.101200332 1.11760950088501 28.0
251.80640626 0.691373586654663 29.0
251.80640626 1.53163325786591 30.0
46.212954044 1.12975120544434 31.0
9.3591867879 0.367830961942673 32.0
20.356574271 1.40037202835083 33.0
20.356574271 0.984716415405273 34.0
20.356574271 0.136616542935371 35.0
9.3803766622 0.0322964787483215 36.0
8.1039013695 0.365212082862854 37.0
0 0.0255139209330082 38.0
0 0.115258261561394 39.0
-12.026789146 0.76848578453064 40.0
-24.175951718 0.454899966716766 41.0
5.4862488187 0.474533945322037 42.0
-32.360310704 0.566989839076996 43.0
97.511480711 0.829735219478607 44.0
97.511480711 0.129544600844383 45.0
7.0876288728 0.702521622180939 46.0
2.615463851 0.355321228504181 47.0
145.93491707 0.137544870376587 48.0
138.97300473 0.374240130186081 49.0
0.48870892336 0.301377594470978 50.0
73.072393264 0.093096561729908 51.0
82.87446039 0.326243668794632 52.0
75.541817955 0.0863900631666183 53.0
75.541817955 0.0698806121945381 54.0
75.541817955 0.0838688835501671 55.0
0 5.35418081283569 56.0
80.517194373 0.0235322844237089 57.0
228.70520593 0.6247199177742 58.0
251.80640626 0.282505869865417 59.0
20.356574271 -0.00442670844495296 60.0
0 0.0921169742941856 61.0
202.8617598 0.840631246566772 0.0
202.8617598 1.76511096954346 1.0
201.55383571 2.37857866287231 2.0
196.12906452 4.31848001480103 3.0
0 3.55338430404663 4.0
0 4.30900716781616 5.0
-58.229448299 0.0441490709781647 6.0
71.201587319 0.664407253265381 7.0
40.531977153 0.443088114261627 8.0
30.669610166 0.551385819911957 9.0
91.314013117 0.995161175727844 10.0
78.634493856 0.211671739816666 11.0
83.447234374 0.868387222290039 12.0
83.447234374 0.127758145332336 13.0
83.447234374 0.102909669280052 14.0
0 0.248806744813919 15.0
0 0.122553125023842 16.0
0 0.813463389873505 17.0
65.372653242 1.32812559604645 18.0
-65.372653242 1.88493585586548 19.0
-66.012789375 0.567444205284119 20.0
-66.079833026 -0.0971294194459915 21.0
87.06709432 0.827944219112396 22.0
137.03958398 0.496172964572906 23.0
103.94798948 0.924437761306763 24.0
22.632214454 0.0792568773031235 25.0
224.1066783 0.532581746578217 26.0
224.1066783 1.73888552188873 27.0
29.134464093 2.25552248954773 28.0
253.24114239 1.57042765617371 29.0
253.24114239 2.01801705360413 30.0
46.558565446 1.17060804367065 31.0
9.2673644253 0.332837492227554 32.0
20.987261294 2.54324960708618 33.0
20.987261294 0.652851283550262 34.0
20.987261294 0.125618442893028 35.0
10.554701116 0.0399029664695263 36.0
9.2351221853 0.330486953258514 37.0
0 0.0409379452466965 38.0
0 0.182282760739326 39.0
-13.063060733 0.638627946376801 40.0
-26.5055602 0.569870948791504 41.0
5.5628964751 0.372686922550201 42.0
-34.438053101 0.733609914779663 43.0
93.451168392 0.51640373468399 44.0
93.451168392 0.159684747457504 45.0
6.7566724227 0.361770272254944 46.0
2.4443123971 0.346055388450623 47.0
202.8617598 0.140852957963943 48.0
196.12906452 0.603736162185669 49.0
0 0.0865815877914429 50.0
71.201587319 0.0113522317260504 51.0
91.314013117 0.172368079423904 52.0
83.447234374 0.121998526155949 53.0
83.447234374 0.0450329594314098 54.0
83.447234374 0.0866779834032059 55.0
0 4.13309335708618 56.0
87.06709432 0.290031254291534 57.0
224.1066783 0.0628402903676033 58.0
253.24114239 1.39540719985962 59.0
20.987261294 -0.00490336492657661 60.0
0 0.0999149829149246 61.0
194.70589143 0.975777387619019 0.0
194.70589143 1.43472766876221 1.0
193.28109888 1.95960032939911 2.0
188.4919936 6.45529651641846 3.0
0 8.20084571838379 4.0
0 3.66272163391113 5.0
-53.078886716 2.199866771698 6.0
62.820341858 1.00618064403534 7.0
62.850313513 0.897729992866516 8.0
-0.029971655268 0.898489534854889 9.0
94.662601215 1.14237475395203 10.0
82.477984186 0.475814431905746 11.0
78.54762665 0.818253993988037 12.0
78.54762665 0.282105684280396 13.0
78.54762665 0.0455398932099342 14.0
0 0.0716715306043625 15.0
0 0.202164694666862 16.0
0 1.3746258020401 17.0
59.239811107 0.884068071842194 18.0
-59.239811107 0.764506340026855 19.0
-59.239811107 0.427639305591583 20.0
-59.239811107 0.0221477076411247 21.0
80.61189373 1.32057058811188 22.0
138.57504636 0.612725853919983 23.0
103.30086977 0.778198659420013 24.0
21.138180746 0.075304701924324 25.0
219.18694009 1.19551777839661 26.0
219.18694009 1.57136476039886 27.0
23.306221306 2.07308435440063 28.0
242.4931614 0.773555397987366 29.0
242.4931614 1.63341248035431 30.0
46.321967244 1.04595875740051 31.0
9.227429371 0.423209011554718 32.0
21.372082623 1.23597121238708 33.0
21.372082623 0.892622053623199 34.0
21.372082623 0.117878489196301 35.0
8.9029337144 0.0376665629446507 36.0
7.5728779912 0.36398646235466 37.0
0 0.0805853456258774 38.0
0 0.263657361268997 39.0
-11.870335346 0.53023898601532 40.0
-24.768737066 0.628011763095856 41.0
4.927804963 0.440926313400269 42.0
-32.377409611 0.474809169769287 43.0
84.33636962 0.561242938041687 44.0
84.33636962 0.0986426174640656 45.0
6.2777454013 0.547538697719574 46.0
2.5957993119 0.339961260557175 47.0
194.70589143 0.0778201818466187 48.0
188.4919936 0.168013393878937 49.0
0 0.288989186286926 50.0
62.820341858 0.145276740193367 51.0
94.662601215 0.150286197662354 52.0
78.54762665 0.178142964839935 53.0
78.54762665 0.0886497050523758 54.0
78.54762665 0.193262562155724 55.0
0 3.40204071998596 56.0
80.61189373 0.0581652671098709 57.0
219.18694009 0.367735207080841 58.0
242.4931614 0.345400542020798 59.0
21.372082623 0.0137336403131485 60.0
0 0.0690833777189255 61.0
168.37826735 1.11150121688843 0.0
168.37826735 0.880989015102386 1.0
166.98157336 1.23021292686462 2.0
162.19246825 5.11236810684204 3.0
0 6.38046360015869 4.0
0 4.25672006607056 5.0
-54.410090915 0.806340575218201 6.0
66.413262595 0.521113634109497 7.0
61.743926172 1.24805104732513 8.0
4.6693364234 1.29861915111542 9.0
80.4821673 1.22458255290985 10.0
66.723296004 0.0947830900549889 11.0
81.511423958 0.786818206310272 12.0
81.511423958 0.197118699550629 13.0
81.511423958 0.07281144708395 14.0
0 0.187444880604744 15.0
0 0.192094013094902 16.0
0 1.1828556060791 17.0
61.653111095 0.923171043395996 18.0
-61.653111095 0.767880797386169 19.0
-61.653111095 0.757550716400146 20.0
-61.653111095 0.0708504095673561 21.0
82.53888248 1.34257972240448 22.0
155.48405631 0.902428448200226 23.0
117.92286975 0.406611770391464 24.0
23.686814743 0.19014835357666 25.0
238.02293879 0.896946370601654 26.0
238.02293879 1.44760715961456 27.0
26.728952941 1.47104465961456 28.0
264.75189173 0.761172533035278 29.0
264.75189173 1.83054840564728 30.0
47.132609675 1.29755580425262 31.0
9.3666569881 0.437535673379898 32.0
20.885771385 0.912011444568634 33.0
20.885771385 0.787425577640533 34.0
20.885771385 0.131163522601128 35.0
9.6741322311 0.0378450304269791 36.0
8.4850314501 0.396779537200928 37.0
0 0.0364936515688896 38.0
0 0.222413897514343 39.0
-11.511296741 0.480968236923218 40.0
-24.422143329 0.531011700630188 41.0
5.4542230306 0.397134244441986 42.0
-32.622370056 0.419211596250534 43.0
96.293017635 0.619608163833618 44.0
96.293017635 0.0689591616392136 45.0
7.8753693537 0.639081418514252 46.0
2.8026975105 0.385272860527039 47.0
168.37826735 0.149966642260551 48.0
162.19246825 0.258838891983032 49.0
0 0.174054101109505 50.0
66.413262595 0.110131710767746 51.0
80.4821673 0.0997714400291443 52.0
81.511423958 0.0608494728803635 53.0
81.511423958 0.0826139748096466 54.0
81.511423958 0.054335854947567 55.0
0 2.396235704422 56.0
82.53888248 -0.0125951338559389 57.0
238.02293879 0.58546394109726 58.0
264.75189173 0.731655836105347 59.0
20.885771385 0.0151343420147896 60.0
0 0.0891993492841721 61.0
125.42766907 0.93813556432724 0.0
125.42766907 1.98895239830017 1.0
124.39678924 2.12733268737793 2.0
119.60768396 2.77954149246216 3.0
0 3.04924559593201 4.0
0 4.28286552429199 5.0
-79.856176823 4.16277742385864 6.0
89.530360254 0.623264849185944 7.0
64.297718093 0.526736259460449 8.0
25.232642161 0.735124230384827 9.0
90.383993763 1.0143449306488 10.0
77.690913108 0.0434154644608498 11.0
86.950328207 0.117492713034153 12.0
86.950328207 0.587132513523102 13.0
86.950328207 0.105159245431423 14.0
0 0.210002169013023 15.0
0 0.213388830423355 16.0
0 1.27430689334869 17.0
48.413660037 0.745489418506622 18.0
-48.413660037 0.5958012342453 19.0
-48.413660037 1.00284707546234 20.0
-48.413660037 0.256037414073944 21.0
70.533960414 1.71522402763367 22.0
154.01222217 2.14311122894287 23.0
117.08138586 0.350483179092407 24.0
24.016232552 0.264555156230927 25.0
224.54618259 1.18476462364197 26.0
224.54618259 2.02413773536682 27.0
22.468314242 1.36858737468719 28.0
247.01449683 1.0704950094223 29.0
247.01449683 1.92698669433594 30.0
47.085582303 1.15493893623352 31.0
9.1982906443 0.346298366785049 32.0
22.120300377 1.19834136962891 33.0
22.120300377 0.738168478012085 34.0
22.120300377 0.0634785890579224 35.0
9.189911122 0.0281310081481934 36.0
7.9471277415 0.402934908866882 37.0
0 0.000592954456806183 38.0
0 0.243850380182266 39.0
-11.93022657 0.762812614440918 40.0
-23.565673273 0.315937519073486 41.0
4.6672606153 0.480265617370605 42.0
-30.874797586 0.262772738933563 43.0
84.368700701 0.791217148303986 44.0
84.368700701 0.0498987212777138 45.0
7.117072097 0.711761355400085 46.0
2.0280408468 0.45911493897438 47.0
125.42766907 0.161085426807404 48.0
119.60768396 0.354207962751389 49.0
0 0.334786355495453 50.0
89.530360254 0.197118371725082 51.0
90.383993763 0.100893422961235 52.0
86.950328207 0.0360320210456848 53.0
86.950328207 0.0355536490678787 54.0
86.950328207 0.0661732628941536 55.0
0 3.48283576965332 56.0
70.533960414 -0.054975725710392 57.0
224.54618259 0.843480050563812 58.0
247.01449683 0.991428017616272 59.0
22.120300377 0.0138512328267097 60.0
0 0.0983565598726273 61.0
192.22629773 0.967084646224976 0.0
192.22629773 0.730340003967285 1.0
190.92234231 0.845138132572174 2.0
186.13323704 3.96142864227295 3.0
0 5.4308762550354 4.0
0 4.21792554855347 5.0
-54.989498687 0.309219837188721 6.0
65.484144445 0.698689520359039 7.0
50.960589608 0.589082956314087 8.0
14.523554838 0.759423136711121 9.0
89.295811666 1.03238713741302 10.0
75.699223914 0.271346926689148 11.0
78.663523278 0.261154472827911 12.0
78.663523278 0.137764513492584 13.0
78.663523278 0.0479591228067875 14.0
0 -0.000453675165772438 15.0
0 0.20523039996624 16.0
0 1.14432168006897 17.0
55.649336253 1.7432541847229 18.0
-55.649336253 2.22700428962708 19.0
-55.649336253 0.257085978984833 20.0
-55.649336253 -0.0348028242588043 21.0
76.719348531 1.80891501903534 22.0
144.42469548 2.06247448921204 23.0
110.30733639 0.320987731218338 24.0
23.749525759 0.113586008548737 25.0
221.14404401 1.00620424747467 26.0
221.14404401 1.84469866752625 27.0
25.863229067 2.19187164306641 28.0
247.00727307 0.667362689971924 29.0
247.00727307 1.74708390235901 30.0
46.280095898 1.05664002895355 31.0
9.3657219556 0.562552511692047 32.0
21.070012278 1.69897139072418 33.0
21.070012278 1.03507661819458 34.0
21.070012278 0.134599506855011 35.0
10.083938175 0.044509157538414 36.0
8.8979293915 0.397281408309937 37.0
0 0.0397430546581745 38.0
0 0.196784734725952 39.0
-11.101445852 0.490324944257736 40.0
-22.587046189 0.550890624523163 41.0
5.1406184202 0.611266613006592 42.0
-30.248398666 0.776995658874512 43.0
92.288504723 0.617596447467804 44.0
92.288504723 0.126130640506744 45.0
7.021099603 0.459084302186966 46.0
2.4508153777 0.365488827228546 47.0
192.22629773 0.142653062939644 48.0
186.13323704 0.17224745452404 49.0
0 0.12406500428915 50.0
65.484144445 0.19583123922348 51.0
89.295811666 0.11474285274744 52.0
78.663523278 0.0921462699770927 53.0
78.663523278 0.103156521916389 54.0
78.663523278 0.0931096598505974 55.0
0 4.05776405334473 56.0
76.719348531 0.0459492392838001 57.0
221.14404401 0.641456425189972 58.0
247.00727307 0.308745265007019 59.0
21.070012278 0.00853167474269867 60.0
0 0.0924383848905563 61.0
164.34510867 1.00592565536499 0.0
164.34510867 1.24349343776703 1.0
163.36266976 1.16943311691284 2.0
158.57359991 4.92084646224976 3.0
0 8.66887950897217 4.0
0 5.42754220962524 5.0
-62.972209986 6.76012802124023 6.0
72.848206311 0.542881369590759 7.0
49.217048314 0.822805941104889 8.0
23.631157998 1.00160503387451 9.0
77.199761672 1.02640974521637 10.0
64.816419114 0.0833679810166359 11.0
79.442141596 0.917498648166656 12.0
79.442141596 0.601197183132172 13.0
79.442141596 0.138436362147331 14.0
0 0.191822931170464 15.0
0 0.138041526079178 16.0
0 1.76280069351196 17.0
55.225848415 0.7780482172966 18.0
-55.225848415 0.432688176631927 19.0
-55.225848415 0.975271284580231 20.0
-55.225848415 -0.0787865370512009 21.0
77.005026896 1.3776388168335 22.0
139.37011754 0.827469289302826 23.0
106.59731122 0.741293668746948 24.0
24.521995815 0.072954848408699 25.0
216.37514443 0.482945442199707 26.0
216.37514443 3.01287078857422 27.0
27.342508559 3.03820323944092 28.0
243.71765299 1.14470624923706 29.0
243.71765299 2.29283952713013 30.0
46.155963076 1.38069689273834 31.0
9.3598629315 0.564749658107758 32.0
21.779178481 2.00499987602234 33.0
21.779178481 1.07558441162109 34.0
21.779178481 0.135686054825783 35.0
9.7150866509 0.0245793871581554 36.0
8.3797289079 0.382425636053085 37.0
0 0.00267563760280609 38.0
0 0.248064652085304 39.0
-12.815946512 0.487266063690186 40.0
-26.935478652 0.577264845371246 41.0
4.8737559209 0.364329129457474 42.0
-34.203922136 0.448437213897705 43.0
105.5610177 0.724206030368805 44.0
105.5610177 0.0726096928119659 45.0
6.4456936365 0.569454252719879 46.0
2.0788281625 0.415615946054459 47.0
164.34510867 0.0885806232690811 48.0
158.57359991 0.0106275118887424 49.0
0 0.597091376781464 50.0
72.848206311 0.232320681214333 51.0
77.199761672 0.105190187692642 52.0
79.442141596 0.0846416875720024 53.0
79.442141596 0.0242943894118071 54.0
79.442141596 0.0444793552160263 55.0
0 3.85197043418884 56.0
77.005026896 0.123526751995087 57.0
216.37514443 0.267204314470291 58.0
243.71765299 0.542314529418945 59.0
21.779178481 -0.00955352745950222 60.0
0 0.0615851394832134 61.0
196.57671863 0.892847836017609 0.0
196.57671863 1.60336947441101 1.0
195.6023891 1.28561663627625 2.0
190.13771852 4.83921384811401 3.0
9.2495796314 5.56974124908447 4.0
9.2495796314 4.71616363525391 5.0
-50.047142868 4.91520309448242 6.0
62.917820942 0.646396994590759 7.0
40.013294119 0.429958879947662 8.0
22.904526823 0.52035391330719 9.0
72.670132239 0.959509909152985 10.0
59.879446563 0.0818854719400406 11.0
72.512049015 0.126211553812027 12.0
72.512049015 0.546026468276978 13.0
72.512049015 0.131240516901016 14.0
0 0.0280755460262299 15.0
0 0.192424267530441 16.0
0 1.02673816680908 17.0
60.145324124 1.5152724981308 18.0
-60.145324124 1.53742504119873 19.0
-60.145324124 0.556606709957123 20.0
-60.145324124 0.241973578929901 21.0
80.610042204 1.99473798274994 22.0
144.03900529 2.5230655670166 23.0
107.2366449 0.321926295757294 24.0
25.333598324 0.142842352390289 25.0
224.6490475 1.75571668148041 26.0
224.6490475 2.25833606719971 27.0
22.563259896 2.31841659545898 28.0
247.21230739 1.06429100036621 29.0
247.21230739 1.89043009281158 30.0
46.135670843 1.18054592609406 31.0
9.2609772059 0.548298597335815 32.0
20.46471808 0.773864984512329 33.0
20.46471808 1.33966767787933 34.0
20.46471808 0.0897915959358215 35.0
8.9179854107 0.0206234138458967 36.0
7.4985569983 0.417414546012878 37.0
0 0.0108694434165955 38.0
0 0.178230464458466 39.0
-12.407806554 0.526142537593842 40.0
-24.448060041 0.573209941387177 41.0
5.4232968293 0.663031935691833 42.0
-32.627274861 0.733677744865417 43.0
92.361340373 0.888447880744934 44.0
92.361340373 0.0911613181233406 45.0
5.886669356 0.627501487731934 46.0
2.1653905403 0.402701586484909 47.0
196.57671863 0.106331087648869 48.0
190.13771852 0.0349182896316051 49.0
9.2495796314 1.16518604755402 50.0
62.917820942 0.221168756484985 51.0
72.670132239 0.148945569992065 52.0
72.512049015 0.0249929167330265 53.0
72.512049015 0.138253182172775 54.0
72.512049015 0.160936146974564 55.0
0 8.03946685791016 56.0
80.610042204 0.0559924244880676 57.0
224.6490475 0.507669806480408 58.0
247.21230739 0.547502219676971 59.0
20.46471808 -0.0268010422587395 60.0
0 0.10111577808857 61.0
130.74071009 1.01705491542816 0.0
130.74071009 1.52336847782135 1.0
129.37561548 1.82193565368652 2.0
124.58651023 4.49504232406616 3.0
0 3.52192664146423 4.0
0 2.79140686988831 5.0
-49.984150408 0.0697344392538071 6.0
61.730606279 0.833032011985779 7.0
55.125195199 1.01121020317078 8.0
6.6054110798 0.664156079292297 9.0
88.298095145 1.13012957572937 10.0
73.557583761 0.128324806690216 11.0
82.674702045 1.34476923942566 12.0
82.674702045 0.805499315261841 13.0
82.674702045 0.0384588241577148 14.0
0 0.0800521150231361 15.0
0 0.168397188186646 16.0
0 1.1378390789032 17.0
55.815306439 1.07779026031494 18.0
-55.815306439 1.39612758159637 19.0
-55.815306439 0.492416024208069 20.0
-55.815306439 0.00711869075894356 21.0
75.421740384 1.34512293338776 22.0
149.05379112 1.59606957435608 23.0
113.80482906 0.337967574596405 24.0
24.271686513 0.0822184979915619 25.0
224.47553151 0.937343180179596 26.0
224.47553151 2.78028631210327 27.0
29.301067093 2.01188063621521 28.0
253.7765986 0.558478474617004 29.0
253.7765986 1.77531337738037 30.0
46.995286378 0.967473208904266 31.0
9.423982441 0.374279767274857 32.0
19.606433945 1.60398077964783 33.0
19.606433945 1.36966216564178 34.0
19.606433945 0.132997497916222 35.0
9.17121406 0.0267916116863489 36.0
7.9142873228 0.34640696644783 37.0
0 0.0197182409465313 38.0
0 0.309297740459442 39.0
-11.777626411 0.357050359249115 40.0
-24.001921066 0.842903792858124 41.0
5.7216970733 0.425484210252762 42.0
-32.224912851 0.641473293304443 43.0
92.436240463 0.786496698856354 44.0
92.436240463 0.0492858216166496 45.0
6.7731602796 0.479510515928268 46.0
2.4940711505 0.313884675502777 47.0
130.74071009 0.130032926797867 48.0
124.58651023 0.267086058855057 49.0
0 0.31519028544426 50.0
61.730606279 0.302241891622543 51.0
88.298095145 0.348017126321793 52.0
82.674702045 0.130868434906006 53.0
82.674702045 0.334155857563019 54.0
82.674702045 0.262313425540924 55.0
0 3.12714171409607 56.0
75.421740384 0.0582211278378963 57.0
224.47553151 0.96523904800415 58.0
253.7765986 0.599932074546814 59.0
19.606433945 -0.0106723476201296 60.0
0 0.104461424052715 61.0
179.99515272 1.00052809715271 0.0
179.99515272 1.71390616893768 1.0
178.61783854 1.69470167160034 2.0
173.82873332 5.39684295654297 3.0
0 8.46319675445557 4.0
0 5.16427278518677 5.0
-70.676334266 0.556678414344788 6.0
83.282500912 0.918359398841858 7.0
49.374058982 0.832898557186127 8.0
33.908441931 0.571453630924225 9.0
89.03158331 1.08790123462677 10.0
74.890940112 0.237859189510345 11.0
88.340735298 0.457869470119476 12.0
88.340735298 0.123581141233444 13.0
88.340735298 0.0983517989516258 14.0
0 0.281215101480484 15.0
0 0.145906686782837 16.0
0 1.35332858562469 17.0
54.676502313 0.825128793716431 18.0
-54.676502313 0.957336127758026 19.0
-54.676502313 0.552456498146057 20.0
-54.676502313 0.0026775635778904 21.0
77.661346531 2.25352430343628 22.0
144.70833545 2.18764162063599 23.0
109.16523995 0.449102073907852 24.0
23.173817115 0.164032742381096 25.0
222.36968198 1.12810385227203 26.0
222.36968198 1.36972761154175 27.0
29.489615479 1.33193004131317 28.0
251.85929746 0.939304113388062 29.0
251.85929746 1.65568578243256 30.0
46.66804689 0.949821829795837 31.0
9.437293849 0.53939026594162 32.0
22.984844218 1.34886515140533 33.0
22.984844218 1.01168167591095 34.0
22.984844218 0.0922533273696899 35.0
9.9994582673 0.0347399041056633 36.0
8.7460664855 0.389272302389145 37.0
0 0.0345074646174908 38.0
0 0.236302092671394 39.0
-12.448199426 0.460388988256454 40.0
-25.439991836 0.766949594020844 41.0
5.2080740136 0.386039704084396 42.0
-33.057450644 0.574468731880188 43.0
98.121435667 0.673100531101227 44.0
98.121435667 0.142970532178879 45.0
7.187621862 0.5688117146492 46.0
2.645982 0.352189779281616 47.0
179.99515272 0.119577869772911 48.0
173.82873332 0.0535852164030075 49.0
0 0.23726661503315 50.0
83.282500912 0.260064214468002 51.0
89.03158331 0.235520616173744 52.0
88.340735298 0.0548679754137993 53.0
88.340735298 0.119100742042065 54.0
88.340735298 0.0757423639297485 55.0
0 1.35792803764343 56.0
77.661346531 0.0107650030404329 57.0
222.36968198 0.371388882398605 58.0
251.85929746 1.04040908813477 59.0
22.984844218 -0.0244759656488895 60.0
0 0.0902929231524467 61.0
182.80937596 0.609510719776154 0.0
182.80937596 1.18465161323547 1.0
181.74126444 1.12140870094299 2.0
176.27662903 4.4084210395813 3.0
29.824772724 5.2455005645752 4.0
29.824772724 6.61607027053833 5.0
-50.25727714 9.19406986236572 6.0
59.685287643 0.624188661575317 7.0
38.967845163 1.00106751918793 8.0
20.717442479 1.19841396808624 9.0
86.290027461 0.873374223709106 10.0
73.336008955 0.256694376468658 11.0
74.75833698 0.990254759788513 12.0
74.75833698 0.121884413063526 13.0
74.75833698 0.0570461675524712 14.0
14.00034613 0.13077138364315 15.0
14.00034613 0.163744404911995 16.0
0 0.442757338285446 17.0
52.49667412 1.36196458339691 18.0
-52.49667412 2.20493841171265 19.0
-52.49667412 0.367405354976654 20.0
-52.49667412 0.131019070744514 21.0
75.443744687 1.02606737613678 22.0
155.71139681 0.533615350723267 23.0
118.12281235 1.07087457180023 24.0
24.596813256 0.0980945006012917 25.0
231.1551415 0.921890377998352 26.0
231.1551415 1.76550531387329 27.0
28.016258452 1.77516508102417 28.0
259.17139995 0.636431515216827 29.0
259.17139995 1.29763734340668 30.0
46.258322034 0.948848724365234 31.0
9.1968643281 0.41998615860939 32.0
22.947070567 0.740844666957855 33.0
22.947070567 0.963202476501465 34.0
22.947070567 0.155538439750671 35.0
10.033828534 0.0356999151408672 36.0
8.7068719767 0.376247614622116 37.0
0 0.0704757422208786 38.0
0 0.100590862333775 39.0
-12.854167958 1.19223618507385 40.0
-24.950016543 0.40526869893074 41.0
5.0907268007 0.430774807929993 42.0
-32.687148269 0.701207280158997 43.0
87.799077366 0.632119357585907 44.0
87.799077366 0.0609563291072845 45.0
7.7227907445 0.352308243513107 46.0
2.6694979595 0.273298233747482 47.0
182.80937596 0.0847866535186768 48.0
176.27662903 0.171025902032852 49.0
29.824772724 0.318632066249847 50.0
59.685287643 0.269510000944138 51.0
86.290027461 0.450557321310043 52.0
74.75833698 0.0956164002418518 53.0
74.75833698 0.139810487627983 54.0
74.75833698 0.0571188777685165 55.0
0 4.41232681274414 56.0
75.443744687 0.172045677900314 57.0
231.1551415 0.376037180423737 58.0
259.17139995 0.227239027619362 59.0
22.947070567 0.0139087401330471 60.0
0 0.069853663444519 61.0
165.38499715 0.94471150636673 0.0
165.38499715 0.558611214160919 1.0
164.32146129 0.663881897926331 2.0
159.53235605 3.74534678459167 3.0
0 4.1371021270752 4.0
0 8.24169826507568 5.0
-80.558319421 6.13104915618896 6.0
91.883181989 1.37092125415802 7.0
43.230296273 0.795637428760529 8.0
48.652885716 0.435056567192078 9.0
100.60082367 0.941127061843872 10.0
87.845612867 0.0783291682600975 11.0
83.538726741 0.0814753025770187 12.0
83.538726741 0.092396579682827 13.0
83.538726741 0.105060108006001 14.0
0 0.247456029057503 15.0
0 0.138828754425049 16.0
0 1.20282125473022 17.0
58.918752002 1.78691458702087 18.0
-58.918752002 1.24436450004578 19.0
-58.918752002 0.992327928543091 20.0
-58.918752002 0.00866369903087616 21.0
79.061034271 0.85090696811676 22.0
134.87640848 0.376560777425766 23.0
99.980987566 1.05820679664612 24.0
14.316981705 0.0892569273710251 25.0
213.93744275 0.527664661407471 26.0
213.93744275 1.53076148033142 27.0
30.980122857 1.09268033504486 28.0
244.9175656 0.942925214767456 29.0
244.9175656 1.95383620262146 30.0
47.147667286 1.25360596179962 31.0
9.3503284528 0.808263182640076 32.0
20.142282269 1.7794041633606 33.0
20.142282269 0.95632666349411 34.0
20.142282269 0.121971555054188 35.0
9.1868984032 0.0566526614129543 36.0
8.022207613 0.386587291955948 37.0
0 0.125447750091553 38.0
0 0.0647228062152863 39.0
-12.380919405 0.681637704372406 40.0
-23.595146132 0.349277824163437 41.0
5.3130625813 0.450217425823212 42.0
-31.097744326 0.279143363237381 43.0
100.64624847 1.05689322948456 44.0
100.64624847 0.114427410066128 45.0
7.4304478273 0.880981922149658 46.0
2.8010259327 0.387589991092682 47.0
165.38499715 0.237999781966209 48.0
159.53235605 0.0316599160432816 49.0
0 1.39015066623688 50.0
91.883181989 0.270737826824188 51.0
100.60082367 0.236575931310654 52.0
83.538726741 0.000943106482736766 53.0
83.538726741 0.101794987916946 54.0
83.538726741 0.0561216846108437 55.0
0 4.67141723632812 56.0
79.061034271 0.0823658481240273 57.0
213.93744275 0.399921774864197 58.0
244.9175656 0.353168070316315 59.0
20.142282269 0.00735059008002281 60.0
0 0.0883940309286118 61.0
142.59657859 0.983258068561554 0.0
142.59657859 1.89939475059509 1.0
141.22585219 1.96737313270569 2.0
135.76118163 4.71084594726562 3.0
4.5597975958 6.61896896362305 4.0
4.5597975958 4.53057050704956 5.0
-65.822106806 0.420847952365875 6.0
77.792793944 1.03270626068115 7.0
44.590595393 1.10440194606781 8.0
33.202198552 1.15921723842621 9.0
104.7294439 1.0064115524292 10.0
90.596703064 0.35434490442276 11.0
87.480146404 0.782663106918335 12.0
87.480146404 0.117195293307304 13.0
87.480146404 0.120338551700115 14.0
0 0.263152420520782 15.0
0 0.151473671197891 16.0
0 1.08065295219421 17.0
61.714019068 1.36708903312683 18.0
-61.714019068 0.962338984012604 19.0
-61.714019068 0.959911525249481 20.0
-61.714019068 0.00315967574715614 21.0
85.102803945 1.59311175346375 22.0
137.84930583 1.26226198673248 23.0
102.09292327 0.59360545873642 24.0
24.263665902 0.181161448359489 25.0
222.95210978 1.16017150878906 26.0
222.95210978 1.95536410808563 27.0
26.163401009 2.33133172988892 28.0
249.11551078 1.03771603107452 29.0
249.11551078 1.87513840198517 30.0
46.957696986 1.07445132732391 31.0
9.3006123819 0.500707030296326 32.0
23.388784877 1.53674077987671 33.0
23.388784877 0.900163650512695 34.0
23.388784877 0.142419010400772 35.0
10.034541882 0.0251729618757963 36.0
8.7886597303 0.387120127677917 37.0
0 0.0224710516631603 38.0
0 0.135781303048134 39.0
-12.663577758 0.690883219242096 40.0
-26.874307406 0.639244258403778 41.0
5.4130742771 0.453570365905762 42.0
-34.997354065 0.686845064163208 43.0
99.252619205 0.696031153202057 44.0
99.252619205 0.0737908482551575 45.0
7.2029756267 0.53766530752182 46.0
2.5373114524 0.33562108874321 47.0
142.59657859 0.0928454324603081 48.0
135.76118163 0.249984338879585 49.0
4.5597975958 0.420739024877548 50.0
77.792793944 0.298825323581696 51.0
104.7294439 0.125644505023956 52.0
87.480146404 0.179908215999603 53.0
87.480146404 0.0409845858812332 54.0
87.480146404 0.0522404387593269 55.0
0 2.50611400604248 56.0
85.102803945 0.0558619871735573 57.0
222.95210978 0.407609432935715 58.0
249.11551078 0.348960816860199 59.0
23.388784877 -0.0128794778138399 60.0
0 0.105332389473915 61.0
154.10558925 0.928677797317505 0.0
154.10558925 0.893018126487732 1.0
153.13440809 1.16307806968689 2.0
147.66973733 4.36632490158081 3.0
18.382430665 4.15229749679565 4.0
18.382430665 4.76418876647949 5.0
-58.279661694 0.98316502571106 6.0
68.013699465 0.307903826236725 7.0
47.879940374 0.726346373558044 8.0
20.13375909 1.02180850505829 9.0
102.86412084 1.09940707683563 10.0
91.965566782 0.147116437554359 11.0
87.502748121 0.429559201002121 12.0
87.502748121 0.519905209541321 13.0
87.502748121 0.0203431844711304 14.0
0 0.109834149479866 15.0
0 0.227282822132111 16.0
0 0.694924354553223 17.0
51.362129419 0.832212328910828 18.0
-51.362129419 0.787867367267609 19.0
-51.362129419 0.246989667415619 20.0
-51.362129419 0.267408549785614 21.0
72.633032796 0.850069165229797 22.0
154.37352403 0.793894112110138 23.0
119.03698371 0.661462306976318 24.0
24.806614217 0.0918300300836563 25.0
227.00655682 1.18007731437683 26.0
227.00655682 1.75118911266327 27.0
25.796871563 2.23316025733948 28.0
252.80342838 0.833690106868744 29.0
252.80342838 1.76037383079529 30.0
46.221081543 1.23365581035614 31.0
9.3115869005 0.472191661596298 32.0
21.270903378 0.984001755714417 33.0
21.270903378 1.3325023651123 34.0
21.270903378 0.09714575111866 35.0
9.8444476551 0.0586517490446568 36.0
8.4935620742 0.4075767993927 37.0
0 0.0838688910007477 38.0
0 0.153616964817047 39.0
-12.081556091 0.985654234886169 40.0
-23.617156739 0.435654044151306 41.0
5.2730851833 0.814154446125031 42.0
-31.636862052 1.06879711151123 43.0
77.509021591 0.722195982933044 44.0
77.509021591 0.103841356933117 45.0
6.3152127919 0.626164138317108 46.0
2.7670811189 0.38183656334877 47.0
154.10558925 0.0691787451505661 48.0
147.66973733 0.658381223678589 49.0
18.382430665 0.1488918364048 50.0
68.013699465 0.144091621041298 51.0
102.86412084 0.239682480692863 52.0
87.502748121 0.0624641329050064 53.0
87.502748121 0.167714864015579 54.0
87.502748121 0.0873119831085205 55.0
0 0.884268879890442 56.0
72.633032796 0.163454115390778 57.0
227.00655682 0.628101587295532 58.0
252.80342838 0.776707530021667 59.0
21.270903378 -0.0130910202860832 60.0
0 0.0451507531106472 61.0
120.8717624 1.34231579303741 0.0
120.8717624 0.819731473922729 1.0
119.35369144 1.25743269920349 2.0
114.56458629 2.26629877090454 3.0
0 3.56905078887939 4.0
0 6.55340051651001 5.0
-59.324986943 0.864928960800171 6.0
68.765433369 0.396961212158203 7.0
62.841831875 1.24193668365479 8.0
5.9236014949 1.44846868515015 9.0
86.768941581 1.37351131439209 10.0
72.392892611 0.11315344274044 11.0
81.70435052 1.04327619075775 12.0
81.70435052 0.105315431952477 13.0
81.70435052 0.0270264856517315 14.0
0 0.0848908796906471 15.0
0 0.195967555046082 16.0
0 0.925689578056335 17.0
49.747365218 1.56569123268127 18.0
-49.747365218 1.36440944671631 19.0
-49.747365218 0.650133728981018 20.0
-49.747365218 -0.0804730355739594 21.0
68.371009278 1.86649298667908 22.0
149.3832222 0.597691535949707 23.0
115.24049672 0.728155016899109 24.0
18.543911348 0.228847339749336 25.0
217.75423147 1.6707718372345 26.0
217.75423147 2.97664880752563 27.0
28.918199571 3.31613039970398 28.0
246.67243104 1.29434978961945 29.0
246.67243104 2.17509627342224 30.0
47.086254941 1.24959278106689 31.0
9.2299840822 0.468954831361771 32.0
18.62364406 1.39195930957794 33.0
18.62364406 0.891736805438995 34.0
18.62364406 0.024098839610815 35.0
8.6132826458 0.0234187804162502 36.0
7.525919323 0.402037680149078 37.0
0 0.0830869376659393 38.0
0 0.291972130537033 39.0
-13.613585497 0.340648591518402 40.0
-27.519608238 1.26302826404572 41.0
4.5660182025 0.388343662023544 42.0
-34.762216177 0.563005685806274 43.0
113.68401179 0.778399050235748 44.0
113.68401179 0.0567955151200294 45.0
6.3726325989 0.492482900619507 46.0
2.7500487303 0.34699746966362 47.0
120.8717624 0.165982827544212 48.0
114.56458629 0.231499210000038 49.0
0 0.468281745910645 50.0
68.765433369 0.116334557533264 51.0
86.768941581 0.171028897166252 52.0
81.70435052 0.0370768383145332 53.0
81.70435052 0.0568574890494347 54.0
81.70435052 0.108942367136478 55.0
0 2.679358959198 56.0
68.371009278 0.123987756669521 57.0
217.75423147 0.651673853397369 58.0
246.67243104 0.506118178367615 59.0
18.62364406 0.0453721508383751 60.0
0 0.0522091016173363 61.0
191.12981772 0.795802056789398 0.0
191.12981772 1.61662971973419 1.0
190.03533845 1.75370573997498 2.0
184.57066777 5.33047533035278 3.0
14.126958914 4.61482286453247 4.0
14.126958914 5.22859144210815 5.0
-54.64905892 0.460290879011154 6.0
66.338221498 0.631862103939056 7.0
42.010450263 0.949423015117645 8.0
24.327771234 1.05430173873901 9.0
89.821303289 0.843594431877136 10.0
77.621084734 0.261764019727707 11.0
75.878331914 0.671233296394348 12.0
75.878331914 0.532141745090485 13.0
75.878331914 0.0770265832543373 14.0
0 0.00953022949397564 15.0
0 0.221247181296349 16.0
0 1.1813337802887 17.0
53.701787593 1.75759088993073 18.0
-53.701787593 1.51200008392334 19.0
-53.701787593 0.839933216571808 20.0
-53.701787593 -0.000933006405830383 21.0
75.323689887 1.97616612911224 22.0
148.23355313 1.0223160982132 23.0
111.72862285 0.558777630329132 24.0
21.134255971 0.197714045643806 25.0
223.55724301 1.55249321460724 26.0
223.55724301 1.84310865402222 27.0
27.9900339 1.48812472820282 28.0
251.54727691 0.653197765350342 29.0
251.54727691 1.86301445960999 30.0
46.658153061 1.15472280979156 31.0
9.3380361414 0.640686869621277 32.0
21.621902294 1.35096800327301 33.0
21.621902294 1.36149227619171 34.0
21.621902294 0.129667997360229 35.0
9.8893682855 0.0330301634967327 36.0
8.754101804 0.415584653615952 37.0
0 -0.00370113179087639 38.0
0 0.123086541891098 39.0
-13.062379913 0.835129797458649 40.0
-27.323985622 0.52447372674942 41.0
5.3454616588 0.529442191123962 42.0
-35.324480752 0.659347772598267 43.0
97.673003607 0.857271194458008 44.0
97.673003607 0.0905807316303253 45.0
6.7931334059 0.742028832435608 46.0
2.7349432938 0.397419780492783 47.0
191.12981772 0.105729959905148 48.0
184.57066777 0.104700617492199 49.0
14.126958914 0.193128019571304 50.0
66.338221498 0.31618919968605 51.0
89.821303289 0.103418782353401 52.0
75.878331914 0.131092429161072 53.0
75.878331914 0.215491756796837 54.0
75.878331914 0.0258693695068359 55.0
0 6.92760562896729 56.0
75.323689887 0.122871644794941 57.0
223.55724301 0.637689530849457 58.0
251.54727691 0.5812047123909 59.0
21.621902294 0.00422641169279814 60.0
0 0.0869672447443008 61.0
207.2359248 0.796268165111542 0.0
207.2359248 2.35519123077393 1.0
206.06750811 2.26715993881226 2.0
201.27840302 3.50759959220886 3.0
0 3.97032403945923 4.0
0 3.53306794166565 5.0
-54.92305443 0.486790895462036 6.0
68.195246497 0.520737409591675 7.0
68.081504913 1.51633620262146 8.0
0.11374158361 1.62097358703613 9.0
71.080196793 1.24463260173798 10.0
57.469544676 0.361122488975525 11.0
76.102801552 0.974703967571259 12.0
76.102801552 0.531147539615631 13.0
76.102801552 0.114950284361839 14.0
0 0.157363250851631 15.0
0 0.119872115552425 16.0
0 0.904855847358704 17.0
55.754364481 0.879215836524963 18.0
-55.754364481 0.926099061965942 19.0
-55.754364481 0.394845396280289 20.0
-55.754364481 0.00441837310791016 21.0
77.252689471 3.54639077186584 22.0
151.45391845 2.78388047218323 23.0
111.93956939 0.194974988698959 24.0
23.965068103 0.346349716186523 25.0
228.70660792 1.64364945888519 26.0
228.70660792 2.09903740882874 27.0
22.554055574 1.71133804321289 28.0
251.26066349 1.19510114192963 29.0
251.26066349 1.60308349132538 30.0
46.432095888 1.13864374160767 31.0
9.3868498703 0.276042401790619 32.0
21.49832499 1.38382005691528 33.0
21.49832499 0.936381220817566 34.0
21.49832499 0.157044738531113 35.0
8.4270267425 0.0484073311090469 36.0
7.3271762517 0.337460964918137 37.0
0 0.0993851274251938 38.0
0 0.229877561330795 39.0
-11.17888138 0.7767094373703 40.0
-23.212016419 0.624776244163513 41.0
4.9917191012 0.497424989938736 42.0
-30.662212002 0.759343564510345 43.0
102.81317729 0.504768431186676 44.0
102.81317729 0.0462247952818871 45.0
6.5760770703 0.56899619102478 46.0
2.8552027214 0.392811506986618 47.0
207.2359248 0.18648461997509 48.0
201.27840302 0.398100763559341 49.0
0 0.100786529481411 50.0
68.195246497 -0.0160223860293627 51.0
71.080196793 0.912559688091278 52.0
76.102801552 0.107591971755028 53.0
76.102801552 0.0664913430809975 54.0
76.102801552 -0.00771320518106222 55.0
0 5.25624847412109 56.0
77.252689471 0.236849397420883 57.0
228.70660792 0.533206105232239 58.0
251.26066349 0.397793561220169 59.0
21.49832499 -0.0258809886872768 60.0
0 0.106101110577583 61.0
192.68230602 0.900251567363739 0.0
192.68230602 1.93661856651306 1.0
191.16620389 1.838010430336 2.0
190.11022345 4.31258249282837 3.0
0 3.92338132858276 4.0
0 7.26633310317993 5.0
-77.182233787 1.76789104938507 6.0
85.676052912 0.943251609802246 7.0
62.95232922 1.12976288795471 8.0
22.723723692 1.23622512817383 9.0
106.15963051 1.02938270568848 10.0
93.157840069 0.292324393987656 11.0
90.952271877 0.107739835977554 12.0
90.952271877 0.167906045913696 13.0
90.952271877 0.0631783306598663 14.0
0 0.172757372260094 15.0
0 0.201793476939201 16.0
0 1.21681463718414 17.0
64.776377605 0.966302812099457 18.0
-64.776377605 0.973114490509033 19.0
-65.444645347 0.732368052005768 20.0
-65.513206946 -0.117985188961029 21.0
88.808558141 1.95663785934448 22.0
135.38324074 1.63192534446716 23.0
105.96021847 0.461347043514252 24.0
10.56059518 0.116110995411873 25.0
224.19179888 1.49691498279572 26.0
224.19179888 2.41922974586487 27.0
25.337226231 2.88480186462402 28.0
249.52902511 1.22930550575256 29.0
249.52902511 1.8361759185791 30.0
46.329492684 1.05124354362488 31.0
9.379514299 0.574056088924408 32.0
23.295351194 2.29080104827881 33.0
23.295351194 1.29950332641602 34.0
23.295351194 0.0807267352938652 35.0
10.06417092 0.0224192030727863 36.0
8.7587312141 0.340343505144119 37.0
0 -0.00219468027353287 38.0
0 0.186261609196663 39.0
-12.196704228 0.469525009393692 40.0
-25.546615191 0.512362837791443 41.0
5.6355722667 0.439626514911652 42.0
-33.901033464 0.543602824211121 43.0
94.437945086 0.614203572273254 44.0
93.201212072 0.0819269791245461 45.0
6.4916744284 0.565818548202515 46.0
1.8462196026 0.453503549098969 47.0
192.68230602 0.10159058123827 48.0
190.11022345 0.227628991007805 49.0
0 0.247261941432953 50.0
85.676052912 0.197379320859909 51.0
106.15963051 0.162118703126907 52.0
90.952271877 0.0142287239432335 53.0
90.952271877 0.0740761831402779 54.0
90.952271877 0.0502629950642586 55.0
0 1.97724759578705 56.0
88.808558141 0.315908432006836 57.0
224.19179888 0.710353553295135 58.0
249.52902511 0.460995614528656 59.0
23.295351194 -0.00383907370269299 60.0
0 0.0822349116206169 61.0
180.98010753 0.930801749229431 0.0
180.98010753 1.25671100616455 1.0
179.7328082 1.71612083911896 2.0
174.9437029 5.57193660736084 3.0
0 4.9686279296875 4.0
0 3.25367021560669 5.0
-53.966924272 0.273118555545807 6.0
65.05925954 1.29229056835175 7.0
50.117121385 0.75544285774231 8.0
14.942138155 0.764559984207153 9.0
89.063163356 1.02160441875458 10.0
76.100274485 0.119800388813019 11.0
84.820633391 0.776859223842621 12.0
84.820633391 0.511381387710571 13.0
84.820633391 0.0365503579378128 14.0
0 0.118217639625072 15.0
0 0.225751668214798 16.0
0 1.0992603302002 17.0
57.34521457 2.0146951675415 18.0
-57.34521457 2.13585352897644 19.0
-57.34521457 0.368134915828705 20.0
-57.34521457 -0.0315152332186699 21.0
76.628048788 1.599365234375 22.0
152.22720824 1.72006726264954 23.0
118.59704993 0.525996088981628 24.0
20.201905132 0.17727467417717 25.0
228.85525702 1.43187832832336 26.0
228.85525702 1.14920878410339 27.0
25.85710135 1.32573115825653 28.0
254.71235838 0.40993732213974 29.0
254.71235838 1.5895482301712 30.0
46.352601902 0.996813237667084 31.0
9.2077474241 0.448706388473511 32.0
19.282834218 1.15670740604401 33.0
19.282834218 0.822998821735382 34.0
19.282834218 0.0928320437669754 35.0
8.043418661 0.0227504502981901 36.0
6.9559993117 0.380153447389603 37.0
0 0.0257689654827118 38.0
0 0.194321349263191 39.0
-11.835108089 0.6715167760849 40.0
-25.701149966 0.510667145252228 41.0
5.2929336734 0.584408402442932 42.0
-33.180944586 0.926644146442413 43.0
75.17158293 0.521413862705231 44.0
75.17158293 0.106475099921227 45.0
6.0019031658 0.454767107963562 46.0
2.1130710155 0.348590642213821 47.0
180.98010753 0.0815254896879196 48.0
174.9437029 0.456690073013306 49.0
0 0.227384582161903 50.0
65.05925954 0.288238793611526 51.0
89.063163356 0.0954461693763733 52.0
84.820633391 0.016655221581459 53.0
84.820633391 0.117441236972809 54.0
84.820633391 0.0917543023824692 55.0
0 2.84528613090515 56.0
76.628048788 -0.0596733205020428 57.0
228.85525702 0.630564451217651 58.0
254.71235838 0.551624715328217 59.0
19.282834218 0.0453718341886997 60.0
0 0.0887553095817566 61.0
183.92209409 0.930019080638885 0.0
183.92209409 1.36034309864044 1.0
182.56477765 1.59873139858246 2.0
177.10010697 3.40782356262207 3.0
10.685513509 3.97249269485474 4.0
10.685513509 3.71268749237061 5.0
-64.965581303 -0.0133057832717896 6.0
77.827163111 0.708077251911163 7.0
38.735571239 1.33177781105042 8.0
39.091591872 1.15085697174072 9.0
82.75302242 0.975663661956787 10.0
69.234297485 0.0789325088262558 11.0
82.435339302 0.726856827735901 12.0
82.435339302 0.0981220304965973 13.0
82.435339302 0.098855584859848 14.0
0 0.191920891404152 15.0
0 0.192135229706764 16.0
0 1.07373571395874 17.0
60.049078602 1.20269072055817 18.0
-60.049078602 1.31120717525482 19.0
-60.049078602 0.852492272853851 20.0
-60.049078602 -0.0574882626533508 21.0
80.659897331 1.68600940704346 22.0
148.5996538 1.09726738929749 23.0
112.5556607 0.461386919021606 24.0
21.209018987 0.218099564313889 25.0
229.25955114 1.15760111808777 26.0
229.25955114 2.35182499885559 27.0
26.300835402 1.6934802532196 28.0
255.56038654 0.701443314552307 29.0
255.56038654 1.66691434383392 30.0
46.687524231 1.01832365989685 31.0
9.3488515535 0.46125403046608 32.0
20.610818729 1.14731061458588 33.0
20.610818729 1.07397198677063 34.0
20.610818729 0.110170491039753 35.0
8.896650125 0.0280669499188662 36.0
7.4961022236 0.343838185071945 37.0
0 0.0335590988397598 38.0
0 0.179228037595749 39.0
-11.818436971 0.740338861942291 40.0
-25.426536674 0.31158509850502 41.0
5.6383730883 0.573376178741455 42.0
-33.403876769 0.564618229866028 43.0
91.688139372 0.672970831394196 44.0
91.688139372 0.0984563753008842 45.0
7.0089458458 0.604072868824005 46.0
2.8342558899 0.431716352701187 47.0
183.92209409 0.112430214881897 48.0
177.10010697 0.154630899429321 49.0
10.685513509 0.0621750503778458 50.0
77.827163111 0.380657523870468 51.0
82.75302242 0.3399318754673 52.0
82.435339302 0.0306903086602688 53.0
82.435339302 0.055685393512249 54.0
82.435339302 0.0830646753311157 55.0
0 2.27594041824341 56.0
80.659897331 0.101178266108036 57.0
229.25955114 0.567456007003784 58.0
255.56038654 0.747710645198822 59.0
20.610818729 -0.00333818048238754 60.0
0 0.112764835357666 61.0
159.75786195 0.951691389083862 0.0
159.75786195 0.979962170124054 1.0
158.39748653 0.991258323192596 2.0
153.60838142 3.50673007965088 3.0
0 7.1812949180603 4.0
0 5.95837020874023 5.0
-61.496929689 3.4666543006897 6.0
72.970008956 1.22883117198944 7.0
53.356860511 0.597552359104156 8.0
19.613148446 0.383938163518906 9.0
81.030716004 1.02044546604156 10.0
68.897532017 0.0699886605143547 11.0
79.285400231 0.735035419464111 12.0
79.285400231 0.673354744911194 13.0
79.285400231 0.144383817911148 14.0
0 0.160843700170517 15.0
0 0.193349033594131 16.0
0 1.6725070476532 17.0
59.311826476 0.895572185516357 18.0
-59.311826476 0.546218872070312 19.0
-59.311826476 0.67928147315979 20.0
-59.311826476 0.0239687114953995 21.0
79.844738449 1.2568610906601 22.0
142.24067739 1.16705930233002 23.0
108.57689973 0.596122980117798 24.0
21.131294872 0.0750424116849899 25.0
222.08541584 0.946007668972015 26.0
222.08541584 1.41306066513062 27.0
23.069995428 1.22921276092529 28.0
245.15541127 0.833790898323059 29.0
245.15541127 1.50458908081055 30.0
47.07111742 1.09682393074036 31.0
9.4401484871 0.397765666246414 32.0
20.532911973 1.00555205345154 33.0
20.532911973 0.768971920013428 34.0
20.532911973 0.101291581988335 35.0
9.6717791532 0.0282948110252619 36.0
8.5167435495 0.367573231458664 37.0
0 0.0242087580263615 38.0
0 0.271106988191605 39.0
-11.859359965 0.611982822418213 40.0
-25.104103132 0.588144779205322 41.0
5.1845123961 0.511038839817047 42.0
-32.958031271 0.679985463619232 43.0
96.003066654 0.635274350643158 44.0
96.003066654 0.0756086930632591 45.0
7.2380307986 0.405006736516953 46.0
2.6564389097 0.305278211832047 47.0
159.75786195 0.16553395986557 48.0
153.60838142 0.309365332126617 49.0
0 0.354335099458694 50.0
72.970008956 0.0732386559247971 51.0
81.030716004 0.174087285995483 52.0
79.285400231 0.0483571290969849 53.0
79.285400231 0.137736812233925 54.0
79.285400231 0.248242750763893 55.0
0 5.51177310943604 56.0
79.844738449 0.0987013876438141 57.0
222.08541584 0.652565598487854 58.0
245.15541127 0.484173774719238 59.0
20.532911973 -0.00040610134601593 60.0
0 0.0882312208414078 61.0
165.27514461 0.946897864341736 0.0
165.27514461 1.4257595539093 1.0
163.88545349 1.71365416049957 2.0
159.09634832 6.54303789138794 3.0
0 7.83424282073975 4.0
0 4.37380647659302 5.0
-56.816308341 2.9274206161499 6.0
68.032738729 0.841618418693542 7.0
68.092789051 1.22875142097473 8.0
-0.060050321744 1.27133703231812 9.0
100.39555629 0.971148908138275 10.0
87.826895019 0.334458589553833 11.0
88.402100949 0.399549871683121 12.0
88.402100949 0.0824873596429825 13.0
88.402100949 0.0947501435875893 14.0
0 0.158107414841652 15.0
0 0.192658320069313 16.0
0 1.32337594032288 17.0
63.31778884 1.953293800354 18.0
-63.31778884 2.25640559196472 19.0
-63.31778884 0.750198841094971 20.0
-63.31778884 0.0166923925280571 21.0
84.306607295 1.60203266143799 22.0
149.18805988 1.62275326251984 23.0
108.3011454 0.181900352239609 24.0
23.232098217 0.173451259732246 25.0
233.49466717 1.04556322097778 26.0
233.49466717 1.86800301074982 27.0
30.03272519 0.897778511047363 28.0
263.52739236 0.739409267902374 29.0
263.52739236 1.50773310661316 30.0
46.991774236 1.00463855266571 31.0
9.168578209 0.472340255975723 32.0
20.988818455 1.0750914812088 33.0
20.988818455 1.4667786359787 34.0
20.988818455 0.116387657821178 35.0
10.241202766 0.0200922563672066 36.0
8.8761590115 0.45097279548645 37.0
0 0.0464802235364914 38.0
0 0.0235004127025604 39.0
-12.166414507 0.677095353603363 40.0
-27.315369191 0.314718455076218 41.0
4.6152995088 0.37532502412796 42.0
-34.164975231 0.299477308988571 43.0
87.636035237 0.806209206581116 44.0
87.636035237 0.089311771094799 45.0
6.7011373796 0.56891942024231 46.0
2.0767349427 0.361137926578522 47.0
165.27514461 0.172528564929962 48.0
159.09634832 0.166406750679016 49.0
0 0.791043162345886 50.0
68.032738729 0.266127556562424 51.0
100.39555629 0.479433685541153 52.0
88.402100949 0.0952855497598648 53.0
88.402100949 0.0457796789705753 54.0
88.402100949 0.0952838659286499 55.0
0 1.95904231071472 56.0
84.306607295 -0.069096527993679 57.0
233.49466717 0.615243494510651 58.0
263.52739236 0.251689076423645 59.0
20.988818455 0.0395825169980526 60.0
0 0.0894623547792435 61.0
165.38203521 0.920887410640717 0.0
165.38203521 1.00406348705292 1.0
164.20138962 1.26707792282104 2.0
159.41228444 3.95613694190979 3.0
0 5.13995933532715 4.0
0 2.82153034210205 5.0
-53.360483455 0.203005343675613 6.0
62.863087 0.507497489452362 7.0
56.028328967 1.28070259094238 8.0
6.8347580335 1.1120525598526 9.0
90.040038921 1.08355700969696 10.0
78.264698014 0.264586597681046 11.0
88.771742642 0.373267382383347 12.0
88.771742642 0.45430400967598 13.0
88.771742642 0.0679778680205345 14.0
0 0.155418395996094 15.0
0 0.225271075963974 16.0
0 0.689728617668152 17.0
54.390240122 1.56222081184387 18.0
-54.390240122 1.88000917434692 19.0
-54.390240122 0.68477076292038 20.0
-54.390240122 -0.0567467994987965 21.0
75.702824257 3.21285772323608 22.0
139.16392332 2.97926783561707 23.0
105.42028486 0.327568292617798 24.0
23.424360294 0.0874478593468666 25.0
214.86674758 1.78797197341919 26.0
214.86674758 2.12026190757751 27.0
25.038916887 2.2639582157135 28.0
239.90566447 1.16944181919098 29.0
239.90566447 1.77001535892487 30.0
47.028693636 1.05271744728088 31.0
9.3039806838 0.355733871459961 32.0
21.312584135 1.41015207767487 33.0
21.312584135 1.11685121059418 34.0
21.312584135 0.128048926591873 35.0
8.8125662883 0.0350000262260437 36.0
7.6198459112 0.324898391962051 37.0
0 0.0075388103723526 38.0
0 0.0689087212085724 39.0
-12.619589879 0.548665285110474 40.0
-24.500076987 0.615066111087799 41.0
5.0479935737 0.408375948667526 42.0
-31.993856655 0.369968861341476 43.0
106.97622335 0.773597717285156 44.0
106.97622335 0.0773588344454765 45.0
6.5790484975 0.497223287820816 46.0
1.7040667116 0.332491666078568 47.0
165.38203521 0.123816512525082 48.0
159.41228444 0.0240463130176067 49.0
0 0.148369058966637 50.0
62.863087 0.136394441127777 51.0
90.040038921 0.332012265920639 52.0
88.771742642 0.0587971657514572 53.0
88.771742642 0.0528578758239746 54.0
88.771742642 0.0737432241439819 55.0
0 2.40735960006714 56.0
75.702824257 0.12063430249691 57.0
214.86674758 0.657360076904297 58.0
239.90566447 1.20349836349487 59.0
21.312584135 -0.00970481708645821 60.0
0 0.0933668836951256 61.0
140.28744287 0.808172166347504 0.0
140.28744287 1.52630841732025 1.0
139.1120886 1.38908934593201 2.0
134.32298339 2.34515929222107 3.0
0 3.84473657608032 4.0
0 3.34975671768188 5.0
-51.742397252 0.178754672408104 6.0
61.392485854 0.575607359409332 7.0
63.570096965 0.83701628446579 8.0
-2.1776111113 1.02501821517944 9.0
86.529859295 1.20348942279816 10.0
72.720717409 0.296113133430481 11.0
82.710456437 1.02714788913727 12.0
82.710456437 1.05594491958618 13.0
82.710456437 0.0475107058882713 14.0
0 0.183391764760017 15.0
0 0.154595226049423 16.0
0 1.36296570301056 17.0
59.552469942 0.95441734790802 18.0
-59.552469942 0.530336081981659 19.0
-59.552469942 0.92014753818512 20.0
-59.552469942 -0.0823575556278229 21.0
82.119049173 1.80285906791687 22.0
136.72593605 0.894276320934296 23.0
104.48563663 0.873296022415161 24.0
25.49060624 0.0629145205020905 25.0
218.84498522 1.20635938644409 26.0
218.84498522 1.49376380443573 27.0
26.332111983 0.870602071285248 28.0
245.1770972 0.547837615013123 29.0
245.1770972 1.24588203430176 30.0
46.235849157 0.982968688011169 31.0
9.4812348226 0.435618877410889 32.0
22.566579231 1.70934391021729 33.0
22.566579231 1.30145859718323 34.0
22.566579231 0.123108729720116 35.0
10.310135121 0.0336519852280617 36.0
9.118571694 0.360618621110916 37.0
0 0.0188017226755619 38.0
0 0.349309176206589 39.0
-11.403751322 0.415426671504974 40.0
-24.907385393 0.570089161396027 41.0
5.2942176562 0.529104709625244 42.0
-33.009856388 0.657632946968079 43.0
98.976728294 0.43833988904953 44.0
98.976728294 0.101144917309284 45.0
7.5944225197 0.521057426929474 46.0
2.7516100331 0.431052416563034 47.0
140.28744287 0.173517197370529 48.0
134.32298339 0.0635550245642662 49.0
0 0.429120719432831 50.0
61.392485854 0.163848355412483 51.0
86.529859295 0.400295346975327 52.0
82.710456437 0.0540882870554924 53.0
82.710456437 0.34651792049408 54.0
82.710456437 0.163572177290916 55.0
0 1.28472399711609 56.0
82.119049173 0.214835852384567 57.0
218.84498522 0.582151353359222 58.0
245.1770972 0.21752168238163 59.0
22.566579231 0.0259109102189541 60.0
0 0.067051112651825 61.0
174.27207215 1.00933384895325 0.0
174.27207215 0.820144653320312 1.0
173.10129466 1.60951638221741 2.0
168.31218947 3.58851361274719 3.0
0 4.73560905456543 4.0
0 4.12097311019897 5.0
-57.975444177 3.53639221191406 6.0
67.913180377 1.02587580680847 7.0
57.380952421 1.91503882408142 8.0
10.532227956 1.74289298057556 9.0
79.955227761 0.973516047000885 10.0
67.555469812 0.0700273364782333 11.0
80.419944451 0.128259658813477 12.0
80.419944451 0.371616512537003 13.0
80.419944451 0.123685941100121 14.0
0 0.102407015860081 15.0
0 0.198440685868263 16.0
0 1.03616487979889 17.0
49.361325024 1.53256821632385 18.0
-49.361325024 1.67608201503754 19.0
-49.361325024 0.322274386882782 20.0
-49.361325024 0.0145473219454288 21.0
72.633307927 1.29541206359863 22.0
138.96779576 0.754830956459045 23.0
104.99916964 0.730558514595032 24.0
19.735645114 0.0586651675403118 25.0
211.60110369 0.702137768268585 26.0
211.60110369 0.912927865982056 27.0
27.365531629 0.714595437049866 28.0
238.96663532 0.501222491264343 29.0
238.96663532 1.47921133041382 30.0
46.089433275 1.1051572561264 31.0
9.2136556391 0.450346261262894 32.0
23.271982903 1.37566566467285 33.0
23.271982903 1.06931185722351 34.0
23.271982903 0.0980419591069221 35.0
10.40680778 0.0247975401580334 36.0
9.2067370391 0.423985451459885 37.0
0 -0.0209394469857216 38.0
0 0.0773936808109283 39.0
-10.319390991 0.585942387580872 40.0
-23.232344172 0.229963690042496 41.0
5.3326557562 0.6236732006073 42.0
-31.171066236 0.485755145549774 43.0
109.98578552 0.660496354103088 44.0
109.98578552 0.146494314074516 45.0
6.2258864458 0.64610880613327 46.0
2.3308793237 0.407869577407837 47.0
174.27207215 0.148728013038635 48.0
168.31218947 0.453670293092728 49.0
0 0.361006379127502 50.0
67.913180377 0.299161076545715 51.0
79.955227761 0.139210730791092 52.0
80.419944451 0.016893507912755 53.0
80.419944451 0.0732644125819206 54.0
80.419944451 0.0411503873765469 55.0
0 7.06666851043701 56.0
72.633307927 0.152824983000755 57.0
211.60110369 0.042805902659893 58.0
238.96663532 0.382014781236649 59.0
23.271982903 0.00941574852913618 60.0
0 0.169061034917831 61.0
137.20916752 1.21788656711578 0.0
137.20916752 1.19658946990967 1.0
135.73416125 1.14333581924438 2.0
130.94505598 4.76742076873779 3.0
0 6.12641143798828 4.0
0 3.94929885864258 5.0
-54.619382752 0.434649258852005 6.0
68.101407649 0.911745429039001 7.0
53.734534479 0.661198377609253 8.0
14.366873169 0.489762365818024 9.0
74.561282031 1.19971859455109 10.0
62.178516385 0.0277635678648949 11.0
72.901254397 0.796512722969055 12.0
72.901254397 0.333674490451813 13.0
72.901254397 0.0497149750590324 14.0
0 0.0272383913397789 15.0
0 0.194189697504044 16.0
0 0.784178614616394 17.0
53.385695542 1.12618005275726 18.0
-53.385695542 0.659412741661072 19.0
-53.385695542 0.550707221031189 20.0
-53.385695542 -0.0317852422595024 21.0
75.18363238 2.32713985443115 22.0
146.72489356 1.71866428852081 23.0
113.31413423 0.534882426261902 24.0
22.084235971 0.173023089766502 25.0
221.90852594 1.0295158624649 26.0
221.90852594 1.28468692302704 27.0
30.750017206 2.42310094833374 28.0
252.65854314 1.2405446767807 29.0
252.65854314 1.58855032920837 30.0
47.016371323 1.04949021339417 31.0
9.4064622246 0.303119838237762 32.0
21.797936838 1.46954298019409 33.0
21.797936838 1.05907094478607 34.0
21.797936838 0.101532034575939 35.0
8.8388190132 0.0502207241952419 36.0
7.4757389977 0.378150403499603 37.0
0 0.023822657763958 38.0
0 0.151759684085846 39.0
-12.568375579 0.450082987546921 40.0
-25.786571713 0.565966844558716 41.0
5.0222616563 0.391191303730011 42.0
-33.218887646 0.519314706325531 43.0
86.296138375 0.66902631521225 44.0
86.296138375 0.0735248699784279 45.0
6.827905259 0.557616174221039 46.0
2.7609095237 0.394366145133972 47.0
137.20916752 0.157970383763313 48.0
130.94505598 0.416809141635895 49.0
0 0.45402804017067 50.0
68.101407649 0.0723391398787498 51.0
74.561282031 0.0411090329289436 52.0
72.901254397 0.0318403951823711 53.0
72.901254397 0.144198104739189 54.0
72.901254397 0.0971216261386871 55.0
0 6.18439292907715 56.0
75.18363238 0.0165001191198826 57.0
221.90852594 0.133521527051926 58.0
252.65854314 0.316476821899414 59.0
21.797936838 0.0322683714330196 60.0
0 0.0939724668860435 61.0
162.66187726 0.925492167472839 0.0
162.66187726 0.660026848316193 1.0
161.4840132 1.21204853057861 2.0
156.69490793 4.07016038894653 3.0
0 7.82217407226562 4.0
0 4.34600067138672 5.0
-55.420138736 0.519864737987518 6.0
65.440133398 0.674588263034821 7.0
42.983308097 0.647034347057343 8.0
22.456825301 0.458739399909973 9.0
92.307602249 1.01474738121033 10.0
79.182290255 0.227893054485321 11.0
81.97537854 0.575705587863922 12.0
81.97537854 0.17479133605957 13.0
81.97537854 0.121505990624428 14.0
0 0.158813998103142 15.0
0 0.112371034920216 16.0
0 1.13054955005646 17.0
54.440933791 1.01235580444336 18.0
-54.440933791 1.28336822986603 19.0
-54.440933791 0.340856909751892 20.0
-54.440933791 -0.018955497071147 21.0
76.040898805 2.2367262840271 22.0
139.43407945 2.33487892150879 23.0
106.89505046 0.370813965797424 24.0
22.716217506 0.18143817782402 25.0
215.47497826 1.60924792289734 26.0
215.47497826 2.13175010681152 27.0
26.10510816 1.12353801727295 28.0
241.58008642 0.468927621841431 29.0
241.58008642 1.39170491695404 30.0
47.074223872 1.19949269294739 31.0
9.4745625608 0.324466675519943 32.0
21.599965013 1.48974716663361 33.0
21.599965013 0.847555577754974 34.0
21.599965013 0.140281170606613 35.0
8.8275290803 0.0369710214436054 36.0
7.5923904709 0.352666020393372 37.0
0 0.0253350473940372 38.0
0 0.246365070343018 39.0
-11.649345442 0.277278006076813 40.0
-24.475176121 0.832323253154755 41.0
4.9579293491 0.36198216676712 42.0
-32.181835808 0.561470448970795 43.0
88.019558589 0.603516340255737 44.0
88.019558589 0.10095402598381 45.0
6.7831792399 0.661461293697357 46.0
2.6611446891 0.401245713233948 47.0
162.66187726 0.160981923341751 48.0
156.69490793 0.399741172790527 49.0
0 0.275306910276413 50.0
65.440133398 0.0113451778888702 51.0
92.307602249 0.0768759325146675 52.0
81.97537854 0.0574774108827114 53.0
81.97537854 0.0942350551486015 54.0
81.97537854 0.0750755444169044 55.0
0 2.17639923095703 56.0
76.040898805 0.159605368971825 57.0
215.47497826 0.758455336093903 58.0
241.58008642 0.0974442660808563 59.0
21.599965013 0.0194782037287951 60.0
0 0.147410914301872 61.0
190.44840538 0.996471762657166 0.0
190.44840538 0.924887776374817 1.0
189.54239311 0.982960999011993 2.0
184.07772239 7.57946109771729 3.0
1.1308408893 10.6872253417969 4.0
1.1308408893 6.79953908920288 5.0
-73.787805576 3.42420697212219 6.0
85.056722213 1.0496871471405 7.0
55.692903071 0.683463454246521 8.0
29.363819142 0.650583505630493 9.0
90.637902074 1.20556592941284 10.0
77.659755926 0.212859988212585 11.0
84.799306885 0.822747886180878 12.0
84.799306885 0.115245215594769 13.0
84.799306885 0.0220128912478685 14.0
0 -0.00729952985420823 15.0
0 0.220133870840073 16.0
0 0.95901358127594 17.0
47.012010717 1.69077634811401 18.0
-47.012010717 2.05489945411682 19.0
-47.012010717 0.338533699512482 20.0
-47.012010717 -0.0832885056734085 21.0
67.520104181 2.01845717430115 22.0
139.15219817 2.55690312385559 23.0
105.16648313 0.0692765489220619 24.0
25.186125365 0.168076276779175 25.0
206.67230235 1.47597229480743 26.0
206.67230235 2.87629342079163 27.0
22.806559227 2.53480458259583 28.0
229.47886158 0.728135526180267 29.0
229.47886158 1.39166688919067 30.0
45.95931694 1.21250867843628 31.0
9.2808276538 0.557657539844513 32.0
20.508093464 2.19920802116394 33.0
20.508093464 1.14238905906677 34.0
20.508093464 0.139847502112389 35.0
9.8997200088 0.0477182641625404 36.0
8.676449679 0.346675425767899 37.0
0 -0.0181721299886703 38.0
0 0.329868167638779 39.0
-11.049116302 0.378196507692337 40.0
-21.880077143 0.526563704013824 41.0
5.4863403283 0.400372445583344 42.0
-29.990430451 0.429993867874146 43.0
86.896718778 0.550443232059479 44.0
86.896718778 0.116180598735809 45.0
6.8965390021 0.442832052707672 46.0
3.0865553898 0.289331823587418 47.0
190.44840538 0.16707019507885 48.0
184.07772239 0.131609708070755 49.0
1.1308408893 1.85994482040405 50.0
85.056722213 0.407061874866486 51.0
90.637902074 0.608491063117981 52.0
84.799306885 0.0878994837403297 53.0
84.799306885 0.0188038609921932 54.0
84.799306885 0.0894712060689926 55.0
0 3.22969508171082 56.0
67.520104181 0.18685157597065 57.0
206.67230235 0.790957868099213 58.0
229.47886158 0.150332078337669 59.0
20.508093464 0.00899784173816442 60.0
0 0.123946875333786 61.0
137.4825583 1.24874210357666 0.0
137.4825583 0.891164302825928 1.0
136.25948925 1.74053657054901 2.0
131.47038402 3.02655625343323 3.0
0 4.04828357696533 4.0
0 3.42274141311646 5.0
-70.005398633 -0.0188833177089691 6.0
81.13050163 1.14752459526062 7.0
35.999312384 0.657488346099854 8.0
45.131189246 0.358558833599091 9.0
98.170535163 1.20625257492065 10.0
84.649668428 0.0730666667222977 11.0
87.21085971 0.134907126426697 12.0
87.21085971 0.069642998278141 13.0
87.21085971 0.0856492146849632 14.0
0 0.185772761702538 15.0
0 0.154222935438156 16.0
0 1.08120822906494 17.0
55.646363957 0.73020339012146 18.0
-55.646363957 0.561159610748291 19.0
-55.646363957 0.596991240978241 20.0
-55.646363957 0.0011230893433094 21.0
77.663500352 1.01654005050659 22.0
125.03621418 0.752412676811218 23.0
93.566243805 0.257930785417557 24.0
21.440625574 0.12438653409481 25.0
202.69971453 0.772030711174011 26.0
202.69971453 1.64871311187744 27.0
21.896591695 3.07752251625061 28.0
224.59630623 1.56007242202759 29.0
224.59630623 1.95272040367126 30.0
47.404866811 1.19462049007416 31.0
9.3410324135 0.193655490875244 32.0
22.017136395 1.15554642677307 33.0
22.017136395 1.29168212413788 34.0
22.017136395 0.123809367418289 35.0
11.465074494 0.0329754501581192 36.0
10.1826638 0.360225021839142 37.0
0 0.0187519676983356 38.0
0 0.276471048593521 39.0
-10.045108614 0.250387370586395 40.0
-22.831064964 1.00313913822174 41.0
5.5238892378 0.400040775537491 42.0
-31.287643347 0.313401877880096 43.0
95.303050177 0.577874183654785 44.0
95.303050177 0.151602268218994 45.0
7.1262327398 0.543967008590698 46.0
2.2798598379 0.381365865468979 47.0
137.4825583 0.125375792384148 48.0
131.47038402 0.437377601861954 49.0
0 0.659983158111572 50.0
81.13050163 0.05715661495924 51.0
98.170535163 0.0313567258417606 52.0
87.21085971 0.0482487715780735 53.0
87.21085971 0.0680046230554581 54.0
87.21085971 0.10701896995306 55.0
0 1.94721186161041 56.0
77.663500352 0.207737237215042 57.0
202.69971453 0.13053859770298 58.0
224.59630623 0.472807437181473 59.0
22.017136395 -0.00202704966068268 60.0
0 0.0480481758713722 61.0
196.43093475 0.64572536945343 0.0
196.43093475 1.63820886611938 1.0
195.34557051 2.0701642036438 2.0
190.52742167 6.17644882202148 3.0
0 5.55454778671265 4.0
0 6.21483469009399 5.0
-64.354803457 5.90118598937988 6.0
74.569360778 0.508782386779785 7.0
36.717331873 0.714152216911316 8.0
37.852028905 0.723673939704895 9.0
69.80176276 0.679532706737518 10.0
57.941119204 0.0315438732504845 11.0
59.101432265 1.36140608787537 12.0
59.101432265 0.654724895954132 13.0
59.101432265 0.0631262734532356 14.0
27.813546009 0.15461003780365 15.0
27.813546009 0.207109048962593 16.0
0 0.596823215484619 17.0
52.997198183 1.25758826732635 18.0
-52.997198183 1.31537115573883 19.0
-52.997198183 0.652977585792542 20.0
-52.997198183 -0.0584354139864445 21.0
75.278552888 2.42135715484619 22.0
142.97141752 1.61545717716217 23.0
106.04119557 0.781408309936523 24.0
24.395451674 0.100696235895157 25.0
218.24997041 1.57857191562653 26.0
218.24997041 1.99663841724396 27.0
29.538658012 1.32574307918549 28.0
247.78862842 0.830193042755127 29.0
247.78862842 1.54665696620941 30.0
47.385625497 1.05755043029785 31.0
9.3419087208 0.530213057994843 32.0
22.281354705 1.59315955638885 33.0
22.281354705 1.3515305519104 34.0
22.281354705 0.120231308043003 35.0
8.8464792716 0.0448082946240902 36.0
7.5798832422 0.396114230155945 37.0
0 0.000564657151699066 38.0
0 0.128325805068016 39.0
-11.177386926 1.06482708454132 40.0
-23.910102821 0.624537348747253 41.0
4.6274291348 0.649253904819489 42.0
-31.449377082 0.542620658874512 43.0
93.687007227 0.854288280010223 44.0
93.687007227 0.0846668258309364 45.0
7.4740047645 0.697148621082306 46.0
2.5260451246 0.419091463088989 47.0
196.43093475 0.0731710866093636 48.0
190.52742167 0.187162101268768 49.0
0 0.582224667072296 50.0
74.569360778 0.0917041674256325 51.0
69.80176276 0.169629782438278 52.0
59.101432265 0.0310964435338974 53.0
59.101432265 0.350673675537109 54.0
59.101432265 0.109145253896713 55.0
0 3.10470414161682 56.0
75.278552888 0.119986616075039 57.0
218.24997041 0.579084396362305 58.0
247.78862842 0.322992771863937 59.0
22.281354705 0.00149074569344521 60.0
0 0.0999625697731972 61.0
183.73650137 0.9050133228302 0.0
183.73650137 1.0203047990799 1.0
182.26846933 1.03773295879364 2.0
177.47936419 3.25621914863586 3.0
0 4.08148956298828 4.0
0 3.87673020362854 5.0
-62.509568904 1.93196260929108 6.0
72.681267167 0.672921717166901 7.0
49.697671064 1.0855770111084 8.0
22.983596103 1.14422655105591 9.0
85.055555573 1.05636250972748 10.0
71.441538405 0.101264275610447 11.0
79.924880966 0.213271290063858 12.0
79.924880966 0.431467831134796 13.0
79.924880966 0.106003142893314 14.0
0 0.205761566758156 15.0
0 0.154022693634033 16.0
0 1.5219339132309 17.0
58.83314039 1.11488997936249 18.0
-58.83314039 0.721616387367249 19.0
-58.83314039 0.979676365852356 20.0
-58.83314039 -0.0740930512547493 21.0
80.642716138 1.57670497894287 22.0
145.0586997 1.05003190040588 23.0
110.56088447 0.535417973995209 24.0
24.698800882 0.121446877717972 25.0
225.70141583 1.40067136287689 26.0
225.70141583 1.91030120849609 27.0
29.760870889 1.66225373744965 28.0
255.46228672 0.536217927932739 29.0
255.46228672 1.64470815658569 30.0
46.349771431 0.918025314807892 31.0
9.2785984592 0.545661330223083 32.0
21.809575747 1.65843331813812 33.0
21.809575747 1.09097647666931 34.0
21.809575747 0.0959826335310936 35.0
8.4527371237 0.0257482379674911 36.0
7.1659260501 0.373727411031723 37.0
0 0.0458468832075596 38.0
0 0.173900753259659 39.0
-12.444727221 0.405454963445663 40.0
-26.002010764 0.712271332740784 41.0
5.5063338575 0.463004946708679 42.0
-34.151700533 0.621265649795532 43.0
92.263205154 0.586873412132263 44.0
92.263205154 0.0593948885798454 45.0
6.6292520479 0.494316309690475 46.0
2.191805973 0.380179762840271 47.0
183.73650137 0.121635653078556 48.0
177.47936419 -0.00464665563777089 49.0
0 0.245331719517708 50.0
72.681267167 0.271818935871124 51.0
85.055555573 0.183983683586121 52.0
79.924880966 0.0231853537261486 53.0
79.924880966 0.0977913066744804 54.0
79.924880966 0.0591913796961308 55.0
0 2.31764125823975 56.0
80.642716138 0.0771777182817459 57.0
225.70141583 0.312478244304657 58.0
255.46228672 0.56210058927536 59.0
21.809575747 -0.00406916439533234 60.0
0 0.0764018893241882 61.0
158.39251804 1.14890909194946 0.0
158.39251804 1.55465888977051 1.0
156.99316738 2.15949749946594 2.0
152.20406228 4.02437210083008 3.0
0 5.69007110595703 4.0
0 3.09245324134827 5.0
-65.560727765 0.229147553443909 6.0
75.161046006 1.1892683506012 7.0
58.425365313 0.722848773002625 8.0
16.735680693 0.542824625968933 9.0
98.209922827 1.18390214443207 10.0
84.596869662 0.12252739071846 11.0
86.790034546 0.772821128368378 12.0
86.790034546 0.548875570297241 13.0
86.790034546 0.102255180478096 14.0
0 0.249311357736588 15.0
0 0.148201242089272 16.0
0 1.34072816371918 17.0
63.416426917 1.28094398975372 18.0
-63.416426917 1.25324845314026 19.0
-63.416426917 0.292750656604767 20.0
-63.416426917 0.0325684770941734 21.0
84.596316532 0.967044234275818 22.0
141.35355123 0.705801486968994 23.0
106.88338834 0.354206830263138 24.0
22.639019946 0.222538754343987 25.0
225.94986776 0.985229969024658 26.0
225.94986776 2.00164818763733 27.0
25.391957116 1.70794987678528 28.0
251.34182488 0.59232634305954 29.0
251.34182488 1.8727502822876 30.0
46.625049407 0.911292254924774 31.0
9.4388823112 0.411047369241714 32.0
21.179889615 1.31620645523071 33.0
21.179889615 0.847319185733795 34.0
21.179889615 0.0976787358522415 35.0
8.9690652536 0.0250043869018555 36.0
7.7122501905 0.404889225959778 37.0
0 0.0120371244847775 38.0
0 0.259710967540741 39.0
-11.970902391 0.349799871444702 40.0
-24.248878047 0.837372183799744 41.0
4.8567263383 0.352611392736435 42.0
-31.520078421 0.564818382263184 43.0
104.06043637 0.696982145309448 44.0
104.06043637 0.0768945813179016 45.0
6.5611236638 0.689053952693939 46.0
2.3377949121 0.430736631155014 47.0
158.39251804 0.12152698636055 48.0
152.20406228 0.358713179826736 49.0
0 0.0801196321845055 50.0
75.161046006 0.180701836943626 51.0
98.209922827 0.185680955648422 52.0
86.790034546 0.129635706543922 53.0
86.790034546 0.0568768009543419 54.0
86.790034546 0.0963831916451454 55.0
0 1.45276618003845 56.0
84.596316532 0.0839805603027344 57.0
225.94986776 0.326361507177353 58.0
251.34182488 1.12983059883118 59.0
21.179889615 0.00206941738724709 60.0
0 0.101954162120819 61.0
180.00532949 0.880903661251068 0.0
180.00532949 1.09216749668121 1.0
178.6719071 1.63207173347473 2.0
173.71327309 3.0923376083374 3.0
0 3.9644923210144 4.0
0 4.12329626083374 5.0
-75.70126784 0.940845131874084 6.0
87.873832153 0.605022966861725 7.0
60.559238745 1.2008353471756 8.0
27.314593408 1.39477634429932 9.0
92.96073505 1.00021123886108 10.0
79.984215815 0.403144598007202 11.0
87.226399406 0.697041451931 12.0
87.226399406 0.462149262428284 13.0
87.226399406 0.131864085793495 14.0
0 0.266400784254074 15.0
0 0.164794474840164 16.0
0 1.62219381332397 17.0
59.492668022 1.69255816936493 18.0
-59.492668022 1.52861511707306 19.0
-59.492668022 0.856028258800507 20.0
-59.492668022 -0.054589007049799 21.0
82.554432524 2.18114614486694 22.0
141.66021214 2.37767386436462 23.0
109.60761815 0.200247466564178 24.0
21.623831613 0.208187162876129 25.0
224.21464467 1.15770292282104 26.0
224.21464467 1.71579122543335 27.0
28.677898261 1.71206748485565 28.0
252.89254293 0.457233905792236 29.0
252.89254293 1.67938017845154 30.0
47.273590229 1.06025528907776 31.0
9.4533816342 0.555745720863342 32.0
23.061764502 1.33859884738922 33.0
23.061764502 1.18672680854797 34.0
23.061764502 0.100485369563103 35.0
10.136169977 0.0266025140881538 36.0
8.8993643411 0.400124341249466 37.0
0 0.0383067168295383 38.0
0 0.0815922170877457 39.0
-12.54236702 0.473773002624512 40.0
-24.482536216 0.624413728713989 41.0
5.447201441 0.359191358089447 42.0
-32.275548386 0.565044105052948 43.0
88.834420894 0.639800786972046 44.0
88.834420894 0.101640820503235 45.0
6.5927695932 0.54926335811615 46.0
2.2279921555 0.404370963573456 47.0
180.00532949 0.0296671316027641 48.0
173.71327309 0.116097331047058 49.0
0 0.0606381967663765 50.0
87.873832153 0.155479326844215 51.0
92.96073505 0.330070525407791 52.0
87.226399406 0.109575845301151 53.0
87.226399406 0.0614892914891243 54.0
87.226399406 0.0433401390910149 55.0
0 2.33506226539612 56.0
82.554432524 0.0308348499238491 57.0
224.21464467 0.370985716581345 58.0
252.89254293 0.252183645963669 59.0
23.061764502 -0.0109763592481613 60.0
0 0.0939025729894638 61.0
164.74225234 1.05281114578247 0.0
164.74225234 1.4755973815918 1.0
163.6809727 1.29863476753235 2.0
158.89186746 2.339515209198 3.0
0 3.18146204948425 4.0
0 3.80055856704712 5.0
-53.06816879 0.188354879617691 6.0
62.590794441 0.580227851867676 7.0
40.91760948 0.221294343471527 8.0
21.673184961 0.234483703970909 9.0
107.36086832 1.03682708740234 10.0
94.270006898 0.0774231627583504 11.0
90.473564422 0.51293283700943 12.0
90.473564422 0.723667502403259 13.0
90.473564422 0.0703097879886627 14.0
0 0.110665485262871 15.0
0 0.247521042823792 16.0
0 0.545147657394409 17.0
49.280867633 1.43515729904175 18.0
-49.280867633 2.20414280891418 19.0
-49.280867633 0.238533407449722 20.0
-49.280867633 -0.0257397331297398 21.0
72.439459712 0.959157288074493 22.0
139.11849294 0.887038230895996 23.0
100.50194269 0.709464728832245 24.0
13.544633454 0.0813520029187202 25.0
211.55795265 0.917183697223663 26.0
211.55795265 1.43670916557312 27.0
24.110044483 0.967178404331207 28.0
235.66799713 1.06749773025513 29.0
235.66799713 1.63018715381622 30.0
47.011454414 0.952565014362335 31.0
9.1896513603 0.368624299764633 32.0
23.158592079 1.09146642684937 33.0
23.158592079 1.18787288665771 34.0
23.158592079 0.128308653831482 35.0
9.9839157946 0.039442703127861 36.0
8.5372329492 0.323849409818649 37.0
0 0.0312278047204018 38.0
0 0.305777460336685 39.0
-11.416870758 0.612160921096802 40.0
-24.054666698 0.511091351509094 41.0
4.8084297409 0.731397986412048 42.0
-30.997102633 1.34787678718567 43.0
94.538176827 0.420757591724396 44.0
94.538176827 0.093692347407341 45.0
6.9630538153 0.503289937973022 46.0
2.938901002 0.473480194807053 47.0
164.74225234 0.178145602345467 48.0
158.89186746 0.237399831414223 49.0
0 0.0830041170120239 50.0
62.590794441 0.19892792403698 51.0
107.36086832 0.12775781750679 52.0
90.473564422 0.115298323333263 53.0
90.473564422 0.13641569018364 54.0
90.473564422 0.109779089689255 55.0
0 3.88304543495178 56.0
72.439459712 0.213941141963005 57.0
211.55795265 0.657740294933319 58.0
235.66799713 0.392956256866455 59.0
23.158592079 0.0257392227649689 60.0
0 0.0776935741305351 61.0
135.25208383 0.896430194377899 0.0
135.25208383 2.12578821182251 1.0
134.2257308 1.61485683917999 2.0
129.43662555 4.17304563522339 3.0
0 5.56962013244629 4.0
0 4.08166027069092 5.0
-64.955711813 1.87691867351532 6.0
77.053991413 0.832681596279144 7.0
57.026079575 0.910705864429474 8.0
20.027911838 1.08046615123749 9.0
87.006208124 1.17280530929565 10.0
73.660371469 0.28271472454071 11.0
78.728101354 0.812385559082031 12.0
78.728101354 0.558946847915649 13.0
78.728101354 0.0927359908819199 14.0
0 0.256500780582428 15.0
0 0.17241695523262 16.0
0 1.47071015834808 17.0
60.47571484 1.22793793678284 18.0
-60.47571484 1.28790628910065 19.0
-60.47571484 0.319443047046661 20.0
-60.47571484 -0.0161255598068237 21.0
81.188167036 1.37376821041107 22.0
144.68540805 1.21875774860382 23.0
110.86618695 0.604468882083893 24.0
25.115836774 0.144828289747238 25.0
225.87357509 1.25144195556641 26.0
225.87357509 2.38513493537903 27.0
27.359391095 1.88594532012939 28.0
253.23296618 0.67900013923645 29.0
253.23296618 1.72436308860779 30.0
46.855457433 1.11956763267517 31.0
9.4598542718 0.410624980926514 32.0
20.712452196 1.73802649974823 33.0
20.712452196 0.670056462287903 34.0
20.712452196 0.118623986840248 35.0
9.0236491911 0.0239423867315054 36.0
7.7707522052 0.357538908720016 37.0
0 0.0469180047512054 38.0
0 0.161758244037628 39.0
-12.105366236 0.547232925891876 40.0
-24.31084072 0.679986774921417 41.0
5.1484821597 0.484311938285828 42.0
-31.783559694 0.457513719797134 43.0
90.330900524 0.698833644390106 44.0
90.330900524 0.0769218876957893 45.0
6.8091119644 0.610904157161713 46.0
2.4506697146 0.364872336387634 47.0
135.25208383 0.172869920730591 48.0
129.43662555 0.203154921531677 49.0
0 0.214992225170135 50.0
77.053991413 0.0680078193545341 51.0
87.006208124 0.347297817468643 52.0
78.728101354 0.0284307599067688 53.0
78.728101354 0.20063790678978 54.0
78.728101354 0.193868830800056 55.0
0 2.55879712104797 56.0
81.188167036 0.0314709916710854 57.0
225.87357509 0.61994457244873 58.0
253.23296618 0.444429129362106 59.0
20.712452196 -0.000547494739294052 60.0
0 0.0958571285009384 61.0
164.18453918 0.936814308166504 0.0
164.18453918 1.10993242263794 1.0
162.7391765 1.9505341053009 2.0
157.95010654 5.38019275665283 3.0
0 3.68916821479797 4.0
0 5.68795871734619 5.0
-58.317502608 6.53493690490723 6.0
67.558827529 0.602931678295135 7.0
52.767194723 1.07701444625854 8.0
14.791632805 1.18665301799774 9.0
99.179083688 0.925859570503235 10.0
85.965869558 0.12693578004837 11.0
87.093094866 0.292286217212677 12.0
87.093094866 0.484388560056686 13.0
87.093094866 0.0488889552652836 14.0
0 0.17592890560627 15.0
0 0.163717672228813 16.0
0 0.634542167186737 17.0
59.655902933 1.26552903652191 18.0
-59.655902933 1.23604333400726 19.0
-59.655902933 0.693248987197876 20.0
-59.655902933 -0.00362324714660645 21.0
79.862396162 1.21134912967682 22.0
135.7466714 0.951672554016113 23.0
101.1553477 0.41876694560051 24.0
23.221572885 0.0835242122411728 25.0
215.60906756 1.12334024906158 26.0
215.60906756 1.46323525905609 27.0
23.918420482 2.13357353210449 28.0
239.52748804 1.20762968063354 29.0
239.52748804 1.87066674232483 30.0
47.205011815 0.987422168254852 31.0
9.4754293984 0.32761824131012 32.0
20.206493228 1.24127197265625 33.0
20.206493228 1.10517001152039 34.0
20.206493228 0.14412647485733 35.0
8.6097657567 0.0420150607824326 36.0
7.3252279143 0.360402584075928 37.0
0 0.0206754170358181 38.0
0 0.312400847673416 39.0
-11.573167071 0.450604856014252 40.0
-24.477745786 0.544759392738342 41.0
4.5020143241 0.638727724552155 42.0
-31.509265857 0.577326714992523 43.0
82.832803347 0.520562648773193 44.0
82.832803347 0.0552549436688423 45.0
7.0703189499 0.417165666818619 46.0
2.5892531283 0.371511906385422 47.0
164.18453918 0.0847762301564217 48.0
157.95010654 0.119208842515945 49.0
0 0.326517015695572 50.0
67.558827529 0.207990363240242 51.0
99.179083688 0.0926925167441368 52.0
87.093094866 0.0201048702001572 53.0
87.093094866 0.166133984923363 54.0
87.093094866 0.163886651396751 55.0
0 5.30309391021729 56.0
79.862396162 0.242780774831772 57.0
215.60906756 0.153224289417267 58.0
239.52748804 0.868821561336517 59.0
20.206493228 -0.0201313346624374 60.0
0 0.0952668786048889 61.0
198.0431791 0.897100865840912 0.0
198.0431791 1.44491100311279 1.0
196.78552329 1.53110957145691 2.0
191.32085255 5.07080554962158 3.0
2.6440086156 6.0230712890625 4.0
2.6440086156 4.6974048614502 5.0
-62.961796487 3.67021131515503 6.0
75.465494208 0.620745539665222 7.0
44.989704479 0.695385754108429 8.0
30.475789729 0.623291969299316 9.0
94.561302692 0.892719328403473 10.0
82.325169027 0.170447960495949 11.0
79.554030735 0.411377310752869 12.0
79.554030735 0.415715277194977 13.0
79.554030735 0.085554450750351 14.0
0.18583781012 0.176607251167297 15.0
0.18583781012 0.170583337545395 16.0
0 1.02416718006134 17.0
62.115607519 1.42897641658783 18.0
-62.115607519 1.37925505638123 19.0
-62.115607519 0.502500712871552 20.0
-62.115607519 0.0176840275526047 21.0
83.332258276 1.71719133853912 22.0
153.23293324 1.83138930797577 23.0
116.29839245 0.523712277412415 24.0
22.12619849 0.170690253376961 25.0
236.56519152 1.37362289428711 26.0
236.56519152 1.66758930683136 27.0
25.547157627 1.55455100536346 28.0
262.11234914 0.874177575111389 29.0
262.11234914 1.8797687292099 30.0
46.240374366 0.991638839244843 31.0
9.3042179353 0.497281134128571 32.0
21.216650757 1.37060582637787 33.0
21.216650757 0.841391444206238 34.0
21.216650757 0.0986367613077164 35.0
9.3564223512 0.0495696365833282 36.0
8.0487976534 0.368789225816727 37.0
0 0.0395497009158134 38.0
0 0.123928047716618 39.0
-12.171680956 0.483595073223114 40.0
-25.847620235 0.619346618652344 41.0
5.1499861693 0.357328712940216 42.0
-33.411653838 0.409128457307816 43.0
88.345908306 0.661407172679901 44.0
88.345908306 0.113471105694771 45.0
6.4007068164 0.624078869819641 46.0
2.6717179116 0.403628557920456 47.0
198.0431791 0.121160894632339 48.0
191.32085255 0.119606785476208 49.0
2.6440086156 0.410790026187897 50.0
75.465494208 0.195028319954872 51.0
94.561302692 0.128617167472839 52.0
79.554030735 0.0925809666514397 53.0
79.554030735 0.185379862785339 54.0
79.554030735 0.0654421448707581 55.0
0 4.28013134002686 56.0
83.332258276 0.0614321790635586 57.0
236.56519152 0.482131749391556 58.0
262.11234914 0.465062230825424 59.0
21.216650757 -0.00440954603254795 60.0
0 0.110953249037266 61.0
169.74291588 0.734752476215363 0.0
169.74291588 1.39956152439117 1.0
168.33406439 1.75488495826721 2.0
163.54495917 4.0051326751709 3.0
0 6.23891878128052 4.0
0 4.34284067153931 5.0
-64.07103879 2.53111147880554 6.0
76.100777835 0.329715967178345 7.0
43.207120469 0.832085967063904 8.0
32.893657366 0.843600869178772 9.0
85.394343494 0.848249733448029 10.0
73.264720598 0.268565803766251 11.0
82.569881826 0.136197715997696 12.0
82.569881826 0.295662611722946 13.0
82.569881826 0.0644274950027466 14.0
0 0.0854136645793915 15.0
0 0.206962093710899 16.0
0 1.2343590259552 17.0
51.903892323 0.992172718048096 18.0
-51.903892323 0.766894280910492 19.0
-51.903892323 0.532044410705566 20.0
-51.903892323 0.022285208106041 21.0
74.247132116 1.87961900234222 22.0
142.38904375 1.75633931159973 23.0
109.30170716 0.693769454956055 24.0
21.425921048 0.122511312365532 25.0
216.63617587 0.909474432468414 26.0
216.63617587 1.33979201316833 27.0
30.137653028 1.46309971809387 28.0
246.77382889 1.05710887908936 29.0
246.77382889 1.70911037921906 30.0
47.13385684 1.13415658473969 31.0
9.295826111 0.347215414047241 32.0
22.343239794 1.37867105007172 33.0
22.343239794 0.839122235774994 34.0
22.343239794 0.122554659843445 35.0
9.8227774697 0.027610182762146 36.0
8.4571676248 0.362033307552338 37.0
0 0.0202281959354877 38.0
0 0.169546216726303 39.0
-11.298944201 0.682530105113983 40.0
-25.143557699 0.80633807182312 41.0
5.5714968792 0.446677327156067 42.0
-33.160157976 0.581135749816895 43.0
94.23471576 0.727472543716431 44.0
94.23471576 0.085391640663147 45.0
6.3328960261 0.556960940361023 46.0
2.5582215228 0.356963843107224 47.0
169.74291588 0.163174539804459 48.0
163.54495917 0.0892776846885681 49.0
0 0.48481360077858 50.0
76.100777835 0.109044544398785 51.0
85.394343494 0.578095078468323 52.0
82.569881826 0.0194390397518873 53.0
82.569881826 0.123387709259987 54.0
82.569881826 0.0976115241646767 55.0
0 4.09771203994751 56.0
74.247132116 -0.0142958201467991 57.0
216.63617587 0.378821462392807 58.0
246.77382889 1.00005161762238 59.0
22.343239794 0.0261584632098675 60.0
0 0.08930803835392 61.0
131.17790806 1.0828595161438 0.0
131.17790806 0.640953540802002 1.0
129.75677489 1.16456592082977 2.0
124.96766956 3.87532997131348 3.0
0 8.77176284790039 4.0
0 4.5034441947937 5.0
-59.464803393 0.324344843626022 6.0
68.290984795 0.600764751434326 7.0
59.249837568 0.616646111011505 8.0
9.0411472269 0.703395247459412 9.0
81.841577936 1.03129458427429 10.0
67.692609166 0.0740360617637634 11.0
78.311464457 0.112850293517113 12.0
78.311464457 0.305405616760254 13.0
78.311464457 0.153226539492607 14.0
0 0.0457800552248955 15.0
0 0.172244414687157 16.0
0 1.55811929702759 17.0
52.855065082 1.13224649429321 18.0
-52.855065082 1.44766569137573 19.0
-52.855065082 0.385235249996185 20.0
-52.855065082 0.0284545831382275 21.0
75.107355676 1.08489513397217 22.0
145.9944696 1.26941156387329 23.0
111.17579035 0.258025854825974 24.0
23.367851107 0.187672153115273 25.0
221.10182527 1.16929244995117 26.0
221.10182527 2.65847182273865 27.0
26.262095601 2.84164977073669 28.0
247.36392087 1.55956208705902 29.0
247.36392087 2.12562918663025 30.0
46.963304289 1.34466540813446 31.0
9.5133845477 0.370947778224945 32.0
22.252290594 1.21278023719788 33.0
22.252290594 1.07963669300079 34.0
22.252290594 0.110148131847382 35.0
9.9835928859 0.0281583201140165 36.0
8.6590706392 0.329469114542007 37.0
0 0.0761570781469345 38.0
0 0.344230353832245 39.0
-12.42373562 0.453204780817032 40.0
-25.752838662 0.499984383583069 41.0
5.0805196008 0.433364152908325 42.0
-33.449961725 0.549675166606903 43.0
78.559904147 0.468087434768677 44.0
78.559904147 0.0862430483102798 45.0
6.4228235826 0.491730779409409 46.0
2.3102812484 0.376830488443375 47.0
131.17790806 0.130911275744438 48.0
124.96766956 0.193240761756897 49.0
0 0.323298901319504 50.0
68.290984795 0.110028192400932 51.0
81.841577936 0.181793242692947 52.0
78.311464457 0.0443887077271938 53.0
78.311464457 0.0565371140837669 54.0
78.311464457 0.171771585941315 55.0
0 6.63462543487549 56.0
75.107355676 0.0597109943628311 57.0
221.10182527 0.78236049413681 58.0
247.36392087 0.921858012676239 59.0
22.252290594 0.00538249313831329 60.0
0 0.0897407084703445 61.0
170.94124623 1.09516870975494 0.0
170.94124623 0.909337103366852 1.0
169.57969343 1.21139168739319 2.0
164.79058815 3.02275466918945 3.0
0 4.34586095809937 4.0
0 5.4812650680542 5.0
-59.583475915 0.794635951519012 6.0
72.431267515 0.990467965602875 7.0
59.416131024 1.54752826690674 8.0
13.015136491 1.78404116630554 9.0
100.58187029 1.22249710559845 10.0
87.677023471 0.419684171676636 11.0
87.465968116 1.12604713439941 12.0
87.465968116 0.348081022500992 13.0
87.465968116 0.0878331959247589 14.0
0 0.193017363548279 15.0
0 0.218262672424316 16.0
0 1.29302763938904 17.0
59.166685588 1.10078167915344 18.0
-59.166685588 0.755248188972473 19.0
-59.166685588 1.03888535499573 20.0
-59.166685588 -0.0538821220397949 21.0
80.385672004 1.77677023410797 22.0
148.91943407 2.09550261497498 23.0
115.81590524 0.484620869159698 24.0
22.249436205 0.119609832763672 25.0
229.30510607 0.924824893474579 26.0
229.30510607 2.0289888381958 27.0
28.271706251 2.64099502563477 28.0
257.57681232 1.44589829444885 29.0
257.57681232 2.30789303779602 30.0
46.469139407 1.19224524497986 31.0
9.2145702691 0.418091088533401 32.0
21.218986416 1.44207978248596 33.0
21.218986416 0.903831601142883 34.0
21.218986416 0.126750275492668 35.0
9.3755518953 0.0269926711916924 36.0
8.1803973645 0.33296138048172 37.0
0 0.0330312997102737 38.0
0 0.140804201364517 39.0
-11.22256602 0.547366499900818 40.0
-24.226249909 0.666729748249054 41.0
5.2306470733 0.51096099615097 42.0
-32.163590061 0.642194986343384 43.0
83.26667409 0.752797544002533 44.0
83.26667409 0.0850615724921227 45.0
6.4887113912 0.689935564994812 46.0
2.3573844313 0.405241459608078 47.0
170.94124623 0.119815319776535 48.0
164.79058815 0.00466031767427921 49.0
0 0.197436362504959 50.0
72.431267515 0.477019846439362 51.0
100.58187029 0.459525853395462 52.0
87.465968116 0.121767267584801 53.0
87.465968116 0.17051337659359 54.0
87.465968116 0.0738190710544586 55.0
0 2.47899913787842 56.0
80.385672004 0.0331000909209251 57.0
229.30510607 0.258026540279388 58.0
257.57681232 1.07615077495575 59.0
21.218986416 0.00286559015512466 60.0
0 0.101255968213081 61.0
130.97884195 1.04095017910004 0.0
130.97884195 0.846101522445679 1.0
129.70669175 0.659481287002563 2.0
124.24202098 3.63214182853699 3.0
4.6078038617 4.48096323013306 4.0
4.6078038617 5.10283613204956 5.0
-67.878438194 0.771638035774231 6.0
80.65848943 0.705407023429871 7.0
47.066430871 1.11264479160309 8.0
33.592058559 1.15862309932709 9.0
103.03364956 1.04873323440552 10.0
89.735644331 0.0806362554430962 11.0
92.652205695 0.355832666158676 12.0
92.652205695 0.249797239899635 13.0
92.652205695 0.0785439014434814 14.0
0 0.225977540016174 15.0
0 0.179079055786133 16.0
0 1.11797821521759 17.0
59.381257757 0.679829657077789 18.0
-59.381257757 0.421878844499588 19.0
-59.381257757 0.234904319047928 20.0
-59.381257757 0.348028182983398 21.0
80.728660583 0.987650156021118 22.0
155.41038806 1.24311256408691 23.0
121.52805256 0.407828599214554 24.0
23.26759797 0.295191526412964 25.0
236.13904865 1.15624749660492 26.0
236.13904865 2.23722124099731 27.0
26.691836384 1.85521769523621 28.0
262.83088503 0.713422238826752 29.0
262.83088503 1.90819358825684 30.0
46.95551741 1.08017003536224 31.0
9.2986733045 0.535441637039185 32.0
21.347402826 0.484845668077469 33.0
21.347402826 1.18316102027893 34.0
21.347402826 0.182107031345367 35.0
10.117989107 0.0515442937612534 36.0
8.7925948528 0.380894303321838 37.0
0 -0.0180451944470406 38.0
0 0.161749064922333 39.0
-11.891671779 0.504807531833649 40.0
-26.849189245 0.714695155620575 41.0
4.8787109724 0.31733375787735 42.0
-34.558964821 0.259199678897858 43.0
74.011833694 0.543642282485962 44.0
74.011833694 0.10297679901123 45.0
7.8362309711 0.678802669048309 46.0
3.0189271951 0.382670760154724 47.0
130.97884195 0.0991822630167007 48.0
124.24202098 0.147506669163704 49.0
4.6078038617 0.550471901893616 50.0
80.65848943 0.167609810829163 51.0
103.03364956 0.03999188169837 52.0
92.652205695 0.0322578586637974 53.0
92.652205695 0.0643288940191269 54.0
92.652205695 0.187023088335991 55.0
0 2.39971995353699 56.0
80.728660583 0.0228934120386839 57.0
236.13904865 0.770404100418091 58.0
262.83088503 0.463903725147247 59.0
21.347402826 -0.00680324248969555 60.0
0 0.0759066119790077 61.0
194.3449343 0.750281035900116 0.0
194.3449343 1.56421291828156 1.0
193.09008819 1.45950734615326 2.0
187.62545278 2.83111906051636 3.0
13.121581277 6.03755140304565 4.0
13.121581277 3.12358713150024 5.0
-60.771823783 2.15499687194824 6.0
70.203182349 0.787913918495178 7.0
45.482905403 0.293350398540497 8.0
24.720276946 0.305066078901291 9.0
94.270698232 0.868177354335785 10.0
81.078650402 0.186220422387123 11.0
83.899319775 1.91138100624084 12.0
83.899319775 1.06553781032562 13.0
83.899319775 0.0886002406477928 14.0
0 0.250809818506241 15.0
0 0.141948729753494 16.0
0 1.57741129398346 17.0
47.560425229 0.80802196264267 18.0
-47.560425229 1.3811948299408 19.0
-47.560425229 0.725094199180603 20.0
-47.560425229 -0.045219361782074 21.0
70.088918286 1.51921474933624 22.0
135.91925591 1.39637362957001 23.0
102.49978228 0.306605368852615 24.0
23.727162011 0.265097349882126 25.0
206.0081742 0.37852269411087 26.0
206.0081742 1.16554844379425 27.0
27.497158301 0.973670482635498 28.0
233.5053325 0.461072444915771 29.0
233.5053325 1.42369961738586 30.0
46.08169845 1.05692732334137 31.0
9.2578878382 0.482118874788284 32.0
22.528493057 1.92546212673187 33.0
22.528493057 1.0036609172821 34.0
22.528493057 0.0944845378398895 35.0
10.019765498 0.0280741546303034 36.0
8.8704877671 0.431788325309753 37.0
0 0.0986258536577225 38.0
0 0.129586413502693 39.0
-13.708927095 0.837825298309326 40.0
-27.755641511 0.428152203559875 41.0
5.6434521069 0.307067722082138 42.0
-36.233938785 0.592514038085938 43.0
81.514174741 0.425288319587708 44.0
81.514174741 0.143070429563522 45.0
6.4044736817 0.465998709201813 46.0
2.1765271123 0.416344583034515 47.0
194.3449343 0.146104767918587 48.0
187.62545278 0.174167826771736 49.0
13.121581277 0.315489709377289 50.0
70.203182349 0.179280832409859 51.0
94.270698232 0.319677114486694 52.0
83.899319775 0.152161255478859 53.0
83.899319775 0.476815581321716 54.0
83.899319775 0.194908529520035 55.0
0 1.26819384098053 56.0
70.088918286 -0.00274152494966984 57.0
206.0081742 0.0132430102676153 58.0
233.5053325 0.259810149669647 59.0
22.528493057 -0.0260967034846544 60.0
0 0.0885366201400757 61.0
168.10287637 0.851527631282806 0.0
168.10287637 1.23072350025177 1.0
166.68630631 1.73677086830139 2.0
161.2216356 5.9545259475708 3.0
10.320710232 4.44486045837402 4.0
10.320710232 2.90597033500671 5.0
-65.428127699 0.573104500770569 6.0
74.740825569 0.612207114696503 7.0
48.312106113 0.459305822849274 8.0
26.428719456 0.56791365146637 9.0
82.220707689 1.02242803573608 10.0
70.373757853 0.333652943372726 11.0
80.282237539 1.01374387741089 12.0
80.282237539 0.117446452379227 13.0
80.282237539 0.136717334389687 14.0
0 0.125351667404175 15.0
0 0.143067508935928 16.0
0 1.35883188247681 17.0
57.457385876 1.4538836479187 18.0
-57.457385876 1.01900148391724 19.0
-57.457385876 0.573300063610077 20.0
-57.457385876 0.019360639154911 21.0
77.70053465 1.65374314785004 22.0
149.58460758 1.0502564907074 23.0
118.71441457 0.541534841060638 24.0
24.246563815 0.178818225860596 25.0
227.28514223 1.22780573368073 26.0
227.28514223 1.80007565021515 27.0
24.895454427 2.04898834228516 28.0
252.18059665 0.893443584442139 29.0
252.18059665 1.62512683868408 30.0
46.38254401 1.11995887756348 31.0
9.3319541446 0.476144105195999 32.0
20.243148773 1.46945667266846 33.0
20.243148773 0.853873550891876 34.0
20.243148773 0.139502167701721 35.0
8.7598610565 0.0413617119193077 36.0
7.4161916156 0.381719946861267 37.0
0 0.0452753193676472 38.0
0 0.159833520650864 39.0
-11.312817727 0.962439715862274 40.0
-23.903199282 0.616334497928619 41.0
5.127047498 0.414123445749283 42.0
-31.576180463 0.401776999235153 43.0
89.682204824 0.865848124027252 44.0
89.682204824 0.105488188564777 45.0
7.6602347919 0.601402223110199 46.0
2.6220100524 0.316504299640656 47.0
168.10287637 0.126605302095413 48.0
161.2216356 0.270308047533035 49.0
10.320710232 0.0688909962773323 50.0
74.740825569 0.057849757373333 51.0
82.220707689 0.2788425385952 52.0
80.282237539 0.16270923614502 53.0
80.282237539 0.0854035466909409 54.0
80.282237539 0.0284284148365259 55.0
0 4.67393255233765 56.0
77.70053465 0.0345265194773674 57.0
227.28514223 0.627467513084412 58.0
252.18059665 0.587988138198853 59.0
20.243148773 0.00614755041897297 60.0
0 0.108541250228882 61.0
132.82020834 1.12913060188293 0.0
132.82020834 1.03837239742279 1.0
131.82166121 1.28550672531128 2.0
127.03255587 5.95555448532104 3.0
0 8.81575107574463 4.0
0 3.31687712669373 5.0
-49.350071617 0.210338741540909 6.0
62.999004872 1.07131898403168 7.0
58.417818722 0.763516426086426 8.0
4.58118615 0.823659539222717 9.0
81.281812059 1.21132349967957 10.0
69.279129956 0.0624898448586464 11.0
87.074901486 0.883642613887787 12.0
87.074901486 0.688582122325897 13.0
87.074901486 0.0896246135234833 14.0
0 0.234113305807114 15.0
0 0.083670973777771 16.0
0 1.15914046764374 17.0
66.073223991 0.847944796085358 18.0
-66.073223991 0.551113903522491 19.0
-66.073223991 0.833557605743408 20.0
-66.073223991 0.0106840021908283 21.0
88.598076949 1.84815227985382 22.0
152.958215 1.13244080543518 23.0
119.93251571 0.479774624109268 24.0
25.948014328 0.262596756219864 25.0
241.55629195 0.889523684978485 26.0
241.55629195 1.50074577331543 27.0
22.834514659 1.56251394748688 28.0
264.3908066 1.01817548274994 29.0
264.3908066 1.68005990982056 30.0
46.968218161 1.14796483516693 31.0
9.3358339736 0.56807804107666 32.0
22.524852958 1.24894714355469 33.0
22.524852958 1.11387348175049 34.0
22.524852958 0.130441203713417 35.0
9.6647404774 0.0303698237985373 36.0
8.4191132002 0.351744115352631 37.0
0 0.0996547192335129 38.0
0 0.137900590896606 39.0
-12.490128395 0.863122642040253 40.0
-25.658860499 0.534382522106171 41.0
5.4379476797 0.64020562171936 42.0
-33.725051372 0.802610039710999 43.0
76.08823751 0.661658704280853 44.0
76.08823751 0.0910406112670898 45.0
6.7284160291 0.561947584152222 46.0
2.4691861052 0.358806729316711 47.0
132.82020834 0.151467561721802 48.0
127.03255587 0.128927707672119 49.0
0 0.545486986637115 50.0
62.999004872 0.2005854845047 51.0
81.281812059 0.0812763720750809 52.0
87.074901486 0.0442865304648876 53.0
87.074901486 0.250890851020813 54.0
87.074901486 0.0414833948016167 55.0
0 1.40099811553955 56.0
88.598076949 0.0389283560216427 57.0
241.55629195 0.304941207170486 58.0
264.3908066 0.839828372001648 59.0
22.524852958 -0.0153975272551179 60.0
0 0.0964606553316116 61.0
178.61220894 1.20425128936768 0.0
178.61220894 0.993411183357239 1.0
177.50325111 1.94279098510742 2.0
172.71414603 5.20316696166992 3.0
0 8.91235065460205 4.0
0 4.14250183105469 5.0
-74.889797907 2.90369367599487 6.0
87.404437215 0.788393795490265 7.0
53.523873812 0.953322887420654 8.0
33.880563403 0.696043133735657 9.0
87.97508438 1.23888194561005 10.0
75.706267583 0.277001917362213 11.0
78.969008158 0.565834224224091 12.0
78.969008158 0.851836681365967 13.0
78.969008158 0.103033393621445 14.0
0 0.126098990440369 15.0
0 0.213577300310135 16.0
0 1.37231338024139 17.0
61.944030133 1.23284578323364 18.0
-61.944030133 0.770994186401367 19.0
-61.944030133 0.50838303565979 20.0
-61.944030133 0.0195136778056622 21.0
83.754143884 2.0154824256897 22.0
149.74949766 2.47427439689636 23.0
114.43313287 0.238524734973907 24.0
25.34109774 0.218923881649971 25.0
233.50364155 1.18633496761322 26.0
233.50364155 1.15915846824646 27.0
29.452380998 1.32838094234467 28.0
262.95602255 0.724400460720062 29.0
262.95602255 1.58457291126251 30.0
46.971880336 0.963012278079987 31.0
9.4217505619 0.37699419260025 32.0
21.810113751 1.13339531421661 33.0
21.810113751 0.972840309143066 34.0
21.810113751 0.0883584767580032 35.0
9.366749117 0.0242623183876276 36.0
8.152426864 0.37528869509697 37.0
0 0.0431535169482231 38.0
0 0.167619079351425 39.0
-12.879183375 0.684387862682343 40.0
-24.80889647 0.526915550231934 41.0
5.5344409245 0.556693255901337 42.0
-33.144522716 0.648990094661713 43.0
102.85752031 0.774541020393372 44.0
102.85752031 0.11661221832037 45.0
7.8254493597 0.622460007667542 46.0
2.9200671223 0.387048780918121 47.0
178.61220894 0.0966584458947182 48.0
172.71414603 0.270519435405731 49.0
0 0.298108011484146 50.0
87.404437215 0.169876873493195 51.0
87.97508438 0.136476635932922 52.0
78.969008158 0.0347452610731125 53.0
78.969008158 0.252983093261719 54.0
78.969008158 0.198980644345284 55.0
0 4.87557029724121 56.0
83.754143884 0.0121256951242685 57.0
233.50364155 0.262143075466156 58.0
262.95602255 0.30643966794014 59.0
21.810113751 0.0312700010836124 60.0
0 0.0806915685534477 61.0
194.04516215 0.810178399085999 0.0
194.04516215 1.6790896654129 1.0
192.74481146 2.26451563835144 2.0
191.64831397 3.92767095565796 3.0
0 4.29097938537598 4.0
0 5.03911781311035 5.0
-77.030202976 1.7036384344101 6.0
87.410993461 0.665039658546448 7.0
66.835520749 0.988170266151428 8.0
20.575472711 0.937873303890228 9.0
67.916649612 0.82853102684021 10.0
54.050331853 0.208919048309326 11.0
55.524452247 1.1912704706192 12.0
55.524452247 1.35661351680756 13.0
55.524452247 0.143429815769196 14.0
19.832536234 0.180554166436195 15.0
19.832536234 0.151417970657349 16.0
0 1.84488880634308 17.0
53.110522727 1.2354052066803 18.0
-53.110522727 0.717138350009918 19.0
-53.110522727 1.06714463233948 20.0
-53.110522727 -0.00863609462976456 21.0
75.161604702 2.62094163894653 22.0
145.62350211 2.29482913017273 23.0
108.65474201 0.29159015417099 24.0
23.608739941 0.218809008598328 25.0
220.78510681 1.59615755081177 26.0
220.78510681 2.07752966880798 27.0
23.173427597 1.5190840959549 28.0
243.9585344 0.537162303924561 29.0
243.9585344 2.03802418708801 30.0
47.389766483 1.17656540870667 31.0
9.3355127317 0.649496495723724 32.0
22.051081975 1.38851320743561 33.0
22.051081975 0.490901619195938 34.0
22.051081975 0.135123401880264 35.0
9.2745920513 0.0344589538872242 36.0
8.0661779217 0.35778534412384 37.0
0 0.0949343368411064 38.0
0 0.119085170328617 39.0
-11.425336597 1.08631455898285 40.0
-24.241261239 0.668462038040161 41.0
5.2685461967 0.38000351190567 42.0
-32.085218702 0.51359486579895 43.0
87.601726722 0.687809348106384 44.0
87.601726722 0.0825933367013931 45.0
7.353007147 0.342798799276352 46.0
3.0065010201 0.296153545379639 47.0
194.04516215 0.00359538942575455 48.0
191.64831397 0.178803384304047 49.0
0 0.109569862484932 50.0
87.410993461 0.286893099546432 51.0
67.916649612 0.255446910858154 52.0
55.524452247 0.0990056321024895 53.0
55.524452247 0.450990915298462 54.0
55.524452247 0.50464916229248 55.0
0 6.38775777816772 56.0
75.161604702 -0.0124266790226102 57.0
220.78510681 0.843958675861359 58.0
243.9585344 0.586226642131805 59.0
22.051081975 0.0301082544028759 60.0
0 0.0831093788146973 61.0
169.87465159 1.15933334827423 0.0
169.87465159 0.812752604484558 1.0
168.53192033 1.19649648666382 2.0
163.06724977 5.31183528900146 3.0
0.28556679045 5.55160474777222 4.0
0.28556679045 2.69121670722961 5.0
-57.996835693 0.739757418632507 6.0
70.670724862 0.490617096424103 7.0
39.830456478 0.548486411571503 8.0
30.840268383 0.602817535400391 9.0
85.709553652 1.12164282798767 10.0
72.209025097 0.0877293720841408 11.0
79.049395819 0.109669521450996 12.0
79.049395819 0.42531031370163 13.0
79.049395819 0.0930986851453781 14.0
0 0.196521565318108 15.0
0 0.204863578081131 16.0
0 1.17549228668213 17.0
53.839581599 1.02470183372498 18.0
-53.839581599 1.02369809150696 19.0
-53.839581599 0.338088691234589 20.0
-53.839581599 0.0142074152827263 21.0
75.290834493 1.09767520427704 22.0
147.30810946 0.673546195030212 23.0
113.0187701 0.513327538967133 24.0
20.592393094 0.271093189716339 25.0
222.59894395 1.52099585533142 26.0
222.59894395 2.34017109870911 27.0
26.054797229 1.16410553455353 28.0
248.65374118 0.454369723796844 29.0
248.65374118 1.74569296836853 30.0
46.606694944 1.00372922420502 31.0
9.4621261326 0.491389721632004 32.0
21.451252894 1.03438687324524 33.0
21.451252894 0.920874178409576 34.0
21.451252894 0.146463871002197 35.0
8.4541942896 0.0280085932463408 36.0
7.2486868044 0.328188359737396 37.0
0 0.0457744039595127 38.0
0 0.233490779995918 39.0
-11.7006798 0.764001071453094 40.0
-24.041171928 0.552068471908569 41.0
5.4142433217 0.583531737327576 42.0
-31.77165445 0.720049798488617 43.0
103.69180757 0.737975895404816 44.0
103.69180757 0.0875820219516754 45.0
7.0067084999 0.617926239967346 46.0
2.7643244544 0.418720901012421 47.0
169.87465159 0.12203523516655 48.0
163.06724977 0.101560302078724 49.0
0.28556679045 0.125365197658539 50.0
70.670724862 0.0937177836894989 51.0
85.709553652 0.0658359304070473 52.0
79.049395819 0.0298253744840622 53.0
79.049395819 0.12883722782135 54.0
79.049395819 0.0733601450920105 55.0
0 2.46373081207275 56.0
75.290834493 0.145760014653206 57.0
222.59894395 1.09951949119568 58.0
248.65374118 0.69823169708252 59.0
21.451252894 -0.000964021310210228 60.0
0 0.0945694670081139 61.0
136.67440061 0.953287482261658 0.0
136.67440061 1.02224802970886 1.0
135.65114677 1.4758129119873 2.0
130.86204152 4.64457416534424 3.0
0 5.52338838577271 4.0
0 4.79151678085327 5.0
-55.174259624 3.70511031150818 6.0
66.651083942 0.855276525020599 7.0
61.977826289 1.36143267154694 8.0
4.6732576535 1.56148016452789 9.0
79.450098098 0.987060904502869 10.0
66.66576813 0.084471121430397 11.0
76.683662396 0.213224679231644 12.0
76.683662396 0.899130165576935 13.0
76.683662396 0.15827864408493 14.0
0 0.105039268732071 15.0
0 0.187481433153152 16.0
0 1.89028120040894 17.0
56.923981615 1.53520607948303 18.0
-56.923981615 1.18806171417236 19.0
-56.923981615 0.259877860546112 20.0
-56.923981615 0.0246180556714535 21.0
77.631414554 1.55648446083069 22.0
141.57050038 1.51737785339355 23.0
106.06795854 0.589535713195801 24.0
21.561504607 0.173519715666771 25.0
219.20191494 0.892382502555847 26.0
219.20191494 1.34304535388947 27.0
24.015919347 1.78428101539612 28.0
243.21783428 0.662899315357208 29.0
243.21783428 1.58904206752777 30.0
46.748060241 1.01028478145599 31.0
9.2797838078 0.475293725728989 32.0
20.707432939 1.31247913837433 33.0
20.707432939 1.13837623596191 34.0
20.707432939 0.117570705711842 35.0
9.3727218291 0.031791053712368 36.0
8.1416023148 0.339092314243317 37.0
0 0.0148403570055962 38.0
0 0.229562908411026 39.0
-11.883033263 0.466486394405365 40.0
-24.697531544 0.73164838552475 41.0
5.1517550272 0.458687543869019 42.0
-32.526995204 0.449780762195587 43.0
95.409007362 0.769720792770386 44.0
95.409007362 0.110288083553314 45.0
6.576341723 0.740400433540344 46.0
2.482259395 0.410789281129837 47.0
136.67440061 0.0979990065097809 48.0
130.86204152 0.164516717195511 49.0
0 0.345897316932678 50.0
66.651083942 0.279571145772934 51.0
79.450098098 0.123396098613739 52.0
76.683662396 0.0324817895889282 53.0
76.683662396 0.214794293045998 54.0
76.683662396 0.296310096979141 55.0
0 4.75521802902222 56.0
77.631414554 0.120334573090076 57.0
219.20191494 0.184942826628685 58.0
243.21783428 0.298900872468948 59.0
20.707432939 0.0152270756661892 60.0
0 0.103026084601879 61.0
194.01630326 0.978266000747681 0.0
194.01630326 1.04157078266144 1.0
192.78895624 1.98442566394806 2.0
187.9998727 5.39200162887573 3.0
0 6.6168909072876 4.0
0 3.37806701660156 5.0
-73.099421605 0.343486577272415 6.0
83.039871917 1.05383205413818 7.0
58.749685457 1.0863094329834 8.0
24.29018646 1.15740251541138 9.0
87.744324746 0.931454241275787 10.0
76.038805209 0.0687994509935379 11.0
81.095887441 0.790906310081482 12.0
81.095887441 0.424177080392838 13.0
81.095887441 0.0345649681985378 14.0
0 0.102676086127758 15.0
0 0.238429889082909 16.0
0 1.09040701389313 17.0
59.404408486 0.944027900695801 18.0
-59.404408486 0.977599501609802 19.0
-59.404408486 0.634504377841949 20.0
-59.404408486 -0.0714278593659401 21.0
79.59597614 2.50264501571655 22.0
144.57599589 2.11251187324524 23.0
111.2481026 0.347487151622772 24.0
23.735163537 0.211641922593117 25.0
224.17197203 1.41259336471558 26.0
224.17197203 1.68129765987396 27.0
24.401946305 1.98163199424744 28.0
248.57391834 0.726604342460632 29.0
248.57391834 2.15249633789062 30.0
46.783252688 1.28197574615479 31.0
9.4530179497 0.487065345048904 32.0
20.191567655 1.92639768123627 33.0
20.191567655 1.04082489013672 34.0
20.191567655 0.0899992659687996 35.0
9.138927138 0.0279931649565697 36.0
7.9180716124 0.371428310871124 37.0
0 0.0417569205164909 38.0
0 0.252377808094025 39.0
-11.687856876 0.815427720546722 40.0
-23.243025896 0.457303106784821 41.0
4.9531106467 0.568712651729584 42.0
-30.706213212 0.475106179714203 43.0
88.892867425 0.820025622844696 44.0
88.892867425 0.0569238439202309 45.0
7.3293493801 0.667213559150696 46.0
2.8440453543 0.406047105789185 47.0
194.01630326 0.157081261277199 48.0
187.9998727 0.466119408607483 49.0
0 0.291159182786942 50.0
83.039871917 0.293527901172638 51.0
87.744324746 0.0492858923971653 52.0
81.095887441 0.0657199993729591 53.0
81.095887441 0.182919442653656 54.0
81.095887441 0.11425919085741 55.0
0 3.06045341491699 56.0
79.59597614 0.183674782514572 57.0
224.17197203 0.354507803916931 58.0
248.57391834 0.583172798156738 59.0
20.191567655 -0.0160197280347347 60.0
0 0.105707623064518 61.0
151.47403726 1.06048691272736 0.0
151.47403726 0.725010812282562 1.0
149.97248775 1.09256100654602 2.0
144.50781717 3.05757665634155 3.0
12.014204067 5.55192232131958 4.0
12.014204067 5.46199941635132 5.0
-60.214797173 1.5130455493927 6.0
70.295560771 0.862174987792969 7.0
43.09602816 1.08862507343292 8.0
27.199532611 1.13208675384521 9.0
97.023380939 1.08657848834991 10.0
84.99893732 0.0617572665214539 11.0
86.906296252 0.555119216442108 12.0
86.906296252 0.481539905071259 13.0
86.906296252 0.0738843083381653 14.0
0 0.218857079744339 15.0
0 0.165649995207787 16.0
0 0.923540115356445 17.0
54.237735185 1.19519007205963 18.0
-54.237735185 1.1640567779541 19.0
-54.237735185 0.922239482402802 20.0
-54.237735185 -0.00680940225720406 21.0
76.02793971 2.03423452377319 22.0
154.1078233 1.70860683917999 23.0
118.85163538 0.308245837688446 24.0
22.307568525 0.274480640888214 25.0
230.13576301 1.91842198371887 26.0
230.13576301 2.54142832756042 27.0
25.450269965 1.79358816146851 28.0
255.58603297 0.745010733604431 29.0
255.58603297 1.5132155418396 30.0
47.318307816 0.970113396644592 31.0
9.4297551447 0.334726482629776 32.0
21.790204525 1.37533283233643 33.0
21.790204525 1.070876121521 34.0
21.790204525 0.11218997836113 35.0
9.268480918 0.0366179198026657 36.0
8.0128095438 0.387817591428757 37.0
0 0.0271999575197697 38.0
0 0.190771862864494 39.0
-11.567588566 0.876041114330292 40.0
-25.117623399 0.563664078712463 41.0
5.2038470441 0.44531387090683 42.0
-32.90908616 0.484129786491394 43.0
92.363589981 0.796689927577972 44.0
92.363589981 0.0892692431807518 45.0
6.3311303867 0.664997816085815 46.0
2.4813875976 0.390319526195526 47.0
151.47403726 0.131703808903694 48.0
144.50781717 0.230541616678238 49.0
12.014204067 0.485972851514816 50.0
70.295560771 0.413293838500977 51.0
97.023380939 0.0663715153932571 52.0
86.906296252 0.0311850253492594 53.0
86.906296252 0.0585749857127666 54.0
86.906296252 0.193000644445419 55.0
0 3.3463020324707 56.0
76.02793971 -0.0366348996758461 57.0
230.13576301 1.15932309627533 58.0
255.58603297 0.266491919755936 59.0
21.790204525 -0.0101131768897176 60.0
0 0.11147828400135 61.0
159.63560561 0.738310933113098 0.0
159.63560561 0.482272535562515 1.0
158.12070565 1.05562138557434 2.0
153.33160048 5.89343214035034 3.0
0 3.8754620552063 4.0
0 3.34916877746582 5.0
-50.035947194 1.05269074440002 6.0
62.236808558 1.19888234138489 7.0
44.487598573 0.692131817340851 8.0
17.749209985 0.478884220123291 9.0
81.106031573 0.85296243429184 10.0
68.506615359 0.183570668101311 11.0
72.708620561 0.910019934177399 12.0
72.708620561 0.249768972396851 13.0
72.708620561 0.0970322862267494 14.0
0 0.0364463068544865 15.0
0 0.160161688923836 16.0
0 1.37508893013 17.0
58.611431831 1.29224264621735 18.0
-58.611431831 1.2678918838501 19.0
-58.611431831 0.249423980712891 20.0
-58.611431831 0.136702999472618 21.0
79.623578593 2.47653341293335 22.0
147.29913143 2.25681734085083 23.0
108.8580836 0.281829833984375 24.0
25.688491887 0.20675203204155 25.0
226.92271003 1.37681066989899 26.0
226.92271003 1.88047468662262 27.0
22.936237236 2.23007488250732 28.0
249.85894726 1.32295417785645 29.0
249.85894726 1.76209890842438 30.0
46.571556383 1.09701108932495 31.0
9.3676625108 0.586342871189117 32.0
21.012146763 1.04097461700439 33.0
21.012146763 1.22289836406708 34.0
21.012146763 0.106005303561687 35.0
8.8780112296 0.025399511680007 36.0
7.5606877595 0.406732231378555 37.0
0 0.0338605865836143 38.0
0 0.159798249602318 39.0
-10.273799261 1.16435253620148 40.0
-22.387198945 0.213654860854149 41.0
5.0512561596 0.437715440988541 42.0
-29.950485201 0.370348751544952 43.0
93.057911001 0.716163456439972 44.0
93.057911001 0.0816242471337318 45.0
7.0362045844 0.43262454867363 46.0
2.4636955011 0.337855100631714 47.0
159.63560561 0.0373626090586185 48.0
153.33160048 0.360487878322601 49.0
0 0.377743124961853 50.0
62.236808558 0.194340690970421 51.0
81.106031573 0.394798547029495 52.0
72.708620561 0.0609808750450611 53.0
72.708620561 0.112796917557716 54.0
72.708620561 0.0789688974618912 55.0
0 5.28570032119751 56.0
79.623578593 -0.00917169079184532 57.0
226.92271003 0.208923846483231 58.0
249.85894726 0.677987933158875 59.0
21.012146763 0.00988351553678513 60.0
0 0.101410612463951 61.0
201.2291511 0.949081540107727 0.0
201.2291511 0.79870617389679 1.0
199.93349641 1.06482672691345 2.0
194.468861 3.72624015808105 3.0
1.4572333722 4.46509027481079 4.0
1.4572333722 3.85849404335022 5.0
-70.195872593 1.91775929927826 6.0
82.215804766 0.591025769710541 7.0
52.922370391 0.654613018035889 8.0
29.293434375 0.674801468849182 9.0
80.246649543 1.06763219833374 10.0
67.123389591 0.266760855913162 11.0
75.574849387 0.851404070854187 12.0
75.574849387 1.06023859977722 13.0
75.574849387 0.135557234287262 14.0
0 0.131536602973938 15.0
0 0.178717643022537 16.0
0 1.33044600486755 17.0
48.480157976 1.38090145587921 18.0
-48.480157976 1.67376053333282 19.0
-48.480157976 0.404395878314972 20.0
-48.480157976 0.0491186641156673 21.0
69.163607651 1.7654150724411 22.0
153.03451616 1.39426600933075 23.0
115.90890758 0.16075037419796 24.0
25.25062989 0.32405287027359 25.0
222.19812381 1.71606278419495 26.0
222.19812381 2.38177585601807 27.0
25.204815409 2.38853216171265 28.0
247.40293922 1.5521787405014 29.0
247.40293922 1.95475089550018 30.0
45.9356807 0.962230145931244 31.0
9.2040510541 0.471307843923569 32.0
20.683449675 1.2298196554184 33.0
20.683449675 1.2070609331131 34.0
20.683449675 0.0850506722927094 35.0
8.9088636602 0.0245058760046959 36.0
7.7174028376 0.401760995388031 37.0
0 0.0348006673157215 38.0
0 0.233351275324821 39.0
-11.915684173 0.482435911893845 40.0
-25.295261144 0.67241382598877 41.0
4.7930628482 0.469155848026276 42.0
-32.65032019 0.576976716518402 43.0
83.302993261 0.554182708263397 44.0
83.302993261 0.103093691170216 45.0
6.130125947 0.452367633581161 46.0
1.1336752605 0.310469925403595 47.0
201.2291511 0.0204101335257292 48.0
194.468861 0.174848407506943 49.0
1.4572333722 0.122624799609184 50.0
82.215804766 0.11803200095892 51.0
80.246649543 0.322223842144012 52.0
75.574849387 0.0710030943155289 53.0
75.574849387 0.180815026164055 54.0
75.574849387 0.285427898168564 55.0
0 3.90267395973206 56.0
69.163607651 0.08417609333992 57.0
222.19812381 0.919435918331146 58.0
247.40293922 1.10006177425385 59.0
20.683449675 -0.0176965408027172 60.0
0 0.090124249458313 61.0
190.44737755 0.709740996360779 0.0
190.44737755 0.998503446578979 1.0
189.36661852 0.657386541366577 2.0
183.90194776 2.65705800056458 3.0
7.7396192502 5.13294839859009 4.0
7.7396192502 3.51829409599304 5.0
-54.828504008 0.71217006444931 6.0
65.594326225 0.457711517810822 7.0
38.655734143 0.798562526702881 8.0
26.938592082 0.684650242328644 9.0
74.638381435 0.837779581546783 10.0
61.977192613 0.35702520608902 11.0
63.251450043 0.404596865177155 12.0
63.251450043 0.218513578176498 13.0
63.251450043 0.0643220096826553 14.0
18.43676522 0.0163863822817802 15.0
18.43676522 0.189592719078064 16.0
0 1.08620059490204 17.0
58.914471085 1.53162121772766 18.0
-58.914471085 1.62357604503632 19.0
-58.914471085 0.404217720031738 20.0
-58.914471085 -0.0507711581885815 21.0
81.643069453 1.21822416782379 22.0
137.79626164 1.35569858551025 23.0
101.60136342 0.54196560382843 24.0
21.435472949 0.078153096139431 25.0
219.43933109 0.679675579071045 26.0
219.43933109 2.2941882610321 27.0
30.609388884 2.79415416717529 28.0
250.04871997 1.27272808551788 29.0
250.04871997 1.80767977237701 30.0
46.041921164 1.1710262298584 31.0
9.2788336821 0.605833530426025 32.0
22.728598368 1.30293691158295 33.0
22.728598368 0.985834777355194 34.0
22.728598368 0.150000035762787 35.0
9.1260407041 0.0276301130652428 36.0
7.8288697765 0.334458023309708 37.0
0 0.000262174755334854 38.0
0 0.107293285429478 39.0
-11.972556854 0.505779027938843 40.0
-25.122032943 0.889462471008301 41.0
5.2035404041 0.424558877944946 42.0
-33.066629155 0.342395931482315 43.0
85.157698704 0.811212420463562 44.0
85.157698704 0.094290092587471 45.0
7.2025807605 0.754161357879639 46.0
2.5294550204 0.439252316951752 47.0
190.44737755 0.124554120004177 48.0
183.90194776 0.135884344577789 49.0
7.7396192502 0.179972574114799 50.0
65.594326225 0.0733848363161087 51.0
74.638381435 0.492531627416611 52.0
63.251450043 0.0605998300015926 53.0
63.251450043 0.0430903770029545 54.0
63.251450043 0.0981000512838364 55.0
0 4.49013566970825 56.0
81.643069453 0.134309351444244 57.0
219.43933109 0.104663521051407 58.0
250.04871997 0.379026889801025 59.0
22.728598368 0.0135817425325513 60.0
0 0.127806633710861 61.0
135.08361409 0.882853925228119 0.0
135.08361409 1.35456907749176 1.0
134.08565005 2.19809246063232 2.0
129.29654474 4.74699592590332 3.0
0 3.85100317001343 4.0
0 3.57819867134094 5.0
-62.635662792 0.107490539550781 6.0
72.334555281 1.21569406986237 7.0
48.83381263 0.37613433599472 8.0
23.50074265 0.330581963062286 9.0
85.898147421 1.2216637134552 10.0
72.282536213 0.755715489387512 11.0
71.348179715 0.128798216581345 12.0
71.348179715 0.150762915611267 13.0
71.348179715 0.109617188572884 14.0
0 0.0629705935716629 15.0
0 0.17061248421669 16.0
0 1.08655631542206 17.0
50.330165521 1.58613073825836 18.0
-50.330165521 1.69015741348267 19.0
-50.330165521 0.280541956424713 20.0
-50.330165521 0.199618965387344 21.0
71.340865196 2.36657738685608 22.0
136.4611225 2.40050840377808 23.0
101.23621534 0.244601428508759 24.0
23.22429519 0.195389702916145 25.0
207.80198769 1.34373354911804 26.0
207.80198769 1.69798707962036 27.0
23.095749554 1.92312502861023 28.0
230.89773725 0.555768609046936 29.0
230.89773725 1.73550760746002 30.0
47.239440596 1.03771495819092 31.0
9.465177316 0.523212194442749 32.0
21.010699675 0.95309591293335 33.0
21.010699675 0.912652492523193 34.0
21.010699675 0.186789989471436 35.0
10.508626595 0.0351891592144966 36.0
9.0358461149 0.277246326208115 37.0
0 0.0182573013007641 38.0
0 0.053129106760025 39.0
-12.653711931 0.728703737258911 40.0
-26.407302685 0.438784390687943 41.0
5.1994627445 0.442628443241119 42.0
-33.994646315 0.397202521562576 43.0
81.884521187 0.769311130046844 44.0
81.884521187 0.0913006141781807 45.0
7.3602091885 0.666578650474548 46.0
2.4486460966 0.432193875312805 47.0
135.08361409 0.0947350189089775 48.0
129.29654474 0.89413583278656 49.0
0 0.222058743238449 50.0
72.334555281 0.160300776362419 51.0
85.898147421 0.683842122554779 52.0
71.348179715 0.0865447074174881 53.0
71.348179715 0.0857808887958527 54.0
71.348179715 0.0715809389948845 55.0
0 6.4199743270874 56.0
71.340865196 -0.0339142270386219 57.0
207.80198769 0.607431590557098 58.0
230.89773725 0.708949267864227 59.0
21.010699675 -0.00548635236918926 60.0
0 0.136132925748825 61.0
182.4505363 0.769860088825226 0.0
182.4505363 1.93371176719666 1.0
180.88774614 1.83042240142822 2.0
176.09864089 3.34689879417419 3.0
0 4.58690929412842 4.0
0 3.72758150100708 5.0
-72.842562688 0.855549395084381 6.0
84.613507607 0.469013601541519 7.0
57.913373133 0.615585923194885 8.0
26.700134474 0.75505393743515 9.0
79.726219597 0.785032451152802 10.0
65.29982922 0.129652053117752 11.0
80.874834192 0.771391570568085 12.0
80.874834192 0.768553137779236 13.0
80.874834192 0.100784197449684 14.0
0 0.0538469031453133 15.0
0 0.217898488044739 16.0
0 1.57336664199829 17.0
55.338725651 1.98177826404572 18.0
-55.338725651 2.50383615493774 19.0
-55.338725651 0.291012078523636 20.0
-55.338725651 0.0143406018614769 21.0
76.320741669 1.91773951053619 22.0
144.65621649 1.7967357635498 23.0
108.55118447 0.34604400396347 24.0
18.245383674 0.0888664722442627 25.0
220.97695816 1.27439141273499 26.0
220.97695816 1.58962285518646 27.0
29.257225629 1.3389447927475 28.0
250.23418379 0.658931016921997 29.0
250.23418379 1.87197613716125 30.0
46.534206845 1.1430811882019 31.0
9.4754150046 0.579009056091309 32.0
20.982016018 0.892897367477417 33.0
20.982016018 1.02971565723419 34.0
20.982016018 0.178005829453468 35.0
9.2884737108 0.0328080207109451 36.0
8.0320857854 0.377956330776215 37.0
0 0.0269741900265217 38.0
0 0.0981882512569427 39.0
-13.011043123 1.046541929245 40.0
-26.909779398 0.690718829631805 41.0
4.9153517307 0.450989931821823 42.0
-34.891488447 0.603914499282837 43.0
93.623812597 1.04421973228455 44.0
93.623812597 0.0811693519353867 45.0
6.010971917 0.836414694786072 46.0
1.4019467248 0.443266689777374 47.0
182.4505363 0.115506261587143 48.0
176.09864089 0.214479997754097 49.0
0 0.0668940320611 50.0
84.613507607 0.0578225292265415 51.0
79.726219597 0.378319978713989 52.0
80.874834192 -0.000116435854579322 53.0
80.874834192 0.185212463140488 54.0
80.874834192 0.00320149399340153 55.0
0 5.21025323867798 56.0
76.320741669 0.166021972894669 57.0
220.97695816 0.880774855613708 58.0
250.23418379 0.657673537731171 59.0
20.982016018 -0.000204259529709816 60.0
0 0.115741237998009 61.0
163.45789993 1.01955652236938 0.0
163.45789993 1.00017130374908 1.0
162.12623276 1.04945540428162 2.0
157.33712977 2.90601396560669 3.0
0 5.34736776351929 4.0
0 3.96579575538635 5.0
-56.528878749 3.25375175476074 6.0
65.741269469 0.405179589986801 7.0
43.537075382 1.05336308479309 8.0
22.204194087 1.19590127468109 9.0
90.214592464 1.06889533996582 10.0
78.546958351 0.0669704675674438 11.0
88.383334295 0.539367377758026 12.0
88.383334295 0.148245066404343 13.0
88.383334295 0.109075665473938 14.0
0 0.201108738780022 15.0
0 0.203736171126366 16.0
0 1.28457856178284 17.0
59.210619007 1.28622233867645 18.0
-59.210619007 1.25617384910583 19.0
-59.210619007 0.792210042476654 20.0
-59.210619007 -0.0123898889869452 21.0
83.033920066 2.27052855491638 22.0
139.46095298 1.64969384670258 23.0
108.12978208 0.506036400794983 24.0
23.261496084 0.0952980369329453 25.0
222.49487304 1.54655003547668 26.0
222.49487304 1.87941646575928 27.0
23.338841106 1.05468738079071 28.0
245.83371415 0.493580281734467 29.0
245.83371415 1.302037358284 30.0
46.528697395 1.13061428070068 31.0
9.4059048474 0.353311359882355 32.0
23.823301059 1.50693535804749 33.0
23.823301059 1.07263875007629 34.0
23.823301059 0.0598609670996666 35.0
10.248737632 0.0255533214658499 36.0
9.0800105614 0.405374705791473 37.0
0 0.0475595369935036 38.0
0 0.217784479260445 39.0
-11.953785224 0.65858256816864 40.0
-25.489501509 0.356694221496582 41.0
5.305619635 0.492775082588196 42.0
-33.400586092 0.476681232452393 43.0
89.452712216 0.455234736204147 44.0
89.452712216 0.104670271277428 45.0
6.6012797785 0.435902327299118 46.0
2.821044988 0.391271948814392 47.0
163.45789993 0.133283406496048 48.0
157.33712977 0.054953970015049 49.0
0 0.187832355499268 50.0
65.741269469 0.140286043286324 51.0
90.214592464 0.140016332268715 52.0
88.383334295 0.033205833286047 53.0
88.383334295 0.0909455940127373 54.0
88.383334295 0.0837825983762741 55.0
0 3.46953153610229 56.0
83.033920066 0.103605128824711 57.0
222.49487304 0.57469230890274 58.0
245.83371415 0.033171758055687 59.0
23.823301059 -0.00665026716887951 60.0
0 0.106808722019196 61.0
};
\addlegendentry{$R^2$=-0.588}
\end{axis}

\end{tikzpicture}
}} 
    \caption{Model results using only the loss associated with nodal flow predictions in the colombian 63-node network.}
    \label{fig:lineal_col_base_results}
\end{figure}


For the current experiment stage, involving the 63-node Colombian network, a t-test was conducted to compare the computation times between the optimization model and the GNN-based model. The test yielded a T-statistic of $47.29$, 198 degrees of freedom and a p-value of $4.92 \times 10^{-110}$, demonstrating a highly significant difference in the computational times. As with the previous experiments, these results confirm that the GNN-based model significantly outperformed the optimization model in terms of prediction speed.


In the next stage of this case study, predictions were made on the same system, but this time, considering not only the losses associated with the flows injected at the nodes but also those corresponding to the flows transported by the system's pipelines and compressors. For this experiment, the hyperparameter tuning resulted in $N \ channels = 25$, $N \ layers = 5$, and $N \ dense = 4$, with a total loss of 34.399. Of this, 10.792 were associated with the predictions of the flows injected at the nodes, while 23,607 corresponded to the flows transported through the edges. The MSE for the gas balance was 279.754.

\Cref{fig:yn_col_base_f} illustrates that, as in the previous case, the network can predict which nodes can inject natural gas and demonstrate a certain degree of accuracy regarding their injection capacity. The critical difference in this test, compared to the previous one, is shown in \cref{fig:ye_col_base_f}. Here, it is evident that by including the losses associated with the edges of the Colombian transportation network, the model significantly improves its predictions for these elements, achieving an $R^2$ value of 0.996 compared to the actual data.


\begin{figure}
    \centering
        \setlength\figurewidth{.53\textwidth}        
        \setlength\figureheight{0.36\textwidth} 
        \subfloat [Actual vs predicted nodal flows.] {\label{fig:yn_col_base_f}\resizebox{\figurewidth}{\figureheight}{% This file was created with tikzplotlib v0.10.1.
\begin{tikzpicture}

\definecolor{darkgray176}{RGB}{176,176,176}
\definecolor{lightgray204}{RGB}{204,204,204}

\begin{axis}[
colorbar,
colorbar style={ylabel={node id}},
colormap={mymap}{[1pt]
 rgb(0pt)=(0.12156862745098,0.466666666666667,0.705882352941177);
  rgb(1pt)=(1,0.498039215686275,0.0549019607843137);
  rgb(2pt)=(0.172549019607843,0.627450980392157,0.172549019607843);
  rgb(3pt)=(0.83921568627451,0.152941176470588,0.156862745098039);
  rgb(4pt)=(0.580392156862745,0.403921568627451,0.741176470588235);
  rgb(5pt)=(0.549019607843137,0.337254901960784,0.294117647058824);
  rgb(6pt)=(0.890196078431372,0.466666666666667,0.76078431372549);
  rgb(7pt)=(0.498039215686275,0.498039215686275,0.498039215686275);
  rgb(8pt)=(0.737254901960784,0.741176470588235,0.133333333333333);
  rgb(9pt)=(0.0901960784313725,0.745098039215686,0.811764705882353)
},
legend cell align={left},
legend style={
  fill opacity=0.8,
  draw opacity=1,
  text opacity=1,
  at={(0.03,0.97)},
  anchor=north west,
  draw=lightgray204
},
point meta max=62,
point meta min=0,
tick align=outside,
tick pos=left,
title={},
x grid style={darkgray176},
xlabel={yn test},
xmajorgrids,
xmin=-16.2727338975, xmax=341.7274118475,
xtick style={color=black},
y grid style={darkgray176},
ylabel={y pred},
ymajorgrids,
ymin=-17.2115550398827, ymax=315.543056285381,
ytick style={color=black}
]
\addplot [
  colormap={mymap}{[1pt]
 rgb(0pt)=(0.12156862745098,0.466666666666667,0.705882352941177);
  rgb(1pt)=(1,0.498039215686275,0.0549019607843137);
  rgb(2pt)=(0.172549019607843,0.627450980392157,0.172549019607843);
  rgb(3pt)=(0.83921568627451,0.152941176470588,0.156862745098039);
  rgb(4pt)=(0.580392156862745,0.403921568627451,0.741176470588235);
  rgb(5pt)=(0.549019607843137,0.337254901960784,0.294117647058824);
  rgb(6pt)=(0.890196078431372,0.466666666666667,0.76078431372549);
  rgb(7pt)=(0.498039215686275,0.498039215686275,0.498039215686275);
  rgb(8pt)=(0.737254901960784,0.741176470588235,0.133333333333333);
  rgb(9pt)=(0.0901960784313725,0.745098039215686,0.811764705882353)
},
  only marks,
  scatter,
  scatter src=explicit
]
table [x=x, y=y, meta=colordata]{%
x  y  colordata
266.47324716 249.639175415039 0.0
0 -0.142389416694641 1.0
0 -0.0449339151382446 2.0
0 0.0713010430335999 3.0
0 0.106789827346802 4.0
0 0.236347138881683 5.0
0 0.171468198299408 6.0
0 0.321928083896637 7.0
0 -0.2691330909729 8.0
0 -0.114482879638672 9.0
0 0.585927248001099 10.0
0 -0.408090472221375 11.0
0 0.0334172248840332 12.0
54.420842408 52.3760604858398 13.0
31.76838707 45.4420013427734 14.0
0 -0.236499667167664 15.0
0 0.557169258594513 16.0
31.767708115 35.5617408752441 17.0
0 0.1447514295578 18.0
0 0.00425702333450317 19.0
0 0.0630048513412476 20.0
0 0.0852915048599243 21.0
0 0.153019845485687 22.0
32.042735172 31.7617301940918 23.0
0 0.0172544717788696 24.0
0 0.539882898330688 25.0
0 -0.139046311378479 26.0
0 -0.016973078250885 27.0
0 0.361409664154053 28.0
0 0.0465011596679688 29.0
0 -0.0713483095169067 30.0
0 -0.151389122009277 31.0
0 0.0655762553215027 32.0
0 -0.184005737304688 33.0
0 0.159492611885071 34.0
0 0.411568343639374 35.0
0 -0.149775922298431 36.0
0 0.0114204883575439 37.0
0 0.126012444496155 38.0
0 -0.194342374801636 39.0
0 0.582290053367615 40.0
0 -0.129912436008453 41.0
0 0.131481289863586 42.0
293.56654935 296.294281005859 43.0
0 0.0724555253982544 44.0
0 0.070831298828125 45.0
0 0.129652857780457 46.0
0 0.235148727893829 47.0
0 0.383313328027725 48.0
0 0.428795397281647 49.0
0.088502709347 0.000664055347442627 50.0
2.2103942585 1.90541684627533 51.0
0 0.0654274821281433 52.0
0 -0.258934020996094 53.0
0 0.0212928056716919 54.0
0 0.0398275852203369 55.0
0 0.171365797519684 56.0
0 0.194833934307098 57.0
0 0.210932195186615 58.0
0 0.120373606681824 59.0
0 0.0517104864120483 60.0
0 0.103727102279663 61.0
0 0.180829167366028 62.0
248.37393055 251.710006713867 0.0
0 0.0708619356155396 1.0
0 0.0428527593612671 2.0
0 0.136177897453308 3.0
0 -0.213192462921143 4.0
0 -0.0769016742706299 5.0
0 0.033043384552002 6.0
0 0.0201961994171143 7.0
0 0.244125008583069 8.0
0 0.30745267868042 9.0
0 0.265757024288177 10.0
0 -0.0718837976455688 11.0
0 -0.124699473381042 12.0
64.885396715 64.2186813354492 13.0
37.251651647 33.5145568847656 14.0
0 -0.0671347379684448 15.0
0 0.0574696063995361 16.0
30.798970514 33.6960067749023 17.0
0 0.0931819677352905 18.0
0 -0.0188233852386475 19.0
0 0.0400761365890503 20.0
0 0.255428850650787 21.0
0 -0.086645245552063 22.0
29.963403657 26.0974235534668 23.0
0 0.00885939598083496 24.0
0 0.0383943319320679 25.0
0 0.0686919093132019 26.0
0 -0.204172194004059 27.0
0 0.642504632472992 28.0
0 -0.0974288582801819 29.0
0 0.0880443453788757 30.0
0 0.00759780406951904 31.0
0 0.0762514472007751 32.0
0 -0.176781833171844 33.0
0 -0.170233011245728 34.0
0 -0.0943754315376282 35.0
0 -0.159420728683472 36.0
0 0.0817056894302368 37.0
0 -0.0761663317680359 38.0
0 -0.255910992622375 39.0
0 0.452268987894058 40.0
0 -0.0821685194969177 41.0
0 -0.0398745536804199 42.0
304.35351591 296.549499511719 43.0
0 0.0947914719581604 44.0
0 0.166132509708405 45.0
0 0.0848214626312256 46.0
0 0.286359906196594 47.0
0 0.239474356174469 48.0
0 0.405463248491287 49.0
0.084699044305 0.0063014030456543 50.0
1.9745309823 1.91203594207764 51.0
0 -0.107413411140442 52.0
0 0.134892821311951 53.0
0 0.144681215286255 54.0
0 0.306658506393433 55.0
0 0.101537585258484 56.0
0 0.231771349906921 57.0
0 0.0895958542823792 58.0
0 -0.0771396160125732 59.0
0 0.871439099311829 60.0
0 -0.0337347388267517 61.0
0 -0.0441908836364746 62.0
302.37464816 250.852478027344 0.0
0 -0.310250401496887 1.0
0 0.0636056661605835 2.0
0 0.315250396728516 3.0
0 -0.0144246220588684 4.0
0 0.095795214176178 5.0
0 0.72277843952179 6.0
0 0.380319118499756 7.0
0 -0.480941772460938 8.0
0 -0.204494535923004 9.0
0 -0.212023019790649 10.0
0 -0.225362300872803 11.0
0 -0.330933809280396 12.0
45.482115614 49.6875839233398 13.0
37.007772682 37.5866088867188 14.0
0 0.0265146493911743 15.0
0 -0.0399916172027588 16.0
29.828999935 32.8346824645996 17.0
0 0.12875097990036 18.0
0 -0.00493097305297852 19.0
0 0.0986930727958679 20.0
0 -0.204740166664124 21.0
0 1.56226515769958 22.0
44.134750702 38.6916656494141 23.0
0 -0.10773366689682 24.0
0 -0.145784914493561 25.0
0 -0.0124692916870117 26.0
0 0.135047733783722 27.0
0 0.709931254386902 28.0
0 0.00422543287277222 29.0
0 0.117492496967316 30.0
0 0.418070077896118 31.0
0 -0.0236167907714844 32.0
0 -0.0446475744247437 33.0
0 0.544621348381042 34.0
0 0.241633951663971 35.0
0 0.521328687667847 36.0
0 0.154515385627747 37.0
0 0.467998951673508 38.0
0 -0.187779784202576 39.0
0 -0.461969256401062 40.0
0 -0.253906726837158 41.0
0 -0.0181766748428345 42.0
291.82175176 293.404663085938 43.0
0 0.100164294242859 44.0
0 0.140608310699463 45.0
0 0.331257224082947 46.0
0 -0.791751861572266 47.0
0 0.0309572219848633 48.0
0 0.4157874584198 49.0
0.075606181489 0.0388871431350708 50.0
1.8743350179 1.80711591243744 51.0
0 0.150869071483612 52.0
0 0.0989822149276733 53.0
0 0.219808042049408 54.0
0 0.3351671397686 55.0
0 0.352408409118652 56.0
0 0.00237840414047241 57.0
0 -0.0414254665374756 58.0
0 0.308968365192413 59.0
0 0.0362465381622314 60.0
0 0.0305611491203308 61.0
0 0.0302001237869263 62.0
265.89131818 247.303726196289 0.0
0 -0.346219539642334 1.0
0 0.195574581623077 2.0
0 0.143782436847687 3.0
0 -0.159537792205811 4.0
0 -0.194557726383209 5.0
0 -0.377765417098999 6.0
0 0.567977726459503 7.0
0 0.697785496711731 8.0
0 -0.0322224497795105 9.0
0 0.611300647258759 10.0
0 -0.855613589286804 11.0
0 -0.141015946865082 12.0
60.373084879 60.2287292480469 13.0
38.622708843 31.054443359375 14.0
0 -0.00105714797973633 15.0
0 -0.06923907995224 16.0
25.188702693 34.7364273071289 17.0
0 0.0856233239173889 18.0
0 -0.0107483863830566 19.0
0 0.0434588193893433 20.0
0 0.205599069595337 21.0
0 0.0941464900970459 22.0
26.672164285 26.8738708496094 23.0
0 0.0691215991973877 24.0
31.524665793 0.957210421562195 25.0
0 -0.131632804870605 26.0
0 -0.265188217163086 27.0
0 0.512789607048035 28.0
0 -0.0197376012802124 29.0
0 0.0512744784355164 30.0
0 -0.196807622909546 31.0
0 -0.0987600088119507 32.0
0 -0.0878394246101379 33.0
0 0.10935378074646 34.0
0 0.116467297077179 35.0
0 -0.383853793144226 36.0
0 -0.00468903779983521 37.0
0 0.283622622489929 38.0
0 -0.458166599273682 39.0
0 0.189829230308533 40.0
0 -0.227734565734863 41.0
0 0.101795077323914 42.0
299.2801212 298.878387451172 43.0
0 0.189864635467529 44.0
0 0.0446534156799316 45.0
0 0.0571035146713257 46.0
0 0.29422527551651 47.0
0 0.228481829166412 48.0
0 0.243814527988434 49.0
0.073176300482 -0.00983351469039917 50.0
1.5103348501 1.85240364074707 51.0
0 0.143408417701721 52.0
0 -0.0268762111663818 53.0
0 0.0281428098678589 54.0
0 -0.0706720352172852 55.0
0 -0.121010184288025 56.0
0 -0.033930778503418 57.0
0 0.199068248271942 58.0
0 0.237312078475952 59.0
0 -0.102500557899475 60.0
0 0.0883126258850098 61.0
0 0.198793888092041 62.0
252.51018499 254.235733032227 0.0
0 0.490179419517517 1.0
0 0.121724188327789 2.0
0 0.127305448055267 3.0
0 -0.134583830833435 4.0
0 0.101994872093201 5.0
0 0.253704726696014 6.0
0 0.129769504070282 7.0
0 -0.57126784324646 8.0
0 0.540296256542206 9.0
0 0.253297328948975 10.0
0 -0.224400043487549 11.0
0 -0.0533052682876587 12.0
70.621115574 70.8891220092773 13.0
16.43797061 23.2563877105713 14.0
0 0.257423043251038 15.0
0 -0.0752222537994385 16.0
40.005265497 36.4772453308105 17.0
0 0.180097818374634 18.0
0 -0.0168256759643555 19.0
0 0.0157145261764526 20.0
0 0.157090067863464 21.0
0 0.110984563827515 22.0
26.836412684 25.2606182098389 23.0
0 0.123265981674194 24.0
0 1.07453000545502 25.0
0 -0.34232223033905 26.0
0 -0.166689336299896 27.0
0 1.28673601150513 28.0
0 0.0753816366195679 29.0
0 0.0222569704055786 30.0
0 0.663333892822266 31.0
0 -0.189689815044403 32.0
0 -0.16421103477478 33.0
0 0.32769513130188 34.0
0 0.250788271427155 35.0
0 0.314369171857834 36.0
0 0.0046156644821167 37.0
0 0.319040060043335 38.0
0 -0.112903594970703 39.0
0 0.241432011127472 40.0
0 -0.316124320030212 41.0
0 0.306641817092896 42.0
300.13171616 296.185668945312 43.0
0 0.182920575141907 44.0
0 0.278100788593292 45.0
0 0.227074563503265 46.0
0 -0.56473445892334 47.0
0 0.237197160720825 48.0
1.420043763 0.465582638978958 49.0
0.077893563853 0.0411819219589233 50.0
1.5864886701 1.88475561141968 51.0
0 0.347023665904999 52.0
0 -0.172107398509979 53.0
0 -0.194981455802917 54.0
0 -0.0185278058052063 55.0
0 0.207524418830872 56.0
0 -0.185159504413605 57.0
0 0.121752321720123 58.0
0 0.311920493841171 59.0
0 -0.216814756393433 60.0
0 0.0883799195289612 61.0
0 0.186255753040314 62.0
263.65710187 252.978057861328 0.0
0 -0.0464619398117065 1.0
0 0.0470383763313293 2.0
0 0.291144192218781 3.0
0 -0.212165951728821 4.0
0 0.0707187056541443 5.0
0 0.0277366638183594 6.0
0 0.14969801902771 7.0
0 0.215933203697205 8.0
0 -0.345485806465149 9.0
0 -0.0769699811935425 10.0
0 -0.210354447364807 11.0
0 -0.0559083223342896 12.0
48.304542497 46.9898872375488 13.0
50.754517728 46.7420043945312 14.0
0 -0.0109446048736572 15.0
0 -0.140868425369263 16.0
37.314581088 34.3828239440918 17.0
0 0.0802102088928223 18.0
0 -0.0472348928451538 19.0
0 0.0450291633605957 20.0
0 0.122682094573975 21.0
0 0.113465368747711 22.0
25.997095401 24.5642681121826 23.0
0 0.0839192867279053 24.0
0 0.751701354980469 25.0
0 -0.370288133621216 26.0
0 -0.0793863534927368 27.0
0 0.524729609489441 28.0
0 0.262770593166351 29.0
0 -0.00241619348526001 30.0
0 0.124412894248962 31.0
0 0.0794090628623962 32.0
0 -0.0111080408096313 33.0
0 -0.0714513659477234 34.0
0 0.234387576580048 35.0
0 0.133424937725067 36.0
0 -0.082790732383728 37.0
0 0.227017939090729 38.0
0 -0.140514850616455 39.0
0 0.469157338142395 40.0
0 -0.227647662162781 41.0
0 0.107690691947937 42.0
297.40720397 295.456787109375 43.0
0 0.114283680915833 44.0
0 0.15294337272644 45.0
0 0.0259655714035034 46.0
0 0.0629487633705139 47.0
0 0.288461804389954 48.0
0.68317674862 0.662104964256287 49.0
0.072270753595 0.0454301238059998 50.0
2.1325840066 1.91492056846619 51.0
0 -0.210419297218323 52.0
0 -0.114704787731171 53.0
0 -0.100747466087341 54.0
0 -0.184861898422241 55.0
0 0.125088691711426 56.0
0 -0.114255547523499 57.0
0 -0.0551935434341431 58.0
0 -0.61805522441864 59.0
0 1.24615788459778 60.0
0 -0.00948131084442139 61.0
0 -0.0249838829040527 62.0
248.68242343 254.472427368164 0.0
0 -0.248040795326233 1.0
0 -0.0067899227142334 2.0
0 -0.160886824131012 3.0
0 -0.191324412822723 4.0
0 0.021512508392334 5.0
0 0.0478034019470215 6.0
0 0.111731171607971 7.0
0 0.0206038951873779 8.0
0 -0.262186050415039 9.0
0 0.184019804000854 10.0
0 -0.401756048202515 11.0
0 -0.0999844670295715 12.0
62.217350368 61.4766120910645 13.0
20.055296633 35.4800033569336 14.0
0 -0.0828515291213989 15.0
0 -0.137904524803162 16.0
31.204645306 35.0573081970215 17.0
0 0.0532411336898804 18.0
0 -0.0665684938430786 19.0
0 0.135240256786346 20.0
0 -0.0838650465011597 21.0
0 0.453499495983124 22.0
33.461561051 33.4253730773926 23.0
0 0.0464522838592529 24.0
0 0.66888165473938 25.0
0 -0.101287186145782 26.0
0 -0.0339115262031555 27.0
0 0.295637190341949 28.0
0 0.0768001079559326 29.0
0 0.109150886535645 30.0
0 0.14132297039032 31.0
0 0.171592712402344 32.0
0 0.0202789306640625 33.0
0 0.105875372886658 34.0
0 0.0421056747436523 35.0
0 0.292094618082047 36.0
0 -0.0645756721496582 37.0
0 0.0721940994262695 38.0
0 -0.2807697057724 39.0
0 0.336117178201675 40.0
0 -0.174513041973114 41.0
0 0.439456462860107 42.0
288.11219534 294.586151123047 43.0
0 0.065984845161438 44.0
0 0.195727407932281 45.0
0 0.285016894340515 46.0
0 -0.0144794583320618 47.0
0 -0.657269954681396 48.0
0 0.427869200706482 49.0
0.10009390756 -0.00224024057388306 50.0
1.4612472871 1.85566294193268 51.0
0 0.339989423751831 52.0
0 0.271715760231018 53.0
0 -0.0499093532562256 54.0
0 -0.0204372406005859 55.0
0 0.295814335346222 56.0
0 -0.00739026069641113 57.0
0 0.346512079238892 58.0
0 0.112122893333435 59.0
0 -0.298760175704956 60.0
0 0.0806254148483276 61.0
0 0.278399765491486 62.0
281.58607857 252.82258605957 0.0
0 -0.189184665679932 1.0
0 0.00927311182022095 2.0
0 0.181897282600403 3.0
0 -0.050011157989502 4.0
0 0.0608826875686646 5.0
0 0.19210946559906 6.0
0 0.42149692773819 7.0
0 -0.411937594413757 8.0
0 -0.294276475906372 9.0
0 0.221258401870728 10.0
0 -0.0708356499671936 11.0
0 -0.0648596882820129 12.0
59.957018581 61.9149169921875 13.0
37.03469917 31.0587997436523 14.0
0 -0.058912992477417 15.0
0 0.066728949546814 16.0
35.944780243 36.108585357666 17.0
0 0.0436286926269531 18.0
0 -0.0634040832519531 19.0
0 0.0752445459365845 20.0
0 0.0849241614341736 21.0
0 0.226464986801147 22.0
30.155357818 31.0589122772217 23.0
0 0.16029304265976 24.0
0 1.31766843795776 25.0
0 0.00141143798828125 26.0
0 -0.109062492847443 27.0
0 0.229643046855927 28.0
0 0.103753566741943 29.0
0 -0.175150454044342 30.0
0 0.0892695188522339 31.0
0 0.0905196070671082 32.0
0 -0.276524305343628 33.0
0 0.395414590835571 34.0
0 -0.0558573603630066 35.0
0 0.34939256310463 36.0
0 -0.0424612760543823 37.0
0 0.20551997423172 38.0
0 -0.37497091293335 39.0
0 0.299857258796692 40.0
0 -0.114816784858704 41.0
0 -0.131039023399353 42.0
283.85723649 297.622436523438 43.0
0 -0.0202429294586182 44.0
0 0.122376024723053 45.0
0 0.0567597150802612 46.0
0 0.0803083181381226 47.0
0 0.368579745292664 48.0
0.59758696653 0.538919687271118 49.0
0.09403947758 0.117895603179932 50.0
1.8894940797 1.84787452220917 51.0
0 -0.140680074691772 52.0
0 0.239976406097412 53.0
0 0.0967204570770264 54.0
0 0.223895847797394 55.0
0 0.178147852420807 56.0
0 0.279572606086731 57.0
0 0.464424341917038 58.0
0 0.306434333324432 59.0
0 -0.128453075885773 60.0
0 0.051064670085907 61.0
0 0.28311949968338 62.0
240.45036669 248.647583007812 0.0
0 -0.215507864952087 1.0
0 0.0669234991073608 2.0
0 0.135317802429199 3.0
0 -0.22533917427063 4.0
0 -0.385740160942078 5.0
0 -0.0128384232521057 6.0
0 0.238366663455963 7.0
0 0.0326083898544312 8.0
0 0.237934350967407 9.0
0 0.0924619436264038 10.0
0 -0.18342387676239 11.0
0 -0.0615793466567993 12.0
61.509187996 60.08203125 13.0
37.846701323 37.4662551879883 14.0
0 -0.0655937194824219 15.0
0 0.196058332920074 16.0
37.498431824 35.2882843017578 17.0
0 0.100399792194366 18.0
0 0.0150405168533325 19.0
0 0.0926545858383179 20.0
0 0.0492419600486755 21.0
0 0.0746706128120422 22.0
38.5066457 38.3714561462402 23.0
0 0.0274489521980286 24.0
0 0.938587665557861 25.0
0 -0.0832886099815369 26.0
0 -0.0242490172386169 27.0
0 0.0423840880393982 28.0
0 -0.0665582418441772 29.0
0 0.102037191390991 30.0
0 0.233468174934387 31.0
0 0.173846423625946 32.0
0 0.0255285501480103 33.0
0 0.101930022239685 34.0
0 0.441055536270142 35.0
0 -0.382112622261047 36.0
0 0.00757747888565063 37.0
0 0.0911620855331421 38.0
0 -0.243330717086792 39.0
0 0.0846260786056519 40.0
0 -0.116020977497101 41.0
0 0.0566712617874146 42.0
285.37898489 298.317596435547 43.0
0 0.0761858820915222 44.0
0 0.0983888506889343 45.0
0 0.157666862010956 46.0
0 -0.294995546340942 47.0
0 0.0587868690490723 48.0
0 0.445959359407425 49.0
0.095311076704 0.0157666206359863 50.0
2.2306514896 1.86655592918396 51.0
0 0.234551548957825 52.0
0 -0.166419923305511 53.0
0 -0.0391783714294434 54.0
0 0.215875804424286 55.0
0 0.160309135913849 56.0
0 0.269287884235382 57.0
0 0.122206330299377 58.0
0 0.0195844769477844 59.0
0 -0.147789537906647 60.0
0 0.0428906679153442 61.0
0 0.0726814270019531 62.0
257.41831347 250.833709716797 0.0
0 -0.0765534639358521 1.0
0 -0.0712713599205017 2.0
0 0.561115920543671 3.0
0 -0.101705312728882 4.0
0 0.0293856859207153 5.0
0 -0.0189464688301086 6.0
0 -0.13189697265625 7.0
0 -0.222559452056885 8.0
0 0.114516079425812 9.0
0 -0.366130232810974 10.0
0 -0.0324553251266479 11.0
0 -0.105867743492126 12.0
57.116207597 57.8328285217285 13.0
43.608587743 34.6106224060059 14.0
0 0.0354987382888794 15.0
0 -0.0085066556930542 16.0
28.671625187 34.3699569702148 17.0
0 0.0544250011444092 18.0
0 -0.0496255159378052 19.0
0 0.0987908244132996 20.0
0 0.0593994855880737 21.0
0 0.368371874094009 22.0
31.518957077 32.6918525695801 23.0
0 0.111549615859985 24.0
0 1.12226378917694 25.0
0 0.0181108117103577 26.0
0 -0.137850284576416 27.0
0 -0.0821064114570618 28.0
0 -0.0406764149665833 29.0
0 0.071539580821991 30.0
0 0.142900228500366 31.0
0 0.144622266292572 32.0
0 0.126474440097809 33.0
0 0.366251200437546 34.0
0 0.300861299037933 35.0
0 0.349731922149658 36.0
0 -0.106798589229584 37.0
0 0.300192356109619 38.0
0 -0.122140645980835 39.0
0 -0.158929586410522 40.0
0 -0.162263870239258 41.0
0 -0.0496773719787598 42.0
295.98166452 299.132507324219 43.0
0 0.0394640564918518 44.0
0 0.154117405414581 45.0
0 0.119514524936676 46.0
0 -0.0276259183883667 47.0
0 0.405900239944458 48.0
0 0.518790662288666 49.0
0.067485059706 0.0706395506858826 50.0
1.8775220283 1.82557511329651 51.0
0 0.123755931854248 52.0
0 -0.135800421237946 53.0
0 0.134958863258362 54.0
0 0.256410002708435 55.0
0 0.118732750415802 56.0
0 0.641311287879944 57.0
0 0.212667107582092 58.0
0 0.0786429643630981 59.0
0 0.0516815185546875 60.0
0 -0.0226016044616699 61.0
0 0.0463218092918396 62.0
277.18428757 255.005355834961 0.0
0 -0.489194750785828 1.0
0 0.0384745597839355 2.0
0 -0.211816787719727 3.0
0 -0.284374713897705 4.0
0 0.566271126270294 5.0
0 -0.104274690151215 6.0
0 0.575531840324402 7.0
0 0.755733788013458 8.0
0 -0.0219369530677795 9.0
0 0.431407958269119 10.0
0 -0.256762504577637 11.0
0 0.0292380452156067 12.0
59.991681026 60.9203720092773 13.0
30.636467525 32.1776847839355 14.0
0 -0.16772735118866 15.0
0 0.230584740638733 16.0
34.714832981 36.3000030517578 17.0
0 0.0904713273048401 18.0
0 0.0258248448371887 19.0
0 0.0645294189453125 20.0
0 0.00380820035934448 21.0
0 0.120341956615448 22.0
39.054364376 38.0342559814453 23.0
0 0.0865470767021179 24.0
0 0.64685070514679 25.0
0 0.132498741149902 26.0
0 -0.0214440226554871 27.0
0 0.469395875930786 28.0
0 -0.0733547210693359 29.0
0 -0.096827507019043 30.0
0 0.0481873750686646 31.0
0 -0.122873723506927 32.0
0 -0.106975674629211 33.0
0 0.0660516023635864 34.0
0 0.165908515453339 35.0
0 0.190959453582764 36.0
0 0.116745352745056 37.0
0 -0.00571143627166748 38.0
0 -0.222320556640625 39.0
0 0.357719957828522 40.0
0 -0.175035715103149 41.0
0 0.153661131858826 42.0
288.53734099 297.604034423828 43.0
0 0.143563807010651 44.0
0 0.19271856546402 45.0
0 0.176237344741821 46.0
0 0.0831221342086792 47.0
0 0.121091783046722 48.0
0 0.294031620025635 49.0
0.097389770807 -0.0120442509651184 50.0
1.9364159224 1.8800675868988 51.0
0 -0.0607776641845703 52.0
0 -0.0407957434654236 53.0
0 0.0292274951934814 54.0
0 0.188465356826782 55.0
0 0.30761331319809 56.0
0 -0.036836564540863 57.0
0 0.206368863582611 58.0
0 0.109735608100891 59.0
0 -0.15599912405014 60.0
0 0.0555351972579956 61.0
0 0.300293862819672 62.0
219.63762772 251.520690917969 0.0
0 0.0342477560043335 1.0
0 -0.0510412454605103 2.0
0 -0.017179548740387 3.0
0 -0.292762517929077 4.0
0 -0.256697654724121 5.0
0 0.0836282968521118 6.0
0 -0.0737047791481018 7.0
0 0.0926357507705688 8.0
0 0.239320039749146 9.0
0 0.237673819065094 10.0
0 -0.388975024223328 11.0
0 -0.0406121015548706 12.0
52.549372331 53.5189437866211 13.0
37.200062017 39.0559616088867 14.0
0 -0.0613862276077271 15.0
0 -0.0698006153106689 16.0
34.268062137 35.1395263671875 17.0
0 0.13084602355957 18.0
0 -0.0196796655654907 19.0
0 0.0835529565811157 20.0
0 0.128361105918884 21.0
0 0.168843388557434 22.0
37.198402042 35.3161430358887 23.0
0 0.0216649770736694 24.0
0 0.67192006111145 25.0
0 -0.137352347373962 26.0
0 0.00442641973495483 27.0
0 0.521946549415588 28.0
0 -0.0175988674163818 29.0
0 0.0965960621833801 30.0
0 -0.152047872543335 31.0
0 -0.173991799354553 32.0
0 -0.159516930580139 33.0
0 0.209357142448425 34.0
0 0.292813032865524 35.0
0 0.156861305236816 36.0
0 0.028769314289093 37.0
0 0.185549974441528 38.0
0 -0.322923541069031 39.0
0 0.423878967761993 40.0
0 -0.122915327548981 41.0
0 -0.0489758253097534 42.0
303.61881332 297.273345947266 43.0
0 0.0841255187988281 44.0
0 0.0870317816734314 45.0
0 0.0443310737609863 46.0
0 0.146785914897919 47.0
0 0.42670539021492 48.0
0 0.411026537418365 49.0
0.081955663656 0.0638116598129272 50.0
1.5765686668 1.89043474197388 51.0
0 -0.0137763619422913 52.0
0 -0.0279219746589661 53.0
0 0.0635692477226257 54.0
0 0.2689328789711 55.0
0 0.0308991074562073 56.0
0 0.0359539389610291 57.0
0 0.214816987514496 58.0
0 0.08794105052948 59.0
0 -0.205814301967621 60.0
0 0.0597395300865173 61.0
0 0.274473667144775 62.0
277.02563844 252.010192871094 0.0
0 -0.321223378181458 1.0
0 0.0137357711791992 2.0
0 0.0498109459877014 3.0
0 0.0528483390808105 4.0
0 0.125378906726837 5.0
0 0.178987324237823 6.0
0 0.387651890516281 7.0
0 -0.418488025665283 8.0
0 0.0819904804229736 9.0
0 -0.175348699092865 10.0
0 -0.0630252361297607 11.0
0 0.00913089513778687 12.0
64.273208711 66.2640914916992 13.0
35.459761569 29.3852005004883 14.0
0 -0.0708400011062622 15.0
0 -0.262699604034424 16.0
28.903957412 33.8021354675293 17.0
0 0.070321798324585 18.0
0 -0.0688656568527222 19.0
0 0.397064834833145 20.0
0 0.0369805097579956 21.0
0 0.263087809085846 22.0
32.432418734 31.0367183685303 23.0
0 0.0102866888046265 24.0
26.791545875 0.316432297229767 25.0
0 -0.00735914707183838 26.0
0 0.102466404438019 27.0
0 -0.0488693118095398 28.0
0 -0.0435751080513 29.0
0 0.173852503299713 30.0
0 -0.0440305471420288 31.0
0 0.0435905456542969 32.0
0 -0.465893745422363 33.0
0 -0.228943705558777 34.0
0 0.0775207281112671 35.0
0 -0.591690182685852 36.0
0 0.0641700029373169 37.0
0 0.120166599750519 38.0
0 -0.166440725326538 39.0
0 0.458573043346405 40.0
0 -0.095527172088623 41.0
0 -0.011939525604248 42.0
287.50648982 296.697662353516 43.0
0 0.0777230262756348 44.0
0 0.304028689861298 45.0
0 0.282584130764008 46.0
0 -0.148389339447021 47.0
0 -0.13433575630188 48.0
0 0.471019566059113 49.0
0.074090112329 -0.0198047161102295 50.0
2.0212578915 1.87452006340027 51.0
0 0.255348920822144 52.0
0 -0.287713766098022 53.0
0 -0.118998944759369 54.0
0 0.242589473724365 55.0
0 0.358528465032578 56.0
0 0.13979434967041 57.0
0 -0.0372249484062195 58.0
0 0.00284814834594727 59.0
0 0.0123157501220703 60.0
0 0.0973272919654846 61.0
0 0.18920224905014 62.0
270.73877896 251.071197509766 0.0
0 -0.170127034187317 1.0
0 0.0156590938568115 2.0
0 0.0725563168525696 3.0
0 0.0139942169189453 4.0
0 -0.160492777824402 5.0
0 -0.100118517875671 6.0
0 0.385204643011093 7.0
0 -0.19142347574234 8.0
0 -0.00655889511108398 9.0
0 0.618793129920959 10.0
0 -0.268854260444641 11.0
0 -0.100413024425507 12.0
68.199669813 69.1105422973633 13.0
30.510971411 27.669994354248 14.0
0 -0.00982451438903809 15.0
0 -0.0553306341171265 16.0
37.989399197 35.0120468139648 17.0
0 0.0994681119918823 18.0
0 0.00573396682739258 19.0
0 0.106334388256073 20.0
0 0.057466983795166 21.0
0 0.231574833393097 22.0
33.222988626 30.644193649292 23.0
0 0.0470770001411438 24.0
0 0.480479121208191 25.0
0 0.015934944152832 26.0
0 -0.145455062389374 27.0
0 0.0809268951416016 28.0
0 -0.17970222234726 29.0
0 0.0926997065544128 30.0
0 0.0256558656692505 31.0
0 0.0397618412971497 32.0
0 -0.0891923308372498 33.0
0 0.288863897323608 34.0
0 -0.0230064392089844 35.0
0 0.469170153141022 36.0
0 0.0199227333068848 37.0
0 0.0961824655532837 38.0
0 -0.436879515647888 39.0
0 0.43788355588913 40.0
0 -0.196033477783203 41.0
0 0.120545864105225 42.0
287.29912801 296.058654785156 43.0
0 0.0753686428070068 44.0
0 0.00560009479522705 45.0
0 0.0761548280715942 46.0
0 0.0717946887016296 47.0
0 -0.0670824646949768 48.0
0 0.353507399559021 49.0
0.087115678208 0.0451923608779907 50.0
2.1212313726 1.79500651359558 51.0
0 0.217925727367401 52.0
0 0.0436916351318359 53.0
0 0.0727764964103699 54.0
0 -0.12846565246582 55.0
0 -0.116285085678101 56.0
0 0.290829002857208 57.0
0 0.231216907501221 58.0
0 0.128282606601715 59.0
0 -0.211269378662109 60.0
0 0.125611782073975 61.0
0 0.370329558849335 62.0
231.19885795 253.597106933594 0.0
0 -0.179466128349304 1.0
0 -0.0319989919662476 2.0
0 0.0519118309020996 3.0
0 -0.195984482765198 4.0
0 -0.143172144889832 5.0
0 -0.116440057754517 6.0
0 -0.236338138580322 7.0
0 0.360693901777267 8.0
0 -0.167525589466095 9.0
0 -0.0803903937339783 10.0
0 -0.0946124792098999 11.0
0 0.0344544649124146 12.0
65.020040517 65.9478530883789 13.0
40.687835721 32.461727142334 14.0
0 -0.0827591419219971 15.0
0 -0.0432261228561401 16.0
30.094040615 33.600399017334 17.0
0 0.0799194574356079 18.0
0 -0.0227508544921875 19.0
0 0.0785308480262756 20.0
0 0.161792993545532 21.0
0 0.0513689517974854 22.0
27.402017585 26.7983303070068 23.0
0 0.0789707899093628 24.0
0 0.415753424167633 25.0
0 0.0466816425323486 26.0
0 -0.252649188041687 27.0
0 0.430670976638794 28.0
0 0.0708717703819275 29.0
0 0.00180190801620483 30.0
0 0.173049449920654 31.0
0 -0.144287168979645 32.0
0 -0.201631009578705 33.0
0 0.185277342796326 34.0
0 0.318369656801224 35.0
0 0.400401592254639 36.0
0 0.132993161678314 37.0
0 0.0169583559036255 38.0
0 -0.0945305824279785 39.0
0 0.307365268468857 40.0
0 -0.101592242717743 41.0
0 -0.0419337749481201 42.0
305.4629829 296.903747558594 43.0
0 0.0294317007064819 44.0
0 0.130658328533173 45.0
0 0.164867639541626 46.0
0 0.125657558441162 47.0
0 -0.157101273536682 48.0
0 0.434554755687714 49.0
0.092406918628 -0.0276568531990051 50.0
1.5232393136 1.87046611309052 51.0
0 0.0614891648292542 52.0
0 -0.06400465965271 53.0
0 0.0964226126670837 54.0
0 0.255793511867523 55.0
0 0.213453114032745 56.0
0 0.467734694480896 57.0
0 0.566052615642548 58.0
0 0.206022560596466 59.0
0 -0.0622808933258057 60.0
0 0.0952317118644714 61.0
0 0.170939087867737 62.0
274.49396386 253.592971801758 0.0
0 0.274251222610474 1.0
0 -0.102489411830902 2.0
0 0.181906580924988 3.0
0 0.147621035575867 4.0
0 0.163892984390259 5.0
0 -0.188896656036377 6.0
0 0.671912312507629 7.0
0 0.0283170938491821 8.0
0 -0.271136403083801 9.0
0 -0.0745251178741455 10.0
0 -0.269632816314697 11.0
0 -0.089064359664917 12.0
59.602986736 59.9095993041992 13.0
29.124264368 34.3636169433594 14.0
0 -0.0533170700073242 15.0
0 0.106054842472076 16.0
33.254527792 36.486198425293 17.0
0 0.13530957698822 18.0
0 -0.0199591517448425 19.0
0 0.0990820527076721 20.0
0 0.127416610717773 21.0
0 0.104501605033875 22.0
34.466574702 30.2721614837646 23.0
0 0.0493814945220947 24.0
0 0.620140910148621 25.0
0 0.0845291614532471 26.0
0 -0.0632461905479431 27.0
0 0.609207928180695 28.0
0 -0.036673903465271 29.0
0 0.0541365146636963 30.0
0 -0.00437939167022705 31.0
0 0.0527130961418152 32.0
0 0.112602233886719 33.0
0 0.185135722160339 34.0
0 0.230583488941193 35.0
0 0.359929174184799 36.0
0 0.175593852996826 37.0
0 0.113367855548859 38.0
0 -0.201107859611511 39.0
0 0.278405487537384 40.0
0 -0.0868133306503296 41.0
0 0.028877854347229 42.0
269.77000823 297.633758544922 43.0
0 -0.0152769684791565 44.0
0 0.0448022484779358 45.0
0 0.285296857357025 46.0
0 -0.582747459411621 47.0
0 0.329242140054703 48.0
0.74326893869 0.640021741390228 49.0
0.080048601195 0.0988919734954834 50.0
1.7339306734 1.87668836116791 51.0
0 0.361524939537048 52.0
0 -0.356401562690735 53.0
0 0.278014302253723 54.0
0 0.197511851787567 55.0
0 -0.0168519020080566 56.0
0 0.144798159599304 57.0
0 0.582005500793457 58.0
0 0.371316283941269 59.0
0 -0.151424765586853 60.0
0 0.0874439477920532 61.0
0 0.213068068027496 62.0
262.18985663 248.38117980957 0.0
0 -0.502588391304016 1.0
0 0.0369812846183777 2.0
0 0.123119711875916 3.0
0 0.0524802207946777 4.0
0 -0.181866943836212 5.0
0 -0.347936391830444 6.0
0 0.432206809520721 7.0
0 0.0971102714538574 8.0
0 -0.217187523841858 9.0
0 0.117425501346588 10.0
0 -0.242723107337952 11.0
0 -0.0603191256523132 12.0
56.365983751 54.5412063598633 13.0
43.80118332 39.9979591369629 14.0
0 -0.326120495796204 15.0
0 -0.176313400268555 16.0
33.066608519 34.6931762695312 17.0
0 0.130391478538513 18.0
0 0.00854849815368652 19.0
0 0.110775589942932 20.0
0 0.0848841667175293 21.0
0 0.111544132232666 22.0
26.521451708 27.7692680358887 23.0
0 0.0406882762908936 24.0
29.879480622 0.702938735485077 25.0
0 -0.307277202606201 26.0
0 -0.260977029800415 27.0
0 0.578253984451294 28.0
0 0.16388064622879 29.0
0 -0.0574929118156433 30.0
0 0.0412650108337402 31.0
0 0.0347514152526855 32.0
0 -0.272141695022583 33.0
0 0.255269408226013 34.0
0 0.18311220407486 35.0
0 0.470744699239731 36.0
0 0.245834231376648 37.0
0 -0.0504967570304871 38.0
0 -0.0404382944107056 39.0
0 0.338432013988495 40.0
0 -0.0741031169891357 41.0
0 0.00697565078735352 42.0
297.33797651 297.086700439453 43.0
0 0.137439250946045 44.0
0 0.0624390840530396 45.0
0 0.145405292510986 46.0
0 -0.0368993878364563 47.0
0 0.319660782814026 48.0
0 0.495276510715485 49.0
0.079504869356 0.0627508759498596 50.0
1.8250729833 1.84266257286072 51.0
0 -0.0446211695671082 52.0
0 0.133489072322845 53.0
0 -0.160234689712524 54.0
0 0.182072520256042 55.0
0 0.194623708724976 56.0
0 -0.0632250308990479 57.0
0 0.18278968334198 58.0
0 0.0927798748016357 59.0
0 -0.0662755966186523 60.0
0 0.0967539548873901 61.0
0 0.177761316299438 62.0
258.46001951 255.033462524414 0.0
0 -0.266051650047302 1.0
0 -0.162897288799286 2.0
0 0.409888178110123 3.0
0 -0.1441570520401 4.0
0 -0.102538645267487 5.0
0 -0.187963485717773 6.0
0 0.270351707935333 7.0
0 -0.310772776603699 8.0
0 0.204542279243469 9.0
0 -0.403929948806763 10.0
0 0.0527344942092896 11.0
0 -0.154062747955322 12.0
58.696723069 60.1695861816406 13.0
46.16623135 36.9584350585938 14.0
0 0.0762825012207031 15.0
0 -0.0733309984207153 16.0
34.130535 33.4562759399414 17.0
0 0.126305937767029 18.0
0 0.024603545665741 19.0
0 0.117138266563416 20.0
0 -0.0457916259765625 21.0
0 -0.00160956382751465 22.0
40.509589291 39.8384246826172 23.0
0 0.0328108072280884 24.0
0 0.600791573524475 25.0
0 0.0186406373977661 26.0
0 -0.0138409733772278 27.0
0 0.508920431137085 28.0
0 0.0927420258522034 29.0
0 0.0645318031311035 30.0
0 -0.225753307342529 31.0
0 -0.127986490726471 32.0
0 -0.178643941879272 33.0
0 0.0881745219230652 34.0
0 -0.0994130969047546 35.0
0 0.437333017587662 36.0
0 0.139502465724945 37.0
0 0.113114774227142 38.0
0 -0.162712335586548 39.0
0 0.988618016242981 40.0
0 -0.0395071506500244 41.0
0 0.0175628662109375 42.0
295.39941846 297.363708496094 43.0
0 0.0823429822921753 44.0
0 0.151357531547546 45.0
0 0.0144355297088623 46.0
0 0.154759109020233 47.0
0 0.0335447192192078 48.0
0 0.356902241706848 49.0
0.096890267417 0.0208244919776917 50.0
2.0204951746 1.86861431598663 51.0
0 0.115697681903839 52.0
0 0.0623255968093872 53.0
0 0.0703750848770142 54.0
0 0.319211453199387 55.0
0 0.146226167678833 56.0
0 -0.0255753397941589 57.0
0 0.0587909817695618 58.0
0 -0.079270601272583 59.0
0 -0.198817431926727 60.0
0 -0.0409010052680969 61.0
0 -0.000344634056091309 62.0
260.84088372 252.901565551758 0.0
0 -0.306170105934143 1.0
0 -0.162651240825653 2.0
0 0.12516176700592 3.0
0 0.0757105946540833 4.0
0 0.210399866104126 5.0
0 0.0186082720756531 6.0
0 0.0379273891448975 7.0
0 -0.256649732589722 8.0
0 0.218977272510529 9.0
0 -0.510602235794067 10.0
0 -0.138624012470245 11.0
0 -0.13842511177063 12.0
77.226145413 75.4880447387695 13.0
30.144033821 18.7727737426758 14.0
0 -0.0346939563751221 15.0
0 -0.133737742900848 16.0
27.472449448 34.7731704711914 17.0
0 0.0775458216667175 18.0
0 0.0203741788864136 19.0
0 0.090002715587616 20.0
0 -0.0494977235794067 21.0
0 0.556456089019775 22.0
41.496493639 39.6774826049805 23.0
0 0.112293541431427 24.0
0 1.0955536365509 25.0
0 0.0323359966278076 26.0
0 0.0635008811950684 27.0
0 0.197672784328461 28.0
0 0.105853736400604 29.0
0 0.0974487662315369 30.0
0 -0.263360261917114 31.0
0 -0.141443192958832 32.0
0 -0.687071919441223 33.0
0 0.272017002105713 34.0
0 0.375413000583649 35.0
0 -0.108852386474609 36.0
0 0.210056483745575 37.0
0 0.238780319690704 38.0
0 -0.313506722450256 39.0
0 0.101037681102753 40.0
0 -0.159843564033508 41.0
0 -0.0615780353546143 42.0
298.19111964 297.00927734375 43.0
0 0.161954283714294 44.0
0 0.173879981040955 45.0
0 0.089377760887146 46.0
0 0.00105726718902588 47.0
0 0.184625148773193 48.0
0.47237531641 0.60929811000824 49.0
0.095352842646 0.0659377574920654 50.0
1.5235113519 1.86496257781982 51.0
0 0.296538054943085 52.0
0 0.0832322835922241 53.0
0 0.173535704612732 54.0
0 0.330875247716904 55.0
0 0.141400218009949 56.0
0 0.1785968542099 57.0
0 0.617182552814484 58.0
0 0.306993722915649 59.0
0 -0.174275755882263 60.0
0 0.0474917888641357 61.0
0 0.16796875 62.0
230.50385669 253.9296875 0.0
0 0.435465931892395 1.0
0 -0.125156104564667 2.0
0 0.285311162471771 3.0
0 -0.249197959899902 4.0
0 -0.221356153488159 5.0
0 -0.0354540944099426 6.0
0 -0.170616149902344 7.0
0 0.691609978675842 8.0
0 0.315325915813446 9.0
0 0.0267258882522583 10.0
0 -0.178041338920593 11.0
0 0.0620357990264893 12.0
65.045428007 66.6165084838867 13.0
29.011970421 29.1961841583252 14.0
0 -0.0840746164321899 15.0
0 0.390335857868195 16.0
40.299083308 35.0178451538086 17.0
0 0.198181390762329 18.0
0 -0.0727521181106567 19.0
0 0.0844635963439941 20.0
0 0.158888459205627 21.0
0 0.0591739416122437 22.0
25.746741789 24.5051860809326 23.0
0 0.0179053544998169 24.0
0 -0.156648695468903 25.0
0 0.0771999359130859 26.0
0 -0.296029806137085 27.0
0 0.00568729639053345 28.0
0 0.0805951952934265 29.0
0 -0.0704114437103271 30.0
0 -0.239735007286072 31.0
0 -0.0480386018753052 32.0
0 0.129516303539276 33.0
0 0.362848997116089 34.0
0 0.0201674699783325 35.0
0 0.564659118652344 36.0
0 0.119416117668152 37.0
0 0.354769855737686 38.0
0 -0.281469583511353 39.0
0 0.137523531913757 40.0
0 -0.185055196285248 41.0
0 -0.237281203269958 42.0
287.98597517 297.181701660156 43.0
0 0.0395460724830627 44.0
0 0.111464083194733 45.0
0 0.223530352115631 46.0
0 -0.652395367622375 47.0
0 -0.233599066734314 48.0
0 0.477350562810898 49.0
0.098778867648 -0.0166183710098267 50.0
1.6607063282 1.91692698001862 51.0
0 -0.043666660785675 52.0
0 -0.219512224197388 53.0
0 -0.148302793502808 54.0
0 0.210651338100433 55.0
0 0.152976393699646 56.0
0 0.1332026720047 57.0
0 0.246800065040588 58.0
0 0.515009999275208 59.0
0 0.145216166973114 60.0
0 0.223773658275604 61.0
0 0.440788447856903 62.0
226.20306875 252.406829833984 0.0
0 -0.151340961456299 1.0
0 -0.361282706260681 2.0
0 0.194267988204956 3.0
0 -0.212207078933716 4.0
0 0.0126117467880249 5.0
0 -0.151561260223389 6.0
0 -0.190447270870209 7.0
0 0.283841967582703 8.0
0 -0.553827166557312 9.0
0 -0.128506660461426 10.0
0 -0.00595879554748535 11.0
0 -0.225407600402832 12.0
76.751917862 76.9816284179688 13.0
25.357078416 25.275218963623 14.0
0 -0.0847548246383667 15.0
0 0.0569267272949219 16.0
38.559164148 35.4017868041992 17.0
0 0.119003772735596 18.0
0 -0.0760159492492676 19.0
0 0.0948067903518677 20.0
0 0.100970506668091 21.0
0 0.192773342132568 22.0
32.584985522 32.1497802734375 23.0
0 0.114602208137512 24.0
0 0.631911277770996 25.0
0 0.0342137813568115 26.0
0 -0.127504527568817 27.0
0 -0.0423629283905029 28.0
0 -0.158421576023102 29.0
0 0.070568323135376 30.0
0 -0.194523572921753 31.0
0 0.00125449895858765 32.0
0 -0.105269730091095 33.0
0 0.0644194483757019 34.0
0 0.150797009468079 35.0
0 0.142806529998779 36.0
0 -0.0208354592323303 37.0
0 0.101938366889954 38.0
0 -0.0680387020111084 39.0
0 0.657467126846313 40.0
0 -0.025364100933075 41.0
0 0.119879364967346 42.0
285.37745411 296.276428222656 43.0
0 0.263519525527954 44.0
0 -0.0177961587905884 45.0
0 0.110601246356964 46.0
0 -0.0107244253158569 47.0
0 0.33370453119278 48.0
0 0.484853774309158 49.0
0.09433340491 0.00244820117950439 50.0
2.0155198805 1.9393116235733 51.0
0 0.0926749110221863 52.0
0 0.0231342911720276 53.0
0 0.172847628593445 54.0
0 0.175040602684021 55.0
0 0.0528414845466614 56.0
0 0.011235773563385 57.0
0 0.107638776302338 58.0
0 0.155964612960815 59.0
0 -0.26089072227478 60.0
0 0.0795705318450928 61.0
0 0.271857380867004 62.0
273.37405008 255.700759887695 0.0
0 -0.391190528869629 1.0
0 -0.167795896530151 2.0
0 -0.223477602005005 3.0
0 0.0570476651191711 4.0
0 0.174912214279175 5.0
0 0.247556686401367 6.0
0 0.292973250150681 7.0
0 -0.351273059844971 8.0
0 0.0648818016052246 9.0
0 0.121098577976227 10.0
0 0.492457926273346 11.0
0 -0.140073776245117 12.0
49.60035969 49.8193702697754 13.0
45.538523217 43.1815071105957 14.0
0 -0.0909192562103271 15.0
0 -0.201702773571014 16.0
26.826627673 32.3012809753418 17.0
0 0.109280109405518 18.0
0 0.0239534378051758 19.0
0 0.0899532437324524 20.0
0 0.21543288230896 21.0
0 0.169298529624939 22.0
35.561575837 32.6311111450195 23.0
0 0.0720204710960388 24.0
0 1.06577527523041 25.0
0 0.0258493423461914 26.0
0 -0.190123915672302 27.0
0 0.189159691333771 28.0
0 0.0570353269577026 29.0
0 0.174879372119904 30.0
0 -0.217466592788696 31.0
0 -0.120737195014954 32.0
0 -0.224200487136841 33.0
0 0.141752123832703 34.0
0 0.0144304037094116 35.0
0 -0.0262501835823059 36.0
0 0.105510413646698 37.0
0 0.170282423496246 38.0
0 -0.155444383621216 39.0
0 0.781315207481384 40.0
0 -0.0826930403709412 41.0
0 -0.0289369821548462 42.0
308.07830011 296.662414550781 43.0
0 0.109692692756653 44.0
0 0.152054667472839 45.0
0 0.202056109905243 46.0
0 0.113285303115845 47.0
0 -0.352983117103577 48.0
0 0.416777223348618 49.0
0.089665729146 0.00826001167297363 50.0
1.6969396624 1.78987407684326 51.0
0 -0.0671279430389404 52.0
0 -0.0575792789459229 53.0
0 -0.0332393646240234 54.0
0 0.251736104488373 55.0
0 0.249206244945526 56.0
0 -0.0555329322814941 57.0
0 -0.00821620225906372 58.0
0 0.135576367378235 59.0
0 -0.0229324102401733 60.0
0 0.0477077960968018 61.0
0 0.0855917930603027 62.0
218.1536978 251.994171142578 0.0
0 -0.00383329391479492 1.0
0 -0.066958487033844 2.0
0 0.285820662975311 3.0
0 -0.256754875183105 4.0
0 -0.259544134140015 5.0
0 -0.00631505250930786 6.0
0 -0.249231576919556 7.0
0 0.472134351730347 8.0
0 -0.0226557850837708 9.0
0 0.969939470291138 10.0
0 -0.131269991397858 11.0
0 -0.0955626964569092 12.0
60.343555122 59.4020385742188 13.0
21.654113421 39.088794708252 14.0
0 -0.135908007621765 15.0
0 -0.121881484985352 16.0
34.781138185 35.3388595581055 17.0
0 0.161324143409729 18.0
0 -0.00562012195587158 19.0
0 0.0996507406234741 20.0
0 0.150804877281189 21.0
0 -0.010807991027832 22.0
33.77347685 29.5929565429688 23.0
0 -0.00845432281494141 24.0
0 -0.0568499565124512 25.0
0 -0.17565530538559 26.0
0 0.0509642362594604 27.0
0 -0.158123791217804 28.0
0 0.00060802698135376 29.0
0 -0.049785315990448 30.0
0 0.00540626049041748 31.0
0 -0.167765617370605 32.0
0 -0.381434202194214 33.0
0 0.1546870470047 34.0
0 0.915964126586914 35.0
0 0.544726133346558 36.0
0 0.136762142181396 37.0
0 -0.294465661048889 38.0
0 -0.0172983407974243 39.0
0 0.484346508979797 40.0
0 -0.113807439804077 41.0
0 0.0310715436935425 42.0
293.30127679 296.665161132812 43.0
0 0.212863683700562 44.0
0 -0.105995714664459 45.0
0 0.229829728603363 46.0
0 -0.127798736095428 47.0
0 -0.799958944320679 48.0
0 0.5828537940979 49.0
0.077405766366 -0.0275883078575134 50.0
1.9521658064 1.84641075134277 51.0
0 0.322775930166245 52.0
0 -0.000509798526763916 53.0
0 -0.189534604549408 54.0
0 0.148181200027466 55.0
0 0.223924398422241 56.0
0 -0.150073707103729 57.0
0 0.214766979217529 58.0
0 0.332917869091034 59.0
0 -0.205225825309753 60.0
0 0.102753043174744 61.0
0 0.357080042362213 62.0
302.03272967 253.274368286133 0.0
0 0.0349577665328979 1.0
0 0.0575474500656128 2.0
0 0.565829932689667 3.0
0 -0.339351415634155 4.0
0 -0.387405872344971 5.0
0 0.991326212882996 6.0
0 0.174474954605103 7.0
0 0.828650116920471 8.0
0 0.280560195446014 9.0
0 0.507809937000275 10.0
0 -0.223924398422241 11.0
0 -0.0937188863754272 12.0
53.511361669 53.7434425354004 13.0
36.923514082 37.9648513793945 14.0
0 -0.00282919406890869 15.0
0 0.0192293524742126 16.0
33.455848132 33.2377243041992 17.0
0 0.128830313682556 18.0
0 -0.0150514841079712 19.0
0 0.149456024169922 20.0
0 0.133464932441711 21.0
0 0.0235874056816101 22.0
35.140796652 32.970100402832 23.0
0 0.0742908120155334 24.0
0 0.741313219070435 25.0
0 0.130274057388306 26.0
0 0.0797021389007568 27.0
0 -0.0261014699935913 28.0
0 -0.0844938158988953 29.0
0 0.0391260385513306 30.0
0 -0.149741888046265 31.0
0 -0.0226179361343384 32.0
0 -0.00495970249176025 33.0
0 0.05389404296875 34.0
0 0.165629923343658 35.0
0 0.216006517410278 36.0
0 -0.0331631898880005 37.0
0 0.109381675720215 38.0
0 -0.273292899131775 39.0
0 0.150168180465698 40.0
0 -0.12065976858139 41.0
0 -0.131060361862183 42.0
285.35254219 296.913513183594 43.0
0 0.165274381637573 44.0
0 0.0737566351890564 45.0
0 0.106834888458252 46.0
0 0.107583343982697 47.0
0 -0.453592777252197 48.0
0 0.346901714801788 49.0
0.080450465431 0.00264972448348999 50.0
2.0296904768 1.83755826950073 51.0
0 -0.141341090202332 52.0
0 -0.14682936668396 53.0
0 0.0947427749633789 54.0
0 0.256372094154358 55.0
0 0.246225774288177 56.0
0 0.358305633068085 57.0
0 0.594208240509033 58.0
0 -0.174192309379578 59.0
0 0.0690103769302368 60.0
0 0.10293710231781 61.0
0 0.175848662853241 62.0
281.65383391 255.240737915039 0.0
0 -0.427886962890625 1.0
0 0.104824066162109 2.0
0 0.393855810165405 3.0
0 0.018207311630249 4.0
0 -0.0599882006645203 5.0
0 0.330942273139954 6.0
0 0.477685868740082 7.0
0 0.357538163661957 8.0
0 -0.281624555587769 9.0
0 0.33657568693161 10.0
0 -0.457095623016357 11.0
0 -0.270484328269958 12.0
69.937782717 69.2075271606445 13.0
22.520216183 23.915470123291 14.0
0 -0.233373045921326 15.0
0 0.308960616588593 16.0
37.859321768 35.9523735046387 17.0
0 0.0907363891601562 18.0
0 -0.0764835476875305 19.0
0 0.0865957736968994 20.0
0 0.0410350561141968 21.0
0 0.235093533992767 22.0
30.101321779 29.88303565979 23.0
0 0.0623639822006226 24.0
0 0.596377015113831 25.0
0 -0.0487803220748901 26.0
0 -0.0537703037261963 27.0
0 0.124755084514618 28.0
0 -0.0525502562522888 29.0
0 0.076001763343811 30.0
0 -0.119002819061279 31.0
0 -0.0874804258346558 32.0
0 -0.0979639887809753 33.0
0 0.305898666381836 34.0
0 -0.0597735047340393 35.0
0 -0.145373821258545 36.0
0 0.188439786434174 37.0
0 -0.374908804893494 38.0
0 -0.00476634502410889 39.0
0 0.273183941841125 40.0
0 -0.146814823150635 41.0
0 -0.000666379928588867 42.0
287.77091394 297.871276855469 43.0
0 0.053699254989624 44.0
0 0.127683043479919 45.0
0 0.113961458206177 46.0
0 0.0965204834938049 47.0
0 0.178547620773315 48.0
0.9774023286 0.4736547768116 49.0
0.078716506487 0.0406458973884583 50.0
1.8785712839 1.88523292541504 51.0
0 0.169203519821167 52.0
0 0.0330702662467957 53.0
0 -0.0361791253089905 54.0
0 -0.184250235557556 55.0
0 -0.150699973106384 56.0
0 0.059681236743927 57.0
0 0.0997322797775269 58.0
0 0.142954349517822 59.0
0 -0.0118530988693237 60.0
0 0.0733222961425781 61.0
0 0.210423111915588 62.0
240.52189135 252.894989013672 0.0
0 -0.174607992172241 1.0
0 -0.172894597053528 2.0
0 -0.233526706695557 3.0
0 -0.123137354850769 4.0
0 0.193988621234894 5.0
0 -0.0173300504684448 6.0
0 -0.235478520393372 7.0
0 0.217470407485962 8.0
0 -0.0815005898475647 9.0
0 0.0397661328315735 10.0
0 0.188011467456818 11.0
0 -0.343357801437378 12.0
50.730950994 51.1810607910156 13.0
36.387536856 35.7963485717773 14.0
0 -0.158040881156921 15.0
0 0.533034682273865 16.0
36.900484103 35.8299217224121 17.0
0 0.192671000957489 18.0
0 -0.0102414488792419 19.0
0 0.0923473834991455 20.0
0 -0.0600703954696655 21.0
0 0.139105021953583 22.0
34.1855562 34.7744140625 23.0
0 0.0978260636329651 24.0
0 0.314852178096771 25.0
0 -0.103521108627319 26.0
0 -0.02486652135849 27.0
0 0.410060375928879 28.0
0 -0.0912788510322571 29.0
0 0.0536090731620789 30.0
0 0.116995573043823 31.0
0 -0.0755511522293091 32.0
0 0.0293383598327637 33.0
0 -0.134952962398529 34.0
0 0.0777364373207092 35.0
0 -0.11514139175415 36.0
0 -0.0790213346481323 37.0
0 0.233812391757965 38.0
0 -0.334014296531677 39.0
0 -0.0114878416061401 40.0
0 -0.150221168994904 41.0
0 0.00448334217071533 42.0
285.05879149 298.270080566406 43.0
0 0.161623954772949 44.0
0 0.0884391069412231 45.0
0 0.296155154705048 46.0
0 -0.0405220985412598 47.0
0 0.121914148330688 48.0
0 0.309320867061615 49.0
0.095180279051 0.0731112957000732 50.0
1.8512660172 1.79282462596893 51.0
0 -0.164374709129333 52.0
0 -0.235733509063721 53.0
0 0.243017792701721 54.0
0 0.0868815183639526 55.0
0 0.117319464683533 56.0
0 0.245908796787262 57.0
0 0.0610795617103577 58.0
0 -0.736638784408569 59.0
0 1.48803162574768 60.0
0 -0.00303983688354492 61.0
0 0.000454723834991455 62.0
236.6339844 252.571319580078 0.0
0 0.0240797996520996 1.0
0 0.0732675194740295 2.0
0 0.179283678531647 3.0
0 0.058513879776001 4.0
0 0.0374070405960083 5.0
0 0.016607940196991 6.0
0 -0.138794839382172 7.0
0 -0.565754413604736 8.0
0 -0.058717668056488 9.0
0 0.177173256874084 10.0
0 0.69887763261795 11.0
0 -0.253432750701904 12.0
59.449372573 59.0719108581543 13.0
15.745915508 28.6496505737305 14.0
0 -0.0544073581695557 15.0
0 -0.0843546390533447 16.0
32.026563891 31.3394451141357 17.0
0 0.0833535194396973 18.0
0 0.0134069919586182 19.0
0 0.189629912376404 20.0
0 0.11994993686676 21.0
0 0.154710710048676 22.0
34.846677492 30.6129817962646 23.0
0 -0.0145723819732666 24.0
0 -0.0888209939002991 25.0
0 0.15507709980011 26.0
0 -0.143237709999084 27.0
0 0.206719040870667 28.0
0 0.138949513435364 29.0
0 0.0946895480155945 30.0
0 -0.132514595985413 31.0
0 0.0770228505134583 32.0
0 -0.382955074310303 33.0
0 0.0508763194084167 34.0
0 0.704210698604584 35.0
0 0.543164491653442 36.0
0 0.102008461952209 37.0
0 -0.124568343162537 38.0
0 -0.109212756156921 39.0
0 0.261516094207764 40.0
0 -0.0182951688766479 41.0
0 -0.0267817974090576 42.0
287.0933292 299.002258300781 43.0
0 0.151805758476257 44.0
0 -0.0528095960617065 45.0
0 0.249186754226685 46.0
0 -0.694431781768799 47.0
0 0.365014493465424 48.0
3.5451557676 0.671913146972656 49.0
0.082024197858 0.0892074108123779 50.0
2.2337714666 1.80923962593079 51.0
0 -0.0927735567092896 52.0
0 0.0325230956077576 53.0
0 -0.197766900062561 54.0
0 -0.24081826210022 55.0
0 0.180016338825226 56.0
0 0.259611904621124 57.0
0 0.225625932216644 58.0
0 0.124716997146606 59.0
0 -0.27599048614502 60.0
0 0.00495481491088867 61.0
0 0.134689271450043 62.0
215.19573634 253.711837768555 0.0
0 -0.139697909355164 1.0
0 -0.079274594783783 2.0
0 0.188204228878021 3.0
0 -0.327497243881226 4.0
0 -0.429286003112793 5.0
0 -0.0361117124557495 6.0
0 -0.245720505714417 7.0
0 -0.0730609893798828 8.0
0 -0.305458903312683 9.0
0 0.0959764122962952 10.0
0 0.151075124740601 11.0
0 -0.159945607185364 12.0
67.540711229 68.8250732421875 13.0
40.6001798 29.2523403167725 14.0
0 -0.0297554731369019 15.0
0 -0.0696212649345398 16.0
39.016725313 34.8227806091309 17.0
0 0.100741267204285 18.0
0 -0.0290226936340332 19.0
0 0.260203659534454 20.0
0 0.00042259693145752 21.0
0 0.0857793092727661 22.0
35.933291588 36.281005859375 23.0
0 0.11639016866684 24.0
0 0.570190250873566 25.0
0 -0.0468100309371948 26.0
0 0.0069965124130249 27.0
0 0.313791424036026 28.0
0 0.0450640916824341 29.0
0 0.250735998153687 30.0
0 -0.135096430778503 31.0
0 -0.0187376737594604 32.0
0 -0.0655586719512939 33.0
0 0.166599631309509 34.0
0 -0.181263148784637 35.0
0 0.22803521156311 36.0
0 -0.0273815393447876 37.0
0 0.0707029700279236 38.0
0 -0.0763795375823975 39.0
0 0.473225593566895 40.0
0 -0.24237334728241 41.0
0 0.0308330059051514 42.0
287.72554061 294.713623046875 43.0
0 -0.0792141556739807 44.0
0 -0.124602437019348 45.0
0 0.131485998630524 46.0
0 0.212824940681458 47.0
0 0.164434194564819 48.0
0 0.368323534727097 49.0
0.08161973328 0.0287113189697266 50.0
1.5830603668 1.82656466960907 51.0
0 0.0142692923545837 52.0
0 -0.0477021336555481 53.0
0 -0.0343837141990662 54.0
0 -0.0328339338302612 55.0
0 0.172034025192261 56.0
0 -0.0754265189170837 57.0
0 -0.0245523452758789 58.0
0 0.088303804397583 59.0
0 -0.303436636924744 60.0
0 0.0538734197616577 61.0
0 0.176997661590576 62.0
251.69285645 252.666976928711 0.0
0 -0.206237435340881 1.0
0 -0.228557705879211 2.0
0 0.423266232013702 3.0
0 -0.473365426063538 4.0
0 0.0802360773086548 5.0
0 0.0801512002944946 6.0
0 0.0781455039978027 7.0
0 0.579928934574127 8.0
0 -0.00772684812545776 9.0
0 -0.027417004108429 10.0
0 -0.159781754016876 11.0
0 -0.0754458904266357 12.0
71.91577744 71.4994201660156 13.0
28.223806798 28.7866649627686 14.0
0 -0.0348962545394897 15.0
0 0.0999217629432678 16.0
33.813247194 35.4685287475586 17.0
0 0.135801613330841 18.0
0 -0.0114296674728394 19.0
0 0.0395078659057617 20.0
0 0.231337785720825 21.0
0 0.00859463214874268 22.0
28.382888812 24.6964511871338 23.0
0 0.0893470644950867 24.0
0 0.779139041900635 25.0
0 -0.134667992591858 26.0
0 0.03129643201828 27.0
0 0.512382864952087 28.0
0 0.0223492980003357 29.0
0 0.0352492928504944 30.0
0 0.0572035312652588 31.0
0 -0.105968296527863 32.0
0 -0.211576461791992 33.0
0 -0.0330210328102112 34.0
0 0.135683596134186 35.0
0 -0.118219375610352 36.0
0 -0.0664973258972168 37.0
0 0.18536639213562 38.0
0 -0.28561806678772 39.0
0 0.059900164604187 40.0
0 -0.137923538684845 41.0
0 0.159181594848633 42.0
308.54984051 299.338073730469 43.0
0 0.210751295089722 44.0
0 0.0293249487876892 45.0
0 0.149011671543121 46.0
0 0.0744540691375732 47.0
0 0.385129481554031 48.0
2.0499608765 0.498519778251648 49.0
0.074958321275 0.0673040151596069 50.0
1.8423908066 1.87913107872009 51.0
0 -0.00890105962753296 52.0
0 -0.698800563812256 53.0
0 0.0168349742889404 54.0
0 -0.0916709899902344 55.0
0 0.264234781265259 56.0
0 0.0228973627090454 57.0
0 0.157138466835022 58.0
0 -0.242052316665649 59.0
0 0.62183803319931 60.0
0 0.0149495005607605 61.0
0 0.152462422847748 62.0
244.98744882 250.639770507812 0.0
0 -0.455058097839355 1.0
0 -0.309494137763977 2.0
0 0.229695081710815 3.0
0 -0.197452008724213 4.0
0 -0.153699278831482 5.0
0 -0.0747928619384766 6.0
0 -0.0330223441123962 7.0
0 -0.418715953826904 8.0
0 0.105471253395081 9.0
0 0.809721052646637 10.0
0 0.461732298135757 11.0
0 0.0508089065551758 12.0
49.221472722 48.8939819335938 13.0
37.243934768 40.3781051635742 14.0
0 -0.351787805557251 15.0
0 -0.0716520547866821 16.0
32.447358494 33.0346603393555 17.0
0 0.0653837919235229 18.0
0 0.039121687412262 19.0
0 0.116507411003113 20.0
0 0.0960525870323181 21.0
0 0.136951684951782 22.0
34.500268397 33.8348159790039 23.0
0 0.0514588356018066 24.0
0 0.492010354995728 25.0
0 -0.112447559833527 26.0
0 0.0645953416824341 27.0
0 -0.117468774318695 28.0
0 -0.0380669832229614 29.0
0 0.0319681167602539 30.0
0 -0.0639802217483521 31.0
0 0.0211687684059143 32.0
0 -0.0575579404830933 33.0
0 0.154309153556824 34.0
0 -0.00788480043411255 35.0
0 -0.419392824172974 36.0
0 0.190309464931488 37.0
0 0.121415376663208 38.0
0 -0.27132260799408 39.0
0 0.573556303977966 40.0
0 -0.110900700092316 41.0
0 -0.0233386754989624 42.0
287.56160535 295.786987304688 43.0
0 -0.0216615200042725 44.0
0 0.0789068341255188 45.0
0 0.198241829872131 46.0
0 -0.0299564003944397 47.0
0 0.410766303539276 48.0
0 0.493741273880005 49.0
0.097107580134 0.0664698481559753 50.0
1.5984762539 1.84481775760651 51.0
0 -0.077389657497406 52.0
0 0.159636080265045 53.0
0 -0.00955742597579956 54.0
0 0.113879919052124 55.0
0 0.304358124732971 56.0
0 -0.172905683517456 57.0
0 0.325949013233185 58.0
0 0.294521003961563 59.0
0 -0.204076409339905 60.0
0 0.0875923633575439 61.0
0 0.377152323722839 62.0
247.09320596 253.990036010742 0.0
0 0.433042645454407 1.0
0 0.0465651750564575 2.0
0 -0.00826627016067505 3.0
0 -0.0592283010482788 4.0
0 -0.183045744895935 5.0
0 -0.019192099571228 6.0
0 -0.10494464635849 7.0
0 0.644147753715515 8.0
0 -0.124273896217346 9.0
0 0.340400516986847 10.0
0 -0.327240943908691 11.0
0 -0.297233700752258 12.0
54.450613912 55.5725250244141 13.0
34.343487461 35.7981567382812 14.0
0 -0.123492002487183 15.0
0 0.448651552200317 16.0
24.392296269 29.5801944732666 17.0
0 0.0327298641204834 18.0
0 -0.0024338960647583 19.0
0 0.158254504203796 20.0
0 -0.059450089931488 21.0
0 0.15096515417099 22.0
43.532837748 39.1117134094238 23.0
0 0.00258028507232666 24.0
0 0.242791593074799 25.0
0 0.0824518203735352 26.0
0 0.0507925748825073 27.0
0 -0.18333101272583 28.0
0 -0.231199145317078 29.0
0 0.0643234848976135 30.0
0 -0.392638325691223 31.0
0 -0.142395257949829 32.0
0 -0.115593791007996 33.0
0 0.0661661028862 34.0
0 -0.0742812156677246 35.0
0 0.84377646446228 36.0
0 -0.115572452545166 37.0
0 0.115101456642151 38.0
0 -0.169468641281128 39.0
0 0.210312604904175 40.0
0 -0.2595134973526 41.0
0 0.148637771606445 42.0
297.73046784 298.028656005859 43.0
0 0.0394913554191589 44.0
0 -0.0388976335525513 45.0
0 0.0251506567001343 46.0
0 0.379770159721375 47.0
0 -0.0537470579147339 48.0
0 0.37528720498085 49.0
0.065017611978 -0.0210473537445068 50.0
2.1002984395 1.81314182281494 51.0
0 -0.068345308303833 52.0
0 -0.0535494685173035 53.0
0 -0.0584785938262939 54.0
0 0.274811863899231 55.0
0 0.32236585021019 56.0
0 0.290444612503052 57.0
0 0.145275831222534 58.0
0 0.12708842754364 59.0
0 0.29329788684845 60.0
0 0.178104639053345 61.0
0 0.422826945781708 62.0
238.56829679 252.721328735352 0.0
0 0.0555830001831055 1.0
0 0.018460750579834 2.0
0 0.232361137866974 3.0
0 -0.177902460098267 4.0
0 -0.289254307746887 5.0
0 -0.205063223838806 6.0
0 -0.215893268585205 7.0
0 0.121391773223877 8.0
0 -0.0857201814651489 9.0
0 -0.0887249708175659 10.0
0 -0.163955390453339 11.0
0 0.119304955005646 12.0
61.203433403 61.2768745422363 13.0
40.373790485 35.0386390686035 14.0
0 0.0362721681594849 15.0
0 -0.118715941905975 16.0
27.610804155 34.0043182373047 17.0
0 0.111018061637878 18.0
0 -0.0503125190734863 19.0
0 0.131798505783081 20.0
0 -0.132588744163513 21.0
0 0.457961142063141 22.0
37.964466863 37.7887229919434 23.0
0 0.054826021194458 24.0
0 0.783295154571533 25.0
0 -0.187815427780151 26.0
0 -0.0812956690788269 27.0
0 -0.0453293323516846 28.0
0 -0.1257483959198 29.0
0 0.000658869743347168 30.0
0 -0.287498116493225 31.0
0 -0.0673967599868774 32.0
0 -0.0410706996917725 33.0
0 0.192695200443268 34.0
0 -0.0484411716461182 35.0
0 0.206712007522583 36.0
0 -0.0322282314300537 37.0
0 0.254605233669281 38.0
0 -0.246810793876648 39.0
0 0.0570793747901917 40.0
0 -0.238608837127686 41.0
0 0.0726540088653564 42.0
289.81499856 298.190856933594 43.0
0 0.193288326263428 44.0
0 -0.426921486854553 45.0
0 0.0863354206085205 46.0
0 -0.196807682514191 47.0
0 0.260804295539856 48.0
0 0.481438338756561 49.0
0.080239982549 0.0448998808860779 50.0
2.0920683675 1.84539020061493 51.0
0 0.0327735543251038 52.0
0 -0.211004376411438 53.0
0 -0.191248655319214 54.0
0 0.0304545164108276 55.0
0 0.382011592388153 56.0
0 -0.127150356769562 57.0
0 0.255030512809753 58.0
0 0.153792321681976 59.0
0 -0.115750074386597 60.0
0 0.0408810377120972 61.0
0 0.15451443195343 62.0
231.4158619 253.347396850586 0.0
0 -0.0265344381332397 1.0
0 -0.0594960451126099 2.0
0 -0.141259133815765 3.0
0 -0.32179594039917 4.0
0 -0.35871434211731 5.0
0 -0.0410183668136597 6.0
0 -0.314441442489624 7.0
0 0.249081075191498 8.0
0 -0.0321522951126099 9.0
0 -0.192487180233002 10.0
0 -0.163498759269714 11.0
0 -0.0162311196327209 12.0
58.892119169 56.3964729309082 13.0
43.022170451 38.2855033874512 14.0
0 -0.221842050552368 15.0
0 0.300944030284882 16.0
26.080215447 29.1650161743164 17.0
0 0.0260365009307861 18.0
0 0.0804962515830994 19.0
0 0.139573574066162 20.0
0 0.0647203922271729 21.0
0 0.124396324157715 22.0
37.167939797 36.0397567749023 23.0
0 0.0842810869216919 24.0
0 0.943174839019775 25.0
0 0.110775649547577 26.0
0 -0.0304673910140991 27.0
0 0.544440090656281 28.0
0 -0.0535936951637268 29.0
0 0.0988867282867432 30.0
0 0.190426349639893 31.0
0 0.0703179836273193 32.0
0 0.155042111873627 33.0
0 0.259941041469574 34.0
0 -0.16983824968338 35.0
0 -0.278966665267944 36.0
0 0.0476905107498169 37.0
0 0.158554255962372 38.0
0 -0.229227423667908 39.0
0 0.305672585964203 40.0
0 -0.0766011476516724 41.0
0 0.0452783107757568 42.0
288.44292383 298.014465332031 43.0
0 0.128217220306396 44.0
0 0.157506823539734 45.0
0 0.231593549251556 46.0
0 0.0201431512832642 47.0
0 0.322328984737396 48.0
0.94196107254 0.616465926170349 49.0
0.093090072867 0.0804862976074219 50.0
1.6760415212 1.88571548461914 51.0
0 0.0808892846107483 52.0
0 -0.014373779296875 53.0
0 0.333555221557617 54.0
0 0.174257934093475 55.0
0 -0.142981946468353 56.0
0 -0.062675952911377 57.0
0 0.11315530538559 58.0
0 -0.134630262851715 59.0
0 -0.0768939852714539 60.0
0 0.0398589372634888 61.0
0 0.108843088150024 62.0
259.83994203 252.724060058594 0.0
0 -0.256796956062317 1.0
0 -0.0589445829391479 2.0
0 0.39160418510437 3.0
0 -0.0144844055175781 4.0
0 0.262483596801758 5.0
0 -0.175245344638824 6.0
0 0.132987380027771 7.0
0 -0.455053329467773 8.0
0 0.419084846973419 9.0
0 0.0448572635650635 10.0
0 0.0468247532844543 11.0
0 -0.265515565872192 12.0
67.357114068 66.8408889770508 13.0
42.735096612 28.6702480316162 14.0
0 0.339280128479004 15.0
0 0.0488113164901733 16.0
37.638693255 35.2897415161133 17.0
0 0.190528273582458 18.0
0 -0.0191581249237061 19.0
0 0.0384057760238647 20.0
0 0.185751736164093 21.0
0 0.0239287614822388 22.0
29.28688007 28.7737045288086 23.0
0 -0.0472756624221802 24.0
0 0.248638689517975 25.0
0 -0.228905916213989 26.0
0 -0.183407068252563 27.0
0 0.160824716091156 28.0
0 0.100051462650299 29.0
0 0.148318946361542 30.0
0 -0.228882431983948 31.0
0 -0.16742479801178 32.0
0 -0.0331355929374695 33.0
0 0.0375868082046509 34.0
0 -0.117842853069305 35.0
0 0.31040757894516 36.0
0 0.0294073820114136 37.0
0 0.192582726478577 38.0
0 -0.147125840187073 39.0
0 0.545273303985596 40.0
0 -0.0983688831329346 41.0
0 0.024127721786499 42.0
291.19377683 297.424041748047 43.0
0 0.0940583944320679 44.0
0 0.134459912776947 45.0
0 0.0778113007545471 46.0
0 0.0546832680702209 47.0
0 -0.357500195503235 48.0
0 0.555967032909393 49.0
0.092492395225 -0.0148139595985413 50.0
2.1822047799 1.81976771354675 51.0
0 -0.0741146206855774 52.0
0 0.0587775707244873 53.0
0 -0.149258315563202 54.0
0 0.235476076602936 55.0
0 0.310879707336426 56.0
0 0.212288618087769 57.0
0 0.398873507976532 58.0
0 0.313038229942322 59.0
0 -0.14573734998703 60.0
0 0.0639987587928772 61.0
0 0.327198684215546 62.0
249.7342568 246.999771118164 0.0
0 -0.265251278877258 1.0
0 0.134872138500214 2.0
0 -0.0334142446517944 3.0
0 -0.23884105682373 4.0
0 0.23759001493454 5.0
0 0.0580788850784302 6.0
0 -0.378865957260132 7.0
0 -0.220979928970337 8.0
0 -0.058635950088501 9.0
0 -0.193998336791992 10.0
0 -0.363893866539001 11.0
0 -0.0291992425918579 12.0
55.983310316 54.6422653198242 13.0
39.349298436 47.4668273925781 14.0
0 -0.360193967819214 15.0
0 0.253782570362091 16.0
28.769802159 28.0570201873779 17.0
0 0.0635565519332886 18.0
0 0.0348666906356812 19.0
0 0.0827716588973999 20.0
0 0.280066192150116 21.0
0 0.0424372553825378 22.0
28.886249846 27.1975154876709 23.0
0 0.1225266456604 24.0
20.064828518 1.2035905122757 25.0
0 -0.241170167922974 26.0
0 -0.348636150360107 27.0
0 0.146038830280304 28.0
0 0.363114833831787 29.0
0 -0.174000084400177 30.0
0 0.422335386276245 31.0
0 -0.152228653430939 32.0
0 -0.0548097491264343 33.0
0 -0.425713300704956 34.0
0 0.426361799240112 35.0
0 0.196880161762238 36.0
0 0.171576619148254 37.0
0 0.267483472824097 38.0
0 -0.253061890602112 39.0
0 -0.0764585733413696 40.0
0 -0.182951748371124 41.0
0 -0.0694733858108521 42.0
319.42421708 296.307891845703 43.0
0 0.141003847122192 44.0
0 -0.0452090501785278 45.0
0 0.237415611743927 46.0
0 0.1078040599823 47.0
0 0.429084837436676 48.0
0 0.465485513210297 49.0
0.10591559171 0.0931888818740845 50.0
2.0195445708 1.79996919631958 51.0
0 0.0233269929885864 52.0
0 0.138141274452209 53.0
0 -0.0150860548019409 54.0
0 -0.160612106323242 55.0
0 -0.194657266139984 56.0
0 -0.0848952531814575 57.0
0 -0.219371914863586 58.0
0 0.225540459156036 59.0
0 -0.217139482498169 60.0
0 0.132336616516113 61.0
0 0.283811569213867 62.0
262.34667389 252.944152832031 0.0
0 -0.15747344493866 1.0
0 -0.00479549169540405 2.0
0 0.171459138393402 3.0
0 -0.221910834312439 4.0
0 -0.0106087327003479 5.0
0 0.109067618846893 6.0
0 0.0226136445999146 7.0
0 0.0839067697525024 8.0
0 -0.0203524231910706 9.0
0 0.108139336109161 10.0
0 0.0570652484893799 11.0
0 -0.0862009525299072 12.0
59.6971132 60.4799766540527 13.0
38.203402497 32.8405952453613 14.0
0 0.0296398401260376 15.0
0 -0.136833071708679 16.0
28.68551223 31.9875049591064 17.0
0 0.0369964838027954 18.0
0 -0.00853133201599121 19.0
0 0.0777456760406494 20.0
0 0.20277863740921 21.0
0 -0.00178378820419312 22.0
27.957151674 25.9473266601562 23.0
0 0.0279344320297241 24.0
0 0.365250170230865 25.0
0 -0.0712609887123108 26.0
0 -0.017002284526825 27.0
0 0.173691153526306 28.0
0 -0.0391898155212402 29.0
0 0.0986111164093018 30.0
0 -0.0896455049514771 31.0
0 -0.0558024048805237 32.0
0 -0.120913445949554 33.0
0 0.233373701572418 34.0
0 0.228371679782867 35.0
0 0.427350580692291 36.0
0 -0.0795362591743469 37.0
0 0.13461709022522 38.0
0 -0.393170952796936 39.0
0 0.516419768333435 40.0
0 -0.16848748922348 41.0
0 0.102166056632996 42.0
311.19147889 296.404388427734 43.0
0 0.106405138969421 44.0
0 0.204663097858429 45.0
0 0.0377796292304993 46.0
0 0.176497459411621 47.0
0 0.407033830881119 48.0
0 0.330649197101593 49.0
0.070464162568 0.0679917335510254 50.0
1.9549566721 1.86071002483368 51.0
0 -0.179430902004242 52.0
0 -0.307029247283936 53.0
0 -0.108958184719086 54.0
0 0.20512068271637 55.0
0 -0.0459432005882263 56.0
0 0.423342227935791 57.0
0 0.311555773019791 58.0
0 0.180617809295654 59.0
0 -0.110572457313538 60.0
0 0.125390231609344 61.0
0 0.264511525630951 62.0
223.06373393 251.906784057617 0.0
0 0.139200091362 1.0
0 -0.0461162328720093 2.0
0 0.053322970867157 3.0
0 -0.404873132705688 4.0
0 -0.256126403808594 5.0
0 0.0898592472076416 6.0
0 -0.254256010055542 7.0
0 0.241960287094116 8.0
0 -0.156097114086151 9.0
0 -0.171779870986938 10.0
0 -0.0234829783439636 11.0
0 0.0329807996749878 12.0
65.9390245 65.8317108154297 13.0
36.628116556 30.8850231170654 14.0
0 -0.102866172790527 15.0
0 -0.00628435611724854 16.0
36.307275383 34.0663032531738 17.0
0 0.0938835144042969 18.0
0 -0.0158432722091675 19.0
0 0.0743881464004517 20.0
0 -0.0358876585960388 21.0
0 0.040449857711792 22.0
40.359030908 38.6923904418945 23.0
0 -0.0738257765769958 24.0
0 0.354566186666489 25.0
0 -0.0831794142723083 26.0
0 -0.0348177552223206 27.0
0 0.761874496936798 28.0
0 0.10752260684967 29.0
0 0.0597802400588989 30.0
0 -0.0968621969223022 31.0
0 -0.0897334218025208 32.0
0 -0.123283326625824 33.0
0 0.0112023949623108 34.0
0 0.0300992131233215 35.0
0 -0.174688458442688 36.0
0 0.0763717889785767 37.0
0 0.239522874355316 38.0
0 -0.425822854042053 39.0
0 0.207503020763397 40.0
0 -0.28594183921814 41.0
0 0.0775047540664673 42.0
313.13159493 293.044006347656 43.0
0 -0.12252014875412 44.0
0 0.0644285082817078 45.0
0 0.152320384979248 46.0
0 -0.273049831390381 47.0
0 -0.16443133354187 48.0
0 0.430859744548798 49.0
0.074620194132 0.0128076076507568 50.0
1.9976541855 1.86453676223755 51.0
0 0.379043251276016 52.0
0 0.149084448814392 53.0
0 -0.0601381063461304 54.0
0 0.00583928823471069 55.0
0 0.0649785995483398 56.0
0 -0.0248723030090332 57.0
0 0.235413670539856 58.0
0 0.0869531631469727 59.0
0 -0.259676337242126 60.0
0 -0.00761663913726807 61.0
0 0.0782831907272339 62.0
258.4712988 253.392578125 0.0
0 -0.0453083515167236 1.0
0 -0.0172264575958252 2.0
0 0.520538449287415 3.0
0 0.0691910982131958 4.0
0 0.331424087285995 5.0
0 -0.249716520309448 6.0
0 0.286772727966309 7.0
0 -0.539772748947144 8.0
0 -0.427918434143066 9.0
0 -0.0400116443634033 10.0
0 0.239998877048492 11.0
0 -0.0900804996490479 12.0
71.157359765 70.6487350463867 13.0
37.723159568 33.2814598083496 14.0
0 -0.123520970344543 15.0
0 0.0695496797561646 16.0
39.499359333 35.4949417114258 17.0
0 0.146712303161621 18.0
0 -0.0736871361732483 19.0
0 0.101442039012909 20.0
0 -0.107462406158447 21.0
0 0.286138355731964 22.0
36.874433179 36.0755195617676 23.0
0 0.130015969276428 24.0
0 0.844201564788818 25.0
0 0.115500807762146 26.0
0 -0.0523659586906433 27.0
0 1.24255311489105 28.0
0 0.0416693091392517 29.0
0 0.064314067363739 30.0
0 -0.10167121887207 31.0
0 -0.034026026725769 32.0
0 -0.373273372650146 33.0
0 -0.261781096458435 34.0
0 0.344247251749039 35.0
0 0.496472269296646 36.0
0 -0.00694632530212402 37.0
0 0.202391982078552 38.0
0 -0.154727935791016 39.0
0 0.543464422225952 40.0
0 -0.142681479454041 41.0
0 -0.0403622388839722 42.0
311.56635111 297.535461425781 43.0
0 0.0766315460205078 44.0
0 0.0508355498313904 45.0
0 0.389602839946747 46.0
0 -0.237134218215942 47.0
0 0.173938989639282 48.0
0 0.393167555332184 49.0
0.099727854963 -0.0239701271057129 50.0
1.4307932821 1.88013553619385 51.0
0 0.362717390060425 52.0
0 -0.269805908203125 53.0
0 -0.00596076250076294 54.0
0 -0.0287662148475647 55.0
0 -0.173536956310272 56.0
0 -0.110805571079254 57.0
0 0.267571330070496 58.0
0 0.317638903856277 59.0
0 -0.26597785949707 60.0
0 0.136643767356873 61.0
0 0.266261398792267 62.0
302.95985872 254.498764038086 0.0
0 0.130227208137512 1.0
0 0.143178582191467 2.0
0 -0.193122267723083 3.0
0 -0.344817399978638 4.0
0 0.271047830581665 5.0
0 0.410360932350159 6.0
0 0.993443608283997 7.0
0 0.663799405097961 8.0
0 -0.222425818443298 9.0
0 0.538043856620789 10.0
0 0.872557044029236 11.0
0 -0.288970470428467 12.0
50.958177298 51.8598289489746 13.0
35.606354115 37.6804885864258 14.0
0 0.0997405052185059 15.0
0 0.828205287456512 16.0
39.441759774 38.3243408203125 17.0
0 0.430345505475998 18.0
0 -0.0211661458015442 19.0
0 0.0342296361923218 20.0
0 0.228654026985168 21.0
0 -0.0482006072998047 22.0
28.253024995 25.1111221313477 23.0
0 0.044299304485321 24.0
0 0.389679431915283 25.0
0 0.167233288288116 26.0
0 -0.256358861923218 27.0
0 0.783488571643829 28.0
0 0.0369131565093994 29.0
0 0.145059525966644 30.0
0 -0.0976365804672241 31.0
0 0.00757777690887451 32.0
0 -0.0933921933174133 33.0
0 0.358109831809998 34.0
0 0.0228973627090454 35.0
0 -0.149003505706787 36.0
0 -0.047063946723938 37.0
0 0.410429626703262 38.0
0 -0.273827314376831 39.0
0 0.399003654718399 40.0
0 -0.173168241977692 41.0
0 -0.0673477649688721 42.0
290.8022228 292.563751220703 43.0
0 0.0918634533882141 44.0
0 -0.00375425815582275 45.0
0 0.127480804920197 46.0
0 -0.474553823471069 47.0
0 0.277541518211365 48.0
0 0.259548902511597 49.0
0.093729941839 0.0998061299324036 50.0
1.7306271921 1.81614351272583 51.0
0 -0.0363801121711731 52.0
0 -0.355644106864929 53.0
0 0.00230324268341064 54.0
0 0.22514820098877 55.0
0 0.344927668571472 56.0
0 -0.139553904533386 57.0
0 0.0227664709091187 58.0
0 0.211908459663391 59.0
0 -0.20358008146286 60.0
0 0.135949432849884 61.0
0 0.193444669246674 62.0
246.75535333 255.460983276367 0.0
0 -0.302999019622803 1.0
0 -0.307594180107117 2.0
0 0.285161852836609 3.0
0 -0.0715363025665283 4.0
0 0.188456475734711 5.0
0 0.0780202746391296 6.0
0 -0.0492181777954102 7.0
0 0.134093523025513 8.0
0 0.189711332321167 9.0
0 -0.0372067093849182 10.0
0 -0.0562567114830017 11.0
0 -0.170807123184204 12.0
67.443635552 68.950553894043 13.0
37.974647887 29.8335876464844 14.0
0 -0.0473473072052002 15.0
0 -0.138184487819672 16.0
31.439478866 35.4271011352539 17.0
0 0.0848730206489563 18.0
0 0.00690877437591553 19.0
0 0.0744423866271973 20.0
0 0.0785921812057495 21.0
0 -0.190566539764404 22.0
40.423062733 38.0741958618164 23.0
0 0.0391740202903748 24.0
0 0.215814828872681 25.0
0 -0.0820432305335999 26.0
0 -0.0881674289703369 27.0
0 0.658991515636444 28.0
0 -0.0445671081542969 29.0
0 -0.0244796872138977 30.0
0 0.273934841156006 31.0
0 0.203363299369812 32.0
0 -0.264619469642639 33.0
0 0.0457772016525269 34.0
0 0.118887543678284 35.0
0 0.353726178407669 36.0
0 -0.0679097175598145 37.0
0 0.159939289093018 38.0
0 -0.215285062789917 39.0
0 0.0417680144309998 40.0
0 -0.17234343290329 41.0
0 -0.0552413463592529 42.0
301.40284389 295.812103271484 43.0
0 0.0486096143722534 44.0
0 0.0116780996322632 45.0
0 0.107365310192108 46.0
0 -0.0469362735748291 47.0
0 -0.0976631641387939 48.0
0 0.461880505084991 49.0
0.07882727628 0.0297439098358154 50.0
1.7281058789 1.7990574836731 51.0
0 -0.0953214168548584 52.0
0 -0.0842733979225159 53.0
0 -0.0824166536331177 54.0
0 -0.0163941979408264 55.0
0 0.140424132347107 56.0
0 0.24537181854248 57.0
0 0.209647834300995 58.0
0 0.175621271133423 59.0
0 -0.239054203033447 60.0
0 0.0754555463790894 61.0
0 0.228281617164612 62.0
239.4102587 252.013412475586 0.0
0 -0.0100975036621094 1.0
0 -0.0488033294677734 2.0
0 0.321253299713135 3.0
0 -0.323551654815674 4.0
0 -0.27983558177948 5.0
0 -0.00488030910491943 6.0
0 -0.0978545546531677 7.0
0 0.234461307525635 8.0
0 -0.118774175643921 9.0
0 -0.100148320198059 10.0
0 -0.0896963477134705 11.0
0 0.0677531361579895 12.0
51.630477482 51.8053894042969 13.0
50.409177176 47.1559371948242 14.0
0 -0.132450461387634 15.0
0 -0.110655188560486 16.0
28.600080496 31.3611068725586 17.0
0 0.0368208885192871 18.0
0 0.000109434127807617 19.0
0 0.167730927467346 20.0
0 0.110488772392273 21.0
0 0.129926264286041 22.0
31.901942293 31.5475597381592 23.0
0 -0.00189042091369629 24.0
0 0.159428119659424 25.0
0 0.0447204113006592 26.0
0 -0.0970008373260498 27.0
0 0.0135354399681091 28.0
0 -0.0669471025466919 29.0
0 0.0382838249206543 30.0
0 0.175486207008362 31.0
0 0.083071231842041 32.0
0 0.0795494914054871 33.0
0 0.300483345985413 34.0
0 0.0702457427978516 35.0
0 0.365053713321686 36.0
0 0.0335950255393982 37.0
0 0.226863026618958 38.0
0 -0.191889405250549 39.0
0 0.586351811885834 40.0
0 -0.170394778251648 41.0
0 0.0269225835800171 42.0
292.28426679 294.316833496094 43.0
0 0.00968003273010254 44.0
0 -0.0396305918693542 45.0
0 0.134547591209412 46.0
0 -0.555745244026184 47.0
0 0.422598749399185 48.0
0.66646253983 0.461467623710632 49.0
0.097045015093 0.111839294433594 50.0
2.0688533976 1.86084842681885 51.0
0 0.0673270225524902 52.0
0 0.091320812702179 53.0
0 0.126892507076263 54.0
0 -0.0219184160232544 55.0
0 -0.0193332433700562 56.0
0 0.0154759287834167 57.0
0 0.158847987651825 58.0
0 0.126838803291321 59.0
0 -0.0521228909492493 60.0
0 0.0704325437545776 61.0
0 0.205746471881866 62.0
279.58684323 251.837829589844 0.0
0 -0.0229456424713135 1.0
0 -0.0259279012680054 2.0
0 0.0467938184738159 3.0
0 0.13691520690918 4.0
0 0.396501898765564 5.0
0 0.577554225921631 6.0
0 0.493010252714157 7.0
0 0.0173217058181763 8.0
0 -0.930577397346497 9.0
0 0.172479033470154 10.0
0 -0.443296074867249 11.0
0 -0.18077540397644 12.0
60.811241434 60.9853515625 13.0
31.046755268 31.2489490509033 14.0
0 -0.032305121421814 15.0
0 -0.0332708358764648 16.0
29.520966241 30.3781871795654 17.0
0 0.0512373447418213 18.0
0 -0.0326361656188965 19.0
0 0.113126754760742 20.0
0 -0.0137506723403931 21.0
0 0.289734363555908 22.0
34.103073711 33.2116775512695 23.0
0 0.0488411784172058 24.0
0 0.41342830657959 25.0
0 -0.053492546081543 26.0
0 0.0350013375282288 27.0
0 0.149948537349701 28.0
0 -0.0564457178115845 29.0
0 -0.00644087791442871 30.0
0 0.18302309513092 31.0
0 0.0685234665870667 32.0
0 -0.123728454113007 33.0
0 0.233815312385559 34.0
0 -0.06756991147995 35.0
0 0.0172780752182007 36.0
0 0.103495538234711 37.0
0 0.0108875632286072 38.0
0 -0.216411828994751 39.0
0 0.575880467891693 40.0
0 -0.196351945400238 41.0
0 0.0649102926254272 42.0
299.16791168 295.391387939453 43.0
0 0.183962345123291 44.0
0 0.0146363377571106 45.0
0 0.165807783603668 46.0
0 -0.0689963102340698 47.0
0 0.320985108613968 48.0
0 0.458461344242096 49.0
0.080006698255 0.0337130427360535 50.0
2.0212136382 1.86176538467407 51.0
0 -0.144658088684082 52.0
0 0.0109297633171082 53.0
0 -0.0205467939376831 54.0
0 0.132476806640625 55.0
0 -0.150449573993683 56.0
0 -0.0231762528419495 57.0
0 0.205077409744263 58.0
0 0.174935698509216 59.0
0 -0.199149549007416 60.0
0 0.0817349553108215 61.0
0 0.173993289470673 62.0
225.41471717 252.778930664062 0.0
0 0.138515472412109 1.0
0 -0.205198049545288 2.0
0 -0.185175061225891 3.0
0 -0.293163061141968 4.0
0 -0.142931997776031 5.0
0 -0.0082632303237915 6.0
0 -0.19908994436264 7.0
0 0.118114173412323 8.0
0 0.128336071968079 9.0
0 0.236239910125732 10.0
0 0.0845361351966858 11.0
0 -0.211423516273499 12.0
75.458597559 75.5509262084961 13.0
28.733769808 26.5257625579834 14.0
0 -0.109377503395081 15.0
0 -0.0747839212417603 16.0
41.290846831 38.5821151733398 17.0
0 0.0624960660934448 18.0
0 0.0178974866867065 19.0
0 -0.0254878401756287 20.0
0 0.0704445838928223 21.0
0 0.224911153316498 22.0
31.922558824 32.0804176330566 23.0
0 0.0295625329017639 24.0
0 0.387315630912781 25.0
0 -0.182198166847229 26.0
0 -0.105785191059113 27.0
0 0.393373429775238 28.0
0 0.127875685691833 29.0
0 -0.133847713470459 30.0
0 -0.067339301109314 31.0
0 -0.0945807099342346 32.0
0 -0.321610689163208 33.0
0 -0.52390193939209 34.0
0 0.280591130256653 35.0
0 0.274526953697205 36.0
0 -0.0309381484985352 37.0
0 0.127749264240265 38.0
0 -0.592633485794067 39.0
0 0.464685380458832 40.0
0 -0.152129948139191 41.0
0 -0.0247387886047363 42.0
316.91099372 297.604705810547 43.0
0 0.0811933279037476 44.0
0 0.0989274382591248 45.0
0 0.0707084536552429 46.0
0 0.180938601493835 47.0
0 0.366576194763184 48.0
0 0.47342774271965 49.0
0.076421123912 0.0308583974838257 50.0
1.8259803841 1.84171915054321 51.0
0 -0.0933699011802673 52.0
0 0.166328549385071 53.0
0 0.350549697875977 54.0
0 0.274139046669006 55.0
0 0.00408977270126343 56.0
0 0.522616267204285 57.0
0 0.00911271572113037 58.0
0 0.193538188934326 59.0
0 -0.0405163168907166 60.0
0 0.0729900598526001 61.0
0 0.0286431312561035 62.0
221.80881799 255.766937255859 0.0
0 -0.126400589942932 1.0
0 -0.0510201454162598 2.0
0 0.350920438766479 3.0
0 -0.29352605342865 4.0
0 -0.371641874313354 5.0
0 0.000828862190246582 6.0
0 -0.383842468261719 7.0
0 0.567840337753296 8.0
0 -0.0522663593292236 9.0
0 1.02395153045654 10.0
0 -0.287653088569641 11.0
0 -0.00909769535064697 12.0
55.080488394 54.9675941467285 13.0
23.806101469 39.5820655822754 14.0
0 -0.0494652986526489 15.0
0 -0.218816041946411 16.0
29.705788544 33.6797180175781 17.0
0 0.123349964618683 18.0
0 -0.058430552482605 19.0
0 0.143296599388123 20.0
0 0.148286819458008 21.0
0 0.0162221789360046 22.0
29.371852987 25.8268814086914 23.0
0 0.0230717062950134 24.0
0 -0.172931551933289 25.0
0 -0.0737882256507874 26.0
0 -0.104433238506317 27.0
0 0.0983668565750122 28.0
0 -0.114115178585052 29.0
0 0.0980689525604248 30.0
0 0.294454455375671 31.0
0 0.240886867046356 32.0
0 -0.0683586597442627 33.0
0 0.0865037441253662 34.0
0 0.139942467212677 35.0
0 -0.511218428611755 36.0
0 -0.0311160683631897 37.0
0 0.127877593040466 38.0
0 -0.268015623092651 39.0
0 0.411669850349426 40.0
0 -0.145237267017365 41.0
0 -0.00536394119262695 42.0
284.83336163 296.564453125 43.0
0 0.0840478539466858 44.0
0 0.033500075340271 45.0
0 0.255036175251007 46.0
0 0.122868716716766 47.0
0 0.436681032180786 48.0
0 0.465951383113861 49.0
0.082940674443 0.0385441780090332 50.0
2.2128975286 1.90743517875671 51.0
0 0.176048278808594 52.0
0 -0.0265558362007141 53.0
0 0.160075545310974 54.0
0 0.0927788019180298 55.0
0 0.229756057262421 56.0
0 -0.256339430809021 57.0
0 -0.288963913917542 58.0
0 0.257906675338745 59.0
0 -0.191667675971985 60.0
0 0.273891925811768 61.0
0 0.381760001182556 62.0
285.37011784 254.675704956055 0.0
0 -0.399811625480652 1.0
0 0.0261242985725403 2.0
0 0.165650546550751 3.0
0 0.16441810131073 4.0
0 0.091206967830658 5.0
0 0.11376965045929 6.0
0 0.605238556861877 7.0
0 -0.455888986587524 8.0
0 0.0827354788780212 9.0
0 0.135628819465637 10.0
0 1.00196218490601 11.0
0 -0.235065698623657 12.0
53.916539587 52.2022247314453 13.0
35.992029034 34.0896263122559 14.0
0 -0.0888701677322388 15.0
0 0.00601112842559814 16.0
32.49615338 34.0882530212402 17.0
0 0.130681872367859 18.0
0 0.0152839422225952 19.0
0 0.0190693736076355 20.0
0 0.205119669437408 21.0
0 -0.0580311417579651 22.0
26.70583649 27.905424118042 23.0
0 0.0885021090507507 24.0
1.2126831654 0.324145376682281 25.0
0 0.0670009851455688 26.0
0 -0.28841757774353 27.0
0 -0.0672823786735535 28.0
0 -0.040931761264801 29.0
0 0.0435488820075989 30.0
0 0.126833558082581 31.0
0 0.166002511978149 32.0
0 0.168628334999084 33.0
0 -0.158155620098114 34.0
0 -0.0328775644302368 35.0
0 0.00540250539779663 36.0
0 0.151856958866119 37.0
0 -0.20690256357193 38.0
0 -0.0936379432678223 39.0
0 0.0129576325416565 40.0
0 -0.0521851778030396 41.0
0 0.140266895294189 42.0
287.26699285 294.401916503906 43.0
0 0.150608539581299 44.0
0 0.055656909942627 45.0
0 0.165043830871582 46.0
0 -0.378845453262329 47.0
0 0.343961447477341 48.0
1.8381315509 0.486930727958679 49.0
0.066300080933 0.0929915308952332 50.0
1.6463063873 1.81398987770081 51.0
0 -0.00334107875823975 52.0
0 -0.0718989968299866 53.0
0 0.232901573181152 54.0
0 0.243105590343475 55.0
0 0.152470707893372 56.0
0 0.389075934886932 57.0
0 0.30184069275856 58.0
0 0.103550434112549 59.0
0 -0.111445069313049 60.0
0 0.0521981120109558 61.0
0 0.160388112068176 62.0
261.52040326 252.7978515625 0.0
0 0.164791464805603 1.0
0 0.125675797462463 2.0
0 0.149729311466217 3.0
0 0.00829070806503296 4.0
0 0.197898507118225 5.0
0 0.103860855102539 6.0
0 0.119361400604248 7.0
0 0.196318328380585 8.0
0 0.2055304646492 9.0
0 -0.298547506332397 10.0
0 0.0521441698074341 11.0
0 -0.0699058771133423 12.0
70.818061754 69.3837585449219 13.0
34.797736183 31.4764499664307 14.0
0 0.0218636989593506 15.0
0 0.066078245639801 16.0
29.615922232 31.4746761322021 17.0
0 0.107164084911346 18.0
0 -0.0415005683898926 19.0
0 0.0960485935211182 20.0
0 0.233436405658722 21.0
0 -0.132580101490021 22.0
27.770254747 25.2403755187988 23.0
0 0.00627583265304565 24.0
0 0.246396005153656 25.0
0 -0.150310039520264 26.0
0 -0.129319071769714 27.0
0 0.0186730623245239 28.0
0 -0.0917161107063293 29.0
0 0.104380130767822 30.0
0 0.22772479057312 31.0
0 0.0453881621360779 32.0
0 -0.0641747713088989 33.0
0 0.16932213306427 34.0
0 -0.113966822624207 35.0
0 0.144416093826294 36.0
0 0.0816580653190613 37.0
0 0.114324450492859 38.0
0 -0.383382201194763 39.0
0 0.19378799200058 40.0
0 -0.139849960803986 41.0
0 -0.0998845100402832 42.0
289.84860216 297.295959472656 43.0
0 -0.0335659384727478 44.0
0 0.126925528049469 45.0
0 0.232585430145264 46.0
0 -0.123623788356781 47.0
0 0.0589832663536072 48.0
0 0.435361534357071 49.0
0.070241881099 0.04204261302948 50.0
2.2117969638 1.80413508415222 51.0
0 0.100993275642395 52.0
0 -0.0427122712135315 53.0
0 -0.0723769664764404 54.0
0 0.229735314846039 55.0
0 0.0648675560951233 56.0
0 0.149182796478271 57.0
0 0.0374672412872314 58.0
0 -0.164599478244781 59.0
0 -0.0773046016693115 60.0
0 0.0301083326339722 61.0
0 0.0448554158210754 62.0
267.52910161 251.948547363281 0.0
0 -0.266194462776184 1.0
0 0.0476634502410889 2.0
0 0.160133123397827 3.0
0 -0.220914006233215 4.0
0 -0.325406312942505 5.0
0 0.09772127866745 6.0
0 0.00110810995101929 7.0
0 -0.420289039611816 8.0
0 0.0216293931007385 9.0
0 -0.00522744655609131 10.0
0 -0.0239956378936768 11.0
0 -0.0274889469146729 12.0
57.948408858 56.6070404052734 13.0
48.69157276 41.1179618835449 14.0
0 -0.250290393829346 15.0
0 -0.154911041259766 16.0
26.179323087 32.583911895752 17.0
0 0.0326634645462036 18.0
0 0.0282651782035828 19.0
0 0.123815953731537 20.0
0 0.0753452777862549 21.0
0 0.176789522171021 22.0
35.083281152 33.2691307067871 23.0
0 -0.0553513169288635 24.0
0 0.532033205032349 25.0
0 0.0712676048278809 26.0
0 -0.262542247772217 27.0
0 1.15873467922211 28.0
0 0.342419922351837 29.0
0 0.0453447103500366 30.0
0 0.20426070690155 31.0
0 0.0188947319984436 32.0
0 0.0798181295394897 33.0
0 0.0639985203742981 34.0
0 0.0687711834907532 35.0
0 0.413985401391983 36.0
0 0.185406267642975 37.0
0 -0.13535064458847 38.0
0 -0.148576498031616 39.0
0 0.335460871458054 40.0
0 -0.11227810382843 41.0
0 -0.0413229465484619 42.0
302.64790712 295.809478759766 43.0
0 0.0819065570831299 44.0
0 0.207095801830292 45.0
0 0.220236957073212 46.0
0 -0.0496398210525513 47.0
0 0.21342408657074 48.0
1.5542339903 0.561231553554535 49.0
0.084084051117 0.0731049776077271 50.0
1.8979146465 1.84741687774658 51.0
0 0.175196051597595 52.0
0 -0.59978199005127 53.0
0 -0.0360153913497925 54.0
0 -0.176300406455994 55.0
0 -0.0951743125915527 56.0
0 0.8205646276474 57.0
0 0.238304972648621 58.0
0 0.106270432472229 59.0
0 -0.200492799282074 60.0
0 0.114364206790924 61.0
0 0.41539603471756 62.0
276.7675781 251.975814819336 0.0
0 -0.0961084365844727 1.0
0 -0.0362995266914368 2.0
0 0.342960566282272 3.0
0 0.00618946552276611 4.0
0 0.0625084638595581 5.0
0 -0.24567699432373 6.0
0 0.224034905433655 7.0
0 -0.476417303085327 8.0
0 0.0632545351982117 9.0
0 -0.134905099868774 10.0
0 -0.0953696966171265 11.0
0 -0.176787376403809 12.0
53.446777152 54.7628784179688 13.0
42.805262223 37.9608993530273 14.0
0 0.0509665012359619 15.0
0 0.0356210470199585 16.0
32.807290346 33.1445236206055 17.0
0 0.0948734283447266 18.0
0 0.013257622718811 19.0
0 0.168501138687134 20.0
0 -0.0823167562484741 21.0
0 0.0334652662277222 22.0
39.955478534 40.6454696655273 23.0
0 0.109372735023499 24.0
0 1.03008675575256 25.0
0 -0.144378662109375 26.0
0 -0.0362616777420044 27.0
0 0.363507986068726 28.0
0 0.0306345224380493 29.0
0 0.0767938494682312 30.0
0 0.0182137489318848 31.0
0 -0.0175285935401917 32.0
0 -0.0138202905654907 33.0
0 0.417810291051865 34.0
0 -0.0877516269683838 35.0
0 -0.114320576190948 36.0
0 -0.0525188446044922 37.0
0 0.227212309837341 38.0
0 -0.245977163314819 39.0
0 0.572516798973083 40.0
0 -0.179126381874084 41.0
0 0.135332822799683 42.0
292.623804 296.799865722656 43.0
0 0.0988425016403198 44.0
0 0.1069375872612 45.0
0 0.311355292797089 46.0
0 -0.0356782674789429 47.0
0 -0.233568906784058 48.0
0 0.51819920539856 49.0
0.099055284191 -0.0138417482376099 50.0
1.7524794653 1.87502574920654 51.0
0 0.12342643737793 52.0
0 -0.142359793186188 53.0
0 0.12641179561615 54.0
0 0.0976510047912598 55.0
0 0.129966795444489 56.0
0 -0.0807867050170898 57.0
0 0.201533257961273 58.0
0 0.107783854007721 59.0
0 -0.10638153553009 60.0
0 0.0399934053421021 61.0
0 0.173730492591858 62.0
271.87419061 252.759521484375 0.0
0 -0.0601969957351685 1.0
0 0.231377482414246 2.0
0 0.323412418365479 3.0
0 0.0846357941627502 4.0
0 0.166386663913727 5.0
0 -0.239779472351074 6.0
0 0.267587125301361 7.0
0 -0.0508122444152832 8.0
0 -0.122335731983185 9.0
0 0.036881148815155 10.0
0 0.139137625694275 11.0
0 -0.07393878698349 12.0
58.560599776 55.6420516967773 13.0
49.13534769 43.5881958007812 14.0
0 0.00253558158874512 15.0
0 -0.126355946063995 16.0
29.051016888 31.9196968078613 17.0
0 0.0803414583206177 18.0
0 -0.0149029493331909 19.0
0 0.387149453163147 20.0
0 -0.100427687168121 21.0
0 0.915819883346558 22.0
38.277677209 38.0692100524902 23.0
0 0.0525637865066528 24.0
0 0.396660566329956 25.0
0 -0.0857565402984619 26.0
0 0.128762483596802 27.0
0 -0.10517406463623 28.0
0 -0.0395466089248657 29.0
0 0.140781402587891 30.0
0 0.0904463529586792 31.0
0 -0.00986498594284058 32.0
0 -0.0916072130203247 33.0
0 0.494327008724213 34.0
0 -0.0212444663047791 35.0
0 -0.151779472827911 36.0
0 -0.119942963123322 37.0
0 0.318495839834213 38.0
0 -0.29831862449646 39.0
0 -0.0918827056884766 40.0
0 -0.196649610996246 41.0
0 -0.0180869102478027 42.0
295.57065636 298.397857666016 43.0
0 0.0738844871520996 44.0
0 0.0712421536445618 45.0
0 0.543915867805481 46.0
0 -0.823224067687988 47.0
0 -0.0248851180076599 48.0
0 0.517105221748352 49.0
0.086185050852 -0.0537996292114258 50.0
1.9707942146 1.89064431190491 51.0
0 -0.0326870679855347 52.0
0 -0.269165515899658 53.0
0 0.121265888214111 54.0
0 0.129581868648529 55.0
0 0.194345712661743 56.0
0 0.159195721149445 57.0
0 0.285911440849304 58.0
0 0.214673638343811 59.0
0 -0.233414888381958 60.0
0 -0.00362223386764526 61.0
0 0.182380139827728 62.0
264.28042817 251.841354370117 0.0
0 -0.206370115280151 1.0
0 -0.0110498666763306 2.0
0 -0.512389183044434 3.0
0 -0.0950347781181335 4.0
0 0.362453192472458 5.0
0 0.972859025001526 6.0
0 0.459830731153488 7.0
0 0.214080929756165 8.0
0 -0.52973484992981 9.0
0 0.315067410469055 10.0
0 -0.854899048805237 11.0
0 -0.181861579418182 12.0
68.738699064 67.271354675293 13.0
22.890194531 30.1561107635498 14.0
0 -0.0238407850265503 15.0
0 -0.0640795826911926 16.0
33.592704445 37.0997734069824 17.0
0 0.122994422912598 18.0
0 -0.0386675596237183 19.0
0 0.0296449661254883 20.0
0 -0.0325011610984802 21.0
0 0.273361563682556 22.0
40.165505483 37.4858627319336 23.0
0 0.0879833698272705 24.0
0 0.749501883983612 25.0
0 -0.20586109161377 26.0
0 -0.0247024893760681 27.0
0 0.297905594110489 28.0
0 -0.00580322742462158 29.0
0 -0.116891503334045 30.0
0 0.061659574508667 31.0
0 0.0869621634483337 32.0
0 -0.0299885272979736 33.0
0 0.12527072429657 34.0
0 -0.130582809448242 35.0
0 0.0405876636505127 36.0
0 -0.0695328712463379 37.0
0 0.0997944474220276 38.0
0 -0.0561656951904297 39.0
0 0.345324754714966 40.0
0 -0.128800332546234 41.0
0 0.31451427936554 42.0
288.41706292 297.424987792969 43.0
0 0.329298138618469 44.0
0 -0.333788394927979 45.0
0 -0.0143494009971619 46.0
0 0.16514927148819 47.0
0 0.31772118806839 48.0
0 0.379558622837067 49.0
0.095738469781 0.0440282225608826 50.0
1.7935757239 1.81960678100586 51.0
0 -0.0846090912818909 52.0
0 0.434779465198517 53.0
0 -0.0379464626312256 54.0
0 0.200624585151672 55.0
0 0.200576603412628 56.0
0 0.103785157203674 57.0
0 0.519906342029572 58.0
0 -0.115006327629089 59.0
0 -0.209791302680969 60.0
0 0.0913430452346802 61.0
0 0.233746767044067 62.0
237.6739775 251.571960449219 0.0
0 0.05744469165802 1.0
0 -0.0979966521263123 2.0
0 0.170774519443512 3.0
0 -0.311835765838623 4.0
0 -0.313704133033752 5.0
0 0.0725623965263367 6.0
0 -0.168085336685181 7.0
0 0.110650181770325 8.0
0 0.0350530743598938 9.0
0 -0.273960590362549 10.0
0 -0.408772230148315 11.0
0 -0.111675441265106 12.0
64.081995144 62.4451141357422 13.0
22.910176443 32.5962600708008 14.0
0 -0.0569597482681274 15.0
0 0.00867933034896851 16.0
26.79861857 31.7563724517822 17.0
0 0.0613455772399902 18.0
0 0.0017930269241333 19.0
0 0.102368712425232 20.0
0 0.0731306672096252 21.0
0 0.122843503952026 22.0
31.14661445 28.828218460083 23.0
0 0.0227853059768677 24.0
0 0.188263416290283 25.0
0 0.0629088878631592 26.0
0 -0.0839455127716064 27.0
0 0.198956966400146 28.0
0 -0.153127789497375 29.0
0 0.00981807708740234 30.0
0 0.586464107036591 31.0
0 -0.219614744186401 32.0
0 -0.142546355724335 33.0
0 -0.0282248258590698 34.0
0 0.629387199878693 35.0
0 0.436353087425232 36.0
0 0.213185489177704 37.0
0 -0.143649876117706 38.0
0 -0.0901405811309814 39.0
0 0.064595103263855 40.0
0 -0.107769072055817 41.0
0 0.0447841882705688 42.0
293.56772263 296.787658691406 43.0
0 0.164853096008301 44.0
0 0.0267592668533325 45.0
0 0.0853169560432434 46.0
0 0.236744344234467 47.0
0 0.0367643237113953 48.0
0 0.381010085344315 49.0
0.080117158891 0.0459935665130615 50.0
2.1270971768 1.80107402801514 51.0
0 -0.033435583114624 52.0
0 -0.227958083152771 53.0
0 -0.143744230270386 54.0
0 0.0481833815574646 55.0
0 0.233744740486145 56.0
0 -0.150102615356445 57.0
0 0.290119886398315 58.0
0 0.11010879278183 59.0
0 0.0540010929107666 60.0
0 0.0761981010437012 61.0
0 0.357876837253571 62.0
238.08467049 254.235412597656 0.0
0 -0.25701379776001 1.0
0 0.113173723220825 2.0
0 0.15743225812912 3.0
0 -0.0103717446327209 4.0
0 -0.155315458774567 5.0
0 -0.0666460394859314 6.0
0 0.118927001953125 7.0
0 -0.16117650270462 8.0
0 -0.0615307688713074 9.0
0 0.0960165858268738 10.0
0 -0.0964657068252563 11.0
0 -0.0368361473083496 12.0
49.393890078 44.4012794494629 13.0
56.169972799 52.7562294006348 14.0
0 -0.0619875192642212 15.0
0 0.239839613437653 16.0
41.442478508 39.4305076599121 17.0
0 0.154118478298187 18.0
0 -0.0879101753234863 19.0
0 0.12015837430954 20.0
0 -0.0246446132659912 21.0
0 0.73529040813446 22.0
42.169465219 39.3367919921875 23.0
0 0.150151610374451 24.0
0 0.86656528711319 25.0
0 0.273876667022705 26.0
0 -0.0868155956268311 27.0
0 0.627145886421204 28.0
0 -0.0198222994804382 29.0
0 0.0722085237503052 30.0
0 -0.239188551902771 31.0
0 -0.249358177185059 32.0
0 -0.573400020599365 33.0
0 0.514560103416443 34.0
0 0.0507856607437134 35.0
0 0.803821802139282 36.0
0 0.0498085021972656 37.0
0 -0.325860619544983 38.0
0 -0.0848345756530762 39.0
0 0.0426552891731262 40.0
0 -0.234139204025269 41.0
0 0.211703062057495 42.0
297.45743926 294.938262939453 43.0
0 0.177623629570007 44.0
0 0.218601167201996 45.0
0 0.0720406174659729 46.0
0 0.25636488199234 47.0
0 0.347908109426498 48.0
2.5193321356 0.0301845669746399 49.0
0.088491126352 -0.00735026597976685 50.0
1.7220568437 1.93224716186523 51.0
0 0.081731379032135 52.0
0 -0.13246351480484 53.0
0 -0.00204908847808838 54.0
0 -0.148627042770386 55.0
0 0.00558948516845703 56.0
0 -0.0898101329803467 57.0
0 0.151138782501221 58.0
0 0.257118821144104 59.0
0 -0.236921668052673 60.0
0 0.226430714130402 61.0
0 0.146802723407745 62.0
231.69073308 254.021102905273 0.0
0 -0.279531121253967 1.0
0 -0.333176732063293 2.0
0 -0.0095900297164917 3.0
0 -0.292514324188232 4.0
0 0.0846329927444458 5.0
0 -0.134829640388489 6.0
0 -0.256863236427307 7.0
0 0.17132568359375 8.0
0 0.105443120002747 9.0
0 -0.0370235443115234 10.0
0 0.714635789394379 11.0
0 -0.03589928150177 12.0
69.329267189 68.4074859619141 13.0
34.02027327 30.8503475189209 14.0
0 -0.0768957138061523 15.0
0 -0.137234151363373 16.0
30.643068212 33.3762359619141 17.0
0 0.0614985227584839 18.0
0 -0.0192990899085999 19.0
0 0.0205085277557373 20.0
0 0.279846429824829 21.0
0 0.0100987553596497 22.0
27.573599633 24.8562297821045 23.0
0 -0.00132662057876587 24.0
0 0.0342631936073303 25.0
0 -7.74860382080078e-07 26.0
0 -0.101004481315613 27.0
0 0.34948268532753 28.0
0 -0.0656393766403198 29.0
0 -0.102065145969391 30.0
0 0.156214594841003 31.0
0 0.131962716579437 32.0
0 -0.180498242378235 33.0
0 -0.151989579200745 34.0
0 -0.149259150028229 35.0
0 -0.0337665677070618 36.0
0 0.0540145039558411 37.0
0 0.36886602640152 38.0
0 -0.347911953926086 39.0
0 -0.0877467393875122 40.0
0 -0.206861555576324 41.0
0 -0.0603073835372925 42.0
304.40464963 297.875427246094 43.0
0 0.0987015962600708 44.0
0 0.0517688989639282 45.0
0 0.172350406646729 46.0
0 0.0507050156593323 47.0
0 0.394292771816254 48.0
0 0.453260242938995 49.0
0.083854484104 0.00700652599334717 50.0
1.8431739396 1.88551759719849 51.0
0 0.018176257610321 52.0
0 0.050258457660675 53.0
0 -0.101286649703979 54.0
0 0.0393027663230896 55.0
0 0.0159852504730225 56.0
0 0.175363004207611 57.0
0 0.123656272888184 58.0
0 -0.126830995082855 59.0
0 -0.115850687026978 60.0
0 0.0814656019210815 61.0
0 0.186166048049927 62.0
278.43598285 252.463134765625 0.0
0 -0.279794812202454 1.0
0 -0.0662463903427124 2.0
0 0.335437893867493 3.0
0 -0.0659759044647217 4.0
0 0.222477912902832 5.0
0 0.103274345397949 6.0
0 0.319270670413971 7.0
0 0.367893218994141 8.0
0 0.0316972732543945 9.0
0 0.163212716579437 10.0
0 -0.126860976219177 11.0
0 -0.0842211842536926 12.0
74.55616663 73.6218185424805 13.0
15.971431952 25.6338977813721 14.0
0 -0.0306297540664673 15.0
0 0.283832967281342 16.0
29.912285666 35.8750228881836 17.0
0 0.119661569595337 18.0
0 0.0361372828483582 19.0
0 0.0514690279960632 20.0
0 0.135296821594238 21.0
0 0.0559828281402588 22.0
29.128546305 27.5412120819092 23.0
0 0.022541344165802 24.0
0 0.438142716884613 25.0
0 -0.0661242008209229 26.0
0 -0.148654699325562 27.0
0 0.251140415668488 28.0
0 -0.0357542634010315 29.0
0 0.0442703366279602 30.0
0 -0.0953270196914673 31.0
0 -0.109823107719421 32.0
0 -0.452712535858154 33.0
0 0.195691645145416 34.0
0 0.446074724197388 35.0
0 -0.158893406391144 36.0
0 0.121488094329834 37.0
0 0.10282951593399 38.0
0 -0.287976384162903 39.0
0 0.395379900932312 40.0
0 -0.162518084049225 41.0
0 0.0862072706222534 42.0
297.30166453 297.527191162109 43.0
0 0.101957023143768 44.0
0 0.139074265956879 45.0
0 0.144896149635315 46.0
0 -0.20271897315979 47.0
0 0.207859516143799 48.0
0 0.336814850568771 49.0
0.096140738484 0.0369945764541626 50.0
1.7937046086 1.87588810920715 51.0
0 0.0187431573867798 52.0
0 -0.215925335884094 53.0
0 0.081831693649292 54.0
0 0.21371054649353 55.0
0 0.225299417972565 56.0
0 0.0264203548431396 57.0
0 0.0506271123886108 58.0
0 0.248929023742676 59.0
0 -0.253164410591125 60.0
0 0.0611361861228943 61.0
0 0.239328026771545 62.0
263.28697621 252.21044921875 0.0
0 0.0918458700180054 1.0
0 -0.0959576964378357 2.0
0 0.136965811252594 3.0
0 -0.0750084519386292 4.0
0 0.382898211479187 5.0
0 0.360531628131866 6.0
0 0.311771154403687 7.0
0 -0.329082727432251 8.0
0 0.0273656249046326 9.0
0 -0.0930694341659546 10.0
0 0.107135355472565 11.0
0 -0.0154669284820557 12.0
50.426212613 50.6924705505371 13.0
53.90100255 49.8136444091797 14.0
0 -0.0849486589431763 15.0
0 0.175467133522034 16.0
38.202167545 38.2499160766602 17.0
0 0.147627949714661 18.0
0 -0.0762038230895996 19.0
0 0.106450796127319 20.0
0 -0.0723140239715576 21.0
0 1.23092532157898 22.0
38.953191523 40.049919128418 23.0
0 0.0105971693992615 24.0
0 0.787169277667999 25.0
0 0.0439155697822571 26.0
0 -0.0638307929039001 27.0
0 0.0759618878364563 28.0
0 0.172162234783173 29.0
0 0.115725934505463 30.0
0 -0.321269273757935 31.0
0 -0.059052586555481 32.0
0 -0.0492644906044006 33.0
0 0.197929322719574 34.0
0 -0.123722672462463 35.0
0 0.284528970718384 36.0
0 -0.0923704504966736 37.0
0 0.208096265792847 38.0
0 -0.363989114761353 39.0
0 0.45008572936058 40.0
0 -0.169476926326752 41.0
0 0.0259053707122803 42.0
301.52837854 297.412048339844 43.0
0 0.243053197860718 44.0
0 0.0126615762710571 45.0
0 0.305976748466492 46.0
0 0.0105795860290527 47.0
0 -0.420786619186401 48.0
0 0.412112295627594 49.0
0.089416773702 -0.00866371393203735 50.0
2.2057718245 1.79766035079956 51.0
0 -0.00439178943634033 52.0
0 0.618076145648956 53.0
0 0.0438565611839294 54.0
0 0.0331792831420898 55.0
0 0.00215601921081543 56.0
0 0.0316193103790283 57.0
0 0.296148627996445 58.0
0 0.208252668380737 59.0
0 -0.150998711585999 60.0
0 0.116032600402832 61.0
0 0.216816306114197 62.0
247.25848636 251.413101196289 0.0
0 -0.0969234704971313 1.0
0 -0.204236745834351 2.0
0 0.151471257209778 3.0
0 -0.280823111534119 4.0
0 0.271858930587769 5.0
0 0.222368776798248 6.0
0 0.149480104446411 7.0
0 0.473771214485168 8.0
0 0.170041441917419 9.0
0 0.378960698843002 10.0
0 -0.24760913848877 11.0
0 -0.01216721534729 12.0
66.40480806 66.953483581543 13.0
26.769849994 28.9768390655518 14.0
0 -0.0296636819839478 15.0
0 0.107696533203125 16.0
35.326647863 33.6506767272949 17.0
0 0.1082524061203 18.0
0 0.0213373899459839 19.0
0 0.141910910606384 20.0
0 0.0582388639450073 21.0
0 0.168878674507141 22.0
35.6833838 37.5077705383301 23.0
0 0.18204402923584 24.0
0 1.37253141403198 25.0
0 -0.0298784375190735 26.0
0 -0.0325064659118652 27.0
0 -0.23189377784729 28.0
0 -0.0582424402236938 29.0
0 -0.015699028968811 30.0
0 -0.0139091014862061 31.0
0 -0.0664324164390564 32.0
0 0.0234832763671875 33.0
0 -0.0988829135894775 34.0
0 0.295968621969223 35.0
0 0.122361302375793 36.0
0 -0.0149433612823486 37.0
0 0.183001160621643 38.0
0 -0.325442314147949 39.0
0 0.386682122945786 40.0
0 -0.136904001235962 41.0
0 0.00756525993347168 42.0
296.08796145 297.832702636719 43.0
0 0.20261013507843 44.0
0 -0.000662267208099365 45.0
0 0.190213441848755 46.0
0 0.105913519859314 47.0
0 0.371950209140778 48.0
0 0.296859383583069 49.0
0.090433382039 0.054269015789032 50.0
1.6931523154 1.89234733581543 51.0
0 -0.00638991594314575 52.0
0 0.0586295127868652 53.0
0 -0.131893038749695 54.0
0 -0.0280610918998718 55.0
0 0.286015450954437 56.0
0 0.217381298542023 57.0
0 0.124738216400146 58.0
0 0.148241579532623 59.0
0 -0.194680511951447 60.0
0 0.000995159149169922 61.0
0 0.093616247177124 62.0
275.35752484 252.242965698242 0.0
0 -0.276887178421021 1.0
0 0.287680208683014 2.0
0 0.369651049375534 3.0
0 0.265768885612488 4.0
0 0.412030726671219 5.0
0 0.430139899253845 6.0
0 0.47969526052475 7.0
0 0.859667301177979 8.0
0 0.198598980903625 9.0
0 -0.0157536268234253 10.0
0 -0.13778692483902 11.0
0 -0.191611170768738 12.0
54.670564935 53.9803123474121 13.0
49.192071519 48.1142578125 14.0
0 -0.288591384887695 15.0
0 0.000366032123565674 16.0
36.467767105 32.9331893920898 17.0
0 0.0245071649551392 18.0
0 0.0193694829940796 19.0
0 0.183249235153198 20.0
0 -0.0502433776855469 21.0
0 0.226052582263947 22.0
40.173398488 41.5793762207031 23.0
0 0.0782905220985413 24.0
0 1.15912199020386 25.0
0 0.206131815910339 26.0
0 -0.0216403007507324 27.0
0 0.0898409485816956 28.0
0 -0.0457034111022949 29.0
0 0.0551032423973083 30.0
0 -0.141347646713257 31.0
0 0.043545663356781 32.0
0 -0.289172410964966 33.0
0 -0.0354269742965698 34.0
0 0.195791721343994 35.0
0 0.416772365570068 36.0
0 -0.0274414420127869 37.0
0 0.020261824131012 38.0
0 -0.121671199798584 39.0
0 1.07348418235779 40.0
0 -0.117345750331879 41.0
0 0.293201088905334 42.0
305.75759646 295.764312744141 43.0
0 0.0859224796295166 44.0
0 0.200181424617767 45.0
0 0.147770762443542 46.0
0 -0.240646123886108 47.0
0 -0.1279456615448 48.0
0 0.439913034439087 49.0
0.10424575315 -0.01925128698349 50.0
2.206269212 1.92823266983032 51.0
0 0.410158574581146 52.0
0 -0.407191872596741 53.0
0 -0.171298742294312 54.0
0 -0.0604099631309509 55.0
0 0.198689222335815 56.0
0 0.377980172634125 57.0
0 0.339885830879211 58.0
0 0.302283734083176 59.0
0 -0.20061057806015 60.0
0 0.0918142795562744 61.0
0 0.220363914966583 62.0
240.05350175 253.267990112305 0.0
0 0.0169776678085327 1.0
0 0.0683102011680603 2.0
0 0.0792520642280579 3.0
0 -0.243739604949951 4.0
0 0.047067403793335 5.0
0 0.164949238300323 6.0
0 0.138784050941467 7.0
0 0.24937516450882 8.0
0 -0.293169498443604 9.0
0 -0.0206339955329895 10.0
0 -0.225063323974609 11.0
0 -0.128444612026215 12.0
63.322695604 62.3354454040527 13.0
29.663199033 38.0196647644043 14.0
0 0.0295186042785645 15.0
0 -0.100898921489716 16.0
32.535684782 34.1940078735352 17.0
0 0.120052695274353 18.0
0 -0.00364220142364502 19.0
0 0.150074899196625 20.0
0 0.117522835731506 21.0
0 0.129163682460785 22.0
38.721581019 36.5183715820312 23.0
0 0.0691112279891968 24.0
0 0.744293332099915 25.0
0 0.190494537353516 26.0
0 -0.0865184664726257 27.0
0 0.592634856700897 28.0
0 -0.131648182868958 29.0
0 0.106393814086914 30.0
0 -0.209469437599182 31.0
0 -0.108885943889618 32.0
0 -0.411497712135315 33.0
0 -1.05517733097076 34.0
0 0.0237146615982056 35.0
0 0.140699982643127 36.0
0 0.0270842313766479 37.0
0 0.17915153503418 38.0
0 -0.0490940809249878 39.0
0 0.768041670322418 40.0
0 -0.0751305818557739 41.0
0 0.0745234489440918 42.0
315.08246822 296.131011962891 43.0
0 0.179610729217529 44.0
0 0.220451176166534 45.0
0 0.00226718187332153 46.0
0 -0.0988945960998535 47.0
0 0.295273423194885 48.0
0.75763162536 0.560908138751984 49.0
0.072996309458 0.10962575674057 50.0
1.8564770029 1.88702178001404 51.0
0 -0.0845472812652588 52.0
0 0.338778376579285 53.0
0 0.158757209777832 54.0
0 -0.00693345069885254 55.0
0 -0.0702351331710815 56.0
0 -0.0665225982666016 57.0
0 0.00864839553833008 58.0
0 0.101923108100891 59.0
0 -0.0511004328727722 60.0
0 0.0502234697341919 61.0
0 0.103562474250793 62.0
253.12686262 253.779998779297 0.0
0 -0.141224503517151 1.0
0 0.00447458028793335 2.0
0 -0.228672742843628 3.0
0 -0.505935907363892 4.0
0 0.112755715847015 5.0
0 0.083298921585083 6.0
0 -0.0572056770324707 7.0
0 0.294584214687347 8.0
0 -0.0232886672019958 9.0
0 0.261406123638153 10.0
0 -0.185032486915588 11.0
0 -0.0479722619056702 12.0
50.893101907 52.7060813903809 13.0
37.764263818 44.5434303283691 14.0
0 -0.221836566925049 15.0
0 -0.179035902023315 16.0
29.409633191 34.267635345459 17.0
0 0.0723552703857422 18.0
0 0.00390303134918213 19.0
0 0.119489789009094 20.0
0 -0.0263768434524536 21.0
0 0.0588973760604858 22.0
37.427133781 36.1383666992188 23.0
0 -0.0371403694152832 24.0
0 -0.106895864009857 25.0
0 -0.135739088058472 26.0
0 0.0133358240127563 27.0
0 -0.0132240056991577 28.0
0 0.0142338871955872 29.0
0 -0.0258457660675049 30.0
0 -0.217373609542847 31.0
0 -0.0473297238349915 32.0
0 0.0047791600227356 33.0
0 0.14282089471817 34.0
0 -0.0583466291427612 35.0
0 -0.207188785076141 36.0
0 0.0261667966842651 37.0
0 -0.0417729020118713 38.0
0 -0.064635157585144 39.0
0 0.0312432050704956 40.0
0 -0.0658514499664307 41.0
0 -0.0257498025894165 42.0
285.5591431 298.025451660156 43.0
0 -0.0513424873352051 44.0
0 0.193158268928528 45.0
0 0.0954530835151672 46.0
0 0.223500669002533 47.0
0 -0.726449370384216 48.0
0 0.370667546987534 49.0
0.066117360775 -0.0229123830795288 50.0
2.1400181273 1.8355975151062 51.0
0 0.0264092683792114 52.0
0 -0.116830408573151 53.0
0 -0.00118148326873779 54.0
0 -0.0881379842758179 55.0
0 -0.0758041739463806 56.0
0 0.168315351009369 57.0
0 0.242653608322144 58.0
0 -0.553736448287964 59.0
0 0.643078625202179 60.0
0 0.0555527210235596 61.0
0 0.093245267868042 62.0
243.31665632 246.560699462891 0.0
0 -0.342884302139282 1.0
0 0.0996772050857544 2.0
0 0.243733644485474 3.0
0 -0.0128840208053589 4.0
0 0.200538992881775 5.0
0 0.0235762000083923 6.0
0 0.0930460095405579 7.0
0 0.594172358512878 8.0
0 -0.168004035949707 9.0
0 0.342422068119049 10.0
0 -0.0138987302780151 11.0
0 -0.235528826713562 12.0
56.307807227 54.4765663146973 13.0
38.135639672 35.6967697143555 14.0
0 -0.178626894950867 15.0
0 0.229598343372345 16.0
41.718718759 37.6034774780273 17.0
0 0.17671674489975 18.0
0 0.00568246841430664 19.0
0 0.0993427038192749 20.0
0 -0.154203295707703 21.0
0 0.799743354320526 22.0
41.364386369 38.9272193908691 23.0
0 -0.0493581891059875 24.0
0 0.574651837348938 25.0
0 -0.133700013160706 26.0
0 0.125867545604706 27.0
0 -0.0539074540138245 28.0
0 0.16754823923111 29.0
0 0.152663230895996 30.0
0 0.093957781791687 31.0
0 0.207860231399536 32.0
0 0.000293493270874023 33.0
0 -1.23454570770264 34.0
0 0.335833787918091 35.0
0 0.22142505645752 36.0
0 0.0218436717987061 37.0
0 0.178942441940308 38.0
0 -0.200926899909973 39.0
0 -0.105195641517639 40.0
0 -0.128217756748199 41.0
0 -0.143145799636841 42.0
308.72303722 297.70849609375 43.0
0 0.120322227478027 44.0
0 -0.0721562504768372 45.0
0 0.0869740843772888 46.0
0 -0.318366169929504 47.0
0 0.184856057167053 48.0
0 0.308563888072968 49.0
0.10466111364 -0.0293474197387695 50.0
2.1360196043 1.88290452957153 51.0
0 -0.245745897293091 52.0
0 0.075514554977417 53.0
0 -0.197958827018738 54.0
0 -0.0158975124359131 55.0
0 0.108241379261017 56.0
0 0.424424111843109 57.0
0 0.314315170049667 58.0
0 0.461205124855042 59.0
0 0.0259243249893188 60.0
0 0.0164698362350464 61.0
0 0.194596707820892 62.0
240.98978493 252.588317871094 0.0
0 0.381684303283691 1.0
0 -0.0125680565834045 2.0
0 0.24254983663559 3.0
0 -0.101656079292297 4.0
0 -0.0423024892807007 5.0
0 0.050951361656189 6.0
0 -0.207038342952728 7.0
0 -0.0577236413955688 8.0
0 -0.0322868227958679 9.0
0 -0.386531233787537 10.0
0 -0.0848139524459839 11.0
0 -0.00335901975631714 12.0
65.386252069 65.9456024169922 13.0
37.448316432 30.1544437408447 14.0
0 -0.0284249782562256 15.0
0 0.205085813999176 16.0
28.490731958 34.1711883544922 17.0
0 0.0849728584289551 18.0
0 0.0231586694717407 19.0
0 0.0703240633010864 20.0
0 0.0526642203330994 21.0
0 0.131782531738281 22.0
36.610466664 36.3298492431641 23.0
0 0.065895676612854 24.0
0 0.545918047428131 25.0
0 -0.07045578956604 26.0
0 0.0329551696777344 27.0
0 0.624068260192871 28.0
0 -0.0168229937553406 29.0
0 0.0796467065811157 30.0
0 0.110651612281799 31.0
0 0.0622975826263428 32.0
0 0.0699530839920044 33.0
0 -0.0722858309745789 34.0
0 0.122920989990234 35.0
0 0.0128615498542786 36.0
0 -0.0533146262168884 37.0
0 0.263599991798401 38.0
0 -0.19500994682312 39.0
0 0.0916460156440735 40.0
0 -0.215434312820435 41.0
0 0.146778583526611 42.0
283.35715287 297.346130371094 43.0
0 0.215044617652893 44.0
0 0.200548231601715 45.0
0 0.0914500951766968 46.0
0 0.0195844769477844 47.0
0 0.383983582258224 48.0
0 0.226287126541138 49.0
0.089966727012 0.0934047698974609 50.0
2.2483497882 1.84208738803864 51.0
0 -0.0528453588485718 52.0
0 0.158668637275696 53.0
0 0.023822009563446 54.0
0 0.180653035640717 55.0
0 -0.0270008444786072 56.0
0 0.314160466194153 57.0
0 -0.0397893786430359 58.0
0 0.146336436271667 59.0
0 -0.179088056087494 60.0
0 0.0821727514266968 61.0
0 0.146114110946655 62.0
232.79105847 253.755355834961 0.0
0 -0.43564510345459 1.0
0 -0.359845876693726 2.0
0 0.141698777675629 3.0
0 -0.317057728767395 4.0
0 0.0257733464241028 5.0
0 0.0349739193916321 6.0
0 -0.092238187789917 7.0
0 -0.210971593856812 8.0
0 0.414993345737457 9.0
0 0.0427623987197876 10.0
0 -0.266847014427185 11.0
0 0.0559405088424683 12.0
68.23933816 68.0055618286133 13.0
19.850124225 29.6165580749512 14.0
0 -0.210482478141785 15.0
0 0.00199174880981445 16.0
37.444178629 37.2661933898926 17.0
0 0.102137804031372 18.0
0 -0.0421237349510193 19.0
0 0.107417702674866 20.0
0 0.00260406732559204 21.0
0 0.106317758560181 22.0
38.713392688 36.7744369506836 23.0
0 0.0653122663497925 24.0
0 0.355251431465149 25.0
0 -0.0170937180519104 26.0
0 -0.16147780418396 27.0
0 0.0233317017555237 28.0
0 -0.066112756729126 29.0
0 0.0323950052261353 30.0
0 -0.0210438966751099 31.0
0 0.090427577495575 32.0
0 -0.263036489486694 33.0
0 -0.0611554980278015 34.0
0 0.222481966018677 35.0
0 0.459138572216034 36.0
0 -0.00191521644592285 37.0
0 0.281770646572113 38.0
0 -0.201143860816956 39.0
0 0.23149049282074 40.0
0 -0.159817814826965 41.0
0 -0.0871235132217407 42.0
294.58802633 298.275360107422 43.0
0 0.184871673583984 44.0
0 0.0592613816261292 45.0
0 0.194948971271515 46.0
0 -0.40335750579834 47.0
0 0.364157170057297 48.0
0 0.351272493600845 49.0
0.095238279977 0.0498101711273193 50.0
2.1742159562 1.88336682319641 51.0
0 -0.0369293689727783 52.0
0 -0.241884231567383 53.0
0 -0.0901997089385986 54.0
0 0.0926588177680969 55.0
0 -0.0755119919776917 56.0
0 0.472636640071869 57.0
0 0.0272061824798584 58.0
0 0.0549925565719604 59.0
0 -0.20462167263031 60.0
0 0.0363723039627075 61.0
0 0.0742059350013733 62.0
273.89602809 253.004989624023 0.0
0 -0.295841097831726 1.0
0 -0.0381672978401184 2.0
0 0.31621265411377 3.0
0 -0.0161855220794678 4.0
0 0.154698550701141 5.0
0 0.128021121025085 6.0
0 0.321351528167725 7.0
0 0.164862990379333 8.0
0 -0.388471722602844 9.0
0 0.0193980932235718 10.0
0 -0.207060098648071 11.0
0 -0.128892183303833 12.0
59.628445198 60.1880798339844 13.0
31.85481259 36.0799140930176 14.0
0 0.261156558990479 15.0
0 -0.164703786373138 16.0
27.946023088 29.8072166442871 17.0
0 0.063996434211731 18.0
0 0.00630456209182739 19.0
0 0.139742910861969 20.0
0 0.16306471824646 21.0
0 0.112227141857147 22.0
36.602684639 34.4823913574219 23.0
0 0.0628392100334167 24.0
0 0.838671267032623 25.0
0 0.00462579727172852 26.0
0 -0.0535461902618408 27.0
0 0.628855168819427 28.0
0 -0.0398301482200623 29.0
0 0.085821807384491 30.0
0 -0.0190269947052002 31.0
0 0.0944306254386902 32.0
0 -0.0493456125259399 33.0
0 0.156909167766571 34.0
0 -0.173265993595123 35.0
0 0.0830182433128357 36.0
0 0.0111395716667175 37.0
0 0.152819752693176 38.0
0 -0.118949770927429 39.0
0 0.435637980699539 40.0
0 -0.147405385971069 41.0
0 0.0373797416687012 42.0
287.01050639 295.098327636719 43.0
0 0.0250029563903809 44.0
0 0.0609121918678284 45.0
0 0.183828234672546 46.0
0 -0.350330352783203 47.0
0 0.350906550884247 48.0
0 0.64056932926178 49.0
0.071999223948 0.0283571481704712 50.0
1.8788176877 1.79151344299316 51.0
0 0.0555342435836792 52.0
0 -0.149416327476501 53.0
0 -0.129258513450623 54.0
0 0.0808281898498535 55.0
0 0.218186140060425 56.0
0 0.105582118034363 57.0
0 0.139592349529266 58.0
0 0.172458231449127 59.0
0 -0.0208060741424561 60.0
0 0.0452069640159607 61.0
0 0.192584633827209 62.0
273.18578668 254.40251159668 0.0
0 -0.141435980796814 1.0
0 0.051897406578064 2.0
0 0.503188133239746 3.0
0 0.0333291292190552 4.0
0 -0.101445317268372 5.0
0 0.017913818359375 6.0
0 0.454408526420593 7.0
0 0.0167440176010132 8.0
0 0.0445391535758972 9.0
0 0.612441182136536 10.0
0 -0.188885033130646 11.0
0 -0.175892055034637 12.0
69.813436847 68.6143493652344 13.0
27.65636747 29.7790298461914 14.0
0 -0.0794327259063721 15.0
0 -0.0472369790077209 16.0
35.885270061 34.888484954834 17.0
0 0.10742974281311 18.0
0 -0.00801211595535278 19.0
0 0.075580358505249 20.0
0 0.0869344472885132 21.0
0 0.118225812911987 22.0
34.728203494 32.9449882507324 23.0
0 0.0984301567077637 24.0
0 0.732280969619751 25.0
0 -0.0687075853347778 26.0
0 -0.0836839079856873 27.0
0 0.0265114903450012 28.0
0 -0.14788818359375 29.0
0 0.0376913547515869 30.0
0 -0.0521508455276489 31.0
0 0.00344675779342651 32.0
0 -0.326974630355835 33.0
0 0.238888025283813 34.0
0 -0.1181800365448 35.0
0 0.379090458154678 36.0
0 0.19281017780304 37.0
0 0.0165243148803711 38.0
0 -0.246336579322815 39.0
0 0.439734607934952 40.0
0 -0.0935601592063904 41.0
0 -0.0497785806655884 42.0
308.06470882 298.291717529297 43.0
0 0.161371231079102 44.0
0 0.0647078156471252 45.0
0 0.165527641773224 46.0
0 -0.0393768548965454 47.0
0 0.269920587539673 48.0
0 0.408922851085663 49.0
0.082051525957 0.0615554451942444 50.0
1.7240553443 1.86459970474243 51.0
0 -0.24149489402771 52.0
0 0.0665255188941956 53.0
0 -0.0470740795135498 54.0
0 -0.0154556035995483 55.0
0 0.0980169177055359 56.0
0 0.0773451328277588 57.0
0 0.0231358408927917 58.0
0 0.0216872692108154 59.0
0 0.284967362880707 60.0
0 0.00183534622192383 61.0
0 0.117059648036957 62.0
267.03090172 254.564727783203 0.0
0 -0.304965615272522 1.0
0 -0.194321155548096 2.0
0 0.15315580368042 3.0
0 -0.224483370780945 4.0
0 0.175662517547607 5.0
0 0.123338043689728 6.0
0 0.0782222151756287 7.0
0 -0.0853638648986816 8.0
0 0.0946893095970154 9.0
0 -0.0444543361663818 10.0
0 -0.0828129053115845 11.0
0 -0.213642597198486 12.0
52.935446864 54.2146759033203 13.0
38.251603723 39.2012748718262 14.0
0 -0.224573016166687 15.0
0 0.055899441242218 16.0
31.53322104 33.0528831481934 17.0
0 0.0567793846130371 18.0
0 -0.0123469829559326 19.0
0 0.275499403476715 20.0
0 0.0294648408889771 21.0
0 0.0683416724205017 22.0
38.536060328 36.8587112426758 23.0
0 0.0455856919288635 24.0
0 0.275124430656433 25.0
0 -0.0132423639297485 26.0
0 0.0485439896583557 27.0
0 0.29764661192894 28.0
0 -0.0965735912322998 29.0
0 0.0795468091964722 30.0
0 -0.132688760757446 31.0
0 -0.0877512097358704 32.0
0 -0.0823926329612732 33.0
0 -0.154199123382568 34.0
0 0.407004714012146 35.0
0 -0.357728362083435 36.0
0 0.218037903308868 37.0
0 -0.152338445186615 38.0
0 -0.116150975227356 39.0
0 0.0108422040939331 40.0
0 -0.217998862266541 41.0
0 0.087954044342041 42.0
290.22042979 294.838287353516 43.0
0 0.116336822509766 44.0
0 -0.025800347328186 45.0
0 0.1645787358284 46.0
0 -0.100047469139099 47.0
0 0.115768790245056 48.0
0 0.454601764678955 49.0
0.088926200362 -0.00434178113937378 50.0
2.1758567949 1.90911555290222 51.0
0 -0.00769311189651489 52.0
0 -0.0472222566604614 53.0
0 -0.149595081806183 54.0
0 0.194715857505798 55.0
0 -0.0458692312240601 56.0
0 0.0146058201789856 57.0
0 0.312415659427643 58.0
0 0.280099213123322 59.0
0 -0.250856757164001 60.0
0 0.0807875990867615 61.0
0 0.380634427070618 62.0
254.48590567 252.854949951172 0.0
0 -0.102711319923401 1.0
0 0.14594829082489 2.0
0 0.351570904254913 3.0
0 0.0479182004928589 4.0
0 -0.182245790958405 5.0
0 -0.0594810843467712 6.0
0 0.234061896800995 7.0
0 0.0461835861206055 8.0
0 0.0137211084365845 9.0
0 0.367478668689728 10.0
0 -0.107190728187561 11.0
0 -0.0996251106262207 12.0
53.620945825 54.9640426635742 13.0
43.729769586 42.5205497741699 14.0
0 -0.012717604637146 15.0
0 -0.0569909811019897 16.0
32.924975407 33.146598815918 17.0
0 0.080596923828125 18.0
0 0.0141266584396362 19.0
0 0.148915767669678 20.0
0 0.010987401008606 21.0
0 -0.010895848274231 22.0
38.87677847 38.8821296691895 23.0
0 0.136107623577118 24.0
0 0.808245778083801 25.0
0 0.0692789554595947 26.0
0 0.0219355821609497 27.0
0 0.369313716888428 28.0
0 -0.107243478298187 29.0
0 0.0903232097625732 30.0
0 -0.00520312786102295 31.0
0 0.0349888205528259 32.0
0 -0.0473979711532593 33.0
0 0.37504568696022 34.0
0 -0.134522676467896 35.0
0 0.257040679454803 36.0
0 0.0129011273384094 37.0
0 0.185621738433838 38.0
0 -0.242879271507263 39.0
0 0.233378291130066 40.0
0 -0.165010809898376 41.0
0 0.0424454212188721 42.0
291.89860567 297.261383056641 43.0
0 0.130317807197571 44.0
0 0.0444064736366272 45.0
0 0.100808322429657 46.0
0 -0.101608097553253 47.0
0 0.436249077320099 48.0
0 0.263268887996674 49.0
0.07718011008 0.120937466621399 50.0
2.2727167458 1.8398824930191 51.0
0 -0.0513434410095215 52.0
0 -0.0674290060997009 53.0
0 0.13907790184021 54.0
0 0.0952498912811279 55.0
0 0.00393140316009521 56.0
0 0.0370684862136841 57.0
0 0.229513883590698 58.0
0 0.0545768141746521 59.0
0 -0.11765843629837 60.0
0 0.0778821706771851 61.0
0 0.222012341022491 62.0
226.70556468 253.944595336914 0.0
0 -0.147742509841919 1.0
0 -0.0829954147338867 2.0
0 0.294978737831116 3.0
0 -0.3525071144104 4.0
0 -0.234787821769714 5.0
0 0.0817705392837524 6.0
0 -0.0232545137405396 7.0
0 -0.256484866142273 8.0
0 0.070112943649292 9.0
0 0.216790795326233 10.0
0 -0.211300849914551 11.0
0 -0.00032508373260498 12.0
59.995149779 60.6946830749512 13.0
25.132018495 32.0820693969727 14.0
0 -0.0602614879608154 15.0
0 -0.126208782196045 16.0
36.172706389 35.1932258605957 17.0
0 0.141712069511414 18.0
0 -0.0312299728393555 19.0
0 0.117732584476471 20.0
0 -0.0614768266677856 21.0
0 0.222169041633606 22.0
38.296255206 36.4142379760742 23.0
0 0.0306714177131653 24.0
0 0.160847306251526 25.0
0 0.0730149745941162 26.0
0 -0.0267055034637451 27.0
0 0.385905295610428 28.0
0 -0.192861139774323 29.0
0 0.0780109167098999 30.0
0 -0.0910563468933105 31.0
0 -0.0891983509063721 32.0
0 -0.071858823299408 33.0
0 0.158787250518799 34.0
0 -0.0561403036117554 35.0
0 -0.24354350566864 36.0
0 0.0251421928405762 37.0
0 0.160876929759979 38.0
0 -0.278450965881348 39.0
0 0.0758731365203857 40.0
0 -0.148684620857239 41.0
0 -0.14236843585968 42.0
304.32988153 297.958618164062 43.0
0 0.0766351222991943 44.0
0 0.116713047027588 45.0
0 0.207341134548187 46.0
0 0.0173003673553467 47.0
0 -0.056876003742218 48.0
0 0.44214192032814 49.0
0.083995459468 0.0249614715576172 50.0
1.9168600104 1.82603430747986 51.0
0 -0.0536288022994995 52.0
0 -0.0233268737792969 53.0
0 -0.00497925281524658 54.0
0 0.299218893051147 55.0
0 0.0335959792137146 56.0
0 -0.0728954672813416 57.0
0 0.364203661680222 58.0
0 -0.0420830249786377 59.0
0 -0.0635084509849548 60.0
0 0.0404893755912781 61.0
0 0.280266344547272 62.0
272.88998576 251.538131713867 0.0
0 -0.673996686935425 1.0
0 0.342022478580475 2.0
0 0.00933164358139038 3.0
0 -0.372306942939758 4.0
0 0.455147832632065 5.0
0 0.597702622413635 6.0
0 1.64419651031494 7.0
0 0.957194805145264 8.0
0 -0.457613587379456 9.0
0 0.634999454021454 10.0
0 -0.206222176551819 11.0
0 -0.129636347293854 12.0
72.888309015 65.0049438476562 13.0
31.394920153 34.704460144043 14.0
0 -0.0932109355926514 15.0
0 -0.0282323956489563 16.0
41.864460265 36.5449180603027 17.0
0 0.176933288574219 18.0
0 0.0286675691604614 19.0
0 0.0788941979408264 20.0
0 0.211650133132935 21.0
0 0.147704243659973 22.0
27.170517125 27.92995262146 23.0
0 0.0304309725761414 24.0
0 0.444403320550919 25.0
0 -0.327092170715332 26.0
0 -0.186574459075928 27.0
0 0.728510797023773 28.0
0 0.179386496543884 29.0
0 -0.0219807624816895 30.0
0 -0.0459976196289062 31.0
0 -0.129220008850098 32.0
0 -0.248804092407227 33.0
0 0.0585088133811951 34.0
0 0.229832589626312 35.0
0 -0.00184726715087891 36.0
0 -0.0572712421417236 37.0
0 0.262561798095703 38.0
0 -0.426128506660461 39.0
0 0.346881031990051 40.0
0 -0.223436594009399 41.0
0 0.180269002914429 42.0
297.81925338 294.274993896484 43.0
0 0.14503002166748 44.0
0 0.0981159806251526 45.0
0 0.0501599311828613 46.0
0 0.0834479331970215 47.0
0 0.0253725051879883 48.0
0 0.564305305480957 49.0
0.097432028081 0.0410788655281067 50.0
1.3328699386 1.78390407562256 51.0
0 0.261723101139069 52.0
0 -0.234694838523865 53.0
0 -0.114664196968079 54.0
0 -0.0755451917648315 55.0
0 0.350557923316956 56.0
0 -0.165492832660675 57.0
0 0.41039976477623 58.0
0 0.389876365661621 59.0
0 -0.0891245603561401 60.0
0 0.126361727714539 61.0
0 0.443582236766815 62.0
243.74712382 252.935028076172 0.0
0 -0.278852224349976 1.0
0 0.0164881944656372 2.0
0 0.100550353527069 3.0
0 -0.305592656135559 4.0
0 0.270039558410645 5.0
0 0.278273403644562 6.0
0 0.121729493141174 7.0
0 -0.0305162668228149 8.0
0 0.266335904598236 9.0
0 -0.123316586017609 10.0
0 -0.0788044929504395 11.0
0 -0.208395719528198 12.0
68.418311195 68.4984359741211 13.0
26.080632503 30.3863334655762 14.0
0 -0.0517690181732178 15.0
0 0.00308477878570557 16.0
37.331059354 35.8309288024902 17.0
0 0.135109066963196 18.0
0 -0.0207141637802124 19.0
0 0.0672833919525146 20.0
0 0.0901051759719849 21.0
0 0.167360901832581 22.0
35.95065135 32.9407691955566 23.0
0 0.0608739256858826 24.0
0 0.57085508108139 25.0
0 -0.0767707824707031 26.0
0 0.0750831365585327 27.0
0 0.649385929107666 28.0
0 0.0639809370040894 29.0
0 0.0556831359863281 30.0
0 -0.0827122926712036 31.0
0 -0.0809453725814819 32.0
0 -0.11706531047821 33.0
0 -0.0477370023727417 34.0
0 0.102778196334839 35.0
0 -0.146462857723236 36.0
0 0.13437432050705 37.0
0 0.0156603455543518 38.0
0 -0.404578804969788 39.0
0 0.257127106189728 40.0
0 -0.314751029014587 41.0
0 0.156919717788696 42.0
309.34496135 297.161437988281 43.0
0 0.107068598270416 44.0
0 0.1368008852005 45.0
0 0.044069766998291 46.0
0 0.294909596443176 47.0
0 0.253620028495789 48.0
0 0.394332885742188 49.0
0.067910842071 0.0375993847846985 50.0
1.6421269724 1.87218725681305 51.0
0 -0.0363644361495972 52.0
0 -0.0877756476402283 53.0
0 -0.00431835651397705 54.0
0 0.217945337295532 55.0
0 0.254280924797058 56.0
0 -0.0442181825637817 57.0
0 0.339292138814926 58.0
0 0.283059298992157 59.0
0 -0.054141104221344 60.0
0 0.12710428237915 61.0
0 0.251316487789154 62.0
288.55106033 255.714584350586 0.0
0 -0.310524225234985 1.0
0 0.0209155082702637 2.0
0 0.175297141075134 3.0
0 -0.334424138069153 4.0
0 0.224869966506958 5.0
0 -0.68989098072052 6.0
0 -0.0409168004989624 7.0
0 -0.296197652816772 8.0
0 0.399662375450134 9.0
0 -0.28234601020813 10.0
0 -0.00886285305023193 11.0
0 -0.127984046936035 12.0
50.735746972 49.3217315673828 13.0
50.046707319 47.1567535400391 14.0
0 -0.0141081809997559 15.0
0 -0.106475055217743 16.0
38.625324097 36.8808555603027 17.0
0 0.165393948554993 18.0
0 0.00557488203048706 19.0
0 0.0121622681617737 20.0
0 0.100443184375763 21.0
0 0.0816335678100586 22.0
36.875431476 33.7644653320312 23.0
0 0.098095178604126 24.0
0 1.02285277843475 25.0
0 -0.0759168863296509 26.0
0 -0.0321105122566223 27.0
0 0.585683465003967 28.0
0 -0.0409114360809326 29.0
0 0.0311458110809326 30.0
0 -0.103887319564819 31.0
0 0.123095631599426 32.0
0 0.0159329771995544 33.0
0 0.0368357300758362 34.0
0 0.12317156791687 35.0
0 -0.420801281929016 36.0
0 -0.058676540851593 37.0
0 0.280642628669739 38.0
0 -0.248669862747192 39.0
0 0.494534879922867 40.0
0 -0.209017038345337 41.0
0 0.0647208690643311 42.0
286.77648037 296.332916259766 43.0
0 -0.0658236145973206 44.0
0 0.172911703586578 45.0
0 0.278248846530914 46.0
0 -0.367773056030273 47.0
0 -1.03517961502075 48.0
0 0.615678310394287 49.0
0.074694231201 -0.0424752235412598 50.0
2.3228931386 1.81147217750549 51.0
0 -0.0615558624267578 52.0
0 -0.00761109590530396 53.0
0 -0.0182062387466431 54.0
0 0.186684846878052 55.0
0 0.296092242002487 56.0
0 0.337010413408279 57.0
0 0.0132239460945129 58.0
0 -0.00702512264251709 59.0
0 -0.182932019233704 60.0
0 -0.113054037094116 61.0
0 -0.193593382835388 62.0
272.11374709 253.084777832031 0.0
0 -0.237271547317505 1.0
0 -0.0420517325401306 2.0
0 0.106189608573914 3.0
0 0.00589895248413086 4.0
0 0.518519163131714 5.0
0 -0.151609122753143 6.0
0 0.250924706459045 7.0
0 0.712066769599915 8.0
0 -0.203563451766968 9.0
0 -0.237101554870605 10.0
0 0.198145151138306 11.0
0 -0.118865847587585 12.0
72.682594412 75.4432373046875 13.0
22.371117678 20.7330703735352 14.0
0 0.0406953096389771 15.0
0 0.0184416770935059 16.0
40.010529099 36.5908050537109 17.0
0 0.0834107398986816 18.0
0 -0.039646327495575 19.0
0 0.110192656517029 20.0
0 -0.051005482673645 21.0
0 0.319601833820343 22.0
33.464957622 31.7730407714844 23.0
0 0.0674524307250977 24.0
0 0.247916221618652 25.0
0 -0.113258183002472 26.0
0 0.0323383808135986 27.0
0 0.514548361301422 28.0
0 0.0290488004684448 29.0
0 0.0486169457435608 30.0
0 -0.205002903938293 31.0
0 -0.0585756301879883 32.0
0 0.00126016139984131 33.0
0 0.0345802307128906 34.0
0 0.060558557510376 35.0
0 -0.207847833633423 36.0
0 0.122682809829712 37.0
0 0.0789780616760254 38.0
0 -0.135195016860962 39.0
0 0.638330221176147 40.0
0 -0.150065660476685 41.0
0 -0.0242384672164917 42.0
285.89643409 296.848785400391 43.0
0 0.0736969709396362 44.0
0 0.0918921232223511 45.0
0 0.113030612468719 46.0
0 -0.384485244750977 47.0
0 0.0378779172897339 48.0
0 0.350452661514282 49.0
0.088212602346 0.0268311500549316 50.0
1.6422412598 1.89172506332397 51.0
0 0.131957769393921 52.0
0 -0.778000354766846 53.0
0 -0.182620763778687 54.0
0 -0.0825974941253662 55.0
0 0.0729296207427979 56.0
0 -0.146390795707703 57.0
0 0.0830363035202026 58.0
0 -0.102990508079529 59.0
0 -0.132804095745087 60.0
0 0.0234529972076416 61.0
0 0.107614994049072 62.0
265.94262827 251.906173706055 0.0
0 -0.364111304283142 1.0
0 -0.0143884420394897 2.0
0 0.365019470453262 3.0
0 0.0773367881774902 4.0
0 0.312352061271667 5.0
0 -0.191964149475098 6.0
0 0.466813087463379 7.0
0 -0.062666654586792 8.0
0 -0.169692993164062 9.0
0 0.293577492237091 10.0
0 -0.200878202915192 11.0
0 0.0295329093933105 12.0
65.600087551 67.4262847900391 13.0
27.841051897 31.6338043212891 14.0
0 -0.0296568870544434 15.0
0 -0.0393557548522949 16.0
33.866469627 35.7536582946777 17.0
0 0.135363936424255 18.0
0 -0.0549800395965576 19.0
0 0.221795439720154 20.0
0 -0.0259890556335449 21.0
0 0.109487533569336 22.0
38.329061281 36.7659378051758 23.0
0 0.0194694995880127 24.0
0 0.0698334574699402 25.0
0 0.0954568386077881 26.0
0 -0.00426042079925537 27.0
0 0.20455265045166 28.0
0 -0.167704164981842 29.0
0 0.0895213484764099 30.0
0 -0.119261264801025 31.0
0 0.023501992225647 32.0
0 -0.187851011753082 33.0
0 -0.00841879844665527 34.0
0 -0.179908871650696 35.0
0 -0.122043609619141 36.0
0 -0.0828253030776978 37.0
0 0.0915502309799194 38.0
0 -0.197517275810242 39.0
0 0.205910503864288 40.0
0 -0.105385065078735 41.0
0 0.16615879535675 42.0
296.67025417 295.208435058594 43.0
0 0.212987542152405 44.0
0 0.00409519672393799 45.0
0 0.0392115712165833 46.0
0 -0.105213701725006 47.0
0 0.378397256135941 48.0
1.6153876527 0.485886722803116 49.0
0.075074589317 0.0380705595016479 50.0
1.8094135455 1.93169438838959 51.0
0 -0.0367317199707031 52.0
0 0.339828431606293 53.0
0 -0.205186724662781 54.0
0 -0.10237854719162 55.0
0 0.332255959510803 56.0
0 -0.0553301572799683 57.0
0 0.269219756126404 58.0
0 0.293260872364044 59.0
0 -0.203573048114777 60.0
0 0.113160133361816 61.0
0 0.338361084461212 62.0
242.6568557 252.444641113281 0.0
0 -0.234285831451416 1.0
0 -0.129812896251678 2.0
0 0.375263929367065 3.0
0 -0.247594118118286 4.0
0 0.00245440006256104 5.0
0 0.0903447270393372 6.0
0 -0.0523234009742737 7.0
0 0.368991911411285 8.0
0 -0.231727600097656 9.0
0 0.280031383037567 10.0
0 -0.17628014087677 11.0
0 -0.13465291261673 12.0
52.531789063 52.8280868530273 13.0
44.942508026 43.2379302978516 14.0
0 -0.0729939937591553 15.0
0 -0.0747337937355042 16.0
28.762331657 34.6709823608398 17.0
0 0.0981407165527344 18.0
0 -0.0485727787017822 19.0
0 0.095150887966156 20.0
0 0.0544670820236206 21.0
0 0.181189060211182 22.0
35.818889156 34.8395957946777 23.0
0 0.102106630802155 24.0
0 0.982924997806549 25.0
0 -0.0452191829681396 26.0
0 0.130315780639648 27.0
0 0.137801826000214 28.0
0 0.0392354130744934 29.0
0 0.0531548261642456 30.0
0 -0.0848982334136963 31.0
0 0.114862263202667 32.0
0 -0.157953023910522 33.0
0 -0.0839992165565491 34.0
0 0.722325503826141 35.0
0 0.421373009681702 36.0
0 0.0435137748718262 37.0
0 0.117695748806 38.0
0 -0.282454967498779 39.0
0 0.564619839191437 40.0
0 -0.141097545623779 41.0
0 -0.00874006748199463 42.0
284.00416805 296.724304199219 43.0
0 0.14513897895813 44.0
0 0.148137927055359 45.0
0 0.0974329710006714 46.0
0 -0.0938925743103027 47.0
0 0.372430711984634 48.0
0 0.431028008460999 49.0
0.10054343751 0.0434156656265259 50.0
1.5610897839 1.90799975395203 51.0
0 -0.0968186855316162 52.0
0 -0.0804541110992432 53.0
0 -0.167382597923279 54.0
0 0.076285719871521 55.0
0 0.231257081031799 56.0
0 0.155782043933868 57.0
0 -0.00511676073074341 58.0
0 0.0684325695037842 59.0
0 -0.0310292840003967 60.0
0 0.0870429277420044 61.0
0 0.158722162246704 62.0
217.60818562 253.260696411133 0.0
0 0.060686469078064 1.0
0 -0.202382922172546 2.0
0 -0.474511623382568 3.0
0 -0.403287768363953 4.0
0 -0.170396685600281 5.0
0 -0.348964333534241 6.0
0 -0.394550323486328 7.0
0 -0.613380670547485 8.0
0 1.13355457782745 9.0
0 1.02719533443451 10.0
0 -0.0301978588104248 11.0
0 -0.12242203950882 12.0
62.346673346 62.1889419555664 13.0
22.381115289 35.8625717163086 14.0
0 0.583462238311768 15.0
0 0.0754373669624329 16.0
37.469880983 33.5868301391602 17.0
0 0.268834412097931 18.0
0 -0.0620135068893433 19.0
0 0.0617077946662903 20.0
0 0.240265846252441 21.0
0 -0.0259068012237549 22.0
28.505442152 25.6677951812744 23.0
0 -0.134189486503601 24.0
0 0.47727969288826 25.0
0 -0.209771037101746 26.0
0 -0.375460267066956 27.0
0 0.47464257478714 28.0
0 0.0911111235618591 29.0
0 0.104580879211426 30.0
0 0.46014666557312 31.0
0 0.102000534534454 32.0
0 -0.0942583084106445 33.0
0 0.416256725788116 34.0
0 0.0978823900222778 35.0
0 -0.395145177841187 36.0
0 0.0983780026435852 37.0
0 -0.104804039001465 38.0
0 -0.148753523826599 39.0
0 0.433571577072144 40.0
0 -0.0705258846282959 41.0
0 0.0444726943969727 42.0
290.94223518 295.096099853516 43.0
0 0.105029106140137 44.0
0 0.0251059532165527 45.0
0 0.290614426136017 46.0
0 -0.620192050933838 47.0
0 -0.476566076278687 48.0
0 0.457076251506805 49.0
0.095158676813 0.0155839920043945 50.0
1.6161062695 1.80755734443665 51.0
0 -0.131602883338928 52.0
0 -0.0230857133865356 53.0
0 -0.0851622819900513 54.0
0 -0.101440787315369 55.0
0 -0.131573557853699 56.0
0 -0.0205897688865662 57.0
0 0.259435832500458 58.0
0 -0.3686363697052 59.0
0 0.80015754699707 60.0
0 0.071465790271759 61.0
0 0.185433626174927 62.0
288.05572503 253.507369995117 0.0
0 -0.302657961845398 1.0
0 -0.031182587146759 2.0
0 0.143731534481049 3.0
0 -0.0486255288124084 4.0
0 0.023077666759491 5.0
0 0.219207346439362 6.0
0 0.330351173877716 7.0
0 0.14589250087738 8.0
0 -0.0332143306732178 9.0
0 0.154823064804077 10.0
0 -0.218294262886047 11.0
0 0.068780243396759 12.0
55.002972775 55.5098495483398 13.0
36.63477958 39.8249778747559 14.0
0 0.0353554487228394 15.0
0 -0.169961214065552 16.0
27.23906901 32.5101623535156 17.0
0 0.0648371577262878 18.0
0 -0.00959259271621704 19.0
0 0.0962810516357422 20.0
0 0.213239550590515 21.0
0 0.01683509349823 22.0
27.775229117 27.9893035888672 23.0
0 0.178658127784729 24.0
0 1.34427952766418 25.0
0 0.0849801898002625 26.0
0 -0.159596621990204 27.0
0 -0.192494571208954 28.0
0 -0.0273871421813965 29.0
0 0.0840284824371338 30.0
0 0.0182549953460693 31.0
0 0.0050818920135498 32.0
0 -0.00483208894729614 33.0
0 0.288844108581543 34.0
0 0.289585471153259 35.0
0 0.328878223896027 36.0
0 0.174606084823608 37.0
0 -0.0304197669029236 38.0
0 -0.273489475250244 39.0
0 0.455887198448181 40.0
0 -0.13942289352417 41.0
0 0.0362919569015503 42.0
297.7852837 297.633178710938 43.0
0 0.272289276123047 44.0
0 -0.179165661334991 45.0
0 0.246249079704285 46.0
0 -0.280196189880371 47.0
0 -0.0238384008407593 48.0
0 0.446232914924622 49.0
0.096024796561 -0.0206332802772522 50.0
1.7382468023 1.86341547966003 51.0
0 -0.0466660261154175 52.0
0 -0.320503950119019 53.0
0 0.379289120435715 54.0
0 0.137860298156738 55.0
0 -0.086727499961853 56.0
0 0.23081636428833 57.0
0 0.272012054920197 58.0
0 0.21424812078476 59.0
0 -0.260111331939697 60.0
0 0.0859428644180298 61.0
0 0.162217915058136 62.0
232.70466059 251.777313232422 0.0
0 0.055994987487793 1.0
0 -0.261510610580444 2.0
0 0.274899363517761 3.0
0 -0.321908712387085 4.0
0 0.149291694164276 5.0
0 0.0691636204719543 6.0
0 -0.12663722038269 7.0
0 -0.331337928771973 8.0
0 0.506150722503662 9.0
0 0.46754714846611 10.0
0 -0.494930982589722 11.0
0 -0.18084180355072 12.0
62.31079969 60.4378318786621 13.0
23.603696329 36.715934753418 14.0
0 0.00806140899658203 15.0
0 0.243299007415771 16.0
36.948327195 38.0454483032227 17.0
0 0.171181321144104 18.0
0 -0.0539082288742065 19.0
0 0.0486876368522644 20.0
0 0.028434157371521 21.0
0 0.417970299720764 22.0
34.717000304 35.0774383544922 23.0
0 0.114734351634979 24.0
0 0.61300265789032 25.0
0 0.124990522861481 26.0
0 -0.130483090877533 27.0
0 0.67955470085144 28.0
0 -0.0373428463935852 29.0
0 0.170171856880188 30.0
0 -0.101826429367065 31.0
0 -0.0689957141876221 32.0
0 -0.256264805793762 33.0
0 -0.114413499832153 34.0
0 0.142824411392212 35.0
0 0.3174709379673 36.0
0 0.128221750259399 37.0
0 -0.409048199653625 38.0
0 -0.159733653068542 39.0
0 0.537688255310059 40.0
0 -0.187115073204041 41.0
0 0.0428619384765625 42.0
302.9707171 294.309997558594 43.0
0 0.0459035634994507 44.0
0 0.139473259449005 45.0
0 0.0280221700668335 46.0
0 0.0688360929489136 47.0
0 0.141187191009521 48.0
0 0.332234025001526 49.0
0.096712199667 0.0323754549026489 50.0
1.8086123757 1.87162554264069 51.0
0 -0.280171632766724 52.0
0 0.277282357215881 53.0
0 -0.139726340770721 54.0
0 0.0510557889938354 55.0
0 -0.0548826456069946 56.0
0 0.298732042312622 57.0
0 0.0102999210357666 58.0
0 0.223989486694336 59.0
0 -0.231204867362976 60.0
0 0.105458557605743 61.0
0 0.236071348190308 62.0
243.09852309 252.211685180664 0.0
0 0.315573930740356 1.0
0 -0.0132927298545837 2.0
0 0.0537564158439636 3.0
0 -0.390457034111023 4.0
0 -0.0209311842918396 5.0
0 0.116199791431427 6.0
0 -0.0412497520446777 7.0
0 -0.364821195602417 8.0
0 0.38311231136322 9.0
0 0.663869142532349 10.0
0 -0.232933282852173 11.0
0 -0.102565944194794 12.0
64.153895264 65.4618072509766 13.0
22.387253142 30.5210456848145 14.0
0 -0.0543438196182251 15.0
0 0.00139445066452026 16.0
27.974983051 32.149974822998 17.0
0 0.0653908252716064 18.0
0 0.0326831340789795 19.0
0 0.14006382226944 20.0
0 0.0870303511619568 21.0
0 0.107579231262207 22.0
32.175482915 30.8867969512939 23.0
0 0.0681703686714172 24.0
0 0.663226425647736 25.0
0 -0.0410307049751282 26.0
0 -0.222333669662476 27.0
0 0.337549239397049 28.0
0 -0.0293798446655273 29.0
0 0.0250255465507507 30.0
0 -0.362725257873535 31.0
0 -0.134802281856537 32.0
0 -0.0599035620689392 33.0
0 0.149152636528015 34.0
0 -0.093853235244751 35.0
0 0.321738868951797 36.0
0 0.00307393074035645 37.0
0 0.180345177650452 38.0
0 -0.214133024215698 39.0
0 0.476594269275665 40.0
0 -0.040263295173645 41.0
0 -0.123176574707031 42.0
286.60866874 298.299102783203 43.0
0 0.0299177169799805 44.0
0 0.136172354221344 45.0
0 0.0460963845252991 46.0
0 0.0555933713912964 47.0
0 0.404969930648804 48.0
0 0.355580359697342 49.0
0.068116531485 0.0729591846466064 50.0
2.0586033484 1.83278465270996 51.0
0 -0.0264868140220642 52.0
0 0.0662904381752014 53.0
0 -0.139246463775635 54.0
0 0.0064769983291626 55.0
0 0.190473437309265 56.0
0 -0.137737274169922 57.0
0 0.143472194671631 58.0
0 0.101033091545105 59.0
0 -0.0733200311660767 60.0
0 0.0355224609375 61.0
0 0.174928903579712 62.0
285.14477651 252.275756835938 0.0
0 -0.235999345779419 1.0
0 0.145477294921875 2.0
0 0.261183619499207 3.0
0 -0.132311046123505 4.0
0 0.233640730381012 5.0
0 1.09412312507629 6.0
0 0.389120787382126 7.0
0 -0.104663252830505 8.0
0 -0.0755985975265503 9.0
0 0.729270875453949 10.0
0 -0.181281566619873 11.0
0 -0.300456404685974 12.0
71.74856789 70.7269287109375 13.0
23.7473168 25.9924449920654 14.0
0 -0.0814180374145508 15.0
0 -0.0854730010032654 16.0
37.177087053 36.5005416870117 17.0
0 0.153106093406677 18.0
0 0.0156693458557129 19.0
0 0.0822925567626953 20.0
0 0.139587044715881 21.0
0 0.169485330581665 22.0
35.397805512 33.7844696044922 23.0
0 0.162435293197632 24.0
0 0.562105596065521 25.0
0 -0.0626720190048218 26.0
0 -0.00650167465209961 27.0
0 0.156604528427124 28.0
0 -0.00722026824951172 29.0
0 0.110240280628204 30.0
0 0.185707569122314 31.0
0 0.122522234916687 32.0
0 -0.0360008478164673 33.0
0 0.284090042114258 34.0
0 0.165657520294189 35.0
0 0.286523997783661 36.0
0 0.141393780708313 37.0
0 -0.954032182693481 38.0
0 -0.0197732448577881 39.0
0 -0.18452376127243 40.0
0 -0.33314573764801 41.0
0 -0.0177842378616333 42.0
288.42086304 296.896240234375 43.0
0 0.167959213256836 44.0
0 0.190353393554688 45.0
0 0.0867701768875122 46.0
0 0.0441240668296814 47.0
0 -0.335607767105103 48.0
0 0.455731272697449 49.0
0.080146593677 -0.0265470743179321 50.0
2.4100387391 1.88327407836914 51.0
0 0.165643572807312 52.0
0 1.02990293502808 53.0
0 -0.14226245880127 54.0
0 0.511597275733948 55.0
0 0.290680766105652 56.0
0 -0.106870055198669 57.0
0 -0.207987129688263 58.0
0 -0.412308931350708 59.0
0 0.934823930263519 60.0
0 0.148019194602966 61.0
0 0.189976036548615 62.0
279.58511863 251.403076171875 0.0
0 -0.260297536849976 1.0
0 0.0287144780158997 2.0
0 0.0501581430435181 3.0
0 0.0966885089874268 4.0
0 0.114077150821686 5.0
0 0.171858727931976 6.0
0 0.363004118204117 7.0
0 0.334448158740997 8.0
0 -0.0557040572166443 9.0
0 0.047872006893158 10.0
0 -0.205652713775635 11.0
0 -0.097220778465271 12.0
60.566279217 61.5055084228516 13.0
34.560449836 31.3317413330078 14.0
0 -0.070784330368042 15.0
0 -0.0862250924110413 16.0
34.805906022 33.9758758544922 17.0
0 0.0763022899627686 18.0
0 0.0174382925033569 19.0
0 0.0928688049316406 20.0
0 0.151863217353821 21.0
0 0.111328125 22.0
30.722654195 28.9752178192139 23.0
0 0.11967134475708 24.0
0 0.863282144069672 25.0
0 -0.0508990883827209 26.0
0 0.0124672055244446 27.0
0 0.497087895870209 28.0
0 0.00590801239013672 29.0
0 -0.160033047199249 30.0
0 0.136208653450012 31.0
0 0.151416599750519 32.0
0 -0.0855149030685425 33.0
0 0.219344437122345 34.0
0 -0.105007648468018 35.0
0 0.123695015907288 36.0
0 -0.0715101361274719 37.0
0 0.17451536655426 38.0
0 -0.220645546913147 39.0
0 0.282748937606812 40.0
0 -0.148820459842682 41.0
0 -0.0751489400863647 42.0
290.11517056 297.778259277344 43.0
0 0.0497759580612183 44.0
0 0.0998984575271606 45.0
0 0.12658154964447 46.0
0 0.146415174007416 47.0
0 -0.239056587219238 48.0
0 0.436317980289459 49.0
0.082650190555 0.0124585628509521 50.0
1.5527584459 1.8194854259491 51.0
0 0.0962609052658081 52.0
0 -0.0644057989120483 53.0
0 0.129050254821777 54.0
0 0.320105731487274 55.0
0 -0.0663515329360962 56.0
0 0.634659945964813 57.0
0 0.179828584194183 58.0
0 0.0871161222457886 59.0
0 -0.0984828472137451 60.0
0 0.0484687685966492 61.0
0 0.156282186508179 62.0
247.75085629 255.569717407227 0.0
0 -0.313139319419861 1.0
0 0.035300076007843 2.0
0 0.430911689996719 3.0
0 -0.0126790404319763 4.0
0 0.307697892189026 5.0
0 0.14531409740448 6.0
0 0.0312264561653137 7.0
0 0.342485725879669 8.0
0 -0.191076815128326 9.0
0 0.254482448101044 10.0
0 -0.0298454761505127 11.0
0 -0.166895627975464 12.0
70.969225915 71.2956314086914 13.0
25.381194152 24.4165077209473 14.0
0 -0.173809051513672 15.0
0 -0.233870267868042 16.0
38.340544179 35.3761520385742 17.0
0 0.0756255984306335 18.0
0 -0.0400989055633545 19.0
0 0.128569841384888 20.0
0 -0.0433623194694519 21.0
0 0.136086463928223 22.0
39.408788561 38.5330619812012 23.0
0 0.0586574673652649 24.0
0 0.660917580127716 25.0
0 -0.0370199084281921 26.0
0 -0.0733497142791748 27.0
0 0.230172693729401 28.0
0 -0.0910237431526184 29.0
0 -0.0411293506622314 30.0
0 0.0397348403930664 31.0
0 0.0677616596221924 32.0
0 0.0975656509399414 33.0
0 0.156540930271149 34.0
0 -0.0383684039115906 35.0
0 -0.00904035568237305 36.0
0 -0.0908080339431763 37.0
0 0.126817107200623 38.0
0 -0.201527118682861 39.0
0 -0.0965371131896973 40.0
0 -0.276238441467285 41.0
0 0.0925112962722778 42.0
295.08684681 298.223083496094 43.0
0 0.0598614811897278 44.0
0 0.142597556114197 45.0
0 0.185460805892944 46.0
0 -0.063957154750824 47.0
0 0.360591650009155 48.0
0 0.504416763782501 49.0
0.068893552949 0.0198155641555786 50.0
2.0996022828 1.85169386863708 51.0
0 -0.176049470901489 52.0
0 -0.100042223930359 53.0
0 0.0904443860054016 54.0
0 0.00778430700302124 55.0
0 -0.0468444228172302 56.0
0 -0.0460116267204285 57.0
0 0.563712060451508 58.0
0 0.254169166088104 59.0
0 -0.196312665939331 60.0
0 0.126346588134766 61.0
0 0.296071708202362 62.0
271.43766513 251.995941162109 0.0
0 -0.25315260887146 1.0
0 -0.149761378765106 2.0
0 0.173397898674011 3.0
0 0.0113805532455444 4.0
0 0.380966246128082 5.0
0 0.194725334644318 6.0
0 0.181916773319244 7.0
0 -0.311109781265259 8.0
0 0.149098932743073 9.0
0 -0.349247217178345 10.0
0 -0.13683295249939 11.0
0 -0.158190369606018 12.0
51.313120058 49.7586708068848 13.0
50.904797578 43.6509094238281 14.0
0 -0.0236456394195557 15.0
0 0.00883609056472778 16.0
28.900113361 32.2443580627441 17.0
0 0.0598369240760803 18.0
0 -0.0128817558288574 19.0
0 0.161853611469269 20.0
0 -0.0718511343002319 21.0
0 0.0813370943069458 22.0
37.694105001 37.2261695861816 23.0
0 0.0379288792610168 24.0
0 0.348845481872559 25.0
0 -0.061134934425354 26.0
0 0.00314730405807495 27.0
0 -0.216860175132751 28.0
0 -0.123254060745239 29.0
0 0.0896985530853271 30.0
0 0.206660032272339 31.0
0 0.112600564956665 32.0
0 -0.0117803812026978 33.0
0 0.402729541063309 34.0
0 0.0970900654792786 35.0
0 0.366294741630554 36.0
0 -0.0248137712478638 37.0
0 0.289088726043701 38.0
0 -0.332955241203308 39.0
0 -0.442095875740051 40.0
0 -0.213316679000854 41.0
0 0.0936157703399658 42.0
302.03226599 297.266510009766 43.0
0 0.0707557201385498 44.0
0 0.0857851505279541 45.0
0 0.134489357471466 46.0
0 0.0998169183731079 47.0
0 0.311509907245636 48.0
0 0.411102861166 49.0
0.10014508264 0.0864297151565552 50.0
1.5575525281 1.8346244096756 51.0
0 -0.213904738426208 52.0
0 0.167239487171173 53.0
0 0.0139298439025879 54.0
0 0.0405162572860718 55.0
0 -0.0832804441452026 56.0
0 0.260588109493256 57.0
0 0.341631531715393 58.0
0 0.298615217208862 59.0
0 -0.197693824768066 60.0
0 0.0593813061714172 61.0
0 0.335627198219299 62.0
284.14537469 252.811172485352 0.0
0 -0.121747851371765 1.0
0 0.27938312292099 2.0
0 0.0336077809333801 3.0
0 0.2409428358078 4.0
0 0.55646014213562 5.0
0 0.294795095920563 6.0
0 1.1188782453537 7.0
0 -0.570232629776001 8.0
0 -1.44981551170349 9.0
0 0.0720211267471313 10.0
0 -0.426817655563354 11.0
0 -0.116686701774597 12.0
76.850307876 74.0159912109375 13.0
7.2856764878 24.3939990997314 14.0
0 -0.00537800788879395 15.0
0 0.185461819171906 16.0
35.07803141 36.4655075073242 17.0
0 0.17327868938446 18.0
0 -0.0225650072097778 19.0
0 0.0615988969802856 20.0
0 -0.00978070497512817 21.0
0 0.226005673408508 22.0
38.930051883 39.0213050842285 23.0
0 0.0633711814880371 24.0
0 1.06758379936218 25.0
0 -0.0465255975723267 26.0
0 0.0401924252510071 27.0
0 -0.262723445892334 28.0
0 -0.00723755359649658 29.0
0 0.071452796459198 30.0
0 -0.0864098072052002 31.0
0 0.12782621383667 32.0
0 0.27213180065155 33.0
0 0.232729375362396 34.0
0 0.124596357345581 35.0
0 0.0104120373725891 36.0
0 0.123218297958374 37.0
0 0.379142910242081 38.0
0 -0.257917881011963 39.0
0 0.12535148859024 40.0
0 -0.215446949005127 41.0
0 0.258769989013672 42.0
290.0754595 297.21630859375 43.0
0 0.217653155326843 44.0
0 0.191033184528351 45.0
0 0.144478499889374 46.0
0 -0.258531332015991 47.0
0 0.0704630613327026 48.0
0 0.594540596008301 49.0
0.066510878162 -0.010313868522644 50.0
1.9182097686 1.83222949504852 51.0
0 -0.141061425209045 52.0
0 0.0207340717315674 53.0
0 -0.064048171043396 54.0
0 0.237307190895081 55.0
0 0.300659269094467 56.0
0 0.043281078338623 57.0
0 0.235276758670807 58.0
0 -0.0107583999633789 59.0
0 0.0772808194160461 60.0
0 -0.0136387348175049 61.0
0 0.0500816106796265 62.0
233.80165969 255.65705871582 0.0
0 -0.281790971755981 1.0
0 -0.396187782287598 2.0
0 0.290158033370972 3.0
0 -0.274712085723877 4.0
0 0.0461474061012268 5.0
0 0.0185784101486206 6.0
0 -0.270984888076782 7.0
0 0.236588835716248 8.0
0 -0.00018543004989624 9.0
0 -0.139032304286957 10.0
0 -0.110168159008026 11.0
0 -0.0423817038536072 12.0
59.150218231 61.4983100891113 13.0
38.104543964 37.6500434875488 14.0
0 -0.0619980096817017 15.0
0 -0.13045346736908 16.0
38.544214466 38.4738388061523 17.0
0 0.151425361633301 18.0
0 0.0415793657302856 19.0
0 -0.00558507442474365 20.0
0 0.0617423057556152 21.0
0 0.2823446393013 22.0
35.170354878 34.776496887207 23.0
0 0.064042329788208 24.0
0 0.696800827980042 25.0
0 -0.0661526918411255 26.0
0 -0.126501441001892 27.0
0 0.74289470911026 28.0
0 0.111418128013611 29.0
0 -0.10310822725296 30.0
0 -0.0298811197280884 31.0
0 -0.0316962003707886 32.0
0 -0.0924955606460571 33.0
0 -0.496716618537903 34.0
0 0.340413719415665 35.0
0 0.128419160842896 36.0
0 -0.0620089769363403 37.0
0 0.225130021572113 38.0
0 -0.257666230201721 39.0
0 0.388141691684723 40.0
0 -0.292415857315063 41.0
0 0.200220823287964 42.0
308.5554924 296.369689941406 43.0
0 0.248219013214111 44.0
0 0.144981861114502 45.0
0 0.126268804073334 46.0
0 0.243717670440674 47.0
0 0.379789859056473 48.0
0 0.514297485351562 49.0
0.086474174759 0.0224346518516541 50.0
1.6945652072 1.8815313577652 51.0
0 0.151913404464722 52.0
0 -0.00683009624481201 53.0
0 0.0505927801132202 54.0
0 0.131746053695679 55.0
0 -0.0280604362487793 56.0
0 -0.298625588417053 57.0
0 0.175042927265167 58.0
0 0.295474648475647 59.0
0 -0.27907383441925 60.0
0 0.0969899296760559 61.0
0 0.245579600334167 62.0
267.97949714 252.714263916016 0.0
0 0.225440621376038 1.0
0 0.0218807458877563 2.0
0 0.218281865119934 3.0
0 -0.204922616481781 4.0
0 0.302341997623444 5.0
0 -0.116066336631775 6.0
0 0.372501075267792 7.0
0 0.0229211449623108 8.0
0 0.132191598415375 9.0
0 0.540324985980988 10.0
0 -0.122177243232727 11.0
0 -0.145808339118958 12.0
58.872971929 58.1337051391602 13.0
38.985602136 42.4284477233887 14.0
0 -0.151361107826233 15.0
0 -0.0320175290107727 16.0
41.152677032 34.3158988952637 17.0
0 0.158710598945618 18.0
0 -0.0264158248901367 19.0
0 0.250977218151093 20.0
0 0.0105301141738892 21.0
0 0.0933467149734497 22.0
36.259094901 33.5345268249512 23.0
0 -0.0297551155090332 24.0
0 -0.0398649573326111 25.0
0 -0.0837938785552979 26.0
0 -0.0637157559394836 27.0
0 0.36638993024826 28.0
0 -0.0455166697502136 29.0
0 0.0468699336051941 30.0
0 -0.124657988548279 31.0
0 0.0755858421325684 32.0
0 -0.0401020050048828 33.0
0 -0.0835322737693787 34.0
0 -0.0333197712898254 35.0
0 0.0233461856842041 36.0
0 -0.0588734745979309 37.0
0 0.0201499462127686 38.0
0 -0.124245762825012 39.0
0 0.188740909099579 40.0
0 -0.0965204238891602 41.0
0 0.00528597831726074 42.0
289.54348864 296.081451416016 43.0
0 0.0407825708389282 44.0
0 0.131665050983429 45.0
0 0.221224367618561 46.0
0 -0.0690399408340454 47.0
0 -0.0693234801292419 48.0
0 0.477469056844711 49.0
0.079610501675 0.0661497712135315 50.0
1.4601166421 1.79078912734985 51.0
0 0.312615811824799 52.0
0 -0.134190380573273 53.0
0 0.0559945106506348 54.0
0 0.0149655342102051 55.0
0 0.00256389379501343 56.0
0 0.712250471115112 57.0
0 0.282457053661346 58.0
0 -0.0270264148712158 59.0
0 -0.11372584104538 60.0
0 0.1542609333992 61.0
0 0.282715797424316 62.0
215.1989388 255.742401123047 0.0
0 0.0268394947052002 1.0
0 -0.0137326121330261 2.0
0 0.111204564571381 3.0
0 -0.334057688713074 4.0
0 -0.441060066223145 5.0
0 0.17488420009613 6.0
0 -0.0852755904197693 7.0
0 0.241224527359009 8.0
0 0.135593235492706 9.0
0 -0.421342968940735 10.0
0 -0.0855675935745239 11.0
0 -0.031149685382843 12.0
58.886066786 58.6168556213379 13.0
46.677659195 40.3062705993652 14.0
0 -0.109883427619934 15.0
0 0.0990570187568665 16.0
39.510103474 37.4870376586914 17.0
0 0.223079562187195 18.0
0 0.000241398811340332 19.0
0 0.0158019661903381 20.0
0 0.0172412991523743 21.0
0 0.121370732784271 22.0
38.859631182 39.7028388977051 23.0
0 0.107500493526459 24.0
0 1.14512324333191 25.0
0 -0.124166250228882 26.0
0 -0.0501067042350769 27.0
0 -0.196887850761414 28.0
0 -0.137797772884369 29.0
0 0.165532350540161 30.0
0 0.439968943595886 31.0
0 0.314663410186768 32.0
0 0.0595896244049072 33.0
0 0.0731043815612793 34.0
0 0.0442149639129639 35.0
0 -0.530909538269043 36.0
0 0.132012605667114 37.0
0 0.24342280626297 38.0
0 -0.51581084728241 39.0
0 0.45240980386734 40.0
0 -0.280113220214844 41.0
0 0.0509116649627686 42.0
286.70841085 298.317565917969 43.0
0 0.0997633337974548 44.0
0 0.105569958686829 45.0
0 0.25143164396286 46.0
0 -0.1537806391716 47.0
0 -0.266144990921021 48.0
0 0.446407407522202 49.0
0.097137871941 -0.00624161958694458 50.0
1.6525891577 1.86163759231567 51.0
0 -0.0207862257957458 52.0
0 -0.497475147247314 53.0
0 -0.0810396671295166 54.0
0 0.287929117679596 55.0
0 0.0106227993965149 56.0
0 0.0373998880386353 57.0
0 0.304288476705551 58.0
0 0.164573788642883 59.0
0 -0.157018005847931 60.0
0 0.111168622970581 61.0
0 0.325328946113586 62.0
275.36439735 249.925491333008 0.0
0 -0.299121975898743 1.0
0 -0.0234757661819458 2.0
0 0.598284125328064 3.0
0 0.000842452049255371 4.0
0 0.354531437158585 5.0
0 0.182069361209869 6.0
0 0.226934015750885 7.0
0 -0.23267674446106 8.0
0 0.369333386421204 9.0
0 0.107089340686798 10.0
0 0.058844268321991 11.0
0 0.00858592987060547 12.0
52.337996399 50.8298454284668 13.0
53.300697866 44.5216102600098 14.0
0 -0.0144259929656982 15.0
0 -0.22524619102478 16.0
35.12402498 34.1638526916504 17.0
0 0.134789228439331 18.0
0 -0.0395373106002808 19.0
0 0.125548243522644 20.0
0 0.241597235202789 21.0
0 -0.0561992526054382 22.0
26.110269326 24.6697731018066 23.0
0 0.126327872276306 24.0
7.1634779453 0.49057811498642 25.0
0 -0.290452361106873 26.0
0 0.0213750004768372 27.0
0 0.454738408327103 28.0
0 0.177158355712891 29.0
0 0.199272632598877 30.0
0 0.0721325874328613 31.0
0 0.269395709037781 32.0
0 0.120267987251282 33.0
0 -0.081285834312439 34.0
0 0.372960954904556 35.0
0 -0.203702509403229 36.0
0 -0.0894009470939636 37.0
0 0.119169473648071 38.0
0 -0.383324861526489 39.0
0 -0.0648641586303711 40.0
0 -0.192930042743683 41.0
0 0.197430491447449 42.0
289.95398947 298.2998046875 43.0
0 0.168311297893524 44.0
0 0.0477411150932312 45.0
0 0.269542634487152 46.0
0 -0.395453453063965 47.0
0 0.365635871887207 48.0
0 0.414293169975281 49.0
0.084047460775 0.0851899385452271 50.0
1.5672531393 1.86707091331482 51.0
0 0.0683104991912842 52.0
0 -0.0392712354660034 53.0
0 -0.112026572227478 54.0
0 0.0883522033691406 55.0
0 0.282792091369629 56.0
0 0.128857612609863 57.0
0 0.257730543613434 58.0
0 0.175455212593079 59.0
0 -0.213993191719055 60.0
0 0.0993586778640747 61.0
0 0.145235300064087 62.0
249.48695709 253.278900146484 0.0
0 -0.0977727174758911 1.0
0 0.101603865623474 2.0
0 0.244566559791565 3.0
0 -0.143590569496155 4.0
0 0.0557665228843689 5.0
0 0.0112628936767578 6.0
0 0.0141983032226562 7.0
0 -0.240440368652344 8.0
0 -0.110042631626129 9.0
0 -0.0229904651641846 10.0
0 -0.0192135572433472 11.0
0 -0.181379199028015 12.0
67.115100688 66.2212219238281 13.0
37.94438071 30.2680740356445 14.0
0 -0.0985957384109497 15.0
0 0.105835139751434 16.0
38.483770383 35.7830657958984 17.0
0 0.160408854484558 18.0
0 -0.0100650787353516 19.0
0 0.0711338520050049 20.0
0 0.245404601097107 21.0
0 0.0877419114112854 22.0
32.171675982 31.7398643493652 23.0
0 0.128672897815704 24.0
0 0.911935806274414 25.0
0 0.0240814685821533 26.0
0 -0.27644956111908 27.0
0 -0.0264413356781006 28.0
0 -0.0629347562789917 29.0
0 0.0405164957046509 30.0
0 -0.332744598388672 31.0
0 0.0700142383575439 32.0
0 -0.056727409362793 33.0
0 0.0978307127952576 34.0
0 0.0938088297843933 35.0
0 -0.265416264533997 36.0
0 -0.00621426105499268 37.0
0 0.195068657398224 38.0
0 -0.562233209609985 39.0
0 0.156713306903839 40.0
0 -0.201784729957581 41.0
0 0.146564364433289 42.0
292.37576908 296.688354492188 43.0
0 0.114622831344604 44.0
0 0.000556647777557373 45.0
0 0.210877895355225 46.0
0 -0.0959746837615967 47.0
0 0.338477462530136 48.0
0 0.503009438514709 49.0
0.09554469375 0.0602447986602783 50.0
1.9031681127 1.85225808620453 51.0
0 -0.150093674659729 52.0
0 -0.149680972099304 53.0
0 -0.00440764427185059 54.0
0 0.0750017166137695 55.0
0 -0.150632500648499 56.0
0 -0.0198302268981934 57.0
0 0.258819341659546 58.0
0 0.0149960517883301 59.0
0 -0.0871217250823975 60.0
0 0.00366151332855225 61.0
0 0.138711512088776 62.0
256.49631561 250.88264465332 0.0
0 -0.0761762857437134 1.0
0 0.0225219130516052 2.0
0 0.17162150144577 3.0
0 -0.19132786989212 4.0
0 0.0435510277748108 5.0
0 0.174428105354309 6.0
0 0.077966034412384 7.0
0 0.426034480333328 8.0
0 0.416831910610199 9.0
0 0.479598462581635 10.0
0 -0.336613655090332 11.0
0 -0.213601112365723 12.0
54.317112778 54.2944412231445 13.0
37.345994672 34.4622535705566 14.0
0 -0.0523635149002075 15.0
0 0.0345651507377625 16.0
29.019435348 30.3245048522949 17.0
0 0.0708369612693787 18.0
0 -0.0556336641311646 19.0
0 0.118336498737335 20.0
0 0.167475163936615 21.0
0 0.0690244436264038 22.0
29.225982654 29.7809543609619 23.0
0 0.150989294052124 24.0
0 1.04441928863525 25.0
0 0.186523973941803 26.0
0 -0.15236359834671 27.0
0 -0.0535609722137451 28.0
0 -0.111029148101807 29.0
0 0.0652025938034058 30.0
0 -0.119081735610962 31.0
0 -0.134609520435333 32.0
0 -0.102606177330017 33.0
0 0.466158807277679 34.0
0 0.211071789264679 35.0
0 0.569399476051331 36.0
0 -0.0300911068916321 37.0
0 0.191473782062531 38.0
0 -0.220852375030518 39.0
0 0.251543283462524 40.0
0 -0.153393030166626 41.0
0 0.103869676589966 42.0
301.83957223 297.991577148438 43.0
0 0.0138302445411682 44.0
0 0.107914566993713 45.0
0 0.224360764026642 46.0
0 -0.262696146965027 47.0
0 -0.104193687438965 48.0
0 0.405013114213943 49.0
0.092675354729 0.000846624374389648 50.0
1.8708407659 1.84610247612 51.0
0 -0.23919403553009 52.0
0 -0.0385201573371887 53.0
0 0.093542218208313 54.0
0 0.0606943964958191 55.0
0 -0.0600454211235046 56.0
0 0.282260537147522 57.0
0 0.270801842212677 58.0
0 0.201567769050598 59.0
0 0.120184421539307 60.0
0 0.0264350175857544 61.0
0 0.1366947889328 62.0
266.41645476 252.659957885742 0.0
0 -0.090755820274353 1.0
0 -0.142071008682251 2.0
0 0.266641139984131 3.0
0 -0.129115462303162 4.0
0 -0.130252301692963 5.0
0 0.0121954083442688 6.0
0 0.340816020965576 7.0
0 -0.301832914352417 8.0
0 0.00010913610458374 9.0
0 -0.0165328979492188 10.0
0 -0.764098405838013 11.0
0 -0.254134178161621 12.0
70.198889707 68.7383804321289 13.0
20.3633639 29.1305866241455 14.0
0 0.126803040504456 15.0
0 -0.0301625728607178 16.0
35.07291187 34.8652992248535 17.0
0 0.132651448249817 18.0
0 -0.000842452049255371 19.0
0 0.0331391096115112 20.0
0 0.182459712028503 21.0
0 2.9444694519043e-05 22.0
27.603924057 26.6619892120361 23.0
0 0.0418247580528259 24.0
0 0.410004913806915 25.0
0 -0.0326725840568542 26.0
0 -0.17728990316391 27.0
0 0.323281645774841 28.0
0 -0.0569881796836853 29.0
0 0.0532382726669312 30.0
0 -0.0558927059173584 31.0
0 -0.0471152663230896 32.0
0 -0.0793671011924744 33.0
0 0.353668332099915 34.0
0 0.109091997146606 35.0
0 -0.450969934463501 36.0
0 -0.075730562210083 37.0
0 0.246133506298065 38.0
0 -0.371706008911133 39.0
0 0.0821352005004883 40.0
0 -0.18043315410614 41.0
0 0.142428755760193 42.0
302.70397342 295.990234375 43.0
0 0.118768095970154 44.0
0 -0.0416815280914307 45.0
0 0.24124014377594 46.0
0 0.00479638576507568 47.0
0 0.134249687194824 48.0
0 0.217638075351715 49.0
0.091579183574 -0.00263643264770508 50.0
1.8798391589 1.92821300029755 51.0
0 0.114957451820374 52.0
0 -0.10297167301178 53.0
0 -0.058604896068573 54.0
0 0.10515969991684 55.0
0 -0.047290563583374 56.0
0 0.0906058549880981 57.0
0 0.277374148368835 58.0
0 0.163891553878784 59.0
0 -0.104995906352997 60.0
0 0.0956788063049316 61.0
0 0.302294492721558 62.0
261.94938102 252.475677490234 0.0
0 -0.390401482582092 1.0
0 0.0648396015167236 2.0
0 0.293832182884216 3.0
0 0.153783798217773 4.0
0 0.213929653167725 5.0
0 0.181053519248962 6.0
0 0.226991236209869 7.0
0 0.129216313362122 8.0
0 0.0603626370429993 9.0
0 0.4117192029953 10.0
0 -0.0690659284591675 11.0
0 -0.19297468662262 12.0
61.907766411 65.1954498291016 13.0
32.505762433 27.1672611236572 14.0
0 -0.144952774047852 15.0
0 -0.0438931584358215 16.0
36.323895019 34.3471260070801 17.0
0 0.0932896137237549 18.0
0 0.107765793800354 19.0
0 0.0507659316062927 20.0
0 0.0684732794761658 21.0
0 0.0355788469314575 22.0
39.19409232 37.9300575256348 23.0
0 0.0478470325469971 24.0
0 0.843994140625 25.0
0 -0.077629566192627 26.0
0 -0.00098186731338501 27.0
0 -0.348741292953491 28.0
0 0.141530156135559 29.0
0 -0.0236895084381104 30.0
0 -0.0473359823226929 31.0
0 -0.030239999294281 32.0
0 -0.201014935970306 33.0
0 0.187810778617859 34.0
0 -0.162846386432648 35.0
0 0.0282504558563232 36.0
0 0.157746613025665 37.0
0 0.209051311016083 38.0
0 -0.296162724494934 39.0
0 -0.347643375396729 40.0
0 -0.171235263347626 41.0
0 0.314412713050842 42.0
296.90844393 296.275268554688 43.0
0 -0.0480965375900269 44.0
0 0.147067785263062 45.0
0 0.164456605911255 46.0
0 0.339583516120911 47.0
0 -0.692397236824036 48.0
0 0.460956484079361 49.0
0.067493080588 -0.023105263710022 50.0
1.4845335907 1.85856413841248 51.0
0 0.287537813186646 52.0
0 -0.380393743515015 53.0
0 0.0570626258850098 54.0
0 0.230819761753082 55.0
0 0.0574787855148315 56.0
0 0.468742668628693 57.0
0 0.0684452652931213 58.0
0 -0.0126877427101135 59.0
0 -0.159790933132172 60.0
0 0.0351778864860535 61.0
0 0.104124188423157 62.0
273.45040258 252.293090820312 0.0
0 -0.363615036010742 1.0
0 0.112419188022614 2.0
0 0.281432867050171 3.0
0 0.177734553813934 4.0
0 0.150338113307953 5.0
0 -0.0516291260719299 6.0
0 0.215299367904663 7.0
0 -0.258267641067505 8.0
0 0.278733372688293 9.0
0 0.117904245853424 10.0
0 0.0109624862670898 11.0
0 -0.00403493642807007 12.0
66.432060698 68.0022354125977 13.0
40.102164442 33.2486534118652 14.0
0 -0.0186480283737183 15.0
0 -0.0543332695960999 16.0
35.670144624 33.2474822998047 17.0
0 0.12127673625946 18.0
0 -0.0758113861083984 19.0
0 0.193160772323608 20.0
0 -0.0029112696647644 21.0
0 0.278117120265961 22.0
33.597006326 32.4884223937988 23.0
0 0.025340735912323 24.0
0 0.398180365562439 25.0
0 -0.0595834255218506 26.0
0 -0.0886135101318359 27.0
0 0.175765454769135 28.0
0 -0.0396180152893066 29.0
0 0.0478448867797852 30.0
0 0.0857429504394531 31.0
0 0.146695852279663 32.0
0 -0.0616886019706726 33.0
0 0.0648666620254517 34.0
0 -0.0299355983734131 35.0
0 -0.21054995059967 36.0
0 -0.0171225070953369 37.0
0 0.202679574489594 38.0
0 -0.208673357963562 39.0
0 0.273154139518738 40.0
0 -0.222729563713074 41.0
0 0.108508586883545 42.0
301.21305439 296.809448242188 43.0
0 0.16199666261673 44.0
0 0.193526804447174 45.0
0 0.0274778604507446 46.0
0 0.199243724346161 47.0
0 0.365807890892029 48.0
0.90558675582 0.424907803535461 49.0
0.082174053256 0.0440037846565247 50.0
1.6497225689 1.9091854095459 51.0
0 0.153334856033325 52.0
0 0.00307101011276245 53.0
0 0.138011336326599 54.0
0 0.0963203310966492 55.0
0 0.0477334260940552 56.0
0 0.000127315521240234 57.0
0 0.212397277355194 58.0
0 0.328501999378204 59.0
0 -0.233842611312866 60.0
0 0.112132489681244 61.0
0 0.228892982006073 62.0
285.9534919 252.918060302734 0.0
0 -0.244835734367371 1.0
0 0.141179323196411 2.0
0 0.0100908875465393 3.0
0 -0.0159260630607605 4.0
0 0.273417472839355 5.0
0 -0.21726393699646 6.0
0 0.135113298892975 7.0
0 -0.041232168674469 8.0
0 0.101689159870148 9.0
0 0.585820436477661 10.0
0 0.240942895412445 11.0
0 -0.128882169723511 12.0
51.508817411 51.3237571716309 13.0
35.802671729 34.3706169128418 14.0
0 0.0598592758178711 15.0
0 -0.178158462047577 16.0
33.153151097 35.3856391906738 17.0
0 0.137510299682617 18.0
0 -0.092057466506958 19.0
0 0.0920624732971191 20.0
0 0.142090082168579 21.0
0 0.041169285774231 22.0
27.182330215 27.3363838195801 23.0
0 0.124141454696655 24.0
0 0.804478526115417 25.0
0 -0.0825955867767334 26.0
0 -0.0322520732879639 27.0
0 0.108578205108643 28.0
0 0.00703173875808716 29.0
0 0.0525333881378174 30.0
0 -0.00296163558959961 31.0
0 0.160660028457642 32.0
0 0.150013029575348 33.0
0 -0.192822575569153 34.0
0 0.016968846321106 35.0
0 -0.222294569015503 36.0
0 -0.00493526458740234 37.0
0 0.294261157512665 38.0
0 -0.0947787761688232 39.0
0 0.279875993728638 40.0
0 -0.292834639549255 41.0
0 0.00657474994659424 42.0
296.47656289 296.412017822266 43.0
0 0.0484567880630493 44.0
0 0.208669006824493 45.0
0 0.168910145759583 46.0
0 0.0799269080162048 47.0
0 -0.38384222984314 48.0
0 0.382679849863052 49.0
0.07302087106 0.0368645191192627 50.0
2.1758798273 1.83433699607849 51.0
0 0.266872107982635 52.0
0 -0.167467832565308 53.0
0 -0.202608704566956 54.0
0 -0.202655673027039 55.0
0 -0.0984364748001099 56.0
0 -0.220506906509399 57.0
0 0.282252967357635 58.0
0 -0.0673091411590576 59.0
0 0.242559909820557 60.0
0 0.223447501659393 61.0
0 0.240725636482239 62.0
240.08129185 255.505874633789 0.0
0 -0.32824718952179 1.0
0 -0.11218249797821 2.0
0 -0.0599421858787537 3.0
0 -0.120380043983459 4.0
0 0.165847063064575 5.0
0 0.145819902420044 6.0
0 0.028262197971344 7.0
0 0.515274584293365 8.0
0 -0.204969763755798 9.0
0 0.110747396945953 10.0
0 -0.287052154541016 11.0
0 0.000446557998657227 12.0
59.305942027 59.3248634338379 13.0
43.151405036 37.8998413085938 14.0
0 -0.190952181816101 15.0
0 -0.180728435516357 16.0
41.227761585 34.1203117370605 17.0
0 0.108846366405487 18.0
0 0.0402261018753052 19.0
0 0.144599199295044 20.0
0 0.0377155542373657 21.0
0 0.231426239013672 22.0
36.003513181 33.296989440918 23.0
0 0.0398923754692078 24.0
0 0.354595422744751 25.0
0 0.101123929023743 26.0
0 0.15054178237915 27.0
0 0.423741936683655 28.0
0 0.146334230899811 29.0
0 0.301288455724716 30.0
0 -0.0344816446304321 31.0
0 -0.0735307335853577 32.0
0 -0.220344543457031 33.0
0 0.441449403762817 34.0
0 0.0482898950576782 35.0
0 0.451154202222824 36.0
0 0.120372653007507 37.0
0 -0.467561602592468 38.0
0 -0.179805636405945 39.0
0 0.332841396331787 40.0
0 -0.211711287498474 41.0
0 0.638217926025391 42.0
292.02125832 293.194213867188 43.0
0 0.0384191274642944 44.0
0 0.510537981987 45.0
0 0.109127521514893 46.0
0 0.0685380697250366 47.0
0 0.292862772941589 48.0
2.5003941897 0.383622765541077 49.0
0.068211725641 0.0592772960662842 50.0
1.9248288681 1.92770338058472 51.0
0 -0.174372315406799 52.0
0 -0.582767844200134 53.0
0 -0.0798148512840271 54.0
0 0.0165874361991882 55.0
0 0.0513445138931274 56.0
0 0.176702380180359 57.0
0 0.548852443695068 58.0
0 -0.168739080429077 59.0
0 0.0161744356155396 60.0
0 0.115144848823547 61.0
0 0.259924173355103 62.0
234.21142557 251.439956665039 0.0
0 0.0146085023880005 1.0
0 -0.0912823081016541 2.0
0 -0.0546858310699463 3.0
0 -0.312268495559692 4.0
0 0.342901319265366 5.0
0 -0.0476881265640259 6.0
0 0.0543628931045532 7.0
0 -0.160698652267456 8.0
0 -0.142502844333649 9.0
0 0.00847768783569336 10.0
0 0.831802666187286 11.0
0 -0.150640487670898 12.0
73.901517486 71.0031661987305 13.0
42.532279673 31.5524234771729 14.0
0 -0.103771924972534 15.0
0 0.147356688976288 16.0
40.528079774 36.358772277832 17.0
0 0.141101717948914 18.0
0 0.018263578414917 19.0
0 0.0976683497428894 20.0
0 0.199858784675598 21.0
0 0.15990287065506 22.0
35.41475823 34.5814437866211 23.0
0 0.126313984394073 24.0
0 1.30129361152649 25.0
0 -0.200211107730865 26.0
0 0.00441604852676392 27.0
0 0.955915093421936 28.0
0 0.34686666727066 29.0
0 0.0886790156364441 30.0
0 -0.0738645792007446 31.0
0 -0.14846408367157 32.0
0 -0.360391139984131 33.0
0 0.0752184391021729 34.0
0 -0.05998694896698 35.0
0 -0.391864418983459 36.0
0 0.170522093772888 37.0
0 -0.0935758352279663 38.0
0 -0.140117049217224 39.0
0 0.542469084262848 40.0
0 -0.0650517344474792 41.0
0 -0.0646648406982422 42.0
310.9767671 296.377410888672 43.0
0 0.0987720489501953 44.0
0 0.13511449098587 45.0
0 0.0402342677116394 46.0
0 0.191846072673798 47.0
0 -0.105482518672943 48.0
0 0.444732040166855 49.0
0.094776332382 0.0269396901130676 50.0
2.0539169074 1.82807040214539 51.0
0 0.158503413200378 52.0
0 -0.144090592861176 53.0
0 -0.0691477060317993 54.0
0 0.143970131874084 55.0
0 0.124878406524658 56.0
0 -0.0868145823478699 57.0
0 0.0388738512992859 58.0
0 -0.373385429382324 59.0
0 1.15293884277344 60.0
0 0.0523087382316589 61.0
0 0.0864118337631226 62.0
287.68438617 251.035797119141 0.0
0 -0.186953902244568 1.0
0 0.147019267082214 2.0
0 -0.412157773971558 3.0
0 -0.205664455890656 4.0
0 0.0456694960594177 5.0
0 0.645907044410706 6.0
0 0.139451384544373 7.0
0 0.382545053958893 8.0
0 0.210616886615753 9.0
0 0.0515056848526001 10.0
0 0.118787884712219 11.0
0 -0.0752110481262207 12.0
63.671751767 65.1755676269531 13.0
48.850657256 36.0861473083496 14.0
0 0.00566720962524414 15.0
0 0.0257776379585266 16.0
32.548068406 31.7591190338135 17.0
0 0.087299644947052 18.0
0 -0.0295177698135376 19.0
0 0.128170251846313 20.0
0 0.306953519582748 21.0
0 -0.166967809200287 22.0
25.294959133 24.5116767883301 23.0
0 0.0757380127906799 24.0
0 3.13512897491455 25.0
0 -0.10983419418335 26.0
0 -0.0494482517242432 27.0
0 0.328761428594589 28.0
0 0.0122956037521362 29.0
0 0.101309537887573 30.0
0 0.0487650632858276 31.0
0 0.107394933700562 32.0
0 0.121399760246277 33.0
0 0.033408522605896 34.0
0 -0.0424802899360657 35.0
0 0.33171671628952 36.0
0 0.207025527954102 37.0
0 -0.743845820426941 38.0
0 -0.267133355140686 39.0
0 -0.14040195941925 40.0
0 -0.0353756546974182 41.0
0 0.0309793949127197 42.0
277.52658185 297.010711669922 43.0
0 0.275206446647644 44.0
0 -0.0466910004615784 45.0
0 0.0484059453010559 46.0
0 -0.205906569957733 47.0
0 0.253272831439972 48.0
1.8344329307 0.23111093044281 49.0
0.085401334938 0.0220930576324463 50.0
2.1749560892 1.85269284248352 51.0
0 -0.0354877710342407 52.0
0 -0.0830766558647156 53.0
0 0.139873266220093 54.0
0 0.153324842453003 55.0
0 0.0714706778526306 56.0
0 0.223515629768372 57.0
0 0.585908770561218 58.0
0 -0.731945037841797 59.0
0 -0.19083708524704 60.0
0 0.00612306594848633 61.0
0 0.0854969620704651 62.0
258.89610182 251.86408996582 0.0
0 0.0764154195785522 1.0
0 -0.197645425796509 2.0
0 0.406513005495071 3.0
0 -0.00953036546707153 4.0
0 0.354349553585052 5.0
0 0.146646857261658 6.0
0 -0.0139862298965454 7.0
0 -0.0371838808059692 8.0
0 0.478448092937469 9.0
0 0.269555985927582 10.0
0 -0.0371572971343994 11.0
0 -0.165108323097229 12.0
59.249459951 59.0564041137695 13.0
43.828784991 36.2621688842773 14.0
0 -0.0639708042144775 15.0
0 0.0283764004707336 16.0
30.708807071 32.6713371276855 17.0
0 0.0684627294540405 18.0
0 -0.0109509229660034 19.0
0 0.0630566477775574 20.0
0 0.174915552139282 21.0
0 0.270653128623962 22.0
35.276215926 34.2496910095215 23.0
0 0.0886935591697693 24.0
0 0.984913349151611 25.0
0 -0.206552445888519 26.0
0 -0.0696166753768921 27.0
0 0.572733521461487 28.0
0 0.0793526768684387 29.0
0 0.016460657119751 30.0
0 -0.00132942199707031 31.0
0 0.204615831375122 32.0
0 0.0347195267677307 33.0
0 0.277085542678833 34.0
0 0.0662045478820801 35.0
0 -0.641802906990051 36.0
0 -0.0612306594848633 37.0
0 0.140997886657715 38.0
0 -0.412919282913208 39.0
0 0.472414135932922 40.0
0 -0.165624737739563 41.0
0 0.072358250617981 42.0
288.85194651 295.62890625 43.0
0 -0.0608501434326172 44.0
0 0.338519275188446 45.0
0 0.142845094203949 46.0
0 -0.273704051971436 47.0
0 0.357513219118118 48.0
0 0.410785615444183 49.0
0.08002008268 0.0495414137840271 50.0
2.213835677 1.85456097126007 51.0
0 -0.0526375770568848 52.0
0 -0.133821427822113 53.0
0 -0.00524997711181641 54.0
0 -0.0794568061828613 55.0
0 -0.113548517227173 56.0
0 0.00174707174301147 57.0
0 0.240218341350555 58.0
0 0.255662083625793 59.0
0 -0.228239059448242 60.0
0 0.0733349323272705 61.0
0 0.21258282661438 62.0
276.56986356 252.754867553711 0.0
0 -0.228983402252197 1.0
0 0.067126452922821 2.0
0 0.380605220794678 3.0
0 -0.0295969247817993 4.0
0 0.324934452772141 5.0
0 -0.269434928894043 6.0
0 0.279476702213287 7.0
0 -0.417059659957886 8.0
0 0.247819662094116 9.0
0 -0.0869024395942688 10.0
0 -0.0693972110748291 11.0
0 -0.065976083278656 12.0
56.579774214 56.3071136474609 13.0
43.904792797 40.9631156921387 14.0
0 -0.0328829288482666 15.0
0 0.0350965261459351 16.0
32.386267832 32.814769744873 17.0
0 0.0951865911483765 18.0
0 0.0021134614944458 19.0
0 0.101481795310974 20.0
0 0.0274810791015625 21.0
0 0.273126840591431 22.0
37.674452637 38.638427734375 23.0
0 0.0813868045806885 24.0
0 0.653842091560364 25.0
0 0.0539168119430542 26.0
0 -0.186942636966705 27.0
0 -0.0324850082397461 28.0
0 -0.150331854820251 29.0
0 0.0571091175079346 30.0
0 -0.0846724510192871 31.0
0 0.0759033560752869 32.0
0 0.203204333782196 33.0
0 0.359578162431717 34.0
0 -0.158012747764587 35.0
0 -0.27923059463501 36.0
0 -0.062653660774231 37.0
0 0.221996307373047 38.0
0 -0.271381378173828 39.0
0 -0.113197445869446 40.0
0 -0.184287130832672 41.0
0 -0.036316990852356 42.0
291.52830217 296.912109375 43.0
0 0.193688154220581 44.0
0 -0.0234364867210388 45.0
0 0.197726428508759 46.0
0 -0.100493133068085 47.0
0 0.262387037277222 48.0
1.503316198 0.563496828079224 49.0
0.092497779242 0.0557341575622559 50.0
1.6932715143 1.84767508506775 51.0
0 0.197137355804443 52.0
0 -0.210639476776123 53.0
0 0.189411878585815 54.0
0 0.0229953527450562 55.0
0 0.112760484218597 56.0
0 0.0924041867256165 57.0
0 -0.0361607074737549 58.0
0 0.061767041683197 59.0
0 -0.0673247575759888 60.0
0 0.0533559322357178 61.0
0 0.126166582107544 62.0
258.6234743 256.259246826172 0.0
0 -0.32812488079071 1.0
0 -0.0460783243179321 2.0
0 0.0554559826850891 3.0
0 -0.240822792053223 4.0
0 0.266721665859222 5.0
0 0.24373722076416 6.0
0 0.120612621307373 7.0
0 -0.0725507736206055 8.0
0 -0.2550288438797 9.0
0 0.0914960503578186 10.0
0 -0.0588213801383972 11.0
0 0.0558691024780273 12.0
66.650757632 66.3178634643555 13.0
38.460615981 31.0052471160889 14.0
0 -0.0874520540237427 15.0
0 0.281780362129211 16.0
37.283374816 37.9044075012207 17.0
0 0.135362386703491 18.0
0 -0.0656694173812866 19.0
0 0.0979843735694885 20.0
0 0.040109395980835 21.0
0 0.237327575683594 22.0
33.991783652 35.8305778503418 23.0
0 0.0289596319198608 24.0
0 1.02851259708405 25.0
0 0.0724062919616699 26.0
0 -0.029556930065155 27.0
0 -0.179112076759338 28.0
0 -0.0818707942962646 29.0
0 0.131802797317505 30.0
0 -0.085735559463501 31.0
0 -0.188246786594391 32.0
0 -0.354497194290161 33.0
0 0.174145102500916 34.0
0 0.146334588527679 35.0
0 0.276386499404907 36.0
0 -0.0327852368354797 37.0
0 0.29198169708252 38.0
0 -0.437107443809509 39.0
0 -0.181650459766388 40.0
0 -0.190500974655151 41.0
0 0.00132858753204346 42.0
300.89341092 294.935821533203 43.0
0 -0.220106244087219 44.0
0 0.130421102046967 45.0
0 0.18532919883728 46.0
0 -0.0736940503120422 47.0
0 0.0489360690116882 48.0
0 0.372522085905075 49.0
0.065693903915 0.0234822034835815 50.0
1.5715701972 1.88037085533142 51.0
0 -0.186119437217712 52.0
0 -0.285584688186646 53.0
0 0.350385457277298 54.0
0 0.289998710155487 55.0
0 0.096919059753418 56.0
0 -0.117107391357422 57.0
0 0.249985694885254 58.0
0 -0.912993192672729 59.0
0 0.488922268152237 60.0
0 0.0396806597709656 61.0
0 0.151140987873077 62.0
231.91637109 253.057891845703 0.0
0 -0.187671542167664 1.0
0 -0.0208245515823364 2.0
0 0.0104424357414246 3.0
0 -0.271278738975525 4.0
0 -0.158427178859711 5.0
0 -0.014629602432251 6.0
0 -0.149189531803131 7.0
0 0.0344970226287842 8.0
0 0.0075523853302002 9.0
0 0.0868276357650757 10.0
0 0.090403139591217 11.0
0 -0.176952838897705 12.0
59.488600737 59.5511779785156 13.0
51.530430294 41.2903938293457 14.0
0 -0.0240066051483154 15.0
0 0.034649670124054 16.0
34.187953013 34.0961990356445 17.0
0 0.103970587253571 18.0
0 -0.0331517457962036 19.0
0 0.115790963172913 20.0
0 0.0392621755599976 21.0
0 0.270200371742249 22.0
32.60891208 30.8136596679688 23.0
0 0.0913389921188354 24.0
0 1.18357121944427 25.0
0 -0.0865615010261536 26.0
0 -0.0201237797737122 27.0
0 0.407458990812302 28.0
0 -0.0307354927062988 29.0
0 0.0398066639900208 30.0
0 0.107136130332947 31.0
0 0.100607454776764 32.0
0 0.127750217914581 33.0
0 0.409075260162354 34.0
0 0.198652625083923 35.0
0 0.259259819984436 36.0
0 -0.0319679975509644 37.0
0 0.0583968162536621 38.0
0 -0.180940270423889 39.0
0 0.307708531618118 40.0
0 -0.187446355819702 41.0
0 0.0286761522293091 42.0
302.22651091 295.951690673828 43.0
0 0.241955995559692 44.0
0 0.0356213450431824 45.0
0 0.311193883419037 46.0
0 -0.287882566452026 47.0
0 0.271989345550537 48.0
0 0.35163152217865 49.0
0.069869145475 0.0623213648796082 50.0
2.2762747618 1.82263040542603 51.0
0 -0.0228228569030762 52.0
0 -0.227033853530884 53.0
0 0.11883157491684 54.0
0 0.191618084907532 55.0
0 0.0702991485595703 56.0
0 -0.0598774552345276 57.0
0 -0.0169914960861206 58.0
0 0.0490936636924744 59.0
0 -0.121516704559326 60.0
0 0.0667511820793152 61.0
0 0.0828615427017212 62.0
254.61785104 254.324508666992 0.0
0 -0.314932107925415 1.0
0 -0.171100974082947 2.0
0 0.486224681138992 3.0
0 -0.0472041368484497 4.0
0 -0.027026891708374 5.0
0 -0.0498146414756775 6.0
0 0.148529827594757 7.0
0 -0.49016809463501 8.0
0 -0.11426168680191 9.0
0 0.304503202438354 10.0
0 -0.510921597480774 11.0
0 -0.152262330055237 12.0
62.852801938 62.9731712341309 13.0
26.437024189 30.0161762237549 14.0
0 -0.0504697561264038 15.0
0 -0.119748651981354 16.0
34.227010803 36.4334678649902 17.0
0 0.150275230407715 18.0
0 0.00174975395202637 19.0
0 0.0357024073600769 20.0
0 0.205068588256836 21.0
0 0.170374751091003 22.0
32.677819824 30.1848201751709 23.0
0 0.0609654188156128 24.0
0 0.67535412311554 25.0
0 -0.0847055912017822 26.0
0 -0.0852081179618835 27.0
0 0.764117956161499 28.0
0 0.0344680547714233 29.0
0 -0.0321661829948425 30.0
0 0.218320846557617 31.0
0 0.192852199077606 32.0
0 -0.306406259536743 33.0
0 0.0497450828552246 34.0
0 -0.0438154935836792 35.0
0 -0.129398703575134 36.0
0 -0.0644200444221497 37.0
0 0.0517400503158569 38.0
0 -0.18312656879425 39.0
0 0.471491008996964 40.0
0 -0.148847818374634 41.0
0 -0.00085759162902832 42.0
305.67300985 296.091094970703 43.0
0 0.129910469055176 44.0
0 0.123683094978333 45.0
0 0.291832685470581 46.0
0 0.218304634094238 47.0
0 0.039122998714447 48.0
0 0.350933611392975 49.0
0.10003403417 0.0684314966201782 50.0
2.2589030731 1.81061899662018 51.0
0 0.191082954406738 52.0
0 -0.107523858547211 53.0
0 0.0695576667785645 54.0
0 0.173021674156189 55.0
0 0.143033444881439 56.0
0 0.156004369258881 57.0
0 0.187279880046844 58.0
0 0.0601969957351685 59.0
0 -0.0894137024879456 60.0
0 0.0536894798278809 61.0
0 0.209485769271851 62.0
236.78590164 252.090621948242 0.0
0 0.135149002075195 1.0
0 -0.0514699220657349 2.0
0 -0.00395107269287109 3.0
0 -0.265608310699463 4.0
0 -0.0928258299827576 5.0
0 0.155692458152771 6.0
0 0.0106039047241211 7.0
0 -0.0710539817810059 8.0
0 -0.0568980574607849 9.0
0 -0.145348727703094 10.0
0 -0.00730329751968384 11.0
0 -0.0183326601982117 12.0
54.894970885 54.1094512939453 13.0
49.969043503 42.5372009277344 14.0
0 -0.0227286815643311 15.0
0 -0.122042536735535 16.0
27.682508324 32.5712394714355 17.0
0 0.0841226577758789 18.0
0 -0.0900323390960693 19.0
0 0.180628657341003 20.0
0 0.0612677335739136 21.0
0 0.177796065807343 22.0
34.447314551 30.8110504150391 23.0
0 -0.0258283019065857 24.0
0 -0.0608399510383606 25.0
0 0.0559430718421936 26.0
0 -0.067889928817749 27.0
0 0.340193629264832 28.0
0 0.0208319425582886 29.0
0 0.0366097688674927 30.0
0 -0.0787078142166138 31.0
0 -0.0691595077514648 32.0
0 -0.0141486525535583 33.0
0 0.0243240594863892 34.0
0 0.31537002325058 35.0
0 0.204337537288666 36.0
0 0.0837889909744263 37.0
0 0.153024673461914 38.0
0 -0.475568056106567 39.0
0 0.549105823040009 40.0
0 -0.162691473960876 41.0
0 0.034376859664917 42.0
311.28143529 297.311462402344 43.0
0 0.0756031274795532 44.0
0 0.108705699443817 45.0
0 0.247072815895081 46.0
0 0.0105624198913574 47.0
0 0.39839118719101 48.0
0 0.387939214706421 49.0
0.080158714448 0.0652793049812317 50.0
2.1896345058 1.89388370513916 51.0
0 -0.01739501953125 52.0
0 0.0503332614898682 53.0
0 -0.054999828338623 54.0
0 0.13544762134552 55.0
0 0.067257285118103 56.0
0 -0.0519988536834717 57.0
0 0.471343636512756 58.0
0 0.336923629045486 59.0
0 -0.250198245048523 60.0
0 0.119494438171387 61.0
0 0.342270016670227 62.0
292.2422711 252.049911499023 0.0
0 -0.245739340782166 1.0
0 0.0805855989456177 2.0
0 0.0981799960136414 3.0
0 0.147005438804626 4.0
0 -0.107335925102234 5.0
0 -0.521739959716797 6.0
0 0.432312369346619 7.0
0 0.472763419151306 8.0
0 -0.198010921478271 9.0
0 0.452487975358963 10.0
0 -0.303166627883911 11.0
0 -0.118468761444092 12.0
72.920203023 72.6820983886719 13.0
10.818918492 21.3757457733154 14.0
0 0.0502136945724487 15.0
0 -0.131025373935699 16.0
25.104022259 32.731689453125 17.0
0 0.0736562013626099 18.0
0 -0.00269317626953125 19.0
0 0.109656095504761 20.0
0 0.186425983905792 21.0
0 -0.0214625000953674 22.0
29.943244679 25.1406803131104 23.0
0 0.0288921594619751 24.0
0 -0.156904995441437 25.0
0 0.0438994169235229 26.0
0 -0.183108925819397 27.0
0 0.86067670583725 28.0
0 0.086397111415863 29.0
0 -0.0498623251914978 30.0
0 0.0614579916000366 31.0
0 -0.0690380930900574 32.0
0 -0.442640781402588 33.0
0 -0.909694314002991 34.0
0 0.244104087352753 35.0
0 0.266065835952759 36.0
0 0.0857247114181519 37.0
0 0.127538800239563 38.0
0 -0.388149380683899 39.0
0 0.983100712299347 40.0
0 -0.17531818151474 41.0
0 0.101613163948059 42.0
314.56458023 296.392639160156 43.0
0 0.259191751480103 44.0
0 0.174723088741302 45.0
0 0.23546314239502 46.0
0 -0.0577136874198914 47.0
0 0.214979529380798 48.0
0 0.368364542722702 49.0
0.083751983565 0.0474791526794434 50.0
1.8076762511 1.82667756080627 51.0
0 -0.0217445492744446 52.0
0 -0.20127671957016 53.0
0 -0.194106221199036 54.0
0 0.00500833988189697 55.0
0 0.296058475971222 56.0
0 0.00601756572723389 57.0
0 -0.0639514923095703 58.0
0 0.0176101326942444 59.0
0 0.0352612137794495 60.0
0 0.0458470582962036 61.0
0 0.0302857756614685 62.0
264.12849863 253.226989746094 0.0
0 -0.174442887306213 1.0
0 -0.107493877410889 2.0
0 0.32343989610672 3.0
0 0.118062019348145 4.0
0 0.302073806524277 5.0
0 -0.0934609770774841 6.0
0 0.104239106178284 7.0
0 -0.0922362804412842 8.0
0 0.103870987892151 9.0
0 0.346251040697098 10.0
0 -0.352425932884216 11.0
0 -0.140346527099609 12.0
58.510001856 58.7485656738281 13.0
33.750505047 37.6319961547852 14.0
0 -0.104845643043518 15.0
0 0.0274142622947693 16.0
39.880846865 34.4408111572266 17.0
0 0.135445952415466 18.0
0 -0.0482050180435181 19.0
0 0.0762671232223511 20.0
0 0.225479900836945 21.0
0 -0.0903211236000061 22.0
31.200704212 29.7357978820801 23.0
0 0.0794746279716492 24.0
0 0.625236928462982 25.0
0 -0.013744592666626 26.0
0 -0.0350525379180908 27.0
0 -0.0754982829093933 28.0
0 -0.132056832313538 29.0
0 0.0415730476379395 30.0
0 0.0445380210876465 31.0
0 0.084841251373291 32.0
0 -0.0156238079071045 33.0
0 0.320104449987411 34.0
0 -0.0812979936599731 35.0
0 -0.229720830917358 36.0
0 -0.0252633094787598 37.0
0 0.162488460540771 38.0
0 -0.270206689834595 39.0
0 -0.0691960453987122 40.0
0 -0.195331573486328 41.0
0 -0.0747759342193604 42.0
285.71845451 297.554901123047 43.0
0 0.095372200012207 44.0
0 -0.0581737160682678 45.0
0 0.12292742729187 46.0
0 -0.137807667255402 47.0
0 0.375228226184845 48.0
0 0.434371709823608 49.0
0.092296352804 0.026520848274231 50.0
1.6774711432 1.90297317504883 51.0
0 -0.0251443982124329 52.0
0 0.191626071929932 53.0
0 -0.172707796096802 54.0
0 0.0133686065673828 55.0
0 0.341534495353699 56.0
0 -0.0822486281394958 57.0
0 0.260663032531738 58.0
0 0.170056343078613 59.0
0 -0.128311634063721 60.0
0 0.0904520153999329 61.0
0 0.278636932373047 62.0
285.83036178 254.214736938477 0.0
0 -0.476810216903687 1.0
0 0.083748459815979 2.0
0 0.177285373210907 3.0
0 -0.167177677154541 4.0
0 -0.111180245876312 5.0
0 0.268773674964905 6.0
0 0.563743591308594 7.0
0 0.91630607843399 8.0
0 -0.00578641891479492 9.0
0 0.507094979286194 10.0
0 -0.185604631900787 11.0
0 -0.211750626564026 12.0
51.270367436 49.9543037414551 13.0
44.579845013 43.456672668457 14.0
0 -0.16767942905426 15.0
0 -0.143958270549774 16.0
27.57866214 33.4625701904297 17.0
0 0.0491943359375 18.0
0 0.00822627544403076 19.0
0 0.092565655708313 20.0
0 -0.113974690437317 21.0
0 0.61368715763092 22.0
39.730743801 42.242790222168 23.0
0 0.0382686853408813 24.0
0 1.13828504085541 25.0
0 0.0850760936737061 26.0
0 -0.0671502947807312 27.0
0 -0.294693827629089 28.0
0 0.238516509532928 29.0
0 -0.108197033405304 30.0
0 -0.0563150644302368 31.0
0 -0.0416543483734131 32.0
0 -0.388684153556824 33.0
0 0.274886071681976 34.0
0 0.0176767110824585 35.0
0 0.792492270469666 36.0
0 0.168630480766296 37.0
0 -0.0227506160736084 38.0
0 -0.193838477134705 39.0
0 0.206664025783539 40.0
0 -0.0610026121139526 41.0
0 0.0721887350082397 42.0
305.64654629 294.116455078125 43.0
0 -0.0945903658866882 44.0
0 0.0900132656097412 45.0
0 0.118324458599091 46.0
0 0.123991847038269 47.0
0 0.279754161834717 48.0
0 0.436618864536285 49.0
0.078425341343 0.0500331521034241 50.0
1.5268959518 1.88553667068481 51.0
0 0.031474232673645 52.0
0 0.0757220387458801 53.0
0 -0.0648957490921021 54.0
0 -0.0877138376235962 55.0
0 0.43533194065094 56.0
0 -0.107141673564911 57.0
0 0.0527264475822449 58.0
0 0.000139236450195312 59.0
0 -0.130448222160339 60.0
0 0.034284234046936 61.0
0 0.0629733800888062 62.0
245.81264008 255.799026489258 0.0
0 -0.287914872169495 1.0
0 -0.179082632064819 2.0
0 0.431166768074036 3.0
0 -0.173771262168884 4.0
0 0.239406883716583 5.0
0 0.18309611082077 6.0
0 0.171205103397369 7.0
0 0.22057032585144 8.0
0 0.0437678098678589 9.0
0 0.234184682369232 10.0
0 -0.304541707038879 11.0
0 -0.0226062536239624 12.0
64.99393289 65.1529006958008 13.0
26.698685448 32.9903259277344 14.0
0 -0.13567042350769 15.0
0 -0.168554484844208 16.0
37.069741489 34.9337615966797 17.0
0 0.103413999080658 18.0
0 -0.00457859039306641 19.0
0 0.0811465382575989 20.0
0 0.108554840087891 21.0
0 0.332194268703461 22.0
35.368887519 34.596565246582 23.0
0 0.0665450096130371 24.0
0 0.74806672334671 25.0
0 -0.0058358907699585 26.0
0 -0.203485727310181 27.0
0 0.411339372396469 28.0
0 0.00336378812789917 29.0
0 0.0523717999458313 30.0
0 -0.0248199701309204 31.0
0 -0.112285971641541 32.0
0 -0.164136528968811 33.0
0 0.0609440803527832 34.0
0 0.0460420250892639 35.0
0 0.550565361976624 36.0
0 0.166620671749115 37.0
0 0.0807530879974365 38.0
0 -0.237770438194275 39.0
0 0.589730262756348 40.0
0 -0.143395662307739 41.0
0 0.0361690521240234 42.0
291.33799704 296.030517578125 43.0
0 0.186827182769775 44.0
0 -0.0424107313156128 45.0
0 0.182377457618713 46.0
0 0.201180279254913 47.0
0 -0.375231862068176 48.0
0 0.509657680988312 49.0
0.099520487438 -0.00613480806350708 50.0
2.1526552288 1.83658647537231 51.0
0 0.113189101219177 52.0
0 -0.169785261154175 53.0
0 -0.133096575737 54.0
0 -0.0892777442932129 55.0
0 0.0846854448318481 56.0
0 0.232106804847717 57.0
0 0.110005855560303 58.0
0 0.270511746406555 59.0
0 -0.10017204284668 60.0
0 0.154041171073914 61.0
0 0.306889772415161 62.0
241.20453296 251.46110534668 0.0
0 0.296088933944702 1.0
0 -0.0566807389259338 2.0
0 0.153954386711121 3.0
0 -0.294179320335388 4.0
0 0.142970621585846 5.0
0 0.128010988235474 6.0
0 -0.0945447683334351 7.0
0 0.150280594825745 8.0
0 0.30749249458313 9.0
0 -0.0808380246162415 10.0
0 0.0610402226448059 11.0
0 -0.194892823696136 12.0
56.83792508 58.7598190307617 13.0
41.050512659 30.5080394744873 14.0
0 -0.0329173803329468 15.0
0 -0.0988622307777405 16.0
28.022522009 32.2977867126465 17.0
0 0.00950241088867188 18.0
0 0.0131230354309082 19.0
0 0.116436004638672 20.0
0 0.175892531871796 21.0
0 0.241156041622162 22.0
36.448437654 35.0666122436523 23.0
0 0.0194059014320374 24.0
0 0.923378348350525 25.0
0 -0.200399160385132 26.0
0 -0.0201477408409119 27.0
0 0.47607496380806 28.0
0 0.203582942485809 29.0
0 -0.12404191493988 30.0
0 -0.342333555221558 31.0
0 -0.119387328624725 32.0
0 -0.64374566078186 33.0
0 0.0533252358436584 34.0
0 0.196864604949951 35.0
0 0.599625110626221 36.0
0 -0.0722599029541016 37.0
0 0.256409585475922 38.0
0 -0.610720872879028 39.0
0 0.401168584823608 40.0
0 -0.182861983776093 41.0
0 0.014910101890564 42.0
298.58716259 297.853424072266 43.0
0 0.0201059579849243 44.0
0 0.13140457868576 45.0
0 0.118840873241425 46.0
0 0.147320449352264 47.0
0 0.038567841053009 48.0
0 0.400254130363464 49.0
0.091440580127 0.00633764266967773 50.0
2.0054477443 1.90006065368652 51.0
0 -0.167670130729675 52.0
0 -0.0384248495101929 53.0
0 -0.145285129547119 54.0
0 -0.064486563205719 55.0
0 -0.045353889465332 56.0
0 -0.0915045142173767 57.0
0 0.148362040519714 58.0
0 0.14174747467041 59.0
0 -0.101243019104004 60.0
0 0.0277391076087952 61.0
0 0.121398389339447 62.0
224.11415861 252.221817016602 0.0
0 -0.0305936336517334 1.0
0 -0.0745561718940735 2.0
0 -0.0534011721611023 3.0
0 -0.399608135223389 4.0
0 -0.280335426330566 5.0
0 -0.0176426768302917 6.0
0 -0.253616094589233 7.0
0 0.0545053482055664 8.0
0 0.385837107896805 9.0
0 0.367231577634811 10.0
0 -0.245875716209412 11.0
0 0.00500088930130005 12.0
68.655427755 67.9651947021484 13.0
25.343893349 30.6408672332764 14.0
0 0.838392376899719 15.0
0 0.00518143177032471 16.0
34.742793499 31.1912155151367 17.0
0 0.161550641059875 18.0
0 -0.00782096385955811 19.0
0 0.0744103193283081 20.0
0 0.207828462123871 21.0
0 0.0515727400779724 22.0
30.883864894 29.6596603393555 23.0
0 0.0451922416687012 24.0
0 0.481468707323074 25.0
0 0.0728312730789185 26.0
0 -0.0145047903060913 27.0
0 -0.0948109030723572 28.0
0 -0.0330802798271179 29.0
0 0.0600021481513977 30.0
0 -0.164324522018433 31.0
0 -0.214895248413086 32.0
0 -0.374647974967957 33.0
0 0.38526439666748 34.0
0 0.25948703289032 35.0
0 0.107325732707977 36.0
0 0.0827803611755371 37.0
0 0.359718501567841 38.0
0 -0.366402983665466 39.0
0 0.202037692070007 40.0
0 -0.164890110492706 41.0
0 -0.0571962594985962 42.0
310.85769402 296.168365478516 43.0
0 0.0933268070220947 44.0
0 0.20288872718811 45.0
0 0.122155427932739 46.0
0 -0.0578619837760925 47.0
0 0.139523804187775 48.0
0 0.428471207618713 49.0
0.063293244035 -0.026542067527771 50.0
1.7952988643 1.87905049324036 51.0
0 -0.114869356155396 52.0
0 -0.270937919616699 53.0
0 0.00892341136932373 54.0
0 0.188338994979858 55.0
0 -0.0372025370597839 56.0
0 0.264047503471375 57.0
0 -0.169906675815582 58.0
0 -0.082650363445282 59.0
0 -0.217939019203186 60.0
0 0.227936625480652 61.0
0 0.176181793212891 62.0
280.14880499 252.957412719727 0.0
0 -0.147467255592346 1.0
0 0.0064157247543335 2.0
0 0.180807709693909 3.0
0 0.158073306083679 4.0
0 0.176526963710785 5.0
0 -0.304221868515015 6.0
0 0.23986542224884 7.0
0 -0.290164470672607 8.0
0 -0.0466368794441223 9.0
0 -0.2031489610672 10.0
0 0.0143027901649475 11.0
0 0.0957778096199036 12.0
62.797252697 64.4159622192383 13.0
38.960147004 31.7430686950684 14.0
0 -0.0292606353759766 15.0
0 0.0731838345527649 16.0
35.919535291 32.017707824707 17.0
0 0.152069747447968 18.0
0 -0.00273466110229492 19.0
0 0.0841013789176941 20.0
0 0.258267879486084 21.0
0 -0.181803107261658 22.0
27.069836156 24.992244720459 23.0
0 0.0794748663902283 24.0
0 0.758613467216492 25.0
0 -0.0185747742652893 26.0
0 0.0868741869926453 27.0
0 -0.674954533576965 28.0
0 0.0708807706832886 29.0
0 0.0611019730567932 30.0
0 0.011134147644043 31.0
0 0.132126212120056 32.0
0 -0.359992742538452 33.0
0 0.208500385284424 34.0
0 0.461272180080414 35.0
0 -0.404311180114746 36.0
0 -0.0522994995117188 37.0
0 0.0428475141525269 38.0
0 -0.478080987930298 39.0
0 0.405571401119232 40.0
0 -0.267561554908752 41.0
0 0.0666251182556152 42.0
286.00464313 294.148315429688 43.0
0 0.058515191078186 44.0
0 0.179635643959045 45.0
0 0.0918336510658264 46.0
0 -0.0837209224700928 47.0
0 0.340579867362976 48.0
0.88651629558 0.612399578094482 49.0
0.094675648822 0.0462930202484131 50.0
1.7437741584 1.96086931228638 51.0
0 -0.0914344787597656 52.0
0 -0.389618277549744 53.0
0 -0.164345502853394 54.0
0 0.181044459342957 55.0
0 0.137943625450134 56.0
0 -0.165770590305328 57.0
0 0.139595627784729 58.0
0 0.331589996814728 59.0
0 -0.0918307900428772 60.0
0 0.182554543018341 61.0
0 0.347284197807312 62.0
224.84921263 254.769546508789 0.0
0 -0.235748529434204 1.0
0 -0.142095804214478 2.0
0 -0.235817909240723 3.0
0 -0.0532170534133911 4.0
0 -0.377158880233765 5.0
0 0.151701331138611 6.0
0 0.106573343276978 7.0
0 -0.0742592811584473 8.0
0 0.71487021446228 9.0
0 1.06350183486938 10.0
0 -0.238344669342041 11.0
0 -0.155639410018921 12.0
63.863949206 62.8046951293945 13.0
27.311140513 36.4164352416992 14.0
0 -0.0326710939407349 15.0
0 -0.219330191612244 16.0
39.04707468 34.240177154541 17.0
0 0.0589616298675537 18.0
0 0.00339359045028687 19.0
0 0.233165562152863 20.0
0 -0.00538069009780884 21.0
0 0.226928114891052 22.0
41.781921956 38.9779357910156 23.0
0 0.0469117760658264 24.0
0 1.07443284988403 25.0
0 0.0697698593139648 26.0
0 -0.0335747003555298 27.0
0 0.231751322746277 28.0
0 0.0897746682167053 29.0
0 0.0618827342987061 30.0
0 -0.20490574836731 31.0
0 -0.0323729515075684 32.0
0 -0.148210167884827 33.0
0 -0.0234001874923706 34.0
0 0.73163378238678 35.0
0 0.218092083930969 36.0
0 -0.10603392124176 37.0
0 0.727470993995667 38.0
0 -0.57314133644104 39.0
0 -0.619387269020081 40.0
0 -0.234504342079163 41.0
0 0.0831570625305176 42.0
291.77721056 297.497100830078 43.0
0 0.38993889093399 44.0
0 -1.03291630744934 45.0
0 0.373932480812073 46.0
0 -0.338802933692932 47.0
0 -0.212649464607239 48.0
0 0.409409195184708 49.0
0.10685570861 -0.0316320061683655 50.0
2.3194510292 1.8681458234787 51.0
0 -0.0580298900604248 52.0
0 -0.192916572093964 53.0
0 0.131141602993011 54.0
0 0.289307415485382 55.0
0 0.00364780426025391 56.0
0 0.249665141105652 57.0
0 0.457198232412338 58.0
0 0.159436464309692 59.0
0 -0.0997763872146606 60.0
0 0.000189602375030518 61.0
0 0.0939534306526184 62.0
268.52452419 250.805603027344 0.0
0 -0.205545663833618 1.0
0 0.0773806571960449 2.0
0 -0.00150233507156372 3.0
0 -0.18710458278656 4.0
0 0.135084986686707 5.0
0 -0.0197907686233521 6.0
0 0.236727476119995 7.0
0 -0.405321359634399 8.0
0 0.0836030244827271 9.0
0 -0.153882503509521 10.0
0 -0.130205512046814 11.0
0 -0.0192369818687439 12.0
74.758009153 74.071174621582 13.0
16.162382967 29.3337383270264 14.0
0 -0.107123851776123 15.0
0 -0.0313076972961426 16.0
30.245221427 33.7524795532227 17.0
0 0.0751068592071533 18.0
0 -0.0510435104370117 19.0
0 0.120348870754242 20.0
0 0.0272862911224365 21.0
0 0.243874907493591 22.0
34.992444182 33.3753700256348 23.0
0 0.0305189490318298 24.0
0 0.413692623376846 25.0
0 -0.0589603781700134 26.0
0 -0.0868831872940063 27.0
0 -0.249776124954224 28.0
0 -0.110031485557556 29.0
0 0.133193612098694 30.0
0 0.0426331758499146 31.0
0 -0.127270340919495 32.0
0 -0.117161452770233 33.0
0 0.226576745510101 34.0
0 0.357173770666122 35.0
0 0.367250263690948 36.0
0 0.104232728481293 37.0
0 -0.155551552772522 38.0
0 -0.224862456321716 39.0
0 -0.0227904319763184 40.0
0 -0.102322936058044 41.0
0 0.162455558776855 42.0
297.29688204 297.283416748047 43.0
0 0.219393253326416 44.0
0 0.0911678075790405 45.0
0 0.0870132446289062 46.0
0 -0.198545575141907 47.0
0 0.386624246835709 48.0
0.64313282151 0.391030550003052 49.0
0.10394210898 0.0635459423065186 50.0
1.934684858 1.87064385414124 51.0
0 0.0357103943824768 52.0
0 -0.0814361572265625 53.0
0 -0.152161717414856 54.0
0 0.09969162940979 55.0
0 0.294383376836777 56.0
0 -0.0475932359695435 57.0
0 0.14375627040863 58.0
0 0.0988880395889282 59.0
0 -0.171607375144958 60.0
0 0.0230207443237305 61.0
0 0.122805655002594 62.0
227.12903339 251.736282348633 0.0
0 0.133166193962097 1.0
0 -0.0994746685028076 2.0
0 0.00158363580703735 3.0
0 -0.202304840087891 4.0
0 -0.30026912689209 5.0
0 0.0268647074699402 6.0
0 -0.367984533309937 7.0
0 0.157744288444519 8.0
0 0.457851022481918 9.0
0 0.0367892980575562 10.0
0 -0.176854074001312 11.0
0 -0.0478016138076782 12.0
62.363471687 61.4809722900391 13.0
34.022535837 37.2722663879395 14.0
0 -0.0842481851577759 15.0
0 -0.0539202690124512 16.0
32.159857575 35.6150169372559 17.0
0 0.0748442411422729 18.0
0 -0.0272536277770996 19.0
0 0.0347438454627991 20.0
0 0.123366117477417 21.0
0 0.264558792114258 22.0
31.054229572 31.4699535369873 23.0
0 0.0471597909927368 24.0
0 1.17701470851898 25.0
0 0.15758228302002 26.0
0 -0.0104453563690186 27.0
0 -0.293794989585876 28.0
0 -0.11085832118988 29.0
0 0.0481575131416321 30.0
0 -0.259789824485779 31.0
0 -0.0383647084236145 32.0
0 0.0467774868011475 33.0
0 -0.0837426781654358 34.0
0 -0.187389969825745 35.0
0 -0.094353199005127 36.0
0 -0.0508612394332886 37.0
0 0.0607906579971313 38.0
0 -0.20252251625061 39.0
0 0.0783905982971191 40.0
0 -0.101276576519012 41.0
0 0.0633251667022705 42.0
287.89130055 295.461029052734 43.0
0 0.379458546638489 44.0
0 -0.00615030527114868 45.0
0 0.196962714195251 46.0
0 -0.01949542760849 47.0
0 0.128863036632538 48.0
0.6660507776 0.547113835811615 49.0
0.086564654553 0.041489839553833 50.0
1.6741268827 1.83340764045715 51.0
0 0.248656332492828 52.0
0 -0.0157264471054077 53.0
0 0.0694747567176819 54.0
0 -0.0516896843910217 55.0
0 -0.172634661197662 56.0
0 0.223917841911316 57.0
0 0.075334370136261 58.0
0 -0.0318137407302856 59.0
0 -0.109702408313751 60.0
0 0.0518832206726074 61.0
0 0.237331807613373 62.0
256.02880642 255.806457519531 0.0
0 -0.128134608268738 1.0
0 -0.106575489044189 2.0
0 0.204639971256256 3.0
0 -0.247640609741211 4.0
0 0.00184714794158936 5.0
0 0.293978154659271 6.0
0 0.00456303358078003 7.0
0 -0.477700710296631 8.0
0 0.117669761180878 9.0
0 0.411101818084717 10.0
0 -0.334039092063904 11.0
0 0.0118470788002014 12.0
55.420499037 55.4885101318359 13.0
36.72611185 39.0512199401855 14.0
0 -0.178128242492676 15.0
0 -0.274834394454956 16.0
27.728908385 32.6745910644531 17.0
0 0.124829769134521 18.0
0 0.00569385290145874 19.0
0 0.136275231838226 20.0
0 0.0777585506439209 21.0
0 0.158742904663086 22.0
35.510548487 32.9525680541992 23.0
0 0.066007673740387 24.0
0 0.0177849531173706 25.0
0 -0.149851858615875 26.0
0 -0.0511981248855591 27.0
0 0.18883603811264 28.0
0 -0.0846447944641113 29.0
0 0.0844997763633728 30.0
0 -0.167805433273315 31.0
0 -0.0748679637908936 32.0
0 -0.086094081401825 33.0
0 -0.55387270450592 34.0
0 0.329800695180893 35.0
0 0.165123820304871 36.0
0 0.0230030417442322 37.0
0 -0.0779536366462708 38.0
0 -0.292058348655701 39.0
0 -0.0342162847518921 40.0
0 -0.123222529888153 41.0
0 0.166814804077148 42.0
301.58470093 297.621704101562 43.0
0 0.0894815921783447 44.0
0 0.205413937568665 45.0
0 0.0682128071784973 46.0
0 -0.034526526927948 47.0
0 -0.162091910839081 48.0
0 0.30594801902771 49.0
0.083739554075 0.0441468954086304 50.0
1.7598980153 1.81489300727844 51.0
0 -0.0390107035636902 52.0
0 -0.292539119720459 53.0
0 0.250052690505981 54.0
0 0.0447441339492798 55.0
0 -0.23290741443634 56.0
0 0.0971791744232178 57.0
0 0.298277884721756 58.0
0 0.310360878705978 59.0
0 -0.234617710113525 60.0
0 0.0772224068641663 61.0
0 0.319850564002991 62.0
236.47592496 252.006851196289 0.0
0 0.119835019111633 1.0
0 -0.236460089683533 2.0
0 0.0614738464355469 3.0
0 -0.375690817832947 4.0
0 0.0045202374458313 5.0
0 0.0936475992202759 6.0
0 0.0126104354858398 7.0
0 -0.0766668319702148 8.0
0 0.0147659182548523 9.0
0 0.244212090969086 10.0
0 -0.327795624732971 11.0
0 0.0344477891921997 12.0
57.951924482 57.4486198425293 13.0
33.361025765 35.2561111450195 14.0
0 -0.116822361946106 15.0
0 0.0964080691337585 16.0
39.877750025 35.9902954101562 17.0
0 0.158494710922241 18.0
0 0.000517725944519043 19.0
0 0.0179818868637085 20.0
0 0.214446604251862 21.0
0 -0.0940080285072327 22.0
28.989765415 28.6916313171387 23.0
0 0.0341511964797974 24.0
0 0.255589604377747 25.0
0 -0.0473240613937378 26.0
0 -0.0881250500679016 27.0
0 -0.717700600624084 28.0
0 -0.142936825752258 29.0
0 0.0332260131835938 30.0
0 0.192652821540833 31.0
0 0.0100760459899902 32.0
0 -0.218850493431091 33.0
0 0.3536456823349 34.0
0 0.236444771289825 35.0
0 0.491864204406738 36.0
0 0.047638475894928 37.0
0 0.14392626285553 38.0
0 -0.28405499458313 39.0
0 0.459827780723572 40.0
0 -0.166710734367371 41.0
0 0.235259890556335 42.0
297.73425418 294.819305419922 43.0
0 0.0210076570510864 44.0
0 0.210858225822449 45.0
0 0.247054517269135 46.0
0 -0.245818972587585 47.0
0 -0.381926417350769 48.0
0 0.396627157926559 49.0
0.082918852283 -0.0214172601699829 50.0
1.6807894815 1.84555411338806 51.0
0 -0.183820307254791 52.0
0 -0.212831735610962 53.0
0 -0.021306037902832 54.0
0 0.0881974101066589 55.0
0 -0.020954966545105 56.0
0 -0.00273025035858154 57.0
0 0.351165533065796 58.0
0 0.187502264976501 59.0
0 0.164173126220703 60.0
0 -0.0047181248664856 61.0
0 0.117641627788544 62.0
261.20913477 252.230850219727 0.0
0 0.0956152677536011 1.0
0 0.0719839334487915 2.0
0 0.161463022232056 3.0
0 -0.0231987237930298 4.0
0 0.40413099527359 5.0
0 0.316905498504639 6.0
0 0.242011606693268 7.0
0 -0.147653222084045 8.0
0 0.0873475670814514 9.0
0 -0.165880739688873 10.0
0 0.0201501846313477 11.0
0 -0.165368795394897 12.0
73.540565728 73.1095809936523 13.0
37.497280481 26.5329933166504 14.0
0 -0.0906631946563721 15.0
0 -0.0530656576156616 16.0
34.065180874 32.3626556396484 17.0
0 0.054304838180542 18.0
0 0.0459764003753662 19.0
0 0.188497126102448 20.0
0 -0.05491042137146 21.0
0 0.00892984867095947 22.0
39.521300559 40.3850479125977 23.0
0 0.134893953800201 24.0
0 1.10915923118591 25.0
0 0.139453887939453 26.0
0 0.0141292810440063 27.0
0 0.146008431911469 28.0
0 -0.0415196418762207 29.0
0 -1.16825103759766e-05 30.0
0 -0.0528113842010498 31.0
0 -0.093166708946228 32.0
0 -0.0552371740341187 33.0
0 0.395000398159027 34.0
0 0.220674097537994 35.0
0 0.488667279481888 36.0
0 -0.0754562616348267 37.0
0 0.18299674987793 38.0
0 -0.244177103042603 39.0
0 0.580037653446198 40.0
0 -0.129539608955383 41.0
0 0.011232852935791 42.0
302.88631852 297.072204589844 43.0
0 0.0244949460029602 44.0
0 0.102476358413696 45.0
0 0.0914987325668335 46.0
0 0.172408938407898 47.0
0 -0.262342095375061 48.0
0 0.386238276958466 49.0
0.090964684029 -0.0125448703765869 50.0
2.1009679515 1.8961284160614 51.0
0 0.0745793581008911 52.0
0 -0.0634012222290039 53.0
0 0.0476899147033691 54.0
0 0.146054327487946 55.0
0 -0.0132851004600525 56.0
0 0.0937799215316772 57.0
0 0.0226613879203796 58.0
0 0.124674081802368 59.0
0 0.0676468014717102 60.0
0 0.0282449126243591 61.0
0 0.0705384612083435 62.0
291.39999961 254.146041870117 0.0
0 -0.320105791091919 1.0
0 0.11156964302063 2.0
0 -0.0869832038879395 3.0
0 -0.437341809272766 4.0
0 0.644248068332672 5.0
0 0.554567575454712 6.0
0 0.867525994777679 7.0
0 0.458438158035278 8.0
0 0.0177961587905884 9.0
0 0.48956573009491 10.0
0 -0.0357053279876709 11.0
0 -0.150542497634888 12.0
52.735730162 51.9005813598633 13.0
45.758196567 42.2133178710938 14.0
0 -0.182326078414917 15.0
0 0.193657517433167 16.0
36.211377639 34.3937568664551 17.0
0 0.058405876159668 18.0
0 -0.0118913650512695 19.0
0 0.0700381994247437 20.0
0 0.117340922355652 21.0
0 0.114556193351746 22.0
30.790039403 28.841724395752 23.0
0 0.105058670043945 24.0
0 1.03282237052917 25.0
0 -0.1048903465271 26.0
0 -0.0542290806770325 27.0
0 0.438871294260025 28.0
0 -0.0442557334899902 29.0
0 0.0828149914741516 30.0
0 0.0300444364547729 31.0
0 -0.0773190855979919 32.0
0 -0.0151433348655701 33.0
0 -0.0839234590530396 34.0
0 0.136915922164917 35.0
0 0.292456835508347 36.0
0 0.184099733829498 37.0
0 0.129013240337372 38.0
0 -0.311594247817993 39.0
0 0.266399681568146 40.0
0 -0.131817102432251 41.0
0 0.153792023658752 42.0
297.66704426 296.131103515625 43.0
0 0.179335355758667 44.0
0 0.229166269302368 45.0
0 0.255265295505524 46.0
0 -0.283872842788696 47.0
0 0.0883620381355286 48.0
0 0.460445821285248 49.0
0.084100237116 0.0196009874343872 50.0
2.3381196811 1.8522310256958 51.0
0 0.109239518642426 52.0
0 -0.0493093729019165 53.0
0 0.0912057161331177 54.0
0 0.229539513587952 55.0
0 0.106749892234802 56.0
0 -0.0808217525482178 57.0
0 -0.0283645391464233 58.0
0 0.0032883882522583 59.0
0 -0.0723603963851929 60.0
0 0.0837101936340332 61.0
0 0.152996242046356 62.0
224.6741042 253.662261962891 0.0
0 -0.16474974155426 1.0
0 -0.354951858520508 2.0
0 0.281800210475922 3.0
0 -0.126063227653503 4.0
0 -0.116940855979919 5.0
0 -0.0344037413597107 6.0
0 -0.321672677993774 7.0
0 0.236060380935669 8.0
0 0.353611886501312 9.0
0 0.310805022716522 10.0
0 -0.818467974662781 11.0
0 -0.110593497753143 12.0
69.974356348 68.6414947509766 13.0
27.050373049 30.550745010376 14.0
0 -0.072449803352356 15.0
0 0.0709335207939148 16.0
31.612646398 33.8581771850586 17.0
0 0.0530329942703247 18.0
0 0.0239478945732117 19.0
0 0.0941050052642822 20.0
0 -0.00972086191177368 21.0
0 0.420849323272705 22.0
37.724724263 40.6881675720215 23.0
0 0.10349977016449 24.0
0 1.38562071323395 25.0
0 -0.106679081916809 26.0
0 0.0114395618438721 27.0
0 0.194863975048065 28.0
0 0.19209212064743 29.0
0 0.0848245620727539 30.0
0 -0.118708729743958 31.0
0 -0.144868612289429 32.0
0 -0.214478254318237 33.0
0 -0.395719051361084 34.0
0 0.293436914682388 35.0
0 -0.0230913758277893 36.0
0 0.0777426958084106 37.0
0 0.66504842042923 38.0
0 -0.531654477119446 39.0
0 -0.686238050460815 40.0
0 -0.247107863426208 41.0
0 0.274894714355469 42.0
321.01608269 294.505981445312 43.0
0 -0.0694718360900879 44.0
0 0.372881472110748 45.0
0 0.0410268902778625 46.0
0 -0.531320095062256 47.0
0 0.0574128031730652 48.0
0 0.476139336824417 49.0
0.073054749805 0.0114687085151672 50.0
1.7290962862 1.86226093769073 51.0
0 -0.118011653423309 52.0
0 0.235698640346527 53.0
0 0.0769485831260681 54.0
0 0.311421304941177 55.0
0 0.202369213104248 56.0
0 0.443062454462051 57.0
0 0.107362866401672 58.0
0 -0.0625773668289185 59.0
0 0.552146375179291 60.0
0 -0.00118672847747803 61.0
0 0.0734213590621948 62.0
282.49040078 247.769729614258 0.0
0 -0.557616829872131 1.0
0 0.495767056941986 2.0
0 -0.0687699913978577 3.0
0 -0.20022040605545 4.0
0 0.684586107730865 5.0
0 -0.0565134286880493 6.0
0 0.194355547428131 7.0
0 0.617944717407227 8.0
0 0.328735828399658 9.0
0 0.177474319934845 10.0
0 -0.291748523712158 11.0
0 -0.0133923888206482 12.0
51.104022208 51.7488479614258 13.0
42.638712553 40.4460945129395 14.0
0 -0.398125529289246 15.0
0 0.582939982414246 16.0
25.083207806 28.2866458892822 17.0
0 -0.0381357669830322 18.0
0 0.0812014937400818 19.0
0 0.0992262363433838 20.0
0 0.12778514623642 21.0
0 0.047243595123291 22.0
39.910102948 36.006721496582 23.0
0 0.117128968238831 24.0
14.426160808 0.942208766937256 25.0
0 0.0386433601379395 26.0
0 -0.361176729202271 27.0
0 0.633227586746216 28.0
0 0.340122699737549 29.0
0 0.0198972821235657 30.0
0 0.133099675178528 31.0
0 0.278475046157837 32.0
0 0.0182779431343079 33.0
0 0.258726835250854 34.0
0 -0.157220661640167 35.0
0 -0.0424764156341553 36.0
0 0.077170193195343 37.0
0 0.461490780115128 38.0
0 -0.270240426063538 39.0
0 0.446702063083649 40.0
0 -0.270255446434021 41.0
0 0.0538439750671387 42.0
287.29363031 295.856536865234 43.0
0 0.045053243637085 44.0
0 0.231762290000916 45.0
0 0.217890739440918 46.0
0 -0.83408534526825 47.0
0 0.405889689922333 48.0
0.90023994722 0.517800390720367 49.0
0.083858032355 0.0711658000946045 50.0
1.4939362423 1.85351824760437 51.0
0 0.0910240411758423 52.0
0 0.31351837515831 53.0
0 0.154582023620605 54.0
0 -0.117214918136597 55.0
0 0.0317879915237427 56.0
0 -0.0134218335151672 57.0
0 0.542672634124756 58.0
0 -0.603419542312622 59.0
0 0.0455352663993835 60.0
0 0.0848618149757385 61.0
0 0.100021660327911 62.0
281.61783302 252.064880371094 0.0
0 -0.207968235015869 1.0
0 -0.0595591068267822 2.0
0 0.261656045913696 3.0
0 0.038273811340332 4.0
0 0.182887673377991 5.0
0 -0.113974571228027 6.0
0 0.412856966257095 7.0
0 0.036216676235199 8.0
0 -0.206360340118408 9.0
0 -0.103970408439636 10.0
0 -0.21497106552124 11.0
0 -0.139340758323669 12.0
73.382989963 71.5062026977539 13.0
16.380195098 27.4249687194824 14.0
0 0.20147180557251 15.0
0 0.0394558310508728 16.0
27.699902224 32.0654563903809 17.0
0 0.13996946811676 18.0
0 0.00483232736587524 19.0
0 0.0811086893081665 20.0
0 0.199794888496399 21.0
0 -0.0677152276039124 22.0
28.919722692 26.6038608551025 23.0
0 0.0562250018119812 24.0
0 0.329106003046036 25.0
0 -0.0357074737548828 26.0
0 -0.201361775398254 27.0
0 0.863649725914001 28.0
0 0.0115002393722534 29.0
0 0.0635147094726562 30.0
0 -0.230862021446228 31.0
0 0.045347273349762 32.0
0 0.00857657194137573 33.0
0 -0.0618795156478882 34.0
0 0.135426580905914 35.0
0 0.158661007881165 36.0
0 0.0337525606155396 37.0
0 0.311424404382706 38.0
0 -0.260393619537354 39.0
0 0.110783576965332 40.0
0 -0.153900325298309 41.0
0 -0.105132102966309 42.0
302.6795406 297.770233154297 43.0
0 0.0476731061935425 44.0
0 0.0263174176216125 45.0
0 0.00342398881912231 46.0
0 0.0849881172180176 47.0
0 0.14713728427887 48.0
0 0.399982631206512 49.0
0.087304560495 0.0446033477783203 50.0
2.0037237592 1.83959126472473 51.0
0 -0.156261146068573 52.0
0 -0.12605482339859 53.0
0 -0.0529710054397583 54.0
0 0.163883090019226 55.0
0 0.00190061330795288 56.0
0 0.0206613540649414 57.0
0 0.367905497550964 58.0
0 0.263744115829468 59.0
0 -0.0330966711044312 60.0
0 0.120129287242889 61.0
0 0.303529977798462 62.0
274.78671435 253.33430480957 0.0
0 -0.340709209442139 1.0
0 -0.345362067222595 2.0
0 -0.152312755584717 3.0
0 -0.187767088413239 4.0
0 -0.0608270168304443 5.0
0 0.218897938728333 6.0
0 0.331775188446045 7.0
0 -0.660280466079712 8.0
0 -0.268836259841919 9.0
0 0.126395881175995 10.0
0 -0.201214194297791 11.0
0 -0.242610692977905 12.0
53.824693066 52.5338287353516 13.0
43.349507567 40.3486824035645 14.0
0 -0.106780767440796 15.0
0 -0.0691174864768982 16.0
40.470992263 38.6942024230957 17.0
0 0.173591375350952 18.0
0 -0.00901931524276733 19.0
0 0.117671370506287 20.0
0 -0.188584506511688 21.0
0 0.440431892871857 22.0
42.377486054 38.5337867736816 23.0
0 -0.0309620499610901 24.0
0 0.44369187951088 25.0
0 -0.141540288925171 26.0
0 0.0841767191886902 27.0
0 0.135389983654022 28.0
0 -0.051033079624176 29.0
0 0.0342088341712952 30.0
0 0.225072145462036 31.0
0 -0.287403583526611 32.0
0 -0.171295821666718 33.0
0 0.249353289604187 34.0
0 0.0827710628509521 35.0
0 -0.0599284768104553 36.0
0 0.160010457038879 37.0
0 0.0807952284812927 38.0
0 -0.247929692268372 39.0
0 0.0111164450645447 40.0
0 -0.0739926099777222 41.0
0 0.0313607454299927 42.0
315.30465686 297.538360595703 43.0
0 0.0949038863182068 44.0
0 0.0626335740089417 45.0
0 0.0865181088447571 46.0
0 0.03663569688797 47.0
0 -0.163243234157562 48.0
0 0.406370013952255 49.0
0.071187458495 0.00552886724472046 50.0
2.0719745164 1.80275022983551 51.0
0 0.0604488253593445 52.0
0 0.0776880383491516 53.0
0 -0.22014594078064 54.0
0 0.0541954636573792 55.0
0 0.311106145381927 56.0
0 -0.0449938774108887 57.0
0 0.0933559536933899 58.0
0 0.0371415019035339 59.0
0 -0.137638628482819 60.0
0 0.0283451080322266 61.0
0 0.0655791163444519 62.0
209.96597811 252.596771240234 0.0
0 0.0311672687530518 1.0
0 -0.126571476459503 2.0
0 0.102198123931885 3.0
0 -0.444318294525146 4.0
0 -0.473878145217896 5.0
0 -0.0573757290840149 6.0
0 -0.344821691513062 7.0
0 0.15883469581604 8.0
0 0.361257016658783 9.0
0 1.32673454284668 10.0
0 -0.634638071060181 11.0
0 -0.0439616441726685 12.0
74.439340236 76.156364440918 13.0
18.235403543 22.7584476470947 14.0
0 -0.0946410894393921 15.0
0 -0.0879500508308411 16.0
38.328187807 37.2292213439941 17.0
0 0.108618855476379 18.0
0 -0.0619174242019653 19.0
0 0.223533868789673 20.0
0 0.0830438137054443 21.0
0 -0.0563796758651733 22.0
42.001709153 37.6593322753906 23.0
0 -0.0138582587242126 24.0
0 -0.0972979664802551 25.0
0 0.0122576951980591 26.0
0 0.0120275616645813 27.0
0 0.0894938111305237 28.0
0 0.0305615067481995 29.0
0 0.147128999233246 30.0
0 -0.100244045257568 31.0
0 -0.131498634815216 32.0
0 -0.207738101482391 33.0
0 0.546174943447113 34.0
0 0.138480365276337 35.0
0 0.708466053009033 36.0
0 0.044770359992981 37.0
0 0.0607996582984924 38.0
0 -0.28790545463562 39.0
0 0.580325722694397 40.0
0 -0.0620400905609131 41.0
0 -0.107679486274719 42.0
296.09550789 296.316741943359 43.0
0 0.112174153327942 44.0
0 0.126869261264801 45.0
0 0.222960412502289 46.0
0 -0.33597469329834 47.0
0 0.34864217042923 48.0
0 0.476063281297684 49.0
0.095773616928 -0.00102943181991577 50.0
1.5691449372 1.91346752643585 51.0
0 -0.196271538734436 52.0
0 0.00539529323577881 53.0
0 0.131762266159058 54.0
0 0.307709038257599 55.0
0 0.179577052593231 56.0
0 0.13256573677063 57.0
0 -0.24079155921936 58.0
0 0.173531651496887 59.0
0 -0.0712853670120239 60.0
0 0.18833976984024 61.0
0 0.276447355747223 62.0
267.3387124 252.04768371582 0.0
0 -0.079833984375 1.0
0 -0.060646116733551 2.0
0 0.508283615112305 3.0
0 0.0369580984115601 4.0
0 0.114760279655457 5.0
0 -0.0772864818572998 6.0
0 0.0995602011680603 7.0
0 -0.0843090415000916 8.0
0 -0.0964742302894592 9.0
0 0.274417161941528 10.0
0 -0.371142864227295 11.0
0 -0.0928925275802612 12.0
76.580262766 76.2479019165039 13.0
21.955859086 28.3026943206787 14.0
0 -0.10744047164917 15.0
0 -0.134700417518616 16.0
34.115091676 34.6109390258789 17.0
0 0.106470584869385 18.0
0 0.0138795971870422 19.0
0 0.0872235298156738 20.0
0 0.226338863372803 21.0
0 -0.0439119338989258 22.0
25.75549732 24.7449493408203 23.0
0 0.0170941948890686 24.0
0 -0.294405817985535 25.0
0 -0.097467303276062 26.0
0 -0.158668220043182 27.0
0 0.14652681350708 28.0
0 0.101630628108978 29.0
0 0.0605056285858154 30.0
0 -0.066490650177002 31.0
0 0.102239668369293 32.0
0 -0.0165642499923706 33.0
0 0.282486796379089 34.0
0 0.159189283847809 35.0
0 -0.269822239875793 36.0
0 -0.0468479990959167 37.0
0 0.597382962703705 38.0
0 -0.392635226249695 39.0
0 -0.444007039070129 40.0
0 -0.236016392707825 41.0
0 -0.00889945030212402 42.0
299.51989916 299.476531982422 43.0
0 0.00346636772155762 44.0
0 0.205304563045502 45.0
0 0.132827818393707 46.0
0 0.252579510211945 47.0
0 0.383947014808655 48.0
0 0.376625657081604 49.0
0.063777482078 0.0676069855690002 50.0
2.3355659938 1.82424163818359 51.0
0 0.19275838136673 52.0
0 -0.120058715343475 53.0
0 -0.0111106634140015 54.0
0 -0.0311191082000732 55.0
0 0.297800481319427 56.0
0 -0.248831033706665 57.0
0 0.0827183127403259 58.0
0 0.265646874904633 59.0
0 -0.203759491443634 60.0
0 0.0265353322029114 61.0
0 -0.0460114479064941 62.0
275.41966082 251.136489868164 0.0
0 -0.225784778594971 1.0
0 -0.151977837085724 2.0
0 0.000785648822784424 3.0
0 -0.0558627843856812 4.0
0 0.310180306434631 5.0
0 0.198479473590851 6.0
0 0.228452324867249 7.0
0 -0.356431484222412 8.0
0 0.422900855541229 9.0
0 -0.484739899635315 10.0
0 0.331047266721725 11.0
0 -0.0443324446678162 12.0
69.107051479 67.5989990234375 13.0
36.713784803 35.2325553894043 14.0
0 0.14764142036438 15.0
0 -0.120386123657227 16.0
33.26894436 33.9074592590332 17.0
0 0.107162475585938 18.0
0 -0.0182197093963623 19.0
0 0.043316125869751 20.0
0 0.185512185096741 21.0
0 0.0111609101295471 22.0
28.125697556 23.9307422637939 23.0
0 0.0376689434051514 24.0
0 -0.248676896095276 25.0
0 -0.0602110624313354 26.0
0 -0.0556700229644775 27.0
0 -0.056738018989563 28.0
0 -0.029572069644928 29.0
0 0.0236527919769287 30.0
0 0.0717180967330933 31.0
0 0.196311414241791 32.0
0 0.24315482378006 33.0
0 0.414485394954681 34.0
0 0.195491075515747 35.0
0 0.0485047101974487 36.0
0 0.0475497245788574 37.0
0 0.49009245634079 38.0
0 -0.304950356483459 39.0
0 -0.420968055725098 40.0
0 -0.212569355964661 41.0
0 0.530995011329651 42.0
286.46455959 293.700531005859 43.0
0 0.0779216289520264 44.0
0 0.20097416639328 45.0
0 0.230822861194611 46.0
0 -0.311047554016113 47.0
0 0.381934493780136 48.0
0 0.485245704650879 49.0
0.063472850804 0.107551574707031 50.0
2.3516517025 1.81069338321686 51.0
0 -0.0943430662155151 52.0
0 0.00386571884155273 53.0
0 -0.136862754821777 54.0
0 -0.221903324127197 55.0
0 0.0224557518959045 56.0
0 0.0761535167694092 57.0
0 0.268671214580536 58.0
0 0.563809990882874 59.0
0 0.0511738061904907 60.0
0 0.13349461555481 61.0
0 0.282541513442993 62.0
289.04765868 251.772918701172 0.0
0 0.0838700532913208 1.0
0 0.0162099599838257 2.0
0 0.0933772921562195 3.0
0 -0.0335761308670044 4.0
0 -0.159937024116516 5.0
0 0.162467241287231 6.0
0 0.251581430435181 7.0
0 -0.413740158081055 8.0
0 0.112857520580292 9.0
0 -0.277558207511902 10.0
0 0.0422430038452148 11.0
0 -0.10657787322998 12.0
64.776636851 66.7888946533203 13.0
37.14044377 31.4161586761475 14.0
0 -0.0890101194381714 15.0
0 -0.0381171703338623 16.0
32.092707593 34.8238105773926 17.0
0 0.124735116958618 18.0
0 -0.0273534059524536 19.0
0 0.033897876739502 20.0
0 0.122110247612 21.0
0 0.0702717304229736 22.0
31.646382895 31.3794345855713 23.0
0 0.107502460479736 24.0
0 1.1082729101181 25.0
0 -0.0682054162025452 26.0
0 -0.13878458738327 27.0
0 -0.138735592365265 28.0
0 -0.0603763461112976 29.0
0 -0.081404983997345 30.0
0 0.0249960422515869 31.0
0 0.11949223279953 32.0
0 0.0356006026268005 33.0
0 -0.60017204284668 34.0
0 0.321751236915588 35.0
0 0.290784955024719 36.0
0 0.033216655254364 37.0
0 0.0362128615379333 38.0
0 -0.25634491443634 39.0
0 0.24906587600708 40.0
0 -0.0635764598846436 41.0
0 0.175048828125 42.0
292.54840512 297.405517578125 43.0
0 -0.0235088467597961 44.0
0 0.146852910518646 45.0
0 0.28233003616333 46.0
0 0.109587669372559 47.0
0 0.295768201351166 48.0
1.111575566 0.339695870876312 49.0
0.10143928363 0.0212129950523376 50.0
2.2296753441 1.87188982963562 51.0
0 0.0458207130432129 52.0
0 -0.042267918586731 53.0
0 0.168968021869659 54.0
0 -0.157337963581085 55.0
0 -0.112128853797913 56.0
0 0.416780710220337 57.0
0 0.421385198831558 58.0
0 0.3332479596138 59.0
0 -0.000666499137878418 60.0
0 0.095702052116394 61.0
0 0.240857243537903 62.0
258.5528846 251.970825195312 0.0
0 -0.0992556810379028 1.0
0 -0.0274701714515686 2.0
0 0.0949097871780396 3.0
0 -0.203592658042908 4.0
0 0.161718010902405 5.0
0 0.355222761631012 6.0
0 0.316953718662262 7.0
0 0.395565390586853 8.0
0 0.158994078636169 9.0
0 0.0613502860069275 10.0
0 -0.0522292852401733 11.0
0 -0.0147560238838196 12.0
68.212114709 69.0219650268555 13.0
28.114542912 26.2233562469482 14.0
0 -0.0767326354980469 15.0
0 -0.0975770950317383 16.0
31.617475974 34.0878524780273 17.0
0 0.0574805736541748 18.0
0 -0.0196104645729065 19.0
0 0.127336740493774 20.0
0 0.0406140089035034 21.0
0 0.191438972949982 22.0
37.466239438 35.1447525024414 23.0
0 0.0686695575714111 24.0
0 0.677812397480011 25.0
0 0.0714761018753052 26.0
0 -0.0886864066123962 27.0
0 0.606549739837646 28.0
0 -0.0358579158782959 29.0
0 0.0422044396400452 30.0
0 0.15939724445343 31.0
0 0.213290214538574 32.0
0 -0.0284693241119385 33.0
0 -0.00245732069015503 34.0
0 0.0191366672515869 35.0
0 0.429699003696442 36.0
0 -0.031950831413269 37.0
0 0.110607445240021 38.0
0 -0.198871970176697 39.0
0 0.476226001977921 40.0
0 -0.117982983589172 41.0
0 0.038838267326355 42.0
307.27672119 296.458892822266 43.0
0 0.187067031860352 44.0
0 0.083292543888092 45.0
0 0.215060770511627 46.0
0 -0.0352964997291565 47.0
0 0.251136064529419 48.0
0 0.418511748313904 49.0
0.074389333406 0.0275198221206665 50.0
2.2516711333 1.83262157440186 51.0
0 0.0095176100730896 52.0
0 -0.127926230430603 53.0
0 0.283425271511078 54.0
0 0.17901599407196 55.0
0 0.111119210720062 56.0
0 0.144399225711823 57.0
0 0.131757259368896 58.0
0 0.0774697065353394 59.0
0 -0.0831789970397949 60.0
0 -0.0201621055603027 61.0
0 0.0716506838798523 62.0
266.36361609 253.136047363281 0.0
0 -0.0857305526733398 1.0
0 -0.0224470496177673 2.0
0 0.0944222807884216 3.0
0 0.0291754007339478 4.0
0 0.299487918615341 5.0
0 0.529272973537445 6.0
0 0.283881425857544 7.0
0 0.174802482128143 8.0
0 0.0628078579902649 9.0
0 0.591432631015778 10.0
0 -0.698352575302124 11.0
0 -0.115807235240936 12.0
66.240177227 65.0585403442383 13.0
26.088771751 31.5138092041016 14.0
0 0.192728042602539 15.0
0 -0.114234685897827 16.0
28.521411783 30.6205272674561 17.0
0 0.112173914909363 18.0
0 0.0509850978851318 19.0
0 0.118872165679932 20.0
0 0.105070233345032 21.0
0 0.263934016227722 22.0
35.263424717 35.039249420166 23.0
0 0.134810507297516 24.0
0 0.945625722408295 25.0
0 -0.0788009762763977 26.0
0 -0.133528113365173 27.0
0 0.41310653090477 28.0
0 -0.033566415309906 29.0
0 0.011462926864624 30.0
0 -0.198309063911438 31.0
0 0.0521429181098938 32.0
0 -0.0524244904518127 33.0
0 0.207671940326691 34.0
0 -0.104480564594269 35.0
0 0.0960562229156494 36.0
0 -0.069070041179657 37.0
0 0.217835903167725 38.0
0 -0.110011339187622 39.0
0 0.356921255588531 40.0
0 -0.152037858963013 41.0
0 0.174116253852844 42.0
290.76819492 297.185546875 43.0
0 0.125798583030701 44.0
0 0.0188372731208801 45.0
0 0.0415551662445068 46.0
0 0.0672237277030945 47.0
0 0.053686797618866 48.0
0 0.484983563423157 49.0
0.068664587311 0.01429283618927 50.0
2.1048909833 1.83616781234741 51.0
0 0.229033410549164 52.0
0 0.0912817716598511 53.0
0 0.101126372814178 54.0
0 -0.0328975915908813 55.0
0 0.0495691299438477 56.0
0 0.0571917295455933 57.0
0 -0.0496295690536499 58.0
0 0.196288406848907 59.0
0 -0.0842807292938232 60.0
0 0.0494275689125061 61.0
0 0.127590775489807 62.0
290.63223739 252.981857299805 0.0
0 0.0214444398880005 1.0
0 0.0699372291564941 2.0
0 0.0589765906333923 3.0
0 -0.244122505187988 4.0
0 -0.0284249186515808 5.0
0 0.232726335525513 6.0
0 0.88839316368103 7.0
0 0.176938891410828 8.0
0 -0.31813633441925 9.0
0 0.21543675661087 10.0
0 -0.0898478627204895 11.0
0 -0.320756196975708 12.0
57.387260065 55.9256134033203 13.0
22.255530525 38.3502807617188 14.0
0 0.0425863265991211 15.0
0 -0.0390693545341492 16.0
30.17586962 35.5544776916504 17.0
0 0.166925191879272 18.0
0 -0.0429927110671997 19.0
0 0.0326855182647705 20.0
0 0.0734792947769165 21.0
0 0.217153787612915 22.0
35.122059061 33.251163482666 23.0
0 0.0960804224014282 24.0
0 0.903784275054932 25.0
0 0.0410059690475464 26.0
0 0.0643066763877869 27.0
0 -0.076448380947113 28.0
0 -0.11640202999115 29.0
0 0.0674834251403809 30.0
0 -0.0141465663909912 31.0
0 0.110536634922028 32.0
0 -0.190192699432373 33.0
0 0.25314724445343 34.0
0 0.33673307299614 35.0
0 -0.372612476348877 36.0
0 -0.0851436853408813 37.0
0 0.203220248222351 38.0
0 -0.0269086360931396 39.0
0 0.711732506752014 40.0
0 -0.122637689113617 41.0
0 0.874274730682373 42.0
288.03750844 292.443206787109 43.0
0 -0.025361955165863 44.0
0 0.47766250371933 45.0
0 0.26920622587204 46.0
0 -0.0912221670150757 47.0
0 0.268329739570618 48.0
0 0.359881013631821 49.0
0.082724115318 0.00990808010101318 50.0
2.4291772059 1.83956873416901 51.0
0 0.12495231628418 52.0
0 -0.0974977612495422 53.0
0 0.180018067359924 54.0
0 0.177361607551575 55.0
0 0.191095352172852 56.0
0 -0.309518337249756 57.0
0 0.0376171469688416 58.0
0 0.0382692217826843 59.0
0 -0.159333169460297 60.0
0 0.0664085149765015 61.0
0 0.208391606807709 62.0
224.79200779 254.624801635742 0.0
0 -0.171408891677856 1.0
0 -0.188834726810455 2.0
0 -0.0827825665473938 3.0
0 -0.363840460777283 4.0
0 -0.0875911712646484 5.0
0 0.181871712207794 6.0
0 -0.179146766662598 7.0
0 -0.141434609889984 8.0
0 0.377170026302338 9.0
0 0.158021152019501 10.0
0 -0.0379613637924194 11.0
0 -0.0770334005355835 12.0
69.024486517 70.7817764282227 13.0
22.167757293 25.0504570007324 14.0
0 -0.111150622367859 15.0
0 0.0618009567260742 16.0
40.242351004 37.9743957519531 17.0
0 0.0983723402023315 18.0
0 -0.0413151979446411 19.0
0 0.035433828830719 20.0
0 0.119920492172241 21.0
0 0.12880802154541 22.0
27.488591991 28.4116039276123 23.0
0 -0.00280147790908813 24.0
0 0.181959629058838 25.0
0 0.0453618764877319 26.0
0 -0.125724852085114 27.0
0 -0.375390291213989 28.0
0 -0.130853593349457 29.0
0 0.0838019251823425 30.0
0 0.113208532333374 31.0
0 0.308754801750183 32.0
0 -0.050203800201416 33.0
0 0.0721398591995239 34.0
0 -0.0242694616317749 35.0
0 0.145432889461517 36.0
0 -0.0364301800727844 37.0
0 0.218059718608856 38.0
0 -0.12221896648407 39.0
0 0.307942450046539 40.0
0 -0.161072313785553 41.0
0 -0.137979626655579 42.0
280.93342873 297.940185546875 43.0
0 0.00724542140960693 44.0
0 0.166484415531158 45.0
0 0.0978341698646545 46.0
0 0.0290879607200623 47.0
0 -0.0122309327125549 48.0
0 0.396296203136444 49.0
0.098561729148 0.0204418301582336 50.0
2.0492601744 1.85681581497192 51.0
0 0.12083774805069 52.0
0 0.433075726032257 53.0
0 -0.0248643159866333 54.0
0 0.158977091312408 55.0
0 0.0745770931243896 56.0
0 0.229743599891663 57.0
0 0.276899099349976 58.0
0 0.212538719177246 59.0
0 -0.0874694585800171 60.0
0 0.0517187118530273 61.0
0 0.13376784324646 62.0
260.470191 251.969116210938 0.0
0 0.0424759387969971 1.0
0 -0.0850879549980164 2.0
0 0.225332796573639 3.0
0 -0.0352691411972046 4.0
0 0.346041142940521 5.0
0 0.291989862918854 6.0
0 0.146737694740295 7.0
0 0.328045696020126 8.0
0 0.0074494481086731 9.0
0 -0.0502300262451172 10.0
0 -0.219910621643066 11.0
0 -0.0466500520706177 12.0
66.688147596 66.6371459960938 13.0
20.138393068 30.958122253418 14.0
0 -0.0711425542831421 15.0
0 0.0592639446258545 16.0
33.314832262 34.4291343688965 17.0
0 0.119430661201477 18.0
0 -0.0570858716964722 19.0
0 0.104054868221283 20.0
0 0.128068387508392 21.0
0 0.133531808853149 22.0
35.160068501 30.9047832489014 23.0
0 0.0384370684623718 24.0
0 0.508216619491577 25.0
0 -0.0901948213577271 26.0
0 -0.0711649060249329 27.0
0 0.157234966754913 28.0
0 -0.191968619823456 29.0
0 0.0798523426055908 30.0
0 -0.0305291414260864 31.0
0 0.113648116588593 32.0
0 0.0225765705108643 33.0
0 0.284229278564453 34.0
0 0.00444120168685913 35.0
0 0.00516378879547119 36.0
0 -0.0768352746963501 37.0
0 0.176838338375092 38.0
0 -0.377699375152588 39.0
0 0.546329081058502 40.0
0 -0.193211495876312 41.0
0 0.160637378692627 42.0
289.9213926 296.779388427734 43.0
0 0.115249991416931 44.0
0 0.0417441725730896 45.0
0 0.25828605890274 46.0
0 -0.231536388397217 47.0
0 0.306620001792908 48.0
1.8482635241 0.573035776615143 49.0
0.070795443155 0.0511255264282227 50.0
2.0211316999 1.88227438926697 51.0
0 0.233526706695557 52.0
0 -0.131668329238892 53.0
0 -0.0945155620574951 54.0
0 0.0139693021774292 55.0
0 -0.0624618530273438 56.0
0 0.0728908777236938 57.0
0 0.0200398564338684 58.0
0 -0.00634580850601196 59.0
0 -0.00969266891479492 60.0
0 0.0381721258163452 61.0
0 0.0391111969947815 62.0
219.19496235 252.013046264648 0.0
0 -0.10837984085083 1.0
0 -0.15657764673233 2.0
0 0.151587724685669 3.0
0 -0.404558062553406 4.0
0 -0.423261284828186 5.0
0 0.0152823328971863 6.0
0 -0.323907732963562 7.0
0 0.288612484931946 8.0
0 0.0770626068115234 9.0
0 -0.247209072113037 10.0
0 -0.279248714447021 11.0
0 -0.100536286830902 12.0
74.223042769 74.4845352172852 13.0
11.858257154 25.6633777618408 14.0
0 0.0291988849639893 15.0
0 -0.0259460806846619 16.0
28.27688234 34.5454940795898 17.0
0 0.0927531719207764 18.0
0 -0.0489001274108887 19.0
0 0.0813353657722473 20.0
0 -0.0446375608444214 21.0
0 0.348861008882523 22.0
34.957583188 35.0196151733398 23.0
0 0.0247268676757812 24.0
0 0.450276851654053 25.0
0 0.0634207725524902 26.0
0 -0.12673419713974 27.0
0 0.519259572029114 28.0
0 -0.0557631850242615 29.0
0 0.0358899831771851 30.0
0 0.225636839866638 31.0
0 -0.229164123535156 32.0
0 -0.00753611326217651 33.0
0 -0.0167970061302185 34.0
0 -0.0942978262901306 35.0
0 0.316188633441925 36.0
0 -0.00512707233428955 37.0
0 0.144608020782471 38.0
0 -0.238858342170715 39.0
0 0.271984875202179 40.0
0 -0.202196359634399 41.0
0 -0.0376583337783813 42.0
290.31785766 296.742492675781 43.0
0 0.15170681476593 44.0
0 0.101799249649048 45.0
0 0.170580744743347 46.0
0 -0.224943518638611 47.0
0 0.381232231855392 48.0
0 0.427240759134293 49.0
0.064847797163 0.11191201210022 50.0
2.0090568338 1.83790683746338 51.0
0 0.144967079162598 52.0
0 -0.184950232505798 53.0
0 -0.102321743965149 54.0
0 0.0208167433738708 55.0
0 -0.127355575561523 56.0
0 0.114191174507141 57.0
0 0.0854507088661194 58.0
0 0.0058513879776001 59.0
0 -0.164431929588318 60.0
0 -0.0136591196060181 61.0
0 0.0589661598205566 62.0
288.93721796 253.095474243164 0.0
0 -0.299704194068909 1.0
0 -0.069598376750946 2.0
0 -0.0306094884872437 3.0
0 -0.258263468742371 4.0
0 0.0509509444236755 5.0
0 -0.374938249588013 6.0
0 0.482234120368958 7.0
0 0.740871846675873 8.0
0 0.194702982902527 9.0
0 0.520873308181763 10.0
0 -0.223860740661621 11.0
0 -0.00631153583526611 12.0
65.384495083 66.1054916381836 13.0
27.501641297 28.1781311035156 14.0
0 -0.023218035697937 15.0
0 0.0267067551612854 16.0
36.122484605 38.2047348022461 17.0
0 0.173655807971954 18.0
0 -0.0726432800292969 19.0
0 0.0754692554473877 20.0
0 0.0647903680801392 21.0
0 0.199235081672668 22.0
32.435889857 31.3891849517822 23.0
0 0.0725488662719727 24.0
0 0.722762227058411 25.0
0 -0.106164395809174 26.0
0 -0.0226139426231384 27.0
0 0.455286890268326 28.0
0 0.0239940285682678 29.0
0 0.0169267058372498 30.0
0 -0.271272420883179 31.0
0 -0.0287073254585266 32.0
0 -0.122365355491638 33.0
0 0.357806712388992 34.0
0 -0.166426062583923 35.0
0 0.156311690807343 36.0
0 -0.0146499872207642 37.0
0 0.237502753734589 38.0
0 -0.33698844909668 39.0
0 0.294092535972595 40.0
0 -0.129511952400208 41.0
0 -0.0247476100921631 42.0
309.01312375 296.617248535156 43.0
0 0.080572247505188 44.0
0 -0.0529985427856445 45.0
0 0.239892423152924 46.0
0 -0.157516181468964 47.0
0 -0.100750207901001 48.0
0 0.483550667762756 49.0
0.08982525606 0.00370126962661743 50.0
1.604908793 1.85129809379578 51.0
0 0.0917933583259583 52.0
0 -0.746798753738403 53.0
0 -0.00769960880279541 54.0
0 -0.0374698638916016 55.0
0 -0.0712713599205017 56.0
0 -0.273961067199707 57.0
0 0.0960897207260132 58.0
0 -0.0428910255432129 59.0
0 -0.00696194171905518 60.0
0 0.0695062875747681 61.0
0 0.166996598243713 62.0
290.66777788 252.325744628906 0.0
0 -0.0515623092651367 1.0
0 0.091971755027771 2.0
0 0.0752759575843811 3.0
0 -0.414706587791443 4.0
0 0.591575264930725 5.0
0 -0.0986659526824951 6.0
0 0.630801558494568 7.0
0 0.0807968974113464 8.0
0 0.279124438762665 9.0
0 0.0169247984886169 10.0
0 0.514526784420013 11.0
0 -0.0468353033065796 12.0
78.006602912 78.3954772949219 13.0
35.903081887 25.1558971405029 14.0
0 -0.0875190496444702 15.0
0 -0.196236968040466 16.0
25.322285054 29.0574016571045 17.0
0 0.0348301529884338 18.0
0 0.143715620040894 19.0
0 0.135907292366028 20.0
0 0.17231148481369 21.0
0 0.0857406258583069 22.0
33.779773957 31.9391632080078 23.0
0 0.0259228944778442 24.0
0 1.01969981193542 25.0
0 -0.0730518698692322 26.0
0 -0.191496551036835 27.0
0 0.139930427074432 28.0
0 0.203396916389465 29.0
0 -0.0989396572113037 30.0
0 0.00777137279510498 31.0
0 -0.0801559686660767 32.0
0 -0.102873206138611 33.0
0 0.379576027393341 34.0
0 0.1180579662323 35.0
0 0.739294290542603 36.0
0 0.0949233770370483 37.0
0 0.0383023023605347 38.0
0 -0.18205201625824 39.0
0 0.416252106428146 40.0
0 -0.111288368701935 41.0
0 -0.0125671625137329 42.0
305.83467021 296.5712890625 43.0
0 0.101862907409668 44.0
0 -0.054125964641571 45.0
0 0.123246073722839 46.0
0 0.275862455368042 47.0
0 0.236476480960846 48.0
0 0.414303183555603 49.0
0.071379511149 0.0403955578804016 50.0
2.2096223 1.89993059635162 51.0
0 0.000325024127960205 52.0
0 0.249920725822449 53.0
0 0.0384196639060974 54.0
0 0.164628863334656 55.0
0 0.214179515838623 56.0
0 -0.11832994222641 57.0
0 0.175400733947754 58.0
0 0.282036542892456 59.0
0 -0.220978498458862 60.0
0 0.116402387619019 61.0
0 0.341391384601593 62.0
272.0474143 247.933959960938 0.0
0 -0.1784907579422 1.0
0 0.151668310165405 2.0
0 0.108251988887787 3.0
0 0.0786694884300232 4.0
0 -0.0160901546478271 5.0
0 0.449829339981079 6.0
0 -0.268062233924866 7.0
0 -0.184427857398987 8.0
0 0.435112297534943 9.0
0 0.343069761991501 10.0
0 -0.315799117088318 11.0
0 -0.00312072038650513 12.0
49.314139113 51.7365798950195 13.0
36.713861648 35.765682220459 14.0
0 0.105563402175903 15.0
0 -0.171846389770508 16.0
40.021672577 37.5512351989746 17.0
0 0.18392688035965 18.0
0 -0.0556779503822327 19.0
0 -0.0337173938751221 20.0
0 0.144553422927856 21.0
0 0.097407341003418 22.0
27.199982913 26.4062595367432 23.0
0 0.00582242012023926 24.0
29.573486731 1.10651028156281 25.0
0 -0.379294037818909 26.0
0 -0.0492039918899536 27.0
0 0.317448437213898 28.0
0 -0.0290336608886719 29.0
0 0.0711674690246582 30.0
0 -0.449054479598999 31.0
0 -0.119647145271301 32.0
0 -0.213079929351807 33.0
0 0.12340897321701 34.0
0 0.107574999332428 35.0
0 0.0396844744682312 36.0
0 -0.0517787933349609 37.0
0 0.31632786989212 38.0
0 -0.189128041267395 39.0
0 0.530955135822296 40.0
0 -0.137168705463409 41.0
0 -0.0436886548995972 42.0
293.39239219 299.232696533203 43.0
0 0.175457119941711 44.0
0 -0.581783652305603 45.0
0 -0.0429016351699829 46.0
0 0.303642213344574 47.0
0 -0.0148194432258606 48.0
0 0.425042241811752 49.0
0.061636533876 0.00989007949829102 50.0
2.0656540364 1.81703758239746 51.0
0 -0.153519868850708 52.0
0 -0.00856864452362061 53.0
0 -0.00795257091522217 54.0
0 0.0302661657333374 55.0
0 0.356562197208405 56.0
0 -0.147072732448578 57.0
0 0.220098733901978 58.0
0 0.236298322677612 59.0
0 -0.194008827209473 60.0
0 0.0359812378883362 61.0
0 0.181977987289429 62.0
254.72235854 253.244583129883 0.0
0 0.409950494766235 1.0
0 0.136293709278107 2.0
0 0.107368588447571 3.0
0 -0.0570775866508484 4.0
0 0.358167350292206 5.0
0 0.327642142772675 6.0
0 0.208065330982208 7.0
0 0.315586507320404 8.0
0 -0.122917115688324 9.0
0 -0.324310898780823 10.0
0 -0.192068874835968 11.0
0 -0.0454504489898682 12.0
54.012941712 53.3620910644531 13.0
42.309681284 43.8006057739258 14.0
0 -0.0506777763366699 15.0
0 0.214231431484222 16.0
34.940489622 31.8918132781982 17.0
0 0.0682212114334106 18.0
0 -0.0121200084686279 19.0
0 0.0783222913742065 20.0
0 0.0689671039581299 21.0
0 0.129545211791992 22.0
36.147926512 36.5777397155762 23.0
0 0.107008218765259 24.0
0 0.914560794830322 25.0
0 -0.0221830010414124 26.0
0 -0.17617392539978 27.0
0 -0.0473285317420959 28.0
0 -0.0775638818740845 29.0
0 0.0395805835723877 30.0
0 -0.103407025337219 31.0
0 -0.195125937461853 32.0
0 -0.351061820983887 33.0
0 0.15790456533432 34.0
0 -0.0758473873138428 35.0
0 0.195211887359619 36.0
0 0.146196305751801 37.0
0 0.0689417719841003 38.0
0 -0.227553009986877 39.0
0 0.521146297454834 40.0
0 -0.107631206512451 41.0
0 0.105361580848694 42.0
303.40690426 297.087524414062 43.0
0 0.140717387199402 44.0
0 0.108792722225189 45.0
0 0.103852570056915 46.0
0 0.0877717733383179 47.0
0 0.16133040189743 48.0
0 0.299380898475647 49.0
0.10031826894 0.0504222512245178 50.0
1.64704654 1.88718616962433 51.0
0 -0.132795631885529 52.0
0 0.0638764500617981 53.0
0 0.00799018144607544 54.0
0 -0.199203670024872 55.0
0 0.0308017730712891 56.0
0 -0.191613554954529 57.0
0 0.182093560695648 58.0
0 0.128470420837402 59.0
0 -0.156379640102386 60.0
0 0.0957337617874146 61.0
0 0.360200881958008 62.0
223.30168101 254.785217285156 0.0
0 -0.0358530282974243 1.0
0 0.0532559156417847 2.0
0 -0.42152726650238 3.0
0 -0.235473275184631 4.0
0 -0.327993988990784 5.0
0 -0.315313816070557 6.0
0 -0.249886035919189 7.0
0 0.213983416557312 8.0
0 -0.0968971848487854 9.0
0 -0.215516209602356 10.0
0 -0.532845616340637 11.0
0 -0.0558899641036987 12.0
75.975722422 72.8686981201172 13.0
13.235427153 24.9105186462402 14.0
0 -0.100294232368469 15.0
0 -0.110318064689636 16.0
32.68582305 32.8144340515137 17.0
0 0.0929350256919861 18.0
0 0.0851773023605347 19.0
0 0.012867271900177 20.0
0 0.327131032943726 21.0
0 -0.292786359786987 22.0
24.145659417 23.5389633178711 23.0
0 0.0591499209403992 24.0
0 0.539496779441833 25.0
0 -0.162034928798676 26.0
0 -0.0885342359542847 27.0
0 0.178042113780975 28.0
0 0.214679956436157 29.0
0 -0.166997015476227 30.0
0 0.325833439826965 31.0
0 0.181319773197174 32.0
0 -0.450262069702148 33.0
0 0.0297573208808899 34.0
0 0.106396496295929 35.0
0 0.256584405899048 36.0
0 0.23611444234848 37.0
0 -0.241307377815247 38.0
0 -0.20714282989502 39.0
0 0.626345038414001 40.0
0 -0.147318661212921 41.0
0 0.0702468156814575 42.0
309.73542332 295.622192382812 43.0
0 0.0141107439994812 44.0
0 -0.0130846500396729 45.0
0 0.0624824166297913 46.0
0 -0.044983983039856 47.0
0 -0.254488945007324 48.0
0 0.377213060855865 49.0
0.082626815004 0.00629270076751709 50.0
2.2206849829 1.8775542974472 51.0
0 -0.00919371843338013 52.0
0 -0.431204557418823 53.0
0 0.358170032501221 54.0
0 -0.22890567779541 55.0
0 -0.0907776355743408 56.0
0 -0.344074606895447 57.0
0 0.253890812397003 58.0
0 0.16250216960907 59.0
0 -0.0349994301795959 60.0
0 -0.0693109035491943 61.0
0 -0.114069163799286 62.0
259.97353359 254.062118530273 0.0
0 -0.274767756462097 1.0
0 0.0118407011032104 2.0
0 -0.145468235015869 3.0
0 -0.395781755447388 4.0
0 0.308616012334824 5.0
0 0.252489805221558 6.0
0 0.00235295295715332 7.0
0 -0.324136972427368 8.0
0 0.0703160762786865 9.0
0 -0.146487176418304 10.0
0 0.970462381839752 11.0
0 -0.336286306381226 12.0
47.79353582 43.202953338623 13.0
34.430654908 34.3083877563477 14.0
0 -0.0406471490859985 15.0
0 0.0170226693153381 16.0
38.71983441 34.4097366333008 17.0
0 0.156436324119568 18.0
0 -0.0499359369277954 19.0
0 0.0787239074707031 20.0
0 0.182293295860291 21.0
0 0.0712952613830566 22.0
28.739520123 25.6639404296875 23.0
0 0.075272262096405 24.0
0 0.474385976791382 25.0
0 -0.324087023735046 26.0
0 0.0523472428321838 27.0
0 0.509179353713989 28.0
0 0.315262198448181 29.0
0 0.0184339284896851 30.0
0 -0.0173615217208862 31.0
0 -0.144378125667572 32.0
0 0.0196044445037842 33.0
0 0.180133581161499 34.0
0 -0.117600679397583 35.0
0 -0.223887205123901 36.0
0 0.112702965736389 37.0
0 -0.0516910552978516 38.0
0 -0.295961380004883 39.0
0 -0.0831342935562134 40.0
0 -0.0885918736457825 41.0
0 -0.155601501464844 42.0
287.49801283 298.669097900391 43.0
0 0.172298073768616 44.0
0 0.0881627202033997 45.0
0 0.184052348136902 46.0
0 -0.0761252641677856 47.0
0 -0.115272641181946 48.0
0 0.356087148189545 49.0
0.10598188183 0.0508063435554504 50.0
2.4356406862 1.74571537971497 51.0
0 -0.079073965549469 52.0
0 0.283824145793915 53.0
0 -0.114089250564575 54.0
0 -0.280025124549866 55.0
0 -0.00193333625793457 56.0
0 0.0750417113304138 57.0
0 0.532319724559784 58.0
0 0.042225182056427 59.0
0 0.169631600379944 60.0
0 0.177992105484009 61.0
0 0.435471415519714 62.0
248.90488222 251.246505737305 0.0
0 0.0651377439498901 1.0
0 -0.0111222863197327 2.0
0 -0.056771457195282 3.0
0 -0.045807421207428 4.0
0 0.0703092813491821 5.0
0 0.144334614276886 6.0
0 0.103258669376373 7.0
0 -0.080971360206604 8.0
0 0.141576766967773 9.0
0 -0.107043862342834 10.0
0 0.0786076188087463 11.0
0 -0.123469710350037 12.0
69.551129863 69.881233215332 13.0
32.540830326 25.5376033782959 14.0
0 -0.0546655654907227 15.0
0 -0.0616441369056702 16.0
32.877838025 32.8701248168945 17.0
0 0.0963403582572937 18.0
0 -0.013664722442627 19.0
0 0.118391513824463 20.0
0 0.00830495357513428 21.0
0 0.185621380805969 22.0
35.404276492 34.3847999572754 23.0
0 0.0413809418678284 24.0
0 0.711823999881744 25.0
0 0.00441068410873413 26.0
0 -0.0434398651123047 27.0
0 0.0911107659339905 28.0
0 -0.0458664298057556 29.0
0 0.0920720100402832 30.0
0 0.0143545866012573 31.0
0 0.0399103760719299 32.0
0 -0.0350264310836792 33.0
0 -0.128418803215027 34.0
0 0.0689681768417358 35.0
0 0.00514662265777588 36.0
0 0.108701109886169 37.0
0 0.144972681999207 38.0
0 -0.330162644386292 39.0
0 0.209738850593567 40.0
0 -0.220320463180542 41.0
0 -0.0248308181762695 42.0
287.09257158 295.408355712891 43.0
0 0.0696722865104675 44.0
0 0.206702888011932 45.0
0 0.104579210281372 46.0
0 -0.0437808036804199 47.0
0 0.425067603588104 48.0
0 0.363365083932877 49.0
0.089455569335 0.0867147445678711 50.0
1.5311249564 1.85632157325745 51.0
0 -0.0536800622940063 52.0
0 -0.00706994533538818 53.0
0 -0.0622536540031433 54.0
0 -0.0229740738868713 55.0
0 -0.0434098243713379 56.0
0 -0.121057868003845 57.0
0 0.514319777488708 58.0
0 0.270707488059998 59.0
0 -0.0023922324180603 60.0
0 0.0732150077819824 61.0
0 0.207385838031769 62.0
245.05914242 252.872970581055 0.0
0 -0.171788811683655 1.0
0 -0.0378026962280273 2.0
0 0.073621392250061 3.0
0 0.0323550701141357 4.0
0 0.31290191411972 5.0
0 0.179003655910492 6.0
0 0.128903329372406 7.0
0 0.0595647096633911 8.0
0 0.132016241550446 9.0
0 -0.0136427283287048 10.0
0 -0.120177388191223 11.0
0 0.0156835913658142 12.0
57.872471477 58.7728500366211 13.0
44.265076846 36.7447090148926 14.0
0 -0.0516295433044434 15.0
0 0.172760128974915 16.0
39.172866734 35.258472442627 17.0
0 0.188697218894958 18.0
0 0.0163848400115967 19.0
0 0.107728123664856 20.0
0 0.0773269534111023 21.0
0 -0.146587133407593 22.0
40.812786084 41.1125984191895 23.0
0 0.142070889472961 24.0
0 1.40715909004211 25.0
0 0.114895224571228 26.0
0 -0.0180326700210571 27.0
0 0.174348056316376 28.0
0 -0.215569615364075 29.0
0 0.0810977220535278 30.0
0 -0.0259268283843994 31.0
0 -0.0474449396133423 32.0
0 -0.118015050888062 33.0
0 0.256128430366516 34.0
0 -0.056178092956543 35.0
0 0.341008841991425 36.0
0 0.0217857956886292 37.0
0 -0.0714438557624817 38.0
0 -0.340782761573792 39.0
0 0.162949919700623 40.0
0 -0.194434285163879 41.0
0 0.230157017707825 42.0
291.21246737 297.543609619141 43.0
0 0.0978414416313171 44.0
0 0.202885329723358 45.0
0 0.304286539554596 46.0
0 -0.0445553064346313 47.0
0 -0.332043409347534 48.0
0 0.481701135635376 49.0
0.096199218641 0.0211889147758484 50.0
2.1336888402 1.79056680202484 51.0
0 0.132066905498505 52.0
0 -0.0691675543785095 53.0
0 -0.0410200357437134 54.0
0 -0.111009657382965 55.0
0 0.0203788876533508 56.0
0 0.416863709688187 57.0
0 0.345641523599625 58.0
0 0.00693774223327637 59.0
0 -0.180023074150085 60.0
0 -0.0463393330574036 61.0
0 0.1207315325737 62.0
294.86048316 252.779632568359 0.0
0 -0.290345907211304 1.0
0 0.0946624279022217 2.0
0 0.404405921697617 3.0
0 -0.31557285785675 4.0
0 0.112555503845215 5.0
0 0.70290744304657 6.0
0 0.460886180400848 7.0
0 -0.00114321708679199 8.0
0 0.226137578487396 9.0
0 -0.261708855628967 10.0
0 0.00703245401382446 11.0
0 -0.181265711784363 12.0
58.903531991 59.3334274291992 13.0
50.304253099 38.1205940246582 14.0
0 -0.190412640571594 15.0
0 0.246847867965698 16.0
34.69172164 36.2430725097656 17.0
0 0.105925321578979 18.0
0 -0.0485717058181763 19.0
0 0.0736316442489624 20.0
0 0.0719163417816162 21.0
0 0.230695962905884 22.0
29.080332385 29.275333404541 23.0
0 0.0969614386558533 24.0
0 1.22102618217468 25.0
0 -0.0812638401985168 26.0
0 -0.0260953903198242 27.0
0 0.248016774654388 28.0
0 0.0558026432991028 29.0
0 0.108559310436249 30.0
0 0.0420172214508057 31.0
0 0.0510649085044861 32.0
0 0.0673114657402039 33.0
0 0.241289615631104 34.0
0 0.18212628364563 35.0
0 0.30223360657692 36.0
0 0.000713169574737549 37.0
0 0.274510681629181 38.0
0 -0.367614388465881 39.0
0 0.0822007656097412 40.0
0 -0.284714698791504 41.0
0 0.12691855430603 42.0
297.29201504 295.423248291016 43.0
0 0.0870404839515686 44.0
0 0.188396394252777 45.0
0 0.174136400222778 46.0
0 -0.054354190826416 47.0
0 0.355523586273193 48.0
0 0.341907382011414 49.0
0.074917853145 0.0534557700157166 50.0
1.9725949338 1.87484157085419 51.0
0 -0.133040130138397 52.0
0 -0.0457596182823181 53.0
0 0.0626797676086426 54.0
0 -0.0404206514358521 55.0
0 -0.143243074417114 56.0
0 -0.0304222106933594 57.0
0 0.150676071643829 58.0
0 0.231113314628601 59.0
0 -0.176492512226105 60.0
0 0.130056977272034 61.0
0 0.323531329631805 62.0
269.13282936 252.15315246582 0.0
0 0.179386615753174 1.0
0 0.215377688407898 2.0
0 -0.0635771751403809 3.0
0 -0.279406785964966 4.0
0 -0.129610121250153 5.0
0 0.245025038719177 6.0
0 0.238352596759796 7.0
0 -0.152626633644104 8.0
0 0.0919637680053711 9.0
0 -0.278830528259277 10.0
0 0.0638009309768677 11.0
0 -0.150674641132355 12.0
49.987269414 50.9128570556641 13.0
34.364241733 37.2223663330078 14.0
0 -0.110027432441711 15.0
0 -0.0284198522567749 16.0
27.694204577 34.0661163330078 17.0
0 0.0722554922103882 18.0
0 0.0451673269271851 19.0
0 -0.0392069816589355 20.0
0 0.32136270403862 21.0
0 0.0643490552902222 22.0
38.291501287 36.0909690856934 23.0
0 0.0472182631492615 24.0
0 5.69036769866943 25.0
0 -0.290896892547607 26.0
0 0.0503199100494385 27.0
0 -0.00802135467529297 28.0
0 0.182650446891785 29.0
0 -0.0236001014709473 30.0
0 1.01342368125916 31.0
0 -0.140005528926849 32.0
0 -0.0975978374481201 33.0
0 0.0564231872558594 34.0
0 0.333637744188309 35.0
0 0.42668205499649 36.0
0 0.0279849171638489 37.0
0 -0.628503680229187 38.0
0 -0.00249564647674561 39.0
0 0.623713076114655 40.0
0 -0.0630755424499512 41.0
0 0.308977842330933 42.0
289.09630278 294.121948242188 43.0
0 0.0700128078460693 44.0
0 0.151688933372498 45.0
0 0.246228218078613 46.0
0 0.0536873936653137 47.0
0 0.158578276634216 48.0
0 0.502501010894775 49.0
0.06533799866 0.0350563526153564 50.0
1.4484521583 1.81739819049835 51.0
0 0.20451158285141 52.0
0 -0.182706892490387 53.0
0 0.202392101287842 54.0
0 0.266076683998108 55.0
0 0.277491867542267 56.0
0 0.0407149791717529 57.0
0 0.14392751455307 58.0
0 -1.45699667930603 59.0
0 2.68449354171753 60.0
0 -0.0783486366271973 61.0
0 0.0757190585136414 62.0
277.47908146 253.680786132812 0.0
0 0.127986550331116 1.0
0 -0.0687296390533447 2.0
0 0.00845301151275635 3.0
0 -0.151046514511108 4.0
0 0.586142778396606 5.0
0 0.124939382076263 6.0
0 0.331203043460846 7.0
0 -0.153002858161926 8.0
0 0.312705397605896 9.0
0 0.297164440155029 10.0
0 -0.167832493782043 11.0
0 -0.0755950808525085 12.0
56.089812067 55.3272399902344 13.0
48.468637005 46.081901550293 14.0
0 0.162279367446899 15.0
0 -0.124076843261719 16.0
33.805103268 33.757698059082 17.0
0 0.0870610475540161 18.0
0 0.024214506149292 19.0
0 0.0432831645011902 20.0
0 0.0199822187423706 21.0
0 0.278200387954712 22.0
33.425997429 33.6312370300293 23.0
0 0.107968270778656 24.0
0 1.20860111713409 25.0
0 -0.0901695489883423 26.0
0 0.0799555778503418 27.0
0 0.140682101249695 28.0
0 -0.0375341773033142 29.0
0 0.0469008684158325 30.0
0 -0.225460052490234 31.0
0 -0.0387318730354309 32.0
0 0.0304076671600342 33.0
0 0.232367873191833 34.0
0 0.31913086771965 35.0
0 0.0586992502212524 36.0
0 0.0385035276412964 37.0
0 -0.0383825302124023 38.0
0 -0.344507575035095 39.0
0 0.148695528507233 40.0
0 -0.122623205184937 41.0
0 0.0974122285842896 42.0
293.42126398 293.783966064453 43.0
0 0.190757632255554 44.0
0 -0.0686324834823608 45.0
0 0.156620025634766 46.0
0 0.0500059127807617 47.0
0 0.0943881869316101 48.0
0 0.394640624523163 49.0
0.078155575475 0.0373991131782532 50.0
2.1598175894 1.87452292442322 51.0
0 0.146862268447876 52.0
0 -0.094089150428772 53.0
0 -0.157865405082703 54.0
0 -0.128211081027985 55.0
0 0.0185903310775757 56.0
0 0.0897165536880493 57.0
0 0.481837093830109 58.0
0 0.257413923740387 59.0
0 -0.272797107696533 60.0
0 0.047547459602356 61.0
0 0.261450588703156 62.0
230.44408218 251.678192138672 0.0
0 -0.142476797103882 1.0
0 0.00653553009033203 2.0
0 0.0919839143753052 3.0
0 -0.43321430683136 4.0
0 -0.111229956150055 5.0
0 0.149696707725525 6.0
0 -0.013760507106781 7.0
0 -0.0503029823303223 8.0
0 -0.0788978934288025 9.0
0 0.195777535438538 10.0
0 -0.362977266311646 11.0
0 -0.0607563257217407 12.0
65.506754529 64.6060028076172 13.0
20.182684266 34.074779510498 14.0
0 -0.0394197702407837 15.0
0 0.0628730654716492 16.0
34.838872478 33.2034683227539 17.0
0 0.141849160194397 18.0
0 0.0512292385101318 19.0
0 0.0897200107574463 20.0
0 0.079297661781311 21.0
0 -0.0443991422653198 22.0
40.269232465 39.4806327819824 23.0
0 0.0374976396560669 24.0
0 0.760169386863708 25.0
0 -0.203599810600281 26.0
0 0.127772271633148 27.0
0 -0.394915223121643 28.0
0 -0.0275970697402954 29.0
0 0.0719882249832153 30.0
0 -0.00919568538665771 31.0
0 2.27689743041992e-05 32.0
0 -0.224317669868469 33.0
0 0.0534766912460327 34.0
0 0.264489352703094 35.0
0 0.317084729671478 36.0
0 0.0467252731323242 37.0
0 -0.0317121148109436 38.0
0 -0.195560932159424 39.0
0 0.881836593151093 40.0
0 -0.0851085186004639 41.0
0 -0.0434054136276245 42.0
297.63265979 297.206146240234 43.0
0 0.0838910937309265 44.0
0 0.219839036464691 45.0
0 0.284882426261902 46.0
0 0.0535514354705811 47.0
0 -0.439218997955322 48.0
0 0.396699547767639 49.0
0.087922028171 -0.0182114839553833 50.0
2.1547050883 1.87549448013306 51.0
0 0.315960556268692 52.0
0 -0.183191657066345 53.0
0 -0.0866715908050537 54.0
0 -0.0542404651641846 55.0
0 -0.0647810101509094 56.0
0 0.366242200136185 57.0
0 0.384070605039597 58.0
0 0.257276594638824 59.0
0 -0.128578960895538 60.0
0 0.0772107839584351 61.0
0 0.20099014043808 62.0
242.38913139 254.16943359375 0.0
0 -0.220248103141785 1.0
0 -0.107824981212616 2.0
0 -0.0438073873519897 3.0
0 -0.376416921615601 4.0
0 -0.27265202999115 5.0
0 -0.0698060393333435 6.0
0 -0.0319251418113708 7.0
0 0.111820340156555 8.0
0 -0.135172724723816 9.0
0 -0.0976481437683105 10.0
0 -0.160165250301361 11.0
0 0.0102639198303223 12.0
51.610605871 52.5415382385254 13.0
38.270892513 39.8340072631836 14.0
0 -0.114252090454102 15.0
0 -0.218847513198853 16.0
34.837471483 32.9028625488281 17.0
0 0.0876839160919189 18.0
0 0.0230317115783691 19.0
0 0.131686270236969 20.0
0 0.101766109466553 21.0
0 0.113191246986389 22.0
30.094018495 31.3849067687988 23.0
0 0.134687006473541 24.0
0 1.42312037944794 25.0
0 -0.128101587295532 26.0
0 0.0314461588859558 27.0
0 0.156924962997437 28.0
0 0.184423387050629 29.0
0 -0.0757433772087097 30.0
0 0.208208560943604 31.0
0 -0.0798869729042053 32.0
0 -0.0198829770088196 33.0
0 0.172106742858887 34.0
0 -0.00422030687332153 35.0
0 -0.0800034403800964 36.0
0 0.174162805080414 37.0
0 0.262993991374969 38.0
0 -0.260771155357361 39.0
0 0.267271876335144 40.0
0 -0.229040384292603 41.0
0 0.0319308042526245 42.0
300.09787078 295.566253662109 43.0
0 0.0407349467277527 44.0
0 0.207891404628754 45.0
0 0.303720951080322 46.0
0 -0.366343259811401 47.0
0 0.224251449108124 48.0
0 0.418332278728485 49.0
0.063910000148 0.0568033456802368 50.0
1.456039889 1.83480656147003 51.0
0 0.0505435466766357 52.0
0 -0.149797737598419 53.0
0 -0.0845053791999817 54.0
0 0.140591144561768 55.0
0 -0.0275761485099792 56.0
0 0.0617923140525818 57.0
0 0.0355233550071716 58.0
0 0.0375490188598633 59.0
0 -0.168171286582947 60.0
0 0.0529505014419556 61.0
0 0.176091134548187 62.0
275.22040505 252.312637329102 0.0
0 -0.321745157241821 1.0
0 0.0565206408500671 2.0
0 0.0988384485244751 3.0
0 0.04007488489151 4.0
0 -0.157028257846832 5.0
0 -0.0216822624206543 6.0
0 0.364484429359436 7.0
0 -0.0317779779434204 8.0
0 0.339059472084045 9.0
0 0.468558818101883 10.0
0 -0.0117870569229126 11.0
0 -0.14289516210556 12.0
66.659907129 66.3453521728516 13.0
35.009623797 32.7091293334961 14.0
0 0.122326016426086 15.0
0 -0.0449520945549011 16.0
30.613093158 33.5150756835938 17.0
0 0.092637300491333 18.0
0 -0.0192426443099976 19.0
0 0.0125216245651245 20.0
0 0.170474529266357 21.0
0 0.195216536521912 22.0
30.724902912 30.1057472229004 23.0
0 0.0686860084533691 24.0
0 0.934519588947296 25.0
0 -0.0407487154006958 26.0
0 -0.0738601684570312 27.0
0 0.25519847869873 28.0
0 -0.00177043676376343 29.0
0 0.00607675313949585 30.0
0 0.0593929290771484 31.0
0 0.0286203622817993 32.0
0 -0.0392142534255981 33.0
0 0.148439943790436 34.0
0 -0.100532293319702 35.0
0 0.304088443517685 36.0
0 0.0655155181884766 37.0
0 0.164147973060608 38.0
0 -0.188681364059448 39.0
0 0.441489815711975 40.0
0 -0.124716818332672 41.0
0 -0.0589942932128906 42.0
302.42822443 297.545349121094 43.0
0 0.00923073291778564 44.0
0 0.0803323984146118 45.0
0 0.325896084308624 46.0
0 -0.218305945396423 47.0
0 0.280622363090515 48.0
0 0.474085479974747 49.0
0.072843792536 0.0282186269760132 50.0
1.6435388387 1.87447738647461 51.0
0 -0.0886096954345703 52.0
0 -0.363034605979919 53.0
0 -0.0980492830276489 54.0
0 -0.153037071228027 55.0
0 -0.0700815320014954 56.0
0 0.454580038785934 57.0
0 0.0451872944831848 58.0
0 -0.0817553400993347 59.0
0 -0.164646685123444 60.0
0 0.0660993456840515 61.0
0 0.17931079864502 62.0
253.53875146 253.508834838867 0.0
0 0.0685042142868042 1.0
0 -0.108047068119049 2.0
0 0.68068528175354 3.0
0 -0.0271709561347961 4.0
0 0.165810704231262 5.0
0 0.146212100982666 6.0
0 0.347513258457184 7.0
0 0.064414381980896 8.0
0 0.165815770626068 9.0
0 0.591034114360809 10.0
0 0.733900189399719 11.0
0 0.0190722346305847 12.0
57.717651483 57.9352607727051 13.0
33.629861082 37.4365425109863 14.0
0 -0.143778204917908 15.0
0 0.33884185552597 16.0
41.111115813 36.9474792480469 17.0
0 0.122525095939636 18.0
0 -0.0867644548416138 19.0
0 0.115638017654419 20.0
0 0.191015541553497 21.0
0 0.0628436207771301 22.0
28.232476034 27.5899066925049 23.0
0 0.0856897830963135 24.0
0 0.62178248167038 25.0
0 -0.150866985321045 26.0
0 -0.233976364135742 27.0
0 0.892474770545959 28.0
0 0.172431468963623 29.0
0 -0.012928307056427 30.0
0 -0.119066596031189 31.0
0 -0.0737828612327576 32.0
0 0.0168018341064453 33.0
0 0.328012436628342 34.0
0 0.31821283698082 35.0
0 0.2988041639328 36.0
0 0.20325756072998 37.0
0 -0.20483660697937 38.0
0 -0.136229395866394 39.0
0 0.394818395376205 40.0
0 -0.275426983833313 41.0
0 0.00442934036254883 42.0
299.73097047 297.640258789062 43.0
0 0.224001526832581 44.0
0 0.121318876743317 45.0
0 0.0139211416244507 46.0
0 0.0169306397438049 47.0
0 0.28039962053299 48.0
3.1362500336 0.603742480278015 49.0
0.065019719745 0.0642744302749634 50.0
2.4569510741 1.90027189254761 51.0
0 0.245869696140289 52.0
0 -0.0795981287956238 53.0
0 -0.226890802383423 54.0
0 -0.0514299869537354 55.0
0 0.40991672873497 56.0
0 -0.128420472145081 57.0
0 0.101510643959045 58.0
0 -0.163408517837524 59.0
0 -0.150708317756653 60.0
0 -0.0876253247261047 61.0
0 -0.10605525970459 62.0
232.65166051 251.264846801758 0.0
0 -0.15376877784729 1.0
0 -0.203243434429169 2.0
0 -0.16231906414032 3.0
0 -0.364391207695007 4.0
0 -0.1868497133255 5.0
0 0.0202683210372925 6.0
0 0.0955613851547241 7.0
0 0.0932096242904663 8.0
0 0.199535429477692 9.0
0 0.244882047176361 10.0
0 -0.280537128448486 11.0
0 0.0408704280853271 12.0
60.810885982 61.1492576599121 13.0
31.929983204 35.734317779541 14.0
0 0.0405764579772949 15.0
0 0.250864088535309 16.0
36.967788076 38.3804779052734 17.0
0 0.160201132297516 18.0
0 -0.0712469816207886 19.0
0 0.081166684627533 20.0
0 0.00866091251373291 21.0
0 0.276508212089539 22.0
35.066385747 35.3003997802734 23.0
0 0.119941592216492 24.0
0 0.934851944446564 25.0
0 -0.183782398700714 26.0
0 0.0504207015037537 27.0
0 -0.0648175477981567 28.0
0 -0.0774044990539551 29.0
0 0.0719068646430969 30.0
0 0.010839581489563 31.0
0 -0.023760199546814 32.0
0 -0.0694479942321777 33.0
0 0.376324206590652 34.0
0 -0.00398492813110352 35.0
0 -0.0151035785675049 36.0
0 -0.0451250076293945 37.0
0 0.104687094688416 38.0
0 -0.266605019569397 39.0
0 0.20880913734436 40.0
0 -0.136807501316071 41.0
0 -0.116897463798523 42.0
288.5187033 298.269378662109 43.0
0 0.0950096845626831 44.0
0 0.258137822151184 45.0
0 0.204804420471191 46.0
0 0.0438563823699951 47.0
0 0.331357657909393 48.0
2.3133490095 0.517670631408691 49.0
0.081148030894 0.0669951438903809 50.0
1.5692991252 1.85858154296875 51.0
0 0.0826690196990967 52.0
0 -0.00988531112670898 53.0
0 0.198768317699432 54.0
0 0.163895428180695 55.0
0 -0.0759731531143188 56.0
0 0.0678276419639587 57.0
0 0.188003182411194 58.0
0 -0.0403494238853455 59.0
0 -0.0036359429359436 60.0
0 0.0942767262458801 61.0
0 0.222589194774628 62.0
258.15531298 252.950469970703 0.0
0 -0.198767423629761 1.0
0 -0.0738265514373779 2.0
0 0.226902544498444 3.0
0 -0.157132506370544 4.0
0 -0.0549357533454895 5.0
0 0.0300300717353821 6.0
0 0.0366184711456299 7.0
0 0.20228511095047 8.0
0 1.18613243103027e-05 9.0
0 -0.266649961471558 10.0
0 -0.271446824073792 11.0
0 -0.0374888777732849 12.0
59.823728168 60.9812393188477 13.0
36.688477536 34.2798843383789 14.0
0 -0.0109298229217529 15.0
0 -0.0232185125350952 16.0
30.383769037 33.4079055786133 17.0
0 0.138805627822876 18.0
0 -0.0507411360740662 19.0
0 0.092486560344696 20.0
0 0.104990839958191 21.0
0 0.159149885177612 22.0
28.557715324 28.8557796478271 23.0
0 0.0787911415100098 24.0
0 0.579519391059875 25.0
0 -0.325669407844543 26.0
0 -0.0803964734077454 27.0
0 0.31827449798584 28.0
0 0.0640299320220947 29.0
0 0.0302456617355347 30.0
0 -0.0767267942428589 31.0
0 -0.0070652961730957 32.0
0 -0.0636881589889526 33.0
0 0.444377720355988 34.0
0 -0.0953096151351929 35.0
0 0.120462834835052 36.0
0 -0.0371665358543396 37.0
0 0.285497784614563 38.0
0 -0.233874678611755 39.0
0 0.141270458698273 40.0
0 -0.292902827262878 41.0
0 0.0795873403549194 42.0
295.10220696 296.221710205078 43.0
0 0.175806641578674 44.0
0 0.12603771686554 45.0
0 0.0838780999183655 46.0
0 -0.0367673635482788 47.0
0 0.367825537919998 48.0
2.7510986584 0.51014506816864 49.0
0.073520814064 0.0219192504882812 50.0
1.5485565509 1.91147208213806 51.0
0 0.0381410121917725 52.0
0 0.0614068508148193 53.0
0 -0.0443682670593262 54.0
0 -0.0783074498176575 55.0
0 0.268346428871155 56.0
0 -0.0222179889678955 57.0
0 0.18083518743515 58.0
0 0.340629190206528 59.0
0 -0.237259387969971 60.0
0 0.0499058365821838 61.0
0 0.140732765197754 62.0
272.98899234 255.755722045898 0.0
0 -0.297186851501465 1.0
0 0.00276905298233032 2.0
0 0.157741546630859 3.0
0 0.0181903839111328 4.0
0 0.239941120147705 5.0
0 0.295590579509735 6.0
0 0.485339105129242 7.0
0 0.272126317024231 8.0
0 -0.440483331680298 9.0
0 -0.00262337923049927 10.0
0 -0.206073760986328 11.0
0 -0.0862528085708618 12.0
58.416888787 57.5562362670898 13.0
28.1961665 39.9746360778809 14.0
0 -0.0778427124023438 15.0
0 -0.10594242811203 16.0
33.943554718 35.0168571472168 17.0
0 0.108130156993866 18.0
0 -0.0235921144485474 19.0
0 0.0793660879135132 20.0
0 0.124180972576141 21.0
0 0.0869613885879517 22.0
33.742980433 29.998348236084 23.0
0 -0.0136033296585083 24.0
0 0.115562558174133 25.0
0 0.0558211803436279 26.0
0 -0.127066075801849 27.0
0 0.389687538146973 28.0
0 -0.0720799565315247 29.0
0 0.0659087300300598 30.0
0 0.131207942962646 31.0
0 -0.0516059398651123 32.0
0 -0.139785766601562 33.0
0 -0.00624930858612061 34.0
0 0.0520476102828979 35.0
0 0.289597749710083 36.0
0 0.0784929990768433 37.0
0 0.140057563781738 38.0
0 -0.229751348495483 39.0
0 0.527699053287506 40.0
0 -0.171070039272308 41.0
0 0.133372783660889 42.0
306.43720587 297.135681152344 43.0
0 0.112351417541504 44.0
0 0.208725929260254 45.0
0 0.133387684822083 46.0
0 -0.161669909954071 47.0
0 0.316597372293472 48.0
0 0.464833974838257 49.0
0.069289000903 0.0508905649185181 50.0
1.7158084527 1.91751706600189 51.0
0 -0.174276173114777 52.0
0 -0.111301779747009 53.0
0 0.148713827133179 54.0
0 0.0394083261489868 55.0
0 0.0172565579414368 56.0
0 0.171352982521057 57.0
0 0.187331080436707 58.0
0 0.300079166889191 59.0
0 -0.162231802940369 60.0
0 0.0937554836273193 61.0
0 0.267704665660858 62.0
253.50736574 246.36669921875 0.0
0 -0.171702146530151 1.0
0 -0.0129291415214539 2.0
0 0.0405178070068359 3.0
0 -0.142239689826965 4.0
0 0.0644775629043579 5.0
0 0.330549299716949 6.0
0 -0.00481224060058594 7.0
0 -0.101795375347137 8.0
0 -0.603010296821594 9.0
0 -0.130846500396729 10.0
0 -0.118162035942078 11.0
0 -0.465848207473755 12.0
47.836731181 46.1436080932617 13.0
42.989432225 42.5078239440918 14.0
0 -0.490860342979431 15.0
0 0.00406908988952637 16.0
36.004107187 33.762996673584 17.0
0 0.0447845458984375 18.0
0 0.0309512615203857 19.0
0 0.0824994444847107 20.0
0 0.112291216850281 21.0
0 -0.0992221832275391 22.0
36.468433913 33.3701629638672 23.0
0 0.0703972578048706 24.0
0 0.74198853969574 25.0
0 -0.0715765953063965 26.0
0 -0.111362934112549 27.0
0 1.10658192634583 28.0
0 0.0246938467025757 29.0
0 -0.0336039662361145 30.0
0 0.00766611099243164 31.0
0 0.0337616801261902 32.0
0 -0.193907856941223 33.0
0 -0.107784748077393 34.0
0 0.179001748561859 35.0
0 0.113611102104187 36.0
0 0.107827067375183 37.0
0 0.174129903316498 38.0
0 -0.364450573921204 39.0
0 0.408569931983948 40.0
0 -0.191092729568481 41.0
0 0.0812786817550659 42.0
303.94303198 296.353363037109 43.0
0 0.0330523252487183 44.0
0 0.107301890850067 45.0
0 0.137591302394867 46.0
0 -0.447478652000427 47.0
0 0.239521324634552 48.0
0 0.492448568344116 49.0
0.090546972208 0.000280678272247314 50.0
1.7581438165 1.90988564491272 51.0
0 -0.0148561596870422 52.0
0 -0.772572755813599 53.0
0 0.220450878143311 54.0
0 -0.0497441291809082 55.0
0 0.0615739226341248 56.0
0 0.179119765758514 57.0
0 0.0214090347290039 58.0
0 0.0183092951774597 59.0
0 -0.264087557792664 60.0
0 -0.0128445029258728 61.0
0 0.0757095813751221 62.0
213.80927252 256.064208984375 0.0
0 -0.10191798210144 1.0
0 -0.108323276042938 2.0
0 0.136579513549805 3.0
0 -0.240604877471924 4.0
0 -0.129855871200562 5.0
0 -0.122925341129303 6.0
0 -0.250808715820312 7.0
0 -0.0801818370819092 8.0
0 -0.122703909873962 9.0
0 0.03121417760849 10.0
0 -0.0654070973396301 11.0
0 -0.0242277979850769 12.0
64.29922089 63.1080741882324 13.0
43.287292406 33.7111892700195 14.0
0 -0.0663881301879883 15.0
0 0.265298128128052 16.0
39.032526494 35.0277061462402 17.0
0 0.100627660751343 18.0
0 0.0448002815246582 19.0
0 0.0883827209472656 20.0
0 0.0851337909698486 21.0
0 0.0956090092658997 22.0
36.738975356 35.4342765808105 23.0
0 0.0770094394683838 24.0
0 0.951335549354553 25.0
0 -0.032798171043396 26.0
0 0.0402201414108276 27.0
0 -0.028713583946228 28.0
0 -0.0877612829208374 29.0
0 0.0913481712341309 30.0
0 -0.16732394695282 31.0
0 -0.104404389858246 32.0
0 -0.132408440113068 33.0
0 0.464189648628235 34.0
0 -0.11144345998764 35.0
0 0.167698621749878 36.0
0 -0.112800180912018 37.0
0 0.169583141803741 38.0
0 -0.191628098487854 39.0
0 0.125956475734711 40.0
0 -0.177448272705078 41.0
0 0.0275747776031494 42.0
294.4924155 298.114501953125 43.0
0 0.0633518695831299 44.0
0 0.0533269643783569 45.0
0 0.0459975004196167 46.0
0 -0.0130741596221924 47.0
0 0.304425448179245 48.0
0 0.314546465873718 49.0
0.07196863207 0.0596199631690979 50.0
1.7036641729 1.85747003555298 51.0
0 -0.11375218629837 52.0
0 0.0357237458229065 53.0
0 -0.0793054103851318 54.0
0 -0.063513457775116 55.0
0 -0.157790958881378 56.0
0 0.11324554681778 57.0
0 0.32495129108429 58.0
0 0.217755794525146 59.0
0 -0.019472062587738 60.0
0 0.090118408203125 61.0
0 0.171841442584991 62.0
233.39653219 253.709686279297 0.0
0 0.184311151504517 1.0
0 0.063348650932312 2.0
0 -0.338819980621338 3.0
0 -0.173582077026367 4.0
0 -0.0738989114761353 5.0
0 -0.355091214179993 6.0
0 -0.334453344345093 7.0
0 0.236811757087708 8.0
0 -0.169305682182312 9.0
0 0.169749617576599 10.0
0 -0.0289381146430969 11.0
0 -0.238427877426147 12.0
73.929151131 67.5606079101562 13.0
34.058096488 24.0969638824463 14.0
0 -0.137386202812195 15.0
0 -0.258487105369568 16.0
25.715121919 33.4088935852051 17.0
0 0.146568298339844 18.0
0 -0.024806022644043 19.0
0 0.0616660118103027 20.0
0 0.194017350673676 21.0
0 -0.0418077707290649 22.0
26.766139073 25.2578945159912 23.0
0 0.0647203922271729 24.0
0 0.100477337837219 25.0
0 -0.0816662907600403 26.0
0 -0.175828099250793 27.0
0 0.497622191905975 28.0
0 -0.0585229396820068 29.0
0 0.00873380899429321 30.0
0 -0.244290471076965 31.0
0 0.0729828476905823 32.0
0 0.117507100105286 33.0
0 0.244190573692322 34.0
0 0.143171370029449 35.0
0 0.377278298139572 36.0
0 -0.0483354330062866 37.0
0 -0.0726081132888794 38.0
0 -0.2386394739151 39.0
0 0.212856829166412 40.0
0 -0.0246052742004395 41.0
0 0.125410914421082 42.0
304.31973609 293.825744628906 43.0
0 0.126816272735596 44.0
0 -0.005271315574646 45.0
0 0.104553937911987 46.0
0 -0.163941621780396 47.0
0 0.121459066867828 48.0
0 0.47426649928093 49.0
0.063046412882 0.0468987822532654 50.0
1.7512188062 1.73833799362183 51.0
0 -0.205779075622559 52.0
0 -0.144005358219147 53.0
0 -0.231815576553345 54.0
0 0.133149921894073 55.0
0 0.429390549659729 56.0
0 0.100895702838898 57.0
0 0.0245106220245361 58.0
0 0.102681875228882 59.0
0 -0.192193329334259 60.0
0 0.0819640755653381 61.0
0 0.126412332057953 62.0
261.08736624 254.571060180664 0.0
0 -0.262064218521118 1.0
0 0.176616072654724 2.0
0 0.207747399806976 3.0
0 -0.0459300875663757 4.0
0 0.0315056443214417 5.0
0 -0.123700439929962 6.0
0 0.341594338417053 7.0
0 -0.183580815792084 8.0
0 -0.12971156835556 9.0
0 -0.000386059284210205 10.0
0 -0.0981181263923645 11.0
0 -0.0463539361953735 12.0
66.179953128 67.1016159057617 13.0
36.407034241 27.3535842895508 14.0
0 -0.105913162231445 15.0
0 0.0106246471405029 16.0
36.044385975 34.3995094299316 17.0
0 0.126359820365906 18.0
0 -0.0256314277648926 19.0
0 0.146368324756622 20.0
0 -0.0772935152053833 21.0
0 0.568081498146057 22.0
40.736021611 40.6993522644043 23.0
0 0.108297407627106 24.0
0 0.883252024650574 25.0
0 0.11746609210968 26.0
0 0.128319978713989 27.0
0 0.103265225887299 28.0
0 -0.145853519439697 29.0
0 0.0637524127960205 30.0
0 -0.0726920366287231 31.0
0 -0.0191100239753723 32.0
0 -0.369981288909912 33.0
0 0.150588631629944 34.0
0 -0.153264880180359 35.0
0 -0.0196897983551025 36.0
0 0.0370579361915588 37.0
0 0.26147449016571 38.0
0 -0.280870199203491 39.0
0 0.278953313827515 40.0
0 -0.186978816986084 41.0
0 0.162732720375061 42.0
292.26257826 297.841522216797 43.0
0 0.17656934261322 44.0
0 0.106018364429474 45.0
0 0.180352568626404 46.0
0 -0.466115355491638 47.0
0 0.0644396543502808 48.0
0 0.426739782094955 49.0
0.066713174088 -0.0156872868537903 50.0
1.8556383549 1.91210412979126 51.0
0 -0.162792563438416 52.0
0 0.0170583128929138 53.0
0 -0.114039599895477 54.0
0 -0.0112666487693787 55.0
0 0.103904068470001 56.0
0 0.0193644762039185 57.0
0 -0.0818696618080139 58.0
0 0.0944612622261047 59.0
0 -0.174440443515778 60.0
0 0.182414710521698 61.0
0 0.416418135166168 62.0
269.42602565 254.821655273438 0.0
0 -0.289685606956482 1.0
0 -0.0559377670288086 2.0
0 0.310179233551025 3.0
0 -0.027410626411438 4.0
0 0.403236657381058 5.0
0 0.133255183696747 6.0
0 0.109131515026093 7.0
0 0.149630546569824 8.0
0 -0.25012731552124 9.0
0 0.267128169536591 10.0
0 0.00414484739303589 11.0
0 -0.184728920459747 12.0
65.799210563 65.7008590698242 13.0
41.067070204 34.3932495117188 14.0
0 0.0239378213882446 15.0
0 -0.153407633304596 16.0
33.01922309 34.065055847168 17.0
0 0.0830593705177307 18.0
0 -0.00786960124969482 19.0
0 0.03266441822052 20.0
0 0.192706763744354 21.0
0 0.0983420610427856 22.0
28.015000106 26.8450355529785 23.0
0 0.0288769006729126 24.0
0 0.461729884147644 25.0
0 -0.00834178924560547 26.0
0 -0.139740228652954 27.0
0 -0.060883641242981 28.0
0 -0.161219596862793 29.0
0 0.0682256817817688 30.0
0 -0.00567328929901123 31.0
0 0.139604032039642 32.0
0 -0.226868152618408 33.0
0 0.31065034866333 34.0
0 0.00267708301544189 35.0
0 0.320500433444977 36.0
0 -0.0645842552185059 37.0
0 0.155771136283875 38.0
0 -0.245026230812073 39.0
0 0.452267467975616 40.0
0 -0.104760527610779 41.0
0 -0.0965801477432251 42.0
298.08422857 297.201324462891 43.0
0 0.11543607711792 44.0
0 0.0221523642539978 45.0
0 0.0565068125724792 46.0
0 0.0291821360588074 47.0
0 0.308766454458237 48.0
0.057292440523 0.438616394996643 49.0
0.093837513548 0.0745035409927368 50.0
1.5961688553 1.83853960037231 51.0
0 0.168201208114624 52.0
0 -0.180195987224579 53.0
0 0.221884071826935 54.0
0 0.268529772758484 55.0
0 -0.0583975315093994 56.0
0 0.414251029491425 57.0
0 -0.0333332419395447 58.0
0 -0.142129600048065 59.0
0 0.075879693031311 60.0
0 0.047116756439209 61.0
0 0.0868182182312012 62.0
270.1104562 255.39599609375 0.0
0 -0.333365201950073 1.0
0 -0.0470143556594849 2.0
0 0.429837703704834 3.0
0 -0.125485956668854 4.0
0 -0.025521457195282 5.0
0 -0.0727726221084595 6.0
0 0.140170335769653 7.0
0 -0.0742722749710083 8.0
0 0.108452558517456 9.0
0 -0.203120231628418 10.0
0 -0.140316605567932 11.0
0 -0.05351722240448 12.0
54.831854958 54.8916969299316 13.0
38.068687422 38.6697235107422 14.0
0 -0.0961828231811523 15.0
0 -0.0380426049232483 16.0
36.462148554 35.8451995849609 17.0
0 0.138401925563812 18.0
0 0.021659255027771 19.0
0 0.0488823056221008 20.0
0 0.0111154317855835 21.0
0 0.350568890571594 22.0
31.741452668 32.3325042724609 23.0
0 0.043380081653595 24.0
0 0.807245492935181 25.0
0 0.060611367225647 26.0
0 -0.213399171829224 27.0
0 0.325845748186111 28.0
0 0.0542336106300354 29.0
0 -0.0157728791236877 30.0
0 -0.0293724536895752 31.0
0 -0.0164833664894104 32.0
0 -0.147289633750916 33.0
0 0.437671482563019 34.0
0 -0.116207242012024 35.0
0 0.486608415842056 36.0
0 0.190649747848511 37.0
0 -0.0684958100318909 38.0
0 -0.0992424488067627 39.0
0 0.454122066497803 40.0
0 -0.150915384292603 41.0
0 -0.0566774606704712 42.0
305.69173281 297.413299560547 43.0
0 0.0473868846893311 44.0
0 0.178118407726288 45.0
0 0.100494503974915 46.0
0 0.0445225238800049 47.0
0 0.412163734436035 48.0
0 0.441219717264175 49.0
0.072399664922 0.0493874549865723 50.0
1.8137849334 1.9168713092804 51.0
0 -0.0080903172492981 52.0
0 0.0220289826393127 53.0
0 0.0855287313461304 54.0
0 0.170923829078674 55.0
0 0.00889098644256592 56.0
0 0.180755496025085 57.0
0 0.204089999198914 58.0
0 0.139294683933258 59.0
0 -0.0763585567474365 60.0
0 0.127865374088287 61.0
0 0.222693800926208 62.0
276.79123355 255.195907592773 0.0
0 -0.389287233352661 1.0
0 -0.0296868681907654 2.0
0 0.00553643703460693 3.0
0 0.0467393398284912 4.0
0 0.383086502552032 5.0
0 0.509496748447418 6.0
0 0.455507040023804 7.0
0 0.437266945838928 8.0
0 -0.0028037428855896 9.0
0 0.346306651830673 10.0
0 -0.0342642068862915 11.0
0 -0.0356183648109436 12.0
63.422758573 61.8922348022461 13.0
37.123995372 36.0599174499512 14.0
0 -0.23817253112793 15.0
0 -0.205628097057343 16.0
36.811108355 35.1765327453613 17.0
0 0.091463565826416 18.0
0 -0.0320537090301514 19.0
0 0.246333658695221 20.0
0 0.111982822418213 21.0
0 0.134985089302063 22.0
36.051308771 35.321891784668 23.0
0 0.101268529891968 24.0
1.5421738041 0.76722639799118 25.0
0 0.0424482822418213 26.0
0 -0.0836206078529358 27.0
0 -0.257282137870789 28.0
0 -0.0624491572380066 29.0
0 0.0848454833030701 30.0
0 -0.00036168098449707 31.0
0 0.0442066192626953 32.0
0 -0.114401936531067 33.0
0 -0.0461814403533936 34.0
0 0.302858769893646 35.0
0 -0.182182312011719 36.0
0 0.157860934734344 37.0
0 0.0993906259536743 38.0
0 -0.28461229801178 39.0
0 0.216298341751099 40.0
0 -0.228444814682007 41.0
0 0.162791728973389 42.0
296.7365224 298.509765625 43.0
0 0.186788320541382 44.0
0 -0.500088691711426 45.0
0 0.139593362808228 46.0
0 0.102486252784729 47.0
0 -0.0697810649871826 48.0
0 0.421485304832458 49.0
0.067873108842 -0.0202169418334961 50.0
2.1565564027 1.86825609207153 51.0
0 0.160729229450226 52.0
0 -0.0646345615386963 53.0
0 0.238966345787048 54.0
0 0.158298254013062 55.0
0 0.135951280593872 56.0
0 0.117984414100647 57.0
0 0.104396343231201 58.0
0 0.176477372646332 59.0
0 0.0282859206199646 60.0
0 0.0451064109802246 61.0
0 0.14673638343811 62.0
248.81457186 254.944137573242 0.0
0 -0.204844117164612 1.0
0 -0.107927799224854 2.0
0 -0.0953869223594666 3.0
0 -0.154410898685455 4.0
0 -0.0514211654663086 5.0
0 0.030644953250885 6.0
0 0.0449473857879639 7.0
0 -0.00798797607421875 8.0
0 0.22751647233963 9.0
0 -0.2938551902771 10.0
0 0.139930605888367 11.0
0 -0.098112940788269 12.0
59.247375114 60.1196517944336 13.0
50.918427684 43.1854209899902 14.0
0 0.00609111785888672 15.0
0 -0.147002279758453 16.0
37.876561457 37.4317588806152 17.0
0 0.163127481937408 18.0
0 0.0339314937591553 19.0
0 0.0146664381027222 20.0
0 0.151227355003357 21.0
0 0.0891922116279602 22.0
29.111025914 30.3738040924072 23.0
0 -0.0785338282585144 24.0
0 0.955309689044952 25.0
0 -0.163324594497681 26.0
0 -0.182466149330139 27.0
0 0.0926814675331116 28.0
0 0.0957764387130737 29.0
0 -0.010037362575531 30.0
0 -0.09993577003479 31.0
0 0.0294426679611206 32.0
0 -0.000555336475372314 33.0
0 0.458896964788437 34.0
0 0.00687456130981445 35.0
0 0.0648689270019531 36.0
0 -0.0999923348426819 37.0
0 0.195373773574829 38.0
0 -0.267516732215881 39.0
0 0.260577440261841 40.0
0 -0.236517906188965 41.0
0 -0.111846685409546 42.0
313.29074583 293.543029785156 43.0
0 -0.0379227995872498 44.0
0 0.130894124507904 45.0
0 0.0229825377464294 46.0
0 0.231037974357605 47.0
0 0.141611039638519 48.0
0 0.568653702735901 49.0
0.09078456601 0.00288307666778564 50.0
1.6286962655 1.81108748912811 51.0
0 0.0613768100738525 52.0
0 -0.280609250068665 53.0
0 0.0157638788223267 54.0
0 0.310325503349304 55.0
0 0.25251042842865 56.0
0 -0.0302809476852417 57.0
0 0.235215663909912 58.0
0 -0.128298044204712 59.0
0 0.817690968513489 60.0
0 0.0822745561599731 61.0
0 0.273037552833557 62.0
232.55699626 253.299957275391 0.0
0 -0.0546793937683105 1.0
0 -0.0160289406776428 2.0
0 0.322888255119324 3.0
0 -0.230057239532471 4.0
0 -0.247105598449707 5.0
0 0.161138534545898 6.0
0 0.0462068915367126 7.0
0 -0.023822546005249 8.0
0 -0.0403965711593628 9.0
0 1.22244095802307 10.0
0 0.112364590167999 11.0
0 -0.108659982681274 12.0
49.465981989 49.2133522033691 13.0
32.225685901 35.2104301452637 14.0
0 -0.0176442861557007 15.0
0 -0.0376119017601013 16.0
33.682527315 31.7422847747803 17.0
0 0.14281153678894 18.0
0 -0.00643062591552734 19.0
0 0.100331783294678 20.0
0 0.195343673229218 21.0
0 0.0498208403587341 22.0
35.282091723 34.2540626525879 23.0
0 0.0530247092247009 24.0
0 0.942600131034851 25.0
0 -0.0166842937469482 26.0
0 -0.201156318187714 27.0
0 -0.699061870574951 28.0
0 -0.157901644706726 29.0
0 -0.168118357658386 30.0
0 0.208804130554199 31.0
0 -0.0735459923744202 32.0
0 -0.126223862171173 33.0
0 0.083462655544281 34.0
0 0.0720852613449097 35.0
0 0.0478172302246094 36.0
0 0.18891716003418 37.0
0 0.124388098716736 38.0
0 -0.0896439552307129 39.0
0 0.390334635972977 40.0
0 -0.149463295936584 41.0
0 0.0511759519577026 42.0
298.45320865 293.914306640625 43.0
0 0.132450342178345 44.0
0 -0.0223548412322998 45.0
0 0.19477790594101 46.0
0 -0.218933343887329 47.0
0 0.308523774147034 48.0
1.9239731665 0.538756549358368 49.0
0.076841748906 0.106866896152496 50.0
1.5279097031 1.84122896194458 51.0
0 0.156172692775726 52.0
0 -0.425539493560791 53.0
0 -0.098391592502594 54.0
0 0.23465633392334 55.0
0 -0.0167334079742432 56.0
0 0.418232500553131 57.0
0 0.550667524337769 58.0
0 0.122465133666992 59.0
0 0.191147804260254 60.0
0 0.0808753967285156 61.0
0 0.172180712223053 62.0
233.65799234 251.397369384766 0.0
0 -0.226395606994629 1.0
0 -0.0202289819717407 2.0
0 -0.0386065244674683 3.0
0 -0.316150188446045 4.0
0 -0.25557553768158 5.0
0 0.088565468788147 6.0
0 0.0320335030555725 7.0
0 0.0538221001625061 8.0
0 0.130369305610657 9.0
0 -0.148663222789764 10.0
0 -0.212146520614624 11.0
0 -0.150040864944458 12.0
54.364116177 52.2119140625 13.0
43.894648376 40.4597511291504 14.0
0 -0.0422735214233398 15.0
0 -0.0695972442626953 16.0
33.116409708 35.9515266418457 17.0
0 0.082190990447998 18.0
0 0.0513790845870972 19.0
0 0.0612820386886597 20.0
0 0.0544747710227966 21.0
0 0.0901138782501221 22.0
36.75857316 36.1118965148926 23.0
0 0.153284549713135 24.0
0 1.36303877830505 25.0
0 0.0621247291564941 26.0
0 0.0416463613510132 27.0
0 0.40915858745575 28.0
0 -0.077052116394043 29.0
0 0.0427389144897461 30.0
0 0.0767185688018799 31.0
0 0.157376885414124 32.0
0 -0.0519787669181824 33.0
0 -0.178109526634216 34.0
0 -0.0129267573356628 35.0
0 -0.103337407112122 36.0
0 -0.0774394273757935 37.0
0 0.302057802677155 38.0
0 -0.265792369842529 39.0
0 0.222666561603546 40.0
0 -0.208229064941406 41.0
0 -0.00464129447937012 42.0
301.72402746 296.517822265625 43.0
0 0.134686827659607 44.0
0 0.1916384100914 45.0
0 0.0586748719215393 46.0
0 -0.0959246754646301 47.0
0 0.374075651168823 48.0
0 0.420500040054321 49.0
0.088404334203 0.0445476174354553 50.0
1.7721548014 1.85383820533752 51.0
0 -0.220044374465942 52.0
0 -0.0835809707641602 53.0
0 -0.0877039432525635 54.0
0 -0.0405023097991943 55.0
0 0.205091714859009 56.0
0 0.0742549896240234 57.0
0 0.204670250415802 58.0
0 0.0790700316429138 59.0
0 -0.175451278686523 60.0
0 0.0768749117851257 61.0
0 0.200137317180634 62.0
225.97606024 253.362319946289 0.0
0 -0.218596339225769 1.0
0 -0.12968897819519 2.0
0 0.0963464379310608 3.0
0 -0.42902660369873 4.0
0 -0.0299710035324097 5.0
0 0.00958716869354248 6.0
0 -0.223030209541321 7.0
0 -0.156095862388611 8.0
0 0.142975330352783 9.0
0 0.255359292030334 10.0
0 -0.418547749519348 11.0
0 -0.0870880484580994 12.0
69.658659775 68.6082611083984 13.0
13.562326857 27.7415065765381 14.0
0 -0.142536997795105 15.0
0 -0.0387251377105713 16.0
30.707251708 31.4615249633789 17.0
0 0.0934202671051025 18.0
0 0.0354211330413818 19.0
0 0.12946355342865 20.0
0 0.0997307896614075 21.0
0 0.136042356491089 22.0
38.01278503 37.4120788574219 23.0
0 0.12826943397522 24.0
0 0.627163887023926 25.0
0 0.00177168846130371 26.0
0 -0.082224428653717 27.0
0 0.103403985500336 28.0
0 -0.183782517910004 29.0
0 0.0196890830993652 30.0
0 -0.00905430316925049 31.0
0 0.154090166091919 32.0
0 -0.39645516872406 33.0
0 0.264141738414764 34.0
0 0.0699126720428467 35.0
0 -0.0148504376411438 36.0
0 -0.0369473099708557 37.0
0 0.106408834457397 38.0
0 -0.160557746887207 39.0
0 0.538708090782166 40.0
0 -0.119478821754456 41.0
0 -0.0531970262527466 42.0
294.85646142 297.9287109375 43.0
0 0.122774362564087 44.0
0 0.0286747813224792 45.0
0 0.0663946270942688 46.0
0 0.217730224132538 47.0
0 0.294219434261322 48.0
0 0.413056194782257 49.0
0.06961985294 0.0550194978713989 50.0
1.8876776666 1.83106899261475 51.0
0 -0.0554901361465454 52.0
0 -0.0414240956306458 53.0
0 0.00550854206085205 54.0
0 0.048285186290741 55.0
0 -0.108213067054749 56.0
0 0.284350037574768 57.0
0 -0.0543889999389648 58.0
0 -0.112122774124146 59.0
0 -0.0951377153396606 60.0
0 0.0496540069580078 61.0
0 0.162576496601105 62.0
268.84642055 252.042846679688 0.0
0 -0.239224076271057 1.0
0 -0.123466491699219 2.0
0 0.254115283489227 3.0
0 -0.133398115634918 4.0
0 -0.0452434420585632 5.0
0 -0.22104024887085 6.0
0 0.332865118980408 7.0
0 -0.311005353927612 8.0
0 -0.0850425958633423 9.0
0 -0.158058941364288 10.0
0 0.0587764382362366 11.0
0 -0.0455924868583679 12.0
64.665590061 66.3249130249023 13.0
36.848695474 31.0577564239502 14.0
0 -0.075230598449707 15.0
0 -0.139635384082794 16.0
33.097476726 34.7266616821289 17.0
0 0.136789679527283 18.0
0 0.0211878418922424 19.0
0 0.0868819952011108 20.0
0 0.0231758952140808 21.0
0 0.0493637919425964 22.0
39.769121878 39.406364440918 23.0
0 0.115552186965942 24.0
0 0.878574132919312 25.0
0 0.120291173458099 26.0
0 0.033230185508728 27.0
0 0.0228549242019653 28.0
0 0.000399231910705566 29.0
0 -0.177460134029388 30.0
0 0.0979106426239014 31.0
0 0.126285433769226 32.0
0 0.0681612491607666 33.0
0 0.0802909731864929 34.0
0 0.432467132806778 35.0
0 0.260267078876495 36.0
0 -0.0680379867553711 37.0
0 0.436836361885071 38.0
0 -0.319552302360535 39.0
0 -0.431218147277832 40.0
0 -0.226879477500916 41.0
0 -0.0518285036087036 42.0
293.94686565 295.872985839844 43.0
0 -0.00493407249450684 44.0
0 0.114269316196442 45.0
0 0.134995818138123 46.0
0 -0.0321260690689087 47.0
0 0.352148413658142 48.0
0 0.359493643045425 49.0
0.092022737967 0.0522781610488892 50.0
2.0683690731 1.86362624168396 51.0
0 0.00787818431854248 52.0
0 -0.0856667160987854 53.0
0 0.276542663574219 54.0
0 0.00488531589508057 55.0
0 -0.165964722633362 56.0
0 0.00441259145736694 57.0
0 0.148968994617462 58.0
0 -0.0600072145462036 59.0
0 -0.026705265045166 60.0
0 0.0795644521713257 61.0
0 0.201343595981598 62.0
225.71843141 253.882797241211 0.0
0 0.657216668128967 1.0
0 -0.00820779800415039 2.0
0 -0.0790061950683594 3.0
0 -0.28040075302124 4.0
0 -0.232723832130432 5.0
0 -0.263204097747803 6.0
0 -0.376964569091797 7.0
0 -0.329315900802612 8.0
0 0.65044504404068 9.0
0 0.368610948324203 10.0
0 -0.233799576759338 11.0
0 -0.144399523735046 12.0
77.126407252 77.6960754394531 13.0
3.8384771544 16.2254314422607 14.0
0 -0.0444886684417725 15.0
0 0.206304132938385 16.0
32.136463664 32.8851928710938 17.0
0 0.12681120634079 18.0
0 0.0124222636222839 19.0
0 0.0687140226364136 20.0
0 0.162837564945221 21.0
0 0.0936065912246704 22.0
27.570393345 28.7155532836914 23.0
0 0.0805859565734863 24.0
0 0.725730955600739 25.0
0 0.130647122859955 26.0
0 -0.215100526809692 27.0
0 -0.06159907579422 28.0
0 -0.0478782057762146 29.0
0 0.126670956611633 30.0
0 -0.0218455791473389 31.0
0 0.0564925074577332 32.0
0 0.035370945930481 33.0
0 0.0231532454490662 34.0
0 0.0389642715454102 35.0
0 -0.174496293067932 36.0
0 0.0454669594764709 37.0
0 0.0575491189956665 38.0
0 -0.370134115219116 39.0
0 0.620659947395325 40.0
0 -0.206607818603516 41.0
0 0.0929462909698486 42.0
291.06252088 297.068176269531 43.0
0 -0.0155148506164551 44.0
0 0.191441535949707 45.0
0 0.140391647815704 46.0
0 0.0670637488365173 47.0
0 0.365746885538101 48.0
0 0.261048257350922 49.0
0.067346752959 0.101280033588409 50.0
2.2143386046 1.81792378425598 51.0
0 -0.0941733121871948 52.0
0 -0.213047862052917 53.0
0 -0.0110421180725098 54.0
0 0.0751102566719055 55.0
0 0.0526474118232727 56.0
0 0.718852043151855 57.0
0 0.507134318351746 58.0
0 0.289878606796265 59.0
0 -0.0221058130264282 60.0
0 0.137829482555389 61.0
0 0.280238270759583 62.0
285.34146214 250.434829711914 0.0
0 -0.0675711631774902 1.0
0 0.106817126274109 2.0
0 0.365394532680511 3.0
0 0.054313063621521 4.0
0 0.39560404419899 5.0
0 0.0184381604194641 6.0
0 0.0872547626495361 7.0
0 -0.0814589262008667 8.0
0 -0.267422556877136 9.0
0 -0.180817604064941 10.0
0 -0.29189658164978 11.0
0 0.0640043616294861 12.0
57.839089168 57.6181907653809 13.0
25.341378056 39.4197540283203 14.0
0 -0.0461400747299194 15.0
0 -0.0871678590774536 16.0
37.987312129 31.4062957763672 17.0
0 0.0804007649421692 18.0
0 0.0324894189834595 19.0
0 0.113832294940948 20.0
0 0.259891986846924 21.0
0 -0.0837489366531372 22.0
26.450075621 25.1951274871826 23.0
0 0.0712098479270935 24.0
0 0.777295470237732 25.0
0 0.0149891376495361 26.0
0 -0.265985488891602 27.0
0 0.500192999839783 28.0
0 0.0891789197921753 29.0
0 0.109843373298645 30.0
0 -0.351544499397278 31.0
0 -0.35521411895752 32.0
0 -0.202243208885193 33.0
0 -0.264478445053101 34.0
0 0.451835215091705 35.0
0 0.090912938117981 36.0
0 0.245344042778015 37.0
0 -0.122346460819244 38.0
0 -0.438590168952942 39.0
0 0.802891850471497 40.0
0 -0.131228148937225 41.0
0 0.224409103393555 42.0
307.16220672 294.845336914062 43.0
0 0.161312937736511 44.0
0 0.154764413833618 45.0
0 0.0857373476028442 46.0
0 -0.0701445937156677 47.0
0 0.234267830848694 48.0
0.12415470289 0.399720191955566 49.0
0.083238980263 0.0763195753097534 50.0
1.8526915615 1.82764685153961 51.0
0 0.233377456665039 52.0
0 -0.108798503875732 53.0
0 -0.161622226238251 54.0
0 -0.0778114199638367 55.0
0 0.262310326099396 56.0
0 0.0547225475311279 57.0
0 0.148056626319885 58.0
0 0.261923134326935 59.0
0 -0.191392958164215 60.0
0 0.0645235776901245 61.0
0 0.0210022330284119 62.0
283.03778031 251.360641479492 0.0
0 0.0228219032287598 1.0
0 -0.0566977262496948 2.0
0 -0.133474230766296 3.0
0 0.112828493118286 4.0
0 0.470190644264221 5.0
0 0.771677136421204 6.0
0 -0.643921256065369 7.0
0 -0.154019594192505 8.0
0 0.293472826480865 9.0
0 -0.492645382881165 10.0
0 -0.0386481881141663 11.0
0 -0.0390436053276062 12.0
70.153369453 73.3915328979492 13.0
44.15327627 31.5576057434082 14.0
0 0.0296505689620972 15.0
0 -0.182458877563477 16.0
25.783204519 31.9852466583252 17.0
0 0.0464605093002319 18.0
0 -0.0164060592651367 19.0
0 0.122244536876678 20.0
0 -0.19886839389801 21.0
0 1.00842893123627 22.0
40.269832152 37.7247314453125 23.0
0 0.0367040634155273 24.0
0 0.19331693649292 25.0
0 -0.123859047889709 26.0
0 -0.0347672700881958 27.0
0 -0.0946072936058044 28.0
0 0.205471515655518 29.0
0 0.0292956829071045 30.0
0 0.296057343482971 31.0
0 0.0855998396873474 32.0
0 0.244377315044403 33.0
0 -0.128562450408936 34.0
0 0.421860933303833 35.0
0 0.0839434266090393 36.0
0 -0.0111057162284851 37.0
0 0.153281331062317 38.0
0 -0.236393094062805 39.0
0 0.321943402290344 40.0
0 -0.0296151638031006 41.0
0 -0.114289522171021 42.0
302.0113475 297.591430664062 43.0
0 -0.00457239151000977 44.0
0 0.195337116718292 45.0
0 0.189282715320587 46.0
0 -0.466558456420898 47.0
0 0.120174705982208 48.0
0 0.426482915878296 49.0
0.082153014496 0.0565168261528015 50.0
1.5016404248 1.86292946338654 51.0
0 0.401505798101425 52.0
0 0.0262634754180908 53.0
0 0.192346572875977 54.0
0 0.0908345580101013 55.0
0 -0.0669873952865601 56.0
0 -0.082335352897644 57.0
0 0.0751979351043701 58.0
0 0.100333869457245 59.0
0 -0.194688677787781 60.0
0 -0.0387930274009705 61.0
0 0.00137549638748169 62.0
252.94528798 253.326583862305 0.0
0 0.24474823474884 1.0
0 0.0433412790298462 2.0
0 0.191782712936401 3.0
0 -0.303946375846863 4.0
0 -0.0966504216194153 5.0
0 0.0862656235694885 6.0
0 -0.0756930112838745 7.0
0 0.0618098378181458 8.0
0 -0.062978982925415 9.0
0 0.00762879848480225 10.0
0 -0.089070737361908 11.0
0 0.0290223956108093 12.0
56.862684157 56.7610397338867 13.0
39.05627099 37.7995185852051 14.0
0 -0.102975726127625 15.0
0 0.20741868019104 16.0
37.590417237 37.3119468688965 17.0
0 0.207682967185974 18.0
0 -0.0377233624458313 19.0
0 0.0521566867828369 20.0
0 0.213760912418365 21.0
0 -0.0679526925086975 22.0
28.352805298 27.137077331543 23.0
0 0.0798853039741516 24.0
0 0.617127537727356 25.0
0 -0.171150863170624 26.0
0 -0.134969592094421 27.0
0 0.32611146569252 28.0
0 0.0345454812049866 29.0
0 -0.0053788423538208 30.0
0 0.137608766555786 31.0
0 0.0740429162979126 32.0
0 -0.0598145127296448 33.0
0 -0.1603844165802 34.0
0 0.350329637527466 35.0
0 -0.278910875320435 36.0
0 0.152601838111877 37.0
0 -0.106012463569641 38.0
0 -0.318167924880981 39.0
0 0.529573500156403 40.0
0 -0.133307754993439 41.0
0 0.0966035127639771 42.0
296.85403669 297.493225097656 43.0
0 0.16639244556427 44.0
0 0.0114437341690063 45.0
0 0.205056011676788 46.0
0 -0.020244300365448 47.0
0 0.268732011318207 48.0
1.919601633 0.526386976242065 49.0
0.073686326159 0.0572952032089233 50.0
2.1239768262 1.8833692073822 51.0
0 0.0563825368881226 52.0
0 -0.156348884105682 53.0
0 0.119250893592834 54.0
0 0.197514355182648 55.0
0 -0.120212852954865 56.0
0 0.118600308895111 57.0
0 0.40272656083107 58.0
0 0.196272432804108 59.0
0 -0.279843926429749 60.0
0 0.0716971755027771 61.0
0 0.260706901550293 62.0
244.38117906 251.201614379883 0.0
0 0.189602136611938 1.0
0 -0.0106397867202759 2.0
0 0.0689654946327209 3.0
0 -0.289119243621826 4.0
0 -0.0305059552192688 5.0
0 0.0438643097877502 6.0
0 -0.0253744721412659 7.0
0 -0.082547664642334 8.0
0 -0.131350159645081 9.0
0 -0.163936495780945 10.0
0 -0.000352680683135986 11.0
0 -0.0880467295646667 12.0
58.817908408 59.9213371276855 13.0
46.722940887 37.7995529174805 14.0
0 -0.0558509826660156 15.0
0 -0.00927776098251343 16.0
30.234262387 34.5353126525879 17.0
0 0.0554505586624146 18.0
0 -0.0366238355636597 19.0
0 0.098855197429657 20.0
0 0.101499319076538 21.0
0 0.174366533756256 22.0
30.874178325 28.5872745513916 23.0
0 0.0789775252342224 24.0
0 0.865626692771912 25.0
0 -0.00646072626113892 26.0
0 -0.0936390161514282 27.0
0 0.540846109390259 28.0
0 -0.114933431148529 29.0
0 0.0744515657424927 30.0
0 -0.0780009031295776 31.0
0 -0.118726968765259 32.0
0 -0.288877725601196 33.0
0 -0.0646398067474365 34.0
0 0.256761789321899 35.0
0 0.520134210586548 36.0
0 0.0733450055122375 37.0
0 0.0227623581886292 38.0
0 -0.2022944688797 39.0
0 0.436737328767776 40.0
0 -0.126589119434357 41.0
0 -0.0420851707458496 42.0
308.17725645 297.66064453125 43.0
0 0.0471948385238647 44.0
0 0.0220649838447571 45.0
0 0.106304824352264 46.0
0 0.200315058231354 47.0
0 0.225288510322571 48.0
0 0.472362399101257 49.0
0.099608544695 0.00810754299163818 50.0
1.9216247472 1.8788183927536 51.0
0 0.0349071025848389 52.0
0 0.0657228231430054 53.0
0 0.166027903556824 54.0
0 0.326064944267273 55.0
0 0.0679121613502502 56.0
0 0.457692325115204 57.0
0 0.202224493026733 58.0
0 0.335219591856003 59.0
0 -0.214387655258179 60.0
0 0.103275060653687 61.0
0 0.378908693790436 62.0
266.66609717 253.034912109375 0.0
0 0.287986278533936 1.0
0 -0.0196131467819214 2.0
0 0.0708162784576416 3.0
0 0.223791122436523 4.0
0 0.242107093334198 5.0
0 0.318507075309753 6.0
0 0.241020858287811 7.0
0 -0.677376985549927 8.0
0 0.067059338092804 9.0
0 0.0566074252128601 10.0
0 -0.392444252967834 11.0
0 -0.0120142698287964 12.0
65.383447241 64.8894500732422 13.0
22.552843657 34.9595680236816 14.0
0 -0.0577800273895264 15.0
0 0.0461638569831848 16.0
31.69747893 31.0083866119385 17.0
0 0.0717553496360779 18.0
0 -0.00570595264434814 19.0
0 0.228737473487854 20.0
0 0.0329282283782959 21.0
0 0.129217267036438 22.0
38.382268246 37.5237579345703 23.0
0 0.0683656334877014 24.0
0 0.507577061653137 25.0
0 -0.129331350326538 26.0
0 0.108209669589996 27.0
0 0.0381919741630554 28.0
0 0.00913304090499878 29.0
0 0.0601488351821899 30.0
0 -0.0174736976623535 31.0
0 -0.0323173403739929 32.0
0 -0.121273398399353 33.0
0 -0.0775682926177979 34.0
0 0.226508617401123 35.0
0 0.22512024641037 36.0
0 -0.0118280649185181 37.0
0 0.242063879966736 38.0
0 -0.162765264511108 39.0
0 0.30517253279686 40.0
0 -0.161019742488861 41.0
0 0.0391159057617188 42.0
302.25395628 297.645111083984 43.0
0 0.142925500869751 44.0
0 0.0877970457077026 45.0
0 0.181679904460907 46.0
0 -0.243651151657104 47.0
0 0.387382060289383 48.0
0 0.439356446266174 49.0
0.088534935379 0.0461716055870056 50.0
1.8435555144 1.89233314990997 51.0
0 0.212601959705353 52.0
0 -0.186616659164429 53.0
0 0.161275207996368 54.0
0 0.223625898361206 55.0
0 0.0776916742324829 56.0
0 0.346941411495209 57.0
0 0.135267376899719 58.0
0 0.0701702833175659 59.0
0 0.0706186890602112 60.0
0 0.0335018634796143 61.0
0 0.129850804805756 62.0
271.79683126 251.33544921875 0.0
0 -0.255114436149597 1.0
0 -0.0806000232696533 2.0
0 0.000764906406402588 3.0
0 -0.0052715539932251 4.0
0 0.482975572347641 5.0
0 -0.271453619003296 6.0
0 0.435152471065521 7.0
0 0.137726664543152 8.0
0 0.138950884342194 9.0
0 0.13991117477417 10.0
0 -0.189026296138763 11.0
0 -0.0375593900680542 12.0
64.138640847 65.770751953125 13.0
33.267947848 31.5625228881836 14.0
0 0.0345364809036255 15.0
0 0.0308237671852112 16.0
30.838204555 32.6964797973633 17.0
0 0.094272255897522 18.0
0 0.00608372688293457 19.0
0 0.130730390548706 20.0
0 -0.00703263282775879 21.0
0 -0.0247666239738464 22.0
40.875194813 41.0710334777832 23.0
0 0.160174489021301 24.0
0 0.819377541542053 25.0
0 0.00621646642684937 26.0
0 0.103872835636139 27.0
0 0.162222504615784 28.0
0 -0.054021418094635 29.0
0 0.01884925365448 30.0
0 0.105445981025696 31.0
0 0.0888894200325012 32.0
0 -0.0232794880867004 33.0
0 -0.021389365196228 34.0
0 -0.143172979354858 35.0
0 -0.0674313902854919 36.0
0 0.00560951232910156 37.0
0 0.243381559848785 38.0
0 -0.22074294090271 39.0
0 0.328712224960327 40.0
0 -0.143281400203705 41.0
0 0.0927469730377197 42.0
288.1242724 294.725708007812 43.0
0 0.108202338218689 44.0
0 -0.016379177570343 45.0
0 0.255468964576721 46.0
0 -0.0297921895980835 47.0
0 0.276884257793427 48.0
0 0.405828356742859 49.0
0.095563565059 0.0247384309768677 50.0
2.1691385168 1.9105019569397 51.0
0 -0.121538877487183 52.0
0 0.221331000328064 53.0
0 -0.0117883682250977 54.0
0 0.0929208397865295 55.0
0 0.347257852554321 56.0
0 0.0673469305038452 57.0
0 0.0972822904586792 58.0
0 0.260131478309631 59.0
0 0.00680595636367798 60.0
0 0.0678766369819641 61.0
0 0.190569043159485 62.0
275.63056689 251.611312866211 0.0
0 -0.709803938865662 1.0
0 0.134400248527527 2.0
0 0.0679775476455688 3.0
0 0.0700423121452332 4.0
0 -0.326281070709229 5.0
0 0.343768715858459 6.0
0 0.37096756696701 7.0
0 0.189897298812866 8.0
0 0.331728518009186 9.0
0 -0.486159801483154 10.0
0 -0.254655241966248 11.0
0 -0.283068776130676 12.0
48.92124188 42.2118148803711 13.0
33.337069839 36.1467742919922 14.0
0 -0.32088029384613 15.0
0 0.32983011007309 16.0
24.855084947 33.1911849975586 17.0
0 -0.00537240505218506 18.0
0 -0.0304150581359863 19.0
0 0.155298829078674 20.0
0 -0.141433238983154 21.0
0 1.5241858959198 22.0
43.283294724 39.6690902709961 23.0
0 0.115714430809021 24.0
0 1.06061089038849 25.0
0 0.0614574551582336 26.0
0 -0.0535761713981628 27.0
0 0.789040982723236 28.0
0 -0.0512065887451172 29.0
0 -0.0702130198478699 30.0
0 0.204301238059998 31.0
0 -0.0248103141784668 32.0
0 -0.356584191322327 33.0
0 0.257063388824463 34.0
0 0.0333384275436401 35.0
0 0.381658226251602 36.0
0 0.227234780788422 37.0
0 -0.104165434837341 38.0
0 -0.183810114860535 39.0
0 0.491323828697205 40.0
0 -0.106828093528748 41.0
0 0.194236278533936 42.0
312.24680318 296.425445556641 43.0
0 0.189080238342285 44.0
0 0.038034200668335 45.0
0 0.145484685897827 46.0
0 -0.267008304595947 47.0
0 -0.135188043117523 48.0
0 0.282245397567749 49.0
0.099569927761 -0.0222981572151184 50.0
1.9400882412 1.87196755409241 51.0
0 -0.108044385910034 52.0
0 -0.0293310880661011 53.0
0 -0.125444412231445 54.0
0 -0.135514855384827 55.0
0 -0.118495941162109 56.0
0 0.0140199661254883 57.0
0 0.273035407066345 58.0
0 0.174303770065308 59.0
0 -0.145772814750671 60.0
0 0.147920250892639 61.0
0 0.36017918586731 62.0
276.42782907 252.715881347656 0.0
0 -0.147544026374817 1.0
0 0.0780555009841919 2.0
0 0.0856573581695557 3.0
0 -0.124618947505951 4.0
0 0.386891901493073 5.0
0 0.37552797794342 6.0
0 0.31579202413559 7.0
0 0.0283637046813965 8.0
0 0.0711060166358948 9.0
0 0.277554392814636 10.0
0 -0.0396015644073486 11.0
0 -0.0758556127548218 12.0
73.507077504 70.9045257568359 13.0
22.789538882 25.3341999053955 14.0
0 -0.0937033891677856 15.0
0 0.109153807163239 16.0
41.661853572 38.2468490600586 17.0
0 0.17158842086792 18.0
0 -0.0858874320983887 19.0
0 0.0873696804046631 20.0
0 -0.0145083665847778 21.0
0 0.288926005363464 22.0
32.744999983 31.1728267669678 23.0
0 0.0161656141281128 24.0
0 0.057709813117981 25.0
0 0.0474557876586914 26.0
0 -0.217962026596069 27.0
0 0.696081280708313 28.0
0 0.0555995106697083 29.0
0 0.25028121471405 30.0
0 -0.157204747200012 31.0
0 -0.160000622272491 32.0
0 -0.262023210525513 33.0
0 0.244539499282837 34.0
0 0.297982424497604 35.0
0 0.0955488681793213 36.0
0 0.0267847776412964 37.0
0 0.0137509107589722 38.0
0 -0.250478744506836 39.0
0 0.299537181854248 40.0
0 -0.0916492342948914 41.0
0 0.0120618343353271 42.0
309.86963318 297.231140136719 43.0
0 -0.0425729751586914 44.0
0 0.204036355018616 45.0
0 0.086018443107605 46.0
0 0.0147842168807983 47.0
0 -0.519430756568909 48.0
0 0.407097935676575 49.0
0.097215947996 -0.0353012681007385 50.0
1.5379566443 1.87630367279053 51.0
0 -0.0385134220123291 52.0
0 0.126123487949371 53.0
0 -0.118045330047607 54.0
0 -0.214653372764587 55.0
0 0.156869292259216 56.0
0 -0.149380326271057 57.0
0 0.0964685678482056 58.0
0 0.262655854225159 59.0
0 -0.0363301634788513 60.0
0 0.0547106266021729 61.0
0 0.0887141227722168 62.0
239.56547112 253.050476074219 0.0
0 -0.153766751289368 1.0
0 -0.0879594087600708 2.0
0 0.236106872558594 3.0
0 -0.291599988937378 4.0
0 -0.264263272285461 5.0
0 -0.093139111995697 6.0
0 -0.11902517080307 7.0
0 -0.266132593154907 8.0
0 0.298205733299255 9.0
0 0.161035120487213 10.0
0 -0.267519354820251 11.0
0 0.0306679606437683 12.0
48.949389317 46.9374389648438 13.0
39.698927028 50.0487442016602 14.0
0 -0.0900974273681641 15.0
0 0.00985032320022583 16.0
31.294059177 35.6881103515625 17.0
0 0.129996478557587 18.0
0 0.0561389327049255 19.0
0 0.00408190488815308 20.0
0 0.337029427289963 21.0
0 -0.123367667198181 22.0
25.58027467 25.4099826812744 23.0
0 0.117676854133606 24.0
0 0.686196327209473 25.0
0 -0.0788799524307251 26.0
0 -0.207769155502319 27.0
0 0.114887118339539 28.0
0 -0.04975426197052 29.0
0 0.047178328037262 30.0
0 -0.130661845207214 31.0
0 -0.0158082246780396 32.0
0 -0.187569141387939 33.0
0 0.324317783117294 34.0
0 0.17994898557663 35.0
0 0.141337215900421 36.0
0 0.133212864398956 37.0
0 -0.0220422744750977 38.0
0 -0.143848896026611 39.0
0 0.0117344856262207 40.0
0 -0.06432044506073 41.0
0 0.17405116558075 42.0
301.28417734 295.663391113281 43.0
0 0.132457494735718 44.0
0 -0.0395380258560181 45.0
0 0.140715420246124 46.0
0 -0.304124355316162 47.0
0 0.102972745895386 48.0
0 0.512792408466339 49.0
0.070410603483 0.0581687688827515 50.0
1.5471955211 1.80844807624817 51.0
0 0.458282917737961 52.0
0 -0.00877273082733154 53.0
0 -0.0675687789916992 54.0
0 0.0843778848648071 55.0
0 -0.0969909429550171 56.0
0 0.362413883209229 57.0
0 0.492873728275299 58.0
0 0.288648843765259 59.0
0 -0.269243359565735 60.0
0 0.0502411723136902 61.0
0 0.187455654144287 62.0
280.00719244 252.061065673828 0.0
0 0.107359290122986 1.0
0 0.00135141611099243 2.0
0 0.130292057991028 3.0
0 0.127469420433044 4.0
0 0.269996285438538 5.0
0 0.0696550011634827 6.0
0 0.399695336818695 7.0
0 -0.202918350696564 8.0
0 -0.0753134489059448 9.0
0 0.13884162902832 10.0
0 -0.407242059707642 11.0
0 -0.0528501272201538 12.0
53.790237666 54.1682586669922 13.0
31.410668941 40.8643798828125 14.0
0 -0.0911962985992432 15.0
0 0.0751293301582336 16.0
33.25132231 33.233268737793 17.0
0 0.113394737243652 18.0
0 -0.0313324928283691 19.0
0 0.141004085540771 20.0
0 0.0513911247253418 21.0
0 0.180049419403076 22.0
32.638287812 30.3742008209229 23.0
0 -0.0515509843826294 24.0
0 0.31660196185112 25.0
0 0.0488389730453491 26.0
0 -0.169794499874115 27.0
0 0.0348041653633118 28.0
0 -0.0355736613273621 29.0
0 0.114119052886963 30.0
0 0.0708252191543579 31.0
0 0.0607320070266724 32.0
0 -0.236951470375061 33.0
0 0.288292586803436 34.0
0 0.0179053544998169 35.0
0 0.293474584817886 36.0
0 -0.0180067420005798 37.0
0 0.255820393562317 38.0
0 -0.238399982452393 39.0
0 -0.102077782154083 40.0
0 -0.157257616519928 41.0
0 -0.0297515392303467 42.0
290.34130433 296.054748535156 43.0
0 0.175207138061523 44.0
0 -0.074647068977356 45.0
0 0.201417326927185 46.0
0 0.0521916747093201 47.0
0 0.199167847633362 48.0
0 0.57785701751709 49.0
0.089386096543 0.00423002243041992 50.0
1.6631265882 1.8903980255127 51.0
0 -0.0332245230674744 52.0
0 0.0865845680236816 53.0
0 0.20680558681488 54.0
0 0.123817920684814 55.0
0 0.0327046513557434 56.0
0 0.0385775566101074 57.0
0 0.206330299377441 58.0
0 0.063859760761261 59.0
0 -0.102222204208374 60.0
0 0.0125141143798828 61.0
0 0.130425572395325 62.0
223.9914769 253.798355102539 0.0
0 -0.0872420072555542 1.0
0 -0.151694416999817 2.0
0 0.251517117023468 3.0
0 -0.445425987243652 4.0
0 -0.650398850440979 5.0
0 -1.04295635223389 6.0
0 0.236073851585388 7.0
0 -0.17936247587204 8.0
0 -0.815640568733215 9.0
0 0.0781669020652771 10.0
0 -0.573260426521301 11.0
0 -0.0713534951210022 12.0
77.911143784 76.5072708129883 13.0
5.7892204574 20.7691192626953 14.0
0 -0.0152062177658081 15.0
0 -0.00735199451446533 16.0
37.335076731 37.4613838195801 17.0
0 0.118054866790771 18.0
0 -0.0254092216491699 19.0
0 0.121014297008514 20.0
0 0.0429756641387939 21.0
0 0.293755292892456 22.0
33.287056097 32.7143020629883 23.0
0 -0.0199451446533203 24.0
0 0.290739953517914 25.0
0 0.240352988243103 26.0
0 -0.00394254922866821 27.0
0 0.660803854465485 28.0
0 0.0665992498397827 29.0
0 0.0191678404808044 30.0
0 0.649162948131561 31.0
0 -0.243974089622498 32.0
0 0.137864470481873 33.0
0 0.263894617557526 34.0
0 0.0882769227027893 35.0
0 -0.341568946838379 36.0
0 0.167773008346558 37.0
0 0.441262483596802 38.0
0 -0.374873161315918 39.0
0 -0.413460254669189 40.0
0 -0.282322645187378 41.0
0 0.684545397758484 42.0
303.74530842 292.414794921875 43.0
0 -0.103849351406097 44.0
0 0.524361312389374 45.0
0 0.00205826759338379 46.0
0 -0.124726474285126 47.0
0 0.306892365217209 48.0
0.51427872687 0.497842192649841 49.0
0.082703252759 0.0904427766799927 50.0
2.0670842045 1.85181820392609 51.0
0 0.17878782749176 52.0
0 -0.0842965841293335 53.0
0 -0.205275475978851 54.0
0 0.0258593559265137 55.0
0 0.317969620227814 56.0
0 0.168563902378082 57.0
0 0.0714177489280701 58.0
0 0.655053734779358 59.0
0 -0.0275303721427917 60.0
0 0.0582958459854126 61.0
0 0.0803616046905518 62.0
257.47549706 251.430389404297 0.0
0 -0.101043343544006 1.0
0 -0.0948195457458496 2.0
0 0.36936542391777 3.0
0 -0.0188698768615723 4.0
0 -0.144146859645844 5.0
0 -0.138362407684326 6.0
0 0.0200687050819397 7.0
0 -0.0609563589096069 8.0
0 0.635122716426849 9.0
0 -0.733160257339478 10.0
0 -0.00148278474807739 11.0
0 -0.0157074332237244 12.0
70.792350397 72.1030731201172 13.0
37.630556774 27.9575748443604 14.0
0 -0.0470073223114014 15.0
0 0.455535709857941 16.0
39.604461073 35.681468963623 17.0
0 0.150324106216431 18.0
0 -0.111738085746765 19.0
0 0.0688412189483643 20.0
0 0.12461644411087 21.0
0 0.123605847358704 22.0
27.41008405 24.5094470977783 23.0
0 0.00939303636550903 24.0
0 -0.0780277252197266 25.0
0 0.0067669153213501 26.0
0 -0.175568342208862 27.0
0 0.957864046096802 28.0
0 0.0147092342376709 29.0
0 0.0958669781684875 30.0
0 -0.0721238851547241 31.0
0 0.0828706622123718 32.0
0 0.0495818853378296 33.0
0 0.125956416130066 34.0
0 0.253433465957642 35.0
0 0.263876438140869 36.0
0 -0.0596409440040588 37.0
0 0.196895480155945 38.0
0 -0.216645956039429 39.0
0 0.584516048431396 40.0
0 -0.125857293605804 41.0
0 0.0697369575500488 42.0
293.10661013 296.566711425781 43.0
0 0.133016109466553 44.0
0 0.112805664539337 45.0
0 -0.0489944815635681 46.0
0 0.0953453183174133 47.0
0 0.244143962860107 48.0
2.4965545754 0.495546102523804 49.0
0.069006927485 0.0705838203430176 50.0
1.6443350752 1.87767291069031 51.0
0 0.309466391801834 52.0
0 -0.255923986434937 53.0
0 0.241583228111267 54.0
0 0.221754193305969 55.0
0 0.155553042888641 56.0
0 0.266578078269958 57.0
0 0.317288607358932 58.0
0 0.0181667804718018 59.0
0 0.139855921268463 60.0
0 0.00649982690811157 61.0
0 0.164126932621002 62.0
264.94160441 254.102584838867 0.0
0 -0.269199371337891 1.0
0 -0.213886380195618 2.0
0 0.120260715484619 3.0
0 -0.219642639160156 4.0
0 0.203938245773315 5.0
0 0.0904496908187866 6.0
0 0.160209178924561 7.0
0 -0.0479645729064941 8.0
0 0.0115768313407898 9.0
0 -0.315397620201111 10.0
0 -0.0612208843231201 11.0
0 0.0721507668495178 12.0
53.983733149 52.9965438842773 13.0
48.837777674 44.5720291137695 14.0
0 -0.0391287803649902 15.0
0 0.183695614337921 16.0
34.295588413 36.6636772155762 17.0
0 0.107402205467224 18.0
0 -0.00813394784927368 19.0
0 0.0303357839584351 20.0
0 0.122904181480408 21.0
0 0.159035384654999 22.0
30.130073022 29.7386207580566 23.0
0 0.113552629947662 24.0
0 1.06730878353119 25.0
0 -0.0539576411247253 26.0
0 -0.0572348237037659 27.0
0 -0.0356314778327942 28.0
0 0.00282418727874756 29.0
0 0.0502009987831116 30.0
0 -0.0411350727081299 31.0
0 0.126440763473511 32.0
0 -0.185716927051544 33.0
0 0.0648058652877808 34.0
0 0.232403635978699 35.0
0 0.45504167675972 36.0
0 -0.0573630928993225 37.0
0 0.0662385225296021 38.0
0 -0.241362571716309 39.0
0 0.537203669548035 40.0
0 -0.0693825483322144 41.0
0 -0.00212764739990234 42.0
296.98357959 297.573760986328 43.0
0 0.092285692691803 44.0
0 0.0869452953338623 45.0
0 0.244195401668549 46.0
0 0.060892641544342 47.0
0 -0.205630540847778 48.0
0 0.285616636276245 49.0
0.084731397219 -0.0145291686058044 50.0
1.9770899799 1.89383125305176 51.0
0 -0.047527015209198 52.0
0 -0.110690712928772 53.0
0 0.0361034870147705 54.0
0 0.273549318313599 55.0
0 -0.0694764256477356 56.0
0 0.125954151153564 57.0
0 0.112537145614624 58.0
0 0.354925900697708 59.0
0 0.0995771884918213 60.0
0 0.0809699296951294 61.0
0 0.205480694770813 62.0
234.76436467 253.441055297852 0.0
0 0.121005892753601 1.0
0 -0.0602946281433105 2.0
0 -0.106285035610199 3.0
0 -0.236403942108154 4.0
0 -0.266645908355713 5.0
0 0.0986161231994629 6.0
0 0.0612297058105469 7.0
0 0.189887642860413 8.0
0 -0.234146356582642 9.0
0 0.307080268859863 10.0
0 -0.0540705919265747 11.0
0 -0.116662740707397 12.0
48.440328131 48.7622032165527 13.0
52.436244207 44.9552307128906 14.0
0 0.125651955604553 15.0
0 0.00897693634033203 16.0
37.514577482 36.6001625061035 17.0
0 0.157044529914856 18.0
0 -0.0030483603477478 19.0
0 0.0602515935897827 20.0
0 -0.0264897346496582 21.0
0 0.302893757820129 22.0
37.230078733 34.9961853027344 23.0
0 0.0967022776603699 24.0
0 0.801776766777039 25.0
0 0.086369514465332 26.0
0 -0.0684962868690491 27.0
0 0.235150396823883 28.0
0 -0.212746500968933 29.0
0 -0.0429509878158569 30.0
0 0.443768739700317 31.0
0 0.0956109166145325 32.0
0 -0.246272802352905 33.0
0 -0.111962199211121 34.0
0 0.303613126277924 35.0
0 0.114747881889343 36.0
0 0.136194467544556 37.0
0 -0.632722973823547 38.0
0 -0.153739213943481 39.0
0 0.243107616901398 40.0
0 -0.250669240951538 41.0
0 0.0747053623199463 42.0
299.91259139 294.520416259766 43.0
0 0.138591468334198 44.0
0 0.0283917784690857 45.0
0 0.284349620342255 46.0
0 0.129113137722015 47.0
0 0.332784384489059 48.0
1.2566849593 0.553195893764496 49.0
0.080848219987 0.0773409605026245 50.0
1.6261997037 1.88605535030365 51.0
0 -0.083773672580719 52.0
0 0.0235832333564758 53.0
0 0.0875961184501648 54.0
0 0.161415159702301 55.0
0 0.289168238639832 56.0
0 -0.0897659063339233 57.0
0 -0.014197051525116 58.0
0 0.204273283481598 59.0
0 -0.258430004119873 60.0
0 0.18121063709259 61.0
0 0.408566415309906 62.0
268.2770799 249.998764038086 0.0
0 -0.324796438217163 1.0
0 0.0196687579154968 2.0
0 0.455514132976532 3.0
0 0.036713182926178 4.0
0 0.0439087152481079 5.0
0 -0.277220606803894 6.0
0 0.390480548143387 7.0
0 -0.098496675491333 8.0
0 0.0791051387786865 9.0
0 -0.02707439661026 10.0
0 -0.0525646209716797 11.0
0 -0.0523120164871216 12.0
63.496752842 63.0455131530762 13.0
40.536821234 35.9669303894043 14.0
0 0.0329123735427856 15.0
0 -0.00496006011962891 16.0
39.234571726 35.7491683959961 17.0
0 0.100940644741058 18.0
0 -0.0101276636123657 19.0
0 0.083061695098877 20.0
0 0.0754711627960205 21.0
0 0.209680259227753 22.0
33.091375035 32.5543403625488 23.0
0 0.0807782411575317 24.0
0 1.20642638206482 25.0
0 -0.0309484004974365 26.0
0 -0.151035130023956 27.0
0 0.157606184482574 28.0
0 -0.00595402717590332 29.0
0 0.0712753534317017 30.0
0 0.0471550226211548 31.0
0 0.0791206359863281 32.0
0 0.129458069801331 33.0
0 0.0115395784378052 34.0
0 -0.0477635264396667 35.0
0 0.0379148721694946 36.0
0 -0.0780268907546997 37.0
0 0.225317239761353 38.0
0 -0.347308874130249 39.0
0 0.307405591011047 40.0
0 -0.20392245054245 41.0
0 0.0867160558700562 42.0
283.01646596 294.802673339844 43.0
0 0.101815819740295 44.0
0 -0.00510710477828979 45.0
0 0.156321704387665 46.0
0 0.0253288149833679 47.0
0 -0.0027042031288147 48.0
0 0.443877279758453 49.0
0.091112931927 -0.0158585906028748 50.0
1.9567162238 1.86161065101624 51.0
0 0.135315656661987 52.0
0 -0.171036541461945 53.0
0 0.219299256801605 54.0
0 -0.0782208442687988 55.0
0 -0.0890266299247742 56.0
0 0.0324589014053345 57.0
0 0.0664602518081665 58.0
0 0.00410604476928711 59.0
0 -0.11209362745285 60.0
0 0.073596179485321 61.0
0 0.080656111240387 62.0
281.00113385 254.751754760742 0.0
0 -0.305575847625732 1.0
0 -0.00673055648803711 2.0
0 0.192800104618073 3.0
0 -0.179251074790955 4.0
0 -0.0293360948562622 5.0
0 -0.409213662147522 6.0
0 0.248093485832214 7.0
0 0.300076007843018 8.0
0 0.0382068157196045 9.0
0 0.315271466970444 10.0
0 -0.262330532073975 11.0
0 -0.063620924949646 12.0
74.753302787 77.3922882080078 13.0
24.489793822 20.9456062316895 14.0
0 -0.0207198858261108 15.0
0 -0.115860283374786 16.0
32.552392045 33.9463272094727 17.0
0 0.0858232975006104 18.0
0 -0.0334628820419312 19.0
0 0.126621842384338 20.0
0 0.036644458770752 21.0
0 0.130743145942688 22.0
38.698858006 37.6262016296387 23.0
0 0.0717726349830627 24.0
0 0.571428596973419 25.0
0 -0.119035661220551 26.0
0 -0.035269558429718 27.0
0 0.395096242427826 28.0
0 0.0135570764541626 29.0
0 -0.017253041267395 30.0
0 0.0975302457809448 31.0
0 0.133634805679321 32.0
0 0.0170391201972961 33.0
0 0.294563382863998 34.0
0 0.210515439510345 35.0
0 0.553576290607452 36.0
0 -0.0298476815223694 37.0
0 -0.031079888343811 38.0
0 -0.0593705177307129 39.0
0 -0.0448698401451111 40.0
0 -0.0770304203033447 41.0
0 0.210723757743835 42.0
303.23468453 294.640991210938 43.0
0 0.0120948553085327 44.0
0 0.0485004782676697 45.0
0 0.287028014659882 46.0
0 -0.40799355506897 47.0
0 0.0496495366096497 48.0
0 0.454161614179611 49.0
0.092712506686 -0.0155832767486572 50.0
1.5301851507 1.86868381500244 51.0
0 -0.0264642834663391 52.0
0 -0.557355761528015 53.0
0 0.0743384957313538 54.0
0 0.0646170377731323 55.0
0 -0.043817400932312 56.0
0 0.00547140836715698 57.0
0 -0.0378732681274414 58.0
0 0.0601584911346436 59.0
0 -0.100348651409149 60.0
0 0.0388140678405762 61.0
0 0.138189852237701 62.0
291.04612406 255.259475708008 0.0
0 0.0530774593353271 1.0
0 0.0587597489356995 2.0
0 0.0801947116851807 3.0
0 0.112993240356445 4.0
0 0.159548759460449 5.0
0 0.705188155174255 6.0
0 0.337502062320709 7.0
0 0.121078372001648 8.0
0 -0.0120319128036499 9.0
0 0.575065672397614 10.0
0 -0.377073049545288 11.0
0 0.0852004289627075 12.0
50.210531035 52.317756652832 13.0
36.662032654 38.5206069946289 14.0
0 0.327972769737244 15.0
0 0.159588158130646 16.0
33.256550229 35.6923370361328 17.0
0 0.200570702552795 18.0
0 0.0475249886512756 19.0
0 -0.013796329498291 20.0
0 0.199556112289429 21.0
0 -0.0395956039428711 22.0
30.045406534 28.41042137146 23.0
0 0.0897451043128967 24.0
0 0.726789712905884 25.0
0 -0.129368782043457 26.0
0 -0.213773012161255 27.0
0 0.791159331798553 28.0
0 0.0266914963722229 29.0
0 0.0417177677154541 30.0
0 -0.0788235664367676 31.0
0 0.0145487785339355 32.0
0 -0.00597476959228516 33.0
0 0.33255672454834 34.0
0 -0.0520568490028381 35.0
0 0.127708792686462 36.0
0 0.0710036754608154 37.0
0 -0.0377351641654968 38.0
0 -0.024311900138855 39.0
0 0.385092735290527 40.0
0 -0.0330036878585815 41.0
0 0.114060282707214 42.0
305.51659408 296.718078613281 43.0
0 0.158248424530029 44.0
0 0.102197229862213 45.0
0 0.215446591377258 46.0
0 0.0885635614395142 47.0
0 0.300469636917114 48.0
0 0.27457195520401 49.0
0.10127332003 0.0749142169952393 50.0
1.5492713777 1.81878471374512 51.0
0 0.125295042991638 52.0
0 0.237013101577759 53.0
0 0.28536581993103 54.0
0 -0.0724796652793884 55.0
0 -0.0999542474746704 56.0
0 0.153965175151825 57.0
0 0.0491093993186951 58.0
0 -0.406399965286255 59.0
0 1.41605722904205 60.0
0 -0.151535093784332 61.0
0 -0.29421854019165 62.0
251.11943256 251.605422973633 0.0
0 -0.172454237937927 1.0
0 -0.145298421382904 2.0
0 0.281445860862732 3.0
0 -0.177576720714569 4.0
0 -0.223962545394897 5.0
0 -0.135245025157928 6.0
0 0.296879917383194 7.0
0 -0.360316634178162 8.0
0 0.0577990412712097 9.0
0 -0.399896740913391 10.0
0 0.0644062161445618 11.0
0 -0.120367586612701 12.0
57.3738703 57.0358047485352 13.0
51.459239401 42.526782989502 14.0
0 0.100573897361755 15.0
0 0.00737857818603516 16.0
38.996225241 36.4793701171875 17.0
0 0.151287257671356 18.0
0 -0.0524667501449585 19.0
0 0.127485871315002 20.0
0 0.0333791971206665 21.0
0 0.1938516497612 22.0
34.594830323 33.9609451293945 23.0
0 -0.0288349986076355 24.0
0 0.183399021625519 25.0
0 -0.109409272670746 26.0
0 -0.0403488278388977 27.0
0 0.139401912689209 28.0
0 -0.0308437943458557 29.0
0 0.0777266025543213 30.0
0 0.0996671915054321 31.0
0 0.0381220579147339 32.0
0 0.0941022634506226 33.0
0 0.00903224945068359 34.0
0 0.0382105112075806 35.0
0 0.356285750865936 36.0
0 -0.0558202862739563 37.0
0 0.282684326171875 38.0
0 -0.144476532936096 39.0
0 -0.00679004192352295 40.0
0 -0.174483180046082 41.0
0 0.0200161933898926 42.0
287.75546475 296.873809814453 43.0
0 0.23918879032135 44.0
0 -0.0258747935295105 45.0
0 0.160993456840515 46.0
0 -0.000384867191314697 47.0
0 0.106609523296356 48.0
0 0.346810430288315 49.0
0.079434909767 0.0445065498352051 50.0
1.7852753187 1.88800132274628 51.0
0 0.124825239181519 52.0
0 -0.231324791908264 53.0
0 0.258989572525024 54.0
0 0.21656334400177 55.0
0 0.16455090045929 56.0
0 0.437901675701141 57.0
0 0.617067933082581 58.0
0 0.140592753887177 59.0
0 0.141052007675171 60.0
0 0.0213147401809692 61.0
0 0.117117583751678 62.0
242.28158491 243.895599365234 0.0
0 -0.208391308784485 1.0
0 -0.232972383499146 2.0
0 -0.377060174942017 3.0
0 -0.268392562866211 4.0
0 0.195140540599823 5.0
0 -0.108307898044586 6.0
0 -0.136775553226471 7.0
0 0.153753399848938 8.0
0 -0.161047697067261 9.0
0 0.240248382091522 10.0
0 -0.369192600250244 11.0
0 -0.172274231910706 12.0
59.514924938 61.5732345581055 13.0
32.06230178 34.4692649841309 14.0
0 -0.00742828845977783 15.0
0 0.110497415065765 16.0
25.793300633 31.9796047210693 17.0
0 0.107811689376831 18.0
0 0.0156612992286682 19.0
0 0.143657088279724 20.0
0 0.0729265213012695 21.0
0 0.0823001861572266 22.0
36.813159216 37.9949913024902 23.0
0 0.0241678953170776 24.0
0 0.492464542388916 25.0
0 -0.0795713067054749 26.0
0 0.0497515201568604 27.0
0 -0.314719796180725 28.0
0 0.0818761587142944 29.0
0 0.0754392147064209 30.0
0 0.119280576705933 31.0
0 0.137304723262787 32.0
0 -0.115187466144562 33.0
0 -0.089749813079834 34.0
0 0.00964915752410889 35.0
0 0.175502061843872 36.0
0 0.287440359592438 37.0
0 -0.122202277183533 38.0
0 -0.231847763061523 39.0
0 0.606728076934814 40.0
0 -0.0608644485473633 41.0
0 0.178719878196716 42.0
278.08586279 297.877746582031 43.0
0 -0.0256732702255249 44.0
0 0.0913744568824768 45.0
0 -0.0109220743179321 46.0
0 -0.555748701095581 47.0
0 0.303215473890305 48.0
2.666721913 0.687828183174133 49.0
0.073982131637 0.00309020280838013 50.0
1.9455021058 1.94783842563629 51.0
0 0.465627282857895 52.0
0 -0.123297512531281 53.0
0 0.0359995365142822 54.0
0 0.0588465929031372 55.0
0 0.201540946960449 56.0
0 -0.197108209133148 57.0
0 0.0726416110992432 58.0
0 0.270001530647278 59.0
0 -0.331436276435852 60.0
0 0.0465890169143677 61.0
0 0.139633357524872 62.0
311.87771469 251.831207275391 0.0
0 -0.00311648845672607 1.0
0 -0.0690736770629883 2.0
0 0.0921697020530701 3.0
0 -0.331441164016724 4.0
0 0.468909472227097 5.0
0 0.449297785758972 6.0
0 0.360127657651901 7.0
0 -0.0944065451622009 8.0
0 0.0810234546661377 9.0
0 0.137188255786896 10.0
0 -0.0244467258453369 11.0
0 -0.293560028076172 12.0
52.283006122 54.2274284362793 13.0
35.526330907 34.9103698730469 14.0
0 0.0183674097061157 15.0
0 -0.104625463485718 16.0
31.976197706 30.8246955871582 17.0
0 0.085993766784668 18.0
0 0.0432467460632324 19.0
0 0.0333381295204163 20.0
0 0.228759050369263 21.0
0 -0.083777904510498 22.0
30.535223792 26.8762969970703 23.0
0 -0.0346994996070862 24.0
0 0.236318111419678 25.0
0 -0.0401923656463623 26.0
0 -0.0463149547576904 27.0
0 0.0395182371139526 28.0
0 -0.0276045799255371 29.0
0 0.2248495221138 30.0
0 -0.139542698860168 31.0
0 0.0485925078392029 32.0
0 -0.0168737173080444 33.0
0 -0.0575742125511169 34.0
0 0.282498717308044 35.0
0 -0.545111298561096 36.0
0 0.0311213731765747 37.0
0 0.0643360018730164 38.0
0 -0.228275418281555 39.0
0 0.0880602598190308 40.0
0 -0.159099340438843 41.0
0 -0.0765358209609985 42.0
295.04952334 298.719696044922 43.0
0 0.196511149406433 44.0
0 0.00788450241088867 45.0
0 0.317976474761963 46.0
0 -0.150673627853394 47.0
0 -0.0178532600402832 48.0
0 0.548799991607666 49.0
0.093484530431 0.021327018737793 50.0
2.253202797 1.76551365852356 51.0
0 0.153268039226532 52.0
0 -0.37297523021698 53.0
0 -0.0833058953285217 54.0
0 0.267158210277557 55.0
0 0.159020483493805 56.0
0 0.151190876960754 57.0
0 0.2923524081707 58.0
0 0.167812705039978 59.0
0 -0.110376954078674 60.0
0 0.0271457433700562 61.0
0 0.220319449901581 62.0
219.31567005 251.266235351562 0.0
0 0.0732048749923706 1.0
0 -0.244555830955505 2.0
0 0.0631389021873474 3.0
0 -0.356328725814819 4.0
0 -0.0452373027801514 5.0
0 0.125039517879486 6.0
0 0.000282883644104004 7.0
0 0.43542006611824 8.0
0 -0.0674839615821838 9.0
0 -0.1299147605896 10.0
0 0.0244545340538025 11.0
0 -0.0170727968215942 12.0
70.379393024 70.7695693969727 13.0
34.207273733 28.1716232299805 14.0
0 -0.0997897386550903 15.0
0 0.0619717240333557 16.0
40.474224841 39.6767349243164 17.0
0 0.179359197616577 18.0
0 0.0113705396652222 19.0
0 0.0914070606231689 20.0
0 0.130344748497009 21.0
0 -0.222542762756348 22.0
42.523834472 40.7098579406738 23.0
0 -0.0171388387680054 24.0
0 0.39020961523056 25.0
0 0.11286461353302 26.0
0 -0.101374268531799 27.0
0 0.847956359386444 28.0
0 0.0907001495361328 29.0
0 0.162776589393616 30.0
0 0.0793478488922119 31.0
0 -0.0298594832420349 32.0
0 -0.155145049095154 33.0
0 0.0407715439796448 34.0
0 0.206659257411957 35.0
0 0.350038021802902 36.0
0 0.121013045310974 37.0
0 0.0746222734451294 38.0
0 -0.36102819442749 39.0
0 0.744084060192108 40.0
0 -0.117614984512329 41.0
0 0.118290901184082 42.0
295.88285251 296.962524414062 43.0
0 0.0176105499267578 44.0
0 0.118553400039673 45.0
0 0.373283743858337 46.0
0 -0.420058488845825 47.0
0 -0.349251747131348 48.0
0 0.467584252357483 49.0
0.090197974917 -0.00662016868591309 50.0
2.1474421058 1.86454594135284 51.0
0 -0.164270162582397 52.0
0 -0.470454812049866 53.0
0 -0.0796239376068115 54.0
0 -0.194765567779541 55.0
0 -0.156046807765961 56.0
0 0.0802902579307556 57.0
0 0.0615476965904236 58.0
0 0.0595551729202271 59.0
0 0.199012398719788 60.0
0 0.197967290878296 61.0
0 0.309309899806976 62.0
220.67157005 252.592102050781 0.0
0 0.457388997077942 1.0
0 -0.0810993313789368 2.0
0 0.174178957939148 3.0
0 -0.296066641807556 4.0
0 -0.219319701194763 5.0
0 -0.212664365768433 6.0
0 -0.2532057762146 7.0
0 0.449880659580231 8.0
0 0.0185598731040955 9.0
0 0.300606310367584 10.0
0 -0.594436049461365 11.0
0 -0.166071772575378 12.0
73.584740272 75.4565963745117 13.0
21.265505822 23.8089408874512 14.0
0 -0.0246704816818237 15.0
0 0.216625392436981 16.0
39.070011874 37.1673545837402 17.0
0 0.180029153823853 18.0
0 -0.0418986082077026 19.0
0 0.100869119167328 20.0
0 0.024456262588501 21.0
0 0.200241804122925 22.0
32.967187562 31.2634620666504 23.0
0 0.0632047653198242 24.0
0 0.758081555366516 25.0
0 -0.095716118812561 26.0
0 -0.0079646110534668 27.0
0 0.610706865787506 28.0
0 0.104064702987671 29.0
0 0.0095868706703186 30.0
0 0.0207834243774414 31.0
0 0.125718116760254 32.0
0 -0.0483331084251404 33.0
0 0.389069080352783 34.0
0 0.0227121114730835 35.0
0 0.0965393781661987 36.0
0 -0.0535961389541626 37.0
0 0.2390416264534 38.0
0 -0.302271604537964 39.0
0 0.688129425048828 40.0
0 -0.111442565917969 41.0
0 -0.0280733108520508 42.0
306.65425679 297.802612304688 43.0
0 0.167188048362732 44.0
0 -0.0486221313476562 45.0
0 0.222137153148651 46.0
0 0.0182679295539856 47.0
0 -0.300745964050293 48.0
0 0.404923945665359 49.0
0.072621856579 -0.043864905834198 50.0
2.005245894 1.89590287208557 51.0
0 0.031386137008667 52.0
0 0.0492687225341797 53.0
0 -0.292508721351624 54.0
0 0.0160428881645203 55.0
0 0.219484210014343 56.0
0 -0.0842114686965942 57.0
0 0.0221219062805176 58.0
0 -0.059141218662262 59.0
0 -0.0372000932693481 60.0
0 0.108038485050201 61.0
0 0.191190242767334 62.0
238.04106374 254.819137573242 0.0
0 -0.119133591651917 1.0
0 -0.22053074836731 2.0
0 -0.17010223865509 3.0
0 -0.520445227622986 4.0
0 0.170156002044678 5.0
0 0.306453287601471 6.0
0 0.139305174350739 7.0
0 0.219775795936584 8.0
0 0.263939321041107 9.0
0 -0.437890529632568 10.0
0 -0.135234951972961 11.0
0 0.0162811875343323 12.0
58.689453163 56.2918014526367 13.0
47.01825198 38.4311943054199 14.0
0 0.0382483005523682 15.0
0 0.0981679558753967 16.0
42.515756931 37.4054565429688 17.0
0 0.26129162311554 18.0
0 -0.00345218181610107 19.0
0 -0.0557447671890259 20.0
0 0.207142949104309 21.0
0 -0.0440952777862549 22.0
27.366168107 26.5596408843994 23.0
0 0.0909461975097656 24.0
0 1.07299494743347 25.0
0 -0.200260698795319 26.0
0 -0.0350214242935181 27.0
0 0.453489392995834 28.0
0 0.315020531415939 29.0
0 -0.096236526966095 30.0
0 0.247753858566284 31.0
0 -0.038410484790802 32.0
0 -0.0926061868667603 33.0
0 -0.0385252237319946 34.0
0 -0.109444916248322 35.0
0 0.632706999778748 36.0
0 0.0726282596588135 37.0
0 0.213353991508484 38.0
0 -0.152838349342346 39.0
0 0.225049495697021 40.0
0 -0.156494975090027 41.0
0 0.220304012298584 42.0
295.99044964 299.526947021484 43.0
0 -1.00340914726257 44.0
0 -2.08634543418884 45.0
0 0.0746340751647949 46.0
0 0.0347197651863098 47.0
0 0.383231431245804 48.0
0 0.374308496713638 49.0
0.067907379296 0.107734203338623 50.0
1.4784783034 1.79722619056702 51.0
0 -0.24267578125 52.0
0 0.203077435493469 53.0
0 -0.0558947324752808 54.0
0 -0.0404532551765442 55.0
0 0.158625483512878 56.0
0 0.65445876121521 57.0
0 0.654978036880493 58.0
0 -0.0198028683662415 59.0
0 -0.0326352119445801 60.0
0 0.0463389158248901 61.0
0 0.122343301773071 62.0
259.92483504 246.463302612305 0.0
0 -0.365177869796753 1.0
0 0.092789351940155 2.0
0 0.148650884628296 3.0
0 -0.215685963630676 4.0
0 -0.353173136711121 5.0
0 -0.00125741958618164 6.0
0 -0.0856517553329468 7.0
0 0.903969764709473 8.0
0 0.383768916130066 9.0
0 0.388992875814438 10.0
0 -0.178481161594391 11.0
0 -0.265249371528625 12.0
47.341977477 45.1040420532227 13.0
50.880785745 44.4618988037109 14.0
0 0.0209161043167114 15.0
0 0.361261487007141 16.0
34.269138489 38.4688835144043 17.0
0 0.222620606422424 18.0
0 -0.00341761112213135 19.0
0 0.0128289461135864 20.0
0 0.231091141700745 21.0
0 0.114508032798767 22.0
31.189564932 30.4128150939941 23.0
0 0.144421696662903 24.0
36.875099932 1.43839108943939 25.0
0 -0.141362071037292 26.0
0 -0.0212987065315247 27.0
0 0.112232267856598 28.0
0 -0.0100645422935486 29.0
0 -0.00569331645965576 30.0
0 -0.188135623931885 31.0
0 0.043408215045929 32.0
0 -0.355875372886658 33.0
0 0.278104603290558 34.0
0 -0.0957756042480469 35.0
0 0.42814165353775 36.0
0 0.0160250663757324 37.0
0 0.448712408542633 38.0
0 -0.397066235542297 39.0
0 0.0602762699127197 40.0
0 -0.246230363845825 41.0
0 0.0588759183883667 42.0
281.15557654 297.708374023438 43.0
0 0.0496708750724792 44.0
0 0.131085693836212 45.0
0 0.088074266910553 46.0
0 -0.136191248893738 47.0
0 0.0947966575622559 48.0
0 0.588387548923492 49.0
0.063974411647 0.0266862511634827 50.0
2.0972755789 1.85259771347046 51.0
0 0.593541979789734 52.0
0 -0.053825855255127 53.0
0 0.328204214572906 54.0
0 0.185592770576477 55.0
0 0.0258803367614746 56.0
0 0.205026388168335 57.0
0 0.410254150629044 58.0
0 -0.00581872463226318 59.0
0 -0.282345652580261 60.0
0 0.0108664631843567 61.0
0 0.0545057058334351 62.0
302.1335631 250.482345581055 0.0
0 -0.0838512182235718 1.0
0 -0.00561392307281494 2.0
0 -0.274403929710388 3.0
0 -0.203031122684479 4.0
0 0.410350561141968 5.0
0 0.00495630502700806 6.0
0 0.30437558889389 7.0
0 -0.382092714309692 8.0
0 0.0910816788673401 9.0
0 -0.191957294940948 10.0
0 -0.0608636140823364 11.0
0 -0.023364245891571 12.0
52.019086093 51.4566307067871 13.0
52.186249782 45.8466148376465 14.0
0 -0.0985317230224609 15.0
0 0.00187206268310547 16.0
28.657082208 32.4423828125 17.0
0 0.102471947669983 18.0
0 -0.00629866123199463 19.0
0 0.0679794549942017 20.0
0 0.274314284324646 21.0
0 -0.0781439542770386 22.0
27.74360472 24.3897933959961 23.0
0 0.0555211305618286 24.0
0 0.342419683933258 25.0
0 -0.178771376609802 26.0
0 -0.303706526756287 27.0
0 0.615212500095367 28.0
0 0.0674231052398682 29.0
0 0.0710783004760742 30.0
0 0.16011917591095 31.0
0 0.227076828479767 32.0
0 0.0146207809448242 33.0
0 0.297502934932709 34.0
0 -0.0572580099105835 35.0
0 0.264854073524475 36.0
0 -0.0827921628952026 37.0
0 0.250630617141724 38.0
0 -0.246923565864563 39.0
0 0.143193960189819 40.0
0 -0.186751663684845 41.0
0 -0.0858374834060669 42.0
300.64079956 297.170440673828 43.0
0 0.0177179574966431 44.0
0 0.111398935317993 45.0
0 0.112792372703552 46.0
0 0.125616788864136 47.0
0 0.139804184436798 48.0
0 0.4077108502388 49.0
0.074691647384 0.0294901132583618 50.0
1.6097897844 1.86039650440216 51.0
0 -0.171836495399475 52.0
0 0.421723276376724 53.0
0 -0.121684491634369 54.0
0 0.123682737350464 55.0
0 0.26917427778244 56.0
0 -0.0775377154350281 57.0
0 0.403212010860443 58.0
0 0.231183648109436 59.0
0 -0.100857377052307 60.0
0 0.0470480918884277 61.0
0 0.171863615512848 62.0
239.31525595 257.375549316406 0.0
0 -0.191838622093201 1.0
0 -0.162340521812439 2.0
0 0.182618498802185 3.0
0 -0.310593605041504 4.0
0 -0.191412210464478 5.0
0 0.0285404324531555 6.0
0 0.0757747888565063 7.0
0 0.0766792297363281 8.0
0 -0.0226151943206787 9.0
0 -0.151134788990021 10.0
0 0.00490134954452515 11.0
0 -0.148792445659637 12.0
72.656255727 72.0770492553711 13.0
35.667377076 23.5826568603516 14.0
0 -0.130190134048462 15.0
0 -0.110400021076202 16.0
32.997486584 33.7927627563477 17.0
0 0.0762577056884766 18.0
0 0.00134086608886719 19.0
0 0.0924570560455322 20.0
0 0.0540843605995178 21.0
0 0.055098295211792 22.0
37.35378644 39.0373840332031 23.0
0 0.148818969726562 24.0
0 0.885859966278076 25.0
0 -0.189217388629913 26.0
0 -0.0407571792602539 27.0
0 0.300941377878189 28.0
0 -0.00288647413253784 29.0
0 0.0534205436706543 30.0
0 -0.237019538879395 31.0
0 0.0858050584793091 32.0
0 0.0866441130638123 33.0
0 0.113251566886902 34.0
0 -0.16701352596283 35.0
0 0.327653139829636 36.0
0 -0.0837109684944153 37.0
0 0.115889370441437 38.0
0 -0.18585205078125 39.0
0 0.489359259605408 40.0
0 -0.13948118686676 41.0
0 0.0813966989517212 42.0
302.08609943 298.236938476562 43.0
0 0.194611132144928 44.0
0 -0.727250099182129 45.0
0 0.199111461639404 46.0
0 0.0788141489028931 47.0
0 -0.223451256752014 48.0
0 0.390927225351334 49.0
0.094985358033 0.0395089983940125 50.0
2.0550890314 1.8065972328186 51.0
0 0.325575828552246 52.0
0 -0.0212808847427368 53.0
0 -0.0821691751480103 54.0
0 0.107821702957153 55.0
0 0.338176429271698 56.0
0 0.057977020740509 57.0
0 -0.0108867287635803 58.0
0 0.0407372713088989 59.0
0 -0.135497212409973 60.0
0 0.0343925356864929 61.0
0 0.149381995201111 62.0
246.39874901 253.700561523438 0.0
0 -0.145272493362427 1.0
0 -0.136505365371704 2.0
0 0.561215281486511 3.0
0 -0.0716124773025513 4.0
0 0.0819237232208252 5.0
0 0.136866509914398 6.0
0 0.292190462350845 7.0
0 -0.324381351470947 8.0
0 -0.108625531196594 9.0
0 -0.0202233195304871 10.0
0 -0.39763081073761 11.0
0 -0.0281975269317627 12.0
71.923019271 71.50341796875 13.0
17.97802291 30.858362197876 14.0
0 0.344730138778687 15.0
0 0.129038989543915 16.0
38.94124841 34.5877418518066 17.0
0 0.203505992889404 18.0
0 -0.00717258453369141 19.0
0 0.10054087638855 20.0
0 0.00120508670806885 21.0
0 0.174106478691101 22.0
37.488216216 36.9782562255859 23.0
0 0.11887001991272 24.0
0 0.998806059360504 25.0
0 -0.000631928443908691 26.0
0 0.081991970539093 27.0
0 0.292839288711548 28.0
0 -0.0906471014022827 29.0
0 0.051360011100769 30.0
0 -0.00114822387695312 31.0
0 0.0100472569465637 32.0
0 -0.0336149930953979 33.0
0 0.208161175251007 34.0
0 0.105001211166382 35.0
0 -0.405702590942383 36.0
0 0.132629871368408 37.0
0 -0.0801388025283813 38.0
0 -0.0845979452133179 39.0
0 -0.0670168995857239 40.0
0 -0.0965278744697571 41.0
0 -0.037445068359375 42.0
286.29513927 296.704010009766 43.0
0 0.158435940742493 44.0
0 -0.0244652032852173 45.0
0 0.138841032981873 46.0
0 0.137122273445129 47.0
0 0.227961421012878 48.0
0 0.411266356706619 49.0
0.096211713991 0.0516749620437622 50.0
1.769375075 1.83689224720001 51.0
0 0.0328083038330078 52.0
0 0.18784236907959 53.0
0 0.252936720848083 54.0
0 0.204671382904053 55.0
0 0.197437644004822 56.0
0 0.274199485778809 57.0
0 0.521692752838135 58.0
0 0.316896617412567 59.0
0 -0.0804493427276611 60.0
0 0.0936140418052673 61.0
0 0.225583493709564 62.0
253.70460744 252.832412719727 0.0
0 -0.229168176651001 1.0
0 -0.160708785057068 2.0
0 0.196403980255127 3.0
0 -0.226288795471191 4.0
0 0.240240454673767 5.0
0 0.189227879047394 6.0
0 -0.029643177986145 7.0
0 -0.108988404273987 8.0
0 0.168307721614838 9.0
0 -0.143242359161377 10.0
0 0.0229949355125427 11.0
0 -0.205300331115723 12.0
64.974404875 64.9009017944336 13.0
38.518314503 32.9237594604492 14.0
0 -0.0565662384033203 15.0
0 -0.145991683006287 16.0
30.329998196 34.1772651672363 17.0
0 0.0585877895355225 18.0
0 -0.0102731585502625 19.0
0 0.0868028998374939 20.0
0 0.191588163375854 21.0
0 0.250194013118744 22.0
33.12737726 30.3225040435791 23.0
0 0.0537440180778503 24.0
0 0.729883193969727 25.0
0 0.0325795412063599 26.0
0 -0.0671393275260925 27.0
0 0.368837833404541 28.0
0 -0.149569988250732 29.0
0 0.0582802891731262 30.0
0 -0.167398810386658 31.0
0 -0.0767937302589417 32.0
0 -0.0947664976119995 33.0
0 0.458626627922058 34.0
0 -0.152627468109131 35.0
0 0.1618292927742 36.0
0 0.215107083320618 37.0
0 -0.0402570366859436 38.0
0 -0.412869811058044 39.0
0 0.123747825622559 40.0
0 -0.182361245155334 41.0
0 0.214077591896057 42.0
303.35380131 295.360656738281 43.0
0 -0.017602801322937 44.0
0 0.197535812854767 45.0
0 0.26083642244339 46.0
0 -0.0974006652832031 47.0
0 0.36476731300354 48.0
0 0.467091739177704 49.0
0.095997323273 0.0359445214271545 50.0
2.0867652098 1.87682032585144 51.0
0 -0.203746259212494 52.0
0 -0.467712163925171 53.0
0 0.087649941444397 54.0
0 0.202467918395996 55.0
0 -0.0357471704483032 56.0
0 -0.183402061462402 57.0
0 0.244809925556183 58.0
0 -0.027739942073822 59.0
0 -0.108646392822266 60.0
0 0.0401132106781006 61.0
0 0.241355776786804 62.0
232.56218415 251.932250976562 0.0
0 -0.526880621910095 1.0
0 -0.0702677965164185 2.0
0 0.325925648212433 3.0
0 -0.2517249584198 4.0
0 -0.12289559841156 5.0
0 0.0447916984558105 6.0
0 0.296428799629211 7.0
0 0.22377872467041 8.0
0 -0.365005254745483 9.0
0 -0.257723569869995 10.0
0 0.333179891109467 11.0
0 0.156426012516022 12.0
66.756231875 67.4778137207031 13.0
15.038318469 26.9020671844482 14.0
0 -0.22726047039032 15.0
0 -0.0426200032234192 16.0
42.733016726 39.1130027770996 17.0
0 0.272828578948975 18.0
0 -0.0117199420928955 19.0
0 0.034188449382782 20.0
0 -0.0299589037895203 21.0
0 0.319553852081299 22.0
40.884618414 39.9467697143555 23.0
0 0.0809949636459351 24.0
0 0.928401589393616 25.0
0 -0.052818775177002 26.0
0 0.224338948726654 27.0
0 -0.232811570167542 28.0
0 0.0883034467697144 29.0
0 0.0439558625221252 30.0
0 -0.0539419054985046 31.0
0 -0.181012213230133 32.0
0 -0.126341223716736 33.0
0 0.11103343963623 34.0
0 1.12846684455872 35.0
0 0.145645380020142 36.0
0 0.0462611317634583 37.0
0 -0.00976580381393433 38.0
0 -0.286251425743103 39.0
0 0.990982949733734 40.0
0 -0.149436950683594 41.0
0 -0.0186485052108765 42.0
278.3503852 297.177398681641 43.0
0 0.130387663841248 44.0
0 0.120447099208832 45.0
0 -0.0609635710716248 46.0
0 -0.650840997695923 47.0
0 0.360062092542648 48.0
2.9809180946 0.657743096351624 49.0
0.080866151121 0.0520861148834229 50.0
1.7815458483 1.87652659416199 51.0
0 0.216408610343933 52.0
0 0.0230596661567688 53.0
0 -0.108893930912018 54.0
0 0.116093635559082 55.0
0 0.18940281867981 56.0
0 0.175163149833679 57.0
0 0.348414868116379 58.0
0 -0.039892315864563 59.0
0 -0.0597188472747803 60.0
0 0.049066424369812 61.0
0 0.273984253406525 62.0
255.10287453 252.727264404297 0.0
0 -0.03956139087677 1.0
0 -0.118040502071381 2.0
0 0.0153075456619263 3.0
0 0.159644246101379 4.0
0 0.232434093952179 5.0
0 0.24026620388031 6.0
0 0.102110743522644 7.0
0 -0.105279982089996 8.0
0 0.0232551097869873 9.0
0 0.151748716831207 10.0
0 -0.324177026748657 11.0
0 0.0740398168563843 12.0
50.880412886 49.1625900268555 13.0
38.691059923 47.5590591430664 14.0
0 -0.153276443481445 15.0
0 -0.0333229899406433 16.0
34.2039283 38.1514854431152 17.0
0 0.131578326225281 18.0
0 -0.0246355533599854 19.0
0 0.119747936725616 20.0
0 -0.0730080604553223 21.0
0 0.394098699092865 22.0
40.519624266 40.4154052734375 23.0
0 -0.0248873233795166 24.0
0 0.231697201728821 25.0
0 -0.07709139585495 26.0
0 0.135784804821014 27.0
0 0.304170727729797 28.0
0 0.0579870343208313 29.0
0 0.0249165892601013 30.0
0 -0.129536867141724 31.0
0 0.0519230365753174 32.0
0 0.0595694780349731 33.0
0 0.393739104270935 34.0
0 -0.106495916843414 35.0
0 0.000751614570617676 36.0
0 -0.031899631023407 37.0
0 0.14074981212616 38.0
0 -0.0902016162872314 39.0
0 0.955572485923767 40.0
0 -0.0906446576118469 41.0
0 0.134799838066101 42.0
288.24456602 297.991424560547 43.0
0 0.0443741083145142 44.0
0 0.171975255012512 45.0
0 0.0650286078453064 46.0
0 -0.0152924656867981 47.0
0 0.139236569404602 48.0
0 0.340386748313904 49.0
0.073702805731 0.0711451768875122 50.0
1.5960864304 1.81260085105896 51.0
0 0.072534441947937 52.0
0 -0.179540038108826 53.0
0 -0.0286495685577393 54.0
0 0.109014987945557 55.0
0 -0.0967021584510803 56.0
0 0.268526196479797 57.0
0 -0.171150207519531 58.0
0 0.0913461446762085 59.0
0 0.114309132099152 60.0
0 0.196495831012726 61.0
0 0.305281758308411 62.0
249.46328796 251.5927734375 0.0
0 -0.055432915687561 1.0
0 0.045932412147522 2.0
0 0.364009261131287 3.0
0 -0.0103142857551575 4.0
0 -0.00872510671615601 5.0
0 0.0348334312438965 6.0
0 0.00574409961700439 7.0
0 0.125898659229279 8.0
0 0.632750689983368 9.0
0 0.454177260398865 10.0
0 -0.512227058410645 11.0
0 -0.00180333852767944 12.0
75.946328884 72.1180267333984 13.0
3.9468699412 30.6769313812256 14.0
0 -0.318069815635681 15.0
0 0.57081139087677 16.0
30.69990668 30.7055587768555 17.0
0 0.0500926971435547 18.0
0 -0.0194596648216248 19.0
0 0.0986140966415405 20.0
0 0.291488587856293 21.0
0 -0.115524411201477 22.0
26.930817774 23.8374557495117 23.0
0 0.0819414854049683 24.0
0 0.50429093837738 25.0
0 -0.0650871396064758 26.0
0 -0.136173009872437 27.0
0 0.122305989265442 28.0
0 -0.0351842045783997 29.0
0 0.056005597114563 30.0
0 0.032956600189209 31.0
0 -0.0188212394714355 32.0
0 0.107392072677612 33.0
0 0.558457016944885 34.0
0 0.407883554697037 35.0
0 0.376464128494263 36.0
0 -0.0632480978965759 37.0
0 0.177118480205536 38.0
0 -0.252446293830872 39.0
0 0.356050819158554 40.0
0 -0.144233524799347 41.0
0 0.0481492280960083 42.0
291.84882837 295.614562988281 43.0
0 0.0327622294425964 44.0
0 0.103143692016602 45.0
0 0.15068644285202 46.0
0 0.0424173474311829 47.0
0 -0.234304189682007 48.0
0 0.442839711904526 49.0
0.090237204057 0.0185044407844543 50.0
1.9243110082 1.82086682319641 51.0
0 0.214379906654358 52.0
0 0.0558806657791138 53.0
0 0.005318284034729 54.0
0 -0.155754089355469 55.0
0 0.255340456962585 56.0
0 -0.132003605365753 57.0
0 0.207684516906738 58.0
0 -0.841109395027161 59.0
0 -0.0429126620292664 60.0
0 0.0170374512672424 61.0
0 0.0740137100219727 62.0
256.92590835 253.707382202148 0.0
0 -0.223954319953918 1.0
0 -0.338715434074402 2.0
0 0.37848824262619 3.0
0 -0.179898083209991 4.0
0 -0.0571276545524597 5.0
0 -0.0292211174964905 6.0
0 -0.162570476531982 7.0
0 0.128958344459534 8.0
0 -0.220344066619873 9.0
0 0.0448378920555115 10.0
0 -0.148453176021576 11.0
0 -0.258722066879272 12.0
55.223623362 53.9833602905273 13.0
36.956893699 38.5862731933594 14.0
0 -0.0155984163284302 15.0
0 0.265799343585968 16.0
28.186577854 33.943775177002 17.0
0 0.11813884973526 18.0
0 -0.0623271465301514 19.0
0 0.143636703491211 20.0
0 -0.055191695690155 21.0
0 0.253940999507904 22.0
37.526342396 35.0490875244141 23.0
0 0.0676602125167847 24.0
0 0.755182147026062 25.0
0 -0.0577663779258728 26.0
0 -0.015699028968811 27.0
0 0.367291539907455 28.0
0 -0.0864841938018799 29.0
0 0.168270945549011 30.0
0 0.169886827468872 31.0
0 0.0810617208480835 32.0
0 -0.0168651342391968 33.0
0 -0.00128006935119629 34.0
0 -0.0653201341629028 35.0
0 -0.121190011501312 36.0
0 0.101925313472748 37.0
0 -0.0574489831924438 38.0
0 -0.0365417003631592 39.0
0 0.473504245281219 40.0
0 -0.126334309577942 41.0
0 -0.0672154426574707 42.0
303.32930529 296.228454589844 43.0
0 -0.002480149269104 44.0
0 0.0354727506637573 45.0
0 0.304368615150452 46.0
0 0.0994730591773987 47.0
0 0.207002997398376 48.0
0 0.40416207909584 49.0
0.083098356939 0.066226065158844 50.0
2.2434371246 1.81163239479065 51.0
0 0.151218235492706 52.0
0 -0.0688019990921021 53.0
0 -0.0873488187789917 54.0
0 0.142881751060486 55.0
0 0.136614143848419 56.0
0 0.510905206203461 57.0
0 0.452434003353119 58.0
0 0.303319811820984 59.0
0 -0.114989995956421 60.0
0 0.149624228477478 61.0
0 0.215153634548187 62.0
256.19397997 251.265960693359 0.0
0 -0.206687808036804 1.0
0 -0.168766260147095 2.0
0 0.119746685028076 3.0
0 -0.301921129226685 4.0
0 0.355574697256088 5.0
0 0.0800789594650269 6.0
0 0.135171473026276 7.0
0 0.688467502593994 8.0
0 -0.458119034767151 9.0
0 0.309009522199631 10.0
0 0.672741651535034 11.0
0 -0.0707245469093323 12.0
72.559710283 72.9268646240234 13.0
22.283979656 23.5634346008301 14.0
0 0.0373783111572266 15.0
0 0.0381592512130737 16.0
36.266881649 35.428882598877 17.0
0 0.117120146751404 18.0
0 -0.0751217007637024 19.0
0 0.105795383453369 20.0
0 0.0375685095787048 21.0
0 0.179599106311798 22.0
30.190988998 30.5533618927002 23.0
0 0.103291094303131 24.0
0 0.506417155265808 25.0
0 -0.127450883388519 26.0
0 -0.0686411261558533 27.0
0 0.0109003186225891 28.0
0 -0.0556252002716064 29.0
0 0.0416589975357056 30.0
0 -0.0703548192977905 31.0
0 -0.0313244462013245 32.0
0 0.0488371849060059 33.0
0 0.0778455138206482 34.0
0 0.144640028476715 35.0
0 -0.225762009620667 36.0
0 -0.0328300595283508 37.0
0 0.190439343452454 38.0
0 -0.181407928466797 39.0
0 0.516474664211273 40.0
0 -0.169134974479675 41.0
0 0.269906282424927 42.0
298.23558351 297.215515136719 43.0
0 0.248804211616516 44.0
0 0.128442585468292 45.0
0 0.11100971698761 46.0
0 -0.11361175775528 47.0
0 0.404916733503342 48.0
0.35345542079 0.500630378723145 49.0
0.083548187829 0.0624656677246094 50.0
2.1847563736 1.87189543247223 51.0
0 0.0207034945487976 52.0
0 -0.0424154996871948 53.0
0 -0.0801195502281189 54.0
0 -0.0117627382278442 55.0
0 -0.00815463066101074 56.0
0 -0.00322681665420532 57.0
0 0.188890397548676 58.0
0 0.36145955324173 59.0
0 -0.180662631988525 60.0
0 0.110077083110809 61.0
0 0.279744744300842 62.0
269.30185065 255.084213256836 0.0
0 -0.302621960639954 1.0
0 -0.0167931318283081 2.0
0 0.231426000595093 3.0
0 0.138276100158691 4.0
0 0.377308964729309 5.0
0 0.267769277095795 6.0
0 0.177311599254608 7.0
0 -0.143995046615601 8.0
0 0.0374615788459778 9.0
0 -0.147853255271912 10.0
0 0.0952804684638977 11.0
0 0.0302430987358093 12.0
52.849592136 51.7057800292969 13.0
52.868304174 48.5539436340332 14.0
0 -0.0402489900588989 15.0
0 -0.0799867510795593 16.0
29.833932817 33.9494514465332 17.0
0 0.130226016044617 18.0
0 -0.0225851535797119 19.0
0 0.0988616943359375 20.0
0 0.088640570640564 21.0
0 0.0439309477806091 22.0
39.382427681 39.489860534668 23.0
0 0.102056741714478 24.0
0 0.920199751853943 25.0
0 -0.00169223546981812 26.0
0 -0.00403249263763428 27.0
0 0.0960589647293091 28.0
0 -0.0945758819580078 29.0
0 0.105100393295288 30.0
0 0.0938761234283447 31.0
0 0.0839489698410034 32.0
0 0.115263819694519 33.0
0 0.0576558709144592 34.0
0 -0.059375524520874 35.0
0 -0.13614559173584 36.0
0 -0.0708181262016296 37.0
0 0.105631351470947 38.0
0 -0.322211861610413 39.0
0 0.173466324806213 40.0
0 -0.269694328308105 41.0
0 0.158984303474426 42.0
283.30942408 297.299285888672 43.0
0 0.233264923095703 44.0
0 -0.820182800292969 45.0
0 0.228845000267029 46.0
0 -0.0152006149291992 47.0
0 -0.293720483779907 48.0
0 0.426570773124695 49.0
0.096816779611 -0.00743341445922852 50.0
2.0557149422 1.84910833835602 51.0
0 0.367311626672745 52.0
0 -0.169517576694489 53.0
0 0.211118340492249 54.0
0 0.296323120594025 55.0
0 0.132193207740784 56.0
0 -0.0911318063735962 57.0
0 0.494848370552063 58.0
0 0.286565124988556 59.0
0 -0.0944908261299133 60.0
0 0.137420535087585 61.0
0 0.341976702213287 62.0
265.80910048 257.967041015625 0.0
0 -0.133344411849976 1.0
0 0.0872201919555664 2.0
0 0.14895623922348 3.0
0 -0.143039703369141 4.0
0 -0.282085776329041 5.0
0 0.0309207439422607 6.0
0 0.128641247749329 7.0
0 -0.0256580114364624 8.0
0 0.30856716632843 9.0
0 0.438448369503021 10.0
0 -0.135835409164429 11.0
0 -0.185607612133026 12.0
47.139764741 50.0271186828613 13.0
50.39038906 44.9800796508789 14.0
0 0.0519458055496216 15.0
0 -0.282765984535217 16.0
24.60814755 33.3015518188477 17.0
0 0.137783586978912 18.0
0 0.00801420211791992 19.0
0 0.121321558952332 20.0
0 0.185760498046875 21.0
0 0.147798657417297 22.0
33.249849595 29.1086540222168 23.0
0 0.156379640102386 24.0
0 1.35567593574524 25.0
0 0.0840181708335876 26.0
0 -0.35166335105896 27.0
0 0.94109308719635 28.0
0 0.1477290391922 29.0
0 0.132631480693817 30.0
0 0.130362153053284 31.0
0 0.0348837971687317 32.0
0 -0.00767147541046143 33.0
0 0.504708349704742 34.0
0 -0.177255153656006 35.0
0 -0.153859913349152 36.0
0 0.207888960838318 37.0
0 -0.332387685775757 38.0
0 -0.230950355529785 39.0
0 0.132653772830963 40.0
0 0.0267228484153748 41.0
0 0.0261455774307251 42.0
299.45936428 294.921325683594 43.0
0 0.0678005814552307 44.0
0 -0.0432468056678772 45.0
0 0.258058726787567 46.0
0 -0.105952143669128 47.0
0 0.262645781040192 48.0
1.1341277689 0.619575917720795 49.0
0.091726057961 0.0611599087715149 50.0
1.8212389282 1.86894774436951 51.0
0 -0.114935874938965 52.0
0 -0.185454845428467 53.0
0 -0.00226616859436035 54.0
0 -0.0436959862709045 55.0
0 -0.190737724304199 56.0
0 -0.0941985845565796 57.0
0 -0.028807520866394 58.0
0 0.219553172588348 59.0
0 -0.212548971176147 60.0
0 0.187252998352051 61.0
0 0.259716272354126 62.0
240.22955994 254.184432983398 0.0
0 -0.161758780479431 1.0
0 -0.124567210674286 2.0
0 0.0527448058128357 3.0
0 -0.20122891664505 4.0
0 -0.136474788188934 5.0
0 0.0789076685905457 6.0
0 0.163154244422913 7.0
0 0.311187237501144 8.0
0 -0.626659154891968 9.0
0 -0.303764700889587 10.0
0 -0.2841557264328 11.0
0 -0.112581849098206 12.0
69.636107372 68.2298431396484 13.0
14.307446932 32.3733100891113 14.0
0 -0.162459373474121 15.0
0 0.00853723287582397 16.0
39.270926501 39.415340423584 17.0
0 0.238279342651367 18.0
0 -0.128158390522003 19.0
0 0.0608954429626465 20.0
0 0.00999253988265991 21.0
0 0.243900954723358 22.0
28.931644896 28.4517135620117 23.0
0 -0.00185716152191162 24.0
0 1.47337341308594 25.0
0 0.024212121963501 26.0
0 -0.131725788116455 27.0
0 0.218156635761261 28.0
0 -0.130895256996155 29.0
0 0.040492057800293 30.0
0 -0.104459524154663 31.0
0 -0.171500682830811 32.0
0 -0.436197519302368 33.0
0 0.140190184116364 34.0
0 -0.0710117816925049 35.0
0 0.321136891841888 36.0
0 0.0856158137321472 37.0
0 0.332273781299591 38.0
0 -0.532142519950867 39.0
0 -0.112072944641113 40.0
0 -0.223813772201538 41.0
0 0.0683645009994507 42.0
308.65474658 297.354095458984 43.0
0 0.0780681371688843 44.0
0 0.210016429424286 45.0
0 0.0539363026618958 46.0
0 0.207854688167572 47.0
0 0.279084444046021 48.0
0 0.363069772720337 49.0
0.087223187366 0.0678104162216187 50.0
1.8949571688 1.80947411060333 51.0
0 0.089208722114563 52.0
0 0.0851667523384094 53.0
0 0.00685727596282959 54.0
0 0.141331493854523 55.0
0 0.425189793109894 56.0
0 -0.0251501202583313 57.0
0 0.11477929353714 58.0
0 0.301464796066284 59.0
0 -0.128983318805695 60.0
0 0.104280591011047 61.0
0 0.201522886753082 62.0
260.35723117 252.658767700195 0.0
0 0.142602801322937 1.0
0 -0.10629403591156 2.0
0 0.224538862705231 3.0
0 0.0867822170257568 4.0
0 0.0761377215385437 5.0
0 -0.0565208792686462 6.0
0 0.256684124469757 7.0
0 -0.174384832382202 8.0
0 -0.260224342346191 9.0
0 -0.0907818078994751 10.0
0 -0.234324216842651 11.0
0 -0.0795348286628723 12.0
48.179879009 46.9329643249512 13.0
38.927279617 48.5355339050293 14.0
0 -0.0887973308563232 15.0
0 -0.070601761341095 16.0
27.590744074 32.5758590698242 17.0
0 0.0617215633392334 18.0
0 0.0167305469512939 19.0
0 0.0838001370429993 20.0
0 0.136137008666992 21.0
0 0.274743795394897 22.0
35.001658069 33.3886947631836 23.0
0 0.0554352998733521 24.0
0 0.68287718296051 25.0
0 -0.0877776741981506 26.0
0 -0.007027268409729 27.0
0 -0.247342109680176 28.0
0 -0.184086978435516 29.0
0 0.0550381541252136 30.0
0 0.0828539133071899 31.0
0 -0.0976637601852417 32.0
0 -0.196612238883972 33.0
0 0.285049676895142 34.0
0 0.10373991727829 35.0
0 0.54081916809082 36.0
0 0.150375247001648 37.0
0 -0.0726317167282104 38.0
0 -0.0973424911499023 39.0
0 -0.216349363327026 40.0
0 -0.0217592120170593 41.0
0 0.0625883340835571 42.0
297.93995323 294.715087890625 43.0
0 0.151729822158813 44.0
0 0.102288067340851 45.0
0 0.257423281669617 46.0
0 0.0864492654800415 47.0
0 0.188185095787048 48.0
0 0.260755300521851 49.0
0.076489205824 0.03278648853302 50.0
2.0169821015 1.87904822826385 51.0
0 0.068376362323761 52.0
0 -0.0784815549850464 53.0
0 0.147460877895355 54.0
0 0.177533566951752 55.0
0 0.0561941266059875 56.0
0 -0.0313112139701843 57.0
0 0.0703592896461487 58.0
0 -0.0251004099845886 59.0
0 0.028225302696228 60.0
0 0.050228476524353 61.0
0 0.0807836055755615 62.0
247.45731025 252.516586303711 0.0
0 0.274618864059448 1.0
0 0.00801801681518555 2.0
0 0.0641599297523499 3.0
0 -0.0661979913711548 4.0
0 -0.638431787490845 5.0
0 0.151785016059875 6.0
0 0.0894461274147034 7.0
0 -0.115974009037018 8.0
0 0.151051461696625 9.0
0 -0.0827110409736633 10.0
0 0.0511460900306702 11.0
0 -0.0390472412109375 12.0
74.734776812 73.646354675293 13.0
28.251256154 24.2598667144775 14.0
0 -0.081279993057251 15.0
0 0.0257983803749084 16.0
42.244453167 34.9527778625488 17.0
0 0.0699013471603394 18.0
0 -0.0166119337081909 19.0
0 0.024743378162384 20.0
0 0.251240491867065 21.0
0 -0.122575521469116 22.0
24.043058012 24.6879692077637 23.0
0 0.134058475494385 24.0
0 0.875103354454041 25.0
0 -0.110222578048706 26.0
0 -0.0136967301368713 27.0
0 -0.125224113464355 28.0
0 -0.187470018863678 29.0
0 -0.0156300663948059 30.0
0 -0.27031672000885 31.0
0 -0.00114679336547852 32.0
0 -0.156728744506836 33.0
0 -0.200702607631683 34.0
0 -0.0984671115875244 35.0
0 -0.224958658218384 36.0
0 0.155494630336761 37.0
0 0.0313882827758789 38.0
0 -0.348046898841858 39.0
0 0.557798683643341 40.0
0 0.013657808303833 41.0
0 0.0976406335830688 42.0
299.88042464 298.643646240234 43.0
0 0.0846837759017944 44.0
0 0.191863834857941 45.0
0 0.124820590019226 46.0
0 0.0862560868263245 47.0
0 -0.438355684280396 48.0
0 0.357332408428192 49.0
0.099544338957 -0.00656712055206299 50.0
1.6187271156 1.87341475486755 51.0
0 -0.0915431976318359 52.0
0 -0.290333867073059 53.0
0 -0.143931031227112 54.0
0 0.275071799755096 55.0
0 0.255455493927002 56.0
0 0.179955840110779 57.0
0 0.539358913898468 58.0
0 -0.902163982391357 59.0
0 0.0296759605407715 60.0
0 0.0872181057929993 61.0
0 0.0567315220832825 62.0
246.69066752 251.263137817383 0.0
0 0.274826049804688 1.0
0 0.0406395196914673 2.0
0 0.377753973007202 3.0
0 -0.290280342102051 4.0
0 0.0629350543022156 5.0
0 0.108420848846436 6.0
0 -0.0112798810005188 7.0
0 0.45097890496254 8.0
0 0.0506031513214111 9.0
0 0.207275390625 10.0
0 -0.0979668498039246 11.0
0 -0.332314491271973 12.0
53.966757899 48.8364295959473 13.0
48.433397519 42.7971839904785 14.0
0 0.215196371078491 15.0
0 -0.0588576793670654 16.0
31.213370155 31.4544734954834 17.0
0 0.122534096240997 18.0
0 0.073158860206604 19.0
0 0.109963774681091 20.0
0 0.164742231369019 21.0
0 0.0211251378059387 22.0
32.067007033 30.7559757232666 23.0
0 -0.0138934850692749 24.0
0 -0.0893984436988831 25.0
0 -0.0959938764572144 26.0
0 -0.18098372220993 27.0
0 -0.194068610668182 28.0
0 -0.156405031681061 29.0
0 0.121557831764221 30.0
0 0.30044424533844 31.0
0 0.204617381095886 32.0
0 -0.107882142066956 33.0
0 0.0136157870292664 34.0
0 0.0572016835212708 35.0
0 -0.361784219741821 36.0
0 0.156822204589844 37.0
0 0.152771532535553 38.0
0 -0.47182023525238 39.0
0 -0.235887765884399 40.0
0 -0.315491199493408 41.0
0 -0.112755656242371 42.0
292.27003107 296.573028564453 43.0
0 0.173000812530518 44.0
0 0.131639063358307 45.0
0 0.210295975208282 46.0
0 0.00120705366134644 47.0
0 0.34153887629509 48.0
0 0.480399668216705 49.0
0.065155186409 0.0788968801498413 50.0
2.3143531112 1.84195363521576 51.0
0 0.0155336260795593 52.0
0 0.0244226455688477 53.0
0 0.305600434541702 54.0
0 0.253642678260803 55.0
0 -0.00704681873321533 56.0
0 0.50070595741272 57.0
0 -0.0526158213615417 58.0
0 -1.03495740890503 59.0
0 1.07304334640503 60.0
0 0.030670166015625 61.0
0 -0.0138763785362244 62.0
266.99238228 253.5830078125 0.0
0 -0.216085195541382 1.0
0 0.0274404883384705 2.0
0 0.156573712825775 3.0
0 0.192569136619568 4.0
0 0.102532863616943 5.0
0 -0.13592791557312 6.0
0 0.451374530792236 7.0
0 0.583335876464844 8.0
0 -0.171290457248688 9.0
0 0.325121313333511 10.0
0 -0.18688040971756 11.0
0 0.0427335500717163 12.0
58.021837003 56.4258880615234 13.0
32.947656746 35.6137161254883 14.0
0 0.0115277767181396 15.0
0 0.0861889719963074 16.0
40.787496385 37.6395683288574 17.0
0 0.154365658760071 18.0
0 0.0283286571502686 19.0
0 0.0469515919685364 20.0
0 0.0378918647766113 21.0
0 0.0416612029075623 22.0
37.103448925 37.766414642334 23.0
0 0.0253188610076904 24.0
0 0.521162509918213 25.0
0 -0.225385427474976 26.0
0 -0.0922127366065979 27.0
0 -0.0273195505142212 28.0
0 0.0042579174041748 29.0
0 0.0762200951576233 30.0
0 0.0761340856552124 31.0
0 0.118764162063599 32.0
0 -0.0652291178703308 33.0
0 0.342396557331085 34.0
0 0.379256546497345 35.0
0 0.132109880447388 36.0
0 -0.0132306218147278 37.0
0 0.204655885696411 38.0
0 -0.36121666431427 39.0
0 0.221013903617859 40.0
0 -0.183617174625397 41.0
0 0.192640781402588 42.0
297.93071406 295.672332763672 43.0
0 0.13025438785553 44.0
0 -0.0139948129653931 45.0
0 0.196593225002289 46.0
0 -0.237877607345581 47.0
0 -0.324338316917419 48.0
0 0.421936392784119 49.0
0.089120521593 0.00996488332748413 50.0
2.2522056604 1.84091663360596 51.0
0 -0.102748274803162 52.0
0 0.172996520996094 53.0
0 0.223569273948669 54.0
0 0.330415666103363 55.0
0 -0.137559831142426 56.0
0 0.0636763572692871 57.0
0 0.159683227539062 58.0
0 0.116494536399841 59.0
0 -0.253713011741638 60.0
0 0.0645606517791748 61.0
0 0.152528285980225 62.0
234.02942146 254.077072143555 0.0
0 -0.189453601837158 1.0
0 -0.0274432897567749 2.0
0 0.482178628444672 3.0
0 -0.0820915699005127 4.0
0 -0.106216013431549 5.0
0 -0.0362812280654907 6.0
0 -0.197096645832062 7.0
0 -0.248427629470825 8.0
0 0.421799957752228 9.0
0 0.698601424694061 10.0
0 -0.350149989128113 11.0
0 0.0583950877189636 12.0
60.630432255 61.5135498046875 13.0
39.253370751 34.528148651123 14.0
0 -0.100228071212769 15.0
0 -0.0480427742004395 16.0
39.092942463 36.2925224304199 17.0
0 0.113848924636841 18.0
0 -0.0218024253845215 19.0
0 0.104837417602539 20.0
0 0.0667519569396973 21.0
0 0.194904565811157 22.0
32.935321283 31.8036613464355 23.0
0 0.134908437728882 24.0
0 0.523055195808411 25.0
0 -0.0438104271888733 26.0
0 -0.00576436519622803 27.0
0 0.641599774360657 28.0
0 -0.0711405277252197 29.0
0 0.141154885292053 30.0
0 -0.0818923711776733 31.0
0 -0.0365521907806396 32.0
0 0.0266709923744202 33.0
0 0.221195042133331 34.0
0 -0.083655834197998 35.0
0 -0.286728382110596 36.0
0 0.106833517551422 37.0
0 0.31634321808815 38.0
0 -0.287710547447205 39.0
0 -0.266471743583679 40.0
0 -0.215891003608704 41.0
0 0.111749053001404 42.0
290.69798825 293.647033691406 43.0
0 0.0166650414466858 44.0
0 0.441018581390381 45.0
0 0.118948698043823 46.0
0 -0.196911871433258 47.0
0 0.175154149532318 48.0
4.1611504403 0.363787293434143 49.0
0.10066955157 0.0212751626968384 50.0
1.9154200387 1.93253350257874 51.0
0 -0.1540247797966 52.0
0 -0.188623547554016 53.0
0 0.104201555252075 54.0
0 0.0336610078811646 55.0
0 -0.0947564840316772 56.0
0 0.0584521889686584 57.0
0 0.0688050985336304 58.0
0 -0.895137906074524 59.0
0 1.47502911090851 60.0
0 0.0352839827537537 61.0
0 0.11490660905838 62.0
238.63475866 252.824645996094 0.0
0 0.199345231056213 1.0
0 0.0623419880867004 2.0
0 0.478128731250763 3.0
0 -0.240070343017578 4.0
0 -0.0836659073829651 5.0
0 -0.0398415327072144 6.0
0 -0.158021211624146 7.0
0 -0.215845942497253 8.0
0 0.340386509895325 9.0
0 0.579435229301453 10.0
0 -0.780157685279846 11.0
0 -0.179744243621826 12.0
71.951354519 72.1983642578125 13.0
19.015484427 25.4050579071045 14.0
0 -0.094064474105835 15.0
0 0.0215091705322266 16.0
35.376981317 34.8843154907227 17.0
0 0.164305686950684 18.0
0 -0.0770151019096375 19.0
0 0.159397482872009 20.0
0 0.00257915258407593 21.0
0 0.253071784973145 22.0
33.600307126 33.0991401672363 23.0
0 0.0560321807861328 24.0
0 0.37376880645752 25.0
0 -0.0565996170043945 26.0
0 -0.0157008171081543 27.0
0 0.285918295383453 28.0
0 0.0423624515533447 29.0
0 0.118249297142029 30.0
0 -0.124920010566711 31.0
0 0.00623035430908203 32.0
0 -0.0615842342376709 33.0
0 0.278687417507172 34.0
0 -0.11700302362442 35.0
0 0.195307850837708 36.0
0 0.00275838375091553 37.0
0 0.0303031206130981 38.0
0 -0.105907440185547 39.0
0 0.354363560676575 40.0
0 -0.0747752785682678 41.0
0 0.142630815505981 42.0
293.1893796 299.452728271484 43.0
0 0.102900981903076 44.0
0 0.119565784931183 45.0
0 0.0823419690132141 46.0
0 0.344770729541779 47.0
0 0.366850256919861 48.0
0 0.432804018259048 49.0
0.07959358186 0.0586379766464233 50.0
2.0187404117 1.8595107793808 51.0
0 -0.16364324092865 52.0
0 0.115013182163239 53.0
0 0.0967172980308533 54.0
0 0.133016884326935 55.0
0 0.0284337401390076 56.0
0 0.571370482444763 57.0
0 0.0884723663330078 58.0
0 0.0591214895248413 59.0
0 0.489570021629333 60.0
0 0.040362536907196 61.0
0 0.155936002731323 62.0
221.67063151 254.213562011719 0.0
0 -0.27684485912323 1.0
0 -0.167679607868195 2.0
0 0.0494164824485779 3.0
0 -0.425289392471313 4.0
0 -0.16388875246048 5.0
0 0.0919111967086792 6.0
0 -0.0619553327560425 7.0
0 -0.227971911430359 8.0
0 0.0320231914520264 9.0
0 0.0453895330429077 10.0
0 -0.236878752708435 11.0
0 0.00582540035247803 12.0
67.13595327 67.5874862670898 13.0
26.766804939 30.3157005310059 14.0
0 -0.158564805984497 15.0
0 0.137623250484467 16.0
37.589080986 36.5478286743164 17.0
0 0.105602383613586 18.0
0 0.0415835380554199 19.0
0 0.0568186640739441 20.0
0 0.0408096313476562 21.0
0 0.000563740730285645 22.0
39.765662904 37.651439666748 23.0
0 0.0567846298217773 24.0
0 0.458557844161987 25.0
0 -0.0997862815856934 26.0
0 -0.0696006417274475 27.0
0 0.270397186279297 28.0
0 -0.0326427817344666 29.0
0 -0.192762553691864 30.0
0 0.151733994483948 31.0
0 0.0203465819358826 32.0
0 -0.0549150109291077 33.0
0 0.121628522872925 34.0
0 -0.131248950958252 35.0
0 0.595100879669189 36.0
0 -0.00276809930801392 37.0
0 0.146188080310822 38.0
0 -0.0983165502548218 39.0
0 0.381992220878601 40.0
0 -0.177654445171356 41.0
0 0.0329734086990356 42.0
290.05078228 295.440612792969 43.0
0 0.146216869354248 44.0
0 0.131871581077576 45.0
0 0.149818539619446 46.0
0 -0.0518200397491455 47.0
0 0.24384742975235 48.0
0 0.444764852523804 49.0
0.078756182425 0.0379324555397034 50.0
1.8235164658 1.83496332168579 51.0
0 -0.0208646059036255 52.0
0 -0.0811992287635803 53.0
0 0.0531444549560547 54.0
0 -0.0126538872718811 55.0
0 0.0110784769058228 56.0
0 0.389127314090729 57.0
0 0.438230097293854 58.0
0 0.228605806827545 59.0
0 -0.135674476623535 60.0
0 0.111048758029938 61.0
0 0.236197531223297 62.0
269.60218236 251.487243652344 0.0
0 0.00490200519561768 1.0
0 -0.0782777667045593 2.0
0 0.281329333782196 3.0
0 0.055077075958252 4.0
0 0.189912796020508 5.0
0 -0.0239421129226685 6.0
0 0.398630917072296 7.0
0 0.316960155963898 8.0
0 -0.0787259340286255 9.0
0 0.0797919034957886 10.0
0 0.0895401239395142 11.0
0 0.0494571328163147 12.0
64.710139171 66.0131912231445 13.0
34.017302934 36.0433044433594 14.0
0 -0.0887655019760132 15.0
0 -0.0518186092376709 16.0
31.216245485 30.5454578399658 17.0
0 0.0729053616523743 18.0
0 -0.0313669443130493 19.0
0 0.141729950904846 20.0
0 0.0124396085739136 21.0
0 0.231602668762207 22.0
34.839461676 33.7008934020996 23.0
0 0.0572041273117065 24.0
0 0.477529525756836 25.0
0 -0.0238507986068726 26.0
0 0.0150858163833618 27.0
0 -0.0404505133628845 28.0
0 -0.149750351905823 29.0
0 0.0496231317520142 30.0
0 0.105603337287903 31.0
0 0.175659716129303 32.0
0 0.0869521498680115 33.0
0 0.134120583534241 34.0
0 -0.0680087208747864 35.0
0 -0.258133292198181 36.0
0 -0.0656205415725708 37.0
0 0.263485431671143 38.0
0 -0.118837237358093 39.0
0 0.136957705020905 40.0
0 -0.189989864826202 41.0
0 0.11033022403717 42.0
295.9116752 296.487396240234 43.0
0 0.185251951217651 44.0
0 0.0719508528709412 45.0
0 0.132564723491669 46.0
0 0.0845111012458801 47.0
0 -0.312100529670715 48.0
0 0.425756812095642 49.0
0.070130376699 -0.00454777479171753 50.0
2.093771922 1.83996653556824 51.0
0 0.20316743850708 52.0
0 -0.180166244506836 53.0
0 -0.00868988037109375 54.0
0 0.0805594921112061 55.0
0 0.340966045856476 56.0
0 -0.268537282943726 57.0
0 0.0294660329818726 58.0
0 -0.177124857902527 59.0
0 0.829613208770752 60.0
0 0.0255935788154602 61.0
0 0.0197808146476746 62.0
254.52474018 256.051666259766 0.0
0 -0.347896575927734 1.0
0 0.0417126417160034 2.0
0 0.312213718891144 3.0
0 -0.073874831199646 4.0
0 0.0973678231239319 5.0
0 -0.214502096176147 6.0
0 0.153190493583679 7.0
0 -0.248320460319519 8.0
0 0.408316254615784 9.0
0 -0.60291063785553 10.0
0 -0.0749980211257935 11.0
0 -0.0683083534240723 12.0
70.03709014 69.1454925537109 13.0
34.893110645 30.3429946899414 14.0
0 -0.0391415357589722 15.0
0 -0.0934513211250305 16.0
36.13946642 36.2432708740234 17.0
0 0.179111361503601 18.0
0 -0.0565533638000488 19.0
0 0.179589688777924 20.0
0 -0.106378674507141 21.0
0 0.622613489627838 22.0
40.064822834 39.8285942077637 23.0
0 0.13896644115448 24.0
0 0.994434058666229 25.0
0 -0.0538887977600098 26.0
0 -9.50098037719727e-05 27.0
0 0.34562075138092 28.0
0 -0.0839294791221619 29.0
0 -0.0200490951538086 30.0
0 0.252281308174133 31.0
0 0.139449656009674 32.0
0 0.150823175907135 33.0
0 0.164012432098389 34.0
0 -0.104588687419891 35.0
0 0.000599026679992676 36.0
0 0.000977993011474609 37.0
0 0.152535319328308 38.0
0 -0.313971638679504 39.0
0 0.475899875164032 40.0
0 -0.103375673294067 41.0
0 -0.000197291374206543 42.0
285.89570968 295.690948486328 43.0
0 0.310710072517395 44.0
0 -0.0166834592819214 45.0
0 0.116893768310547 46.0
0 0.0607671141624451 47.0
0 0.0718957185745239 48.0
1.5905183078 0.557084918022156 49.0
0.068831653867 0.0547372102737427 50.0
2.0531431832 1.86539936065674 51.0
0 0.224523186683655 52.0
0 -0.0913999080657959 53.0
0 0.12302041053772 54.0
0 0.262987971305847 55.0
0 0.257076263427734 56.0
0 -0.0582005977630615 57.0
0 0.15214341878891 58.0
0 0.203812003135681 59.0
0 -0.25237774848938 60.0
0 0.0949881076812744 61.0
0 0.300126969814301 62.0
253.20000185 253.154220581055 0.0
0 -0.229806184768677 1.0
0 -0.222513675689697 2.0
0 0.0898733139038086 3.0
0 -0.0524148344993591 4.0
0 0.219842374324799 5.0
0 0.132041454315186 6.0
0 -0.0848646759986877 7.0
0 -0.157478213310242 8.0
0 0.00683259963989258 9.0
0 -0.296963214874268 10.0
0 -0.340077996253967 11.0
0 -0.224663734436035 12.0
78.400270341 79.8179016113281 13.0
30.45134704 22.2812175750732 14.0
0 -0.136246562004089 15.0
0 -0.170411348342896 16.0
28.155920514 33.8414611816406 17.0
0 0.112393319606781 18.0
0 0.0173652172088623 19.0
0 0.098391056060791 20.0
0 0.192510485649109 21.0
0 0.0651675462722778 22.0
25.551756648 27.480432510376 23.0
0 0.14172625541687 24.0
0 1.39043760299683 25.0
0 0.018667459487915 26.0
0 -0.385514974594116 27.0
0 0.113235533237457 28.0
0 0.128305494785309 29.0
0 -0.124564290046692 30.0
0 -0.0251063108444214 31.0
0 0.187005162239075 32.0
0 0.102340817451477 33.0
0 0.166940212249756 34.0
0 -0.045799732208252 35.0
0 -0.204535067081451 36.0
0 0.0822319984436035 37.0
0 0.270028829574585 38.0
0 -0.384612798690796 39.0
0 -0.451505303382874 40.0
0 -0.156560897827148 41.0
0 0.145513892173767 42.0
280.91359608 295.059875488281 43.0
0 0.0390816926956177 44.0
0 0.115761995315552 45.0
0 0.24350780248642 46.0
0 -0.677767515182495 47.0
0 0.349889159202576 48.0
2.9527220798 0.690561473369598 49.0
0.087160510679 0.0567977428436279 50.0
1.5003556838 1.79088258743286 51.0
0 -0.137627840042114 52.0
0 -0.270986557006836 53.0
0 -0.0138571262359619 54.0
0 0.10032856464386 55.0
0 0.0137190818786621 56.0
0 0.336416900157928 57.0
0 0.346686065196991 58.0
0 0.280064642429352 59.0
0 0.148486912250519 60.0
0 0.157376050949097 61.0
0 0.540888965129852 62.0
264.47989877 252.057052612305 0.0
0 -0.0706616640090942 1.0
0 -0.214050650596619 2.0
0 0.267758369445801 3.0
0 -0.0827686190605164 4.0
0 0.236418724060059 5.0
0 0.139753103256226 6.0
0 0.129808902740479 7.0
0 -0.186333298683167 8.0
0 0.0549106597900391 9.0
0 -0.111569046974182 10.0
0 -0.1533163189888 11.0
0 -0.1490718126297 12.0
61.562181259 60.4845161437988 13.0
30.04947805 37.5599479675293 14.0
0 -0.0312337875366211 15.0
0 -0.176351189613342 16.0
29.63005733 30.0599956512451 17.0
0 0.0394699573516846 18.0
0 -0.00635778903961182 19.0
0 0.133033514022827 20.0
0 0.146846055984497 21.0
0 0.0909537076950073 22.0
31.069100403 28.4419345855713 23.0
0 0.0580105781555176 24.0
0 0.830556869506836 25.0
0 -0.0267925262451172 26.0
0 -0.092864453792572 27.0
0 0.0471352338790894 28.0
0 -0.0654050707817078 29.0
0 -0.226264715194702 30.0
0 0.0476183891296387 31.0
0 0.0168108940124512 32.0
0 0.0661662817001343 33.0
0 0.267021000385284 34.0
0 -0.0600419640541077 35.0
0 0.300776213407516 36.0
0 -0.0208758115768433 37.0
0 0.305597871541977 38.0
0 -0.256969809532166 39.0
0 -0.275140762329102 40.0
0 -0.178170144557953 41.0
0 -0.034110426902771 42.0
293.03229757 297.762542724609 43.0
0 0.0638707876205444 44.0
0 0.0770320892333984 45.0
0 0.199635744094849 46.0
0 -0.0943006873130798 47.0
0 0.350417643785477 48.0
0 0.490973591804504 49.0
0.082113524514 0.0146547555923462 50.0
2.1334813876 1.90030443668365 51.0
0 -0.127619981765747 52.0
0 -0.00920575857162476 53.0
0 0.0505024790763855 54.0
0 0.294353902339935 55.0
0 -0.0889441967010498 56.0
0 0.166193962097168 57.0
0 0.311720877885818 58.0
0 0.0801180601119995 59.0
0 0.0320455431938171 60.0
0 0.0571821928024292 61.0
0 0.104700565338135 62.0
248.62109118 255.081085205078 0.0
0 -0.17409873008728 1.0
0 -0.169198513031006 2.0
0 0.167503833770752 3.0
0 -0.322903156280518 4.0
0 -0.232111692428589 5.0
0 0.0334083437919617 6.0
0 0.153155088424683 7.0
0 0.037388801574707 8.0
0 -0.221319675445557 9.0
0 -0.113846063613892 10.0
0 -0.324866056442261 11.0
0 -0.180021226406097 12.0
61.562914844 62.2610549926758 13.0
25.064835695 34.4766845703125 14.0
0 -0.0252194404602051 15.0
0 0.122851312160492 16.0
32.794741914 34.6822738647461 17.0
0 0.0651904940605164 18.0
0 -0.0570316314697266 19.0
0 0.127397894859314 20.0
0 0.0970866680145264 21.0
0 0.0664470195770264 22.0
29.866216802 29.0018749237061 23.0
0 0.0584239959716797 24.0
0 0.725435376167297 25.0
0 -0.279334664344788 26.0
0 -0.162621319293976 27.0
0 0.409194201231003 28.0
0 0.0550428032875061 29.0
0 0.054290771484375 30.0
0 0.00398397445678711 31.0
0 -0.144006907939911 32.0
0 -0.178066492080688 33.0
0 0.0935410261154175 34.0
0 -0.100897133350372 35.0
0 0.445030778646469 36.0
0 0.155469536781311 37.0
0 0.099042534828186 38.0
0 -0.076532244682312 39.0
0 0.821493625640869 40.0
0 -0.0761374235153198 41.0
0 0.107023596763611 42.0
293.00686217 297.213989257812 43.0
0 0.0948984622955322 44.0
0 0.063825249671936 45.0
0 0.259861350059509 46.0
0 -0.118612468242645 47.0
0 0.00741392374038696 48.0
0 0.41845977306366 49.0
0.08402808032 0.0665513277053833 50.0
2.1993978314 1.86678028106689 51.0
0 0.35840779542923 52.0
0 -0.174015939235687 53.0
0 -0.162058234214783 54.0
0 0.0159967541694641 55.0
0 0.306643664836884 56.0
0 0.0630309581756592 57.0
0 0.592885971069336 58.0
0 0.10097861289978 59.0
0 0.198605179786682 60.0
0 0.117931723594666 61.0
0 0.210774600505829 62.0
281.61087054 252.068374633789 0.0
0 0.228803277015686 1.0
0 -0.0632198452949524 2.0
0 0.0866191387176514 3.0
0 -0.271500825881958 4.0
0 0.233692109584808 5.0
0 -0.299856305122375 6.0
0 0.725225985050201 7.0
0 0.131709337234497 8.0
0 -0.430919647216797 9.0
0 -0.15853625535965 10.0
0 -0.338534832000732 11.0
0 0.000266313552856445 12.0
59.159966396 60.6585998535156 13.0
30.17111157 35.7030982971191 14.0
0 -0.0606796741485596 15.0
0 0.118739783763885 16.0
30.985182362 33.8242111206055 17.0
0 0.0616604089736938 18.0
0 0.00644600391387939 19.0
0 0.0585078001022339 20.0
0 0.0560300350189209 21.0
0 0.187938868999481 22.0
36.34464829 33.3576049804688 23.0
0 0.0495421886444092 24.0
0 0.687970876693726 25.0
0 0.0576580166816711 26.0
0 -0.115110039710999 27.0
0 0.696291148662567 28.0
0 0.0607452392578125 29.0
0 0.00528723001480103 30.0
0 0.0755696296691895 31.0
0 0.158458530902863 32.0
0 -0.117232024669647 33.0
0 -0.0456516742706299 34.0
0 0.140218377113342 35.0
0 -0.0860500335693359 36.0
0 -0.103471577167511 37.0
0 0.296372950077057 38.0
0 -0.536327838897705 39.0
0 0.165136218070984 40.0
0 -0.273038983345032 41.0
0 0.0139168500900269 42.0
306.70099068 296.693908691406 43.0
0 0.0506287217140198 44.0
0 0.101612627506256 45.0
0 0.211226880550385 46.0
0 -0.178605914115906 47.0
0 0.0803237557411194 48.0
0 0.356560826301575 49.0
0.069129266665 -0.0334360003471375 50.0
1.8001557747 1.90140652656555 51.0
0 -0.177024364471436 52.0
0 -0.0200719833374023 53.0
0 -0.115221619606018 54.0
0 0.317336827516556 55.0
0 0.108370721340179 56.0
0 0.260280549526215 57.0
0 0.120164394378662 58.0
0 0.0140376687049866 59.0
0 -0.0882936120033264 60.0
0 0.0553081035614014 61.0
0 0.222079992294312 62.0
245.42931705 253.819747924805 0.0
0 -0.266741394996643 1.0
0 -0.261387705802917 2.0
0 0.149663746356964 3.0
0 -0.369258642196655 4.0
0 0.0692722201347351 5.0
0 0.118691682815552 6.0
0 0.0836800336837769 7.0
0 0.231366276741028 8.0
0 -0.150232076644897 9.0
0 -0.109773993492126 10.0
0 -0.252266883850098 11.0
0 -0.00813233852386475 12.0
54.965823083 54.3890419006348 13.0
35.444878905 39.8210525512695 14.0
0 0.00130379199981689 15.0
0 -0.139818131923676 16.0
30.791429567 31.3491706848145 17.0
0 0.104801774024963 18.0
0 -0.0208253860473633 19.0
0 0.235227227210999 20.0
0 0.0635553598403931 21.0
0 0.120914041996002 22.0
31.39693113 28.2994213104248 23.0
0 0.0442063808441162 24.0
0 0.38375985622406 25.0
0 -0.0576499104499817 26.0
0 0.0177996754646301 27.0
0 0.64492130279541 28.0
0 -0.00363457202911377 29.0
0 0.0301170945167542 30.0
0 -0.0895580053329468 31.0
0 0.00370550155639648 32.0
0 0.0542401671409607 33.0
0 0.241374015808105 34.0
0 -0.113663673400879 35.0
0 0.198782861232758 36.0
0 -0.0796782970428467 37.0
0 0.268583655357361 38.0
0 -0.210178971290588 39.0
0 -0.151827156543732 40.0
0 -0.20091849565506 41.0
0 0.0308935642242432 42.0
301.41947472 297.702758789062 43.0
0 0.0614558458328247 44.0
0 0.056435763835907 45.0
0 0.119798064231873 46.0
0 0.156105518341064 47.0
0 0.276025772094727 48.0
0 0.380167365074158 49.0
0.086372717601 0.0133238434791565 50.0
1.9782131787 1.90388989448547 51.0
0 -0.092157244682312 52.0
0 -0.0285882353782654 53.0
0 0.120867848396301 54.0
0 0.161777079105377 55.0
0 0.194555401802063 56.0
0 0.557766675949097 57.0
0 0.179696142673492 58.0
0 0.0892460942268372 59.0
0 -0.117299497127533 60.0
0 0.0442955493927002 61.0
0 0.15706080198288 62.0
247.30722208 252.588455200195 0.0
0 0.201885581016541 1.0
0 -0.0711482763290405 2.0
0 0.368576347827911 3.0
0 -0.188280463218689 4.0
0 -0.0648506283760071 5.0
0 -0.00506281852722168 6.0
0 0.0559051632881165 7.0
0 -0.16639643907547 8.0
0 0.157883822917938 9.0
0 0.0633739233016968 10.0
0 -0.112404227256775 11.0
0 0.0338269472122192 12.0
51.76171997 52.3122215270996 13.0
38.01176415 42.6655502319336 14.0
0 -0.0564199686050415 15.0
0 0.0944488644599915 16.0
33.669348058 31.7176342010498 17.0
0 0.0959498286247253 18.0
0 -0.0280921459197998 19.0
0 0.129144906997681 20.0
0 0.0462718605995178 21.0
0 0.158501505851746 22.0
31.110092225 28.9580020904541 23.0
0 0.0382599234580994 24.0
0 0.0839568972587585 25.0
0 0.0405067205429077 26.0
0 -0.0587950944900513 27.0
0 0.703392446041107 28.0
0 0.0313761234283447 29.0
0 0.0933221578598022 30.0
0 -0.0194787979125977 31.0
0 -0.0184280872344971 32.0
0 -0.015578031539917 33.0
0 0.420476526021957 34.0
0 0.0754299163818359 35.0
0 0.262962937355042 36.0
0 0.0399730801582336 37.0
0 0.201818943023682 38.0
0 -0.145437717437744 39.0
0 -0.159006953239441 40.0
0 -0.294509410858154 41.0
0 0.0558404922485352 42.0
300.2608113 297.722869873047 43.0
0 0.176489949226379 44.0
0 0.0347360968589783 45.0
0 0.16604095697403 46.0
0 0.0693956017494202 47.0
0 0.0856854915618896 48.0
0 0.391409039497375 49.0
0.072681129229 0.0429006814956665 50.0
1.955695114 1.84262537956238 51.0
0 0.109003901481628 52.0
0 -0.187978029251099 53.0
0 0.0145375728607178 54.0
0 -0.0430939793586731 55.0
0 0.0781006813049316 56.0
0 0.0757592916488647 57.0
0 -0.0093725323677063 58.0
0 0.0115739703178406 59.0
0 -0.0623697638511658 60.0
0 0.0210472345352173 61.0
0 0.0853515863418579 62.0
276.19814493 251.217956542969 0.0
0 -0.246844410896301 1.0
0 -0.00147223472595215 2.0
0 0.175016582012177 3.0
0 0.000411868095397949 4.0
0 0.184344530105591 5.0
0 0.034299910068512 6.0
0 0.297949284315109 7.0
0 0.0563649535179138 8.0
0 0.225748002529144 9.0
0 0.349851727485657 10.0
0 0.043793261051178 11.0
0 -0.0604853630065918 12.0
62.047479561 60.4752388000488 13.0
37.360148293 38.3654289245605 14.0
0 0.0232594013214111 15.0
0 -0.0890641212463379 16.0
32.145155216 34.49072265625 17.0
0 0.0586729645729065 18.0
0 -0.00431239604949951 19.0
0 0.0242385268211365 20.0
0 0.149133741855621 21.0
0 0.213481545448303 22.0
30.360410063 29.5945491790771 23.0
0 0.0811034440994263 24.0
0 1.18321895599365 25.0
0 0.0601567029953003 26.0
0 -0.109527230262756 27.0
0 0.192943572998047 28.0
0 0.0148926973342896 29.0
0 0.0705716609954834 30.0
0 -0.00101137161254883 31.0
0 0.0116175413131714 32.0
0 -0.0227473378181458 33.0
0 0.341015726327896 34.0
0 -0.138362407684326 35.0
0 0.0419358015060425 36.0
0 0.0393295288085938 37.0
0 0.196011543273926 38.0
0 -0.326362490653992 39.0
0 0.161701321601868 40.0
0 -0.206983983516693 41.0
0 0.185359716415405 42.0
293.27436776 296.146453857422 43.0
0 0.189197659492493 44.0
0 0.0126558542251587 45.0
0 0.0961363315582275 46.0
0 0.0609390139579773 47.0
0 0.378313511610031 48.0
0.17418863458 0.535687148571014 49.0
0.098072052474 0.0459012985229492 50.0
1.7359085401 1.8556113243103 51.0
0 0.0323427319526672 52.0
0 -0.030204176902771 53.0
0 -0.107821047306061 54.0
0 -0.0126105546951294 55.0
0 0.0170384645462036 56.0
0 0.136048376560211 57.0
0 0.420386463403702 58.0
0 0.292101591825485 59.0
0 -0.125784456729889 60.0
0 0.123978853225708 61.0
0 0.208646774291992 62.0
210.08432496 218.786697387695 0.0
0 -0.672206997871399 1.0
0 -0.250247716903687 2.0
0 -0.157901167869568 3.0
0 -0.290267467498779 4.0
0 -0.206009566783905 5.0
0 0.0186063647270203 6.0
0 -0.192647814750671 7.0
0 0.280226349830627 8.0
0 0.0956286787986755 9.0
0 0.316330015659332 10.0
0 -0.018486499786377 11.0
0 -0.176819562911987 12.0
73.376851198 73.0581741333008 13.0
32.887369122 27.3280448913574 14.0
0 -0.457944631576538 15.0
0 0.259086847305298 16.0
29.330682953 35.6409301757812 17.0
0 0.123329401016235 18.0
0 0.0363273620605469 19.0
0 0.100494623184204 20.0
0 0.00367480516433716 21.0
0 0.217650949954987 22.0
42.076584507 40.0263290405273 23.0
0 0.0921607613563538 24.0
0 1.27037870883942 25.0
0 -0.0017198920249939 26.0
0 0.0410205721855164 27.0
0 5.89745616912842 28.0
0 0.0453764200210571 29.0
0 0.0145388841629028 30.0
0 0.105854511260986 31.0
0 0.0776375532150269 32.0
0 -0.242255926132202 33.0
0 0.220814049243927 34.0
0 0.062121570110321 35.0
0 0.701914191246033 36.0
0 -0.00479453802108765 37.0
0 0.0924620628356934 38.0
0 -0.107990503311157 39.0
0 -0.338527917861938 40.0
0 -0.0663641691207886 41.0
0 0.172900915145874 42.0
298.38481952 296.7080078125 43.0
0 0.200477957725525 44.0
0 -0.00701946020126343 45.0
0 0.12399286031723 46.0
0 -0.0251255035400391 47.0
0 0.379968196153641 48.0
0 0.40594556927681 49.0
0.077927456927 0.0610136985778809 50.0
1.3317238482 1.8725790977478 51.0
0 -0.266356468200684 52.0
0 0.100359797477722 53.0
0 0.226872146129608 54.0
0 0.21896368265152 55.0
0 0.00680303573608398 56.0
0 0.481357902288437 57.0
0 0.0361807942390442 58.0
0 0.242486298084259 59.0
0 -0.134459257125854 60.0
0 0.0350402593612671 61.0
0 0.127630650997162 62.0
264.63435244 255.174911499023 0.0
0 -0.204591035842896 1.0
0 -0.011208176612854 2.0
0 0.333873331546783 3.0
0 -0.0432484149932861 4.0
0 0.369568884372711 5.0
0 0.268356382846832 6.0
0 0.194102764129639 7.0
0 0.0578192472457886 8.0
0 0.262088775634766 9.0
0 -0.160427808761597 10.0
0 -0.0479198098182678 11.0
0 -0.0926105976104736 12.0
51.7203565 51.640926361084 13.0
44.075659543 38.8652877807617 14.0
0 -0.114807724952698 15.0
0 -0.0342733860015869 16.0
39.335482198 39.1072959899902 17.0
0 0.204018712043762 18.0
0 -0.0517027974128723 19.0
0 0.0634944438934326 20.0
0 0.100730717182159 21.0
0 0.141922533512115 22.0
31.898294025 29.7926998138428 23.0
0 0.0200423002243042 24.0
0 0.48113802075386 25.0
0 -0.0939448475837708 26.0
0 0.0298911929130554 27.0
0 0.0451474785804749 28.0
0 0.00761151313781738 29.0
0 0.0191522836685181 30.0
0 -0.0500234365463257 31.0
0 0.0141098499298096 32.0
0 -0.0907778739929199 33.0
0 0.390993773937225 34.0
0 -0.071340799331665 35.0
0 -0.110959529876709 36.0
0 -0.0775054693222046 37.0
0 0.261129856109619 38.0
0 -0.226927995681763 39.0
0 0.122722506523132 40.0
0 -0.203788101673126 41.0
0 -0.0267856121063232 42.0
297.74596701 296.9072265625 43.0
0 0.0364309549331665 44.0
0 0.100517630577087 45.0
0 0.20117849111557 46.0
0 -0.149320483207703 47.0
0 0.347592532634735 48.0
0 0.352299422025681 49.0
0.083566976261 0.0674738883972168 50.0
1.9863997017 1.8592381477356 51.0
0 -0.0222967863082886 52.0
0 0.0668456554412842 53.0
0 0.190647125244141 54.0
0 0.224577724933624 55.0
0 0.155977368354797 56.0
0 0.00397396087646484 57.0
0 -0.143037736415863 58.0
0 0.161908805370331 59.0
0 -0.10692572593689 60.0
0 0.14782726764679 61.0
0 0.19021999835968 62.0
237.77470758 251.533920288086 0.0
0 -0.28704559803009 1.0
0 -0.347178339958191 2.0
0 0.454869538545609 3.0
0 -0.312987446784973 4.0
0 0.201742708683014 5.0
0 0.0234856009483337 6.0
0 -0.16629958152771 7.0
0 0.351516515016556 8.0
0 0.383262574672699 9.0
0 0.274029076099396 10.0
0 0.67865788936615 11.0
0 -0.0711565613746643 12.0
75.145997498 73.2988815307617 13.0
23.448961281 28.9304733276367 14.0
0 -0.23410177230835 15.0
0 -0.386087536811829 16.0
30.32079587 33.080078125 17.0
0 0.0128230452537537 18.0
0 0.0310477018356323 19.0
0 0.251935124397278 20.0
0 -0.0341253280639648 21.0
0 0.115484118461609 22.0
41.2078355 39.6521797180176 23.0
0 0.0867739915847778 24.0
0 0.555802762508392 25.0
0 -0.150696396827698 26.0
0 -0.15112030506134 27.0
0 -0.176999092102051 28.0
0 -0.0965744853019714 29.0
0 -0.0621288418769836 30.0
0 0.056464672088623 31.0
0 -0.248348593711853 32.0
0 -0.102162897586823 33.0
0 0.629411935806274 34.0
0 0.0818307995796204 35.0
0 0.540139436721802 36.0
0 0.130665838718414 37.0
0 0.303175628185272 38.0
0 -0.19764244556427 39.0
0 0.623749017715454 40.0
0 -0.205335795879364 41.0
0 0.230995178222656 42.0
277.78384492 297.609558105469 43.0
0 0.509251475334167 44.0
0 -1.13672399520874 45.0
0 0.246599316596985 46.0
0 -0.302040338516235 47.0
0 0.0468569397926331 48.0
0 0.294994413852692 49.0
0.098542726474 0.070384681224823 50.0
1.3397887989 1.83022892475128 51.0
0 -0.0950324535369873 52.0
0 0.241098344326019 53.0
0 -0.100926995277405 54.0
0 0.137913227081299 55.0
0 0.0789888501167297 56.0
0 0.595206499099731 57.0
0 0.445643931627274 58.0
0 0.181008696556091 59.0
0 0.262701153755188 60.0
0 0.12066239118576 61.0
0 0.332040786743164 62.0
261.69516383 254.87174987793 0.0
0 -0.385035753250122 1.0
0 -0.06026691198349 2.0
0 0.295332491397858 3.0
0 -0.167237460613251 4.0
0 -0.0888983011245728 5.0
0 0.0579781532287598 6.0
0 0.187881290912628 7.0
0 -0.139595985412598 8.0
0 0.0523529052734375 9.0
0 -0.0310872793197632 10.0
0 -0.162200927734375 11.0
0 -0.0139122009277344 12.0
75.495362196 76.2017669677734 13.0
12.579817416 21.3077640533447 14.0
0 -0.139753699302673 15.0
0 -0.000856935977935791 16.0
29.229829856 32.0763893127441 17.0
0 0.0851331949234009 18.0
0 -0.00515294075012207 19.0
0 0.117181301116943 20.0
0 0.218122482299805 21.0
0 -0.109011888504028 22.0
27.969855091 25.3145122528076 23.0
0 -0.0155608654022217 24.0
0 -0.255943655967712 25.0
0 -0.163838028907776 26.0
0 -0.361016273498535 27.0
0 0.744383454322815 28.0
0 0.312870651483536 29.0
0 0.145748913288116 30.0
0 0.0798180103302002 31.0
0 -0.00796139240264893 32.0
0 -0.130388021469116 33.0
0 0.426911532878876 34.0
0 -0.0410733819007874 35.0
0 -0.130264401435852 36.0
0 -0.0620538592338562 37.0
0 0.108955144882202 38.0
0 -0.116443157196045 39.0
0 0.498529106378555 40.0
0 -0.149809777736664 41.0
0 0.16829240322113 42.0
292.41731407 297.424560546875 43.0
0 0.0531043410301208 44.0
0 0.0727031826972961 45.0
0 0.159538567066193 46.0
0 -0.176842033863068 47.0
0 0.0585072636604309 48.0
0 0.394988775253296 49.0
0.070407563472 0.0407489538192749 50.0
2.1042127785 1.90327715873718 51.0
0 0.31402450799942 52.0
0 0.0123782753944397 53.0
0 -0.0805288553237915 54.0
0 -0.203560590744019 55.0
0 -0.0438554286956787 56.0
0 -0.0451353788375854 57.0
0 0.220731616020203 58.0
0 0.166157960891724 59.0
0 -0.230832934379578 60.0
0 0.114889740943909 61.0
0 0.295298457145691 62.0
251.95961803 251.508285522461 0.0
0 -0.388798236846924 1.0
0 -0.0648348927497864 2.0
0 0.181623041629791 3.0
0 -0.139648079872131 4.0
0 0.24814361333847 5.0
0 0.0994513034820557 6.0
0 0.147667288780212 7.0
0 0.0456565618515015 8.0
0 0.517270922660828 9.0
0 0.194409549236298 10.0
0 -0.585973620414734 11.0
0 -0.170440971851349 12.0
70.956213442 71.5988235473633 13.0
28.192665005 22.1056957244873 14.0
0 -0.134484887123108 15.0
0 -0.116581559181213 16.0
25.295494207 31.1535530090332 17.0
0 0.0212403535842896 18.0
0 0.0315120816230774 19.0
0 0.0905154943466187 20.0
0 -0.0804030895233154 21.0
0 0.609169483184814 22.0
42.260446544 41.3751602172852 23.0
0 0.0611151456832886 24.0
0 0.806045293807983 25.0
0 0.0102249383926392 26.0
0 0.0586214661598206 27.0
0 0.476209044456482 28.0
0 -0.0773091912269592 29.0
0 0.0351245999336243 30.0
0 0.498802900314331 31.0
0 -0.0532263517379761 32.0
0 -0.126677811145782 33.0
0 0.342738956212997 34.0
0 -0.112414419651031 35.0
0 0.363947927951813 36.0
0 0.146473228931427 37.0
0 0.330121785402298 38.0
0 -0.248245239257812 39.0
0 0.2978835105896 40.0
0 -0.173459410667419 41.0
0 0.0349005460739136 42.0
294.51108639 295.136199951172 43.0
0 0.155124604701996 44.0
0 0.294992089271545 45.0
0 0.121163725852966 46.0
0 0.104757845401764 47.0
0 -0.376412153244019 48.0
0 0.376989722251892 49.0
0.094241441293 0.0256409645080566 50.0
2.1252169638 1.82020998001099 51.0
0 0.2223801612854 52.0
0 -0.0015869140625 53.0
0 -0.0300689339637756 54.0
0 0.107083380222321 55.0
0 0.262365639209747 56.0
0 0.192828059196472 57.0
0 0.141093790531158 58.0
0 -0.0394407510757446 59.0
0 -0.0475475192070007 60.0
0 0.0129565000534058 61.0
0 0.0682439208030701 62.0
226.65796938 252.599945068359 0.0
0 0.192398190498352 1.0
0 -0.153937876224518 2.0
0 -0.000403761863708496 3.0
0 -0.388060331344604 4.0
0 -0.27254581451416 5.0
0 -0.027621865272522 6.0
0 -0.242877721786499 7.0
0 0.115555882453918 8.0
0 -0.384076714515686 9.0
0 0.0430812835693359 10.0
0 -0.133520901203156 11.0
0 0.0430529713630676 12.0
63.560308703 63.4637908935547 13.0
42.924527383 37.8294906616211 14.0
0 -0.0252370834350586 15.0
0 -0.225185394287109 16.0
31.009641071 32.2507095336914 17.0
0 0.0939943790435791 18.0
0 0.0174388885498047 19.0
0 0.158950984477997 20.0
0 -0.14112913608551 21.0
0 0.133027136325836 22.0
38.335664262 36.9668121337891 23.0
0 0.0352588295936584 24.0
0 0.227246046066284 25.0
0 -0.00990915298461914 26.0
0 0.188322603702545 27.0
0 0.299953311681747 28.0
0 -0.0868536233901978 29.0
0 -0.0029672384262085 30.0
0 0.147693395614624 31.0
0 0.261248409748077 32.0
0 0.103232741355896 33.0
0 0.0748318433761597 34.0
0 0.0315678119659424 35.0
0 -0.219063520431519 36.0
0 0.0357885360717773 37.0
0 0.147873938083649 38.0
0 -0.26392924785614 39.0
0 0.411318480968475 40.0
0 -0.308499932289124 41.0
0 0.0912744998931885 42.0
288.52182452 294.920318603516 43.0
0 0.119640171527863 44.0
0 0.00267845392227173 45.0
0 0.0640754699707031 46.0
0 0.0317301154136658 47.0
0 0.406528741121292 48.0
0.26651186107 0.470117568969727 49.0
0.093873115245 0.11108672618866 50.0
2.1969197565 1.84388220310211 51.0
0 0.0423828363418579 52.0
0 -0.206170201301575 53.0
0 0.229766964912415 54.0
0 0.191944181919098 55.0
0 0.272177278995514 56.0
0 0.370985984802246 57.0
0 0.160524785518646 58.0
0 0.103585839271545 59.0
0 -0.108440101146698 60.0
0 -0.0476011633872986 61.0
0 -0.0190489888191223 62.0
237.51783235 250.890090942383 0.0
0 -0.175535917282104 1.0
0 -0.0704799890518188 2.0
0 0.0126615166664124 3.0
0 -0.145752906799316 4.0
0 -0.331833958625793 5.0
0 -0.0705832839012146 6.0
0 -0.183134078979492 7.0
0 0.125438690185547 8.0
0 0.240066230297089 9.0
0 0.318366885185242 10.0
0 0.0290533304214478 11.0
0 -0.173073530197144 12.0
64.927611907 65.0175704956055 13.0
36.307376942 31.4581356048584 14.0
0 -0.000430107116699219 15.0
0 -0.0828365683555603 16.0
25.120222883 35.399169921875 17.0
0 0.0512141585350037 18.0
0 -0.0613200664520264 19.0
0 0.14074182510376 20.0
0 -0.0201750993728638 21.0
0 -0.0967262387275696 22.0
40.689263114 39.1364860534668 23.0
0 0.0416368842124939 24.0
0 0.830931067466736 25.0
0 -0.162160754203796 26.0
0 0.182742416858673 27.0
0 -0.212271213531494 28.0
0 -0.119555234909058 29.0
0 0.0705660581588745 30.0
0 0.377890467643738 31.0
0 0.265394389629364 32.0
0 0.307599782943726 33.0
0 0.0665897130966187 34.0
0 -0.155342936515808 35.0
0 -0.367302894592285 36.0
0 -0.0242460966110229 37.0
0 0.232700884342194 38.0
0 -0.370120048522949 39.0
0 0.0751140117645264 40.0
0 -0.232319593429565 41.0
0 0.160147070884705 42.0
295.37218523 300.417846679688 43.0
0 -0.0727107524871826 44.0
0 0.239329695701599 45.0
0 0.290549695491791 46.0
0 -0.0589781403541565 47.0
0 0.342243641614914 48.0
2.6238521576 0.242298722267151 49.0
0.082208944562 0.0588414669036865 50.0
2.3178064489 1.87391424179077 51.0
0 0.0471894145011902 52.0
0 0.0694586038589478 53.0
0 -0.093663215637207 54.0
0 -0.0235962867736816 55.0
0 0.34501975774765 56.0
0 0.142347395420074 57.0
0 0.459755092859268 58.0
0 0.384839355945587 59.0
0 -0.164589881896973 60.0
0 0.197826564311981 61.0
0 0.398495733737946 62.0
232.93067687 251.339691162109 0.0
0 -0.159353494644165 1.0
0 -0.0904754996299744 2.0
0 0.264418482780457 3.0
0 -0.374918460845947 4.0
0 -0.0100433230400085 5.0
0 0.117105305194855 6.0
0 -0.135726988315582 7.0
0 -0.141960859298706 8.0
0 0.070862352848053 9.0
0 -0.134738326072693 10.0
0 -0.221060752868652 11.0
0 -0.0998982191085815 12.0
67.076604449 66.1557693481445 13.0
27.355363573 32.8630332946777 14.0
0 -0.0188822746276855 15.0
0 -0.133263826370239 16.0
27.127755128 31.5146827697754 17.0
0 0.0555703639984131 18.0
0 0.00850284099578857 19.0
0 0.121896862983704 20.0
0 -0.00818383693695068 21.0
0 0.160795211791992 22.0
38.906369125 38.7517547607422 23.0
0 0.0254365801811218 24.0
0 0.498893439769745 25.0
0 -0.171489536762238 26.0
0 0.0381909012794495 27.0
0 0.773031949996948 28.0
0 0.117889523506165 29.0
0 0.102407038211823 30.0
0 0.0911589860916138 31.0
0 0.0742377042770386 32.0
0 0.0549805760383606 33.0
0 0.202968835830688 34.0
0 -0.147384226322174 35.0
0 0.0522512197494507 36.0
0 -0.0609776973724365 37.0
0 0.190017461776733 38.0
0 -0.253197312355042 39.0
0 0.299298167228699 40.0
0 -0.193095147609711 41.0
0 0.072641134262085 42.0
293.22306812 298.132232666016 43.0
0 0.145037949085236 44.0
0 0.156565189361572 45.0
0 0.088586151599884 46.0
0 0.168348252773285 47.0
0 -0.00572741031646729 48.0
0 0.395819187164307 49.0
0.069960196202 0.0114821791648865 50.0
1.6420656098 1.8717919588089 51.0
0 -0.0506342649459839 52.0
0 -0.14445698261261 53.0
0 0.102690041065216 54.0
0 0.112985908985138 55.0
0 -0.0881967544555664 56.0
0 0.467147648334503 57.0
0 0.20361453294754 58.0
0 0.269083797931671 59.0
0 0.21008574962616 60.0
0 0.0964576005935669 61.0
0 0.178192794322968 62.0
262.67040224 254.41943359375 0.0
0 0.0637705326080322 1.0
0 -0.13888680934906 2.0
0 0.319262266159058 3.0
0 -0.19195830821991 4.0
0 0.0494945049285889 5.0
0 -0.0207539200782776 6.0
0 0.220899224281311 7.0
0 -0.1453458070755 8.0
0 -0.0861016511917114 9.0
0 0.249750852584839 10.0
0 -0.235608339309692 11.0
0 0.030771791934967 12.0
74.541289994 72.3114700317383 13.0
12.444886739 30.464054107666 14.0
0 -0.0816735029220581 15.0
0 0.480055630207062 16.0
36.371880693 34.3831977844238 17.0
0 0.0843917727470398 18.0
0 -0.0446795225143433 19.0
0 0.0699226856231689 20.0
0 -0.011605441570282 21.0
0 0.294189810752869 22.0
29.040941702 29.5093898773193 23.0
0 0.0632917881011963 24.0
0 0.586504101753235 25.0
0 -0.237098693847656 26.0
0 -0.0539892315864563 27.0
0 0.281794965267181 28.0
0 0.150561571121216 29.0
0 0.0752918720245361 30.0
0 -0.104688048362732 31.0
0 0.0175129771232605 32.0
0 0.0526056289672852 33.0
0 0.303025245666504 34.0
0 0.299399763345718 35.0
0 0.381110966205597 36.0
0 0.0302452445030212 37.0
0 -0.306577205657959 38.0
0 -0.0428266525268555 39.0
0 0.176805257797241 40.0
0 -0.136242985725403 41.0
0 0.154611706733704 42.0
284.53842756 296.666656494141 43.0
0 -0.055997908115387 44.0
0 0.127987563610077 45.0
0 0.280347764492035 46.0
0 -0.194568693637848 47.0
0 -0.407328128814697 48.0
0 0.504574537277222 49.0
0.096328519372 -0.000117719173431396 50.0
2.3150551281 1.83549690246582 51.0
0 0.0801659822463989 52.0
0 -0.113940417766571 53.0
0 -0.0415483713150024 54.0
0 0.158183336257935 55.0
0 -0.0406900644302368 56.0
0 0.741875767707825 57.0
0 -0.00621485710144043 58.0
0 0.216513156890869 59.0
0 -0.268771886825562 60.0
0 0.0757935643196106 61.0
0 0.202570974826813 62.0
264.75550282 254.134506225586 0.0
0 -0.258825302124023 1.0
0 -0.00674837827682495 2.0
0 0.388284891843796 3.0
0 -0.00670480728149414 4.0
0 0.145931601524353 5.0
0 -0.13567453622818 6.0
0 0.377161175012589 7.0
0 -0.0408975481987 8.0
0 0.1734818816185 9.0
0 -0.421682238578796 10.0
0 -0.27585232257843 11.0
0 -0.106759965419769 12.0
54.01039412 52.7925796508789 13.0
38.664463813 37.2732048034668 14.0
0 -0.233994245529175 15.0
0 0.0551955103874207 16.0
40.095653999 38.9710845947266 17.0
0 0.110097050666809 18.0
0 0.0111362934112549 19.0
0 0.0895763635635376 20.0
0 -0.0811716318130493 21.0
0 -0.0374631285667419 22.0
42.485921193 41.8326873779297 23.0
0 0.126095056533813 24.0
0 1.17411625385284 25.0
0 -0.0614287853240967 26.0
0 0.00662392377853394 27.0
0 0.780888259410858 28.0
0 0.016304075717926 29.0
0 -0.0441504120826721 30.0
0 0.222535252571106 31.0
0 0.127315402030945 32.0
0 -0.049938976764679 33.0
0 0.339986324310303 34.0
0 0.0152584910392761 35.0
0 -0.280405521392822 36.0
0 0.0943329930305481 37.0
0 0.123919665813446 38.0
0 -0.0382575988769531 39.0
0 0.44949135184288 40.0
0 -0.121812582015991 41.0
0 0.212393760681152 42.0
293.9107329 296.211730957031 43.0
0 0.0611268877983093 44.0
0 0.149809598922729 45.0
0 0.176007866859436 46.0
0 0.140145659446716 47.0
0 0.282659888267517 48.0
0 0.416353523731232 49.0
0.077491245465 0.0405721664428711 50.0
2.072437679 1.82616305351257 51.0
0 0.0298457145690918 52.0
0 -0.176896333694458 53.0
0 -0.0528210997581482 54.0
0 0.288740813732147 55.0
0 0.19723117351532 56.0
0 -0.020478367805481 57.0
0 0.26012772321701 58.0
0 -0.662317872047424 59.0
0 0.329729586839676 60.0
0 0.0342485308647156 61.0
0 0.192629039287567 62.0
280.28484605 251.863632202148 0.0
0 0.214729785919189 1.0
0 -0.105870962142944 2.0
0 0.225192487239838 3.0
0 0.0337871313095093 4.0
0 -0.181562483310699 5.0
0 0.0344232320785522 6.0
0 0.29617777466774 7.0
0 -0.369571447372437 8.0
0 0.0114203095436096 9.0
0 -0.114769518375397 10.0
0 0.020143985748291 11.0
0 -0.150167882442474 12.0
50.998664849 52.4557418823242 13.0
41.999509436 44.0202980041504 14.0
0 -0.107764482498169 15.0
0 0.0542416572570801 16.0
32.276948366 33.6918067932129 17.0
0 0.126890897750854 18.0
0 0.0123410224914551 19.0
0 0.130040466785431 20.0
0 -0.0348265767097473 21.0
0 0.0531489849090576 22.0
39.279484566 37.1735763549805 23.0
0 0.0275512933731079 24.0
0 0.185447990894318 25.0
0 0.0717434883117676 26.0
0 -0.0891528725624084 27.0
0 0.0642353892326355 28.0
0 -0.138573467731476 29.0
0 0.0931071639060974 30.0
0 0.154467701911926 31.0
0 0.128085136413574 32.0
0 0.0199925303459167 33.0
0 -0.00115221738815308 34.0
0 0.139415323734283 35.0
0 -0.014337420463562 36.0
0 0.1042720079422 37.0
0 0.143386602401733 38.0
0 -0.196616291999817 39.0
0 0.27919590473175 40.0
0 -0.180894196033478 41.0
0 0.125445365905762 42.0
299.6503347 296.6123046875 43.0
0 0.0797407031059265 44.0
0 0.125605821609497 45.0
0 0.156025230884552 46.0
0 0.215170741081238 47.0
0 0.36500671505928 48.0
0 0.422336399555206 49.0
0.069007167321 0.0323127508163452 50.0
1.9052116522 1.85782551765442 51.0
0 0.0208826661109924 52.0
0 -0.217532396316528 53.0
0 0.184194445610046 54.0
0 0.195034503936768 55.0
0 0.252844333648682 56.0
0 0.477769911289215 57.0
0 0.137141048908234 58.0
0 0.114197134971619 59.0
0 -0.134096801280975 60.0
0 0.0606371760368347 61.0
0 0.151621758937836 62.0
217.56839131 251.452484130859 0.0
0 0.457677483558655 1.0
0 -0.102338671684265 2.0
0 -0.068063497543335 3.0
0 -0.422348856925964 4.0
0 -0.565986633300781 5.0
0 0.219373762607574 6.0
0 -0.13693630695343 7.0
0 0.158391892910004 8.0
0 0.38992828130722 9.0
0 -0.079367458820343 10.0
0 -0.429458022117615 11.0
0 0.0729546546936035 12.0
67.224249853 69.3337936401367 13.0
30.493504799 26.8639621734619 14.0
0 0.0147614479064941 15.0
0 -0.20949649810791 16.0
30.959584783 33.4415283203125 17.0
0 0.0737640857696533 18.0
0 -0.03961181640625 19.0
0 0.0256487131118774 20.0
0 0.127341866493225 21.0
0 0.142133116722107 22.0
33.288480266 32.7789115905762 23.0
0 0.103822588920593 24.0
0 0.353864550590515 25.0
0 -0.0199066400527954 26.0
0 -0.0513013601303101 27.0
0 0.459519803524017 28.0
0 -0.117914259433746 29.0
0 0.0678218603134155 30.0
0 -0.448748588562012 31.0
0 -0.0746772885322571 32.0
0 -0.288923740386963 33.0
0 0.26264613866806 34.0
0 0.223235130310059 35.0
0 0.379082769155502 36.0
0 -0.0185278654098511 37.0
0 0.512317776679993 38.0
0 -0.41058874130249 39.0
0 0.091492235660553 40.0
0 -0.188973069190979 41.0
0 0.0332690477371216 42.0
302.25549224 288.905029296875 43.0
0 0.114442348480225 44.0
0 -0.102111399173737 45.0
0 0.221435368061066 46.0
0 -0.271355390548706 47.0
0 0.301784604787827 48.0
0 0.407741844654083 49.0
0.10002265775 0.0434829592704773 50.0
1.7015037391 1.91796028614044 51.0
0 -0.0287132859230042 52.0
0 0.39596751332283 53.0
0 -0.0110913515090942 54.0
0 -0.0852344036102295 55.0
0 -0.107079565525055 56.0
0 0.0986998677253723 57.0
0 0.40920102596283 58.0
0 0.142820000648499 59.0
0 -0.259797692298889 60.0
0 0.112703025341034 61.0
0 0.201751887798309 62.0
259.34352174 252.493270874023 0.0
0 -0.0132699012756348 1.0
0 -0.185335516929626 2.0
0 0.0845809578895569 3.0
0 -0.155883312225342 4.0
0 0.331628262996674 5.0
0 0.268708407878876 6.0
0 0.177422881126404 7.0
0 0.0585553646087646 8.0
0 0.0819125175476074 9.0
0 -0.101508975028992 10.0
0 -0.0365463495254517 11.0
0 0.0593639612197876 12.0
64.870012892 65.9142227172852 13.0
27.012132241 35.573070526123 14.0
0 -0.0289225578308105 15.0
0 -0.0852499604225159 16.0
28.272072668 31.4591045379639 17.0
0 0.0639113187789917 18.0
0 0.0101629495620728 19.0
0 0.038257360458374 20.0
0 0.236760258674622 21.0
0 0.17952162027359 22.0
31.273759965 27.6154003143311 23.0
0 -0.00511085987091064 24.0
0 0.346850454807281 25.0
0 -0.00739169120788574 26.0
0 -0.125923216342926 27.0
0 0.425260961055756 28.0
0 0.0794956684112549 29.0
0 -0.0243081450462341 30.0
0 -0.173621892929077 31.0
0 -0.092106819152832 32.0
0 0.0164132118225098 33.0
0 0.405224680900574 34.0
0 0.267536044120789 35.0
0 0.390239506959915 36.0
0 -0.0720542669296265 37.0
0 0.198732376098633 38.0
0 -0.302251935005188 39.0
0 0.176978290081024 40.0
0 -0.183323800563812 41.0
0 0.0710195302963257 42.0
307.11549337 297.516571044922 43.0
0 0.344422578811646 44.0
0 0.108264684677124 45.0
0 0.143285870552063 46.0
0 -0.0348952412605286 47.0
0 0.029838502407074 48.0
0 0.504216194152832 49.0
0.095771582441 0.0256984829902649 50.0
2.1251155727 1.9088294506073 51.0
0 0.383715450763702 52.0
0 -0.0733965039253235 53.0
0 -0.143663704395294 54.0
0 -0.0682474970817566 55.0
0 0.0742142200469971 56.0
0 0.440165519714355 57.0
0 0.150891363620758 58.0
0 -0.208962917327881 59.0
0 -0.0402641296386719 60.0
0 0.0237186551094055 61.0
0 0.157488822937012 62.0
239.78732382 252.642822265625 0.0
0 -0.0633134841918945 1.0
0 -0.179300308227539 2.0
0 0.198452174663544 3.0
0 -0.21856427192688 4.0
0 0.0696277618408203 5.0
0 0.120252013206482 6.0
0 -0.0697214603424072 7.0
0 0.287905216217041 8.0
0 0.117620646953583 9.0
0 0.00462764501571655 10.0
0 -0.0309438705444336 11.0
0 -0.0463208556175232 12.0
61.200613139 59.7199859619141 13.0
39.764265483 34.2009315490723 14.0
0 0.0269511938095093 15.0
0 -0.03502357006073 16.0
34.274509081 32.2621994018555 17.0
0 0.10386860370636 18.0
0 -0.0670250654220581 19.0
0 0.114897489547729 20.0
0 0.0734279155731201 21.0
0 0.15224015712738 22.0
30.347505698 28.3593769073486 23.0
0 -0.00166779756546021 24.0
0 -0.308582663536072 25.0
0 -0.0361872315406799 26.0
0 -0.0622416734695435 27.0
0 0.142091751098633 28.0
0 -0.0644872784614563 29.0
0 0.116452753543854 30.0
0 0.0118967294692993 31.0
0 0.0119661092758179 32.0
0 0.010919988155365 33.0
0 0.260284543037415 34.0
0 0.184421122074127 35.0
0 0.373772710561752 36.0
0 -0.0943198800086975 37.0
0 0.449392586946487 38.0
0 -0.252809882164001 39.0
0 -0.0917902588844299 40.0
0 -0.24450159072876 41.0
0 -0.0368086099624634 42.0
290.0538782 298.153411865234 43.0
0 0.10194456577301 44.0
0 0.100218713283539 45.0
0 0.164244771003723 46.0
0 0.0720865726470947 47.0
0 0.196991205215454 48.0
0 0.350204974412918 49.0
0.072487331402 0.0870563983917236 50.0
1.8163158111 1.79109191894531 51.0
0 0.00926917791366577 52.0
0 -0.0337993502616882 53.0
0 -0.0442018508911133 54.0
0 0.033752977848053 55.0
0 0.192788124084473 56.0
0 0.0609061717987061 57.0
0 0.195568680763245 58.0
0 0.286643862724304 59.0
0 -0.199485957622528 60.0
0 0.100774526596069 61.0
0 0.194467127323151 62.0
230.47627669 251.559921264648 0.0
0 -0.107303023338318 1.0
0 -0.069841742515564 2.0
0 -0.195285618305206 3.0
0 -0.276247024536133 4.0
0 -0.0641078948974609 5.0
0 -0.361817598342896 6.0
0 -0.0755249261856079 7.0
0 0.316107213497162 8.0
0 -0.330665946006775 9.0
0 0.560859143733978 10.0
0 -0.259043335914612 11.0
0 -0.209583520889282 12.0
65.137025965 64.19482421875 13.0
23.248803719 34.4926109313965 14.0
0 0.036620020866394 15.0
0 0.0763190984725952 16.0
38.266280008 37.5029296875 17.0
0 0.1469686627388 18.0
0 0.000948071479797363 19.0
0 -0.0195894837379456 20.0
0 0.229401826858521 21.0
0 -0.125314772129059 22.0
28.494969988 28.6506862640381 23.0
0 0.0608029961585999 24.0
0 0.741720259189606 25.0
0 -0.00720703601837158 26.0
0 -0.165009319782257 27.0
0 -0.176470220088959 28.0
0 0.02701735496521 29.0
0 0.0765379071235657 30.0
0 0.0337355136871338 31.0
0 0.0321084856987 32.0
0 0.0307092666625977 33.0
0 0.293430656194687 34.0
0 -0.0508679151535034 35.0
0 0.187626361846924 36.0
0 -0.090968132019043 37.0
0 0.215925335884094 38.0
0 -0.284552693367004 39.0
0 4.73260879516602e-05 40.0
0 -0.207964181900024 41.0
0 0.0151576995849609 42.0
292.29046354 295.965667724609 43.0
0 0.0903142690658569 44.0
0 0.120640218257904 45.0
0 0.179288804531097 46.0
0 -0.342785596847534 47.0
0 0.370675295591354 48.0
1.4625533864 0.562559247016907 49.0
0.087135632418 0.0651941895484924 50.0
2.2131491394 1.90785336494446 51.0
0 -0.169952392578125 52.0
0 -0.0787389874458313 53.0
0 -0.163083791732788 54.0
0 0.0809061527252197 55.0
0 0.319014489650726 56.0
0 0.12264496088028 57.0
0 0.19391006231308 58.0
0 0.263472437858582 59.0
0 -0.226224064826965 60.0
0 0.0687006711959839 61.0
0 0.247486472129822 62.0
249.38223153 252.30793762207 0.0
0 -0.0465285778045654 1.0
0 -0.066544771194458 2.0
0 0.0897117257118225 3.0
0 -0.433031797409058 4.0
0 -0.179438173770905 5.0
0 0.123368680477142 6.0
0 -0.0139980316162109 7.0
0 -0.00754475593566895 8.0
0 0.101079225540161 9.0
0 -0.355511426925659 10.0
0 -0.00226539373397827 11.0
0 -0.125300824642181 12.0
58.357653569 58.1917839050293 13.0
40.868829352 37.5262718200684 14.0
0 -0.158145308494568 15.0
0 0.0069928765296936 16.0
27.539115201 32.5827522277832 17.0
0 0.0832160115242004 18.0
0 0.0073782205581665 19.0
0 0.119416058063507 20.0
0 0.00841629505157471 21.0
0 0.262713015079498 22.0
32.431771665 32.6821098327637 23.0
0 0.00920569896697998 24.0
0 0.0877374410629272 25.0
0 0.00530993938446045 26.0
0 -0.0882970690727234 27.0
0 0.431628972291946 28.0
0 -0.0239295363426208 29.0
0 -0.0287973284721375 30.0
0 0.0553443431854248 31.0
0 0.0676661729812622 32.0
0 -0.0661376118659973 33.0
0 0.41739809513092 34.0
0 0.112967550754547 35.0
0 0.48181214928627 36.0
0 0.139314353466034 37.0
0 0.0206384062767029 38.0
0 -0.183483481407166 39.0
0 0.417024612426758 40.0
0 -0.0895894169807434 41.0
0 0.0159330368041992 42.0
305.68967484 296.255798339844 43.0
0 0.0860088467597961 44.0
0 0.142362236976624 45.0
0 0.131476759910583 46.0
0 0.327413320541382 47.0
0 0.154495120048523 48.0
0 0.419657230377197 49.0
0.091698441537 0.00774973630905151 50.0
1.525319138 1.89686226844788 51.0
0 -0.00715959072113037 52.0
0 -0.106621563434601 53.0
0 -0.0108233690261841 54.0
0 0.00186276435852051 55.0
0 0.0802187919616699 56.0
0 0.0823702812194824 57.0
0 0.173257291316986 58.0
0 0.0541547536849976 59.0
0 -0.18380343914032 60.0
0 -0.0499802231788635 61.0
0 0.0664281845092773 62.0
232.3109274 252.999145507812 0.0
0 -0.200241327285767 1.0
0 -0.112315893173218 2.0
0 0.233865737915039 3.0
0 -0.250340580940247 4.0
0 -0.109704673290253 5.0
0 0.0465686321258545 6.0
0 -0.344343543052673 7.0
0 0.177474021911621 8.0
0 0.139086484909058 9.0
0 0.0134796500205994 10.0
0 -0.276495218276978 11.0
0 -0.140382170677185 12.0
65.478391051 64.5418701171875 13.0
24.03198437 33.0099868774414 14.0
0 -0.0260833501815796 15.0
0 -0.16670960187912 16.0
30.692681519 32.0899505615234 17.0
0 0.109595894813538 18.0
0 -0.00840073823928833 19.0
0 0.108932375907898 20.0
0 0.117929816246033 21.0
0 0.112279176712036 22.0
34.26777897 32.5244750976562 23.0
0 0.0214240550994873 24.0
0 0.457379251718521 25.0
0 -0.147355914115906 26.0
0 -0.0384120345115662 27.0
0 -0.163412630558014 28.0
0 -0.057713508605957 29.0
0 -0.0257633924484253 30.0
0 0.0189130306243896 31.0
0 0.0673679709434509 32.0
0 -0.0755360722541809 33.0
0 0.305991560220718 34.0
0 0.167161524295807 35.0
0 -0.165780425071716 36.0
0 -0.0485367774963379 37.0
0 0.143372893333435 38.0
0 -0.141949415206909 39.0
0 0.407403349876404 40.0
0 -0.0405429005622864 41.0
0 -0.0284547805786133 42.0
298.62863694 295.961090087891 43.0
0 0.153948903083801 44.0
0 -0.017290472984314 45.0
0 0.178418397903442 46.0
0 0.0865573287010193 47.0
0 -0.0213279724121094 48.0
0 0.390226781368256 49.0
0.096521692643 -0.0233695507049561 50.0
2.2012805908 1.89995312690735 51.0
0 0.147835373878479 52.0
0 -0.267800211906433 53.0
0 -0.0955188274383545 54.0
0 0.0333781838417053 55.0
0 0.325076997280121 56.0
0 -0.210092544555664 57.0
0 -0.310160875320435 58.0
0 0.172094464302063 59.0
0 -0.226263523101807 60.0
0 0.117349147796631 61.0
0 0.156070947647095 62.0
272.13189377 253.558029174805 0.0
0 -0.38460099697113 1.0
0 -0.0583156943321228 2.0
0 0.0559560060501099 3.0
0 -0.224327325820923 4.0
0 0.360636532306671 5.0
0 0.398333728313446 6.0
0 0.376279175281525 7.0
0 -0.0913577079772949 8.0
0 0.463882625102997 9.0
0 -0.745505928993225 10.0
0 -0.0565899610519409 11.0
0 -0.0272393822669983 12.0
65.029524617 64.8640594482422 13.0
43.37505515 35.6976280212402 14.0
0 -0.0357290506362915 15.0
0 -0.0373771190643311 16.0
38.910189204 36.3210945129395 17.0
0 0.123967051506042 18.0
0 -0.0599439144134521 19.0
0 0.124362051486969 20.0
0 -0.130153059959412 21.0
0 0.395655870437622 22.0
39.104963414 40.2911033630371 23.0
0 0.11427515745163 24.0
0 0.703923642635345 25.0
0 -0.000355124473571777 26.0
0 -0.11192262172699 27.0
0 -0.0866749882698059 28.0
0 0.0943145155906677 29.0
0 0.0321697592735291 30.0
0 0.0321764945983887 31.0
0 0.129999995231628 32.0
0 -0.157618820667267 33.0
0 0.150414049625397 34.0
0 -0.101031899452209 35.0
0 0.228430509567261 36.0
0 0.00351822376251221 37.0
0 0.183934271335602 38.0
0 -0.430212736129761 39.0
0 -0.248676419258118 40.0
0 -0.312373757362366 41.0
0 -0.107364416122437 42.0
305.85476575 297.222778320312 43.0
0 0.0636293888092041 44.0
0 0.0843091011047363 45.0
0 0.122384071350098 46.0
0 0.242290675640106 47.0
0 0.107239484786987 48.0
0 0.517439126968384 49.0
0.075277228732 0.00849610567092896 50.0
1.6232026091 1.85078382492065 51.0
0 -0.279070734977722 52.0
0 0.163844227790833 53.0
0 -0.111478209495544 54.0
0 0.12111222743988 55.0
0 0.366643458604813 56.0
0 0.0718281269073486 57.0
0 0.145526051521301 58.0
0 -0.18005383014679 59.0
0 -0.00287830829620361 60.0
0 0.0521693825721741 61.0
0 0.200980603694916 62.0
237.52721954 252.481353759766 0.0
0 -0.259652018547058 1.0
0 0.0530769228935242 2.0
0 0.157031416893005 3.0
0 -0.227971434593201 4.0
0 -0.0694351196289062 5.0
0 -0.0709230899810791 6.0
0 -0.0292986035346985 7.0
0 0.120294690132141 8.0
0 0.492400765419006 9.0
0 0.011417031288147 10.0
0 -0.171523153781891 11.0
0 -0.0847791433334351 12.0
57.484695243 56.5455780029297 13.0
42.971066646 36.8670959472656 14.0
0 -0.0258568525314331 15.0
0 0.231810390949249 16.0
39.718516674 36.9148292541504 17.0
0 0.192069053649902 18.0
0 -0.0322578549385071 19.0
0 0.0384859442710876 20.0
0 0.157614469528198 21.0
0 0.0966780185699463 22.0
33.886835177 31.75026512146 23.0
0 -0.00296461582183838 24.0
0 0.541033029556274 25.0
0 0.0276114940643311 26.0
0 -0.107368886470795 27.0
0 0.404628247022629 28.0
0 -0.0629454851150513 29.0
0 0.0770427584648132 30.0
0 -0.11461067199707 31.0
0 -0.0179325342178345 32.0
0 -0.106441617012024 33.0
0 -0.207700371742249 34.0
0 0.534736633300781 35.0
0 -0.082398533821106 36.0
0 0.0193893909454346 37.0
0 0.204812526702881 38.0
0 -0.231432557106018 39.0
0 0.384841650724411 40.0
0 -0.148454427719116 41.0
0 0.00615799427032471 42.0
288.13440796 297.528656005859 43.0
0 0.215352177619934 44.0
0 0.0599451065063477 45.0
0 0.124538660049438 46.0
0 -0.286581039428711 47.0
0 0.366552621126175 48.0
0 0.43316924571991 49.0
0.082844985084 0.0551655292510986 50.0
2.2802321592 1.87028527259827 51.0
0 -0.116262793540955 52.0
0 -0.133488416671753 53.0
0 -0.12718403339386 54.0
0 0.0179710388183594 55.0
0 0.124623775482178 56.0
0 0.0374096035957336 57.0
0 0.240072727203369 58.0
0 -0.244370341300964 59.0
0 0.406894713640213 60.0
0 0.0560925006866455 61.0
0 0.127617299556732 62.0
246.05935244 254.294540405273 0.0
0 -0.265814661979675 1.0
0 -0.205263674259186 2.0
0 0.45621195435524 3.0
0 -0.0220011472702026 4.0
0 0.102552771568298 5.0
0 0.0278773307800293 6.0
0 0.0922907590866089 7.0
0 -0.0208828449249268 8.0
0 0.044705331325531 9.0
0 0.112829685211182 10.0
0 -0.142875254154205 11.0
0 -0.0509135127067566 12.0
60.648794993 60.0102500915527 13.0
30.297955667 38.2553520202637 14.0
0 -0.150112867355347 15.0
0 0.103768050670624 16.0
40.023682347 36.7810020446777 17.0
0 0.129781723022461 18.0
0 -0.0157099962234497 19.0
0 0.0227023363113403 20.0
0 0.100062489509583 21.0
0 0.211496472358704 22.0
31.872507403 31.4827671051025 23.0
0 0.112178325653076 24.0
0 1.27426314353943 25.0
0 -0.0196582078933716 26.0
0 -0.0784634351730347 27.0
0 0.390383273363113 28.0
0 0.0158736705780029 29.0
0 -0.132080852985382 30.0
0 0.141458749771118 31.0
0 0.0250499844551086 32.0
0 -0.0901403427124023 33.0
0 0.322275280952454 34.0
0 -0.0916356444358826 35.0
0 0.102770566940308 36.0
0 0.0919896364212036 37.0
0 -0.0103668570518494 38.0
0 -0.177573800086975 39.0
0 0.488457024097443 40.0
0 -0.166925370693207 41.0
0 0.103596448898315 42.0
293.72206789 297.112976074219 43.0
0 0.0849993228912354 44.0
0 0.152991414070129 45.0
0 0.148064851760864 46.0
0 -0.160378098487854 47.0
0 0.300306707620621 48.0
0 0.34256649017334 49.0
0.069468918632 0.0636987090110779 50.0
1.9536012898 1.86483561992645 51.0
0 -0.0236291885375977 52.0
0 -0.0309571027755737 53.0
0 0.0988488793373108 54.0
0 0.0275183916091919 55.0
0 -0.050821840763092 56.0
0 0.306383669376373 57.0
0 0.11904776096344 58.0
0 0.0212652087211609 59.0
0 0.0830934047698975 60.0
0 -0.0371325016021729 61.0
0 -0.0508320927619934 62.0
220.32226889 251.730239868164 0.0
0 0.204695582389832 1.0
0 -0.276285409927368 2.0
0 0.0501081347465515 3.0
0 -0.432857871055603 4.0
0 -0.354173302650452 5.0
0 0.110081851482391 6.0
0 -0.337243795394897 7.0
0 0.218362390995026 8.0
0 -0.37753164768219 9.0
0 0.273337304592133 10.0
0 0.0856490135192871 11.0
0 -0.230829954147339 12.0
48.320742735 46.1031112670898 13.0
56.588202567 50.8851089477539 14.0
0 0.228847861289978 15.0
0 0.0143376588821411 16.0
36.02255836 31.4529247283936 17.0
0 0.058346152305603 18.0
0 -0.062023401260376 19.0
0 0.19647091627121 20.0
0 -0.128408193588257 21.0
0 0.150212526321411 22.0
41.213348433 38.0468788146973 23.0
0 0.0584022402763367 24.0
0 0.671346604824066 25.0
0 0.0564112663269043 26.0
0 0.0166313648223877 27.0
0 0.152667939662933 28.0
0 -0.0362371206283569 29.0
0 0.0759257078170776 30.0
0 -0.313087105751038 31.0
0 -0.100666880607605 32.0
0 -0.22386646270752 33.0
0 0.435002714395523 34.0
0 0.920221745967865 35.0
0 0.355110645294189 36.0
0 -0.0788371562957764 37.0
0 0.445538461208344 38.0
0 -0.418517351150513 39.0
0 -0.197713375091553 40.0
0 -0.283040642738342 41.0
0 0.143609166145325 42.0
282.03177645 297.162170410156 43.0
0 0.0168057680130005 44.0
0 0.229083955287933 45.0
0 0.169616162776947 46.0
0 0.365018606185913 47.0
0 0.422475963830948 48.0
0 0.372205555438995 49.0
0.084078616469 0.0879401564598083 50.0
2.1349774065 1.81301152706146 51.0
0 0.0573737025260925 52.0
0 -0.00608938932418823 53.0
0 0.0513478517532349 54.0
0 0.208025991916656 55.0
0 0.0973811745643616 56.0
0 0.501381039619446 57.0
0 0.0550394058227539 58.0
0 0.263317823410034 59.0
0 -0.133650541305542 60.0
0 -0.0319591760635376 61.0
0 -0.0100768208503723 62.0
287.78096686 255.410614013672 0.0
0 -0.483260154724121 1.0
0 0.0763705968856812 2.0
0 0.316543340682983 3.0
0 0.0438297390937805 4.0
0 0.0226041674613953 5.0
0 0.480419993400574 6.0
0 0.646900057792664 7.0
0 0.460728228092194 8.0
0 -0.732466220855713 9.0
0 0.547926902770996 10.0
0 -0.445719361305237 11.0
0 -0.0485435724258423 12.0
73.753106042 73.0024185180664 13.0
20.9075625 22.3984966278076 14.0
0 -0.129946947097778 15.0
0 -0.192362666130066 16.0
28.153379787 34.8469390869141 17.0
0 0.0891001224517822 18.0
0 -0.0278662443161011 19.0
0 0.077216625213623 20.0
0 0.113917469978333 21.0
0 0.132318258285522 22.0
29.328173564 28.5081672668457 23.0
0 0.071302592754364 24.0
6.1047708807 0.883603811264038 25.0
0 -0.0125625133514404 26.0
0 -0.2290358543396 27.0
0 0.174275875091553 28.0
0 -0.0666654109954834 29.0
0 0.114012658596039 30.0
0 0.054987907409668 31.0
0 0.00881928205490112 32.0
0 -0.149488866329193 33.0
0 0.252516686916351 34.0
0 0.00282317399978638 35.0
0 0.302544981241226 36.0
0 -0.0888633131980896 37.0
0 0.109424591064453 38.0
0 -0.0901098251342773 39.0
0 0.35161417722702 40.0
0 -0.0772866010665894 41.0
0 0.0365338325500488 42.0
291.71875663 297.911865234375 43.0
0 0.0162164568901062 44.0
0 0.0735323429107666 45.0
0 0.297679543495178 46.0
0 -0.0249298214912415 47.0
0 -0.236975789070129 48.0
0 0.516735911369324 49.0
0.069292825703 -0.011574923992157 50.0
2.202704643 1.8635665178299 51.0
0 0.219625473022461 52.0
0 -0.278239369392395 53.0
0 -0.0396630764007568 54.0
0 0.190462052822113 55.0
0 0.0424277186393738 56.0
0 0.148214936256409 57.0
0 0.24425745010376 58.0
0 0.287779986858368 59.0
0 -0.124890565872192 60.0
0 0.00357437133789062 61.0
0 0.183944344520569 62.0
251.86766432 253.403228759766 0.0
0 -0.230432748794556 1.0
0 -0.0915099382400513 2.0
0 0.53636646270752 3.0
0 -0.176875472068787 4.0
0 -0.397814869880676 5.0
0 -0.0528384447097778 6.0
0 0.321785748004913 7.0
0 0.116549015045166 8.0
0 0.389255493879318 9.0
0 0.53197979927063 10.0
0 -0.0238958597183228 11.0
0 -0.149601578712463 12.0
78.150987325 76.2311859130859 13.0
7.9669064394 23.1540260314941 14.0
0 -0.0417585372924805 15.0
0 -0.026930034160614 16.0
38.452334072 33.6280174255371 17.0
0 0.0894529223442078 18.0
0 -0.00854170322418213 19.0
0 0.145344495773315 20.0
0 0.180649638175964 21.0
0 0.157152414321899 22.0
32.905687815 30.9896602630615 23.0
0 0.0836063027381897 24.0
0 1.12513947486877 25.0
0 0.188259601593018 26.0
0 -0.135380387306213 27.0
0 0.538752675056458 28.0
0 -0.131688356399536 29.0
0 0.117890179157257 30.0
0 0.0683399438858032 31.0
0 -0.00448298454284668 32.0
0 -0.0873046517372131 33.0
0 0.0414332151412964 34.0
0 -0.111814856529236 35.0
0 0.203537583351135 36.0
0 -0.094184935092926 37.0
0 -0.26496946811676 38.0
0 0.00934278964996338 39.0
0 0.527470886707306 40.0
0 -0.168203949928284 41.0
0 0.342203259468079 42.0
291.18965567 296.36474609375 43.0
0 -0.0857665538787842 44.0
0 0.121499240398407 45.0
0 0.220950365066528 46.0
0 -0.780620574951172 47.0
0 0.334963500499725 48.0
2.2963563198 0.635241150856018 49.0
0.074811963145 0.0935275554656982 50.0
1.46082485 1.88145709037781 51.0
0 -0.0640255808830261 52.0
0 -0.440845489501953 53.0
0 0.34942889213562 54.0
0 0.133370995521545 55.0
0 0.200210332870483 56.0
0 0.691149830818176 57.0
0 -0.08267742395401 58.0
0 0.271524906158447 59.0
0 -0.113028585910797 60.0
0 0.0944814682006836 61.0
0 0.065850555896759 62.0
271.67658987 244.804275512695 0.0
0 -0.327777147293091 1.0
0 0.105688333511353 2.0
0 0.0363378524780273 3.0
0 0.0421704053878784 4.0
0 -0.292969465255737 5.0
0 0.0625838041305542 6.0
0 0.35382354259491 7.0
0 -0.828455328941345 8.0
0 -0.530380606651306 9.0
0 -0.131470501422882 10.0
0 -0.0550597906112671 11.0
0 -0.12824547290802 12.0
77.972772209 78.0148315429688 13.0
39.491035465 19.3026027679443 14.0
0 -0.0825117826461792 15.0
0 0.378398656845093 16.0
36.071717401 35.2897453308105 17.0
0 0.066508412361145 18.0
0 -0.0546231269836426 19.0
0 0.0843045711517334 20.0
0 0.0665278434753418 21.0
0 0.193829894065857 22.0
29.54791832 28.7785816192627 23.0
0 0.0756761431694031 24.0
1.0647807124 0.521648645401001 25.0
0 -0.13311904668808 26.0
0 -0.127748310565948 27.0
0 0.00272774696350098 28.0
0 0.0530163049697876 29.0
0 -0.0640369057655334 30.0
0 -0.0351161956787109 31.0
0 -0.0253951549530029 32.0
0 -0.027010440826416 33.0
0 -0.0591323971748352 34.0
0 -0.202320098876953 35.0
0 0.177465081214905 36.0
0 -0.0334973335266113 37.0
0 0.0956854820251465 38.0
0 -0.227230429649353 39.0
0 0.0909441709518433 40.0
0 0.0158029794692993 41.0
0 -0.151516914367676 42.0
289.94363973 293.324401855469 43.0
0 -0.00441837310791016 44.0
0 0.0456637740135193 45.0
0 0.0599367618560791 46.0
0 -0.19629031419754 47.0
0 0.0465209484100342 48.0
0 0.391499131917953 49.0
0.099570813844 0.0797344446182251 50.0
2.3310354819 1.81741917133331 51.0
0 0.423382848501205 52.0
0 0.210638344287872 53.0
0 0.0478242635726929 54.0
0 -0.0187278389930725 55.0
0 0.314366638660431 56.0
0 -0.0637433528900146 57.0
0 -0.0972082614898682 58.0
0 0.227371633052826 59.0
0 -0.031585693359375 60.0
0 0.088506817817688 61.0
0 0.10600483417511 62.0
248.62534282 252.269332885742 0.0
0 0.0245195627212524 1.0
0 -0.19181090593338 2.0
0 0.245755732059479 3.0
0 -0.238478422164917 4.0
0 -0.0932453274726868 5.0
0 0.102711021900177 6.0
0 -0.362765312194824 7.0
0 -0.0470951795578003 8.0
0 0.129466652870178 9.0
0 -0.470452547073364 10.0
0 0.0909184217453003 11.0
0 -0.118858754634857 12.0
59.32989404 60.3506851196289 13.0
41.547820596 34.5429992675781 14.0
0 -0.021262526512146 15.0
0 0.0896791219711304 16.0
40.848720742 37.2934684753418 17.0
0 0.13380229473114 18.0
0 -0.0358273983001709 19.0
0 0.0325086116790771 20.0
0 0.278165638446808 21.0
0 -0.0167317390441895 22.0
29.915444503 28.4293117523193 23.0
0 0.079628586769104 24.0
0 0.90317577123642 25.0
0 -0.130789458751678 26.0
0 -0.0796828269958496 27.0
0 0.190357565879822 28.0
0 0.0323706269264221 29.0
0 0.0881094336509705 30.0
0 0.0951932668685913 31.0
0 0.124185919761658 32.0
0 -0.256747245788574 33.0
0 -0.131445825099945 34.0
0 1.34542417526245 35.0
0 0.37391585111618 36.0
0 0.16983163356781 37.0
0 -0.0920183062553406 38.0
0 -0.0146831274032593 39.0
0 -0.0338704586029053 40.0
0 -0.0791131854057312 41.0
0 -0.0125715732574463 42.0
289.04453648 293.074005126953 43.0
0 -0.0268110632896423 44.0
0 -0.0215208530426025 45.0
0 0.131649374961853 46.0
0 0.162997722625732 47.0
0 -0.467962503433228 48.0
0 0.3692946434021 49.0
0.082713070301 -0.0255317687988281 50.0
2.1216175769 1.84898221492767 51.0
0 0.0856004953384399 52.0
0 -0.0521030426025391 53.0
0 0.116501152515411 54.0
0 -0.085841178894043 55.0
0 -0.187271356582642 56.0
0 0.0562827587127686 57.0
0 0.396503806114197 58.0
0 -0.075581431388855 59.0
0 0.168592989444733 60.0
0 0.0943828821182251 61.0
0 0.290366530418396 62.0
286.63029923 254.718521118164 0.0
0 0.211971879005432 1.0
0 -0.102903425693512 2.0
0 0.0469082593917847 3.0
0 -0.0539753437042236 4.0
0 -0.161233305931091 5.0
0 -0.168362021446228 6.0
0 0.307178109884262 7.0
0 0.494699835777283 8.0
0 -0.0285723209381104 9.0
0 0.527281761169434 10.0
0 -0.364291667938232 11.0
0 -0.09064781665802 12.0
72.591744301 71.3132400512695 13.0
23.790355498 28.6104946136475 14.0
0 -0.0619301795959473 15.0
0 -0.0787304043769836 16.0
30.986996245 31.7303333282471 17.0
0 0.0596903562545776 18.0
0 -0.0723025798797607 19.0
0 0.141424000263214 20.0
0 0.0879967212677002 21.0
0 0.156771898269653 22.0
26.001174025 25.9941921234131 23.0
0 0.0520249605178833 24.0
0 0.4896380007267 25.0
0 -0.0224318504333496 26.0
0 -0.198835074901581 27.0
0 0.241462409496307 28.0
0 0.246987998485565 29.0
0 -0.0978432893753052 30.0
0 -0.0338771343231201 31.0
0 -0.217856168746948 32.0
0 -0.394549965858459 33.0
0 0.384882628917694 34.0
0 -0.0853624939918518 35.0
0 0.304515838623047 36.0
0 0.0175646543502808 37.0
0 -0.302044749259949 38.0
0 -0.126327514648438 39.0
0 0.416155964136124 40.0
0 -0.0961164832115173 41.0
0 0.195868492126465 42.0
303.19393774 293.317840576172 43.0
0 0.0435030460357666 44.0
0 0.198639631271362 45.0
0 0.230180203914642 46.0
0 -0.0347425937652588 47.0
0 0.0853647589683533 48.0
0 0.283953309059143 49.0
0.081068813634 -0.016441285610199 50.0
1.4305229253 1.88791799545288 51.0
0 0.0787426233291626 52.0
0 -0.356383323669434 53.0
0 -0.0966276526451111 54.0
0 -0.128093600273132 55.0
0 0.0523233413696289 56.0
0 0.121962428092957 57.0
0 0.281963706016541 58.0
0 -0.158242702484131 59.0
0 0.0443428754806519 60.0
0 0.0765357613563538 61.0
0 0.298492312431335 62.0
243.02479961 254.512237548828 0.0
0 0.0588870048522949 1.0
0 0.0938289761543274 2.0
0 0.0567429661750793 3.0
0 -0.101779580116272 4.0
0 0.0657602548599243 5.0
0 0.195728898048401 6.0
0 -0.20152473449707 7.0
0 0.234563827514648 8.0
0 0.200940787792206 9.0
0 1.0727926492691 10.0
0 -0.258529901504517 11.0
0 -0.0851030349731445 12.0
73.33205714 72.1750793457031 13.0
15.807538868 27.3564682006836 14.0
0 0.152462363243103 15.0
0 -0.181418597698212 16.0
36.673572127 34.4900283813477 17.0
0 0.125281155109406 18.0
0 -0.0569891929626465 19.0
0 0.121646881103516 20.0
0 0.146432816982269 21.0
0 0.0876829624176025 22.0
30.497911887 27.0186176300049 23.0
0 -0.0138730406761169 24.0
0 -0.287139296531677 25.0
0 0.0657849311828613 26.0
0 -0.160967707633972 27.0
0 0.303793966770172 28.0
0 0.119239568710327 29.0
0 -0.0325376987457275 30.0
0 0.0797003507614136 31.0
0 0.00571835041046143 32.0
0 -0.297989249229431 33.0
0 0.496582210063934 34.0
0 0.129881858825684 35.0
0 0.626384139060974 36.0
0 -0.112182557582855 37.0
0 0.507189631462097 38.0
0 -0.627670168876648 39.0
0 -0.13798576593399 40.0
0 -0.228103518486023 41.0
0 0.0998548269271851 42.0
308.90230742 295.375030517578 43.0
0 0.0354211330413818 44.0
0 0.133966326713562 45.0
0 0.163794040679932 46.0
0 0.215910255908966 47.0
0 -0.179637610912323 48.0
0 0.541562139987946 49.0
0.067390503591 -0.00961011648178101 50.0
1.5566688125 1.8338919878006 51.0
0 -0.171994566917419 52.0
0 0.13508540391922 53.0
0 0.0848225951194763 54.0
0 0.122762084007263 55.0
0 0.13958477973938 56.0
0 0.480861008167267 57.0
0 0.412561476230621 58.0
0 0.193732380867004 59.0
0 -0.185196042060852 60.0
0 0.0905430912971497 61.0
0 0.330292642116547 62.0
257.20001115 253.286331176758 0.0
0 -0.252927899360657 1.0
0 -0.134132266044617 2.0
0 -0.190913259983063 3.0
0 -0.198864996433258 4.0
0 0.106457352638245 5.0
0 0.077569305896759 6.0
0 0.00974166393280029 7.0
0 -0.452266454696655 8.0
0 0.226625084877014 9.0
0 -0.00369739532470703 10.0
0 0.0807309150695801 11.0
0 -0.237627744674683 12.0
77.431067014 78.3202514648438 13.0
34.231020978 24.1392135620117 14.0
0 0.0460407733917236 15.0
0 -0.181841194629669 16.0
29.417780872 32.4921760559082 17.0
0 0.187788009643555 18.0
0 0.0287047624588013 19.0
0 0.124619662761688 20.0
0 -0.00277304649353027 21.0
0 0.183341979980469 22.0
36.473307099 34.3532028198242 23.0
0 0.0717007517814636 24.0
0 0.617063045501709 25.0
0 -0.0905085802078247 26.0
0 -0.0376994609832764 27.0
0 0.474335610866547 28.0
0 0.108886539936066 29.0
0 0.153764963150024 30.0
0 0.0782651901245117 31.0
0 0.163642287254333 32.0
0 0.0115149617195129 33.0
0 0.21719753742218 34.0
0 -0.0940370559692383 35.0
0 0.0207587480545044 36.0
0 -0.0890077352523804 37.0
0 0.306650578975677 38.0
0 -0.273199319839478 39.0
0 0.303400903940201 40.0
0 -0.139448881149292 41.0
0 -0.112415194511414 42.0
297.67759516 296.836151123047 43.0
0 -0.0367516279220581 44.0
0 0.26033753156662 45.0
0 0.18286794424057 46.0
0 0.379693508148193 47.0
0 0.201163947582245 48.0
0 0.349895715713501 49.0
0.096062353499 0.0716468095779419 50.0
2.3217491514 1.90529847145081 51.0
0 -0.105127811431885 52.0
0 0.492518663406372 53.0
0 0.169659435749054 54.0
0 0.328903585672379 55.0
0 -0.0501772165298462 56.0
0 -0.0709015130996704 57.0
0 0.303555101156235 58.0
0 0.216389834880829 59.0
0 -0.150029540061951 60.0
0 0.0875090360641479 61.0
0 0.369508564472198 62.0
257.99846208 253.097366333008 0.0
0 -0.270077466964722 1.0
0 0.0426337718963623 2.0
0 0.253851592540741 3.0
0 -0.0494469404220581 4.0
0 0.280376553535461 5.0
0 -0.0243608951568604 6.0
0 0.0469966530799866 7.0
0 -0.11813759803772 8.0
0 0.0299206376075745 9.0
0 0.00834959745407104 10.0
0 -0.0450403690338135 11.0
0 -0.00168079137802124 12.0
55.399678833 57.1027565002441 13.0
53.889953377 44.2192726135254 14.0
0 -0.0821678638458252 15.0
0 -0.0977375507354736 16.0
35.892887873 36.2329864501953 17.0
0 0.102462768554688 18.0
0 0.00356113910675049 19.0
0 0.0759345889091492 20.0
0 -0.0308129787445068 21.0
0 0.238872468471527 22.0
39.067789787 38.9027557373047 23.0
0 0.0249589681625366 24.0
0 0.564779758453369 25.0
0 -0.181206107139587 26.0
0 -0.0096127986907959 27.0
0 0.051909863948822 28.0
0 -0.079664409160614 29.0
0 0.0808584094047546 30.0
0 0.0312414169311523 31.0
0 -0.0490073561668396 32.0
0 -0.192675888538361 33.0
0 0.215221226215363 34.0
0 0.0311852693557739 35.0
0 0.485547423362732 36.0
0 0.11202734708786 37.0
0 0.0716162919998169 38.0
0 -0.195092797279358 39.0
0 0.205483019351959 40.0
0 -0.120713531970978 41.0
0 0.0112038850784302 42.0
303.13362215 296.8525390625 43.0
0 0.0762896537780762 44.0
0 0.0308799743652344 45.0
0 0.328314006328583 46.0
0 -0.300271987915039 47.0
0 0.122770845890045 48.0
0 0.538361310958862 49.0
0.097882409403 0.00694113969802856 50.0
1.9692237301 1.81503748893738 51.0
0 0.0309690237045288 52.0
0 -0.107001781463623 53.0
0 0.145699620246887 54.0
0 0.149926662445068 55.0
0 0.0820038318634033 56.0
0 0.222650527954102 57.0
0 0.287448048591614 58.0
0 0.21514630317688 59.0
0 -0.139901340007782 60.0
0 0.108041703701019 61.0
0 0.259414851665497 62.0
221.18963464 251.99951171875 0.0
0 -0.122874617576599 1.0
0 -0.162885010242462 2.0
0 0.398787200450897 3.0
0 -0.274670958518982 4.0
0 -0.190498769283295 5.0
0 -0.0938681960105896 6.0
0 -0.192448377609253 7.0
0 0.322956323623657 8.0
0 -0.382909536361694 9.0
0 -0.171013295650482 10.0
0 0.0192437171936035 11.0
0 -0.114985406398773 12.0
72.297798706 70.7234725952148 13.0
23.612076894 29.7039947509766 14.0
0 -0.00797748565673828 15.0
0 0.31569629907608 16.0
36.469931345 38.4868850708008 17.0
0 0.112480401992798 18.0
0 0.0140959620475769 19.0
0 0.0657917261123657 20.0
0 -0.0261352062225342 21.0
0 0.596523106098175 22.0
40.70308422 40.1690826416016 23.0
0 0.0843281745910645 24.0
0 0.930257797241211 25.0
0 -0.0592844486236572 26.0
0 0.0849150419235229 27.0
0 0.327736496925354 28.0
0 0.0150518417358398 29.0
0 -0.117864489555359 30.0
0 0.076732873916626 31.0
0 0.0378565788269043 32.0
0 -0.111481964588165 33.0
0 0.215869545936584 34.0
0 0.0874238610267639 35.0
0 -0.694051623344421 36.0
0 0.0470466613769531 37.0
0 0.0676605701446533 38.0
0 -0.157683372497559 39.0
0 0.457198798656464 40.0
0 -0.143038928508759 41.0
0 0.0283925533294678 42.0
293.36798227 297.115783691406 43.0
0 0.0122966170310974 44.0
0 0.0983410477638245 45.0
0 0.298547744750977 46.0
0 -0.562232971191406 47.0
0 0.287732720375061 48.0
0 0.56296443939209 49.0
0.098220466819 0.00186651945114136 50.0
2.2695639013 1.88590431213379 51.0
0 0.370320200920105 52.0
0 -0.127019047737122 53.0
0 0.173133134841919 54.0
0 0.168941795825958 55.0
0 -0.144798994064331 56.0
0 0.250147938728333 57.0
0 0.537872910499573 58.0
0 0.198456346988678 59.0
0 -0.0632714033126831 60.0
0 0.078248918056488 61.0
0 0.183740258216858 62.0
279.17660995 252.248062133789 0.0
0 0.123317837715149 1.0
0 0.0509060025215149 2.0
0 0.192246735095978 3.0
0 -0.0389487147331238 4.0
0 0.15194046497345 5.0
0 0.117309331893921 6.0
0 0.398045837879181 7.0
0 0.117898106575012 8.0
0 -0.200843393802643 9.0
0 0.0181405544281006 10.0
0 0.0042988657951355 11.0
0 -0.172271609306335 12.0
73.536429152 72.63525390625 13.0
25.242637195 25.9891262054443 14.0
0 -0.0336364507675171 15.0
0 0.0564167499542236 16.0
33.620101475 32.1276168823242 17.0
0 0.125088930130005 18.0
0 -0.0135881900787354 19.0
0 0.113633990287781 20.0
0 0.15397709608078 21.0
0 0.133652627468109 22.0
30.838792679 31.2959995269775 23.0
0 0.0947196483612061 24.0
0 1.10062456130981 25.0
0 0.125810444355011 26.0
0 -0.193973481655121 27.0
0 0.534918189048767 28.0
0 -0.116242706775665 29.0
0 0.0684432983398438 30.0
0 0.110471844673157 31.0
0 0.089421272277832 32.0
0 -0.106244325637817 33.0
0 0.135169267654419 34.0
0 0.235572516918182 35.0
0 0.161978840827942 36.0
0 0.11842554807663 37.0
0 0.0284916758537292 38.0
0 -0.186021327972412 39.0
0 0.235864877700806 40.0
0 -0.123213291168213 41.0
0 -0.0873271226882935 42.0
301.51073201 298.286590576172 43.0
0 0.0698205232620239 44.0
0 0.0283409357070923 45.0
0 0.0846415758132935 46.0
0 0.307147264480591 47.0
0 0.0487477779388428 48.0
0 0.46319168806076 49.0
0.079042540691 0.000790953636169434 50.0
1.6246454731 1.92169630527496 51.0
0 0.240274846553802 52.0
0 0.0789681673049927 53.0
0 0.228044033050537 54.0
0 0.273011684417725 55.0
0 -0.0126501321792603 56.0
0 -0.0267554521560669 57.0
0 0.156709134578705 58.0
0 -0.0542234182357788 59.0
0 -0.212895274162292 60.0
0 -0.0261059403419495 61.0
0 -0.00107306241989136 62.0
289.13540036 256.780700683594 0.0
0 -0.333852529525757 1.0
0 0.0719650387763977 2.0
0 0.0677794218063354 3.0
0 -0.00362026691436768 4.0
0 0.308003723621368 5.0
0 0.312951326370239 6.0
0 0.0137205123901367 7.0
0 -0.126831114292145 8.0
0 -0.254954099655151 9.0
0 0.277508020401001 10.0
0 -0.0113757252693176 11.0
0 0.0898727774620056 12.0
68.866693791 68.8755722045898 13.0
41.288334264 42.3628158569336 14.0
0 -0.0172145366668701 15.0
0 -0.23665452003479 16.0
38.383454877 35.9645919799805 17.0
0 0.165393948554993 18.0
0 0.0123327374458313 19.0
0 0.147689461708069 20.0
0 0.0399986505508423 21.0
0 0.261025249958038 22.0
30.047276578 33.4328804016113 23.0
0 0.140190899372101 24.0
0 1.25487554073334 25.0
0 -0.0960847139358521 26.0
0 -0.327349185943604 27.0
0 0.453884452581406 28.0
0 0.440178364515305 29.0
0 0.040996253490448 30.0
0 0.184938073158264 31.0
0 0.0630568861961365 32.0
0 -0.539415836334229 33.0
0 0.199580788612366 34.0
0 -0.126266539096832 35.0
0 0.487899750471115 36.0
0 0.0182014107704163 37.0
0 -0.0441946983337402 38.0
0 -0.0607457160949707 39.0
0 0.187244236469269 40.0
0 -0.0639975666999817 41.0
0 0.0900251865386963 42.0
299.41961875 295.706512451172 43.0
0 -0.036937952041626 44.0
0 0.208165347576141 45.0
0 0.644257187843323 46.0
0 -0.714321613311768 47.0
0 0.154407203197479 48.0
0 0.476153016090393 49.0
0.095425536491 0.0355099439620972 50.0
2.251991223 1.8123232126236 51.0
0 0.129679620265961 52.0
0 -0.60208785533905 53.0
0 0.0184544324874878 54.0
0 -0.167052865028381 55.0
0 0.226085305213928 56.0
0 -0.257778882980347 57.0
0 0.0414068698883057 58.0
0 0.616070628166199 59.0
0 0.0606527328491211 60.0
0 0.0110610723495483 61.0
0 0.0796841979026794 62.0
220.71939728 254.938446044922 0.0
0 0.428646445274353 1.0
0 -0.0825310349464417 2.0
0 0.25928395986557 3.0
0 -0.261913895606995 4.0
0 -0.300461530685425 5.0
0 0.0626516938209534 6.0
0 -0.380597352981567 7.0
0 -0.431029081344604 8.0
0 0.583539128303528 9.0
0 1.1205997467041 10.0
0 -0.818452954292297 11.0
0 -0.139410257339478 12.0
74.884040347 70.3531341552734 13.0
19.861244368 33.5405807495117 14.0
0 -0.0722895860671997 15.0
0 0.182980477809906 16.0
40.464876681 39.1919937133789 17.0
0 0.265782117843628 18.0
0 -0.0515187978744507 19.0
0 -0.0287332534790039 20.0
0 0.152743399143219 21.0
0 0.05796217918396 22.0
29.094090129 27.8865451812744 23.0
0 0.0169121623039246 24.0
0 0.657438457012177 25.0
0 -0.0415092706680298 26.0
0 -0.238028645515442 27.0
0 0.482508718967438 28.0
0 0.0630340576171875 29.0
0 -0.00960958003997803 30.0
0 0.323402762413025 31.0
0 0.258001446723938 32.0
0 0.174255192279816 33.0
0 0.23625260591507 34.0
0 0.0906733274459839 35.0
0 0.298872113227844 36.0
0 0.100873947143555 37.0
0 0.0671650767326355 38.0
0 -0.241544842720032 39.0
0 0.0714708566665649 40.0
0 -0.0296310186386108 41.0
0 0.106703042984009 42.0
300.80218241 294.548797607422 43.0
0 0.235207557678223 44.0
0 -0.0556381344795227 45.0
0 0.296576917171478 46.0
0 0.162264704704285 47.0
0 -0.0333895683288574 48.0
0 0.388938248157501 49.0
0.084481283998 -0.0192517638206482 50.0
2.2467770895 1.87126183509827 51.0
0 0.0698554515838623 52.0
0 -0.268476366996765 53.0
0 0.0422796607017517 54.0
0 0.14367139339447 55.0
0 0.160922288894653 56.0
0 0.322914481163025 57.0
0 0.018534779548645 58.0
0 -1.0300133228302 59.0
0 1.26663446426392 60.0
0 0.00707334280014038 61.0
0 0.0820832848548889 62.0
207.41855917 230.642181396484 0.0
0 -0.511325359344482 1.0
0 -0.0961751937866211 2.0
0 -0.275131940841675 3.0
0 -0.24158775806427 4.0
0 -0.376699805259705 5.0
0 0.0108787417411804 6.0
0 -0.0789726972579956 7.0
0 0.389663487672806 8.0
0 -0.423964500427246 9.0
0 1.57639873027802 10.0
0 -0.456368207931519 11.0
0 -0.178385555744171 12.0
57.291286378 56.5667114257812 13.0
22.617069621 29.8976650238037 14.0
0 -0.451687932014465 15.0
0 -0.041773796081543 16.0
37.04270841 37.3361473083496 17.0
0 0.215731143951416 18.0
0 -0.0419139862060547 19.0
0 0.0260640382766724 20.0
0 -0.0644296407699585 21.0
0 0.0930007696151733 22.0
39.755573851 40.0445938110352 23.0
0 -0.0228785872459412 24.0
0 0.261036813259125 25.0
0 -0.374713540077209 26.0
0 -0.1694575548172 27.0
0 -0.675257802009583 28.0
0 0.256542921066284 29.0
0 0.0164727568626404 30.0
0 -0.100653648376465 31.0
0 -0.0814162492752075 32.0
0 -0.0521130561828613 33.0
0 0.404412031173706 34.0
0 -0.0602253675460815 35.0
0 0.233052015304565 36.0
0 -0.0999234318733215 37.0
0 0.278443872928619 38.0
0 -0.14142644405365 39.0
0 -0.296695590019226 40.0
0 -0.133877635002136 41.0
0 -0.0671406984329224 42.0
293.73933739 298.056060791016 43.0
0 0.0704329013824463 44.0
0 0.186695516109467 45.0
0 0.239861667156219 46.0
0 -0.0138412714004517 47.0
0 0.213389813899994 48.0
0 0.545357942581177 49.0
0.080225425828 0.0723855495452881 50.0
2.1768402116 1.79812574386597 51.0
0 0.21792334318161 52.0
0 -0.0597990751266479 53.0
0 0.289498209953308 54.0
0 0.19833117723465 55.0
0 -0.0261523723602295 56.0
0 0.651185512542725 57.0
0 0.284197688102722 58.0
0 0.34304004907608 59.0
0 -0.203016519546509 60.0
0 0.123278796672821 61.0
0 0.210763931274414 62.0
250.08172929 256.17529296875 0.0
0 -0.120342493057251 1.0
0 -0.286616683006287 2.0
0 0.172971189022064 3.0
0 -0.165763676166534 4.0
0 -0.647753238677979 5.0
0 -0.0768970847129822 6.0
0 0.491175472736359 7.0
0 -0.0806804895401001 8.0
0 -0.109562635421753 9.0
0 -0.258252143859863 10.0
0 -0.228935122489929 11.0
0 -0.0764893293380737 12.0
50.54904622 48.2738723754883 13.0
35.276093557 45.2964859008789 14.0
0 0.0135211944580078 15.0
0 0.429018080234528 16.0
35.694720552 30.509449005127 17.0
0 0.0899151563644409 18.0
0 -0.0798851251602173 19.0
0 0.450195610523224 20.0
0 -0.100450456142426 21.0
0 0.71454781293869 22.0
38.65859477 38.2724113464355 23.0
0 -0.0648579597473145 24.0
0 0.113506376743317 25.0
0 -0.0362293720245361 26.0
0 -0.0836697220802307 27.0
0 0.0955100655555725 28.0
0 -0.115184903144836 29.0
0 0.0449645519256592 30.0
0 -0.326256513595581 31.0
0 0.0769268274307251 32.0
0 -0.248310208320618 33.0
0 -0.0327180027961731 34.0
0 -0.121734082698822 35.0
0 0.47489532828331 36.0
0 -0.0685847997665405 37.0
0 0.322799056768417 38.0
0 -0.226806163787842 39.0
0 0.596055805683136 40.0
0 -0.21600079536438 41.0
0 0.146289110183716 42.0
287.03245529 295.756408691406 43.0
0 0.236122846603394 44.0
0 0.0355024337768555 45.0
0 0.0547412633895874 46.0
0 0.296297311782837 47.0
0 0.0479617118835449 48.0
0 0.329989284276962 49.0
0.10180198242 0.0560247302055359 50.0
2.0508946936 1.83332824707031 51.0
0 -0.369851112365723 52.0
0 -0.23082447052002 53.0
0 0.0403341054916382 54.0
0 -0.0304400324821472 55.0
0 0.129419207572937 56.0
0 0.00629937648773193 57.0
0 0.403404355049133 58.0
0 -0.0353693962097168 59.0
0 -0.00708645582199097 60.0
0 0.157324433326721 61.0
0 0.433687508106232 62.0
258.68463186 256.856231689453 0.0
0 0.00518560409545898 1.0
0 -0.0129051208496094 2.0
0 -0.248146414756775 3.0
0 -0.286023616790771 4.0
0 -0.151774287223816 5.0
0 -0.253770351409912 6.0
0 -0.169568240642548 7.0
0 -0.245376229286194 8.0
0 -0.28042197227478 9.0
0 0.257672846317291 10.0
0 -0.251016020774841 11.0
0 -0.357943892478943 12.0
46.141564417 37.9190368652344 13.0
43.307929872 36.628734588623 14.0
0 -0.0188450813293457 15.0
0 0.0887810587882996 16.0
25.364805385 31.9031047821045 17.0
0 0.052595853805542 18.0
0 -0.0152227282524109 19.0
0 0.248215615749359 20.0
0 0.0380966663360596 21.0
0 0.195943713188171 22.0
30.114389469 32.9798355102539 23.0
0 0.143777847290039 24.0
0 1.17286205291748 25.0
0 -0.0149584412574768 26.0
0 0.0569533705711365 27.0
0 -0.309078335762024 28.0
0 -0.140255272388458 29.0
0 -0.00642651319503784 30.0
0 0.736783623695374 31.0
0 -0.153656005859375 32.0
0 -0.188531398773193 33.0
0 -0.259561896324158 34.0
0 0.217301368713379 35.0
0 0.402176946401596 36.0
0 -0.0458204746246338 37.0
0 0.0539267063140869 38.0
0 -0.0870330333709717 39.0
0 0.335571438074112 40.0
0 -0.160517334938049 41.0
0 0.0461616516113281 42.0
282.98633165 297.0068359375 43.0
0 0.149915099143982 44.0
0 0.103485226631165 45.0
0 0.0236266255378723 46.0
0 -0.356482744216919 47.0
0 0.136263132095337 48.0
3.810066187 0.644688189029694 49.0
0.10348498179 0.0768341422080994 50.0
1.7098376902 1.87586426734924 51.0
0 0.53340756893158 52.0
0 -0.0997092723846436 53.0
0 -0.0894996523857117 54.0
0 0.0683642625808716 55.0
0 -0.0981934666633606 56.0
0 0.413046419620514 57.0
0 0.441079705953598 58.0
0 0.30744856595993 59.0
0 -0.17677891254425 60.0
0 0.107913672924042 61.0
0 0.2452272772789 62.0
283.5009113 251.670455932617 0.0
0 -0.184139013290405 1.0
0 -0.032034158706665 2.0
0 0.315678864717484 3.0
0 0.00581133365631104 4.0
0 0.0546009540557861 5.0
0 0.418698310852051 6.0
0 0.466613590717316 7.0
0 -0.202936708927155 8.0
0 0.287080407142639 9.0
0 0.124970138072968 10.0
0 -0.269178748130798 11.0
0 -0.118369579315186 12.0
55.690441054 55.1390724182129 13.0
39.88448581 38.0527610778809 14.0
0 -0.0552076101303101 15.0
0 -0.0134146213531494 16.0
35.411546351 34.3567504882812 17.0
0 0.154256701469421 18.0
0 -0.00124335289001465 19.0
0 0.114878356456757 20.0
0 0.188791632652283 21.0
0 -0.0501824617385864 22.0
30.055030089 27.3467330932617 23.0
0 -0.000934720039367676 24.0
0 -0.0148431658744812 25.0
0 -0.0895454287528992 26.0
0 -0.141039431095123 27.0
0 -0.0398481488227844 28.0
0 -0.0912773609161377 29.0
0 0.0229775309562683 30.0
0 -0.0818527936935425 31.0
0 0.0441750288009644 32.0
0 -0.0478061437606812 33.0
0 0.325842201709747 34.0
0 -0.020332396030426 35.0
0 0.109803915023804 36.0
0 0.0745473504066467 37.0
0 -0.169106721878052 38.0
0 -0.145041823387146 39.0
0 0.321065932512283 40.0
0 -0.163322925567627 41.0
0 -0.0690817832946777 42.0
297.11379986 296.087615966797 43.0
0 0.140655398368835 44.0
0 0.0512439608573914 45.0
0 0.0972219705581665 46.0
0 -0.0475915670394897 47.0
0 0.0774763822555542 48.0
0 0.358269721269608 49.0
0.082478329306 0.0460007190704346 50.0
2.1049435022 1.84142470359802 51.0
0 -0.0771324634552002 52.0
0 0.0472680926322937 53.0
0 0.0213321447372437 54.0
0 0.199043035507202 55.0
0 -0.0650341510772705 56.0
0 0.223923206329346 57.0
0 0.575213193893433 58.0
0 0.117641687393188 59.0
0 -0.232731699943542 60.0
0 0.114659786224365 61.0
0 0.214148879051208 62.0
252.38042851 252.647247314453 0.0
0 0.052420973777771 1.0
0 -0.168654263019562 2.0
0 0.220763444900513 3.0
0 0.0836710929870605 4.0
0 0.050290048122406 5.0
0 0.0260945558547974 6.0
0 0.123820960521698 7.0
0 -0.122844576835632 8.0
0 0.16599303483963 9.0
0 -0.0832809209823608 10.0
0 -0.152103185653687 11.0
0 -0.169249176979065 12.0
59.439065846 59.7829208374023 13.0
36.295059662 37.6916275024414 14.0
0 0.0160413980484009 15.0
0 -0.0133512020111084 16.0
36.204838544 32.9497833251953 17.0
0 0.0946030020713806 18.0
0 -0.0486728549003601 19.0
0 0.0413190126419067 20.0
0 0.240256428718567 21.0
0 0.112087249755859 22.0
27.925007411 26.5353012084961 23.0
0 0.10025954246521 24.0
0 0.514989256858826 25.0
0 -0.010722815990448 26.0
0 -0.171728312969208 27.0
0 0.501899719238281 28.0
0 -0.0345975160598755 29.0
0 0.126289665699005 30.0
0 -0.0580404996871948 31.0
0 -0.0425994396209717 32.0
0 -0.0328604578971863 33.0
0 0.129746377468109 34.0
0 -0.0850494503974915 35.0
0 -0.0142524242401123 36.0
0 0.160057067871094 37.0
0 0.129089415073395 38.0
0 -0.241129875183105 39.0
0 0.354365348815918 40.0
0 -0.171910405158997 41.0
0 0.215425372123718 42.0
299.19094952 297.165924072266 43.0
0 0.0881333947181702 44.0
0 0.177982747554779 45.0
0 0.178793728351593 46.0
0 -0.00237768888473511 47.0
0 0.220871806144714 48.0
0 0.430001974105835 49.0
0.085345708936 0.0170406699180603 50.0
1.9300965745 1.86326456069946 51.0
0 0.0216193199157715 52.0
0 -0.0612519979476929 53.0
0 0.0984216928482056 54.0
0 0.148696660995483 55.0
0 0.11396312713623 56.0
0 0.45587283372879 57.0
0 0.00284874439239502 58.0
0 -0.018474280834198 59.0
0 -0.0484437346458435 60.0
0 0.0799822807312012 61.0
0 0.128350615501404 62.0
245.88010282 254.675216674805 0.0
0 -0.248890042304993 1.0
0 -0.0145965218544006 2.0
0 0.0406190156936646 3.0
0 -0.324905395507812 4.0
0 0.0761451125144958 5.0
0 0.0670129060745239 6.0
0 0.0801113247871399 7.0
0 0.0432521104812622 8.0
0 0.0374094247817993 9.0
0 -0.219127416610718 10.0
0 -0.170154511928558 11.0
0 -0.0834687948226929 12.0
66.12451197 66.1587448120117 13.0
27.060630449 32.9593658447266 14.0
0 -0.0826395750045776 15.0
0 0.145175039768219 16.0
35.394268756 36.9175567626953 17.0
0 0.198365330696106 18.0
0 -0.0535988807678223 19.0
0 0.0587543249130249 20.0
0 0.0112909078598022 21.0
0 0.179640889167786 22.0
36.764392787 35.4799880981445 23.0
0 0.0909392833709717 24.0
0 0.423945307731628 25.0
0 0.00151437520980835 26.0
0 -0.037030041217804 27.0
0 0.967755794525146 28.0
0 0.041455864906311 29.0
0 0.0291425585746765 30.0
0 -0.176841497421265 31.0
0 0.0222650766372681 32.0
0 -0.0986688733100891 33.0
0 0.28057861328125 34.0
0 -0.173107802867889 35.0
0 -0.119242191314697 36.0
0 0.2276850938797 37.0
0 -0.0505744814872742 38.0
0 -0.161752700805664 39.0
0 0.264913022518158 40.0
0 -0.13909500837326 41.0
0 -0.0766348838806152 42.0
302.78565925 296.433776855469 43.0
0 -0.0245672464370728 44.0
0 0.163385391235352 45.0
0 -0.000489532947540283 46.0
0 0.109276711940765 47.0
0 0.416881233453751 48.0
0.67679063142 0.387137591838837 49.0
0.088097603945 0.0618670582771301 50.0
1.7242136023 1.8796694278717 51.0
0 0.0651390552520752 52.0
0 0.0333189964294434 53.0
0 0.0285776257514954 54.0
0 0.100103557109833 55.0
0 -0.129690527915955 56.0
0 0.197023451328278 57.0
0 0.236275911331177 58.0
0 0.275077998638153 59.0
0 -0.240757465362549 60.0
0 0.0684814453125 61.0
0 0.272422432899475 62.0
242.19384869 251.584625244141 0.0
0 -0.228029131889343 1.0
0 0.0764155387878418 2.0
0 0.174518465995789 3.0
0 -0.176867246627808 4.0
0 0.0792930126190186 5.0
0 0.193043172359467 6.0
0 0.0437783598899841 7.0
0 -0.10012948513031 8.0
0 0.386283874511719 9.0
0 0.579262137413025 10.0
0 -0.312868475914001 11.0
0 -0.104109883308411 12.0
73.228093622 73.8611297607422 13.0
9.7815139041 19.9809646606445 14.0
0 -0.186995267868042 15.0
0 0.0384659767150879 16.0
26.744888326 32.0513381958008 17.0
0 0.0124492645263672 18.0
0 0.0168260931968689 19.0
0 0.188236773014069 20.0
0 0.0794364213943481 21.0
0 0.161713600158691 22.0
41.89800623 39.4417152404785 23.0
0 0.0811477303504944 24.0
0 0.812096893787384 25.0
0 -0.0627750158309937 26.0
0 -0.163913249969482 27.0
0 0.330884337425232 28.0
0 -0.177555322647095 29.0
0 0.0444446206092834 30.0
0 -0.14068067073822 31.0
0 -0.16343891620636 32.0
0 -0.335750341415405 33.0
0 -0.00690215826034546 34.0
0 0.0184255838394165 35.0
0 0.491310626268387 36.0
0 0.0983663201332092 37.0
0 0.0679441690444946 38.0
0 -0.273986220359802 39.0
0 0.623520374298096 40.0
0 -0.123929679393768 41.0
0 0.16706120967865 42.0
305.26233141 293.799926757812 43.0
0 -0.0713260173797607 44.0
0 0.433527767658234 45.0
0 0.209712028503418 46.0
0 -0.0369642972946167 47.0
0 0.328231394290924 48.0
0.32470042567 0.615798652172089 49.0
0.079557474583 0.00432997941970825 50.0
1.5291629244 1.91110014915466 51.0
0 0.0306912660598755 52.0
0 0.0175552368164062 53.0
0 -0.0806553959846497 54.0
0 -0.18072909116745 55.0
0 -0.0107318162918091 56.0
0 0.192445755004883 57.0
0 0.0396073460578918 58.0
0 -1.13711583614349 59.0
0 1.59499931335449 60.0
0 0.0611903667449951 61.0
0 0.0703098177909851 62.0
222.53475815 251.421768188477 0.0
0 0.4210045337677 1.0
0 0.0446836352348328 2.0
0 0.0162321329116821 3.0
0 -0.156079351902008 4.0
0 -0.0990419983863831 5.0
0 0.127869963645935 6.0
0 -0.0589393973350525 7.0
0 0.574252486228943 8.0
0 0.0449649691581726 9.0
0 -0.313209295272827 10.0
0 -0.0642216801643372 11.0
0 -0.110356628894806 12.0
74.282548098 73.5386810302734 13.0
23.315430257 24.908239364624 14.0
0 -0.0740725994110107 15.0
0 0.0213082432746887 16.0
41.436963763 38.3480911254883 17.0
0 0.0884487628936768 18.0
0 -0.0251392722129822 19.0
0 0.0643079280853271 20.0
0 0.0177043080329895 21.0
0 0.226792573928833 22.0
37.793914061 39.4032096862793 23.0
0 0.116593241691589 24.0
0 0.888200461864471 25.0
0 -0.0699607133865356 26.0
0 -0.161857664585114 27.0
0 0.123527228832245 28.0
0 0.130912244319916 29.0
0 0.0583110451698303 30.0
0 -0.154799818992615 31.0
0 -0.0233102440834045 32.0
0 -0.0925338268280029 33.0
0 0.0858874320983887 34.0
0 0.0609757900238037 35.0
0 0.378084629774094 36.0
0 -0.0483960509300232 37.0
0 0.336680173873901 38.0
0 -0.636745810508728 39.0
0 0.377409398555756 40.0
0 -0.199449717998505 41.0
0 0.184728264808655 42.0
294.20229469 298.025085449219 43.0
0 0.183702945709229 44.0
0 0.110635161399841 45.0
0 0.112369179725647 46.0
0 0.166380047798157 47.0
0 -0.914838910102844 48.0
0 0.587599635124207 49.0
0.083547178043 -0.0680896639823914 50.0
2.0829406106 1.89062333106995 51.0
0 -0.0507376790046692 52.0
0 -0.449769735336304 53.0
0 0.192286133766174 54.0
0 0.132975518703461 55.0
0 0.0947037935256958 56.0
0 0.731201529502869 57.0
0 0.0399639010429382 58.0
0 -0.212747573852539 59.0
0 1.14454996585846 60.0
0 -0.00997966527938843 61.0
0 0.0152891278266907 62.0
255.16083761 251.425033569336 0.0
0 -0.238911628723145 1.0
0 -0.110423684120178 2.0
0 0.377833247184753 3.0
0 -0.0717805624008179 4.0
0 0.317872822284698 5.0
0 -0.0812376141548157 6.0
0 0.169305205345154 7.0
0 0.0559521317481995 8.0
0 0.196207761764526 9.0
0 -0.169116079807281 10.0
0 0.103343784809113 11.0
0 -0.0872375965118408 12.0
72.669480389 74.1426391601562 13.0
27.773062189 23.3938732147217 14.0
0 -0.00570976734161377 15.0
0 -0.044555127620697 16.0
40.682801768 35.24755859375 17.0
0 0.0966536998748779 18.0
0 -0.0348933935165405 19.0
0 0.146687507629395 20.0
0 -0.0460957288742065 21.0
0 0.266479849815369 22.0
35.658010278 35.3215255737305 23.0
0 0.102237343788147 24.0
0 0.453456997871399 25.0
0 -0.0995227098464966 26.0
0 0.0484194159507751 27.0
0 0.147331058979034 28.0
0 0.016990602016449 29.0
0 -0.118492662906647 30.0
0 -0.0215585231781006 31.0
0 -0.114183306694031 32.0
0 -0.449085712432861 33.0
0 0.227227628231049 34.0
0 0.0210481882095337 35.0
0 0.248481154441833 36.0
0 0.157195270061493 37.0
0 0.137028992176056 38.0
0 -0.194607853889465 39.0
0 0.57184624671936 40.0
0 -0.0985071063041687 41.0
0 0.175199747085571 42.0
302.83771367 295.524139404297 43.0
0 0.153921246528625 44.0
0 0.227406084537506 45.0
0 0.184686720371246 46.0
0 -0.20785516500473 47.0
0 0.370347499847412 48.0
0.54247811833 0.561082243919373 49.0
0.098496921765 0.0711587071418762 50.0
2.2923141728 1.85395133495331 51.0
0 -0.0408070683479309 52.0
0 -0.0226823091506958 53.0
0 0.098967432975769 54.0
0 0.27346283197403 55.0
0 0.166626393795013 56.0
0 0.579753041267395 57.0
0 0.250024378299713 58.0
0 0.309232801198959 59.0
0 0.0285720825195312 60.0
0 0.0337674617767334 61.0
0 0.226731657981873 62.0
237.87281016 253.87809753418 0.0
0 -0.330720543861389 1.0
0 -0.424281239509583 2.0
0 0.442294716835022 3.0
0 -0.148161351680756 4.0
0 -0.0713193416595459 5.0
0 -0.0216981768608093 6.0
0 -0.164340078830719 7.0
0 0.259590625762939 8.0
0 0.150325059890747 9.0
0 0.0810167789459229 10.0
0 -0.206029951572418 11.0
0 0.106304168701172 12.0
64.824007446 64.439697265625 13.0
23.009797594 31.3439788818359 14.0
0 -0.126888036727905 15.0
0 0.227358639240265 16.0
30.819566998 35.0208320617676 17.0
0 0.0466405153274536 18.0
0 0.0295854210853577 19.0
0 0.0108678340911865 20.0
0 0.158981740474701 21.0
0 0.148550987243652 22.0
30.730806784 29.3554096221924 23.0
0 0.113591313362122 24.0
0 1.17353963851929 25.0
0 -0.0092465877532959 26.0
0 -0.152002513408661 27.0
0 0.666280508041382 28.0
0 -0.0434377193450928 29.0
0 0.0545679330825806 30.0
0 0.0688722133636475 31.0
0 0.184244751930237 32.0
0 0.215634226799011 33.0
0 0.329018443822861 34.0
0 -0.106133699417114 35.0
0 0.0741812586784363 36.0
0 -0.00152057409286499 37.0
0 0.285091400146484 38.0
0 -0.204169631004333 39.0
0 0.549296021461487 40.0
0 -0.155363321304321 41.0
0 -0.0015789270401001 42.0
297.38788536 293.717254638672 43.0
0 0.0774217247962952 44.0
0 -0.054335355758667 45.0
0 0.244177281856537 46.0
0 -0.0107120275497437 47.0
0 -0.091041088104248 48.0
0 0.552225768566132 49.0
0.089327612676 0.0342845916748047 50.0
1.8685452927 1.77817583084106 51.0
0 0.236158430576324 52.0
0 -0.00446397066116333 53.0
0 0.0189715623855591 54.0
0 0.287373185157776 55.0
0 0.30271989107132 56.0
0 0.396055936813354 57.0
0 0.225094199180603 58.0
0 0.153281092643738 59.0
0 0.192671477794647 60.0
0 0.079612135887146 61.0
0 0.265199899673462 62.0
264.33470731 252.176788330078 0.0
0 -0.10442066192627 1.0
0 -0.0683389902114868 2.0
0 -0.00548899173736572 3.0
0 0.0319710969924927 4.0
0 0.345187038183212 5.0
0 0.873123049736023 6.0
0 0.426087081432343 7.0
0 0.726883709430695 8.0
0 -0.116604089736938 9.0
0 0.607086539268494 10.0
0 -0.112269163131714 11.0
0 -0.0578873157501221 12.0
65.994429092 64.8062057495117 13.0
23.132654157 32.6173515319824 14.0
0 -0.0671144723892212 15.0
0 0.0580401420593262 16.0
37.835481072 36.4390106201172 17.0
0 0.17570436000824 18.0
0 -0.00984287261962891 19.0
0 0.146983921527863 20.0
0 -0.0205339193344116 21.0
0 0.796726047992706 22.0
44.059005567 40.2640800476074 23.0
0 -0.0575398802757263 24.0
0 0.325201243162155 25.0
0 -0.116770267486572 26.0
0 0.166050910949707 27.0
0 -0.363157391548157 28.0
0 0.00339376926422119 29.0
0 -0.144946098327637 30.0
0 0.0644183158874512 31.0
0 0.0895063281059265 32.0
0 -0.104495406150818 33.0
0 -0.00194311141967773 34.0
0 0.177230954170227 35.0
0 0.369712620973587 36.0
0 0.00248533487319946 37.0
0 -0.487537264823914 38.0
0 -0.209612488746643 39.0
0 -0.0421043634414673 40.0
0 -0.135442435741425 41.0
0 0.0828031301498413 42.0
296.4681872 296.920013427734 43.0
0 0.231453061103821 44.0
0 0.147152662277222 45.0
0 0.128607273101807 46.0
0 -0.365902662277222 47.0
0 -0.462997436523438 48.0
0 0.430243939161301 49.0
0.084110514746 0.0149809718132019 50.0
1.8160922236 1.83292317390442 51.0
0 0.275389730930328 52.0
0 0.159033179283142 53.0
0 0.102187156677246 54.0
0 -0.168506801128387 55.0
0 -0.034229040145874 56.0
0 -0.0623420476913452 57.0
0 0.42259818315506 58.0
0 0.365564107894897 59.0
0 -0.164973556995392 60.0
0 0.0392130613327026 61.0
0 0.032456636428833 62.0
239.11522201 252.38525390625 0.0
0 0.113930463790894 1.0
0 -0.100706517696381 2.0
0 0.0459663271903992 3.0
0 -0.133798241615295 4.0
0 0.0155097246170044 5.0
0 0.0431183576583862 6.0
0 -0.0268355011940002 7.0
0 0.192326068878174 8.0
0 -0.0894313454627991 9.0
0 -0.233020663261414 10.0
0 -0.102916896343231 11.0
0 -0.118817567825317 12.0
68.169675423 67.7233428955078 13.0
26.454188959 33.0965690612793 14.0
0 0.639293789863586 15.0
0 0.0141552686691284 16.0
39.486161947 34.214729309082 17.0
0 0.169879913330078 18.0
0 -0.0446945428848267 19.0
0 0.0928937196731567 20.0
0 0.135190606117249 21.0
0 0.124860465526581 22.0
29.900615106 31.4389839172363 23.0
0 0.111992239952087 24.0
0 0.923290610313416 25.0
0 -0.101685345172882 26.0
0 0.115223288536072 27.0
0 -0.341737866401672 28.0
0 0.0339277386665344 29.0
0 0.108245074748993 30.0
0 0.0952779054641724 31.0
0 0.093087375164032 32.0
0 0.00337547063827515 33.0
0 0.228847324848175 34.0
0 -0.190917730331421 35.0
0 0.0628547072410583 36.0
0 -0.0504238605499268 37.0
0 0.174743592739105 38.0
0 -0.13975989818573 39.0
0 0.197582900524139 40.0
0 -0.159354746341705 41.0
0 0.106082797050476 42.0
294.9843684 297.713928222656 43.0
0 0.0547085404396057 44.0
0 0.0671259760856628 45.0
0 0.173410713672638 46.0
0 -0.0163227319717407 47.0
0 0.0522763729095459 48.0
0.19979740263 0.518510818481445 49.0
0.096007441204 0.0428594350814819 50.0
2.0951637449 1.83724403381348 51.0
0 0.317367970943451 52.0
0 -0.379953026771545 53.0
0 0.00997960567474365 54.0
0 0.253444015979767 55.0
0 0.17645674943924 56.0
0 0.00448763370513916 57.0
0 0.138940751552582 58.0
0 -0.000702023506164551 59.0
0 -0.153540253639221 60.0
0 0.0409255623817444 61.0
0 0.18881094455719 62.0
217.3160003 246.361618041992 0.0
0 0.0467265844345093 1.0
0 -0.171884834766388 2.0
0 -0.200688660144806 3.0
0 -0.408438205718994 4.0
0 -0.158676981925964 5.0
0 0.110056340694427 6.0
0 -0.420279860496521 7.0
0 0.175526142120361 8.0
0 -0.384160399436951 9.0
0 -0.249480485916138 10.0
0 -0.0950648188591003 11.0
0 -0.196241855621338 12.0
70.623865828 70.3705673217773 13.0
36.874925793 20.0241584777832 14.0
0 -0.0911087989807129 15.0
0 -0.250904560089111 16.0
34.427348197 33.9002838134766 17.0
0 0.123186409473419 18.0
0 0.0248648524284363 19.0
0 0.0588722229003906 20.0
0 0.206672191619873 21.0
0 0.147737145423889 22.0
31.511185911 29.5715274810791 23.0
0 0.0805196166038513 24.0
0 1.12297296524048 25.0
0 -0.159991204738617 26.0
0 -0.373724699020386 27.0
0 0.325992405414581 28.0
0 0.220671772956848 29.0
0 -0.0469842553138733 30.0
0 0.170695424079895 31.0
0 -0.082943320274353 32.0
0 -0.104123175144196 33.0
0 0.0531558394432068 34.0
0 0.531706750392914 35.0
0 -0.368200898170471 36.0
0 0.116374850273132 37.0
0 0.487851530313492 38.0
0 -0.596623659133911 39.0
0 -0.500826716423035 40.0
0 -0.293543100357056 41.0
0 -0.00808238983154297 42.0
297.53187305 295.884552001953 43.0
0 0.189414620399475 44.0
0 -0.165023982524872 45.0
0 0.382185161113739 46.0
0 -0.267146944999695 47.0
0 -0.267514705657959 48.0
0 0.521852195262909 49.0
0.058845978523 -0.0444380640983582 50.0
1.5574254181 1.81570971012115 51.0
0 -0.193017065525055 52.0
0 0.388520866632462 53.0
0 0.448849618434906 54.0
0 0.261649966239929 55.0
0 0.0198863744735718 56.0
0 0.260605871677399 57.0
0 0.350965112447739 58.0
0 -0.100849747657776 59.0
0 0.260786831378937 60.0
0 0.109272480010986 61.0
0 0.164129137992859 62.0
263.67565341 252.707321166992 0.0
0 -0.184757232666016 1.0
0 -0.000449299812316895 2.0
0 0.402498483657837 3.0
0 -0.310106873512268 4.0
0 -0.069837749004364 5.0
0 0.155126690864563 6.0
0 -0.0190451145172119 7.0
0 0.205463767051697 8.0
0 0.194206237792969 9.0
0 0.173396468162537 10.0
0 -0.254931449890137 11.0
0 -0.174242854118347 12.0
52.311710681 53.0566520690918 13.0
37.083596385 37.4485168457031 14.0
0 -0.132126808166504 15.0
0 0.145818531513214 16.0
30.759255512 35.6925506591797 17.0
0 0.0403976440429688 18.0
0 -0.0301980376243591 19.0
0 0.0754534006118774 20.0
0 -0.00258433818817139 21.0
0 0.213149130344391 22.0
35.578611312 34.2947196960449 23.0
0 0.0629385709762573 24.0
0 0.504527807235718 25.0
0 0.109425365924835 26.0
0 -0.0363306403160095 27.0
0 0.147066116333008 28.0
0 -0.183649480342865 29.0
0 0.107947051525116 30.0
0 0.049094557762146 31.0
0 0.00425314903259277 32.0
0 0.00656259059906006 33.0
0 -0.0759415626525879 34.0
0 0.30290424823761 35.0
0 0.10191285610199 36.0
0 0.192272126674652 37.0
0 -0.0998916625976562 38.0
0 -0.179929256439209 39.0
0 0.27476692199707 40.0
0 -0.135925889015198 41.0
0 0.0477815866470337 42.0
302.5665184 296.818237304688 43.0
0 0.167618751525879 44.0
0 0.105560123920441 45.0
0 0.283568620681763 46.0
0 -0.0871163606643677 47.0
0 -0.145662665367126 48.0
0 0.450667560100555 49.0
0.068799032607 0.0320398807525635 50.0
1.5304762444 1.8624792098999 51.0
0 0.230692028999329 52.0
0 -0.074652373790741 53.0
0 -0.00303500890731812 54.0
0 0.0485161542892456 55.0
0 -0.0946741700172424 56.0
0 0.197767436504364 57.0
0 0.607966661453247 58.0
0 0.274190366268158 59.0
0 -0.204171240329742 60.0
0 0.0620357990264893 61.0
0 0.0775281190872192 62.0
232.3307681 254.011840820312 0.0
0 0.104854106903076 1.0
0 0.0107625722885132 2.0
0 0.136985242366791 3.0
0 -0.248820066452026 4.0
0 -0.504185438156128 5.0
0 0.120814442634583 6.0
0 -0.154948711395264 7.0
0 -0.0810531377792358 8.0
0 0.224360466003418 9.0
0 0.237585663795471 10.0
0 -0.111414790153503 11.0
0 -0.0906747579574585 12.0
48.15489493 44.1912689208984 13.0
53.035406725 48.2010269165039 14.0
0 -0.0655773878097534 15.0
0 0.29574853181839 16.0
42.231424022 39.1746215820312 17.0
0 0.288682460784912 18.0
0 -0.117211222648621 19.0
0 -0.0304892063140869 20.0
0 0.275572538375854 21.0
0 -0.0646063685417175 22.0
25.01101635 26.9206581115723 23.0
0 0.126094460487366 24.0
0 0.717592179775238 25.0
0 -0.0572760701179504 26.0
0 -0.263445854187012 27.0
0 0.485432684421539 28.0
0 -0.103546142578125 29.0
0 0.0668864846229553 30.0
0 -0.0709266662597656 31.0
0 -0.219497203826904 32.0
0 -0.368122816085815 33.0
0 0.362407028675079 34.0
0 0.660939931869507 35.0
0 0.231338262557983 36.0
0 -0.0741084814071655 37.0
0 -0.113442778587341 38.0
0 -0.00838637351989746 39.0
0 -0.0154950618743896 40.0
0 -0.0439521074295044 41.0
0 0.126840114593506 42.0
291.21315359 295.158874511719 43.0
0 0.00773066282272339 44.0
0 0.279838800430298 45.0
0 0.0692710280418396 46.0
0 -0.188988506793976 47.0
0 0.329874187707901 48.0
0.67752155086 0.411469668149948 49.0
0.066172989946 0.115107595920563 50.0
2.2350539324 1.85891735553741 51.0
0 -0.175790071487427 52.0
0 -0.285414814949036 53.0
0 0.319475024938583 54.0
0 -0.198209822177887 55.0
0 0.0552277565002441 56.0
0 -0.1004958152771 57.0
0 -0.0822131037712097 58.0
0 -0.150518655776978 59.0
0 -0.22291362285614 60.0
0 0.135626256465912 61.0
0 0.179650723934174 62.0
220.38595671 252.338973999023 0.0
0 -0.112014174461365 1.0
0 0.00345224142074585 2.0
0 -0.422775983810425 3.0
0 -0.267800569534302 4.0
0 -0.168868660926819 5.0
0 -0.376370429992676 6.0
0 -0.341706037521362 7.0
0 0.647046446800232 8.0
0 0.154186308383942 9.0
0 -0.195954263210297 10.0
0 -0.448235750198364 11.0
0 -0.013471245765686 12.0
75.809005303 72.9406204223633 13.0
23.098544417 26.839225769043 14.0
0 -0.0741183757781982 15.0
0 -0.0230236649513245 16.0
40.625354281 39.6648445129395 17.0
0 0.210902750492096 18.0
0 -0.094524621963501 19.0
0 0.12127149105072 20.0
0 0.120219111442566 21.0
0 0.156481385231018 22.0
28.186593851 28.4874458312988 23.0
0 0.075792670249939 24.0
0 1.18525397777557 25.0
0 -0.0244091749191284 26.0
0 -0.325480937957764 27.0
0 0.0403901934623718 28.0
0 0.308911710977554 29.0
0 -0.0356204509735107 30.0
0 0.220562815666199 31.0
0 0.0817213654518127 32.0
0 -0.0977532863616943 33.0
0 -0.0295472145080566 34.0
0 0.726608633995056 35.0
0 0.277830064296722 36.0
0 0.125524044036865 37.0
0 -0.014294445514679 38.0
0 -0.407614231109619 39.0
0 0.126187562942505 40.0
0 -0.0975188612937927 41.0
0 -0.00336992740631104 42.0
287.7375369 297.39794921875 43.0
0 0.110797047615051 44.0
0 0.0562324523925781 45.0
0 0.355215609073639 46.0
0 -0.206825375556946 47.0
0 0.283040523529053 48.0
0 0.296323895454407 49.0
0.068954088859 0.0779215097427368 50.0
1.9287466489 1.81967520713806 51.0
0 -0.199788331985474 52.0
0 -0.335399866104126 53.0
0 0.179715991020203 54.0
0 0.287837743759155 55.0
0 0.295643627643585 56.0
0 0.310542941093445 57.0
0 0.0510403513908386 58.0
0 -0.00464016199111938 59.0
0 -0.175892472267151 60.0
0 0.0606755018234253 61.0
0 0.212817072868347 62.0
235.26173313 252.226898193359 0.0
0 -0.204447031021118 1.0
0 0.0641534328460693 2.0
0 0.0694031715393066 3.0
0 -0.309824228286743 4.0
0 0.0231013298034668 5.0
0 0.342347323894501 6.0
0 0.0860641002655029 7.0
0 0.0542484521865845 8.0
0 0.489431321620941 9.0
0 0.593684673309326 10.0
0 -0.18246603012085 11.0
0 -0.0370879173278809 12.0
60.171260524 60.4043846130371 13.0
37.644956247 34.6142196655273 14.0
0 -0.0540338754653931 15.0
0 -0.0369725227355957 16.0
36.278208065 36.0266036987305 17.0
0 0.122396230697632 18.0
0 -0.0705533027648926 19.0
0 0.140856504440308 20.0
0 -0.0223953127861023 21.0
0 0.0740439295768738 22.0
37.437973804 35.6298637390137 23.0
0 0.014832079410553 24.0
0 0.343344449996948 25.0
0 0.0315179228782654 26.0
0 -0.0306428074836731 27.0
0 0.421968013048172 28.0
0 0.0445792078971863 29.0
0 0.0835115313529968 30.0
0 -0.0638076066970825 31.0
0 -0.131320297718048 32.0
0 0.0352762937545776 33.0
0 0.358577460050583 34.0
0 0.0447565317153931 35.0
0 0.253822445869446 36.0
0 0.0406628847122192 37.0
0 0.142702698707581 38.0
0 -0.237523794174194 39.0
0 0.461935579776764 40.0
0 -0.178072273731232 41.0
0 0.265854239463806 42.0
302.52394381 297.943969726562 43.0
0 0.305602073669434 44.0
0 -0.31696629524231 45.0
0 0.163088202476501 46.0
0 -0.0849575400352478 47.0
0 0.0884665250778198 48.0
0 0.263038158416748 49.0
0.067407977203 0.081159770488739 50.0
1.889961565 1.82446932792664 51.0
0 0.223494291305542 52.0
0 -0.0931311845779419 53.0
0 0.224122822284698 54.0
0 0.277633249759674 55.0
0 -0.0220354199409485 56.0
0 -0.0905664563179016 57.0
0 -0.120426058769226 58.0
0 0.289685606956482 59.0
0 0.0385837554931641 60.0
0 0.0712575316429138 61.0
0 0.100881397724152 62.0
298.05816727 252.632598876953 0.0
0 0.1019606590271 1.0
0 0.115738689899445 2.0
0 0.288221955299377 3.0
0 -0.41084361076355 4.0
0 -0.0172533392906189 5.0
0 0.659508943557739 6.0
0 0.495149284601212 7.0
0 1.01181757450104 8.0
0 0.534638404846191 9.0
0 -0.0606445670127869 10.0
0 -0.186580836772919 11.0
0 -0.236200213432312 12.0
52.219233525 49.74609375 13.0
43.000396797 38.1446647644043 14.0
0 -0.170399188995361 15.0
0 0.260238587856293 16.0
31.235369923 35.6449928283691 17.0
0 0.163682878017426 18.0
0 -0.00155466794967651 19.0
0 0.0762351751327515 20.0
0 0.0295881032943726 21.0
0 0.196245193481445 22.0
34.983959259 33.7362251281738 23.0
0 0.0502320528030396 24.0
0 0.674417972564697 25.0
0 -0.140524625778198 26.0
0 0.0765872001647949 27.0
0 0.192613542079926 28.0
0 0.0494859218597412 29.0
0 0.00482642650604248 30.0
0 -0.0327893495559692 31.0
0 0.0533693432807922 32.0
0 -0.124432981014252 33.0
0 0.10724264383316 34.0
0 0.149422228336334 35.0
0 -0.324003219604492 36.0
0 0.00550282001495361 37.0
0 0.355587124824524 38.0
0 -0.150994300842285 39.0
0 -0.175754904747009 40.0
0 -0.200528979301453 41.0
0 0.0833659172058105 42.0
296.56821776 297.768707275391 43.0
0 0.0336362719535828 44.0
0 0.0679225921630859 45.0
0 0.0997737646102905 46.0
0 0.0262598395347595 47.0
0 0.268353700637817 48.0
3.6351245034 0.575602412223816 49.0
0.068282151705 0.105925619602203 50.0
1.8112243603 1.87706136703491 51.0
0 -0.130121111869812 52.0
0 0.376702308654785 53.0
0 -0.167739689350128 54.0
0 0.266629695892334 55.0
0 0.348506450653076 56.0
0 -0.13137823343277 57.0
0 -0.0189188718795776 58.0
0 0.0123279094696045 59.0
0 -0.112213015556335 60.0
0 0.0499734282493591 61.0
0 0.0272432565689087 62.0
270.23927835 252.389175415039 0.0
0 -0.239105582237244 1.0
0 -0.0627013444900513 2.0
0 0.224865078926086 3.0
0 -0.0219680666923523 4.0
0 0.405191510915756 5.0
0 0.246267914772034 6.0
0 0.254448175430298 7.0
0 -0.316875696182251 8.0
0 0.209744393825531 9.0
0 -0.401871800422668 10.0
0 0.101776003837585 11.0
0 -0.00704807043075562 12.0
56.860920445 59.4686813354492 13.0
51.046141537 39.2390785217285 14.0
0 0.0278728008270264 15.0
0 0.0838443636894226 16.0
39.209754044 36.0743026733398 17.0
0 0.153515815734863 18.0
0 -0.0196312665939331 19.0
0 0.0716850757598877 20.0
0 0.142953395843506 21.0
0 0.216798365116119 22.0
30.397317481 30.455717086792 23.0
0 0.0803342461585999 24.0
0 0.932746887207031 25.0
0 -0.0421573519706726 26.0
0 -0.102674663066864 27.0
0 0.303741723299026 28.0
0 0.0192834734916687 29.0
0 -0.0763526558876038 30.0
0 0.206198930740356 31.0
0 0.112330555915833 32.0
0 -0.120407342910767 33.0
0 0.223697066307068 34.0
0 -0.192492246627808 35.0
0 0.286786675453186 36.0
0 0.00905829668045044 37.0
0 0.214223027229309 38.0
0 -0.23665976524353 39.0
0 0.413404762744904 40.0
0 -0.183762311935425 41.0
0 0.134125351905823 42.0
298.39678256 295.251007080078 43.0
0 0.100195527076721 44.0
0 0.0116106867790222 45.0
0 0.249098837375641 46.0
0 0.0709972977638245 47.0
0 0.11618560552597 48.0
0 0.314023822546005 49.0
0.082274691007 0.066184401512146 50.0
2.178440716 1.88616180419922 51.0
0 0.252920567989349 52.0
0 0.0347381830215454 53.0
0 0.0166784524917603 54.0
0 0.0550422668457031 55.0
0 0.191834211349487 56.0
0 0.322253406047821 57.0
0 0.11686372756958 58.0
0 0.304716289043427 59.0
0 -0.213373184204102 60.0
0 0.125094592571259 61.0
0 0.257502675056458 62.0
240.59310271 253.415267944336 0.0
0 -0.194972038269043 1.0
0 -0.148919582366943 2.0
0 0.26220041513443 3.0
0 -0.241172671318054 4.0
0 0.0810586810112 5.0
0 -0.0803955793380737 6.0
0 -0.138484120368958 7.0
0 0.149286329746246 8.0
0 -0.0528503656387329 9.0
0 -0.0495186448097229 10.0
0 -0.725841641426086 11.0
0 -0.0465668439865112 12.0
77.619795451 76.3213653564453 13.0
13.211694308 23.494607925415 14.0
0 -0.128383874893188 15.0
0 -0.119179546833038 16.0
32.241471181 31.0732479095459 17.0
0 0.0133624672889709 18.0
0 -0.0115712881088257 19.0
0 0.0837217569351196 20.0
0 0.170277118682861 21.0
0 0.104156851768494 22.0
33.003388115 31.2722492218018 23.0
0 0.043437659740448 24.0
0 0.400606602430344 25.0
0 -0.0322107672691345 26.0
0 -0.110609233379364 27.0
0 0.484872102737427 28.0
0 -0.00838387012481689 29.0
0 0.00999099016189575 30.0
0 0.0349607467651367 31.0
0 0.0412015914916992 32.0
0 0.0197930932044983 33.0
0 -0.0644435882568359 34.0
0 0.0512553453445435 35.0
0 -0.334382772445679 36.0
0 0.122527241706848 37.0
0 0.00772351026535034 38.0
0 -0.092347264289856 39.0
0 0.68531745672226 40.0
0 -0.0510326623916626 41.0
0 0.319530487060547 42.0
280.85618501 293.300659179688 43.0
0 -0.066282331943512 44.0
0 0.195289492607117 45.0
0 0.265011787414551 46.0
0 -0.542667031288147 47.0
0 0.202759861946106 48.0
0 0.56846034526825 49.0
0.087094521659 0.00155985355377197 50.0
2.1225821682 1.86745989322662 51.0
0 0.062391996383667 52.0
0 -0.437281012535095 53.0
0 -0.0389732718467712 54.0
0 -0.292843341827393 55.0
0 -0.0675506591796875 56.0
0 0.0404755473136902 57.0
0 0.455682575702667 58.0
0 0.0795129537582397 59.0
0 -0.23335599899292 60.0
0 -0.0172390937805176 61.0
0 0.329895734786987 62.0
288.68310947 253.248413085938 0.0
0 -0.38839054107666 1.0
0 -0.0397356748580933 2.0
0 0.0955413579940796 3.0
0 -0.0941746234893799 4.0
0 0.24592798948288 5.0
0 -0.031751275062561 6.0
0 0.385012686252594 7.0
0 0.582557082176208 8.0
0 0.0182979106903076 9.0
0 0.301617652177811 10.0
0 -0.166962325572968 11.0
0 0.138646006584167 12.0
54.927716053 55.7013702392578 13.0
42.515306541 40.569507598877 14.0
0 -0.0856075286865234 15.0
0 -0.13484799861908 16.0
28.038908665 32.5997543334961 17.0
0 0.0543943643569946 18.0
0 0.0403858423233032 19.0
0 0.0443670749664307 20.0
0 0.215311050415039 21.0
0 0.0431661605834961 22.0
29.083724455 28.9313945770264 23.0
0 0.16691255569458 24.0
0 0.905161559581757 25.0
0 0.0675811767578125 26.0
0 -0.233211994171143 27.0
0 0.33144274353981 28.0
0 0.195430099964142 29.0
0 -0.132939875125885 30.0
0 0.138993501663208 31.0
0 0.0886059999465942 32.0
0 0.12139219045639 33.0
0 0.356171846389771 34.0
0 0.233462333679199 35.0
0 0.102510571479797 36.0
0 -0.0698270201683044 37.0
0 0.263218402862549 38.0
0 -0.251668930053711 39.0
0 0.254396796226501 40.0
0 -0.168075203895569 41.0
0 0.0958051681518555 42.0
294.89439326 295.978454589844 43.0
0 0.314489603042603 44.0
0 0.0032227635383606 45.0
0 0.179052352905273 46.0
0 0.168604910373688 47.0
0 0.00144773721694946 48.0
0 0.355289906263351 49.0
0.092006087383 -0.00531095266342163 50.0
2.1377905596 1.85965144634247 51.0
0 -0.0477832555770874 52.0
0 -0.102840423583984 53.0
0 -0.0215179920196533 54.0
0 -0.0832415819168091 55.0
0 -0.124478578567505 56.0
0 0.175171554088593 57.0
0 0.285268306732178 58.0
0 0.210508286952972 59.0
0 -0.253240823745728 60.0
0 0.0578355193138123 61.0
0 0.183832228183746 62.0
241.01445311 252.957290649414 0.0
0 -0.218259215354919 1.0
0 -0.111905813217163 2.0
0 0.0973502993583679 3.0
0 -0.337981462478638 4.0
0 -0.00753402709960938 5.0
0 0.0310088992118835 6.0
0 0.0103741884231567 7.0
0 -0.0298131704330444 8.0
0 -0.0389285087585449 9.0
0 -0.0789439678192139 10.0
0 -0.276779174804688 11.0
0 -0.0257559418678284 12.0
69.131995679 68.7860870361328 13.0
17.662589933 29.8609828948975 14.0
0 0.0282477140426636 15.0
0 0.0204483270645142 16.0
28.411521536 31.7789478302002 17.0
0 0.090894341468811 18.0
0 0.0453636646270752 19.0
0 0.131612062454224 20.0
0 0.0158790349960327 21.0
0 0.251038312911987 22.0
35.740490402 35.7948570251465 23.0
0 0.0837982296943665 24.0
0 0.560677886009216 25.0
0 -0.0765126943588257 26.0
0 -0.0608071088790894 27.0
0 0.514111399650574 28.0
0 0.0498923063278198 29.0
0 -0.0422143936157227 30.0
0 -0.0992431640625 31.0
0 -0.126762926578522 32.0
0 -0.152819514274597 33.0
0 0.167031824588776 34.0
0 0.0537157654762268 35.0
0 0.0313417911529541 36.0
0 0.141614317893982 37.0
0 0.0718517899513245 38.0
0 -0.175357341766357 39.0
0 0.250916242599487 40.0
0 -0.0854390859603882 41.0
0 -0.057171106338501 42.0
297.30877898 295.745239257812 43.0
0 -0.094836950302124 44.0
0 0.299676418304443 45.0
0 0.224420428276062 46.0
0 0.200975716114044 47.0
0 -0.323856592178345 48.0
0 0.483762949705124 49.0
0.078903344481 -0.00402772426605225 50.0
2.0753104941 1.79499793052673 51.0
0 0.237197518348694 52.0
0 0.00553226470947266 53.0
0 0.181394696235657 54.0
0 0.232336521148682 55.0
0 0.0736110210418701 56.0
0 -0.0105034708976746 57.0
0 -0.221843600273132 58.0
0 -0.0611719489097595 59.0
0 -0.214870810508728 60.0
0 0.126516580581665 61.0
0 0.110017955303192 62.0
219.85847512 254.184524536133 0.0
0 0.173567533493042 1.0
0 -0.0841406583786011 2.0
0 0.306354939937592 3.0
0 -0.380749106407166 4.0
0 -0.342140913009644 5.0
0 -0.00483590364456177 6.0
0 -0.1925328373909 7.0
0 0.211854338645935 8.0
0 0.10502690076828 9.0
0 0.162614107131958 10.0
0 -0.123073220252991 11.0
0 -0.176681995391846 12.0
54.373589458 51.3977355957031 13.0
45.400741121 45.4945602416992 14.0
0 -0.155200362205505 15.0
0 -0.231796741485596 16.0
34.824889582 33.5317306518555 17.0
0 0.0472038984298706 18.0
0 0.00945103168487549 19.0
0 0.196635544300079 20.0
0 -0.12229323387146 21.0
0 -0.0252488851547241 22.0
39.860077407 38.7092399597168 23.0
0 0.0485013127326965 24.0
0 0.523292183876038 25.0
0 0.127119541168213 26.0
0 0.0730965733528137 27.0
0 0.80991131067276 28.0
0 0.0346345901489258 29.0
0 -0.00242400169372559 30.0
0 -0.149254202842712 31.0
0 -0.0972942113876343 32.0
0 0.0682363510131836 33.0
0 0.0882610678672791 34.0
0 0.275581777095795 35.0
0 -0.0973559617996216 36.0
0 -0.0725423097610474 37.0
0 0.277563571929932 38.0
0 -0.249982476234436 39.0
0 0.192361056804657 40.0
0 -0.203550577163696 41.0
0 -0.00137042999267578 42.0
302.41016237 297.170776367188 43.0
0 0.145531415939331 44.0
0 0.218875765800476 45.0
0 -0.00755304098129272 46.0
0 0.0057794451713562 47.0
0 0.272039949893951 48.0
2.5023707677 0.516409993171692 49.0
0.065299455483 0.0809760689735413 50.0
1.5048429251 1.8407084941864 51.0
0 -0.0903382897377014 52.0
0 0.0252331495285034 53.0
0 -0.112310707569122 54.0
0 -0.279098629951477 55.0
0 -0.040730893611908 56.0
0 -0.0458468198776245 57.0
0 0.151046335697174 58.0
0 -1.12710082530975 59.0
0 1.68858695030212 60.0
0 -0.00735336542129517 61.0
0 0.109219670295715 62.0
212.26781159 253.277633666992 0.0
0 0.124260663986206 1.0
0 -0.109107434749603 2.0
0 0.095892608165741 3.0
0 -0.241381883621216 4.0
0 -0.25320565700531 5.0
0 -0.0433143973350525 6.0
0 -0.371476411819458 7.0
0 0.208321332931519 8.0
0 0.154081404209137 9.0
0 -0.239709496498108 10.0
0 -0.121768474578857 11.0
0 -0.126840710639954 12.0
69.331706772 69.5698394775391 13.0
34.542107074 25.0115146636963 14.0
0 -0.0298895835876465 15.0
0 -0.250094652175903 16.0
40.854444328 38.441047668457 17.0
0 0.216565489768982 18.0
0 -0.0585086345672607 19.0
0 0.0627226829528809 20.0
0 -0.0343728065490723 21.0
0 0.303781986236572 22.0
35.644893517 35.2049293518066 23.0
0 0.0616980195045471 24.0
0 0.445822238922119 25.0
0 0.0423967838287354 26.0
0 -0.17038768529892 27.0
0 0.0688530802726746 28.0
0 -0.00168824195861816 29.0
0 0.0236380100250244 30.0
0 -0.262460470199585 31.0
0 -0.138234257698059 32.0
0 0.00176435708999634 33.0
0 0.0187304615974426 34.0
0 0.178490102291107 35.0
0 0.074185848236084 36.0
0 0.17462694644928 37.0
0 -0.0289409160614014 38.0
0 -0.144525647163391 39.0
0 0.44004824757576 40.0
0 -0.186309218406677 41.0
0 0.202504992485046 42.0
295.20429435 296.510406494141 43.0
0 0.196640908718109 44.0
0 0.201845109462738 45.0
0 0.169108748435974 46.0
0 -0.50691282749176 47.0
0 -0.592325925827026 48.0
0 0.465530782938004 49.0
0.082006154749 0.0101378560066223 50.0
1.7727399108 1.8213951587677 51.0
0 -0.0324608087539673 52.0
0 -0.621560096740723 53.0
0 -0.0433074235916138 54.0
0 0.202968239784241 55.0
0 -0.0251120328903198 56.0
0 0.0609106421470642 57.0
0 0.480051398277283 58.0
0 0.255193710327148 59.0
0 -0.247591733932495 60.0
0 0.0338031649589539 61.0
0 0.326053261756897 62.0
286.87036867 252.174301147461 0.0
0 -0.331234812736511 1.0
0 -0.0176225304603577 2.0
0 0.396594613790512 3.0
0 0.499641060829163 4.0
0 -0.222045302391052 5.0
0 0.554902076721191 6.0
0 0.454622983932495 7.0
0 0.523631513118744 8.0
0 0.00748169422149658 9.0
0 0.764685571193695 10.0
0 0.451721787452698 11.0
0 -0.254492282867432 12.0
46.68744948 44.4764404296875 13.0
42.538988478 39.8647956848145 14.0
0 -0.171733736991882 15.0
0 0.0720202326774597 16.0
41.726721964 38.916374206543 17.0
0 0.259329795837402 18.0
0 0.0167144536972046 19.0
0 -0.0763814449310303 20.0
0 0.290775537490845 21.0
0 0.0866267681121826 22.0
31.800116824 29.223030090332 23.0
0 0.0463694930076599 24.0
6.0105823476 0.446329116821289 25.0
0 -0.187714517116547 26.0
0 0.077538788318634 27.0
0 -0.00477981567382812 28.0
0 0.0663113594055176 29.0
0 0.0962340235710144 30.0
0 0.11325991153717 31.0
0 0.118826389312744 32.0
0 0.195504903793335 33.0
0 -0.0694684982299805 34.0
0 0.0379664897918701 35.0
0 -0.127066373825073 36.0
0 -0.0662226676940918 37.0
0 -0.0189792513847351 38.0
0 -0.208718299865723 39.0
0 -0.0484418272972107 40.0
0 -0.328062176704407 41.0
0 0.0689107179641724 42.0
268.67405968 295.609619140625 43.0
0 0.0558821558952332 44.0
0 -0.00360953807830811 45.0
0 0.138714790344238 46.0
0 0.0267524719238281 47.0
0 0.270399749279022 48.0
0.68782398798 0.156444072723389 49.0
0.059818228005 0.0530290007591248 50.0
1.664009326 1.85301065444946 51.0
0 0.158141136169434 52.0
0 -0.384621381759644 53.0
0 -0.142131865024567 54.0
0 0.0181351900100708 55.0
0 0.40267139673233 56.0
0 -0.00832146406173706 57.0
0 0.279458820819855 58.0
0 -0.623432278633118 59.0
0 0.069710910320282 60.0
0 0.0401392579078674 61.0
0 0.133064925670624 62.0
260.32997917 255.1181640625 0.0
0 -0.247223854064941 1.0
0 0.0781985521316528 2.0
0 -0.0258334279060364 3.0
0 0.247116327285767 4.0
0 0.217213034629822 5.0
0 0.479806423187256 6.0
0 0.151838958263397 7.0
0 -0.429352402687073 8.0
0 1.03618395328522 9.0
0 0.509849309921265 10.0
0 -0.158867716789246 11.0
0 -0.0252047181129456 12.0
53.794217428 55.1088600158691 13.0
29.555356257 44.4863815307617 14.0
0 0.0240253210067749 15.0
0 0.0848049521446228 16.0
36.132153611 36.8485794067383 17.0
0 0.223734498023987 18.0
0 0.0128636956214905 19.0
0 0.0303496718406677 20.0
0 0.167524635791779 21.0
0 0.0285512208938599 22.0
32.684494684 29.894474029541 23.0
0 0.0540900826454163 24.0
0 0.747822344303131 25.0
0 -0.0210738182067871 26.0
0 -0.132508158683777 27.0
0 0.694100975990295 28.0
0 -0.054685115814209 29.0
0 0.0901992917060852 30.0
0 0.149143218994141 31.0
0 -0.231601476669312 32.0
0 -0.349848747253418 33.0
0 0.252049028873444 34.0
0 0.38567066192627 35.0
0 0.316500931978226 36.0
0 0.0261155962944031 37.0
0 0.472240924835205 38.0
0 -0.375954031944275 39.0
0 -0.329127669334412 40.0
0 -0.19830846786499 41.0
0 -0.158195972442627 42.0
298.25777384 296.590270996094 43.0
0 -0.0553626418113708 44.0
0 0.134718596935272 45.0
0 0.283160209655762 46.0
0 0.08477783203125 47.0
0 0.391193985939026 48.0
0 0.620926380157471 49.0
0.10499294142 0.0139678120613098 50.0
1.8857461447 1.80436384677887 51.0
0 0.14689964056015 52.0
0 -0.25629734992981 53.0
0 0.0894715189933777 54.0
0 0.0722545385360718 55.0
0 0.341936886310577 56.0
0 -0.0742709040641785 57.0
0 -0.102780163288116 58.0
0 0.0136508345603943 59.0
0 -0.0654841065406799 60.0
0 0.0891258716583252 61.0
0 0.0157164931297302 62.0
276.44899054 256.427154541016 0.0
0 -0.320270299911499 1.0
0 0.0215480327606201 2.0
0 0.290115177631378 3.0
0 -0.136104047298431 4.0
0 0.0868834257125854 5.0
0 -0.109255313873291 6.0
0 0.239728033542633 7.0
0 0.152765154838562 8.0
0 0.0737131834030151 9.0
0 0.193073868751526 10.0
0 -0.165257811546326 11.0
0 -0.176190555095673 12.0
51.067754552 52.6444511413574 13.0
38.532063453 42.3882179260254 14.0
0 -0.0301802158355713 15.0
0 0.042557418346405 16.0
39.585584893 38.4210662841797 17.0
0 0.304988861083984 18.0
0 -0.0755243897438049 19.0
0 -0.0403555631637573 20.0
0 0.221082448959351 21.0
0 0.0966235399246216 22.0
27.558684383 26.872766494751 23.0
0 0.0799661874771118 24.0
0 0.807979643344879 25.0
0 -0.0047764778137207 26.0
0 -0.0849176049232483 27.0
0 0.332974165678024 28.0
0 0.0255724191665649 29.0
0 0.0819140672683716 30.0
0 -0.0718048810958862 31.0
0 0.0128598809242249 32.0
0 0.0232195854187012 33.0
0 0.283550798892975 34.0
0 0.0691357851028442 35.0
0 0.20966112613678 36.0
0 -0.0572916269302368 37.0
0 0.0674005746841431 38.0
0 -0.114437341690063 39.0
0 0.629135727882385 40.0
0 -0.117042541503906 41.0
0 0.0634284019470215 42.0
288.39837811 297.929077148438 43.0
0 0.0234556794166565 44.0
0 0.196554720401764 45.0
0 0.181949019432068 46.0
0 -0.189414381980896 47.0
0 0.364633560180664 48.0
1.2085313439 0.434526860713959 49.0
0.08918143139 0.0660233497619629 50.0
1.5537229536 1.87956237792969 51.0
0 -0.198808550834656 52.0
0 -0.220049023628235 53.0
0 0.160079598426819 54.0
0 -0.0829463601112366 55.0
0 -0.136263906955719 56.0
0 0.0450804233551025 57.0
0 0.284298300743103 58.0
0 0.191664040088654 59.0
0 -0.0926127433776855 60.0
0 0.146131753921509 61.0
0 0.371071934700012 62.0
286.04089005 252.20524597168 0.0
0 0.0625841617584229 1.0
0 0.11505001783371 2.0
0 0.225047945976257 3.0
0 0.144084990024567 4.0
0 0.470212936401367 5.0
0 -0.202252864837646 6.0
0 0.758159756660461 7.0
0 0.386914789676666 8.0
0 -0.374189257621765 9.0
0 0.221906185150146 10.0
0 -0.747443675994873 11.0
0 -0.265805721282959 12.0
67.496237578 67.6674499511719 13.0
21.767624332 27.5069274902344 14.0
0 -0.0582354068756104 15.0
0 -0.140147626399994 16.0
28.503457818 31.5791110992432 17.0
0 0.0705608129501343 18.0
0 0.0212833881378174 19.0
0 0.0282827615737915 20.0
0 0.307723343372345 21.0
0 0.0643616914749146 22.0
28.919316228 24.5637454986572 23.0
0 0.123758852481842 24.0
0 0.893301367759705 25.0
0 -0.11473822593689 26.0
0 -0.254509449005127 27.0
0 0.430717021226883 28.0
0 0.146727204322815 29.0
0 -0.00540971755981445 30.0
0 0.0266205072402954 31.0
0 -0.0912235975265503 32.0
0 -0.497658967971802 33.0
0 -0.440264463424683 34.0
0 0.00412267446517944 35.0
0 -0.157528638839722 36.0
0 0.256788492202759 37.0
0 0.266130805015564 38.0
0 -0.127471208572388 39.0
0 0.608070611953735 40.0
0 -0.102439999580383 41.0
0 -0.0691924095153809 42.0
298.82843799 297.844116210938 43.0
0 0.310091018676758 44.0
0 -0.224937200546265 45.0
0 0.139047861099243 46.0
0 -0.00354701280593872 47.0
0 0.340127348899841 48.0
0 0.336285352706909 49.0
0.090661758896 0.0773018598556519 50.0
1.620437118 1.80305159091949 51.0
0 0.0192317366600037 52.0
0 0.460508108139038 53.0
0 0.0852758884429932 54.0
0 0.0775084495544434 55.0
0 -0.156278967857361 56.0
0 -0.0499857664108276 57.0
0 0.455002427101135 58.0
0 0.180337131023407 59.0
0 -0.276075482368469 60.0
0 0.0142277479171753 61.0
0 0.210981607437134 62.0
262.06197997 252.425033569336 0.0
0 -0.24746561050415 1.0
0 -0.0965042114257812 2.0
0 0.370877921581268 3.0
0 0.116035997867584 4.0
0 -0.0363789200782776 5.0
0 -0.0405884385108948 6.0
0 0.307927906513214 7.0
0 -0.0410938858985901 8.0
0 -0.133296847343445 9.0
0 0.130328297615051 10.0
0 -0.204580843448639 11.0
0 0.17206859588623 12.0
52.012106648 53.7878150939941 13.0
41.3491415 44.6928558349609 14.0
0 -0.089091420173645 15.0
0 -0.0304330587387085 16.0
35.292382721 35.9592819213867 17.0
0 0.0553199052810669 18.0
0 0.0126208066940308 19.0
0 -0.0106506943702698 20.0
0 0.114126741886139 21.0
0 0.147080302238464 22.0
29.988754544 30.1996593475342 23.0
0 0.114137828350067 24.0
0 0.556739091873169 25.0
0 -0.0403265953063965 26.0
0 -0.127755582332611 27.0
0 0.107088506221771 28.0
0 -0.0604393482208252 29.0
0 0.0399618148803711 30.0
0 -0.225836515426636 31.0
0 -0.0848903656005859 32.0
0 -0.107374906539917 33.0
0 0.165036916732788 34.0
0 -0.0602673292160034 35.0
0 -0.544037103652954 36.0
0 -0.0749890208244324 37.0
0 0.176955044269562 38.0
0 -0.19349217414856 39.0
0 0.152126371860504 40.0
0 -0.16427081823349 41.0
0 -0.0727722644805908 42.0
306.312585 295.633453369141 43.0
0 0.19620668888092 44.0
0 0.0616793632507324 45.0
0 0.0652785301208496 46.0
0 0.07657390832901 47.0
0 0.28341543674469 48.0
0 0.311612755060196 49.0
0.09814811083 0.0655739307403564 50.0
1.8141370145 1.83156108856201 51.0
0 0.271352648735046 52.0
0 -0.242421746253967 53.0
0 -0.019628643989563 54.0
0 -0.203059673309326 55.0
0 -0.0496134161949158 56.0
0 0.360472857952118 57.0
0 0.508033514022827 58.0
0 0.14753782749176 59.0
0 0.436645925045013 60.0
0 0.0547281503677368 61.0
0 0.191366314888 62.0
282.29800653 253.61994934082 0.0
0 -0.412787318229675 1.0
0 0.0331396460533142 2.0
0 -0.0106827020645142 3.0
0 -0.384422898292542 4.0
0 0.286339521408081 5.0
0 -0.199289560317993 6.0
0 0.419977128505707 7.0
0 0.829216718673706 8.0
0 -0.215185165405273 9.0
0 0.180465221405029 10.0
0 -0.0954039096832275 11.0
0 -0.085770845413208 12.0
64.522614861 65.3556976318359 13.0
26.340123109 30.0214347839355 14.0
0 -0.0641411542892456 15.0
0 0.0572206377983093 16.0
27.440309454 35.3443832397461 17.0
0 0.0865616798400879 18.0
0 -0.0216753482818604 19.0
0 0.123957633972168 20.0
0 -0.0551110506057739 21.0
0 0.2471684217453 22.0
38.725814425 35.6988372802734 23.0
0 0.0230705142021179 24.0
0 0.31089323759079 25.0
0 -0.0255204439163208 26.0
0 -0.0214950442314148 27.0
0 0.752955913543701 28.0
0 0.0159443020820618 29.0
0 -0.0986787080764771 30.0
0 0.00709152221679688 31.0
0 0.0988532900810242 32.0
0 0.109894931316376 33.0
0 0.319880723953247 34.0
0 -0.163141250610352 35.0
0 -0.148854196071625 36.0
0 -0.0849798917770386 37.0
0 0.222182989120483 38.0
0 -0.244038224220276 39.0
0 0.0476486682891846 40.0
0 -0.170964539051056 41.0
0 -0.0488135814666748 42.0
295.22529275 297.829254150391 43.0
0 0.184563994407654 44.0
0 -0.0163527727127075 45.0
0 0.204408943653107 46.0
0 -0.216580867767334 47.0
0 0.317565649747849 48.0
0 0.35484179854393 49.0
0.084778115279 0.0473722815513611 50.0
1.9927972116 1.89982199668884 51.0
0 0.115506410598755 52.0
0 -0.0914251208305359 53.0
0 0.130849719047546 54.0
0 0.238360524177551 55.0
0 0.172404587268829 56.0
0 0.519239485263824 57.0
0 0.185660839080811 58.0
0 -0.0398176312446594 59.0
0 -0.0246809124946594 60.0
0 0.0049431324005127 61.0
0 0.0851495862007141 62.0
281.0586102 254.637680053711 0.0
0 -0.314082980155945 1.0
0 0.0413156747817993 2.0
0 0.258672118186951 3.0
0 0.0181964039802551 4.0
0 0.0399019718170166 5.0
0 0.182217121124268 6.0
0 0.364252239465714 7.0
0 0.239571213722229 8.0
0 0.194218993186951 9.0
0 0.355595111846924 10.0
0 -0.23437774181366 11.0
0 -0.09693443775177 12.0
53.325408415 54.3219032287598 13.0
37.395221491 39.859790802002 14.0
0 -0.0442098379135132 15.0
0 0.0860838890075684 16.0
32.285681142 34.8663482666016 17.0
0 0.132346987724304 18.0
0 -0.0323784947395325 19.0
0 0.14129626750946 20.0
0 -0.0607656240463257 21.0
0 0.259097874164581 22.0
38.741625692 37.8378639221191 23.0
0 0.101832389831543 24.0
0 0.709013938903809 25.0
0 0.0345728397369385 26.0
0 0.125128984451294 27.0
0 0.110202074050903 28.0
0 -0.0578833222389221 29.0
0 0.0692017674446106 30.0
0 0.321388125419617 31.0
0 0.16387403011322 32.0
0 -0.0309011340141296 33.0
0 0.203243732452393 34.0
0 -0.157978475093842 35.0
0 0.0549362897872925 36.0
0 -0.0927539467811584 37.0
0 0.107836067676544 38.0
0 -0.105308413505554 39.0
0 0.339414656162262 40.0
0 -0.174027979373932 41.0
0 -0.0111572742462158 42.0
298.47339062 295.531982421875 43.0
0 0.136181712150574 44.0
0 0.10927826166153 45.0
0 0.328933656215668 46.0
0 -0.289974927902222 47.0
0 0.132184863090515 48.0
0 0.387522459030151 49.0
0.074228196102 -0.0168039202690125 50.0
1.714387499 1.90023839473724 51.0
0 -0.106041312217712 52.0
0 -0.288488626480103 53.0
0 -0.0222106575965881 54.0
0 0.0573905110359192 55.0
0 0.0511055588722229 56.0
0 0.0461200475692749 57.0
0 0.38097083568573 58.0
0 0.370005309581757 59.0
0 -0.163448214530945 60.0
0 0.0954682230949402 61.0
0 0.323950409889221 62.0
267.36616709 252.224349975586 0.0
0 -0.239927530288696 1.0
0 -0.103434801101685 2.0
0 0.492177963256836 3.0
0 -0.0536589026451111 4.0
0 0.335816740989685 5.0
0 0.0672512054443359 6.0
0 0.165054857730865 7.0
0 -0.0351725220680237 8.0
0 -0.195704102516174 9.0
0 -0.191999018192291 10.0
0 -0.340330719947815 11.0
0 -0.0389015674591064 12.0
59.51851307 59.4449272155762 13.0
32.089554859 37.4425849914551 14.0
0 -0.147760987281799 15.0
0 0.00305420160293579 16.0
32.613319972 34.6257476806641 17.0
0 0.106751263141632 18.0
0 -0.00952982902526855 19.0
0 0.0631054639816284 20.0
0 0.186223447322845 21.0
0 0.0467154383659363 22.0
28.455497503 27.8527278900146 23.0
0 0.0687276124954224 24.0
0 0.717046141624451 25.0
0 -0.0227215886116028 26.0
0 -0.0787056684494019 27.0
0 0.330496311187744 28.0
0 0.0946787595748901 29.0
0 0.130169987678528 30.0
0 -0.00354719161987305 31.0
0 -0.0948197245597839 32.0
0 -0.0247296094894409 33.0
0 -0.00223666429519653 34.0
0 0.156705498695374 35.0
0 0.161962270736694 36.0
0 0.13103574514389 37.0
0 -0.0387719869613647 38.0
0 -0.0925168991088867 39.0
0 0.273957967758179 40.0
0 -0.123058676719666 41.0
0 0.0792515277862549 42.0
295.09069198 296.785858154297 43.0
0 0.108933925628662 44.0
0 0.0519645214080811 45.0
0 0.150413393974304 46.0
0 0.111414611339569 47.0
0 0.286581099033356 48.0
0 0.487246751785278 49.0
0.077222902259 0.0446434020996094 50.0
2.2243827985 1.91608428955078 51.0
0 0.292454719543457 52.0
0 -0.0616775155067444 53.0
0 0.0572072267532349 54.0
0 0.23891294002533 55.0
0 0.330863356590271 56.0
0 0.0221383571624756 57.0
0 0.0496426820755005 58.0
0 0.316710501909256 59.0
0 0.00290268659591675 60.0
0 0.115188300609589 61.0
0 0.237564921379089 62.0
279.69863451 250.649261474609 0.0
0 -0.251819491386414 1.0
0 0.0482200384140015 2.0
0 0.19103991985321 3.0
0 -0.24013614654541 4.0
0 0.365714520215988 5.0
0 0.525178968906403 6.0
0 0.364152610301971 7.0
0 0.0550576448440552 8.0
0 0.0186184048652649 9.0
0 -0.0387647747993469 10.0
0 -0.0826236605644226 11.0
0 -0.167157292366028 12.0
52.722140636 49.9628639221191 13.0
49.177139343 49.8520851135254 14.0
0 -0.173841953277588 15.0
0 -0.2078537940979 16.0
26.778336194 32.9982986450195 17.0
0 0.083396852016449 18.0
0 -0.105425357818604 19.0
0 0.153681039810181 20.0
0 0.0248628854751587 21.0
0 0.208469986915588 22.0
24.585865192 25.9132785797119 23.0
0 0.126641631126404 24.0
0 0.747483909130096 25.0
0 -0.0813473463058472 26.0
0 -0.0436220765113831 27.0
0 0.00780785083770752 28.0
0 0.220645427703857 29.0
0 0.108547687530518 30.0
0 0.146937966346741 31.0
0 0.0669587850570679 32.0
0 0.179250240325928 33.0
0 0.103487253189087 34.0
0 -0.0422198176383972 35.0
0 -0.0570981502532959 36.0
0 -0.0899749994277954 37.0
0 0.0606994032859802 38.0
0 -0.304410457611084 39.0
0 0.0165780782699585 40.0
0 -0.332532405853271 41.0
0 0.0558023452758789 42.0
282.13151882 295.782745361328 43.0
0 0.455065250396729 44.0
0 -0.057727038860321 45.0
0 0.179483115673065 46.0
0 -0.0912917852401733 47.0
0 0.381809502840042 48.0
0 0.385479390621185 49.0
0.064330153962 0.0666570663452148 50.0
1.73731272 1.88008451461792 51.0
0 0.0470985174179077 52.0
0 -0.0488781929016113 53.0
0 -0.16509485244751 54.0
0 0.0653973817825317 55.0
0 0.00468510389328003 56.0
0 0.0529863834381104 57.0
0 -0.355316758155823 58.0
0 0.0558429956436157 59.0
0 -0.0673577785491943 60.0
0 0.232333779335022 61.0
0 0.353418290615082 62.0
247.90429988 251.475311279297 0.0
0 0.439363956451416 1.0
0 -0.0532659292221069 2.0
0 -0.00435084104537964 3.0
0 -0.33306622505188 4.0
0 -0.3829185962677 5.0
0 0.120867967605591 6.0
0 0.0624599456787109 7.0
0 0.578768968582153 8.0
0 -0.672520875930786 9.0
0 0.111146032810211 10.0
0 -0.41745400428772 11.0
0 -0.208437204360962 12.0
67.228804805 67.8154907226562 13.0
31.005208712 28.3557395935059 14.0
0 -0.0506672859191895 15.0
0 0.189389705657959 16.0
29.020012376 29.5542812347412 17.0
0 0.0628392696380615 18.0
0 0.0566091537475586 19.0
0 0.0519949197769165 20.0
0 0.193963646888733 21.0
0 0.0773683786392212 22.0
29.093371863 28.0942325592041 23.0
0 0.131386816501617 24.0
0 1.10342073440552 25.0
0 -0.0774307250976562 26.0
0 -0.240476131439209 27.0
0 0.844072103500366 28.0
0 -0.0255832672119141 29.0
0 0.0598945021629333 30.0
0 0.102451086044312 31.0
0 -0.0458595752716064 32.0
0 -0.194573104381561 33.0
0 0.0217831134796143 34.0
0 0.33068522810936 35.0
0 0.627572357654572 36.0
0 0.107150614261627 37.0
0 0.275187432765961 38.0
0 -0.449225068092346 39.0
0 0.0594899654388428 40.0
0 -0.211273431777954 41.0
0 0.0139156579971313 42.0
305.5731127 293.811492919922 43.0
0 -0.713957190513611 44.0
0 -0.211546182632446 45.0
0 0.143764495849609 46.0
0 0.187180638313293 47.0
0 0.259766101837158 48.0
0 0.378442168235779 49.0
0.094667944011 0.0749192833900452 50.0
1.4407456639 1.81048834323883 51.0
0 0.11416232585907 52.0
0 -0.344654679298401 53.0
0 0.0114120244979858 54.0
0 0.0134778022766113 55.0
0 -0.0998077392578125 56.0
0 0.236111998558044 57.0
0 0.31803959608078 58.0
0 0.235650539398193 59.0
0 0.0268327593803406 60.0
0 -0.0160419344902039 61.0
0 0.110752940177917 62.0
278.63259851 256.167633056641 0.0
0 -0.337453126907349 1.0
0 -0.00300109386444092 2.0
0 -0.129583358764648 3.0
0 -0.301868915557861 4.0
0 0.344660997390747 5.0
0 -0.0054326057434082 6.0
0 0.448536574840546 7.0
0 0.280518770217896 8.0
0 0.164377868175507 9.0
0 0.39180913567543 10.0
0 -0.2476646900177 11.0
0 -0.279957056045532 12.0
52.502901386 53.1422119140625 13.0
35.312734919 36.1855316162109 14.0
0 -0.0577919483184814 15.0
0 -0.138899743556976 16.0
33.47810802 34.772762298584 17.0
0 0.0890838503837585 18.0
0 -0.0174847841262817 19.0
0 0.0910866260528564 20.0
0 0.0534945130348206 21.0
0 0.160613059997559 22.0
37.645697034 35.0834922790527 23.0
0 0.0279799103736877 24.0
0 0.44699490070343 25.0
0 0.00673079490661621 26.0
0 -0.0297527313232422 27.0
0 -0.470627188682556 28.0
0 -0.024941086769104 29.0
0 0.0262438058853149 30.0
0 -0.0854183435440063 31.0
0 0.0973727703094482 32.0
0 0.0373057723045349 33.0
0 -0.119176745414734 34.0
0 -0.0365910530090332 35.0
0 0.271375775337219 36.0
0 0.0425268411636353 37.0
0 -0.13692033290863 38.0
0 -0.196319341659546 39.0
0 0.563494741916656 40.0
0 -0.179272532463074 41.0
0 0.0653753280639648 42.0
275.12698703 295.573944091797 43.0
0 0.123619556427002 44.0
0 0.251375138759613 45.0
0 -0.00738328695297241 46.0
0 0.143395900726318 47.0
0 0.311528235673904 48.0
1.3918829858 0.376232862472534 49.0
0.095072127127 0.00716203451156616 50.0
2.0717044307 1.92618846893311 51.0
0 0.0157530903816223 52.0
0 -0.119637548923492 53.0
0 -0.0148560404777527 54.0
0 0.187783777713776 55.0
0 0.300010621547699 56.0
0 0.4256452023983 57.0
0 0.273771405220032 58.0
0 0.15881359577179 59.0
0 0.0493764877319336 60.0
0 -0.00456106662750244 61.0
0 0.218026757240295 62.0
283.24356847 253.586318969727 0.0
0 -0.298165917396545 1.0
0 0.350130438804626 2.0
0 -0.487814903259277 3.0
0 -0.0467729568481445 4.0
0 0.623813152313232 5.0
0 -0.401149868965149 6.0
0 0.929032146930695 7.0
0 0.990547299385071 8.0
0 0.18788480758667 9.0
0 0.501224756240845 10.0
0 -0.488678574562073 11.0
0 -0.164014577865601 12.0
78.793043392 75.9119720458984 13.0
5.9688303395 20.5006694793701 14.0
0 -0.0674930810928345 15.0
0 0.164091467857361 16.0
37.401295111 36.5528373718262 17.0
0 0.223506569862366 18.0
0 -0.0132026076316833 19.0
0 -0.0324140787124634 20.0
0 0.261640787124634 21.0
0 0.106272220611572 22.0
37.515664868 38.337890625 23.0
0 0.0666708946228027 24.0
0 0.670984745025635 25.0
0 0.137977719306946 26.0
0 0.104072093963623 27.0
0 0.478308945894241 28.0
0 0.0513627529144287 29.0
0 0.225623726844788 30.0
0 -0.152877688407898 31.0
0 -0.165505886077881 32.0
0 -0.0208842754364014 33.0
0 0.156009316444397 34.0
0 0.164834082126617 35.0
0 -0.119698345661163 36.0
0 0.080946683883667 37.0
0 0.00368374586105347 38.0
0 -0.166526913642883 39.0
0 0.713625848293304 40.0
0 -0.125114798545837 41.0
0 0.411241054534912 42.0
287.09574607 296.540710449219 43.0
0 -0.0951595902442932 44.0
0 0.172917664051056 45.0
0 0.0593215227127075 46.0
0 0.155798137187958 47.0
0 -0.23270308971405 48.0
0 0.588774085044861 49.0
0.073718297104 -0.00866973400115967 50.0
1.4351162071 1.7902135848999 51.0
0 0.145186901092529 52.0
0 -0.190328180789948 53.0
0 -0.195124387741089 54.0
0 0.286969184875488 55.0
0 0.35277384519577 56.0
0 0.0712588429450989 57.0
0 -0.00443953275680542 58.0
0 -1.03025579452515 59.0
0 1.84315979480743 60.0
0 0.0293154716491699 61.0
0 0.108105778694153 62.0
287.03912195 254.736511230469 0.0
0 0.105507135391235 1.0
0 0.139402866363525 2.0
0 0.00409424304962158 3.0
0 0.248415052890778 4.0
0 -0.360042333602905 5.0
0 -0.259665608406067 6.0
0 0.261356353759766 7.0
0 -0.338349580764771 8.0
0 -0.0588072538375854 9.0
0 -0.196185410022736 10.0
0 1.18940949440002 11.0
0 0.0395706295967102 12.0
74.904855998 74.0215225219727 13.0
35.309664555 31.9283924102783 14.0
0 0.201087594032288 15.0
0 -0.181869089603424 16.0
31.201515712 30.6964130401611 17.0
0 0.0127038955688477 18.0
0 0.00676059722900391 19.0
0 0.155448019504547 20.0
0 0.153862535953522 21.0
0 0.159675061702728 22.0
33.624895939 31.0946063995361 23.0
0 0.0699157118797302 24.0
0 0.836190342903137 25.0
0 -0.150109708309174 26.0
0 -0.260220527648926 27.0
0 0.931578814983368 28.0
0 -0.00126266479492188 29.0
0 0.134814858436584 30.0
0 0.151211261749268 31.0
0 0.173850476741791 32.0
0 0.00358366966247559 33.0
0 0.296451032161713 34.0
0 -0.0173498392105103 35.0
0 0.2607461810112 36.0
0 0.0639736652374268 37.0
0 -0.0924869775772095 38.0
0 -0.233743190765381 39.0
0 -0.0553767681121826 40.0
0 -0.308048725128174 41.0
0 -0.119074106216431 42.0
290.01885354 296.275726318359 43.0
0 -0.00312024354934692 44.0
0 0.13383412361145 45.0
0 0.143283903598785 46.0
0 0.225863695144653 47.0
0 0.369823038578033 48.0
2.38711896 0.39552104473114 49.0
0.10105636817 0.03874272108078 50.0
1.8364142282 1.90336108207703 51.0
0 -0.0523377060890198 52.0
0 0.0733038783073425 53.0
0 -0.0677173137664795 54.0
0 0.285506427288055 55.0
0 0.163813889026642 56.0
0 0.293245911598206 57.0
0 0.0482162833213806 58.0
0 -0.150777339935303 59.0
0 -0.16517037153244 60.0
0 -0.0217179656028748 61.0
0 -0.0483905673027039 62.0
210.3732136 253.708557128906 0.0
0 -0.0844913721084595 1.0
0 -0.0675557851791382 2.0
0 -0.437165975570679 3.0
0 -0.299936771392822 4.0
0 -0.319853186607361 5.0
0 -0.453268527984619 6.0
0 -0.290265083312988 7.0
0 0.65199738740921 8.0
0 -0.33857536315918 9.0
0 1.87996292114258 10.0
0 -0.0240510106086731 11.0
0 -0.156749546527863 12.0
70.222261391 70.7504806518555 13.0
9.5193853789 31.9469680786133 14.0
0 0.0010608434677124 15.0
0 -0.285299777984619 16.0
32.995929292 33.226245880127 17.0
0 0.13109564781189 18.0
0 -0.00200545787811279 19.0
0 0.113735318183899 20.0
0 -0.00283396244049072 21.0
0 0.279717206954956 22.0
28.556578551 29.0044727325439 23.0
0 0.111441552639008 24.0
0 0.756137371063232 25.0
0 -0.195460617542267 26.0
0 -0.14542019367218 27.0
0 0.0182142853736877 28.0
0 0.0434970855712891 29.0
0 -0.0912819504737854 30.0
0 0.0388665199279785 31.0
0 -0.0648741722106934 32.0
0 -0.453192830085754 33.0
0 -0.0761228799819946 34.0
0 0.239756464958191 35.0
0 0.222369909286499 36.0
0 0.133697509765625 37.0
0 -0.116116642951965 38.0
0 -0.0726609230041504 39.0
0 0.560636281967163 40.0
0 -0.0668272376060486 41.0
0 0.531576633453369 42.0
291.7244426 292.319641113281 43.0
0 -0.0279485583305359 44.0
0 0.276320576667786 45.0
0 0.076315701007843 46.0
0 0.00500130653381348 47.0
0 -1.18592464923859 48.0
0 0.557383239269257 49.0
0.068969803772 -0.0263758897781372 50.0
1.505824475 1.79262614250183 51.0
0 -0.0266069173812866 52.0
0 0.0982417464256287 53.0
0 -0.0379016399383545 54.0
0 -0.0672787427902222 55.0
0 0.244805932044983 56.0
0 -0.0572682619094849 57.0
0 0.182576775550842 58.0
0 0.245633900165558 59.0
0 -0.20256781578064 60.0
0 0.142645418643951 61.0
0 0.399393439292908 62.0
280.13984062 252.090606689453 0.0
0 -0.100390791893005 1.0
0 0.078359842300415 2.0
0 0.150788307189941 3.0
0 0.246216535568237 4.0
0 0.165888845920563 5.0
0 0.295756638050079 6.0
0 0.333462983369827 7.0
0 0.068174421787262 8.0
0 0.21729838848114 9.0
0 -0.00703537464141846 10.0
0 -0.697834134101868 11.0
0 -0.156884372234344 12.0
76.869278494 76.4887237548828 13.0
8.516960043 20.1036529541016 14.0
0 -0.263198375701904 15.0
0 -0.212650537490845 16.0
31.69258874 33.4635047912598 17.0
0 0.0235092639923096 18.0
0 -0.0394519567489624 19.0
0 0.102247059345245 20.0
0 0.203298687934875 21.0
0 0.0204619169235229 22.0
28.345926367 29.5313034057617 23.0
0 0.103900790214539 24.0
0 0.977118492126465 25.0
0 0.165460228919983 26.0
0 -0.166513204574585 27.0
0 0.739845454692841 28.0
0 -0.0558738112449646 29.0
0 0.0535216331481934 30.0
0 -0.0175970792770386 31.0
0 0.0793066620826721 32.0
0 0.155768156051636 33.0
0 0.29257270693779 34.0
0 -0.0661135911941528 35.0
0 0.0450171232223511 36.0
0 -0.0209600329399109 37.0
0 0.1346195936203 38.0
0 -0.243254899978638 39.0
0 0.605776190757751 40.0
0 -0.0713438391685486 41.0
0 0.132691025733948 42.0
281.95434016 296.741271972656 43.0
0 0.119067430496216 44.0
0 0.177230060100555 45.0
0 0.182077348232269 46.0
0 0.0098758339881897 47.0
0 0.308796763420105 48.0
2.1983389161 0.685673475265503 49.0
0.093345219357 0.0657796859741211 50.0
2.0193049116 1.90018117427826 51.0
0 0.154643654823303 52.0
0 0.0722783803939819 53.0
0 0.0490398406982422 54.0
0 -0.0411549210548401 55.0
0 -0.0642781257629395 56.0
0 0.121390342712402 57.0
0 0.19851154088974 58.0
0 -0.0149698257446289 59.0
0 -0.00071406364440918 60.0
0 0.12920743227005 61.0
0 0.384694933891296 62.0
242.57443679 253.398147583008 0.0
0 -0.188178896903992 1.0
0 -0.159292995929718 2.0
0 0.0783643126487732 3.0
0 -0.144128382205963 4.0
0 -0.0247495770454407 5.0
0 0.0925622582435608 6.0
0 0.205977141857147 7.0
0 -0.0300712585449219 8.0
0 -0.114453434944153 9.0
0 0.287790656089783 10.0
0 -0.41676938533783 11.0
0 -0.081102728843689 12.0
72.261032236 71.4586029052734 13.0
16.752060994 27.7662925720215 14.0
0 -0.0350663661956787 15.0
0 0.0261229276657104 16.0
39.82912236 35.6243591308594 17.0
0 0.186782598495483 18.0
0 -0.0561226606369019 19.0
0 0.157026052474976 20.0
0 -0.114774584770203 21.0
0 0.485472470521927 22.0
39.829513031 39.3229598999023 23.0
0 0.0125501155853271 24.0
0 0.527874529361725 25.0
0 -0.158381462097168 26.0
0 0.0795438289642334 27.0
0 -0.352481722831726 28.0
0 -0.0153977870941162 29.0
0 0.100680351257324 30.0
0 0.0417739152908325 31.0
0 0.00646728277206421 32.0
0 0.000711560249328613 33.0
0 0.339156478643417 34.0
0 -0.185620605945587 35.0
0 -0.244987487792969 36.0
0 -0.0234749317169189 37.0
0 0.141786813735962 38.0
0 -0.142115592956543 39.0
0 0.454432308673859 40.0
0 -0.159658193588257 41.0
0 0.0240547657012939 42.0
287.70999018 295.6416015625 43.0
0 0.139378666877747 44.0
0 0.0751659274101257 45.0
0 0.222047865390778 46.0
0 -0.0267175436019897 47.0
0 0.402146190404892 48.0
0 0.389561355113983 49.0
0.074529071282 0.0380678772926331 50.0
2.020271849 1.85739755630493 51.0
0 -0.0703147649765015 52.0
0 0.148935735225677 53.0
0 0.226035118103027 54.0
0 0.277910590171814 55.0
0 -0.088787317276001 56.0
0 0.191463708877563 57.0
0 0.61881148815155 58.0
0 0.14511102437973 59.0
0 0.00279659032821655 60.0
0 0.0614312887191772 61.0
0 0.0906511545181274 62.0
220.34798063 252.102478027344 0.0
0 0.0562484264373779 1.0
0 -0.0761899352073669 2.0
0 0.20292717218399 3.0
0 -0.29827356338501 4.0
0 -0.126369774341583 5.0
0 -0.0632807612419128 6.0
0 -0.205359518527985 7.0
0 0.27795022726059 8.0
0 0.154082953929901 9.0
0 -0.0746529102325439 10.0
0 -0.0467003583908081 11.0
0 -0.0380067229270935 12.0
52.579990729 53.3396949768066 13.0
44.160734846 40.9035186767578 14.0
0 -0.0695754289627075 15.0
0 0.0203323364257812 16.0
38.715976435 39.2837753295898 17.0
0 0.137032866477966 18.0
0 -0.0245679020881653 19.0
0 0.0485852956771851 20.0
0 -0.0781739354133606 21.0
0 0.296979069709778 22.0
39.910497891 38.0815582275391 23.0
0 0.000867128372192383 24.0
0 0.193370401859283 25.0
0 -0.11082124710083 26.0
0 0.20093446969986 27.0
0 -0.136463940143585 28.0
0 -0.0301060676574707 29.0
0 -0.211508870124817 30.0
0 0.429342150688171 31.0
0 0.0511257648468018 32.0
0 0.0330085158348083 33.0
0 0.193091750144958 34.0
0 0.0776033401489258 35.0
0 -0.327677726745605 36.0
0 0.0163091421127319 37.0
0 0.282069504261017 38.0
0 -0.500548124313354 39.0
0 0.123529314994812 40.0
0 -0.208314716815948 41.0
0 -0.0157972574234009 42.0
291.37576827 296.312866210938 43.0
0 0.160297751426697 44.0
0 0.10030472278595 45.0
0 0.265315532684326 46.0
0 -0.0613897442817688 47.0
0 0.127177894115448 48.0
0 0.454012393951416 49.0
0.096834131342 0.0251449942588806 50.0
1.8362021857 1.8683979511261 51.0
0 0.0825187563896179 52.0
0 0.0110775828361511 53.0
0 -0.169167280197144 54.0
0 -0.15296196937561 55.0
0 0.131739497184753 56.0
0 -0.110313951969147 57.0
0 0.355147421360016 58.0
0 -0.047890305519104 59.0
0 -0.118446946144104 60.0
0 0.0691942572593689 61.0
0 0.173504114151001 62.0
281.46447196 251.553756713867 0.0
0 -0.0723417997360229 1.0
0 -0.0604557991027832 2.0
0 0.289070129394531 3.0
0 -0.0515435338020325 4.0
0 0.1740602850914 5.0
0 0.089615523815155 6.0
0 0.384899646043777 7.0
0 0.206084251403809 8.0
0 0.0214282274246216 9.0
0 0.491491883993149 10.0
0 -0.06537264585495 11.0
0 -0.0510797500610352 12.0
64.784267831 64.6562271118164 13.0
29.040946745 34.8525314331055 14.0
0 -0.106659293174744 15.0
0 0.0289495587348938 16.0
29.357539053 34.371036529541 17.0
0 0.101648330688477 18.0
0 0.013353705406189 19.0
0 0.0400561094284058 20.0
0 0.156229734420776 21.0
0 0.136914849281311 22.0
28.80285437 28.673547744751 23.0
0 0.12163507938385 24.0
0 0.782593071460724 25.0
0 -0.0507340431213379 26.0
0 -0.0842263698577881 27.0
0 0.250743925571442 28.0
0 -0.0441763997077942 29.0
0 0.0838205218315125 30.0
0 -0.141605377197266 31.0
0 -0.0262128114700317 32.0
0 -0.0614085793495178 33.0
0 0.406681805849075 34.0
0 -0.0774985551834106 35.0
0 0.0651737451553345 36.0
0 -0.0615007877349854 37.0
0 0.174714326858521 38.0
0 -0.138317108154297 39.0
0 0.47839167714119 40.0
0 -0.155464887619019 41.0
0 0.0842987298965454 42.0
297.54156386 296.054809570312 43.0
0 0.0859179496765137 44.0
0 -0.0150091052055359 45.0
0 0.130529463291168 46.0
0 0.133134365081787 47.0
0 -0.0654101371765137 48.0
0 0.316714435815811 49.0
0.070555030227 0.0279836654663086 50.0
2.0991595382 1.86022746562958 51.0
0 -0.0662692189216614 52.0
0 -0.16186785697937 53.0
0 0.13852858543396 54.0
0 0.0438777208328247 55.0
0 -0.106737375259399 56.0
0 0.224652588367462 57.0
0 0.144255816936493 58.0
0 0.0854129195213318 59.0
0 -0.0534703135490417 60.0
0 0.0482069253921509 61.0
0 0.129321277141571 62.0
262.90394992 255.896209716797 0.0
0 -0.240055561065674 1.0
0 -0.0439729690551758 2.0
0 0.342136144638062 3.0
0 -0.177826046943665 4.0
0 0.246645331382751 5.0
0 0.224739134311676 6.0
0 0.136346101760864 7.0
0 0.128370642662048 8.0
0 -0.0390376448631287 9.0
0 0.1632479429245 10.0
0 -0.209626793861389 11.0
0 -0.124125897884369 12.0
74.733737372 76.3102493286133 13.0
26.065559514 23.9343528747559 14.0
0 -0.116554021835327 15.0
0 0.0984919667243958 16.0
27.352612065 34.0962753295898 17.0
0 0.0941399931907654 18.0
0 0.00939285755157471 19.0
0 0.10036563873291 20.0
0 0.103185653686523 21.0
0 0.0977933406829834 22.0
35.317924428 34.4537773132324 23.0
0 0.114619135856628 24.0
0 0.696120798587799 25.0
0 -0.00251519680023193 26.0
0 -0.0216209888458252 27.0
0 0.490889430046082 28.0
0 -0.00543785095214844 29.0
0 0.0192010998725891 30.0
0 -0.33372163772583 31.0
0 -0.147522807121277 32.0
0 -0.0854368209838867 33.0
0 -0.170148968696594 34.0
0 0.216908931732178 35.0
0 -0.236925601959229 36.0
0 0.187204301357269 37.0
0 -0.129549205303192 38.0
0 -0.151559829711914 39.0
0 0.171440660953522 40.0
0 -0.12781298160553 41.0
0 0.114751815795898 42.0
308.1921571 297.183197021484 43.0
0 0.104577422142029 44.0
0 -0.0367395877838135 45.0
0 -9.51290130615234e-05 46.0
0 0.337744474411011 47.0
0 0.343403905630112 48.0
0 0.497206807136536 49.0
0.07315531581 0.0263268947601318 50.0
2.2179871765 1.86782503128052 51.0
0 0.0238355398178101 52.0
0 -0.206041574478149 53.0
0 0.331517815589905 54.0
0 0.120811462402344 55.0
0 -0.129153370857239 56.0
0 0.0994748473167419 57.0
0 0.226855337619781 58.0
0 -0.00172638893127441 59.0
0 0.238515257835388 60.0
0 -0.0317624807357788 61.0
0 0.0799673795700073 62.0
229.90229003 253.510223388672 0.0
0 -0.0559872388839722 1.0
0 -0.00438511371612549 2.0
0 0.161374092102051 3.0
0 -0.224611043930054 4.0
0 -0.303132891654968 5.0
0 -0.082477331161499 6.0
0 -0.268182158470154 7.0
0 0.556970596313477 8.0
0 -0.140942573547363 9.0
0 0.231969177722931 10.0
0 -0.0672531127929688 11.0
0 -0.0596077442169189 12.0
66.560621126 66.0610046386719 13.0
28.904167579 34.0602722167969 14.0
0 -0.120209693908691 15.0
0 -0.071599543094635 16.0
34.630979768 36.5518951416016 17.0
0 0.150029122829437 18.0
0 -0.0779799222946167 19.0
0 0.125945746898651 20.0
0 0.0613558888435364 21.0
0 0.170729279518127 22.0
33.474994478 30.4586868286133 23.0
0 0.0252755284309387 24.0
0 0.321561455726624 25.0
0 -0.0738681554794312 26.0
0 -0.181806683540344 27.0
0 0.161595225334167 28.0
0 -0.0263367891311646 29.0
0 0.0751441717147827 30.0
0 -0.0912724733352661 31.0
0 0.0012962818145752 32.0
0 -0.157572150230408 33.0
0 -0.173588633537292 34.0
0 0.284022092819214 35.0
0 0.32756519317627 36.0
0 0.175395369529724 37.0
0 0.100552439689636 38.0
0 -0.328045845031738 39.0
0 0.2602618932724 40.0
0 -0.160473465919495 41.0
0 -0.138812303543091 42.0
305.3693217 297.020751953125 43.0
0 0.0227260589599609 44.0
0 0.138468205928802 45.0
0 0.177830696105957 46.0
0 0.131672739982605 47.0
0 0.0934637784957886 48.0
0 0.421511143445969 49.0
0.096074437849 0.0500668883323669 50.0
1.9362114934 1.82811498641968 51.0
0 0.207618355751038 52.0
0 -0.264808893203735 53.0
0 0.240140557289124 54.0
0 0.213762104511261 55.0
0 -0.121890306472778 56.0
0 -0.0493593215942383 57.0
0 -0.127137899398804 58.0
0 0.0666912794113159 59.0
0 0.261136293411255 60.0
0 0.0914432406425476 61.0
0 0.204353332519531 62.0
262.85456753 252.267150878906 0.0
0 -0.22112250328064 1.0
0 -0.147109925746918 2.0
0 0.110867261886597 3.0
0 0.0916562676429749 4.0
0 0.197851538658142 5.0
0 0.55634081363678 6.0
0 0.196639716625214 7.0
0 0.331026077270508 8.0
0 0.106014311313629 9.0
0 0.214046895503998 10.0
0 -0.00535297393798828 11.0
0 -0.0130957365036011 12.0
65.716654485 68.2416534423828 13.0
31.158174152 28.2533931732178 14.0
0 0.0798667669296265 15.0
0 -0.0635446906089783 16.0
40.854153771 38.3098449707031 17.0
0 0.204116344451904 18.0
0 -0.00636422634124756 19.0
0 0.0878734588623047 20.0
0 0.0661526918411255 21.0
0 0.202452778816223 22.0
33.320946837 28.8756580352783 23.0
0 -0.0207080245018005 24.0
0 0.458996534347534 25.0
0 -0.0911705493927002 26.0
0 -0.00530582666397095 27.0
0 0.341299057006836 28.0
0 -0.127111554145813 29.0
0 -0.0684215426445007 30.0
0 -0.0931061506271362 31.0
0 0.0586287975311279 32.0
0 0.186613619327545 33.0
0 0.297702342271805 34.0
0 -0.0698081254959106 35.0
0 -0.286024451255798 36.0
0 0.154449164867401 37.0
0 -0.300218343734741 38.0
0 -0.0247198343276978 39.0
0 0.268884778022766 40.0
0 -0.0191447734832764 41.0
0 0.4001305103302 42.0
277.70491776 293.235473632812 43.0
0 -0.0274943113327026 44.0
0 0.387960016727448 45.0
0 0.128162384033203 46.0
0 0.0247758626937866 47.0
0 0.200818836688995 48.0
0 0.417185246944427 49.0
0.094162097161 0.0790655016899109 50.0
1.5907494099 1.81209087371826 51.0
0 0.282562553882599 52.0
0 -0.0386070013046265 53.0
0 -0.102742195129395 54.0
0 0.0913277268409729 55.0
0 0.183602333068848 56.0
0 0.0933142304420471 57.0
0 0.575786948204041 58.0
0 0.261445343494415 59.0
0 -0.0594013333320618 60.0
0 0.0572004914283752 61.0
0 0.159839451313019 62.0
247.27688166 251.217391967773 0.0
0 -0.290899872779846 1.0
0 0.138386487960815 2.0
0 -0.0153924226760864 3.0
0 0.00844442844390869 4.0
0 0.00288486480712891 5.0
0 0.442357659339905 6.0
0 0.208782434463501 7.0
0 -0.140392780303955 8.0
0 -0.338835120201111 9.0
0 0.431244403123856 10.0
0 0.807105958461761 11.0
0 -0.338749885559082 12.0
47.8586967 46.3189926147461 13.0
41.494657128 41.5184020996094 14.0
0 -0.178764224052429 15.0
0 0.251387357711792 16.0
40.943948714 37.9247894287109 17.0
0 0.147467732429504 18.0
0 -0.0432272553443909 19.0
0 0.122396349906921 20.0
0 -0.0158259868621826 21.0
0 0.0178604125976562 22.0
41.820883601 38.2125854492188 23.0
0 0.00501704216003418 24.0
0 0.162802457809448 25.0
0 -0.0238991975784302 26.0
0 0.014929473400116 27.0
0 0.254974901676178 28.0
0 -0.195201516151428 29.0
0 0.0595022439956665 30.0
0 0.0179866552352905 31.0
0 -0.0336312651634216 32.0
0 -0.211641311645508 33.0
0 0.0880379676818848 34.0
0 -0.0553752779960632 35.0
0 0.271261632442474 36.0
0 0.133851766586304 37.0
0 -0.0998091697692871 38.0
0 -0.257207155227661 39.0
0 0.447905302047729 40.0
0 -0.177113711833954 41.0
0 0.101021766662598 42.0
287.34909198 297.216522216797 43.0
0 0.0624773502349854 44.0
0 0.108922898769379 45.0
0 0.262860774993896 46.0
0 -0.0815596580505371 47.0
0 0.404132515192032 48.0
0 0.436324417591095 49.0
0.092769818517 0.0490875244140625 50.0
1.8816234448 1.87447452545166 51.0
0 -0.129589319229126 52.0
0 -0.0855868458747864 53.0
0 -0.0261377096176147 54.0
0 0.261190891265869 55.0
0 0.241515517234802 56.0
0 0.35185495018959 57.0
0 0.118189632892609 58.0
0 -0.840454459190369 59.0
0 1.57609629631042 60.0
0 -0.112815141677856 61.0
0 -0.145868301391602 62.0
199.56853871 254.724639892578 0.0
0 0.628296136856079 1.0
0 -0.259518980979919 2.0
0 -0.638007879257202 3.0
0 -0.600342750549316 4.0
0 -0.427808403968811 5.0
0 0.0472760796546936 6.0
0 -0.302554726600647 7.0
0 0.157285213470459 8.0
0 -0.155857086181641 9.0
0 0.0695149898529053 10.0
0 0.0333502292633057 11.0
0 -0.130425691604614 12.0
73.914887701 77.0658111572266 13.0
23.192194968 20.199239730835 14.0
0 -0.00433051586151123 15.0
0 0.151898682117462 16.0
34.787687303 34.9917221069336 17.0
0 0.0912007093429565 18.0
0 0.0725125074386597 19.0
0 0.0870222449302673 20.0
0 -0.0299894213676453 21.0
0 0.436963617801666 22.0
40.311613214 39.7592163085938 23.0
0 -0.0185078382492065 24.0
0 1.0556606054306 25.0
0 -0.0349418520927429 26.0
0 0.0436426401138306 27.0
0 -0.240531802177429 28.0
0 -0.122637391090393 29.0
0 -0.0883998870849609 30.0
0 -0.0181294679641724 31.0
0 -0.0770110487937927 32.0
0 -0.0931625366210938 33.0
0 0.249503612518311 34.0
0 -0.0605553984642029 35.0
0 -0.237947702407837 36.0
0 0.0359690189361572 37.0
0 0.105684041976929 38.0
0 -0.230105638504028 39.0
0 0.158533692359924 40.0
0 -0.0573774576187134 41.0
0 0.19561493396759 42.0
276.72008582 298.573303222656 43.0
0 0.0914930701255798 44.0
0 0.227839350700378 45.0
0 0.130322277545929 46.0
0 -0.406926870346069 47.0
0 0.327408134937286 48.0
0 0.474675118923187 49.0
0.10273295143 0.0643347501754761 50.0
1.5686037941 1.95015215873718 51.0
0 -0.0457574129104614 52.0
0 0.359253942966461 53.0
0 -0.055999755859375 54.0
0 0.109583973884583 55.0
0 0.157442092895508 56.0
0 -0.00622683763504028 57.0
0 0.462634742259979 58.0
0 0.367376536130905 59.0
0 -0.228309631347656 60.0
0 0.125678300857544 61.0
0 0.465126156806946 62.0
208.32376007 251.930908203125 0.0
0 0.334326505661011 1.0
0 -0.00545591115951538 2.0
0 0.0761346220970154 3.0
0 -0.308877944946289 4.0
0 -0.0678622126579285 5.0
0 -0.0885855555534363 6.0
0 -0.414778590202332 7.0
0 0.388415157794952 8.0
0 0.089970588684082 9.0
0 0.389721125364304 10.0
0 -0.0783578753471375 11.0
0 0.0939447283744812 12.0
65.046757542 63.7625427246094 13.0
36.941752993 32.645938873291 14.0
0 -0.0662013292312622 15.0
0 0.0466957688331604 16.0
34.306826137 32.8809700012207 17.0
0 0.144197046756744 18.0
0 -0.0194722414016724 19.0
0 0.274300634860992 20.0
0 -0.0605226159095764 21.0
0 1.22359073162079 22.0
41.839379136 39.6939239501953 23.0
0 0.122661530971527 24.0
0 0.765777289867401 25.0
0 -0.341060638427734 26.0
0 0.105543673038483 27.0
0 -0.319305300712585 28.0
0 -0.1012242436409 29.0
0 0.0227471590042114 30.0
0 -0.320103049278259 31.0
0 -0.0252951383590698 32.0
0 0.0835428833961487 33.0
0 0.598523378372192 34.0
0 0.136987864971161 35.0
0 0.275820016860962 36.0
0 0.00785708427429199 37.0
0 0.317506045103073 38.0
0 -0.179658174514771 39.0
0 0.947694599628448 40.0
0 -0.118481397628784 41.0
0 0.10599672794342 42.0
293.60079286 297.419555664062 43.0
0 0.148662328720093 44.0
0 -0.0177066922187805 45.0
0 0.0300235748291016 46.0
0 0.176496863365173 47.0
0 0.0480192899703979 48.0
0 0.445548683404922 49.0
0.081214047268 -0.0479921102523804 50.0
1.5010535613 1.89463257789612 51.0
0 0.0159016251564026 52.0
0 0.0407056212425232 53.0
0 0.237950444221497 54.0
0 0.131243586540222 55.0
0 -0.101468801498413 56.0
0 0.363180547952652 57.0
0 -0.0647857785224915 58.0
0 -0.249950528144836 59.0
0 0.964573204517365 60.0
0 -0.0849443674087524 61.0
0 -0.17559152841568 62.0
248.49726433 251.16081237793 0.0
0 -0.230982780456543 1.0
0 -0.0342386364936829 2.0
0 0.372298181056976 3.0
0 -0.0214449167251587 4.0
0 -0.0306460857391357 5.0
0 -0.00703209638595581 6.0
0 0.118870556354523 7.0
0 0.0849636793136597 8.0
0 0.0639764666557312 9.0
0 -0.338356256484985 10.0
0 -0.0531266927719116 11.0
0 0.0686190128326416 12.0
59.267151492 57.1477813720703 13.0
43.789416536 39.0969619750977 14.0
0 0.0184546709060669 15.0
0 0.367021143436432 16.0
38.552934444 40.3950691223145 17.0
0 0.316053450107574 18.0
0 -0.0507403612136841 19.0
0 0.0189411640167236 20.0
0 -0.0833299160003662 21.0
0 0.182493090629578 22.0
37.980961885 37.3105621337891 23.0
0 0.0666259527206421 24.0
0 0.808936536312103 25.0
0 0.0246939659118652 26.0
0 -0.0207312703132629 27.0
0 -0.00823676586151123 28.0
0 -0.198312163352966 29.0
0 0.127368092536926 30.0
0 0.0191323757171631 31.0
0 0.0262845158576965 32.0
0 -0.0474375486373901 33.0
0 0.258740723133087 34.0
0 0.285801649093628 35.0
0 0.210043907165527 36.0
0 -0.0565052032470703 37.0
0 0.324103772640228 38.0
0 -0.128281235694885 39.0
0 0.446937918663025 40.0
0 -0.189451992511749 41.0
0 0.19011127948761 42.0
295.5148869 298.438293457031 43.0
0 -0.0330727100372314 44.0
0 0.190834045410156 45.0
0 0.104691386222839 46.0
0 -0.0788767337799072 47.0
0 0.333256125450134 48.0
1.6844036159 0.70777440071106 49.0
0.074911347609 0.0775371789932251 50.0
2.1596089522 1.86102485656738 51.0
0 0.0204648375511169 52.0
0 0.032648503780365 53.0
0 -0.210375189781189 54.0
0 -0.17225456237793 55.0
0 0.0337446928024292 56.0
0 0.0356935858726501 57.0
0 0.0188963413238525 58.0
0 -0.769248247146606 59.0
0 1.49346268177032 60.0
0 -0.0328093767166138 61.0
0 -0.0239010453224182 62.0
275.17389549 252.537063598633 0.0
0 -0.167420029640198 1.0
0 0.048227846622467 2.0
0 0.342236518859863 3.0
0 -0.0127829313278198 4.0
0 0.127352774143219 5.0
0 -0.29143214225769 6.0
0 0.363724023103714 7.0
0 0.289907336235046 8.0
0 -0.524978518486023 9.0
0 0.417556822299957 10.0
0 -0.119501948356628 11.0
0 0.0315666198730469 12.0
71.317386717 74.2230606079102 13.0
23.301003826 28.3769474029541 14.0
0 0.0171617269515991 15.0
0 -0.0422912240028381 16.0
36.429389996 34.4917182922363 17.0
0 0.125352919101715 18.0
0 -0.0678154230117798 19.0
0 0.154057919979095 20.0
0 -0.0270178914070129 21.0
0 0.247260689735413 22.0
36.725830706 37.5780258178711 23.0
0 0.162127792835236 24.0
0 1.00677669048309 25.0
0 -0.0360807180404663 26.0
0 -0.121962368488312 27.0
0 -0.2920161485672 28.0
0 -0.139118134975433 29.0
0 0.11112380027771 30.0
0 0.358356714248657 31.0
0 0.225140154361725 32.0
0 0.0595033168792725 33.0
0 0.225057005882263 34.0
0 0.0133835077285767 35.0
0 0.192320108413696 36.0
0 -0.0108208656311035 37.0
0 -0.134678244590759 38.0
0 -0.115051865577698 39.0
0 0.479521751403809 40.0
0 -0.272257804870605 41.0
0 0.131450295448303 42.0
287.06527813 297.108154296875 43.0
0 0.20518958568573 44.0
0 0.131566345691681 45.0
0 0.418930351734161 46.0
0 -0.539020299911499 47.0
0 0.367195725440979 48.0
0 0.471466660499573 49.0
0.099655138521 0.0244454145431519 50.0
1.6582109955 1.85178995132446 51.0
0 -0.17762964963913 52.0
0 0.0591254830360413 53.0
0 -0.064058244228363 54.0
0 0.132320463657379 55.0
0 0.173306465148926 56.0
0 0.230087995529175 57.0
0 0.117884278297424 58.0
0 0.281343519687653 59.0
0 -0.273699641227722 60.0
0 0.137469530105591 61.0
0 0.306600272655487 62.0
271.60358718 251.748901367188 0.0
0 -0.232115745544434 1.0
0 -0.127860605716705 2.0
0 -0.0510497689247131 3.0
0 0.135054528713226 4.0
0 0.219311594963074 5.0
0 -0.133363842964172 6.0
0 0.466055512428284 7.0
0 0.449722290039062 8.0
0 0.0559042692184448 9.0
0 0.274723410606384 10.0
0 -0.118657767772675 11.0
0 -0.0338935852050781 12.0
56.984253009 54.0480461120605 13.0
37.774192673 40.9160003662109 14.0
0 -0.168563723564148 15.0
0 -0.107933282852173 16.0
26.070424668 31.4519309997559 17.0
0 -0.0278823375701904 18.0
0 0.066960334777832 19.0
0 0.134846210479736 20.0
0 0.0768986344337463 21.0
0 -0.0127499103546143 22.0
37.981946431 37.0336723327637 23.0
0 0.0862783193588257 24.0
0 1.34286463260651 25.0
0 0.123081982135773 26.0
0 -0.148675382137299 27.0
0 -0.071246862411499 28.0
0 0.028731107711792 29.0
0 -0.132228255271912 30.0
0 0.146646857261658 31.0
0 0.0635372996330261 32.0
0 0.0153490900993347 33.0
0 0.400108218193054 34.0
0 -0.0144473314285278 35.0
0 -0.320407867431641 36.0
0 0.0656779408454895 37.0
0 0.208275258541107 38.0
0 -0.198534965515137 39.0
0 0.28618448972702 40.0
0 -0.188271939754486 41.0
0 0.236361742019653 42.0
291.07342298 297.791687011719 43.0
0 0.217540383338928 44.0
0 0.117962956428528 45.0
0 0.256760120391846 46.0
0 0.0093996524810791 47.0
0 0.181263267993927 48.0
0 0.342582643032074 49.0
0.083522871053 0.0831648111343384 50.0
1.4431618393 1.80248153209686 51.0
0 0.275522768497467 52.0
0 0.0158348679542542 53.0
0 0.0253161191940308 54.0
0 0.0779030323028564 55.0
0 0.102590084075928 56.0
0 0.0137673020362854 57.0
0 0.0428553223609924 58.0
0 0.214259028434753 59.0
0 -0.297919869422913 60.0
0 0.12319004535675 61.0
0 0.195669770240784 62.0
232.20106629 253.841171264648 0.0
0 0.00963544845581055 1.0
0 -0.141588449478149 2.0
0 -0.0478709936141968 3.0
0 -0.344800472259521 4.0
0 0.151486277580261 5.0
0 -0.0997970700263977 6.0
0 -0.20776492357254 7.0
0 -0.00196409225463867 8.0
0 0.557166218757629 9.0
0 0.751480162143707 10.0
0 -0.197999179363251 11.0
0 -0.0707920789718628 12.0
70.230648464 69.6913757324219 13.0
15.313269025 28.4831256866455 14.0
0 0.00730490684509277 15.0
0 -0.0493123531341553 16.0
31.917085252 34.272102355957 17.0
0 0.114812433719635 18.0
0 0.0728192925453186 19.0
0 0.0369380712509155 20.0
0 0.13552987575531 21.0
0 0.0701425075531006 22.0
30.614824325 28.6397647857666 23.0
0 0.0712471604347229 24.0
0 0.729307651519775 25.0
0 -0.122468948364258 26.0
0 -0.0303815603256226 27.0
0 0.497071981430054 28.0
0 -0.0539799928665161 29.0
0 0.114532589912415 30.0
0 0.30565869808197 31.0
0 0.170579254627228 32.0
0 0.0491366982460022 33.0
0 0.411743342876434 34.0
0 0.366449445486069 35.0
0 0.318916499614716 36.0
0 0.0502575039863586 37.0
0 0.0943170189857483 38.0
0 -0.324892997741699 39.0
0 0.504047572612762 40.0
0 -0.139069736003876 41.0
0 0.0357446670532227 42.0
305.61089148 297.154022216797 43.0
0 0.110487103462219 44.0
0 -0.0192506909370422 45.0
0 0.130986452102661 46.0
0 -0.171379566192627 47.0
0 -0.354133248329163 48.0
0 0.373544484376907 49.0
0.073289204953 0.00854772329330444 50.0
1.8032371736 1.86180591583252 51.0
0 -0.206107974052429 52.0
0 0.218928694725037 53.0
0 -0.0510441064834595 54.0
0 0.133084237575531 55.0
0 0.328331977128983 56.0
0 0.0277809500694275 57.0
0 -0.0907001495361328 58.0
0 0.169900596141815 59.0
0 0.0605570077896118 60.0
0 0.0937773585319519 61.0
0 0.178057849407196 62.0
275.78057615 255.401733398438 0.0
0 -0.304426193237305 1.0
0 0.0885152816772461 2.0
0 0.0957579016685486 3.0
0 0.138942420482635 4.0
0 0.0920662879943848 5.0
0 -0.374558448791504 6.0
0 0.422135382890701 7.0
0 -0.104005336761475 8.0
0 0.0858319997787476 9.0
0 -0.243074536323547 10.0
0 -0.129600167274475 11.0
0 -0.0409594774246216 12.0
74.801876637 77.1970520019531 13.0
27.012304854 26.8510799407959 14.0
0 -0.039171576499939 15.0
0 0.0358449816703796 16.0
34.852657472 38.1332893371582 17.0
0 0.151331424713135 18.0
0 -0.0139744281768799 19.0
0 0.0696083307266235 20.0
0 0.0428495407104492 21.0
0 0.290519058704376 22.0
35.290274454 35.0415725708008 23.0
0 0.140546441078186 24.0
0 1.17133712768555 25.0
0 -0.0150032043457031 26.0
0 -0.153510928153992 27.0
0 0.547147452831268 28.0
0 -0.0400686860084534 29.0
0 0.157024800777435 30.0
0 0.224410653114319 31.0
0 0.133382797241211 32.0
0 0.026755690574646 33.0
0 0.177687287330627 34.0
0 -0.0959345698356628 35.0
0 0.269127488136292 36.0
0 -0.03276127576828 37.0
0 0.251066982746124 38.0
0 -0.181754469871521 39.0
0 0.245484471321106 40.0
0 -0.160742700099945 41.0
0 0.0785598754882812 42.0
292.06700254 298.371520996094 43.0
0 0.0141600966453552 44.0
0 0.0749538540840149 45.0
0 0.281254410743713 46.0
0 -0.182612299919128 47.0
0 0.420543670654297 48.0
0.48186238659 0.430585205554962 49.0
0.073508409272 0.0398976802825928 50.0
1.9355381709 1.88321161270142 51.0
0 -0.0420444011688232 52.0
0 -0.0618168711662292 53.0
0 0.00341784954071045 54.0
0 0.196490705013275 55.0
0 0.20697009563446 56.0
0 0.0401254296302795 57.0
0 0.552940130233765 58.0
0 0.353383183479309 59.0
0 0.160937786102295 60.0
0 0.0865020155906677 61.0
0 0.174372017383575 62.0
257.579695 251.476303100586 0.0
0 -0.501283884048462 1.0
0 0.105928599834442 2.0
0 0.111746788024902 3.0
0 -0.145626306533813 4.0
0 0.36040210723877 5.0
0 0.0246535539627075 6.0
0 0.332267493009567 7.0
0 0.379302084445953 8.0
0 0.163660049438477 9.0
0 0.41306871175766 10.0
0 -0.189957976341248 11.0
0 -0.169556498527527 12.0
55.457659519 54.15966796875 13.0
46.405566784 40.3176727294922 14.0
0 -0.332054495811462 15.0
0 0.478655576705933 16.0
30.202759877 34.4963417053223 17.0
0 0.112985849380493 18.0
0 0.000216126441955566 19.0
0 0.0927401185035706 20.0
0 0.203673601150513 21.0
0 0.0316375494003296 22.0
34.108071713 31.6523361206055 23.0
0 -0.00450176000595093 24.0
29.516487592 0.0188377499580383 25.0
0 -0.0329997539520264 26.0
0 -0.0297427177429199 27.0
0 0.287297129631042 28.0
0 -0.0359582901000977 29.0
0 0.0501150488853455 30.0
0 -0.230956554412842 31.0
0 -0.0139465928077698 32.0
0 -0.0273321866989136 33.0
0 0.379595875740051 34.0
0 0.28080290555954 35.0
0 0.113584995269775 36.0
0 0.0977506637573242 37.0
0 0.118238687515259 38.0
0 -0.198300242424011 39.0
0 0.0740363001823425 40.0
0 -0.110121846199036 41.0
0 0.0283690690994263 42.0
297.18644464 297.433654785156 43.0
0 0.144267082214355 44.0
0 0.0973208546638489 45.0
0 0.0643321871757507 46.0
0 0.126612305641174 47.0
0 -0.180387556552887 48.0
0 0.486942052841187 49.0
0.083608823224 -0.0190044641494751 50.0
1.8884394892 1.87191355228424 51.0
0 0.0808982253074646 52.0
0 0.132171750068665 53.0
0 0.0408061742782593 54.0
0 0.207171440124512 55.0
0 0.283678293228149 56.0
0 -0.0554107427597046 57.0
0 0.233238339424133 58.0
0 -0.050200343132019 59.0
0 -0.109920680522919 60.0
0 0.0113455653190613 61.0
0 0.170213758945465 62.0
270.99625891 251.310592651367 0.0
0 -0.0614405870437622 1.0
0 -0.0409826040267944 2.0
0 0.254399120807648 3.0
0 0.192896842956543 4.0
0 0.407808125019073 5.0
0 -0.0037270188331604 6.0
0 0.071510910987854 7.0
0 -0.0101948380470276 8.0
0 0.44614839553833 9.0
0 -0.475370645523071 10.0
0 -0.205818891525269 11.0
0 -0.160810828208923 12.0
48.072261826 49.6483840942383 13.0
36.684633278 40.8359489440918 14.0
0 -0.0483132600784302 15.0
0 0.116230070590973 16.0
37.841888144 38.4057769775391 17.0
0 0.196586549282074 18.0
0 -0.0649081468582153 19.0
0 0.0710679292678833 20.0
0 0.161048173904419 21.0
0 0.002555251121521 22.0
39.616845866 34.7192649841309 23.0
0 0.0117547512054443 24.0
18.564394511 0.201078832149506 25.0
0 0.0527403354644775 26.0
0 0.147799968719482 27.0
0 0.549925327301025 28.0
0 -0.0489311218261719 29.0
0 -0.0505109429359436 30.0
0 0.0890066623687744 31.0
0 0.216496646404266 32.0
0 -0.305202484130859 33.0
0 0.0150960087776184 34.0
0 -0.081246018409729 35.0
0 -0.102942287921906 36.0
0 -0.0391702055931091 37.0
0 0.325633525848389 38.0
0 -0.18120801448822 39.0
0 0.366682529449463 40.0
0 -0.185742855072021 41.0
0 -0.0132526159286499 42.0
289.37334476 298.593444824219 43.0
0 0.113219201564789 44.0
0 0.114428281784058 45.0
0 0.239014744758606 46.0
0 -0.160783767700195 47.0
0 0.220858871936798 48.0
0 0.451271086931229 49.0
0.07558203861 0.0561100244522095 50.0
1.8300725009 1.79672658443451 51.0
0 -0.195446193218231 52.0
0 -0.0695803165435791 53.0
0 -0.0205304622650146 54.0
0 0.191309452056885 55.0
0 0.0629352331161499 56.0
0 0.502467930316925 57.0
0 0.521711826324463 58.0
0 0.294936776161194 59.0
0 0.073952317237854 60.0
0 -0.00400024652481079 61.0
0 0.047935962677002 62.0
241.916803 253.095291137695 0.0
0 -0.189413666725159 1.0
0 -0.0389675498008728 2.0
0 0.166366100311279 3.0
0 -0.427569389343262 4.0
0 0.105612099170685 5.0
0 0.190612196922302 6.0
0 -0.0690482258796692 7.0
0 0.349586963653564 8.0
0 0.536535680294037 9.0
0 0.235416173934937 10.0
0 -0.49225378036499 11.0
0 -0.070934534072876 12.0
71.601108604 71.1997985839844 13.0
28.549216602 26.2476787567139 14.0
0 -0.09712815284729 15.0
0 -0.199301600456238 16.0
34.395313311 33.1312713623047 17.0
0 0.11553418636322 18.0
0 -0.000574111938476562 19.0
0 0.14836049079895 20.0
0 0.209805846214294 21.0
0 -0.0485737323760986 22.0
28.545406097 25.8330345153809 23.0
0 0.0977875590324402 24.0
0 0.987263023853302 25.0
0 -0.146968245506287 26.0
0 0.0621381402015686 27.0
0 -0.0344146490097046 28.0
0 0.0313956737518311 29.0
0 0.14185756444931 30.0
0 0.154151678085327 31.0
0 0.0979040861129761 32.0
0 0.150334596633911 33.0
0 0.276743412017822 34.0
0 0.127934753894806 35.0
0 0.26941704750061 36.0
0 0.211870968341827 37.0
0 -0.0662814974784851 38.0
0 -0.082857608795166 39.0
0 0.279957175254822 40.0
0 -0.0631459951400757 41.0
0 -0.00643324851989746 42.0
289.98488078 297.457092285156 43.0
0 0.07309889793396 44.0
0 0.144029498100281 45.0
0 0.160113275051117 46.0
0 -0.516191720962524 47.0
0 0.146459579467773 48.0
0 0.334080338478088 49.0
0.084098265965 0.0338427424430847 50.0
1.7053764041 1.84561681747437 51.0
0 -0.242425560951233 52.0
0 -0.265066146850586 53.0
0 -0.0617619156837463 54.0
0 0.0951944589614868 55.0
0 -0.150510609149933 56.0
0 0.043833315372467 57.0
0 0.276302337646484 58.0
0 0.360927730798721 59.0
0 -0.16847836971283 60.0
0 0.147453725337982 61.0
0 0.273782730102539 62.0
234.92622212 252.801116943359 0.0
0 -0.178049802780151 1.0
0 -0.151946246623993 2.0
0 0.122757196426392 3.0
0 -0.277317404747009 4.0
0 -0.153939187526703 5.0
0 0.132198214530945 6.0
0 -0.0337333679199219 7.0
0 0.256085634231567 8.0
0 0.133065640926361 9.0
0 -0.205418169498444 10.0
0 -0.356687784194946 11.0
0 0.0274014472961426 12.0
73.078557418 74.0436019897461 13.0
15.170612388 22.3788032531738 14.0
0 -0.117252826690674 15.0
0 0.230309307575226 16.0
37.527367898 36.4914512634277 17.0
0 0.124790847301483 18.0
0 -0.0517450571060181 19.0
0 0.0909602642059326 20.0
0 0.0809144377708435 21.0
0 0.120734930038452 22.0
31.171631065 29.9929237365723 23.0
0 0.0608701705932617 24.0
0 0.52294135093689 25.0
0 -0.124857127666473 26.0
0 -0.106768131256104 27.0
0 0.194718778133392 28.0
0 -0.0131787657737732 29.0
0 -0.11032110452652 30.0
0 0.067071795463562 31.0
0 -0.147377669811249 32.0
0 -0.107017517089844 33.0
0 0.339581280946732 34.0
0 -0.112988889217377 35.0
0 0.782854914665222 36.0
0 0.171207845211029 37.0
0 -0.355613112449646 38.0
0 -0.199667692184448 39.0
0 0.0526630878448486 40.0
0 -0.19778037071228 41.0
0 -0.0969913005828857 42.0
299.45085629 297.751861572266 43.0
0 0.0180028080940247 44.0
0 0.0784509181976318 45.0
0 0.245897114276886 46.0
0 -0.0300218462944031 47.0
0 -0.143227696418762 48.0
0 0.333292573690414 49.0
0.083773669426 0.00410902500152588 50.0
1.8235104933 1.86783373355865 51.0
0 0.218695163726807 52.0
0 -0.273590445518494 53.0
0 0.192912876605988 54.0
0 0.0465615391731262 55.0
0 -0.12673145532608 56.0
0 -0.146015465259552 57.0
0 0.238672494888306 58.0
0 0.102422595024109 59.0
0 -0.135563731193542 60.0
0 0.0366047620773315 61.0
0 0.161735773086548 62.0
265.53335386 254.43132019043 0.0
0 -0.330430865287781 1.0
0 -0.117311954498291 2.0
0 -0.0128670334815979 3.0
0 -0.0331002473831177 4.0
0 0.256867051124573 5.0
0 0.605901598930359 6.0
0 0.223268985748291 7.0
0 -0.0541033744812012 8.0
0 0.22734522819519 9.0
0 -0.0708760619163513 10.0
0 0.0157485008239746 11.0
0 -0.2548588514328 12.0
59.468570686 59.1644515991211 13.0
44.364680904 38.278881072998 14.0
0 -0.0761950016021729 15.0
0 0.241663753986359 16.0
34.817014922 34.9938735961914 17.0
0 0.100329637527466 18.0
0 0.0208007097244263 19.0
0 0.0262986421585083 20.0
0 0.180667102336884 21.0
0 -0.157078981399536 22.0
38.239443516 36.2570343017578 23.0
0 0.0868841409683228 24.0
0 0.592865645885468 25.0
0 0.0656338930130005 26.0
0 -0.100641369819641 27.0
0 0.283326148986816 28.0
0 -0.0996557474136353 29.0
0 0.120704829692841 30.0
0 0.0992200374603271 31.0
0 -0.104756474494934 32.0
0 0.0238600969314575 33.0
0 -0.0569693446159363 34.0
0 -0.218475818634033 35.0
0 0.419719189405441 36.0
0 -0.0043485164642334 37.0
0 0.227320492267609 38.0
0 -0.167783379554749 39.0
0 0.316357165575027 40.0
0 -0.153945744037628 41.0
0 0.0488275289535522 42.0
284.06596081 297.487487792969 43.0
0 0.214293360710144 44.0
0 0.108594596385956 45.0
0 0.19838547706604 46.0
0 0.183823943138123 47.0
0 0.169947683811188 48.0
0 0.405177593231201 49.0
0.078807583439 0.0386176705360413 50.0
1.6498847923 1.82898306846619 51.0
0 -0.0218407511711121 52.0
0 -0.115340828895569 53.0
0 0.215299725532532 54.0
0 -0.0101180076599121 55.0
0 -0.113912343978882 56.0
0 0.183014631271362 57.0
0 0.347441345453262 58.0
0 0.346438020467758 59.0
0 -0.172914266586304 60.0
0 0.122397065162659 61.0
0 0.310489773750305 62.0
231.37290218 252.054382324219 0.0
0 -0.152666926383972 1.0
0 -0.212255001068115 2.0
0 0.0836414098739624 3.0
0 -0.179706454277039 4.0
0 -0.0934668779373169 5.0
0 0.0404773354530334 6.0
0 -0.128635287284851 7.0
0 0.152553081512451 8.0
0 0.317678928375244 9.0
0 -0.150911808013916 10.0
0 -0.171040832996368 11.0
0 -0.0367785692214966 12.0
73.729266251 74.077880859375 13.0
23.803730181 26.5420780181885 14.0
0 -0.0214141607284546 15.0
0 -0.0267369151115417 16.0
31.429577842 34.5469970703125 17.0
0 0.0819319486618042 18.0
0 -0.0519123077392578 19.0
0 0.14622950553894 20.0
0 0.00640463829040527 21.0
0 0.196577847003937 22.0
37.704617022 35.2850875854492 23.0
0 0.0177223682403564 24.0
0 -0.0420013666152954 25.0
0 0.0049099326133728 26.0
0 0.0560342669487 27.0
0 0.235616147518158 28.0
0 -0.117602825164795 29.0
0 0.049251914024353 30.0
0 0.194475173950195 31.0
0 0.228964388370514 32.0
0 0.10840380191803 33.0
0 -0.030872642993927 34.0
0 0.696496307849884 35.0
0 0.545513033866882 36.0
0 -0.0695604681968689 37.0
0 0.0915053486824036 38.0
0 -0.0735077857971191 39.0
0 0.576463341712952 40.0
0 -0.0599015355110168 41.0
0 0.169428467750549 42.0
284.30898456 298.774444580078 43.0
0 0.035213828086853 44.0
0 0.11706417798996 45.0
0 0.167555212974548 46.0
0 0.00684899091720581 47.0
0 0.211481928825378 48.0
1.2882250007 0.520305752754211 49.0
0.072113733817 0.0261902809143066 50.0
2.1939200668 1.89823949337006 51.0
0 0.0872651338577271 52.0
0 0.00883615016937256 53.0
0 0.114377558231354 54.0
0 0.147997736930847 55.0
0 -0.103454530239105 56.0
0 0.383517056703568 57.0
0 -0.0123884081840515 58.0
0 0.168358027935028 59.0
0 -0.0101599097251892 60.0
0 0.00963121652603149 61.0
0 0.0563340783119202 62.0
294.95874107 252.721450805664 0.0
0 -0.24895441532135 1.0
0 -0.0455106496810913 2.0
0 -0.0637372136116028 3.0
0 -0.0215383768081665 4.0
0 0.148036062717438 5.0
0 -0.212385773658752 6.0
0 0.463038206100464 7.0
0 -0.376063346862793 8.0
0 0.238953948020935 9.0
0 0.179911196231842 10.0
0 0.150992870330811 11.0
0 -0.0431139469146729 12.0
76.259398208 75.3915100097656 13.0
30.880485354 25.8123302459717 14.0
0 0.00202107429504395 15.0
0 -0.0746620893478394 16.0
26.489415048 31.758451461792 17.0
0 0.101182639598846 18.0
0 -0.0572724342346191 19.0
0 0.165071725845337 20.0
0 0.12966001033783 21.0
0 0.114044070243835 22.0
28.909933772 27.3448333740234 23.0
0 0.0576168894767761 24.0
0 0.527877986431122 25.0
0 -0.30224871635437 26.0
0 -0.344644784927368 27.0
0 0.589147806167603 28.0
0 0.093680739402771 29.0
0 0.047471821308136 30.0
0 -0.1439448595047 31.0
0 0.104901969432831 32.0
0 0.24505889415741 33.0
0 0.131820797920227 34.0
0 0.248022198677063 35.0
0 0.201277792453766 36.0
0 -0.0207253098487854 37.0
0 0.159965813159943 38.0
0 -0.0433816909790039 39.0
0 0.869538307189941 40.0
0 -0.0409957766532898 41.0
0 0.00977003574371338 42.0
295.63098587 296.205596923828 43.0
0 0.102960467338562 44.0
0 0.0747066736221313 45.0
0 0.17116904258728 46.0
0 -0.176084756851196 47.0
0 -1.29037117958069 48.0
0 0.654975593090057 49.0
0.080492369965 -0.0849335789680481 50.0
2.2720643893 1.90187740325928 51.0
0 -0.031883716583252 52.0
0 0.169289112091064 53.0
0 -0.0593527555465698 54.0
0 0.248266398906708 55.0
0 0.261070013046265 56.0
0 0.083685040473938 57.0
0 0.556491494178772 58.0
0 0.100678503513336 59.0
0 -0.0241134762763977 60.0
0 0.128003001213074 61.0
0 0.245642066001892 62.0
235.9834042 252.781021118164 0.0
0 -0.0483304262161255 1.0
0 -0.0286867618560791 2.0
0 -0.00768643617630005 3.0
0 -0.302402019500732 4.0
0 -0.19037789106369 5.0
0 0.1235311627388 6.0
0 -0.118889927864075 7.0
0 0.175666689872742 8.0
0 0.343916475772858 9.0
0 0.308189034461975 10.0
0 -0.189420163631439 11.0
0 -0.161407470703125 12.0
60.453966266 59.7984962463379 13.0
42.07636244 35.4091796875 14.0
0 0.00279736518859863 15.0
0 0.0861777663230896 16.0
32.55161438 34.6068496704102 17.0
0 0.143402576446533 18.0
0 -0.00786161422729492 19.0
0 0.0704501867294312 20.0
0 0.205811142921448 21.0
0 -0.0868948698043823 22.0
28.175531348 25.4001712799072 23.0
0 0.0351357460021973 24.0
0 0.328687012195587 25.0
0 -0.156791508197784 26.0
0 -0.259149551391602 27.0
0 0.615433216094971 28.0
0 -0.0413921475410461 29.0
0 0.0401750802993774 30.0
0 -0.146698117256165 31.0
0 -0.0537345409393311 32.0
0 -0.281008958816528 33.0
0 -0.307750463485718 34.0
0 -0.032206654548645 35.0
0 -0.0784943699836731 36.0
0 -0.0300628542900085 37.0
0 0.247443020343781 38.0
0 -0.155482649803162 39.0
0 0.601318299770355 40.0
0 -0.114194989204407 41.0
0 -0.0473215579986572 42.0
302.53042299 297.689331054688 43.0
0 0.174318909645081 44.0
0 0.0464459657669067 45.0
0 0.169184327125549 46.0
0 0.156173169612885 47.0
0 -0.0874091982841492 48.0
0 0.442987263202667 49.0
0.085883718722 0.0215266346931458 50.0
1.805869249 1.82167720794678 51.0
0 0.112741649150848 52.0
0 -0.372277498245239 53.0
0 0.300597220659256 54.0
0 0.259969770908356 55.0
0 0.0162187218666077 56.0
0 -0.119991064071655 57.0
0 0.252708256244659 58.0
0 0.292019635438919 59.0
0 -0.0905040502548218 60.0
0 0.143937349319458 61.0
0 0.353585541248322 62.0
278.9552749 254.860992431641 0.0
0 -0.264780759811401 1.0
0 0.146812498569489 2.0
0 0.15210622549057 3.0
0 0.17009449005127 4.0
0 0.345473408699036 5.0
0 0.204899847507477 6.0
0 0.201256811618805 7.0
0 -0.400970458984375 8.0
0 0.178378522396088 9.0
0 -0.0866143703460693 10.0
0 -0.0142168998718262 11.0
0 -0.167205929756165 12.0
48.988235784 52.2138671875 13.0
50.913591413 42.9273414611816 14.0
0 -0.128838658332825 15.0
0 -0.0666418671607971 16.0
33.498128009 34.2975425720215 17.0
0 0.0204781293869019 18.0
0 -0.00289070606231689 19.0
0 0.147860169410706 20.0
0 0.0494875907897949 21.0
0 0.192220449447632 22.0
38.712898969 38.3980331420898 23.0
0 0.0902388691902161 24.0
9.1277749745 1.18842470645905 25.0
0 -0.22200608253479 26.0
0 0.151803553104401 27.0
0 -0.232683777809143 28.0
0 -0.0321218967437744 29.0
0 -0.109066426753998 30.0
0 0.107572555541992 31.0
0 -0.00975358486175537 32.0
0 0.0313780903816223 33.0
0 0.0507879257202148 34.0
0 -0.0853655338287354 35.0
0 0.0544848442077637 36.0
0 -0.0611059665679932 37.0
0 0.00315433740615845 38.0
0 -0.0243864059448242 39.0
0 0.296892285346985 40.0
0 -0.0247597098350525 41.0
0 -0.0149223804473877 42.0
285.31678736 296.887878417969 43.0
0 0.212199687957764 44.0
0 0.0609827637672424 45.0
0 0.332028865814209 46.0
0 -0.417954564094543 47.0
0 -0.104596972465515 48.0
0 0.54335081577301 49.0
0.080393239348 -0.0271286964416504 50.0
1.5151120086 1.89493227005005 51.0
0 0.0690098404884338 52.0
0 -0.200813233852386 53.0
0 -0.00130236148834229 54.0
0 0.294680029153824 55.0
0 0.267112016677856 56.0
0 0.537637412548065 57.0
0 0.0905353426933289 58.0
0 0.24350905418396 59.0
0 -0.00148630142211914 60.0
0 0.0862037539482117 61.0
0 0.162119805812836 62.0
236.79043036 251.49592590332 0.0
0 0.075451135635376 1.0
0 -0.0251991748809814 2.0
0 -0.00213706493377686 3.0
0 -0.182100594043732 4.0
0 -0.254477977752686 5.0
0 0.0273686647415161 6.0
0 0.0471037030220032 7.0
0 0.103356182575226 8.0
0 -0.20322972536087 9.0
0 -0.227996468544006 10.0
0 -0.295092463493347 11.0
0 -0.174866020679474 12.0
52.196457312 49.8136825561523 13.0
34.316352077 41.4986343383789 14.0
0 -0.0745627880096436 15.0
0 0.114037930965424 16.0
33.637875312 33.6843605041504 17.0
0 0.185065269470215 18.0
0 -0.0221527814865112 19.0
0 0.128933906555176 20.0
0 0.0112812519073486 21.0
0 0.281725287437439 22.0
36.838440484 34.8733100891113 23.0
0 0.0630247592926025 24.0
0 1.03237128257751 25.0
0 -0.00102663040161133 26.0
0 -0.0756365060806274 27.0
0 0.484494298696518 28.0
0 0.0268470048904419 29.0
0 0.100008904933929 30.0
0 -0.119203567504883 31.0
0 -0.0144448280334473 32.0
0 0.0154926776885986 33.0
0 0.358793199062347 34.0
0 -0.164762496948242 35.0
0 0.222716927528381 36.0
0 0.125848114490509 37.0
0 0.00678586959838867 38.0
0 -0.139350533485413 39.0
0 0.313858419656754 40.0
0 -0.158108234405518 41.0
0 0.13025963306427 42.0
302.01659948 295.738067626953 43.0
0 0.114436626434326 44.0
0 0.16797411441803 45.0
0 0.152570009231567 46.0
0 -0.113206028938293 47.0
0 -0.0501993298530579 48.0
0 0.382481634616852 49.0
0.087202340172 0.0429314374923706 50.0
2.1894028871 1.88442277908325 51.0
0 0.303845584392548 52.0
0 -0.221632838249207 53.0
0 -0.0372992157936096 54.0
0 0.184793889522552 55.0
0 0.220631659030914 56.0
0 -0.143440902233124 57.0
0 -0.111789286136627 58.0
0 0.12309330701828 59.0
0 0.122568905353546 60.0
0 0.0643317699432373 61.0
0 0.0724700689315796 62.0
275.52811002 253.71321105957 0.0
0 -0.286377906799316 1.0
0 -0.114448547363281 2.0
0 0.00265389680862427 3.0
0 0.0103441476821899 4.0
0 0.136127650737762 5.0
0 -0.404668211936951 6.0
0 0.428082972764969 7.0
0 0.277148008346558 8.0
0 -0.0161799788475037 9.0
0 0.253841817378998 10.0
0 -0.0645118951797485 11.0
0 -0.0673119425773621 12.0
69.258033846 71.916862487793 13.0
30.641860792 24.1512031555176 14.0
0 0.0278520584106445 15.0
0 -0.177752435207367 16.0
26.915620124 31.6078777313232 17.0
0 0.0457297563552856 18.0
0 -0.0446298122406006 19.0
0 0.18543529510498 20.0
0 -0.0648956298828125 21.0
0 0.303126335144043 22.0
35.785623654 35.6207618713379 23.0
0 0.173503756523132 24.0
0 1.07060194015503 25.0
0 0.0527117252349854 26.0
0 0.0386917591094971 27.0
0 0.741089940071106 28.0
0 0.0252202749252319 29.0
0 0.00739789009094238 30.0
0 -0.0293715000152588 31.0
0 -0.0803808569908142 32.0
0 -0.198919177055359 33.0
0 0.405047804117203 34.0
0 -0.0669785141944885 35.0
0 -0.0650235414505005 36.0
0 0.118512272834778 37.0
0 0.198517560958862 38.0
0 -0.298665523529053 39.0
0 0.188690185546875 40.0
0 -0.178598582744598 41.0
0 0.082371711730957 42.0
298.85453126 297.853759765625 43.0
0 0.11663806438446 44.0
0 0.14956659078598 45.0
0 0.113714873790741 46.0
0 0.264580547809601 47.0
0 0.10934978723526 48.0
0 0.287148177623749 49.0
0.070408304074 0.0713873505592346 50.0
2.2216969395 1.83042299747467 51.0
0 0.134275197982788 52.0
0 -0.0869325995445251 53.0
0 -0.0499494671821594 54.0
0 0.130707085132599 55.0
0 0.229812741279602 56.0
0 0.182461321353912 57.0
0 0.240594089031219 58.0
0 0.146605849266052 59.0
0 -0.160058796405792 60.0
0 0.078259289264679 61.0
0 0.300086677074432 62.0
258.0898583 254.979476928711 0.0
0 -0.181776642799377 1.0
0 0.00521153211593628 2.0
0 0.268522262573242 3.0
0 -0.120059847831726 4.0
0 0.156106531620026 5.0
0 0.168886959552765 6.0
0 -0.28919506072998 7.0
0 -0.0348369479179382 8.0
0 0.158747136592865 9.0
0 -0.204150855541229 10.0
0 -0.35154926776886 11.0
0 -0.179486513137817 12.0
58.684710823 57.3642234802246 13.0
28.362083797 33.0004844665527 14.0
0 -0.216182589530945 15.0
0 -0.231296300888062 16.0
28.00511878 33.1044540405273 17.0
0 0.11117947101593 18.0
0 -0.0222301483154297 19.0
0 0.0995628833770752 20.0
0 0.17994236946106 21.0
0 -0.0607988238334656 22.0
26.136116014 24.4952602386475 23.0
0 0.00345414876937866 24.0
0 0.280125617980957 25.0
0 -0.191482961177826 26.0
0 -0.171805262565613 27.0
0 0.413459688425064 28.0
0 0.113766849040985 29.0
0 0.238132894039154 30.0
0 -0.117480278015137 31.0
0 0.128102540969849 32.0
0 -0.230671167373657 33.0
0 -0.153006374835968 34.0
0 0.310501515865326 35.0
0 0.0508097410202026 36.0
0 0.1187903881073 37.0
0 0.00150740146636963 38.0
0 -0.0218994617462158 39.0
0 -0.0140050649642944 40.0
0 -0.113795280456543 41.0
0 0.282555818557739 42.0
303.41825155 296.950714111328 43.0
0 0.166174054145813 44.0
0 0.18972259759903 45.0
0 0.159437537193298 46.0
0 -0.446371674537659 47.0
0 0.0752841830253601 48.0
0 0.318426072597504 49.0
0.079684317409 0.0393370389938354 50.0
1.684521442 1.8192286491394 51.0
0 -0.0878657102584839 52.0
0 -0.328039407730103 53.0
0 -0.168633937835693 54.0
0 0.0770419836044312 55.0
0 0.270323216915131 56.0
0 -0.0998348593711853 57.0
0 -0.0064588189125061 58.0
0 -0.0179558992385864 59.0
0 -0.150405585765839 60.0
0 -0.0302945971488953 61.0
0 0.00614702701568604 62.0
259.2429644 250.996963500977 0.0
0 -0.113364815711975 1.0
0 0.16403865814209 2.0
0 0.122442305088043 3.0
0 -0.0398394465446472 4.0
0 -0.078769862651825 5.0
0 0.0565107464790344 6.0
0 0.267691969871521 7.0
0 -0.155770838260651 8.0
0 0.198469758033752 9.0
0 -0.229965209960938 10.0
0 -0.213132381439209 11.0
0 -0.0248112678527832 12.0
72.491194817 72.2167282104492 13.0
18.87755827 23.5315914154053 14.0
0 -0.0726112127304077 15.0
0 0.0505412817001343 16.0
35.266665616 36.1928634643555 17.0
0 0.112897872924805 18.0
0 -0.019997239112854 19.0
0 0.0587154626846313 20.0
0 0.115270495414734 21.0
0 0.129339456558228 22.0
29.261050709 28.8796844482422 23.0
0 0.0854426622390747 24.0
0 0.90299254655838 25.0
0 -0.00166183710098267 26.0
0 -0.0817803144454956 27.0
0 0.190549373626709 28.0
0 -0.122438192367554 29.0
0 0.0618818402290344 30.0
0 -0.123844742774963 31.0
0 0.0597712993621826 32.0
0 0.0863443613052368 33.0
0 -0.0820083618164062 34.0
0 -0.0285868048667908 35.0
0 0.0888855457305908 36.0
0 0.011962890625 37.0
0 0.180342018604279 38.0
0 -0.203108549118042 39.0
0 0.335291564464569 40.0
0 -0.167865931987762 41.0
0 -0.0369824171066284 42.0
282.50309314 298.153198242188 43.0
0 0.132387757301331 44.0
0 0.0434285402297974 45.0
0 0.225875854492188 46.0
0 0.00821834802627563 47.0
0 -0.193574488162994 48.0
0 0.479230463504791 49.0
0.072925460211 0.0303094983100891 50.0
2.2388166397 1.83843100070953 51.0
0 0.307378113269806 52.0
0 -0.197465121746063 53.0
0 0.115697681903839 54.0
0 0.226642608642578 55.0
0 -0.0884353518486023 56.0
0 0.172222793102264 57.0
0 0.36924147605896 58.0
0 0.151114821434021 59.0
0 0.260734915733337 60.0
0 0.122148036956787 61.0
0 0.233854293823242 62.0
229.46714368 253.965789794922 0.0
0 -0.28453254699707 1.0
0 0.144436836242676 2.0
0 0.0595634579658508 3.0
0 -0.384506225585938 4.0
0 -0.184822857379913 5.0
0 0.108047902584076 6.0
0 -0.0538290739059448 7.0
0 0.394560188055038 8.0
0 -0.192305564880371 9.0
0 -0.151729643344879 10.0
0 -0.297012805938721 11.0
0 -0.164809823036194 12.0
81.816986617 74.527473449707 13.0
15.768766602 24.1379222869873 14.0
0 -0.0739314556121826 15.0
0 0.0480348467826843 16.0
43.190744436 36.3934516906738 17.0
0 0.153746008872986 18.0
0 -0.121361494064331 19.0
0 0.0482999086380005 20.0
0 0.117617011070251 21.0
0 0.0983599424362183 22.0
27.095534136 27.1189365386963 23.0
0 -0.00199443101882935 24.0
0 -0.0109041333198547 25.0
0 -0.100219488143921 26.0
0 0.0453535318374634 27.0
0 0.15158748626709 28.0
0 0.0728501081466675 29.0
0 -0.258811831474304 30.0
0 0.490872502326965 31.0
0 0.145512700080872 32.0
0 -0.0460290908813477 33.0
0 0.0481440424919128 34.0
0 0.0741173624992371 35.0
0 0.392641603946686 36.0
0 -0.0464879870414734 37.0
0 -0.0388983488082886 38.0
0 -0.150485992431641 39.0
0 -0.0305769443511963 40.0
0 -0.0925258994102478 41.0
0 0.17848265171051 42.0
289.39495836 295.650054931641 43.0
0 0.242174744606018 44.0
0 0.116845369338989 45.0
0 0.233956277370453 46.0
0 -0.267662048339844 47.0
0 -0.00419235229492188 48.0
0 0.389378130435944 49.0
0.091526673166 0.0313736796379089 50.0
1.4856117511 1.86339116096497 51.0
0 -0.157769620418549 52.0
0 -0.506763458251953 53.0
0 -0.00947725772857666 54.0
0 0.0876213312149048 55.0
0 0.241948485374451 56.0
0 -0.132938742637634 57.0
0 -0.0494752526283264 58.0
0 -0.790114164352417 59.0
0 0.849849343299866 60.0
0 0.00685036182403564 61.0
0 -0.000242292881011963 62.0
278.75643369 254.829238891602 0.0
0 -0.252873182296753 1.0
0 0.203159332275391 2.0
0 0.163962721824646 3.0
0 0.0419392585754395 4.0
0 -0.0164512395858765 5.0
0 -0.113162875175476 6.0
0 0.221088945865631 7.0
0 -0.219834446907043 8.0
0 0.130915760993958 9.0
0 -0.301899671554565 10.0
0 0.000956416130065918 11.0
0 0.155851304531097 12.0
60.591985627 61.7472648620605 13.0
40.756077199 35.8289260864258 14.0
0 -0.125155091285706 15.0
0 -0.109025120735168 16.0
34.426570234 38.0165061950684 17.0
0 0.138122797012329 18.0
0 -0.0416481494903564 19.0
0 0.0331780910491943 20.0
0 0.158017635345459 21.0
0 0.0349079370498657 22.0
29.591181045 27.4647121429443 23.0
0 0.0562602281570435 24.0
0 0.188017427921295 25.0
0 0.0242835879325867 26.0
0 -0.064240574836731 27.0
0 0.0069003701210022 28.0
0 -0.143531799316406 29.0
0 0.0252625346183777 30.0
0 -0.15498960018158 31.0
0 -0.0977749824523926 32.0
0 -0.20097541809082 33.0
0 0.37885507941246 34.0
0 -0.0339541435241699 35.0
0 0.0700291395187378 36.0
0 -0.057708203792572 37.0
0 0.472909063100815 38.0
0 -0.271701574325562 39.0
0 -0.370215177536011 40.0
0 -0.188258647918701 41.0
0 -0.127376437187195 42.0
293.26135685 297.279205322266 43.0
0 0.102724432945251 44.0
0 0.138470709323883 45.0
0 0.10267847776413 46.0
0 0.0990386009216309 47.0
0 0.0982401967048645 48.0
0 0.385046750307083 49.0
0.071721666677 0.0230260491371155 50.0
1.6649417791 1.89597356319427 51.0
0 0.0722748637199402 52.0
0 -0.264603018760681 53.0
0 0.218252003192902 54.0
0 0.346057564020157 55.0
0 -0.0636115074157715 56.0
0 0.23469066619873 57.0
0 0.304007977247238 58.0
0 0.260796785354614 59.0
0 -0.270780801773071 60.0
0 0.0960386991500854 61.0
0 0.320007503032684 62.0
279.70088099 252.283081054688 0.0
0 -0.375125169754028 1.0
0 -0.00556689500808716 2.0
0 0.0441634058952332 3.0
0 -0.270514488220215 4.0
0 0.32353413105011 5.0
0 -0.16070568561554 6.0
0 0.308116734027863 7.0
0 -0.010708749294281 8.0
0 -0.0511584877967834 9.0
0 0.379340082406998 10.0
0 0.00185370445251465 11.0
0 -0.167304039001465 12.0
63.959198209 65.3080749511719 13.0
44.11608437 32.2807006835938 14.0
0 -0.148298859596252 15.0
0 -0.286848545074463 16.0
29.07866159 33.3130531311035 17.0
0 0.0518349409103394 18.0
0 -0.00326371192932129 19.0
0 0.195323050022125 20.0
0 -0.0489168167114258 21.0
0 0.558620691299438 22.0
38.388323032 39.1257781982422 23.0
0 0.0371243953704834 24.0
0.22530124412 0.942650318145752 25.0
0 -0.195508122444153 26.0
0 0.161672353744507 27.0
0 -0.440839171409607 28.0
0 -0.0615137815475464 29.0
0 -0.0292186737060547 30.0
0 0.205616593360901 31.0
0 -0.068385899066925 32.0
0 -0.375232219696045 33.0
0 0.258192956447601 34.0
0 0.192447662353516 35.0
0 0.48774841427803 36.0
0 -0.037029504776001 37.0
0 0.0535865426063538 38.0
0 -0.0785039663314819 39.0
0 0.368699252605438 40.0
0 -0.0966564416885376 41.0
0 0.134386658668518 42.0
294.06010271 297.438598632812 43.0
0 0.0925729274749756 44.0
0 0.0301569700241089 45.0
0 -0.0504297018051147 46.0
0 0.147290170192719 47.0
0 -0.11333841085434 48.0
0 0.343617111444473 49.0
0.077057333571 -0.00410962104797363 50.0
2.064094059 1.85146355628967 51.0
0 -0.0956965684890747 52.0
0 -0.268193483352661 53.0
0 0.0521356463432312 54.0
0 -0.0108324289321899 55.0
0 0.106671690940857 56.0
0 0.330896496772766 57.0
0 0.217468440532684 58.0
0 0.000548899173736572 59.0
0 -0.264607191085815 60.0
0 -0.087813675403595 61.0
0 -0.120389759540558 62.0
285.87677014 253.901809692383 0.0
0 -0.198691129684448 1.0
0 0.0705962777137756 2.0
0 0.0506913661956787 3.0
0 0.173679649829865 4.0
0 0.120176792144775 5.0
0 0.209059417247772 6.0
0 0.398674428462982 7.0
0 0.201222002506256 8.0
0 0.0459557175636292 9.0
0 0.682235062122345 10.0
0 -0.45819890499115 11.0
0 -0.250280618667603 12.0
67.682685537 66.6173782348633 13.0
23.908782334 31.793306350708 14.0
0 -0.0245821475982666 15.0
0 -0.0285043120384216 16.0
34.699577358 32.9233818054199 17.0
0 0.106625854969025 18.0
0 -0.0698918104171753 19.0
0 0.0838327407836914 20.0
0 0.0772639513015747 21.0
0 0.214002966880798 22.0
34.914879306 31.6193466186523 23.0
0 0.0727046728134155 24.0
0 0.528242230415344 25.0
0 -0.144672513008118 26.0
0 0.123281419277191 27.0
0 -0.307653665542603 28.0
0 0.020715057849884 29.0
0 0.00991988182067871 30.0
0 -0.0646203756332397 31.0
0 -0.052107572555542 32.0
0 -0.25956916809082 33.0
0 0.374506235122681 34.0
0 -0.134320735931396 35.0
0 0.367629408836365 36.0
0 0.153238952159882 37.0
0 0.111892819404602 38.0
0 -0.0875128507614136 39.0
0 0.62512868642807 40.0
0 -0.0877642035484314 41.0
0 0.154393792152405 42.0
294.97744891 298.840637207031 43.0
0 0.150593996047974 44.0
0 0.037062406539917 45.0
0 0.143201053142548 46.0
0 -0.579820156097412 47.0
0 0.333435267210007 48.0
0 0.519608974456787 49.0
0.10859238404 0.0801368355751038 50.0
1.943095784 1.82018935680389 51.0
0 -0.232741475105286 52.0
0 -0.226140379905701 53.0
0 -0.0411683320999146 54.0
0 -0.00596177577972412 55.0
0 0.28108823299408 56.0
0 0.286301612854004 57.0
0 0.0341364741325378 58.0
0 -0.86163318157196 59.0
0 0.519228219985962 60.0
0 0.106382191181183 61.0
0 0.132630288600922 62.0
233.74491392 250.785049438477 0.0
0 0.121167302131653 1.0
0 -0.0407364368438721 2.0
0 0.177596390247345 3.0
0 -0.45540452003479 4.0
0 -0.125693678855896 5.0
0 -0.102592289447784 6.0
0 -0.212335705757141 7.0
0 0.124284088611603 8.0
0 0.207356750965118 9.0
0 -0.46652364730835 10.0
0 0.00198948383331299 11.0
0 -0.0428269505500793 12.0
63.226397139 63.7654685974121 13.0
40.629856699 32.6499404907227 14.0
0 -0.0256527662277222 15.0
0 -0.0741685628890991 16.0
36.937310964 35.6589660644531 17.0
0 0.100994467735291 18.0
0 -0.0358210206031799 19.0
0 0.00137150287628174 20.0
0 0.255054414272308 21.0
0 0.0892714262008667 22.0
28.357222253 24.4521656036377 23.0
0 0.121160626411438 24.0
0 1.05100536346436 25.0
0 -0.131035387516022 26.0
0 0.0942245125770569 27.0
0 0.33221772313118 28.0
0 0.0651102066040039 29.0
0 -0.0544494986534119 30.0
0 -0.15295946598053 31.0
0 -0.0216695666313171 32.0
0 0.0259729027748108 33.0
0 0.2005615234375 34.0
0 -0.057328462600708 35.0
0 0.31832891702652 36.0
0 0.179170310497284 37.0
0 0.0547589659690857 38.0
0 -0.261082410812378 39.0
0 0.0574510097503662 40.0
0 -0.193340957164764 41.0
0 0.0366873741149902 42.0
290.8565828 296.167358398438 43.0
0 0.182462930679321 44.0
0 0.218893826007843 45.0
0 0.0101240277290344 46.0
0 0.303991794586182 47.0
0 0.365098297595978 48.0
0 0.405404567718506 49.0
0.092757093424 0.0184389352798462 50.0
1.5847780696 1.90349912643433 51.0
0 -0.0153380632400513 52.0
0 0.152752161026001 53.0
0 -0.00102853775024414 54.0
0 -0.0265380144119263 55.0
0 0.0640111565589905 56.0
0 0.238709807395935 57.0
0 0.343336910009384 58.0
0 -0.0178255438804626 59.0
0 0.332103878259659 60.0
0 -0.0674757957458496 61.0
0 0.120149612426758 62.0
292.32594393 253.513778686523 0.0
0 -0.368872046470642 1.0
0 0.0541895627975464 2.0
0 0.124793350696564 3.0
0 -0.0332102179527283 4.0
0 0.3738833963871 5.0
0 0.54081666469574 6.0
0 0.442789196968079 7.0
0 0.495097935199738 8.0
0 -0.108903467655182 9.0
0 0.143527567386627 10.0
0 -0.0897695422172546 11.0
0 -0.159711122512817 12.0
69.523303678 70.4916687011719 13.0
26.563109338 24.4298305511475 14.0
0 -0.151806116104126 15.0
0 -0.179193615913391 16.0
28.255373994 32.1859588623047 17.0
0 0.0685253143310547 18.0
0 0.0862033367156982 19.0
0 0.0806425809860229 20.0
0 0.220986127853394 21.0
0 -0.00644385814666748 22.0
28.809435015 28.0378971099854 23.0
0 0.047989547252655 24.0
0 0.490007281303406 25.0
0 -0.10117781162262 26.0
0 -0.0152114033699036 27.0
0 0.341953068971634 28.0
0 0.0723112225532532 29.0
0 0.0968562364578247 30.0
0 0.0899826288223267 31.0
0 0.102274775505066 32.0
0 -0.0694748759269714 33.0
0 0.199252128601074 34.0
0 -0.0481539368629456 35.0
0 0.344160109758377 36.0
0 0.00621604919433594 37.0
0 0.0833943486213684 38.0
0 -0.0821232795715332 39.0
0 0.166330516338348 40.0
0 -0.081251323223114 41.0
0 0.21866774559021 42.0
292.9634541 298.366424560547 43.0
0 0.189773917198181 44.0
0 0.047234058380127 45.0
0 0.268859803676605 46.0
0 -0.0607300996780396 47.0
0 0.240989863872528 48.0
0.4262285813 0.319965064525604 49.0
0.093864799732 -0.00799840688705444 50.0
1.7240773046 1.88552558422089 51.0
0 -0.0912662744522095 52.0
0 -0.472089886665344 53.0
0 0.102619886398315 54.0
0 0.220394134521484 55.0
0 0.136970281600952 56.0
0 0.143210053443909 57.0
0 -0.0623236894607544 58.0
0 0.242111504077911 59.0
0 0.147059559822083 60.0
0 0.0207436084747314 61.0
0 -0.0145681500434875 62.0
239.1078801 252.994476318359 0.0
0 -0.123567819595337 1.0
0 0.0147926807403564 2.0
0 0.244212210178375 3.0
0 0.00786960124969482 4.0
0 -0.199323058128357 5.0
0 -0.274205803871155 6.0
0 0.0334653854370117 7.0
0 0.364045411348343 8.0
0 -0.605427861213684 9.0
0 0.0892901420593262 10.0
0 -0.103119015693665 11.0
0 0.00815439224243164 12.0
66.235542845 64.1937637329102 13.0
32.112309833 32.9789123535156 14.0
0 0.0331845283508301 15.0
0 0.301384150981903 16.0
41.504587676 39.652099609375 17.0
0 0.138988077640533 18.0
0 0.0316698551177979 19.0
0 -0.0395888090133667 20.0
0 0.199387192726135 21.0
0 0.133695602416992 22.0
30.037984899 28.3755970001221 23.0
0 0.0799927711486816 24.0
0 0.952996850013733 25.0
0 0.14135867357254 26.0
0 -0.306524753570557 27.0
0 0.367935508489609 28.0
0 0.145813703536987 29.0
0 -0.0139046311378479 30.0
0 0.125016689300537 31.0
0 0.0748276710510254 32.0
0 0.0312924981117249 33.0
0 0.118846297264099 34.0
0 -0.158444881439209 35.0
0 0.349335581064224 36.0
0 0.0813487768173218 37.0
0 -0.0433743596076965 38.0
0 -0.252411127090454 39.0
0 0.357289582490921 40.0
0 -0.233449101448059 41.0
0 0.0901203155517578 42.0
301.61068488 296.736968994141 43.0
0 0.0306112766265869 44.0
0 0.11816793680191 45.0
0 0.0972793102264404 46.0
0 -0.0357020497322083 47.0
0 -0.0662639737129211 48.0
0 0.457655370235443 49.0
0.075579692182 0.0113075375556946 50.0
1.6644130673 1.79496049880981 51.0
0 0.103085994720459 52.0
0 -0.282486081123352 53.0
0 -0.112284779548645 54.0
0 0.0478911399841309 55.0
0 -0.0263503789901733 56.0
0 0.465118527412415 57.0
0 0.627745568752289 58.0
0 0.216508805751801 59.0
0 -0.217501282691956 60.0
0 0.0849999189376831 61.0
0 0.361103177070618 62.0
269.65509803 252.863845825195 0.0
0 0.0467160940170288 1.0
0 0.0134304165840149 2.0
0 0.535391569137573 3.0
0 0.0432490110397339 4.0
0 0.164309859275818 5.0
0 -0.337652206420898 6.0
0 0.230651438236237 7.0
0 0.458462536334991 8.0
0 -0.27820611000061 9.0
0 0.157577991485596 10.0
0 -0.0650420188903809 11.0
0 -0.0825357437133789 12.0
61.084048008 61.4691696166992 13.0
37.771865436 34.4063034057617 14.0
0 -0.0826108455657959 15.0
0 -0.101786255836487 16.0
33.559687325 32.0509796142578 17.0
0 0.0744515657424927 18.0
0 0.00102603435516357 19.0
0 0.122576832771301 20.0
0 0.0883893966674805 21.0
0 0.161688685417175 22.0
35.250780522 36.3356628417969 23.0
0 0.12899649143219 24.0
0 0.766750812530518 25.0
0 0.0910205841064453 26.0
0 0.0422192811965942 27.0
0 0.0123265385627747 28.0
0 -0.0385770201683044 29.0
0 0.0202567577362061 30.0
0 0.0152527093887329 31.0
0 -0.057311475276947 32.0
0 -0.161334693431854 33.0
0 0.177049517631531 34.0
0 -0.0127848386764526 35.0
0 0.342037558555603 36.0
0 -0.0743428468704224 37.0
0 0.279012858867645 38.0
0 -0.290446519851685 39.0
0 0.100364446640015 40.0
0 -0.202287256717682 41.0
0 0.0602865219116211 42.0
290.19431974 297.132263183594 43.0
0 0.120447754859924 44.0
0 0.116629958152771 45.0
0 0.0721980929374695 46.0
0 0.243628859519958 47.0
0 0.398304641246796 48.0
0 0.400535970926285 49.0
0.095594335305 0.0546214580535889 50.0
1.6282663224 1.85270524024963 51.0
0 -0.0438216924667358 52.0
0 -0.194168329238892 53.0
0 -0.0951325297355652 54.0
0 0.17143726348877 55.0
0 -0.0872808694839478 56.0
0 -0.0750823020935059 57.0
0 0.0220993757247925 58.0
0 -0.093887984752655 59.0
0 -0.0935844779014587 60.0
0 0.0294893980026245 61.0
0 0.0706385374069214 62.0
251.96702745 254.442596435547 0.0
0 -0.206364035606384 1.0
0 0.0616196990013123 2.0
0 0.476930141448975 3.0
0 -0.256354570388794 4.0
0 0.251473248004913 5.0
0 0.0107299089431763 6.0
0 0.333326697349548 7.0
0 -0.158697366714478 8.0
0 0.25575464963913 9.0
0 -0.0239554047584534 10.0
0 0.191724598407745 11.0
0 -0.233140707015991 12.0
69.992569034 71.6558609008789 13.0
38.122754273 28.9820957183838 14.0
0 -0.0983151197433472 15.0
0 -0.121139883995056 16.0
39.512086424 35.9024887084961 17.0
0 0.121676564216614 18.0
0 0.030446469783783 19.0
0 0.0415647029876709 20.0
0 0.24123203754425 21.0
0 0.0137984752655029 22.0
37.853688804 36.0844573974609 23.0
0 0.0658200979232788 24.0
0 1.16644859313965 25.0
0 -0.0964124202728271 26.0
0 -0.0503539443016052 27.0
0 -0.0472622513771057 28.0
0 -0.086916446685791 29.0
0 0.0819772481918335 30.0
0 0.0683766603469849 31.0
0 0.147250652313232 32.0
0 0.130165219306946 33.0
0 0.00289559364318848 34.0
0 -0.14874804019928 35.0
0 0.317736238241196 36.0
0 0.0972779393196106 37.0
0 -0.378048539161682 38.0
0 -0.520906686782837 39.0
0 0.396978914737701 40.0
0 -0.278228402137756 41.0
0 0.0416253805160522 42.0
290.23449131 298.052917480469 43.0
0 0.0235806703567505 44.0
0 0.0800002217292786 45.0
0 0.166334807872772 46.0
0 0.0586381554603577 47.0
0 0.283091366291046 48.0
0 0.348806381225586 49.0
0.098581653162 0.00146389007568359 50.0
2.2157008611 1.91451597213745 51.0
0 -0.0907742977142334 52.0
0 -0.308126449584961 53.0
0 0.0223833322525024 54.0
0 0.145868062973022 55.0
0 0.0182498693466187 56.0
0 0.307404518127441 57.0
0 0.343639761209488 58.0
0 0.168764352798462 59.0
0 -0.275654792785645 60.0
0 0.00382739305496216 61.0
0 0.143060922622681 62.0
221.87361643 251.679611206055 0.0
0 -0.114915490150452 1.0
0 -0.162535011768341 2.0
0 -0.177330732345581 3.0
0 -0.327726602554321 4.0
0 -0.119607269763947 5.0
0 -0.18333774805069 6.0
0 -0.211910605430603 7.0
0 0.478225618600845 8.0
0 0.06844562292099 9.0
0 -0.157333791255951 10.0
0 -0.357988953590393 11.0
0 -0.110654354095459 12.0
59.034370612 57.7110595703125 13.0
27.619327639 34.0368499755859 14.0
0 0.0558714866638184 15.0
0 0.0674241185188293 16.0
30.852876702 34.3614921569824 17.0
0 0.114476919174194 18.0
0 0.0154803991317749 19.0
0 0.109974265098572 20.0
0 0.0391541123390198 21.0
0 0.201764106750488 22.0
39.781319642 38.4350814819336 23.0
0 0.0623050928115845 24.0
0 0.912988066673279 25.0
0 -0.118456184864044 26.0
0 -0.0318599343299866 27.0
0 -0.0351579785346985 28.0
0 -0.0598020553588867 29.0
0 0.167620658874512 30.0
0 -0.155273199081421 31.0
0 0.00403279066085815 32.0
0 0.143205642700195 33.0
0 -0.0337324142456055 34.0
0 0.368776917457581 35.0
0 -0.0643998980522156 36.0
0 0.0697678327560425 37.0
0 0.0991227626800537 38.0
0 -0.359149098396301 39.0
0 -0.172630548477173 40.0
0 -0.144318103790283 41.0
0 -0.187735676765442 42.0
295.29156438 297.491638183594 43.0
0 0.187714099884033 44.0
0 0.0334533452987671 45.0
0 0.110853374004364 46.0
0 -0.117075383663177 47.0
0 0.0669565200805664 48.0
0 0.526831567287445 49.0
0.065545535818 -0.0579107403755188 50.0
1.8379005543 1.93081116676331 51.0
0 -0.132588565349579 52.0
0 -0.447596907615662 53.0
0 0.392325043678284 54.0
0 0.0953508615493774 55.0
0 -0.0972204208374023 56.0
0 0.0961105227470398 57.0
0 0.270913541316986 58.0
0 -0.293619394302368 59.0
0 0.482024490833282 60.0
0 0.0174319744110107 61.0
0 0.223025977611542 62.0
214.98327155 254.191619873047 0.0
0 -0.1774742603302 1.0
0 -0.145562410354614 2.0
0 -0.000387072563171387 3.0
0 -0.294005870819092 4.0
0 -0.318760991096497 5.0
0 -0.0104127526283264 6.0
0 -0.340446710586548 7.0
0 0.342941373586655 8.0
0 -0.696556925773621 9.0
0 -0.188274681568146 10.0
0 -0.268061399459839 11.0
0 -0.17838716506958 12.0
71.887816988 72.7350692749023 13.0
30.595490687 23.1802978515625 14.0
0 -0.0691690444946289 15.0
0 0.00419831275939941 16.0
33.049350699 35.8344268798828 17.0
0 0.173885107040405 18.0
0 -0.110958158969879 19.0
0 0.430693000555038 20.0
0 -0.0731843113899231 21.0
0 0.304839879274368 22.0
39.875402078 39.71875 23.0
0 0.00732332468032837 24.0
0 0.647266507148743 25.0
0 0.182409882545471 26.0
0 0.0684866905212402 27.0
0 -0.247546315193176 28.0
0 -0.121305167675018 29.0
0 0.171992659568787 30.0
0 -0.14721405506134 31.0
0 -0.0667488574981689 32.0
0 -0.103733956813812 33.0
0 0.509968280792236 34.0
0 0.216971337795258 35.0
0 0.545462906360626 36.0
0 0.0352534055709839 37.0
0 0.187234163284302 38.0
0 -0.145335674285889 39.0
0 0.169716477394104 40.0
0 -0.175941944122314 41.0
0 0.0391454696655273 42.0
298.53518202 297.802307128906 43.0
0 0.216750383377075 44.0
0 0.0981172323226929 45.0
0 0.131145179271698 46.0
0 -0.0759792923927307 47.0
0 0.0957072973251343 48.0
0 0.364587724208832 49.0
0.065279441021 0.021555483341217 50.0
1.684512634 1.9008936882019 51.0
0 0.353224277496338 52.0
0 0.267651498317719 53.0
0 0.255321264266968 54.0
0 -0.0136854648590088 55.0
0 -0.205234706401825 56.0
0 0.0426075458526611 57.0
0 0.345601886510849 58.0
0 0.217975735664368 59.0
0 0.0621358752250671 60.0
0 0.134665906429291 61.0
0 0.360323309898376 62.0
238.28621966 252.108444213867 0.0
0 -0.00761759281158447 1.0
0 -0.0841590762138367 2.0
0 -0.0801221132278442 3.0
0 -0.432262182235718 4.0
0 -0.066215991973877 5.0
0 -0.023323655128479 6.0
0 -0.0910020470619202 7.0
0 0.114824771881104 8.0
0 -0.00257271528244019 9.0
0 -0.0549582242965698 10.0
0 0.00800842046737671 11.0
0 -0.00407254695892334 12.0
54.90366638 54.6891441345215 13.0
52.777318959 45.347599029541 14.0
0 -0.0871973037719727 15.0
0 -0.0693449378013611 16.0
32.32894565 31.7722721099854 17.0
0 0.0971475839614868 18.0
0 0.0377216935157776 19.0
0 0.104102969169617 20.0
0 0.154145956039429 21.0
0 -0.0241929292678833 22.0
27.986443286 26.5790309906006 23.0
0 0.0538209080696106 24.0
0 0.444700837135315 25.0
0 -0.131872355937958 26.0
0 -0.0258302092552185 27.0
0 0.0883040428161621 28.0
0 -0.00902330875396729 29.0
0 0.000991880893707275 30.0
0 0.17139744758606 31.0
0 -0.00632214546203613 32.0
0 -0.123827457427979 33.0
0 0.343102097511292 34.0
0 -0.0657193064689636 35.0
0 0.347272604703903 36.0
0 0.194023072719574 37.0
0 0.0216159820556641 38.0
0 -0.168713569641113 39.0
0 0.304637253284454 40.0
0 -0.168953359127045 41.0
0 -0.00042414665222168 42.0
297.56626058 295.882141113281 43.0
0 0.0807411670684814 44.0
0 0.358136534690857 45.0
0 0.220834910869598 46.0
0 0.032447338104248 47.0
0 0.316624194383621 48.0
1.485817682 0.394768536090851 49.0
0.076859983049 0.0819088220596313 50.0
1.7325096791 1.87093615531921 51.0
0 0.160349130630493 52.0
0 -0.121724367141724 53.0
0 -0.147787809371948 54.0
0 0.0356122255325317 55.0
0 0.409172534942627 56.0
0 0.0162020325660706 57.0
0 0.0475988984107971 58.0
0 0.0268486142158508 59.0
0 -0.00948965549468994 60.0
0 0.013594925403595 61.0
0 0.099677562713623 62.0
268.73683534 252.360214233398 0.0
0 -0.0952715873718262 1.0
0 -0.10363382101059 2.0
0 0.3778116106987 3.0
0 -0.0920989513397217 4.0
0 0.111080229282379 5.0
0 -0.00657522678375244 6.0
0 0.238513171672821 7.0
0 -0.255850791931152 8.0
0 0.332387804985046 9.0
0 -0.485111236572266 10.0
0 -0.0634877681732178 11.0
0 -0.026090681552887 12.0
60.984075135 61.8220329284668 13.0
38.439238265 32.0869140625 14.0
0 -0.0108389854431152 15.0
0 0.0500178337097168 16.0
39.179407764 36.2402992248535 17.0
0 0.183144509792328 18.0
0 -0.0423818826675415 19.0
0 0.0736183524131775 20.0
0 0.0928565263748169 21.0
0 0.306214332580566 22.0
33.611903438 32.5326805114746 23.0
0 0.113436818122864 24.0
0 1.29105603694916 25.0
0 0.0165109038352966 26.0
0 -0.00418192148208618 27.0
0 -0.108024716377258 28.0
0 -0.17053210735321 29.0
0 0.134472548961639 30.0
0 -0.143598079681396 31.0
0 -0.175172924995422 32.0
0 -0.198931396007538 33.0
0 0.35209783911705 34.0
0 -0.0467341542243958 35.0
0 0.145581841468811 36.0
0 0.149040162563324 37.0
0 0.0340375304222107 38.0
0 -0.168094277381897 39.0
0 0.48717188835144 40.0
0 -0.117589950561523 41.0
0 -0.0326709747314453 42.0
287.1059853 296.867370605469 43.0
0 0.018492579460144 44.0
0 0.0745551586151123 45.0
0 0.185028433799744 46.0
0 0.230310916900635 47.0
0 0.219088435173035 48.0
0 0.407423555850983 49.0
0.091702639074 0.0467577576637268 50.0
2.1224256986 1.81223917007446 51.0
0 0.022144079208374 52.0
0 0.0384958386421204 53.0
0 0.177165150642395 54.0
0 0.13548731803894 55.0
0 -0.0809717774391174 56.0
0 0.287666320800781 57.0
0 -0.0114456415176392 58.0
0 0.0513761043548584 59.0
0 -0.0782170295715332 60.0
0 0.0589649677276611 61.0
0 0.199513971805573 62.0
255.53472348 255.18359375 0.0
0 -0.183787226676941 1.0
0 -0.0901392698287964 2.0
0 0.0934965014457703 3.0
0 -0.360074996948242 4.0
0 -0.287237882614136 5.0
0 0.0417927503585815 6.0
0 -0.00889575481414795 7.0
0 0.325706422328949 8.0
0 -0.551366329193115 9.0
0 -0.216028332710266 10.0
0 0.597599267959595 11.0
0 -0.318124413490295 12.0
47.318277851 42.2802238464355 13.0
32.79229526 38.3827972412109 14.0
0 -0.0711439847946167 15.0
0 -0.127865791320801 16.0
25.453134098 32.4895248413086 17.0
0 0.0263047814369202 18.0
0 0.0436850190162659 19.0
0 0.0415630340576172 20.0
0 0.0803142786026001 21.0
0 0.334045052528381 22.0
37.581278578 37.0478782653809 23.0
0 0.111481130123138 24.0
0 1.43540620803833 25.0
0 -0.127777218818665 26.0
0 -0.0431806445121765 27.0
0 -0.149976015090942 28.0
0 -0.0385387539863586 29.0
0 0.012505829334259 30.0
0 -0.0601738691329956 31.0
0 0.129447102546692 32.0
0 -0.0532239675521851 33.0
0 -0.0533947944641113 34.0
0 0.256843090057373 35.0
0 0.284553408622742 36.0
0 -0.0370612144470215 37.0
0 0.246346294879913 38.0
0 -0.333010077476501 39.0
0 0.212149858474731 40.0
0 -0.10334038734436 41.0
0 -0.0355793237686157 42.0
304.65070894 294.930023193359 43.0
0 0.0760225057601929 44.0
0 0.255096554756165 45.0
0 0.188357353210449 46.0
0 0.0935667753219604 47.0
0 0.138646006584167 48.0
0 0.483264088630676 49.0
0.065633409226 0.0476753115653992 50.0
1.6199125903 1.8281512260437 51.0
0 0.239395141601562 52.0
0 -0.170278549194336 53.0
0 -0.0289461612701416 54.0
0 -0.141943335533142 55.0
0 -0.0765281915664673 56.0
0 -0.30975079536438 57.0
0 0.262101054191589 58.0
0 0.367991328239441 59.0
0 -0.15681517124176 60.0
0 0.177929103374481 61.0
0 0.440597534179688 62.0
251.31625314 252.975219726562 0.0
0 -0.257391571998596 1.0
0 0.0192652940750122 2.0
0 0.344575345516205 3.0
0 -0.0855730772018433 4.0
0 0.220262944698334 5.0
0 0.0889045000076294 6.0
0 0.0359615683555603 7.0
0 0.192229688167572 8.0
0 -0.112848520278931 9.0
0 0.0587397217750549 10.0
0 -0.210601449012756 11.0
0 -0.0148464441299438 12.0
57.708967037 56.8933639526367 13.0
43.747324778 37.6877746582031 14.0
0 -0.00145399570465088 15.0
0 0.0519608855247498 16.0
40.036539983 34.6730690002441 17.0
0 0.150138378143311 18.0
0 -0.0206464529037476 19.0
0 0.0783585906028748 20.0
0 0.111202836036682 21.0
0 0.270568311214447 22.0
34.16217594 33.5569801330566 23.0
0 0.0810396075248718 24.0
0 1.11910712718964 25.0
0 -0.0543930530548096 26.0
0 -0.140957534313202 27.0
0 -0.123902916908264 28.0
0 -0.0596747398376465 29.0
0 0.00739580392837524 30.0
0 -0.0624887943267822 31.0
0 -0.0144644975662231 32.0
0 0.0299146175384521 33.0
0 0.325713336467743 34.0
0 0.155689418315887 35.0
0 0.247468829154968 36.0
0 -0.0452343821525574 37.0
0 0.195967674255371 38.0
0 -0.278394103050232 39.0
0 -0.0802936553955078 40.0
0 -0.167954087257385 41.0
0 -0.0568171739578247 42.0
300.39455794 297.961578369141 43.0
0 0.0301203727722168 44.0
0 0.126907527446747 45.0
0 0.103164732456207 46.0
0 0.000996947288513184 47.0
0 -0.0226737260818481 48.0
0 0.300586372613907 49.0
0.082316863142 0.0372915267944336 50.0
1.6393939244 1.85828340053558 51.0
0 0.0438672304153442 52.0
0 -0.0520028471946716 53.0
0 -0.136190414428711 54.0
0 -0.232971906661987 55.0
0 -0.113210439682007 56.0
0 0.28616726398468 57.0
0 0.222654819488525 58.0
0 0.0576602816581726 59.0
0 -0.148534297943115 60.0
0 0.0389912128448486 61.0
0 0.125586807727814 62.0
263.81034548 248.975387573242 0.0
0 -0.350197434425354 1.0
0 0.0210232138633728 2.0
0 0.22202479839325 3.0
0 -0.301644086837769 4.0
0 0.455989003181458 5.0
0 0.0661587119102478 6.0
0 0.365104585886002 7.0
0 0.434707820415497 8.0
0 0.271407544612885 9.0
0 0.417211174964905 10.0
0 -0.396935701370239 11.0
0 -0.183341443538666 12.0
73.294769378 72.5963516235352 13.0
34.665665358 26.8139762878418 14.0
0 -0.0813243389129639 15.0
0 0.113774120807648 16.0
33.22317322 33.9538536071777 17.0
0 0.0215397477149963 18.0
0 0.0195075869560242 19.0
0 0.0327227115631104 20.0
0 0.206326961517334 21.0
0 0.112213134765625 22.0
27.7619274 28.2343807220459 23.0
0 0.0903748273849487 24.0
20.552521613 0.77549946308136 25.0
0 -0.019199550151825 26.0
0 -0.370238304138184 27.0
0 0.203114688396454 28.0
0 0.0843651294708252 29.0
0 0.0251168012619019 30.0
0 -0.295542120933533 31.0
0 -0.11289644241333 32.0
0 0.0265480875968933 33.0
0 0.480936467647552 34.0
0 0.240001320838928 35.0
0 0.37076610326767 36.0
0 -0.0119485855102539 37.0
0 0.0491248965263367 38.0
0 -0.157952785491943 39.0
0 0.0271490812301636 40.0
0 -0.230666756629944 41.0
0 0.0740728378295898 42.0
298.56473116 296.197814941406 43.0
0 0.11917108297348 44.0
0 0.0338745713233948 45.0
0 0.193045079708099 46.0
0 -0.0139052867889404 47.0
0 0.191191732883453 48.0
0.035281019878 0.375199019908905 49.0
0.068100821363 0.0256189107894897 50.0
1.8890982422 1.85943531990051 51.0
0 0.285678029060364 52.0
0 -0.259180068969727 53.0
0 -0.00437641143798828 54.0
0 0.0681740641593933 55.0
0 0.315009593963623 56.0
0 0.251867830753326 57.0
0 0.36228409409523 58.0
0 0.185945570468903 59.0
0 -0.143321514129639 60.0
0 0.0569568276405334 61.0
0 0.180390596389771 62.0
228.55439973 252.297973632812 0.0
0 0.0187582969665527 1.0
0 -0.0586711168289185 2.0
0 -0.0771154165267944 3.0
0 -0.342079877853394 4.0
0 -0.0431218147277832 5.0
0 0.164710760116577 6.0
0 -0.0293468832969666 7.0
0 0.187620639801025 8.0
0 -0.0352176427841187 9.0
0 -0.0679920315742493 10.0
0 -0.164815664291382 11.0
0 -0.184837698936462 12.0
65.119129974 64.4168319702148 13.0
29.687186289 34.9943161010742 14.0
0 0.034915566444397 15.0
0 0.0060771107673645 16.0
33.280789397 33.2249641418457 17.0
0 0.0841400623321533 18.0
0 -0.0277827978134155 19.0
0 0.169248878955841 20.0
0 -0.0209338665008545 21.0
0 0.2464839220047 22.0
36.42474548 37.4507942199707 23.0
0 0.169459044933319 24.0
0 1.24124836921692 25.0
0 -0.0155285596847534 26.0
0 0.0628654360771179 27.0
0 0.384627372026443 28.0
0 -0.0394406318664551 29.0
0 -0.0780975818634033 30.0
0 -0.15769898891449 31.0
0 0.0445332527160645 32.0
0 0.131729483604431 33.0
0 0.195867538452148 34.0
0 -0.0700435638427734 35.0
0 0.2125403881073 36.0
0 0.0436057448387146 37.0
0 0.25935971736908 38.0
0 -0.257950782775879 39.0
0 0.0303863286972046 40.0
0 -0.201909422874451 41.0
0 0.0796245336532593 42.0
298.18324209 297.654541015625 43.0
0 0.00187891721725464 44.0
0 0.146046042442322 45.0
0 0.0657244324684143 46.0
0 0.0986062288284302 47.0
0 0.380112051963806 48.0
0 0.443467527627945 49.0
0.073575693848 0.0423405766487122 50.0
2.2216393747 1.86302924156189 51.0
0 0.0198663473129272 52.0
0 -0.0116034150123596 53.0
0 -0.14641410112381 54.0
0 -0.162938117980957 55.0
0 -0.0835065841674805 56.0
0 0.147389471530914 57.0
0 0.284428358078003 58.0
0 0.181682527065277 59.0
0 -0.191219985485077 60.0
0 0.0570685863494873 61.0
0 0.168771386146545 62.0
222.34075331 254.855346679688 0.0
0 -0.266197681427002 1.0
0 -0.199630856513977 2.0
0 0.344329714775085 3.0
0 -0.265297770500183 4.0
0 -0.120693683624268 5.0
0 -0.0372822880744934 6.0
0 -0.229730844497681 7.0
0 -0.16343879699707 8.0
0 -0.129412770271301 9.0
0 -0.208978474140167 10.0
0 0.0280754566192627 11.0
0 -0.0478418469429016 12.0
68.797360723 69.6438293457031 13.0
40.570969191 31.3603324890137 14.0
0 -0.121442914009094 15.0
0 0.416176617145538 16.0
38.213385301 38.271354675293 17.0
0 0.191789746284485 18.0
0 -0.0521336793899536 19.0
0 0.0794690847396851 20.0
0 0.0318653583526611 21.0
0 0.258530735969543 22.0
36.283005157 35.1579475402832 23.0
0 0.108758866786957 24.0
0 0.568650603294373 25.0
0 0.0136078596115112 26.0
0 0.00549626350402832 27.0
0 0.877811849117279 28.0
0 0.0617393851280212 29.0
0 0.00522106885910034 30.0
0 -0.187288284301758 31.0
0 -0.11605703830719 32.0
0 -0.173321604728699 33.0
0 0.0381980538368225 34.0
0 -0.101867914199829 35.0
0 0.592555522918701 36.0
0 0.170925915241241 37.0
0 0.166303873062134 38.0
0 -0.199509739875793 39.0
0 0.791556477546692 40.0
0 -0.114185929298401 41.0
0 0.0950146913528442 42.0
304.72779736 297.953460693359 43.0
0 0.297677278518677 44.0
0 0.0261297225952148 45.0
0 0.0294580459594727 46.0
0 0.189629971981049 47.0
0 0.135416209697723 48.0
0 0.370918542146683 49.0
0.086192040543 0.04441237449646 50.0
1.5608557562 1.85240197181702 51.0
0 0.0839828252792358 52.0
0 -0.142182767391205 53.0
0 -0.130971312522888 54.0
0 0.294981837272644 55.0
0 0.349921524524689 56.0
0 -0.0706555247306824 57.0
0 0.0680786967277527 58.0
0 0.151954054832458 59.0
0 -0.17623633146286 60.0
0 0.0549406409263611 61.0
0 0.0837621688842773 62.0
264.82383215 252.636184692383 0.0
0 0.0177298784255981 1.0
0 -0.0356932878494263 2.0
0 0.0491686463356018 3.0
0 0.0597051382064819 4.0
0 0.314912736415863 5.0
0 0.511586666107178 6.0
0 0.422567188739777 7.0
0 -0.540593147277832 8.0
0 0.0402425527572632 9.0
0 -0.0436282157897949 10.0
0 -0.146126985549927 11.0
0 0.105338156223297 12.0
70.282789219 71.4032669067383 13.0
10.808473721 24.4296741485596 14.0
0 0.338653564453125 15.0
0 0.111088037490845 16.0
42.127497532 35.9399375915527 17.0
0 0.207194924354553 18.0
0 0.0699315667152405 19.0
0 -0.111754536628723 20.0
0 0.417789280414581 21.0
0 -0.226425886154175 22.0
27.470765527 22.9487915039062 23.0
0 0.0804628729820251 24.0
0 0.544169306755066 25.0
0 -0.15096914768219 26.0
0 -0.129818022251129 27.0
0 0.849891841411591 28.0
0 -0.00183337926864624 29.0
0 0.138128519058228 30.0
0 0.169679999351501 31.0
0 0.0537580251693726 32.0
0 -0.0107278823852539 33.0
0 0.412972688674927 34.0
0 0.125784397125244 35.0
0 0.397219747304916 36.0
0 0.0485450029373169 37.0
0 -0.0422602295875549 38.0
0 -0.391853094100952 39.0
0 0.887359797954559 40.0
0 -0.055585503578186 41.0
0 -0.0077970027923584 42.0
293.70366515 295.589996337891 43.0
0 -0.0288407802581787 44.0
0 0.112605214118958 45.0
0 0.327947854995728 46.0
0 0.0800489783287048 47.0
0 -0.288920521736145 48.0
0 0.436777144670486 49.0
0.096470687751 0.0253688097000122 50.0
2.1292153747 1.78358137607574 51.0
0 0.253730118274689 52.0
0 -0.235095739364624 53.0
0 -0.161865532398224 54.0
0 -0.0811662673950195 55.0
0 0.319806665182114 56.0
0 -0.177793622016907 57.0
0 0.204489767551422 58.0
0 0.212061882019043 59.0
0 -0.297844648361206 60.0
0 0.0761868953704834 61.0
0 0.301243007183075 62.0
240.99431336 252.765182495117 0.0
0 -0.163634657859802 1.0
0 -0.019769549369812 2.0
0 0.0264155268669128 3.0
0 -0.278519511222839 4.0
0 0.00922167301177979 5.0
0 0.12324321269989 6.0
0 -0.143122494220734 7.0
0 -0.056915819644928 8.0
0 0.185921192169189 9.0
0 -0.427432060241699 10.0
0 0.149571657180786 11.0
0 -0.0698789954185486 12.0
74.845624911 75.267951965332 13.0
32.84370916 25.5271224975586 14.0
0 0.000300765037536621 15.0
0 -0.202057659626007 16.0
30.540146745 36.2211761474609 17.0
0 0.115076899528503 18.0
0 -0.00814551115036011 19.0
0 0.103480935096741 20.0
0 0.0151983499526978 21.0
0 0.00577515363693237 22.0
38.431047004 39.5987319946289 23.0
0 0.115653336048126 24.0
0 1.59298586845398 25.0
0 -0.110603094100952 26.0
0 0.0347800254821777 27.0
0 0.0440536737442017 28.0
0 -0.0467762351036072 29.0
0 0.0112044811248779 30.0
0 0.0325846672058105 31.0
0 -0.0461023449897766 32.0
0 -0.0158717036247253 33.0
0 -0.0274582505226135 34.0
0 0.385046660900116 35.0
0 0.158809661865234 36.0
0 0.156260311603546 37.0
0 0.123239636421204 38.0
0 -0.121174693107605 39.0
0 0.673479974269867 40.0
0 -0.125131964683533 41.0
0 0.0464781522750854 42.0
302.85766885 299.031433105469 43.0
0 0.17513906955719 44.0
0 0.00107276439666748 45.0
0 0.163506805896759 46.0
0 -0.00316727161407471 47.0
0 0.054422914981842 48.0
0 0.349134027957916 49.0
0.083547671646 0.0348095893859863 50.0
2.0818586082 1.87718558311462 51.0
0 0.187361657619476 52.0
0 0.0333444476127625 53.0
0 0.129421234130859 54.0
0 -0.00205212831497192 55.0
0 0.0367972850799561 56.0
0 0.0897992849349976 57.0
0 0.0535604357719421 58.0
0 0.0434831380844116 59.0
0 -0.0766043066978455 60.0
0 0.0165523290634155 61.0
0 0.138503670692444 62.0
227.73235607 253.008209228516 0.0
0 0.126630187034607 1.0
0 -0.138516545295715 2.0
0 0.266102612018585 3.0
0 -0.407264828681946 4.0
0 -0.0722123384475708 5.0
0 0.108257651329041 6.0
0 -0.210687756538391 7.0
0 0.20951509475708 8.0
0 0.121848344802856 9.0
0 1.29659104347229 10.0
0 -0.279873251914978 11.0
0 -0.197717308998108 12.0
61.260581122 60.8989143371582 13.0
33.386442718 31.5880107879639 14.0
0 -0.0586673021316528 15.0
0 -0.25922703742981 16.0
26.778130921 32.0603942871094 17.0
0 0.00973385572433472 18.0
0 0.00345873832702637 19.0
0 0.161824703216553 20.0
0 -0.0609266757965088 21.0
0 0.778705060482025 22.0
42.401683417 40.9959373474121 23.0
0 0.123559832572937 24.0
0 1.07461333274841 25.0
0 -0.145127654075623 26.0
0 -0.116154551506042 27.0
0 0.504223465919495 28.0
0 -0.0976197123527527 29.0
0 0.0988755226135254 30.0
0 0.138751745223999 31.0
0 0.0712692737579346 32.0
0 -0.0575119256973267 33.0
0 0.138975620269775 34.0
0 0.154998362064362 35.0
0 0.292880237102509 36.0
0 -0.0553585290908813 37.0
0 0.142582893371582 38.0
0 -0.260494589805603 39.0
0 0.720511198043823 40.0
0 -0.0537893176078796 41.0
0 -0.0973360538482666 42.0
297.8409968 297.719268798828 43.0
0 0.0972588062286377 44.0
0 0.197039425373077 45.0
0 0.0848782658576965 46.0
0 0.0456427931785583 47.0
0 0.186511516571045 48.0
0 0.525587260723114 49.0
0.085741025074 0.0167713165283203 50.0
1.8075028649 1.81275129318237 51.0
0 -0.0295557379722595 52.0
0 -0.00439685583114624 53.0
0 0.347387105226517 54.0
0 0.12544310092926 55.0
0 0.034679651260376 56.0
0 0.468929260969162 57.0
0 0.50016725063324 58.0
0 0.289506316184998 59.0
0 -0.212256550788879 60.0
0 0.0772333145141602 61.0
0 0.286956548690796 62.0
252.02164271 254.533645629883 0.0
0 -0.322121262550354 1.0
0 0.0173676013946533 2.0
0 0.279101490974426 3.0
0 -0.0185757875442505 4.0
0 0.167273759841919 5.0
0 0.0634233355522156 6.0
0 0.0347897410392761 7.0
0 0.482281923294067 8.0
0 -0.238227605819702 9.0
0 -0.296928763389587 10.0
0 -0.0882265567779541 11.0
0 -0.086653470993042 12.0
53.911576808 55.707649230957 13.0
46.392411857 41.9326705932617 14.0
0 -0.0365620851516724 15.0
0 0.0144477486610413 16.0
39.476462652 37.7801475524902 17.0
0 0.158822178840637 18.0
0 -0.0584843158721924 19.0
0 0.0454773902893066 20.0
0 0.0600252151489258 21.0
0 0.145954489707947 22.0
30.240960458 28.841381072998 23.0
0 0.0312653183937073 24.0
0 -0.0371100306510925 25.0
0 -0.0549153685569763 26.0
0 -0.08939129114151 27.0
0 -0.146414637565613 28.0
0 -0.140842914581299 29.0
0 -0.0050513744354248 30.0
0 0.180325865745544 31.0
0 0.0733701586723328 32.0
0 0.0982519388198853 33.0
0 0.134788870811462 34.0
0 -0.117062151432037 35.0
0 0.293988436460495 36.0
0 0.112544357776642 37.0
0 0.102398812770844 38.0
0 -0.192326068878174 39.0
0 0.45065900683403 40.0
0 -0.133278787136078 41.0
0 0.00849783420562744 42.0
290.55793085 296.572814941406 43.0
0 0.0847883224487305 44.0
0 0.141268014907837 45.0
0 0.189820885658264 46.0
0 -0.0534837245941162 47.0
0 -0.120681643486023 48.0
0 0.507527768611908 49.0
0.077159891457 -0.0067521333694458 50.0
1.5222842832 1.85801446437836 51.0
0 0.264541268348694 52.0
0 -0.0273189544677734 53.0
0 0.100141763687134 54.0
0 0.192637324333191 55.0
0 0.00659871101379395 56.0
0 0.306840181350708 57.0
0 0.153299570083618 58.0
0 0.125751852989197 59.0
0 0.017377495765686 60.0
0 0.0724331140518188 61.0
0 0.192388117313385 62.0
270.80122478 257.3544921875 0.0
0 -0.306731343269348 1.0
0 -0.074313759803772 2.0
0 0.1316819190979 3.0
0 0.0828614234924316 4.0
0 -0.0902948379516602 5.0
0 0.0083237886428833 6.0
0 0.230736613273621 7.0
0 -0.204848110675812 8.0
0 0.0904151201248169 9.0
0 0.112348437309265 10.0
0 0.0273205041885376 11.0
0 -0.185069024562836 12.0
59.501070188 59.9986534118652 13.0
48.97467817 39.029483795166 14.0
0 0.0127145051956177 15.0
0 -0.135184645652771 16.0
30.661504065 35.3017997741699 17.0
0 0.122933924198151 18.0
0 -0.0872970223426819 19.0
0 0.057847261428833 20.0
0 0.162403583526611 21.0
0 0.185587763786316 22.0
29.212725305 27.4208507537842 23.0
0 0.0975062251091003 24.0
0 0.625470995903015 25.0
0 0.0773459076881409 26.0
0 -0.0970847010612488 27.0
0 0.220547735691071 28.0
0 -0.126751065254211 29.0
0 0.0612320899963379 30.0
0 -0.114494800567627 31.0
0 0.0112506151199341 32.0
0 -0.0263451933860779 33.0
0 0.169038355350494 34.0
0 0.211920142173767 35.0
0 -0.324244499206543 36.0
0 0.175145268440247 37.0
0 -0.0818071365356445 38.0
0 -0.145631432533264 39.0
0 0.391765236854553 40.0
0 -0.0792697668075562 41.0
0 0.0444835424423218 42.0
292.32051973 295.942993164062 43.0
0 0.129459500312805 44.0
0 0.211883902549744 45.0
0 0.208084523677826 46.0
0 0.166032373905182 47.0
0 -0.252372026443481 48.0
0 0.417014509439468 49.0
0.073333910532 0.00201922655105591 50.0
2.0356751961 1.85075330734253 51.0
0 -0.00837349891662598 52.0
0 -0.292781114578247 53.0
0 0.170853734016418 54.0
0 0.174619317054749 55.0
0 -0.124063849449158 56.0
0 -0.05790114402771 57.0
0 0.131663799285889 58.0
0 0.0304337739944458 59.0
0 -0.0759326219558716 60.0
0 0.0390528440475464 61.0
0 0.14461213350296 62.0
231.87285439 251.732330322266 0.0
0 0.677572250366211 1.0
0 -0.120275139808655 2.0
0 0.117508828639984 3.0
0 -0.441563487052917 4.0
0 0.045807957649231 5.0
0 -0.424591660499573 6.0
0 -0.0444546341896057 7.0
0 0.239361643791199 8.0
0 0.000773012638092041 9.0
0 0.60037899017334 10.0
0 -0.174041390419006 11.0
0 -0.137548327445984 12.0
67.909543857 68.8024826049805 13.0
22.837573375 30.961483001709 14.0
0 0.197187423706055 15.0
0 0.170764923095703 16.0
36.520247142 35.9264907836914 17.0
0 0.15048885345459 18.0
0 -0.0594838261604309 19.0
0 0.0188404321670532 20.0
0 0.0980567336082458 21.0
0 0.251955330371857 22.0
32.951719865 30.7804431915283 23.0
0 0.0388280749320984 24.0
0 0.551064610481262 25.0
0 -0.0756456255912781 26.0
0 -0.161455869674683 27.0
0 -0.122170448303223 28.0
0 0.0460507273674011 29.0
0 0.206615447998047 30.0
0 0.00740110874176025 31.0
0 0.0558139681816101 32.0
0 -0.365663647651672 33.0
0 0.0888409018516541 34.0
0 -0.175290763378143 35.0
0 0.190492451190948 36.0
0 0.110534429550171 37.0
0 0.0669918060302734 38.0
0 -0.182579636573792 39.0
0 0.0967388153076172 40.0
0 -0.19102817773819 41.0
0 0.0963494777679443 42.0
305.99982188 295.188018798828 43.0
0 0.145882308483124 44.0
0 0.280436813831329 45.0
0 0.236287653446198 46.0
0 0.0611289143562317 47.0
0 0.333875089883804 48.0
1.2446015156 0.529864907264709 49.0
0.076453735783 0.0487740635871887 50.0
1.8098506405 1.87477719783783 51.0
0 0.0603351593017578 52.0
0 -0.202519834041595 53.0
0 -0.0340654850006104 54.0
0 0.0235642194747925 55.0
0 0.0261064171791077 56.0
0 0.0335211753845215 57.0
0 0.492359340190887 58.0
0 -0.558194160461426 59.0
0 0.282465219497681 60.0
0 0.0594801902770996 61.0
0 0.260901212692261 62.0
249.17938777 254.899673461914 0.0
0 -0.332572221755981 1.0
0 -0.255309343338013 2.0
0 0.0960038900375366 3.0
0 -0.119633734226227 4.0
0 0.358309984207153 5.0
0 0.192662298679352 6.0
0 0.09614098072052 7.0
0 0.45692503452301 8.0
0 0.0536454916000366 9.0
0 0.379834055900574 10.0
0 -0.155751466751099 11.0
0 -0.137162566184998 12.0
61.093667433 59.2786674499512 13.0
37.050920359 37.7789421081543 14.0
0 -0.0675457715988159 15.0
0 -0.24123215675354 16.0
28.45714564 32.8370933532715 17.0
0 0.076322078704834 18.0
0 0.0210778713226318 19.0
0 0.0807733535766602 20.0
0 0.0467897653579712 21.0
0 0.406287163496017 22.0
34.826109181 35.3111610412598 23.0
0 0.129318177700043 24.0
0 1.30950593948364 25.0
0 0.00982213020324707 26.0
0 -0.0980594754219055 27.0
0 0.150535345077515 28.0
0 -0.04311603307724 29.0
0 0.093244194984436 30.0
0 -0.310269117355347 31.0
0 -0.0572859048843384 32.0
0 -0.138436675071716 33.0
0 -0.0410919785499573 34.0
0 -0.00855505466461182 35.0
0 0.211609542369843 36.0
0 0.226591348648071 37.0
0 -0.079963207244873 38.0
0 -0.184859991073608 39.0
0 0.532309651374817 40.0
0 -0.0538830161094666 41.0
0 -0.11313807964325 42.0
288.60034068 297.572021484375 43.0
0 0.0138393640518188 44.0
0 0.0852364301681519 45.0
0 0.098497748374939 46.0
0 -0.00420963764190674 47.0
0 -0.0310081243515015 48.0
0 0.43848529458046 49.0
0.073481875478 0.0299220681190491 50.0
2.1610603202 1.81612157821655 51.0
0 -0.0212044715881348 52.0
0 -0.168223679065704 53.0
0 0.105877101421356 54.0
0 0.321571588516235 55.0
0 0.164897561073303 56.0
0 0.0905603170394897 57.0
0 -0.0202882885932922 58.0
0 0.142135560512543 59.0
0 0.0269237756729126 60.0
0 0.0570261478424072 61.0
0 0.11122852563858 62.0
253.22285958 252.68701171875 0.0
0 0.0874831676483154 1.0
0 0.144623160362244 2.0
0 0.0111747980117798 3.0
0 0.0323226451873779 4.0
0 -0.0304844975471497 5.0
0 0.0807525515556335 6.0
0 0.233393549919128 7.0
0 -0.243399143218994 8.0
0 0.829527616500854 9.0
0 0.634936809539795 10.0
0 -0.158112406730652 11.0
0 -0.132884860038757 12.0
68.006981795 68.3037567138672 13.0
20.445209871 28.1001243591309 14.0
0 0.0125327110290527 15.0
0 0.0821070671081543 16.0
39.691836515 36.847599029541 17.0
0 0.144526660442352 18.0
0 -0.0690352916717529 19.0
0 0.0685316324234009 20.0
0 0.0130723118782043 21.0
0 0.115990161895752 22.0
37.841063406 37.2636451721191 23.0
0 -0.0229045152664185 24.0
0 0.262771427631378 25.0
0 -0.223574995994568 26.0
0 -0.0343180894851685 27.0
0 0.207655191421509 28.0
0 0.0595313906669617 29.0
0 0.00786334276199341 30.0
0 -0.0400751829147339 31.0
0 -0.0660816431045532 32.0
0 -0.287963628768921 33.0
0 0.454377770423889 34.0
0 0.160961449146271 35.0
0 0.570647537708282 36.0
0 0.030156135559082 37.0
0 0.310714781284332 38.0
0 -0.440024495124817 39.0
0 -0.160879731178284 40.0
0 -0.22715961933136 41.0
0 0.13751757144928 42.0
307.09163169 295.950073242188 43.0
0 0.191830992698669 44.0
0 0.154887080192566 45.0
0 0.28480452299118 46.0
0 -0.466930866241455 47.0
0 0.330341726541519 48.0
0 0.578160047531128 49.0
0.071420927419 0.0158066153526306 50.0
1.836011505 1.85998404026031 51.0
0 0.024954080581665 52.0
0 -0.417683362960815 53.0
0 -0.140392184257507 54.0
0 0.205567240715027 55.0
0 0.180859684944153 56.0
0 0.191919147968292 57.0
0 0.408804774284363 58.0
0 0.152263104915619 59.0
0 -0.0950537919998169 60.0
0 0.0174084901809692 61.0
0 0.142991900444031 62.0
268.64522096 253.787033081055 0.0
0 -0.222432374954224 1.0
0 -0.0954870581626892 2.0
0 0.326669186353683 3.0
0 -0.00329440832138062 4.0
0 0.187472820281982 5.0
0 -0.11400705575943 6.0
0 0.263557910919189 7.0
0 -0.234844207763672 8.0
0 0.0699059367179871 9.0
0 0.0290143489837646 10.0
0 0.0672571063041687 11.0
0 -0.178217649459839 12.0
66.672308843 67.4280242919922 13.0
45.606255479 33.1144561767578 14.0
0 -0.0457044839859009 15.0
0 -0.0662656426429749 16.0
34.678730911 35.9882125854492 17.0
0 0.150812864303589 18.0
0 -0.00913548469543457 19.0
0 0.0780119299888611 20.0
0 0.148021578788757 21.0
0 0.0508834719657898 22.0
31.468319249 28.9558181762695 23.0
0 0.0733267068862915 24.0
0 0.548210740089417 25.0
0 -0.00696659088134766 26.0
0 -0.132150411605835 27.0
0 0.326338350772858 28.0
0 -0.00765568017959595 29.0
0 0.0461267828941345 30.0
0 -0.248476982116699 31.0
0 -0.0497401356697083 32.0
0 -0.182021975517273 33.0
0 -0.272607445716858 34.0
0 0.132328271865845 35.0
0 -0.0486260652542114 36.0
0 0.0118056535720825 37.0
0 0.332010328769684 38.0
0 -0.246522307395935 39.0
0 0.342048168182373 40.0
0 -0.237874031066895 41.0
0 0.151036739349365 42.0
294.55961386 296.570434570312 43.0
0 0.138705432415009 44.0
0 0.186334013938904 45.0
0 0.12960022687912 46.0
0 -0.315347671508789 47.0
0 0.312595278024673 48.0
0 0.293359994888306 49.0
0.094246789642 0.0588852167129517 50.0
1.5733816559 1.90246868133545 51.0
0 0.176370918750763 52.0
0 0.277291297912598 53.0
0 -0.179935097694397 54.0
0 -0.14626145362854 55.0
0 0.182548999786377 56.0
0 -0.109126269817352 57.0
0 0.0626808404922485 58.0
0 0.0493135452270508 59.0
0 -0.134161949157715 60.0
0 0.0278705954551697 61.0
0 0.170962572097778 62.0
266.78658714 249.122619628906 0.0
0 -0.231636762619019 1.0
0 0.0897916555404663 2.0
0 0.473092049360275 3.0
0 0.0960368514060974 4.0
0 0.15054976940155 5.0
0 -0.187388241291046 6.0
0 0.388591289520264 7.0
0 -0.0873799324035645 8.0
0 0.338169038295746 9.0
0 0.250059127807617 10.0
0 -1.87920022010803 11.0
0 -0.0488669276237488 12.0
71.1798157 71.3830795288086 13.0
8.9644641545 25.0185585021973 14.0
0 0.00588774681091309 15.0
0 -0.0093156099319458 16.0
39.520555938 38.3854484558105 17.0
0 0.150057077407837 18.0
0 -0.0811945796012878 19.0
0 0.0869901180267334 20.0
0 -0.0821009874343872 21.0
0 0.187664806842804 22.0
38.212851516 36.460994720459 23.0
0 0.0993525385856628 24.0
0 0.58361941576004 25.0
0 -0.0289040803909302 26.0
0 0.151239156723022 27.0
0 0.311037003993988 28.0
0 -0.0294495820999146 29.0
0 0.0273663997650146 30.0
0 0.245872020721436 31.0
0 0.292695194482803 32.0
0 0.113693714141846 33.0
0 0.134233713150024 34.0
0 0.0695872902870178 35.0
0 0.0569864511489868 36.0
0 -0.0917198061943054 37.0
0 -0.0473639965057373 38.0
0 -0.0794228315353394 39.0
0 0.0642829537391663 40.0
0 -0.0412321090698242 41.0
0 0.149815082550049 42.0
292.92229748 293.601715087891 43.0
0 0.112776875495911 44.0
0 0.0118863582611084 45.0
0 0.12281197309494 46.0
0 0.0672241449356079 47.0
0 0.44567745923996 48.0
0 0.359711587429047 49.0
0.080275186023 0.0873501896858215 50.0
1.909064239 1.81043696403503 51.0
0 -0.148289382457733 52.0
0 0.214297652244568 53.0
0 0.0775021314620972 54.0
0 0.0199384689331055 55.0
0 -0.161896228790283 56.0
0 0.0521485805511475 57.0
0 0.204938352108002 58.0
0 -0.0463705062866211 59.0
0 -0.204391121864319 60.0
0 0.171725690364838 61.0
0 0.293684899806976 62.0
261.67836324 253.904632568359 0.0
0 -0.329481363296509 1.0
0 -0.0450839996337891 2.0
0 -0.110532820224762 3.0
0 -0.061758816242218 4.0
0 0.0410091280937195 5.0
0 -0.448379516601562 6.0
0 -0.350273132324219 7.0
0 -0.214685916900635 8.0
0 0.178430438041687 9.0
0 -0.391692042350769 10.0
0 0.648819744586945 11.0
0 -0.0590912103652954 12.0
70.987775366 72.5611343383789 13.0
40.743132975 31.0331611633301 14.0
0 -0.0806460380554199 15.0
0 -0.166574120521545 16.0
26.909981327 33.1138305664062 17.0
0 0.0492289662361145 18.0
0 -0.033444881439209 19.0
0 0.13436758518219 20.0
0 -0.0723656415939331 21.0
0 0.0493661761283875 22.0
40.142785367 38.3327217102051 23.0
0 0.0651013255119324 24.0
0 0.563423931598663 25.0
0 -0.0475400686264038 26.0
0 -0.00373274087905884 27.0
0 0.12284255027771 28.0
0 -0.0953258275985718 29.0
0 0.132679760456085 30.0
0 -0.21189558506012 31.0
0 -0.145181119441986 32.0
0 -0.00292009115219116 33.0
0 0.124167382717133 34.0
0 -0.101915061473846 35.0
0 0.0717921257019043 36.0
0 0.0094681978225708 37.0
0 0.0845224857330322 38.0
0 -0.212550044059753 39.0
0 0.285377562046051 40.0
0 -0.0899966955184937 41.0
0 0.029481053352356 42.0
289.19542161 294.738494873047 43.0
0 0.37941038608551 44.0
0 -0.0442512631416321 45.0
0 0.164821863174438 46.0
0 -0.191409349441528 47.0
0 0.311859667301178 48.0
0 0.367641746997833 49.0
0.095731972915 0.0523205399513245 50.0
2.1552807351 1.8878161907196 51.0
0 0.103477716445923 52.0
0 0.124068558216095 53.0
0 0.238598585128784 54.0
0 0.205914556980133 55.0
0 0.134702265262604 56.0
0 0.118330597877502 57.0
0 0.239369332790375 58.0
0 0.264089345932007 59.0
0 -0.169018268585205 60.0
0 0.0919002294540405 61.0
0 0.327623903751373 62.0
228.7724457 253.644805908203 0.0
0 -0.0516242980957031 1.0
0 -0.0980331301689148 2.0
0 0.0969666838645935 3.0
0 -0.326335906982422 4.0
0 -0.26541805267334 5.0
0 0.0536124706268311 6.0
0 -0.127156972885132 7.0
0 -0.0752191543579102 8.0
0 0.247100234031677 9.0
0 0.224738955497742 10.0
0 0.700218796730042 11.0
0 -0.189204335212708 12.0
44.189536873 43.597225189209 13.0
45.346981184 46.7917213439941 14.0
0 0.00584113597869873 15.0
0 -0.0547153949737549 16.0
33.125635603 33.112434387207 17.0
0 0.078931450843811 18.0
0 -0.0746855735778809 19.0
0 0.240129709243774 20.0
0 -0.1384437084198 21.0
0 0.66548079252243 22.0
37.031766391 35.8536567687988 23.0
0 0.0899027585983276 24.0
0 0.449412643909454 25.0
0 -0.187770843505859 26.0
0 -0.122477829456329 27.0
0 0.703576266765594 28.0
0 0.0540801286697388 29.0
0 -0.0471847057342529 30.0
0 0.143649101257324 31.0
0 0.225216448307037 32.0
0 0.251253306865692 33.0
0 -0.0662954449653625 34.0
0 -0.0731578469276428 35.0
0 0.226330280303955 36.0
0 -0.0464967489242554 37.0
0 -0.0543950796127319 38.0
0 -0.293147802352905 39.0
0 0.488882899284363 40.0
0 -0.163232147693634 41.0
0 0.184828281402588 42.0
283.51323095 294.164154052734 43.0
0 0.199386239051819 44.0
0 -0.102298617362976 45.0
0 0.0371589064598083 46.0
0 0.475095927715302 47.0
0 -1.02732336521149 48.0
0 0.610992312431335 49.0
0.070220588432 -0.0414628982543945 50.0
2.0563324054 1.79339480400085 51.0
0 0.028566837310791 52.0
0 -0.413379907608032 53.0
0 -0.162617146968842 54.0
0 0.0137592554092407 55.0
0 0.0684969425201416 56.0
0 -0.260724186897278 57.0
0 0.0225456357002258 58.0
0 0.197169184684753 59.0
0 -0.134641230106354 60.0
0 0.0923565626144409 61.0
0 0.145114958286285 62.0
280.53253657 253.110061645508 0.0
0 -0.333043217658997 1.0
0 0.103707075119019 2.0
0 0.236014306545258 3.0
0 0.03892982006073 4.0
0 0.395985901355743 5.0
0 0.428819060325623 6.0
0 0.496176183223724 7.0
0 -0.113634049892426 8.0
0 -0.30636191368103 9.0
0 0.364642411470413 10.0
0 -0.136033356189728 11.0
0 -0.153224050998688 12.0
73.231345313 73.2123870849609 13.0
12.805478206 22.592378616333 14.0
0 -0.0847388505935669 15.0
0 -0.265625238418579 16.0
26.260972395 31.4767818450928 17.0
0 0.0388592481613159 18.0
0 -0.00760114192962646 19.0
0 0.195153295993805 20.0
0 -0.0278787016868591 21.0
0 0.271260380744934 22.0
39.075018611 38.0171165466309 23.0
0 0.0357198715209961 24.0
0 0.908302426338196 25.0
0 0.00515931844711304 26.0
0 -0.115759253501892 27.0
0 0.141068994998932 28.0
0 -0.0908232927322388 29.0
0 0.128865957260132 30.0
0 0.221142649650574 31.0
0 0.18800675868988 32.0
0 0.13777232170105 33.0
0 0.354493856430054 34.0
0 0.164648950099945 35.0
0 -0.496187210083008 36.0
0 0.00177913904190063 37.0
0 -0.0235686302185059 38.0
0 -0.0391005277633667 39.0
0 0.40661895275116 40.0
0 -0.137000381946564 41.0
0 0.114294648170471 42.0
289.02295542 294.158996582031 43.0
0 0.194444060325623 44.0
0 -0.0432510375976562 45.0
0 0.252342104911804 46.0
0 -0.245125532150269 47.0
0 -0.00894391536712646 48.0
0 0.531609773635864 49.0
0.090703303809 -0.0203448534011841 50.0
2.2094275485 1.87526392936707 51.0
0 0.254939794540405 52.0
0 -0.196436107158661 53.0
0 -0.0614581108093262 54.0
0 0.225485503673553 55.0
0 0.124257802963257 56.0
0 0.121990025043488 57.0
0 0.0834017395973206 58.0
0 0.174676418304443 59.0
0 -0.00708425045013428 60.0
0 0.0366436839103699 61.0
0 0.0664380788803101 62.0
244.6339829 251.883331298828 0.0
0 0.0307594537734985 1.0
0 0.0119115114212036 2.0
0 0.275010168552399 3.0
0 -0.227510452270508 4.0
0 -0.270591974258423 5.0
0 0.0980052351951599 6.0
0 0.0717645883560181 7.0
0 -0.151811361312866 8.0
0 0.22171276807785 9.0
0 0.185829401016235 10.0
0 -0.232108950614929 11.0
0 -0.0378244519233704 12.0
71.37361797 72.9666137695312 13.0
20.437904387 23.9810676574707 14.0
0 -0.00946438312530518 15.0
0 -0.0861282348632812 16.0
32.822877275 32.6725883483887 17.0
0 0.0730447173118591 18.0
0 -0.062883734703064 19.0
0 0.0557239651679993 20.0
0 0.285226285457611 21.0
0 -0.0258994102478027 22.0
27.43871654 24.5284938812256 23.0
0 0.0813146829605103 24.0
0 0.670260190963745 25.0
0 -0.222977995872498 26.0
0 -0.198039829730988 27.0
0 0.628513097763062 28.0
0 0.0480509400367737 29.0
0 0.00498467683792114 30.0
0 -0.10242760181427 31.0
0 0.0696824789047241 32.0
0 0.0544893741607666 33.0
0 -0.00391554832458496 34.0
0 -0.117428660392761 35.0
0 0.0119505524635315 36.0
0 0.0130690932273865 37.0
0 0.074670135974884 38.0
0 -0.140128254890442 39.0
0 0.492275297641754 40.0
0 -0.142583847045898 41.0
0 0.0402752161026001 42.0
288.4585863 296.256164550781 43.0
0 0.172335624694824 44.0
0 0.256576895713806 45.0
0 0.102746605873108 46.0
0 -0.175677537918091 47.0
0 -0.122357666492462 48.0
0 0.512895345687866 49.0
0.089233224846 0.00383323431015015 50.0
2.0497031 1.83357870578766 51.0
0 0.0460078120231628 52.0
0 -0.163812279701233 53.0
0 -0.125383198261261 54.0
0 -0.213626027107239 55.0
0 0.0469592213630676 56.0
0 0.0192443132400513 57.0
0 -0.0638530850410461 58.0
0 -0.0610373616218567 59.0
0 -0.0961158275604248 60.0
0 0.113105952739716 61.0
0 0.181814193725586 62.0
242.51087869 254.424087524414 0.0
0 -0.226346015930176 1.0
0 0.03218674659729 2.0
0 -0.245945692062378 3.0
0 -0.13252866268158 4.0
0 0.201779365539551 5.0
0 -0.125642418861389 6.0
0 -0.202702641487122 7.0
0 -0.024099588394165 8.0
0 -0.0721361637115479 9.0
0 0.378067404031754 10.0
0 -0.105711579322815 11.0
0 -0.244945287704468 12.0
52.080455426 47.7837028503418 13.0
52.775834398 44.1420593261719 14.0
0 -0.0560365915298462 15.0
0 -0.124212622642517 16.0
24.857065148 32.0477485656738 17.0
0 0.0342516899108887 18.0
0 0.159656286239624 19.0
0 0.181353032588959 20.0
0 0.129178524017334 21.0
0 -0.136339902877808 22.0
43.119778152 39.9093933105469 23.0
0 0.000246286392211914 24.0
0 0.184678673744202 25.0
0 0.0160439610481262 26.0
0 0.0188025236129761 27.0
0 0.29994010925293 28.0
0 -0.0285436511039734 29.0
0 0.0720752477645874 30.0
0 -0.202813863754272 31.0
0 0.0834529399871826 32.0
0 -0.110520124435425 33.0
0 0.472905933856964 34.0
0 0.0801175832748413 35.0
0 0.461093217134476 36.0
0 -0.0636679530143738 37.0
0 0.611023962497711 38.0
0 -0.742951154708862 39.0
0 -0.775732636451721 40.0
0 -0.243183493614197 41.0
0 0.0498729944229126 42.0
292.22120951 295.583221435547 43.0
0 0.0608353614807129 44.0
0 0.119801461696625 45.0
0 0.029076874256134 46.0
0 0.113298058509827 47.0
0 -0.639650821685791 48.0
0 0.3466756939888 49.0
0.075099622862 0.0197837352752686 50.0
2.4670316479 1.86327111721039 51.0
0 -0.0287041664123535 52.0
0 0.186095535755157 53.0
0 0.00971907377243042 54.0
0 0.0975914001464844 55.0
0 0.273355007171631 56.0
0 0.0573453903198242 57.0
0 0.557941794395447 58.0
0 0.370100229978561 59.0
0 -0.0317631959915161 60.0
0 0.0591835379600525 61.0
0 0.0886378288269043 62.0
239.32061051 251.828140258789 0.0
0 -0.118152260780334 1.0
0 -0.110678911209106 2.0
0 0.101828992366791 3.0
0 -0.295838117599487 4.0
0 -0.215274572372437 5.0
0 0.0563194155693054 6.0
0 0.152548730373383 7.0
0 -0.0856592059135437 8.0
0 0.138737499713898 9.0
0 0.378217071294785 10.0
0 -0.155765056610107 11.0
0 -0.0394939184188843 12.0
65.232969204 66.1841506958008 13.0
31.66158515 29.8833713531494 14.0
0 -0.13361132144928 15.0
0 0.0853691101074219 16.0
39.737265476 37.2579574584961 17.0
0 0.191502571105957 18.0
0 -0.0393210649490356 19.0
0 0.102696657180786 20.0
0 -0.0581684112548828 21.0
0 0.186573386192322 22.0
36.526522588 38.0256423950195 23.0
0 0.067212700843811 24.0
0 0.632956504821777 25.0
0 0.0198342800140381 26.0
0 -0.124812841415405 27.0
0 -0.170524716377258 28.0
0 -0.0733566880226135 29.0
0 0.0765084028244019 30.0
0 -0.0419173240661621 31.0
0 0.0588650107383728 32.0
0 0.0376938581466675 33.0
0 0.125006079673767 34.0
0 -0.0862880349159241 35.0
0 0.0975939035415649 36.0
0 -0.0755479335784912 37.0
0 0.0972849130630493 38.0
0 -0.120746254920959 39.0
0 0.306006580591202 40.0
0 -0.0591886043548584 41.0
0 0.0365183353424072 42.0
287.74965554 296.607360839844 43.0
0 0.137213587760925 44.0
0 0.0563790798187256 45.0
0 0.204409122467041 46.0
0 0.00653350353240967 47.0
0 -0.281744122505188 48.0
0 0.532213687896729 49.0
0.087298901602 0.00184124708175659 50.0
1.9383259302 1.8158175945282 51.0
0 0.07642662525177 52.0
0 -0.073367178440094 53.0
0 -0.0704606771469116 54.0
0 -0.0191704034805298 55.0
0 0.108506977558136 56.0
0 0.352513313293457 57.0
0 0.360722839832306 58.0
0 0.310718148946762 59.0
0 -0.184047222137451 60.0
0 0.0719858407974243 61.0
0 0.171938538551331 62.0
242.84022059 252.498489379883 0.0
0 -0.0842059850692749 1.0
0 -0.00400537252426147 2.0
0 -0.0468310117721558 3.0
0 -0.312090396881104 4.0
0 0.0762892961502075 5.0
0 0.215697288513184 6.0
0 0.0179000496864319 7.0
0 0.224440693855286 8.0
0 -0.0206223726272583 9.0
0 0.108897566795349 10.0
0 0.0670963525772095 11.0
0 0.000206649303436279 12.0
59.076364698 60.1306991577148 13.0
45.733434644 39.2320861816406 14.0
0 -0.000766634941101074 15.0
0 -0.105640888214111 16.0
29.589494977 34.4176788330078 17.0
0 0.0771548748016357 18.0
0 -0.0751652717590332 19.0
0 0.12737500667572 20.0
0 0.0675821304321289 21.0
0 0.181853890419006 22.0
38.333210999 36.9470748901367 23.0
0 0.146714985370636 24.0
0 0.550434589385986 25.0
0 -0.023323655128479 26.0
0 0.0682141780853271 27.0
0 0.424214631319046 28.0
0 -0.0425283908843994 29.0
0 0.00832724571228027 30.0
0 0.0992926359176636 31.0
0 0.0877770185470581 32.0
0 0.154065191745758 33.0
0 0.398460388183594 34.0
0 0.253428936004639 35.0
0 0.143718838691711 36.0
0 -0.0805714726448059 37.0
0 0.560399174690247 38.0
0 -0.345954179763794 39.0
0 -0.479399800300598 40.0
0 -0.217414975166321 41.0
0 -0.0895814895629883 42.0
298.16822197 296.253143310547 43.0
0 0.0949737429618835 44.0
0 0.175702393054962 45.0
0 0.161589026451111 46.0
0 0.00110876560211182 47.0
0 0.314700663089752 48.0
0 0.477153867483139 49.0
0.094011662225 0.0766432285308838 50.0
1.5220511172 1.87050318717957 51.0
0 0.0548052191734314 52.0
0 -0.170064687728882 53.0
0 -0.0304294228553772 54.0
0 0.00437319278717041 55.0
0 -0.0163554549217224 56.0
0 0.223934292793274 57.0
0 0.146738111972809 58.0
0 0.237176179885864 59.0
0 -0.120363056659698 60.0
0 0.163594603538513 61.0
0 0.303800225257874 62.0
248.52516911 247.670806884766 0.0
0 -0.449303269386292 1.0
0 0.131887435913086 2.0
0 -0.18580836057663 3.0
0 -0.048723578453064 4.0
0 -0.11827951669693 5.0
0 0.229482352733612 6.0
0 0.205892860889435 7.0
0 0.174971342086792 8.0
0 -0.120727062225342 9.0
0 0.344600409269333 10.0
0 0.571401715278625 11.0
0 -0.0901573896408081 12.0
50.906049038 50.845401763916 13.0
37.72052794 36.5743942260742 14.0
0 -0.317852377891541 15.0
0 -0.10638153553009 16.0
37.090920197 37.8897743225098 17.0
0 0.177842020988464 18.0
0 0.0455253124237061 19.0
0 -0.0366355180740356 20.0
0 0.130751669406891 21.0
0 0.0692384243011475 22.0
29.270662451 26.3460140228271 23.0
0 0.0473644733428955 24.0
18.383009797 0.292167127132416 25.0
0 -0.0443794131278992 26.0
0 -0.0478246808052063 27.0
0 0.452906012535095 28.0
0 0.0599991083145142 29.0
0 -0.183244585990906 30.0
0 0.0623451471328735 31.0
0 0.0659860372543335 32.0
0 -0.00471115112304688 33.0
0 0.57419490814209 34.0
0 0.354158967733383 35.0
0 -0.568912982940674 36.0
0 0.0720009207725525 37.0
0 0.120872855186462 38.0
0 -0.20352566242218 39.0
0 0.632036030292511 40.0
0 -0.157861530780792 41.0
0 0.0685139894485474 42.0
302.59218598 294.129943847656 43.0
0 0.135902047157288 44.0
0 -0.0647658705711365 45.0
0 0.113552927970886 46.0
0 0.589198112487793 47.0
0 -0.0412387847900391 48.0
0 0.469079673290253 49.0
0.065175261791 0.0259205102920532 50.0
1.9167984978 1.77143347263336 51.0
0 -0.151489913463593 52.0
0 0.0331224799156189 53.0
0 0.0430786609649658 54.0
0 -0.141712129116058 55.0
0 -0.0637091398239136 56.0
0 0.573860108852386 57.0
0 0.31791964173317 58.0
0 0.0344703793525696 59.0
0 -0.241451978683472 60.0
0 0.0274631977081299 61.0
0 0.245951771736145 62.0
241.76568572 252.142807006836 0.0
0 -0.0490632057189941 1.0
0 0.0648206472396851 2.0
0 0.128022372722626 3.0
0 -0.236286044120789 4.0
0 0.0775684714317322 5.0
0 0.190962970256805 6.0
0 0.126354396343231 7.0
0 0.209780216217041 8.0
0 -0.290074110031128 9.0
0 -0.214972734451294 10.0
0 -0.106303751468658 11.0
0 0.0426122546195984 12.0
58.868076311 59.3951034545898 13.0
46.200559404 38.5564994812012 14.0
0 -0.132280349731445 15.0
0 0.104821562767029 16.0
38.372324709 35.4886283874512 17.0
0 0.117499351501465 18.0
0 -0.0492781400680542 19.0
0 0.1025390625 20.0
0 -0.0155330896377563 21.0
0 0.0391371846199036 22.0
36.076071049 37.7839813232422 23.0
0 0.0611979961395264 24.0
0 0.481000959873199 25.0
0 -0.0125699043273926 26.0
0 -0.057831883430481 27.0
0 0.403500854969025 28.0
0 -0.0386580228805542 29.0
0 0.0371817946434021 30.0
0 -0.0377572774887085 31.0
0 0.0956369042396545 32.0
0 -0.17041426897049 33.0
0 -0.224939584732056 34.0
0 0.173516571521759 35.0
0 -0.0796012282371521 36.0
0 0.0508893132209778 37.0
0 0.145614504814148 38.0
0 -0.163115978240967 39.0
0 0.556899964809418 40.0
0 -0.0812535285949707 41.0
0 0.0214883089065552 42.0
294.10897261 297.087188720703 43.0
0 0.221386075019836 44.0
0 0.0758270621299744 45.0
0 0.213999927043915 46.0
0 -0.0581457614898682 47.0
0 -0.468406319618225 48.0
0 0.544929087162018 49.0
0.074706904832 -0.00981181859970093 50.0
2.2516411669 1.81870818138123 51.0
0 0.0249295830726624 52.0
0 -0.18204265832901 53.0
0 -0.133952379226685 54.0
0 0.00461173057556152 55.0
0 0.11812949180603 56.0
0 -0.11075222492218 57.0
0 0.191764891147614 58.0
0 0.268767595291138 59.0
0 -0.182273507118225 60.0
0 0.103510677814484 61.0
0 0.282558143138885 62.0
223.17712182 251.864532470703 0.0
0 -0.323843598365784 1.0
0 -0.245838642120361 2.0
0 0.270579695701599 3.0
0 -0.158261299133301 4.0
0 -0.26066517829895 5.0
0 -0.157722651958466 6.0
0 -0.132889807224274 7.0
0 0.174705028533936 8.0
0 -0.00121736526489258 9.0
0 0.0496748089790344 10.0
0 -0.047642707824707 11.0
0 -0.097133994102478 12.0
69.239895581 69.4252471923828 13.0
27.325769563 25.1415538787842 14.0
0 -0.161430716514587 15.0
0 -0.0339189171791077 16.0
39.012309105 37.6870880126953 17.0
0 0.137707591056824 18.0
0 -0.0400110483169556 19.0
0 0.0656885504722595 20.0
0 -0.0591207146644592 21.0
0 0.252217173576355 22.0
37.521861794 37.7232360839844 23.0
0 0.0955750942230225 24.0
0 0.901144027709961 25.0
0 -0.0712515711784363 26.0
0 0.0157338976860046 27.0
0 -0.272843599319458 28.0
0 -0.112817645072937 29.0
0 0.0345430374145508 30.0
0 -0.047558069229126 31.0
0 -0.0353614091873169 32.0
0 -0.0010300874710083 33.0
0 0.0439595580101013 34.0
0 -0.00275641679763794 35.0
0 -0.283826947212219 36.0
0 0.0380054116249084 37.0
0 -0.0187345743179321 38.0
0 -0.0919809341430664 39.0
0 -0.00395739078521729 40.0
0 -0.0695341229438782 41.0
0 -0.0148023366928101 42.0
301.03927732 291.919250488281 43.0
0 0.103393614292145 44.0
0 -0.0836576819419861 45.0
0 0.214928209781647 46.0
0 -0.0156804323196411 47.0
0 0.0662463903427124 48.0
0.63362981642 0.445142537355423 49.0
0.100625172 0.0719531774520874 50.0
2.0060420618 1.84780144691467 51.0
0 0.325088620185852 52.0
0 -0.264391183853149 53.0
0 0.249596476554871 54.0
0 0.147973001003265 55.0
0 -0.0811352729797363 56.0
0 0.10183846950531 57.0
0 0.179458856582642 58.0
0 0.209692120552063 59.0
0 -0.301060557365417 60.0
0 0.10766339302063 61.0
0 0.234474301338196 62.0
245.91799878 252.843490600586 0.0
0 -0.237169742584229 1.0
0 -0.00480633974075317 2.0
0 0.247649312019348 3.0
0 -0.159869492053986 4.0
0 0.075126588344574 5.0
0 0.090278148651123 6.0
0 0.0398233532905579 7.0
0 0.310331076383591 8.0
0 0.0247134566307068 9.0
0 0.0511881709098816 10.0
0 0.022813618183136 11.0
0 -0.12572968006134 12.0
58.516824428 58.6640319824219 13.0
40.202772132 38.2397918701172 14.0
0 -0.030892014503479 15.0
0 -0.0380865931510925 16.0
33.188625311 31.6040534973145 17.0
0 0.106466114521027 18.0
0 -0.0292086601257324 19.0
0 0.370914131402969 20.0
0 -0.0335234999656677 21.0
0 0.107355535030365 22.0
38.565008501 37.7432556152344 23.0
0 0.0109719634056091 24.0
0 0.261232256889343 25.0
0 -0.192918181419373 26.0
0 0.0452674627304077 27.0
0 -0.0434593558311462 28.0
0 -0.0169131755828857 29.0
0 0.0509431958198547 30.0
0 0.0859673023223877 31.0
0 0.17081606388092 32.0
0 -0.238781929016113 33.0
0 0.232911705970764 34.0
0 -0.0306524038314819 35.0
0 0.0754562616348267 36.0
0 -0.0453662872314453 37.0
0 0.226323962211609 38.0
0 -0.207038998603821 39.0
0 0.347440421581268 40.0
0 -0.201131761074066 41.0
0 0.0152323246002197 42.0
292.17769438 297.2080078125 43.0
0 0.0996842384338379 44.0
0 0.0965192914009094 45.0
0 0.0560203790664673 46.0
0 0.0381636023521423 47.0
0 0.133844077587128 48.0
0 0.368189543485641 49.0
0.069654888426 0.0720235705375671 50.0
2.1826425875 1.80926775932312 51.0
0 0.0389493703842163 52.0
0 -0.119892656803131 53.0
0 -0.0733500719070435 54.0
0 0.139742016792297 55.0
0 0.105118036270142 56.0
0 0.0497161746025085 57.0
0 0.175000548362732 58.0
0 0.201896786689758 59.0
0 -0.0588031411170959 60.0
0 0.112790524959564 61.0
0 0.201298713684082 62.0
215.67279661 253.236190795898 0.0
0 -0.139210820198059 1.0
0 -0.15376603603363 2.0
0 0.0753073692321777 3.0
0 -0.376562118530273 4.0
0 -0.307737827301025 5.0
0 0.0205908417701721 6.0
0 -0.242796182632446 7.0
0 -0.0229933261871338 8.0
0 0.0409829020500183 9.0
0 0.175597131252289 10.0
0 0.00571489334106445 11.0
0 -0.167256891727448 12.0
75.495040757 77.2886734008789 13.0
39.319202175 20.8945140838623 14.0
0 -0.130101442337036 15.0
0 0.222499310970306 16.0
39.744508752 37.4673347473145 17.0
0 0.0930472612380981 18.0
0 -0.0416635274887085 19.0
0 0.0837982892990112 20.0
0 -0.0521724820137024 21.0
0 0.946923673152924 22.0
40.030031598 40.1688499450684 23.0
0 0.0856505632400513 24.0
0 0.965243816375732 25.0
0 -0.131448030471802 26.0
0 0.23237818479538 27.0
0 0.0667514801025391 28.0
0 0.175845861434937 29.0
0 0.203731596469879 30.0
0 0.091118335723877 31.0
0 0.104191660881042 32.0
0 0.125683605670929 33.0
0 0.0220552086830139 34.0
0 -0.0739338397979736 35.0
0 0.381206631660461 36.0
0 -0.0637566447257996 37.0
0 0.125980317592621 38.0
0 -0.221390724182129 39.0
0 0.677634179592133 40.0
0 -0.0950087904930115 41.0
0 0.0601279735565186 42.0
288.8494564 296.148803710938 43.0
0 0.0170081853866577 44.0
0 0.134071111679077 45.0
0 0.0815918445587158 46.0
0 0.195213198661804 47.0
0 -0.886098027229309 48.0
0 0.535917162895203 49.0
0.10131861186 -0.0175113677978516 50.0
1.8214781665 1.84217321872711 51.0
0 0.105741441249847 52.0
0 -0.222283720970154 53.0
0 0.137277960777283 54.0
0 0.207493126392365 55.0
0 -0.0573382973670959 56.0
0 0.276674330234528 57.0
0 0.124937057495117 58.0
0 0.245222628116608 59.0
0 -0.0217047929763794 60.0
0 0.0996403694152832 61.0
0 0.185500741004944 62.0
282.62822862 251.125106811523 0.0
0 -0.258976459503174 1.0
0 -0.0101269483566284 2.0
0 0.0434445142745972 3.0
0 0.067788302898407 4.0
0 0.15295946598053 5.0
0 0.209845006465912 6.0
0 0.446400433778763 7.0
0 0.0512505769729614 8.0
0 -0.502719640731812 9.0
0 -0.140653431415558 10.0
0 0.961560130119324 11.0
0 -0.267279505729675 12.0
49.397152635 48.0248146057129 13.0
34.934617835 35.9602470397949 14.0
0 -0.13673996925354 15.0
0 -0.155054271221161 16.0
34.718311237 34.7127876281738 17.0
0 0.078927755355835 18.0
0 0.0137175321578979 19.0
0 0.0302113890647888 20.0
0 0.11142909526825 21.0
0 0.108854949474335 22.0
29.797624182 29.3643608093262 23.0
0 0.0797892212867737 24.0
0 0.878793716430664 25.0
0 -0.221883654594421 26.0
0 -0.0809970498085022 27.0
0 0.0144042372703552 28.0
0 -0.000522494316101074 29.0
0 0.0398761034011841 30.0
0 -0.162845373153687 31.0
0 0.0649605989456177 32.0
0 0.0568661093711853 33.0
0 0.0748207569122314 34.0
0 0.343394726514816 35.0
0 -0.335606575012207 36.0
0 -0.0902259349822998 37.0
0 0.0736621022224426 38.0
0 -0.469366431236267 39.0
0 0.447890520095825 40.0
0 -0.313266634941101 41.0
0 0.00925874710083008 42.0
296.17525446 295.931976318359 43.0
0 0.229082584381104 44.0
0 0.173222839832306 45.0
0 0.0963273048400879 46.0
0 0.0808354020118713 47.0
0 0.344443678855896 48.0
0 0.483485460281372 49.0
0.074654459631 0.0168473124504089 50.0
2.3276120766 1.84420871734619 51.0
0 -0.0420093536376953 52.0
0 -0.139670670032501 53.0
0 0.303366482257843 54.0
0 -0.144160270690918 55.0
0 -0.10613226890564 56.0
0 0.350778698921204 57.0
0 0.373945325613022 58.0
0 -0.048734188079834 59.0
0 -0.0277678370475769 60.0
0 0.114768505096436 61.0
0 0.364280343055725 62.0
248.11337624 254.001190185547 0.0
0 -0.231922268867493 1.0
0 0.0431979298591614 2.0
0 0.423605978488922 3.0
0 -0.0824006795883179 4.0
0 0.0673136711120605 5.0
0 0.0390662550926208 6.0
0 0.0334592461585999 7.0
0 0.221188545227051 8.0
0 0.0863351821899414 9.0
0 -0.00468844175338745 10.0
0 0.00451278686523438 11.0
0 0.0171749591827393 12.0
62.600649148 61.446891784668 13.0
39.057579714 36.5355682373047 14.0
0 -0.100221157073975 15.0
0 0.187968254089355 16.0
39.642478925 36.9413452148438 17.0
0 0.0869234800338745 18.0
0 -0.026530385017395 19.0
0 0.0350469350814819 20.0
0 0.112620711326599 21.0
0 0.129229307174683 22.0
29.294634714 27.8200969696045 23.0
0 -0.0129328966140747 24.0
0 0.0687786340713501 25.0
0 -0.139641582965851 26.0
0 -0.0626616477966309 27.0
0 -0.223734974861145 28.0
0 -0.0278305411338806 29.0
0 -0.003193199634552 30.0
0 -0.0199776887893677 31.0
0 -0.0488467812538147 32.0
0 -0.18102365732193 33.0
0 0.0662218332290649 34.0
0 -0.00966328382492065 35.0
0 -0.0843186378479004 36.0
0 0.213964581489563 37.0
0 -0.230751872062683 38.0
0 -0.0243077278137207 39.0
0 0.571915507316589 40.0
0 -0.180532038211823 41.0
0 0.127367377281189 42.0
300.07997214 298.223999023438 43.0
0 0.155104637145996 44.0
0 0.171486020088196 45.0
0 0.0891714096069336 46.0
0 0.184217870235443 47.0
0 0.292884707450867 48.0
1.7667891302 0.417225629091263 49.0
0.086356501254 0.0627612471580505 50.0
1.6979166454 1.86932396888733 51.0
0 0.101014137268066 52.0
0 -0.18651294708252 53.0
0 0.0684623718261719 54.0
0 0.196610391139984 55.0
0 0.0360686779022217 56.0
0 0.224094867706299 57.0
0 0.00111168622970581 58.0
0 -0.0910635590553284 59.0
0 -0.0771183967590332 60.0
0 0.0183835029602051 61.0
0 -0.00647914409637451 62.0
267.33503675 253.037216186523 0.0
0 -0.0303744077682495 1.0
0 -0.0282376408576965 2.0
0 0.427471905946732 3.0
0 0.0146735906600952 4.0
0 0.331052958965302 5.0
0 -0.0608391761779785 6.0
0 0.283775627613068 7.0
0 -0.301002025604248 8.0
0 0.114645957946777 9.0
0 0.187552630901337 10.0
0 0.0211741924285889 11.0
0 -0.0249477624893188 12.0
57.528379983 59.6800270080566 13.0
45.357883924 37.9787902832031 14.0
0 -0.0236951112747192 15.0
0 0.136954605579376 16.0
38.009591225 35.8510055541992 17.0
0 0.131434261798859 18.0
0 -0.0694469213485718 19.0
0 0.139269232749939 20.0
0 0.0328437089920044 21.0
0 0.285330772399902 22.0
33.413340572 34.1990051269531 23.0
0 0.0919687151908875 24.0
0 0.578413486480713 25.0
0 -0.083253026008606 26.0
0 -0.0761784315109253 27.0
0 -0.0338374972343445 28.0
0 -0.0702176094055176 29.0
0 0.0231677293777466 30.0
0 0.0656284093856812 31.0
0 -0.0975457429885864 32.0
0 -0.0751781463623047 33.0
0 0.120008111000061 34.0
0 -0.103007197380066 35.0
0 0.0410938262939453 36.0
0 0.023962140083313 37.0
0 0.0649456381797791 38.0
0 -0.232325077056885 39.0
0 0.792944312095642 40.0
0 -0.131538569927216 41.0
0 0.0846760272979736 42.0
297.98101369 297.823150634766 43.0
0 0.0561841726303101 44.0
0 0.161542177200317 45.0
0 0.167283177375793 46.0
0 -0.125914573669434 47.0
0 -0.244945168495178 48.0
0 0.548943519592285 49.0
0.098001075448 -0.072411060333252 50.0
2.0491327275 1.91657209396362 51.0
0 0.278675198554993 52.0
0 -0.0954634547233582 53.0
0 -0.226431369781494 54.0
0 -0.169277608394623 55.0
0 0.181076407432556 56.0
0 0.102250695228577 57.0
0 0.298756301403046 58.0
0 0.117697656154633 59.0
0 -0.186049222946167 60.0
0 0.118005514144897 61.0
0 0.41592538356781 62.0
282.1134423 253.937850952148 0.0
0 -0.273738026618958 1.0
0 -0.0409932136535645 2.0
0 0.0138171911239624 3.0
0 0.0901565551757812 4.0
0 0.28077632188797 5.0
0 0.0716818571090698 6.0
0 0.290312647819519 7.0
0 -0.0242756009101868 8.0
0 0.105583012104034 9.0
0 -0.42409360408783 10.0
0 -0.126242220401764 11.0
0 -0.053779661655426 12.0
56.673653393 56.7767601013184 13.0
41.178294141 38.5684280395508 14.0
0 0.10181200504303 15.0
0 -0.0579926371574402 16.0
29.127740126 31.598783493042 17.0
0 0.126996278762817 18.0
0 0.00576090812683105 19.0
0 0.0819226503372192 20.0
0 0.108615875244141 21.0
0 0.213180780410767 22.0
35.140980611 33.7291946411133 23.0
0 0.132852077484131 24.0
0 0.569886565208435 25.0
0 0.00602030754089355 26.0
0 -0.0329452753067017 27.0
0 0.581747889518738 28.0
0 0.0124408006668091 29.0
0 0.0528088808059692 30.0
0 0.063897967338562 31.0
0 0.265866160392761 32.0
0 0.16922253370285 33.0
0 -0.0911074876785278 34.0
0 0.0524424910545349 35.0
0 0.0314298868179321 36.0
0 -0.0368475914001465 37.0
0 0.0949349999427795 38.0
0 -0.353878736495972 39.0
0 0.399179577827454 40.0
0 -0.165584683418274 41.0
0 0.0454868078231812 42.0
288.47798426 297.458923339844 43.0
0 0.117233991622925 44.0
0 0.197300910949707 45.0
0 0.1434406042099 46.0
0 -0.264449834823608 47.0
0 0.0256178379058838 48.0
0 0.432660222053528 49.0
0.097042098806 0.0355404615402222 50.0
1.6554447004 1.83441686630249 51.0
0 -0.12435108423233 52.0
0 -0.00381553173065186 53.0
0 -0.0128935575485229 54.0
0 0.257509410381317 55.0
0 0.22490519285202 56.0
0 0.555920958518982 57.0
0 0.314462721347809 58.0
0 0.0950263738632202 59.0
0 -0.126113772392273 60.0
0 0.0377349853515625 61.0
0 0.126397550106049 62.0
240.78681227 253.670791625977 0.0
0 -0.206035017967224 1.0
0 0.0360501408576965 2.0
0 0.147090792655945 3.0
0 -0.181460082530975 4.0
0 0.0428991913795471 5.0
0 0.147552251815796 6.0
0 0.036578893661499 7.0
0 0.17566978931427 8.0
0 0.182604193687439 9.0
0 0.126326143741608 10.0
0 -0.0629108548164368 11.0
0 -0.0587408542633057 12.0
75.206888943 75.2185897827148 13.0
20.899782081 23.2920703887939 14.0
0 0.0143277645111084 15.0
0 0.2109335064888 16.0
37.470895907 34.7558898925781 17.0
0 0.15069580078125 18.0
0 -0.0106105804443359 19.0
0 0.0829730033874512 20.0
0 0.0926879048347473 21.0
0 0.118892550468445 22.0
38.561854171 38.3039741516113 23.0
0 0.0853261947631836 24.0
0 1.03347706794739 25.0
0 -0.0133150815963745 26.0
0 0.00774097442626953 27.0
0 0.45546567440033 28.0
0 -0.0619440078735352 29.0
0 0.0691597461700439 30.0
0 0.166563749313354 31.0
0 0.131181716918945 32.0
0 0.123725175857544 33.0
0 0.0130996108055115 34.0
0 -0.0373228788375854 35.0
0 -0.034489631652832 36.0
0 -0.00855958461761475 37.0
0 0.167324244976044 38.0
0 -0.131837606430054 39.0
0 0.463953256607056 40.0
0 -0.171056687831879 41.0
0 0.131126046180725 42.0
289.14350512 299.091247558594 43.0
0 0.0672253370285034 44.0
0 0.0709534287452698 45.0
0 0.174706041812897 46.0
0 0.0279456377029419 47.0
0 0.000554263591766357 48.0
0 0.390860438346863 49.0
0.086084604431 -0.0409188866615295 50.0
2.1870333365 1.91136085987091 51.0
0 -0.0873644948005676 52.0
0 0.0163642168045044 53.0
0 -0.129416763782501 54.0
0 0.0280730724334717 55.0
0 0.116897940635681 56.0
0 0.0993630886077881 57.0
0 0.125183582305908 58.0
0 0.0798081755638123 59.0
0 0.0670186281204224 60.0
0 0.120415568351746 61.0
0 0.192837953567505 62.0
278.34523574 253.606521606445 0.0
0 0.188371181488037 1.0
0 -0.0897554755210876 2.0
0 0.0256559252738953 3.0
0 0.0738604664802551 4.0
0 -0.198996245861053 5.0
0 0.222272753715515 6.0
0 0.437930405139923 7.0
0 -0.209694385528564 8.0
0 0.0951287150382996 9.0
0 -0.228758335113525 10.0
0 -0.0046159029006958 11.0
0 -0.050667405128479 12.0
74.10057323 73.0472793579102 13.0
34.108813467 27.163703918457 14.0
0 -0.08211350440979 15.0
0 0.285440623760223 16.0
36.425239961 35.7685737609863 17.0
0 0.194868624210358 18.0
0 -0.0473513603210449 19.0
0 -0.0111683011054993 20.0
0 0.161337733268738 21.0
0 0.120853245258331 22.0
31.026102014 28.0660228729248 23.0
0 0.023576021194458 24.0
0 0.348545849323273 25.0
0 -0.13696551322937 26.0
0 -0.0683346390724182 27.0
0 0.536724448204041 28.0
0 0.096387505531311 29.0
0 0.0808994174003601 30.0
0 -0.0946767330169678 31.0
0 -0.0874128937721252 32.0
0 -0.492200374603271 33.0
0 -0.0895227193832397 34.0
0 0.279802441596985 35.0
0 0.389960467815399 36.0
0 0.0200669765472412 37.0
0 0.420996010303497 38.0
0 -0.0842182636260986 39.0
0 0.00669956207275391 40.0
0 -0.191634118556976 41.0
0 -0.0721805095672607 42.0
305.83049434 297.8251953125 43.0
0 0.0528429746627808 44.0
0 0.0195075869560242 45.0
0 0.280236005783081 46.0
0 -0.181734025478363 47.0
0 0.184156835079193 48.0
0 0.357599407434464 49.0
0.091873957003 0.0690608620643616 50.0
1.8961682634 1.81545901298523 51.0
0 0.0854423046112061 52.0
0 -0.129168272018433 53.0
0 -0.0559630990028381 54.0
0 0.0297510623931885 55.0
0 0.027996301651001 56.0
0 -0.0148581266403198 57.0
0 0.171329379081726 58.0
0 0.518546998500824 59.0
0 0.0838286876678467 60.0
0 0.0715764164924622 61.0
0 0.213947474956512 62.0
250.48217647 252.315399169922 0.0
0 -0.108136177062988 1.0
0 -0.0484371185302734 2.0
0 0.42575067281723 3.0
0 -0.126288115978241 4.0
0 0.441698282957077 5.0
0 0.284410357475281 6.0
0 0.131184101104736 7.0
0 0.586315155029297 8.0
0 0.139777958393097 9.0
0 0.303342372179031 10.0
0 -0.0903517603874207 11.0
0 -0.0916025042533875 12.0
57.753628372 58.5050086975098 13.0
38.659660508 32.4610939025879 14.0
0 0.0601973533630371 15.0
0 -0.0202332139015198 16.0
34.428022302 31.3162307739258 17.0
0 0.0748076438903809 18.0
0 0.0121334791183472 19.0
0 0.151908397674561 20.0
0 -0.035649836063385 21.0
0 -0.078627347946167 22.0
41.165432125 40.6202812194824 23.0
0 0.0230826735496521 24.0
0 0.629865407943726 25.0
0 -0.220982551574707 26.0
0 0.296033799648285 27.0
0 -0.353918790817261 28.0
0 -0.182202041149139 29.0
0 0.0488113760948181 30.0
0 0.101936340332031 31.0
0 0.121924042701721 32.0
0 0.134236693382263 33.0
0 0.302981793880463 34.0
0 -0.172482967376709 35.0
0 -0.419199585914612 36.0
0 -0.0637763142585754 37.0
0 0.311310231685638 38.0
0 -0.280557632446289 39.0
0 -0.00364542007446289 40.0
0 -0.183088004589081 41.0
0 0.051153302192688 42.0
284.03060757 297.851776123047 43.0
0 0.077881932258606 44.0
0 0.0874407291412354 45.0
0 0.126652538776398 46.0
0 0.0560416579246521 47.0
0 0.415881872177124 48.0
0 0.27814507484436 49.0
0.095986899865 0.0892508029937744 50.0
1.4737200266 1.84823143482208 51.0
0 -0.0271337032318115 52.0
0 0.0264260172843933 53.0
0 -0.0840643048286438 54.0
0 -0.0428537726402283 55.0
0 0.134208559989929 56.0
0 0.125369668006897 57.0
0 0.619620084762573 58.0
0 -0.310949206352234 59.0
0 0.0331542491912842 60.0
0 0.039606511592865 61.0
0 0.08429354429245 62.0
271.68861281 250.781295776367 0.0
0 -0.30479884147644 1.0
0 -0.0468655824661255 2.0
0 -0.0778155326843262 3.0
0 -0.0632921457290649 4.0
0 0.354886472225189 5.0
0 0.375237822532654 6.0
0 0.148837447166443 7.0
0 0.0707939863204956 8.0
0 0.327795207500458 9.0
0 -0.0574597120285034 10.0
0 -0.418783187866211 11.0
0 -0.155225515365601 12.0
59.477389131 58.8764305114746 13.0
48.418345172 36.8115463256836 14.0
0 -0.00850784778594971 15.0
0 -0.10097998380661 16.0
32.315475183 32.9488945007324 17.0
0 0.0590921640396118 18.0
0 -0.0182771682739258 19.0
0 0.0215954780578613 20.0
0 0.289993584156036 21.0
0 -0.0455982089042664 22.0
26.343315095 26.1492080688477 23.0
0 0.0667836666107178 24.0
0 0.719782829284668 25.0
0 0.021446704864502 26.0
0 -0.156157255172729 27.0
0 0.0364997386932373 28.0
0 0.145224332809448 29.0
0 0.0024259090423584 30.0
0 -0.122244954109192 31.0
0 -0.0110903382301331 32.0
0 -0.0283328890800476 33.0
0 0.469552010297775 34.0
0 0.278022706508636 35.0
0 0.300521105527878 36.0
0 -0.0538578629493713 37.0
0 0.131969451904297 38.0
0 -0.0461914539337158 39.0
0 0.573097407817841 40.0
0 -0.0607445240020752 41.0
0 0.0759824514389038 42.0
288.55103347 297.481414794922 43.0
0 0.0726392865180969 44.0
0 0.106939613819122 45.0
0 0.253710865974426 46.0
0 -0.136598229408264 47.0
0 0.323860168457031 48.0
0 0.339846849441528 49.0
0.075545697364 -0.00836193561553955 50.0
2.0645924029 1.88459515571594 51.0
0 -0.0807862877845764 52.0
0 -0.304027438163757 53.0
0 -0.099071741104126 54.0
0 0.159167587757111 55.0
0 0.371220946311951 56.0
0 -0.0398804545402527 57.0
0 -0.207447469234467 58.0
0 -0.0713791847229004 59.0
0 -0.265422821044922 60.0
0 0.134424865245819 61.0
0 0.15754246711731 62.0
273.58160435 252.278854370117 0.0
0 -0.0507564544677734 1.0
0 0.136978507041931 2.0
0 0.32101970911026 3.0
0 0.0326611995697021 4.0
0 0.161997020244598 5.0
0 -0.0671753883361816 6.0
0 0.292511820793152 7.0
0 -0.042291522026062 8.0
0 0.346922039985657 9.0
0 -0.33388888835907 10.0
0 -0.0889290571212769 11.0
0 -0.170307099819183 12.0
71.155582163 73.3294219970703 13.0
37.947552344 23.8039836883545 14.0
0 0.176955223083496 15.0
0 0.0887268781661987 16.0
39.317887367 36.6004371643066 17.0
0 0.176637291908264 18.0
0 -0.00366240739822388 19.0
0 0.0324045419692993 20.0
0 0.0189983248710632 21.0
0 0.183495402336121 22.0
36.02562392 36.9618453979492 23.0
0 0.128607392311096 24.0
0 0.9467933177948 25.0
0 -0.153518795967102 26.0
0 0.00930124521255493 27.0
0 0.24869030714035 28.0
0 -0.00317513942718506 29.0
0 0.0853313207626343 30.0
0 -0.204796552658081 31.0
0 -0.0418274402618408 32.0
0 -0.498939990997314 33.0
0 0.258869826793671 34.0
0 0.362157225608826 35.0
0 -0.284268975257874 36.0
0 0.00424569845199585 37.0
0 0.211396396160126 38.0
0 -0.31456446647644 39.0
0 0.587070226669312 40.0
0 -0.0637677311897278 41.0
0 0.00869035720825195 42.0
294.12687127 295.587951660156 43.0
0 0.0572075843811035 44.0
0 -0.0160790085792542 45.0
0 0.136302411556244 46.0
0 -0.105243563652039 47.0
0 0.343075901269913 48.0
0 0.439726531505585 49.0
0.076329561421 0.020752489566803 50.0
1.9961822968 1.92300987243652 51.0
0 -0.0996283888816833 52.0
0 -0.116590261459351 53.0
0 0.140704751014709 54.0
0 0.124096751213074 55.0
0 0.118452727794647 56.0
0 0.392632186412811 57.0
0 0.301774859428406 58.0
0 0.0357620716094971 59.0
0 -0.188206255435944 60.0
0 0.0327161550521851 61.0
0 0.241646826267242 62.0
242.01835932 256.279571533203 0.0
0 -0.253742694854736 1.0
0 -0.476262211799622 2.0
0 0.224466860294342 3.0
0 -0.441707372665405 4.0
0 0.0319599509239197 5.0
0 0.0506779551506042 6.0
0 0.0540974140167236 7.0
0 0.0226963758468628 8.0
0 0.153689205646515 9.0
0 -0.406381964683533 10.0
0 -0.131463408470154 11.0
0 -0.00776427984237671 12.0
66.657444029 63.937255859375 13.0
39.755328297 31.5810070037842 14.0
0 0.0215275287628174 15.0
0 -0.000953793525695801 16.0
30.77732024 36.0860290527344 17.0
0 0.142624258995056 18.0
0 0.0427870750427246 19.0
0 0.0467036962509155 20.0
0 0.157721340656281 21.0
0 -0.331949234008789 22.0
40.508005282 38.2438316345215 23.0
0 -0.0263347029685974 24.0
0 -0.216436862945557 25.0
0 -0.0676537752151489 26.0
0 0.0406889319419861 27.0
0 0.666147649288177 28.0
0 -0.094300389289856 29.0
0 0.0618810653686523 30.0
0 0.367729902267456 31.0
0 0.0698050856590271 32.0
0 0.129855692386627 33.0
0 0.514399290084839 34.0
0 0.219066619873047 35.0
0 -0.0799640417098999 36.0
0 -0.12414813041687 37.0
0 0.0324096083641052 38.0
0 -0.142386317253113 39.0
0 0.125630617141724 40.0
0 -0.139862596988678 41.0
0 -0.103953123092651 42.0
300.97383752 297.272064208984 43.0
0 0.18443775177002 44.0
0 0.0981916785240173 45.0
0 0.208661794662476 46.0
0 0.135919451713562 47.0
0 -0.397001147270203 48.0
0 0.407670140266418 49.0
0.072223119636 0.0140776038169861 50.0
1.3742987298 1.79664611816406 51.0
0 0.436406552791595 52.0
0 0.27001690864563 53.0
0 -0.0683881640434265 54.0
0 0.0663526058197021 55.0
0 0.361758649349213 56.0
0 -0.0698006749153137 57.0
0 0.522301554679871 58.0
0 0.15753585100174 59.0
0 0.0924500226974487 60.0
0 0.175281286239624 61.0
0 0.297896206378937 62.0
239.98168467 247.460037231445 0.0
0 -0.173704385757446 1.0
0 -0.313291430473328 2.0
0 0.268101394176483 3.0
0 -0.184466779232025 4.0
0 -0.247927904129028 5.0
0 -0.13004344701767 6.0
0 -0.191294610500336 7.0
0 0.0672414302825928 8.0
0 0.187815546989441 9.0
0 0.0837644934654236 10.0
0 0.215627789497375 11.0
0 -0.119209885597229 12.0
50.039153767 47.474437713623 13.0
36.427092006 39.8877639770508 14.0
0 -0.229465007781982 15.0
0 -0.155836582183838 16.0
33.525782324 35.4335479736328 17.0
0 0.146207332611084 18.0
0 -0.0316758155822754 19.0
0 0.0684923529624939 20.0
0 0.275587677955627 21.0
0 -0.234744429588318 22.0
26.085415411 22.7516040802002 23.0
0 0.0775777101516724 24.0
0 0.201781690120697 25.0
0 -0.181424617767334 26.0
0 0.0367113947868347 27.0
0 0.0575359463691711 28.0
0 -0.0243167281150818 29.0
0 0.141808152198792 30.0
0 0.0312042236328125 31.0
0 0.049069881439209 32.0
0 0.203799962997437 33.0
0 0.0724817514419556 34.0
0 0.0446264743804932 35.0
0 -0.374614596366882 36.0
0 -0.00489729642868042 37.0
0 0.143009126186371 38.0
0 -0.452080249786377 39.0
0 -0.148783266544342 40.0
0 -0.158566474914551 41.0
0 -0.157027959823608 42.0
277.49014027 298.328643798828 43.0
0 0.171881198883057 44.0
0 0.046174943447113 45.0
0 0.07488614320755 46.0
0 -0.150588512420654 47.0
0 0.345743000507355 48.0
0.2701626643 0.398715436458588 49.0
0.10474411195 0.0946426391601562 50.0
2.3354762464 1.88268148899078 51.0
0 0.204116821289062 52.0
0 -0.00420272350311279 53.0
0 0.397544384002686 54.0
0 0.0130090713500977 55.0
0 -0.0489857196807861 56.0
0 -0.191751599311829 57.0
0 0.0702927112579346 58.0
0 0.0638574361801147 59.0
0 -0.0428348779678345 60.0
0 0.0126581192016602 61.0
0 0.0596733093261719 62.0
215.52495939 252.875396728516 0.0
0 0.240941524505615 1.0
0 -0.0429296493530273 2.0
0 -0.0222057700157166 3.0
0 -0.27291202545166 4.0
0 -0.154653072357178 5.0
0 0.156504273414612 6.0
0 0.0533734560012817 7.0
0 0.336141973733902 8.0
0 -0.216476321220398 9.0
0 0.186731934547424 10.0
0 -0.409399032592773 11.0
0 -0.103396654129028 12.0
55.794266121 55.4209022521973 13.0
29.13528326 37.814884185791 14.0
0 -0.210015058517456 15.0
0 -0.0369500517845154 16.0
39.255764081 35.2776679992676 17.0
0 0.131560206413269 18.0
0 -0.0213246941566467 19.0
0 0.0971286296844482 20.0
0 -0.0249603986740112 21.0
0 0.0727171301841736 22.0
39.606053873 39.7843742370605 23.0
0 0.134681403636932 24.0
0 0.86510294675827 25.0
0 0.0240198969841003 26.0
0 0.0819136500358582 27.0
0 0.202475368976593 28.0
0 -0.0965124368667603 29.0
0 -0.0282851457595825 30.0
0 0.0542093515396118 31.0
0 -0.0806295275688171 32.0
0 -0.211638808250427 33.0
0 -0.0309150218963623 34.0
0 1.29071652889252 35.0
0 0.521393656730652 36.0
0 0.146022915840149 37.0
0 0.069156289100647 38.0
0 -0.329824447631836 39.0
0 0.478328227996826 40.0
0 -0.152769684791565 41.0
0 -0.00197958946228027 42.0
288.45124051 296.393463134766 43.0
0 0.2004314661026 44.0
0 0.0389963984489441 45.0
0 0.287986278533936 46.0
0 -0.144943594932556 47.0
0 0.385893046855927 48.0
0.47773281517 0.528216481208801 49.0
0.10042677391 0.0853278040885925 50.0
2.2089702451 1.83480501174927 51.0
0 0.127010703086853 52.0
0 0.011416494846344 53.0
0 0.1397305727005 54.0
0 0.112503528594971 55.0
0 -0.165248513221741 56.0
0 -0.0389553904533386 57.0
0 0.276525974273682 58.0
0 0.017602801322937 59.0
0 0.034265398979187 60.0
0 0.0605084300041199 61.0
0 0.233097195625305 62.0
246.20674477 253.81591796875 0.0
0 0.651283383369446 1.0
0 -0.0225802659988403 2.0
0 0.162128865718842 3.0
0 -0.273448348045349 4.0
0 0.133384525775909 5.0
0 0.221985399723053 6.0
0 -0.135258078575134 7.0
0 0.310618191957474 8.0
0 0.348535180091858 9.0
0 -0.282676935195923 10.0
0 -0.00737935304641724 11.0
0 -0.143643498420715 12.0
76.606186672 79.4300537109375 13.0
21.163892638 17.1942234039307 14.0
0 0.817608714103699 15.0
0 0.0543079972267151 16.0
28.177117665 31.1508655548096 17.0
0 0.100224614143372 18.0
0 -0.0326772928237915 19.0
0 0.0874520540237427 20.0
0 0.235751271247864 21.0
0 0.0405005812644958 22.0
29.834672084 28.3897075653076 23.0
0 0.0238267183303833 24.0
0 0.512022256851196 25.0
0 0.035155177116394 26.0
0 -0.196274816989899 27.0
0 -0.641821503639221 28.0
0 -0.0300267934799194 29.0
0 0.0144612789154053 30.0
0 -0.0888364315032959 31.0
0 -0.164718151092529 32.0
0 -0.178869128227234 33.0
0 0.134704411029816 34.0
0 0.374031782150269 35.0
0 0.433353066444397 36.0
0 0.0759841799736023 37.0
0 0.194857954978943 38.0
0 -0.368375420570374 39.0
0 0.568951666355133 40.0
0 -0.0872386693954468 41.0
0 0.00729751586914062 42.0
302.78527312 298.226104736328 43.0
0 0.198077440261841 44.0
0 0.0260589718818665 45.0
0 0.267302811145782 46.0
0 -0.171285510063171 47.0
0 0.323197662830353 48.0
3.5005886028 0.681789755821228 49.0
0.093187626515 0.0562386512756348 50.0
2.0253616759 1.88610768318176 51.0
0 -0.0603126883506775 52.0
0 0.125774145126343 53.0
0 -0.0977317094802856 54.0
0 0.231892704963684 55.0
0 0.331632196903229 56.0
0 0.0943232774734497 57.0
0 0.255599141120911 58.0
0 0.228009283542633 59.0
0 -0.085102915763855 60.0
0 0.108879327774048 61.0
0 0.255412936210632 62.0
249.97135326 255.338607788086 0.0
0 0.297253012657166 1.0
0 0.234523952007294 2.0
0 0.241960465908051 3.0
0 0.00308322906494141 4.0
0 0.17489892244339 5.0
0 0.058002769947052 6.0
0 -0.0238606929779053 7.0
0 0.0567612648010254 8.0
0 -0.0814022421836853 9.0
0 0.113146305084229 10.0
0 0.0580595731735229 11.0
0 -0.0925793647766113 12.0
68.542152303 70.1681823730469 13.0
32.960545583 25.2105560302734 14.0
0 0.266101598739624 15.0
0 -0.0290539264678955 16.0
39.452684336 36.4450149536133 17.0
0 0.147804856300354 18.0
0 0.0140517950057983 19.0
0 0.017250120639801 20.0
0 0.243707716464996 21.0
0 -0.204923689365387 22.0
29.283652282 27.169584274292 23.0
0 0.0268151164054871 24.0
0 1.07085537910461 25.0
0 -0.171042919158936 26.0
0 0.139500975608826 27.0
0 -0.182334125041962 28.0
0 0.207859516143799 29.0
0 -0.00880825519561768 30.0
0 0.0146867036819458 31.0
0 -0.12842208147049 32.0
0 -0.229206085205078 33.0
0 -0.0898584723472595 34.0
0 0.860887110233307 35.0
0 0.431585311889648 36.0
0 0.14182436466217 37.0
0 0.313475102186203 38.0
0 -0.355413436889648 39.0
0 0.0460454225540161 40.0
0 -0.254585266113281 41.0
0 0.0902127027511597 42.0
281.32166198 293.808502197266 43.0
0 0.0957740545272827 44.0
0 0.0315367579460144 45.0
0 0.335483133792877 46.0
0 0.041243314743042 47.0
0 0.348134309053421 48.0
0 0.204399108886719 49.0
0.071041341186 0.0233427882194519 50.0
2.2134749445 1.86910092830658 51.0
0 0.0476247072219849 52.0
0 -0.238966345787048 53.0
0 0.208040714263916 54.0
0 0.290858268737793 55.0
0 -0.0130133628845215 56.0
0 -0.109385251998901 57.0
0 0.128281772136688 58.0
0 0.211456120014191 59.0
0 -0.212563395500183 60.0
0 0.191346287727356 61.0
0 0.333453416824341 62.0
272.50304604 254.371292114258 0.0
0 0.0697345733642578 1.0
0 0.111159384250641 2.0
0 0.265714168548584 3.0
0 0.0730689764022827 4.0
0 -0.0528014302253723 5.0
0 0.00944811105728149 6.0
0 -0.299071907997131 7.0
0 -0.319884181022644 8.0
0 0.0997312068939209 9.0
0 0.332468926906586 10.0
0 0.142683148384094 11.0
0 -0.433414936065674 12.0
44.234207789 44.0632209777832 13.0
33.727151645 33.5073661804199 14.0
0 -0.0214064121246338 15.0
0 0.00295895338058472 16.0
35.080437188 31.9784336090088 17.0
0 0.138619422912598 18.0
0 -0.0281556844711304 19.0
0 0.0765976905822754 20.0
0 0.0224353075027466 21.0
0 0.416627466678619 22.0
32.150843258 33.3580169677734 23.0
0 0.079916775226593 24.0
0 0.447529703378677 25.0
0 0.125164747238159 26.0
0 -0.21601402759552 27.0
0 0.271647989749908 28.0
0 -0.105102181434631 29.0
0 0.0484020709991455 30.0
0 -0.293159365653992 31.0
0 -0.0978050231933594 32.0
0 -0.488611698150635 33.0
0 0.0821114778518677 34.0
0 -0.102588474750519 35.0
0 0.398124158382416 36.0
0 0.0628963708877563 37.0
0 0.447154939174652 38.0
0 -0.36602246761322 39.0
0 -0.376273632049561 40.0
0 -0.227527141571045 41.0
0 -0.00244903564453125 42.0
300.78900995 294.853881835938 43.0
0 0.165992736816406 44.0
0 -0.110514998435974 45.0
0 -0.0643420815467834 46.0
0 -0.353471755981445 47.0
0 0.238346457481384 48.0
0 0.292559444904327 49.0
0.064249910472 0.0380886197090149 50.0
2.2288691353 1.90394330024719 51.0
0 0.074002206325531 52.0
0 -0.279656052589417 53.0
0 -0.138450384140015 54.0
0 -0.175163269042969 55.0
0 0.363758742809296 56.0
0 0.0893741250038147 57.0
0 -0.336867213249207 58.0
0 0.325434327125549 59.0
0 0.144003629684448 60.0
0 0.193920016288757 61.0
0 0.290603637695312 62.0
244.36375251 252.438186645508 0.0
0 0.0096668004989624 1.0
0 -0.0287948250770569 2.0
0 0.163212180137634 3.0
0 -0.358080387115479 4.0
0 -0.161117613315582 5.0
0 0.0781930088996887 6.0
0 -0.05361008644104 7.0
0 0.484133392572403 8.0
0 -0.455065250396729 9.0
0 0.222831547260284 10.0
0 -0.055755615234375 11.0
0 -0.108256042003632 12.0
59.109724165 60.6953773498535 13.0
38.727902611 32.7887725830078 14.0
0 -0.0872626304626465 15.0
0 -0.00089716911315918 16.0
32.281848932 32.923755645752 17.0
0 0.118667006492615 18.0
0 -0.0573204159736633 19.0
0 0.14738392829895 20.0
0 0.0615809559822083 21.0
0 0.149137556552887 22.0
30.154971234 27.8691692352295 23.0
0 -0.0359707474708557 24.0
0 -0.00724583864212036 25.0
0 -0.0914131999015808 26.0
0 -0.0320428609848022 27.0
0 0.440581232309341 28.0
0 -0.0400890707969666 29.0
0 0.039643406867981 30.0
0 -0.0442265272140503 31.0
0 -0.156509876251221 32.0
0 -0.153114974498749 33.0
0 0.0989530086517334 34.0
0 -0.183288514614105 35.0
0 0.232636630535126 36.0
0 0.189648687839508 37.0
0 0.0471619367599487 38.0
0 -0.206230163574219 39.0
0 0.541816234588623 40.0
0 -0.138331711292267 41.0
0 -0.0323183536529541 42.0
294.8443707 296.613006591797 43.0
0 0.00238007307052612 44.0
0 0.149643898010254 45.0
0 0.180575251579285 46.0
0 -0.241250991821289 47.0
0 -0.0726164579391479 48.0
0 0.362293720245361 49.0
0.074048340663 0.0280032157897949 50.0
2.2253786567 1.87428045272827 51.0
0 0.359141141176224 52.0
0 -0.297778129577637 53.0
0 0.145514726638794 54.0
0 0.267494082450867 55.0
0 0.285577654838562 56.0
0 0.307068854570389 57.0
0 0.162476718425751 58.0
0 -0.0799378752708435 59.0
0 -0.0312008857727051 60.0
0 0.0740593075752258 61.0
0 0.200317680835724 62.0
225.98772482 250.578475952148 0.0
0 0.100823402404785 1.0
0 -0.158508658409119 2.0
0 -0.158575475215912 3.0
0 -0.296565771102905 4.0
0 -0.159552574157715 5.0
0 -0.221619129180908 6.0
0 -0.212745666503906 7.0
0 0.491511285305023 8.0
0 -0.513299226760864 9.0
0 -0.22890841960907 10.0
0 -0.186578035354614 11.0
0 -0.174844264984131 12.0
58.656289413 58.9521064758301 13.0
36.373793884 35.6532402038574 14.0
0 0.0160400867462158 15.0
0 -0.106925666332245 16.0
31.464186454 31.8341007232666 17.0
0 0.000874638557434082 18.0
0 0.0178742408752441 19.0
0 0.121978878974915 20.0
0 0.0542511343955994 21.0
0 0.282661259174347 22.0
36.488292164 34.219841003418 23.0
0 0.0343173146247864 24.0
0 0.368416011333466 25.0
0 0.0825690031051636 26.0
0 -0.0278614163398743 27.0
0 0.878138482570648 28.0
0 0.0852738618850708 29.0
0 0.0693175792694092 30.0
0 0.0243585109710693 31.0
0 0.0516619682312012 32.0
0 0.146008193492889 33.0
0 -0.0749293565750122 34.0
0 0.0444782376289368 35.0
0 -0.35734486579895 36.0
0 -0.00209861993789673 37.0
0 0.301766812801361 38.0
0 -0.242180705070496 39.0
0 0.105623960494995 40.0
0 -0.224364042282104 41.0
0 -0.114606738090515 42.0
271.9449345 297.649291992188 43.0
0 0.0824321508407593 44.0
0 0.130111157894135 45.0
0 0.201461553573608 46.0
0 -0.232913732528687 47.0
0 0.281715273857117 48.0
1.6328395511 0.200972080230713 49.0
0.063136153826 -0.00921761989593506 50.0
1.8547471443 1.93762636184692 51.0
0 -0.0943405628204346 52.0
0 -0.0390167236328125 53.0
0 0.0660815238952637 54.0
0 0.208372712135315 55.0
0 0.0496953725814819 56.0
0 0.102440655231476 57.0
0 0.266888916492462 58.0
0 0.579442083835602 59.0
0 0.10314154624939 60.0
0 0.0747737288475037 61.0
0 0.164614200592041 62.0
259.02738298 253.895217895508 0.0
0 -0.2750084400177 1.0
0 -0.0778168439865112 2.0
0 -0.0706229209899902 3.0
0 -0.3223797082901 4.0
0 0.0596209764480591 5.0
0 0.117140412330627 6.0
0 -0.0620675683021545 7.0
0 -0.0459997057914734 8.0
0 0.106011569499969 9.0
0 -0.165404558181763 10.0
0 -0.145233392715454 11.0
0 -0.132986605167389 12.0
54.086684197 54.2034912109375 13.0
36.648849385 40.0634918212891 14.0
0 0.110616326332092 15.0
0 -0.211934447288513 16.0
30.509370584 32.6304664611816 17.0
0 0.109944224357605 18.0
0 -0.0526939630508423 19.0
0 0.195744097232819 20.0
0 -0.122698724269867 21.0
0 0.649449229240417 22.0
42.159636048 39.6824951171875 23.0
0 -0.00262194871902466 24.0
0 -0.00581616163253784 25.0
0 0.0861325263977051 26.0
0 0.115033388137817 27.0
0 0.130693137645721 28.0
0 -0.229592800140381 29.0
0 0.0275382995605469 30.0
0 -0.213558554649353 31.0
0 -0.210661053657532 32.0
0 -0.0980015993118286 33.0
0 0.29486745595932 34.0
0 0.247315943241119 35.0
0 -0.216333508491516 36.0
0 -0.0502029061317444 37.0
0 0.365208268165588 38.0
0 -0.331049561500549 39.0
0 0.393011420965195 40.0
0 -0.190953493118286 41.0
0 0.070095419883728 42.0
285.55075658 296.649169921875 43.0
0 0.139516353607178 44.0
0 0.240396440029144 45.0
0 0.0736860632896423 46.0
0 0.049924910068512 47.0
0 0.259572684764862 48.0
0 0.405065536499023 49.0
0.065499378212 0.0239488482475281 50.0
1.8006695232 1.87756538391113 51.0
0 0.00914144515991211 52.0
0 -0.0869146585464478 53.0
0 -0.151106774806976 54.0
0 0.158670663833618 55.0
0 0.298733353614807 56.0
0 0.0193967819213867 57.0
0 0.112837433815002 58.0
0 0.150617003440857 59.0
0 -0.319855451583862 60.0
0 -0.00284945964813232 61.0
0 0.120896279811859 62.0
287.83676607 251.776580810547 0.0
0 -0.147800207138062 1.0
0 0.0847129225730896 2.0
0 0.10981410741806 3.0
0 -0.22402024269104 4.0
0 -0.117442905902863 5.0
0 0.421867251396179 6.0
0 0.595347046852112 7.0
0 0.842413425445557 8.0
0 0.55589884519577 9.0
0 0.0522664785385132 10.0
0 0.242049038410187 11.0
0 0.0445032715797424 12.0
60.337048269 56.5230522155762 13.0
24.870050508 32.7442398071289 14.0
0 -0.00115084648132324 15.0
0 -0.0323686003684998 16.0
37.495261248 33.8732528686523 17.0
0 0.101147055625916 18.0
0 -0.0399941802024841 19.0
0 0.0923531055450439 20.0
0 -0.0160369277000427 21.0
0 0.290840446949005 22.0
33.381015837 32.9957542419434 23.0
0 -0.079728364944458 24.0
0 0.456368505954742 25.0
0 -0.0590602159500122 26.0
0 -0.0511571168899536 27.0
0 -0.0205029249191284 28.0
0 -0.0215378403663635 29.0
0 0.0618662238121033 30.0
0 0.147501230239868 31.0
0 0.266531527042389 32.0
0 -0.240978956222534 33.0
0 0.23760050535202 34.0
0 -0.0846459269523621 35.0
0 0.0986436605453491 36.0
0 0.207090795040131 37.0
0 -0.535385131835938 38.0
0 -0.0772793292999268 39.0
0 0.258516252040863 40.0
0 -0.199541866779327 41.0
0 0.0243326425552368 42.0
293.33065665 294.40234375 43.0
0 0.0997242331504822 44.0
0 0.0509265661239624 45.0
0 0.187306225299835 46.0
0 -0.151423513889313 47.0
0 -0.0283887386322021 48.0
0 0.551789939403534 49.0
0.10523408049 -0.00985556840896606 50.0
2.1242517802 1.91059947013855 51.0
0 0.273137629032135 52.0
0 -0.517046332359314 53.0
0 -0.0838107466697693 54.0
0 -0.0352770090103149 55.0
0 -0.0482381582260132 56.0
0 0.31983757019043 57.0
0 0.68866491317749 58.0
0 0.223409473896027 59.0
0 -0.224848508834839 60.0
0 0.0495978593826294 61.0
0 0.186728000640869 62.0
243.86735991 253.39811706543 0.0
0 -0.14525043964386 1.0
0 -0.0265958309173584 2.0
0 0.143746852874756 3.0
0 -0.238876461982727 4.0
0 0.132164835929871 5.0
0 -0.225239515304565 6.0
0 -0.103988349437714 7.0
0 0.349456906318665 8.0
0 -0.380585312843323 9.0
0 0.0286096334457397 10.0
0 0.124412000179291 11.0
0 0.0579491853713989 12.0
67.127002544 66.8244171142578 13.0
37.158914484 33.8587188720703 14.0
0 -0.0215553045272827 15.0
0 0.0838600993156433 16.0
33.329475029 36.8624534606934 17.0
0 0.178017377853394 18.0
0 -0.0818736553192139 19.0
0 0.0457851886749268 20.0
0 0.209006428718567 21.0
0 0.0425997972488403 22.0
25.899205427 24.6251792907715 23.0
0 0.054193377494812 24.0
0 0.500149726867676 25.0
0 0.0131525993347168 26.0
0 -0.24442720413208 27.0
0 0.289046943187714 28.0
0 -0.193568706512451 29.0
0 0.118714809417725 30.0
0 -0.0395965576171875 31.0
0 -0.168908298015594 32.0
0 -0.281532287597656 33.0
0 0.456414043903351 34.0
0 -0.119400799274445 35.0
0 0.517631053924561 36.0
0 0.144816875457764 37.0
0 -0.0552642941474915 38.0
0 0.00543844699859619 39.0
0 -0.34068489074707 40.0
0 -0.0289506316184998 41.0
0 -0.0247961282730103 42.0
295.98959212 296.084777832031 43.0
0 0.101113557815552 44.0
0 0.198146820068359 45.0
0 0.135301232337952 46.0
0 -0.00445806980133057 47.0
0 0.187942743301392 48.0
0 0.473156154155731 49.0
0.068121909443 -0.00496536493301392 50.0
1.4359158275 1.87641358375549 51.0
0 0.0120140314102173 52.0
0 -0.0649421215057373 53.0
0 -0.0379152894020081 54.0
0 0.0519804954528809 55.0
0 -0.13504946231842 56.0
0 0.0685441493988037 57.0
0 -0.268682837486267 58.0
0 0.27857917547226 59.0
0 -0.214915990829468 60.0
0 0.248312771320343 61.0
0 0.334884405136108 62.0
204.27638936 253.000564575195 0.0
0 -0.00116622447967529 1.0
0 -0.160814702510834 2.0
0 -0.114648282527924 3.0
0 -0.32138729095459 4.0
0 -0.191507458686829 5.0
0 -0.528066635131836 6.0
0 -0.245093822479248 7.0
0 0.842330038547516 8.0
0 -0.091375470161438 9.0
0 1.22707450389862 10.0
0 -0.425671219825745 11.0
0 0.0160560607910156 12.0
65.610447414 66.2749252319336 13.0
24.386461509 29.9735336303711 14.0
0 -0.0717262029647827 15.0
0 -0.0977271795272827 16.0
34.923882534 36.4336967468262 17.0
0 0.181094825267792 18.0
0 -0.038949191570282 19.0
0 0.13621062040329 20.0
0 -0.112452447414398 21.0
0 0.71151477098465 22.0
40.262979718 39.8276710510254 23.0
0 0.0215687155723572 24.0
0 1.09690093994141 25.0
0 0.0167958736419678 26.0
0 0.0763300061225891 27.0
0 0.635621666908264 28.0
0 0.230940699577332 29.0
0 0.127653360366821 30.0
0 -0.0239930152893066 31.0
0 0.0201670527458191 32.0
0 -0.101830244064331 33.0
0 0.257034003734589 34.0
0 0.0663501620292664 35.0
0 0.486324638128281 36.0
0 0.0573247671127319 37.0
0 0.123246967792511 38.0
0 -0.282339572906494 39.0
0 0.5923992395401 40.0
0 -0.155074834823608 41.0
0 0.0129982233047485 42.0
305.85007787 295.942047119141 43.0
0 0.0827782154083252 44.0
0 0.228317439556122 45.0
0 0.13349711894989 46.0
0 0.177422881126404 47.0
0 0.32497450709343 48.0
1.1637784227 0.370930701494217 49.0
0.097492309361 0.0735469460487366 50.0
1.6658639051 1.88883781433105 51.0
0 -0.169637858867645 52.0
0 -0.265411734580994 53.0
0 -0.0772802233695984 54.0
0 0.16509747505188 55.0
0 0.163089990615845 56.0
0 0.393006324768066 57.0
0 0.284751951694489 58.0
0 0.0378300547599792 59.0
0 -0.220855355262756 60.0
0 -0.00890600681304932 61.0
0 0.15660572052002 62.0
278.2407868 253.495681762695 0.0
0 -0.295074582099915 1.0
0 0.012082040309906 2.0
0 0.241776406764984 3.0
0 0.0989078879356384 4.0
0 0.233824491500854 5.0
0 -0.269716262817383 6.0
0 0.305377900600433 7.0
0 0.0585356950759888 8.0
0 0.179531097412109 9.0
0 0.303702294826508 10.0
0 -0.329404830932617 11.0
0 -0.159224808216095 12.0
72.807957719 73.2457122802734 13.0
25.519022516 26.2794933319092 14.0
0 0.0422217845916748 15.0
0 -0.0676039457321167 16.0
32.95020058 35.5499076843262 17.0
0 0.126688659191132 18.0
0 0.0103981494903564 19.0
0 0.0963942408561707 20.0
0 -0.00198841094970703 21.0
0 -0.0398327708244324 22.0
37.99715404 37.7790489196777 23.0
0 0.0786463022232056 24.0
0 0.529954075813293 25.0
0 0.0764180421829224 26.0
0 0.0384236574172974 27.0
0 0.423101723194122 28.0
0 -0.080707311630249 29.0
0 -0.0103233456611633 30.0
0 -0.00260543823242188 31.0
0 0.00166714191436768 32.0
0 -0.113598942756653 33.0
0 0.121236801147461 34.0
0 0.149572491645813 35.0
0 0.105994462966919 36.0
0 -0.00670093297958374 37.0
0 0.088107705116272 38.0
0 -0.26282262802124 39.0
0 0.506803870201111 40.0
0 -0.132257699966431 41.0
0 0.00832664966583252 42.0
302.16587076 296.477996826172 43.0
0 0.125185251235962 44.0
0 0.0631476044654846 45.0
0 0.163543105125427 46.0
0 0.172690272331238 47.0
0 0.275672614574432 48.0
1.2473462301 0.554470241069794 49.0
0.078781832126 0.0558823347091675 50.0
2.0193618966 1.85130834579468 51.0
0 0.186804056167603 52.0
0 0.0135302543640137 53.0
0 0.0922366380691528 54.0
0 0.241653561592102 55.0
0 0.104876756668091 56.0
0 0.333448201417923 57.0
0 0.0854707956314087 58.0
0 0.016402006149292 59.0
0 0.619365274906158 60.0
0 0.00106930732727051 61.0
0 0.0872951149940491 62.0
280.18100927 252.267349243164 0.0
0 0.00673568248748779 1.0
0 -0.0357246398925781 2.0
0 0.148557960987091 3.0
0 -0.00641828775405884 4.0
0 0.0529958009719849 5.0
0 0.113692820072174 6.0
0 0.393914043903351 7.0
0 0.0155084729194641 8.0
0 -0.301543235778809 9.0
0 0.445425927639008 10.0
0 -0.261260986328125 11.0
0 -0.176562905311584 12.0
64.034107254 64.8063507080078 13.0
27.453889419 29.1242828369141 14.0
0 -0.0338735580444336 15.0
0 -0.025904655456543 16.0
31.922249331 34.0983848571777 17.0
0 0.111764192581177 18.0
0 0.0362993478775024 19.0
0 0.0228162407875061 20.0
0 0.241795897483826 21.0
0 -0.00409138202667236 22.0
31.277426049 29.0433254241943 23.0
0 0.0777944326400757 24.0
0 0.79002434015274 25.0
0 -0.0149965286254883 26.0
0 -0.164353549480438 27.0
0 0.674222946166992 28.0
0 0.0402876138687134 29.0
0 0.0306528210639954 30.0
0 -0.147334694862366 31.0
0 0.00439155101776123 32.0
0 -0.086248517036438 33.0
0 0.168184638023376 34.0
0 -0.0192513465881348 35.0
0 -0.103776931762695 36.0
0 -0.0529328584671021 37.0
0 0.25072979927063 38.0
0 -0.356636047363281 39.0
0 0.151538252830505 40.0
0 -0.140490829944611 41.0
0 0.120841383934021 42.0
297.14844558 296.390167236328 43.0
0 0.096500039100647 44.0
0 0.127982258796692 45.0
0 0.173105955123901 46.0
0 0.063632607460022 47.0
0 -0.0980674028396606 48.0
0 0.436632513999939 49.0
0.07284401556 -0.00875496864318848 50.0
1.5912764728 1.84317231178284 51.0
0 0.235227584838867 52.0
0 -0.0301215648651123 53.0
0 0.0639227032661438 54.0
0 0.024220883846283 55.0
0 -0.14007556438446 56.0
0 0.267191827297211 57.0
0 0.208452463150024 58.0
0 -0.0443820953369141 59.0
0 -0.131165981292725 60.0
0 0.0455379486083984 61.0
0 0.222683429718018 62.0
238.99100309 253.961715698242 0.0
0 -0.0782389640808105 1.0
0 -0.0025780200958252 2.0
0 -0.136427342891693 3.0
0 -0.25680935382843 4.0
0 0.0797892212867737 5.0
0 -0.019163966178894 6.0
0 -0.314544439315796 7.0
0 -1.07563126087189 8.0
0 0.932836651802063 9.0
0 0.52830445766449 10.0
0 0.833508133888245 11.0
0 -0.127707362174988 12.0
56.874574283 55.3687400817871 13.0
26.222699521 39.8941307067871 14.0
0 0.00347578525543213 15.0
0 0.308811247348785 16.0
35.490898359 34.7593955993652 17.0
0 0.0844261050224304 18.0
0 -0.0245479345321655 19.0
0 0.0354887247085571 20.0
0 0.136040091514587 21.0
0 0.108938217163086 22.0
27.382919413 24.0444850921631 23.0
0 0.0529288053512573 24.0
0 0.493019461631775 25.0
0 -0.366658210754395 26.0
0 -0.0639331340789795 27.0
0 3.7742805480957 28.0
0 0.133601903915405 29.0
0 0.0697247982025146 30.0
0 -0.0752602815628052 31.0
0 0.0258104205131531 32.0
0 -0.17432963848114 33.0
0 0.311396062374115 34.0
0 -0.0511155724525452 35.0
0 0.17122095823288 36.0
0 -0.0228212475776672 37.0
0 0.211020112037659 38.0
0 -0.137586116790771 39.0
0 0.979033052921295 40.0
0 -0.102562427520752 41.0
0 0.424075841903687 42.0
310.10594086 295.567352294922 43.0
0 0.0944869518280029 44.0
0 0.243566632270813 45.0
0 0.105152308940887 46.0
0 -0.192348003387451 47.0
0 0.244403541088104 48.0
0 0.574384689331055 49.0
0.073393748773 0.0153895616531372 50.0
2.4182322292 1.83877110481262 51.0
0 -0.247693300247192 52.0
0 -0.0145061612129211 53.0
0 -0.109638094902039 54.0
0 -0.107091128826141 55.0
0 -0.0719020962715149 56.0
0 0.159757077693939 57.0
0 0.398587971925735 58.0
0 0.253321170806885 59.0
0 -0.101132571697235 60.0
0 0.0843005776405334 61.0
0 0.307514309883118 62.0
290.4737658 255.126327514648 0.0
0 -0.360400915145874 1.0
0 -0.0417086482048035 2.0
0 0.00342166423797607 3.0
0 -0.216850757598877 4.0
0 0.149584889411926 5.0
0 0.461073398590088 6.0
0 0.398554980754852 7.0
0 -0.185121536254883 8.0
0 0.423220455646515 9.0
0 -0.55445921421051 10.0
0 0.0432981252670288 11.0
0 -0.129816174507141 12.0
74.47712597 75.6207580566406 13.0
31.018134641 21.1586990356445 14.0
0 -0.0549987554550171 15.0
0 -0.0925387144088745 16.0
28.377817682 31.4295902252197 17.0
0 0.08357834815979 18.0
0 0.0263662338256836 19.0
0 0.111516237258911 20.0
0 0.0908806324005127 21.0
0 0.263329863548279 22.0
35.202012653 35.2058906555176 23.0
0 0.12419992685318 24.0
0 1.09834635257721 25.0
0 -0.0271899700164795 26.0
0 -0.185648798942566 27.0
0 0.00169289112091064 28.0
0 -0.140203058719635 29.0
0 0.0591503977775574 30.0
0 -0.0690543651580811 31.0
0 0.0105777978897095 32.0
0 0.113959908485413 33.0
0 0.308686792850494 34.0
0 0.0520787239074707 35.0
0 0.203100562095642 36.0
0 0.0401677489280701 37.0
0 0.0976338386535645 38.0
0 -0.182634353637695 39.0
0 0.630049705505371 40.0
0 -0.0824127793312073 41.0
0 -0.0602853298187256 42.0
289.39060655 296.603546142578 43.0
0 0.109235286712646 44.0
0 0.127769291400909 45.0
0 0.232062458992004 46.0
0 -0.206580221652985 47.0
0 -0.0493284463882446 48.0
0 0.568952322006226 49.0
0.079677720901 -0.0268694162368774 50.0
1.6928243377 1.87651705741882 51.0
0 0.260598301887512 52.0
0 -0.212504386901855 53.0
0 -0.162382006645203 54.0
0 -0.109144866466522 55.0
0 -0.0246195197105408 56.0
0 -0.0586221218109131 57.0
0 0.123538613319397 58.0
0 0.230130791664124 59.0
0 -0.248576521873474 60.0
0 0.0540657639503479 61.0
0 0.242278814315796 62.0
248.43958602 254.596145629883 0.0
0 -0.257029294967651 1.0
0 -0.0252641439437866 2.0
0 0.334650665521622 3.0
0 -0.0173696875572205 4.0
0 0.0785141587257385 5.0
0 0.00403159856796265 6.0
0 0.319891273975372 7.0
0 -0.213943719863892 8.0
0 -0.223517775535583 9.0
0 -0.0876393914222717 10.0
0 -0.314676642417908 11.0
0 -0.175366044044495 12.0
63.766245353 63.7500648498535 13.0
22.956768577 34.0427017211914 14.0
0 -0.150014996528625 15.0
0 0.144439995288849 16.0
39.839276995 36.8686485290527 17.0
0 0.167099952697754 18.0
0 0.00706624984741211 19.0
0 0.148152232170105 20.0
0 0.021762490272522 21.0
0 0.127568304538727 22.0
38.350925468 37.3115196228027 23.0
0 0.0431733131408691 24.0
0 0.261549770832062 25.0
0 0.073820948600769 26.0
0 0.095400869846344 27.0
0 0.316158443689346 28.0
0 -0.0599068999290466 29.0
0 0.0833374857902527 30.0
0 0.16634738445282 31.0
0 0.218233108520508 32.0
0 0.188223004341125 33.0
0 0.210035741329193 34.0
0 -0.00747811794281006 35.0
0 -0.157674491405487 36.0
0 -0.0381024479866028 37.0
0 0.346746474504471 38.0
0 -0.163030028343201 39.0
0 -0.246584415435791 40.0
0 -0.203493118286133 41.0
0 0.00649917125701904 42.0
292.8071353 297.005798339844 43.0
0 0.206764221191406 44.0
0 0.122271180152893 45.0
0 0.267112612724304 46.0
0 -0.403470873832703 47.0
0 0.0142893195152283 48.0
0 0.365446805953979 49.0
0.085439234167 -0.000421106815338135 50.0
1.7366201606 1.83551263809204 51.0
0 -0.206149935722351 52.0
0 -0.0589576959609985 53.0
0 -0.0966461896896362 54.0
0 -0.0809205770492554 55.0
0 -0.0731849670410156 56.0
0 -0.0166090726852417 57.0
0 0.05645352602005 58.0
0 0.159361481666565 59.0
0 -0.0273625254631042 60.0
0 0.0964087843894958 61.0
0 0.211299300193787 62.0
278.46626545 252.946487426758 0.0
0 -0.343790650367737 1.0
0 0.0461902022361755 2.0
0 0.427285254001617 3.0
0 -0.0303784608840942 4.0
0 0.114832878112793 5.0
0 -0.154235422611237 6.0
0 0.176007032394409 7.0
0 0.0313948392868042 8.0
0 0.00848895311355591 9.0
0 0.464108854532242 10.0
0 -0.441028833389282 11.0
0 -0.189618289470673 12.0
51.008070864 50.6980400085449 13.0
42.164999155 38.4182624816895 14.0
0 -0.116131901741028 15.0
0 -0.114827752113342 16.0
30.013560124 34.4046287536621 17.0
0 0.069547176361084 18.0
0 -0.0396162867546082 19.0
0 0.0485358238220215 20.0
0 0.149579584598541 21.0
0 0.0276452302932739 22.0
27.988556239 27.098087310791 23.0
0 0.0183829665184021 24.0
0 0.00540119409561157 25.0
0 -0.061970055103302 26.0
0 -0.0896362662315369 27.0
0 -0.272689342498779 28.0
0 -0.10735958814621 29.0
0 0.0211794972419739 30.0
0 -0.206364631652832 31.0
0 -0.0438634157180786 32.0
0 0.0335556268692017 33.0
0 -0.01082444190979 34.0
0 -0.0152344703674316 35.0
0 -0.0191325545310974 36.0
0 -0.0571763515472412 37.0
0 0.138028800487518 38.0
0 -0.135286450386047 39.0
0 0.576951801776886 40.0
0 -0.105344116687775 41.0
0 0.10556697845459 42.0
290.07483721 298.701263427734 43.0
0 0.0394682884216309 44.0
0 0.118433594703674 45.0
0 0.153557360172272 46.0
0 -0.161236464977264 47.0
0 0.368390053510666 48.0
2.7750411744 0.473505139350891 49.0
0.081625496609 0.0921077728271484 50.0
1.7007950898 1.90091586112976 51.0
0 -0.148856043815613 52.0
0 -0.145322918891907 53.0
0 -0.166696429252625 54.0
0 -0.218393921852112 55.0
0 -0.117642223834991 56.0
0 0.153470098972321 57.0
0 0.163689136505127 58.0
0 0.0259072184562683 59.0
0 0.0725765228271484 60.0
0 0.0774199962615967 61.0
0 0.218326091766357 62.0
211.85605503 254.969314575195 0.0
0 0.0270258188247681 1.0
0 -0.0389180779457092 2.0
0 -0.102905213832855 3.0
0 -0.26801872253418 4.0
0 -0.152012586593628 5.0
0 -0.260963439941406 6.0
0 -0.360252976417542 7.0
0 0.0633704662322998 8.0
0 0.502612709999084 9.0
0 1.2575626373291 10.0
0 0.692886352539062 11.0
0 -0.190979242324829 12.0
55.314151002 55.8790588378906 13.0
38.009230018 39.9127502441406 14.0
0 -0.153573393821716 15.0
0 -0.234747052192688 16.0
35.889405338 34.3498802185059 17.0
0 0.0585299730300903 18.0
0 0.0260897278785706 19.0
0 0.190557301044464 20.0
0 -0.187273740768433 21.0
0 0.852395832538605 22.0
43.183188086 41.7103843688965 23.0
0 0.0900116562843323 24.0
0 1.301225066185 25.0
0 -0.0705339312553406 26.0
0 0.0121535062789917 27.0
0 0.522379636764526 28.0
0 0.0524128079414368 29.0
0 -0.0269984006881714 30.0
0 1.04172742366791 31.0
0 -0.0604214668273926 32.0
0 -0.252792239189148 33.0
0 -0.0665125846862793 34.0
0 0.783742904663086 35.0
0 -0.121832609176636 36.0
0 -0.0378258228302002 37.0
0 0.271111309528351 38.0
0 -0.10045051574707 39.0
0 0.257290959358215 40.0
0 -0.115252077579498 41.0
0 -0.157827019691467 42.0
302.40578242 293.294891357422 43.0
0 -0.0530769824981689 44.0
0 0.0774238109588623 45.0
0 0.0146524906158447 46.0
0 -0.276405692100525 47.0
0 0.22150993347168 48.0
0 0.631824493408203 49.0
0.1044434842 0.0284416675567627 50.0
2.202523898 1.70865845680237 51.0
0 -0.192065000534058 52.0
0 -0.254405736923218 53.0
0 0.189657747745514 54.0
0 0.192246317863464 55.0
0 0.299284040927887 56.0
0 0.440047442913055 57.0
0 -0.341105222702026 58.0
0 0.298923015594482 59.0
0 -0.167781710624695 60.0
0 0.226605176925659 61.0
0 0.298622846603394 62.0
302.27147038 252.209106445312 0.0
0 -0.308248519897461 1.0
0 0.101375877857208 2.0
0 -0.153676450252533 3.0
0 -0.395318984985352 4.0
0 0.569132626056671 5.0
0 0.209168195724487 6.0
0 0.735047817230225 7.0
0 0.00497257709503174 8.0
0 0.288085877895355 9.0
0 -0.36550509929657 10.0
0 -0.110687375068665 11.0
0 -0.320831775665283 12.0
56.885010396 52.4745178222656 13.0
48.455075711 39.3468360900879 14.0
0 -0.0513203144073486 15.0
0 0.0893326997756958 16.0
31.571768759 33.6408615112305 17.0
0 0.0756670236587524 18.0
0 -0.0508340001106262 19.0
0 0.0742338895797729 20.0
0 0.0664833784103394 21.0
0 0.225306630134583 22.0
27.090642875 29.2526264190674 23.0
0 0.12372875213623 24.0
0 0.878160119056702 25.0
0 -0.147819638252258 26.0
0 -0.325085759162903 27.0
0 0.750466883182526 28.0
0 0.277999401092529 29.0
0 0.0606400966644287 30.0
0 -0.142856597900391 31.0
0 0.0128178000450134 32.0
0 -0.00644659996032715 33.0
0 0.078298032283783 34.0
0 -0.0129643678665161 35.0
0 -0.443127393722534 36.0
0 0.140922725200653 37.0
0 0.196236312389374 38.0
0 -0.28423547744751 39.0
0 0.563681960105896 40.0
0 -0.0531609654426575 41.0
0 0.0121139287948608 42.0
304.213328 294.941040039062 43.0
0 0.0812873840332031 44.0
0 0.0626106262207031 45.0
0 0.0588480234146118 46.0
0 0.0110979080200195 47.0
0 0.316557794809341 48.0
1.415409114 0.587415337562561 49.0
0.096039305992 0.0883252024650574 50.0
1.743682165 1.88393831253052 51.0
0 -0.0838199853897095 52.0
0 -0.293612957000732 53.0
0 0.150997877120972 54.0
0 0.234723210334778 55.0
0 -0.0270238518714905 56.0
0 0.302745819091797 57.0
0 0.194940090179443 58.0
0 0.314728081226349 59.0
0 -0.00707226991653442 60.0
0 0.110853374004364 61.0
0 0.226469874382019 62.0
224.12577429 255.161163330078 0.0
0 -0.0923612117767334 1.0
0 -0.200171172618866 2.0
0 -0.143869042396545 3.0
0 -0.221621632575989 4.0
0 -0.378965854644775 5.0
0 -0.305723428726196 6.0
0 -0.212781667709351 7.0
0 0.0292388200759888 8.0
0 0.094944953918457 9.0
0 0.493625938892365 10.0
0 -0.232358574867249 11.0
0 -0.078822135925293 12.0
80.760436044 77.4812927246094 13.0
11.578269525 26.2437343597412 14.0
0 -0.132413148880005 15.0
0 -0.18517655134201 16.0
25.819864654 31.6459579467773 17.0
0 0.0836450457572937 18.0
0 -0.00559109449386597 19.0
0 0.1702601313591 20.0
0 -0.0136505961418152 21.0
0 0.67491489648819 22.0
43.05718695 40.0970878601074 23.0
0 0.116017580032349 24.0
0 0.507373690605164 25.0
0 -0.0529197454452515 26.0
0 0.0341665744781494 27.0
0 -0.170302510261536 28.0
0 0.0288205146789551 29.0
0 -0.0669518709182739 30.0
0 0.39007306098938 31.0
0 0.302108258008957 32.0
0 0.340596199035645 33.0
0 0.217332065105438 34.0
0 -0.111254334449768 35.0
0 -0.297325611114502 36.0
0 0.130288183689117 37.0
0 0.336993634700775 38.0
0 -0.159145593643188 39.0
0 0.586380124092102 40.0
0 -0.118672847747803 41.0
0 0.0987046957015991 42.0
282.29057351 296.096984863281 43.0
0 0.0622575283050537 44.0
0 0.00720685720443726 45.0
0 0.067918062210083 46.0
0 -0.0146854519844055 47.0
0 0.329100072383881 48.0
1.482510048 0.668060898780823 49.0
0.10706010674 0.0423023104667664 50.0
1.8007778272 1.87804698944092 51.0
0 -0.0109263062477112 52.0
0 -0.481452703475952 53.0
0 -0.0365891456604004 54.0
0 0.224652528762817 55.0
0 0.360223531723022 56.0
0 0.263222932815552 57.0
0 0.22275584936142 58.0
0 0.169467449188232 59.0
0 -0.096096396446228 60.0
0 0.138941645622253 61.0
0 0.238779842853546 62.0
285.68938743 252.773773193359 0.0
0 -0.364253640174866 1.0
0 -0.100867211818695 2.0
0 0.22434675693512 3.0
0 -0.287907361984253 4.0
0 -0.104715645313263 5.0
0 0.89640200138092 6.0
0 -0.26868200302124 7.0
0 -0.229868769645691 8.0
0 0.237892389297485 9.0
0 -0.144309401512146 10.0
0 -0.148361086845398 11.0
0 -0.0936249494552612 12.0
60.224962003 61.286865234375 13.0
34.50476733 29.9200992584229 14.0
0 -0.0747920274734497 15.0
0 0.0143822431564331 16.0
30.038763068 35.8090057373047 17.0
0 0.122708678245544 18.0
0 -0.025728166103363 19.0
0 0.116750597953796 20.0
0 0.205550312995911 21.0
0 -0.103407621383667 22.0
31.937172262 30.0300636291504 23.0
0 0.056358814239502 24.0
26.303015044 0.379159986972809 25.0
0 -0.0928561687469482 26.0
0 -0.00130456686019897 27.0
0 -0.283936500549316 28.0
0 0.030064582824707 29.0
0 0.0399412512779236 30.0
0 -0.230689525604248 31.0
0 -0.107242524623871 32.0
0 -0.220595598220825 33.0
0 0.121004283428192 34.0
0 0.208508312702179 35.0
0 0.488458871841431 36.0
0 0.0806089043617249 37.0
0 0.161274492740631 38.0
0 -0.171519875526428 39.0
0 0.138815939426422 40.0
0 -0.0831435918807983 41.0
0 -0.0611124038696289 42.0
299.48295738 297.818603515625 43.0
0 0.069044828414917 44.0
0 0.155178129673004 45.0
0 0.192400753498077 46.0
0 -0.0487194061279297 47.0
0 0.273366093635559 48.0
0 0.437626481056213 49.0
0.063905944587 0.0349001884460449 50.0
1.4339563034 1.82580637931824 51.0
0 0.102348268032074 52.0
0 0.0104296207427979 53.0
0 0.211557269096375 54.0
0 0.26392674446106 55.0
0 -0.155105292797089 56.0
0 0.379767894744873 57.0
0 0.143171727657318 58.0
0 0.141359329223633 59.0
0 0.113693952560425 60.0
0 0.0803769826889038 61.0
0 0.253596425056458 62.0
243.47223828 239.098159790039 0.0
0 -0.445218205451965 1.0
0 0.0736136436462402 2.0
0 0.256439685821533 3.0
0 0.0910831689834595 4.0
0 -0.0152524709701538 5.0
0 0.136062383651733 6.0
0 0.199922323226929 7.0
0 -0.666868209838867 8.0
0 0.0661293864250183 9.0
0 -0.147091746330261 10.0
0 0.284340500831604 11.0
0 -0.141764163970947 12.0
62.264105959 59.4314994812012 13.0
55.784385164 39.5187149047852 14.0
0 -0.394859433174133 15.0
0 -0.106391906738281 16.0
24.170314652 32.4339599609375 17.0
0 0.0905965566635132 18.0
0 -0.0160877704620361 19.0
0 0.109094321727753 20.0
0 0.0678534507751465 21.0
0 0.106349050998688 22.0
27.019698714 28.04833984375 23.0
0 -0.0230056643486023 24.0
33.024345954 0.530391097068787 25.0
0 0.0664207339286804 26.0
0 -0.255985379219055 27.0
0 -0.57318651676178 28.0
0 -0.0607731342315674 29.0
0 0.0205028057098389 30.0
0 -0.213611125946045 31.0
0 0.116015195846558 32.0
0 0.115310668945312 33.0
0 0.317545086145401 34.0
0 0.201671123504639 35.0
0 0.077358603477478 36.0
0 0.0730947852134705 37.0
0 -0.260227918624878 38.0
0 -0.153189182281494 39.0
0 0.40630891919136 40.0
0 -0.139233589172363 41.0
0 -0.0202169418334961 42.0
291.10179427 295.005706787109 43.0
0 0.0524917244911194 44.0
0 0.236816883087158 45.0
0 0.250331819057465 46.0
0 -0.254177808761597 47.0
0 0.352129548788071 48.0
0 0.535863280296326 49.0
0.072210044653 0.018318235874176 50.0
2.0891500799 1.8020327091217 51.0
0 -0.173276960849762 52.0
0 -0.442548274993896 53.0
0 0.2826247215271 54.0
0 -0.0706958174705505 55.0
0 -0.0702599287033081 56.0
0 -0.0940603017807007 57.0
0 0.628868341445923 58.0
0 0.209130883216858 59.0
0 -0.230425834655762 60.0
0 0.0214765071868896 61.0
0 0.233457148075104 62.0
231.44172736 255.622100830078 0.0
0 -0.109544634819031 1.0
0 0.0552120208740234 2.0
0 -0.668634176254272 3.0
0 -0.387713670730591 4.0
0 -0.104903697967529 5.0
0 0.140061974525452 6.0
0 0.00470662117004395 7.0
0 0.426943838596344 8.0
0 -0.06268310546875 9.0
0 0.264343917369843 10.0
0 0.0389323830604553 11.0
0 0.0667033195495605 12.0
66.979039259 65.0723495483398 13.0
36.082329342 44.1823616027832 14.0
0 0.711282849311829 15.0
0 0.226704359054565 16.0
36.355899886 33.6683731079102 17.0
0 0.265473961830139 18.0
0 -0.0626325607299805 19.0
0 0.221627116203308 20.0
0 -0.0678493976593018 21.0
0 1.04066836833954 22.0
40.479611254 39.3692932128906 23.0
0 0.00228959321975708 24.0
0 0.303019821643829 25.0
0 0.0369724035263062 26.0
0 -0.00722163915634155 27.0
0 0.0533636212348938 28.0
0 -0.0476865768432617 29.0
0 0.0666813254356384 30.0
0 -0.178982377052307 31.0
0 -0.0246304869651794 32.0
0 0.103522717952728 33.0
0 -0.0247522592544556 34.0
0 -0.181825041770935 35.0
0 -0.187865138053894 36.0
0 0.0199695825576782 37.0
0 0.22237491607666 38.0
0 -0.36487877368927 39.0
0 0.517230868339539 40.0
0 -0.167552351951599 41.0
0 0.560965299606323 42.0
282.16731331 291.315673828125 43.0
0 -0.0374950170516968 44.0
0 0.347629249095917 45.0
0 0.383469104766846 46.0
0 -0.0413873791694641 47.0
0 -0.0104870796203613 48.0
0 0.323766857385635 49.0
0.098510456341 0.0497097373008728 50.0
1.8216671832 1.87753057479858 51.0
0 0.137282907962799 52.0
0 -0.191358685493469 53.0
0 -0.0924859046936035 54.0
0 -0.174719214439392 55.0
0 -0.0370548367500305 56.0
0 -0.0450168251991272 57.0
0 0.109288156032562 58.0
0 0.344634413719177 59.0
0 0.0236376523971558 60.0
0 -0.0269384384155273 61.0
0 0.0425739884376526 62.0
256.04335226 256.60595703125 0.0
0 -0.225867629051208 1.0
0 0.18935614824295 2.0
0 0.0989806056022644 3.0
0 0.0715236663818359 4.0
0 -0.257702112197876 5.0
0 -0.359637141227722 6.0
0 0.20925098657608 7.0
0 -0.176236271858215 8.0
0 0.0874485373497009 9.0
0 -0.204281032085419 10.0
0 -0.116932213306427 11.0
0 -0.058775782585144 12.0
49.58600719 50.3124656677246 13.0
50.933527123 44.9896545410156 14.0
0 -0.141067028045654 15.0
0 -0.126595258712769 16.0
37.832984153 33.7433853149414 17.0
0 0.0876250267028809 18.0
0 -0.0104648470878601 19.0
0 0.0876845121383667 20.0
0 0.143834352493286 21.0
0 0.133269369602203 22.0
35.199212953 33.6653480529785 23.0
0 0.023855984210968 24.0
0 0.219523072242737 25.0
0 -0.223435997962952 26.0
0 0.161335945129395 27.0
0 -0.570763349533081 28.0
0 -0.0563911199569702 29.0
0 -0.0809488296508789 30.0
0 0.0590217113494873 31.0
0 0.0331144332885742 32.0
0 0.0387836098670959 33.0
0 0.0406702756881714 34.0
0 0.153092265129089 35.0
0 0.145537972450256 36.0
0 0.00427818298339844 37.0
0 0.213762044906616 38.0
0 -0.293435573577881 39.0
0 -0.273342370986938 40.0
0 -0.207421481609344 41.0
0 -0.0201020240783691 42.0
285.94291385 296.003448486328 43.0
0 0.101553320884705 44.0
0 0.165831327438354 45.0
0 0.0201816558837891 46.0
0 -0.115393459796906 47.0
0 0.207458734512329 48.0
0.1069576001 0.529126405715942 49.0
0.10204606399 0.0194047689437866 50.0
1.735372669 1.88926005363464 51.0
0 0.106416881084442 52.0
0 -0.174949109554291 53.0
0 -0.0829538106918335 54.0
0 -0.230065584182739 55.0
0 0.0457981824874878 56.0
0 0.189595818519592 57.0
0 0.154438436031342 58.0
0 0.017608106136322 59.0
0 -0.005748450756073 60.0
0 0.0667893290519714 61.0
0 0.173191845417023 62.0
249.99431079 252.457229614258 0.0
0 0.0800442695617676 1.0
0 0.0627956986427307 2.0
0 -0.138368248939514 3.0
0 -0.185224413871765 4.0
0 0.121590971946716 5.0
0 -0.0714364051818848 6.0
0 0.0451976656913757 7.0
0 0.504604399204254 8.0
0 -0.161356091499329 9.0
0 0.0637110471725464 10.0
0 -0.127621650695801 11.0
0 -0.0914586782455444 12.0
54.466281631 53.0822715759277 13.0
33.833278203 40.5587615966797 14.0
0 0.0129774808883667 15.0
0 0.178270995616913 16.0
27.226103391 32.1714134216309 17.0
0 0.0518369674682617 18.0
0 0.0204585790634155 19.0
0 0.0640709400177002 20.0
0 0.0941239595413208 21.0
0 0.239219069480896 22.0
33.033023389 32.5713348388672 23.0
0 0.0659227967262268 24.0
0 0.587524771690369 25.0
0 0.0264847278594971 26.0
0 -0.0782828330993652 27.0
0 0.185118019580841 28.0
0 -0.0303234457969666 29.0
0 -0.0535279512405396 30.0
0 -0.159222602844238 31.0
0 -0.183933556079865 32.0
0 -0.335748434066772 33.0
0 0.186317920684814 34.0
0 0.907511413097382 35.0
0 0.566096782684326 36.0
0 0.181630492210388 37.0
0 0.0606306195259094 38.0
0 -0.18410849571228 39.0
0 0.550889670848846 40.0
0 -0.174684882164001 41.0
0 0.0591228008270264 42.0
294.14701013 297.095916748047 43.0
0 0.177189469337463 44.0
0 0.04683518409729 45.0
0 0.216111242771149 46.0
0 -0.110656678676605 47.0
0 -0.351552248001099 48.0
0 0.557948350906372 49.0
0.071906328711 -0.00207173824310303 50.0
1.8575115052 1.812420129776 51.0
0 0.270863056182861 52.0
0 -0.0754247307777405 53.0
0 0.130694925785065 54.0
0 -0.158777356147766 55.0
0 -0.0749863386154175 56.0
0 -0.0929591655731201 57.0
0 0.167161583900452 58.0
0 0.229965567588806 59.0
0 -0.203030586242676 60.0
0 0.0710281133651733 61.0
0 0.218169689178467 62.0
229.55700833 252.45378112793 0.0
0 -0.0154918432235718 1.0
0 -0.0674894452095032 2.0
0 0.0641407370567322 3.0
0 -0.392084836959839 4.0
0 -0.0382185578346252 5.0
0 -0.113527059555054 6.0
0 -0.177568912506104 7.0
0 0.188360095024109 8.0
0 -0.0698431134223938 9.0
0 0.0616835951805115 10.0
0 0.0972608327865601 11.0
0 -0.0876373052597046 12.0
73.270746549 75.6084976196289 13.0
25.213872912 21.1722316741943 14.0
0 -0.0110613107681274 15.0
0 -0.148145914077759 16.0
30.659179873 33.7102394104004 17.0
0 0.0750062465667725 18.0
0 -0.0377671122550964 19.0
0 0.11779111623764 20.0
0 0.122859835624695 21.0
0 0.0626437664031982 22.0
34.007752525 32.2929878234863 23.0
0 0.111126959323883 24.0
0 0.572220802307129 25.0
0 -0.0525468587875366 26.0
0 -0.14309561252594 27.0
0 0.504078447818756 28.0
0 -0.0379788279533386 29.0
0 0.141507983207703 30.0
0 0.105676889419556 31.0
0 0.0130266547203064 32.0
0 0.0163973569869995 33.0
0 0.374906480312347 34.0
0 0.00115102529525757 35.0
0 0.21426260471344 36.0
0 -0.0305609107017517 37.0
0 0.204621970653534 38.0
0 -0.25230872631073 39.0
0 0.0762837529182434 40.0
0 -0.13638973236084 41.0
0 -0.07586669921875 42.0
297.53709612 297.044036865234 43.0
0 0.0955444574356079 44.0
0 0.126559376716614 45.0
0 0.307586848735809 46.0
0 0.0991168022155762 47.0
0 0.129654705524445 48.0
0 0.373240947723389 49.0
0.076997497906 0.04853355884552 50.0
1.7632970171 1.87572908401489 51.0
0 0.285243451595306 52.0
0 0.00812488794326782 53.0
0 -0.0297863483428955 54.0
0 0.0436359643936157 55.0
0 0.36990612745285 56.0
0 0.00257360935211182 57.0
0 -0.00639510154724121 58.0
0 -0.582560896873474 59.0
0 -0.0437050461769104 60.0
0 0.0353558659553528 61.0
0 0.0532174706459045 62.0
290.32108164 250.725677490234 0.0
0 -0.388897657394409 1.0
0 0.0786800384521484 2.0
0 0.498093903064728 3.0
0 0.231312215328217 4.0
0 0.239108085632324 5.0
0 -0.118109881877899 6.0
0 0.821828007698059 7.0
0 1.07817888259888 8.0
0 0.0942890644073486 9.0
0 0.185968697071075 10.0
0 -0.388919830322266 11.0
0 0.0848018527030945 12.0
64.363623498 64.3065643310547 13.0
23.731238707 27.6082305908203 14.0
0 -0.259561777114868 15.0
0 -0.189559817314148 16.0
41.973632891 34.4516716003418 17.0
0 0.0546076893806458 18.0
0 -0.0677303075790405 19.0
0 0.168926060199738 20.0
0 -0.0822399854660034 21.0
0 0.378413558006287 22.0
37.798567681 40.310302734375 23.0
0 0.137992799282074 24.0
0 1.26048564910889 25.0
0 -0.217912673950195 26.0
0 -0.046604335308075 27.0
0 0.595311522483826 28.0
0 0.092890739440918 29.0
0 0.0529944896697998 30.0
0 -0.110598087310791 31.0
0 0.183569073677063 32.0
0 0.224892675876617 33.0
0 0.038077712059021 34.0
0 0.350773185491562 35.0
0 0.0702213048934937 36.0
0 0.150964915752411 37.0
0 0.137187778949738 38.0
0 -0.306575536727905 39.0
0 0.50166654586792 40.0
0 -0.179130077362061 41.0
0 0.111670970916748 42.0
292.94971953 297.750946044922 43.0
0 -0.177091658115387 44.0
0 0.191630423069 45.0
0 0.140381395816803 46.0
0 0.292758703231812 47.0
0 0.358156353235245 48.0
0 0.452170372009277 49.0
0.10854584856 0.0760765671730042 50.0
2.2162483298 1.84667778015137 51.0
0 0.288881659507751 52.0
0 0.324081122875214 53.0
0 -0.0933089256286621 54.0
0 0.305644154548645 55.0
0 -0.0699911713600159 56.0
0 0.369457542896271 57.0
0 -0.21575927734375 58.0
0 0.182793200016022 59.0
0 -0.220047473907471 60.0
0 0.156159639358521 61.0
0 0.376589477062225 62.0
270.70358841 250.35285949707 0.0
0 -0.211496710777283 1.0
0 0.0945085883140564 2.0
0 0.419689536094666 3.0
0 -0.0833213329315186 4.0
0 -0.0877041816711426 5.0
0 -0.145684599876404 6.0
0 0.076521635055542 7.0
0 -0.17436671257019 8.0
0 -0.163454532623291 9.0
0 -0.049790620803833 10.0
0 0.025051474571228 11.0
0 -0.0560151934623718 12.0
58.730379266 56.4227104187012 13.0
44.024796473 40.3440742492676 14.0
0 0.0392076969146729 15.0
0 -0.249234318733215 16.0
27.685896229 31.404899597168 17.0
0 0.0316066741943359 18.0
0 0.0609053373336792 19.0
0 0.0787180066108704 20.0
0 0.131261110305786 21.0
0 0.204195320606232 22.0
30.61702817 29.9672031402588 23.0
0 0.0363774299621582 24.0
0 0.693513870239258 25.0
0 0.0111116170883179 26.0
0 -0.106430411338806 27.0
0 -0.183542013168335 28.0
0 -0.0213791728019714 29.0
0 0.0474062561988831 30.0
0 0.0693483352661133 31.0
0 0.157038629055023 32.0
0 0.0242935419082642 33.0
0 0.2566859126091 34.0
0 -0.13210928440094 35.0
0 -0.222234129905701 36.0
0 0.0406859517097473 37.0
0 0.033822238445282 38.0
0 -0.156079649925232 39.0
0 0.461732447147369 40.0
0 -0.139299988746643 41.0
0 0.0299054384231567 42.0
290.83558994 295.815643310547 43.0
0 0.0924564599990845 44.0
0 0.155639886856079 45.0
0 0.132775247097015 46.0
0 -0.480156183242798 47.0
0 0.395164012908936 48.0
0 0.409462690353394 49.0
0.098612944162 0.0537590384483337 50.0
2.2175752334 1.88089621067047 51.0
0 0.0285324454307556 52.0
0 -0.251579165458679 53.0
0 0.199519395828247 54.0
0 0.0392547845840454 55.0
0 0.110165417194366 56.0
0 -0.149213135242462 57.0
0 0.0356150269508362 58.0
0 -0.0333267450332642 59.0
0 -0.0175514817237854 60.0
0 0.12742555141449 61.0
0 0.23081761598587 62.0
229.67722721 252.004531860352 0.0
0 -0.0716793537139893 1.0
0 -0.263624310493469 2.0
0 -0.000909626483917236 3.0
0 -0.314092636108398 4.0
0 -0.111997604370117 5.0
0 -0.450210213661194 6.0
0 -0.129744172096252 7.0
0 -0.519105911254883 8.0
0 0.150993645191193 9.0
0 0.279242932796478 10.0
0 -0.142298579216003 11.0
0 -0.0453554391860962 12.0
72.197980483 72.2104263305664 13.0
22.10056248 28.79150390625 14.0
0 0.205966114997864 15.0
0 -0.297495126724243 16.0
30.645359998 32.7790946960449 17.0
0 0.029163122177124 18.0
0 -0.00901186466217041 19.0
0 0.144197702407837 20.0
0 -0.0179857015609741 21.0
0 0.399389743804932 22.0
34.989811965 34.3033103942871 23.0
0 0.0404825210571289 24.0
0 0.286089539527893 25.0
0 0.073639988899231 26.0
0 -0.0966236591339111 27.0
0 0.67943149805069 28.0
0 -0.0266536474227905 29.0
0 0.0444700717926025 30.0
0 -0.0768519639968872 31.0
0 0.11085045337677 32.0
0 0.235206723213196 33.0
0 -0.192746937274933 34.0
0 -0.226216197013855 35.0
0 0.0844598412513733 36.0
0 0.149462401866913 37.0
0 0.163380205631256 38.0
0 -0.198656916618347 39.0
0 0.623036623001099 40.0
0 -0.150995671749115 41.0
0 0.482404351234436 42.0
274.15371495 293.473052978516 43.0
0 0.141383290290833 44.0
0 0.220698416233063 45.0
0 0.00768983364105225 46.0
0 0.2559694647789 47.0
0 -0.20888763666153 48.0
0 0.274861454963684 49.0
0.059528125774 0.0664666295051575 50.0
2.3204037312 1.79625928401947 51.0
0 -0.194584012031555 52.0
0 -0.0535427331924438 53.0
0 -0.0483245849609375 54.0
0 -0.13373851776123 55.0
0 0.246499001979828 56.0
0 0.263146281242371 57.0
0 0.277934968471527 58.0
0 0.246741056442261 59.0
0 -0.249849557876587 60.0
0 0.00258123874664307 61.0
0 0.0743710398674011 62.0
284.55288592 253.668212890625 0.0
0 -0.399788022041321 1.0
0 0.0970315933227539 2.0
0 0.320198088884354 3.0
0 -0.128659844398499 4.0
0 0.209908187389374 5.0
0 -0.220043301582336 6.0
0 0.409851044416428 7.0
0 -0.11126172542572 8.0
0 0.146336019039154 9.0
0 -0.456809997558594 10.0
0 -0.0573351383209229 11.0
0 0.115544319152832 12.0
64.612632853 66.1932601928711 13.0
41.664207064 32.5243873596191 14.0
0 -0.232973217964172 15.0
0 0.180981695652008 16.0
33.14027959 33.761791229248 17.0
0 0.10009503364563 18.0
0 0.00600004196166992 19.0
0 0.0286951661109924 20.0
0 0.219014108181 21.0
0 0.0906798839569092 22.0
27.384997931 26.9351844787598 23.0
0 0.0403164625167847 24.0
2.4481295822 0.691281497478485 25.0
0 0.0688005685806274 26.0
0 -0.395235061645508 27.0
0 0.737821936607361 28.0
0 0.0733465552330017 29.0
0 0.0918454527854919 30.0
0 0.478844523429871 31.0
0 0.398124068975449 32.0
0 0.205970764160156 33.0
0 0.242278695106506 34.0
0 -0.221226811408997 35.0
0 -0.251402378082275 36.0
0 0.00594484806060791 37.0
0 -0.00527119636535645 38.0
0 -0.102120876312256 39.0
0 0.33624666929245 40.0
0 -0.0928118824958801 41.0
0 0.166794061660767 42.0
304.46962616 299.158325195312 43.0
0 0.169845700263977 44.0
0 0.0228875875473022 45.0
0 0.235490500926971 46.0
0 -0.010728657245636 47.0
0 0.0090596079826355 48.0
0 0.377309054136276 49.0
0.097826748779 -0.0490965843200684 50.0
1.6617190146 1.89519393444061 51.0
0 -0.259616017341614 52.0
0 0.0355579853057861 53.0
0 -0.0977746844291687 54.0
0 0.11846786737442 55.0
0 0.079767107963562 56.0
0 -0.0897996425628662 57.0
0 -0.281233549118042 58.0
0 0.070199191570282 59.0
0 -0.265802621841431 60.0
0 0.176863431930542 61.0
0 0.160222768783569 62.0
254.6715707 252.56022644043 0.0
0 -0.0780291557312012 1.0
0 -0.25439178943634 2.0
0 0.436952382326126 3.0
0 -0.0147898197174072 4.0
0 -0.112674117088318 5.0
0 0.0596917271614075 6.0
0 0.101579904556274 7.0
0 0.248167514801025 8.0
0 0.551932811737061 9.0
0 0.290258169174194 10.0
0 0.29383310675621 11.0
0 -0.0697705745697021 12.0
55.91474104 55.9253387451172 13.0
39.143796666 33.1785659790039 14.0
0 -0.0592601299285889 15.0
0 0.0282682180404663 16.0
29.39783079 34.5771865844727 17.0
0 0.109945356845856 18.0
0 0.0479515790939331 19.0
0 0.0932672023773193 20.0
0 0.0678446292877197 21.0
0 0.0803819894790649 22.0
35.489715219 35.3095588684082 23.0
0 0.0676316022872925 24.0
0 0.944873213768005 25.0
0 -0.0984011888504028 26.0
0 0.0671631097793579 27.0
0 0.148029923439026 28.0
0 0.0101234912872314 29.0
0 -0.115787923336029 30.0
0 -0.0332883596420288 31.0
0 -0.108067870140076 32.0
0 -0.186940491199493 33.0
0 0.233325600624084 34.0
0 0.0728577971458435 35.0
0 0.344090789556503 36.0
0 0.102217197418213 37.0
0 0.322004467248917 38.0
0 -0.188803434371948 39.0
0 0.141939640045166 40.0
0 -0.310648083686829 41.0
0 -0.0583091974258423 42.0
293.14026509 297.571655273438 43.0
0 0.078140914440155 44.0
0 0.0407657623291016 45.0
0 0.130806684494019 46.0
0 0.0384491086006165 47.0
0 -0.318781614303589 48.0
0 0.460390537977219 49.0
0.098867996991 -0.00671499967575073 50.0
1.7472478128 1.82854819297791 51.0
0 0.0421636700630188 52.0
0 0.132388353347778 53.0
0 -0.127886056900024 54.0
0 -0.0600834488868713 55.0
0 0.376308083534241 56.0
0 -0.0107944011688232 57.0
0 0.004799485206604 58.0
0 -0.065868079662323 59.0
0 0.00327438116073608 60.0
0 0.00607770681381226 61.0
0 0.0125805139541626 62.0
212.20393165 253.075225830078 0.0
0 -0.0597493648529053 1.0
0 -0.126690030097961 2.0
0 0.377607583999634 3.0
0 -0.320528268814087 4.0
0 -0.270084500312805 5.0
0 0.0447071194648743 6.0
0 -0.405708074569702 7.0
0 0.0434056520462036 8.0
0 -0.130261600017548 9.0
0 0.0302481055259705 10.0
0 -0.0882551670074463 11.0
0 -0.186081409454346 12.0
71.128505893 71.1180572509766 13.0
40.901673931 29.3089427947998 14.0
0 0.18134605884552 15.0
0 -0.0897663235664368 16.0
27.293677358 30.0029754638672 17.0
0 0.0439323782920837 18.0
0 -0.0252945423126221 19.0
0 0.161807060241699 20.0
0 0.00509995222091675 21.0
0 0.038486123085022 22.0
37.440146235 35.0515251159668 23.0
0 0.020423412322998 24.0
0 0.320922583341599 25.0
0 0.0633608102798462 26.0
0 0.0624033808708191 27.0
0 0.54827743768692 28.0
0 -0.0921100378036499 29.0
0 0.0694817304611206 30.0
0 -0.126519918441772 31.0
0 -0.0238104462623596 32.0
0 -0.24900484085083 33.0
0 0.363913983106613 34.0
0 -0.00723254680633545 35.0
0 0.21428120136261 36.0
0 0.2334144115448 37.0
0 0.135780334472656 38.0
0 -0.300308108329773 39.0
0 0.373063385486603 40.0
0 -0.19176971912384 41.0
0 0.0353425741195679 42.0
298.30976256 295.889312744141 43.0
0 0.012885570526123 44.0
0 0.209497570991516 45.0
0 0.0950964689254761 46.0
0 0.137455821037292 47.0
0 0.0585240125656128 48.0
0 0.392832159996033 49.0
0.073350875899 0.045997679233551 50.0
1.4500164192 1.87838387489319 51.0
0 0.334431618452072 52.0
0 0.118891775608063 53.0
0 9.76324081420898e-05 54.0
0 -0.0632015466690063 55.0
0 -0.039880633354187 56.0
0 0.394208073616028 57.0
0 0.53675103187561 58.0
0 0.227075457572937 59.0
0 -0.146280884742737 60.0
0 0.0965713262557983 61.0
0 0.341612696647644 62.0
247.88849941 253.971054077148 0.0
0 0.500690460205078 1.0
0 0.019245982170105 2.0
0 0.316729128360748 3.0
0 -0.284910678863525 4.0
0 -0.129894256591797 5.0
0 -0.0648271441459656 6.0
0 -0.155815720558167 7.0
0 0.329464077949524 8.0
0 -0.615214109420776 9.0
0 0.149652302265167 10.0
0 -0.241110563278198 11.0
0 -0.296494603157043 12.0
50.733748978 50.0053367614746 13.0
37.92975737 42.1177711486816 14.0
0 -0.0856438875198364 15.0
0 0.340894281864166 16.0
32.540317113 32.9878768920898 17.0
0 0.148636341094971 18.0
0 0.0538517236709595 19.0
0 0.0955094695091248 20.0
0 0.0323995351791382 21.0
0 0.164014637470245 22.0
42.542608801 40.8129539489746 23.0
0 0.0058588981628418 24.0
0 0.612969100475311 25.0
0 -0.512023448944092 26.0
0 0.185712516307831 27.0
0 -0.718328356742859 28.0
0 -0.120775461196899 29.0
0 0.0509141683578491 30.0
0 0.204665422439575 31.0
0 -0.120649099349976 32.0
0 -0.399041891098022 33.0
0 0.216720342636108 34.0
0 -0.0871561765670776 35.0
0 0.055439829826355 36.0
0 -0.0073818564414978 37.0
0 0.139918386936188 38.0
0 -0.631402492523193 39.0
0 0.449596881866455 40.0
0 -0.265287756919861 41.0
0 0.000959396362304688 42.0
301.97600572 298.275451660156 43.0
0 0.0865761041641235 44.0
0 0.339124262332916 45.0
0 0.295130431652069 46.0
0 0.141873359680176 47.0
0 -0.593429327011108 48.0
0 0.419764816761017 49.0
0.090966138698 -0.0048871636390686 50.0
2.0947647829 1.8093900680542 51.0
0 0.0580376386642456 52.0
0 -0.256020426750183 53.0
0 0.186721444129944 54.0
0 -0.169951975345612 55.0
0 -0.13846755027771 56.0
0 -0.102327764034271 57.0
0 0.0903538465499878 58.0
0 0.180572807788849 59.0
0 -0.0642566680908203 60.0
0 0.266870737075806 61.0
0 0.290305376052856 62.0
260.89285463 253.114974975586 0.0
0 -0.283226847648621 1.0
0 -0.0334450006484985 2.0
0 0.217206716537476 3.0
0 0.219426453113556 4.0
0 -0.211693167686462 5.0
0 -0.334119200706482 6.0
0 0.554956555366516 7.0
0 -0.0515393018722534 8.0
0 -0.766132473945618 9.0
0 -0.112244546413422 10.0
0 -0.43838095664978 11.0
0 -0.076158881187439 12.0
58.285007757 57.4371299743652 13.0
32.113945564 27.8014106750488 14.0
0 -0.16618812084198 15.0
0 -0.289021849632263 16.0
36.008457059 38.1113052368164 17.0
0 0.140700340270996 18.0
0 -0.0419397354125977 19.0
0 0.102123379707336 20.0
0 -0.0970350503921509 21.0
0 0.393179059028625 22.0
42.147832913 42.303897857666 23.0
0 0.113702535629272 24.0
0 0.748503029346466 25.0
0 0.141614973545074 26.0
0 0.064542293548584 27.0
0 0.217284262180328 28.0
0 -0.0845588445663452 29.0
0 0.0702165961265564 30.0
0 -0.139173984527588 31.0
0 0.253548860549927 32.0
0 -0.391171336174011 33.0
0 -0.204718768596649 34.0
0 -0.162865996360779 35.0
0 -0.461406588554382 36.0
0 0.0523855686187744 37.0
0 0.413094609975815 38.0
0 -0.296579599380493 39.0
0 0.10340690612793 40.0
0 -0.274045705795288 41.0
0 0.104853391647339 42.0
294.83808443 299.581634521484 43.0
0 0.16249406337738 44.0
0 0.0660918951034546 45.0
0 0.0165629982948303 46.0
0 0.0963525772094727 47.0
0 0.262925148010254 48.0
0.17172879965 0.377733886241913 49.0
0.060800106113 0.0235692858695984 50.0
1.9995505506 1.93192446231842 51.0
0 0.097423791885376 52.0
0 -0.21172034740448 53.0
0 0.142281830310822 54.0
0 -0.00366306304931641 55.0
0 -0.0226883888244629 56.0
0 -0.455873250961304 57.0
0 0.346117854118347 58.0
0 -0.864118695259094 59.0
0 -0.0639622211456299 60.0
0 0.144690275192261 61.0
0 0.113976538181305 62.0
214.57075343 253.094436645508 0.0
0 0.247322797775269 1.0
0 -0.153290450572968 2.0
0 0.050885021686554 3.0
0 -0.30247175693512 4.0
0 -0.361271142959595 5.0
0 0.0926363468170166 6.0
0 -0.278758406639099 7.0
0 0.314576596021652 8.0
0 -0.432440519332886 9.0
0 0.103097796440125 10.0
0 -0.0648596882820129 11.0
0 0.0151708722114563 12.0
57.599323047 57.4118843078613 13.0
40.177904194 38.3315391540527 14.0
0 -0.0209194421768188 15.0
0 0.112204074859619 16.0
40.63475257 36.1017875671387 17.0
0 0.174883961677551 18.0
0 -0.00867772102355957 19.0
0 0.135386288166046 20.0
0 -0.0736618041992188 21.0
0 0.880774736404419 22.0
41.297173024 37.9579124450684 23.0
0 0.00191271305084229 24.0
0 0.733713805675507 25.0
0 0.067753791809082 26.0
0 0.116812944412231 27.0
0 0.00156545639038086 28.0
0 -0.0361918210983276 29.0
0 0.0388441681861877 30.0
0 0.100771307945251 31.0
0 0.0679466724395752 32.0
0 -0.0532301664352417 33.0
0 0.0690805315971375 34.0
0 0.140812933444977 35.0
0 0.133200645446777 36.0
0 -0.105151772499084 37.0
0 0.180999875068665 38.0
0 -0.196435332298279 39.0
0 0.336133658885956 40.0
0 -0.1051384806633 41.0
0 0.038744330406189 42.0
295.99639626 296.549072265625 43.0
0 0.023068904876709 44.0
0 0.172315955162048 45.0
0 0.0348606705665588 46.0
0 -0.00880253314971924 47.0
0 0.322895258665085 48.0
1.9146032212 0.49542635679245 49.0
0.096800124531 0.0849581956863403 50.0
2.0968843075 1.86672914028168 51.0
0 0.195787370204926 52.0
0 0.0286316871643066 53.0
0 0.0192298889160156 54.0
0 0.0785180926322937 55.0
0 0.0945628881454468 56.0
0 -0.216615319252014 57.0
0 0.505865335464478 58.0
0 0.303891062736511 59.0
0 0.00809156894683838 60.0
0 0.0610761642456055 61.0
0 0.210501849651337 62.0
217.91946451 252.456649780273 0.0
0 -0.0529606342315674 1.0
0 -0.225420236587524 2.0
0 0.0651484727859497 3.0
0 -0.394914865493774 4.0
0 -0.253963708877563 5.0
0 0.158975481987 6.0
0 -0.172566533088684 7.0
0 0.292846083641052 8.0
0 -0.370996117591858 9.0
0 -0.25170111656189 10.0
0 0.510730624198914 11.0
0 -0.146316170692444 12.0
59.478834199 57.7918739318848 13.0
46.409562916 36.7384071350098 14.0
0 -0.00750112533569336 15.0
0 -0.0306667685508728 16.0
28.692105235 31.1832370758057 17.0
0 0.12610250711441 18.0
0 -0.0367223024368286 19.0
0 0.155087351799011 20.0
0 0.0132268667221069 21.0
0 0.116984665393829 22.0
37.997745539 38.1818885803223 23.0
0 0.164865851402283 24.0
0 0.834787786006927 25.0
0 -0.0626564025878906 26.0
0 -0.160306215286255 27.0
0 0.510809302330017 28.0
0 -0.000342726707458496 29.0
0 0.186724781990051 30.0
0 0.119624137878418 31.0
0 0.179990470409393 32.0
0 -0.00553238391876221 33.0
0 0.289358079433441 34.0
0 0.27023059129715 35.0
0 0.139975786209106 36.0
0 0.0624163150787354 37.0
0 0.0127449035644531 38.0
0 0.0273995399475098 39.0
0 0.1820228099823 40.0
0 -0.0949894189834595 41.0
0 0.204218149185181 42.0
291.23321184 295.321350097656 43.0
0 0.168207406997681 44.0
0 0.0722628831863403 45.0
0 0.12715095281601 46.0
0 0.177984833717346 47.0
0 0.306145817041397 48.0
0 0.564040184020996 49.0
0.063417660552 0.0223599076271057 50.0
2.0206333439 1.87073564529419 51.0
0 0.266118228435516 52.0
0 0.0501289963722229 53.0
0 0.0759997367858887 54.0
0 0.0495836734771729 55.0
0 0.227188289165497 56.0
0 0.0699658989906311 57.0
0 0.02229905128479 58.0
0 0.0194936394691467 59.0
0 -0.172647297382355 60.0
0 0.118550002574921 61.0
0 0.169088006019592 62.0
262.76678226 253.517166137695 0.0
0 -0.242236018180847 1.0
0 -0.144802570343018 2.0
0 0.0462941527366638 3.0
0 -0.0633887648582458 4.0
0 0.308631956577301 5.0
0 0.543815910816193 6.0
0 0.290221631526947 7.0
0 0.2631915807724 8.0
0 0.121235966682434 9.0
0 -0.0321550965309143 10.0
0 -0.00596517324447632 11.0
0 -0.0382668972015381 12.0
57.824283741 56.4824256896973 13.0
45.487282754 41.8595123291016 14.0
0 0.0399632453918457 15.0
0 0.0060843825340271 16.0
39.603523949 35.2679328918457 17.0
0 0.105020701885223 18.0
0 -0.0430435538291931 19.0
0 0.100431561470032 20.0
0 0.161326885223389 21.0
0 0.176232933998108 22.0
31.806319326 30.7740001678467 23.0
0 0.08161860704422 24.0
0 0.593542098999023 25.0
0 0.135161697864532 26.0
0 -0.074201762676239 27.0
0 0.180777192115784 28.0
0 -0.0208743214607239 29.0
0 0.00285083055496216 30.0
0 -0.010987401008606 31.0
0 -0.0995376110076904 32.0
0 -0.350972056388855 33.0
0 -0.0801239609718323 34.0
0 -0.115966618061066 35.0
0 0.0501567125320435 36.0
0 -0.0926018953323364 37.0
0 0.241963088512421 38.0
0 -0.200772762298584 39.0
0 0.540271818637848 40.0
0 -0.121247887611389 41.0
0 0.0790005922317505 42.0
304.30449262 295.398803710938 43.0
0 0.159503042697906 44.0
0 0.0126522183418274 45.0
0 0.155012488365173 46.0
0 0.0371679663658142 47.0
0 0.43880558013916 48.0
0 0.343824058771133 49.0
0.07597002613 0.0752201080322266 50.0
1.5939782681 1.85556817054749 51.0
0 0.0723190307617188 52.0
0 0.255243360996246 53.0
0 0.21026885509491 54.0
0 0.309446632862091 55.0
0 0.194614052772522 56.0
0 0.283871233463287 57.0
0 0.372750759124756 58.0
0 0.195972800254822 59.0
0 -0.277234435081482 60.0
0 0.0543429851531982 61.0
0 0.301955699920654 62.0
218.4108762 251.034530639648 0.0
0 -0.0329862833023071 1.0
0 -0.122136533260345 2.0
0 0.246895015239716 3.0
0 -0.374651789665222 4.0
0 0.0779525637626648 5.0
0 0.192176759243011 6.0
0 -0.19869589805603 7.0
0 0.166987359523773 8.0
0 -0.328819990158081 9.0
0 -0.137233853340149 10.0
0 0.0384547710418701 11.0
0 -0.0692269802093506 12.0
78.369453694 76.6323547363281 13.0
26.678807942 21.8306865692139 14.0
0 -0.237597823143005 15.0
0 0.0803963541984558 16.0
33.882306077 35.2667694091797 17.0
0 0.0906515121459961 18.0
0 0.0126050710678101 19.0
0 0.0222954750061035 20.0
0 0.0125089883804321 21.0
0 0.109397530555725 22.0
36.920556958 34.9587593078613 23.0
0 0.0201689600944519 24.0
0 0.48031622171402 25.0
0 0.0463161468505859 26.0
0 -0.00418412685394287 27.0
0 0.43607771396637 28.0
0 -0.0910640954971313 29.0
0 -0.0734755992889404 30.0
0 0.0922433137893677 31.0
0 0.116714537143707 32.0
0 0.146842360496521 33.0
0 0.435824275016785 34.0
0 0.267489373683929 35.0
0 0.433623462915421 36.0
0 -0.0911968946456909 37.0
0 -0.033815324306488 38.0
0 -0.115630030632019 39.0
0 0.384872049093246 40.0
0 -0.0220958590507507 41.0
0 0.00954782962799072 42.0
293.48660797 297.116577148438 43.0
0 0.118919968605042 44.0
0 0.156383275985718 45.0
0 0.188439548015594 46.0
0 -0.394903898239136 47.0
0 -0.183135151863098 48.0
0 0.266487300395966 49.0
0.077486446423 0.0538175106048584 50.0
1.8858513395 1.83649325370789 51.0
0 0.178568124771118 52.0
0 -0.00849366188049316 53.0
0 -0.0177697539329529 54.0
0 0.270263493061066 55.0
0 0.252046883106232 56.0
0 0.366136938333511 57.0
0 0.260123670101166 58.0
0 0.208712518215179 59.0
0 -0.014310896396637 60.0
0 0.0721975564956665 61.0
0 0.24516749382019 62.0
225.35068032 247.399291992188 0.0
0 -0.356130361557007 1.0
0 -0.0253424644470215 2.0
0 -0.266126871109009 3.0
0 -0.296106815338135 4.0
0 -0.0882505178451538 5.0
0 -0.126179873943329 6.0
0 -0.234584450721741 7.0
0 -0.290032744407654 8.0
0 -0.135265946388245 9.0
0 0.464198768138885 10.0
0 1.61674523353577 11.0
0 -0.0503864288330078 12.0
78.746372517 75.8704681396484 13.0
42.900920484 23.234016418457 14.0
0 -0.252153515815735 15.0
0 -0.06993168592453 16.0
41.654272338 36.6443214416504 17.0
0 0.246658205986023 18.0
0 -0.0605565309524536 19.0
0 0.0428816080093384 20.0
0 0.241965174674988 21.0
0 0.0498003363609314 22.0
25.618625159 26.0934925079346 23.0
0 0.117808401584625 24.0
0 1.00657951831818 25.0
0 0.0233597755432129 26.0
0 -0.0630602240562439 27.0
0 -0.484380960464478 28.0
0 -0.15825355052948 29.0
0 0.0169674158096313 30.0
0 -0.123408675193787 31.0
0 -0.00857377052307129 32.0
0 -0.363861918449402 33.0
0 -0.184573948383331 34.0
0 0.347268164157867 35.0
0 0.300905734300613 36.0
0 0.118209779262543 37.0
0 -0.049998939037323 38.0
0 -0.253299117088318 39.0
0 0.23039311170578 40.0
0 -0.0437890887260437 41.0
0 0.165754437446594 42.0
291.88263004 297.788330078125 43.0
0 0.0113444328308105 44.0
0 0.104885756969452 45.0
0 -0.042760968208313 46.0
0 0.154128789901733 47.0
0 -0.235154390335083 48.0
0 0.352370262145996 49.0
0.098550668453 0.0193905830383301 50.0
1.3765129248 1.8992635011673 51.0
0 -0.00376087427139282 52.0
0 0.139207720756531 53.0
0 -0.165338397026062 54.0
0 -0.140501856803894 55.0
0 0.223056197166443 56.0
0 -0.0467190742492676 57.0
0 0.076860785484314 58.0
0 0.253724694252014 59.0
0 -0.216061115264893 60.0
0 0.163602411746979 61.0
0 0.31423282623291 62.0
238.68326094 253.571578979492 0.0
0 0.119780659675598 1.0
0 -0.293054819107056 2.0
0 0.0886507630348206 3.0
0 -0.356717348098755 4.0
0 0.0740649104118347 5.0
0 0.221379518508911 6.0
0 0.112571179866791 7.0
0 0.121941566467285 8.0
0 -0.0160531997680664 9.0
0 -0.166745185852051 10.0
0 -0.248014450073242 11.0
0 -0.106869459152222 12.0
64.238030054 63.8104515075684 13.0
26.748580019 35.5426139831543 14.0
0 -0.0235128402709961 15.0
0 0.0681027770042419 16.0
37.640256677 33.7804298400879 17.0
0 0.144158720970154 18.0
0 0.0222352743148804 19.0
0 0.14026141166687 20.0
0 0.0884575247764587 21.0
0 -0.0711888670921326 22.0
40.562912515 37.9988555908203 23.0
0 0.13227367401123 24.0
0 0.899878799915314 25.0
0 -0.0137090682983398 26.0
0 0.198269903659821 27.0
0 1.15006041526794 28.0
0 0.0714380741119385 29.0
0 0.0569051504135132 30.0
0 -0.0053175687789917 31.0
0 0.0260792374610901 32.0
0 -0.0978966951370239 33.0
0 0.395098954439163 34.0
0 0.12222820520401 35.0
0 -0.491836071014404 36.0
0 -0.041930615901947 37.0
0 0.196727931499481 38.0
0 -0.223586559295654 39.0
0 0.529878079891205 40.0
0 -0.168500006198883 41.0
0 -0.0676405429840088 42.0
285.16804358 297.034637451172 43.0
0 -0.07480788230896 44.0
0 0.0736928582191467 45.0
0 0.202807247638702 46.0
0 -0.0832141637802124 47.0
0 0.30579674243927 48.0
1.3722630632 0.548296570777893 49.0
0.088298272681 0.0390718579292297 50.0
2.0439038349 1.94426298141479 51.0
0 0.0914199352264404 52.0
0 -0.501301884651184 53.0
0 0.234933972358704 54.0
0 0.275888025760651 55.0
0 0.152164101600647 56.0
0 0.628353357315063 57.0
0 0.0528899431228638 58.0
0 0.0899221897125244 59.0
0 -0.0443979501724243 60.0
0 0.0322957038879395 61.0
0 0.105202734470367 62.0
280.56752019 251.554336547852 0.0
0 -0.286393642425537 1.0
0 -0.0337358713150024 2.0
0 0.0121447443962097 3.0
0 -0.377960562705994 4.0
0 0.00261318683624268 5.0
0 -0.0517987608909607 6.0
0 0.460114806890488 7.0
0 0.523292183876038 8.0
0 0.19038999080658 9.0
0 0.328549027442932 10.0
0 -0.569738030433655 11.0
0 -0.236106991767883 12.0
75.068642675 74.7605133056641 13.0
25.632167587 22.645528793335 14.0
0 0.0377823114395142 15.0
0 -0.1864413022995 16.0
32.104267336 32.0670852661133 17.0
0 0.0263196229934692 18.0
0 0.0536489486694336 19.0
0 0.0844904184341431 20.0
0 0.0125821828842163 21.0
0 0.286080777645111 22.0
36.117439943 33.6895332336426 23.0
0 -0.0256583094596863 24.0
0 -0.0363730788230896 25.0
0 -0.0871736407279968 26.0
0 -0.0533121228218079 27.0
0 0.2939393222332 28.0
0 0.0401489734649658 29.0
0 0.0721632838249207 30.0
0 0.225376725196838 31.0
0 0.153770327568054 32.0
0 0.0017390251159668 33.0
0 0.10342276096344 34.0
0 0.187871754169464 35.0
0 0.476319879293442 36.0
0 -0.0990138053894043 37.0
0 0.362176775932312 38.0
0 -0.268564343452454 39.0
0 0.0404181480407715 40.0
0 -0.221167206764221 41.0
0 0.000324726104736328 42.0
299.69281027 298.216461181641 43.0
0 0.0793041586875916 44.0
0 0.134879350662231 45.0
0 0.0495564937591553 46.0
0 0.213819205760956 47.0
0 0.364090085029602 48.0
0 0.379303127527237 49.0
0.068228669897 0.0646026134490967 50.0
1.9021331776 1.80438208580017 51.0
0 0.0275651216506958 52.0
0 -0.340100526809692 53.0
0 0.0822922587394714 54.0
0 -0.041224479675293 55.0
0 -0.00730955600738525 56.0
0 -0.217354416847229 57.0
0 0.393796771764755 58.0
0 0.284490942955017 59.0
0 -0.0822494029998779 60.0
0 0.163826942443848 61.0
0 0.427957653999329 62.0
254.54534295 253.332595825195 0.0
0 0.0800188779830933 1.0
0 -0.0155817270278931 2.0
0 0.069344699382782 3.0
0 -0.174357175827026 4.0
0 -0.114828586578369 5.0
0 0.0127884149551392 6.0
0 0.215950429439545 7.0
0 0.172299385070801 8.0
0 0.00264596939086914 9.0
0 0.0346167087554932 10.0
0 -0.0708538293838501 11.0
0 -0.108387231826782 12.0
67.208876715 66.5945739746094 13.0
31.372426399 32.7191772460938 14.0
0 -0.0113946199417114 15.0
0 0.124250650405884 16.0
34.5289667 35.160961151123 17.0
0 0.136818885803223 18.0
0 -0.0223143100738525 19.0
0 0.0537852048873901 20.0
0 0.117317020893097 21.0
0 0.150034725666046 22.0
29.077970191 29.9804153442383 23.0
0 0.144053816795349 24.0
0 1.07382643222809 25.0
0 -0.238278269767761 26.0
0 -0.263839483261108 27.0
0 0.530240178108215 28.0
0 0.0308335423469543 29.0
0 0.10390305519104 30.0
0 0.0348542928695679 31.0
0 0.0424318313598633 32.0
0 -0.0920252203941345 33.0
0 0.0699521899223328 34.0
0 0.0717751979827881 35.0
0 0.168845176696777 36.0
0 0.135152876377106 37.0
0 0.171021103858948 38.0
0 -0.262861847877502 39.0
0 -0.104213535785675 40.0
0 -0.255520701408386 41.0
0 0.0470119714736938 42.0
301.66498052 293.167602539062 43.0
0 0.114150702953339 44.0
0 -0.0942026972770691 45.0
0 0.184706211090088 46.0
0 -0.090215802192688 47.0
0 -0.224393844604492 48.0
0 0.381293267011642 49.0
0.06734684049 0.0350935459136963 50.0
2.1559368052 1.80137360095978 51.0
0 0.00814539194107056 52.0
0 -0.222868084907532 53.0
0 0.363633543252945 54.0
0 0.0754549503326416 55.0
0 0.0278946757316589 56.0
0 -0.0351768732070923 57.0
0 0.238186240196228 58.0
0 -0.0191103219985962 59.0
0 -0.0407223701477051 60.0
0 0.0740388035774231 61.0
0 0.20262223482132 62.0
217.62329508 251.604217529297 0.0
0 -0.0221011638641357 1.0
0 -0.209068238735199 2.0
0 -0.0149571895599365 3.0
0 -0.253203868865967 4.0
0 -0.418493390083313 5.0
0 -0.02881920337677 6.0
0 -0.246245265007019 7.0
0 0.493289083242416 8.0
0 -0.0625967383384705 9.0
0 0.182519197463989 10.0
0 -0.0506269931793213 11.0
0 -0.0180827379226685 12.0
68.580553107 69.0943222045898 13.0
24.059072432 23.8297481536865 14.0
0 -0.0793576240539551 15.0
0 -0.0280867218971252 16.0
40.604433122 37.8689994812012 17.0
0 0.193458795547485 18.0
0 -0.0619791150093079 19.0
0 0.0695767998695374 20.0
0 0.211121082305908 21.0
0 0.054007887840271 22.0
33.390634093 32.6749305725098 23.0
0 0.196948707103729 24.0
0 1.24982023239136 25.0
0 0.0850780010223389 26.0
0 -0.305091500282288 27.0
0 0.346279591321945 28.0
0 0.206983685493469 29.0
0 -0.0149887204170227 30.0
0 0.117642879486084 31.0
0 0.200783550739288 32.0
0 -0.00589776039123535 33.0
0 0.188696384429932 34.0
0 -0.00415009260177612 35.0
0 0.311398088932037 36.0
0 -0.052773118019104 37.0
0 0.525136888027191 38.0
0 -0.265128493309021 39.0
0 0.0170570611953735 40.0
0 -0.192408323287964 41.0
0 -0.0564409494400024 42.0
290.0880264 298.008697509766 43.0
0 0.375989317893982 44.0
0 0.134061694145203 45.0
0 0.358541011810303 46.0
0 -0.729129195213318 47.0
0 0.19574761390686 48.0
0 0.56942617893219 49.0
0.083890348419 0.00952684879302979 50.0
2.1308860983 1.85119462013245 51.0
0 -0.10783839225769 52.0
0 -0.198709785938263 53.0
0 -0.0179908275604248 54.0
0 0.288991987705231 55.0
0 0.227852463722229 56.0
0 0.39827898144722 57.0
0 0.457257777452469 58.0
0 0.0673028230667114 59.0
0 0.124888479709625 60.0
0 0.0352100133895874 61.0
0 0.211471080780029 62.0
252.8776666 254.187103271484 0.0
0 -0.24549663066864 1.0
0 0.104755938053131 2.0
0 0.331004023551941 3.0
0 -0.18711256980896 4.0
0 -0.138305425643921 5.0
0 -0.0953482985496521 6.0
0 0.269005656242371 7.0
0 -0.353404641151428 8.0
0 -0.0166040658950806 9.0
0 0.043784499168396 10.0
0 -0.377699375152588 11.0
0 -0.0004006028175354 12.0
53.371839696 53.4473457336426 13.0
36.182797605 44.185474395752 14.0
0 -0.100056886672974 15.0
0 -0.0807220935821533 16.0
35.488146765 34.7678489685059 17.0
0 0.106951653957367 18.0
0 0.0277094841003418 19.0
0 0.0875868201255798 20.0
0 0.0633825063705444 21.0
0 0.191395998001099 22.0
35.464251111 35.1372985839844 23.0
0 0.139330506324768 24.0
0 0.668382823467255 25.0
0 0.0575797557830811 26.0
0 -0.043559730052948 27.0
0 0.569834530353546 28.0
0 -0.00261801481246948 29.0
0 0.0556104183197021 30.0
0 0.213528990745544 31.0
0 0.0953535437583923 32.0
0 -0.0334456562995911 33.0
0 0.368489533662796 34.0
0 -0.0982785820960999 35.0
0 0.357616007328033 36.0
0 -0.071090042591095 37.0
0 0.150640904903412 38.0
0 -0.117160797119141 39.0
0 0.854465901851654 40.0
0 -0.0756282210350037 41.0
0 0.047987699508667 42.0
295.73657671 296.122467041016 43.0
0 0.074241042137146 44.0
0 0.186536431312561 45.0
0 0.269663989543915 46.0
0 -0.0639630556106567 47.0
0 -0.0587656497955322 48.0
0 0.345870703458786 49.0
0.075506082823 0.0585065484046936 50.0
2.2164233124 1.81561017036438 51.0
0 0.27320808172226 52.0
0 0.0506893992424011 53.0
0 0.179354727268219 54.0
0 0.252900719642639 55.0
0 0.319848030805588 56.0
0 -0.115948617458344 57.0
0 0.249008595943451 58.0
0 0.267764091491699 59.0
0 -0.234193205833435 60.0
0 0.0986005067825317 61.0
0 0.262162268161774 62.0
274.19193798 251.766632080078 0.0
0 -0.307957768440247 1.0
0 0.00250029563903809 2.0
0 0.361242711544037 3.0
0 -0.0376499891281128 4.0
0 0.0933785438537598 5.0
0 0.313313841819763 6.0
0 0.391687393188477 7.0
0 -0.06683748960495 8.0
0 0.240051925182343 9.0
0 -0.144623458385468 10.0
0 -0.527234554290771 11.0
0 0.00193405151367188 12.0
75.643082264 74.6911010742188 13.0
13.75747602 27.3135795593262 14.0
0 -0.0527645349502563 15.0
0 -0.0726127028465271 16.0
34.773097815 35.9376640319824 17.0
0 0.124595642089844 18.0
0 -0.0619392395019531 19.0
0 0.0693478584289551 20.0
0 0.0942362546920776 21.0
0 0.100898444652557 22.0
27.666336732 27.5908794403076 23.0
0 0.0784342288970947 24.0
0 0.738011837005615 25.0
0 -0.313973188400269 26.0
0 -0.207397878170013 27.0
0 0.567468881607056 28.0
0 0.00386261940002441 29.0
0 0.0469977259635925 30.0
0 -0.158801198005676 31.0
0 -0.0897310376167297 32.0
0 -0.0704221725463867 33.0
0 0.0178491473197937 34.0
0 -0.0597652792930603 35.0
0 0.300869435071945 36.0
0 0.159840762615204 37.0
0 0.0744946002960205 38.0
0 -0.318243980407715 39.0
0 0.550073206424713 40.0
0 -0.0644144415855408 41.0
0 0.016412615776062 42.0
297.56864332 297.494659423828 43.0
0 0.0450429916381836 44.0
0 0.0548178553581238 45.0
0 0.0683580040931702 46.0
0 0.188491225242615 47.0
0 0.33007338643074 48.0
0 0.452370285987854 49.0
0.085654161353 0.0574102401733398 50.0
1.7178663352 1.8379989862442 51.0
0 0.0396904945373535 52.0
0 -0.159263908863068 53.0
0 0.22611790895462 54.0
0 -0.0304108858108521 55.0
0 -0.0340445637702942 56.0
0 0.232280492782593 57.0
0 0.262715935707092 58.0
0 0.231567978858948 59.0
0 -0.177295923233032 60.0
0 0.0804269909858704 61.0
0 0.205240905284882 62.0
232.90973734 252.644500732422 0.0
0 -0.2120521068573 1.0
0 -0.074065625667572 2.0
0 0.0754503011703491 3.0
0 -0.372767090797424 4.0
0 -0.0283717513084412 5.0
0 0.13577127456665 6.0
0 0.0178622603416443 7.0
0 0.51894211769104 8.0
0 -0.359971761703491 9.0
0 -0.234905958175659 10.0
0 -0.036099374294281 11.0
0 -0.164715766906738 12.0
51.189231602 50.8611526489258 13.0
50.818993875 44.5245780944824 14.0
0 0.163574814796448 15.0
0 -0.0275288224220276 16.0
36.602084839 35.5948600769043 17.0
0 0.136385083198547 18.0
0 -0.0227826237678528 19.0
0 0.0752832889556885 20.0
0 0.0909084677696228 21.0
0 0.170906662940979 22.0
34.919519712 32.7834281921387 23.0
0 0.0174605846405029 24.0
0 0.1827552318573 25.0
0 -0.0783933401107788 26.0
0 -0.0388131737709045 27.0
0 0.159814059734344 28.0
0 0.000677406787872314 29.0
0 -0.0945221781730652 30.0
0 0.00681829452514648 31.0
0 -0.0667389035224915 32.0
0 -0.143369436264038 33.0
0 0.413572460412979 34.0
0 -0.105859041213989 35.0
0 0.219612956047058 36.0
0 -0.0345326662063599 37.0
0 0.0248743295669556 38.0
0 -0.243452072143555 39.0
0 0.572260677814484 40.0
0 -0.166179656982422 41.0
0 0.136215806007385 42.0
300.74704362 297.877166748047 43.0
0 0.24708878993988 44.0
0 0.0917093753814697 45.0
0 0.32385790348053 46.0
0 -0.800535202026367 47.0
0 -0.00144976377487183 48.0
0 0.505024671554565 49.0
0.082471374992 -0.0125905275344849 50.0
1.6308822718 1.84618973731995 51.0
0 -0.169321954250336 52.0
0 -0.167687892913818 53.0
0 -0.109275102615356 54.0
0 0.0847225785255432 55.0
0 0.234977126121521 56.0
0 -0.12809556722641 57.0
0 0.162918150424957 58.0
0 0.245717167854309 59.0
0 -0.133099555969238 60.0
0 0.109990239143372 61.0
0 0.354668080806732 62.0
294.20739513 252.339813232422 0.0
0 -0.156223237514496 1.0
0 0.221475541591644 2.0
0 -0.145871758460999 3.0
0 -0.125646710395813 4.0
0 0.0315589308738708 5.0
0 0.954409241676331 6.0
0 -0.299631595611572 7.0
0 0.990703344345093 8.0
0 0.551871299743652 9.0
0 -0.680332660675049 10.0
0 0.297817349433899 11.0
0 -0.1541907787323 12.0
81.741061226 79.6776504516602 13.0
18.484079879 13.9713306427002 14.0
0 -0.0157819986343384 15.0
0 0.172326624393463 16.0
30.218778637 35.095043182373 17.0
0 0.150816202163696 18.0
0 -0.0240007638931274 19.0
0 0.124103903770447 20.0
0 0.0263160467147827 21.0
0 -0.118295907974243 22.0
40.358921969 40.2552375793457 23.0
0 0.0529035925865173 24.0
0 0.866716802120209 25.0
0 0.0111809968948364 26.0
0 0.237689197063446 27.0
0 0.81276524066925 28.0
0 0.319161176681519 29.0
0 -0.0554189682006836 30.0
0 0.0134718418121338 31.0
0 0.096782922744751 32.0
0 -0.059209406375885 33.0
0 -0.138409554958344 34.0
0 -0.0816177725791931 35.0
0 -0.663786888122559 36.0
0 0.0269991159439087 37.0
0 -0.0192473530769348 38.0
0 -0.0997456312179565 39.0
0 0.45869255065918 40.0
0 -0.103218913078308 41.0
0 0.0936105251312256 42.0
275.13017626 295.83203125 43.0
0 0.123294174671173 44.0
0 0.267283976078033 45.0
0 0.0866326689720154 46.0
0 0.2803915143013 47.0
0 -0.642404794692993 48.0
0 0.485471844673157 49.0
0.086390391679 -0.025978147983551 50.0
1.5119311696 1.71349263191223 51.0
0 0.213996708393097 52.0
0 -0.285183072090149 53.0
0 -0.166205704212189 54.0
0 0.17035186290741 55.0
0 0.0364644527435303 56.0
0 -0.00477856397628784 57.0
0 0.0739542841911316 58.0
0 0.324857622385025 59.0
0 -0.228817939758301 60.0
0 0.181682169437408 61.0
0 0.519321501255035 62.0
244.8816621 253.461959838867 0.0
0 -0.225815057754517 1.0
0 -0.26116156578064 2.0
0 0.386329919099808 3.0
0 -0.0640917420387268 4.0
0 0.205464661121368 5.0
0 0.163295745849609 6.0
0 -0.107213377952576 7.0
0 0.01125168800354 8.0
0 0.200768947601318 9.0
0 0.26130610704422 10.0
0 -0.226844787597656 11.0
0 -0.0625455975532532 12.0
67.759926245 68.0754318237305 13.0
21.004208216 30.1791019439697 14.0
0 -0.0546590089797974 15.0
0 0.0449234843254089 16.0
26.933622107 31.1059741973877 17.0
0 0.053659200668335 18.0
0 -0.0136678218841553 19.0
0 0.144498229026794 20.0
0 0.0706049203872681 21.0
0 0.228662848472595 22.0
37.033500591 37.6609344482422 23.0
0 0.160035490989685 24.0
0 0.958322703838348 25.0
0 -0.114990472793579 26.0
0 -0.0422834753990173 27.0
0 0.314906924962997 28.0
0 -0.0974537134170532 29.0
0 0.051129937171936 30.0
0 -0.0158064365386963 31.0
0 0.00828665494918823 32.0
0 -0.200426399707794 33.0
0 -0.244712233543396 34.0
0 -0.0754041075706482 35.0
0 -0.053192138671875 36.0
0 -0.0130141973495483 37.0
0 0.184333980083466 38.0
0 -0.122619390487671 39.0
0 0.571812331676483 40.0
0 -0.129763424396515 41.0
0 0.165760040283203 42.0
305.32437519 298.249542236328 43.0
0 0.222652792930603 44.0
0 0.163079023361206 45.0
0 0.20455539226532 46.0
0 0.0296077728271484 47.0
0 0.281446516513824 48.0
0 0.43959254026413 49.0
0.081088773646 0.0327457785606384 50.0
1.8786827144 1.82251298427582 51.0
0 0.11351490020752 52.0
0 -0.203322529792786 53.0
0 0.109865665435791 54.0
0 0.290244340896606 55.0
0 0.123090028762817 56.0
0 0.20415598154068 57.0
0 0.10445237159729 58.0
0 -0.0839852094650269 59.0
0 -0.0125368237495422 60.0
0 -0.0816320776939392 61.0
0 -0.0648619532585144 62.0
237.13006186 252.886291503906 0.0
0 -0.0848995447158813 1.0
0 -0.121466338634491 2.0
0 0.051875114440918 3.0
0 -0.327617406845093 4.0
0 -0.12622344493866 5.0
0 0.16801929473877 6.0
0 -0.231813311576843 7.0
0 0.157147943973541 8.0
0 -0.461236476898193 9.0
0 -0.148344933986664 10.0
0 0.066436767578125 11.0
0 -0.0770432353019714 12.0
54.571975238 54.6387519836426 13.0
46.119781581 43.8485832214355 14.0
0 0.0224810838699341 15.0
0 -0.0648859143257141 16.0
36.110648646 35.1663436889648 17.0
0 0.0786892771720886 18.0
0 -0.0448417067527771 19.0
0 0.0319797396659851 20.0
0 0.245661735534668 21.0
0 0.00519478321075439 22.0
27.160436254 25.6482372283936 23.0
0 0.0384588241577148 24.0
0 0.19868677854538 25.0
0 -0.0186606645584106 26.0
0 -0.367127895355225 27.0
0 0.142506659030914 28.0
0 0.0536057353019714 29.0
0 0.156085133552551 30.0
0 -0.0448378324508667 31.0
0 0.00702333450317383 32.0
0 -0.179294168949127 33.0
0 0.421874046325684 34.0
0 -0.149944961071014 35.0
0 0.369877874851227 36.0
0 -0.0802329778671265 37.0
0 0.198801219463348 38.0
0 -0.257149696350098 39.0
0 0.207789778709412 40.0
0 -0.191454231739044 41.0
0 -0.0605731010437012 42.0
298.08076804 297.854949951172 43.0
0 0.0439335703849792 44.0
0 0.226405322551727 45.0
0 0.203537940979004 46.0
0 0.211491167545319 47.0
0 0.398864924907684 48.0
0 0.409402817487717 49.0
0.098746881887 0.0836112499237061 50.0
1.9460942729 1.83544874191284 51.0
0 0.247270941734314 52.0
0 0.180568695068359 53.0
0 0.210241675376892 54.0
0 0.0191637277603149 55.0
0 -0.031629741191864 56.0
0 0.125664830207825 57.0
0 0.355715781450272 58.0
0 0.0600398778915405 59.0
0 0.0958543419837952 60.0
0 0.105861306190491 61.0
0 0.210977494716644 62.0
234.65716883 252.307266235352 0.0
0 -0.186114430427551 1.0
0 -0.198191463947296 2.0
0 -0.967500925064087 3.0
0 -0.357810854911804 4.0
0 -0.540976285934448 5.0
0 0.00110667943954468 6.0
0 0.0719636082649231 7.0
0 0.0293369293212891 8.0
0 -0.181886851787567 9.0
0 0.0737130641937256 10.0
0 0.059022843837738 11.0
0 0.0757527351379395 12.0
47.298295101 49.3966903686523 13.0
50.720897389 48.8740158081055 14.0
0 0.0240658521652222 15.0
0 0.0585653781890869 16.0
35.364516817 37.6231994628906 17.0
0 0.181120276451111 18.0
0 -0.00983071327209473 19.0
0 0.0099109411239624 20.0
0 -0.0151741504669189 21.0
0 0.244000196456909 22.0
32.200745388 30.6189212799072 23.0
0 0.0372344255447388 24.0
0 0.454819321632385 25.0
0 -0.0723319053649902 26.0
0 -0.169192552566528 27.0
0 0.211048245429993 28.0
0 0.24964702129364 29.0
0 -0.000205039978027344 30.0
0 0.177644371986389 31.0
0 0.146607041358948 32.0
0 -0.192201197147369 33.0
0 -0.12161922454834 34.0
0 0.213761448860168 35.0
0 0.249873995780945 36.0
0 -0.000201940536499023 37.0
0 0.904877901077271 38.0
0 -0.532735347747803 39.0
0 -0.866650581359863 40.0
0 -0.238273739814758 41.0
0 0.324254155158997 42.0
307.96706574 296.193908691406 43.0
0 0.0768074989318848 44.0
0 0.152096748352051 45.0
0 0.231215059757233 46.0
0 0.173762261867523 47.0
0 0.373950123786926 48.0
0 0.377251803874969 49.0
0.087987982218 0.0627807974815369 50.0
1.344756985 1.86049818992615 51.0
0 -0.180096685886383 52.0
0 -0.229881167411804 53.0
0 0.292389214038849 54.0
0 0.254191994667053 55.0
0 -0.100039958953857 56.0
0 0.204158842563629 57.0
0 0.0879712700843811 58.0
0 0.409224778413773 59.0
0 -0.117846071720123 60.0
0 0.214787542819977 61.0
0 0.428520262241364 62.0
258.88237442 250.205520629883 0.0
0 -0.283167362213135 1.0
0 0.0925695896148682 2.0
0 -0.0753230452537537 3.0
0 -0.0236643552780151 4.0
0 0.357712507247925 5.0
0 0.405344426631927 6.0
0 0.0594484806060791 7.0
0 0.233876943588257 8.0
0 0.377162337303162 9.0
0 -0.460518836975098 10.0
0 0.785000562667847 11.0
0 -0.128098011016846 12.0
47.88675572 44.248592376709 13.0
47.92879095 36.4190368652344 14.0
0 -0.203172922134399 15.0
0 0.569436132907867 16.0
32.812988538 39.4369201660156 17.0
0 0.129954874515533 18.0
0 0.0113204717636108 19.0
0 -0.0097353458404541 20.0
0 0.104063272476196 21.0
0 0.350535273551941 22.0
34.549200188 32.720832824707 23.0
0 0.0902093648910522 24.0
5.4426552722 1.14997029304504 25.0
0 -0.194797217845917 26.0
0 -0.198428750038147 27.0
0 -0.364623546600342 28.0
0 0.161784052848816 29.0
0 -0.0643402338027954 30.0
0 0.329527974128723 31.0
0 0.179741501808167 32.0
0 0.210670173168182 33.0
0 -0.861577033996582 34.0
0 -0.208596050739288 35.0
0 -0.647321701049805 36.0
0 0.251182913780212 37.0
0 -0.334208250045776 38.0
0 -0.165485978126526 39.0
0 0.340405136346817 40.0
0 -0.130247890949249 41.0
0 0.132088303565979 42.0
295.43668419 291.505218505859 43.0
0 0.13068687915802 44.0
0 0.2165487408638 45.0
0 0.0498578548431396 46.0
0 0.0802938342094421 47.0
0 0.297203540802002 48.0
0 0.32032498717308 49.0
0.081433129384 -0.0237289071083069 50.0
1.7621144139 1.86650991439819 51.0
0 -0.169402360916138 52.0
0 -0.406723856925964 53.0
0 -0.011574923992157 54.0
0 0.080244779586792 55.0
0 0.246450304985046 56.0
0 -0.24261462688446 57.0
0 0.342097878456116 58.0
0 0.363369822502136 59.0
0 -0.190888643264771 60.0
0 0.114581108093262 61.0
0 0.467734456062317 62.0
325.45467795 253.708648681641 0.0
0 -0.014865517616272 1.0
0 0.218698441982269 2.0
0 0.164007008075714 3.0
0 -0.286085963249207 4.0
0 0.00455355644226074 5.0
0 1.90574812889099 6.0
0 0.0118443965911865 7.0
0 -0.216693758964539 8.0
0 1.2585117816925 9.0
0 0.0274390578269958 10.0
0 -0.0530655384063721 11.0
0 -0.409278035163879 12.0
52.444244042 53.3222198486328 13.0
34.823352652 36.3557929992676 14.0
0 0.0318046808242798 15.0
0 -0.192082285881042 16.0
27.263400709 32.6186485290527 17.0
0 0.135510802268982 18.0
0 0.00588977336883545 19.0
0 0.165952146053314 20.0
0 -0.060826301574707 21.0
0 0.638455927371979 22.0
39.698222495 40.4324531555176 23.0
0 0.121168553829193 24.0
0 1.07581126689911 25.0
0 0.245938777923584 26.0
0 -0.0140244960784912 27.0
0 0.914740741252899 28.0
0 0.115713536739349 29.0
0 0.0316348075866699 30.0
0 0.316689252853394 31.0
0 -0.0587207078933716 32.0
0 -0.202010035514832 33.0
0 0.237668693065643 34.0
0 0.26727569103241 35.0
0 0.406331807374954 36.0
0 0.0901980400085449 37.0
0 -0.49302613735199 38.0
0 -0.252469062805176 39.0
0 -0.112860023975372 40.0
0 -0.267296195030212 41.0
0 0.152458429336548 42.0
304.23807588 296.394866943359 43.0
0 0.139127373695374 44.0
0 0.0408535003662109 45.0
0 0.0360260009765625 46.0
0 0.23692786693573 47.0
0 0.167557299137115 48.0
0 0.446298062801361 49.0
0.093613550149 0.0413297414779663 50.0
2.4171198018 1.78149211406708 51.0
0 0.0659249424934387 52.0
0 0.60171914100647 53.0
0 0.250316143035889 54.0
0 -0.0667481422424316 55.0
0 -0.000925600528717041 56.0
0 0.525961875915527 57.0
0 0.12195611000061 58.0
0 0.168942093849182 59.0
0 -0.179849028587341 60.0
0 0.032168447971344 61.0
0 0.138322651386261 62.0
254.03495808 253.089157104492 0.0
0 -0.0420236587524414 1.0
0 0.131552815437317 2.0
0 0.423053979873657 3.0
0 -0.063179075717926 4.0
0 0.150690376758575 5.0
0 0.0373439192771912 6.0
0 0.065099835395813 7.0
0 -0.129973411560059 8.0
0 0.105831801891327 9.0
0 0.0351773500442505 10.0
0 -0.0589033961296082 11.0
0 -0.0223184823989868 12.0
53.61746442 53.5341415405273 13.0
47.018244347 41.2431907653809 14.0
0 -0.0132001638412476 15.0
0 0.0551052689552307 16.0
39.013527275 35.2955780029297 17.0
0 0.134498834609985 18.0
0 -0.0211293697357178 19.0
0 0.0404422283172607 20.0
0 0.0837876796722412 21.0
0 0.176577270030975 22.0
36.303448129 35.3374290466309 23.0
0 0.063978374004364 24.0
0 0.569972693920135 25.0
0 0.0331902503967285 26.0
0 -0.012936532497406 27.0
0 -0.105695366859436 28.0
0 -0.114149272441864 29.0
0 0.00149470567703247 30.0
0 -0.0850851535797119 31.0
0 -0.0528688430786133 32.0
0 0.0469585657119751 33.0
0 0.0653167366981506 34.0
0 0.144357979297638 35.0
0 0.338838309049606 36.0
0 0.000526905059814453 37.0
0 0.146313667297363 38.0
0 -0.242489337921143 39.0
0 0.50560998916626 40.0
0 -0.113579630851746 41.0
0 -0.0160410404205322 42.0
291.74951304 296.366180419922 43.0
0 0.0105822086334229 44.0
0 0.120028555393219 45.0
0 0.166113376617432 46.0
0 -0.248101234436035 47.0
0 0.37859508395195 48.0
0.49257156538 0.601609349250793 49.0
0.08159813432 0.0324591994285583 50.0
2.2415451044 1.83500492572784 51.0
0 0.0187391638755798 52.0
0 -0.136438131332397 53.0
0 -0.0418234467506409 54.0
0 0.0121353268623352 55.0
0 0.291621148586273 56.0
0 0.235959768295288 57.0
0 0.296790242195129 58.0
0 0.106765806674957 59.0
0 -0.190539717674255 60.0
0 0.0310131311416626 61.0
0 0.18313604593277 62.0
236.04417897 254.212341308594 0.0
0 -0.115665912628174 1.0
0 -0.136774957180023 2.0
0 -0.113116264343262 3.0
0 -0.281330823898315 4.0
0 -0.282167315483093 5.0
0 0.0719603300094604 6.0
0 0.0815449953079224 7.0
0 0.593609094619751 8.0
0 0.0180985331535339 9.0
0 0.131239950656891 10.0
0 -0.0728611946105957 11.0
0 -0.0281594395637512 12.0
60.331493039 61.2652206420898 13.0
35.566921063 38.2901725769043 14.0
0 -0.0637936592102051 15.0
0 -0.0488249659538269 16.0
33.109169994 36.2560958862305 17.0
0 0.130741715431213 18.0
0 -0.0750325918197632 19.0
0 0.139509558677673 20.0
0 0.0187314748764038 21.0
0 0.115022540092468 22.0
36.1682862 35.5900192260742 23.0
0 0.0322509407997131 24.0
0 0.307739704847336 25.0
0 -0.0669097900390625 26.0
0 -0.00622081756591797 27.0
0 -0.0587713718414307 28.0
0 -0.143835186958313 29.0
0 0.143211543560028 30.0
0 0.172524690628052 31.0
0 0.128750324249268 32.0
0 -0.206707715988159 33.0
0 0.205782115459442 34.0
0 0.174591481685638 35.0
0 0.294306755065918 36.0
0 0.159024000167847 37.0
0 -0.0151658058166504 38.0
0 -0.233645439147949 39.0
0 0.509316682815552 40.0
0 -0.165271401405334 41.0
0 0.0609995126724243 42.0
295.78077173 296.490020751953 43.0
0 0.138303399085999 44.0
0 0.0867120027542114 45.0
0 0.301074087619781 46.0
0 0.022354781627655 47.0
0 -0.0937680006027222 48.0
0 0.308093518018723 49.0
0.099450149283 -0.0431438088417053 50.0
1.9770940485 1.89175343513489 51.0
0 -0.132169604301453 52.0
0 -0.129721343517303 53.0
0 -0.154728770256042 54.0
0 0.00100654363632202 55.0
0 0.0791572332382202 56.0
0 0.0805548429489136 57.0
0 0.543730318546295 58.0
0 0.175719797611237 59.0
0 -0.23590612411499 60.0
0 0.0816424489021301 61.0
0 0.260209977626801 62.0
257.03468776 252.706741333008 0.0
0 -0.272337436676025 1.0
0 -0.0723505616188049 2.0
0 0.385393351316452 3.0
0 0.116328358650208 4.0
0 0.13630622625351 5.0
0 -0.0895062685012817 6.0
0 0.114681601524353 7.0
0 -0.23680305480957 8.0
0 -0.127556264400482 9.0
0 0.200243473052979 10.0
0 -0.311160206794739 11.0
0 -0.148229479789734 12.0
75.561863144 76.3462600708008 13.0
33.25320531 27.3087425231934 14.0
0 0.0358158349990845 15.0
0 -0.0892953872680664 16.0
36.912263212 34.6863098144531 17.0
0 0.081612765789032 18.0
0 -0.0380219221115112 19.0
0 0.133702158927917 20.0
0 -0.0109906792640686 21.0
0 0.279025673866272 22.0
35.852809871 35.164119720459 23.0
0 0.112009882926941 24.0
0 0.994986057281494 25.0
0 0.0117002725601196 26.0
0 -0.0933859944343567 27.0
0 0.613089501857758 28.0
0 0.062796950340271 29.0
0 0.0111918449401855 30.0
0 -0.0170766115188599 31.0
0 0.0547515749931335 32.0
0 0.00668799877166748 33.0
0 0.0124194025993347 34.0
0 0.849292397499084 35.0
0 0.586494922637939 36.0
0 -0.0596257448196411 37.0
0 0.281501233577728 38.0
0 -0.172963857650757 39.0
0 0.328730463981628 40.0
0 -0.144214630126953 41.0
0 0.0746440887451172 42.0
291.67765658 297.083251953125 43.0
0 0.109199643135071 44.0
0 0.0215743184089661 45.0
0 0.148206233978271 46.0
0 0.159317016601562 47.0
0 -0.0423120260238647 48.0
0 0.375485122203827 49.0
0.07345652707 0.0402409434318542 50.0
1.7718718791 1.82399809360504 51.0
0 -0.241221904754639 52.0
0 0.18921422958374 53.0
0 0.303176164627075 54.0
0 0.319256484508514 55.0
0 -0.0821167230606079 56.0
0 0.596494436264038 57.0
0 -0.00742626190185547 58.0
0 0.32452791929245 59.0
0 -0.209921360015869 60.0
0 0.100577056407928 61.0
0 0.24024772644043 62.0
247.14978929 254.156402587891 0.0
0 0.0200591087341309 1.0
0 0.129316568374634 2.0
0 0.351919770240784 3.0
0 -0.0246009826660156 4.0
0 -0.233906507492065 5.0
0 0.00174552202224731 6.0
0 0.00413143634796143 7.0
0 0.0190335512161255 8.0
0 0.090074360370636 9.0
0 0.404014229774475 10.0
0 -0.10701459646225 11.0
0 -0.16885232925415 12.0
70.269927493 71.7211227416992 13.0
26.019537196 27.9861087799072 14.0
0 -0.0436397790908813 15.0
0 -0.0603583455085754 16.0
37.112595429 36.4846115112305 17.0
0 0.151346564292908 18.0
0 -0.0702863931655884 19.0
0 0.0535247325897217 20.0
0 0.100415825843811 21.0
0 0.24278724193573 22.0
34.357191016 33.6168746948242 23.0
0 0.0716332197189331 24.0
0 0.953931927680969 25.0
0 0.117091059684753 26.0
0 -0.184347629547119 27.0
0 0.692201614379883 28.0
0 -0.0869784951210022 29.0
0 0.0458057522773743 30.0
0 -0.359439134597778 31.0
0 -0.0971336960792542 32.0
0 -0.240729808807373 33.0
0 0.179937124252319 34.0
0 0.158576786518097 35.0
0 0.223824262619019 36.0
0 0.0554877519607544 37.0
0 -0.0935039520263672 38.0
0 -0.379364728927612 39.0
0 0.513782501220703 40.0
0 -0.328165173530579 41.0
0 0.0856342315673828 42.0
304.49214157 294.625732421875 43.0
0 0.0845980644226074 44.0
0 -0.0901025533676147 45.0
0 0.102195203304291 46.0
0 -0.491691827774048 47.0
0 0.342731833457947 48.0
0 0.317893624305725 49.0
0.086462505248 0.0475846529006958 50.0
2.0149638687 1.90391767024994 51.0
0 -0.143419981002808 52.0
0 -0.263319969177246 53.0
0 -0.137945771217346 54.0
0 0.288029193878174 55.0
0 0.323690235614777 56.0
0 0.12316906452179 57.0
0 0.582615494728088 58.0
0 -0.585580587387085 59.0
0 0.20170921087265 60.0
0 0.010549783706665 61.0
0 0.1068274974823 62.0
248.40144179 252.454940795898 0.0
0 -0.162933588027954 1.0
0 -0.15013313293457 2.0
0 0.479044735431671 3.0
0 -0.224414706230164 4.0
0 -0.162474632263184 5.0
0 -0.0188630819320679 6.0
0 -0.0893136858940125 7.0
0 -0.0884996652603149 8.0
0 0.559020638465881 9.0
0 0.411131650209427 10.0
0 -0.0494168400764465 11.0
0 -0.151504755020142 12.0
72.444532492 72.66162109375 13.0
19.209369275 28.291223526001 14.0
0 -0.0625582933425903 15.0
0 -0.138502836227417 16.0
28.940090047 30.8867359161377 17.0
0 0.016624927520752 18.0
0 -0.00448548793792725 19.0
0 0.0497989654541016 20.0
0 0.180366218090057 21.0
0 0.0894631147384644 22.0
28.966577482 29.0582180023193 23.0
0 0.125643134117126 24.0
0 0.846071839332581 25.0
0 0.0595062375068665 26.0
0 -0.148127138614655 27.0
0 0.0319359302520752 28.0
0 -0.150238454341888 29.0
0 0.0948036909103394 30.0
0 0.144734025001526 31.0
0 0.0858474969863892 32.0
0 -0.0208120346069336 33.0
0 0.508882701396942 34.0
0 -0.0753574967384338 35.0
0 0.118926048278809 36.0
0 0.177093684673309 37.0
0 0.331029653549194 38.0
0 -0.283434748649597 39.0
0 -0.804253578186035 40.0
0 -0.207492828369141 41.0
0 0.323584318161011 42.0
294.08510851 295.207427978516 43.0
0 0.0443247556686401 44.0
0 0.215345203876495 45.0
0 0.173052787780762 46.0
0 -0.0850932002067566 47.0
0 0.398545354604721 48.0
0 0.5142782330513 49.0
0.088052253603 0.0476645231246948 50.0
1.695596824 1.85903215408325 51.0
0 -0.107960939407349 52.0
0 -0.554010391235352 53.0
0 -0.00167346000671387 54.0
0 0.0527569055557251 55.0
0 -0.0896478891372681 56.0
0 -0.00230652093887329 57.0
0 0.213223159313202 58.0
0 0.0601829886436462 59.0
0 0.0969946980476379 60.0
0 0.0985734462738037 61.0
0 0.291255474090576 62.0
272.30572715 252.452926635742 0.0
0 -0.310413002967834 1.0
0 -0.0136634111404419 2.0
0 0.290325105190277 3.0
0 0.0628166198730469 4.0
0 0.196515023708344 5.0
0 -0.0754789710044861 6.0
0 0.335452169179916 7.0
0 0.275915741920471 8.0
0 -0.0558592081069946 9.0
0 0.301729321479797 10.0
0 -0.118291258811951 11.0
0 -0.10564136505127 12.0
62.622463872 62.1500053405762 13.0
33.431457743 34.6609344482422 14.0
0 -0.0152592658996582 15.0
0 0.0402604341506958 16.0
33.761975149 36.1546020507812 17.0
0 0.161586403846741 18.0
0 -0.050492525100708 19.0
0 0.119065165519714 20.0
0 0.0653394460678101 21.0
0 0.00604701042175293 22.0
36.881935514 36.9976501464844 23.0
0 0.0519341826438904 24.0
0 0.413292527198792 25.0
0 -0.085541307926178 26.0
0 -0.0629656314849854 27.0
0 0.215010225772858 28.0
0 -0.0487948656082153 29.0
0 0.0785562992095947 30.0
0 0.0162540674209595 31.0
0 0.0619134306907654 32.0
0 0.0518652200698853 33.0
0 0.0329532623291016 34.0
0 -0.221155047416687 35.0
0 0.232643604278564 36.0
0 -0.0469623208045959 37.0
0 0.111846089363098 38.0
0 -0.345443248748779 39.0
0 0.312778890132904 40.0
0 -0.132970094680786 41.0
0 0.0258907079696655 42.0
297.01918654 297.496826171875 43.0
0 0.0841463208198547 44.0
0 0.169339597225189 45.0
0 0.18846595287323 46.0
0 0.310099303722382 47.0
0 0.348248898983002 48.0
0 0.493732243776321 49.0
0.097947078953 0.0261877179145813 50.0
1.9221338643 1.83073699474335 51.0
0 0.0295971632003784 52.0
0 0.781033933162689 53.0
0 0.205945730209351 54.0
0 0.0988284945487976 55.0
0 -0.144081115722656 56.0
0 -0.0900754928588867 57.0
0 0.292409539222717 58.0
0 0.162755489349365 59.0
0 -0.0556824207305908 60.0
0 0.0667188167572021 61.0
0 0.316286385059357 62.0
266.25209636 254.026184082031 0.0
0 -0.317532777786255 1.0
0 0.235754251480103 2.0
0 -0.15729284286499 3.0
0 -0.480300426483154 4.0
0 0.563078999519348 5.0
0 -0.0223509669303894 6.0
0 0.0985209941864014 7.0
0 0.0349083542823792 8.0
0 0.43408989906311 9.0
0 0.383630961179733 10.0
0 0.153567790985107 11.0
0 -0.116633534431458 12.0
71.704330614 69.4461441040039 13.0
35.84333892 32.278751373291 14.0
0 -0.156803369522095 15.0
0 -0.123499274253845 16.0
42.220579872 38.3909568786621 17.0
0 0.147064089775085 18.0
0 0.0844789743423462 19.0
0 0.0971226692199707 20.0
0 -0.134560763835907 21.0
0 0.986598491668701 22.0
42.791119017 39.3854331970215 23.0
0 0.00865888595581055 24.0
0 0.578337073326111 25.0
0 0.0152246952056885 26.0
0 -0.00503599643707275 27.0
0 -0.351071238517761 28.0
0 -0.119283556938171 29.0
0 0.00865727663040161 30.0
0 0.51840615272522 31.0
0 -0.0381242632865906 32.0
0 0.0884379148483276 33.0
0 0.349458932876587 34.0
0 -0.0999927520751953 35.0
0 0.397150337696075 36.0
0 0.0561075210571289 37.0
0 1.05076503753662 38.0
0 -0.651462078094482 39.0
0 -1.196702003479 40.0
0 -0.310074329376221 41.0
0 0.293616771697998 42.0
282.79239019 294.755950927734 43.0
0 0.137005150318146 44.0
0 0.266633868217468 45.0
0 0.258529961109161 46.0
0 -0.271057844161987 47.0
0 0.317382872104645 48.0
0.83089773726 0.657331705093384 49.0
0.10567670745 0.12128347158432 50.0
2.1006475019 1.89752388000488 51.0
0 0.0213063359260559 52.0
0 -0.177014827728271 53.0
0 0.139683067798615 54.0
0 0.240155637264252 55.0
0 0.169786691665649 56.0
0 -0.0943695902824402 57.0
0 -0.03900545835495 58.0
0 0.341291010379791 59.0
0 0.036091148853302 60.0
0 0.0585950613021851 61.0
0 -0.00787341594696045 62.0
273.27337162 252.380340576172 0.0
0 -0.26910388469696 1.0
0 0.0317623615264893 2.0
0 0.30753755569458 3.0
0 -0.0943278074264526 4.0
0 0.18004310131073 5.0
0 0.0665937066078186 6.0
0 0.0801664590835571 7.0
0 -0.184373915195465 8.0
0 -0.0421224236488342 9.0
0 0.0789106488227844 10.0
0 0.0257987976074219 11.0
0 -0.0466145873069763 12.0
66.652338872 68.3815383911133 13.0
36.787301651 30.1428928375244 14.0
0 -0.0164057016372681 15.0
0 0.109631597995758 16.0
34.949145668 36.3326034545898 17.0
0 0.241200029850006 18.0
0 -0.0609115958213806 19.0
0 0.0514259338378906 20.0
0 0.125638484954834 21.0
0 0.0821912288665771 22.0
28.85372796 27.2442569732666 23.0
0 0.068597674369812 24.0
0 0.382689297199249 25.0
0 -0.0249491930007935 26.0
0 -0.136290788650513 27.0
0 0.208163857460022 28.0
0 -0.104582965373993 29.0
0 0.0380705595016479 30.0
0 0.168067097663879 31.0
0 0.213468253612518 32.0
0 0.111979126930237 33.0
0 0.237673282623291 34.0
0 -0.0610293745994568 35.0
0 0.380452334880829 36.0
0 -0.0221641659736633 37.0
0 0.214149951934814 38.0
0 -0.330199360847473 39.0
0 0.0452398061752319 40.0
0 -0.210960745811462 41.0
0 0.0224300622940063 42.0
283.22351272 297.182220458984 43.0
0 0.244527816772461 44.0
0 0.0467572212219238 45.0
0 0.15997052192688 46.0
0 0.0610837936401367 47.0
0 0.26104748249054 48.0
0 0.398691833019257 49.0
0.076424670483 0.0328945517539978 50.0
2.2403455676 1.83643364906311 51.0
0 0.0477237701416016 52.0
0 -0.196179151535034 53.0
0 0.185482919216156 54.0
0 0.250193357467651 55.0
0 -0.0814496278762817 56.0
0 0.217949211597443 57.0
0 0.583078324794769 58.0
0 0.204335808753967 59.0
0 -0.22876501083374 60.0
0 0.035517156124115 61.0
0 0.155430555343628 62.0
240.19196184 253.663497924805 0.0
0 0.293170094490051 1.0
0 -0.203656077384949 2.0
0 0.325283765792847 3.0
0 0.0830570459365845 4.0
0 0.00975435972213745 5.0
0 -0.058385968208313 6.0
0 0.0375818610191345 7.0
0 -0.0029597282409668 8.0
0 0.0532398223876953 9.0
0 -0.266541957855225 10.0
0 -0.122116208076477 11.0
0 -0.0818140506744385 12.0
68.300996291 67.2242584228516 13.0
24.313683288 33.3562698364258 14.0
0 0.406067609786987 15.0
0 -0.0542306900024414 16.0
37.184641099 32.9010963439941 17.0
0 0.0967662334442139 18.0
0 -0.00624525547027588 19.0
0 0.0292212963104248 20.0
0 0.139055252075195 21.0
0 0.12445741891861 22.0
37.381731222 36.1385078430176 23.0
0 0.102311491966248 24.0
0 0.57256543636322 25.0
0 0.230838298797607 26.0
0 -0.00440472364425659 27.0
0 0.842372894287109 28.0
0 0.00799334049224854 29.0
0 -0.0604484677314758 30.0
0 0.0306650400161743 31.0
0 0.128605842590332 32.0
0 0.160938084125519 33.0
0 -0.0769309401512146 34.0
0 -0.124385714530945 35.0
0 -0.381974816322327 36.0
0 -0.066010057926178 37.0
0 0.303129822015762 38.0
0 -0.18445611000061 39.0
0 0.608194589614868 40.0
0 -0.165513277053833 41.0
0 -0.0838754177093506 42.0
284.07177981 296.619201660156 43.0
0 0.0847325325012207 44.0
0 0.328113973140717 45.0
0 0.246591448783875 46.0
0 0.126944780349731 47.0
0 -0.0136567950248718 48.0
0 0.128353655338287 49.0
0.083536194658 0.0557271838188171 50.0
1.4643799391 1.8480829000473 51.0
0 -0.157508373260498 52.0
0 -0.0235605835914612 53.0
0 0.319137215614319 54.0
0 0.0372898578643799 55.0
0 -0.122346580028534 56.0
0 -0.0441799163818359 57.0
0 0.0906268954277039 58.0
0 -0.37312376499176 59.0
0 1.33865880966187 60.0
0 0.0809338688850403 61.0
0 0.104349911212921 62.0
279.91525713 254.615829467773 0.0
0 -0.448813080787659 1.0
0 0.0600100755691528 2.0
0 0.097326397895813 3.0
0 -0.0415298342704773 4.0
0 0.220529854297638 5.0
0 -0.388208866119385 6.0
0 0.361099064350128 7.0
0 -0.118765115737915 8.0
0 0.268142342567444 9.0
0 -0.462346911430359 10.0
0 -0.134664714336395 11.0
0 -0.06927490234375 12.0
61.099661207 60.1774482727051 13.0
44.151335023 36.6876525878906 14.0
0 -0.10062050819397 15.0
0 -0.0642369389533997 16.0
29.401995711 31.9315395355225 17.0
0 0.0218967795372009 18.0
0 0.0134546160697937 19.0
0 0.140779912471771 20.0
0 0.105172991752625 21.0
0 0.195687472820282 22.0
31.183532442 30.9790802001953 23.0
0 0.0881069302558899 24.0
16.127765537 1.16821622848511 25.0
0 0.0830973386764526 26.0
0 -0.222547769546509 27.0
0 0.59861946105957 28.0
0 0.125270903110504 29.0
0 0.0133785009384155 30.0
0 -0.229990243911743 31.0
0 0.0377378463745117 32.0
0 0.097631573677063 33.0
0 0.259231686592102 34.0
0 -0.112566769123077 35.0
0 -0.0122928619384766 36.0
0 0.11245721578598 37.0
0 0.275529742240906 38.0
0 -0.174209475517273 39.0
0 0.688672125339508 40.0
0 -0.127229452133179 41.0
0 0.0449742078781128 42.0
297.37118817 298.396636962891 43.0
0 0.0257492661476135 44.0
0 0.154038310050964 45.0
0 0.0180309414863586 46.0
0 0.191771447658539 47.0
0 0.272588670253754 48.0
0 0.457325160503387 49.0
0.06952100797 0.0265821814537048 50.0
1.6299858697 1.84859800338745 51.0
0 -0.0447781682014465 52.0
0 0.0374516248703003 53.0
0 -0.0721601247787476 54.0
0 0.134331703186035 55.0
0 -0.0180379152297974 56.0
0 0.115082025527954 57.0
0 -0.0953343510627747 58.0
0 0.0136392116546631 59.0
0 -0.110752761363983 60.0
0 0.0388877391815186 61.0
0 0.0434418320655823 62.0
299.84376297 251.734710693359 0.0
0 -0.314212322235107 1.0
0 0.0785982012748718 2.0
0 0.0330883860588074 3.0
0 -0.033208966255188 4.0
0 0.579771995544434 5.0
0 -0.225607395172119 6.0
0 0.335691869258881 7.0
0 -0.244130492210388 8.0
0 -0.0852444767951965 9.0
0 -0.0284133553504944 10.0
0 0.0818931460380554 11.0
0 -0.121963977813721 12.0
72.077804465 72.2179260253906 13.0
34.644952896 31.2544841766357 14.0
0 0.116354942321777 15.0
0 -0.0987698435783386 16.0
24.921083892 28.6380634307861 17.0
0 0.118452608585358 18.0
0 4.80413436889648e-05 19.0
0 0.0386507511138916 20.0
0 0.246889591217041 21.0
0 -0.197216868400574 22.0
25.851399123 27.1670894622803 23.0
0 0.0515910983085632 24.0
0 0.374026745557785 25.0
0 -0.0975327491760254 26.0
0 -0.174972236156464 27.0
0 0.241991519927979 28.0
0 0.0653922557830811 29.0
0 0.0703235864639282 30.0
0 -0.050891637802124 31.0
0 0.0802929997444153 32.0
0 -0.0473634004592896 33.0
0 0.333591729402542 34.0
0 0.161194920539856 35.0
0 0.33535835146904 36.0
0 0.0701148509979248 37.0
0 0.369231700897217 38.0
0 -0.54548442363739 39.0
0 -0.340373635292053 40.0
0 -0.231967449188232 41.0
0 -0.0824532508850098 42.0
302.29095183 298.607940673828 43.0
0 0.0978938341140747 44.0
0 0.169243812561035 45.0
0 0.0594854354858398 46.0
0 0.0729081034660339 47.0
0 -0.271611094474792 48.0
0 0.299022495746613 49.0
0.10358520651 0.0192687511444092 50.0
1.7617632978 1.82602262496948 51.0
0 -0.0449081659317017 52.0
0 0.256882429122925 53.0
0 -0.0282433032989502 54.0
0 0.0256661176681519 55.0
0 -0.206187844276428 56.0
0 0.067697286605835 57.0
0 0.45727676153183 58.0
0 0.365903586149216 59.0
0 -0.0334601402282715 60.0
0 0.19257515668869 61.0
0 0.494659721851349 62.0
256.63018531 253.555267333984 0.0
0 -0.0784714221954346 1.0
0 0.0448637008666992 2.0
0 0.162276566028595 3.0
0 -0.108401119709015 4.0
0 -0.0990630984306335 5.0
0 -0.0140122771263123 6.0
0 0.0841414928436279 7.0
0 0.207683742046356 8.0
0 0.131872951984406 9.0
0 0.0737529993057251 10.0
0 -0.139966666698456 11.0
0 -0.0738541483879089 12.0
56.862618449 56.7879333496094 13.0
36.920438112 33.5243797302246 14.0
0 -0.108740091323853 15.0
0 -0.0654366612434387 16.0
31.667790177 35.991340637207 17.0
0 0.121397495269775 18.0
0 -0.0460182428359985 19.0
0 0.0931388735771179 20.0
0 -0.0296534299850464 21.0
0 0.179695129394531 22.0
37.712980342 37.4760551452637 23.0
0 0.06525057554245 24.0
0 0.545989990234375 25.0
0 -0.150547087192535 26.0
0 0.111993670463562 27.0
0 -0.565147638320923 28.0
0 0.0145503878593445 29.0
0 0.115597665309906 30.0
0 -0.135332822799683 31.0
0 0.0234766006469727 32.0
0 -0.0940901041030884 33.0
0 0.107175230979919 34.0
0 0.170358538627625 35.0
0 -0.0999301075935364 36.0
0 -0.0260836482048035 37.0
0 0.107306897640228 38.0
0 -0.0675977468490601 39.0
0 0.608629584312439 40.0
0 -0.109111487865448 41.0
0 0.1108238697052 42.0
290.44649166 297.411956787109 43.0
0 0.0726208686828613 44.0
0 0.202099025249481 45.0
0 0.164703726768494 46.0
0 -0.00541681051254272 47.0
0 0.196354866027832 48.0
0 0.315142810344696 49.0
0.095250515062 0.0507065057754517 50.0
1.7892221208 1.85499596595764 51.0
0 -0.153495252132416 52.0
0 -0.116071581840515 53.0
0 0.102324187755585 54.0
0 0.0223035216331482 55.0
0 -0.162039160728455 56.0
0 -0.10801362991333 57.0
0 0.525459170341492 58.0
0 0.136252105236053 59.0
0 -0.199772655963898 60.0
0 0.153390824794769 61.0
0 0.354188680648804 62.0
243.62646505 252.695510864258 0.0
0 -0.0461516380310059 1.0
0 -0.201839566230774 2.0
0 0.316127240657806 3.0
0 -0.0681369304656982 4.0
0 0.00267177820205688 5.0
0 0.00426900386810303 6.0
0 0.0990832448005676 7.0
0 -0.674980998039246 8.0
0 0.061296820640564 9.0
0 0.0554651618003845 10.0
0 -0.343724370002747 11.0
0 -0.0686032772064209 12.0
69.219982648 68.8395156860352 13.0
22.810244942 31.992826461792 14.0
0 -0.05555260181427 15.0
0 0.0391788482666016 16.0
31.60101736 32.5996170043945 17.0
0 0.119577646255493 18.0
0 -0.0198837518692017 19.0
0 0.115794479846954 20.0
0 -0.0288138389587402 21.0
0 -0.136513590812683 22.0
39.728784695 39.8699264526367 23.0
0 0.00159949064254761 24.0
0 0.28874135017395 25.0
0 0.0534803867340088 26.0
0 0.0626888871192932 27.0
0 0.161591708660126 28.0
0 -0.0817074179649353 29.0
0 0.0353013277053833 30.0
0 0.201350927352905 31.0
0 0.0915355086326599 32.0
0 -0.007579505443573 33.0
0 0.329432487487793 34.0
0 -0.0498020648956299 35.0
0 0.387863874435425 36.0
0 -0.073853075504303 37.0
0 0.0672463178634644 38.0
0 -0.327324867248535 39.0
0 0.451078504323959 40.0
0 -0.117065727710724 41.0
0 0.0305322408676147 42.0
291.62949503 297.455017089844 43.0
0 0.00274902582168579 44.0
0 0.125016927719116 45.0
0 0.287610471248627 46.0
0 0.167662620544434 47.0
0 0.106548309326172 48.0
0 0.391727238893509 49.0
0.095931872684 -0.0356970429420471 50.0
2.2548959947 1.89457583427429 51.0
0 -0.13279801607132 52.0
0 -0.435876965522766 53.0
0 -0.012562096118927 54.0
0 0.12901645898819 55.0
0 -0.0749759674072266 56.0
0 0.303221940994263 57.0
0 0.371593058109283 58.0
0 0.223900079727173 59.0
0 -0.0556844472885132 60.0
0 0.0794357657432556 61.0
0 0.289679169654846 62.0
289.97585618 251.933959960938 0.0
0 -0.192509889602661 1.0
0 0.138224840164185 2.0
0 0.19144731760025 3.0
0 0.53535670042038 4.0
0 -0.23792552947998 5.0
0 0.605511665344238 6.0
0 0.783661365509033 7.0
0 0.992492079734802 8.0
0 -0.325461030006409 9.0
0 0.767243802547455 10.0
0 -0.276185274124146 11.0
0 -0.0423324704170227 12.0
64.908904284 65.3134841918945 13.0
21.872592549 29.7888946533203 14.0
0 -0.0448900461196899 15.0
0 -0.0418174862861633 16.0
25.787085088 31.5425071716309 17.0
0 0.0834789276123047 18.0
0 0.00218343734741211 19.0
0 0.223458826541901 20.0
0 -0.0594443678855896 21.0
0 0.281881153583527 22.0
39.69505896 39.2631225585938 23.0
0 0.0608863830566406 24.0
0 0.404518842697144 25.0
0 0.0483196973800659 26.0
0 0.144500553607941 27.0
0 0.375072181224823 28.0
0 -0.0854082107543945 29.0
0 0.032600998878479 30.0
0 0.111294507980347 31.0
0 -0.060547411441803 32.0
0 -0.410719394683838 33.0
0 0.0562806725502014 34.0
0 -0.11875969171524 35.0
0 -0.0629459023475647 36.0
0 0.10554838180542 37.0
0 0.152422308921814 38.0
0 -0.400390267372131 39.0
0 0.428186893463135 40.0
0 -0.235336899757385 41.0
0 -0.0591858625411987 42.0
294.42834998 295.288391113281 43.0
0 0.00526732206344604 44.0
0 0.185910880565643 45.0
0 0.0659184455871582 46.0
0 0.0863364338874817 47.0
0 0.271078646183014 48.0
1.6124177994 0.560993671417236 49.0
0.087749272705 0.0273091793060303 50.0
2.0730645242 1.93240141868591 51.0
0 0.145808935165405 52.0
0 0.0439899563789368 53.0
0 -0.0798193216323853 54.0
0 -0.00799155235290527 55.0
0 0.00100225210189819 56.0
0 0.0835945606231689 57.0
0 0.383970975875854 58.0
0 0.261088967323303 59.0
0 -0.103513419628143 60.0
0 0.0491542220115662 61.0
0 0.237949788570404 62.0
293.12166541 252.459991455078 0.0
0 -0.207045912742615 1.0
0 -0.00756740570068359 2.0
0 0.122380018234253 3.0
0 -0.0846917629241943 4.0
0 0.111245572566986 5.0
0 0.0994277000427246 6.0
0 0.426147878170013 7.0
0 0.400510191917419 8.0
0 0.122846961021423 9.0
0 0.173542082309723 10.0
0 -0.179660797119141 11.0
0 0.0140259265899658 12.0
59.574156399 57.5026893615723 13.0
36.741091148 33.9407806396484 14.0
0 -0.040557861328125 15.0
0 0.0497941374778748 16.0
34.454393456 34.0678215026855 17.0
0 0.11765044927597 18.0
0 -0.0373808741569519 19.0
0 0.0610759854316711 20.0
0 0.150535941123962 21.0
0 0.0440927743911743 22.0
28.591421896 28.2886905670166 23.0
0 0.128515779972076 24.0
0 1.15755498409271 25.0
0 -0.301288843154907 26.0
0 -0.215705394744873 27.0
0 0.738889992237091 28.0
0 0.130619883537292 29.0
0 -0.0822241902351379 30.0
0 0.0948548316955566 31.0
0 -0.0554324388504028 32.0
0 -0.213865041732788 33.0
0 -0.143775403499603 34.0
0 0.10702919960022 35.0
0 0.0771138668060303 36.0
0 -0.0819762349128723 37.0
0 0.162858486175537 38.0
0 -0.237085223197937 39.0
0 0.0416550636291504 40.0
0 -0.181297302246094 41.0
0 -0.0182374715805054 42.0
313.48518079 296.247589111328 43.0
0 0.0787182450294495 44.0
0 0.177811622619629 45.0
0 0.369647145271301 46.0
0 -0.729455947875977 47.0
0 0.364323526620865 48.0
0.93406215243 0.588750123977661 49.0
0.079948515492 0.0524308085441589 50.0
1.9847203711 1.91988062858582 51.0
0 0.165432751178741 52.0
0 -0.192672908306122 53.0
0 0.0641747117042542 54.0
0 0.238708317279816 55.0
0 -0.0728573203086853 56.0
0 0.427996635437012 57.0
0 0.125377416610718 58.0
0 0.222954630851746 59.0
0 0.250052332878113 60.0
0 0.104450702667236 61.0
0 0.137778580188751 62.0
226.42842874 251.305038452148 0.0
0 0.142298460006714 1.0
0 -0.0933211445808411 2.0
0 0.109888076782227 3.0
0 -0.313538789749146 4.0
0 -0.310024976730347 5.0
0 -0.0737453699111938 6.0
0 -0.314660310745239 7.0
0 0.0539311170578003 8.0
0 0.594241976737976 9.0
0 1.10035300254822 10.0
0 -0.361001253128052 11.0
0 0.0547010898590088 12.0
57.420917904 58.897891998291 13.0
31.106988255 37.5585327148438 14.0
0 0.0253310203552246 15.0
0 -0.0828702449798584 16.0
30.423663646 33.1269569396973 17.0
0 0.0612975358963013 18.0
0 -0.0216715335845947 19.0
0 0.0430304408073425 20.0
0 0.207566738128662 21.0
0 0.0844018459320068 22.0
28.300460891 25.2545490264893 23.0
0 0.105046987533569 24.0
0 0.569394469261169 25.0
0 0.0215744972229004 26.0
0 -0.257434964179993 27.0
0 0.921525657176971 28.0
0 -0.0271291732788086 29.0
0 0.0998928546905518 30.0
0 0.0245050191879272 31.0
0 0.0408859252929688 32.0
0 -0.126460015773773 33.0
0 0.450852304697037 34.0
0 -0.106100380420685 35.0
0 0.268092572689056 36.0
0 0.177039086818695 37.0
0 -0.0110588669776917 38.0
0 -0.203229188919067 39.0
0 0.0373272299766541 40.0
0 -0.103617012500763 41.0
0 0.0738363265991211 42.0
306.42735889 297.211120605469 43.0
0 0.13659143447876 44.0
0 0.128769099712372 45.0
0 0.271092712879181 46.0
0 -0.0957099199295044 47.0
0 -0.0304020643234253 48.0
0.14862723358 0.665262043476105 49.0
0.092112695056 0.0226221084594727 50.0
1.4814205284 1.76313924789429 51.0
0 0.133925437927246 52.0
0 -0.318046092987061 53.0
0 0.0923565626144409 54.0
0 0.277887403964996 55.0
0 -0.0272724628448486 56.0
0 0.388473987579346 57.0
0 0.504228591918945 58.0
0 0.156765103340149 59.0
0 -0.105229377746582 60.0
0 0.123463034629822 61.0
0 0.267230927944183 62.0
284.91632861 252.005462646484 0.0
0 -0.248194098472595 1.0
0 0.0123503804206848 2.0
0 0.0115431547164917 3.0
0 -0.175766229629517 4.0
0 -0.109041273593903 5.0
0 0.508323550224304 6.0
0 -0.138670682907104 7.0
0 0.128376245498657 8.0
0 -0.177076101303101 9.0
0 0.0573080778121948 10.0
0 0.473097532987595 11.0
0 -0.286743640899658 12.0
54.934044737 54.417797088623 13.0
37.188702488 36.0084457397461 14.0
0 -0.0673906803131104 15.0
0 -0.132529199123383 16.0
28.338153492 32.0220794677734 17.0
0 0.0331085324287415 18.0
0 -0.00601845979690552 19.0
0 0.214883327484131 20.0
0 -0.037028431892395 21.0
0 0.00551795959472656 22.0
38.643895006 35.6401557922363 23.0
0 0.0165159106254578 24.0
0 0.343840599060059 25.0
0 0.0588428974151611 26.0
0 0.000221729278564453 27.0
0 0.475565820932388 28.0
0 0.0121363997459412 29.0
0 0.0644206404685974 30.0
0 -0.153887271881104 31.0
0 0.132124602794647 32.0
0 -0.0366185307502747 33.0
0 0.267504096031189 34.0
0 -0.138555109500885 35.0
0 -0.113917589187622 36.0
0 0.112657904624939 37.0
0 0.314105927944183 38.0
0 -0.396461844444275 39.0
0 0.41261038184166 40.0
0 -0.171890497207642 41.0
0 -0.048689603805542 42.0
283.81026563 296.949768066406 43.0
0 0.0288166403770447 44.0
0 0.210381388664246 45.0
0 0.214922845363617 46.0
0 -0.0711082816123962 47.0
0 0.312166929244995 48.0
0 0.42739462852478 49.0
0.098263669205 0.017812192440033 50.0
2.1297101723 1.89294135570526 51.0
0 -0.300331115722656 52.0
0 -0.224308013916016 53.0
0 -0.102172255516052 54.0
0 0.252979099750519 55.0
0 0.265834331512451 56.0
0 0.333909779787064 57.0
0 0.52295982837677 58.0
0 0.290644228458405 59.0
0 -0.198445737361908 60.0
0 0.0971900820732117 61.0
0 0.286545515060425 62.0
267.16641274 253.384124755859 0.0
0 -0.285732507705688 1.0
0 0.0172044634819031 2.0
0 0.0237097144126892 3.0
0 0.00380140542984009 4.0
0 0.0897121429443359 5.0
0 0.34423565864563 6.0
0 0.198677003383636 7.0
0 -0.58232593536377 8.0
0 0.432108104228973 9.0
0 0.305197060108185 10.0
0 -0.168862104415894 11.0
0 -0.0879756212234497 12.0
73.622815665 74.5391845703125 13.0
12.585848687 19.629186630249 14.0
0 0.0416563749313354 15.0
0 -0.114223778247833 16.0
28.171901681 32.4276351928711 17.0
0 0.0819356441497803 18.0
0 0.0278185606002808 19.0
0 0.107799232006073 20.0
0 0.118114709854126 21.0
0 0.113409519195557 22.0
35.519281615 33.6374778747559 23.0
0 0.0440042018890381 24.0
0 0.724399149417877 25.0
0 -0.0227079391479492 26.0
0 -0.117142915725708 27.0
0 0.467523038387299 28.0
0 0.00105750560760498 29.0
0 0.00950509309768677 30.0
0 0.0641344785690308 31.0
0 0.176858246326447 32.0
0 -0.0435090065002441 33.0
0 0.143996238708496 34.0
0 0.259006023406982 35.0
0 0.329947233200073 36.0
0 0.00465065240859985 37.0
0 0.192311346530914 38.0
0 -0.392514228820801 39.0
0 0.505170047283173 40.0
0 -0.112052917480469 41.0
0 0.0465841293334961 42.0
299.72437057 294.468078613281 43.0
0 0.100567698478699 44.0
0 0.150705933570862 45.0
0 0.100081086158752 46.0
0 0.0166687965393066 47.0
0 0.237432241439819 48.0
0 0.424506217241287 49.0
0.06770095198 0.050101101398468 50.0
2.1275261575 1.8843138217926 51.0
0 0.159187018871307 52.0
0 0.0281554460525513 53.0
0 -0.0279065370559692 54.0
0 0.124693870544434 55.0
0 0.00743407011032104 56.0
0 0.0915999412536621 57.0
0 0.197285532951355 58.0
0 0.199556231498718 59.0
0 -0.241236329078674 60.0
0 0.0958658456802368 61.0
0 0.305223643779755 62.0
255.56707141 255.328964233398 0.0
0 -0.198530316352844 1.0
0 -0.0676681399345398 2.0
0 0.138306140899658 3.0
0 -0.0398568511009216 4.0
0 0.0718434453010559 5.0
0 0.16414612531662 6.0
0 0.241562366485596 7.0
0 0.356863766908646 8.0
0 -0.0673375129699707 9.0
0 0.383908927440643 10.0
0 -0.259324550628662 11.0
0 0.0553321838378906 12.0
54.39590443 54.2618675231934 13.0
33.204580577 41.3523712158203 14.0
0 -0.21725857257843 15.0
0 -0.152048945426941 16.0
33.145094906 34.9089508056641 17.0
0 0.0838639736175537 18.0
0 -0.037408709526062 19.0
0 0.13627678155899 20.0
0 0.000853776931762695 21.0
0 0.263648748397827 22.0
33.915489764 31.7418613433838 23.0
0 0.0761353373527527 24.0
0 0.153410673141479 25.0
0 0.028093695640564 26.0
0 0.0598646998405457 27.0
0 0.502296268939972 28.0
0 0.0496621131896973 29.0
0 0.0853224396705627 30.0
0 0.0094752311706543 31.0
0 0.12805312871933 32.0
0 -0.0346094369888306 33.0
0 -0.813718199729919 34.0
0 0.306957960128784 35.0
0 0.198209166526794 36.0
0 0.0182942152023315 37.0
0 0.0402127504348755 38.0
0 -0.00544381141662598 39.0
0 0.325378596782684 40.0
0 -0.084736704826355 41.0
0 0.0123789310455322 42.0
303.20201703 297.501983642578 43.0
0 0.180434823036194 44.0
0 0.0676190853118896 45.0
0 0.00424408912658691 46.0
0 0.108411967754364 47.0
0 0.19588977098465 48.0
0 0.34681648015976 49.0
0.090061249676 -0.00212383270263672 50.0
1.7849062669 1.89132165908813 51.0
0 -0.134226381778717 52.0
0 -0.0710195302963257 53.0
0 0.0146504044532776 54.0
0 0.111641407012939 55.0
0 -0.0816329717636108 56.0
0 0.0868276357650757 57.0
0 0.437265783548355 58.0
0 0.258241236209869 59.0
0 0.134581863880157 60.0
0 0.155158460140228 61.0
0 0.289018332958221 62.0
249.90293714 252.891555786133 0.0
0 -0.283328771591187 1.0
0 0.123495757579803 2.0
0 0.222789764404297 3.0
0 -0.0713685154914856 4.0
0 0.0965744853019714 5.0
0 0.0975630879402161 6.0
0 0.0825998783111572 7.0
0 -0.131292462348938 8.0
0 -0.0616059303283691 9.0
0 0.65373283624649 10.0
0 -0.304409384727478 11.0
0 -0.0757790207862854 12.0
52.495871358 52.9653396606445 13.0
40.528860485 40.9543342590332 14.0
0 -0.0751992464065552 15.0
0 -0.00207048654556274 16.0
36.52351367 34.686710357666 17.0
0 0.120213031768799 18.0
0 0.00753343105316162 19.0
0 0.0796407461166382 20.0
0 0.0703301429748535 21.0
0 -0.126437485218048 22.0
38.16165703 38.8016014099121 23.0
0 0.0432993173599243 24.0
0 0.506725549697876 25.0
0 -0.0347310304641724 26.0
0 -0.0188057422637939 27.0
0 0.250144600868225 28.0
0 -0.0639830827713013 29.0
0 0.0662827491760254 30.0
0 0.123109936714172 31.0
0 0.157107293605804 32.0
0 0.123397588729858 33.0
0 0.0294228792190552 34.0
0 0.0274379849433899 35.0
0 0.371791243553162 36.0
0 -0.0718560814857483 37.0
0 0.0749075412750244 38.0
0 -0.10357654094696 39.0
0 0.178838729858398 40.0
0 -0.0924508571624756 41.0
0 0.056011438369751 42.0
291.58404701 296.1162109375 43.0
0 0.109601855278015 44.0
0 0.000753402709960938 45.0
0 0.128844738006592 46.0
0 0.214160621166229 47.0
0 0.42105558514595 48.0
0 0.495312929153442 49.0
0.072260930937 0.0421543717384338 50.0
2.02134151 1.85095405578613 51.0
0 -0.0327059030532837 52.0
0 -0.0516154766082764 53.0
0 0.103443384170532 54.0
0 0.0854284763336182 55.0
0 -0.068327009677887 56.0
0 -0.0993363857269287 57.0
0 -0.0510244965553284 58.0
0 0.146546423435211 59.0
0 0.0918095111846924 60.0
0 0.109559178352356 61.0
0 0.172112643718719 62.0
246.47853885 253.994293212891 0.0
0 -0.211024641990662 1.0
0 -0.248040795326233 2.0
0 0.149389028549194 3.0
0 -0.335449695587158 4.0
0 -0.114209055900574 5.0
0 0.1390620470047 6.0
0 0.00971877574920654 7.0
0 -0.0206025838851929 8.0
0 0.455786913633347 9.0
0 0.21722149848938 10.0
0 0.0247624516487122 11.0
0 -0.109904289245605 12.0
55.153480377 53.8922805786133 13.0
39.001223103 39.0275955200195 14.0
0 -0.0799767971038818 15.0
0 -0.192190527915955 16.0
29.890045797 33.3568229675293 17.0
0 0.0305667519569397 18.0
0 -0.0374360680580139 19.0
0 0.128196537494659 20.0
0 0.0954132080078125 21.0
0 0.256038010120392 22.0
35.594433896 35.4278602600098 23.0
0 0.0357109904289246 24.0
0 0.378564476966858 25.0
0 -0.0426190495491028 26.0
0 -0.19238406419754 27.0
0 0.255129277706146 28.0
0 0.0401874780654907 29.0
0 0.0630037188529968 30.0
0 -0.0567730665206909 31.0
0 0.0256118178367615 32.0
0 -0.0183196663856506 33.0
0 -0.109562754631042 34.0
0 0.270168304443359 35.0
0 0.253529489040375 36.0
0 -7.3552131652832e-05 37.0
0 0.0593830943107605 38.0
0 -0.227150201797485 39.0
0 0.342134714126587 40.0
0 -0.0900716185569763 41.0
0 0.0132440328598022 42.0
308.96250714 295.502502441406 43.0
0 0.0660642385482788 44.0
0 -0.0086638331413269 45.0
0 0.137384176254272 46.0
0 -0.0054091215133667 47.0
0 0.341283202171326 48.0
0 0.468915104866028 49.0
0.076557151485 0.0588585138320923 50.0
2.1349900561 1.90213215351105 51.0
0 -0.0237103700637817 52.0
0 0.0307931900024414 53.0
0 -0.168440341949463 54.0
0 0.231953382492065 55.0
0 0.25967925786972 56.0
0 0.325434625148773 57.0
0 0.0746188759803772 58.0
0 0.384947061538696 59.0
0 -0.0538522601127625 60.0
0 0.153786420822144 61.0
0 0.35084331035614 62.0
246.27384936 233.961120605469 0.0
0 -0.649028539657593 1.0
0 0.0308095216751099 2.0
0 0.338201820850372 3.0
0 -0.449849247932434 4.0
0 -0.0741754770278931 5.0
0 -0.563700914382935 6.0
0 0.558340191841125 7.0
0 0.574467957019806 8.0
0 0.111603856086731 9.0
0 0.620295763015747 10.0
0 -0.172389566898346 11.0
0 -0.13008588552475 12.0
67.764320873 65.7627258300781 13.0
36.06674367 30.5880279541016 14.0
0 -0.453273415565491 15.0
0 -0.10361385345459 16.0
28.879855975 30.1713886260986 17.0
0 0.0835418701171875 18.0
0 0.0549670457839966 19.0
0 0.211948990821838 20.0
0 0.191782832145691 21.0
0 0.0766806602478027 22.0
33.674342195 28.8721237182617 23.0
0 0.0225458145141602 24.0
29.904366056 0.269060492515564 25.0
0 -0.0282254815101624 26.0
0 -0.245188236236572 27.0
0 0.301249831914902 28.0
0 0.140333414077759 29.0
0 0.0859975218772888 30.0
0 0.633591294288635 31.0
0 -0.0627298355102539 32.0
0 -0.478114485740662 33.0
0 0.258949100971222 34.0
0 0.0349020957946777 35.0
0 0.309788316488266 36.0
0 0.0335414409637451 37.0
0 0.200336456298828 38.0
0 -0.180433750152588 39.0
0 0.462009429931641 40.0
0 -0.122739434242249 41.0
0 0.0734397172927856 42.0
279.49976993 297.124633789062 43.0
0 0.113790929317474 44.0
0 0.219247877597809 45.0
0 0.243389844894409 46.0
0 -0.203580737113953 47.0
0 -0.560510516166687 48.0
0 0.333295524120331 49.0
0.074540763721 -0.000496149063110352 50.0
1.9977340937 1.87410449981689 51.0
0 0.28589391708374 52.0
0 -0.078432559967041 53.0
0 0.148944795131683 54.0
0 0.0137648582458496 55.0
0 -0.104040622711182 56.0
0 0.343204915523529 57.0
0 0.7242112159729 58.0
0 0.115931570529938 59.0
0 -0.268659472465515 60.0
0 0.207696437835693 61.0
0 0.512636423110962 62.0
223.26609178 251.51350402832 0.0
0 0.483691573143005 1.0
0 -0.0915277004241943 2.0
0 0.0188664197921753 3.0
0 -0.289864301681519 4.0
0 -0.487275242805481 5.0
0 0.116945505142212 6.0
0 -0.13821667432785 7.0
0 -0.018986701965332 8.0
0 -0.106765329837799 9.0
0 0.409179508686066 10.0
0 -0.177073001861572 11.0
0 0.00309443473815918 12.0
54.585999735 51.8491249084473 13.0
51.400147099 44.1281585693359 14.0
0 0.385860443115234 15.0
0 0.0565657019615173 16.0
41.964693888 34.754337310791 17.0
0 0.128628969192505 18.0
0 -0.0597313642501831 19.0
0 0.108305871486664 20.0
0 0.0529531240463257 21.0
0 0.211477398872375 22.0
33.234306113 34.2726020812988 23.0
0 0.0656477212905884 24.0
0 0.43658834695816 25.0
0 0.0704132914543152 26.0
0 -0.197351813316345 27.0
0 -0.302801489830017 28.0
0 0.101554870605469 29.0
0 0.121046304702759 30.0
0 -0.0910675525665283 31.0
0 0.0807833075523376 32.0
0 0.00310671329498291 33.0
0 -0.437111854553223 34.0
0 0.0373986959457397 35.0
0 -0.481779098510742 36.0
0 0.00500816106796265 37.0
0 -0.0634899735450745 38.0
0 -0.129226803779602 39.0
0 0.268320620059967 40.0
0 -0.125699162483215 41.0
0 0.234389543533325 42.0
304.85191717 296.134521484375 43.0
0 0.166084766387939 44.0
0 0.155662894248962 45.0
0 0.148715317249298 46.0
0 0.0993115901947021 47.0
0 0.278136968612671 48.0
2.0246724542 0.359192192554474 49.0
0.078205828447 -0.0177519917488098 50.0
1.7566524133 1.87498497962952 51.0
0 -0.224074244499207 52.0
0 0.277884125709534 53.0
0 -0.157098770141602 54.0
0 -0.049394965171814 55.0
0 -0.127682030200958 56.0
0 0.0441115498542786 57.0
0 0.000679194927215576 58.0
0 0.541442394256592 59.0
0 0.105756342411041 60.0
0 0.0673950910568237 61.0
0 0.193714439868927 62.0
253.78856919 253.661712646484 0.0
0 -0.066689133644104 1.0
0 -0.102772533893585 2.0
0 0.404389053583145 3.0
0 -0.211756825447083 4.0
0 -0.0318129062652588 5.0
0 0.00538575649261475 6.0
0 -0.11754035949707 7.0
0 -0.18554949760437 8.0
0 -0.212817311286926 9.0
0 0.0861808061599731 10.0
0 0.101160049438477 11.0
0 -0.134214043617249 12.0
58.856230718 60.6535377502441 13.0
46.66904074 38.106315612793 14.0
0 -0.192997694015503 15.0
0 0.478460311889648 16.0
28.581460582 35.1583786010742 17.0
0 0.142981648445129 18.0
0 -0.0432053804397583 19.0
0 0.0442506074905396 20.0
0 0.102654933929443 21.0
0 0.133963346481323 22.0
30.850875241 28.1433887481689 23.0
0 -0.0170838832855225 24.0
0 0.396936982870102 25.0
0 0.00338137149810791 26.0
0 -0.223351716995239 27.0
0 0.265446126461029 28.0
0 -0.0963529944419861 29.0
0 0.115933895111084 30.0
0 0.318193078041077 31.0
0 0.233622789382935 32.0
0 -0.00321638584136963 33.0
0 0.300316035747528 34.0
0 -0.174220621585846 35.0
0 0.0712000131607056 36.0
0 -0.040746808052063 37.0
0 0.0417793393135071 38.0
0 -0.0638289451599121 39.0
0 0.523896217346191 40.0
0 -0.0550234913825989 41.0
0 -0.0167515277862549 42.0
286.17129642 295.880645751953 43.0
0 0.221423745155334 44.0
0 0.202182948589325 45.0
0 0.0948864221572876 46.0
0 -0.314541101455688 47.0
0 0.326801121234894 48.0
1.3629653983 0.66364711523056 49.0
0.076665818564 0.0338114500045776 50.0
2.0756467539 1.90056824684143 51.0
0 0.0920994281768799 52.0
0 -0.0924105644226074 53.0
0 -0.0145097970962524 54.0
0 0.291325956583023 55.0
0 0.331196516752243 56.0
0 0.0396628379821777 57.0
0 0.234504759311676 58.0
0 0.044539213180542 59.0
0 -0.200590252876282 60.0
0 0.0286796092987061 61.0
0 0.234862923622131 62.0
239.10623981 252.315673828125 0.0
0 -0.0912187099456787 1.0
0 -0.129621803760529 2.0
0 0.169588029384613 3.0
0 -0.242344856262207 4.0
0 -0.251745820045471 5.0
0 0.0119373202323914 6.0
0 0.0580941438674927 7.0
0 -0.0208333730697632 8.0
0 -0.324542999267578 9.0
0 -0.0174582600593567 10.0
0 0.0386248826980591 11.0
0 0.0190529823303223 12.0
72.884736446 72.74365234375 13.0
36.719566195 27.2423439025879 14.0
0 -0.133652091026306 15.0
0 0.387471377849579 16.0
34.726962542 34.7558479309082 17.0
0 0.123335003852844 18.0
0 -0.0198848247528076 19.0
0 0.129324018955231 20.0
0 -0.0694499015808105 21.0
0 0.0570504665374756 22.0
38.2427777 38.1778564453125 23.0
0 0.0421169400215149 24.0
0 0.377223968505859 25.0
0 -0.0797715783119202 26.0
0 0.0495631694793701 27.0
0 -0.0574562549591064 28.0
0 -0.0174368023872375 29.0
0 0.0490740537643433 30.0
0 -0.150450468063354 31.0
0 0.0739812850952148 32.0
0 -0.325563788414001 33.0
0 0.0830751657485962 34.0
0 0.0132261514663696 35.0
0 -0.025686502456665 36.0
0 -0.0788917541503906 37.0
0 0.444948524236679 38.0
0 -0.600911855697632 39.0
0 0.27663391828537 40.0
0 -0.204449951648712 41.0
0 -0.0351536273956299 42.0
307.23998536 296.583404541016 43.0
0 0.0341610312461853 44.0
0 0.139615416526794 45.0
0 0.170888900756836 46.0
0 -0.0724812746047974 47.0
0 0.186039924621582 48.0
0 0.432953387498856 49.0
0.093856312152 0.06258624792099 50.0
2.0102677936 1.86382460594177 51.0
0 -0.146347403526306 52.0
0 -0.0273371934890747 53.0
0 0.463137239217758 54.0
0 0.128177642822266 55.0
0 -0.124248027801514 56.0
0 0.481853842735291 57.0
0 0.302079886198044 58.0
0 0.101176381111145 59.0
0 -0.0791336894035339 60.0
0 -0.00394248962402344 61.0
0 0.100907385349274 62.0
249.47224269 254.72868347168 0.0
0 -0.000231027603149414 1.0
0 0.0282449126243591 2.0
0 0.442882478237152 3.0
0 0.0182255506515503 4.0
0 -0.397759675979614 5.0
0 -0.258847951889038 6.0
0 0.253048002719879 7.0
0 -0.924559831619263 8.0
0 -0.671915888786316 9.0
0 1.20693445205688 10.0
0 0.0483288764953613 11.0
0 -0.148167610168457 12.0
44.530182814 42.8472595214844 13.0
39.326904163 47.8801765441895 14.0
0 0.0814121961593628 15.0
0 -0.105734288692474 16.0
30.126958638 35.4128837585449 17.0
0 0.198182463645935 18.0
0 -0.0808130502700806 19.0
0 0.123802363872528 20.0
0 -0.128887832164764 21.0
0 0.946754455566406 22.0
39.192553873 37.190544128418 23.0
0 -0.00841617584228516 24.0
0 0.675537407398224 25.0
0 -0.0407710075378418 26.0
0 0.00638514757156372 27.0
0 -0.519425988197327 28.0
0 -0.205524504184723 29.0
0 0.0535305738449097 30.0
0 0.281441211700439 31.0
0 0.281211256980896 32.0
0 -0.349487543106079 33.0
0 0.0630356669425964 34.0
0 0.266612946987152 35.0
0 -0.225489377975464 36.0
0 -0.0807452201843262 37.0
0 0.0866405367851257 38.0
0 -0.167644262313843 39.0
0 0.718822121620178 40.0
0 -0.193479478359222 41.0
0 0.155693054199219 42.0
292.42597538 295.090545654297 43.0
0 0.0252137184143066 44.0
0 0.182328701019287 45.0
0 0.54173481464386 46.0
0 -0.598389029502869 47.0
0 -0.84177303314209 48.0
0 0.623964309692383 49.0
0.09665012652 -0.0484555959701538 50.0
2.4211588333 1.84156382083893 51.0
0 -0.17999804019928 52.0
0 0.162080228328705 53.0
0 -0.17835658788681 54.0
0 -0.188875019550323 55.0
0 -0.123768448829651 56.0
0 -0.286482214927673 57.0
0 0.297501236200333 58.0
0 0.0914135575294495 59.0
0 -0.183172106742859 60.0
0 0.0254509449005127 61.0
0 0.189623713493347 62.0
255.99925825 254.189041137695 0.0
0 -0.161969661712646 1.0
0 0.0772925615310669 2.0
0 0.314968347549438 3.0
0 -0.245269656181335 4.0
0 0.0363172888755798 5.0
0 0.0782415866851807 6.0
0 0.0915979146957397 7.0
0 0.52292275428772 8.0
0 -0.322052240371704 9.0
0 0.190490484237671 10.0
0 0.0809116959571838 11.0
0 -0.00208252668380737 12.0
63.378122859 62.0942001342773 13.0
35.461713515 38.0472450256348 14.0
0 -0.0043950080871582 15.0
0 -0.115229487419128 16.0
23.823333818 30.5265693664551 17.0
0 0.0459027886390686 18.0
0 0.0500873923301697 19.0
0 0.199946880340576 20.0
0 0.243852913379669 21.0
0 -0.0387579202651978 22.0
31.601187126 25.8728733062744 23.0
0 0.000402331352233887 24.0
0 0.240214347839355 25.0
0 -0.00925552845001221 26.0
0 0.139597356319427 27.0
0 -0.0371410846710205 28.0
0 -0.0336369276046753 29.0
0 -0.0162807106971741 30.0
0 0.010358452796936 31.0
0 -0.0377344489097595 32.0
0 0.0481163859367371 33.0
0 0.1251580119133 34.0
0 0.0611851215362549 35.0
0 -0.00611674785614014 36.0
0 0.177680253982544 37.0
0 -0.00894773006439209 38.0
0 -0.0773999691009521 39.0
0 0.444549918174744 40.0
0 -0.0978207588195801 41.0
0 0.0707547664642334 42.0
281.42504354 297.669921875 43.0
0 0.071302592754364 44.0
0 0.145527541637421 45.0
0 0.257879316806793 46.0
0 -0.0271697640419006 47.0
0 0.287467956542969 48.0
3.8216077332 0.570688009262085 49.0
0.064380623846 0.0740444660186768 50.0
2.0002402086 1.86960995197296 51.0
0 -0.230640530586243 52.0
0 -0.31225311756134 53.0
0 -0.223901033401489 54.0
0 -0.141189634799957 55.0
0 0.220754981040955 56.0
0 0.182594060897827 57.0
0 0.00439947843551636 58.0
0 0.266122877597809 59.0
0 -0.181855201721191 60.0
0 0.151235222816467 61.0
0 0.382031977176666 62.0
277.8785134 250.131042480469 0.0
0 -0.24557900428772 1.0
0 0.0682199001312256 2.0
0 0.101409018039703 3.0
0 -0.0251025557518005 4.0
0 0.0466779470443726 5.0
0 0.0178806781768799 6.0
0 0.386934369802475 7.0
0 0.0871551036834717 8.0
0 0.292125582695007 9.0
0 -0.557440400123596 10.0
0 -0.175067603588104 11.0
0 -0.0469050407409668 12.0
52.149117455 51.9481391906738 13.0
43.921465091 41.2573699951172 14.0
0 -0.218827486038208 15.0
0 0.207340955734253 16.0
30.038375892 31.3893280029297 17.0
0 0.135007798671722 18.0
0 -0.0350189208984375 19.0
0 0.143278956413269 20.0
0 0.0543632507324219 21.0
0 0.150733530521393 22.0
35.006707538 33.2484474182129 23.0
0 -0.0340882539749146 24.0
0 -0.277569890022278 25.0
0 -0.0767366886138916 26.0
0 -0.118565797805786 27.0
0 0.331078916788101 28.0
0 -0.117487788200378 29.0
0 0.0350120067596436 30.0
0 0.0463211536407471 31.0
0 0.152413487434387 32.0
0 0.123256325721741 33.0
0 -0.146377205848694 34.0
0 -0.231626510620117 35.0
0 0.270272254943848 36.0
0 -0.102140605449677 37.0
0 0.286469757556915 38.0
0 -0.435917019844055 39.0
0 0.490308701992035 40.0
0 -0.244603514671326 41.0
0 0.0493662357330322 42.0
280.03533859 293.904479980469 43.0
0 0.119791030883789 44.0
0 0.167124569416046 45.0
0 0.05427086353302 46.0
0 -0.988937139511108 47.0
0 0.182441532611847 48.0
0 0.414912462234497 49.0
0.10704007417 0.0800673961639404 50.0
1.7283463559 1.76228141784668 51.0
0 -0.100689470767975 52.0
0 -0.0686116814613342 53.0
0 -0.145256400108337 54.0
0 -0.0210610628128052 55.0
0 0.193384528160095 56.0
0 0.328392207622528 57.0
0 0.025590717792511 58.0
0 0.0353039503097534 59.0
0 -0.00769174098968506 60.0
0 0.115012645721436 61.0
0 0.0985153317451477 62.0
224.26623383 251.953201293945 0.0
0 -0.151583075523376 1.0
0 -0.0775977969169617 2.0
0 0.0365826487541199 3.0
0 -0.278868198394775 4.0
0 -0.147893249988556 5.0
0 0.153868615627289 6.0
0 -0.264371275901794 7.0
0 -0.367440938949585 8.0
0 0.386136889457703 9.0
0 0.325103789567947 10.0
0 0.606820344924927 11.0
0 -0.116397380828857 12.0
54.995673976 52.6221389770508 13.0
35.408442778 38.4427871704102 14.0
0 0.0419291257858276 15.0
0 0.0527704954147339 16.0
29.342136396 32.8623657226562 17.0
0 0.123436570167542 18.0
0 -0.00654387474060059 19.0
0 0.208119928836823 20.0
0 -0.0526782274246216 21.0
0 0.0987230539321899 22.0
40.57176485 37.6390800476074 23.0
0 0.0265222191810608 24.0
0 0.583790421485901 25.0
0 0.0094267725944519 26.0
0 0.0149704813957214 27.0
0 -0.246802926063538 28.0
0 -0.139140367507935 29.0
0 0.0544874668121338 30.0
0 -0.355037569999695 31.0
0 -0.0938601493835449 32.0
0 -0.262980699539185 33.0
0 0.139523148536682 34.0
0 -0.0338883996009827 35.0
0 0.571982562541962 36.0
0 0.234763979911804 37.0
0 -0.50630795955658 38.0
0 -0.0503607988357544 39.0
0 0.418068140745163 40.0
0 -0.318056106567383 41.0
0 0.0464491844177246 42.0
293.95017357 295.411346435547 43.0
0 0.0798574686050415 44.0
0 0.0548496842384338 45.0
0 0.209174156188965 46.0
0 -0.114166915416718 47.0
0 0.183172643184662 48.0
0 0.210108757019043 49.0
0.10203926264 0.0659883618354797 50.0
2.2384175158 1.83622360229492 51.0
0 0.0256547927856445 52.0
0 -0.50098979473114 53.0
0 -0.0794817209243774 54.0
0 0.158838868141174 55.0
0 0.106487691402435 56.0
0 0.324489027261734 57.0
0 -0.0269826650619507 58.0
0 0.153522729873657 59.0
0 0.0463740229606628 60.0
0 0.00444352626800537 61.0
0 -0.0532224774360657 62.0
237.9749393 252.599517822266 0.0
0 -0.267160654067993 1.0
0 -0.00323104858398438 2.0
0 0.0469697117805481 3.0
0 -0.305498957633972 4.0
0 0.0676226615905762 5.0
0 0.190459251403809 6.0
0 0.178408622741699 7.0
0 0.278436303138733 8.0
0 0.0629947185516357 9.0
0 -0.0436846017837524 10.0
0 -0.208450317382812 11.0
0 -0.0153966546058655 12.0
53.01598376 52.406551361084 13.0
42.106576182 44.8892593383789 14.0
0 -0.109025835990906 15.0
0 0.0555400252342224 16.0
30.24209057 33.8810005187988 17.0
0 0.0351197719573975 18.0
0 0.00909733772277832 19.0
0 0.156299591064453 20.0
0 -0.000176012516021729 21.0
0 0.160275876522064 22.0
37.721346758 35.0021286010742 23.0
0 0.109951853752136 24.0
0 0.390498161315918 25.0
0 -0.11982923746109 26.0
0 -0.0997179746627808 27.0
0 1.02319777011871 28.0
0 -0.0430593490600586 29.0
0 0.177193284034729 30.0
0 0.00849342346191406 31.0
0 -0.157791674137115 32.0
0 -0.21303403377533 33.0
0 0.306137889623642 34.0
0 -0.00974124670028687 35.0
0 0.276890993118286 36.0
0 0.15659111738205 37.0
0 0.0205320119857788 38.0
0 -0.184474349021912 39.0
0 0.707290530204773 40.0
0 -0.0922883749008179 41.0
0 0.151959180831909 42.0
305.97915306 296.640197753906 43.0
0 0.0892139673233032 44.0
0 0.0194360017776489 45.0
0 0.277662813663483 46.0
0 -0.0247103571891785 47.0
0 -1.14452636241913 48.0
0 0.535500228404999 49.0
0.089228017656 -0.0468489527702332 50.0
1.8739509055 1.85280704498291 51.0
0 0.183251976966858 52.0
0 -0.048349142074585 53.0
0 -0.019792914390564 54.0
0 0.0299278497695923 55.0
0 -0.0636423230171204 56.0
0 0.257896900177002 57.0
0 0.3233702480793 58.0
0 0.25180983543396 59.0
0 -0.205122292041779 60.0
0 0.127271354198456 61.0
0 0.323346853256226 62.0
243.5053937 253.399200439453 0.0
0 0.275060653686523 1.0
0 0.0714315176010132 2.0
0 0.362432301044464 3.0
0 -0.294955253601074 4.0
0 -0.468411684036255 5.0
0 0.00995123386383057 6.0
0 0.0633487701416016 7.0
0 -0.0461117029190063 8.0
0 0.254713714122772 9.0
0 0.0621575713157654 10.0
0 0.0459641218185425 11.0
0 -0.213217735290527 12.0
58.517422758 59.019947052002 13.0
44.983247631 37.215217590332 14.0
0 -0.0570520162582397 15.0
0 0.11443692445755 16.0
39.213407233 35.836009979248 17.0
0 0.176892936229706 18.0
0 -0.0120480060577393 19.0
0 0.0566537380218506 20.0
0 0.107158958911896 21.0
0 0.290941178798676 22.0
33.424482537 33.219654083252 23.0
0 0.060643196105957 24.0
0 0.649627864360809 25.0
0 -0.0370572805404663 26.0
0 -0.184044599533081 27.0
0 0.414347350597382 28.0
0 -0.0296134352684021 29.0
0 0.059143602848053 30.0
0 -0.0100218057632446 31.0
0 -0.00208044052124023 32.0
0 -0.18141633272171 33.0
0 0.1418097615242 34.0
0 0.268391132354736 35.0
0 0.16785192489624 36.0
0 -0.0681792497634888 37.0
0 0.298349440097809 38.0
0 -0.366873979568481 39.0
0 0.155680418014526 40.0
0 -0.242059350013733 41.0
0 -0.0536402463912964 42.0
302.23032285 297.329833984375 43.0
0 0.0617835521697998 44.0
0 0.0206116437911987 45.0
0 0.181795299053192 46.0
0 -0.328746557235718 47.0
0 0.384403735399246 48.0
0.33193961482 0.522757411003113 49.0
0.079595171134 0.0989545583724976 50.0
1.9698338108 1.83503937721252 51.0
0 0.452872037887573 52.0
0 0.144295215606689 53.0
0 0.184258580207825 54.0
0 0.285774707794189 55.0
0 -0.0743985772132874 56.0
0 0.180390894412994 57.0
0 0.271356284618378 58.0
0 0.241644382476807 59.0
0 -0.196622312068939 60.0
0 0.0791488885879517 61.0
0 0.294707775115967 62.0
252.29878881 252.257293701172 0.0
0 -0.154705762863159 1.0
0 -0.0112746953964233 2.0
0 0.0707172751426697 3.0
0 -0.102408170700073 4.0
0 0.347776591777802 5.0
0 0.241323709487915 6.0
0 0.157631576061249 7.0
0 -0.0241801738739014 8.0
0 -0.332694530487061 9.0
0 -0.248842000961304 10.0
0 0.0226717591285706 11.0
0 -0.0670350193977356 12.0
65.30730968 65.4462890625 13.0
43.929372721 34.3506698608398 14.0
0 -0.0122194290161133 15.0
0 -0.115178883075714 16.0
29.692732258 32.1323432922363 17.0
0 0.051457941532135 18.0
0 -0.0233911275863647 19.0
0 0.158618807792664 20.0
0 -0.00649946928024292 21.0
0 0.89454174041748 22.0
40.252212846 40.282341003418 23.0
0 0.160796046257019 24.0
0 0.907861590385437 25.0
0 0.0216784477233887 26.0
0 0.0787232518196106 27.0
0 -0.117909967899323 28.0
0 -0.180384397506714 29.0
0 0.0482197403907776 30.0
0 -0.0982370376586914 31.0
0 0.0730081796646118 32.0
0 -0.0171689391136169 33.0
0 -0.0481252074241638 34.0
0 0.230168461799622 35.0
0 -0.190411806106567 36.0
0 -0.0387980341911316 37.0
0 0.366993218660355 38.0
0 -0.384527802467346 39.0
0 -0.363723516464233 40.0
0 -0.240183711051941 41.0
0 0.0523974895477295 42.0
290.52591934 297.337890625 43.0
0 0.0284295678138733 44.0
0 0.226174652576447 45.0
0 0.088126540184021 46.0
0 0.00519955158233643 47.0
0 0.191969633102417 48.0
0 0.309386879205704 49.0
0.082579872985 -0.00860857963562012 50.0
1.8766716864 1.86620807647705 51.0
0 -0.172766447067261 52.0
0 -0.179577767848969 53.0
0 0.0413157939910889 54.0
0 -0.189278841018677 55.0
0 -0.0961267948150635 56.0
0 0.0272765159606934 57.0
0 0.215605497360229 58.0
0 0.447669386863708 59.0
0 0.0568674802780151 60.0
0 0.140542447566986 61.0
0 0.215630412101746 62.0
228.19421449 255.3955078125 0.0
0 -0.212936282157898 1.0
0 -0.0681290626525879 2.0
0 0.0903786420822144 3.0
0 -0.237834215164185 4.0
0 -0.232446789741516 5.0
0 0.0516031384468079 6.0
0 -0.126751899719238 7.0
0 0.324303269386292 8.0
0 0.276396214962006 9.0
0 0.337658643722534 10.0
0 -0.133723735809326 11.0
0 -0.111144781112671 12.0
55.992410422 55.9463233947754 13.0
44.106995702 39.5724830627441 14.0
0 -0.0622373819351196 15.0
0 0.0268344879150391 16.0
32.027255686 33.5982818603516 17.0
0 0.07767653465271 18.0
0 -0.0211000442504883 19.0
0 0.0979496240615845 20.0
0 0.0841624736785889 21.0
0 0.12562221288681 22.0
34.324535744 32.3167495727539 23.0
0 0.0347260236740112 24.0
0 0.412320911884308 25.0
0 -0.17292046546936 26.0
0 -0.0394327044487 27.0
0 0.375846058130264 28.0
0 0.0790090560913086 29.0
0 0.103489279747009 30.0
0 0.0438646078109741 31.0
0 0.060810923576355 32.0
0 -0.3504638671875 33.0
0 -0.253757357597351 34.0
0 0.106047213077545 35.0
0 0.216582775115967 36.0
0 -0.0451576113700867 37.0
0 0.35483330488205 38.0
0 -0.215680241584778 39.0
0 0.00811076164245605 40.0
0 -0.194300413131714 41.0
0 0.0645767450332642 42.0
306.1345352 298.153747558594 43.0
0 0.0148054957389832 44.0
0 0.167156100273132 45.0
0 0.2361781001091 46.0
0 0.0645787119865417 47.0
0 0.159126877784729 48.0
1.4974476211 0.481060236692429 49.0
0.093620291271 0.0325409173965454 50.0
1.8684514663 1.91011118888855 51.0
0 0.132282853126526 52.0
0 -0.289436101913452 53.0
0 0.0817135572433472 54.0
0 0.22777783870697 55.0
0 0.185789525508881 56.0
0 0.177035331726074 57.0
0 0.172611057758331 58.0
0 -0.0754047632217407 59.0
0 -0.109184980392456 60.0
0 0.0287622809410095 61.0
0 0.0922597050666809 62.0
246.10036775 251.472412109375 0.0
0 -0.193207144737244 1.0
0 -0.13138484954834 2.0
0 0.296145856380463 3.0
0 -0.0685569047927856 4.0
0 0.0736313462257385 5.0
0 -0.0782703757286072 6.0
0 -0.278191924095154 7.0
0 0.0502166748046875 8.0
0 0.330953896045685 9.0
0 0.335683643817902 10.0
0 -0.072706937789917 11.0
0 -0.253419160842896 12.0
60.16591839 58.7005729675293 13.0
42.230332396 37.3405647277832 14.0
0 0.0881087779998779 15.0
0 -0.110251665115356 16.0
33.765235929 36.0369186401367 17.0
0 0.148728430271149 18.0
0 -0.038901686668396 19.0
0 0.125233173370361 20.0
0 0.116216063499451 21.0
0 0.0775277614593506 22.0
31.095887022 30.9572887420654 23.0
0 0.0133275389671326 24.0
0 0.486186385154724 25.0
0 0.0415481328964233 26.0
0 -0.115670442581177 27.0
0 -0.0118584036827087 28.0
0 -0.1114821434021 29.0
0 0.0447824001312256 30.0
0 0.206865668296814 31.0
0 0.0837240219116211 32.0
0 -0.204327166080475 33.0
0 -0.159885764122009 34.0
0 0.223584175109863 35.0
0 0.560309052467346 36.0
0 0.111258625984192 37.0
0 0.0460180640220642 38.0
0 -0.236460208892822 39.0
0 0.181391537189484 40.0
0 -0.144014537334442 41.0
0 -0.140238761901855 42.0
299.23666777 297.230987548828 43.0
0 0.0990457534790039 44.0
0 0.180787861347198 45.0
0 0.102139592170715 46.0
0 -0.0257649421691895 47.0
0 0.174927830696106 48.0
0 0.288556694984436 49.0
0.078311438954 0.0789592266082764 50.0
1.8185687446 1.81926870346069 51.0
0 0.106295526027679 52.0
0 -0.125784993171692 53.0
0 0.177886247634888 54.0
0 0.122611999511719 55.0
0 -0.0809393525123596 56.0
0 0.117952406406403 57.0
0 0.261859655380249 58.0
0 -0.047715961933136 59.0
0 -0.156474053859711 60.0
0 -0.00256597995758057 61.0
0 0.168526530265808 62.0
246.95520436 254.452728271484 0.0
0 -0.184022307395935 1.0
0 -0.11296159029007 2.0
0 0.0944125056266785 3.0
0 -0.0327275991439819 4.0
0 0.0481364130973816 5.0
0 0.0464929938316345 6.0
0 0.14729917049408 7.0
0 0.339730322360992 8.0
0 -0.0701541304588318 9.0
0 -0.232480525970459 10.0
0 -0.127657413482666 11.0
0 -0.122373938560486 12.0
53.989704928 52.8634986877441 13.0
47.414005905 42.7596321105957 14.0
0 0.0679640769958496 15.0
0 -0.151058316230774 16.0
33.393887828 35.367790222168 17.0
0 0.177413046360016 18.0
0 0.0268404483795166 19.0
0 0.323742002248764 20.0
0 -0.0219348669052124 21.0
0 0.156948447227478 22.0
35.250246598 35.5689468383789 23.0
0 0.00858521461486816 24.0
0 0.437893509864807 25.0
0 -0.0563751459121704 26.0
0 -0.111300408840179 27.0
0 -0.157212793827057 28.0
0 -0.130060732364655 29.0
0 0.0592517256736755 30.0
0 -0.167861700057983 31.0
0 -0.0616206526756287 32.0
0 -0.164743661880493 33.0
0 0.0702778100967407 34.0
0 -0.0802417993545532 35.0
0 -0.223480820655823 36.0
0 -0.0440316200256348 37.0
0 0.163020551204681 38.0
0 -0.179158806800842 39.0
0 0.288160502910614 40.0
0 -0.178288400173187 41.0
0 0.0299410820007324 42.0
295.17538214 295.682098388672 43.0
0 0.0984550714492798 44.0
0 0.219662249088287 45.0
0 0.234331965446472 46.0
0 -0.21442985534668 47.0
0 0.408325046300888 48.0
0 0.428094685077667 49.0
0.079214648542 0.054754912853241 50.0
1.548652431 1.83835673332214 51.0
0 0.057887077331543 52.0
0 0.0368080735206604 53.0
0 0.172213077545166 54.0
0 -0.0252600312232971 55.0
0 -0.0448682308197021 56.0
0 -0.188081622123718 57.0
0 0.127899408340454 58.0
0 -0.150547742843628 59.0
0 -0.0443965196609497 60.0
0 0.0590656399726868 61.0
0 0.159496366977692 62.0
214.77891698 255.722885131836 0.0
0 0.242672562599182 1.0
0 -0.202155947685242 2.0
0 -0.192427515983582 3.0
0 -0.368860363960266 4.0
0 -0.138798534870148 5.0
0 -0.312938809394836 6.0
0 -0.326969504356384 7.0
0 -0.138351917266846 8.0
0 0.196961283683777 9.0
0 -0.0222809910774231 10.0
0 -0.299802899360657 11.0
0 -0.0286153554916382 12.0
61.401887446 61.7944145202637 13.0
23.041916514 30.9244747161865 14.0
0 -0.117498397827148 15.0
0 -0.153166651725769 16.0
38.097280504 37.0342407226562 17.0
0 0.119077265262604 18.0
0 -0.0852776765823364 19.0
0 0.103178858757019 20.0
0 0.050540030002594 21.0
0 0.214949369430542 22.0
37.038688915 33.2741050720215 23.0
0 0.0894253253936768 24.0
0 0.384976297616959 25.0
0 0.0625964403152466 26.0
0 -0.143287062644958 27.0
0 1.14276123046875 28.0
0 0.143929004669189 29.0
0 0.143230736255646 30.0
0 0.131937623023987 31.0
0 0.226221680641174 32.0
0 0.144899606704712 33.0
0 -0.0847804546356201 34.0
0 -0.258659839630127 35.0
0 0.175770878791809 36.0
0 0.0159223675727844 37.0
0 0.284164667129517 38.0
0 -0.237639427185059 39.0
0 0.161729454994202 40.0
0 -0.143421053886414 41.0
0 0.45322847366333 42.0
293.23891824 293.881317138672 43.0
0 -0.0400832295417786 44.0
0 0.318760812282562 45.0
0 0.222343623638153 46.0
0 0.0493430495262146 47.0
0 0.171182036399841 48.0
0.56460484691 0.463209509849548 49.0
0.089891975643 0.0221667289733887 50.0
1.9454678792 1.8693231344223 51.0
0 0.134749174118042 52.0
0 -0.0472914576530457 53.0
0 0.240111708641052 54.0
0 0.277657926082611 55.0
0 -0.0930971503257751 56.0
0 0.235292792320251 57.0
0 0.149160802364349 58.0
0 0.0820388793945312 59.0
0 0.0588149428367615 60.0
0 0.123257339000702 61.0
0 0.188812375068665 62.0
227.1505211 256.887908935547 0.0
0 -0.231315732002258 1.0
0 -0.0368392467498779 2.0
0 -0.0423551201820374 3.0
0 -0.129043996334076 4.0
0 -0.328339099884033 5.0
0 0.0664876103401184 6.0
0 0.0814356803894043 7.0
0 -0.92743968963623 8.0
0 0.0559639930725098 9.0
0 0.982570648193359 10.0
0 0.718731284141541 11.0
0 0.0199509263038635 12.0
54.629164531 55.2700347900391 13.0
30.547711843 36.0277862548828 14.0
0 -0.195078730583191 15.0
0 0.0884073972702026 16.0
39.448261244 37.2626342773438 17.0
0 0.157052874565125 18.0
0 -0.0986932516098022 19.0
0 0.176519513130188 20.0
0 -0.0377116203308105 21.0
0 0.230073094367981 22.0
37.44736164 36.5490379333496 23.0
0 0.0625293850898743 24.0
0 0.965473830699921 25.0
0 0.105389595031738 26.0
0 -0.0204805731773376 27.0
0 0.151874542236328 28.0
0 0.0985772013664246 29.0
0 0.0776845216751099 30.0
0 1.13657116889954 31.0
0 -0.308967351913452 32.0
0 -0.15471088886261 33.0
0 -0.0201770067214966 34.0
0 -0.0947003364562988 35.0
0 0.106876969337463 36.0
0 0.117757022380829 37.0
0 0.395763278007507 38.0
0 -0.305593729019165 39.0
0 -0.227028489112854 40.0
0 -0.180854678153992 41.0
0 -0.0441173315048218 42.0
288.43737327 296.785797119141 43.0
0 0.149606943130493 44.0
0 0.0994179248809814 45.0
0 0.246532797813416 46.0
0 -0.11468094587326 47.0
0 0.344964206218719 48.0
0 0.391918838024139 49.0
0.097052147027 0.0332909226417542 50.0
2.0553872314 1.84792172908783 51.0
0 0.0623910427093506 52.0
0 -0.168160617351532 53.0
0 -0.0288991928100586 54.0
0 -0.0208456516265869 55.0
0 0.0103868842124939 56.0
0 -0.0129733681678772 57.0
0 0.153787612915039 58.0
0 0.290818750858307 59.0
0 -0.070152759552002 60.0
0 0.0857944488525391 61.0
0 0.27152681350708 62.0
253.37155263 252.299606323242 0.0
0 -0.114549875259399 1.0
0 -0.142886102199554 2.0
0 0.0853633284568787 3.0
0 -0.456597805023193 4.0
0 0.278139889240265 5.0
0 0.222762763500214 6.0
0 -0.0199373960494995 7.0
0 0.303864568471909 8.0
0 -0.152087986469269 9.0
0 -0.0204706192016602 10.0
0 -0.215681195259094 11.0
0 -0.0552495121955872 12.0
67.207264462 66.8557739257812 13.0
28.29809793 30.976131439209 14.0
0 -0.108035802841187 15.0
0 0.174171924591064 16.0
26.532257296 30.7793426513672 17.0
0 0.0953788757324219 18.0
0 0.0191378593444824 19.0
0 0.0358415246009827 20.0
0 0.203851342201233 21.0
0 0.0496823191642761 22.0
32.482596425 29.1467304229736 23.0
0 -0.0645937919616699 24.0
0 0.544559836387634 25.0
0 -0.122008979320526 26.0
0 -0.111973643302917 27.0
0 -0.0652457475662231 28.0
0 -0.0479276776313782 29.0
0 0.0414013266563416 30.0
0 -0.0708454847335815 31.0
0 -0.0104340314865112 32.0
0 -0.389185905456543 33.0
0 -0.640928864479065 34.0
0 0.186564743518829 35.0
0 0.153089225292206 36.0
0 0.0511057376861572 37.0
0 0.480562746524811 38.0
0 -0.404873728752136 39.0
0 -0.3503577709198 40.0
0 -0.308448910713196 41.0
0 -0.109422922134399 42.0
305.21051439 298.498596191406 43.0
0 0.0986809730529785 44.0
0 0.174577295780182 45.0
0 0.152387678623199 46.0
0 0.146173000335693 47.0
0 0.387140393257141 48.0
0 0.471191436052322 49.0
0.064116596024 0.0551989674568176 50.0
1.6610570194 1.80415201187134 51.0
0 0.0513545870780945 52.0
0 0.196201622486115 53.0
0 0.181457698345184 54.0
0 -0.161369025707245 55.0
0 -0.106734871864319 56.0
0 -0.044094443321228 57.0
0 0.381041944026947 58.0
0 0.328526675701141 59.0
0 -0.0714884400367737 60.0
0 0.0962463021278381 61.0
0 0.292519748210907 62.0
248.55034147 251.072387695312 0.0
0 0.212791085243225 1.0
0 0.000370979309082031 2.0
0 -0.52993106842041 3.0
0 -0.428463935852051 4.0
0 -0.149662494659424 5.0
0 0.0904491543769836 6.0
0 -0.10776150226593 7.0
0 0.0668791532516479 8.0
0 0.503177225589752 9.0
0 0.20779025554657 10.0
0 -0.0943204164505005 11.0
0 -0.101874649524689 12.0
78.786265315 77.6580657958984 13.0
25.028728706 19.0185737609863 14.0
0 -0.0338267087936401 15.0
0 0.105717301368713 16.0
23.565910309 29.1977596282959 17.0
0 0.0619467496871948 18.0
0 -0.0127259492874146 19.0
0 0.287741661071777 20.0
0 0.0266541242599487 21.0
0 0.230957388877869 22.0
31.229943222 30.689302444458 23.0
0 -0.0149177312850952 24.0
0 0.0822744965553284 25.0
0 0.194501638412476 26.0
0 -0.259368062019348 27.0
0 0.627977132797241 28.0
0 0.377608627080917 29.0
0 -0.0786845684051514 30.0
0 0.393915414810181 31.0
0 0.0309047698974609 32.0
0 0.0922212600708008 33.0
0 -0.11681604385376 34.0
0 -0.0976779460906982 35.0
0 0.143077492713928 36.0
0 0.010499894618988 37.0
0 0.0362375974655151 38.0
0 -0.298842549324036 39.0
0 0.101154625415802 40.0
0 -0.302090287208557 41.0
0 -0.132324934005737 42.0
281.8352294 298.278686523438 43.0
0 0.0592406988143921 44.0
0 0.0435163378715515 45.0
0 -0.00284290313720703 46.0
0 0.113401055335999 47.0
0 -0.529892444610596 48.0
0 0.575945019721985 49.0
0.10367579902 -0.0280948281288147 50.0
2.3842427632 1.82258129119873 51.0
0 -0.21921968460083 52.0
0 0.124257802963257 53.0
0 -0.272639513015747 54.0
0 -0.0645641088485718 55.0
0 0.420198559761047 56.0
0 -0.0747813582420349 57.0
0 -0.0325008034706116 58.0
0 0.557033121585846 59.0
0 0.104249775409698 60.0
0 0.0499014854431152 61.0
0 0.0330101847648621 62.0
274.71338076 254.962890625 0.0
0 0.323067545890808 1.0
0 -0.216790199279785 2.0
0 -0.385511875152588 3.0
0 0.0990598797798157 4.0
0 0.0101010799407959 5.0
0 0.0276162624359131 6.0
0 0.453419357538223 7.0
0 -0.209926605224609 8.0
0 0.313389956951141 9.0
0 0.515946388244629 10.0
0 -0.261040210723877 11.0
0 -0.00316882133483887 12.0
48.592168345 46.891529083252 13.0
48.066042103 45.3486366271973 14.0
0 0.021884560585022 15.0
0 0.139740228652954 16.0
31.925967301 32.0643653869629 17.0
0 0.124653100967407 18.0
0 -0.0695809125900269 19.0
0 0.188549041748047 20.0
0 -0.00146299600601196 21.0
0 0.714365899562836 22.0
36.00091564 37.8864784240723 23.0
0 0.0808044672012329 24.0
0 1.03414809703827 25.0
0 0.0620368719100952 26.0
0 -0.0637275576591492 27.0
0 -0.373575091362 28.0
0 -0.03281170129776 29.0
0 0.0287202596664429 30.0
0 -0.17999005317688 31.0
0 -0.0217363834381104 32.0
0 -0.0450875163078308 33.0
0 0.248416006565094 34.0
0 0.152289688587189 35.0
0 0.462086290121078 36.0
0 -0.0823282599449158 37.0
0 0.158368945121765 38.0
0 -0.54964542388916 39.0
0 0.445799469947815 40.0
0 -0.230238318443298 41.0
0 0.206417798995972 42.0
299.78474926 295.966644287109 43.0
0 0.159204959869385 44.0
0 -0.00349462032318115 45.0
0 0.0985905528068542 46.0
0 -0.00351786613464355 47.0
0 0.293037831783295 48.0
0 0.340170502662659 49.0
0.082233883225 0.0708644390106201 50.0
2.1052938168 1.86837220191956 51.0
0 -0.165690422058105 52.0
0 0.0473926663398743 53.0
0 -0.144863903522491 54.0
0 -0.192517280578613 55.0
0 0.0881648063659668 56.0
0 -0.0629627704620361 57.0
0 0.335914850234985 58.0
0 0.247672140598297 59.0
0 -0.23110818862915 60.0
0 0.121666312217712 61.0
0 0.292760133743286 62.0
245.92235752 252.00146484375 0.0
0 -0.129040837287903 1.0
0 -0.249554514884949 2.0
0 0.292922616004944 3.0
0 -0.23767876625061 4.0
0 0.22444486618042 5.0
0 0.0308098793029785 6.0
0 -0.302908658981323 7.0
0 -0.0789110064506531 8.0
0 0.0980855822563171 9.0
0 0.0127769112586975 10.0
0 -0.0278443098068237 11.0
0 -0.195947289466858 12.0
63.611338926 64.0169677734375 13.0
43.487659217 29.0517845153809 14.0
0 0.0148967504501343 15.0
0 0.00913751125335693 16.0
28.186473688 31.9330215454102 17.0
0 0.0701016187667847 18.0
0 -0.00608313083648682 19.0
0 0.129987955093384 20.0
0 -0.0116736888885498 21.0
0 0.252158403396606 22.0
36.111356254 35.9031829833984 23.0
0 0.0660194158554077 24.0
0 0.342429459095001 25.0
0 0.0520123243331909 26.0
0 -0.0890327095985413 27.0
0 0.421847283840179 28.0
0 -0.0790624618530273 29.0
0 0.0561029314994812 30.0
0 -0.0460447072982788 31.0
0 0.150976777076721 32.0
0 0.191794753074646 33.0
0 0.269093930721283 34.0
0 -0.120735287666321 35.0
0 0.0348870158195496 36.0
0 -0.00830948352813721 37.0
0 -0.0260208249092102 38.0
0 -0.292587876319885 39.0
0 0.0272345542907715 40.0
0 -0.136592209339142 41.0
0 0.0355933904647827 42.0
283.44922613 297.280578613281 43.0
0 -0.00433474779129028 44.0
0 0.159807324409485 45.0
0 0.251847088336945 46.0
0 -0.126485168933868 47.0
0 0.0898461937904358 48.0
2.4850982791 0.463044077157974 49.0
0.094109062831 0.0290524959564209 50.0
2.3624543252 1.89384043216705 51.0
0 0.284684062004089 52.0
0 0.149728238582611 53.0
0 -0.0751038789749146 54.0
0 0.0177738666534424 55.0
0 -0.05323326587677 56.0
0 -0.0117800831794739 57.0
0 0.330074816942215 58.0
0 -0.220235466957092 59.0
0 0.318746089935303 60.0
0 0.079648494720459 61.0
0 0.143901407718658 62.0
236.23977496 252.857940673828 0.0
0 -0.147156119346619 1.0
0 -0.0543696880340576 2.0
0 -0.377645254135132 3.0
0 -0.473527073860168 4.0
0 -0.184461414813995 5.0
0 0.102086901664734 6.0
0 -0.0796255469322205 7.0
0 0.0510956048965454 8.0
0 0.550649762153625 9.0
0 0.286661982536316 10.0
0 -0.0501915812492371 11.0
0 -0.047898530960083 12.0
62.305242826 63.0850067138672 13.0
38.160287119 34.8796997070312 14.0
0 -0.0388615131378174 15.0
0 0.170875310897827 16.0
41.464854029 38.5688323974609 17.0
0 0.205747663974762 18.0
0 -0.00284820795059204 19.0
0 -0.0171697735786438 20.0
0 0.111954510211945 21.0
0 0.110746383666992 22.0
28.961416808 29.7075290679932 23.0
0 0.0654708743095398 24.0
0 0.491304516792297 25.0
0 -0.110933721065521 26.0
0 -0.0393655300140381 27.0
0 -0.00574475526809692 28.0
0 -0.0297394394874573 29.0
0 0.145253121852875 30.0
0 -0.00860428810119629 31.0
0 -0.00469827651977539 32.0
0 -0.0794498324394226 33.0
0 -1.65371465682983 34.0
0 -0.119160652160645 35.0
0 -0.00487065315246582 36.0
0 0.0171725749969482 37.0
0 0.175833582878113 38.0
0 -0.227520823478699 39.0
0 0.687025368213654 40.0
0 -0.127019882202148 41.0
0 0.0443217754364014 42.0
300.86026577 295.11279296875 43.0
0 0.0903880000114441 44.0
0 -0.00156813859939575 45.0
0 0.0581156611442566 46.0
0 0.289276361465454 47.0
0 0.274823427200317 48.0
0 0.353341519832611 49.0
0.083252628919 0.091966986656189 50.0
2.1624799006 1.85780441761017 51.0
0 -0.143608152866364 52.0
0 0.337530374526978 53.0
0 0.0129473805427551 54.0
0 0.304997682571411 55.0
0 0.245126187801361 56.0
0 0.332036972045898 57.0
0 0.19396710395813 58.0
0 0.106737732887268 59.0
0 -0.19534307718277 60.0
0 0.0927289724349976 61.0
0 0.181416094303131 62.0
251.6335405 255.504623413086 0.0
0 -0.103434920310974 1.0
0 0.147720158100128 2.0
0 0.125614821910858 3.0
0 -0.135362088680267 4.0
0 0.272274911403656 5.0
0 0.112990915775299 6.0
0 0.133012235164642 7.0
0 0.442314684391022 8.0
0 0.055622398853302 9.0
0 0.220308184623718 10.0
0 -0.0708826184272766 11.0
0 -0.0801542401313782 12.0
68.743829998 71.2883453369141 13.0
26.677010757 26.9273052215576 14.0
0 -0.105676293373108 15.0
0 -0.0634782314300537 16.0
29.639555081 34.4419136047363 17.0
0 0.127770721912384 18.0
0 -0.000590324401855469 19.0
0 0.0459085702896118 20.0
0 0.21309757232666 21.0
0 -0.100662767887115 22.0
36.124095328 33.7745170593262 23.0
0 0.0968912839889526 24.0
0 0.831925749778748 25.0
0 0.176388382911682 26.0
0 -0.0472767949104309 27.0
0 0.581743717193604 28.0
0 0.0110806822776794 29.0
0 0.162437379360199 30.0
0 0.00378775596618652 31.0
0 0.0701629519462585 32.0
0 0.00116205215454102 33.0
0 0.183490216732025 34.0
0 0.0951849818229675 35.0
0 -0.0288441777229309 36.0
0 0.0200695991516113 37.0
0 0.505843043327332 38.0
0 -0.201486229896545 39.0
0 -0.0598884224891663 40.0
0 -0.235939979553223 41.0
0 0.35131824016571 42.0
287.72703475 295.080535888672 43.0
0 0.078747034072876 44.0
0 0.0946851372718811 45.0
0 0.0284156203269958 46.0
0 0.0781322121620178 47.0
0 0.10479074716568 48.0
0 0.299566894769669 49.0
0.083055145228 0.0104024410247803 50.0
2.3711110216 1.91951084136963 51.0
0 -0.100869834423065 52.0
0 -0.0792078375816345 53.0
0 0.146929025650024 54.0
0 0.311550378799438 55.0
0 0.0314779281616211 56.0
0 0.809405326843262 57.0
0 0.131988823413849 58.0
0 0.0865359902381897 59.0
0 -0.0741382837295532 60.0
0 0.0767040252685547 61.0
0 0.161070883274078 62.0
275.88544297 252.447509765625 0.0
0 -0.117578387260437 1.0
0 -0.0434309244155884 2.0
0 0.19000518321991 3.0
0 0.170390903949738 4.0
0 -0.170718908309937 5.0
0 0.129758954048157 6.0
0 0.27048921585083 7.0
0 -0.263019800186157 8.0
0 0.382426023483276 9.0
0 -0.306242704391479 10.0
0 0.115710973739624 11.0
0 -0.156021177768707 12.0
70.721540296 70.4834136962891 13.0
35.210300513 29.9514503479004 14.0
0 -0.0198111534118652 15.0
0 0.0560297966003418 16.0
30.22864456 30.7721385955811 17.0
0 0.108422875404358 18.0
0 -0.0154536962509155 19.0
0 0.0866124629974365 20.0
0 0.181899070739746 21.0
0 0.23717075586319 22.0
29.910647956 28.5973491668701 23.0
0 0.0762222409248352 24.0
0 0.525602102279663 25.0
0 -0.0767163634300232 26.0
0 -0.124554991722107 27.0
0 0.645345687866211 28.0
0 0.0763912796974182 29.0
0 0.0647650361061096 30.0
0 0.0200599431991577 31.0
0 -0.0293515920639038 32.0
0 -0.0489220023155212 33.0
0 0.223031342029572 34.0
0 -0.121018350124359 35.0
0 0.246375203132629 36.0
0 0.0654377341270447 37.0
0 0.266924500465393 38.0
0 -0.440396904945374 39.0
0 0.029894232749939 40.0
0 -0.2005255818367 41.0
0 0.389214277267456 42.0
305.76546092 293.333190917969 43.0
0 0.0828920602798462 44.0
0 0.152756452560425 45.0
0 0.12173193693161 46.0
0 0.0466291308403015 47.0
0 0.0270408987998962 48.0
0 0.406628429889679 49.0
0.09097757744 0.0292574763298035 50.0
1.6970982364 1.79919266700745 51.0
0 -0.22074830532074 52.0
0 -0.0304405093193054 53.0
0 -0.0574923753738403 54.0
0 0.0475153923034668 55.0
0 -0.106966853141785 56.0
0 0.0315690636634827 57.0
0 0.170335233211517 58.0
0 -0.0943615436553955 59.0
0 -0.00470495223999023 60.0
0 0.0868706703186035 61.0
0 0.172424912452698 62.0
256.11859023 253.319519042969 0.0
0 -0.309353232383728 1.0
0 -0.0810763835906982 2.0
0 -0.00996023416519165 3.0
0 0.090612530708313 4.0
0 0.0807756781578064 5.0
0 0.123499870300293 6.0
0 0.15071702003479 7.0
0 -0.0453726649284363 8.0
0 0.00790363550186157 9.0
0 0.225029110908508 10.0
0 -0.208236396312714 11.0
0 -0.23234760761261 12.0
61.406157636 61.4589920043945 13.0
32.150833402 36.4509315490723 14.0
0 -0.19277548789978 15.0
0 -0.13904219865799 16.0
30.327468976 33.8932876586914 17.0
0 0.0546430349349976 18.0
0 0.0174903273582458 19.0
0 0.0922370553016663 20.0
0 0.00218325853347778 21.0
0 0.0395135283470154 22.0
40.707992043 39.0983009338379 23.0
0 0.00646024942398071 24.0
0 0.0145257115364075 25.0
0 -0.0523736476898193 26.0
0 -0.0215649604797363 27.0
0 0.0817000865936279 28.0
0 -0.0681130290031433 29.0
0 0.0533332824707031 30.0
0 0.0435405969619751 31.0
0 0.0234100818634033 32.0
0 -0.133740901947021 33.0
0 0.270440697669983 34.0
0 0.410854518413544 35.0
0 -0.000566959381103516 36.0
0 0.170199573040009 37.0
0 -0.202806949615479 38.0
0 -0.0949662923812866 39.0
0 0.559590458869934 40.0
0 -0.281421780586243 41.0
0 0.0745648145675659 42.0
289.76088901 294.505462646484 43.0
0 0.0937026739120483 44.0
0 -0.0392619371414185 45.0
0 0.0550068020820618 46.0
0 -0.00808733701705933 47.0
0 0.323932766914368 48.0
1.0488815448 0.42809784412384 49.0
0.080977034108 -0.00564455986022949 50.0
2.2198448049 1.92207396030426 51.0
0 -0.0848929882049561 52.0
0 -0.330764412879944 53.0
0 0.167810559272766 54.0
0 0.0137093067169189 55.0
0 -0.0460000038146973 56.0
0 -0.00518256425857544 57.0
0 0.0599332451820374 58.0
0 0.307881653308868 59.0
0 0.194394409656525 60.0
0 0.0466021299362183 61.0
0 0.109444379806519 62.0
225.35791818 252.490417480469 0.0
0 -0.290315270423889 1.0
0 -0.226502299308777 2.0
0 -1.09608697891235 3.0
0 -0.41328239440918 4.0
0 -0.32580828666687 5.0
0 0.1510329246521 6.0
0 -0.130991399288177 7.0
0 -0.228597283363342 8.0
0 1.01614546775818 9.0
0 0.984222292900085 10.0
0 -0.101979076862335 11.0
0 -0.113714933395386 12.0
74.140283542 73.6851654052734 13.0
8.2389699424 25.4817886352539 14.0
0 -0.195622086524963 15.0
0 -0.00089716911315918 16.0
27.284019722 32.210807800293 17.0
0 0.082542896270752 18.0
0 -0.0145472884178162 19.0
0 0.122030854225159 20.0
0 0.0469316244125366 21.0
0 0.209935486316681 22.0
33.992217249 31.7529354095459 23.0
0 -0.100641429424286 24.0
0 0.436805009841919 25.0
0 -0.0327385663986206 26.0
0 -0.109833180904388 27.0
0 -0.460807323455811 28.0
0 -0.0203626155853271 29.0
0 -0.0418581366539001 30.0
0 -0.103784680366516 31.0
0 -0.15452516078949 32.0
0 -0.246592164039612 33.0
0 -0.641547203063965 34.0
0 0.351778447628021 35.0
0 0.097423791885376 36.0
0 -0.0758127570152283 37.0
0 0.382317870855331 38.0
0 -0.282328724861145 39.0
0 -0.310327768325806 40.0
0 -0.203692555427551 41.0
0 0.0607137680053711 42.0
301.78185481 295.258178710938 43.0
0 0.0358992218971252 44.0
0 0.0947003364562988 45.0
0 0.187479078769684 46.0
0 -0.313017725944519 47.0
0 -0.592469930648804 48.0
0 0.518118739128113 49.0
0.082599040763 0.000584721565246582 50.0
2.1902961354 1.84673237800598 51.0
0 0.0248115062713623 52.0
0 -0.403541922569275 53.0
0 0.188248038291931 54.0
0 0.201967477798462 55.0
0 0.271363854408264 56.0
0 0.490370750427246 57.0
0 0.162432014942169 58.0
0 0.0545815825462341 59.0
0 0.414798200130463 60.0
0 -0.00564694404602051 61.0
0 0.105354428291321 62.0
222.2919626 254.416015625 0.0
0 -0.140599608421326 1.0
0 -0.347168922424316 2.0
0 0.113560676574707 3.0
0 -0.214088916778564 4.0
0 -0.0035662055015564 5.0
0 -0.261090159416199 6.0
0 -0.260311245918274 7.0
0 0.467872738838196 8.0
0 -0.04071044921875 9.0
0 0.0233450531959534 10.0
0 -0.195817291736603 11.0
0 -0.0964558124542236 12.0
77.473098842 74.7223815917969 13.0
12.334580639 21.6380615234375 14.0
0 -0.0353035926818848 15.0
0 -0.0716819763183594 16.0
35.764464649 36.2105674743652 17.0
0 0.0959321260452271 18.0
0 -0.0100350975990295 19.0
0 0.0296245813369751 20.0
0 -0.0234434604644775 21.0
0 0.38667219877243 22.0
32.733605826 34.9518203735352 23.0
0 0.173068404197693 24.0
0 1.09137117862701 25.0
0 -0.113783478736877 26.0
0 0.0955148935317993 27.0
0 0.0265641212463379 28.0
0 -0.058812141418457 29.0
0 0.0388726592063904 30.0
0 0.161474466323853 31.0
0 0.170291364192963 32.0
0 0.201195776462555 33.0
0 0.388901174068451 34.0
0 0.300278961658478 35.0
0 0.191888153553009 36.0
0 0.00452703237533569 37.0
0 0.0653696656227112 38.0
0 -0.183745265007019 39.0
0 0.709090113639832 40.0
0 -0.097125768661499 41.0
0 -0.0583120584487915 42.0
284.17695069 296.2744140625 43.0
0 -0.0254663228988647 44.0
0 0.140748500823975 45.0
0 0.00715655088424683 46.0
0 0.144491195678711 47.0
0 0.247751414775848 48.0
0 0.424980372190475 49.0
0.075431432874 0.0448850393295288 50.0
2.0171134421 1.9000723361969 51.0
0 0.332454591989517 52.0
0 1.82121872901917 53.0
0 0.158121705055237 54.0
0 0.302862197160721 55.0
0 -0.000355362892150879 56.0
0 0.365663170814514 57.0
0 0.000272095203399658 58.0
0 -0.145697951316833 59.0
0 0.56348991394043 60.0
0 0.115442097187042 61.0
0 0.144462645053864 62.0
250.40868616 251.000106811523 0.0
0 -0.261766195297241 1.0
0 0.192239999771118 2.0
0 0.027620792388916 3.0
0 0.268806636333466 4.0
0 -0.0486164689064026 5.0
0 0.294014811515808 6.0
0 0.265748143196106 7.0
0 -1.15606284141541 8.0
0 -0.940146088600159 9.0
0 1.59659862518311 10.0
0 0.964253306388855 11.0
0 -0.158970415592194 12.0
48.986813036 46.8517913818359 13.0
34.521160553 37.4072036743164 14.0
0 -0.248424887657166 15.0
0 -0.141192317008972 16.0
28.579983834 31.9168128967285 17.0
0 0.166559457778931 18.0
0 -0.0811823606491089 19.0
0 0.282261729240417 20.0
0 -0.00334072113037109 21.0
0 0.34812393784523 22.0
32.437606246 32.263500213623 23.0
0 0.022347092628479 24.0
20.612312119 0.452981650829315 25.0
0 0.0455712676048279 26.0
0 -0.26462185382843 27.0
0 -0.00861823558807373 28.0
0 0.158098101615906 29.0
0 0.142492353916168 30.0
0 -0.285172581672668 31.0
0 -0.0654042959213257 32.0
0 -0.448029041290283 33.0
0 0.0691754817962646 34.0
0 -0.0949572324752808 35.0
0 0.00360143184661865 36.0
0 0.0514619946479797 37.0
0 -0.0748804807662964 38.0
0 -0.246469855308533 39.0
0 0.00435596704483032 40.0
0 0.00116479396820068 41.0
0 0.0312644243240356 42.0
309.11879795 298.438720703125 43.0
0 0.165178179740906 44.0
0 0.145915269851685 45.0
0 0.41577935218811 46.0
0 -0.43037223815918 47.0
0 -0.342034459114075 48.0
0 0.509755969047546 49.0
0.061378010895 0.0172705054283142 50.0
1.3499419302 1.78888654708862 51.0
0 -0.0514211654663086 52.0
0 -0.240436553955078 53.0
0 0.303897887468338 54.0
0 0.175678670406342 55.0
0 0.346936941146851 56.0
0 -0.00127029418945312 57.0
0 0.368033349514008 58.0
0 0.313574731349945 59.0
0 -0.225771903991699 60.0
0 0.0554453730583191 61.0
0 0.15974348783493 62.0
250.49109626 253.812072753906 0.0
0 -0.239948630332947 1.0
0 -0.221853375434875 2.0
0 0.21160614490509 3.0
0 -0.118154525756836 4.0
0 0.0549991130828857 5.0
0 0.0414066314697266 6.0
0 0.242757976055145 7.0
0 0.125629782676697 8.0
0 0.316672384738922 9.0
0 0.200887382030487 10.0
0 -0.157792806625366 11.0
0 0.0273904800415039 12.0
65.127533516 67.0040969848633 13.0
36.304656101 28.6816997528076 14.0
0 -0.0791577100753784 15.0
0 -0.112286031246185 16.0
32.931399479 33.6075820922852 17.0
0 0.0705198049545288 18.0
0 0.000797569751739502 19.0
0 0.112253725528717 20.0
0 0.108483076095581 21.0
0 0.173316657543182 22.0
36.4068468 34.868293762207 23.0
0 0.0386018753051758 24.0
0 0.433268249034882 25.0
0 -0.0619485378265381 26.0
0 0.000360965728759766 27.0
0 0.050836443901062 28.0
0 -0.145687401294708 29.0
0 0.0897251963615417 30.0
0 -0.0354897975921631 31.0
0 -0.0379577875137329 32.0
0 -0.289858818054199 33.0
0 -0.0340733528137207 34.0
0 -0.0238188505172729 35.0
0 -0.0964455604553223 36.0
0 0.1395183801651 37.0
0 0.0799668431282043 38.0
0 -0.204154491424561 39.0
0 0.33424574136734 40.0
0 -0.125821530818939 41.0
0 -0.0468109846115112 42.0
303.72890241 297.245635986328 43.0
0 0.0739596486091614 44.0
0 0.0589523911476135 45.0
0 0.107701063156128 46.0
0 0.0221506357192993 47.0
0 -0.550437092781067 48.0
0 0.420520484447479 49.0
0.082687637955 -0.0147259831428528 50.0
1.948681639 1.86697483062744 51.0
0 0.00103354454040527 52.0
0 -0.0696409344673157 53.0
0 0.178279995918274 54.0
0 -0.0411602854728699 55.0
0 0.0246838331222534 56.0
0 0.220615983009338 57.0
0 0.124128520488739 58.0
0 -0.041070818901062 59.0
0 0.0701518058776855 60.0
0 0.0740706920623779 61.0
0 0.137532353401184 62.0
229.26889286 251.069900512695 0.0
0 -0.265161037445068 1.0
0 -0.200834631919861 2.0
0 -0.392767310142517 3.0
0 -0.25917649269104 4.0
0 -0.351010680198669 5.0
0 0.0281761288642883 6.0
0 -0.217284321784973 7.0
0 -0.18612152338028 8.0
0 0.239534795284271 9.0
0 0.396060943603516 10.0
0 0.676802933216095 11.0
0 0.0054861307144165 12.0
76.716671706 74.9652252197266 13.0
33.1084879 26.154088973999 14.0
0 -0.104422926902771 15.0
0 0.310958981513977 16.0
28.217203973 35.5846061706543 17.0
0 0.00571310520172119 18.0
0 0.0581411123275757 19.0
0 0.0637662410736084 20.0
0 0.0920326709747314 21.0
0 0.162187993526459 22.0
33.852899347 31.6415348052979 23.0
0 0.033538281917572 24.0
0 0.874395906925201 25.0
0 -0.0852159261703491 26.0
0 0.0419418811798096 27.0
0 -0.467833042144775 28.0
0 -0.135966777801514 29.0
0 -0.0969305634498596 30.0
0 0.173452615737915 31.0
0 0.0160368680953979 32.0
0 -0.291552305221558 33.0
0 0.266653597354889 34.0
0 0.192269802093506 35.0
0 0.335180521011353 36.0
0 -0.0883786082267761 37.0
0 0.0534115433692932 38.0
0 -0.551863789558411 39.0
0 0.736261665821075 40.0
0 -0.177418768405914 41.0
0 0.0958740711212158 42.0
293.81677313 294.523010253906 43.0
0 0.0745655298233032 44.0
0 0.121308088302612 45.0
0 0.164553046226501 46.0
0 0.00688815116882324 47.0
0 -0.749027490615845 48.0
0 0.549752056598663 49.0
0.10014390631 -0.0661467909812927 50.0
1.8850201977 1.92427277565002 51.0
0 0.175271272659302 52.0
0 -0.148504376411438 53.0
0 0.319867551326752 54.0
0 0.16775506734848 55.0
0 0.0373300909996033 56.0
0 0.356961011886597 57.0
0 0.537041306495667 58.0
0 -0.309259414672852 59.0
0 -0.20466810464859 60.0
0 0.0392112731933594 61.0
0 0.123308420181274 62.0
269.68141378 251.896240234375 0.0
0 -0.166424512863159 1.0
0 -0.0197104215621948 2.0
0 0.500617861747742 3.0
0 -0.131688058376312 4.0
0 -0.0246407985687256 5.0
0 -0.177229702472687 6.0
0 0.254720449447632 7.0
0 -0.0527729988098145 8.0
0 0.315463125705719 9.0
0 -0.507918357849121 10.0
0 -0.243119597434998 11.0
0 -0.0840756297111511 12.0
55.552462707 55.4089279174805 13.0
39.319588548 34.4000663757324 14.0
0 -0.0209990739822388 15.0
0 -0.0644538998603821 16.0
33.709039357 33.793041229248 17.0
0 0.127209663391113 18.0
0 0.0478187799453735 19.0
0 0.090967059135437 20.0
0 -0.00973868370056152 21.0
0 0.300608694553375 22.0
37.436243462 35.5063133239746 23.0
0 0.0751940011978149 24.0
0 0.808501362800598 25.0
0 0.0358819365501404 26.0
0 -0.165548503398895 27.0
0 0.790946841239929 28.0
0 -0.090714156627655 29.0
0 0.0823898315429688 30.0
0 0.0101642608642578 31.0
0 0.0254976153373718 32.0
0 -0.0988452434539795 33.0
0 0.181186676025391 34.0
0 0.346141785383224 35.0
0 -0.147414565086365 36.0
0 0.14085191488266 37.0
0 0.0388635396957397 38.0
0 -0.0993208885192871 39.0
0 0.720590889453888 40.0
0 -0.113023042678833 41.0
0 0.113407850265503 42.0
294.64026527 295.307373046875 43.0
0 0.159460783004761 44.0
0 0.183458864688873 45.0
0 0.321506977081299 46.0
0 -0.509451985359192 47.0
0 0.394584000110626 48.0
0.59764420091 0.498376905918121 49.0
0.086474010776 0.0367860198020935 50.0
1.7747647144 1.88362860679626 51.0
0 -0.143571555614471 52.0
0 -0.504554986953735 53.0
0 -0.000898778438568115 54.0
0 0.25566565990448 55.0
0 0.179131865501404 56.0
0 0.0232130289077759 57.0
0 0.273587822914124 58.0
0 0.0510485768318176 59.0
0 0.204111218452454 60.0
0 0.134641945362091 61.0
0 0.215712785720825 62.0
291.08171095 253.150100708008 0.0
0 0.205832958221436 1.0
0 -0.137763500213623 2.0
0 0.182329058647156 3.0
0 -0.106135368347168 4.0
0 0.232070922851562 5.0
0 0.297288358211517 6.0
0 0.446556508541107 7.0
0 -0.118195712566376 8.0
0 -0.138012886047363 9.0
0 0.0591163039207458 10.0
0 0.733744740486145 11.0
0 0.0813339948654175 12.0
53.78858457 52.7382774353027 13.0
27.039991377 38.254508972168 14.0
0 -0.108723759651184 15.0
0 0.163480937480927 16.0
27.122923807 32.7682800292969 17.0
0 0.17328155040741 18.0
0 -0.0212162733078003 19.0
0 -0.00736767053604126 20.0
0 0.231001198291779 21.0
0 0.102763772010803 22.0
26.368936495 27.9871101379395 23.0
0 0.152825713157654 24.0
0 1.16967487335205 25.0
0 0.0393227934837341 26.0
0 -0.357341766357422 27.0
0 0.228697776794434 28.0
0 0.0330566763877869 29.0
0 0.0315893292427063 30.0
0 -0.166796684265137 31.0
0 -0.0844101309776306 32.0
0 -0.10695493221283 33.0
0 0.38334932923317 34.0
0 0.213172197341919 35.0
0 0.558373928070068 36.0
0 0.0882891416549683 37.0
0 -0.0149526596069336 38.0
0 -0.196328043937683 39.0
0 0.279355406761169 40.0
0 -0.0583614110946655 41.0
0 0.0717779397964478 42.0
305.87768424 295.026336669922 43.0
0 0.111804604530334 44.0
0 -0.0157886743545532 45.0
0 0.220776915550232 46.0
0 -0.100311934947968 47.0
0 0.275569260120392 48.0
0 0.471948593854904 49.0
0.068253031613 0.0523209571838379 50.0
1.5096010009 1.88023948669434 51.0
0 -0.185492336750031 52.0
0 -0.166891515254974 53.0
0 -0.0198159217834473 54.0
0 -0.169434309005737 55.0
0 -0.121543288230896 56.0
0 -0.131240248680115 57.0
0 0.0730947852134705 58.0
0 0.133517146110535 59.0
0 -0.19481486082077 60.0
0 0.0664219260215759 61.0
0 0.140947580337524 62.0
307.75593221 254.152420043945 0.0
0 -0.360378861427307 1.0
0 0.444006502628326 2.0
0 -0.560419082641602 3.0
0 -0.193444430828094 4.0
0 0.515464305877686 5.0
0 0.821786165237427 6.0
0 0.865301728248596 7.0
0 0.219036221504211 8.0
0 0.210876286029816 9.0
0 -0.0552394986152649 10.0
0 -0.0709412693977356 11.0
0 -0.670619487762451 12.0
51.086684797 49.2782287597656 13.0
33.310005547 38.0853691101074 14.0
0 -0.0766818523406982 15.0
0 0.012048602104187 16.0
30.008862643 29.4987010955811 17.0
0 0.0178403258323669 18.0
0 -0.00485843420028687 19.0
0 0.105630815029144 20.0
0 0.21868634223938 21.0
0 0.159432172775269 22.0
30.729808814 29.3194141387939 23.0
0 0.139589548110962 24.0
0 1.12641727924347 25.0
0 -0.131807625293732 26.0
0 0.0389468669891357 27.0
0 0.589126884937286 28.0
0 -0.105308294296265 29.0
0 0.040316641330719 30.0
0 1.02109694480896 31.0
0 -0.109809935092926 32.0
0 -0.622717261314392 33.0
0 -0.646697402000427 34.0
0 0.265555620193481 35.0
0 0.388702422380447 36.0
0 0.0115466117858887 37.0
0 0.205542027950287 38.0
0 -0.0836989879608154 39.0
0 0.563639521598816 40.0
0 -0.0317345857620239 41.0
0 0.0758332014083862 42.0
307.12761265 295.637542724609 43.0
0 0.164199709892273 44.0
0 -0.0262722373008728 45.0
0 0.127845764160156 46.0
0 0.226443648338318 47.0
0 0.32622554898262 48.0
0.97733821256 0.624120593070984 49.0
0.092395832605 0.113923251628876 50.0
2.2745948555 1.85370564460754 51.0
0 0.0193632841110229 52.0
0 0.240504086017609 53.0
0 0.14338743686676 54.0
0 0.36339282989502 55.0
0 -0.12688535451889 56.0
0 0.212669014930725 57.0
0 0.419419437646866 58.0
0 0.128583192825317 59.0
0 -0.304599523544312 60.0
0 -0.00550419092178345 61.0
0 0.195855557918549 62.0
279.76466269 249.304809570312 0.0
0 -0.273305416107178 1.0
0 -0.185143947601318 2.0
0 -0.0692093968391418 3.0
0 -0.321411967277527 4.0
0 0.129161477088928 5.0
0 -0.127595186233521 6.0
0 0.466721773147583 7.0
0 0.440470159053802 8.0
0 0.128051042556763 9.0
0 0.0660586357116699 10.0
0 -0.315561890602112 11.0
0 -0.0134166479110718 12.0
56.702884893 58.0317230224609 13.0
30.215250916 35.0638694763184 14.0
0 -0.38394570350647 15.0
0 0.0548557043075562 16.0
32.583872401 33.3668403625488 17.0
0 0.1277174949646 18.0
0 0.0117781162261963 19.0
0 0.0983206033706665 20.0
0 -0.011005163192749 21.0
0 0.0821366310119629 22.0
38.783805736 37.4956741333008 23.0
0 0.0739867687225342 24.0
0 0.470468997955322 25.0
0 -0.0451561212539673 26.0
0 -0.0538877844810486 27.0
0 0.363564848899841 28.0
0 -0.0839279294013977 29.0
0 0.0110896229743958 30.0
0 0.0774704217910767 31.0
0 0.0467283129692078 32.0
0 0.00859200954437256 33.0
0 -0.0278639793395996 34.0
0 0.208886682987213 35.0
0 0.291143834590912 36.0
0 0.207143127918243 37.0
0 -0.0281212329864502 38.0
0 -0.339932322502136 39.0
0 0.256867945194244 40.0
0 -0.178067326545715 41.0
0 -0.0174695253372192 42.0
287.83828545 296.354034423828 43.0
0 -0.0292311310768127 44.0
0 0.191429376602173 45.0
0 0.126800298690796 46.0
0 -0.212122678756714 47.0
0 0.318507820367813 48.0
0 0.495177388191223 49.0
0.065079268487 0.0942898988723755 50.0
1.6267071442 1.8374354839325 51.0
0 -0.174292683601379 52.0
0 0.203585505485535 53.0
0 0.0803492069244385 54.0
0 -0.0382730960845947 55.0
0 0.0387436747550964 56.0
0 0.331426739692688 57.0
0 0.348168462514877 58.0
0 0.181974470615387 59.0
0 -0.140360891819 60.0
0 0.0358495712280273 61.0
0 0.187367796897888 62.0
274.38931176 253.456497192383 0.0
0 -0.0338056087493896 1.0
0 0.0840893983840942 2.0
0 0.31313419342041 3.0
0 0.109889328479767 4.0
0 0.236503839492798 5.0
0 -0.00450170040130615 6.0
0 0.509866416454315 7.0
0 0.600117683410645 8.0
0 -0.24730384349823 9.0
0 -0.0968580842018127 10.0
0 -0.223843932151794 11.0
0 -0.0375632047653198 12.0
50.596074853 50.9501533508301 13.0
37.448987404 42.5498313903809 14.0
0 -0.00944066047668457 15.0
0 0.0298068523406982 16.0
36.513332711 32.5236396789551 17.0
0 0.111920595169067 18.0
0 0.0121842622756958 19.0
0 0.189387202262878 20.0
0 0.085368812084198 21.0
0 0.18458878993988 22.0
34.967581683 34.2438659667969 23.0
0 0.15179181098938 24.0
0 0.71302193403244 25.0
0 -0.0104143619537354 26.0
0 -0.0364121198654175 27.0
0 0.176282167434692 28.0
0 -0.147540330886841 29.0
0 0.0613736510276794 30.0
0 -0.110233902931213 31.0
0 -0.0508460998535156 32.0
0 -0.192353665828705 33.0
0 -0.0305178761482239 34.0
0 -0.0818094611167908 35.0
0 0.168621063232422 36.0
0 0.0968129634857178 37.0
0 -0.209058821201324 38.0
0 -0.118204116821289 39.0
0 0.0171413421630859 40.0
0 -0.145444929599762 41.0
0 0.135942220687866 42.0
307.84354457 295.772888183594 43.0
0 0.147979497909546 44.0
0 0.1288743019104 45.0
0 0.206843793392181 46.0
0 0.0665277242660522 47.0
0 0.388256758451462 48.0
0 0.439877837896347 49.0
0.097997714581 0.0275708436965942 50.0
2.2543396636 1.86742770671844 51.0
0 -0.106613874435425 52.0
0 -0.0302271246910095 53.0
0 0.00798702239990234 54.0
0 0.14154052734375 55.0
0 -0.190876960754395 56.0
0 0.121848404407501 57.0
0 0.261653244495392 58.0
0 0.108352661132812 59.0
0 -0.221952080726624 60.0
0 -0.0394037961959839 61.0
0 0.0407170653343201 62.0
255.92779805 251.904067993164 0.0
0 -0.136675953865051 1.0
0 -0.136713802814484 2.0
0 0.0838713049888611 3.0
0 -0.297648429870605 4.0
0 0.0736403465270996 5.0
0 0.30097097158432 6.0
0 0.260482907295227 7.0
0 0.225297212600708 8.0
0 -0.157108545303345 9.0
0 0.1528280377388 10.0
0 0.00507068634033203 11.0
0 -0.0614891052246094 12.0
54.109720708 55.4086418151855 13.0
47.114327095 41.8870658874512 14.0
0 -0.0563560724258423 15.0
0 -0.0142262578010559 16.0
39.72562017 35.9749717712402 17.0
0 0.17319643497467 18.0
0 -0.024962842464447 19.0
0 0.0177547931671143 20.0
0 0.18669867515564 21.0
0 0.0831519961357117 22.0
29.470716346 29.1013202667236 23.0
0 0.085906982421875 24.0
0 0.91496741771698 25.0
0 0.0334801077842712 26.0
0 -0.145134270191193 27.0
0 -0.145938754081726 28.0
0 -0.107617497444153 29.0
0 0.0295345783233643 30.0
0 -0.281148910522461 31.0
0 -0.168061077594757 32.0
0 -0.187562763690948 33.0
0 0.373199135065079 34.0
0 -0.0217218399047852 35.0
0 -0.343229651451111 36.0
0 0.179753065109253 37.0
0 0.038906455039978 38.0
0 -0.2552490234375 39.0
0 0.571855485439301 40.0
0 -0.105442464351654 41.0
0 -0.0586651563644409 42.0
296.15998686 297.632537841797 43.0
0 0.164320707321167 44.0
0 0.0908983945846558 45.0
0 0.196682751178741 46.0
0 -0.346995830535889 47.0
0 0.34537398815155 48.0
0 0.493995845317841 49.0
0.0681768052 0.0353425741195679 50.0
2.206507185 1.84186983108521 51.0
0 0.159162223339081 52.0
0 -0.0200716853141785 53.0
0 -0.13577538728714 54.0
0 -0.0584163665771484 55.0
0 0.012852668762207 56.0
0 0.377262592315674 57.0
0 0.316365092992783 58.0
0 0.0383938550949097 59.0
0 -0.145481467247009 60.0
0 0.0712153315544128 61.0
0 0.166536748409271 62.0
252.34967923 252.655548095703 0.0
0 0.473336577415466 1.0
0 0.0249375700950623 2.0
0 -0.066243588924408 3.0
0 -0.273264527320862 4.0
0 0.104935944080353 5.0
0 0.0290022492408752 6.0
0 -0.119408190250397 7.0
0 0.0861494541168213 8.0
0 -0.0253514647483826 9.0
0 -0.145653784275055 10.0
0 -0.185452461242676 11.0
0 -0.160236954689026 12.0
46.762597964 44.9814987182617 13.0
35.003813457 40.2989349365234 14.0
0 -0.0551131963729858 15.0
0 0.291206121444702 16.0
39.053064695 36.4007987976074 17.0
0 0.306782305240631 18.0
0 -0.0901079177856445 19.0
0 0.544353842735291 20.0
0 -0.101122200489044 21.0
0 0.0224757790565491 22.0
40.406945709 41.9765625 23.0
0 0.0291959047317505 24.0
0 0.820621848106384 25.0
0 0.158701181411743 26.0
0 -0.263276338577271 27.0
0 0.2464799284935 28.0
0 0.0433214902877808 29.0
0 0.0683831572532654 30.0
0 0.136296153068542 31.0
0 0.0964118242263794 32.0
0 0.114090740680695 33.0
0 -0.0346366763114929 34.0
0 -0.126009106636047 35.0
0 0.316763520240784 36.0
0 0.00630044937133789 37.0
0 0.287788271903992 38.0
0 -0.209336161613464 39.0
0 -0.0676626563072205 40.0
0 -0.208824872970581 41.0
0 0.026250958442688 42.0
292.8233387 295.139373779297 43.0
0 0.0585222840309143 44.0
0 0.185543239116669 45.0
0 0.266926050186157 46.0
0 -0.612205505371094 47.0
0 0.260917067527771 48.0
0 0.318110257387161 49.0
0.099050126349 0.0879995822906494 50.0
1.6664045094 1.78267693519592 51.0
0 0.250684857368469 52.0
0 0.338296592235565 53.0
0 -0.0107449293136597 54.0
0 -0.28027617931366 55.0
0 0.0237631797790527 56.0
0 -0.434422135353088 57.0
0 0.282855927944183 58.0
0 -0.516168475151062 59.0
0 1.13299012184143 60.0
0 0.253184735774994 61.0
0 0.0983628034591675 62.0
275.3043677 252.08039855957 0.0
0 -0.180782198905945 1.0
0 -0.116359829902649 2.0
0 -0.145835161209106 3.0
0 -0.038446307182312 4.0
0 0.296314001083374 5.0
0 0.298313736915588 6.0
0 0.163242280483246 7.0
0 0.0437434315681458 8.0
0 0.200212299823761 9.0
0 -0.685707926750183 10.0
0 -0.09317946434021 11.0
0 0.0175162553787231 12.0
64.600019834 63.9707832336426 13.0
36.268573339 31.3740367889404 14.0
0 0.0619173049926758 15.0
0 -0.14798766374588 16.0
30.887518839 35.660701751709 17.0
0 0.0994288921356201 18.0
0 -0.015160083770752 19.0
0 0.0771510004997253 20.0
0 0.048281729221344 21.0
0 0.220190942287445 22.0
32.934773826 32.3282814025879 23.0
0 0.150383234024048 24.0
5.4103287499 0.932688653469086 25.0
0 -0.109889388084412 26.0
0 -0.113648235797882 27.0
0 0.726263642311096 28.0
0 -0.053313672542572 29.0
0 0.0659857988357544 30.0
0 -0.111341118812561 31.0
0 -0.041199803352356 32.0
0 -0.172133803367615 33.0
0 0.496844828128815 34.0
0 0.15217137336731 35.0
0 0.461067885160446 36.0
0 -0.0911007523536682 37.0
0 0.362724363803864 38.0
0 -0.270178556442261 39.0
0 0.149735927581787 40.0
0 -0.231313943862915 41.0
0 -0.030214786529541 42.0
300.4473475 295.202423095703 43.0
0 -0.0386273860931396 44.0
0 0.1624835729599 45.0
0 0.0438367128372192 46.0
0 -0.204621434211731 47.0
0 0.33235689997673 48.0
1.9204469612 0.13213586807251 49.0
0.083858651558 0.0227769017219543 50.0
1.7433263219 1.9252678155899 51.0
0 -0.0185167789459229 52.0
0 -0.177747309207916 53.0
0 0.0453261137008667 54.0
0 0.168176651000977 55.0
0 -0.0800086259841919 56.0
0 0.489510387182236 57.0
0 0.254080057144165 58.0
0 0.0418819189071655 59.0
0 -0.0465215444564819 60.0
0 0.126717686653137 61.0
0 0.352822005748749 62.0
284.68101815 253.566070556641 0.0
0 0.234144806861877 1.0
0 0.0203210711479187 2.0
0 0.407835632562637 3.0
0 -0.343898773193359 4.0
0 0.496678441762924 5.0
0 0.315283358097076 6.0
0 0.265801548957825 7.0
0 -0.177021205425262 8.0
0 0.220495343208313 9.0
0 -0.124624729156494 10.0
0 0.0201631784439087 11.0
0 -0.125390648841858 12.0
65.063950928 66.1605072021484 13.0
39.603872045 30.3454780578613 14.0
0 -0.0312345027923584 15.0
0 0.125959038734436 16.0
31.704376234 30.5887432098389 17.0
0 0.0495857000350952 18.0
0 -0.0199686288833618 19.0
0 0.19320011138916 20.0
0 0.10483455657959 21.0
0 0.320505797863007 22.0
34.678551321 34.6836585998535 23.0
0 0.0550106167793274 24.0
0 0.564692974090576 25.0
0 0.10975193977356 26.0
0 -0.116708695888519 27.0
0 0.0776187777519226 28.0
0 0.0317977666854858 29.0
0 0.0442518591880798 30.0
0 0.0562388896942139 31.0
0 0.255998075008392 32.0
0 0.0184966325759888 33.0
0 0.0555698871612549 34.0
0 0.506623983383179 35.0
0 0.270991861820221 36.0
0 -0.0305221080780029 37.0
0 0.388616740703583 38.0
0 -0.391652584075928 39.0
0 -0.0206871032714844 40.0
0 -0.250235795974731 41.0
0 0.0542349815368652 42.0
288.75249874 297.035461425781 43.0
0 0.17434287071228 44.0
0 0.00889497995376587 45.0
0 0.258536279201508 46.0
0 0.286759495735168 47.0
0 -0.477658867835999 48.0
0 0.336797416210175 49.0
0.10433613882 -0.0240225195884705 50.0
1.9666064809 1.88357520103455 51.0
0 0.0885286927223206 52.0
0 -0.155673563480377 53.0
0 -0.0438122749328613 54.0
0 0.112764716148376 55.0
0 0.342347085475922 56.0
0 -0.234660983085632 57.0
0 0.10040271282196 58.0
0 0.256224691867828 59.0
0 0.332620054483414 60.0
0 0.00722271203994751 61.0
0 -0.0190876722335815 62.0
287.28037552 250.958251953125 0.0
0 -0.199446201324463 1.0
0 0.00677108764648438 2.0
0 -0.0606650114059448 3.0
0 -0.11867094039917 4.0
0 0.3198282122612 5.0
0 -0.142199397087097 6.0
0 1.06371772289276 7.0
0 -0.623144507408142 8.0
0 -0.858262777328491 9.0
0 0.256776094436646 10.0
0 -0.486741185188293 11.0
0 -0.195140242576599 12.0
67.935250884 65.6638412475586 13.0
15.197371828 28.3174877166748 14.0
0 -0.114290952682495 15.0
0 0.0337536334991455 16.0
31.122765614 36.0083465576172 17.0
0 0.0875564813613892 18.0
0 -0.0159174203872681 19.0
0 0.0016101598739624 20.0
0 0.173240780830383 21.0
0 0.111840903759003 22.0
29.325986481 28.7773761749268 23.0
0 0.0518162250518799 24.0
0 1.10857236385345 25.0
0 0.162606775760651 26.0
0 -0.143014073371887 27.0
0 -0.573930382728577 28.0
0 -0.203946709632874 29.0
0 0.0928250551223755 30.0
0 -0.0376183986663818 31.0
0 -0.0704293847084045 32.0
0 -0.184678137302399 33.0
0 0.433012962341309 34.0
0 -0.0335151553153992 35.0
0 0.0394715070724487 36.0
0 0.0220102667808533 37.0
0 0.265581727027893 38.0
0 -0.0915728807449341 39.0
0 0.0338373184204102 40.0
0 -0.185400903224945 41.0
0 -0.0592765808105469 42.0
280.86975552 296.836242675781 43.0
0 0.149771571159363 44.0
0 0.0551167726516724 45.0
0 0.0762555599212646 46.0
0 -0.25089955329895 47.0
0 0.358519613742828 48.0
2.5469392214 0.613511145114899 49.0
0.091455845681 0.119743764400482 50.0
1.6066944094 1.86648678779602 51.0
0 0.0279370546340942 52.0
0 -0.142310619354248 53.0
0 0.0626103281974792 54.0
0 0.161502361297607 55.0
0 0.155087649822235 56.0
0 -0.0277738571166992 57.0
0 0.272459864616394 58.0
0 0.265376329421997 59.0
0 -0.171243131160736 60.0
0 0.122295022010803 61.0
0 0.470890939235687 62.0
237.54105769 254.442321777344 0.0
0 -0.0132917165756226 1.0
0 0.0362475514411926 2.0
0 0.443333327770233 3.0
0 -0.16916823387146 4.0
0 -0.203951358795166 5.0
0 -0.0412317514419556 6.0
0 0.0538897514343262 7.0
0 0.259337782859802 8.0
0 -0.324593544006348 9.0
0 0.199145317077637 10.0
0 -0.0458933711051941 11.0
0 0.00945746898651123 12.0
56.893944802 56.7267189025879 13.0
43.526331043 43.2189750671387 14.0
0 -0.0252348184585571 15.0
0 -0.0243930816650391 16.0
37.807135045 34.6242942810059 17.0
0 0.166188478469849 18.0
0 0.00398099422454834 19.0
0 0.10998260974884 20.0
0 0.0194311141967773 21.0
0 0.000804305076599121 22.0
39.210994689 38.9955368041992 23.0
0 0.118740916252136 24.0
0 0.681594371795654 25.0
0 0.0216085910797119 26.0
0 0.00964844226837158 27.0
0 0.100631296634674 28.0
0 -0.0866312980651855 29.0
0 0.010452926158905 30.0
0 0.104556441307068 31.0
0 -0.0346807241439819 32.0
0 -0.146305561065674 33.0
0 0.116237878799438 34.0
0 0.129230618476868 35.0
0 0.141265273094177 36.0
0 0.072037398815155 37.0
0 0.188597023487091 38.0
0 -0.418572425842285 39.0
0 -0.057237982749939 40.0
0 -0.180842816829681 41.0
0 -0.027006983757019 42.0
292.56560845 294.891326904297 43.0
0 0.0782390236854553 44.0
0 -0.00108319520950317 45.0
0 0.06386399269104 46.0
0 0.0437878370285034 47.0
0 0.352878034114838 48.0
0 0.412583291530609 49.0
0.08942268068 0.0671871304512024 50.0
2.1812558595 1.88436555862427 51.0
0 -0.204043865203857 52.0
0 -0.0555108189582825 53.0
0 -0.0462196469306946 54.0
0 -0.0325490236282349 55.0
0 -0.0856417417526245 56.0
0 0.0093122124671936 57.0
0 0.292632400989532 58.0
0 0.141023397445679 59.0
0 -0.144432067871094 60.0
0 0.0735596418380737 61.0
0 0.241599142551422 62.0
275.98517783 253.076705932617 0.0
0 -0.124473452568054 1.0
0 -0.241114139556885 2.0
0 0.128058314323425 3.0
0 -0.17460685968399 4.0
0 0.0829334259033203 5.0
0 -0.131855487823486 6.0
0 0.190928637981415 7.0
0 -0.0681502819061279 8.0
0 0.308782696723938 9.0
0 -0.493039011955261 10.0
0 -0.162890911102295 11.0
0 0.0467271208763123 12.0
50.994106635 53.3774337768555 13.0
48.216071814 44.4918518066406 14.0
0 -0.105366468429565 15.0
0 -0.107644021511078 16.0
25.931175343 32.3760795593262 17.0
0 -0.00534725189208984 18.0
0 -0.0129179954528809 19.0
0 0.211373090744019 20.0
0 -0.143030524253845 21.0
0 0.274934768676758 22.0
41.861145546 42.0054664611816 23.0
0 0.0849348306655884 24.0
0 0.990874409675598 25.0
0 -0.0960570573806763 26.0
0 -0.0605983138084412 27.0
0 0.338626652956009 28.0
0 0.162799775600433 29.0
0 0.038843035697937 30.0
0 0.196474432945251 31.0
0 0.13585239648819 32.0
0 0.175170421600342 33.0
0 0.197398483753204 34.0
0 0.160255193710327 35.0
0 0.199475765228271 36.0
0 -0.010032057762146 37.0
0 0.228919684886932 38.0
0 -0.429511547088623 39.0
0 0.411505341529846 40.0
0 -0.174932479858398 41.0
0 0.0429565906524658 42.0
278.74188031 295.986328125 43.0
0 0.0830034017562866 44.0
0 0.239856719970703 45.0
0 0.243363261222839 46.0
0 0.118348598480225 47.0
0 -0.114018201828003 48.0
0 0.393492251634598 49.0
0.098592901533 -0.0303181409835815 50.0
1.9757061643 1.89114141464233 51.0
0 0.0470382571220398 52.0
0 -0.0867570042610168 53.0
0 0.0217643976211548 54.0
0 0.173579216003418 55.0
0 0.11378800868988 56.0
0 0.853243112564087 57.0
0 0.0630931854248047 58.0
0 -0.0201848149299622 59.0
0 -0.0840678811073303 60.0
0 0.0243166089057922 61.0
0 0.0802852511405945 62.0
285.22617903 251.326324462891 0.0
0 -0.146582007408142 1.0
0 -0.0256175398826599 2.0
0 -0.26875102519989 3.0
0 -0.12874698638916 4.0
0 0.197733879089355 5.0
0 0.728716254234314 6.0
0 0.526198387145996 7.0
0 0.0693643093109131 8.0
0 0.534807741641998 9.0
0 0.219879508018494 10.0
0 -0.175454795360565 11.0
0 -0.0558815598487854 12.0
66.956987539 67.1147689819336 13.0
33.427380462 28.4465637207031 14.0
0 -0.0499521493911743 15.0
0 0.013035774230957 16.0
34.300763357 32.0511093139648 17.0
0 0.111452460289001 18.0
0 0.00706863403320312 19.0
0 0.061141312122345 20.0
0 0.100837349891663 21.0
0 0.150470435619354 22.0
30.935732544 26.1071834564209 23.0
0 0.0607349276542664 24.0
0 0.378888309001923 25.0
0 -0.0292767882347107 26.0
0 -0.252550363540649 27.0
0 1.22650480270386 28.0
0 0.0837975740432739 29.0
0 0.101032972335815 30.0
0 -0.208778858184814 31.0
0 -0.0596016645431519 32.0
0 -0.494935393333435 33.0
0 0.132991671562195 34.0
0 0.226400017738342 35.0
0 0.503388822078705 36.0
0 0.0553100109100342 37.0
0 0.311005890369415 38.0
0 -0.217210412025452 39.0
0 0.0425333380699158 40.0
0 -0.191058039665222 41.0
0 0.0345679521560669 42.0
306.61975992 297.421508789062 43.0
0 0.030759871006012 44.0
0 0.0663979053497314 45.0
0 0.180328965187073 46.0
0 0.227985322475433 47.0
0 0.253347992897034 48.0
0 0.26844596862793 49.0
0.094456536044 0.0868308544158936 50.0
2.0822127351 1.8081226348877 51.0
0 -0.187942743301392 52.0
0 -0.183681607246399 53.0
0 0.144878149032593 54.0
0 -0.059528112411499 55.0
0 0.000367224216461182 56.0
0 -0.114355385303497 57.0
0 0.170725643634796 58.0
0 0.145999550819397 59.0
0 -0.292945981025696 60.0
0 0.0104212760925293 61.0
0 0.191334962844849 62.0
249.84386924 253.615676879883 0.0
0 0.0132738351821899 1.0
0 0.0331035852432251 2.0
0 0.328700184822083 3.0
0 -0.307433128356934 4.0
0 0.33150053024292 5.0
0 0.210328280925751 6.0
0 -0.044495165348053 7.0
0 0.0415208339691162 8.0
0 0.00813442468643188 9.0
0 0.198240041732788 10.0
0 -0.0115879774093628 11.0
0 -0.160778999328613 12.0
61.609053655 61.5494995117188 13.0
37.073592711 31.5685043334961 14.0
0 0.0706340074539185 15.0
0 -0.096605122089386 16.0
37.484953156 37.4620590209961 17.0
0 0.125616908073425 18.0
0 -0.0359149575233459 19.0
0 0.0871602296829224 20.0
0 0.014673113822937 21.0
0 0.258802056312561 22.0
30.947753272 30.7459087371826 23.0
0 0.124286890029907 24.0
0 0.95922714471817 25.0
0 0.040286660194397 26.0
0 -0.0400106906890869 27.0
0 0.545657575130463 28.0
0 0.0372451543807983 29.0
0 -0.0760817527770996 30.0
0 0.150435328483582 31.0
0 0.0932655334472656 32.0
0 -0.00526738166809082 33.0
0 0.138095378875732 34.0
0 0.0282052159309387 35.0
0 0.227736234664917 36.0
0 -0.0188182592391968 37.0
0 0.111608564853668 38.0
0 -0.18886935710907 39.0
0 0.481779366731644 40.0
0 -0.0826334953308105 41.0
0 -0.0114191770553589 42.0
294.59599448 295.829376220703 43.0
0 0.101322650909424 44.0
0 0.0702543258666992 45.0
0 0.124545991420746 46.0
0 0.196727633476257 47.0
0 0.397920936346054 48.0
0 0.405280768871307 49.0
0.079013306321 0.0939052104949951 50.0
2.1399526795 1.82340264320374 51.0
0 -0.017876923084259 52.0
0 -0.0630440711975098 53.0
0 0.0760397911071777 54.0
0 -0.130419015884399 55.0
0 -0.144858956336975 56.0
0 0.148133039474487 57.0
0 0.195555031299591 58.0
0 0.0356639623641968 59.0
0 0.0586615204811096 60.0
0 0.139043688774109 61.0
0 0.285617709159851 62.0
269.05819337 249.703399658203 0.0
0 -0.431182980537415 1.0
0 0.28298020362854 2.0
0 0.0588877201080322 3.0
0 0.149705946445465 4.0
0 0.270854890346527 5.0
0 0.507396697998047 6.0
0 0.258313417434692 7.0
0 0.143798291683197 8.0
0 0.317647606134415 9.0
0 0.217358589172363 10.0
0 -0.0682437419891357 11.0
0 -0.00723874568939209 12.0
54.319430824 55.1279220581055 13.0
41.365266406 37.2397193908691 14.0
0 -0.361646890640259 15.0
0 -0.0100042819976807 16.0
32.865457466 36.3521347045898 17.0
0 0.12157940864563 18.0
0 -0.0336846113204956 19.0
0 -0.00688689947128296 20.0
0 0.425687670707703 21.0
0 -0.00584423542022705 22.0
24.165976484 21.5727348327637 23.0
0 0.0927342772483826 24.0
24.947118847 0.528627216815948 25.0
0 -0.0116304159164429 26.0
0 -0.0959222912788391 27.0
0 -0.366858124732971 28.0
0 -0.211507320404053 29.0
0 0.0803536772727966 30.0
0 -0.401616334915161 31.0
0 -0.193316340446472 32.0
0 -0.266015887260437 33.0
0 0.529037058353424 34.0
0 -0.158976018428802 35.0
0 0.949233114719391 36.0
0 -0.116596519947052 37.0
0 0.263500034809113 38.0
0 -0.331584215164185 39.0
0 0.372998952865601 40.0
0 -0.0901626348495483 41.0
0 -0.0342360734939575 42.0
276.33555039 298.096435546875 43.0
0 0.0743697881698608 44.0
0 0.116295456886292 45.0
0 0.160705089569092 46.0
0 0.286726832389832 47.0
0 0.320914745330811 48.0
0 0.225633263587952 49.0
0.073028407159 0.00863248109817505 50.0
2.4295335331 1.838250041008 51.0
0 -0.129678964614868 52.0
0 -0.160806894302368 53.0
0 0.215689480304718 54.0
0 0.116797983646393 55.0
0 0.118464350700378 56.0
0 0.270349383354187 57.0
0 0.38266259431839 58.0
0 0.148400008678436 59.0
0 -0.145788073539734 60.0
0 0.126080870628357 61.0
0 0.326497793197632 62.0
221.91762877 255.018035888672 0.0
0 0.422930479049683 1.0
0 -0.180895388126373 2.0
0 0.00415956974029541 3.0
0 -0.33825957775116 4.0
0 -0.0907996296882629 5.0
0 -0.182649254798889 6.0
0 -0.186678826808929 7.0
0 0.251181721687317 8.0
0 0.0926653742790222 9.0
0 0.370074152946472 10.0
0 0.0452865362167358 11.0
0 -0.0449715852737427 12.0
64.538315837 65.0067825317383 13.0
39.189849814 36.227710723877 14.0
0 0.0861369371414185 15.0
0 0.118742406368256 16.0
40.842555222 35.2745475769043 17.0
0 0.171460151672363 18.0
0 -0.039379358291626 19.0
0 0.32507261633873 20.0
0 0.0324000120162964 21.0
0 0.311368405818939 22.0
31.832918477 32.2656478881836 23.0
0 0.157237648963928 24.0
0 0.978624701499939 25.0
0 -0.247951626777649 26.0
0 -0.0742921829223633 27.0
0 0.124699234962463 28.0
0 -0.0509177446365356 29.0
0 0.0246443748474121 30.0
0 0.112621307373047 31.0
0 0.203000664710999 32.0
0 0.285928666591644 33.0
0 -0.0730587840080261 34.0
0 -0.149737060070038 35.0
0 -0.286803483963013 36.0
0 0.062922477722168 37.0
0 0.19587779045105 38.0
0 -0.165468573570251 39.0
0 0.866633653640747 40.0
0 -0.123269498348236 41.0
0 0.110192537307739 42.0
284.78024272 296.576782226562 43.0
0 0.0612621903419495 44.0
0 0.199576079845428 45.0
0 0.202702879905701 46.0
0 -0.355518341064453 47.0
0 0.36533111333847 48.0
4.3289940177 0.556466817855835 49.0
0.099792830817 0.0665733218193054 50.0
2.0006023392 1.94181096553802 51.0
0 -0.0747799873352051 52.0
0 0.351643711328506 53.0
0 0.0858732461929321 54.0
0 -0.115593194961548 55.0
0 -0.185281753540039 56.0
0 0.144206166267395 57.0
0 0.466506540775299 58.0
0 0.192988038063049 59.0
0 -0.218176484107971 60.0
0 0.109930157661438 61.0
0 0.302412271499634 62.0
249.48345762 252.853408813477 0.0
0 0.464453816413879 1.0
0 -0.049602210521698 2.0
0 -0.0626670122146606 3.0
0 -0.363652944564819 4.0
0 0.26188051700592 5.0
0 0.255163192749023 6.0
0 -0.0152131319046021 7.0
0 0.22722327709198 8.0
0 -0.146514058113098 9.0
0 0.0203900933265686 10.0
0 -0.0243765115737915 11.0
0 0.0590439438819885 12.0
58.77821622 58.5981407165527 13.0
48.153231664 41.0136451721191 14.0
0 -0.138225674629211 15.0
0 0.407753646373749 16.0
26.235728915 32.8334312438965 17.0
0 0.148119151592255 18.0
0 -0.0616872906684875 19.0
0 0.0813445448875427 20.0
0 0.127164900302887 21.0
0 0.249553024768829 22.0
32.439661467 31.72509765625 23.0
0 0.123136401176453 24.0
0 0.960572481155396 25.0
0 0.0472452640533447 26.0
0 -0.0650443434715271 27.0
0 0.406515538692474 28.0
0 -0.00688481330871582 29.0
0 0.0913859605789185 30.0
0 -0.142017126083374 31.0
0 -0.141599595546722 32.0
0 -0.270513415336609 33.0
0 0.0508603453636169 34.0
0 0.836534440517426 35.0
0 0.457969546318054 36.0
0 0.135712623596191 37.0
0 -0.0131083130836487 38.0
0 -0.224816203117371 39.0
0 0.391127824783325 40.0
0 -0.132601380348206 41.0
0 0.0922123193740845 42.0
283.70113661 294.569152832031 43.0
0 0.067145824432373 44.0
0 0.186675548553467 45.0
0 -0.0406008958816528 46.0
0 0.22945499420166 47.0
0 0.216354668140411 48.0
0 0.495326459407806 49.0
0.095829782669 0.0222913026809692 50.0
2.179861404 1.89548063278198 51.0
0 0.183519661426544 52.0
0 0.281582772731781 53.0
0 0.0409661531448364 54.0
0 0.218988418579102 55.0
0 0.158279359340668 56.0
0 -0.0395801067352295 57.0
0 0.192640721797943 58.0
0 0.206456184387207 59.0
0 0.02082359790802 60.0
0 0.0830621123313904 61.0
0 0.192499399185181 62.0
286.59959479 256.142639160156 0.0
0 -0.397030830383301 1.0
0 0.0281669497489929 2.0
0 0.271401703357697 3.0
0 0.0694765448570251 4.0
0 0.433867156505585 5.0
0 0.535106301307678 6.0
0 0.393189251422882 7.0
0 0.0956407785415649 8.0
0 0.179786026477814 9.0
0 0.486888259649277 10.0
0 -0.104477047920227 11.0
0 -0.0238436460494995 12.0
53.49258881 54.7209892272949 13.0
37.796194569 39.534782409668 14.0
0 -0.0210322141647339 15.0
0 -0.173711597919464 16.0
31.263005395 32.6214714050293 17.0
0 0.105761408805847 18.0
0 0.0316812992095947 19.0
0 0.166658639907837 20.0
0 0.0898643732070923 21.0
0 -0.124517738819122 22.0
38.506253573 37.8304214477539 23.0
0 0.120925664901733 24.0
0 0.62894082069397 25.0
0 0.0144169330596924 26.0
0 -0.0742653012275696 27.0
0 0.499519765377045 28.0
0 0.10178005695343 29.0
0 -0.0807333588600159 30.0
0 0.237690925598145 31.0
0 0.135017812252045 32.0
0 -0.0477378368377686 33.0
0 0.343964070081711 34.0
0 0.0313742756843567 35.0
0 0.298375993967056 36.0
0 -0.097726583480835 37.0
0 0.279233694076538 38.0
0 -0.291340589523315 39.0
0 0.348236322402954 40.0
0 -0.226505875587463 41.0
0 0.0575459003448486 42.0
304.25826801 296.479278564453 43.0
0 0.219581842422485 44.0
0 0.187476754188538 45.0
0 0.164205193519592 46.0
0 -0.378843307495117 47.0
0 0.353009968996048 48.0
0.33061588588 0.343068778514862 49.0
0.081300359236 0.0830196738243103 50.0
1.5505018716 1.87855267524719 51.0
0 -0.264168262481689 52.0
0 -0.230672597885132 53.0
0 0.11786413192749 54.0
0 0.189516603946686 55.0
0 0.241073846817017 56.0
0 0.29138970375061 57.0
0 0.132816255092621 58.0
0 0.145658135414124 59.0
0 0.16605931520462 60.0
0 0.0724246501922607 61.0
0 0.203437149524689 62.0
231.44689297 218.404357910156 0.0
0 -0.73657763004303 1.0
0 -0.395931243896484 2.0
0 -0.333268046379089 3.0
0 -0.115468621253967 4.0
0 -0.0696203708648682 5.0
0 -0.208342552185059 6.0
0 -0.14118105173111 7.0
0 -0.0703502893447876 8.0
0 0.110295355319977 9.0
0 -0.431691288948059 10.0
0 -0.0295137763023376 11.0
0 -0.0169374942779541 12.0
68.702625864 69.6384506225586 13.0
39.240581123 30.1033897399902 14.0
0 -1.01765704154968 15.0
0 -0.198728501796722 16.0
23.826472374 32.8012580871582 17.0
0 0.0226716995239258 18.0
0 0.0557483434677124 19.0
0 0.0395779609680176 20.0
0 0.14482319355011 21.0
0 0.0914962291717529 22.0
33.8809781 30.3115310668945 23.0
0 0.00760757923126221 24.0
0 0.952045202255249 25.0
0 -0.157867312431335 26.0
0 0.16480827331543 27.0
0 0.0930627584457397 28.0
0 -0.0615782141685486 29.0
0 0.104625225067139 30.0
0 0.315740823745728 31.0
0 0.0928406119346619 32.0
0 -0.0601835250854492 33.0
0 0.0627466440200806 34.0
0 0.0699360370635986 35.0
0 0.564589262008667 36.0
0 0.189752757549286 37.0
0 -0.0631704926490784 38.0
0 -0.173087477684021 39.0
0 0.513589143753052 40.0
0 -0.0810277462005615 41.0
0 0.101787328720093 42.0
296.79678915 296.820068359375 43.0
0 0.437305212020874 44.0
0 -0.423352360725403 45.0
0 0.485132277011871 46.0
0 -0.544432044029236 47.0
0 0.0348888039588928 48.0
0 0.450585961341858 49.0
0.073471162123 0.018231987953186 50.0
2.3957648299 1.85144925117493 51.0
0 0.145427405834198 52.0
0 -0.117408633232117 53.0
0 -0.00441968441009521 54.0
0 0.236924648284912 55.0
0 0.305743753910065 56.0
0 0.00927096605300903 57.0
0 0.0474487543106079 58.0
0 0.07602858543396 59.0
0 -0.17081743478775 60.0
0 0.0907009840011597 61.0
0 0.138296604156494 62.0
281.15404842 252.263595581055 0.0
0 0.0554103851318359 1.0
0 0.124492466449738 2.0
0 0.389835953712463 3.0
0 -0.330042243003845 4.0
0 -0.101790189743042 5.0
0 0.184774994850159 6.0
0 0.370790153741837 7.0
0 0.559548795223236 8.0
0 0.567577123641968 9.0
0 -0.339489579200745 10.0
0 -0.0314269661903381 11.0
0 -0.107246398925781 12.0
57.680016296 57.1364974975586 13.0
51.890261069 39.2730522155762 14.0
0 -0.0969514846801758 15.0
0 0.17029345035553 16.0
39.748491369 35.5435447692871 17.0
0 0.138335287570953 18.0
0 -0.055121123790741 19.0
0 0.0366424918174744 20.0
0 -0.051310658454895 21.0
0 0.373774200677872 22.0
33.378792628 34.2060852050781 23.0
0 0.0548550486564636 24.0
0 0.810770153999329 25.0
0 -0.0447115302085876 26.0
0 -0.0429678559303284 27.0
0 0.0198956727981567 28.0
0 -0.141296029090881 29.0
0 0.0448064208030701 30.0
0 -0.170653343200684 31.0
0 0.00501823425292969 32.0
0 -0.02069491147995 33.0
0 0.330021947622299 34.0
0 0.0279778838157654 35.0
0 0.140424609184265 36.0
0 -0.0951322913169861 37.0
0 0.156982123851776 38.0
0 -0.365822196006775 39.0
0 0.575716614723206 40.0
0 -0.273372769355774 41.0
0 -0.0436578989028931 42.0
285.4502082 297.615783691406 43.0
0 0.0189157724380493 44.0
0 0.056873083114624 45.0
0 0.0607295036315918 46.0
0 -0.060437023639679 47.0
0 0.38355627655983 48.0
0 0.417000234127045 49.0
0.066171708892 0.0900433659553528 50.0
2.1785227876 1.83252644538879 51.0
0 -0.147285521030426 52.0
0 0.122874557971954 53.0
0 0.0648263692855835 54.0
0 -0.091098427772522 55.0
0 -0.0284217596054077 56.0
0 0.033515453338623 57.0
0 0.455799043178558 58.0
0 0.506807386875153 59.0
0 0.0361791253089905 60.0
0 0.0826828479766846 61.0
0 0.215412139892578 62.0
287.24195977 258.242401123047 0.0
0 0.228390455245972 1.0
0 -0.215373039245605 2.0
0 0.154099822044373 3.0
0 0.0706382989883423 4.0
0 0.11697244644165 5.0
0 -0.217456817626953 6.0
0 0.118010401725769 7.0
0 -0.854728221893311 8.0
0 -0.740431547164917 9.0
0 0.307586163282394 10.0
0 -0.116947591304779 11.0
0 -0.283448696136475 12.0
53.982494033 53.3643951416016 13.0
42.644261488 38.5950469970703 14.0
0 0.0393607616424561 15.0
0 0.826467514038086 16.0
29.584683627 34.3715667724609 17.0
0 0.136404633522034 18.0
0 -0.00204426050186157 19.0
0 0.17684805393219 20.0
0 0.0581197738647461 21.0
0 0.161034107208252 22.0
33.223247148 32.9964218139648 23.0
0 0.150793015956879 24.0
0 0.894740164279938 25.0
0 -0.236351370811462 26.0
0 -0.138484477996826 27.0
0 0.343907177448273 28.0
0 0.103090345859528 29.0
0 0.0476281642913818 30.0
0 0.340765118598938 31.0
0 0.0463184118270874 32.0
0 -0.479101419448853 33.0
0 0.169080853462219 34.0
0 0.2037313580513 35.0
0 -0.483032941818237 36.0
0 -0.0284029245376587 37.0
0 0.188131749629974 38.0
0 -0.180684447288513 39.0
0 0.305357098579407 40.0
0 -0.104169189929962 41.0
0 -0.165215253829956 42.0
297.85780169 292.839965820312 43.0
0 0.0665169358253479 44.0
0 -0.158339321613312 45.0
0 0.579972922801971 46.0
0 -1.02902984619141 47.0
0 0.342044234275818 48.0
0 0.69364333152771 49.0
0.062641527489 -0.0178961753845215 50.0
1.6283256948 1.86912894248962 51.0
0 0.540031433105469 52.0
0 0.310646712779999 53.0
0 0.0530120730400085 54.0
0 0.077597975730896 55.0
0 -0.137109577655792 56.0
0 0.0987087488174438 57.0
0 0.297574788331985 58.0
0 0.375279605388641 59.0
0 -0.147079467773438 60.0
0 0.125397205352783 61.0
0 0.234616756439209 62.0
229.42819507 256.664367675781 0.0
0 -0.247062206268311 1.0
0 -0.12598842382431 2.0
0 -0.206064105033875 3.0
0 -0.339707493782043 4.0
0 -0.180881261825562 5.0
0 0.219083487987518 6.0
0 -0.195044934749603 7.0
0 -0.621533393859863 8.0
0 0.839342951774597 9.0
0 0.542769074440002 10.0
0 -0.250115156173706 11.0
0 -0.0589346885681152 12.0
69.241795975 70.5624237060547 13.0
24.117330889 29.419246673584 14.0
0 -0.098851203918457 15.0
0 -0.194479286670685 16.0
31.095099259 34.2827262878418 17.0
0 0.114576458930969 18.0
0 -0.0255992412567139 19.0
0 0.152807831764221 20.0
0 0.192509770393372 21.0
0 -0.00150996446609497 22.0
33.522673275 29.7801723480225 23.0
0 -0.0194352865219116 24.0
0 0.266251564025879 25.0
0 0.0260546803474426 26.0
0 0.0279763340950012 27.0
0 0.0258392095565796 28.0
0 -0.11506974697113 29.0
0 0.178920745849609 30.0
0 0.0473122596740723 31.0
0 -0.00866293907165527 32.0
0 0.0575711131095886 33.0
0 0.106280744075775 34.0
0 0.34101727604866 35.0
0 0.378682553768158 36.0
0 0.134861528873444 37.0
0 -0.193986415863037 38.0
0 -0.0246363878250122 39.0
0 0.334693670272827 40.0
0 -0.0968893766403198 41.0
0 0.0972424745559692 42.0
294.8248876 292.906829833984 43.0
0 0.134674429893494 44.0
0 -0.0415512919425964 45.0
0 0.167437851428986 46.0
0 -0.25759744644165 47.0
0 0.342677682638168 48.0
0.076642592144 0.354388654232025 49.0
0.095903779645 -0.0158889889717102 50.0
2.0353398607 1.89924252033234 51.0
0 -0.159008085727692 52.0
0 -0.0687168836593628 53.0
0 -0.136124014854431 54.0
0 0.0320650339126587 55.0
0 0.343539774417877 56.0
0 0.31668347120285 57.0
0 0.0307995080947876 58.0
0 -0.0357074737548828 59.0
0 -0.19276624917984 60.0
0 0.0428709983825684 61.0
0 0.0896686315536499 62.0
235.4123612 254.199905395508 0.0
0 0.372018575668335 1.0
0 0.133139133453369 2.0
0 -0.0683688521385193 3.0
0 -0.1772700548172 4.0
0 -0.318722486495972 5.0
0 0.188208520412445 6.0
0 0.145975172519684 7.0
0 0.106421589851379 8.0
0 0.325133204460144 9.0
0 0.213089823722839 10.0
0 -0.604771852493286 11.0
0 -0.0638287663459778 12.0
77.712714085 68.7300109863281 13.0
16.915932079 40.2848968505859 14.0
0 -0.0159693956375122 15.0
0 0.396415710449219 16.0
32.179464189 34.5666542053223 17.0
0 0.12005090713501 18.0
0 -0.0202369689941406 19.0
0 0.0573671460151672 20.0
0 -0.0470985770225525 21.0
0 0.655916452407837 22.0
40.398705858 39.5070648193359 23.0
0 0.0273649096488953 24.0
0 0.635989010334015 25.0
0 0.0659801959991455 26.0
0 -0.190118849277496 27.0
0 0.858431398868561 28.0
0 0.168524324893951 29.0
0 0.055794894695282 30.0
0 -0.295507788658142 31.0
0 -0.122471213340759 32.0
0 -0.0660223364830017 33.0
0 0.4576775431633 34.0
0 0.0868196487426758 35.0
0 0.247597396373749 36.0
0 -0.0356670618057251 37.0
0 0.0627451539039612 38.0
0 -0.0367211103439331 39.0
0 0.150148630142212 40.0
0 -0.176847398281097 41.0
0 -0.0196514129638672 42.0
311.82351695 297.449676513672 43.0
0 0.28108286857605 44.0
0 -0.787263154983521 45.0
0 0.244726181030273 46.0
0 -0.291840195655823 47.0
0 0.0994653701782227 48.0
0 0.239195168018341 49.0
0.090250072652 0.0927649140357971 50.0
1.9950449487 1.84175038337708 51.0
0 0.350768744945526 52.0
0 -0.279528021812439 53.0
0 0.0552442073822021 54.0
0 0.348558068275452 55.0
0 0.353620052337646 56.0
0 0.0959105491638184 57.0
0 0.477055758237839 58.0
0 0.178045988082886 59.0
0 -0.0884720087051392 60.0
0 0.00593900680541992 61.0
0 0.207630395889282 62.0
217.5594009 253.951843261719 0.0
0 -0.233730792999268 1.0
0 -0.0995500683784485 2.0
0 -0.269431352615356 3.0
0 -0.361980676651001 4.0
0 -0.344033122062683 5.0
0 -0.0215468406677246 6.0
0 -0.284829378128052 7.0
0 0.0943315029144287 8.0
0 0.0403186678886414 9.0
0 -0.180896937847137 10.0
0 0.0694078207015991 11.0
0 -0.283109903335571 12.0
53.558440821 51.510181427002 13.0
56.777720657 44.071403503418 14.0
0 -0.160799503326416 15.0
0 -0.241138458251953 16.0
31.342182789 32.754467010498 17.0
0 0.0841001868247986 18.0
0 0.0566076040267944 19.0
0 0.11732941865921 20.0
0 0.196441173553467 21.0
0 0.229401648044586 22.0
32.813352055 30.0562400817871 23.0
0 0.105561435222626 24.0
0 1.1055862903595 25.0
0 -0.0034869909286499 26.0
0 -0.115788698196411 27.0
0 0.914965808391571 28.0
0 -0.0607930421829224 29.0
0 0.0376375913619995 30.0
0 -0.00557923316955566 31.0
0 -0.0255670547485352 32.0
0 -0.178030252456665 33.0
0 0.157908976078033 34.0
0 -0.173169314861298 35.0
0 -0.0573529601097107 36.0
0 0.0195242166519165 37.0
0 -0.0655466318130493 38.0
0 0.0158795118331909 39.0
0 0.448934316635132 40.0
0 -0.0643941760063171 41.0
0 0.0726082324981689 42.0
296.42640978 295.763427734375 43.0
0 0.167075634002686 44.0
0 0.126899361610413 45.0
0 0.228186130523682 46.0
0 -0.117010295391083 47.0
0 0.394017517566681 48.0
0 0.466397881507874 49.0
0.088495673637 0.058466911315918 50.0
1.905946706 1.83004677295685 51.0
0 0.16826456785202 52.0
0 0.0818635225296021 53.0
0 0.190499901771545 54.0
0 0.0922855734825134 55.0
0 -0.0270044207572937 56.0
0 0.0395891070365906 57.0
0 0.595279455184937 58.0
0 -0.119024634361267 59.0
0 0.0935742855072021 60.0
0 0.0543280839920044 61.0
0 -0.0255994200706482 62.0
223.30155042 251.94889831543 0.0
0 0.149019360542297 1.0
0 -0.0475688576698303 2.0
0 0.26655387878418 3.0
0 -0.311616659164429 4.0
0 -0.330450654029846 5.0
0 0.0130041241645813 6.0
0 -0.0969570279121399 7.0
0 0.510138273239136 8.0
0 -0.497145652770996 9.0
0 0.0889307856559753 10.0
0 -0.12350594997406 11.0
0 -0.0225063562393188 12.0
51.667320304 52.0667572021484 13.0
40.205274185 40.2699890136719 14.0
0 -0.00691497325897217 15.0
0 -0.03449547290802 16.0
34.868075732 33.1825218200684 17.0
0 0.0962668657302856 18.0
0 -0.0175529718399048 19.0
0 0.0697579383850098 20.0
0 0.114523470401764 21.0
0 0.258034288883209 22.0
35.37909763 33.4256248474121 23.0
0 0.0478371977806091 24.0
0 0.363640248775482 25.0
0 -0.0565135478973389 26.0
0 -0.23650074005127 27.0
0 0.789249897003174 28.0
0 0.0488972663879395 29.0
0 0.0557342171669006 30.0
0 -0.254433155059814 31.0
0 -0.0193777084350586 32.0
0 0.0676535367965698 33.0
0 0.324731916189194 34.0
0 0.104785561561584 35.0
0 0.245114326477051 36.0
0 0.00817036628723145 37.0
0 0.00743067264556885 38.0
0 -0.307237029075623 39.0
0 0.294880270957947 40.0
0 -0.120164275169373 41.0
0 0.0544309616088867 42.0
300.58826483 299.471130371094 43.0
0 0.177737951278687 44.0
0 0.050632119178772 45.0
0 0.194291293621063 46.0
0 -0.0617705583572388 47.0
0 0.0845927000045776 48.0
0 0.333349406719208 49.0
0.070607258342 0.0547135472297668 50.0
2.1217518689 1.85342442989349 51.0
0 0.274566531181335 52.0
0 -0.192144811153412 53.0
0 0.214940130710602 54.0
0 0.252830862998962 55.0
0 -0.097372829914093 56.0
0 -0.0224860310554504 57.0
0 0.163630664348602 58.0
0 0.373982310295105 59.0
0 0.01857590675354 60.0
0 0.118248105049133 61.0
0 0.199689328670502 62.0
263.04232814 251.609634399414 0.0
0 0.104845881462097 1.0
0 0.0285071134567261 2.0
0 0.507942497730255 3.0
0 -0.128803968429565 4.0
0 0.245412468910217 5.0
0 0.0878853797912598 6.0
0 0.183253288269043 7.0
0 -0.0136803388595581 8.0
0 0.100597977638245 9.0
0 -0.223321795463562 10.0
0 0.139721274375916 11.0
0 -0.0262575149536133 12.0
71.877396091 71.8212509155273 13.0
34.873003447 32.1374206542969 14.0
0 -0.0710614919662476 15.0
0 0.410066068172455 16.0
28.481891263 31.4062538146973 17.0
0 0.168664693832397 18.0
0 -0.0425944924354553 19.0
0 0.160412073135376 20.0
0 0.0717968940734863 21.0
0 0.162126660346985 22.0
30.807792881 31.2954635620117 23.0
0 0.0637227296829224 24.0
0 0.744456171989441 25.0
0 0.0579944849014282 26.0
0 -0.212673902511597 27.0
0 -0.195868849754333 28.0
0 -0.0862599611282349 29.0
0 0.0757514238357544 30.0
0 -0.408644556999207 31.0
0 -0.125853002071381 32.0
0 -0.491605281829834 33.0
0 0.128360569477081 34.0
0 -0.0923665165901184 35.0
0 -0.189358115196228 36.0
0 0.163296222686768 37.0
0 0.0462486743927002 38.0
0 -0.174618244171143 39.0
0 0.244622409343719 40.0
0 -0.0706930756568909 41.0
0 0.176870346069336 42.0
304.19256538 296.065795898438 43.0
0 0.0671324729919434 44.0
0 0.0894017219543457 45.0
0 0.242418825626373 46.0
0 0.0121704339981079 47.0
0 0.335093855857849 48.0
3.083003142 0.451334416866302 49.0
0.086866421568 0.0174946784973145 50.0
1.5623424096 1.89358592033386 51.0
0 -0.159419059753418 52.0
0 0.115520477294922 53.0
0 0.215511500835419 54.0
0 0.101543307304382 55.0
0 -0.187681674957275 56.0
0 0.510306119918823 57.0
0 0.227908730506897 58.0
0 0.314225435256958 59.0
0 -0.189585268497467 60.0
0 0.117526233196259 61.0
0 0.449211955070496 62.0
239.55765498 252.693893432617 0.0
0 -0.0584559440612793 1.0
0 -0.0507239699363708 2.0
0 0.0898860096931458 3.0
0 -0.276215434074402 4.0
0 -0.0610673427581787 5.0
0 0.172268867492676 6.0
0 -0.00385308265686035 7.0
0 0.173078656196594 8.0
0 0.303348422050476 9.0
0 -0.451496720314026 10.0
0 0.08670574426651 11.0
0 -0.0583232641220093 12.0
52.849057059 52.3021965026855 13.0
49.746252464 43.0316390991211 14.0
0 -0.257781386375427 15.0
0 0.225998938083649 16.0
37.48258718 35.2166481018066 17.0
0 0.116991877555847 18.0
0 -0.0224840044975281 19.0
0 0.0789996981620789 20.0
0 0.0741199851036072 21.0
0 0.157311677932739 22.0
29.142831035 27.4293918609619 23.0
0 0.028545618057251 24.0
0 0.377757608890533 25.0
0 -0.0421843528747559 26.0
0 -0.0843819975852966 27.0
0 0.268493056297302 28.0
0 -0.00295090675354004 29.0
0 -0.13716048002243 30.0
0 0.277812004089355 31.0
0 0.0878181457519531 32.0
0 -0.0193218588829041 33.0
0 -0.543012976646423 34.0
0 0.384456634521484 35.0
0 -0.101804852485657 36.0
0 -0.0927044153213501 37.0
0 0.220650494098663 38.0
0 -0.383699059486389 39.0
0 0.549509048461914 40.0
0 -0.245379567146301 41.0
0 0.0541927814483643 42.0
301.09598584 298.245422363281 43.0
0 0.0568774342536926 44.0
0 0.116633415222168 45.0
0 0.188850820064545 46.0
0 -0.261324286460876 47.0
0 0.378972917795181 48.0
1.283928521 0.528516829013824 49.0
0.072443526567 0.0389705300331116 50.0
1.8559469499 1.94778895378113 51.0
0 -0.0632534027099609 52.0
0 -0.149933576583862 53.0
0 -0.00421321392059326 54.0
0 0.306097388267517 55.0
0 0.198941111564636 56.0
0 0.456447243690491 57.0
0 -0.0160691142082214 58.0
0 0.00478589534759521 59.0
0 -0.0503559708595276 60.0
0 -0.00648969411849976 61.0
0 0.0673936605453491 62.0
256.966466 252.445373535156 0.0
0 -0.238730430603027 1.0
0 -0.154554963111877 2.0
0 0.480649441480637 3.0
0 0.0412437915802002 4.0
0 -0.0993976593017578 5.0
0 -0.237908959388733 6.0
0 0.128745794296265 7.0
0 -0.00152003765106201 8.0
0 0.105133175849915 9.0
0 -0.0295547842979431 10.0
0 -0.161471128463745 11.0
0 0.0456228852272034 12.0
62.205015268 62.3494720458984 13.0
40.365267653 36.7817001342773 14.0
0 -0.0152724981307983 15.0
0 -0.033374011516571 16.0
31.143264192 33.271800994873 17.0
0 0.0980101823806763 18.0
0 0.00331419706344604 19.0
0 0.0662732124328613 20.0
0 0.158888339996338 21.0
0 0.206340670585632 22.0
31.626512779 31.548210144043 23.0
0 0.0627651810646057 24.0
0 0.6863654255867 25.0
0 -0.049553394317627 26.0
0 -0.188105463981628 27.0
0 -0.194522798061371 28.0
0 -0.0753812193870544 29.0
0 0.0313315987586975 30.0
0 0.0476336479187012 31.0
0 0.0262810587882996 32.0
0 -0.017237663269043 33.0
0 -0.125132441520691 34.0
0 -0.171291410923004 35.0
0 0.45980116724968 36.0
0 0.045529305934906 37.0
0 0.0274160504341125 38.0
0 -0.141769409179688 39.0
0 -0.0904103517532349 40.0
0 -0.101771235466003 41.0
0 0.16867458820343 42.0
282.57474048 295.775451660156 43.0
0 0.0684787034988403 44.0
0 0.00107896327972412 45.0
0 0.12101411819458 46.0
0 -0.0380377173423767 47.0
0 0.311745375394821 48.0
0.11916375769 0.436039447784424 49.0
0.073328214887 0.0722716450691223 50.0
1.9828109933 1.88408970832825 51.0
0 0.19852751493454 52.0
0 -0.0672176480293274 53.0
0 0.178881585597992 54.0
0 0.0639503002166748 55.0
0 -0.107587456703186 56.0
0 0.129846453666687 57.0
0 0.344220370054245 58.0
0 0.299367725849152 59.0
0 -0.1515833735466 60.0
0 0.129963040351868 61.0
0 0.244222819805145 62.0
273.32145127 256.1318359375 0.0
0 -0.41539454460144 1.0
0 0.0273405313491821 2.0
0 0.299615085124969 3.0
0 -0.123910546302795 4.0
0 -0.0738006830215454 5.0
0 -0.132813334465027 6.0
0 0.302635818719864 7.0
0 0.355763375759125 8.0
0 0.38703840970993 9.0
0 0.280916750431061 10.0
0 0.118589401245117 11.0
0 -0.242970943450928 12.0
74.242447016 76.7383728027344 13.0
34.473087534 25.876091003418 14.0
0 -0.0948218107223511 15.0
0 -0.197813987731934 16.0
28.091003229 30.7931041717529 17.0
0 0.0862160921096802 18.0
0 -0.0138008594512939 19.0
0 0.078767716884613 20.0
0 0.1593297123909 21.0
0 0.23306542634964 22.0
30.687873131 30.920389175415 23.0
0 0.11931049823761 24.0
0 1.06204152107239 25.0
0 0.101942598819733 26.0
0 -0.117004871368408 27.0
0 0.349095016717911 28.0
0 0.00515937805175781 29.0
0 0.0404868125915527 30.0
0 -0.000925779342651367 31.0
0 -0.0349387526512146 32.0
0 -0.410767316818237 33.0
0 -0.0314035415649414 34.0
0 0.278189122676849 35.0
0 0.418576151132584 36.0
0 0.223036348819733 37.0
0 0.12781435251236 38.0
0 -0.344376921653748 39.0
0 0.606983304023743 40.0
0 -0.084505558013916 41.0
0 -0.00716674327850342 42.0
304.30228807 296.394073486328 43.0
0 0.0106080770492554 44.0
0 -0.0583791732788086 45.0
0 -0.0168410539627075 46.0
0 0.270281612873077 47.0
0 0.347407609224319 48.0
1.1598405993 0.625333547592163 49.0
0.10030573585 0.0686892867088318 50.0
2.015265386 1.885418176651 51.0
0 -0.207074761390686 52.0
0 0.0651254653930664 53.0
0 0.230773389339447 54.0
0 0.248552203178406 55.0
0 -0.00621610879898071 56.0
0 0.107992053031921 57.0
0 0.156149685382843 58.0
0 -0.11868941783905 59.0
0 -0.0346947312355042 60.0
0 0.039516806602478 61.0
0 0.193828642368317 62.0
229.21611805 251.428771972656 0.0
0 -0.117864727973938 1.0
0 -0.0247731804847717 2.0
0 0.234504818916321 3.0
0 -0.304021716117859 4.0
0 0.013049840927124 5.0
0 -0.210889339447021 6.0
0 -0.234232664108276 7.0
0 0.152393460273743 8.0
0 -0.339708089828491 9.0
0 -0.165919959545135 10.0
0 0.132105469703674 11.0
0 -0.0437957644462585 12.0
70.358368493 70.2046127319336 13.0
33.193124011 29.8976669311523 14.0
0 -0.0433000326156616 15.0
0 0.0828559994697571 16.0
36.435208471 35.2283592224121 17.0
0 0.156249523162842 18.0
0 -0.0703921318054199 19.0
0 0.0957056879997253 20.0
0 0.118028402328491 21.0
0 0.139973402023315 22.0
28.975753635 27.6010208129883 23.0
0 0.093841016292572 24.0
0 0.606615245342255 25.0
0 -0.0655335783958435 26.0
0 -0.0332610607147217 27.0
0 0.313600778579712 28.0
0 -0.0258030891418457 29.0
0 -0.0129120945930481 30.0
0 -0.00417661666870117 31.0
0 0.023672342300415 32.0
0 -0.28089714050293 33.0
0 0.230501115322113 34.0
0 0.0214924812316895 35.0
0 0.0754510760307312 36.0
0 0.18149721622467 37.0
0 0.0328434705734253 38.0
0 -0.351107835769653 39.0
0 0.881735861301422 40.0
0 -0.155534267425537 41.0
0 0.034353494644165 42.0
308.01557923 296.165466308594 43.0
0 0.103783130645752 44.0
0 0.132099449634552 45.0
0 0.149179339408875 46.0
0 -0.0806179642677307 47.0
0 0.118474125862122 48.0
0 0.457856178283691 49.0
0.085043832368 0.028775691986084 50.0
1.6958797211 1.83653783798218 51.0
0 -0.217581391334534 52.0
0 0.0754671096801758 53.0
0 -0.0825076103210449 54.0
0 -0.098468005657196 55.0
0 0.0792908072471619 56.0
0 0.245659530162811 57.0
0 0.303312391042709 58.0
0 0.36575585603714 59.0
0 -0.237992882728577 60.0
0 0.146408200263977 61.0
0 0.382557332515717 62.0
233.80079462 253.140518188477 0.0
0 -0.04120934009552 1.0
0 -0.132914841175079 2.0
0 0.26557457447052 3.0
0 -0.325704574584961 4.0
0 -0.043974757194519 5.0
0 0.0958540439605713 6.0
0 0.062400758266449 7.0
0 0.289593935012817 8.0
0 0.128996014595032 9.0
0 0.0544306039810181 10.0
0 -0.0603940486907959 11.0
0 -0.00758373737335205 12.0
70.068538987 72.4545059204102 13.0
26.101550786 24.6771392822266 14.0
0 0.000632166862487793 15.0
0 0.0416325330734253 16.0
38.28403119 34.0365791320801 17.0
0 0.0854032039642334 18.0
0 -0.0651108026504517 19.0
0 0.0872728824615479 20.0
0 0.0871190428733826 21.0
0 0.214210093021393 22.0
29.348629225 29.3971347808838 23.0
0 0.0954787731170654 24.0
0 0.779758751392365 25.0
0 0.0286484956741333 26.0
0 -0.20024162530899 27.0
0 0.35229155421257 28.0
0 0.104392528533936 29.0
0 0.00914239883422852 30.0
0 0.0168465375900269 31.0
0 -0.0304408073425293 32.0
0 -0.0668307542800903 33.0
0 0.45087993144989 34.0
0 -0.123018741607666 35.0
0 -0.186863005161285 36.0
0 0.0015103816986084 37.0
0 0.23516571521759 38.0
0 -0.298651337623596 39.0
0 0.166059255599976 40.0
0 -0.17975389957428 41.0
0 -0.109059572219849 42.0
294.51245268 296.529541015625 43.0
0 -0.0122404098510742 44.0
0 0.0928505659103394 45.0
0 0.0382162928581238 46.0
0 0.189193248748779 47.0
0 0.106127560138702 48.0
0 0.478272199630737 49.0
0.07149536779 -0.00914794206619263 50.0
1.7674595022 1.87231969833374 51.0
0 0.00559806823730469 52.0
0 -0.0208078026771545 53.0
0 -0.0375937223434448 54.0
0 0.0463963747024536 55.0
0 0.0431509017944336 56.0
0 0.281039476394653 57.0
0 0.441352963447571 58.0
0 0.273362994194031 59.0
0 -0.227697849273682 60.0
0 0.112026214599609 61.0
0 0.297569990158081 62.0
250.78670267 256.5009765625 0.0
0 0.543524622917175 1.0
0 0.0712428689002991 2.0
0 0.0943974852561951 3.0
0 -0.244844436645508 4.0
0 0.0537525415420532 5.0
0 0.0370939970016479 6.0
0 0.013745129108429 7.0
0 -0.185235619544983 8.0
0 -0.206841349601746 9.0
0 0.10318124294281 10.0
0 -0.107120335102081 11.0
0 -0.226550459861755 12.0
55.073028492 57.1760749816895 13.0
39.595488293 36.3618392944336 14.0
0 -0.113994002342224 15.0
0 0.249009430408478 16.0
43.159996453 34.5780258178711 17.0
0 0.162310004234314 18.0
0 -0.0284396409988403 19.0
0 0.070639967918396 20.0
0 0.24946141242981 21.0
0 0.0206617116928101 22.0
31.336650876 28.6013450622559 23.0
0 0.0925601124763489 24.0
0 0.989641904830933 25.0
0 -0.0543058514595032 26.0
0 -0.103865325450897 27.0
0 0.218489825725555 28.0
0 -0.155468881130219 29.0
0 0.145293474197388 30.0
0 0.213598728179932 31.0
0 -0.0630001425743103 32.0
0 -0.35595977306366 33.0
0 -0.0500727891921997 34.0
0 0.334806472063065 35.0
0 0.115851521492004 36.0
0 0.0265699625015259 37.0
0 -0.102845907211304 38.0
0 0.038305401802063 39.0
0 0.493127316236496 40.0
0 -0.105601489543915 41.0
0 0.189625024795532 42.0
305.31578846 296.319244384766 43.0
0 0.252094030380249 44.0
0 0.193770468235016 45.0
0 0.126491725444794 46.0
0 -0.280303835868835 47.0
0 -0.0707727670669556 48.0
0 0.369815289974213 49.0
0.097085774175 0.044577956199646 50.0
1.953680014 1.80674934387207 51.0
0 -0.202154219150543 52.0
0 0.0396395921707153 53.0
0 0.0027955174446106 54.0
0 0.0579646825790405 55.0
0 0.243763148784637 56.0
0 -0.0888674855232239 57.0
0 -0.0659569501876831 58.0
0 -0.293887615203857 59.0
0 0.851133525371552 60.0
0 0.0279420018196106 61.0
0 -0.0310518145561218 62.0
243.13752788 254.773941040039 0.0
0 -0.250833034515381 1.0
0 -0.027082085609436 2.0
0 0.401149123907089 3.0
0 0.0558359622955322 4.0
0 0.00600874423980713 5.0
0 -0.13374787569046 6.0
0 -0.048453688621521 7.0
0 0.265593886375427 8.0
0 -0.0410177111625671 9.0
0 0.382418900728226 10.0
0 -0.430897235870361 11.0
0 -0.139078795909882 12.0
73.745884554 74.9553070068359 13.0
20.783112889 23.5936679840088 14.0
0 -0.0410339832305908 15.0
0 -0.0967960357666016 16.0
31.324940445 35.7038078308105 17.0
0 0.111137390136719 18.0
0 -0.0332502722740173 19.0
0 0.102003157138824 20.0
0 0.0467312932014465 21.0
0 0.178060531616211 22.0
36.482584146 33.3789215087891 23.0
0 0.0368595719337463 24.0
0 0.399109601974487 25.0
0 0.0276763439178467 26.0
0 -0.0456709265708923 27.0
0 0.0672084093093872 28.0
0 -0.131864726543427 29.0
0 -0.025865375995636 30.0
0 0.0866076946258545 31.0
0 0.0160439014434814 32.0
0 -0.246716022491455 33.0
0 -0.757141947746277 34.0
0 0.31310510635376 35.0
0 -0.024902880191803 36.0
0 -0.0677200555801392 37.0
0 0.258194446563721 38.0
0 -0.32917058467865 39.0
0 0.170905351638794 40.0
0 -0.205809712409973 41.0
0 0.145401239395142 42.0
299.93648471 297.389465332031 43.0
0 0.0903496146202087 44.0
0 0.121069729328156 45.0
0 0.226560473442078 46.0
0 -0.353418946266174 47.0
0 0.223749697208405 48.0
0 0.401297092437744 49.0
0.094606658523 0.0340996980667114 50.0
1.7452614474 1.87232637405396 51.0
0 -0.181678771972656 52.0
0 -0.175069510936737 53.0
0 0.189568877220154 54.0
0 0.252914190292358 55.0
0 0.0151472091674805 56.0
0 0.384315222501755 57.0
0 0.0614528059959412 58.0
0 0.493601828813553 59.0
0 0.158463716506958 60.0
0 0.0682167410850525 61.0
0 0.113520979881287 62.0
260.21596519 252.611068725586 0.0
0 -0.223953366279602 1.0
0 -0.000189483165740967 2.0
0 0.158379733562469 3.0
0 -0.0336523056030273 4.0
0 0.181105256080627 5.0
0 0.0228410959243774 6.0
0 0.256505787372589 7.0
0 0.687778115272522 8.0
0 -0.0297771692276001 9.0
0 -0.0936623215675354 10.0
0 -0.119667530059814 11.0
0 -0.286065816879272 12.0
53.099206087 54.0163192749023 13.0
41.356023468 39.1114959716797 14.0
0 -0.0682228803634644 15.0
0 -0.107006907463074 16.0
33.019139068 32.4461555480957 17.0
0 0.0952751040458679 18.0
0 -0.0012514591217041 19.0
0 0.0896231532096863 20.0
0 0.212037801742554 21.0
0 -0.0793893337249756 22.0
31.756504593 27.9469013214111 23.0
0 0.0575557351112366 24.0
0 0.701929926872253 25.0
0 -0.126567006111145 26.0
0 -0.0315027236938477 27.0
0 0.189513385295868 28.0
0 -0.0530933141708374 29.0
0 0.134573042392731 30.0
0 0.11024010181427 31.0
0 0.082943320274353 32.0
0 0.0294253826141357 33.0
0 0.156758368015289 34.0
0 -0.178411662578583 35.0
0 0.135360240936279 36.0
0 0.0543188452720642 37.0
0 0.111201167106628 38.0
0 -0.138632297515869 39.0
0 0.61804336309433 40.0
0 -0.0809516906738281 41.0
0 -0.130618214607239 42.0
288.6975651 297.950225830078 43.0
0 0.0531808137893677 44.0
0 0.0790091753005981 45.0
0 0.22972697019577 46.0
0 -0.0718564391136169 47.0
0 0.191093504428864 48.0
0 0.32800018787384 49.0
0.069620648676 0.0313906669616699 50.0
1.892026531 1.8779274225235 51.0
0 -0.151577591896057 52.0
0 -0.00646233558654785 53.0
0 -0.0412540435791016 54.0
0 0.117324709892273 55.0
0 0.318073600530624 56.0
0 0.00038224458694458 57.0
0 0.0871195197105408 58.0
0 0.0922493934631348 59.0
0 -0.137375712394714 60.0
0 0.0953726768493652 61.0
0 0.335099279880524 62.0
258.84835025 251.916107177734 0.0
0 -0.249664425849915 1.0
0 0.0578056573867798 2.0
0 0.268956184387207 3.0
0 -0.0885415077209473 4.0
0 0.00168108940124512 5.0
0 -0.0486940145492554 6.0
0 0.298370212316513 7.0
0 -0.365549087524414 8.0
0 0.0343122482299805 9.0
0 0.700567841529846 10.0
0 -0.241331577301025 11.0
0 -0.155406951904297 12.0
72.231083023 71.1136322021484 13.0
22.391961858 26.8822689056396 14.0
0 -0.114209771156311 15.0
0 0.356497168540955 16.0
35.064090269 35.0309867858887 17.0
0 0.11032497882843 18.0
0 -0.0277381539344788 19.0
0 0.135281980037689 20.0
0 -0.0514706373214722 21.0
0 0.143234610557556 22.0
38.664490381 37.2332649230957 23.0
0 0.0291954874992371 24.0
0 0.325345695018768 25.0
0 -0.0765294432640076 26.0
0 -0.135373175144196 27.0
0 -0.212386012077332 28.0
0 -0.0500635504722595 29.0
0 -0.0769412517547607 30.0
0 0.0153043270111084 31.0
0 0.127554059028625 32.0
0 0.151296675205231 33.0
0 0.170049846172333 34.0
0 0.341751843690872 35.0
0 0.21341073513031 36.0
0 -0.0864979028701782 37.0
0 0.199536025524139 38.0
0 -0.283041954040527 39.0
0 0.339180707931519 40.0
0 -0.194912850856781 41.0
0 -0.0410377979278564 42.0
294.03980122 296.049621582031 43.0
0 0.0667978525161743 44.0
0 0.170706629753113 45.0
0 0.0441380143165588 46.0
0 0.131106555461884 47.0
0 -0.268874526023865 48.0
0 0.333476096391678 49.0
0.078676989074 -0.0117053985595703 50.0
2.1521124148 1.86331677436829 51.0
0 -0.0087130069732666 52.0
0 -0.340408563613892 53.0
0 0.0664790868759155 54.0
0 0.185377895832062 55.0
0 -0.0730851888656616 56.0
0 -0.00887554883956909 57.0
0 0.0963672399520874 58.0
0 -0.03449547290802 59.0
0 -0.0578986406326294 60.0
0 0.0507394075393677 61.0
0 0.105085968971252 62.0
278.14158706 251.24055480957 0.0
0 -0.15648341178894 1.0
0 0.0910689234733582 2.0
0 0.345719635486603 3.0
0 -0.0842458605766296 4.0
0 0.274419367313385 5.0
0 -0.205827355384827 6.0
0 0.344960063695908 7.0
0 0.714704632759094 8.0
0 -0.11628246307373 9.0
0 0.310537487268448 10.0
0 -0.320410013198853 11.0
0 0.100728631019592 12.0
53.599903491 54.2532119750977 13.0
38.219984152 39.5268440246582 14.0
0 -0.0189418792724609 15.0
0 -0.00712859630584717 16.0
36.402517354 36.2498664855957 17.0
0 0.10144305229187 18.0
0 -0.0750254988670349 19.0
0 0.0866498351097107 20.0
0 0.0394161939620972 21.0
0 0.261856734752655 22.0
30.137476528 28.9471454620361 23.0
0 0.0364723205566406 24.0
0 0.461770415306091 25.0
0 0.104453086853027 26.0
0 -0.186956584453583 27.0
0 0.143854677677155 28.0
0 0.0796313881874084 29.0
0 -0.0120871663093567 30.0
0 0.154565095901489 31.0
0 0.103582084178925 32.0
0 -0.236145734786987 33.0
0 -0.0343151688575745 34.0
0 0.234071671962738 35.0
0 0.0634835958480835 36.0
0 -0.0701025128364563 37.0
0 0.446706652641296 38.0
0 -0.212509036064148 39.0
0 0.139261722564697 40.0
0 -0.218095183372498 41.0
0 0.113053202629089 42.0
301.40603488 296.452606201172 43.0
0 0.234758138656616 44.0
0 0.0644657015800476 45.0
0 0.218462228775024 46.0
0 -0.0329182744026184 47.0
0 0.381297171115875 48.0
0 0.397405505180359 49.0
0.070955344363 0.0704470872879028 50.0
2.0021030415 1.90440034866333 51.0
0 0.112750470638275 52.0
0 0.177977025508881 53.0
0 0.158097147941589 54.0
0 0.147656857967377 55.0
0 0.239514052867889 56.0
0 0.158465445041656 57.0
0 0.0772924423217773 58.0
0 0.22750848531723 59.0
0 -0.0425461530685425 60.0
0 0.0328968167304993 61.0
0 0.143207669258118 62.0
222.53694425 253.387924194336 0.0
0 -0.00981581211090088 1.0
0 -0.061871349811554 2.0
0 -0.0320730805397034 3.0
0 -0.345996856689453 4.0
0 -0.263083577156067 5.0
0 0.111280500888824 6.0
0 -0.0591926574707031 7.0
0 0.648478806018829 8.0
0 -0.207477986812592 9.0
0 -0.0377335548400879 10.0
0 -0.399248600006104 11.0
0 -0.0569307804107666 12.0
58.566110583 57.8061218261719 13.0
31.244729912 35.1609497070312 14.0
0 -0.0853836536407471 15.0
0 -0.13873291015625 16.0
31.916165464 34.309326171875 17.0
0 0.0592085123062134 18.0
0 -0.0181896686553955 19.0
0 0.3120456635952 20.0
0 -0.0373349785804749 21.0
0 0.0424953699111938 22.0
37.946938735 38.5445404052734 23.0
0 0.03658127784729 24.0
0 0.815956354141235 25.0
0 0.0256109237670898 26.0
0 -0.0510006546974182 27.0
0 -0.0482139587402344 28.0
0 -0.0569504499435425 29.0
0 0.0597116947174072 30.0
0 0.103359818458557 31.0
0 0.0449855327606201 32.0
0 0.0669870972633362 33.0
0 0.226882994174957 34.0
0 0.116124272346497 35.0
0 -0.0955355763435364 36.0
0 -0.00625407695770264 37.0
0 0.0864757895469666 38.0
0 -0.0991840362548828 39.0
0 0.53008097410202 40.0
0 -0.185302734375 41.0
0 0.177693605422974 42.0
300.48097757 297.614288330078 43.0
0 0.0774504542350769 44.0
0 0.168746590614319 45.0
0 0.140405833721161 46.0
0 0.00943237543106079 47.0
0 -0.13884174823761 48.0
0 0.490749001502991 49.0
0.095123967199 -0.0145426988601685 50.0
1.8145255127 1.88856506347656 51.0
0 0.26077926158905 52.0
0 0.0982704162597656 53.0
0 0.145064294338226 54.0
0 -0.134965777397156 55.0
0 -0.166485190391541 56.0
0 0.0706899166107178 57.0
0 0.183579444885254 58.0
0 0.2989621758461 59.0
0 -0.26414155960083 60.0
0 0.101781785488129 61.0
0 0.348863959312439 62.0
212.06917478 253.399291992188 0.0
0 -0.0525902509689331 1.0
0 -0.193720459938049 2.0
0 0.161150634288788 3.0
0 -0.300605416297913 4.0
0 -0.234909892082214 5.0
0 -0.29874849319458 6.0
0 -0.339535236358643 7.0
0 0.832422316074371 8.0
0 0.00102776288986206 9.0
0 0.444621503353119 10.0
0 -0.274850487709045 11.0
0 -0.0518438816070557 12.0
63.615569412 63.7823066711426 13.0
22.844709003 36.2708625793457 14.0
0 0.0480740070343018 15.0
0 0.250553905963898 16.0
35.260481058 36.144645690918 17.0
0 0.142572700977325 18.0
0 0.0175724029541016 19.0
0 0.0230934619903564 20.0
0 0.167850732803345 21.0
0 0.302540004253387 22.0
34.333765282 32.6500587463379 23.0
0 0.0739912390708923 24.0
0 1.01923823356628 25.0
0 -0.0414029359817505 26.0
0 -0.0457074642181396 27.0
0 0.339297443628311 28.0
0 0.0571706295013428 29.0
0 -0.0926030874252319 30.0
0 -0.145396828651428 31.0
0 -0.183311760425568 32.0
0 -0.0583972334861755 33.0
0 0.494842827320099 34.0
0 1.0869562625885 35.0
0 -1.30307674407959 36.0
0 -0.0115753412246704 37.0
0 0.299206614494324 38.0
0 -0.389231324195862 39.0
0 0.171177983283997 40.0
0 -0.199570119380951 41.0
0 0.0220513343811035 42.0
309.06944081 297.525573730469 43.0
0 0.0659298896789551 44.0
0 0.10507196187973 45.0
0 0.0720016360282898 46.0
0 0.0502160787582397 47.0
0 0.329637765884399 48.0
0.43309061828 0.445161759853363 49.0
0.068300639146 0.0607087016105652 50.0
2.0357855881 1.89017057418823 51.0
0 -0.179110765457153 52.0
0 0.283604085445404 53.0
0 0.147350549697876 54.0
0 0.0995483994483948 55.0
0 -0.159893929958344 56.0
0 -0.0861474275588989 57.0
0 0.184662520885468 58.0
0 -0.116397619247437 59.0
0 -0.0802021622657776 60.0
0 0.233752489089966 61.0
0 0.298348724842072 62.0
277.3372159 251.872406005859 0.0
0 -0.567695021629333 1.0
0 -0.0319991111755371 2.0
0 -0.00121438503265381 3.0
0 -0.163185656070709 4.0
0 0.386711448431015 5.0
0 0.253092110157013 6.0
0 0.311517536640167 7.0
0 0.106503486633301 8.0
0 -0.100991725921631 9.0
0 -0.0734143853187561 10.0
0 -0.0502234697341919 11.0
0 -0.0154882073402405 12.0
52.135840533 52.6896514892578 13.0
45.26284574 43.5825614929199 14.0
0 -0.169034123420715 15.0
0 0.27717262506485 16.0
32.806505844 34.4349098205566 17.0
0 0.0536162853240967 18.0
0 -0.0332573056221008 19.0
0 0.0767301321029663 20.0
0 0.17452609539032 21.0
0 -0.00445020198822021 22.0
28.305452457 26.477388381958 23.0
0 0.0489174127578735 24.0
16.213087482 0.394532084465027 25.0
0 0.0156276226043701 26.0
0 -0.173975944519043 27.0
0 0.277515470981598 28.0
0 -0.0913810729980469 29.0
0 -0.0209715366363525 30.0
0 0.122638821601868 31.0
0 0.0712911486625671 32.0
0 0.0128987431526184 33.0
0 0.164649486541748 34.0
0 -0.182015895843506 35.0
0 0.154619932174683 36.0
0 -0.0393522381782532 37.0
0 0.111675202846527 38.0
0 -0.168274164199829 39.0
0 0.328066259622574 40.0
0 -0.125699043273926 41.0
0 0.0546644926071167 42.0
292.152561 296.283843994141 43.0
0 0.0261817574501038 44.0
0 0.1389200091362 45.0
0 0.29237824678421 46.0
0 -0.095198929309845 47.0
0 -0.277984142303467 48.0
0 0.392909198999405 49.0
0.076956602856 -0.0342848300933838 50.0
1.5658015475 1.91283273696899 51.0
0 -0.0840308666229248 52.0
0 0.0536285638809204 53.0
0 0.00959169864654541 54.0
0 0.288784265518188 55.0
0 0.0671898126602173 56.0
0 0.55034327507019 57.0
0 0.25665682554245 58.0
0 0.0717085599899292 59.0
0 -0.0383689403533936 60.0
0 0.050664484500885 61.0
0 0.222335755825043 62.0
240.35396961 252.319885253906 0.0
0 -0.251824259757996 1.0
0 -0.0318490266799927 2.0
0 0.11512279510498 3.0
0 -0.371098518371582 4.0
0 -0.173277974128723 5.0
0 0.0298171043395996 6.0
0 -0.103107273578644 7.0
0 0.119494557380676 8.0
0 -0.0318719744682312 9.0
0 -0.0335497260093689 10.0
0 -0.498206257820129 11.0
0 -0.130938410758972 12.0
70.913581622 71.1125946044922 13.0
25.855697556 28.9852981567383 14.0
0 -0.0443059206008911 15.0
0 -0.0122422575950623 16.0
27.858357745 33.9352874755859 17.0
0 0.0826065540313721 18.0
0 0.00208938121795654 19.0
0 0.0896992683410645 20.0
0 0.058855414390564 21.0
0 0.244405508041382 22.0
35.305742439 34.2566413879395 23.0
0 0.0908049345016479 24.0
0 0.727607131004333 25.0
0 -0.00655484199523926 26.0
0 -0.133739292621613 27.0
0 0.301204770803452 28.0
0 -0.0837391018867493 29.0
0 0.0721011161804199 30.0
0 -0.046305775642395 31.0
0 0.0497956871986389 32.0
0 -0.344313859939575 33.0
0 0.0864068269729614 34.0
0 -0.0642398595809937 35.0
0 0.509359657764435 36.0
0 0.1652472615242 37.0
0 -0.0316457152366638 38.0
0 -0.397836685180664 39.0
0 -0.055523157119751 40.0
0 -0.272379755973816 41.0
0 0.0253602266311646 42.0
303.54388244 297.052185058594 43.0
0 0.158754944801331 44.0
0 0.0851505994796753 45.0
0 0.0642457008361816 46.0
0 0.246776461601257 47.0
0 0.317284345626831 48.0
0.93647173163 0.480924814939499 49.0
0.096348310978 0.0305425524711609 50.0
2.280504512 1.89003956317902 51.0
0 -0.0360195636749268 52.0
0 -0.22385835647583 53.0
0 -0.0485465526580811 54.0
0 0.251927733421326 55.0
0 0.15556013584137 56.0
0 -0.058950662612915 57.0
0 0.238699615001678 58.0
0 -0.1934894323349 59.0
0 0.334494352340698 60.0
0 0.0818203687667847 61.0
0 0.271628618240356 62.0
263.2494073 255.034805297852 0.0
0 -0.198262929916382 1.0
0 0.0718845725059509 2.0
0 -0.244319796562195 3.0
0 -0.256385326385498 4.0
0 0.0959751009941101 5.0
0 0.254948973655701 6.0
0 0.247151970863342 7.0
0 0.398486435413361 8.0
0 -0.0271555185317993 9.0
0 0.0163789391517639 10.0
0 -0.147051572799683 11.0
0 0.0907293558120728 12.0
50.264681029 50.9296493530273 13.0
38.649297015 41.8352813720703 14.0
0 -0.0122231245040894 15.0
0 -0.240379333496094 16.0
29.322728733 32.0632629394531 17.0
0 0.0760884284973145 18.0
0 0.0396636724472046 19.0
0 0.139857769012451 20.0
0 0.014165461063385 21.0
0 0.0742858648300171 22.0
38.673965994 38.976390838623 23.0
0 0.0565102100372314 24.0
0 0.796785175800323 25.0
0 -0.162361443042755 26.0
0 -0.203364849090576 27.0
0 0.639571130275726 28.0
0 0.219698190689087 29.0
0 -0.0617262125015259 30.0
0 0.127476215362549 31.0
0 0.0670788288116455 32.0
0 0.0871239900588989 33.0
0 0.308450043201447 34.0
0 0.278071284294128 35.0
0 0.375445991754532 36.0
0 0.121822118759155 37.0
0 -0.0624064207077026 38.0
0 -0.272963523864746 39.0
0 0.466044187545776 40.0
0 -0.114946782588959 41.0
0 -0.0090184211730957 42.0
307.18118865 293.691375732422 43.0
0 0.115515947341919 44.0
0 0.00493568181991577 45.0
0 0.356220960617065 46.0
0 -0.574671983718872 47.0
0 0.392499059438705 48.0
1.7545256831 0.443036079406738 49.0
0.1032949599 0.0506702065467834 50.0
1.4554644357 1.88061785697937 51.0
0 -0.356150507926941 52.0
0 -0.400395512580872 53.0
0 -0.128774702548981 54.0
0 -0.16559237241745 55.0
0 0.15132737159729 56.0
0 0.0568726658821106 57.0
0 0.154195547103882 58.0
0 0.14657735824585 59.0
0 -0.138194799423218 60.0
0 0.0813771486282349 61.0
0 0.278044521808624 62.0
222.59798319 252.479263305664 0.0
0 -0.0144221782684326 1.0
0 -0.0939986109733582 2.0
0 -0.321719646453857 3.0
0 -0.276402473449707 4.0
0 -0.0213943123817444 5.0
0 -0.334148645401001 6.0
0 -0.263845801353455 7.0
0 0.102429091930389 8.0
0 0.169512987136841 9.0
0 0.212579071521759 10.0
0 -0.322859048843384 11.0
0 -0.0649653673171997 12.0
67.648901167 65.1710586547852 13.0
16.925311353 32.8296051025391 14.0
0 -0.141857266426086 15.0
0 -0.0570774078369141 16.0
33.533115246 33.8632354736328 17.0
0 0.0654631853103638 18.0
0 -0.0151024460792542 19.0
0 0.12335991859436 20.0
0 0.0615918636322021 21.0
0 0.292840868234634 22.0
32.663695397 31.5750160217285 23.0
0 0.0904305577278137 24.0
0 0.978974640369415 25.0
0 0.043940007686615 26.0
0 -0.0846848487854004 27.0
0 0.420762211084366 28.0
0 -0.00689452886581421 29.0
0 0.06243896484375 30.0
0 0.0538910627365112 31.0
0 0.176847159862518 32.0
0 0.13331264257431 33.0
0 0.361528396606445 34.0
0 0.154459178447723 35.0
0 0.353835850954056 36.0
0 -0.088444709777832 37.0
0 0.187080204486847 38.0
0 -0.422021865844727 39.0
0 0.315945148468018 40.0
0 -0.187068939208984 41.0
0 0.0921610593795776 42.0
295.29786484 297.606536865234 43.0
0 0.0633041858673096 44.0
0 0.0136188864707947 45.0
0 0.258745551109314 46.0
0 -0.126534819602966 47.0
0 0.114362776279449 48.0
0 0.498147428035736 49.0
0.078235822518 -0.0282211899757385 50.0
2.0603740868 1.90590810775757 51.0
0 0.0381549000740051 52.0
0 -0.567211389541626 53.0
0 -0.0597728490829468 54.0
0 0.154177486896515 55.0
0 -0.0439794063568115 56.0
0 -0.0376237630844116 57.0
0 0.230548143386841 58.0
0 0.14347243309021 59.0
0 -0.230195164680481 60.0
0 0.0626811981201172 61.0
0 0.121349513530731 62.0
246.6890976 252.524642944336 0.0
0 -0.0596184730529785 1.0
0 -0.115967929363251 2.0
0 0.422155380249023 3.0
0 -0.11512964963913 4.0
0 0.260470330715179 5.0
0 0.170524179935455 6.0
0 0.149102687835693 7.0
0 -0.117217004299164 8.0
0 -0.0299239754676819 9.0
0 0.402957916259766 10.0
0 -0.30472469329834 11.0
0 0.0469098091125488 12.0
75.44972482 74.9308013916016 13.0
13.304151842 26.6119785308838 14.0
0 0.139048099517822 15.0
0 -0.295803785324097 16.0
31.968047318 30.8446598052979 17.0
0 0.0718920230865479 18.0
0 0.0539995431900024 19.0
0 0.226328611373901 20.0
0 -0.165413498878479 21.0
0 1.33802700042725 22.0
43.245574436 40.9041175842285 23.0
0 0.0657656192779541 24.0
0 1.38369297981262 25.0
0 -0.0888662338256836 26.0
0 0.0666466355323792 27.0
0 1.00050318241119 28.0
0 0.0332774519920349 29.0
0 0.02568119764328 30.0
0 0.178394675254822 31.0
0 0.260026216506958 32.0
0 0.0920793414115906 33.0
0 -0.109506607055664 34.0
0 -0.00227087736129761 35.0
0 0.0928937196731567 36.0
0 -0.0308531522750854 37.0
0 0.204772889614105 38.0
0 -0.185211777687073 39.0
0 0.882206439971924 40.0
0 -0.0747172832489014 41.0
0 -0.118416786193848 42.0
297.94637097 296.217071533203 43.0
0 0.0595916509628296 44.0
0 0.182316720485687 45.0
0 0.10918790102005 46.0
0 0.175464987754822 47.0
0 -0.0704123973846436 48.0
0 0.228555083274841 49.0
0.077762557051 0.0786998867988586 50.0
1.9892170678 1.82672023773193 51.0
0 0.129285931587219 52.0
0 0.269182026386261 53.0
0 0.176367223262787 54.0
0 0.269740343093872 55.0
0 0.0732951164245605 56.0
0 0.0778228044509888 57.0
0 0.4599729180336 58.0
0 0.368691146373749 59.0
0 -0.116259872913361 60.0
0 0.169678270816803 61.0
0 0.377353310585022 62.0
267.16714832 252.302154541016 0.0
0 -0.233604073524475 1.0
0 -0.0651408433914185 2.0
0 0.355446636676788 3.0
0 -0.182950496673584 4.0
0 0.220701932907104 5.0
0 0.0417980551719666 6.0
0 0.26366913318634 7.0
0 -0.08392333984375 8.0
0 -0.204575300216675 9.0
0 0.205306053161621 10.0
0 -0.205627620220184 11.0
0 -0.0234240293502808 12.0
69.653781677 69.6832962036133 13.0
22.436049422 29.5561637878418 14.0
0 0.0821430683135986 15.0
0 -0.249571800231934 16.0
33.012393054 33.8458557128906 17.0
0 0.0811996459960938 18.0
0 -0.0295332670211792 19.0
0 0.056766152381897 20.0
0 0.202182114124298 21.0
0 0.13520210981369 22.0
33.770314693 31.4209804534912 23.0
0 0.0315991640090942 24.0
0 0.357713729143143 25.0
0 -0.141534686088562 26.0
0 -0.123561918735504 27.0
0 0.471903026103973 28.0
0 -0.0347211360931396 29.0
0 0.0847228169441223 30.0
0 0.37142276763916 31.0
0 0.190639734268188 32.0
0 0.12888890504837 33.0
0 0.409607797861099 34.0
0 -0.150569558143616 35.0
0 0.12697035074234 36.0
0 -0.0314879417419434 37.0
0 -0.0464434623718262 38.0
0 -0.231566309928894 39.0
0 0.0137044191360474 40.0
0 -0.120116829872131 41.0
0 0.0720069408416748 42.0
288.85092307 297.051849365234 43.0
0 0.165387630462646 44.0
0 0.0281069874763489 45.0
0 0.193514168262482 46.0
0 -0.928463459014893 47.0
0 -0.00320565700531006 48.0
0 0.457821607589722 49.0
0.081741139749 0.0201064944267273 50.0
1.6631941818 1.81237614154816 51.0
0 -0.262545228004456 52.0
0 -0.880646944046021 53.0
0 -0.00722920894622803 54.0
0 -0.00348317623138428 55.0
0 -0.0292591452598572 56.0
0 -0.185969114303589 57.0
0 0.153439521789551 58.0
0 -0.472547888755798 59.0
0 0.537346959114075 60.0
0 0.0155983567237854 61.0
0 0.0867536664009094 62.0
230.31567264 254.280014038086 0.0
0 -0.155969023704529 1.0
0 -0.311014771461487 2.0
0 0.187014639377594 3.0
0 -0.284797072410583 4.0
0 -0.178353130817413 5.0
0 0.11212831735611 6.0
0 0.0425660610198975 7.0
0 0.0973992347717285 8.0
0 0.241878926753998 9.0
0 0.168015778064728 10.0
0 -0.194657385349274 11.0
0 -0.0364944934844971 12.0
65.267520376 64.2482757568359 13.0
34.263413919 31.4926071166992 14.0
0 -0.23924708366394 15.0
0 0.046761691570282 16.0
37.098704112 36.5812568664551 17.0
0 0.072512686252594 18.0
0 -0.0472829341888428 19.0
0 0.0883563756942749 20.0
0 0.127275407314301 21.0
0 0.0531980991363525 22.0
34.595929362 30.4039268493652 23.0
0 0.0474935173988342 24.0
0 0.539765119552612 25.0
0 -0.170096933841705 26.0
0 -0.107288599014282 27.0
0 1.08320903778076 28.0
0 0.0186313986778259 29.0
0 0.132640779018402 30.0
0 0.0688537359237671 31.0
0 -0.157773554325104 32.0
0 -0.413110733032227 33.0
0 -0.176424205303192 34.0
0 -0.00945347547531128 35.0
0 0.220548570156097 36.0
0 0.144940078258514 37.0
0 -0.502227187156677 38.0
0 -0.0853947401046753 39.0
0 0.345406711101532 40.0
0 -0.258181571960449 41.0
0 0.0712082386016846 42.0
313.37392863 295.819702148438 43.0
0 0.15363883972168 44.0
0 0.0612763166427612 45.0
0 0.0270107388496399 46.0
0 0.0956943035125732 47.0
0 0.305320739746094 48.0
0 0.471136689186096 49.0
0.067200362493 0.00863152742385864 50.0
2.2521492529 1.89679074287415 51.0
0 0.0360221862792969 52.0
0 0.144720315933228 53.0
0 0.247981071472168 54.0
0 0.115939617156982 55.0
0 0.003318190574646 56.0
0 0.424545109272003 57.0
0 0.158588469028473 58.0
0 -0.0236921906471252 59.0
0 -0.0265631079673767 60.0
0 0.0995610952377319 61.0
0 0.321874141693115 62.0
275.44364401 251.015899658203 0.0
0 -0.267341732978821 1.0
0 -0.160192668437958 2.0
0 0.186857521533966 3.0
0 0.100671589374542 4.0
0 0.35455334186554 5.0
0 0.0944822430610657 6.0
0 0.113422870635986 7.0
0 -0.0717930793762207 8.0
0 0.107345223426819 9.0
0 -0.300202012062073 10.0
0 -0.0142968893051147 11.0
0 0.0349381566047668 12.0
57.672905438 58.7820129394531 13.0
39.881812559 40.106559753418 14.0
0 -0.0453211069107056 15.0
0 0.0716030597686768 16.0
35.573422543 34.4400215148926 17.0
0 0.137207925319672 18.0
0 -0.0327547192573547 19.0
0 0.165079414844513 20.0
0 0.13678503036499 21.0
0 0.155254602432251 22.0
36.965364417 33.7597122192383 23.0
0 0.0428963899612427 24.0
0 0.431611359119415 25.0
0 -0.137901604175568 26.0
0 -0.12736040353775 27.0
0 0.24462503194809 28.0
0 0.0170860290527344 29.0
0 0.0287966728210449 30.0
0 -0.161758422851562 31.0
0 -0.0979514122009277 32.0
0 -0.0846934914588928 33.0
0 -0.065169632434845 34.0
0 -0.00134444236755371 35.0
0 0.473356425762177 36.0
0 0.151444733142853 37.0
0 0.139928162097931 38.0
0 -0.374260187149048 39.0
0 0.494424790143967 40.0
0 -0.02297043800354 41.0
0 -0.136969923973083 42.0
282.71161507 296.489990234375 43.0
0 0.0345497131347656 44.0
0 0.383710920810699 45.0
0 0.0584039092063904 46.0
0 -0.204878628253937 47.0
0 0.314892292022705 48.0
0 0.418202459812164 49.0
0.067063441153 0.0799381732940674 50.0
2.1714743454 1.89988100528717 51.0
0 0.134252309799194 52.0
0 0.0186684727668762 53.0
0 0.00192511081695557 54.0
0 0.260331034660339 55.0
0 0.271728575229645 56.0
0 0.0559327602386475 57.0
0 0.514943838119507 58.0
0 -0.122910022735596 59.0
0 -0.160368204116821 60.0
0 0.0496819019317627 61.0
0 0.147168099880219 62.0
244.75775225 254.869384765625 0.0
0 0.0866513252258301 1.0
0 -0.0666296482086182 2.0
0 0.276440560817719 3.0
0 0.00449860095977783 4.0
0 -0.250222444534302 5.0
0 0.06683349609375 6.0
0 0.404347062110901 7.0
0 0.243612289428711 8.0
0 -0.51146125793457 9.0
0 0.445579886436462 10.0
0 -0.21994960308075 11.0
0 -0.11672306060791 12.0
76.700689116 73.4517288208008 13.0
6.3886619918 27.9311618804932 14.0
0 0.0240588188171387 15.0
0 -0.254308700561523 16.0
28.483837565 30.5049419403076 17.0
0 0.0636473894119263 18.0
0 0.054652214050293 19.0
0 0.187067627906799 20.0
0 0.170589327812195 21.0
0 -0.214156985282898 22.0
38.299228571 39.63037109375 23.0
0 0.127793431282043 24.0
0 0.573613226413727 25.0
0 -0.0934543609619141 26.0
0 0.0278064012527466 27.0
0 0.300265967845917 28.0
0 0.0101699829101562 29.0
0 0.0591263175010681 30.0
0 0.0999249219894409 31.0
0 0.142516195774078 32.0
0 0.0510504841804504 33.0
0 0.302632987499237 34.0
0 0.282733917236328 35.0
0 0.342439234256744 36.0
0 -0.0733287930488586 37.0
0 0.282362997531891 38.0
0 -0.330194234848022 39.0
0 0.530722200870514 40.0
0 -0.209228992462158 41.0
0 0.16938328742981 42.0
291.61028005 295.730651855469 43.0
0 0.0786287188529968 44.0
0 0.142572104930878 45.0
0 0.175873279571533 46.0
0 0.106596231460571 47.0
0 -0.332318902015686 48.0
0 0.473476767539978 49.0
0.08486051157 -0.0103384852409363 50.0
2.2265903285 1.86112236976624 51.0
0 0.179738640785217 52.0
0 -0.175731420516968 53.0
0 -0.0942086577415466 54.0
0 0.130965709686279 55.0
0 0.0644600391387939 56.0
0 0.401743769645691 57.0
0 0.627886831760406 58.0
0 0.302700877189636 59.0
0 -0.200730979442596 60.0
0 0.0710071921348572 61.0
0 0.189730942249298 62.0
288.10627486 254.186279296875 0.0
0 -0.361962676048279 1.0
0 0.0699641704559326 2.0
0 0.493511974811554 3.0
0 -0.0599216222763062 4.0
0 -0.0413950681686401 5.0
0 -0.329426884651184 6.0
0 0.381684064865112 7.0
0 -0.0962592363357544 8.0
0 0.423167586326599 9.0
0 0.381559818983078 10.0
0 -0.101117610931396 11.0
0 -0.214889287948608 12.0
62.914499448 60.998966217041 13.0
38.025730684 36.8532638549805 14.0
0 -0.102809309959412 15.0
0 -0.244613885879517 16.0
29.391276531 30.9258842468262 17.0
0 0.0567855834960938 18.0
0 0.0144193172454834 19.0
0 0.135396718978882 20.0
0 0.220905542373657 21.0
0 -0.111034393310547 22.0
29.343137579 26.9708843231201 23.0
0 0.0575447082519531 24.0
0 0.608338952064514 25.0
0 -0.0130972862243652 26.0
0 -0.105267345905304 27.0
0 0.31302547454834 28.0
0 -0.0153868198394775 29.0
0 0.100052773952484 30.0
0 -0.0701605081558228 31.0
0 -0.0864931344985962 32.0
0 -0.133029997348785 33.0
0 0.198724806308746 34.0
0 0.285932600498199 35.0
0 0.325565308332443 36.0
0 0.109088480472565 37.0
0 0.190632700920105 38.0
0 -0.19811749458313 39.0
0 0.218867063522339 40.0
0 -0.150031864643097 41.0
0 -0.0527547597885132 42.0
293.55354108 297.487609863281 43.0
0 0.0791410207748413 44.0
0 0.0851070880889893 45.0
0 0.120014250278473 46.0
0 0.173967778682709 47.0
0 0.110237300395966 48.0
0 0.533787488937378 49.0
0.097549822087 0.0045619010925293 50.0
1.9293970625 1.85113453865051 51.0
0 0.00104564428329468 52.0
0 -0.208174228668213 53.0
0 -0.0191745758056641 54.0
0 0.0120737552642822 55.0
0 0.0439876317977905 56.0
0 -0.0243116021156311 57.0
0 0.0138392448425293 58.0
0 -0.0854580402374268 59.0
0 -0.113960266113281 60.0
0 0.0605756044387817 61.0
0 0.132853448390961 62.0
269.35934311 252.98503112793 0.0
0 -0.177655696868896 1.0
0 0.121633410453796 2.0
0 0.504690051078796 3.0
0 0.0649025440216064 4.0
0 0.174097895622253 5.0
0 -0.317468523979187 6.0
0 0.445180833339691 7.0
0 0.515405893325806 8.0
0 -0.278394937515259 9.0
0 0.335061192512512 10.0
0 -0.035423755645752 11.0
0 -0.207860231399536 12.0
62.960999471 64.2965850830078 13.0
36.710074915 26.5747337341309 14.0
0 -0.0028841495513916 15.0
0 0.020421028137207 16.0
40.197456205 36.1093063354492 17.0
0 0.144399285316467 18.0
0 -0.0897289514541626 19.0
0 0.0586497783660889 20.0
0 0.0721468925476074 21.0
0 0.198165416717529 22.0
34.603412945 31.4714679718018 23.0
0 0.0769387483596802 24.0
0 0.479793965816498 25.0
0 -0.150553941726685 26.0
0 0.0554413795471191 27.0
0 0.423866480588913 28.0
0 -0.0515242218971252 29.0
0 0.0993219017982483 30.0
0 -0.0347758531570435 31.0
0 0.115691661834717 32.0
0 -0.0622435212135315 33.0
0 0.409716695547104 34.0
0 0.245523691177368 35.0
0 0.443704962730408 36.0
0 0.0240360498428345 37.0
0 0.405605733394623 38.0
0 -0.26738166809082 39.0
0 0.486417055130005 40.0
0 -0.234299659729004 41.0
0 0.0758367776870728 42.0
291.21260012 296.401733398438 43.0
0 0.0505968332290649 44.0
0 0.0630683898925781 45.0
0 -0.0450088381767273 46.0
0 0.430761635303497 47.0
0 0.340003550052643 48.0
0.54675711017 0.567347228527069 49.0
0.070668795436 0.0591608285903931 50.0
2.3651984301 1.87779545783997 51.0
0 0.0349048376083374 52.0
0 -0.0497210621833801 53.0
0 0.446106016635895 54.0
0 0.211478650569916 55.0
0 -0.0823583006858826 56.0
0 0.0226285457611084 57.0
0 0.619905710220337 58.0
0 0.206421613693237 59.0
0 0.0434005856513977 60.0
0 0.0106103420257568 61.0
0 0.128996312618256 62.0
241.0293595 253.16926574707 0.0
0 0.180439114570618 1.0
0 -0.0999429225921631 2.0
0 0.097124457359314 3.0
0 -0.272828698158264 4.0
0 -0.162239551544189 5.0
0 0.0418323278427124 6.0
0 -0.0828307867050171 7.0
0 0.542259275913239 8.0
0 -0.765317559242249 9.0
0 -0.258445978164673 10.0
0 -0.0969675779342651 11.0
0 -0.00980472564697266 12.0
62.809096643 63.918098449707 13.0
31.011153532 30.4275684356689 14.0
0 -0.048608660697937 15.0
0 0.00376224517822266 16.0
33.532459082 34.6966323852539 17.0
0 0.152114629745483 18.0
0 -0.0280561447143555 19.0
0 0.049315333366394 20.0
0 0.116830706596375 21.0
0 0.238038897514343 22.0
33.546000417 32.209041595459 23.0
0 0.0852624773979187 24.0
0 1.05257081985474 25.0
0 -0.0484099984169006 26.0
0 -0.00475484132766724 27.0
0 -0.0259114503860474 28.0
0 -0.119684815406799 29.0
0 0.0655697584152222 30.0
0 -0.0135139226913452 31.0
0 -0.0224235057830811 32.0
0 -0.222596168518066 33.0
0 0.0884635448455811 34.0
0 0.296135872602463 35.0
0 0.15771496295929 36.0
0 0.155404090881348 37.0
0 0.0278941988945007 38.0
0 -0.243237257003784 39.0
0 0.239379525184631 40.0
0 -0.255824565887451 41.0
0 0.0472644567489624 42.0
292.51363609 295.878479003906 43.0
0 0.117729067802429 44.0
0 0.153493821620941 45.0
0 0.264608979225159 46.0
0 -0.0267747044563293 47.0
0 0.207604825496674 48.0
0 0.466922283172607 49.0
0.075020841311 0.0161522626876831 50.0
2.1502454019 1.88555669784546 51.0
0 0.0112361907958984 52.0
0 -0.10705155134201 53.0
0 0.0775078535079956 54.0
0 0.293290078639984 55.0
0 0.130806505680084 56.0
0 0.01945960521698 57.0
0 0.00148004293441772 58.0
0 0.135502338409424 59.0
0 -0.236650586128235 60.0
0 0.102277934551239 61.0
0 0.244254469871521 62.0
257.60342654 242.028701782227 0.0
0 -0.208669662475586 1.0
0 -0.0663943290710449 2.0
0 0.294533014297485 3.0
0 -0.430061340332031 4.0
0 -0.107339084148407 5.0
0 0.0341734886169434 6.0
0 -0.203400671482086 7.0
0 0.0552734136581421 8.0
0 -0.0287826657295227 9.0
0 -0.128076374530792 10.0
0 0.252611517906189 11.0
0 -0.147138357162476 12.0
79.472037348 82.4358367919922 13.0
29.734155259 17.9211006164551 14.0
0 -0.0998327732086182 15.0
0 -0.0239939093589783 16.0
30.444792609 33.7880973815918 17.0
0 0.157961249351501 18.0
0 0.0456043481826782 19.0
0 -0.01880943775177 20.0
0 0.315838515758514 21.0
0 -0.261597275733948 22.0
26.815846394 25.7992534637451 23.0
0 0.0805667042732239 24.0
0 0.338886916637421 25.0
0 -0.120570540428162 26.0
0 -0.181616127490997 27.0
0 -0.716468095779419 28.0
0 -0.0190326571464539 29.0
0 -0.0999840497970581 30.0
0 0.298919200897217 31.0
0 0.299775660037994 32.0
0 0.0371710062026978 33.0
0 0.19989937543869 34.0
0 -0.0571103096008301 35.0
0 -0.350785851478577 36.0
0 0.14210456609726 37.0
0 0.19421774148941 38.0
0 -0.129122614860535 39.0
0 0.595785856246948 40.0
0 -0.0630951523780823 41.0
0 0.015898585319519 42.0
281.87258828 295.555145263672 43.0
0 0.290702700614929 44.0
0 0.0249031782150269 45.0
0 0.174137473106384 46.0
0 -0.79290771484375 47.0
0 0.29504606127739 48.0
0 0.413230776786804 49.0
0.082652401439 0.0841031074523926 50.0
1.5683646733 1.82497572898865 51.0
0 -0.121439933776855 52.0
0 0.429169178009033 53.0
0 0.05177903175354 54.0
0 0.0861415863037109 55.0
0 0.21604710817337 56.0
0 -0.115307688713074 57.0
0 -0.122532665729523 58.0
0 0.152078449726105 59.0
0 0.236338078975677 60.0
0 0.105475902557373 61.0
0 0.151672065258026 62.0
274.11611659 254.804977416992 0.0
0 -0.0488462448120117 1.0
0 0.0233635306358337 2.0
0 0.446593880653381 3.0
0 -0.28741180896759 4.0
0 0.199557662010193 5.0
0 -0.0170501470565796 6.0
0 0.410667836666107 7.0
0 0.248067140579224 8.0
0 0.11430948972702 9.0
0 0.126678824424744 10.0
0 -0.120925009250641 11.0
0 -0.145143389701843 12.0
74.933474076 72.5908203125 13.0
21.708000001 25.683177947998 14.0
0 0.00202250480651855 15.0
0 0.0643793344497681 16.0
34.476638225 37.2477912902832 17.0
0 0.189170360565186 18.0
0 0.00448876619338989 19.0
0 0.0489982962608337 20.0
0 0.0118592977523804 21.0
0 -0.0515021085739136 22.0
41.994809507 39.0230102539062 23.0
0 0.0556954145431519 24.0
0 0.927114844322205 25.0
0 0.00707721710205078 26.0
0 0.250030994415283 27.0
0 -0.0664452314376831 28.0
0 -0.106833457946777 29.0
0 0.126098990440369 30.0
0 0.196068048477173 31.0
0 0.204731643199921 32.0
0 0.0999325513839722 33.0
0 -0.325744152069092 34.0
0 -0.0024266242980957 35.0
0 -0.13742059469223 36.0
0 0.088894248008728 37.0
0 0.163122475147247 38.0
0 -0.175646543502808 39.0
0 0.864586472511292 40.0
0 -0.0203803777694702 41.0
0 -0.0373222827911377 42.0
304.05714558 297.753662109375 43.0
0 0.199406385421753 44.0
0 0.0109820961952209 45.0
0 0.0251664519309998 46.0
0 -0.00909161567687988 47.0
0 -0.0252061486244202 48.0
0 0.5205157995224 49.0
0.097559138507 0.0287480354309082 50.0
2.2657307849 1.79203581809998 51.0
0 0.0940350890159607 52.0
0 -0.119712352752686 53.0
0 -0.105919122695923 54.0
0 -0.0177444815635681 55.0
0 0.164724409580231 56.0
0 0.283609807491302 57.0
0 0.220807552337646 58.0
0 0.117906212806702 59.0
0 -0.0747923851013184 60.0
0 0.0481627583503723 61.0
0 0.227807283401489 62.0
254.8697006 253.985290527344 0.0
0 -0.262784481048584 1.0
0 0.0166058540344238 2.0
0 -0.591169357299805 3.0
0 -0.474157810211182 4.0
0 0.312384814023972 5.0
0 -0.0162020325660706 6.0
0 0.0795441865921021 7.0
0 0.0794053077697754 8.0
0 0.0552045106887817 9.0
0 0.207948744297028 10.0
0 0.36038863658905 11.0
0 -0.335399508476257 12.0
72.287598451 76.5258560180664 13.0
38.217445805 26.0840473175049 14.0
0 -0.000548362731933594 15.0
0 -0.0202984809875488 16.0
26.550774978 33.5232391357422 17.0
0 0.129404723644257 18.0
0 -0.0028148889541626 19.0
0 0.0905991196632385 20.0
0 0.120152592658997 21.0
0 0.0954780578613281 22.0
34.696864364 35.0016746520996 23.0
0 0.148986339569092 24.0
0 0.971344172954559 25.0
0 0.12168562412262 26.0
0 -0.0231472253799438 27.0
0 0.195819437503815 28.0
0 -0.0184926986694336 29.0
0 -0.0500047206878662 30.0
0 0.0550270080566406 31.0
0 0.100318193435669 32.0
0 0.0839026570320129 33.0
0 0.00438183546066284 34.0
0 0.151625275611877 35.0
0 0.234931409358978 36.0
0 0.0390284657478333 37.0
0 0.258404612541199 38.0
0 -0.235969185829163 39.0
0 0.625206887722015 40.0
0 -0.170162618160248 41.0
0 0.0197670459747314 42.0
295.24219041 295.787078857422 43.0
0 0.0415050983428955 44.0
0 0.0295382738113403 45.0
0 0.229845225811005 46.0
0 -0.427190065383911 47.0
0 0.372946947813034 48.0
0 0.572274506092072 49.0
0.10146427836 0.0453991889953613 50.0
2.3399098852 1.80368375778198 51.0
0 -0.0665856599807739 52.0
0 -0.370056509971619 53.0
0 0.0294795632362366 54.0
0 -0.00308632850646973 55.0
0 -0.0404226779937744 56.0
0 -0.359662055969238 57.0
0 -0.372606873512268 58.0
0 0.230552434921265 59.0
0 0.223644733428955 60.0
0 0.237829029560089 61.0
0 0.425055980682373 62.0
250.97533467 252.439544677734 0.0
0 -0.115180373191833 1.0
0 -0.209275960922241 2.0
0 -0.15385377407074 3.0
0 -0.382667541503906 4.0
0 -0.0328342914581299 5.0
0 0.168386578559875 6.0
0 -0.24461030960083 7.0
0 0.122546434402466 8.0
0 0.0699053406715393 9.0
0 -0.494326233863831 10.0
0 -0.00191634893417358 11.0
0 -0.0639723539352417 12.0
57.215070168 58.6367416381836 13.0
32.617684704 29.6027393341064 14.0
0 -0.0749441385269165 15.0
0 -0.00191420316696167 16.0
33.134670835 33.5298042297363 17.0
0 0.169710993766785 18.0
0 -0.101119995117188 19.0
0 0.177242875099182 20.0
0 0.0983526706695557 21.0
0 0.188551783561707 22.0
28.009451411 23.8182163238525 23.0
0 -0.0229897499084473 24.0
6.3752186108 -0.181941628456116 25.0
0 0.0182210803031921 26.0
0 -0.11999899148941 27.0
0 -0.401204705238342 28.0
0 0.0525691509246826 29.0
0 0.0686611533164978 30.0
0 -0.148273587226868 31.0
0 -0.0219142436981201 32.0
0 0.012113094329834 33.0
0 0.228236138820648 34.0
0 -0.147387027740479 35.0
0 -0.189568400382996 36.0
0 0.0123714804649353 37.0
0 -0.0117714405059814 38.0
0 -0.342099785804749 39.0
0 0.264235377311707 40.0
0 -0.25076699256897 41.0
0 0.108111262321472 42.0
287.02234003 291.033630371094 43.0
0 0.610212206840515 44.0
0 -0.492079496383667 45.0
0 0.392001628875732 46.0
0 -0.317410588264465 47.0
0 -0.229389309883118 48.0
0 0.695543944835663 49.0
0.061394233851 -0.0914289355278015 50.0
2.4102527111 1.90193068981171 51.0
0 0.203393995761871 52.0
0 -0.45491099357605 53.0
0 0.253400146961212 54.0
0 0.207478940486908 55.0
0 0.208849430084229 56.0
0 -0.274804472923279 57.0
0 0.312216609716415 58.0
0 -0.879042029380798 59.0
0 0.266656160354614 60.0
0 0.0487344264984131 61.0
0 0.101008534431458 62.0
231.1341196 254.624176025391 0.0
0 -0.157001376152039 1.0
0 -0.108643114566803 2.0
0 0.353161215782166 3.0
0 -0.191025018692017 4.0
0 0.139042317867279 5.0
0 0.0899332165718079 6.0
0 -0.112663507461548 7.0
0 0.136530876159668 8.0
0 0.0564259886741638 9.0
0 -0.119468927383423 10.0
0 -0.208073973655701 11.0
0 -0.017759382724762 12.0
56.166767391 55.7287368774414 13.0
36.51423982 39.5688514709473 14.0
0 -0.0678156614303589 15.0
0 0.181577563285828 16.0
38.633350611 37.7945594787598 17.0
0 0.134466528892517 18.0
0 0.00725752115249634 19.0
0 0.0635724067687988 20.0
0 0.00918912887573242 21.0
0 -0.15692150592804 22.0
40.767065161 39.7668533325195 23.0
0 0.0347061157226562 24.0
0 0.436290055513382 25.0
0 -0.068726122379303 26.0
0 0.127561986446381 27.0
0 0.371939867734909 28.0
0 0.100647807121277 29.0
0 0.0392318964004517 30.0
0 -0.0455987453460693 31.0
0 0.0878369808197021 32.0
0 -0.201619148254395 33.0
0 0.196686029434204 34.0
0 0.202894628047943 35.0
0 0.161149263381958 36.0
0 -0.0647314786911011 37.0
0 -0.0423295497894287 38.0
0 -0.398139238357544 39.0
0 0.249009966850281 40.0
0 -0.0588762164115906 41.0
0 0.112366318702698 42.0
303.98146919 298.293304443359 43.0
0 0.303022384643555 44.0
0 -0.00485682487487793 45.0
0 0.137773513793945 46.0
0 -0.0627936124801636 47.0
0 0.330210149288177 48.0
0 0.423105955123901 49.0
0.084321224124 0.0613546371459961 50.0
2.135597622 1.88355350494385 51.0
0 -0.0707606077194214 52.0
0 -0.15723067522049 53.0
0 -0.187044024467468 54.0
0 0.31796133518219 55.0
0 0.127892732620239 56.0
0 0.267004549503326 57.0
0 0.226480722427368 58.0
0 0.232300698757172 59.0
0 -0.133192777633667 60.0
0 0.0924912691116333 61.0
0 0.223394870758057 62.0
213.99382493 254.10221862793 0.0
0 -0.133142590522766 1.0
0 -0.1824791431427 2.0
0 0.100026726722717 3.0
0 -0.395206928253174 4.0
0 -0.461853742599487 5.0
0 0.122200667858124 6.0
0 -0.263331174850464 7.0
0 0.405174344778061 8.0
0 -0.0989174842834473 9.0
0 0.148120999336243 10.0
0 -0.168806850910187 11.0
0 -0.121895670890808 12.0
57.947298952 55.8032722473145 13.0
37.894288682 37.1821212768555 14.0
0 0.00325179100036621 15.0
0 -0.188260793685913 16.0
29.35575339 33.8785781860352 17.0
0 0.0947593450546265 18.0
0 -0.0556464195251465 19.0
0 0.111721754074097 20.0
0 0.0130585432052612 21.0
0 0.197600603103638 22.0
36.63369869 35.2227172851562 23.0
0 0.109939694404602 24.0
0 0.541115760803223 25.0
0 -0.0540721416473389 26.0
0 0.180590867996216 27.0
0 0.248451232910156 28.0
0 0.0255889892578125 29.0
0 0.035808801651001 30.0
0 -0.152164220809937 31.0
0 -0.00432127714157104 32.0
0 -0.464197874069214 33.0
0 0.252966165542603 34.0
0 0.0551552772521973 35.0
0 0.535914540290833 36.0
0 -0.0366888046264648 37.0
0 0.166412115097046 38.0
0 -0.291865706443787 39.0
0 0.406985104084015 40.0
0 -0.0757262110710144 41.0
0 -0.130252599716187 42.0
301.13895711 295.370086669922 43.0
0 0.0717200040817261 44.0
0 0.0817478895187378 45.0
0 0.163760900497437 46.0
0 -0.476134538650513 47.0
0 0.287939131259918 48.0
0.60406110081 0.370136916637421 49.0
0.072416596431 0.101096153259277 50.0
1.7912737136 1.86655616760254 51.0
0 0.320461452007294 52.0
0 0.202084064483643 53.0
0 -0.135560870170593 54.0
0 0.264632046222687 55.0
0 0.252871751785278 56.0
0 0.157217919826508 57.0
0 0.30436384677887 58.0
0 0.453137218952179 59.0
0 0.0899088382720947 60.0
0 0.0274912714958191 61.0
0 0.254963874816895 62.0
256.23920654 255.634536743164 0.0
0 -0.287196636199951 1.0
0 -0.0988394021987915 2.0
0 -0.131370604038239 3.0
0 -0.0629881024360657 4.0
0 0.0647525191307068 5.0
0 0.169699966907501 6.0
0 0.25784170627594 7.0
0 0.212207734584808 8.0
0 0.00849384069442749 9.0
0 -0.120896637439728 10.0
0 -0.294178009033203 11.0
0 -0.155589580535889 12.0
62.003353772 61.4005470275879 13.0
27.473281213 36.3666496276855 14.0
0 -0.0891220569610596 15.0
0 0.0862659215927124 16.0
28.877149546 34.5507011413574 17.0
0 0.093156099319458 18.0
0 -0.000829935073852539 19.0
0 0.14715713262558 20.0
0 -0.058086633682251 21.0
0 0.333947062492371 22.0
40.674969883 40.3574867248535 23.0
0 0.0701473951339722 24.0
0 1.01211702823639 25.0
0 0.0841937065124512 26.0
0 0.0410594344139099 27.0
0 0.229772090911865 28.0
0 -0.0984066724777222 29.0
0 -0.0179663896560669 30.0
0 -0.132175326347351 31.0
0 -0.0333915948867798 32.0
0 -0.100024521350861 33.0
0 0.0866870284080505 34.0
0 -0.0982557535171509 35.0
0 -0.144655823707581 36.0
0 0.24781060218811 37.0
0 -0.282820343971252 38.0
0 -0.106800556182861 39.0
0 0.146001815795898 40.0
0 -0.136761069297791 41.0
0 0.132329940795898 42.0
301.38190128 296.924224853516 43.0
0 0.114729642868042 44.0
0 -0.0133506059646606 45.0
0 0.152698040008545 46.0
0 0.150789499282837 47.0
0 0.365362346172333 48.0
0 0.500828146934509 49.0
0.074101118079 0.0331289172172546 50.0
1.8553044493 1.8939768075943 51.0
0 -0.123279750347137 52.0
0 -0.0709211826324463 53.0
0 -0.0747455358505249 54.0
0 -0.104291021823883 55.0
0 0.068569540977478 56.0
0 0.0706332921981812 57.0
0 0.0433440208435059 58.0
0 0.291771829128265 59.0
0 -0.209844589233398 60.0
0 0.110924184322357 61.0
0 0.203584730625153 62.0
283.15171873 254.899276733398 0.0
0 -0.237220406532288 1.0
0 0.0145435929298401 2.0
0 0.424951136112213 3.0
0 -0.0804626941680908 4.0
0 0.0511553883552551 5.0
0 -0.266947150230408 6.0
0 0.365368396043777 7.0
0 0.244376361370087 8.0
0 -0.180209577083588 9.0
0 -0.017513632774353 10.0
0 -0.0404726266860962 11.0
0 -0.0334099531173706 12.0
51.179446404 50.8447875976562 13.0
43.690931356 40.2199058532715 14.0
0 -0.113767147064209 15.0
0 -0.0694222450256348 16.0
36.246890975 35.4499702453613 17.0
0 0.0820472240447998 18.0
0 -0.0104608535766602 19.0
0 0.0242321491241455 20.0
0 0.241096019744873 21.0
0 0.168855428695679 22.0
29.861079349 28.0974655151367 23.0
0 0.136252164840698 24.0
0 1.01745998859406 25.0
0 0.0790063738822937 26.0
0 -0.140221834182739 27.0
0 0.492701262235641 28.0
0 -0.0617973804473877 29.0
0 -0.0280007123947144 30.0
0 0.0524077415466309 31.0
0 0.120256781578064 32.0
0 0.0883007049560547 33.0
0 -0.0976158380508423 34.0
0 -0.0411243438720703 35.0
0 -0.231908559799194 36.0
0 -0.0711323618888855 37.0
0 0.338800311088562 38.0
0 -0.210127115249634 39.0
0 0.427300214767456 40.0
0 -0.181261599063873 41.0
0 0.162664532661438 42.0
286.19624746 297.620758056641 43.0
0 0.105927109718323 44.0
0 0.041867196559906 45.0
0 0.163341224193573 46.0
0 -0.224141240119934 47.0
0 0.175691306591034 48.0
0 0.49643337726593 49.0
0.076541066447 0.0237316489219666 50.0
2.2538349655 1.88474929332733 51.0
0 0.157289087772369 52.0
0 -0.147713601589203 53.0
0 0.142063975334167 54.0
0 0.00540810823440552 55.0
0 0.216325461864471 56.0
0 -0.127927601337433 57.0
0 0.212383270263672 58.0
0 0.174566149711609 59.0
0 -0.214585304260254 60.0
0 0.125285267829895 61.0
0 0.288315713405609 62.0
270.64992015 252.953491210938 0.0
0 -0.267997026443481 1.0
0 0.123279511928558 2.0
0 -0.0900292992591858 3.0
0 0.176242411136627 4.0
0 0.16315108537674 5.0
0 0.333627104759216 6.0
0 0.406504511833191 7.0
0 0.113408803939819 8.0
0 0.120434880256653 9.0
0 0.232126712799072 10.0
0 -0.215119242668152 11.0
0 -0.0680131912231445 12.0
68.909100336 69.6360473632812 13.0
28.212666377 24.6800708770752 14.0
0 -0.0928949117660522 15.0
0 0.184906005859375 16.0
36.712224368 36.9847106933594 17.0
0 0.152923583984375 18.0
0 0.00129026174545288 19.0
0 0.0428258180618286 20.0
0 0.0644852519035339 21.0
0 0.25540292263031 22.0
33.23723737 31.8690643310547 23.0
0 0.0757532119750977 24.0
0 0.903333187103271 25.0
0 0.0734280347824097 26.0
0 -0.116614699363708 27.0
0 0.152336955070496 28.0
0 -0.0613380670547485 29.0
0 0.0226192474365234 30.0
0 -0.154773473739624 31.0
0 -0.164782702922821 32.0
0 -0.127102553844452 33.0
0 0.21485447883606 34.0
0 -0.0137966871261597 35.0
0 -0.0675652027130127 36.0
0 0.241743743419647 37.0
0 -0.161589026451111 38.0
0 -0.208878040313721 39.0
0 0.295786499977112 40.0
0 -0.161846339702606 41.0
0 -0.011507511138916 42.0
297.31427227 298.205017089844 43.0
0 0.121273756027222 44.0
0 0.0937193036079407 45.0
0 0.0183885097503662 46.0
0 0.253988146781921 47.0
0 0.113239169120789 48.0
0.28523358932 0.523449301719666 49.0
0.076343919732 0.0149772763252258 50.0
1.7036314385 1.88262248039246 51.0
0 0.235166311264038 52.0
0 0.0865494608879089 53.0
0 -0.00422495603561401 54.0
0 0.239224433898926 55.0
0 0.244824111461639 56.0
0 0.10402238368988 57.0
0 0.428090214729309 58.0
0 0.313979595899582 59.0
0 -0.168128132820129 60.0
0 0.111942231655121 61.0
0 0.3039630651474 62.0
253.44551137 253.587814331055 0.0
0 -0.335185170173645 1.0
0 -0.1937575340271 2.0
0 0.347555637359619 3.0
0 -0.109856128692627 4.0
0 0.0671239495277405 5.0
0 -0.0366697311401367 6.0
0 0.0483927726745605 7.0
0 -0.122932076454163 8.0
0 0.244092762470245 9.0
0 0.161158680915833 10.0
0 -0.381500720977783 11.0
0 0.0198003053665161 12.0
72.732976148 72.2732162475586 13.0
11.426996931 23.0212707519531 14.0
0 -0.0530288219451904 15.0
0 -0.069048285484314 16.0
31.043602909 34.6561241149902 17.0
0 0.118502020835876 18.0
0 -0.0508134961128235 19.0
0 0.143491327762604 20.0
0 0.0153500437736511 21.0
0 0.261379241943359 22.0
35.518940528 34.6859817504883 23.0
0 0.0102929472923279 24.0
0 0.63879668712616 25.0
0 -0.128262162208557 26.0
0 -0.0271219611167908 27.0
0 0.284399032592773 28.0
0 -0.00187772512435913 29.0
0 0.0364954471588135 30.0
0 0.0253320932388306 31.0
0 -0.0420804023742676 32.0
0 -0.621847629547119 33.0
0 -1.48920178413391 34.0
0 0.299051851034164 35.0
0 0.24704897403717 36.0
0 -0.0494245290756226 37.0
0 0.312265127897263 38.0
0 -0.251000642776489 39.0
0 0.0429288148880005 40.0
0 -0.174146950244904 41.0
0 -0.0203558206558228 42.0
307.5000653 297.024749755859 43.0
0 0.137014389038086 44.0
0 0.097551167011261 45.0
0 0.194397211074829 46.0
0 0.064507782459259 47.0
0 0.442038804292679 48.0
0 0.365641176700592 49.0
0.096742985543 0.0836074352264404 50.0
2.0028143471 1.84149551391602 51.0
0 -0.0316641330718994 52.0
0 -0.314986705780029 53.0
0 -0.0649424195289612 54.0
0 0.0461407899856567 55.0
0 -0.102736115455627 56.0
0 0.219123959541321 57.0
0 0.306493937969208 58.0
0 -0.0128801465034485 59.0
0 -0.0238749384880066 60.0
0 0.0891821384429932 61.0
0 0.129318535327911 62.0
240.65918219 256.652496337891 0.0
0 -0.170019745826721 1.0
0 -0.3611980676651 2.0
0 0.276173770427704 3.0
0 -0.276019334793091 4.0
0 -0.117050468921661 5.0
0 -0.0751667022705078 6.0
0 -0.0251541137695312 7.0
0 -0.0266314744949341 8.0
0 -0.185910940170288 9.0
0 0.107402563095093 10.0
0 0.165070414543152 11.0
0 -0.0538734793663025 12.0
73.915328619 74.2290267944336 13.0
33.540645481 30.592414855957 14.0
0 -0.0465042591094971 15.0
0 0.250419795513153 16.0
31.743220704 34.7744483947754 17.0
0 0.103930234909058 18.0
0 -0.0318602323532104 19.0
0 0.140399932861328 20.0
0 -0.0291843414306641 21.0
0 0.12489265203476 22.0
37.486363232 38.6017723083496 23.0
0 0.0567598342895508 24.0
0 0.486460626125336 25.0
0 -0.0843460559844971 26.0
0 0.0258095264434814 27.0
0 0.184778094291687 28.0
0 0.0158563256263733 29.0
0 -0.0706613659858704 30.0
0 0.104645252227783 31.0
0 -0.0138758420944214 32.0
0 0.0651282072067261 33.0
0 0.281795918941498 34.0
0 0.058540403842926 35.0
0 0.124454855918884 36.0
0 0.184882998466492 37.0
0 0.0574536919593811 38.0
0 -0.222295165061951 39.0
0 0.43669992685318 40.0
0 -0.145819008350372 41.0
0 0.123058557510376 42.0
295.79157961 297.07470703125 43.0
0 0.112414836883545 44.0
0 0.0659117698669434 45.0
0 0.0825801491737366 46.0
0 0.113475322723389 47.0
0 -0.124576210975647 48.0
0 0.427215486764908 49.0
0.075284133411 0.0395409464836121 50.0
2.2420668996 1.83153414726257 51.0
0 0.198211908340454 52.0
0 0.302481144666672 53.0
0 0.0706768035888672 54.0
0 0.212146103382111 55.0
0 0.0153131484985352 56.0
0 0.13053297996521 57.0
0 0.31153279542923 58.0
0 0.059125542640686 59.0
0 -0.220110416412354 60.0
0 0.0139385461807251 61.0
0 0.122073829174042 62.0
236.21554762 252.095840454102 0.0
0 -0.263063788414001 1.0
0 -0.276375770568848 2.0
0 0.464063316583633 3.0
0 -0.230336666107178 4.0
0 -0.0268570780754089 5.0
0 -0.11025458574295 6.0
0 0.00266057252883911 7.0
0 0.175261974334717 8.0
0 -0.129411101341248 9.0
0 -0.0592320561408997 10.0
0 -0.0372432470321655 11.0
0 -0.126781046390533 12.0
67.704846618 67.9897537231445 13.0
28.673826264 29.3722095489502 14.0
0 0.0699852705001831 15.0
0 -0.15347957611084 16.0
33.144127623 32.780330657959 17.0
0 0.0949198007583618 18.0
0 0.0195732712745667 19.0
0 0.155642151832581 20.0
0 0.0191814303398132 21.0
0 0.10541296005249 22.0
40.277228746 39.3159866333008 23.0
0 0.0935450792312622 24.0
0 0.596021950244904 25.0
0 0.0193659067153931 26.0
0 -0.0840830206871033 27.0
0 0.194600939750671 28.0
0 -0.109032869338989 29.0
0 0.0287841558456421 30.0
0 -0.140782594680786 31.0
0 0.0917665362358093 32.0
0 0.0708699822425842 33.0
0 0.382990121841431 34.0
0 0.29700842499733 35.0
0 0.21697986125946 36.0
0 -0.0559363961219788 37.0
0 0.0199965238571167 38.0
0 -0.149138450622559 39.0
0 0.483316451311111 40.0
0 -0.149661123752594 41.0
0 0.119029879570007 42.0
298.92320284 296.661773681641 43.0
0 0.136407971382141 44.0
0 0.0393010377883911 45.0
0 -0.0138002634048462 46.0
0 0.183905243873596 47.0
0 -0.0386338233947754 48.0
0 0.394537568092346 49.0
0.074537723492 0.046097457408905 50.0
2.211995118 1.85796499252319 51.0
0 0.29710590839386 52.0
0 -0.0699543356895447 53.0
0 0.108205795288086 54.0
0 0.266375958919525 55.0
0 0.182366847991943 56.0
0 0.332087516784668 57.0
0 0.180245041847229 58.0
0 -0.0443947315216064 59.0
0 -0.0751795768737793 60.0
0 0.0127387046813965 61.0
0 0.119228839874268 62.0
226.21907454 252.731643676758 0.0
0 -0.149939179420471 1.0
0 -0.277472019195557 2.0
0 0.397379875183105 3.0
0 -0.144965887069702 4.0
0 -0.2965247631073 5.0
0 0.0995661020278931 6.0
0 -0.0505701303482056 7.0
0 0.0105052590370178 8.0
0 -0.0250704288482666 9.0
0 0.14178478717804 10.0
0 -0.22050666809082 11.0
0 -0.0904800891876221 12.0
65.1229249 64.6483306884766 13.0
33.023361695 29.4577331542969 14.0
0 0.0381407737731934 15.0
0 0.141659736633301 16.0
38.797380035 34.291576385498 17.0
0 0.104418635368347 18.0
0 0.00420618057250977 19.0
0 0.126778960227966 20.0
0 0.0578426718711853 21.0
0 0.00961971282958984 22.0
37.532445479 38.726245880127 23.0
0 0.103617310523987 24.0
0 1.35477530956268 25.0
0 -0.111061632633209 26.0
0 0.0161837935447693 27.0
0 0.303892999887466 28.0
0 0.0102025270462036 29.0
0 0.0387973189353943 30.0
0 0.235765933990479 31.0
0 0.0182623863220215 32.0
0 -0.182444274425507 33.0
0 -0.229428291320801 34.0
0 0.307080805301666 35.0
0 0.0415383577346802 36.0
0 0.183780133724213 37.0
0 -0.00339454412460327 38.0
0 -0.297220468521118 39.0
0 0.408081024885178 40.0
0 -0.0866854190826416 41.0
0 -0.139967203140259 42.0
307.42168115 296.847686767578 43.0
0 0.00608843564987183 44.0
0 0.156845390796661 45.0
0 0.149222493171692 46.0
0 0.00955909490585327 47.0
0 0.377098977565765 48.0
0 0.372599542140961 49.0
0.093473233093 0.0652992129325867 50.0
2.0775558148 1.89629316329956 51.0
0 0.142693221569061 52.0
0 -0.038291335105896 53.0
0 0.0768007636070251 54.0
0 0.0225799083709717 55.0
0 0.0937361717224121 56.0
0 0.258595108985901 57.0
0 0.275617659091949 58.0
0 0.250382125377655 59.0
0 -0.21649432182312 60.0
0 0.0789715647697449 61.0
0 0.286906778812408 62.0
293.42562446 251.474716186523 0.0
0 0.000921010971069336 1.0
0 0.133815765380859 2.0
0 -0.209023237228394 3.0
0 -0.312401413917542 4.0
0 -0.0416945219039917 5.0
0 0.405068516731262 6.0
0 0.231520354747772 7.0
0 -0.241832971572876 8.0
0 0.0223759412765503 9.0
0 0.493572324514389 10.0
0 -0.0062718391418457 11.0
0 -0.145365357398987 12.0
60.797147896 60.587474822998 13.0
43.438183544 33.9410018920898 14.0
0 -0.105979561805725 15.0
0 0.155985295772552 16.0
31.033777536 34.9737091064453 17.0
0 0.156952857971191 18.0
0 -0.0540835857391357 19.0
0 0.248727560043335 20.0
0 -0.105488240718842 21.0
0 1.04835605621338 22.0
40.679160118 39.7832069396973 23.0
0 0.102234125137329 24.0
0 0.942192196846008 25.0
0 0.201081454753876 26.0
0 0.243485331535339 27.0
0 0.288372337818146 28.0
0 -0.103670358657837 29.0
0 0.166558504104614 30.0
0 0.034553050994873 31.0
0 -0.0573689341545105 32.0
0 -0.243507623672485 33.0
0 0.124884009361267 34.0
0 0.00743198394775391 35.0
0 -0.111268639564514 36.0
0 0.110058009624481 37.0
0 0.174530267715454 38.0
0 -0.358542561531067 39.0
0 0.493885159492493 40.0
0 -0.1137455701828 41.0
0 0.0976103544235229 42.0
286.04620846 296.444183349609 43.0
0 0.23622465133667 44.0
0 -1.48471593856812 45.0
0 0.0312062501907349 46.0
0 0.227498114109039 47.0
0 -0.110457181930542 48.0
0 0.319750010967255 49.0
0.07243459247 0.072658896446228 50.0
1.9127589685 1.81709289550781 51.0
0 0.2786865234375 52.0
0 0.230214059352875 53.0
0 0.171628654003143 54.0
0 0.308359801769257 55.0
0 0.150065720081329 56.0
0 0.0941745042800903 57.0
0 -0.312222719192505 58.0
0 0.312426894903183 59.0
0 -0.248766303062439 60.0
0 0.279535830020905 61.0
0 0.385620713233948 62.0
224.05559464 251.601501464844 0.0
0 0.168718218803406 1.0
0 -0.0780073404312134 2.0
0 0.0353752970695496 3.0
0 -0.29069709777832 4.0
0 -0.116064310073853 5.0
0 -0.225707173347473 6.0
0 -0.259153723716736 7.0
0 0.217005610466003 8.0
0 0.13479083776474 9.0
0 0.248457014560699 10.0
0 -0.220326066017151 11.0
0 -0.057249903678894 12.0
61.094894998 61.1616058349609 13.0
36.132673421 33.5955772399902 14.0
0 -0.0831277370452881 15.0
0 0.023706316947937 16.0
31.305123228 34.2645568847656 17.0
0 0.0924545526504517 18.0
0 -0.0464152097702026 19.0
0 0.0995522141456604 20.0
0 0.0634322166442871 21.0
0 0.31520003080368 22.0
32.831461818 30.8953323364258 23.0
0 0.0585706233978271 24.0
0 0.719082593917847 25.0
0 -0.0677748322486877 26.0
0 0.00682282447814941 27.0
0 0.475181102752686 28.0
0 0.0433481335639954 29.0
0 0.148505330085754 30.0
0 -0.0794363021850586 31.0
0 -0.0209476947784424 32.0
0 -0.209907054901123 33.0
0 0.234656393527985 34.0
0 -0.0828788876533508 35.0
0 0.0950880646705627 36.0
0 0.181748509407043 37.0
0 -0.0204803347587585 38.0
0 -0.265851616859436 39.0
0 0.465157240629196 40.0
0 -0.0891931653022766 41.0
0 0.229228854179382 42.0
302.45417644 294.401763916016 43.0
0 0.0174496173858643 44.0
0 0.237431228160858 45.0
0 0.164868056774139 46.0
0 0.102641105651855 47.0
0 0.110644340515137 48.0
0 0.494235247373581 49.0
0.082952498376 0.0298922061920166 50.0
2.0574994956 1.83164548873901 51.0
0 0.0133238434791565 52.0
0 0.106796860694885 53.0
0 0.037248969078064 54.0
0 -0.0205827951431274 55.0
0 -0.00984138250350952 56.0
0 0.0322884917259216 57.0
0 -0.0299152731895447 58.0
0 0.257117509841919 59.0
0 0.150974810123444 60.0
0 0.140979290008545 61.0
0 0.249781608581543 62.0
248.39687961 253.824417114258 0.0
0 -0.249345302581787 1.0
0 -0.128839015960693 2.0
0 0.0863265991210938 3.0
0 -0.367284297943115 4.0
0 0.0625694394111633 5.0
0 0.15274316072464 6.0
0 0.171047270298004 7.0
0 0.385677814483643 8.0
0 0.0795937180519104 9.0
0 -0.111746191978455 10.0
0 -0.0777493715286255 11.0
0 -0.193678736686707 12.0
62.275924241 62.2228393554688 13.0
32.766061485 36.4642906188965 14.0
0 -0.0264250040054321 15.0
0 0.00982320308685303 16.0
28.213173378 33.5374412536621 17.0
0 0.0975830554962158 18.0
0 0.0244096517562866 19.0
0 0.106915414333344 20.0
0 0.0898629426956177 21.0
0 0.118206143379211 22.0
35.840460596 33.7245788574219 23.0
0 0.0412791967391968 24.0
0 0.435327470302582 25.0
0 0.107420682907104 26.0
0 -0.069242000579834 27.0
0 -0.0910460948944092 28.0
0 -0.104428589344025 29.0
0 0.0814884901046753 30.0
0 0.228479743003845 31.0
0 0.167183637619019 32.0
0 0.0127032399177551 33.0
0 0.285992085933685 34.0
0 -0.028611958026886 35.0
0 0.395682871341705 36.0
0 0.00392347574234009 37.0
0 0.0945841073989868 38.0
0 -0.205477356910706 39.0
0 0.524160623550415 40.0
0 -0.109703779220581 41.0
0 0.120216965675354 42.0
294.44592481 296.429260253906 43.0
0 -0.0335400700569153 44.0
0 0.160591244697571 45.0
0 0.101487755775452 46.0
0 -0.0326510071754456 47.0
0 0.184137642383575 48.0
0 0.436637997627258 49.0
0.10029351321 0.0149503946304321 50.0
2.1883315265 1.8810179233551 51.0
0 0.0532186627388 52.0
0 -0.0958261489868164 53.0
0 0.134342730045319 54.0
0 0.165185511112213 55.0
0 -0.0492600202560425 56.0
0 0.4217669069767 57.0
0 0.132146179676056 58.0
0 -0.149473667144775 59.0
0 0.00675857067108154 60.0
0 0.043312668800354 61.0
0 0.0861903429031372 62.0
265.50025976 254.582717895508 0.0
0 -0.282534837722778 1.0
0 0.047288715839386 2.0
0 0.142853975296021 3.0
0 -0.579036474227905 4.0
0 -0.266555786132812 5.0
0 -0.0719554424285889 6.0
0 0.535965740680695 7.0
0 0.617048740386963 8.0
0 1.7070152759552 9.0
0 0.109936594963074 10.0
0 0.070589542388916 11.0
0 -0.289837002754211 12.0
68.813804404 69.0811996459961 13.0
42.034308852 29.9987010955811 14.0
0 0.137313723564148 15.0
0 -0.0673507452011108 16.0
42.130653 39.5636215209961 17.0
0 0.271841883659363 18.0
0 -0.116235554218292 19.0
0 0.0722551941871643 20.0
0 -0.0221014022827148 21.0
0 0.241803407669067 22.0
31.840193682 29.755651473999 23.0
0 0.0156557559967041 24.0
0 -0.36901068687439 25.0
0 0.0171275734901428 26.0
0 -0.253358602523804 27.0
0 0.523612082004547 28.0
0 0.113804221153259 29.0
0 0.201213657855988 30.0
0 -0.0940674543380737 31.0
0 0.0383266806602478 32.0
0 -0.222100615501404 33.0
0 0.287766516208649 34.0
0 0.635820865631104 35.0
0 0.511104464530945 36.0
0 0.00920969247817993 37.0
0 -0.061892032623291 38.0
0 -0.10155725479126 39.0
0 0.439932584762573 40.0
0 -0.0558444857597351 41.0
0 0.0588607788085938 42.0
306.81070046 297.828857421875 43.0
0 0.0610061883926392 44.0
0 0.0285667777061462 45.0
0 0.211936354637146 46.0
0 -0.179395437240601 47.0
0 0.327885121107101 48.0
0 0.477959513664246 49.0
0.070712532429 0.0124750733375549 50.0
1.5085296887 1.90528428554535 51.0
0 -0.0797765254974365 52.0
0 -0.128798604011536 53.0
0 -0.0301762223243713 54.0
0 -0.00516504049301147 55.0
0 0.144766747951508 56.0
0 -0.00199645757675171 57.0
0 -0.120662748813629 58.0
0 0.162091672420502 59.0
0 -0.212118029594421 60.0
0 0.270641088485718 61.0
0 0.411846697330475 62.0
241.58928825 252.337539672852 0.0
0 0.293964385986328 1.0
0 -0.0297067165374756 2.0
0 -0.045998215675354 3.0
0 -0.263789892196655 4.0
0 0.0155149102210999 5.0
0 -0.286959648132324 6.0
0 -0.328786969184875 7.0
0 0.180074274539948 8.0
0 -0.498653054237366 9.0
0 -0.244579076766968 10.0
0 -0.168280839920044 11.0
0 -0.142789900302887 12.0
53.502340535 54.1411170959473 13.0
39.818146014 38.8610610961914 14.0
0 0.0259325504302979 15.0
0 0.0314825177192688 16.0
27.810410691 34.2149467468262 17.0
0 0.104561567306519 18.0
0 -0.0389816164970398 19.0
0 0.0822383165359497 20.0
0 -0.137528598308563 21.0
0 1.22325050830841 22.0
40.870808402 40.4107933044434 23.0
0 0.047234058380127 24.0
0 0.858903467655182 25.0
0 -0.148889303207397 26.0
0 0.119767129421234 27.0
0 0.241585731506348 28.0
0 0.0152041912078857 29.0
0 0.123717784881592 30.0
0 0.0111783742904663 31.0
0 0.0892667174339294 32.0
0 -0.010217010974884 33.0
0 0.307633727788925 34.0
0 -0.148212015628815 35.0
0 0.3134386241436 36.0
0 0.0115607976913452 37.0
0 0.138596773147583 38.0
0 -0.182915091514587 39.0
0 0.323346972465515 40.0
0 -0.0694312453269958 41.0
0 0.113398551940918 42.0
289.7219725 296.247863769531 43.0
0 0.153807878494263 44.0
0 -8.6665153503418e-05 45.0
0 0.154613018035889 46.0
0 0.0343284010887146 47.0
0 0.403406649827957 48.0
1.023356564 0.395966589450836 49.0
0.083489362205 0.0207431316375732 50.0
1.9078581876 1.90067076683044 51.0
0 -0.0320874452590942 52.0
0 -0.0406420826911926 53.0
0 -0.111316502094269 54.0
0 -0.000368654727935791 55.0
0 0.106681168079376 56.0
0 0.167191743850708 57.0
0 0.0202018618583679 58.0
0 0.198463499546051 59.0
0 -0.108645975589752 60.0
0 0.139613032341003 61.0
0 0.278971195220947 62.0
282.60905189 249.967636108398 0.0
0 -0.592554926872253 1.0
0 0.122621059417725 2.0
0 -0.536667823791504 3.0
0 -0.221749305725098 4.0
0 0.364025712013245 5.0
0 -0.0434533357620239 6.0
0 0.313740372657776 7.0
0 0.167095959186554 8.0
0 0.385787129402161 9.0
0 -0.722733497619629 10.0
0 -0.0239623785018921 11.0
0 -0.225581288337708 12.0
73.98314465 72.8586730957031 13.0
38.991697736 24.0474662780762 14.0
0 -0.0381871461868286 15.0
0 -0.134712338447571 16.0
37.868908354 35.5263404846191 17.0
0 0.225900292396545 18.0
0 -0.0781481862068176 19.0
0 0.0862339735031128 20.0
0 0.127936005592346 21.0
0 0.377974569797516 22.0
33.287120718 32.3991165161133 23.0
0 0.139174699783325 24.0
0 1.12933564186096 25.0
0 -0.0860766172409058 26.0
0 -0.214967250823975 27.0
0 1.01156640052795 28.0
0 0.0454094409942627 29.0
0 0.0506988763809204 30.0
0 0.871731996536255 31.0
0 -0.0586427450180054 32.0
0 0.0474990010261536 33.0
0 -0.132830798625946 34.0
0 0.24016398191452 35.0
0 -0.30055832862854 36.0
0 0.0946258902549744 37.0
0 0.358353227376938 38.0
0 -0.274610638618469 39.0
0 -0.186162710189819 40.0
0 -0.223575830459595 41.0
0 0.0826131105422974 42.0
304.56667036 296.859771728516 43.0
0 0.143440544605255 44.0
0 0.149259328842163 45.0
0 0.141707420349121 46.0
0 0.343198299407959 47.0
0 0.251538038253784 48.0
0 0.416871517896652 49.0
0.08638238824 0.0605653524398804 50.0
1.6765776913 1.81429052352905 51.0
0 -0.213736653327942 52.0
0 0.0618478655815125 53.0
0 -0.211782693862915 54.0
0 -0.0775730013847351 55.0
0 0.0318812727928162 56.0
0 -0.00956249237060547 57.0
0 0.148734271526337 58.0
0 0.00051647424697876 59.0
0 -0.159420490264893 60.0
0 -0.115044593811035 61.0
0 -0.120787918567657 62.0
223.40470918 254.443420410156 0.0
0 0.113799691200256 1.0
0 -0.113986611366272 2.0
0 -0.302048921585083 3.0
0 -0.528848528862 4.0
0 -0.217144012451172 5.0
0 0.147381961345673 6.0
0 0.0359853506088257 7.0
0 0.387884050607681 8.0
0 -0.196002244949341 9.0
0 0.22847580909729 10.0
0 -0.198586821556091 11.0
0 -0.185361981391907 12.0
71.541785175 71.1537017822266 13.0
11.349998831 29.629093170166 14.0
0 0.0123593807220459 15.0
0 0.0526238679885864 16.0
32.923070611 31.7048034667969 17.0
0 0.0215404629707336 18.0
0 -0.0521788597106934 19.0
0 0.650270342826843 20.0
0 -0.266871571540833 21.0
0 3.12022185325623 22.0
44.093301122 38.3435745239258 23.0
0 0.000633835792541504 24.0
0 1.27892112731934 25.0
0 -0.105229139328003 26.0
0 0.144198894500732 27.0
0 0.462732017040253 28.0
0 -0.118841886520386 29.0
0 0.103789746761322 30.0
0 0.174727201461792 31.0
0 0.220152497291565 32.0
0 0.108708024024963 33.0
0 0.0136992931365967 34.0
0 0.290399491786957 35.0
0 -0.208148002624512 36.0
0 0.129007339477539 37.0
0 0.366752594709396 38.0
0 -0.298070549964905 39.0
0 -0.227537989616394 40.0
0 -0.150994300842285 41.0
0 -0.179938316345215 42.0
270.51918219 296.564910888672 43.0
0 -0.034969687461853 44.0
0 0.285759449005127 45.0
0 0.354227662086487 46.0
0 -0.62819766998291 47.0
0 0.326002925634384 48.0
1.3709660702 0.579674601554871 49.0
0.06752003309 0.0198553204536438 50.0
1.5543782603 1.91282522678375 51.0
0 0.241388022899628 52.0
0 0.715879380702972 53.0
0 -0.153622448444366 54.0
0 0.244595468044281 55.0
0 0.258563280105591 56.0
0 0.229797661304474 57.0
0 0.569368004798889 58.0
0 0.189781308174133 59.0
0 -0.114740133285522 60.0
0 -0.0594373941421509 61.0
0 -0.120226085186005 62.0
285.48498988 253.370239257812 0.0
0 -0.277214169502258 1.0
0 0.0473906993865967 2.0
0 0.173434376716614 3.0
0 0.0212339162826538 4.0
0 -0.0390639305114746 5.0
0 -0.00694739818572998 6.0
0 0.331428408622742 7.0
0 0.0413574576377869 8.0
0 0.463540554046631 9.0
0 -0.5265873670578 10.0
0 0.0390465259552002 11.0
0 -0.0497462749481201 12.0
66.04390803 66.1871566772461 13.0
46.092653103 32.3851013183594 14.0
0 -0.0351636409759521 15.0
0 0.297899186611176 16.0
27.74666352 32.557559967041 17.0
0 0.0530985593795776 18.0
0 -0.0446987152099609 19.0
0 0.219281673431396 20.0
0 0.00742650032043457 21.0
0 0.192692637443542 22.0
33.760387342 33.1292457580566 23.0
0 0.0442128777503967 24.0
0 0.392464071512222 25.0
0 -0.0501649379730225 26.0
0 -0.0853601098060608 27.0
0 0.490271210670471 28.0
0 0.0211762189865112 29.0
0 -0.0118845701217651 30.0
0 0.139637589454651 31.0
0 0.163709282875061 32.0
0 0.0202637910842896 33.0
0 0.282129943370819 34.0
0 0.0397211313247681 35.0
0 0.193086266517639 36.0
0 -0.00420606136322021 37.0
0 0.154549419879913 38.0
0 -0.316651105880737 39.0
0 0.514987647533417 40.0
0 -0.202306270599365 41.0
0 -0.00662946701049805 42.0
296.2529259 295.497985839844 43.0
0 0.157540321350098 44.0
0 0.192200899124146 45.0
0 0.189206004142761 46.0
0 -0.658013820648193 47.0
0 0.384012758731842 48.0
0.29221340826 0.526416599750519 49.0
0.075475703301 0.0110854506492615 50.0
2.2286424598 1.88810443878174 51.0
0 0.0514962077140808 52.0
0 -0.327194929122925 53.0
0 0.194298267364502 54.0
0 0.01459801197052 55.0
0 -0.121008455753326 56.0
0 -0.0415095686912537 57.0
0 -0.163033664226532 58.0
0 0.362577050924301 59.0
0 -0.140315592288971 60.0
0 0.21850848197937 61.0
0 0.316654145717621 62.0
271.00144083 244.548294067383 0.0
0 -0.33286714553833 1.0
0 0.144489705562592 2.0
0 0.386663556098938 3.0
0 0.124832034111023 4.0
0 0.402952611446381 5.0
0 -0.13264787197113 6.0
0 0.259465038776398 7.0
0 -0.0197954773902893 8.0
0 0.230117380619049 9.0
0 0.00291430950164795 10.0
0 -0.315127491950989 11.0
0 -0.175479888916016 12.0
54.786238493 53.5191383361816 13.0
40.357366933 35.8876304626465 14.0
0 -0.104163289070129 15.0
0 -0.249879002571106 16.0
29.386375666 33.9380340576172 17.0
0 0.0563661456108093 18.0
0 0.00285744667053223 19.0
0 0.0302099585533142 20.0
0 0.185213029384613 21.0
0 -0.0391008853912354 22.0
26.271691248 26.2818737030029 23.0
0 0.0753917098045349 24.0
13.919405726 -0.00974124670028687 25.0
0 -0.368958473205566 26.0
0 -0.104219675064087 27.0
0 0.404951632022858 28.0
0 0.0373557806015015 29.0
0 0.057196319103241 30.0
0 -0.013239860534668 31.0
0 0.00157332420349121 32.0
0 0.0562474727630615 33.0
0 0.513622879981995 34.0
0 0.269241631031036 35.0
0 0.390695571899414 36.0
0 -0.0768468379974365 37.0
0 0.100837051868439 38.0
0 -0.341216564178467 39.0
0 0.460717678070068 40.0
0 -0.166674613952637 41.0
0 -0.0119831562042236 42.0
292.036164 298.178588867188 43.0
0 0.00619757175445557 44.0
0 0.145371794700623 45.0
0 0.0301058888435364 46.0
0 0.00961565971374512 47.0
0 0.325588524341583 48.0
0.27722198621 0.537333071231842 49.0
0.073426801253 0.0446165800094604 50.0
1.8559008831 1.87769258022308 51.0
0 -0.145185589790344 52.0
0 -0.300117135047913 53.0
0 -0.0577341318130493 54.0
0 0.215763747692108 55.0
0 0.228991985321045 56.0
0 0.576921880245209 57.0
0 0.52017205953598 58.0
0 0.150197803974152 59.0
0 0.195205271244049 60.0
0 0.0821043252944946 61.0
0 0.311605036258698 62.0
214.52766973 252.949508666992 0.0
0 0.135636687278748 1.0
0 -0.208778858184814 2.0
0 -0.321535587310791 3.0
0 -0.447201013565063 4.0
0 -0.14582771062851 5.0
0 -0.208550810813904 6.0
0 -0.281776189804077 7.0
0 0.393832802772522 8.0
0 0.471195578575134 9.0
0 0.384716212749481 10.0
0 -0.0431922078132629 11.0
0 -0.0375809669494629 12.0
57.515316571 58.493579864502 13.0
40.714787251 38.4988861083984 14.0
0 -0.0869487524032593 15.0
0 0.0937657356262207 16.0
38.779347539 36.9602241516113 17.0
0 0.193622648715973 18.0
0 -0.00930392742156982 19.0
0 0.0374714732170105 20.0
0 -0.0148664712905884 21.0
0 0.278872430324554 22.0
30.724533826 31.7727756500244 23.0
0 0.0756727457046509 24.0
0 0.95674192905426 25.0
0 -0.083362340927124 26.0
0 -0.0611900091171265 27.0
0 0.440824091434479 28.0
0 -0.0628185272216797 29.0
0 0.029066264629364 30.0
0 -0.12360692024231 31.0
0 -0.0437465310096741 32.0
0 -0.508880138397217 33.0
0 0.26306289434433 34.0
0 0.0293482542037964 35.0
0 0.052386999130249 36.0
0 0.250923156738281 37.0
0 -0.131133735179901 38.0
0 -0.143972277641296 39.0
0 0.556415975093842 40.0
0 -0.0660434365272522 41.0
0 0.0871925354003906 42.0
301.68348717 295.951293945312 43.0
0 0.127269685268402 44.0
0 0.0843432545661926 45.0
0 0.156609058380127 46.0
0 -0.360363960266113 47.0
0 0.39146962761879 48.0
1.3204330239 0.477097392082214 49.0
0.099268470486 0.0377369523048401 50.0
2.0982922507 1.87984919548035 51.0
0 -0.121793568134308 52.0
0 -0.34484076499939 53.0
0 0.0754988193511963 54.0
0 0.285457789897919 55.0
0 0.218127608299255 56.0
0 0.437168538570404 57.0
0 0.155693650245667 58.0
0 0.273255348205566 59.0
0 0.0213316082954407 60.0
0 0.0555878877639771 61.0
0 0.200902998447418 62.0
228.63026125 253.118896484375 0.0
0 0.399523973464966 1.0
0 -0.272528409957886 2.0
0 -0.084962010383606 3.0
0 -0.539681911468506 4.0
0 -0.0685975551605225 5.0
0 -0.259802579879761 6.0
0 -0.158741235733032 7.0
0 -0.711955428123474 8.0
0 0.273102700710297 9.0
0 1.26644444465637 10.0
0 -0.314324259757996 11.0
0 -0.00279879570007324 12.0
57.945354588 56.2387580871582 13.0
23.781682019 38.1840515136719 14.0
0 -0.0771712064743042 15.0
0 0.12480354309082 16.0
35.802414904 36.0392723083496 17.0
0 0.0923300981521606 18.0
0 -0.0458660125732422 19.0
0 0.0498901605606079 20.0
0 0.168783068656921 21.0
0 0.173644244670868 22.0
31.770109286 29.8366832733154 23.0
0 0.139835715293884 24.0
0 1.07660317420959 25.0
0 0.0653195381164551 26.0
0 -0.221136689186096 27.0
0 0.208742618560791 28.0
0 -0.147874236106873 29.0
0 0.0509529113769531 30.0
0 -0.0662953853607178 31.0
0 -0.106677234172821 32.0
0 -0.139640808105469 33.0
0 0.319056510925293 34.0
0 -0.10259222984314 35.0
0 -0.128192782402039 36.0
0 0.196613788604736 37.0
0 -0.473814010620117 38.0
0 -0.114599704742432 39.0
0 0.470025807619095 40.0
0 -0.178664684295654 41.0
0 0.0425459146499634 42.0
292.61727791 296.326354980469 43.0
0 0.184774875640869 44.0
0 -0.0140877366065979 45.0
0 0.137358605861664 46.0
0 0.00401020050048828 47.0
0 0.364820003509521 48.0
0 0.380830734968185 49.0
0.094657686223 0.0383085608482361 50.0
1.4716266321 1.9290759563446 51.0
0 -0.072100818157196 52.0
0 -0.177047312259674 53.0
0 -0.139958441257477 54.0
0 0.261063575744629 55.0
0 0.313571482896805 56.0
0 0.044269323348999 57.0
0 0.278277039527893 58.0
0 0.196818590164185 59.0
0 -0.132116913795471 60.0
0 0.0876476168632507 61.0
0 0.27274763584137 62.0
267.86063716 252.248153686523 0.0
0 -0.0413949489593506 1.0
0 0.010196328163147 2.0
0 0.39159083366394 3.0
0 -0.154266357421875 4.0
0 -0.302191734313965 5.0
0 -0.00363993644714355 6.0
0 1.05321645736694 7.0
0 -0.605891466140747 8.0
0 -0.710162878036499 9.0
0 1.72470760345459 10.0
0 0.475560963153839 11.0
0 -0.188766300678253 12.0
49.680704203 49.5969772338867 13.0
31.138162122 38.8475379943848 14.0
0 0.0576014518737793 15.0
0 0.0939421653747559 16.0
38.705343062 34.950252532959 17.0
0 0.0934008359909058 18.0
0 -0.109782516956329 19.0
0 0.142861723899841 20.0
0 0.219270944595337 21.0
0 -0.149207830429077 22.0
23.907913078 26.3942356109619 23.0
0 0.133626580238342 24.0
0 0.603253126144409 25.0
0 -0.145703971385956 26.0
0 -0.112452983856201 27.0
0 0.435269445180893 28.0
0 0.0532225370407104 29.0
0 0.122299134731293 30.0
0 0.0763802528381348 31.0
0 0.0912153720855713 32.0
0 -0.579882860183716 33.0
0 -0.818923950195312 34.0
0 0.17481517791748 35.0
0 0.0248869657516479 36.0
0 0.143633723258972 37.0
0 0.0479288101196289 38.0
0 -0.302767753601074 39.0
0 0.439140021800995 40.0
0 -0.256754636764526 41.0
0 0.339341282844543 42.0
303.39003554 295.704772949219 43.0
0 0.0919420719146729 44.0
0 0.118730485439301 45.0
0 0.02629554271698 46.0
0 -0.272844076156616 47.0
0 0.246258854866028 48.0
0 0.485559165477753 49.0
0.080973326939 0.0663129687309265 50.0
1.823735673 1.92916703224182 51.0
0 -0.289116621017456 52.0
0 -0.144785761833191 53.0
0 -0.036138117313385 54.0
0 0.126562654972076 55.0
0 0.0671660900115967 56.0
0 0.885514736175537 57.0
0 0.590782463550568 58.0
0 -0.466281175613403 59.0
0 0.48035341501236 60.0
0 0.0635716915130615 61.0
0 0.117265641689301 62.0
233.66574661 252.265350341797 0.0
0 -0.241929888725281 1.0
0 -0.0832250118255615 2.0
0 0.280987620353699 3.0
0 -0.407496929168701 4.0
0 -0.0501236319541931 5.0
0 0.069891631603241 6.0
0 0.0575845837593079 7.0
0 0.128614485263824 8.0
0 -0.329420566558838 9.0
0 -0.169499099254608 10.0
0 -0.10944652557373 11.0
0 -0.0675837397575378 12.0
55.318754283 53.7463264465332 13.0
53.099984013 44.5336151123047 14.0
0 -0.122868776321411 15.0
0 0.00761890411376953 16.0
39.810743542 39.0650444030762 17.0
0 0.188575387001038 18.0
0 -0.0203248262405396 19.0
0 0.0179474353790283 20.0
0 0.0995671153068542 21.0
0 0.247537314891815 22.0
31.926482239 30.1516456604004 23.0
0 0.0486650466918945 24.0
0 0.426119804382324 25.0
0 -0.00280487537384033 26.0
0 -0.0808959603309631 27.0
0 0.00814259052276611 28.0
0 -0.0284410715103149 29.0
0 0.0830258727073669 30.0
0 -0.148034691810608 31.0
0 -0.0959979295730591 32.0
0 0.0840489864349365 33.0
0 0.0523813962936401 34.0
0 0.221842408180237 35.0
0 0.0839040279388428 36.0
0 -0.0866556167602539 37.0
0 0.196402728557587 38.0
0 -0.156474471092224 39.0
0 0.0376461744308472 40.0
0 -0.227207779884338 41.0
0 0.00164163112640381 42.0
293.57096095 295.478210449219 43.0
0 0.123011946678162 44.0
0 0.211262106895447 45.0
0 0.140769422054291 46.0
0 0.0363117456436157 47.0
0 -0.0414725542068481 48.0
0 0.395402014255524 49.0
0.078673569841 0.0122370719909668 50.0
2.0643049163 1.91542792320251 51.0
0 0.251399397850037 52.0
0 -0.161540269851685 53.0
0 0.136191010475159 54.0
0 0.140417337417603 55.0
0 -0.0568438768386841 56.0
0 0.259340405464172 57.0
0 0.158137798309326 58.0
0 0.187220215797424 59.0
0 -0.244525671005249 60.0
0 0.0777624845504761 61.0
0 0.20347648859024 62.0
279.89236076 248.165985107422 0.0
0 -0.279931545257568 1.0
0 -0.040648341178894 2.0
0 0.396391838788986 3.0
0 -0.0996113419532776 4.0
0 0.550319671630859 5.0
0 0.123323082923889 6.0
0 0.0634945631027222 7.0
0 -0.285032033920288 8.0
0 0.437937676906586 9.0
0 -0.498426198959351 10.0
0 0.0979032516479492 11.0
0 -0.181289315223694 12.0
47.464357266 48.7503128051758 13.0
55.771398927 45.9572296142578 14.0
0 0.0562148094177246 15.0
0 0.330408751964569 16.0
30.956313966 35.7747383117676 17.0
0 0.0240710973739624 18.0
0 -0.0465056300163269 19.0
0 0.0806500315666199 20.0
0 0.13114470243454 21.0
0 0.0116041898727417 22.0
33.074831409 30.6417427062988 23.0
0 0.0550736784934998 24.0
2.7247727146 0.638694405555725 25.0
0 0.0568875074386597 26.0
0 -0.0238905549049377 27.0
0 0.143224537372589 28.0
0 0.0577462315559387 29.0
0 -0.0229249596595764 30.0
0 0.28528094291687 31.0
0 0.200154960155487 32.0
0 -0.326794624328613 33.0
0 -0.692728996276855 34.0
0 0.205874919891357 35.0
0 0.436487585306168 36.0
0 -0.0616602301597595 37.0
0 0.14679628610611 38.0
0 -0.295306324958801 39.0
0 0.206495106220245 40.0
0 -0.265419483184814 41.0
0 0.154336094856262 42.0
295.72236269 296.916015625 43.0
0 0.510784149169922 44.0
0 -1.17475485801697 45.0
0 0.237778127193451 46.0
0 -0.772589802742004 47.0
0 0.335468173027039 48.0
0 0.390107542276382 49.0
0.079396926519 0.0375046730041504 50.0
1.6879972655 1.88662350177765 51.0
0 -0.268382787704468 52.0
0 -0.137974739074707 53.0
0 -0.119204878807068 54.0
0 0.143969178199768 55.0
0 0.0394837856292725 56.0
0 0.275717496871948 57.0
0 0.448326915502548 58.0
0 0.281131863594055 59.0
0 -0.231305837631226 60.0
0 0.150480687618256 61.0
0 0.410451471805573 62.0
268.53906409 253.740737915039 0.0
0 -0.0430309772491455 1.0
0 0.128330945968628 2.0
0 -0.0475884675979614 3.0
0 -0.0118557214736938 4.0
0 0.147927224636078 5.0
0 0.639917016029358 6.0
0 0.224118709564209 7.0
0 0.443664729595184 8.0
0 -0.220088481903076 9.0
0 0.041678786277771 10.0
0 0.011921226978302 11.0
0 -0.249594807624817 12.0
62.190257203 61.4401168823242 13.0
35.618354132 28.3470554351807 14.0
0 -0.0694719552993774 15.0
0 -0.151529490947723 16.0
28.347314208 31.7282123565674 17.0
0 0.0911793112754822 18.0
0 0.0524242520332336 19.0
0 0.115232229232788 20.0
0 0.0656442642211914 21.0
0 0.138312101364136 22.0
30.200406904 28.0507946014404 23.0
0 0.0708972215652466 24.0
0 0.743968427181244 25.0
0 -0.040533721446991 26.0
0 -0.0978121757507324 27.0
0 0.490440249443054 28.0
0 0.00141966342926025 29.0
0 0.0317835807800293 30.0
0 -0.0704019069671631 31.0
0 -0.0289630889892578 32.0
0 0.0387381315231323 33.0
0 -0.0405174493789673 34.0
0 -0.015343189239502 35.0
0 0.439215570688248 36.0
0 0.219827234745026 37.0
0 -0.0159403085708618 38.0
0 -0.257828950881958 39.0
0 0.567456305027008 40.0
0 -0.105826914310455 41.0
0 -0.0933356285095215 42.0
284.76938136 294.933776855469 43.0
0 0.104217171669006 44.0
0 0.0233067870140076 45.0
0 0.058293879032135 46.0
0 -0.216912388801575 47.0
0 -0.0656865835189819 48.0
0 0.341405093669891 49.0
0.06767479772 0.00949770212173462 50.0
2.277999865 1.92001235485077 51.0
0 -0.15926730632782 52.0
0 -0.353762269020081 53.0
0 0.101355075836182 54.0
0 0.240728080272675 55.0
0 0.237434446811676 56.0
0 0.668797671794891 57.0
0 0.264836549758911 58.0
0 -0.0755348205566406 59.0
0 -0.0999372601509094 60.0
0 0.174639523029327 61.0
0 0.488180339336395 62.0
254.01407252 247.225311279297 0.0
0 -0.261956691741943 1.0
0 -0.284109354019165 2.0
0 -0.133772432804108 3.0
0 -0.433298587799072 4.0
0 0.289940714836121 5.0
0 -0.241442799568176 6.0
0 0.190999925136566 7.0
0 0.262923896312714 8.0
0 -0.00633430480957031 9.0
0 0.512618780136108 10.0
0 -0.340033173561096 11.0
0 -0.140259802341461 12.0
60.753999644 60.7805328369141 13.0
36.027191319 34.7586441040039 14.0
0 -0.126490116119385 15.0
0 0.241997063159943 16.0
23.804559827 32.1863670349121 17.0
0 0.0387487411499023 18.0
0 0.00674045085906982 19.0
0 0.149659216403961 20.0
0 0.0576795339584351 21.0
0 0.226283311843872 22.0
36.208347868 33.486888885498 23.0
0 0.0763449668884277 24.0
0 0.994603157043457 25.0
0 -0.0114167928695679 26.0
0 0.17318856716156 27.0
0 -0.0857613086700439 28.0
0 -0.00308871269226074 29.0
0 -0.15160870552063 30.0
0 -0.201865077018738 31.0
0 -0.0490124821662903 32.0
0 0.149447202682495 33.0
0 0.30218842625618 34.0
0 0.352431684732437 35.0
0 0.0113184452056885 36.0
0 0.132263779640198 37.0
0 -0.0518126487731934 38.0
0 -0.223687410354614 39.0
0 0.553685903549194 40.0
0 -0.0933247804641724 41.0
0 -0.000920414924621582 42.0
286.2246376 295.986541748047 43.0
0 0.121439933776855 44.0
0 0.0970225930213928 45.0
0 0.0360435247421265 46.0
0 -0.0287628173828125 47.0
0 -0.247162818908691 48.0
0 0.111885070800781 49.0
0.064525577131 0.111233830451965 50.0
1.7539313445 1.76829814910889 51.0
0 -0.0801816582679749 52.0
0 -0.247896432876587 53.0
0 0.0997438430786133 54.0
0 0.14429634809494 55.0
0 0.367732137441635 56.0
0 -0.0337564945220947 57.0
0 0.097947359085083 58.0
0 -0.152950823307037 59.0
0 0.166691184043884 60.0
0 -0.197040021419525 61.0
0 -0.465374350547791 62.0
241.84263471 253.524398803711 0.0
0 -0.033576488494873 1.0
0 0.00700730085372925 2.0
0 0.315507709980011 3.0
0 -0.115058660507202 4.0
0 -0.223267555236816 5.0
0 -0.0422614812850952 6.0
0 -0.194310247898102 7.0
0 -0.124891817569733 8.0
0 -0.176281750202179 9.0
0 -0.348543643951416 10.0
0 0.306168437004089 11.0
0 -0.0236247181892395 12.0
78.429406507 77.9948883056641 13.0
32.166202418 19.1726245880127 14.0
0 -0.236322164535522 15.0
0 0.0735378861427307 16.0
27.245902491 31.2519760131836 17.0
0 0.0190314054489136 18.0
0 -0.0289638042449951 19.0
0 0.134153485298157 20.0
0 -0.00311434268951416 21.0
0 0.272852778434753 22.0
33.283640356 32.6627464294434 23.0
0 0.105135142803192 24.0
0 0.604748606681824 25.0
0 -0.0168868899345398 26.0
0 -0.375619411468506 27.0
0 0.278365254402161 28.0
0 0.241046071052551 29.0
0 0.060046374797821 30.0
0 -0.127724289894104 31.0
0 -0.0548965930938721 32.0
0 -0.0557183027267456 33.0
0 0.559025287628174 34.0
0 0.204818964004517 35.0
0 0.507467150688171 36.0
0 -0.107750058174133 37.0
0 0.269205570220947 38.0
0 -0.164036870002747 39.0
0 0.56578916311264 40.0
0 -0.0903238654136658 41.0
0 0.0604466199874878 42.0
300.14380934 294.584014892578 43.0
0 0.18235182762146 44.0
0 -0.0205950736999512 45.0
0 0.242531955242157 46.0
0 -0.158376634120941 47.0
0 -0.0946475863456726 48.0
0 0.522485613822937 49.0
0.092362456615 -0.0253143310546875 50.0
2.2606329454 1.8631911277771 51.0
0 0.192671060562134 52.0
0 -0.205833613872528 53.0
0 0.195252060890198 54.0
0 0.171862602233887 55.0
0 0.00569832324981689 56.0
0 0.224812686443329 57.0
0 0.0246408581733704 58.0
0 0.0237973928451538 59.0
0 0.965324759483337 60.0
0 0.0179150104522705 61.0
0 0.0874508619308472 62.0
234.03854131 254.465042114258 0.0
0 -0.206579446792603 1.0
0 -0.272232413291931 2.0
0 0.477331101894379 3.0
0 -0.23248028755188 4.0
0 -0.437654614448547 5.0
0 0.0512675642967224 6.0
0 0.0750867128372192 7.0
0 -0.107950747013092 8.0
0 0.0497406721115112 9.0
0 0.32893979549408 10.0
0 -1.15011274814606 11.0
0 -0.147286772727966 12.0
76.838364452 74.156867980957 13.0
9.1102989106 27.4616298675537 14.0
0 -0.0669678449630737 15.0
0 -0.108614802360535 16.0
35.053977073 36.32861328125 17.0
0 0.0779587030410767 18.0
0 -0.000778913497924805 19.0
0 0.0362817049026489 20.0
0 0.0706940889358521 21.0
0 0.213774621486664 22.0
37.34908796 35.9308280944824 23.0
0 0.140545427799225 24.0
0 0.937431633472443 25.0
0 -0.024639368057251 26.0
0 -0.0477898716926575 27.0
0 0.353516221046448 28.0
0 -0.046846330165863 29.0
0 0.143945634365082 30.0
0 0.144519448280334 31.0
0 0.0514225959777832 32.0
0 -0.0967577695846558 33.0
0 0.230405449867249 34.0
0 -0.116246283054352 35.0
0 0.266708076000214 36.0
0 0.148043513298035 37.0
0 0.177171289920807 38.0
0 -0.206919431686401 39.0
0 0.26216995716095 40.0
0 -0.114876866340637 41.0
0 -0.0684891939163208 42.0
303.81540708 296.182373046875 43.0
0 0.101590514183044 44.0
0 0.239876985549927 45.0
0 0.128056943416595 46.0
0 -0.618522644042969 47.0
0 0.413690626621246 48.0
0.1147134639 0.444464921951294 49.0
0.081908723024 0.0802655220031738 50.0
2.1369753677 1.87664926052094 51.0
0 0.000800490379333496 52.0
0 -0.102096378803253 53.0
0 0.167464375495911 54.0
0 0.222645163536072 55.0
0 -0.0482672452926636 56.0
0 -0.0904929041862488 57.0
0 0.0801690220832825 58.0
0 0.17829704284668 59.0
0 -0.18076080083847 60.0
0 0.18742448091507 61.0
0 0.417107284069061 62.0
221.46793065 253.233261108398 0.0
0 -0.116986274719238 1.0
0 -0.0307521820068359 2.0
0 0.204447865486145 3.0
0 -0.339458107948303 4.0
0 -0.180938005447388 5.0
0 0.046438455581665 6.0
0 -0.136752128601074 7.0
0 0.0775339603424072 8.0
0 -0.2142573595047 9.0
0 0.187661528587341 10.0
0 0.07700514793396 11.0
0 -0.147192239761353 12.0
71.824143599 70.547966003418 13.0
31.99489351 28.9465885162354 14.0
0 -0.0628365278244019 15.0
0 0.0895132422447205 16.0
40.414019712 35.8375930786133 17.0
0 0.151289999485016 18.0
0 -0.0311985611915588 19.0
0 0.044958770275116 20.0
0 0.0924351215362549 21.0
0 0.0164957046508789 22.0
37.061387067 35.9231719970703 23.0
0 0.0663971900939941 24.0
0 0.822929859161377 25.0
0 0.00739586353302002 26.0
0 0.0903258919715881 27.0
0 0.241142213344574 28.0
0 -0.0579327940940857 29.0
0 0.113140165805817 30.0
0 0.0580604076385498 31.0
0 0.0436598062515259 32.0
0 -0.220423698425293 33.0
0 0.367484867572784 34.0
0 0.75343531370163 35.0
0 0.0455477833747864 36.0
0 0.0548319220542908 37.0
0 0.150162160396576 38.0
0 -0.183268070220947 39.0
0 0.659394025802612 40.0
0 -0.151300668716431 41.0
0 0.215458035469055 42.0
291.81780057 298.888366699219 43.0
0 0.0510305166244507 44.0
0 0.226035833358765 45.0
0 0.0777051448822021 46.0
0 0.0635156035423279 47.0
0 0.392958641052246 48.0
0 0.350450694561005 49.0
0.10256904065 0.0978537797927856 50.0
1.5964032373 1.85548675060272 51.0
0 -0.0537378787994385 52.0
0 0.00804871320724487 53.0
0 0.0996580123901367 54.0
0 0.147442221641541 55.0
0 0.180136919021606 56.0
0 -0.258429527282715 57.0
0 0.225045204162598 58.0
0 -1.04527115821838 59.0
0 0.868007838726044 60.0
0 0.00417953729629517 61.0
0 0.121149063110352 62.0
247.6131174 252.006607055664 0.0
0 -0.107166647911072 1.0
0 -0.222055315971375 2.0
0 0.0878258347511292 3.0
0 -0.344229936599731 4.0
0 0.068219780921936 5.0
0 0.0637453198432922 6.0
0 -0.0200366973876953 7.0
0 0.0941144824028015 8.0
0 -0.481196522712708 9.0
0 0.213987648487091 10.0
0 0.0388584733009338 11.0
0 -0.106242895126343 12.0
76.319203029 77.8487243652344 13.0
25.649087238 20.3906536102295 14.0
0 0.0984313488006592 15.0
0 0.224712252616882 16.0
25.341914732 32.0563468933105 17.0
0 0.0873795747756958 18.0
0 0.0455337762832642 19.0
0 0.15915060043335 20.0
0 0.0172908306121826 21.0
0 0.24385678768158 22.0
34.531093199 33.135181427002 23.0
0 0.0029296875 24.0
0 -0.0100384950637817 25.0
0 -0.0480034351348877 26.0
0 -0.121465444564819 27.0
0 0.711372435092926 28.0
0 0.144596993923187 29.0
0 -0.0174592137336731 30.0
0 0.0397968292236328 31.0
0 0.0536757707595825 32.0
0 -0.237241983413696 33.0
0 -0.595760107040405 34.0
0 0.133344054222107 35.0
0 0.318687438964844 36.0
0 0.0627782344818115 37.0
0 0.21043199300766 38.0
0 -0.226852297782898 39.0
0 0.348070323467255 40.0
0 -0.122960984706879 41.0
0 0.0711889266967773 42.0
314.28944471 297.195556640625 43.0
0 0.0486661195755005 44.0
0 0.0577101111412048 45.0
0 0.113207936286926 46.0
0 0.262946367263794 47.0
0 -0.169018447399139 48.0
0 0.419735938310623 49.0
0.095151660427 -0.0123644471168518 50.0
2.2334348031 1.89544486999512 51.0
0 0.0768805146217346 52.0
0 -0.136954963207245 53.0
0 -0.0551071166992188 54.0
0 0.202513635158539 55.0
0 0.216432690620422 56.0
0 0.0584641098976135 57.0
0 0.00235909223556519 58.0
0 0.0100822448730469 59.0
0 -0.117748200893402 60.0
0 0.0385739207267761 61.0
0 0.0219319462776184 62.0
261.74610948 252.454086303711 0.0
0 -0.218968391418457 1.0
0 -0.131602466106415 2.0
0 0.490225791931152 3.0
0 -0.19307154417038 4.0
0 -0.199155688285828 5.0
0 -0.337779521942139 6.0
0 0.349857926368713 7.0
0 0.00051581859588623 8.0
0 -0.256932258605957 9.0
0 0.250405848026276 10.0
0 0.121009349822998 11.0
0 -0.101004481315613 12.0
69.53284277 69.6635055541992 13.0
38.059677214 32.5531349182129 14.0
0 0.000475406646728516 15.0
0 -0.180772006511688 16.0
31.594816246 33.6948280334473 17.0
0 0.0775306224822998 18.0
0 -0.0646417737007141 19.0
0 0.169420599937439 20.0
0 -0.0839959383010864 21.0
0 0.283046007156372 22.0
35.149564525 34.5537414550781 23.0
0 0.0634089708328247 24.0
0 0.168252229690552 25.0
0 -0.0224272608757019 26.0
0 0.0692480802536011 27.0
0 0.392613619565964 28.0
0 0.0549033284187317 29.0
0 0.0238396525382996 30.0
0 -0.136387348175049 31.0
0 -0.0231361389160156 32.0
0 -0.26461124420166 33.0
0 0.0870320200920105 34.0
0 0.432264268398285 35.0
0 -0.42832362651825 36.0
0 -7.80820846557617e-05 37.0
0 0.246868073940277 38.0
0 -0.32242214679718 39.0
0 0.145263075828552 40.0
0 -0.191119611263275 41.0
0 -0.0326292514801025 42.0
294.81175233 298.739898681641 43.0
0 0.097419261932373 44.0
0 0.169802010059357 45.0
0 0.11767840385437 46.0
0 0.0483881235122681 47.0
0 0.0534404516220093 48.0
0 0.494971215724945 49.0
0.070426020313 0.0262004733085632 50.0
2.2325551085 1.8426867723465 51.0
0 0.09279465675354 52.0
0 -0.299207091331482 53.0
0 -0.0497686862945557 54.0
0 0.19133049249649 55.0
0 -0.0566709041595459 56.0
0 -0.0333964824676514 57.0
0 0.246952414512634 58.0
0 -0.00262337923049927 59.0
0 -0.16911917924881 60.0
0 0.0069509744644165 61.0
0 0.168364644050598 62.0
277.09290868 252.758590698242 0.0
0 0.284511804580688 1.0
0 -0.171834647655487 2.0
0 0.485008716583252 3.0
0 -0.20461368560791 4.0
0 0.198856711387634 5.0
0 0.0176013708114624 6.0
0 -0.130909442901611 7.0
0 0.486459285020828 8.0
0 -0.0401172041893005 9.0
0 -0.258287310600281 10.0
0 -0.158629655838013 11.0
0 -0.126253426074982 12.0
53.477481834 55.611011505127 13.0
36.912436926 35.8007392883301 14.0
0 -0.0672311782836914 15.0
0 0.161662340164185 16.0
25.912651484 32.4003257751465 17.0
0 0.0629158020019531 18.0
0 0.0215849876403809 19.0
0 0.0372160673141479 20.0
0 0.171537280082703 21.0
0 0.182377696037292 22.0
28.335864102 26.799488067627 23.0
0 0.0484461188316345 24.0
0 0.948423385620117 25.0
0 0.0398803949356079 26.0
0 -0.160808145999908 27.0
0 0.503559112548828 28.0
0 0.0270374417304993 29.0
0 0.0495703816413879 30.0
0 -0.0788285732269287 31.0
0 0.047152042388916 32.0
0 -0.0244436264038086 33.0
0 -0.150943398475647 34.0
0 0.938665390014648 35.0
0 0.419590651988983 36.0
0 0.183470129966736 37.0
0 0.515956938266754 38.0
0 -0.842766165733337 39.0
0 -0.777274966239929 40.0
0 -0.230506658554077 41.0
0 -0.0617502927780151 42.0
290.95705833 298.732543945312 43.0
0 0.00575047731399536 44.0
0 0.170129418373108 45.0
0 -0.00649160146713257 46.0
0 -0.0287796258926392 47.0
0 0.298271358013153 48.0
2.108941273 0.258817553520203 49.0
0.096002606855 -0.0104483962059021 50.0
1.9648751656 1.88817477226257 51.0
0 -0.233169198036194 52.0
0 -0.146990776062012 53.0
0 -0.0801321268081665 54.0
0 0.0248463153839111 55.0
0 -0.099858283996582 56.0
0 0.181154191493988 57.0
0 0.298386603593826 58.0
0 0.26734858751297 59.0
0 -0.261540651321411 60.0
0 0.106191337108612 61.0
0 0.364495158195496 62.0
249.45492195 240.225021362305 0.0
0 -0.353239774703979 1.0
0 0.0572580695152283 2.0
0 0.389988422393799 3.0
0 0.090793251991272 4.0
0 0.0933722853660583 5.0
0 -0.189807534217834 6.0
0 0.495372712612152 7.0
0 0.234797954559326 8.0
0 0.452949166297913 9.0
0 0.207742393016815 10.0
0 0.0220661163330078 11.0
0 -0.0306671261787415 12.0
55.251927053 56.2925453186035 13.0
51.677657302 42.0656623840332 14.0
0 -0.211641788482666 15.0
0 0.0344806313514709 16.0
40.790354674 34.8325157165527 17.0
0 0.1164311170578 18.0
0 -0.0459303855895996 19.0
0 0.0589738488197327 20.0
0 0.279478669166565 21.0
0 -0.155750751495361 22.0
26.034732853 22.8232479095459 23.0
0 0.1538006067276 24.0
25.222853634 0.68874990940094 25.0
0 -0.346311569213867 26.0
0 -0.114409327507019 27.0
0 0.753466129302979 28.0
0 0.00820779800415039 29.0
0 -0.082781195640564 30.0
0 0.114330887794495 31.0
0 0.0658099055290222 32.0
0 0.155436456203461 33.0
0 -0.24356472492218 34.0
0 0.219964325428009 35.0
0 0.335922211408615 36.0
0 -0.0407837629318237 37.0
0 0.284777641296387 38.0
0 -0.401389122009277 39.0
0 -0.192115187644958 40.0
0 -0.168968558311462 41.0
0 -0.0905622243881226 42.0
301.2916209 297.462066650391 43.0
0 -0.00543546676635742 44.0
0 -0.0612583756446838 45.0
0 0.101892054080963 46.0
0 -0.60593843460083 47.0
0 0.222780287265778 48.0
2.2964556695 0.373427212238312 49.0
0.094805283665 -0.0721757411956787 50.0
2.0505827376 1.92137086391449 51.0
0 -0.0920129418373108 52.0
0 2.38257360458374 53.0
0 0.255975961685181 54.0
0 -0.0819323658943176 55.0
0 -0.0264540314674377 56.0
0 0.200220942497253 57.0
0 0.0177338123321533 58.0
0 0.100854754447937 59.0
0 -0.176276504993439 60.0
0 0.120678603649139 61.0
0 0.217418432235718 62.0
243.21563184 252.477600097656 0.0
0 -0.167411208152771 1.0
0 -0.157587766647339 2.0
0 0.341730356216431 3.0
0 -0.295351266860962 4.0
0 -0.408159971237183 5.0
0 0.0520311594009399 6.0
0 -0.228434920310974 7.0
0 -0.490378379821777 8.0
0 0.115288257598877 9.0
0 -0.142476975917816 10.0
0 -1.12566328048706 11.0
0 0.0262459516525269 12.0
78.088979864 77.1386337280273 13.0
9.8632372385 24.6527576446533 14.0
0 0.0276129245758057 15.0
0 -0.0113765001296997 16.0
25.826757671 32.0092926025391 17.0
0 0.119951486587524 18.0
0 -1.03116035461426e-05 19.0
0 0.0763156414031982 20.0
0 0.144774079322815 21.0
0 0.01250159740448 22.0
26.298612433 25.5355606079102 23.0
0 0.0657126903533936 24.0
0 1.30862998962402 25.0
0 -0.227286338806152 26.0
0 -0.180225670337677 27.0
0 0.586972057819366 28.0
0 0.0538976192474365 29.0
0 0.0779018998146057 30.0
0 0.0765472650527954 31.0
0 0.104383409023285 32.0
0 -0.219766020774841 33.0
0 0.361710458993912 34.0
0 0.202163100242615 35.0
0 0.352300941944122 36.0
0 0.0497477054595947 37.0
0 0.00892925262451172 38.0
0 -0.172435879707336 39.0
0 0.0180436372756958 40.0
0 -0.0764903426170349 41.0
0 0.160994648933411 42.0
299.96133095 294.573089599609 43.0
0 0.11264181137085 44.0
0 0.138023555278778 45.0
0 0.127321183681488 46.0
0 -0.264452815055847 47.0
0 0.341168403625488 48.0
3.9638102895 0.604527771472931 49.0
0.097987304558 0.0592945218086243 50.0
2.0080214324 1.89212942123413 51.0
0 0.0323916673660278 52.0
0 -0.496945858001709 53.0
0 0.0110105276107788 54.0
0 0.108576834201813 55.0
0 0.184143304824829 56.0
0 -0.100034236907959 57.0
0 0.209026873111725 58.0
0 0.0762128829956055 59.0
0 -0.273822426795959 60.0
0 0.030988335609436 61.0
0 0.17259818315506 62.0
250.29175705 253.608489990234 0.0
0 -0.0981380939483643 1.0
0 -2.14576721191406e-05 2.0
0 0.0381057858467102 3.0
0 -0.0224807262420654 4.0
0 0.150214433670044 5.0
0 -0.0735414028167725 6.0
0 -0.172040402889252 7.0
0 -0.183613419532776 8.0
0 0.261267125606537 9.0
0 -0.336791396141052 10.0
0 0.0757777690887451 11.0
0 -0.236190795898438 12.0
78.279068904 80.3502502441406 13.0
28.558928874 19.958324432373 14.0
0 -0.0229932069778442 15.0
0 0.0617563128471375 16.0
32.921552998 35.6894836425781 17.0
0 0.0809080004692078 18.0
0 -0.0338582992553711 19.0
0 0.0645688772201538 20.0
0 0.116203129291534 21.0
0 0.25345915555954 22.0
37.093937737 34.6808204650879 23.0
0 0.0341339707374573 24.0
0 1.10925996303558 25.0
0 -0.109967350959778 26.0
0 -0.0551044940948486 27.0
0 0.18326073884964 28.0
0 0.058210015296936 29.0
0 -0.131868004798889 30.0
0 0.0614895820617676 31.0
0 -0.194751262664795 32.0
0 -0.15283066034317 33.0
0 0.415190875530243 34.0
0 -0.112946152687073 35.0
0 -0.142775058746338 36.0
0 -0.00584566593170166 37.0
0 -0.0716032385826111 38.0
0 -0.0682274103164673 39.0
0 -0.0104446411132812 40.0
0 -0.0314704179763794 41.0
0 0.115007877349854 42.0
291.13352056 297.968963623047 43.0
0 -0.0267291665077209 44.0
0 0.119182646274567 45.0
0 0.140662789344788 46.0
0 -0.175177037715912 47.0
0 0.383995413780212 48.0
1.5982567953 0.489256799221039 49.0
0.097149398972 0.0476078391075134 50.0
1.7920967951 1.88778471946716 51.0
0 0.053273618221283 52.0
0 -0.170219779014587 53.0
0 0.00848335027694702 54.0
0 0.186830580234528 55.0
0 0.032618522644043 56.0
0 -0.108133912086487 57.0
0 -0.20106303691864 58.0
0 0.248180627822876 59.0
0 -0.253612995147705 60.0
0 0.176704287528992 61.0
0 0.246916890144348 62.0
239.51127979 252.636764526367 0.0
0 -0.139972805976868 1.0
0 -0.00978100299835205 2.0
0 0.198299586772919 3.0
0 -0.256445407867432 4.0
0 -0.110576987266541 5.0
0 0.066101610660553 6.0
0 -0.206070244312286 7.0
0 0.203854084014893 8.0
0 -0.214594841003418 9.0
0 0.287066698074341 10.0
0 -0.230223178863525 11.0
0 -0.216548800468445 12.0
64.485923547 65.722297668457 13.0
42.325195667 31.5725479125977 14.0
0 -0.130008459091187 15.0
0 -0.0565881133079529 16.0
29.925599919 33.2869606018066 17.0
0 0.119399070739746 18.0
0 -0.0266663432121277 19.0
0 0.11699241399765 20.0
0 0.035057544708252 21.0
0 0.209059178829193 22.0
33.488868177 32.584415435791 23.0
0 0.088218629360199 24.0
0 0.950427412986755 25.0
0 -0.0424728393554688 26.0
0 0.0897487998008728 27.0
0 0.337209194898605 28.0
0 0.0558958649635315 29.0
0 0.0306122899055481 30.0
0 0.0552163124084473 31.0
0 0.16017872095108 32.0
0 0.11196506023407 33.0
0 0.291487216949463 34.0
0 -0.154607713222504 35.0
0 0.180970549583435 36.0
0 -0.0300173163414001 37.0
0 -0.0520323514938354 38.0
0 -0.101898670196533 39.0
0 0.535929560661316 40.0
0 -0.171264350414276 41.0
0 0.0695545673370361 42.0
291.15549498 297.283172607422 43.0
0 0.190236926078796 44.0
0 -7.70092010498047e-05 45.0
0 -0.017713725566864 46.0
0 0.151100099086761 47.0
0 0.340649485588074 48.0
0.36362271719 0.313677310943604 49.0
0.071924548879 -0.00402981042861938 50.0
1.9133313534 1.89465165138245 51.0
0 -0.0846115350723267 52.0
0 -0.141505122184753 53.0
0 0.161741614341736 54.0
0 0.0782386064529419 55.0
0 -0.00251913070678711 56.0
0 0.0568384528160095 57.0
0 0.348888635635376 58.0
0 0.0146982669830322 59.0
0 0.0653854608535767 60.0
0 0.0405558347702026 61.0
0 0.14789742231369 62.0
284.18896347 254.252944946289 0.0
0 -0.355383276939392 1.0
0 0.136292695999146 2.0
0 0.36376628279686 3.0
0 -0.0419062972068787 4.0
0 0.225121796131134 5.0
0 -0.0232334136962891 6.0
0 0.516219854354858 7.0
0 0.435578525066376 8.0
0 -0.110738635063171 9.0
0 0.266462385654449 10.0
0 -0.23317563533783 11.0
0 -0.214724779129028 12.0
72.138429721 69.6615219116211 13.0
10.4424253 26.5468101501465 14.0
0 -0.1175297498703 15.0
0 0.251025378704071 16.0
25.136219035 33.2243576049805 17.0
0 0.116409778594971 18.0
0 -0.0560003519058228 19.0
0 0.122449040412903 20.0
0 -0.0102461576461792 21.0
0 0.372496992349625 22.0
35.850716914 35.1756019592285 23.0
0 0.0366822481155396 24.0
0 0.846396923065186 25.0
0 -0.12141340970993 26.0
0 -0.0257696509361267 27.0
0 0.503777861595154 28.0
0 0.0752658247947693 29.0
0 0.172347903251648 30.0
0 -0.0871297121047974 31.0
0 0.0341842770576477 32.0
0 -0.138276875019073 33.0
0 0.242375493049622 34.0
0 0.462244540452957 35.0
0 0.157515645027161 36.0
0 0.00574207305908203 37.0
0 -0.036662220954895 38.0
0 -0.0687105655670166 39.0
0 0.541722536087036 40.0
0 -0.202555179595947 41.0
0 0.0626252889633179 42.0
295.85704234 296.139068603516 43.0
0 0.0733084678649902 44.0
0 0.172903656959534 45.0
0 0.20618736743927 46.0
0 -0.443910837173462 47.0
0 0.150705099105835 48.0
0 0.420115232467651 49.0
0.089656547463 0.0459138154983521 50.0
1.6011260186 1.81328415870667 51.0
0 -0.0947890281677246 52.0
0 0.0448951721191406 53.0
0 0.218259930610657 54.0
0 0.280696630477905 55.0
0 -0.0414183139801025 56.0
0 0.0214762687683105 57.0
0 0.159888386726379 58.0
0 0.416587710380554 59.0
0 -0.0699261426925659 60.0
0 -0.0423416495323181 61.0
0 0.0600079298019409 62.0
261.86699589 252.073806762695 0.0
0 -0.277551412582397 1.0
0 -0.0485561490058899 2.0
0 0.23280942440033 3.0
0 0.19492244720459 4.0
0 0.106896102428436 5.0
0 0.0384607315063477 6.0
0 0.480244815349579 7.0
0 -0.0949417352676392 8.0
0 0.196393489837646 9.0
0 0.39224249124527 10.0
0 0.761208772659302 11.0
0 -0.0483582615852356 12.0
44.901317496 41.5200424194336 13.0
35.760039936 48.4669990539551 14.0
0 0.0120840072631836 15.0
0 -0.00680363178253174 16.0
29.486854089 33.9902114868164 17.0
0 0.0867478847503662 18.0
0 0.069774866104126 19.0
0 0.0633569955825806 20.0
0 0.174973845481873 21.0
0 0.132582545280457 22.0
33.336136368 31.8324775695801 23.0
0 0.0478888154029846 24.0
0 0.470111608505249 25.0
0 -0.182915151119232 26.0
0 0.0503939986228943 27.0
0 -0.0463061928749084 28.0
0 0.0825108289718628 29.0
0 0.0240581035614014 30.0
0 0.364015460014343 31.0
0 0.1885005235672 32.0
0 0.168020009994507 33.0
0 0.270613312721252 34.0
0 -0.124928295612335 35.0
0 -0.52899956703186 36.0
0 0.0494536161422729 37.0
0 0.105133235454559 38.0
0 -0.305464148521423 39.0
0 0.286880135536194 40.0
0 -0.128257989883423 41.0
0 -0.163671016693115 42.0
283.00956435 293.992279052734 43.0
0 0.0871881246566772 44.0
0 0.186986029148102 45.0
0 0.131250619888306 46.0
0 -0.246359348297119 47.0
0 0.330167889595032 48.0
1.8665259843 0.51373815536499 49.0
0.074070759966 0.0752745270729065 50.0
2.183104921 1.90294790267944 51.0
0 0.138473570346832 52.0
0 0.0232686996459961 53.0
0 0.0394735932350159 54.0
0 0.139924705028534 55.0
0 0.074289083480835 56.0
0 0.385217010974884 57.0
0 0.55396842956543 58.0
0 0.316179722547531 59.0
0 -0.2103590965271 60.0
0 0.0118786692619324 61.0
0 0.220596075057983 62.0
265.67200758 254.058242797852 0.0
0 -0.00486302375793457 1.0
0 -0.210538983345032 2.0
0 -0.0218281745910645 3.0
0 -0.100946187973022 4.0
0 0.339770883321762 5.0
0 0.222285032272339 6.0
0 0.142283320426941 7.0
0 -0.879916191101074 8.0
0 0.588631868362427 9.0
0 1.01968419551849 10.0
0 -0.521225094795227 11.0
0 0.0779263973236084 12.0
76.710259106 75.8988876342773 13.0
11.391431096 23.2546672821045 14.0
0 -0.168910622596741 15.0
0 -0.0935900211334229 16.0
31.994360087 34.5126266479492 17.0
0 0.150274038314819 18.0
0 -0.0199817419052124 19.0
0 0.0648000836372375 20.0
0 0.167836725711823 21.0
0 0.23974221944809 22.0
35.111561049 31.2402839660645 23.0
0 0.0162546634674072 24.0
0 0.661328613758087 25.0
0 0.0858148336410522 26.0
0 -0.209147691726685 27.0
0 0.917921483516693 28.0
0 -0.0142744779586792 29.0
0 0.129318475723267 30.0
0 0.183916568756104 31.0
0 0.153031229972839 32.0
0 0.0260733962059021 33.0
0 0.00842124223709106 34.0
0 -0.0207123160362244 35.0
0 -0.467533349990845 36.0
0 0.200155973434448 37.0
0 -0.00236034393310547 38.0
0 -0.0550967454910278 39.0
0 0.417155414819717 40.0
0 -0.0118065476417542 41.0
0 0.0495287179946899 42.0
283.58730026 297.337493896484 43.0
0 0.037407398223877 44.0
0 -0.0281927585601807 45.0
0 9.57250595092773e-05 46.0
0 -0.346744537353516 47.0
0 0.291592508554459 48.0
0 0.456002116203308 49.0
0.10643368626 0.0540409088134766 50.0
2.1690377987 1.91340065002441 51.0
0 0.366536319255829 52.0
0 0.15336549282074 53.0
0 -0.0264210104942322 54.0
0 0.00666892528533936 55.0
0 0.185529291629791 56.0
0 0.0365253686904907 57.0
0 -0.294215202331543 58.0
0 0.262622952461243 59.0
0 -0.237406134605408 60.0
0 -0.0592478513717651 61.0
0 -0.158432960510254 62.0
267.48054756 252.471603393555 0.0
0 -0.114800572395325 1.0
0 0.0355077981948853 2.0
0 0.435631990432739 3.0
0 -0.0924365520477295 4.0
0 -0.388506889343262 5.0
0 -0.241170406341553 6.0
0 0.247492134571075 7.0
0 0.218463182449341 8.0
0 -0.280598044395447 9.0
0 0.326982408761978 10.0
0 -0.263877153396606 11.0
0 -0.0130452513694763 12.0
57.703916354 56.5036315917969 13.0
37.020759439 35.9263076782227 14.0
0 -0.0752104520797729 15.0
0 -0.0237022042274475 16.0
27.115647354 32.8164291381836 17.0
0 0.0834323167800903 18.0
0 -0.0117706656455994 19.0
0 0.0674382448196411 20.0
0 0.0947155952453613 21.0
0 0.138369381427765 22.0
30.978534807 29.6457805633545 23.0
0 -0.019317626953125 24.0
0 0.479972958564758 25.0
0 0.00724506378173828 26.0
0 -0.16094958782196 27.0
0 0.738277912139893 28.0
0 0.0476799607276917 29.0
0 0.0753329992294312 30.0
0 -0.106682300567627 31.0
0 0.0438288450241089 32.0
0 0.0683495998382568 33.0
0 0.0675579905509949 34.0
0 -0.00656276941299438 35.0
0 0.109002113342285 36.0
0 0.00335657596588135 37.0
0 0.0598176717758179 38.0
0 -0.352950930595398 39.0
0 0.115980267524719 40.0
0 -0.164498507976532 41.0
0 -0.082370400428772 42.0
286.01325435 297.706726074219 43.0
0 -0.0313942432403564 44.0
0 0.129482448101044 45.0
0 0.161653399467468 46.0
0 -0.152162790298462 47.0
0 0.242877602577209 48.0
0 0.541531562805176 49.0
0.077024026516 0.0530484318733215 50.0
1.7915774925 1.78559303283691 51.0
0 0.0212644338607788 52.0
0 -0.557935237884521 53.0
0 -0.106124043464661 54.0
0 0.0277412533760071 55.0
0 0.153541505336761 56.0
0 -0.0411404371261597 57.0
0 0.0775750279426575 58.0
0 -0.0235559940338135 59.0
0 -0.216536998748779 60.0
0 0.0341402292251587 61.0
0 0.0898252129554749 62.0
227.07067814 251.386276245117 0.0
0 -0.0796937942504883 1.0
0 -0.0445614457130432 2.0
0 0.125867247581482 3.0
0 -0.415549278259277 4.0
0 -0.164645791053772 5.0
0 0.119671285152435 6.0
0 -0.0380252599716187 7.0
0 0.123388767242432 8.0
0 0.44365119934082 9.0
0 0.483135104179382 10.0
0 -0.0164332389831543 11.0
0 -0.0917214751243591 12.0
72.16094778 73.0929489135742 13.0
30.151610644 23.5643100738525 14.0
0 -0.0489653348922729 15.0
0 0.000561535358428955 16.0
38.252456121 33.0246315002441 17.0
0 0.106699109077454 18.0
0 0.0449612140655518 19.0
0 0.118033349514008 20.0
0 0.0148642063140869 21.0
0 0.228661298751831 22.0
32.388080696 32.2728462219238 23.0
0 -0.0146400928497314 24.0
0 0.0666993260383606 25.0
0 -0.095469057559967 26.0
0 0.0164037942886353 27.0
0 -0.0697977542877197 28.0
0 -0.0603474378585815 29.0
0 0.0540799498558044 30.0
0 -0.159295082092285 31.0
0 -0.0476369261741638 32.0
0 -0.113403141498566 33.0
0 0.141712665557861 34.0
0 0.000368893146514893 35.0
0 0.346021324396133 36.0
0 0.000138640403747559 37.0
0 0.123813688755035 38.0
0 -0.216501235961914 39.0
0 0.255323946475983 40.0
0 -0.091658353805542 41.0
0 0.0685045719146729 42.0
286.70055721 296.608825683594 43.0
0 0.145140528678894 44.0
0 0.0813790559768677 45.0
0 0.176910638809204 46.0
0 0.0148590207099915 47.0
0 -0.169969260692596 48.0
0 0.535204470157623 49.0
0.071867333211 -0.0119772553443909 50.0
2.2525413002 1.85063791275024 51.0
0 0.13093113899231 52.0
0 -0.132136225700378 53.0
0 0.138827085494995 54.0
0 -0.0428882241249084 55.0
0 -0.0902143716812134 56.0
0 0.146130919456482 57.0
0 0.0150392055511475 58.0
0 0.0317908525466919 59.0
0 -0.178347706794739 60.0
0 0.0992974042892456 61.0
0 0.13707959651947 62.0
261.14042368 253.557083129883 0.0
0 -0.313146829605103 1.0
0 0.16437166929245 2.0
0 -0.0128908157348633 3.0
0 0.0259091854095459 4.0
0 0.278394758701324 5.0
0 -0.113057971000671 6.0
0 0.214560449123383 7.0
0 -0.256123661994934 8.0
0 0.230263411998749 9.0
0 -0.33967924118042 10.0
0 0.0649049878120422 11.0
0 -0.115145802497864 12.0
71.083858121 72.3579711914062 13.0
50.826591861 31.7833042144775 14.0
0 -0.194744110107422 15.0
0 0.408963680267334 16.0
36.651673303 36.7369232177734 17.0
0 0.122484385967255 18.0
0 -0.0642070770263672 19.0
0 0.0304068922996521 20.0
0 0.0559366941452026 21.0
0 0.34694105386734 22.0
31.494727069 29.9023895263672 23.0
0 0.060702383518219 24.0
26.743126621 0.83066737651825 25.0
0 0.115637719631195 26.0
0 -0.267883777618408 27.0
0 0.86458420753479 28.0
0 0.275940775871277 29.0
0 0.0424387454986572 30.0
0 0.0619001388549805 31.0
0 0.0419407486915588 32.0
0 -0.288586616516113 33.0
0 -0.737361669540405 34.0
0 0.329372406005859 35.0
0 0.286349177360535 36.0
0 -0.0397064089775085 37.0
0 0.150147557258606 38.0
0 -0.295083165168762 39.0
0 -0.0558708906173706 40.0
0 -0.113006353378296 41.0
0 -0.0897114276885986 42.0
311.11158598 297.154144287109 43.0
0 0.0912927389144897 44.0
0 0.0684862732887268 45.0
0 0.0363892912864685 46.0
0 0.132136702537537 47.0
0 -0.030346155166626 48.0
0 0.403505146503448 49.0
0.096624606515 0.0147730112075806 50.0
2.1237207782 1.87090182304382 51.0
0 0.0213513970375061 52.0
0 -0.103778302669525 53.0
0 -0.120810866355896 54.0
0 -0.105924189090729 55.0
0 -0.023493766784668 56.0
0 0.156916379928589 57.0
0 0.463135272264481 58.0
0 0.397423654794693 59.0
0 0.266860961914062 60.0
0 0.0507187247276306 61.0
0 0.13584303855896 62.0
245.12148855 252.341293334961 0.0
0 -0.209338188171387 1.0
0 0.0320131182670593 2.0
0 -0.0910293459892273 3.0
0 -0.191214323043823 4.0
0 -0.106591999530792 5.0
0 0.0994682312011719 6.0
0 0.000444173812866211 7.0
0 0.147042155265808 8.0
0 -0.372225761413574 9.0
0 0.00162613391876221 10.0
0 -0.0231717228889465 11.0
0 -0.103484451770782 12.0
65.895475576 67.2085342407227 13.0
41.657727304 30.8951072692871 14.0
0 0.0357480049133301 15.0
0 -0.113767743110657 16.0
28.640773079 31.1968364715576 17.0
0 0.0219910144805908 18.0
0 0.0170098543167114 19.0
0 0.0904362201690674 20.0
0 0.0461598634719849 21.0
0 0.294537425041199 22.0
34.636532129 35.8804397583008 23.0
0 0.0619898438453674 24.0
0 0.827214479446411 25.0
0 0.111802935600281 26.0
0 -0.0653664469718933 27.0
0 -0.262724041938782 28.0
0 -0.081706702709198 29.0
0 0.0776560306549072 30.0
0 -0.0414693355560303 31.0
0 0.231955170631409 32.0
0 -0.00293773412704468 33.0
0 0.167876422405243 34.0
0 -0.00135445594787598 35.0
0 0.196478605270386 36.0
0 0.061779260635376 37.0
0 0.186071157455444 38.0
0 -0.120960354804993 39.0
0 0.530448794364929 40.0
0 -0.102199494838715 41.0
0 -0.0530680418014526 42.0
290.75527972 298.525085449219 43.0
0 0.121775269508362 44.0
0 0.185519516468048 45.0
0 0.00783079862594604 46.0
0 -0.00118178129196167 47.0
0 0.391044199466705 48.0
1.0477668826 0.482469946146011 49.0
0.090740649412 0.0511385798454285 50.0
1.5957167434 1.89819812774658 51.0
0 0.124558687210083 52.0
0 -0.0728797912597656 53.0
0 -0.146209120750427 54.0
0 0.140159726142883 55.0
0 0.284205377101898 56.0
0 0.112472355365753 57.0
0 0.108406484127045 58.0
0 -0.071189820766449 59.0
0 0.0125104784965515 60.0
0 0.049915611743927 61.0
0 0.176173388957977 62.0
253.52106179 251.646530151367 0.0
0 -0.214928030967712 1.0
0 -0.127172231674194 2.0
0 0.359827697277069 3.0
0 0.0425787568092346 4.0
0 0.0152685642242432 5.0
0 -0.00914400815963745 6.0
0 0.0547112226486206 7.0
0 -0.683195829391479 8.0
0 0.270838022232056 9.0
0 -0.102982819080353 10.0
0 -0.305591106414795 11.0
0 -0.116755962371826 12.0
71.875574123 71.3768539428711 13.0
12.838606711 22.9943923950195 14.0
0 0.0483276844024658 15.0
0 0.0174645185470581 16.0
27.05299929 31.5577144622803 17.0
0 0.0844266414642334 18.0
0 0.0069272518157959 19.0
0 0.268492221832275 20.0
0 0.00184059143066406 21.0
0 0.0852386951446533 22.0
40.274733803 37.6258125305176 23.0
0 0.107025384902954 24.0
0 0.486451685428619 25.0
0 0.12506103515625 26.0
0 0.0752905011177063 27.0
0 0.0115954875946045 28.0
0 -0.00793099403381348 29.0
0 0.0574442744255066 30.0
0 0.00648534297943115 31.0
0 0.144938230514526 32.0
0 0.0742830038070679 33.0
0 0.353290259838104 34.0
0 -0.0957954525947571 35.0
0 0.507789611816406 36.0
0 0.0352555513381958 37.0
0 -0.0640645623207092 38.0
0 -0.237329006195068 39.0
0 0.630081355571747 40.0
0 -0.147231817245483 41.0
0 -0.0253587961196899 42.0
298.75882548 298.749298095703 43.0
0 0.0867034196853638 44.0
0 0.153098702430725 45.0
0 0.121694147586823 46.0
0 -0.0397971272468567 47.0
0 0.362902402877808 48.0
0 0.446708887815475 49.0
0.096344271167 0.0558925271034241 50.0
1.9984780734 1.87562942504883 51.0
0 -0.0117701292037964 52.0
0 -0.0430148839950562 53.0
0 0.133057355880737 54.0
0 -0.00196266174316406 55.0
0 0.044913649559021 56.0
0 -0.360273718833923 57.0
0 -0.271686673164368 58.0
0 0.183985829353333 59.0
0 -0.260536074638367 60.0
0 0.217797935009003 61.0
0 0.403513550758362 62.0
267.2703099 254.02360534668 0.0
0 -0.54422926902771 1.0
0 0.152608275413513 2.0
0 -0.11122852563858 3.0
0 -0.364988803863525 4.0
0 0.0588781237602234 5.0
0 1.25175189971924 6.0
0 -0.189032316207886 7.0
0 0.379375517368317 8.0
0 0.0516160130500793 9.0
0 0.36661297082901 10.0
0 0.0186471939086914 11.0
0 -0.386937260627747 12.0
52.553203475 52.748104095459 13.0
33.5564501 35.411190032959 14.0
0 -0.123571038246155 15.0
0 -0.222596883773804 16.0
27.2933085 32.8994674682617 17.0
0 0.145420670509338 18.0
0 -0.0375584363937378 19.0
0 0.0899382829666138 20.0
0 -0.0916652679443359 21.0
0 0.54896605014801 22.0
37.442280897 38.6325149536133 23.0
0 0.0862654447555542 24.0
0 0.415209650993347 25.0
0 -0.134116411209106 26.0
0 -0.0176766514778137 27.0
0 -0.212907314300537 28.0
0 -0.00284767150878906 29.0
0 0.104647397994995 30.0
0 -0.0234410762786865 31.0
0 0.0733456015586853 32.0
0 -0.0944041609764099 33.0
0 0.280238211154938 34.0
0 -0.0910204648971558 35.0
0 0.191799879074097 36.0
0 0.115026831626892 37.0
0 -0.0782114267349243 38.0
0 -0.163810610771179 39.0
0 0.0533418655395508 40.0
0 -0.0581958889961243 41.0
0 -0.091701865196228 42.0
287.3955766 297.033355712891 43.0
0 -0.0315985679626465 44.0
0 0.0895852446556091 45.0
0 0.278201103210449 46.0
0 -0.0338405966758728 47.0
0 -0.193297326564789 48.0
0 0.24840235710144 49.0
0.063175023323 0.0884526968002319 50.0
1.6377028112 1.76078939437866 51.0
0 0.115723967552185 52.0
0 -0.446105480194092 53.0
0 0.262044191360474 54.0
0 0.0555276870727539 55.0
0 -0.194451451301575 56.0
0 -0.127752006053925 57.0
0 0.347924649715424 58.0
0 0.0583570003509521 59.0
0 -0.0947839617729187 60.0
0 -0.0730316042900085 61.0
0 0.0830452442169189 62.0
251.16310876 252.23779296875 0.0
0 -0.252090573310852 1.0
0 -0.0288075804710388 2.0
0 -0.578339099884033 3.0
0 -0.501063585281372 4.0
0 -0.00982666015625 5.0
0 0.286843478679657 6.0
0 -0.132911920547485 7.0
0 -0.249148845672607 8.0
0 0.037010133266449 9.0
0 -0.00307327508926392 10.0
0 -0.177569448947906 11.0
0 -0.0401121377944946 12.0
50.813518549 49.9971237182617 13.0
40.036298479 41.2866020202637 14.0
0 -0.171499609947205 15.0
0 -0.038651704788208 16.0
32.836430912 35.6096420288086 17.0
0 0.138288199901581 18.0
0 0.00524270534515381 19.0
0 0.200469613075256 20.0
0 -0.0685455799102783 21.0
0 0.300589799880981 22.0
38.982384552 37.8751564025879 23.0
0 0.0549704432487488 24.0
2.1279160358 0.31153067946434 25.0
0 0.31032407283783 26.0
0 -0.264504909515381 27.0
0 0.324435412883759 28.0
0 0.142724692821503 29.0
0 -0.0773015022277832 30.0
0 0.18851900100708 31.0
0 -0.191616654396057 32.0
0 0.0302078723907471 33.0
0 0.379977822303772 34.0
0 0.331045657396317 35.0
0 0.152745962142944 36.0
0 0.117709577083588 37.0
0 -0.106888890266418 38.0
0 -0.231587052345276 39.0
0 0.22314977645874 40.0
0 -0.0196977853775024 41.0
0 -0.0254838466644287 42.0
294.89315889 293.846099853516 43.0
0 0.15377402305603 44.0
0 -0.104720532894135 45.0
0 0.220257222652435 46.0
0 -0.153218448162079 47.0
0 0.219200551509857 48.0
1.4800181918 0.582299172878265 49.0
0.068843932195 0.131385385990143 50.0
1.9172830808 1.83783626556396 51.0
0 -0.219256162643433 52.0
0 -0.0900296568870544 53.0
0 -0.125370025634766 54.0
0 0.00842118263244629 55.0
0 0.255874514579773 56.0
0 0.989907145500183 57.0
0 0.195785284042358 58.0
0 -0.267558574676514 59.0
0 0.0743671059608459 60.0
0 0.0463064908981323 61.0
0 0.235679388046265 62.0
246.18008873 252.145401000977 0.0
0 0.207686305046082 1.0
0 0.0901353359222412 2.0
0 0.0284256339073181 3.0
0 -0.189276158809662 4.0
0 -0.48973536491394 5.0
0 -0.211369037628174 6.0
0 -0.0813961625099182 7.0
0 0.0945023894309998 8.0
0 0.202757596969604 9.0
0 -0.430136561393738 10.0
0 -0.212770104408264 11.0
0 -0.0884512662887573 12.0
51.925210294 51.7497406005859 13.0
47.576853963 40.9365501403809 14.0
0 -0.0635608434677124 15.0
0 0.237061321735382 16.0
29.227506981 35.1278610229492 17.0
0 0.157905101776123 18.0
0 0.0194939374923706 19.0
0 0.247814536094666 20.0
0 -0.002524733543396 21.0
0 0.0441487431526184 22.0
40.726369818 39.061595916748 23.0
0 0.107930243015289 24.0
0 0.762527287006378 25.0
0 -0.089552640914917 26.0
0 0.0921979546546936 27.0
0 0.860683679580688 28.0
0 -0.025451123714447 29.0
0 0.0653525590896606 30.0
0 0.22679078578949 31.0
0 0.173970222473145 32.0
0 -0.00267684459686279 33.0
0 0.222767353057861 34.0
0 -0.149322032928467 35.0
0 0.336281061172485 36.0
0 0.189785242080688 37.0
0 -0.00670504570007324 38.0
0 -0.379420399665833 39.0
0 0.674310564994812 40.0
0 -0.0304689407348633 41.0
0 -0.0554341077804565 42.0
303.15273324 298.185577392578 43.0
0 0.193160057067871 44.0
0 0.0488681197166443 45.0
0 0.280530273914337 46.0
0 -0.658196687698364 47.0
0 -0.0967535376548767 48.0
0 0.462551563978195 49.0
0.070269781269 0.00678026676177979 50.0
1.5756982828 1.83523380756378 51.0
0 -0.136706590652466 52.0
0 -0.0746543407440186 53.0
0 0.113482475280762 54.0
0 0.193857908248901 55.0
0 0.130498468875885 56.0
0 -0.230403423309326 57.0
0 -0.330136656761169 58.0
0 0.183625936508179 59.0
0 -0.247221112251282 60.0
0 0.234114706516266 61.0
0 0.363162517547607 62.0
280.56167198 252.956695556641 0.0
0 -0.183181405067444 1.0
0 -0.0866056680679321 2.0
0 0.0535677075386047 3.0
0 -0.131701350212097 4.0
0 0.111339688301086 5.0
0 -0.407456755638123 6.0
0 0.240802347660065 7.0
0 -0.00574249029159546 8.0
0 0.0823457837104797 9.0
0 0.0357754826545715 10.0
0 0.0866848826408386 11.0
0 -0.137004137039185 12.0
64.059476262 66.6239776611328 13.0
40.988447534 30.4782447814941 14.0
0 0.0106765031814575 15.0
0 0.00193381309509277 16.0
37.877956099 35.3997039794922 17.0
0 0.11987578868866 18.0
0 -0.0910496115684509 19.0
0 0.0904062986373901 20.0
0 0.0634500980377197 21.0
0 0.20587831735611 22.0
32.491973167 30.5520992279053 23.0
0 0.094576358795166 24.0
0 0.588636696338654 25.0
0 -0.0914474725723267 26.0
0 0.0148735642433167 27.0
0 0.588263869285583 28.0
0 0.028254508972168 29.0
0 0.178848743438721 30.0
0 0.110363006591797 31.0
0 0.225628435611725 32.0
0 0.0209729671478271 33.0
0 -0.0131618976593018 34.0
0 -0.0160731077194214 35.0
0 -0.348754644393921 36.0
0 0.000737428665161133 37.0
0 0.229400038719177 38.0
0 -0.188957810401917 39.0
0 0.57001930475235 40.0
0 -0.098996102809906 41.0
0 -0.0187243223190308 42.0
285.93045757 298.664306640625 43.0
0 0.0864394903182983 44.0
0 0.0628696084022522 45.0
0 0.0756015181541443 46.0
0 0.0503056645393372 47.0
0 0.130452334880829 48.0
0 0.36505651473999 49.0
0.072908591894 0.0539958477020264 50.0
1.9118310413 1.8724776506424 51.0
0 0.293220996856689 52.0
0 -0.201909184455872 53.0
0 -0.0279501676559448 54.0
0 -0.0523725748062134 55.0
0 0.0939626693725586 56.0
0 0.179200291633606 57.0
0 0.246715426445007 58.0
0 0.202497601509094 59.0
0 -0.21574854850769 60.0
0 0.1417316198349 61.0
0 0.402551412582397 62.0
268.84593865 253.21125793457 0.0
0 -0.0995955467224121 1.0
0 -0.0834726095199585 2.0
0 0.209715723991394 3.0
0 0.0672023296356201 4.0
0 -0.0142028331756592 5.0
0 0.111357808113098 6.0
0 0.428458452224731 7.0
0 -0.328866958618164 8.0
0 0.0566229224205017 9.0
0 -0.293844938278198 10.0
0 -0.129977226257324 11.0
0 -0.336527943611145 12.0
73.288245755 74.0155487060547 13.0
33.653586836 28.7527523040771 14.0
0 0.0239294767379761 15.0
0 0.0356494188308716 16.0
38.056911978 38.216438293457 17.0
0 0.132741928100586 18.0
0 -0.0347181558609009 19.0
0 0.0270224213600159 20.0
0 0.0551228523254395 21.0
0 0.229327261447906 22.0
37.427885457 35.8674468994141 23.0
0 0.0486059784889221 24.0
0 0.617012619972229 25.0
0 -0.0793699622154236 26.0
0 -0.0516997575759888 27.0
0 -0.234719038009644 28.0
0 0.0425024032592773 29.0
0 -0.07344651222229 30.0
0 0.0318945646286011 31.0
0 0.0447612404823303 32.0
0 -0.237691879272461 33.0
0 0.22480171918869 34.0
0 0.935904204845428 35.0
0 0.167980790138245 36.0
0 0.0716967582702637 37.0
0 0.33319753408432 38.0
0 -0.276422381401062 39.0
0 0.00508671998977661 40.0
0 -0.190362513065338 41.0
0 -0.050625205039978 42.0
286.67563441 298.018920898438 43.0
0 0.181708216667175 44.0
0 0.182148575782776 45.0
0 0.163093090057373 46.0
0 0.0280827283859253 47.0
0 -0.0165048837661743 48.0
2.8441207667 0.264579832553864 49.0
0.09403044084 0.0217402577400208 50.0
1.852969128 1.88913822174072 51.0
0 0.246569693088531 52.0
0 -0.212582588195801 53.0
0 -0.0126530528068542 54.0
0 0.273093402385712 55.0
0 0.309487789869308 56.0
0 0.434158444404602 57.0
0 0.192535877227783 58.0
0 -0.0576522946357727 59.0
0 -0.119833588600159 60.0
0 0.0646117329597473 61.0
0 0.178103566169739 62.0
259.76311628 252.703506469727 0.0
0 -0.104145169258118 1.0
0 -0.0872887372970581 2.0
0 0.189463675022125 3.0
0 -0.107281744480133 4.0
0 -0.0452011823654175 5.0
0 0.00821930170059204 6.0
0 0.278345942497253 7.0
0 -0.23036003112793 8.0
0 0.377727627754211 9.0
0 0.257748186588287 10.0
0 0.0449373722076416 11.0
0 -0.190274059772491 12.0
59.980435683 61.3961982727051 13.0
40.125500029 34.4397621154785 14.0
0 -0.0389517545700073 15.0
0 0.00519806146621704 16.0
36.309769656 35.4375419616699 17.0
0 0.12419867515564 18.0
0 -0.0199637413024902 19.0
0 0.0586245059967041 20.0
0 0.0640453100204468 21.0
0 0.300258547067642 22.0
33.802271857 35.0045928955078 23.0
0 0.079776406288147 24.0
0 0.974428057670593 25.0
0 -0.117996156215668 26.0
0 -0.105762124061584 27.0
0 0.0309404134750366 28.0
0 -0.00195157527923584 29.0
0 0.0766949057579041 30.0
0 0.00705766677856445 31.0
0 -0.0867264270782471 32.0
0 -0.109025239944458 33.0
0 0.252023816108704 34.0
0 0.171873450279236 35.0
0 0.524389743804932 36.0
0 0.0866153836250305 37.0
0 0.0657389163970947 38.0
0 -0.219281315803528 39.0
0 0.283773124217987 40.0
0 -0.174368798732758 41.0
0 0.123157620429993 42.0
306.11527063 296.549865722656 43.0
0 0.0623862743377686 44.0
0 0.0658013820648193 45.0
0 0.0916246771812439 46.0
0 -0.0913875102996826 47.0
0 0.221299588680267 48.0
0 0.457425057888031 49.0
0.090106932242 0.00896716117858887 50.0
2.2742264445 1.88332319259644 51.0
0 -0.0333306789398193 52.0
0 -0.0494397282600403 53.0
0 0.0535047650337219 54.0
0 0.0795515775680542 55.0
0 -0.141356468200684 56.0
0 0.353828281164169 57.0
0 0.130207419395447 58.0
0 0.257147490978241 59.0
0 -0.135497093200684 60.0
0 0.0791037082672119 61.0
0 0.240854620933533 62.0
273.58449628 252.783874511719 0.0
0 -0.189244747161865 1.0
0 -0.000396609306335449 2.0
0 -0.187063574790955 3.0
0 -0.194615960121155 4.0
0 0.270649969577789 5.0
0 0.00742918252944946 6.0
0 0.356272757053375 7.0
0 0.0537349581718445 8.0
0 0.729804515838623 9.0
0 -0.55426299571991 10.0
0 0.809049069881439 11.0
0 -0.0744830965995789 12.0
71.739535599 72.3970336914062 13.0
36.289904094 27.4278812408447 14.0
0 0.0286284685134888 15.0
0 0.0883298516273499 16.0
38.336049513 35.8459701538086 17.0
0 0.16383969783783 18.0
0 -0.0313708782196045 19.0
0 0.0659292936325073 20.0
0 0.129904568195343 21.0
0 0.21360981464386 22.0
37.069804395 36.2068634033203 23.0
0 0.0940236449241638 24.0
0 0.778362274169922 25.0
0 -0.0662854909896851 26.0
0 -0.0348910093307495 27.0
0 0.00146424770355225 28.0
0 -0.0540297031402588 29.0
0 0.00925588607788086 30.0
0 -0.422606110572815 31.0
0 -0.102203190326691 32.0
0 -0.120419025421143 33.0
0 0.0772835612297058 34.0
0 0.0237877368927002 35.0
0 -0.029300332069397 36.0
0 0.102268040180206 37.0
0 0.0974082350730896 38.0
0 -0.152595996856689 39.0
0 0.780945062637329 40.0
0 -0.0678203105926514 41.0
0 -0.00748145580291748 42.0
289.85898582 296.267730712891 43.0
0 0.145520091056824 44.0
0 0.149713218212128 45.0
0 0.0277464389801025 46.0
0 0.0789591670036316 47.0
0 0.284510970115662 48.0
0 0.39704293012619 49.0
0.082269033937 0.0162172913551331 50.0
2.1728415722 1.88700437545776 51.0
0 0.0520989894866943 52.0
0 0.318970084190369 53.0
0 -0.0719331502914429 54.0
0 0.228372097015381 55.0
0 0.38329666852951 56.0
0 0.136275947093964 57.0
0 0.465946555137634 58.0
0 0.171541810035706 59.0
0 -0.115406394004822 60.0
0 0.078571081161499 61.0
0 0.166538596153259 62.0
255.3310723 252.304580688477 0.0
0 -0.291888833045959 1.0
0 -0.0582285523414612 2.0
0 0.33413290977478 3.0
0 -0.0798497200012207 4.0
0 0.0770394802093506 5.0
0 -0.158468008041382 6.0
0 0.202943801879883 7.0
0 0.0856842994689941 8.0
0 0.223799049854279 9.0
0 0.135497033596039 10.0
0 -0.141197860240936 11.0
0 0.0789011120796204 12.0
68.725793606 66.9772186279297 13.0
32.966682747 28.9061489105225 14.0
0 -0.0153014659881592 15.0
0 0.0643385052680969 16.0
35.656370849 37.3561706542969 17.0
0 0.157763540744781 18.0
0 -0.0649176836013794 19.0
0 0.24440610408783 20.0
0 -0.0215810537338257 21.0
0 0.124187469482422 22.0
39.390824651 36.6876258850098 23.0
0 0.0772793292999268 24.0
0 0.751126646995544 25.0
0 0.0659072399139404 26.0
0 0.0591091513633728 27.0
0 0.136499643325806 28.0
0 -0.164667844772339 29.0
0 0.130518436431885 30.0
0 0.0608960390090942 31.0
0 0.0984764695167542 32.0
0 -0.0390675663948059 33.0
0 0.309426099061966 34.0
0 0.291109591722488 35.0
0 0.396982938051224 36.0
0 -0.0764061808586121 37.0
0 0.0425844788551331 38.0
0 -0.571399331092834 39.0
0 0.754441916942596 40.0
0 -0.305568814277649 41.0
0 0.236953139305115 42.0
297.47305988 294.936767578125 43.0
0 0.218911826610565 44.0
0 0.118061065673828 45.0
0 0.128889977931976 46.0
0 -0.288633942604065 47.0
0 0.109809219837189 48.0
0 0.272990167140961 49.0
0.085130585563 0.0846289992332458 50.0
2.1527465318 1.77696907520294 51.0
0 0.0879872441291809 52.0
0 -0.0748249292373657 53.0
0 -0.114552736282349 54.0
0 0.0895298719406128 55.0
0 -0.0390632748603821 56.0
0 0.180050849914551 57.0
0 -0.035222053527832 58.0
0 0.113559484481812 59.0
0 -0.0313254594802856 60.0
0 0.0557392835617065 61.0
0 0.0716298222541809 62.0
230.6589112 252.761245727539 0.0
0 -0.172111868858337 1.0
0 -0.0196605324745178 2.0
0 -0.207444489002228 3.0
0 -0.397061347961426 4.0
0 -0.322706460952759 5.0
0 0.116474211215973 6.0
0 -0.22734010219574 7.0
0 -0.01983243227005 8.0
0 -0.24459981918335 9.0
0 0.160591065883636 10.0
0 -0.0394223928451538 11.0
0 -0.247023820877075 12.0
52.067495891 53.6673545837402 13.0
40.420871628 42.8684959411621 14.0
0 -0.0487974882125854 15.0
0 0.0077856183052063 16.0
29.560458525 32.2193870544434 17.0
0 0.0865659713745117 18.0
0 0.0372611284255981 19.0
0 0.126457571983337 20.0
0 -0.00491470098495483 21.0
0 -0.0812473297119141 22.0
41.093404166 40.4765777587891 23.0
0 -0.00369811058044434 24.0
0 0.574323713779449 25.0
0 -0.0545386075973511 26.0
0 -0.0433038473129272 27.0
0 -0.178047716617584 28.0
0 -0.157048285007477 29.0
0 0.04487144947052 30.0
0 -0.120118498802185 31.0
0 -0.108777225017548 32.0
0 -0.214691877365112 33.0
0 0.140508711338043 34.0
0 -0.125872671604156 35.0
0 0.487671881914139 36.0
0 -0.0123438239097595 37.0
0 -0.00168019533157349 38.0
0 -0.305819869041443 39.0
0 0.62275630235672 40.0
0 -0.109245836734772 41.0
0 0.0231599807739258 42.0
292.01538521 295.487823486328 43.0
0 0.0625333189964294 44.0
0 0.199792563915253 45.0
0 0.213831067085266 46.0
0 0.0474620461463928 47.0
0 0.0711484551429749 48.0
0.18665609771 0.378397971391678 49.0
0.10455569011 0.0799720287322998 50.0
1.4587178627 1.84886860847473 51.0
0 0.415830731391907 52.0
0 -0.263829708099365 53.0
0 -0.176921308040619 54.0
0 -0.112928330898285 55.0
0 0.371828496456146 56.0
0 0.163265705108643 57.0
0 0.0117231607437134 58.0
0 0.360401839017868 59.0
0 -0.150113880634308 60.0
0 0.122355699539185 61.0
0 0.225623369216919 62.0
256.91643352 252.288787841797 0.0
0 -0.23672878742218 1.0
0 0.140532314777374 2.0
0 0.270744442939758 3.0
0 -0.179448843002319 4.0
0 -0.0227769613265991 5.0
0 -0.022338330745697 6.0
0 0.277201592922211 7.0
0 -0.424723625183105 8.0
0 0.266110777854919 9.0
0 -0.134684562683105 10.0
0 -0.227657914161682 11.0
0 -0.00498861074447632 12.0
54.374880093 52.5185279846191 13.0
34.997839413 45.2326850891113 14.0
0 0.0135091543197632 15.0
0 -0.029990017414093 16.0
32.398470798 34.3967742919922 17.0
0 0.0776504278182983 18.0
0 0.00528711080551147 19.0
0 0.0848630666732788 20.0
0 -0.00553786754608154 21.0
0 0.298345685005188 22.0
34.515914595 35.4476852416992 23.0
0 0.15634822845459 24.0
0 1.1470034122467 25.0
0 -0.159619212150574 26.0
0 0.0653033852577209 27.0
0 0.00111031532287598 28.0
0 -0.0177906155586243 29.0
0 0.100236237049103 30.0
0 0.111849665641785 31.0
0 0.143196284770966 32.0
0 -0.168674886226654 33.0
0 0.22631847858429 34.0
0 0.323873043060303 35.0
0 -0.194957911968231 36.0
0 -0.0381292104721069 37.0
0 0.13429456949234 38.0
0 -0.208040952682495 39.0
0 -0.167689263820648 40.0
0 -0.260471701622009 41.0
0 0.032126784324646 42.0
300.16829508 297.177673339844 43.0
0 0.171481609344482 44.0
0 0.0419343113899231 45.0
0 0.173145651817322 46.0
0 0.209672093391418 47.0
0 0.0387060642242432 48.0
0 0.371983915567398 49.0
0.095201597017 0.0543923377990723 50.0
2.1267604477 1.84548664093018 51.0
0 0.181166112422943 52.0
0 0.0658960342407227 53.0
0 -0.0190899968147278 54.0
0 0.235625445842743 55.0
0 0.276083052158356 56.0
0 0.338966131210327 57.0
0 0.507790744304657 58.0
0 0.114580154418945 59.0
0 0.0970538854598999 60.0
0 0.163467288017273 61.0
0 0.297931313514709 62.0
244.15121697 253.779342651367 0.0
0 -0.191857576370239 1.0
0 0.0126957297325134 2.0
0 0.353431582450867 3.0
0 -0.255236268043518 4.0
0 0.0857225656509399 5.0
0 0.0775444507598877 6.0
0 0.000410556793212891 7.0
0 0.0320485234260559 8.0
0 0.16903555393219 9.0
0 0.152003705501556 10.0
0 0.00256466865539551 11.0
0 -0.00677698850631714 12.0
57.201152815 55.0555191040039 13.0
36.618320941 42.5290222167969 14.0
0 -0.0536820888519287 15.0
0 -0.0853596329689026 16.0
35.737651117 34.2528343200684 17.0
0 0.144550919532776 18.0
0 -0.00943607091903687 19.0
0 0.105862975120544 20.0
0 0.0561245083808899 21.0
0 0.245101451873779 22.0
33.328296936 31.9489974975586 23.0
0 0.0729274153709412 24.0
0 1.01607143878937 25.0
0 -0.0661120414733887 26.0
0 -0.0909614562988281 27.0
0 0.144983768463135 28.0
0 -0.0303874015808105 29.0
0 0.0353284478187561 30.0
0 0.124158024787903 31.0
0 -0.141183137893677 32.0
0 -0.202069878578186 33.0
0 0.157541275024414 34.0
0 -0.0683684945106506 35.0
0 0.341214656829834 36.0
0 0.183280467987061 37.0
0 -0.0522204637527466 38.0
0 -0.0937830209732056 39.0
0 0.483772307634354 40.0
0 -0.121804893016815 41.0
0 -0.034480094909668 42.0
290.09975398 295.968566894531 43.0
0 0.19139564037323 44.0
0 -0.0769156217575073 45.0
0 0.116809487342834 46.0
0 0.254025340080261 47.0
0 0.15769761800766 48.0
0 0.274481415748596 49.0
0.086474966022 0.00726664066314697 50.0
2.2054620147 1.8819887638092 51.0
0 0.0280015468597412 52.0
0 -0.0852227210998535 53.0
0 0.0591064691543579 54.0
0 0.24719774723053 55.0
0 0.273959994316101 56.0
0 -0.0739344954490662 57.0
0 0.0893077254295349 58.0
0 0.19593334197998 59.0
0 -0.13431704044342 60.0
0 0.137849807739258 61.0
0 0.305791735649109 62.0
267.84318832 253.357604980469 0.0
0 -0.236912727355957 1.0
0 -0.224244832992554 2.0
0 0.624188542366028 3.0
0 -0.155660390853882 4.0
0 -0.217674970626831 5.0
0 -0.103081285953522 6.0
0 0.0576386451721191 7.0
0 -0.467221975326538 8.0
0 -0.333067655563354 9.0
0 -0.00467443466186523 10.0
0 0.0382381081581116 11.0
0 -0.184137344360352 12.0
57.727400957 59.1739807128906 13.0
41.444529192 34.2642974853516 14.0
0 -0.160828709602356 15.0
0 -0.139746963977814 16.0
33.959410074 32.0018348693848 17.0
0 0.061087965965271 18.0
0 0.0667951107025146 19.0
0 0.0581424832344055 20.0
0 0.249921083450317 21.0
0 -0.162087202072144 22.0
26.137877313 25.0198364257812 23.0
0 0.146909117698669 24.0
0 1.16285693645477 25.0
0 0.184231042861938 26.0
0 -0.35052490234375 27.0
0 0.510840356349945 28.0
0 -0.131335616111755 29.0
0 0.0232478976249695 30.0
0 -0.155145168304443 31.0
0 -0.0552114248275757 32.0
0 -0.181897103786469 33.0
0 -2.07759094238281 34.0
0 0.168461918830872 35.0
0 0.153870463371277 36.0
0 0.207883715629578 37.0
0 0.05781489610672 38.0
0 -0.109941005706787 39.0
0 0.442140758037567 40.0
0 -0.206357717514038 41.0
0 0.0432453155517578 42.0
314.75526169 295.981597900391 43.0
0 0.0465734004974365 44.0
0 0.102080047130585 45.0
0 0.251720428466797 46.0
0 -0.274507522583008 47.0
0 -0.697397351264954 48.0
0 0.452589333057404 49.0
0.070525096769 -0.0585551857948303 50.0
2.1869748781 1.86937165260315 51.0
0 -0.0872387886047363 52.0
0 -0.0431170463562012 53.0
0 0.17384797334671 54.0
0 -0.0314898490905762 55.0
0 -0.03876131772995 56.0
0 0.0379244089126587 57.0
0 0.128370761871338 58.0
0 0.015641987323761 59.0
0 0.504065871238708 60.0
0 0.0481287240982056 61.0
0 0.151225864887238 62.0
289.26783722 253.104034423828 0.0
0 -0.303835511207581 1.0
0 -0.04271399974823 2.0
0 0.0123928785324097 3.0
0 -0.117268681526184 4.0
0 0.181213855743408 5.0
0 0.172039270401001 6.0
0 0.359358847141266 7.0
0 0.244122207164764 8.0
0 -0.0161523222923279 9.0
0 -0.0868453979492188 10.0
0 -0.290545225143433 11.0
0 -0.183794617652893 12.0
52.878613904 51.9805297851562 13.0
38.986217854 41.1285400390625 14.0
0 -0.0187984704971313 15.0
0 -0.0811601281166077 16.0
30.489808975 32.8856468200684 17.0
0 0.0638130307197571 18.0
0 -0.00728732347488403 19.0
0 0.151808202266693 20.0
0 0.142797410488129 21.0
0 0.0615330338478088 22.0
28.77247142 28.9109592437744 23.0
0 0.123845756053925 24.0
0 1.15844941139221 25.0
0 -0.059834361076355 26.0
0 -0.0137452483177185 27.0
0 0.406744986772537 28.0
0 0.00400769710540771 29.0
0 0.0763697028160095 30.0
0 -0.0132945775985718 31.0
0 0.110229969024658 32.0
0 -0.00211924314498901 33.0
0 0.359153985977173 34.0
0 -0.0852853059768677 35.0
0 -0.0491354465484619 36.0
0 -0.00718188285827637 37.0
0 0.147949457168579 38.0
0 -0.190182209014893 39.0
0 0.450352668762207 40.0
0 -0.117688894271851 41.0
0 -0.00778758525848389 42.0
297.18569359 296.117034912109 43.0
0 0.00100547075271606 44.0
0 0.127605736255646 45.0
0 0.117294788360596 46.0
0 -0.107762038707733 47.0
0 0.21248459815979 48.0
0.28612007582 0.345450699329376 49.0
0.072037425039 0.058617889881134 50.0
2.0593312865 1.8924732208252 51.0
0 0.197769999504089 52.0
0 -0.147355914115906 53.0
0 -0.0914291143417358 54.0
0 -0.15550172328949 55.0
0 -0.0438606739044189 56.0
0 -0.0124121904373169 57.0
0 0.314126640558243 58.0
0 0.101469933986664 59.0
0 -0.115704596042633 60.0
0 0.0365656614303589 61.0
0 0.073137104511261 62.0
233.97170705 254.999038696289 0.0
0 -0.281424641609192 1.0
0 -0.0379081964492798 2.0
0 0.0334866046905518 3.0
0 -0.29108726978302 4.0
0 -0.0647867321968079 5.0
0 0.235637187957764 6.0
0 -0.0709567070007324 7.0
0 -0.0257554054260254 8.0
0 0.247259497642517 9.0
0 -0.137972056865692 10.0
0 -0.130735337734222 11.0
0 -0.154220223426819 12.0
51.823540481 51.7795906066895 13.0
40.010832852 42.012939453125 14.0
0 -0.0438665151596069 15.0
0 0.460908234119415 16.0
40.487353553 36.791576385498 17.0
0 0.218720674514771 18.0
0 0.0487974882125854 19.0
0 -0.0527297258377075 20.0
0 0.274356007575989 21.0
0 -0.135457992553711 22.0
26.191972839 25.5902824401855 23.0
0 0.110114395618439 24.0
0 1.09634506702423 25.0
0 0.0492491126060486 26.0
0 -0.337667226791382 27.0
0 -0.238026022911072 28.0
0 0.134035170078278 29.0
0 0.0698458552360535 30.0
0 0.276791214942932 31.0
0 -0.0513927936553955 32.0
0 0.000889301300048828 33.0
0 0.447769343852997 34.0
0 -0.00478929281234741 35.0
0 -0.162988662719727 36.0
0 0.108272552490234 37.0
0 0.251030802726746 38.0
0 -0.386499643325806 39.0
0 -0.254836678504944 40.0
0 -0.269121289253235 41.0
0 0.327495574951172 42.0
290.76195089 299.045654296875 43.0
0 -0.105543911457062 44.0
0 0.28538054227829 45.0
0 0.353332102298737 46.0
0 -0.467460751533508 47.0
0 -0.0300180912017822 48.0
0 0.438195258378983 49.0
0.10372269847 -0.00570368766784668 50.0
1.5850435502 1.91451573371887 51.0
0 -0.208296716213226 52.0
0 -0.580688953399658 53.0
0 0.114749193191528 54.0
0 -0.0556447505950928 55.0
0 0.0340391993522644 56.0
0 0.203281700611115 57.0
0 0.288519561290741 58.0
0 0.286275148391724 59.0
0 -0.240395665168762 60.0
0 0.130142569541931 61.0
0 0.363796770572662 62.0
267.59447758 254.320205688477 0.0
0 -0.227240800857544 1.0
0 -0.00876933336257935 2.0
0 0.245726466178894 3.0
0 0.14171439409256 4.0
0 0.251747131347656 5.0
0 -0.00886136293411255 6.0
0 0.351576268672943 7.0
0 0.136250734329224 8.0
0 0.460146903991699 9.0
0 0.0695705413818359 10.0
0 -0.025174617767334 11.0
0 -0.0725256204605103 12.0
59.547623861 60.609920501709 13.0
41.228472187 37.0610504150391 14.0
0 -0.0825839042663574 15.0
0 -0.0300940871238708 16.0
36.228725719 35.997200012207 17.0
0 0.13109302520752 18.0
0 -0.0131655931472778 19.0
0 0.0903539061546326 20.0
0 0.164030253887177 21.0
0 0.16108250617981 22.0
32.949529401 32.9029731750488 23.0
0 0.0552269220352173 24.0
0 0.83141565322876 25.0
0 0.104135632514954 26.0
0 -0.0763933062553406 27.0
0 -0.132560312747955 28.0
0 -0.196120500564575 29.0
0 0.0509931445121765 30.0
0 0.0336745977401733 31.0
0 -0.0385048389434814 32.0
0 -0.0492596030235291 33.0
0 -0.303083419799805 34.0
0 0.359137237071991 35.0
0 0.320047557353973 36.0
0 -0.056784987449646 37.0
0 0.499493718147278 38.0
0 -0.273354172706604 39.0
0 -0.44101870059967 40.0
0 -0.197612583637238 41.0
0 0.053552508354187 42.0
303.55499756 295.630859375 43.0
0 0.0871527194976807 44.0
0 0.15434855222702 45.0
0 0.230440616607666 46.0
0 -0.0782358050346375 47.0
0 0.372441053390503 48.0
0 0.42687913775444 49.0
0.083083736066 0.0385058522224426 50.0
2.1818918748 1.88925647735596 51.0
0 -0.200454235076904 52.0
0 -0.0430861711502075 53.0
0 0.0317856073379517 54.0
0 0.220635116100311 55.0
0 -0.0367562770843506 56.0
0 0.381442725658417 57.0
0 -0.0126794576644897 58.0
0 0.160534977912903 59.0
0 -0.178196728229523 60.0
0 0.0496693849563599 61.0
0 0.126896858215332 62.0
268.15967924 254.050643920898 0.0
0 0.243772745132446 1.0
0 -0.0461697578430176 2.0
0 0.308312803506851 3.0
0 -0.0629391670227051 4.0
0 0.106268405914307 5.0
0 -0.0763461589813232 6.0
0 0.435039460659027 7.0
0 0.127548575401306 8.0
0 0.111351251602173 9.0
0 -0.145016431808472 10.0
0 -0.229180335998535 11.0
0 -0.111464321613312 12.0
67.023692763 65.898078918457 13.0
35.795601632 34.4970321655273 14.0
0 -0.0971572399139404 15.0
0 0.0967478156089783 16.0
36.646952669 32.5407295227051 17.0
0 0.10678493976593 18.0
0 0.0907555818557739 19.0
0 0.111373603343964 20.0
0 0.0858447551727295 21.0
0 0.108637154102325 22.0
36.47325566 34.6769752502441 23.0
0 0.0446993708610535 24.0
0 0.937792599201202 25.0
0 -0.15828549861908 26.0
0 0.00544857978820801 27.0
0 0.319486647844315 28.0
0 -0.0185573101043701 29.0
0 -0.00102430582046509 30.0
0 -0.0718622207641602 31.0
0 -0.0716167688369751 32.0
0 -0.183271706104279 33.0
0 0.396231412887573 34.0
0 -0.147946059703827 35.0
0 0.335025370121002 36.0
0 -0.0266093611717224 37.0
0 0.218901753425598 38.0
0 -0.30548894405365 39.0
0 -0.117852926254272 40.0
0 -0.213492870330811 41.0
0 0.0152262449264526 42.0
304.64355556 297.318481445312 43.0
0 0.0968436002731323 44.0
0 0.0866497755050659 45.0
0 0.0527437925338745 46.0
0 -0.0291772484779358 47.0
0 -0.374579668045044 48.0
0 0.4045090675354 49.0
0.068784346755 -0.0209969878196716 50.0
1.6468977015 1.86365032196045 51.0
0 0.385076105594635 52.0
0 -0.0295448303222656 53.0
0 0.0145895481109619 54.0
0 -0.0929602980613708 55.0
0 -0.136583089828491 56.0
0 0.380820006132126 57.0
0 0.410592079162598 58.0
0 0.189272522926331 59.0
0 0.0366065502166748 60.0
0 0.127880096435547 61.0
0 0.330107033252716 62.0
250.91901102 253.270584106445 0.0
0 -0.263288736343384 1.0
0 0.0228673815727234 2.0
0 0.374617099761963 3.0
0 -0.127341866493225 4.0
0 -0.0201531052589417 5.0
0 0.0186008810997009 6.0
0 -0.184717059135437 7.0
0 -0.321563363075256 8.0
0 0.440701484680176 9.0
0 -0.145324051380157 10.0
0 -0.317313551902771 11.0
0 0.0286033749580383 12.0
64.888911081 64.8137512207031 13.0
23.468201469 35.4887046813965 14.0
0 -0.0967297554016113 15.0
0 -0.0645270943641663 16.0
31.837037613 36.3759956359863 17.0
0 0.106743276119232 18.0
0 -0.0116504430770874 19.0
0 0.0522588491439819 20.0
0 0.201161026954651 21.0
0 -0.0359429121017456 22.0
27.739125931 26.0323867797852 23.0
0 0.0893344283103943 24.0
0 0.759074449539185 25.0
0 -0.0611668825149536 26.0
0 -0.0347164273262024 27.0
0 0.0392733812332153 28.0
0 -0.0660587549209595 29.0
0 -0.194375693798065 30.0
0 0.257603287696838 31.0
0 0.0909072160720825 32.0
0 0.113083183765411 33.0
0 0.277305543422699 34.0
0 -0.150887668132782 35.0
0 0.130591154098511 36.0
0 0.00403523445129395 37.0
0 0.123940050601959 38.0
0 -0.220758199691772 39.0
0 0.0252499580383301 40.0
0 -0.206813335418701 41.0
0 0.137752652168274 42.0
287.88521548 297.368347167969 43.0
0 0.183744192123413 44.0
0 0.148781776428223 45.0
0 0.357187509536743 46.0
0 -0.346608519554138 47.0
0 -0.759338140487671 48.0
0 0.567589402198792 49.0
0.10046419202 -0.0288988947868347 50.0
2.2455310135 1.84178674221039 51.0
0 0.154300332069397 52.0
0 -0.0704464912414551 53.0
0 -0.147021412849426 54.0
0 -0.0979283452033997 55.0
0 -0.0136356949806213 56.0
0 -0.0983072519302368 57.0
0 0.316428035497665 58.0
0 0.283107876777649 59.0
0 -0.112377524375916 60.0
0 0.069408655166626 61.0
0 0.240177512168884 62.0
234.64110996 250.604553222656 0.0
0 0.0539648532867432 1.0
0 -0.188584089279175 2.0
0 0.272032976150513 3.0
0 -0.247158288955688 4.0
0 0.134508311748505 5.0
0 -0.00326943397521973 6.0
0 -0.267183303833008 7.0
0 0.157999038696289 8.0
0 0.159829616546631 9.0
0 0.230474054813385 10.0
0 -0.159639000892639 11.0
0 -0.0785646438598633 12.0
62.498959374 62.6320457458496 13.0
26.562161582 36.0098114013672 14.0
0 -0.153529524803162 15.0
0 0.0279550552368164 16.0
37.924949753 38.0755233764648 17.0
0 0.164140343666077 18.0
0 -0.0307469367980957 19.0
0 0.0278365015983582 20.0
0 0.0424054265022278 21.0
0 0.254908561706543 22.0
32.203883507 32.8402290344238 23.0
0 0.116175413131714 24.0
0 1.18289542198181 25.0
0 -0.0282641053199768 26.0
0 -0.0824390053749084 27.0
0 0.508431375026703 28.0
0 0.0538238883018494 29.0
0 0.0309942364692688 30.0
0 -0.0183333158493042 31.0
0 0.0570725798606873 32.0
0 0.0242304801940918 33.0
0 0.0787400007247925 34.0
0 0.138896524906158 35.0
0 -0.630427360534668 36.0
0 0.167414128780365 37.0
0 -0.106416940689087 38.0
0 -0.192432761192322 39.0
0 0.336106359958649 40.0
0 -0.134988069534302 41.0
0 -0.0647077560424805 42.0
289.35397848 296.843078613281 43.0
0 0.199939727783203 44.0
0 0.146551966667175 45.0
0 0.205689787864685 46.0
0 -0.0309638381004333 47.0
0 -0.282987952232361 48.0
0 0.467530310153961 49.0
0.079800267759 -0.00940608978271484 50.0
2.0561708833 1.88246321678162 51.0
0 -0.0706961154937744 52.0
0 -0.0539919137954712 53.0
0 -0.0596280097961426 54.0
0 0.05925452709198 55.0
0 -0.0126369595527649 56.0
0 0.13631397485733 57.0
0 0.338199019432068 58.0
0 -0.0823884606361389 59.0
0 -0.169243693351746 60.0
0 0.0463203191757202 61.0
0 0.201204180717468 62.0
239.90514893 251.542785644531 0.0
0 0.0762064456939697 1.0
0 0.0213574767112732 2.0
0 0.394135117530823 3.0
0 -0.111165642738342 4.0
0 -0.254094243049622 5.0
0 0.0572807788848877 6.0
0 -0.113614439964294 7.0
0 0.152329802513123 8.0
0 0.366732716560364 9.0
0 0.162741005420685 10.0
0 -0.0421613454818726 11.0
0 0.0713005065917969 12.0
52.337781603 54.6048774719238 13.0
46.169391446 44.4933395385742 14.0
0 -0.0590879917144775 15.0
0 0.0559152364730835 16.0
40.0065843 35.9954490661621 17.0
0 0.134481191635132 18.0
0 -0.0121711492538452 19.0
0 0.0225611925125122 20.0
0 0.140469372272491 21.0
0 0.0443641543388367 22.0
29.790150676 29.2050285339355 23.0
0 0.0668894648551941 24.0
0 0.781171202659607 25.0
0 -0.0704858303070068 26.0
0 -0.0329563021659851 27.0
0 0.00541293621063232 28.0
0 -0.0320930480957031 29.0
0 0.0240499973297119 30.0
0 0.317834734916687 31.0
0 0.0295635461807251 32.0
0 -0.163663923740387 33.0
0 0.281827747821808 34.0
0 0.232533276081085 35.0
0 0.342658638954163 36.0
0 -0.0170091986656189 37.0
0 0.120265603065491 38.0
0 -0.284865379333496 39.0
0 0.630359470844269 40.0
0 -0.12802255153656 41.0
0 0.0851747989654541 42.0
290.9599611 296.371154785156 43.0
0 0.157591819763184 44.0
0 -0.0271109342575073 45.0
0 0.0466268658638 46.0
0 -0.00545370578765869 47.0
0 0.408107101917267 48.0
0 0.460243046283722 49.0
0.078253458353 0.0774767398834229 50.0
1.571422133 1.85062658786774 51.0
0 0.147541284561157 52.0
0 0.105249643325806 53.0
0 -0.23058295249939 54.0
0 -0.173269927501678 55.0
0 0.190146446228027 56.0
0 0.230126619338989 57.0
0 0.233243763446808 58.0
0 0.0278955698013306 59.0
0 -0.234797954559326 60.0
0 0.0111042857170105 61.0
0 0.112282156944275 62.0
266.26660172 254.369995117188 0.0
0 -0.355921149253845 1.0
0 -0.17378181219101 2.0
0 0.499702870845795 3.0
0 -0.127732217311859 4.0
0 -0.0542818903923035 5.0
0 -0.0804902911186218 6.0
0 0.0523877143859863 7.0
0 -0.117984056472778 8.0
0 -0.239299297332764 9.0
0 0.447896033525467 10.0
0 0.0570200085639954 11.0
0 -0.16444456577301 12.0
68.044549987 70.4434585571289 13.0
36.453784743 28.1470031738281 14.0
0 -0.0893485546112061 15.0
0 -0.100115478038788 16.0
27.876166939 31.5750598907471 17.0
0 0.0496909618377686 18.0
0 0.00700867176055908 19.0
0 0.046261191368103 20.0
0 0.276199221611023 21.0
0 0.000927329063415527 22.0
30.561186441 28.7125930786133 23.0
0 0.0993177890777588 24.0
0 0.909366250038147 25.0
0 -0.013984739780426 26.0
0 -0.0833299160003662 27.0
0 0.493709355592728 28.0
0 0.0999419093132019 29.0
0 -0.120137453079224 30.0
0 -0.0452468395233154 31.0
0 0.206275463104248 32.0
0 0.0163856148719788 33.0
0 -0.0404160022735596 34.0
0 0.125343978404999 35.0
0 -0.0629541873931885 36.0
0 0.0700774192810059 37.0
0 0.16724419593811 38.0
0 -0.242219924926758 39.0
0 0.637985825538635 40.0
0 -0.11652135848999 41.0
0 -0.0281931161880493 42.0
290.74070957 297.894927978516 43.0
0 0.0846290588378906 44.0
0 0.050275444984436 45.0
0 -0.020833432674408 46.0
0 0.110970795154572 47.0
0 -0.413013458251953 48.0
0 0.443416476249695 49.0
0.089725810217 0.0103120803833008 50.0
1.5642982562 1.82212793827057 51.0
0 -0.160597681999207 52.0
0 0.199619710445404 53.0
0 0.208633184432983 54.0
0 0.271910846233368 55.0
0 -0.00313633680343628 56.0
0 0.0142793655395508 57.0
0 0.458887904882431 58.0
0 0.299123972654343 59.0
0 -0.232051372528076 60.0
0 0.0608087778091431 61.0
0 0.0860077738761902 62.0
228.80937746 252.277191162109 0.0
0 -0.122795820236206 1.0
0 -0.0194260478019714 2.0
0 0.253262341022491 3.0
0 -0.228142261505127 4.0
0 -0.252718210220337 5.0
0 0.056441068649292 6.0
0 -0.19813084602356 7.0
0 0.0188541412353516 8.0
0 0.337445318698883 9.0
0 0.527562499046326 10.0
0 -0.279016971588135 11.0
0 -0.0894087553024292 12.0
55.28209034 54.3787384033203 13.0
43.691123642 38.8970069885254 14.0
0 0.0673589706420898 15.0
0 0.0542219877243042 16.0
31.536494247 33.2061462402344 17.0
0 0.125525414943695 18.0
0 -0.0216828584671021 19.0
0 0.14713329076767 20.0
0 -0.00301849842071533 21.0
0 0.00931006669998169 22.0
40.151034909 37.9265518188477 23.0
0 0.0882846713066101 24.0
0 0.466088593006134 25.0
0 -0.0366804599761963 26.0
0 0.111167371273041 27.0
0 0.975264549255371 28.0
0 0.0467318296432495 29.0
0 -0.0433151721954346 30.0
0 -0.124690532684326 31.0
0 -0.0830857753753662 32.0
0 -0.0765253305435181 33.0
0 -0.146940469741821 34.0
0 0.0680069327354431 35.0
0 0.0464587211608887 36.0
0 0.015520453453064 37.0
0 0.139313220977783 38.0
0 -0.226253867149353 39.0
0 0.426460772752762 40.0
0 -0.179012417793274 41.0
0 0.0618432760238647 42.0
298.01936031 295.973175048828 43.0
0 0.0512464046478271 44.0
0 -0.0170588493347168 45.0
0 0.121448636054993 46.0
0 -0.23122239112854 47.0
0 0.380947142839432 48.0
0 0.299896657466888 49.0
0.091773301616 0.0871516466140747 50.0
2.1862461828 1.83156824111938 51.0
0 0.0207537412643433 52.0
0 -0.218941569328308 53.0
0 0.281607031822205 54.0
0 0.193076133728027 55.0
0 -0.143238306045532 56.0
0 0.0903594493865967 57.0
0 0.0536412000656128 58.0
0 -0.0228142738342285 59.0
0 -0.0449097156524658 60.0
0 0.080729603767395 61.0
0 0.139877617359161 62.0
294.17577292 254.17561340332 0.0
0 -0.445650339126587 1.0
0 0.0661327242851257 2.0
0 -0.755279302597046 3.0
0 -0.156348347663879 4.0
0 0.422748208045959 5.0
0 0.83325731754303 6.0
0 0.45695972442627 7.0
0 0.393376141786575 8.0
0 0.47585666179657 9.0
0 0.102180898189545 10.0
0 -0.0941972732543945 11.0
0 0.085147500038147 12.0
53.643704598 52.1945648193359 13.0
40.391699893 35.7245979309082 14.0
0 -0.0769071578979492 15.0
0 0.141463041305542 16.0
37.112968431 38.1282157897949 17.0
0 0.178369164466858 18.0
0 -0.0809862613677979 19.0
0 0.098842978477478 20.0
0 0.108829498291016 21.0
0 0.176194727420807 22.0
34.292670167 32.3156585693359 23.0
0 0.0829077959060669 24.0
0 0.497521072626114 25.0
0 -0.0945922136306763 26.0
0 -0.174170553684235 27.0
0 0.548760890960693 28.0
0 0.116612553596497 29.0
0 0.0670679807662964 30.0
0 -0.310507893562317 31.0
0 -0.00380265712738037 32.0
0 -0.0976348519325256 33.0
0 -0.00748711824417114 34.0
0 0.668116509914398 35.0
0 0.361054807901382 36.0
0 0.113471567630768 37.0
0 0.0254516005516052 38.0
0 -0.228124856948853 39.0
0 0.674896836280823 40.0
0 -0.0834148526191711 41.0
0 0.0949341058731079 42.0
299.79970784 297.487060546875 43.0
0 -0.0365628004074097 44.0
0 -0.722660899162292 45.0
0 0.254534482955933 46.0
0 -0.179450035095215 47.0
0 0.227143228054047 48.0
0 0.24952495098114 49.0
0.068864734936 0.0715401768684387 50.0
1.553257107 1.84138917922974 51.0
0 0.134513199329376 52.0
0 -0.211489081382751 53.0
0 0.266229867935181 54.0
0 0.180170893669128 55.0
0 0.0683898329734802 56.0
0 0.601617097854614 57.0
0 0.176154315471649 58.0
0 0.270766913890839 59.0
0 -0.266841411590576 60.0
0 0.0658685564994812 61.0
0 0.294773161411285 62.0
289.36849264 253.579513549805 0.0
0 0.140289545059204 1.0
0 -0.0337680578231812 2.0
0 -0.0308317542076111 3.0
0 -0.403521299362183 4.0
0 0.343780040740967 5.0
0 -0.167591392993927 6.0
0 0.466947019100189 7.0
0 0.536605954170227 8.0
0 0.17184054851532 9.0
0 -0.15794974565506 10.0
0 -0.315802335739136 11.0
0 -0.111228346824646 12.0
75.334707876 73.9435119628906 13.0
12.044067138 25.742166519165 14.0
0 -0.113810896873474 15.0
0 -0.00218468904495239 16.0
26.981781413 32.8595657348633 17.0
0 0.100946605205536 18.0
0 0.0462979674339294 19.0
0 0.0894129276275635 20.0
0 0.163649678230286 21.0
0 -0.0847538709640503 22.0
36.238728744 35.2122077941895 23.0
0 0.080127477645874 24.0
0 0.970019638538361 25.0
0 0.155969977378845 26.0
0 -0.250619769096375 27.0
0 0.594322681427002 28.0
0 -0.0578429698944092 29.0
0 0.0937445759773254 30.0
0 0.173075914382935 31.0
0 0.0112963914871216 32.0
0 -0.1679967045784 33.0
0 0.270656526088715 34.0
0 0.267309844493866 35.0
0 0.193337500095367 36.0
0 0.0311474800109863 37.0
0 0.173701465129852 38.0
0 -0.176288843154907 39.0
0 0.537105023860931 40.0
0 -0.126586556434631 41.0
0 -0.00379931926727295 42.0
288.81512864 297.081298828125 43.0
0 0.110793352127075 44.0
0 0.122942447662354 45.0
0 0.323285818099976 46.0
0 -0.115150392055511 47.0
0 0.391070485115051 48.0
0 0.483523190021515 49.0
0.098769944511 0.0338299870491028 50.0
1.8146554525 1.89750671386719 51.0
0 0.0700084567070007 52.0
0 -0.0929782390594482 53.0
0 -0.0910301208496094 54.0
0 0.0303735733032227 55.0
0 -0.108421266078949 56.0
0 0.156988143920898 57.0
0 0.328674703836441 58.0
0 -0.0277745723724365 59.0
0 0.166252970695496 60.0
0 0.0791065692901611 61.0
0 0.261310636997223 62.0
248.86043466 252.867904663086 0.0
0 -0.163038492202759 1.0
0 0.0931180119514465 2.0
0 0.149224579334259 3.0
0 -0.00412392616271973 4.0
0 0.0726038217544556 5.0
0 0.00180017948150635 6.0
0 0.178374290466309 7.0
0 0.014556884765625 8.0
0 -0.161531150341034 9.0
0 -0.0715105533599854 10.0
0 -0.432571530342102 11.0
0 -0.145082950592041 12.0
74.073636845 73.2378463745117 13.0
14.612720101 26.6477642059326 14.0
0 -0.0577131509780884 15.0
0 -0.123237371444702 16.0
39.334098123 37.7868499755859 17.0
0 0.140240430831909 18.0
0 -0.0458710193634033 19.0
0 0.0216350555419922 20.0
0 0.159874260425568 21.0
0 0.320611178874969 22.0
32.577972335 32.6020278930664 23.0
0 0.0442529320716858 24.0
0 0.935984015464783 25.0
0 -0.0531740784645081 26.0
0 -0.195348918437958 27.0
0 0.416001439094543 28.0
0 0.0605422854423523 29.0
0 0.0720006227493286 30.0
0 -0.0725221633911133 31.0
0 -0.0969815254211426 32.0
0 -0.279186725616455 33.0
0 -0.01615971326828 34.0
0 -0.102659463882446 35.0
0 0.0906858444213867 36.0
0 0.179693281650543 37.0
0 -0.0650789141654968 38.0
0 -0.129491925239563 39.0
0 0.347424864768982 40.0
0 -0.193956315517426 41.0
0 0.0512295961380005 42.0
311.88450141 296.694244384766 43.0
0 0.0193800926208496 44.0
0 0.194133937358856 45.0
0 0.179555058479309 46.0
0 0.0737611651420593 47.0
0 0.301524817943573 48.0
1.0286332098 0.469674527645111 49.0
0.087048106312 0.0710986256599426 50.0
1.8618221487 1.86179304122925 51.0
0 0.106533646583557 52.0
0 -0.0246970653533936 53.0
0 -0.0807830691337585 54.0
0 0.117551445960999 55.0
0 0.200970888137817 56.0
0 -0.0230138301849365 57.0
0 0.228368520736694 58.0
0 0.162446737289429 59.0
0 -0.195796430110931 60.0
0 0.0524596571922302 61.0
0 0.141794264316559 62.0
215.81166283 250.886215209961 0.0
0 0.322360396385193 1.0
0 -0.0577037930488586 2.0
0 -0.153573513031006 3.0
0 -0.376192212104797 4.0
0 -0.123403787612915 5.0
0 -0.0391381978988647 6.0
0 -0.174559056758881 7.0
0 0.16313636302948 8.0
0 -0.339324712753296 9.0
0 -0.103066742420197 10.0
0 0.129321813583374 11.0
0 -0.0146268010139465 12.0
76.194331866 75.2270965576172 13.0
36.855815904 29.4352340698242 14.0
0 -0.0844082832336426 15.0
0 0.603609979152679 16.0
35.022344943 36.9938087463379 17.0
0 0.170720219612122 18.0
0 0.0306200981140137 19.0
0 -0.0271405577659607 20.0
0 0.206382989883423 21.0
0 0.0718348026275635 22.0
27.975272082 28.516996383667 23.0
0 0.0121751427650452 24.0
0 0.859963119029999 25.0
0 0.193518817424774 26.0
0 -0.239976167678833 27.0
0 -0.508886337280273 28.0
0 0.119954586029053 29.0
0 0.154297590255737 30.0
0 -0.112854599952698 31.0
0 -0.0651653409004211 32.0
0 -0.678208112716675 33.0
0 -0.419721722602844 34.0
0 0.0647484660148621 35.0
0 0.393287986516953 36.0
0 0.108292579650879 37.0
0 0.404828816652298 38.0
0 -0.262747883796692 39.0
0 -0.203090369701385 40.0
0 -0.209764003753662 41.0
0 0.0709871053695679 42.0
294.10007913 298.204010009766 43.0
0 0.0798071026802063 44.0
0 0.173260033130646 45.0
0 0.415246188640594 46.0
0 -0.577307224273682 47.0
0 -0.169094204902649 48.0
0 0.515377402305603 49.0
0.065448348059 -0.0389919877052307 50.0
1.7024858596 1.8675525188446 51.0
0 -0.169734418392181 52.0
0 -0.072394847869873 53.0
0 -0.0146178007125854 54.0
0 -0.00777131319046021 55.0
0 -0.0388206839561462 56.0
0 0.233652889728546 57.0
0 0.494003504514694 58.0
0 0.320636302232742 59.0
0 -0.150330185890198 60.0
0 0.187865555286407 61.0
0 0.267550230026245 62.0
281.5221094 253.716781616211 0.0
0 -0.262477040290833 1.0
0 0.110833466053009 2.0
0 0.146772563457489 3.0
0 0.0625125765800476 4.0
0 0.032049834728241 5.0
0 -0.13461971282959 6.0
0 0.465844959020615 7.0
0 0.59488433599472 8.0
0 0.00184041261672974 9.0
0 0.591709852218628 10.0
0 -0.380216240882874 11.0
0 -0.154692769050598 12.0
64.093319474 62.843994140625 13.0
27.029944624 33.041618347168 14.0
0 -0.0414533615112305 15.0
0 0.0388046503067017 16.0
29.937958665 33.3938026428223 17.0
0 0.0705505013465881 18.0
0 0.0136044025421143 19.0
0 0.118588030338287 20.0
0 0.0541146993637085 21.0
0 -0.00411504507064819 22.0
37.625776526 36.3676376342773 23.0
0 0.119853019714355 24.0
0 0.681594431400299 25.0
0 0.0177870392799377 26.0
0 0.00387561321258545 27.0
0 -0.0402123928070068 28.0
0 -0.104862868785858 29.0
0 0.137413740158081 30.0
0 -0.1277095079422 31.0
0 -0.0211076736450195 32.0
0 -0.130067348480225 33.0
0 0.378631889820099 34.0
0 -0.167852401733398 35.0
0 0.273196458816528 36.0
0 0.0556641221046448 37.0
0 -0.00022202730178833 38.0
0 -0.149795413017273 39.0
0 0.452941566705704 40.0
0 -0.118585705757141 41.0
0 0.0495685338973999 42.0
293.28736897 296.468414306641 43.0
0 0.109874844551086 44.0
0 0.0547292828559875 45.0
0 0.132188439369202 46.0
0 0.075712263584137 47.0
0 0.414951711893082 48.0
1.6012509281 0.468892395496368 49.0
0.08581378977 0.0818634629249573 50.0
1.7005494871 1.85865247249603 51.0
0 -0.0748100280761719 52.0
0 -0.0802071690559387 53.0
0 0.027248740196228 54.0
0 0.232909202575684 55.0
0 0.0645667314529419 56.0
0 0.140341341495514 57.0
0 0.0696009397506714 58.0
0 0.186163544654846 59.0
0 0.241546928882599 60.0
0 0.034570574760437 61.0
0 0.147149324417114 62.0
241.54487034 253.621170043945 0.0
0 -0.208361864089966 1.0
0 -0.169055342674255 2.0
0 0.397814929485321 3.0
0 -0.173778295516968 4.0
0 0.0347765684127808 5.0
0 -0.302199363708496 6.0
0 0.153170943260193 7.0
0 0.149602472782135 8.0
0 -0.0347414016723633 9.0
0 -0.00130897760391235 10.0
0 -0.197964310646057 11.0
0 0.0660592317581177 12.0
62.907480898 63.3823776245117 13.0
35.446862978 35.2904663085938 14.0
0 -0.166080832481384 15.0
0 0.138655543327332 16.0
40.864363228 37.7737159729004 17.0
0 0.0779871344566345 18.0
0 -0.0338705778121948 19.0
0 0.118919730186462 20.0
0 -0.0189696550369263 21.0
0 0.298474937677383 22.0
36.655961474 36.5094833374023 23.0
0 0.107155382633209 24.0
0 0.998470783233643 25.0
0 0.026429295539856 26.0
0 -0.213284969329834 27.0
0 0.226785182952881 28.0
0 0.0554038882255554 29.0
0 0.103278636932373 30.0
0 -0.185235619544983 31.0
0 0.034338116645813 32.0
0 0.106150448322296 33.0
0 -0.0702812671661377 34.0
0 0.0125948190689087 35.0
0 0.144979119300842 36.0
0 0.158627092838287 37.0
0 0.141071975231171 38.0
0 -0.238434433937073 39.0
0 0.598463475704193 40.0
0 -0.0973430871963501 41.0
0 0.632818222045898 42.0
289.87361607 294.034027099609 43.0
0 -0.0504183769226074 44.0
0 0.517010629177094 45.0
0 0.289883255958557 46.0
0 0.11235499382019 47.0
0 0.0361359715461731 48.0
0 0.361942708492279 49.0
0.063718342578 0.0285807251930237 50.0
2.2509597032 1.82854902744293 51.0
0 0.15556263923645 52.0
0 -0.156670570373535 53.0
0 0.107380449771881 54.0
0 0.293255239725113 55.0
0 0.347378015518188 56.0
0 -0.218987345695496 57.0
0 0.0542475581169128 58.0
0 0.0246097445487976 59.0
0 -0.0860215425491333 60.0
0 0.0205016136169434 61.0
0 0.181329607963562 62.0
269.24685086 252.136917114258 0.0
0 0.217897534370422 1.0
0 0.0223875045776367 2.0
0 0.38576528429985 3.0
0 0.00452733039855957 4.0
0 -0.00934284925460815 5.0
0 0.189852833747864 6.0
0 0.874270617961884 7.0
0 0.156789779663086 8.0
0 -0.162773370742798 9.0
0 -0.0106388330459595 10.0
0 0.470488041639328 11.0
0 -0.125568568706512 12.0
48.943975668 51.539192199707 13.0
35.301209358 39.6022491455078 14.0
0 -0.0900393724441528 15.0
0 0.217710494995117 16.0
40.580190546 39.3155632019043 17.0
0 0.292095720767975 18.0
0 0.025937020778656 19.0
0 0.054252028465271 20.0
0 -0.0594332218170166 21.0
0 0.366796731948853 22.0
41.398914492 39.5658340454102 23.0
0 0.141489863395691 24.0
0 0.703868508338928 25.0
0 -0.0523375868797302 26.0
0 0.117428064346313 27.0
0 0.752869069576263 28.0
0 0.17748099565506 29.0
0 0.134506225585938 30.0
0 0.120110869407654 31.0
0 0.14272540807724 32.0
0 -0.00769710540771484 33.0
0 0.419848024845123 34.0
0 0.00133496522903442 35.0
0 0.220920085906982 36.0
0 0.0759910345077515 37.0
0 0.411115944385529 38.0
0 -0.201541066169739 39.0
0 0.316934734582901 40.0
0 -0.22281014919281 41.0
0 -0.0239903926849365 42.0
293.34797824 295.933197021484 43.0
0 -0.100246250629425 44.0
0 0.0841201543807983 45.0
0 0.143136441707611 46.0
0 0.221847951412201 47.0
0 0.14622688293457 48.0
0 0.26842987537384 49.0
0.069019708511 0.000987410545349121 50.0
2.3446128202 1.8755145072937 51.0
0 0.23423045873642 52.0
0 0.131720185279846 53.0
0 -0.102601289749146 54.0
0 -0.19643247127533 55.0
0 -0.0925544500350952 56.0
0 0.0729610919952393 57.0
0 0.325964033603668 58.0
0 0.312429189682007 59.0
0 -0.186564981937408 60.0
0 0.104243993759155 61.0
0 0.363490045070648 62.0
219.03880523 252.505142211914 0.0
0 -0.0793116092681885 1.0
0 0.0152650475502014 2.0
0 0.12186473608017 3.0
0 -0.240081310272217 4.0
0 -0.154436528682709 5.0
0 0.0925765037536621 6.0
0 -0.36613667011261 7.0
0 -0.0936604142189026 8.0
0 0.81020176410675 9.0
0 0.150076389312744 10.0
0 -0.221966147422791 11.0
0 -0.214829683303833 12.0
65.768556534 65.7577056884766 13.0
20.281544045 29.3777027130127 14.0
0 -0.108867764472961 15.0
0 -0.114301860332489 16.0
38.421209201 35.6481742858887 17.0
0 0.0927711725234985 18.0
0 -0.0910698175430298 19.0
0 0.109587907791138 20.0
0 0.110298037528992 21.0
0 0.127184391021729 22.0
34.384870097 31.4308490753174 23.0
0 -0.0735478401184082 24.0
0 0.452507704496384 25.0
0 -0.163571774959564 26.0
0 -0.0361049771308899 27.0
0 -0.341283917427063 28.0
0 -0.0512412786483765 29.0
0 0.0368911623954773 30.0
0 -0.195531845092773 31.0
0 -0.0737295150756836 32.0
0 0.0333840847015381 33.0
0 0.240704655647278 34.0
0 -0.0506284236907959 35.0
0 0.043269157409668 36.0
0 0.0545664429664612 37.0
0 0.160548090934753 38.0
0 -0.38507878780365 39.0
0 0.340009808540344 40.0
0 -0.192446291446686 41.0
0 0.0381659269332886 42.0
300.77188498 298.167602539062 43.0
0 0.0861632823944092 44.0
0 0.162156760692596 45.0
0 0.0726824402809143 46.0
0 0.401809334754944 47.0
0 0.357064813375473 48.0
2.5255057647 0.208035230636597 49.0
0.071621489294 0.0114180445671082 50.0
2.0981095452 1.93391132354736 51.0
0 0.034643828868866 52.0
0 -0.120794773101807 53.0
0 -0.195680618286133 54.0
0 0.06312096118927 55.0
0 0.236379027366638 56.0
0 0.13147360086441 57.0
0 0.257265627384186 58.0
0 0.36163318157196 59.0
0 0.0135365724563599 60.0
0 0.0944874286651611 61.0
0 0.312172830104828 62.0
269.02673603 255.996627807617 0.0
0 -0.325876712799072 1.0
0 -0.0784163475036621 2.0
0 0.167719781398773 3.0
0 0.0814162492752075 4.0
0 0.252359747886658 5.0
0 -0.181554675102234 6.0
0 0.144043922424316 7.0
0 -0.175465404987335 8.0
0 0.105750560760498 9.0
0 -0.0936437249183655 10.0
0 -0.170944690704346 11.0
0 -0.222578287124634 12.0
61.629423576 60.4068183898926 13.0
43.934907988 34.1421356201172 14.0
0 -0.0779808759689331 15.0
0 -0.0566959977149963 16.0
40.108860822 38.2853393554688 17.0
0 0.160041630268097 18.0
0 -0.0471018552780151 19.0
0 0.0168901681900024 20.0
0 0.236342668533325 21.0
0 0.0538341999053955 22.0
29.108944914 28.326208114624 23.0
0 0.0332497358322144 24.0
0 0.394808948040009 25.0
0 0.0995370149612427 26.0
0 -0.306802868843079 27.0
0 -0.260868310928345 28.0
0 -0.0249676704406738 29.0
0 0.0534977912902832 30.0
0 0.115844011306763 31.0
0 0.0915818214416504 32.0
0 -0.253685235977173 33.0
0 0.420907735824585 34.0
0 0.221171677112579 35.0
0 0.387881308794022 36.0
0 0.0692707300186157 37.0
0 0.0524182319641113 38.0
0 -0.231455087661743 39.0
0 0.0298875570297241 40.0
0 -0.177482485771179 41.0
0 0.0963249206542969 42.0
298.52734435 294.557312011719 43.0
0 0.299532651901245 44.0
0 -0.00620168447494507 45.0
0 0.231211304664612 46.0
0 -0.0103834271430969 47.0
0 -0.325513362884521 48.0
0 0.468814730644226 49.0
0.081479054694 -0.00475633144378662 50.0
1.6282451454 1.8649297952652 51.0
0 0.267918765544891 52.0
0 0.0471267104148865 53.0
0 -0.148932337760925 54.0
0 0.261680722236633 55.0
0 0.266865968704224 56.0
0 0.100835025310516 57.0
0 0.106362879276276 58.0
0 -0.0277346968650818 59.0
0 -0.0322808027267456 60.0
0 0.0510574579238892 61.0
0 0.131038546562195 62.0
269.09940342 250.790023803711 0.0
0 -0.242988467216492 1.0
0 -0.0881057977676392 2.0
0 0.361335754394531 3.0
0 -0.378650903701782 4.0
0 -0.0129126310348511 5.0
0 0.0753142833709717 6.0
0 0.0874956846237183 7.0
0 -0.0915613174438477 8.0
0 -0.319764137268066 9.0
0 0.0646257400512695 10.0
0 -0.0271828174591064 11.0
0 -0.231977462768555 12.0
53.660829229 55.1137008666992 13.0
35.803408955 38.5157737731934 14.0
0 0.0210375785827637 15.0
0 0.390721321105957 16.0
33.267401752 36.2640647888184 17.0
0 0.187134742736816 18.0
0 -0.120223820209503 19.0
0 0.108020305633545 20.0
0 0.268878400325775 21.0
0 -0.0628961324691772 22.0
24.772446718 24.4716358184814 23.0
0 0.0728651881217957 24.0
14.00034613 0.652551591396332 25.0
0 -0.0527290105819702 26.0
0 -0.194833159446716 27.0
0 -0.258055925369263 28.0
0 -0.141721308231354 29.0
0 0.0412248969078064 30.0
0 -0.0823712348937988 31.0
0 -0.122316539287567 32.0
0 -0.446585297584534 33.0
0 0.181486546993256 34.0
0 0.208347380161285 35.0
0 -0.431976318359375 36.0
0 0.0368648767471313 37.0
0 -0.103615820407867 38.0
0 -0.553179144859314 39.0
0 0.523175299167633 40.0
0 -0.269323468208313 41.0
0 0.199787139892578 42.0
305.42972198 293.552856445312 43.0
0 0.194762527942657 44.0
0 0.0138574242591858 45.0
0 0.464432299137115 46.0
0 0.0177763104438782 47.0
0 0.414373457431793 48.0
0 0.378641068935394 49.0
0.060243533815 0.0417646169662476 50.0
1.447844503 1.89070630073547 51.0
0 -0.0760581493377686 52.0
0 -0.40143871307373 53.0
0 0.386182546615601 54.0
0 0.0162721276283264 55.0
0 0.170143723487854 56.0
0 -0.473046779632568 57.0
0 -0.207821249961853 58.0
0 0.189939141273499 59.0
0 -0.112392842769623 60.0
0 0.243167340755463 61.0
0 0.33217978477478 62.0
265.98582083 251.891052246094 0.0
0 0.489941120147705 1.0
0 0.174245178699493 2.0
0 0.113957166671753 3.0
0 -0.176988899707794 4.0
0 0.294678002595901 5.0
0 0.232130348682404 6.0
0 0.168751955032349 7.0
0 -0.124749779701233 8.0
0 -0.593207836151123 9.0
0 -0.0689003467559814 10.0
0 0.0600677728652954 11.0
0 -0.0895150899887085 12.0
54.655463424 54.375244140625 13.0
57.185213586 43.0381889343262 14.0
0 -0.16687798500061 15.0
0 0.328572511672974 16.0
27.896524894 33.3948287963867 17.0
0 0.126290798187256 18.0
0 0.0181220769882202 19.0
0 0.105734825134277 20.0
0 0.257406175136566 21.0
0 0.171062469482422 22.0
33.068486558 30.4832096099854 23.0
0 0.151138007640839 24.0
0 0.892957925796509 25.0
0 -0.0532296299934387 26.0
0 -0.0777069926261902 27.0
0 1.08356142044067 28.0
0 0.155203223228455 29.0
0 0.0930190682411194 30.0
0 0.393740773200989 31.0
0 -0.171337962150574 32.0
0 -0.248963952064514 33.0
0 0.101647734642029 34.0
0 0.843049764633179 35.0
0 0.564521491527557 36.0
0 0.106765985488892 37.0
0 0.14458429813385 38.0
0 -0.221868753433228 39.0
0 -0.762650728225708 40.0
0 -0.188170671463013 41.0
0 0.493999123573303 42.0
292.06523289 293.914428710938 43.0
0 0.199748873710632 44.0
0 0.275306105613708 45.0
0 0.0999683141708374 46.0
0 0.131709337234497 47.0
0 0.247325301170349 48.0
0 0.298828989267349 49.0
0.10544921492 0.0618595480918884 50.0
1.8552489341 1.86285281181335 51.0
0 0.198039174079895 52.0
0 0.261441588401794 53.0
0 0.367940068244934 54.0
0 0.104218244552612 55.0
0 -0.124782204627991 56.0
0 -0.0010143518447876 57.0
0 0.0904075503349304 58.0
0 0.171612858772278 59.0
0 -0.0813714265823364 60.0
0 0.0164752602577209 61.0
0 0.0338720083236694 62.0
247.32602249 256.262756347656 0.0
0 -0.151320338249207 1.0
0 -0.136272370815277 2.0
0 0.167678356170654 3.0
0 -0.217482686042786 4.0
0 -0.210164904594421 5.0
0 0.0558544397354126 6.0
0 -0.23334527015686 7.0
0 0.102130770683289 8.0
0 -0.00460171699523926 9.0
0 0.12455826997757 10.0
0 0.131917834281921 11.0
0 -0.124716341495514 12.0
56.092974634 57.2599182128906 13.0
46.283204416 40.090892791748 14.0
0 -0.163543343544006 15.0
0 0.0160621404647827 16.0
27.739093856 32.8545455932617 17.0
0 0.0623140335083008 18.0
0 -0.059148907661438 19.0
0 0.131030797958374 20.0
0 0.124735891819 21.0
0 0.183728396892548 22.0
29.972609585 26.9391479492188 23.0
0 0.0517269968986511 24.0
0 0.574977278709412 25.0
0 -0.51201593875885 26.0
0 -0.126992702484131 27.0
0 0.958792209625244 28.0
0 0.128363609313965 29.0
0 0.110539436340332 30.0
0 -0.0370298624038696 31.0
0 0.0776746273040771 32.0
0 0.0193190574645996 33.0
0 -0.125458896160126 34.0
0 -0.199039101600647 35.0
0 0.303572475910187 36.0
0 -0.0140864849090576 37.0
0 0.20734703540802 38.0
0 -0.245067715644836 39.0
0 0.594995021820068 40.0
0 -0.146712839603424 41.0
0 -0.0615122318267822 42.0
296.07320777 297.838806152344 43.0
0 0.039985716342926 44.0
0 0.189706921577454 45.0
0 0.0973897576332092 46.0
0 0.112019896507263 47.0
0 0.182846128940582 48.0
0 0.407886922359467 49.0
0.068173736242 0.0414857268333435 50.0
1.8145119985 1.85396385192871 51.0
0 0.0201759934425354 52.0
0 -0.255569338798523 53.0
0 0.103801786899567 54.0
0 0.0602762699127197 55.0
0 0.109768509864807 56.0
0 0.22996985912323 57.0
0 0.154736697673798 58.0
0 0.104055762290955 59.0
0 -0.132865726947784 60.0
0 0.0337327718734741 61.0
0 0.185503125190735 62.0
256.96971009 251.628295898438 0.0
0 0.639374732971191 1.0
0 -0.27401602268219 2.0
0 -0.79461407661438 3.0
0 -0.587016463279724 4.0
0 -0.0362424850463867 5.0
0 -0.168530642986298 6.0
0 -0.228932619094849 7.0
0 -0.156764507293701 8.0
0 0.0883002281188965 9.0
0 0.202696681022644 10.0
0 -0.185780942440033 11.0
0 -0.216330528259277 12.0
58.80702744 57.1934928894043 13.0
33.249885907 34.1139793395996 14.0
0 -0.150331377983093 15.0
0 0.440025746822357 16.0
24.218607905 32.6860809326172 17.0
0 0.153422355651855 18.0
0 -0.0177022218704224 19.0
0 0.0210089683532715 20.0
0 0.149159789085388 21.0
0 0.0247118473052979 22.0
29.712272429 30.4679775238037 23.0
0 0.0506995320320129 24.0
0 0.492321223020554 25.0
0 0.115372776985168 26.0
0 -0.134505748748779 27.0
0 -0.432882785797119 28.0
0 -0.151097416877747 29.0
0 0.230179965496063 30.0
0 -0.176176309585571 31.0
0 0.09068763256073 32.0
0 -0.119953751564026 33.0
0 0.0474158525466919 34.0
0 -0.0339429974555969 35.0
0 0.0769274234771729 36.0
0 0.135626614093781 37.0
0 0.0794254541397095 38.0
0 -0.176199913024902 39.0
0 0.33441373705864 40.0
0 -0.144054532051086 41.0
0 0.244986414909363 42.0
299.02450993 298.560272216797 43.0
0 0.177072286605835 44.0
0 0.16739547252655 45.0
0 0.496364295482635 46.0
0 -0.113448560237885 47.0
0 0.0371649265289307 48.0
0 0.532597124576569 49.0
0.083887248505 0.0665284395217896 50.0
1.5296717106 1.78781652450562 51.0
0 0.438823342323303 52.0
0 0.30343633890152 53.0
0 0.229298889636993 54.0
0 0.181792855262756 55.0
0 0.075426459312439 56.0
0 0.860363483428955 57.0
0 0.512239694595337 58.0
0 0.0970237255096436 59.0
0 -0.0877140164375305 60.0
0 0.134139478206635 61.0
0 0.231077075004578 62.0
207.64070398 254.849365234375 0.0
0 -0.139236450195312 1.0
0 -0.0822080969810486 2.0
0 -0.169662296772003 3.0
0 -0.343008518218994 4.0
0 -0.310299158096313 5.0
0 -0.139819741249084 6.0
0 -0.130880057811737 7.0
0 0.672663033008575 8.0
0 -0.0108513832092285 9.0
0 0.151511430740356 10.0
0 -0.0320570468902588 11.0
0 -0.147292971611023 12.0
75.015263001 73.8658905029297 13.0
18.181752466 21.6769752502441 14.0
0 -0.0762500762939453 15.0
0 -0.106809020042419 16.0
39.771118013 38.1716804504395 17.0
0 0.191589713096619 18.0
0 -0.0983108282089233 19.0
0 0.248244524002075 20.0
0 0.0755910873413086 21.0
0 0.207913041114807 22.0
32.604031648 32.5252723693848 23.0
0 -0.0616896152496338 24.0
0 0.613346576690674 25.0
0 0.168992519378662 26.0
0 0.119180798530579 27.0
0 -0.480012178421021 28.0
0 -0.218447089195251 29.0
0 0.0599256157875061 30.0
0 -0.0298143625259399 31.0
0 0.0972224473953247 32.0
0 -0.0928274989128113 33.0
0 0.0697163939476013 34.0
0 -0.135695517063141 35.0
0 -0.188647091388702 36.0
0 -0.0828531384468079 37.0
0 0.235015392303467 38.0
0 -0.0669430494308472 39.0
0 0.721081733703613 40.0
0 -0.115818321704865 41.0
0 0.0633158683776855 42.0
293.75868599 296.947967529297 43.0
0 -0.0376795530319214 44.0
0 0.0842606425285339 45.0
0 0.280267357826233 46.0
0 -0.274280905723572 47.0
0 0.0927597880363464 48.0
2.4189310742 0.066497266292572 49.0
0.10746107231 -0.0158913731575012 50.0
1.6621454437 1.91358137130737 51.0
0 0.414251029491425 52.0
0 -0.0963994264602661 53.0
0 -0.196284830570221 54.0
0 0.220963358879089 55.0
0 0.268204748630524 56.0
0 0.532586991786957 57.0
0 0.0925498604774475 58.0
0 -0.0874595642089844 59.0
0 -0.106841921806335 60.0
0 0.019220232963562 61.0
0 -0.0529720783233643 62.0
280.951121 252.308639526367 0.0
0 -0.27178955078125 1.0
0 -0.0737235546112061 2.0
0 0.336113393306732 3.0
0 -0.0997218489646912 4.0
0 0.309525579214096 5.0
0 -0.117434799671173 6.0
0 0.269868552684784 7.0
0 -0.193856358528137 8.0
0 0.284226596355438 9.0
0 -0.182145833969116 10.0
0 -0.153858125209808 11.0
0 -0.148898959159851 12.0
53.434081889 52.7349090576172 13.0
37.406246049 39.0663642883301 14.0
0 0.00115883350372314 15.0
0 -0.0697660446166992 16.0
28.347029314 34.1131362915039 17.0
0 0.067537784576416 18.0
0 -0.0173929333686829 19.0
0 0.102849721908569 20.0
0 -0.0277144908905029 21.0
0 0.0186826586723328 22.0
40.765449803 39.3188362121582 23.0
0 0.0212309956550598 24.0
0 0.384210705757141 25.0
0 -0.031810998916626 26.0
0 0.125201463699341 27.0
0 0.184076547622681 28.0
0 -0.0608416199684143 29.0
0 0.184599339962006 30.0
0 0.0129467248916626 31.0
0 0.0529762506484985 32.0
0 0.0528411865234375 33.0
0 0.272482931613922 34.0
0 0.161818623542786 35.0
0 -0.0333085060119629 36.0
0 0.00760316848754883 37.0
0 0.0690929293632507 38.0
0 -0.608357787132263 39.0
0 0.295224577188492 40.0
0 -0.171526432037354 41.0
0 0.076093316078186 42.0
298.20542997 295.550903320312 43.0
0 0.198514819145203 44.0
0 0.149779498577118 45.0
0 -0.0270901918411255 46.0
0 0.0181061029434204 47.0
0 0.246874988079071 48.0
0 0.31475841999054 49.0
0.07749536628 0.0945934057235718 50.0
1.5808263779 1.82464671134949 51.0
0 0.114138007164001 52.0
0 0.17757773399353 53.0
0 0.285512804985046 54.0
0 0.14260345697403 55.0
0 -0.108658790588379 56.0
0 0.36871737241745 57.0
0 0.103859901428223 58.0
0 0.0953102111816406 59.0
0 -0.0104493498802185 60.0
0 0.0681586265563965 61.0
0 0.206062495708466 62.0
278.31612159 253.762969970703 0.0
0 -0.157385230064392 1.0
0 0.159269452095032 2.0
0 -0.446241855621338 3.0
0 0.469575256109238 4.0
0 0.12722510099411 5.0
0 -0.68359375 6.0
0 0.183658719062805 7.0
0 0.514855086803436 8.0
0 0.012731671333313 9.0
0 0.447158575057983 10.0
0 -0.293581604957581 11.0
0 0.0194132328033447 12.0
76.833462599 73.7587203979492 13.0
8.760186625 33.7248153686523 14.0
0 0.0612713098526001 15.0
0 0.0631992816925049 16.0
41.777726903 35.2408638000488 17.0
0 0.161930203437805 18.0
0 -0.091555118560791 19.0
0 0.127180874347687 20.0
0 -0.131158769130707 21.0
0 1.13652074337006 22.0
39.078663899 38.3493003845215 23.0
0 -0.0584949851036072 24.0
0 0.442533195018768 25.0
0 -0.109836578369141 26.0
0 0.0269023180007935 27.0
0 -0.195180177688599 28.0
0 0.05842524766922 29.0
0 0.15904974937439 30.0
0 0.155272364616394 31.0
0 0.165542960166931 32.0
0 -0.151880443096161 33.0
0 0.0942797660827637 34.0
0 0.139931738376617 35.0
0 0.0689825415611267 36.0
0 0.0095561146736145 37.0
0 0.615045785903931 38.0
0 -0.577295303344727 39.0
0 -0.208812952041626 40.0
0 -0.189995169639587 41.0
0 -0.100460290908813 42.0
297.69275938 295.540679931641 43.0
0 -0.0761237144470215 44.0
0 0.0389944314956665 45.0
0 0.0855832099914551 46.0
0 0.146185398101807 47.0
0 0.376134991645813 48.0
0 0.4911148250103 49.0
0.093646781367 0.0586130023002625 50.0
2.0923895741 1.81335937976837 51.0
0 0.00450688600540161 52.0
0 0.0758621692657471 53.0
0 0.0584766268730164 54.0
0 0.328176766633987 55.0
0 0.176070690155029 56.0
0 0.415025740861893 57.0
0 0.525515198707581 58.0
0 0.137198805809021 59.0
0 -0.227717995643616 60.0
0 0.0423082113265991 61.0
0 0.187817215919495 62.0
298.84193652 255.166046142578 0.0
0 -0.466304063796997 1.0
0 -0.0551716685295105 2.0
0 -0.247241973876953 3.0
0 -0.207994759082794 4.0
0 0.260004937648773 5.0
0 -0.358000755310059 6.0
0 0.480238378047943 7.0
0 -0.248012065887451 8.0
0 0.439223647117615 9.0
0 -0.195444822311401 10.0
0 1.23677706718445 11.0
0 0.054363489151001 12.0
72.911870594 73.1284637451172 13.0
33.78222214 26.1495342254639 14.0
0 -0.086456298828125 15.0
0 0.0471808314323425 16.0
25.689592062 33.9413833618164 17.0
0 0.164142251014709 18.0
0 -0.0049593448638916 19.0
0 0.03419029712677 20.0
0 0.20493495464325 21.0
0 -0.0438576340675354 22.0
25.849205605 28.0386428833008 23.0
0 0.116169393062592 24.0
0 1.19870805740356 25.0
0 -0.193249344825745 26.0
0 -0.15646106004715 27.0
0 0.476835817098618 28.0
0 0.217394709587097 29.0
0 -0.0104930400848389 30.0
0 -0.105011343955994 31.0
0 0.235517382621765 32.0
0 0.251527070999146 33.0
0 0.326200425624847 34.0
0 0.236183762550354 35.0
0 0.360759437084198 36.0
0 0.0228413343429565 37.0
0 0.350666403770447 38.0
0 -0.325053453445435 39.0
0 0.333360373973846 40.0
0 -0.180978417396545 41.0
0 -0.134620189666748 42.0
295.85851779 297.947845458984 43.0
0 0.089616060256958 44.0
0 0.0156856775283813 45.0
0 -0.0475689172744751 46.0
0 0.142159879207611 47.0
0 -0.792175650596619 48.0
0 0.495423495769501 49.0
0.098874716655 -0.0493344068527222 50.0
1.7657961462 1.85743832588196 51.0
0 0.247260749340057 52.0
0 0.235590577125549 53.0
0 0.0158015489578247 54.0
0 -0.180923223495483 55.0
0 -0.0848448872566223 56.0
0 0.0775068998336792 57.0
0 0.0420101881027222 58.0
0 -0.0358387231826782 59.0
0 -0.159586906433105 60.0
0 0.0985973477363586 61.0
0 0.270979821681976 62.0
270.04327089 253.942138671875 0.0
0 -0.302653789520264 1.0
0 0.0910164713859558 2.0
0 0.344700187444687 3.0
0 0.222048223018646 4.0
0 -0.131189703941345 5.0
0 -0.67584228515625 6.0
0 0.635788261890411 7.0
0 -0.59515905380249 8.0
0 -0.180749535560608 9.0
0 0.410869091749191 10.0
0 -0.360540151596069 11.0
0 0.0137214660644531 12.0
62.477462346 62.1558723449707 13.0
26.067242464 30.808219909668 14.0
0 -0.0174059867858887 15.0
0 -0.310963869094849 16.0
37.879259581 34.4968872070312 17.0
0 0.0771605968475342 18.0
0 -0.0394295454025269 19.0
0 0.111501634120941 20.0
0 0.0615034103393555 21.0
0 0.228860139846802 22.0
29.818677147 26.6532325744629 23.0
0 0.00271672010421753 24.0
0 -0.241349220275879 25.0
0 -0.0550729036331177 26.0
0 -0.119033932685852 27.0
0 0.524743437767029 28.0
0 -0.102704644203186 29.0
0 0.0955652594566345 30.0
0 -0.11997652053833 31.0
0 -0.129886984825134 32.0
0 -0.0508140325546265 33.0
0 0.159785509109497 34.0
0 0.26845121383667 35.0
0 0.438853591680527 36.0
0 -0.0399309396743774 37.0
0 -0.0712004899978638 38.0
0 -0.16116189956665 39.0
0 -0.198669016361237 40.0
0 -0.0684565305709839 41.0
0 0.0987256765365601 42.0
301.06496028 297.032501220703 43.0
0 -0.00606775283813477 44.0
0 0.207807600498199 45.0
0 0.249829709529877 46.0
0 -0.212134122848511 47.0
0 0.150370717048645 48.0
0 0.403008013963699 49.0
0.080111170894 -0.0104507803916931 50.0
2.3327080822 1.85609149932861 51.0
0 -0.232987761497498 52.0
0 -0.331416487693787 53.0
0 0.101712703704834 54.0
0 0.110943913459778 55.0
0 0.0721774101257324 56.0
0 -0.0256088972091675 57.0
0 -0.190722703933716 58.0
0 -0.0394188165664673 59.0
0 -0.0496730804443359 60.0
0 0.171796441078186 61.0
0 0.256725668907166 62.0
266.67511651 252.78288269043 0.0
0 -0.330606698989868 1.0
0 -0.0969586968421936 2.0
0 0.231844067573547 3.0
0 -0.0405028462409973 4.0
0 0.276040196418762 5.0
0 0.372524440288544 6.0
0 0.144399106502533 7.0
0 -0.130796194076538 8.0
0 -0.0119322538375854 9.0
0 -0.0616275668144226 10.0
0 0.0220574140548706 11.0
0 -0.216360807418823 12.0
50.891133219 50.2016105651855 13.0
51.400086423 44.5234031677246 14.0
0 0.0306262969970703 15.0
0 -0.0247640609741211 16.0
38.825851899 33.853816986084 17.0
0 0.143211305141449 18.0
0 -0.0538827180862427 19.0
0 0.153529644012451 20.0
0 0.0447736978530884 21.0
0 0.298500806093216 22.0
33.862619708 33.0733070373535 23.0
0 0.0790020227432251 24.0
0 0.692514002323151 25.0
0 -0.0893160104751587 26.0
0 -0.158287286758423 27.0
0 0.429482728242874 28.0
0 0.0622828602790833 29.0
0 0.0838833451271057 30.0
0 -0.353684663772583 31.0
0 -0.0533006191253662 32.0
0 -0.158718407154083 33.0
0 0.157701969146729 34.0
0 -0.1451016664505 35.0
0 0.107656836509705 36.0
0 -0.0497331619262695 37.0
0 0.453236728906631 38.0
0 -0.384106874465942 39.0
0 0.132364213466644 40.0
0 -0.211910605430603 41.0
0 0.123172402381897 42.0
302.24791077 297.084167480469 43.0
0 0.0909934043884277 44.0
0 0.0738772749900818 45.0
0 0.263011872768402 46.0
0 0.0326212644577026 47.0
0 -0.211170554161072 48.0
0 0.44475457072258 49.0
0.0877416668 0.0474363565444946 50.0
1.9574083831 1.81183791160583 51.0
0 0.220781207084656 52.0
0 -0.360084891319275 53.0
0 0.0727421045303345 54.0
0 -0.0509451627731323 55.0
0 -0.117784142494202 56.0
0 0.111641526222229 57.0
0 0.322647154331207 58.0
0 0.184920251369476 59.0
0 -0.0776593089103699 60.0
0 0.106117844581604 61.0
0 0.279743492603302 62.0
240.78857796 254.452682495117 0.0
0 -0.259755492210388 1.0
0 0.0757343769073486 2.0
0 0.123989284038544 3.0
0 -0.328354835510254 4.0
0 -0.00730317831039429 5.0
0 0.146714210510254 6.0
0 0.142217516899109 7.0
0 0.398967057466507 8.0
0 -0.263239979743958 9.0
0 0.354919821023941 10.0
0 -0.0412325859069824 11.0
0 -0.103194177150726 12.0
62.782662062 62.2958908081055 13.0
33.559625782 34.0100784301758 14.0
0 -0.0170413255691528 15.0
0 0.422651410102844 16.0
34.468654705 36.0567855834961 17.0
0 0.177459359169006 18.0
0 -0.00484633445739746 19.0
0 0.105572462081909 20.0
0 0.0178565979003906 21.0
0 0.0822018384933472 22.0
36.506732829 35.8138885498047 23.0
0 0.0685821771621704 24.0
0 0.661137163639069 25.0
0 -0.036029577255249 26.0
0 -0.127349078655243 27.0
0 0.463532328605652 28.0
0 -0.00445377826690674 29.0
0 0.12018609046936 30.0
0 0.0494756698608398 31.0
0 0.151010751724243 32.0
0 -0.0366530418395996 33.0
0 0.199868381023407 34.0
0 0.0659058094024658 35.0
0 0.296822488307953 36.0
0 0.192288875579834 37.0
0 -0.0428176522254944 38.0
0 -0.264835715293884 39.0
0 0.662472546100616 40.0
0 -0.130792319774628 41.0
0 0.0159980058670044 42.0
292.22652869 294.685974121094 43.0
0 0.0292719602584839 44.0
0 0.0120874047279358 45.0
0 0.31293123960495 46.0
0 -0.12870842218399 47.0
0 -0.0830588936805725 48.0
0 0.372880607843399 49.0
0.096966011451 -0.0251705050468445 50.0
1.7970417546 1.87033224105835 51.0
0 -0.202115774154663 52.0
0 0.0538260340690613 53.0
0 0.0340361595153809 54.0
0 0.240017652511597 55.0
0 0.0730856657028198 56.0
0 0.158322095870972 57.0
0 0.223814129829407 58.0
0 0.263481557369232 59.0
0 -0.224834442138672 60.0
0 0.0978149175643921 61.0
0 0.332033097743988 62.0
265.6707009 253.93017578125 0.0
0 -0.351007461547852 1.0
0 -0.226685523986816 2.0
0 0.561424493789673 3.0
0 0.122612059116364 4.0
0 0.0768099427223206 5.0
0 -0.273975372314453 6.0
0 0.122102975845337 7.0
0 0.293772101402283 8.0
0 0.0716100931167603 9.0
0 0.116685569286346 10.0
0 -1.32820677757263 11.0
0 -0.0856390595436096 12.0
78.359867536 76.6958160400391 13.0
11.933209304 25.6259155273438 14.0
0 -0.239865899085999 15.0
0 -0.0883988738059998 16.0
27.772835339 31.2286357879639 17.0
0 0.0992332696914673 18.0
0 -0.0157601833343506 19.0
0 0.13009512424469 20.0
0 0.227102160453796 21.0
0 -0.0506668090820312 22.0
27.704634855 23.304141998291 23.0
0 0.0511064529418945 24.0
0 0.118096709251404 25.0
0 -0.304540395736694 26.0
0 -0.221657395362854 27.0
0 1.17391240596771 28.0
0 0.223206758499146 29.0
0 -0.089726984500885 30.0
0 0.233153462409973 31.0
0 0.0189905166625977 32.0
0 -0.285061597824097 33.0
0 0.321445643901825 34.0
0 -0.122048377990723 35.0
0 0.249851226806641 36.0
0 0.0967050790786743 37.0
0 -0.0595661401748657 38.0
0 -0.321753144264221 39.0
0 0.369408547878265 40.0
0 -0.0327012538909912 41.0
0 0.0164244174957275 42.0
310.5191666 296.832946777344 43.0
0 0.180839776992798 44.0
0 0.0587734580039978 45.0
0 0.0917931795120239 46.0
0 0.273391366004944 47.0
0 0.337973058223724 48.0
0 0.440511584281921 49.0
0.07203800088 0.0410297513008118 50.0
1.4037075818 1.8506166934967 51.0
0 0.199817001819611 52.0
0 -0.121236145496368 53.0
0 0.256373047828674 54.0
0 -0.104705214500427 55.0
0 -0.195188403129578 56.0
0 -0.163199245929718 57.0
0 0.37576100230217 58.0
0 0.354404211044312 59.0
0 -0.127542078495026 60.0
0 0.0998049378395081 61.0
0 0.222517848014832 62.0
255.42207413 253.69987487793 0.0
0 0.426257967948914 1.0
0 0.0251075625419617 2.0
0 0.238432228565216 3.0
0 -0.128222942352295 4.0
0 0.12913304567337 5.0
0 0.098534882068634 6.0
0 0.0905896425247192 7.0
0 -0.495605826377869 8.0
0 0.168847739696503 9.0
0 0.311602830886841 10.0
0 -0.364037990570068 11.0
0 -0.0367540717124939 12.0
68.838099062 68.5032958984375 13.0
16.27012008 27.5576763153076 14.0
0 -0.0700629949569702 15.0
0 0.273239135742188 16.0
34.960859225 34.9154777526855 17.0
0 0.113333284854889 18.0
0 -0.096339225769043 19.0
0 0.0846801996231079 20.0
0 0.0838285684585571 21.0
0 0.108926773071289 22.0
28.963336219 26.1106014251709 23.0
0 0.0457581281661987 24.0
0 0.230516672134399 25.0
0 -0.10340017080307 26.0
0 -0.0586877465248108 27.0
0 -0.296512603759766 28.0
0 -0.0831448435783386 29.0
0 -0.000296473503112793 30.0
0 0.1551753282547 31.0
0 0.272391498088837 32.0
0 -0.164592683315277 33.0
0 0.0130414962768555 34.0
0 0.391438007354736 35.0
0 -0.246910929679871 36.0
0 0.105084180831909 37.0
0 0.260868072509766 38.0
0 -0.235128998756409 39.0
0 0.501481533050537 40.0
0 -0.137449622154236 41.0
0 0.1342533826828 42.0
286.9343581 296.392791748047 43.0
0 0.122138857841492 44.0
0 0.0851330757141113 45.0
0 0.130349516868591 46.0
0 0.0359258651733398 47.0
0 0.122535288333893 48.0
0 0.298631101846695 49.0
0.067615316952 0.0307122468948364 50.0
2.064808949 1.89220321178436 51.0
0 0.204040825366974 52.0
0 0.0350937843322754 53.0
0 0.0966227054595947 54.0
0 0.0789897441864014 55.0
0 0.0425344705581665 56.0
0 -0.103857636451721 57.0
0 0.168723821640015 58.0
0 0.138809680938721 59.0
0 -0.246959328651428 60.0
0 0.0561370849609375 61.0
0 0.26965457201004 62.0
226.81730217 252.134170532227 0.0
0 0.0346280336380005 1.0
0 -0.0401244163513184 2.0
0 -0.247088193893433 3.0
0 -0.329393744468689 4.0
0 -0.0230430960655212 5.0
0 0.162743091583252 6.0
0 -0.106086552143097 7.0
0 -0.345450162887573 8.0
0 0.415669560432434 9.0
0 -0.18528938293457 10.0
0 -0.286032199859619 11.0
0 0.0609830021858215 12.0
73.038313919 73.7537536621094 13.0
10.172848308 25.7768802642822 14.0
0 -0.0666940212249756 15.0
0 0.133556842803955 16.0
33.702681249 36.1199188232422 17.0
0 0.160346269607544 18.0
0 -0.0652440786361694 19.0
0 0.0599548816680908 20.0
0 -0.027486264705658 21.0
0 0.37017834186554 22.0
34.617372759 34.2752113342285 23.0
0 0.115478992462158 24.0
0 0.738110840320587 25.0
0 -6.83069229125977e-05 26.0
0 -0.192835807800293 27.0
0 0.498211473226547 28.0
0 -0.0916858911514282 29.0
0 0.0663748979568481 30.0
0 0.061009407043457 31.0
0 0.175815939903259 32.0
0 0.0587265491485596 33.0
0 -0.181708037853241 34.0
0 0.134001076221466 35.0
0 -0.259800553321838 36.0
0 0.0422473549842834 37.0
0 -0.215160131454468 38.0
0 -0.14161217212677 39.0
0 0.224826574325562 40.0
0 -0.225801467895508 41.0
0 0.0161669254302979 42.0
291.41294636 293.706848144531 43.0
0 0.171572148799896 44.0
0 -0.0702142715454102 45.0
0 0.0619550943374634 46.0
0 -0.194578170776367 47.0
0 -0.28228235244751 48.0
0 0.340595781803131 49.0
0.07352751765 0.0396023392677307 50.0
1.9170223032 1.81698262691498 51.0
0 -0.0855963826179504 52.0
0 -0.0121333599090576 53.0
0 -0.0366345643997192 54.0
0 0.157333254814148 55.0
0 0.126255571842194 56.0
0 0.188736617565155 57.0
0 0.281574070453644 58.0
0 -0.0596858263015747 59.0
0 -0.12297910451889 60.0
0 0.0163362622261047 61.0
0 0.15750253200531 62.0
254.22729991 254.412475585938 0.0
0 -0.172447085380554 1.0
0 -0.0456864833831787 2.0
0 0.328488528728485 3.0
0 -0.129280090332031 4.0
0 0.0542531609535217 5.0
0 0.368232011795044 6.0
0 0.343367785215378 7.0
0 -0.170764803886414 8.0
0 -0.0711326003074646 9.0
0 0.728381454944611 10.0
0 -1.09167897701263 11.0
0 -0.100030601024628 12.0
71.906615193 69.5181274414062 13.0
20.144743924 31.2952175140381 14.0
0 -0.251323103904724 15.0
0 -0.0711017251014709 16.0
40.685579683 37.569164276123 17.0
0 0.172340035438538 18.0
0 0.0105575323104858 19.0
0 0.0494061708450317 20.0
0 0.0621130466461182 21.0
0 0.194374620914459 22.0
37.29534209 35.7912979125977 23.0
0 0.125483870506287 24.0
0 1.05914747714996 25.0
0 -0.10883092880249 26.0
0 0.105258762836456 27.0
0 0.213312029838562 28.0
0 -0.0214122533798218 29.0
0 -0.0134978890419006 30.0
0 -0.183181166648865 31.0
0 -0.166086196899414 32.0
0 -0.156617105007172 33.0
0 0.0290217399597168 34.0
0 0.20663595199585 35.0
0 0.148668766021729 36.0
0 0.168663799762726 37.0
0 -0.264540433883667 38.0
0 -0.199389576911926 39.0
0 0.477705001831055 40.0
0 -0.186680316925049 41.0
0 0.0682039260864258 42.0
285.05606859 295.809265136719 43.0
0 0.0853360891342163 44.0
0 0.0132659077644348 45.0
0 0.161491274833679 46.0
0 -0.112453579902649 47.0
0 -0.420315146446228 48.0
0 0.504156351089478 49.0
0.075947219563 -0.0245617032051086 50.0
1.5336980308 1.87824869155884 51.0
0 -0.0150846242904663 52.0
0 -0.123134672641754 53.0
0 0.220064997673035 54.0
0 0.238436043262482 55.0
0 0.0163292288780212 56.0
0 0.353847533464432 57.0
0 0.0441659688949585 58.0
0 0.00618433952331543 59.0
0 -0.0836352109909058 60.0
0 -0.0037691593170166 61.0
0 -0.0632197260856628 62.0
211.77044955 254.120010375977 0.0
0 -0.0476903915405273 1.0
0 -0.102809727191925 2.0
0 -0.368317246437073 3.0
0 -0.436073541641235 4.0
0 -0.237927317619324 5.0
0 -0.135699868202209 6.0
0 -0.198344647884369 7.0
0 0.181658267974854 8.0
0 0.0722230672836304 9.0
0 -0.244044423103333 10.0
0 -0.190991759300232 11.0
0 -0.0928393006324768 12.0
63.442485258 62.2134742736816 13.0
24.833422147 35.820255279541 14.0
0 -0.0978538990020752 15.0
0 0.0490264892578125 16.0
36.432652912 37.1169853210449 17.0
0 0.185869097709656 18.0
0 -0.0627157688140869 19.0
0 0.292217135429382 20.0
0 -0.170884132385254 21.0
0 1.14194631576538 22.0
42.400400962 41.4947891235352 23.0
0 0.0678807497024536 24.0
0 0.893987715244293 25.0
0 0.0991281270980835 26.0
0 0.0432536005973816 27.0
0 0.461880564689636 28.0
0 -0.0969498157501221 29.0
0 0.0155737996101379 30.0
0 -0.268468618392944 31.0
0 -0.0717475414276123 32.0
0 -0.438785076141357 33.0
0 0.246290981769562 34.0
0 0.050527811050415 35.0
0 0.411982417106628 36.0
0 0.0440477728843689 37.0
0 0.126924395561218 38.0
0 -0.182689785957336 39.0
0 1.01441264152527 40.0
0 -0.0467054843902588 41.0
0 -0.0101044178009033 42.0
299.67491447 297.61865234375 43.0
0 -0.00607955455780029 44.0
0 0.00216573476791382 45.0
0 0.193937003612518 46.0
0 -0.230072736740112 47.0
0 0.00973308086395264 48.0
0 0.286537289619446 49.0
0.066256797848 0.0524702072143555 50.0
1.8653538742 1.86521482467651 51.0
0 0.0551207065582275 52.0
0 -0.430264234542847 53.0
0 0.0640040636062622 54.0
0 0.249077498912811 55.0
0 0.0700377225875854 56.0
0 0.303921580314636 57.0
0 0.232148408889771 58.0
0 0.305746495723724 59.0
0 0.0403443574905396 60.0
0 0.0917187333106995 61.0
0 0.310343086719513 62.0
254.96947951 252.427139282227 0.0
0 -0.1922527551651 1.0
0 0.110169947147369 2.0
0 0.048423171043396 3.0
0 -0.176855564117432 4.0
0 -0.182319939136505 5.0
0 0.0223957896232605 6.0
0 0.120994210243225 7.0
0 0.154352962970734 8.0
0 0.0505793690681458 9.0
0 0.616798639297485 10.0
0 -0.273299932479858 11.0
0 0.00854885578155518 12.0
56.718667316 57.5661582946777 13.0
34.980680886 38.6489181518555 14.0
0 0.0615973472595215 15.0
0 0.0378738641738892 16.0
32.716913425 36.1971397399902 17.0
0 0.108406484127045 18.0
0 -0.0699585676193237 19.0
0 0.100685894489288 20.0
0 0.138651490211487 21.0
0 0.0442812442779541 22.0
33.71086084 30.2256736755371 23.0
0 -0.0330150723457336 24.0
0 0.49548265337944 25.0
0 -0.123658061027527 26.0
0 -0.0851137638092041 27.0
0 -0.424729108810425 28.0
0 -0.123739004135132 29.0
0 -0.000962674617767334 30.0
0 0.0758594274520874 31.0
0 0.187704861164093 32.0
0 0.217834174633026 33.0
0 0.428205370903015 34.0
0 0.200752735137939 35.0
0 -0.000896275043487549 36.0
0 -0.0361167192459106 37.0
0 0.211373627185822 38.0
0 -0.219472646713257 39.0
0 0.19672030210495 40.0
0 -0.234158873558044 41.0
0 0.125766158103943 42.0
288.65431029 298.184234619141 43.0
0 0.0476909279823303 44.0
0 0.116741061210632 45.0
0 0.134232878684998 46.0
0 -0.324903011322021 47.0
0 0.386205226182938 48.0
0 0.457248121500015 49.0
0.073767664733 0.0537904500961304 50.0
1.6558579922 1.87846875190735 51.0
0 -0.238227128982544 52.0
0 -0.549544930458069 53.0
0 -0.161117196083069 54.0
0 -0.0560966730117798 55.0
0 0.275951862335205 56.0
0 0.161713242530823 57.0
0 0.345584124326706 58.0
0 -0.0424011945724487 59.0
0 0.0837979316711426 60.0
0 0.0383729338645935 61.0
0 0.193196356296539 62.0
281.08630745 252.233505249023 0.0
0 0.120298504829407 1.0
0 0.141014933586121 2.0
0 0.515254378318787 3.0
0 -0.0420783162117004 4.0
0 0.261029660701752 5.0
0 -0.468157291412354 6.0
0 0.364918887615204 7.0
0 -0.070307195186615 8.0
0 0.93822306394577 9.0
0 -0.368951439857483 10.0
0 0.0997333526611328 11.0
0 -0.225172996520996 12.0
69.248218313 66.7650833129883 13.0
40.129497209 27.8339805603027 14.0
0 0.0247261524200439 15.0
0 0.0571460127830505 16.0
36.221627729 33.1564865112305 17.0
0 0.158541202545166 18.0
0 -0.0741909146308899 19.0
0 0.272169172763824 20.0
0 0.0334119200706482 21.0
0 0.247664511203766 22.0
30.496907509 31.1723232269287 23.0
0 0.0722211003303528 24.0
0 0.699791193008423 25.0
0 0.0869423151016235 26.0
0 0.0323132276535034 27.0
0 -0.276062250137329 28.0
0 -0.238610863685608 29.0
0 0.100940883159637 30.0
0 -0.46486759185791 31.0
0 0.106263875961304 32.0
0 0.130535006523132 33.0
0 -0.0551753640174866 34.0
0 0.114461302757263 35.0
0 0.414790153503418 36.0
0 0.18558257818222 37.0
0 0.0708425045013428 38.0
0 -0.146144032478333 39.0
0 0.624840259552002 40.0
0 -0.0439967513084412 41.0
0 0.0994874238967896 42.0
275.43817852 293.535675048828 43.0
0 0.104710578918457 44.0
0 0.239722549915314 45.0
0 0.102738559246063 46.0
0 0.104851484298706 47.0
0 0.25637412071228 48.0
3.2979221618 0.495425790548325 49.0
0.081460762242 0.104727447032928 50.0
1.9573401666 1.85938930511475 51.0
0 0.117693424224854 52.0
0 -0.22177529335022 53.0
0 -0.176105260848999 54.0
0 0.124940872192383 55.0
0 0.0969130992889404 56.0
0 0.0647459030151367 57.0
0 0.441345751285553 58.0
0 0.0652593374252319 59.0
0 -0.231456637382507 60.0
0 0.112998902797699 61.0
0 0.216002702713013 62.0
235.65309346 248.698791503906 0.0
0 -0.0292167663574219 1.0
0 -0.0225982666015625 2.0
0 -0.161051750183105 3.0
0 -0.368189811706543 4.0
0 -0.237779140472412 5.0
0 0.215895593166351 6.0
0 -0.137998878955841 7.0
0 0.230820536613464 8.0
0 -0.336012005805969 9.0
0 0.0944150686264038 10.0
0 0.0373981595039368 11.0
0 -0.0805490016937256 12.0
50.631545966 49.2005271911621 13.0
53.85550055 46.6657638549805 14.0
0 -0.310798168182373 15.0
0 0.309471905231476 16.0
34.503528592 33.0615005493164 17.0
0 0.184490740299225 18.0
0 -0.0893146395683289 19.0
0 0.28378438949585 20.0
0 0.0606247186660767 21.0
0 0.265457332134247 22.0
26.495236597 28.7823314666748 23.0
0 0.0402103662490845 24.0
0 1.0593364238739 25.0
0 0.00825202465057373 26.0
0 -0.352942943572998 27.0
0 0.0743477344512939 28.0
0 -0.0126228928565979 29.0
0 0.13993626832962 30.0
0 -0.149834871292114 31.0
0 -0.154352784156799 32.0
0 0.0380496978759766 33.0
0 -0.0724458694458008 34.0
0 -0.0125016570091248 35.0
0 0.383756667375565 36.0
0 -0.0629861950874329 37.0
0 0.400391459465027 38.0
0 -0.155711889266968 39.0
0 0.673274040222168 40.0
0 -0.100016295909882 41.0
0 0.0432475805282593 42.0
272.00117304 294.348754882812 43.0
0 -1.37904524803162 44.0
0 -0.89627742767334 45.0
0 0.156737148761749 46.0
0 -0.0584144592285156 47.0
0 0.329184144735336 48.0
0.85863454833 0.448112607002258 49.0
0.079931329352 0.0401332378387451 50.0
1.4301202321 1.92134726047516 51.0
0 0.0806324481964111 52.0
0 0.119772434234619 53.0
0 -0.148178100585938 54.0
0 0.0336123108863831 55.0
0 0.387580871582031 56.0
0 0.0788226723670959 57.0
0 0.0867059826850891 58.0
0 0.118741750717163 59.0
0 -0.0363239049911499 60.0
0 -0.008647620677948 61.0
0 0.127611458301544 62.0
266.23269751 250.673400878906 0.0
0 -0.461084246635437 1.0
0 0.312355995178223 2.0
0 0.183798849582672 3.0
0 -0.183766305446625 4.0
0 0.138538360595703 5.0
0 0.163513541221619 6.0
0 0.749708950519562 7.0
0 0.844911575317383 8.0
0 -0.16315108537674 9.0
0 -0.0297151803970337 10.0
0 -0.222335815429688 11.0
0 -0.0682178139686584 12.0
49.919664279 49.2958564758301 13.0
49.815782402 46.2634925842285 14.0
0 -0.230577111244202 15.0
0 -0.190210580825806 16.0
33.79741348 34.056827545166 17.0
0 0.108513534069061 18.0
0 0.00383651256561279 19.0
0 0.154534459114075 20.0
0 0.151326358318329 21.0
0 -0.0133856534957886 22.0
28.942475729 27.7596054077148 23.0
0 -0.0145624279975891 24.0
27.813546009 0.189659416675568 25.0
0 -0.0415878295898438 26.0
0 -0.0368475317955017 27.0
0 -0.42765998840332 28.0
0 -0.0646791458129883 29.0
0 0.0228229165077209 30.0
0 -0.192649245262146 31.0
0 0.0498976111412048 32.0
0 -0.381583094596863 33.0
0 -0.0476028919219971 34.0
0 0.893403589725494 35.0
0 0.589826703071594 36.0
0 -0.0351703763008118 37.0
0 0.300609439611435 38.0
0 -0.254830956459045 39.0
0 -0.0537176132202148 40.0
0 -0.191644430160522 41.0
0 -0.0679118633270264 42.0
295.17425392 298.133666992188 43.0
0 -0.00449347496032715 44.0
0 0.312495708465576 45.0
0 0.0243402719497681 46.0
0 -0.128946602344513 47.0
0 0.210330009460449 48.0
0 0.268136978149414 49.0
0.1040052238 0.0898409485816956 50.0
1.8392533256 1.84954702854156 51.0
0 -0.0812056660652161 52.0
0 0.224526762962341 53.0
0 0.276677250862122 54.0
0 0.0218825340270996 55.0
0 -0.0179241895675659 56.0
0 -0.112261950969696 57.0
0 0.0660930275917053 58.0
0 0.217644155025482 59.0
0 -0.026974081993103 60.0
0 0.0916004776954651 61.0
0 0.207262396812439 62.0
268.79205694 251.004257202148 0.0
0 0.107588529586792 1.0
0 -0.0126653909683228 2.0
0 -0.0460867285728455 3.0
0 0.0902146100997925 4.0
0 0.272791922092438 5.0
0 0.362363338470459 6.0
0 0.411558419466019 7.0
0 0.0534145832061768 8.0
0 -0.0999678373336792 9.0
0 -0.102253139019012 10.0
0 -0.120263457298279 11.0
0 -0.0738987922668457 12.0
60.231424158 60.8714485168457 13.0
34.475105244 37.0812797546387 14.0
0 -0.0538424253463745 15.0
0 0.117680668830872 16.0
35.378015577 33.7894058227539 17.0
0 0.127590894699097 18.0
0 0.00731790065765381 19.0
0 0.0636149048805237 20.0
0 0.0168529748916626 21.0
0 0.291947245597839 22.0
36.143409194 35.5197410583496 23.0
0 0.0480623245239258 24.0
0 0.494688332080841 25.0
0 -0.142562091350555 26.0
0 -0.0815278291702271 27.0
0 0.313026398420334 28.0
0 -0.098347544670105 29.0
0 0.0484523773193359 30.0
0 -0.111794352531433 31.0
0 0.0393819808959961 32.0
0 -0.119199991226196 33.0
0 0.430175721645355 34.0
0 0.0241904854774475 35.0
0 -0.0281131267547607 36.0
0 -0.119804084300995 37.0
0 0.351106405258179 38.0
0 -0.51147472858429 39.0
0 0.206125319004059 40.0
0 -0.289703965187073 41.0
0 0.0307950973510742 42.0
301.81205815 296.907104492188 43.0
0 0.11475396156311 44.0
0 0.129642307758331 45.0
0 0.171729743480682 46.0
0 -0.255813717842102 47.0
0 0.19935017824173 48.0
0 0.514808535575867 49.0
0.08544656885 0.0197693705558777 50.0
1.8297222574 1.86606216430664 51.0
0 0.319307923316956 52.0
0 -0.160676181316376 53.0
0 -0.0931560397148132 54.0
0 -0.0115472674369812 55.0
0 0.244040489196777 56.0
0 -0.0719224214553833 57.0
0 0.318154960870743 58.0
0 0.320995926856995 59.0
0 -0.209429264068604 60.0
0 0.0704100728034973 61.0
0 0.27165961265564 62.0
256.60244087 253.214660644531 0.0
0 -0.143363833427429 1.0
0 -0.186266481876373 2.0
0 -0.0297108888626099 3.0
0 -0.133471071720123 4.0
0 0.227684736251831 5.0
0 0.200922310352325 6.0
0 -0.0704739093780518 7.0
0 0.00410652160644531 8.0
0 0.218679249286652 9.0
0 0.401894420385361 10.0
0 -0.280837297439575 11.0
0 -0.0839232206344604 12.0
69.917635869 72.6988143920898 13.0
30.147755862 27.8173313140869 14.0
0 -0.012959361076355 15.0
0 0.010414183139801 16.0
32.485373732 34.1872100830078 17.0
0 0.0844486355781555 18.0
0 -0.0415868759155273 19.0
0 0.0795343518257141 20.0
0 0.119072020053864 21.0
0 0.0572702884674072 22.0
27.996445706 27.0395488739014 23.0
0 -0.0123817920684814 24.0
0 0.0256591439247131 25.0
0 0.0135865211486816 26.0
0 -0.203276813030243 27.0
0 0.32234987616539 28.0
0 -0.0501354932785034 29.0
0 0.0479468703269958 30.0
0 -0.0261335372924805 31.0
0 0.0182486176490784 32.0
0 -0.0268411636352539 33.0
0 0.439792394638062 34.0
0 -0.156492114067078 35.0
0 0.0582429170608521 36.0
0 0.00553864240646362 37.0
0 0.290362060070038 38.0
0 -0.469489574432373 39.0
0 0.32436740398407 40.0
0 -0.18793910741806 41.0
0 0.0538212060928345 42.0
297.96687429 296.800323486328 43.0
0 0.142130732536316 44.0
0 0.016534686088562 45.0
0 0.0925975441932678 46.0
0 -0.0124789476394653 47.0
0 0.229383409023285 48.0
0 0.553654730319977 49.0
0.083651493473 0.00636774301528931 50.0
2.259056085 1.8605569601059 51.0
0 -0.10862398147583 52.0
0 0.178111851215363 53.0
0 -0.150108575820923 54.0
0 0.297460675239563 55.0
0 0.344756066799164 56.0
0 0.066835880279541 57.0
0 0.0586917996406555 58.0
0 0.038021445274353 59.0
0 -0.202783703804016 60.0
0 0.0384241342544556 61.0
0 0.0894898772239685 62.0
272.96606454 253.902221679688 0.0
0 -0.292057752609253 1.0
0 -0.0108084678649902 2.0
0 -0.0362265706062317 3.0
0 0.208732545375824 4.0
0 0.425220251083374 5.0
0 0.591812431812286 6.0
0 0.301442623138428 7.0
0 -0.249264717102051 8.0
0 0.202958703041077 9.0
0 -0.0771380662918091 10.0
0 0.127483665943146 11.0
0 -0.100547313690186 12.0
70.041546531 73.4500045776367 13.0
38.802781104 25.5232982635498 14.0
0 -0.0594546794891357 15.0
0 -0.188399970531464 16.0
30.823773941 34.3085899353027 17.0
0 0.0794070959091187 18.0
0 0.0100833773612976 19.0
0 0.0629329681396484 20.0
0 0.194351673126221 21.0
0 -0.00845217704772949 22.0
27.68642223 28.3650302886963 23.0
0 0.153221845626831 24.0
0 0.998503684997559 25.0
0 -0.0134427547454834 26.0
0 -0.0865381360054016 27.0
0 0.3508041203022 28.0
0 -0.0764139890670776 29.0
0 0.060655951499939 30.0
0 -0.190449953079224 31.0
0 0.0153911709785461 32.0
0 -0.139738976955414 33.0
0 0.32784503698349 34.0
0 -0.101502180099487 35.0
0 0.224449872970581 36.0
0 -0.015639066696167 37.0
0 0.0831583142280579 38.0
0 -0.443425893783569 39.0
0 0.98677259683609 40.0
0 -0.185107588768005 41.0
0 0.102163434028625 42.0
300.16613316 295.0283203125 43.0
0 0.0539947748184204 44.0
0 0.126551926136017 45.0
0 0.152483344078064 46.0
0 0.178871750831604 47.0
0 -0.275358557701111 48.0
0 0.327659487724304 49.0
0.087766014177 -0.00761854648590088 50.0
1.5782738992 1.8780300617218 51.0
0 0.236722767353058 52.0
0 -0.0544718503952026 53.0
0 0.113696098327637 54.0
0 0.302957445383072 55.0
0 0.269926786422729 56.0
0 0.454948663711548 57.0
0 0.253939747810364 58.0
0 0.187158584594727 59.0
0 -0.133460402488708 60.0
0 0.111803710460663 61.0
0 0.381504535675049 62.0
272.10312066 251.97395324707 0.0
0 -0.0140699148178101 1.0
0 -0.0919506549835205 2.0
0 -0.253531098365784 3.0
0 -0.386586666107178 4.0
0 0.169651329517365 5.0
0 0.568864047527313 6.0
0 0.166472971439362 7.0
0 -0.0129951238632202 8.0
0 0.118979632854462 9.0
0 -0.140012621879578 10.0
0 0.904781699180603 11.0
0 -0.204363703727722 12.0
49.290461569 44.2190437316895 13.0
33.459932538 37.2755393981934 14.0
0 -0.0227507352828979 15.0
0 -0.107190728187561 16.0
25.341852774 29.8511772155762 17.0
0 -0.0227944850921631 18.0
0 -0.0240033268928528 19.0
0 0.0547399520874023 20.0
0 0.183439552783966 21.0
0 0.0134304761886597 22.0
26.594372985 26.3844909667969 23.0
0 -0.0323408842086792 24.0
0 0.184317231178284 25.0
0 -0.0252934694290161 26.0
0 0.0802664160728455 27.0
0 -0.734010577201843 28.0
0 -0.104035615921021 29.0
0 -0.0355902910232544 30.0
0 0.399213671684265 31.0
0 0.0207546949386597 32.0
0 -0.20361602306366 33.0
0 0.0899774432182312 34.0
0 -0.0684010982513428 35.0
0 0.521714329719543 36.0
0 0.162641227245331 37.0
0 0.124223351478577 38.0
0 -0.457100987434387 39.0
0 0.151559829711914 40.0
0 -0.116172134876251 41.0
0 -0.101352572441101 42.0
282.67945155 293.849395751953 43.0
0 0.131145119667053 44.0
0 -0.0115178227424622 45.0
0 0.224609673023224 46.0
0 -0.452176690101624 47.0
0 0.317116379737854 48.0
0 0.296002000570297 49.0
0.079732273256 0.0544905662536621 50.0
1.7811991652 1.8634660243988 51.0
0 -0.123467743396759 52.0
0 0.156697511672974 53.0
0 0.147993803024292 54.0
0 0.289240300655365 55.0
0 0.164066076278687 56.0
0 0.889472186565399 57.0
0 -0.242006063461304 58.0
0 -0.147384881973267 59.0
0 -0.0942097306251526 60.0
0 0.139988362789154 61.0
0 0.169464290142059 62.0
222.25829195 251.914260864258 0.0
0 0.116333961486816 1.0
0 -0.0909108519554138 2.0
0 0.161982297897339 3.0
0 -0.338792443275452 4.0
0 -0.106163322925568 5.0
0 -0.120535671710968 6.0
0 -0.269887208938599 7.0
0 0.326345860958099 8.0
0 -0.00447559356689453 9.0
0 -0.00579386949539185 10.0
0 -0.00317895412445068 11.0
0 0.0367833375930786 12.0
70.232569249 71.3316650390625 13.0
31.057192014 27.9757537841797 14.0
0 -0.116456747055054 15.0
0 0.139240860939026 16.0
34.056307764 35.5177688598633 17.0
0 0.154346287250519 18.0
0 -0.0589537620544434 19.0
0 0.0910394191741943 20.0
0 0.118820548057556 21.0
0 0.303968042135239 22.0
35.485392222 35.7550773620605 23.0
0 0.10205340385437 24.0
0 0.750092506408691 25.0
0 -0.0675821900367737 26.0
0 -0.0781033635139465 27.0
0 0.217215776443481 28.0
0 -0.104734480381012 29.0
0 0.0574876070022583 30.0
0 -0.135222554206848 31.0
0 0.0175685882568359 32.0
0 0.00493240356445312 33.0
0 0.294636279344559 34.0
0 -0.0515404939651489 35.0
0 -0.0699859261512756 36.0
0 -0.0757550001144409 37.0
0 0.187144100666046 38.0
0 -0.257291316986084 39.0
0 0.273508727550507 40.0
0 -0.197996497154236 41.0
0 0.215416073799133 42.0
300.08842362 297.569519042969 43.0
0 0.202395260334015 44.0
0 0.107532262802124 45.0
0 0.200289845466614 46.0
0 0.0231909155845642 47.0
0 0.365472763776779 48.0
0 0.491325497627258 49.0
0.09150960594 0.0357233285903931 50.0
2.1570394777 1.82494854927063 51.0
0 -0.119062006473541 52.0
0 0.0838522911071777 53.0
0 -0.0419739484786987 54.0
0 -0.173004388809204 55.0
0 -0.1441850066185 56.0
0 0.265766620635986 57.0
0 0.00578218698501587 58.0
0 -0.0343428254127502 59.0
0 -0.0661587715148926 60.0
0 0.0486969351768494 61.0
0 0.17662787437439 62.0
263.36362287 251.83039855957 0.0
0 -0.213379383087158 1.0
0 -0.127426266670227 2.0
0 0.0771268010139465 3.0
0 0.0605027675628662 4.0
0 0.319161951541901 5.0
0 0.238955855369568 6.0
0 0.241948783397675 7.0
0 0.863250195980072 8.0
0 0.228970646858215 9.0
0 0.202064156532288 10.0
0 -0.0352181196212769 11.0
0 -0.0405259132385254 12.0
67.275669575 65.916374206543 13.0
24.536927009 35.2880859375 14.0
0 -0.00456202030181885 15.0
0 0.154987514019012 16.0
26.608947155 34.2694549560547 17.0
0 0.0987688302993774 18.0
0 -0.00368136167526245 19.0
0 0.0514479875564575 20.0
0 0.116343855857849 21.0
0 0.15804648399353 22.0
29.998122641 29.927059173584 23.0
0 0.0799316167831421 24.0
0 0.774712979793549 25.0
0 -0.106674075126648 26.0
0 -0.0141115188598633 27.0
0 0.263580501079559 28.0
0 0.137767672538757 29.0
0 0.139021873474121 30.0
0 -0.117680907249451 31.0
0 0.0406728982925415 32.0
0 0.0494837164878845 33.0
0 -0.0211837291717529 34.0
0 -0.166427910327911 35.0
0 0.31112003326416 36.0
0 -0.0717722177505493 37.0
0 0.223742723464966 38.0
0 -0.149993062019348 39.0
0 0.5693678855896 40.0
0 -0.12048602104187 41.0
0 -0.0476022958755493 42.0
286.73249986 297.394470214844 43.0
0 0.0359987616539001 44.0
0 0.0724291205406189 45.0
0 0.0879186987876892 46.0
0 0.191559791564941 47.0
0 0.163238883018494 48.0
0 0.363443732261658 49.0
0.069692944211 0.0318552851676941 50.0
2.238508584 1.90506386756897 51.0
0 0.180598318576813 52.0
0 0.0564436912536621 53.0
0 -0.103686451911926 54.0
0 0.0225127935409546 55.0
0 -0.120007812976837 56.0
0 -0.0907086133956909 57.0
0 0.299350708723068 58.0
0 -0.0408920645713806 59.0
0 -0.0913623571395874 60.0
0 0.146389007568359 61.0
0 0.416217923164368 62.0
292.60448179 254.735610961914 0.0
0 -0.442366361618042 1.0
0 0.132040739059448 2.0
0 -0.245003938674927 3.0
0 -0.20742654800415 4.0
0 0.425366818904877 5.0
0 0.108427166938782 6.0
0 0.345415830612183 7.0
0 0.44535619020462 8.0
0 0.172316193580627 9.0
0 0.321596056222916 10.0
0 -0.0692777633666992 11.0
0 -0.0697254538536072 12.0
57.901475166 59.7528419494629 13.0
43.752026033 37.8503379821777 14.0
0 -0.15630316734314 15.0
0 -0.115788578987122 16.0
28.216139949 32.7084121704102 17.0
0 0.10204291343689 18.0
0 0.0313599109649658 19.0
0 0.0909517407417297 20.0
0 0.0173643827438354 21.0
0 0.340388983488083 22.0
34.912379346 35.3819313049316 23.0
0 0.113385558128357 24.0
0.18583781012 0.615131497383118 25.0
0 0.0217423439025879 26.0
0 -0.107102751731873 27.0
0 0.560833632946014 28.0
0 0.0310605764389038 29.0
0 -0.0573952198028564 30.0
0 0.18461012840271 31.0
0 0.13736891746521 32.0
0 -0.167618632316589 33.0
0 0.0596716403961182 34.0
0 0.102334141731262 35.0
0 0.198115348815918 36.0
0 -0.0515351295471191 37.0
0 0.200165629386902 38.0
0 -0.0982849597930908 39.0
0 -0.00547319650650024 40.0
0 -0.167890608310699 41.0
0 -9.85860824584961e-05 42.0
308.35272351 297.616455078125 43.0
0 0.260331571102142 44.0
0 -0.248010158538818 45.0
0 0.231211066246033 46.0
0 -0.130021035671234 47.0
0 -0.0188630223274231 48.0
0 0.403475850820541 49.0
0.07752868841 0.0318629741668701 50.0
2.1447460787 1.85763812065125 51.0
0 0.160535097122192 52.0
0 -0.206054270267487 53.0
0 0.0275980234146118 54.0
0 0.0604867935180664 55.0
0 0.156435310840607 56.0
0 0.128273785114288 57.0
0 0.0591066479682922 58.0
0 -0.131978094577789 59.0
0 -0.0343879461288452 60.0
0 0.0124664902687073 61.0
0 0.0704931616783142 62.0
255.13725937 252.371353149414 0.0
0 -0.180951714515686 1.0
0 -0.187248289585114 2.0
0 0.385158896446228 3.0
0 -0.0612407326698303 4.0
0 0.1295445561409 5.0
0 0.0120308995246887 6.0
0 0.190696597099304 7.0
0 0.332063585519791 8.0
0 -0.168615520000458 9.0
0 0.0254217982292175 10.0
0 -0.228011727333069 11.0
0 -0.0497930645942688 12.0
54.453909777 54.5110778808594 13.0
45.174893065 42.3723335266113 14.0
0 -0.0166416168212891 15.0
0 0.237743556499481 16.0
36.926493281 34.2880210876465 17.0
0 0.140690326690674 18.0
0 -0.046937108039856 19.0
0 0.0954711437225342 20.0
0 0.0475358963012695 21.0
0 0.311594367027283 22.0
34.781931745 34.6355133056641 23.0
0 0.0927053689956665 24.0
0 1.11414670944214 25.0
0 0.143545150756836 26.0
0 -0.0517325401306152 27.0
0 -0.237568736076355 28.0
0 -0.15091747045517 29.0
0 0.0767756104469299 30.0
0 0.031519889831543 31.0
0 -0.0139460563659668 32.0
0 -0.177485406398773 33.0
0 0.390420317649841 34.0
0 0.0535752773284912 35.0
0 0.0380749106407166 36.0
0 0.100110650062561 37.0
0 0.00451910495758057 38.0
0 -0.203158617019653 39.0
0 0.474752187728882 40.0
0 -0.10949182510376 41.0
0 0.100259780883789 42.0
293.90768573 296.823638916016 43.0
0 0.137871742248535 44.0
0 0.0263704657554626 45.0
0 0.176579415798187 46.0
0 0.0619596838951111 47.0
0 0.0361067056655884 48.0
0 0.372647702693939 49.0
0.086099480986 0.0233941674232483 50.0
1.8814101777 1.8744912147522 51.0
0 0.0709474086761475 52.0
0 -0.419144868850708 53.0
0 -0.0263451933860779 54.0
0 -0.0221575498580933 55.0
0 0.154189705848694 56.0
0 0.286344528198242 57.0
0 0.125808238983154 58.0
0 0.221869111061096 59.0
0 -0.162340641021729 60.0
0 0.102184295654297 61.0
0 0.250161647796631 62.0
213.019486 254.113403320312 0.0
0 0.125222206115723 1.0
0 -0.159312427043915 2.0
0 0.0619837045669556 3.0
0 -0.469054102897644 4.0
0 -0.475976347923279 5.0
0 -0.0745947360992432 6.0
0 -0.213638305664062 7.0
0 0.0934817790985107 8.0
0 0.0210124254226685 9.0
0 0.226676881313324 10.0
0 -0.142399370670319 11.0
0 -0.129868268966675 12.0
73.783142672 70.048210144043 13.0
22.265905218 30.5679340362549 14.0
0 -0.0583975315093994 15.0
0 -0.077064037322998 16.0
38.841657731 35.240779876709 17.0
0 0.11576509475708 18.0
0 0.056219220161438 19.0
0 0.0449203848838806 20.0
0 0.0790018439292908 21.0
0 -0.1615971326828 22.0
38.845232914 39.0730209350586 23.0
0 0.146781206130981 24.0
0 0.88190758228302 25.0
0 0.150759875774384 26.0
0 -0.0218322277069092 27.0
0 -0.24776017665863 28.0
0 -0.153577864170074 29.0
0 0.0768685340881348 30.0
0 0.090643048286438 31.0
0 0.0221507549285889 32.0
0 -0.0258327126502991 33.0
0 0.30767285823822 34.0
0 0.21732097864151 35.0
0 0.443172544240952 36.0
0 -0.0378499031066895 37.0
0 0.350895404815674 38.0
0 -0.229626178741455 39.0
0 -0.128035366535187 40.0
0 -0.259030103683472 41.0
0 0.0424174070358276 42.0
294.32722516 296.930694580078 43.0
0 0.0132757425308228 44.0
0 0.21820455789566 45.0
0 0.299999594688416 46.0
0 0.0926564931869507 47.0
0 -0.574466705322266 48.0
0 0.431682556867599 49.0
0.10141859103 0.0100102424621582 50.0
1.80237172 1.83819675445557 51.0
0 0.322553932666779 52.0
0 -0.146558165550232 53.0
0 -0.137081563472748 54.0
0 0.212719261646271 55.0
0 0.203685820102692 56.0
0 0.130065083503723 57.0
0 0.441257059574127 58.0
0 -0.0946513414382935 59.0
0 0.894136548042297 60.0
0 0.0252782106399536 61.0
0 0.104133307933807 62.0
271.52311652 254.18830871582 0.0
0 -0.268235802650452 1.0
0 -0.0362012386322021 2.0
0 -0.0543298125267029 3.0
0 0.0937473773956299 4.0
0 0.360091835260391 5.0
0 0.32609236240387 6.0
0 0.270941376686096 7.0
0 0.432517290115356 8.0
0 -0.205929577350616 9.0
0 -0.0513905882835388 10.0
0 -0.0411126613616943 11.0
0 -0.18613475561142 12.0
70.82903193 70.9754257202148 13.0
25.536232397 25.1812114715576 14.0
0 -0.0364747047424316 15.0
0 -0.13622260093689 16.0
30.653274688 32.8956565856934 17.0
0 0.10675048828125 18.0
0 -0.0416222810745239 19.0
0 0.107399225234985 20.0
0 0.159783720970154 21.0
0 0.077711820602417 22.0
29.466170284 27.9473114013672 23.0
0 0.0829039216041565 24.0
0 1.04970490932465 25.0
0 -0.0516277551651001 26.0
0 -0.25829815864563 27.0
0 0.459548056125641 28.0
0 -0.00575900077819824 29.0
0 0.0314865708351135 30.0
0 -0.265362024307251 31.0
0 -0.0751202702522278 32.0
0 -0.112682700157166 33.0
0 0.275342285633087 34.0
0 0.253126800060272 35.0
0 0.360381305217743 36.0
0 -0.0323799252510071 37.0
0 0.148159146308899 38.0
0 -0.191441535949707 39.0
0 0.815808057785034 40.0
0 -0.0566372871398926 41.0
0 0.141017913818359 42.0
304.04595173 296.385467529297 43.0
0 0.0817661285400391 44.0
0 0.159071385860443 45.0
0 0.0931620597839355 46.0
0 0.0614557266235352 47.0
0 0.204340577125549 48.0
0.72893303034 0.441619694232941 49.0
0.095219236659 0.0487908124923706 50.0
2.2450697539 1.8732293844223 51.0
0 0.337340474128723 52.0
0 0.0809569954872131 53.0
0 0.127200961112976 54.0
0 -0.00586479902267456 55.0
0 -0.0907276272773743 56.0
0 0.526914536952972 57.0
0 0.526024401187897 58.0
0 0.191685914993286 59.0
0 0.26533567905426 60.0
0 0.11406022310257 61.0
0 0.277703285217285 62.0
234.01249151 251.379638671875 0.0
0 0.257401704788208 1.0
0 -0.201161623001099 2.0
0 -0.948011875152588 3.0
0 -0.402771234512329 4.0
0 -0.238010048866272 5.0
0 -0.275944352149963 6.0
0 -0.195746004581451 7.0
0 -0.00899398326873779 8.0
0 0.213211417198181 9.0
0 -0.0755610466003418 10.0
0 0.0667320489883423 11.0
0 -0.247779130935669 12.0
57.855332768 57.8877487182617 13.0
48.617548726 39.703987121582 14.0
0 0.829921007156372 15.0
0 0.0314409136772156 16.0
34.474307471 32.8975296020508 17.0
0 0.133343458175659 18.0
0 0.0231351852416992 19.0
0 0.0364282131195068 20.0
0 0.300255805253983 21.0
0 -0.308688163757324 22.0
24.736601216 23.633092880249 23.0
0 0.0224679708480835 24.0
0 0.269284784793854 25.0
0 0.138466060161591 26.0
0 -0.263222932815552 27.0
0 -0.0577042102813721 28.0
0 0.151792347431183 29.0
0 0.241281569004059 30.0
0 -0.0300211906433105 31.0
0 0.112285196781158 32.0
0 -0.155422985553741 33.0
0 -0.440407276153564 34.0
0 0.274060964584351 35.0
0 0.30927050113678 36.0
0 0.187373876571655 37.0
0 0.190243721008301 38.0
0 -0.335029602050781 39.0
0 0.166958928108215 40.0
0 -0.213622450828552 41.0
0 -0.0178488492965698 42.0
309.78640244 297.335571289062 43.0
0 -0.00281083583831787 44.0
0 0.157743871212006 45.0
0 0.251296103000641 46.0
0 -0.361798644065857 47.0
0 0.10595577955246 48.0
0 0.133254408836365 49.0
0.093012964264 0.105051577091217 50.0
1.4458678613 1.84443926811218 51.0
0 -0.193020164966583 52.0
0 0.635710656642914 53.0
0 -0.104288756847382 54.0
0 -0.243870854377747 55.0
0 0.0318498611450195 56.0
0 0.0838905572891235 57.0
0 0.546308994293213 58.0
0 -0.142130792140961 59.0
0 -0.313007235527039 60.0
0 0.0562439560890198 61.0
0 0.150766551494598 62.0
288.61563253 249.827285766602 0.0
0 -0.0444594621658325 1.0
0 -0.0204625725746155 2.0
0 0.0777629017829895 3.0
0 0.0163586139678955 4.0
0 -0.242080211639404 5.0
0 -0.545392990112305 6.0
0 -0.122818470001221 7.0
0 -0.212116479873657 8.0
0 0.184905230998993 9.0
0 0.0408507585525513 10.0
0 -0.229793787002563 11.0
0 -0.0628287792205811 12.0
56.716801511 53.3610191345215 13.0
35.499780911 33.9427032470703 14.0
0 0.00817787647247314 15.0
0 -0.162559866905212 16.0
34.03683588 33.7745895385742 17.0
0 0.0537867546081543 18.0
0 -0.0114501714706421 19.0
0 0.0659396648406982 20.0
0 0.133297085762024 21.0
0 0.165163695812225 22.0
33.182667881 30.8545360565186 23.0
0 -0.0466810464859009 24.0
0 0.250178515911102 25.0
0 0.0583868026733398 26.0
0 -0.246424317359924 27.0
0 0.363628417253494 28.0
0 0.244663715362549 29.0
0 0.0694583654403687 30.0
0 -0.247289657592773 31.0
0 -0.192224740982056 32.0
0 0.0842463970184326 33.0
0 0.376831203699112 34.0
0 -0.0629035234451294 35.0
0 -0.324531555175781 36.0
0 0.0560908913612366 37.0
0 -0.154784619808197 38.0
0 -0.111410617828369 39.0
0 0.0361764430999756 40.0
0 -0.19027578830719 41.0
0 0.0620763301849365 42.0
279.58703095 296.092590332031 43.0
0 0.165502309799194 44.0
0 0.132593035697937 45.0
0 0.1639044880867 46.0
0 -0.115904569625854 47.0
0 -0.646620869636536 48.0
0 0.387756198644638 49.0
0.10559023829 0.0343230366706848 50.0
1.3893158872 1.80461525917053 51.0
0 0.302696406841278 52.0
0 -0.333014726638794 53.0
0 0.0687329769134521 54.0
0 0.219017565250397 55.0
0 0.307856142520905 56.0
0 0.403318494558334 57.0
0 -0.111628770828247 58.0
0 0.331192374229431 59.0
0 -0.18481320142746 60.0
0 0.205069899559021 61.0
0 0.315849781036377 62.0
250.32358406 250.15071105957 0.0
0 -0.232559442520142 1.0
0 0.0456334352493286 2.0
0 0.146097898483276 3.0
0 0.0681354999542236 4.0
0 0.144868195056915 5.0
0 0.0886762738227844 6.0
0 -0.188998818397522 7.0
0 0.0773304104804993 8.0
0 0.286849975585938 9.0
0 -0.477783918380737 10.0
0 -0.193668603897095 11.0
0 0.00587046146392822 12.0
60.267495789 58.7533760070801 13.0
36.83290804 34.9797058105469 14.0
0 0.00635766983032227 15.0
0 -0.154625773429871 16.0
33.940891289 34.6849136352539 17.0
0 0.0776690244674683 18.0
0 -0.0607483386993408 19.0
0 0.13203376531601 20.0
0 0.160638451576233 21.0
0 0.196857154369354 22.0
37.323872616 35.0037231445312 23.0
0 0.0564443469047546 24.0
0 0.419628977775574 25.0
0 -0.139333724975586 26.0
0 0.0745803117752075 27.0
0 0.350394278764725 28.0
0 0.0301071405410767 29.0
0 -0.00657260417938232 30.0
0 0.0388039350509644 31.0
0 0.0303264856338501 32.0
0 -0.0505481958389282 33.0
0 0.0093611478805542 34.0
0 0.251467287540436 35.0
0 0.0579349994659424 36.0
0 -0.00316911935806274 37.0
0 0.114918768405914 38.0
0 -0.128724098205566 39.0
0 0.520416736602783 40.0
0 -0.148386299610138 41.0
0 0.142623066902161 42.0
298.56314066 296.299713134766 43.0
0 0.0966399312019348 44.0
0 0.0769103169441223 45.0
0 0.13298487663269 46.0
0 -0.133209407329559 47.0
0 0.373342335224152 48.0
0.35393219234 0.512939214706421 49.0
0.098564822749 0.0987635850906372 50.0
1.5658450335 1.85271167755127 51.0
0 -0.0427040457725525 52.0
0 -0.13286828994751 53.0
0 0.00212854146957397 54.0
0 -0.0134220719337463 55.0
0 -0.0751634836196899 56.0
0 -0.0611813068389893 57.0
0 0.295413047075272 58.0
0 0.280274331569672 59.0
0 -0.164864778518677 60.0
0 0.129502713680267 61.0
0 0.225205421447754 62.0
214.1020204 253.389144897461 0.0
0 -0.0564684867858887 1.0
0 -0.122829794883728 2.0
0 0.132049977779388 3.0
0 -0.524369716644287 4.0
0 -0.452628493309021 5.0
0 -0.229432582855225 6.0
0 -0.230602025985718 7.0
0 -0.275671720504761 8.0
0 0.160399854183197 9.0
0 -0.196205377578735 10.0
0 -0.261906266212463 11.0
0 -0.0703197121620178 12.0
67.385368973 66.1575164794922 13.0
13.399399728 36.1903381347656 14.0
0 0.0278534889221191 15.0
0 -0.13518100976944 16.0
41.613555255 39.0026817321777 17.0
0 0.17172384262085 18.0
0 -0.0514596700668335 19.0
0 0.0312768220901489 20.0
0 0.158721566200256 21.0
0 0.0742427110671997 22.0
27.063185977 26.2190685272217 23.0
0 -0.0543335676193237 24.0
0 0.893873751163483 25.0
0 -0.181379556655884 26.0
0 -0.398697137832642 27.0
0 0.958756148815155 28.0
0 0.065106987953186 29.0
0 0.0567975640296936 30.0
0 -0.0567233562469482 31.0
0 -0.00926738977432251 32.0
0 -0.0767238140106201 33.0
0 0.0652605891227722 34.0
0 0.216737926006317 35.0
0 0.431905895471573 36.0
0 0.0568116307258606 37.0
0 0.154834270477295 38.0
0 -0.183176517486572 39.0
0 0.45319139957428 40.0
0 -0.172079563140869 41.0
0 0.161592960357666 42.0
311.35902477 296.830871582031 43.0
0 0.166166067123413 44.0
0 0.128586530685425 45.0
0 0.097590446472168 46.0
0 0.300102651119232 47.0
0 0.0914199352264404 48.0
0 0.489501774311066 49.0
0.091164836751 0.00627535581588745 50.0
2.0366008191 1.87804245948792 51.0
0 0.0298047065734863 52.0
0 -0.427032589912415 53.0
0 0.0822892785072327 54.0
0 0.258531033992767 55.0
0 0.0483702421188354 56.0
0 0.304383516311646 57.0
0 -0.0151056051254272 58.0
0 -0.973348021507263 59.0
0 1.24251770973206 60.0
0 0.0387481451034546 61.0
0 0.0477080941200256 62.0
266.58729332 253.952651977539 0.0
0 -0.262078285217285 1.0
0 -0.0934077501296997 2.0
0 0.15440559387207 3.0
0 0.143146514892578 4.0
0 0.205938637256622 5.0
0 -0.252820372581482 6.0
0 0.17575204372406 7.0
0 -0.41801643371582 8.0
0 0.0819855928421021 9.0
0 0.118514239788055 10.0
0 -0.115362823009491 11.0
0 -0.213129878044128 12.0
66.590593742 67.222526550293 13.0
46.924865732 31.7724895477295 14.0
0 -0.0694806575775146 15.0
0 -0.0884276628494263 16.0
28.054461186 32.7139549255371 17.0
0 0.0860182046890259 18.0
0 0.0459686517715454 19.0
0 0.0690410137176514 20.0
0 0.0499482154846191 21.0
0 0.0340030789375305 22.0
38.251210974 37.6608009338379 23.0
0 0.0924009680747986 24.0
0 0.987218201160431 25.0
0 -0.0466823577880859 26.0
0 -0.0369865298271179 27.0
0 0.630582392215729 28.0
0 -0.0370510816574097 29.0
0 0.0581210851669312 30.0
0 0.0561529397964478 31.0
0 0.0673874616622925 32.0
0 -0.110272288322449 33.0
0 -0.117069721221924 34.0
0 0.293586850166321 35.0
0 0.0817803144454956 36.0
0 -0.0250941514968872 37.0
0 -0.020530641078949 38.0
0 -0.0696612596511841 39.0
0 0.576715052127838 40.0
0 -0.0721520781517029 41.0
0 -0.0310094356536865 42.0
309.92790288 298.993560791016 43.0
0 0.164634466171265 44.0
0 0.00410723686218262 45.0
0 0.218127131462097 46.0
0 0.126537203788757 47.0
0 -0.0732787251472473 48.0
0 0.356878340244293 49.0
0.078719991196 -0.0287966728210449 50.0
1.8622598671 1.87732887268066 51.0
0 0.160207033157349 52.0
0 -0.109865665435791 53.0
0 0.0813998579978943 54.0
0 0.0950701832771301 55.0
0 -0.114027976989746 56.0
0 0.453118085861206 57.0
0 0.362273037433624 58.0
0 0.195397198200226 59.0
0 -0.239032864570618 60.0
0 0.0702812671661377 61.0
0 0.169132351875305 62.0
261.96181176 254.408981323242 0.0
0 -0.409701228141785 1.0
0 -0.0376662611961365 2.0
0 -0.109899759292603 3.0
0 -0.26451301574707 4.0
0 0.471322417259216 5.0
0 0.757445216178894 6.0
0 0.457967638969421 7.0
0 -0.23664402961731 8.0
0 0.375360548496246 9.0
0 0.0771906971931458 10.0
0 -0.0267562866210938 11.0
0 -0.152124166488647 12.0
76.270477982 76.8286895751953 13.0
32.316074149 25.6818027496338 14.0
0 -0.200525760650635 15.0
0 -0.18794709444046 16.0
25.12726935 33.4812202453613 17.0
0 0.0353002548217773 18.0
0 -0.0171895027160645 19.0
0 0.127544522285461 20.0
0 -0.0833983421325684 21.0
0 1.01073431968689 22.0
41.131144405 40.7091903686523 23.0
0 0.119542002677917 24.0
19.832536234 1.04690074920654 25.0
0 0.0189235210418701 26.0
0 0.105576038360596 27.0
0 -0.0694465637207031 28.0
0 -0.0463531613349915 29.0
0 0.0511345863342285 30.0
0 -0.058991551399231 31.0
0 0.0776750445365906 32.0
0 -0.388236522674561 33.0
0 0.222644329071045 34.0
0 0.0745317339897156 35.0
0 -0.443504810333252 36.0
0 -0.052461564540863 37.0
0 0.162205994129181 38.0
0 -0.247417330741882 39.0
0 0.446186125278473 40.0
0 -0.176878094673157 41.0
0 0.0783026218414307 42.0
291.34830089 296.473876953125 43.0
0 0.107142090797424 44.0
0 0.149406671524048 45.0
0 0.232483446598053 46.0
0 -0.137866675853729 47.0
0 0.370875716209412 48.0
0 0.410353183746338 49.0
0.078672534299 0.0267317295074463 50.0
2.1571340247 1.90002679824829 51.0
0 -0.127104043960571 52.0
0 -0.287091732025146 53.0
0 0.0541608929634094 54.0
0 0.0821243524551392 55.0
0 0.124601185321808 56.0
0 0.33033674955368 57.0
0 0.300002634525299 58.0
0 0.206436574459076 59.0
0 -0.203694820404053 60.0
0 0.105886340141296 61.0
0 0.405011296272278 62.0
255.58420524 253.293395996094 0.0
0 -0.309851050376892 1.0
0 0.0567983388900757 2.0
0 0.205911934375763 3.0
0 0.0143100023269653 4.0
0 0.135208070278168 5.0
0 0.0726622343063354 6.0
0 0.00269556045532227 7.0
0 -0.471062541007996 8.0
0 -2.40802764892578e-05 9.0
0 -0.00600981712341309 10.0
0 0.34476500749588 11.0
0 -0.087411642074585 12.0
52.563922912 52.9289093017578 13.0
41.889489708 40.663745880127 14.0
0 -0.188837885856628 15.0
0 0.0230676531791687 16.0
33.898973105 36.5564002990723 17.0
0 0.13017201423645 18.0
0 -0.0461522340774536 19.0
0 0.0709013938903809 20.0
0 -0.0345274209976196 21.0
0 0.248779237270355 22.0
35.872247347 35.8003578186035 23.0
0 0.0424273014068604 24.0
0 0.269687712192535 25.0
0 -0.0893326997756958 26.0
0 -0.0460659861564636 27.0
0 -0.346218943595886 28.0
0 -0.0766074061393738 29.0
0 0.00609564781188965 30.0
0 0.119581937789917 31.0
0 0.0485889911651611 32.0
0 -0.117977976799011 33.0
0 0.398056238889694 34.0
0 0.27169394493103 35.0
0 0.419741809368134 36.0
0 -0.00795328617095947 37.0
0 0.130366086959839 38.0
0 -0.156887412071228 39.0
0 0.394516229629517 40.0
0 -0.241924643516541 41.0
0 -0.0317788124084473 42.0
295.26043613 297.564666748047 43.0
0 0.0802344083786011 44.0
0 0.00376147031784058 45.0
0 0.257401645183563 46.0
0 0.146074593067169 47.0
0 0.0968580842018127 48.0
0 0.382081240415573 49.0
0.090542566347 0.0125861763954163 50.0
1.8092060272 1.85840427875519 51.0
0 0.0487298965454102 52.0
0 -0.0618197917938232 53.0
0 -0.0769853591918945 54.0
0 0.0874772071838379 55.0
0 0.0887919664382935 56.0
0 0.188041210174561 57.0
0 0.141901612281799 58.0
0 0.0726897120475769 59.0
0 -0.0913776755332947 60.0
0 0.000487804412841797 61.0
0 0.0926957130432129 62.0
216.12449871 254.102188110352 0.0
0 -0.0726544857025146 1.0
0 -0.0239802598953247 2.0
0 0.0269640684127808 3.0
0 -0.1908860206604 4.0
0 -0.194316565990448 5.0
0 -0.0873211622238159 6.0
0 -0.172250747680664 7.0
0 0.0382657051086426 8.0
0 0.297091424465179 9.0
0 1.46753585338593 10.0
0 -0.228818416595459 11.0
0 -0.176408529281616 12.0
71.999606471 72.4804763793945 13.0
16.026230369 26.0268955230713 14.0
0 -0.186797142028809 15.0
0 0.0508198738098145 16.0
33.734032877 33.4641647338867 17.0
0 0.0220181345939636 18.0
0 0.0124411582946777 19.0
0 0.0820473432540894 20.0
0 0.0353100299835205 21.0
0 -0.0978756546974182 22.0
40.43445633 38.4267768859863 23.0
0 0.0541142225265503 24.0
0 0.824119329452515 25.0
0 -0.0603845119476318 26.0
0 0.0925571322441101 27.0
0 0.360931634902954 28.0
0 -0.0863080024719238 29.0
0 0.0199710130691528 30.0
0 0.226278185844421 31.0
0 0.145370244979858 32.0
0 -0.26559591293335 33.0
0 0.247629284858704 34.0
0 -0.0887491703033447 35.0
0 0.0235424041748047 36.0
0 0.126263499259949 37.0
0 0.0889546275138855 38.0
0 -0.177867650985718 39.0
0 0.430087715387344 40.0
0 -0.126291275024414 41.0
0 0.179405450820923 42.0
289.96589452 296.708374023438 43.0
0 0.238109230995178 44.0
0 -0.0187127590179443 45.0
0 0.230929672718048 46.0
0 -0.103871285915375 47.0
0 0.324616134166718 48.0
1.4913021635 0.591868460178375 49.0
0.068805983454 0.0463987588882446 50.0
1.9151616894 1.8898983001709 51.0
0 0.0755639672279358 52.0
0 -0.139696776866913 53.0
0 -0.101941823959351 54.0
0 0.170118689537048 55.0
0 0.0394018888473511 56.0
0 0.0539402961730957 57.0
0 -0.0394071936607361 58.0
0 -0.00280857086181641 59.0
0 0.0179018974304199 60.0
0 0.0217845439910889 61.0
0 0.091326892375946 62.0
281.76062801 251.546432495117 0.0
0 -0.0437947511672974 1.0
0 -0.0259609222412109 2.0
0 0.309295624494553 3.0
0 -0.254772782325745 4.0
0 0.0474111437797546 5.0
0 0.722702622413635 6.0
0 0.378648579120636 7.0
0 0.19706791639328 8.0
0 0.441581249237061 9.0
0 -0.224340200424194 10.0
0 0.409581422805786 11.0
0 -0.175478160381317 12.0
71.713724232 73.1538009643555 13.0
37.222719945 28.0871162414551 14.0
0 -0.109794855117798 15.0
0 0.0377470850944519 16.0
33.239477933 32.0240631103516 17.0
0 0.0925908088684082 18.0
0 0.00619965791702271 19.0
0 0.0570266246795654 20.0
0 0.176983594894409 21.0
0 0.25053071975708 22.0
34.03275686 33.7825927734375 23.0
0 0.0845191478729248 24.0
0 1.09399080276489 25.0
0 -0.00107628107070923 26.0
0 -0.236878752708435 27.0
0 0.211833000183105 28.0
0 -0.00140500068664551 29.0
0 0.0326187610626221 30.0
0 0.0296381711959839 31.0
0 0.158263027667999 32.0
0 -0.0917243361473083 33.0
0 0.169124603271484 34.0
0 -0.144273281097412 35.0
0 0.0402433276176453 36.0
0 0.0138065218925476 37.0
0 0.0848998427391052 38.0
0 -0.158060669898987 39.0
0 0.829571127891541 40.0
0 -0.0951066613197327 41.0
0 0.403860926628113 42.0
295.35717102 293.172760009766 43.0
0 0.00882637500762939 44.0
0 0.263205230236053 45.0
0 0.10003650188446 46.0
0 -0.0209540724754333 47.0
0 -0.100570142269135 48.0
0 0.522109150886536 49.0
0.088910956254 -0.00423222780227661 50.0
2.0012011238 1.87271630764008 51.0
0 -0.0211716294288635 52.0
0 -0.263736486434937 53.0
0 0.129735767841339 54.0
0 0.0857000350952148 55.0
0 -0.196541666984558 56.0
0 -0.0899651050567627 57.0
0 0.212137997150421 58.0
0 0.220216810703278 59.0
0 -0.208904206752777 60.0
0 0.116376399993896 61.0
0 0.304611504077911 62.0
248.4974182 252.644775390625 0.0
0 0.095786452293396 1.0
0 -0.0936058163642883 2.0
0 -0.0894041657447815 3.0
0 -0.359108448028564 4.0
0 -0.168134748935699 5.0
0 0.158855736255646 6.0
0 -0.0868022441864014 7.0
0 0.157493472099304 8.0
0 -0.155079901218414 9.0
0 0.15269547700882 10.0
0 0.0090218186378479 11.0
0 -0.05258709192276 12.0
53.73577113 54.6455230712891 13.0
40.033022614 40.5559730529785 14.0
0 -0.0709562301635742 15.0
0 0.0878087282180786 16.0
33.320084664 35.0413932800293 17.0
0 0.11157363653183 18.0
0 -0.0210295915603638 19.0
0 0.00641113519668579 20.0
0 0.219536185264587 21.0
0 -0.0207697153091431 22.0
29.539669225 28.9062175750732 23.0
0 0.0797868967056274 24.0
0 0.613589823246002 25.0
0 -0.0215722918510437 26.0
0 -0.176420569419861 27.0
0 -0.14243757724762 28.0
0 -0.0286824107170105 29.0
0 0.0407490730285645 30.0
0 -0.0499860048294067 31.0
0 0.00278168916702271 32.0
0 -0.300751447677612 33.0
0 0.0149199366569519 34.0
0 0.236433029174805 35.0
0 0.642411828041077 36.0
0 -0.0938840508460999 37.0
0 0.347642153501511 38.0
0 -0.333752989768982 39.0
0 0.0781186819076538 40.0
0 -0.204589486122131 41.0
0 0.0405281782150269 42.0
302.90434079 297.287872314453 43.0
0 0.0279164910316467 44.0
0 0.0254055261611938 45.0
0 0.0980377793312073 46.0
0 -0.0110952854156494 47.0
0 0.130662858486176 48.0
0 0.35025542974472 49.0
0.080782961659 0.0351237058639526 50.0
1.9569118886 1.87469553947449 51.0
0 0.0275655388832092 52.0
0 -0.245517253875732 53.0
0 -0.0596277117729187 54.0
0 0.202234148979187 55.0
0 -0.0036550760269165 56.0
0 0.034500777721405 57.0
0 0.530868470668793 58.0
0 0.288514733314514 59.0
0 -0.141226530075073 60.0
0 0.112863779067993 61.0
0 0.276054203510284 62.0
240.74163718 252.087097167969 0.0
0 -0.245644688606262 1.0
0 -0.147529244422913 2.0
0 0.016473114490509 3.0
0 -0.103096485137939 4.0
0 0.0712063908576965 5.0
0 0.182798385620117 6.0
0 -0.0233105421066284 7.0
0 -0.661567449569702 8.0
0 0.149806022644043 9.0
0 0.357561767101288 10.0
0 -0.271966934204102 11.0
0 0.0178672671318054 12.0
55.298184691 54.9900398254395 13.0
29.44232705 41.8721542358398 14.0
0 0.0224192142486572 15.0
0 -0.0863486528396606 16.0
34.202829277 33.4820404052734 17.0
0 0.117728412151337 18.0
0 -0.0481947660446167 19.0
0 0.299984127283096 20.0
0 -0.12354439496994 21.0
0 0.0902228951454163 22.0
41.510042373 38.4922714233398 23.0
0 0.035173773765564 24.0
0 0.27035003900528 25.0
0 -0.0173053741455078 26.0
0 -0.125004827976227 27.0
0 0.588372528553009 28.0
0 0.0417743325233459 29.0
0 -0.0369727611541748 30.0
0 0.076809287071228 31.0
0 0.054561972618103 32.0
0 -0.270580053329468 33.0
0 0.385594427585602 34.0
0 -0.106662034988403 35.0
0 0.262339115142822 36.0
0 -0.0769504308700562 37.0
0 0.220633447170258 38.0
0 -0.260254859924316 39.0
0 0.395873218774796 40.0
0 -0.0682108402252197 41.0
0 0.11103880405426 42.0
296.43050365 296.978973388672 43.0
0 0.198492050170898 44.0
0 0.0400528311729431 45.0
0 0.172328412532806 46.0
0 0.26324063539505 47.0
0 0.342923790216446 48.0
0 0.485491186380386 49.0
0.078597635548 0.0598583221435547 50.0
1.9626192259 1.82127809524536 51.0
0 -0.0864719152450562 52.0
0 -0.559011697769165 53.0
0 0.100963294506073 54.0
0 0.102991044521332 55.0
0 0.0544503927230835 56.0
0 0.094069242477417 57.0
0 0.220780193805695 58.0
0 0.256677329540253 59.0
0 0.0967764854431152 60.0
0 0.045541524887085 61.0
0 0.0980968475341797 62.0
281.47580065 253.948913574219 0.0
0 0.129021406173706 1.0
0 0.132755875587463 2.0
0 0.293286681175232 3.0
0 0.0221982002258301 4.0
0 -0.0997462868690491 5.0
0 1.04126107692719 6.0
0 -0.351293206214905 7.0
0 -0.00529015064239502 8.0
0 -0.0498906373977661 9.0
0 0.162920653820038 10.0
0 0.00167852640151978 11.0
0 -0.166333138942719 12.0
65.369333217 66.4459762573242 13.0
38.10009818 27.05344581604 14.0
0 0.170454859733582 15.0
0 -0.0100815296173096 16.0
36.107295901 34.4790458679199 17.0
0 0.14742112159729 18.0
0 0.0602411031723022 19.0
0 0.170227825641632 20.0
0 0.00113916397094727 21.0
0 -0.153998613357544 22.0
41.036021393 41.0633697509766 23.0
0 0.0268422365188599 24.0
0 1.02105593681335 25.0
0 -0.108914852142334 26.0
0 0.149244070053101 27.0
0 0.178989291191101 28.0
0 0.00421321392059326 29.0
0 0.153756618499756 30.0
0 -0.271771550178528 31.0
0 -0.106852650642395 32.0
0 -0.29461145401001 33.0
0 0.107440233230591 34.0
0 0.196555256843567 35.0
0 0.545078635215759 36.0
0 -0.0918022394180298 37.0
0 0.191715240478516 38.0
0 -0.228934764862061 39.0
0 0.327586561441422 40.0
0 -0.0076184868812561 41.0
0 -0.168375015258789 42.0
293.33861992 296.759155273438 43.0
0 0.0627985000610352 44.0
0 0.25788950920105 45.0
0 0.132488429546356 46.0
0 0.0316382050514221 47.0
0 -0.219282031059265 48.0
0 0.416543364524841 49.0
0.069263331402 -0.022835373878479 50.0
2.2690501695 1.8598427772522 51.0
0 -0.0525200366973877 52.0
0 0.0856939554214478 53.0
0 -0.111654579639435 54.0
0 -0.357194781303406 55.0
0 -0.135312438011169 56.0
0 -0.00556379556655884 57.0
0 -0.0837847590446472 58.0
0 0.124440252780914 59.0
0 -0.110850632190704 60.0
0 0.177140235900879 61.0
0 0.49521416425705 62.0
265.08575898 249.722122192383 0.0
0 -0.278734683990479 1.0
0 0.132479786872864 2.0
0 -0.208131074905396 3.0
0 -0.198205709457397 4.0
0 0.252872824668884 5.0
0 0.250280857086182 6.0
0 0.144037365913391 7.0
0 0.00579947233200073 8.0
0 0.034977912902832 9.0
0 0.140912294387817 10.0
0 -0.317564368247986 11.0
0 0.00917589664459229 12.0
47.541737455 48.9051208496094 13.0
37.588862226 39.8627166748047 14.0
0 -0.0964864492416382 15.0
0 -0.129796087741852 16.0
26.832490607 32.5241546630859 17.0
0 0.0822843313217163 18.0
0 -0.0156934261322021 19.0
0 0.138381123542786 20.0
0 0.00526726245880127 21.0
0 0.117227375507355 22.0
36.183459938 34.7875518798828 23.0
0 0.0865540504455566 24.0
18.43676522 0.624096393585205 25.0
0 -0.0237876176834106 26.0
0 -0.125639855861664 27.0
0 0.833900809288025 28.0
0 -0.0810473561286926 29.0
0 0.0686085224151611 30.0
0 -0.228015184402466 31.0
0 -0.0190162658691406 32.0
0 -0.0860924124717712 33.0
0 -0.14295506477356 34.0
0 0.268194973468781 35.0
0 -0.19485741853714 36.0
0 0.102265000343323 37.0
0 0.145502328872681 38.0
0 -0.287372589111328 39.0
0 0.873071610927582 40.0
0 -0.0817493200302124 41.0
0 -0.0204291343688965 42.0
296.09064114 296.1201171875 43.0
0 0.0342612266540527 44.0
0 0.202466487884521 45.0
0 0.350306034088135 46.0
0 -0.0738999843597412 47.0
0 -0.0537277460098267 48.0
0 0.261727333068848 49.0
0.092621249111 0.0275352597236633 50.0
2.0179596507 1.88442146778107 51.0
0 -0.107426762580872 52.0
0 -0.121059238910675 53.0
0 -0.0698725581169128 54.0
0 -0.191868364810944 55.0
0 -0.0337778925895691 56.0
0 -0.0626645088195801 57.0
0 -0.10664027929306 58.0
0 -0.166819989681244 59.0
0 -0.153502106666565 60.0
0 0.110864698886871 61.0
0 0.0920462012290955 62.0
220.98176151 254.950210571289 0.0
0 0.589629888534546 1.0
0 -0.0658588409423828 2.0
0 0.141311287879944 3.0
0 -0.328164577484131 4.0
0 -0.171991765499115 5.0
0 0.0941436886787415 6.0
0 -0.210091829299927 7.0
0 0.202791333198547 8.0
0 0.0917106866836548 9.0
0 0.409781038761139 10.0
0 -0.142813146114349 11.0
0 0.05443274974823 12.0
60.664408657 59.7294960021973 13.0
35.626502929 34.5781059265137 14.0
0 0.00140154361724854 15.0
0 0.151372253894806 16.0
25.177379079 30.6898975372314 17.0
0 0.12158215045929 18.0
0 -0.0648069381713867 19.0
0 0.37053570151329 20.0
0 -0.108331918716431 21.0
0 0.925640404224396 22.0
42.389579004 40.0512161254883 23.0
0 -0.0617853999137878 24.0
0 0.182678580284119 25.0
0 -0.138440370559692 26.0
0 0.182027280330658 27.0
0 -0.116819381713867 28.0
0 -0.0910367965698242 29.0
0 -0.137703537940979 30.0
0 0.383551597595215 31.0
0 0.187336564064026 32.0
0 -0.0396407842636108 33.0
0 -0.163784027099609 34.0
0 -0.0544319152832031 35.0
0 0.481120407581329 36.0
0 0.0384520888328552 37.0
0 0.110889971256256 38.0
0 -0.200446009635925 39.0
0 0.208435833454132 40.0
0 -0.131519317626953 41.0
0 0.207368731498718 42.0
278.13717784 296.383361816406 43.0
0 0.211805820465088 44.0
0 0.102075397968292 45.0
0 0.232985317707062 46.0
0 0.185689747333527 47.0
0 0.350744634866714 48.0
0 0.381751716136932 49.0
0.077020632143 0.0226370692253113 50.0
1.6556282596 1.87839007377625 51.0
0 -0.170262038707733 52.0
0 -0.0581432580947876 53.0
0 0.265484690666199 54.0
0 -0.0165801048278809 55.0
0 -0.0852437019348145 56.0
0 0.109973132610321 57.0
0 -0.0283663868904114 58.0
0 0.494861334562302 59.0
0 0.0646917819976807 60.0
0 0.0513719320297241 61.0
0 0.116245925426483 62.0
262.1767559 251.908721923828 0.0
0 -0.293178915977478 1.0
0 -0.0379697680473328 2.0
0 0.110501646995544 3.0
0 0.114880323410034 4.0
0 0.133788049221039 5.0
0 0.252423763275146 6.0
0 0.201700329780579 7.0
0 -0.167629837989807 8.0
0 0.115543603897095 9.0
0 -0.0770845413208008 10.0
0 0.0887497067451477 11.0
0 -0.0160731673240662 12.0
66.614808386 66.0141983032227 13.0
37.375906697 37.8002815246582 14.0
0 0.0397216081619263 15.0
0 0.0740303993225098 16.0
38.921121013 38.8795394897461 17.0
0 0.224135994911194 18.0
0 -0.0644658207893372 19.0
0 0.0663502216339111 20.0
0 -0.0629048347473145 21.0
0 0.343566387891769 22.0
37.602040355 34.0566558837891 23.0
0 0.047760009765625 24.0
0 0.581363439559937 25.0
0 0.00495409965515137 26.0
0 -0.0339542031288147 27.0
0 0.48383355140686 28.0
0 0.243681669235229 29.0
0 -0.159506261348724 30.0
0 0.242853999137878 31.0
0 -0.00298774242401123 32.0
0 -0.0886821150779724 33.0
0 0.415579915046692 34.0
0 -0.102735936641693 35.0
0 0.257235765457153 36.0
0 0.100010097026825 37.0
0 -0.121070563793182 38.0
0 -0.267982482910156 39.0
0 0.784818112850189 40.0
0 -0.108745515346527 41.0
0 0.472025394439697 42.0
296.76839063 293.350738525391 43.0
0 0.154958367347717 44.0
0 0.315975725650787 45.0
0 0.310524761676788 46.0
0 0.198300719261169 47.0
0 0.433562129735947 48.0
0 0.452313601970673 49.0
0.075718564897 -0.0058022141456604 50.0
2.4441437361 1.9331419467926 51.0
0 -0.139957904815674 52.0
0 0.101013839244843 53.0
0 0.249059200286865 54.0
0 0.0603210926055908 55.0
0 0.0252132415771484 56.0
0 0.414701521396637 57.0
0 0.248253762722015 58.0
0 0.203151524066925 59.0
0 -0.0645382404327393 60.0
0 -0.0799131393432617 61.0
0 -0.15998375415802 62.0
253.67249239 255.56037902832 0.0
0 -0.239634037017822 1.0
0 -0.00186032056808472 2.0
0 -0.0131158232688904 3.0
0 -0.191901087760925 4.0
0 0.163555264472961 5.0
0 0.182823598384857 6.0
0 0.192088901996613 7.0
0 0.151621460914612 8.0
0 0.0625450015068054 9.0
0 0.325873523950577 10.0
0 0.392979621887207 11.0
0 -0.144027829170227 12.0
54.805968869 56.1743278503418 13.0
33.350198973 34.8357925415039 14.0
0 -0.199798464775085 15.0
0 0.0562037229537964 16.0
39.058830177 34.4841461181641 17.0
0 0.151700735092163 18.0
0 -0.0735924243927002 19.0
0 0.145463883876801 20.0
0 0.111485600471497 21.0
0 0.15476655960083 22.0
29.392543309 29.5499286651611 23.0
0 0.12546306848526 24.0
0 1.13794732093811 25.0
0 0.114943981170654 26.0
0 -0.194876551628113 27.0
0 0.348997503519058 28.0
0 0.0761018991470337 29.0
0 0.0293236374855042 30.0
0 0.13021445274353 31.0
0 0.234034240245819 32.0
0 0.273308217525482 33.0
0 0.112736225128174 34.0
0 -0.042805552482605 35.0
0 0.200493454933167 36.0
0 -0.0381311774253845 37.0
0 0.161464810371399 38.0
0 -0.237456798553467 39.0
0 0.294733822345734 40.0
0 -0.294603586196899 41.0
0 0.161390900611877 42.0
292.36241154 296.433959960938 43.0
0 0.106010437011719 44.0
0 0.0247740149497986 45.0
0 0.0897596478462219 46.0
0 -0.0472095608711243 47.0
0 -0.764325380325317 48.0
0 0.604651093482971 49.0
0.098122924048 -0.0627974271774292 50.0
1.5516535049 1.9051239490509 51.0
0 0.126414656639099 52.0
0 -0.235765099525452 53.0
0 -0.0235946774482727 54.0
0 -0.00194424390792847 55.0
0 -0.171412944793701 56.0
0 -0.0712313055992126 57.0
0 0.345645099878311 58.0
0 0.203449547290802 59.0
0 -0.183318853378296 60.0
0 0.0714544057846069 61.0
0 0.28680408000946 62.0
};
\addlegendentry{$R^2$=0.996}
\end{axis}

\end{tikzpicture}
}}
        \subfloat [Actual vs predicted edge flows.] {\label{fig:ye_col_base_f}\resizebox{\figurewidth}{\figureheight}{% This file was created with tikzplotlib v0.10.1.
\begin{tikzpicture}

\definecolor{darkgray176}{RGB}{176,176,176}
\definecolor{lightgray204}{RGB}{204,204,204}

\begin{axis}[
colorbar,
colorbar style={ylabel={edge id}},
colormap={mymap}{[1pt]
 rgb(0pt)=(0.12156862745098,0.466666666666667,0.705882352941177);
  rgb(1pt)=(1,0.498039215686275,0.0549019607843137);
  rgb(2pt)=(0.172549019607843,0.627450980392157,0.172549019607843);
  rgb(3pt)=(0.83921568627451,0.152941176470588,0.156862745098039);
  rgb(4pt)=(0.580392156862745,0.403921568627451,0.741176470588235);
  rgb(5pt)=(0.549019607843137,0.337254901960784,0.294117647058824);
  rgb(6pt)=(0.890196078431372,0.466666666666667,0.76078431372549);
  rgb(7pt)=(0.498039215686275,0.498039215686275,0.498039215686275);
  rgb(8pt)=(0.737254901960784,0.741176470588235,0.133333333333333);
  rgb(9pt)=(0.0901960784313725,0.745098039215686,0.811764705882353)
},
legend cell align={left},
legend style={fill opacity=0.8, draw opacity=1, text opacity=1, draw=lightgray204},
point meta max=61,
point meta min=0,
tick align=outside,
tick pos=left,
title={},
x grid style={darkgray176},
xlabel={True},
xmajorgrids,
xmin=-121.044251301977, xmax=624.82380659057,
xtick={-100,0,100,200,300,400,500,600},
xticklabels={-100,0,100,200,300,400,500,$f_e$},
xtick style={color=black},
y grid style={darkgray176},
ylabel={Predicted},
ymajorgrids,
ymin=-130.255865478516, ymax=387.355902099609,
ytick={-100,0,100,200,300},
yticklabels={-100,0,100,200,$f_e$},
ytick style={color=black}
]
\addplot [
  colormap={mymap}{[1pt]
 rgb(0pt)=(0.12156862745098,0.466666666666667,0.705882352941177);
  rgb(1pt)=(1,0.498039215686275,0.0549019607843137);
  rgb(2pt)=(0.172549019607843,0.627450980392157,0.172549019607843);
  rgb(3pt)=(0.83921568627451,0.152941176470588,0.156862745098039);
  rgb(4pt)=(0.580392156862745,0.403921568627451,0.741176470588235);
  rgb(5pt)=(0.549019607843137,0.337254901960784,0.294117647058824);
  rgb(6pt)=(0.890196078431372,0.466666666666667,0.76078431372549);
  rgb(7pt)=(0.498039215686275,0.498039215686275,0.498039215686275);
  rgb(8pt)=(0.737254901960784,0.741176470588235,0.133333333333333);
  rgb(9pt)=(0.0901960784313725,0.745098039215686,0.811764705882353)
},
  only marks,
  scatter,
  scatter src=explicit
]
table [x=x, y=y, meta=colordata]{%
x  y  colordata
156.83234799 158.746398925781 0.0
156.83234799 157.190841674805 1.0
155.53998088 158.539306640625 2.0
150.75091116 148.806823730469 3.0
0 -0.550102531909943 4.0
0 2.02057003974915 5.0
-70.351804676 -65.2748794555664 6.0
80.750074137 74.3665542602539 7.0
46.531024486 50.0589752197266 8.0
34.219049641 25.5131397247314 9.0
87.446462721 86.4352798461914 10.0
74.719546131 75.1993789672852 11.0
82.177551722 78.8154754638672 12.0
82.177551722 78.9507675170898 13.0
82.177551722 80.4209976196289 14.0
0 -3.48472356796265 15.0
0 -0.909306943416595 16.0
0 1.2164523601532 17.0
60.892314064 57.8691749572754 18.0
-60.892314074 -58.298828125 19.0
-60.892314052 -59.2076606750488 20.0
-60.892314064 -60.955394744873 21.0
81.96180066 76.8051300048828 22.0
143.16750759 141.247894287109 23.0
106.719705 110.113632202148 24.0
21.656633986 22.4772338867188 25.0
225.12930826 223.70002746582 26.0
225.12930826 225.729095458984 27.0
27.378944891 24.777551651001 28.0
252.50825316 255.221878051758 29.0
252.50825316 258.220306396484 30.0
46.715220681 46.1986274719238 31.0
9.3479103297 9.122239112854 32.0
21.069486586 20.1278533935547 33.0
21.069486586 18.8489627838135 34.0
21.069486586 21.2967205047607 35.0
9.1751953504 7.52423572540283 36.0
7.9225374016 6.81124401092529 37.0
-2.3021364033e-10 0.423498809337616 38.0
-2.3021413624e-10 -0.554073631763458 39.0
-11.468377867 -14.9679346084595 40.0
-24.199109633 -25.4148139953613 41.0
4.8824960524 4.27468967437744 42.0
-31.656519129 -33.375072479248 43.0
93.253690146 91.2723999023438 44.0
93.253690159 95.2701263427734 45.0
6.8365455229 7.46168041229248 46.0
2.5355427436 2.64920401573181 47.0
156.832348 163.685516357422 48.0
150.75091117 148.024139404297 49.0
0 -2.12741827964783 50.0
80.750074147 77.5399322509766 51.0
87.44646273 89.2044067382812 52.0
82.177551732 81.0141830444336 53.0
82.177551732 79.2189483642578 54.0
82.177551732 80.3916015625 55.0
0 2.03908276557922 56.0
81.96180067 80.5405349731445 57.0
225.12930827 220.066268920898 58.0
252.50825317 251.363693237305 59.0
21.069486596 26.268684387207 60.0
0 1.73217487335205 61.0
172.188551194307 165.528060913086 0.0
172.188551194307 166.413177490234 1.0
170.754562744856 158.973693847656 2.0
165.96549285 163.719543457031 3.0
1.5423518217e-09 5.11746263504028 4.0
1.54235182e-09 2.60798501968384 5.0
-53.6734697514663 -61.2328987121582 6.0
62.7142349561321 81.9588470458984 7.0
57.9545874841321 59.7164421081543 8.0
4.75964747213196 41.2237548828125 9.0
91.6979590331321 88.8060455322266 10.0
79.5961755931321 74.4312057495117 11.0
85.4224415601321 75.3353271484375 12.0
85.4224415601321 77.1122131347656 13.0
85.4224415601321 80.3796539306641 14.0
1.9262465079e-08 -1.37970280647278 15.0
1.9262465081e-08 -0.501269042491913 16.0
8.67058163000001e-08 -0.483811914920807 17.0
60.4769839701015 57.3510131835938 18.0
-60.4769839361981 -60.5245399475098 19.0
-60.4769839341981 -59.1617965698242 20.0
-60.4769837781981 -56.8208160400391 21.0
81.9238693380898 78.6046981811523 22.0
137.424649442254 131.302001953125 23.0
104.799675327817 118.212692260742 24.0
21.2731341928173 20.6229267120361 25.0
219.348518778614 230.859771728516 26.0
219.348518778614 216.776916503906 27.0
30.054314069614 30.6156597137451 28.0
249.402832848614 248.972351074219 29.0
249.402832848614 253.497406005859 30.0
46.6715722957763 44.5382652282715 31.0
9.23058475845532 9.22140407562256 32.0
21.4468855603989 20.1591892242432 33.0
21.4468855603989 15.893102645874 34.0
21.4468857571835 22.734411239624 35.0
8.57446183504062 7.82262134552002 36.0
7.36494536394061 7.55616569519043 37.0
8.90893710000002e-09 -7.00286340713501 38.0
8.90893710000002e-09 -4.3419885635376 39.0
-12.3819438645991 -13.9982595443726 40.0
-24.4265337158468 -27.1775741577148 41.0
4.97758282791 4.04443979263306 42.0
-31.7801274441101 -30.4395751953125 43.0
96.7576486161198 82.5388870239258 44.0
96.7576490301101 96.7024993896484 45.0
7.12333720750651 3.37304949760437 46.0
2.46072288042122 1.44243597984314 47.0
172.18855119 166.6962890625 48.0
165.96549285 166.765533447266 49.0
1.5423518218e-09 -5.80873584747314 50.0
62.714234956 73.1149749755859 51.0
91.697959033 91.7118301391602 52.0
85.42244156 79.3073348999023 53.0
85.42244156 75.8380584716797 54.0
85.42244156 80.2763061523438 55.0
8.6705816308e-08 1.91650915145874 56.0
81.923869338 78.757194519043 57.0
219.34851878 217.007873535156 58.0
249.40283285 248.407577514648 59.0
21.446885757 27.666820526123 60.0
8.9089370924e-09 2.06781053543091 61.0
158.78492799 158.511871337891 0.0
158.78492799 160.345504760742 1.0
157.55845776 161.634689331055 2.0
152.76938803 157.02604675293 3.0
0 2.5329372882843 4.0
-2.544593235e-09 0.373229801654816 5.0
-58.2326681 -59.9811553955078 6.0
69.0042065680114 79.5702362060547 7.0
45.3968947500114 51.3475608825684 8.0
23.6073118180114 23.5053615570068 9.0
95.0640429550114 87.0657730102539 10.0
82.0546797530114 74.3658676147461 11.0
85.3699658280114 81.8947525024414 12.0
85.3699658280114 78.724479675293 13.0
85.3699658280114 78.1794052124023 14.0
0 -3.1383318901062 15.0
0 -0.0864683389663696 16.0
-8.1062745031e-09 0.911620855331421 17.0
55.25467914 57.9483222961426 18.0
-55.2546791610172 -58.7306785583496 19.0
-55.2546791390172 -60.5880126953125 20.0
-55.2546791500172 -56.6350975036621 21.0
77.304837526 77.6958618164062 22.0
141.66523917 140.632919311523 23.0
108.75521162 110.377182006836 24.0
23.392796791 20.246021270752 25.0
218.970076706167 224.754730224609 26.0
218.970076706167 223.898422241211 27.0
25.798573808167 25.7207260131836 28.0
244.768650516167 255.512435913086 29.0
244.768650516167 256.97998046875 30.0
46.784455177 45.4250679016113 31.0
9.3568982446 9.14182376861572 32.0
22.0501583827241 19.82204246521 33.0
22.0501583827241 19.2031345367432 34.0
22.050158379 20.3097171783447 35.0
9.7730771557 7.7411003112793 36.0
8.5197689385 7.00214004516602 37.0
-9.7930293767e-09 -0.729664027690887 38.0
-9.793029377e-09 -2.27411699295044 39.0
-11.879575757 -14.2215900421143 40.0
-24.686485019 -25.7833938598633 41.0
5.2909415675 3.25091695785522 42.0
-32.416208745 -32.5273628234863 43.0
99.885911084 93.4572219848633 44.0
99.885911089 94.8323364257812 45.0
7.0851753283 4.62867450714111 46.0
2.7442576215 2.01078462600708 47.0
158.78492799 161.392395019531 48.0
152.76938804 148.182556152344 49.0
0 -2.55120730400085 50.0
69.004206568 75.5496444702148 51.0
95.064042955 88.6372833251953 52.0
85.369965828 81.0701217651367 53.0
85.369965828 79.9349746704102 54.0
85.369965828 80.065803527832 55.0
0 1.26070809364319 56.0
77.304837532 80.0971832275391 57.0
218.97007671 221.678405761719 58.0
244.76865052 251.536270141602 59.0
22.050158385 24.7950115203857 60.0
0 0.563408076763153 61.0
151.64624176 158.84765625 0.0
151.64624176 165.787063598633 1.0
150.38828208 164.369125366211 2.0
145.59920524 153.706726074219 3.0
-5.3252550901e-11 6.02484130859375 4.0
-5.3299257115e-11 1.3813304901123 5.0
-61.604778542 -65.460578918457 6.0
71.939024658 74.6935043334961 7.0
58.719205568 53.016902923584 8.0
13.219819079 22.5619659423828 9.0
84.159605919 86.0759582519531 10.0
71.55339741 76.109977722168 11.0
77.786502181 82.6060104370117 12.0
77.786502181 79.1815719604492 13.0
77.786502181 81.5312271118164 14.0
0 -1.94463753700256 15.0
0 -0.773880422115326 16.0
0 -1.47571706771851 17.0
60.565003059 59.0563087463379 18.0
-60.565003069 -59.5979499816895 19.0
-60.565003047 -59.8511238098145 20.0
-60.56500306 -58.535961151123 21.0
82.439979301 76.3897552490234 22.0
139.92442618 141.520965576172 23.0
106.01089807 109.384071350098 24.0
21.723927641 18.8281269073486 25.0
222.36440549 230.435638427734 26.0
222.36440549 224.566131591797 27.0
26.102021825 26.8473663330078 28.0
248.46642732 255.192993164062 29.0
248.46642732 253.590515136719 30.0
46.595926078 44.9616622924805 31.0
9.3411053986 8.85549545288086 32.0
21.874976231 19.8073902130127 33.0
21.874976231 19.0252304077148 34.0
21.874976231 21.2135524749756 35.0
9.4067347628 6.26200389862061 36.0
8.1136646441 7.41254377365112 37.0
-1.119689488e-13 -1.96453285217285 38.0
-1.1907128515e-13 -0.97162252664566 39.0
-12.131503059 -12.2143669128418 40.0
-24.869162558 -29.3370761871338 41.0
5.2760376598 3.84713768959045 42.0
-32.732582688 -33.1023139953613 43.0
93.428576185 89.0642166137695 44.0
93.428576198 95.1086196899414 45.0
6.8515603131 7.30044269561768 46.0
2.6779769278 2.33660936355591 47.0
151.64624177 159.097808837891 48.0
145.59920525 153.741394042969 49.0
0 -1.63738894462585 50.0
71.939024668 75.2205352783203 51.0
84.159605929 90.2451095581055 52.0
77.786502191 81.4206924438477 53.0
77.786502191 81.41650390625 54.0
77.786502191 80.282112121582 55.0
0 3.75088381767273 56.0
82.439979311 79.6773223876953 57.0
222.3644055 222.277633666992 58.0
248.46642733 249.351791381836 59.0
21.874976241 22.9760704040527 60.0
0 1.41318845748901 61.0
170.60319807 158.037536621094 0.0
170.60319807 159.982330322266 1.0
169.43562546 166.981063842773 2.0
164.64655573 163.061340332031 3.0
0 2.16191697120667 4.0
0 2.88785815238953 5.0
-60.14313647 -65.0493392944336 6.0
70.08138323 75.3288192749023 7.0
54.953517976 51.5064163208008 8.0
15.127865244 24.7839546203613 9.0
87.861713091 88.6258239746094 10.0
75.085335522 76.7249755859375 11.0
85.117897201 81.2533569335938 12.0
85.117897201 77.8865737915039 13.0
85.117897201 77.4321975708008 14.0
0 -3.68081378936768 15.0
0 -0.843175709247589 16.0
0 2.15332818031311 17.0
58.900063342 58.1665496826172 18.0
-58.90006336 -59.8762855529785 19.0
-58.900063338 -59.0597953796387 20.0
-58.90006335 -59.898063659668 21.0
80.909169582 78.9380569458008 22.0
147.84205711 143.088088989258 23.0
112.54285648 108.837051391602 24.0
22.127104375 19.0511856079102 25.0
228.75122671 228.351028442383 26.0
228.75122671 225.462173461914 27.0
26.689506413 25.6229476928711 28.0
255.44073313 257.707366943359 29.0
255.44073313 257.128509521484 30.0
46.777400949 46.2585906982422 31.0
9.3834268056 9.01065540313721 32.0
22.009106221 20.5918216705322 33.0
22.009106221 17.7936878204346 34.0
22.009106219 20.2432861328125 35.0
9.4215328669 8.70091915130615 36.0
8.1627662754 7.89605045318604 37.0
0 -1.60021018981934 38.0
0 -3.52996921539307 39.0
-11.463464557 -14.236798286438 40.0
-24.24925898 -26.9932613372803 41.0
5.235497529 2.70149898529053 42.0
-31.933803643 -32.8002281188965 43.0
99.652103113 95.9946823120117 44.0
99.652103126 95.9328765869141 45.0
7.0996218909 4.61811017990112 46.0
2.63877172 2.03792834281921 47.0
170.60319808 163.235748291016 48.0
164.64655574 159.644485473633 49.0
0 -4.34657192230225 50.0
70.08138324 75.7954940795898 51.0
87.861713101 89.797966003418 52.0
85.117897211 80.6523132324219 53.0
85.117897211 81.5453262329102 54.0
85.117897211 79.8261642456055 55.0
0 3.37748336791992 56.0
80.909169592 79.4521408081055 57.0
228.75122672 223.439102172852 58.0
255.44073314 250.315505981445 59.0
22.009106229 23.0748558044434 60.0
0 1.17029571533203 61.0
167.851578580291 157.624694824219 0.0
167.851578580291 157.893371582031 1.0
166.57735499 164.799667358398 2.0
161.78828525 167.081008911133 3.0
0 1.32436752319336 4.0
-6.64285253e-09 0.0777676701545715 5.0
-61.3956489183344 -65.1298904418945 6.0
71.8097325321702 76.6603546142578 7.0
48.9969343011702 53.6090507507324 8.0
22.8127982311702 23.6905727386475 9.0
89.7781064941702 89.7309112548828 10.0
76.3594286621702 77.3215179443359 11.0
84.7871816201702 85.5937728881836 12.0
84.7871816201702 81.909309387207 13.0
84.7871816201702 81.6051864624023 14.0
0 -1.08381700515747 15.0
0 -1.30826950073242 16.0
-4.41900778500001e-09 -1.9726459980011 17.0
57.8835417824982 58.042781829834 18.0
-57.8835418297554 -59.7814826965332 19.0
-57.8835418087553 -59.0279655456543 20.0
-57.8835418117554 -60.7680854797363 21.0
79.8701809153974 78.1941757202148 22.0
146.451452522456 140.681259155273 23.0
111.648017253071 107.876846313477 24.0
21.3297494220706 21.9726047515869 25.0
226.321633440581 227.088104248047 26.0
226.321633440581 225.171005249023 27.0
25.914641882581 27.543342590332 28.0
252.236275320581 255.906646728516 29.0
252.236275320581 253.653778076172 30.0
46.4206329649525 45.5576324462891 31.0
9.28241392350397 8.83930969238281 32.0
21.9866391040031 20.2806625366211 33.0
21.9866391040031 23.0100383758545 34.0
21.9866391152656 20.2263431549072 35.0
9.8363200119817 6.7127046585083 36.0
8.5244416084817 7.61852979660034 37.0
-8.913903959e-09 1.64067280292511 38.0
-8.913903959e-09 -2.53642416000366 39.0
-11.7185840903777 -13.2036571502686 40.0
-24.5302552240018 -27.2293472290039 41.0
5.05447173854186 4.57785081863403 42.0
-32.2067254496419 -32.3760643005371 43.0
97.5205461086419 91.2004699707031 44.0
97.5205460082612 94.4583282470703 45.0
6.9910232021 5.53251647949219 46.0
2.7931508456 1.15151596069336 47.0
167.85157858 159.698196411133 48.0
161.78828526 158.298217773438 49.0
0 -1.23219132423401 50.0
71.809732532 76.7780075073242 51.0
89.778106494 90.8864059448242 52.0
84.78718162 81.4463577270508 53.0
84.78718162 81.3097152709961 54.0
84.78718162 80.0934753417969 55.0
0 1.49703764915466 56.0
79.870180915 80.2734832763672 57.0
226.32163344 223.388259887695 58.0
252.23627532 255.827041625977 59.0
21.986639115 17.8986434936523 60.0
0 1.27972865104675 61.0
155.72149572 166.822128295898 0.0
155.72149572 167.83854675293 1.0
154.36209964 159.873916625977 2.0
149.57302992 154.213043212891 3.0
0 1.00091886520386 4.0
-2.1329449713e-14 4.05695390701294 5.0
-59.029042726 -64.6119079589844 6.0
69.637068519 74.0829086303711 7.0
46.760599862 46.6488189697266 8.0
22.876468647 23.217248916626 9.0
92.327884871 84.3157196044922 10.0
79.671073078 74.6905899047852 11.0
83.891089171 81.943489074707 12.0
83.891089171 79.1677627563477 13.0
83.891089171 79.104850769043 14.0
0 -3.7549262046814 15.0
0 -0.817016661167145 16.0
-4.9703502458e-14 3.94266295433044 17.0
58.600096264 57.2898139953613 18.0
-58.600096284 -57.911979675293 19.0
-58.600096262 -58.8154487609863 20.0
-58.600096275 -59.1742210388184 21.0
80.404302592 76.5683670043945 22.0
151.38892495 141.837020874023 23.0
115.86888952 105.457878112793 24.0
24.055438403 21.5884704589844 25.0
231.79322755 226.150680541992 26.0
231.79322755 226.115051269531 27.0
27.193361005 27.110803604126 28.0
258.98658857 252.017517089844 29.0
258.98658857 252.512680053711 30.0
46.823890995 45.8057518005371 31.0
9.3792601913 8.49391841888428 32.0
21.804206307 20.2965641021729 33.0
21.804206307 21.852710723877 34.0
21.804206307 20.5895042419434 35.0
9.6155620046 8.04371452331543 36.0
8.358955766 7.01451253890991 37.0
-3.1662993643e-14 -0.733949482440948 38.0
-3.1662993637e-14 -3.28981256484985 39.0
-11.807313532 -15.0730028152466 40.0
-24.829814885 -26.2302989959717 41.0
5.3128321851 6.1045503616333 42.0
-32.663139354 -33.1430435180664 43.0
101.01406899 92.7848892211914 44.0
101.014069 94.5856018066406 45.0
6.8367133115 6.12080860137939 46.0
2.5495157899 2.27223634719849 47.0
155.72149573 164.35595703125 48.0
149.57302993 148.828948974609 49.0
0 -2.56835126876831 50.0
69.637068529 74.6783981323242 51.0
92.327884881 86.7743530273438 52.0
83.891089181 80.7299652099609 53.0
83.891089181 81.3804550170898 54.0
83.891089181 80.2576141357422 55.0
0 4.9667181968689 56.0
80.404302602 79.8279647827148 57.0
231.79322756 222.422393798828 58.0
258.98658858 249.107757568359 59.0
21.804206317 21.4210338592529 60.0
0 0.899785339832306 61.0
137.374995184379 167.431610107422 0.0
137.374995184379 168.116363525391 1.0
136.3102256 147.96061706543 2.0
131.52115587 138.277587890625 3.0
0 7.75590705871582 4.0
-6.333032104e-09 -2.41415929794312 5.0
-66.2396845214069 -65.5914840698242 6.0
75.2081867339241 71.0615539550781 7.0
53.059418127924 52.6302680969238 8.0
22.148768605924 5.08159017562866 9.0
94.161286678924 89.9552001953125 10.0
80.538240061924 87.063362121582 11.0
83.674359737924 84.0154876708984 12.0
83.674359737924 73.7836608886719 13.0
83.674359737924 80.2213134765625 14.0
0 -6.39899301528931 15.0
0 -0.322709619998932 16.0
-3.915399699e-09 -1.25803756713867 17.0
59.2715141168677 54.8433036804199 18.0
-59.2715141163861 -60.0240058898926 19.0
-59.2715140963861 -59.5690765380859 20.0
-59.2715140973861 -56.798038482666 21.0
82.1956198616281 79.4920120239258 22.0
146.666070116868 141.594573974609 23.0
111.374568646085 108.022399902344 24.0
20.2017082260846 22.2541351318359 25.0
228.861689978759 227.367248535156 26.0
228.861689978759 226.758560180664 27.0
25.6355753607585 31.0771675109863 28.0
254.497265338758 243.41471862793 29.0
254.497265338758 244.169891357422 30.0
46.5601388079682 45.5739479064941 31.0
9.47121734989491 8.14004611968994 32.0
22.9241057642206 16.1393127441406 33.0
22.9241057642206 22.220308303833 34.0
22.9241057767324 19.4652290344238 35.0
10.1220978289628 6.28947973251343 36.0
8.87400961456285 6.59697437286377 37.0
-8.76623282e-09 0.150171458721161 38.0
-8.76623281999999e-09 5.52019691467285 39.0
-12.695174089693 -15.9745111465454 40.0
-24.9924547757919 -19.9606285095215 41.0
4.83749505730338 6.81323194503784 42.0
-32.3056239071034 -35.373706817627 43.0
85.5363773001034 84.8748550415039 44.0
85.5363771695882 88.1779479980469 45.0
7.4262603606 5.58299827575684 46.0
2.6530002252 2.05302095413208 47.0
137.37499518 171.981369018555 48.0
131.52115588 134.866836547852 49.0
0 3.11496019363403 50.0
75.208186734 74.8920974731445 51.0
94.161286679 83.7756881713867 52.0
83.674359738 81.9499740600586 53.0
83.674359738 78.828010559082 54.0
83.674359738 79.2952728271484 55.0
0 4.30617475509644 56.0
82.195619862 79.753288269043 57.0
228.86168998 225.364013671875 58.0
254.49726534 242.576126098633 59.0
22.924105777 17.9782180786133 60.0
0 1.27670609951019 61.0
160.01582397 160.546875 0.0
160.01582397 164.467071533203 1.0
158.85952038 160.635971069336 2.0
154.07045066 155.67985534668 3.0
0 0.333596169948578 4.0
0 4.47875308990479 5.0
-67.574651288 -59.5486755371094 6.0
78.492135886 75.3117294311523 7.0
56.626794927 51.3640251159668 8.0
21.865340949 21.9085216522217 9.0
93.017289512 92.3890075683594 10.0
79.796423925 77.6202163696289 11.0
85.771420467 87.4445419311523 12.0
85.771420467 81.7160034179688 13.0
85.771420467 82.3588027954102 14.0
0 -0.0411826372146606 15.0
0 -0.0819154977798462 16.0
0 -0.175158679485321 17.0
56.642530877 59.1359558105469 18.0
-56.642530887 -59.7608108520508 19.0
-56.642530865 -59.2791862487793 20.0
-56.642530877 -61.4495162963867 21.0
78.740122294 78.3712615966797 22.0
146.96045729 143.845291137695 23.0
111.70155071 112.202178955078 24.0
23.554718076 20.9785194396973 25.0
225.70057959 225.795425415039 26.0
225.70057959 224.478088378906 27.0
26.294236319 27.8840370178223 28.0
251.99481592 256.230682373047 29.0
251.99481592 256.483734130859 30.0
46.900787267 46.1537971496582 31.0
9.3641391652 8.89545059204102 32.0
22.097591407 22.0585689544678 33.0
22.097591407 25.7006778717041 34.0
22.097591409 20.5135154724121 35.0
9.4995605065 7.62343311309814 36.0
8.296552137 7.67874431610107 37.0
-2.0636932738e-09 1.98853230476379 38.0
-2.0636933037e-09 -1.746098279953 39.0
-12.030853227 -15.0867700576782 40.0
-24.945261012 -26.4139595031738 41.0
5.0097297805 5.06824445724487 42.0
-32.621496875 -33.2409973144531 43.0
95.892521389 94.883674621582 44.0
95.892521402 94.8365707397461 45.0
7.2248013033 6.44018507003784 46.0
2.6932152991 2.58672904968262 47.0
160.01582398 165.635559082031 48.0
154.07045067 155.25651550293 49.0
0 0.719836711883545 50.0
78.492135896 76.7763595581055 51.0
93.017289522 89.0708160400391 52.0
85.771420477 81.7571563720703 53.0
85.771420477 81.6886901855469 54.0
85.771420477 80.5338363647461 55.0
0 2.54747176170349 56.0
78.740122304 78.7477798461914 57.0
225.7005796 219.340270996094 58.0
251.99481593 252.599456787109 59.0
22.097591419 18.6205577850342 60.0
0 1.74484467506409 61.0
165.351029781406 157.878570556641 0.0
165.351029781406 162.381591796875 1.0
164.08770701 165.636184692383 2.0
158.62307174 155.311874389648 3.0
1.0832880939 2.47760844230652 4.0
1.08328810226321 8.18432426452637 5.0
-67.7555482903632 -61.1807518005371 6.0
78.5853610227712 75.5879669189453 7.0
49.4458210727712 51.3133583068848 8.0
29.1395399497712 23.7947292327881 9.0
89.1073980397712 83.4504318237305 10.0
76.7986695777712 74.4040222167969 11.0
81.8736442707712 80.9371109008789 12.0
81.8736442707712 78.3630065917969 13.0
81.8736442707712 80.6130523681641 14.0
3.7602291097e-09 -0.0713506937026978 15.0
1.6272515452e-09 -0.0682880282402039 16.0
9.55669260000003e-09 1.25840258598328 17.0
57.8610660729661 59.728572845459 18.0
-57.8610660316569 -60.1925582885742 19.0
-57.8610660136569 -57.8545188903809 20.0
-57.8610659926569 -58.7698059082031 21.0
80.0843260790733 77.0896530151367 22.0
147.797166008966 141.997604370117 23.0
112.755926366208 108.210830688477 24.0
23.3683622332077 21.67160987854 25.0
227.881492092813 225.019256591797 26.0
227.881492092813 226.250411987305 27.0
26.901197311813 24.5954608917236 28.0
254.782689402813 256.296020507812 29.0
254.782689402813 257.09765625 30.0
46.4226135160305 46.9759521484375 31.0
9.28952276629632 8.97862720489502 32.0
22.2232600870896 21.1368560791016 33.0
22.2232600870896 25.127046585083 34.0
22.2232601261625 21.121654510498 35.0
9.86013207565584 8.47025203704834 36.0
8.62443095355584 7.59949207305908 37.0
-6.21446807000003e-09 -2.18093037605286 38.0
-6.21446807000003e-09 -2.29818391799927 39.0
-11.6745441648284 -14.040864944458 40.0
-24.6294752288498 -25.0429153442383 41.0
5.21946170480936 6.99011611938477 42.0
-32.3541859148094 -34.2182998657227 43.0
96.8442895768094 95.5808563232422 44.0
96.8442894218094 92.9817581176758 45.0
7.12667677344732 6.46695756912231 46.0
2.47454266054732 2.56767535209656 47.0
165.35102978 162.666473388672 48.0
158.62307175 153.156524658203 49.0
1.0832881023 1.32762098312378 50.0
78.585361023 77.681999206543 51.0
89.10739804 88.1610565185547 52.0
81.873644271 81.6924667358398 53.0
81.873644271 81.0575866699219 54.0
81.873644271 80.1906585693359 55.0
9.556692601e-09 1.85244917869568 56.0
80.084326079 79.4581756591797 57.0
227.88149209 222.508331298828 58.0
254.7826894 251.537719726562 59.0
22.223260126 18.91335105896 60.0
0 1.36672592163086 61.0
167.553581723244 161.729125976562 0.0
167.553581723244 166.587524414062 1.0
166.34051561 152.938430786133 2.0
160.87588041 148.695068359375 3.0
0.07443217065 -1.69536685943604 4.0
0.0744321803303848 9.64665508270264 5.0
-71.9832955547494 -58.3377151489258 6.0
83.6356714486229 77.2297058105469 7.0
46.2686053676229 55.8860054016113 8.0
37.3670660806229 26.8739986419678 9.0
80.5862279426229 88.8506317138672 10.0
67.1094132356229 78.7582244873047 11.0
76.9737490936229 83.3227310180664 12.0
76.9737490936229 75.3675537109375 13.0
76.9737490936229 76.1182327270508 14.0
0 -5.86562061309814 15.0
0 -1.0022246837616 16.0
-5.408911194e-09 -2.96489357948303 17.0
52.4755535595627 55.6439476013184 18.0
-52.4755535914343 -59.2691841125488 19.0
-52.4755535704343 -60.353946685791 20.0
-52.4755535764343 -58.8188629150391 21.0
73.3996972724656 73.2803039550781 22.0
143.041160184703 149.606903076172 23.0
110.856823048379 109.340538024902 24.0
22.7097946413792 21.3275775909424 25.0
216.440857456488 223.272705078125 26.0
216.440857456488 220.47053527832 27.0
29.4519952804879 27.7737464904785 28.0
245.892852736488 248.232498168945 29.0
245.892852736488 251.146209716797 30.0
46.6444792144921 42.8906707763672 31.0
9.4470507578 7.6134295463562 32.0
20.9241436962482 16.4485282897949 33.0
20.9241436962482 15.3453779220581 34.0
20.9241437030079 26.5982818603516 35.0
9.4172577167 12.2832746505737 36.0
8.2650889788 6.61757755279541 37.0
-9.25244505399999e-09 -6.74445009231567 38.0
-9.252445055e-09 -1.6116828918457 39.0
-11.9780049362171 -18.2689743041992 40.0
-25.9524022634184 -23.8079967498779 41.0
4.92695565085238 6.14523983001709 42.0
-33.3264788503524 -34.1615676879883 43.0
88.2464570983524 86.0989761352539 44.0
88.2464570415785 89.2647171020508 45.0
7.2137318066 9.87790203094482 46.0
2.8224558435 2.58502411842346 47.0
167.55358172 158.868850708008 48.0
160.87588042 148.166061401367 49.0
0.07443218033 3.71821880340576 50.0
83.635671449 70.618278503418 51.0
80.586227943 86.6785125732422 52.0
76.973749094 80.6233673095703 53.0
76.973749094 77.1133193969727 54.0
76.973749094 78.8096694946289 55.0
0 0.0758295655250549 56.0
73.399697272 79.2366409301758 57.0
216.44085746 218.775924682617 58.0
245.89285274 249.626708984375 59.0
20.924143703 20.0493106842041 60.0
0 3.16526579856873 61.0
133.96447814 156.484268188477 0.0
133.96447814 158.69873046875 1.0
132.82417083 158.257568359375 2.0
127.35953564 158.636291503906 3.0
6.9324334406 -3.07842350006104 4.0
6.9324334407 11.8928127288818 5.0
-75.396993483 -31.5629863739014 6.0
82.579096537 49.2709197998047 7.0
56.966683588 46.1846618652344 8.0
25.612412939 13.753514289856 9.0
107.99938727 101.844917297363 10.0
95.327787904 83.2505493164062 11.0
95.787695119 86.7883453369141 12.0
95.787695119 93.5574493408203 13.0
95.787695119 49.8998031616211 14.0
0 8.48309516906738 15.0
0 3.82146835327148 16.0
0 1.09526300430298 17.0
60.825454134 57.9217948913574 18.0
-60.825454144 -61.5902099609375 19.0
-60.825454122 -35.0293159484863 20.0
-60.825454134 -61.0265808105469 21.0
82.881445175 78.1933822631836 22.0
132.51398669 123.663558959961 23.0
96.547829206 108.719528198242 24.0
19.226695444 26.3640766143799 25.0
215.39543187 231.842742919922 26.0
215.39543187 215.280090332031 27.0
24.894436475 26.0250625610352 28.0
240.28986836 219.113784790039 29.0
240.28986836 187.782821655273 30.0
46.308874158 43.5092620849609 31.0
9.1907036593 7.53676605224609 32.0
22.055991031 27.6432456970215 33.0
22.055991031 23.8381061553955 34.0
22.055991031 22.5114707946777 35.0
9.3675431257 0.207445204257965 36.0
8.1441339579 0.867900669574738 37.0
-3.3839689981e-11 -7.23617076873779 38.0
-5.0898929458e-11 5.38095569610596 39.0
-14.078492871 -22.0102939605713 40.0
-26.851151096 -18.1554660797119 41.0
4.343056661 15.1537284851074 42.0
-34.269056436 -38.4561042785645 43.0
90.280993891 71.0094757080078 44.0
90.280993904 113.503524780273 45.0
6.2510272122 5.36233043670654 46.0
2.7194826251 -1.93530201911926 47.0
133.96447815 157.435180664062 48.0
127.35953565 141.216506958008 49.0
6.9324334506 2.55555558204651 50.0
82.579096547 17.9202690124512 51.0
107.99938728 80.0791091918945 52.0
95.787695128 74.9440383911133 53.0
95.787695128 78.4908676147461 54.0
95.787695128 80.2130737304688 55.0
0 7.01712417602539 56.0
82.881445185 77.2958831787109 57.0
215.39543188 216.850936889648 58.0
240.28986837 238.672897338867 59.0
22.055991041 23.0276584625244 60.0
0 14.5080785751343 61.0
175.706422112054 159.601943969727 0.0
175.706422112054 157.137359619141 1.0
174.33733556 161.827819824219 2.0
169.54826579 153.448379516602 3.0
0 -1.86016917228699 4.0
-5.55836977e-09 9.21675491333008 5.0
-63.7346356289219 -56.0812187194824 6.0
74.7073562792267 71.0563125610352 7.0
45.9912394332267 51.9045448303223 8.0
28.7161168452268 21.938570022583 9.0
89.6519887592268 87.940803527832 10.0
77.0901447872267 77.6189880371094 11.0
85.0701401072267 87.7401580810547 12.0
85.0701401072267 80.4932403564453 13.0
85.0701401072267 79.3829727172852 14.0
1.818650972e-09 -1.04033899307251 15.0
1.8186509716e-09 -0.732830822467804 16.0
1.544446326e-08 -1.37163949012756 17.0
56.1804717526764 57.3487739562988 18.0
-56.1804718873401 -59.0750350952148 19.0
-56.1804718703401 -58.1939544677734 20.0
-56.1804718373401 -55.9028167724609 21.0
78.8305113919532 77.6848602294922 22.0
144.020082732305 145.140319824219 23.0
111.750808827882 112.757568359375 24.0
21.5375415518819 19.4748802185059 25.0
222.850594124108 229.32600402832 26.0
222.850594124108 220.734298706055 27.0
24.4508406311078 26.052713394165 28.0
247.301434754108 260.256988525391 29.0
247.301434754108 257.740753173828 30.0
46.3661002741146 45.1001472473145 31.0
9.33517575652289 9.31139469146729 32.0
22.6500395547658 21.3931980133057 33.0
22.6500395547658 5.70928478240967 34.0
22.6500396076644 20.4942874908447 35.0
9.89761356117041 6.58216190338135 36.0
8.65092282737043 7.86821031570435 37.0
-4.98269157000003e-09 7.35635805130005 38.0
-4.98269156999997e-09 8.42078018188477 39.0
-11.5960186386701 -10.5329122543335 40.0
-25.3611226063526 -25.5153541564941 41.0
5.00706462432234 4.88421964645386 42.0
-33.0957925755223 -31.4123058319092 43.0
93.8241078555223 94.6019668579102 44.0
93.8241073253902 95.5746688842773 45.0
6.57560281524712 4.55637645721436 46.0
2.34044698064712 2.27259063720703 47.0
175.70642211 160.249374389648 48.0
169.5482658 160.735504150391 49.0
0 -4.06069850921631 50.0
74.707356279 75.0306091308594 51.0
89.651988759 90.6996231079102 52.0
85.070140108 80.5338439941406 53.0
85.070140108 80.6412048339844 54.0
85.070140108 80.2041168212891 55.0
1.5444463253e-08 0.321158587932587 56.0
78.830511392 78.9087066650391 57.0
222.85059412 217.436340332031 58.0
247.30143476 252.286392211914 59.0
22.650039608 21.7111225128174 60.0
0 6.01372909545898 61.0
181.1821658 157.900985717773 0.0
181.1821658 163.289321899414 1.0
180.21342245 177.867095947266 2.0
179.11688484 139.412002563477 3.0
-4.3847347347e-09 130.802612304688 4.0
-4.3847811105e-09 2.42138719558716 5.0
-55.801199946 -54.0600128173828 6.0
64.149270125 71.4222183227539 7.0
64.149270125 62.8218231201172 8.0
0 47.848804473877 9.0
83.889955065 109.681121826172 10.0
69.908477834 74.6972274780273 11.0
83.157495096 77.0538482666016 12.0
83.157495096 74.9897918701172 13.0
83.157495096 76.3853759765625 14.0
0 -5.85904169082642 15.0
0 -1.74360251426697 16.0
0 -0.437414824962616 17.0
51.232682739 58.9941101074219 18.0
-51.232682749 -57.7681579589844 19.0
-51.232682727 -58.8051910400391 20.0
-51.232682739 -55.3772392272949 21.0
73.741860452 78.412223815918 22.0
147.28957609 138.725082397461 23.0
107.85851804 105.323516845703 24.0
21.395254059 17.7509860992432 25.0
221.03143655 225.42578125 26.0
221.03143655 216.883941650391 27.0
26.506236652 29.6016464233398 28.0
247.53767321 264.289886474609 29.0
247.53767321 254.138336181641 30.0
46.314357556 46.1159934997559 31.0
9.2082868403 9.56864452362061 32.0
22.509177703 17.2197532653809 33.0
22.509177703 19.1549186706543 34.0
22.509177707 18.1371040344238 35.0
9.8730648562 9.88022613525391 36.0
8.5779212016 5.31985950469971 37.0
-4.3848622858e-09 -3.15506982803345 38.0
-4.3848622867e-09 -2.0541410446167 39.0
-10.157002873 -16.1597652435303 40.0
-22.382747678 -28.6011276245117 41.0
4.6217172636 2.81594061851501 42.0
-29.638636697 -28.1725120544434 43.0
101.08510702 72.10986328125 44.0
101.08510703 121.046783447266 45.0
7.820850159 8.48478507995605 46.0
2.7257580952 2.95245552062988 47.0
181.18216581 173.100173950195 48.0
179.11688485 169.613510131836 49.0
0 38.6413993835449 50.0
64.149270135 76.6653366088867 51.0
83.889955075 95.2996597290039 52.0
83.157495106 76.4390411376953 53.0
83.157495106 76.7992324829102 54.0
83.157495106 80.6597595214844 55.0
0 -0.948798000812531 56.0
73.741860462 79.1239624023438 57.0
221.03143656 221.055206298828 58.0
247.53767322 252.454467773438 59.0
22.509177717 28.0094604492188 60.0
0 0.754416167736053 61.0
165.37282172 164.771591186523 0.0
165.37282172 165.59538269043 1.0
164.08621151 163.676712036133 2.0
159.29714179 167.432281494141 3.0
-1.3888822338e-14 5.86176204681396 4.0
-4.7192277145e-14 8.77202796936035 5.0
-66.005992672 -61.3398475646973 6.0
75.961399788 76.3741836547852 7.0
45.913450672 51.3402900695801 8.0
30.047949106 25.0838871002197 9.0
93.031398984 88.3005294799805 10.0
80.004953217 76.4509124755859 11.0
84.258851389 80.7977828979492 12.0
84.258851389 81.2753372192383 13.0
84.258851389 80.1710052490234 14.0
0 -2.71896481513977 15.0
0 -1.34778261184692 16.0
0 -1.89199686050415 17.0
57.540175565 58.7489356994629 18.0
-57.540175575 -58.3593559265137 19.0
-57.540175554 -57.5392189025879 20.0
-57.540175566 -55.945198059082 21.0
79.7360664 75.8618316650391 22.0
145.25207475 141.160629272461 23.0
110.43041955 107.821685791016 24.0
23.347608587 18.2015056610107 25.0
224.98814116 227.524459838867 26.0
224.98814116 226.288146972656 27.0
25.828111373 25.6544818878174 28.0
250.81625254 254.621795654297 29.0
250.81625254 255.857360839844 30.0
46.663023555 46.9519081115723 31.0
9.2981412245 8.83577060699463 32.0
22.195890824 21.7209815979004 33.0
22.195890824 18.6042194366455 34.0
22.195890824 22.0220108032227 35.0
10.027075124 8.91591453552246 36.0
8.7661727812 8.14398765563965 37.0
-3.5740541168e-17 0.823405861854553 38.0
0 -1.72273635864258 39.0
-11.826009581 -16.6547870635986 40.0
-24.366354123 -23.3950366973877 41.0
5.2004399562 7.3001971244812 42.0
-32.158958363 -33.5394248962402 43.0
102.50665687 95.3285522460938 44.0
102.50665688 94.5260162353516 45.0
7.1120677723 7.85497713088989 46.0
2.5830831519 2.62177586555481 47.0
165.37282173 161.207443237305 48.0
159.2971418 158.772521972656 49.0
0 2.92333197593689 50.0
75.961399798 76.3285980224609 51.0
93.031398994 91.6590805053711 52.0
84.258851399 81.1545104980469 53.0
84.258851399 80.7502899169922 54.0
84.258851399 80.411750793457 55.0
0 0.316005885601044 56.0
79.73606641 80.2386474609375 57.0
224.98814117 218.659759521484 58.0
250.81625255 249.927993774414 59.0
22.195890834 22.4010620117188 60.0
0 2.1990225315094 61.0
164.58328001 162.778747558594 0.0
164.58328001 164.676956176758 1.0
163.48846591 167.106384277344 2.0
158.69939618 190.850860595703 3.0
0 5.54310894012451 4.0
-2.5694952765e-12 3.57540583610535 5.0
-66.030374652 -46.684009552002 6.0
75.748186354 86.2411193847656 7.0
53.928597355 26.7707633972168 8.0
21.819588989 22.2929382324219 9.0
85.346785249 92.963508605957 10.0
72.28886925 79.7825012207031 11.0
77.060322163 80.9975433349609 12.0
77.060322163 70.2804107666016 13.0
77.060322163 77.7778778076172 14.0
0 1.55915141105652 15.0
0 0.437254726886749 16.0
-6.4982172206e-13 2.54799199104309 17.0
58.685746701 57.2248001098633 18.0
-58.685746721 -55.7280426025391 19.0
-58.685746699 -58.165454864502 20.0
-58.685746711 -61.3686294555664 21.0
80.726705539 78.8855209350586 22.0
137.12085662 137.622833251953 23.0
105.4521144 111.922592163086 24.0
23.350383004 24.5733757019043 25.0
217.84756217 223.580169677734 26.0
217.84756217 217.089416503906 27.0
24.50774448 31.1229057312012 28.0
242.35530666 254.59001159668 29.0
242.35530666 249.690078735352 30.0
46.991592628 45.1640243530273 31.0
9.4380682961 10.0436143875122 32.0
22.040958818 24.6665992736816 33.0
22.040958818 15.8842258453369 34.0
22.040958818 20.4890289306641 35.0
9.0336922681 5.73582553863525 36.0
7.8456315029 8.33577060699463 37.0
-1.6126984745e-11 0.188456356525421 38.0
-1.6126997886e-11 0.0804041624069214 39.0
-11.962981483 -18.2951908111572 40.0
-24.218896123 -24.5339126586914 41.0
5.5105523825 2.20825862884521 42.0
-32.240533973 -31.249303817749 43.0
108.77585045 92.3369369506836 44.0
108.77585047 100.005844116211 45.0
7.0644886267 4.21744251251221 46.0
2.3791763176 2.18349099159241 47.0
164.58328002 166.90608215332 48.0
158.69939619 180.135452270508 49.0
0 -2.49729037284851 50.0
75.748186364 71.5849304199219 51.0
85.346785259 89.1382522583008 52.0
77.060322173 79.5747604370117 53.0
77.060322173 75.283935546875 54.0
77.060322173 78.9232788085938 55.0
0 7.73783016204834 56.0
80.726705549 80.5338516235352 57.0
217.84756218 209.096801757812 58.0
242.35530667 246.213226318359 59.0
22.040958828 21.9083290100098 60.0
0 3.01537418365479 61.0
156.56602634 166.996856689453 0.0
156.56602634 163.216842651367 1.0
155.52811462 170.640487670898 2.0
150.7390449 151.848907470703 3.0
0 4.94050264358521 4.0
-4.2660657474e-11 -1.41317081451416 5.0
-68.731927322 -66.0900115966797 6.0
77.866121724 80.3051452636719 7.0
54.790496082 46.7500190734863 8.0
23.075625632 32.5637359619141 9.0
98.648221469 87.4646377563477 10.0
84.841879345 76.28076171875 11.0
82.413322651 81.6635131835938 12.0
82.413322651 79.5570602416992 13.0
82.413322651 81.3304672241211 14.0
0 4.93020153045654 15.0
0 0.34750235080719 16.0
-8.1178606482e-11 4.74170827865601 17.0
56.504815131 60.2819137573242 18.0
-56.504815151 -51.4048843383789 19.0
-56.504815129 -60.8882675170898 20.0
-56.504815141 -58.2508201599121 21.0
77.90135043 75.4257125854492 22.0
153.77241756 140.865768432617 23.0
117.22080763 111.613304138184 24.0
24.48983216 19.244457244873 25.0
231.673768 227.543914794922 26.0
231.673768 220.664215087891 27.0
23.990083289 29.0249137878418 28.0
255.66385129 255.848693847656 29.0
255.66385129 248.753280639648 30.0
46.75308208 46.4244575500488 31.0
9.3064218121 7.82740592956543 32.0
21.396535279 19.7544956207275 33.0
21.396535279 13.4163761138916 34.0
21.396535279 20.6565799713135 35.0
8.9771209311 11.1744432449341 36.0
7.7013958298 5.92961311340332 37.0
-1.2841912078e-10 1.2199182510376 38.0
-1.2841912089e-10 -0.85528177022934 39.0
-12.303226357 -15.8131647109985 40.0
-26.19435029 -24.2702789306641 41.0
4.9156547157 11.4817390441895 42.0
-33.638238155 -36.9418678283691 43.0
87.857086331 89.1479721069336 44.0
87.857086344 89.78125 45.0
7.2063665523 5.07337284088135 46.0
2.7195235577 2.20142197608948 47.0
156.56602635 169.271057128906 48.0
150.73904491 152.083435058594 49.0
0 1.83639597892761 50.0
77.866121733 76.2368850708008 51.0
98.648221479 90.2155609130859 52.0
82.413322661 80.6285781860352 53.0
82.413322661 81.4595336914062 54.0
82.413322661 80.2137832641602 55.0
0 0.546156346797943 56.0
77.90135044 78.7353744506836 57.0
231.67376801 220.579742431641 58.0
255.6638513 247.508041381836 59.0
21.396535289 16.7020797729492 60.0
0 4.17463970184326 61.0
161.464323483046 163.383651733398 0.0
161.464323483046 152.833740234375 1.0
160.19873735 157.677688598633 2.0
155.40966762 159.434783935547 3.0
0 -2.52708196640015 4.0
-8.782604249e-09 6.43675565719604 5.0
-53.1017804526513 -57.8128204345703 6.0
65.3489096532794 69.3166885375977 7.0
52.9975025992794 50.601936340332 8.0
12.3514070542794 27.7344303131104 9.0
85.5884879572794 86.1074600219727 10.0
73.8637784712794 72.0203552246094 11.0
75.7731636902794 80.6948699951172 12.0
75.7731636902794 83.8072357177734 13.0
75.7731636902794 78.4548492431641 14.0
0 -5.08563232421875 15.0
0 -1.04890012741089 16.0
-5.436396189e-09 -5.80834102630615 17.0
52.5459279188993 58.3730163574219 18.0
-52.5459279519191 -58.6178588867188 19.0
-52.5459279309191 -61.0719718933105 20.0
-52.5459279359191 -56.0626029968262 21.0
74.4799720193728 74.7979507446289 22.0
140.252255884516 144.303253173828 23.0
107.953168718145 109.778038024902 24.0
20.087941068145 18.9377727508545 25.0
214.732227906092 223.229339599609 26.0
214.732227906092 226.594879150391 27.0
24.7549790840916 25.7979679107666 28.0
239.487206986092 254.562881469727 29.0
239.487206986092 256.168975830078 30.0
46.4226951943921 47.4500350952148 31.0
9.2972348456 9.45430088043213 32.0
21.9340440829538 17.9560146331787 33.0
21.9340440829538 21.3133792877197 34.0
21.9340440908894 20.988166809082 35.0
9.205955454 12.1505517959595 36.0
7.8661622202 6.61413812637329 37.0
-9.22286532e-09 -6.14931344985962 38.0
-9.22286532e-09 -2.11518311500549 39.0
-12.7618526829595 -8.93564510345459 40.0
-25.7379024782438 -27.8157844543457 41.0
4.63881470206616 7.281174659729 42.0
-32.9734878310662 -34.9303321838379 43.0
94.186957054052 93.4640884399414 44.0
94.1869571050662 96.5966644287109 45.0
7.3911972129 5.03342723846436 46.0
2.7138170514 1.33035409450531 47.0
161.46432349 173.43701171875 48.0
155.40966763 147.898818969727 49.0
0 -1.2395281791687 50.0
65.348909654 72.1827087402344 51.0
85.588487958 82.1266555786133 52.0
75.773163691 79.9139709472656 53.0
75.773163691 76.5281066894531 54.0
75.773163691 80.4564208984375 55.0
0 -2.32747364044189 56.0
74.479972019 78.4493103027344 57.0
214.7322279 216.704940795898 58.0
239.48720699 246.550704956055 59.0
21.934044091 21.9634418487549 60.0
0 1.44568991661072 61.0
139.81114941 172.960586547852 0.0
139.81114941 163.743499755859 1.0
138.72371154 162.726257324219 2.0
133.25907635 139.098007202148 3.0
5.5821240868 5.601393699646 4.0
5.5821240872 8.69606971740723 5.0
-67.648715297 -46.5973510742188 6.0
77.140169597 76.7492980957031 7.0
50.26987356 52.6912727355957 8.0
26.870296028 28.2903938293457 9.0
91.660201736 74.2676239013672 10.0
79.593270253 83.2192840576172 11.0
79.152063418 87.2714614868164 12.0
79.152063418 84.8142929077148 13.0
79.152063418 80.5010604858398 14.0
0 1.75771641731262 15.0
0 -1.00091528892517 16.0
-2.4962131261e-10 -3.18909978866577 17.0
55.207716411 59.0972061157227 18.0
-55.207716432 -57.2512435913086 19.0
-55.207716411 -62.5128440856934 20.0
-55.207716423 -59.7905235290527 21.0
77.299299234 77.015266418457 22.0
144.21040881 146.919906616211 23.0
113.17796305 109.035057067871 24.0
21.090090984 21.6090049743652 25.0
221.50970805 220.214736938477 26.0
221.50970805 222.326293945312 27.0
27.369192479 25.0262985229492 28.0
248.87890054 254.409912109375 29.0
248.87890054 249.222778320312 30.0
46.402939231 46.7055397033691 31.0
9.3869740051 8.57632637023926 32.0
22.091582805 18.1158866882324 33.0
22.091582805 21.7424583435059 34.0
22.091582804 22.1324806213379 35.0
9.4773175635 10.0925178527832 36.0
8.3005898756 7.59938144683838 37.0
-5.2605100372e-10 -0.195957005023956 38.0
-5.2605100368e-10 -1.74140310287476 39.0
-11.751540186 -13.1289091110229 40.0
-23.189330043 -24.2517471313477 41.0
5.3867767012 6.74650573730469 42.0
-31.427511193 -33.0878295898438 43.0
85.859865353 91.6721267700195 44.0
85.859865365 96.119384765625 45.0
7.2040498429 3.69613766670227 46.0
2.5412092818 2.63412261009216 47.0
139.81114942 171.721771240234 48.0
133.25907636 140.233779907227 49.0
5.5821240968 3.48210573196411 50.0
77.140169605 73.6385726928711 51.0
91.660201744 85.2067413330078 52.0
79.152063426 81.7283782958984 53.0
79.152063426 80.6709747314453 54.0
79.152063426 78.910026550293 55.0
0 -1.31793212890625 56.0
77.299299244 74.585807800293 57.0
221.50970806 219.083526611328 58.0
248.87890055 249.451721191406 59.0
22.091582814 21.5800094604492 60.0
0 1.68747019767761 61.0
137.17687742 143.006500244141 0.0
137.17687742 143.031036376953 1.0
135.86355256 143.535263061523 2.0
131.0744823 135.766204833984 3.0
0 -2.11607456207275 4.0
0 0.342018187046051 5.0
-59.42861149 -59.2346534729004 6.0
69.705106679 70.1772918701172 7.0
65.43029471 62.4528312683105 8.0
4.2748119615 8.48783111572266 9.0
78.669161727 80.2065200805664 10.0
65.229460649 71.2585144042969 11.0
77.500548395 80.8047180175781 12.0
77.500548395 77.9000701904297 13.0
77.500548395 76.9198608398438 14.0
0 -0.64198762178421 15.0
0 0.28728449344635 16.0
0 -3.15485000610352 17.0
59.754945382 58.5015258789062 18.0
-59.754945394 -59.4383087158203 19.0
-59.754945372 -60.0649642944336 20.0
-59.754945384 -61.8801422119141 21.0
80.656571382 77.8813858032227 22.0
140.87903669 140.394561767578 23.0
106.41719675 107.357681274414 24.0
22.90053999 20.4083709716797 25.0
221.53560808 223.594223022461 26.0
221.53560808 221.987747192383 27.0
27.282645659 25.3551406860352 28.0
248.81825375 253.574676513672 29.0
248.81825375 252.408157348633 30.0
47.033013387 47.2299919128418 31.0
9.3131031344 9.18030452728271 32.0
20.901625993 20.4819431304932 33.0
20.901625993 21.1129970550537 34.0
20.901625993 20.8195400238037 35.0
9.3083381265 8.28715133666992 36.0
8.044767894 8.38281917572021 37.0
-2.5262790395e-09 -0.804296553134918 38.0
-2.5262785966e-09 0.113371789455414 39.0
-12.387792857 -14.5510091781616 40.0
-23.81592281 -27.5234241485596 41.0
4.9278776077 3.5162615776062 42.0
-31.108913619 -34.7394485473633 43.0
92.052766479 91.3064651489258 44.0
92.05276649 90.2343292236328 45.0
6.7273095346 3.00315618515015 46.0
2.6568330431 1.83220052719116 47.0
137.17687743 136.681884765625 48.0
131.07448231 127.542411804199 49.0
0 3.5438928604126 50.0
69.705106686 70.9862594604492 51.0
78.669161734 84.2529830932617 52.0
77.500548402 81.0479965209961 53.0
77.500548402 80.9212951660156 54.0
77.500548402 79.4541549682617 55.0
0 -0.0822644829750061 56.0
80.656571391 80.3950500488281 57.0
221.53560809 218.26057434082 58.0
248.81825375 246.469757080078 59.0
20.901626002 19.9398994445801 60.0
0 2.89454770088196 61.0
137.17687742 143.006271362305 0.0
137.17687742 143.039321899414 1.0
135.86355256 143.477111816406 2.0
131.0744823 135.658432006836 3.0
0 -1.99662303924561 4.0
0 0.341170310974121 5.0
-59.42861149 -59.2195014953613 6.0
69.705106679 70.1771087646484 7.0
65.43029471 62.4528312683105 8.0
4.2748119615 8.48783111572266 9.0
78.669161727 80.2065200805664 10.0
65.229460649 71.2585144042969 11.0
77.500548395 80.8047180175781 12.0
77.500548395 77.9000701904297 13.0
77.500548395 76.9198608398438 14.0
0 -0.64198762178421 15.0
0 0.28728449344635 16.0
0 -3.15485000610352 17.0
59.754945382 58.5015258789062 18.0
-59.754945394 -59.4383087158203 19.0
-59.754945372 -60.0649642944336 20.0
-59.754945384 -61.8801422119141 21.0
80.656571382 77.8813858032227 22.0
140.87903669 140.394561767578 23.0
106.41719675 107.357681274414 24.0
22.90053999 20.4083709716797 25.0
221.53560808 223.594223022461 26.0
221.53560808 221.987747192383 27.0
27.282645659 25.3551406860352 28.0
248.81825375 253.574676513672 29.0
248.81825375 252.408157348633 30.0
47.033013387 47.2299919128418 31.0
9.3131031344 9.18030452728271 32.0
20.901625993 20.4819431304932 33.0
20.901625993 21.1129970550537 34.0
20.901625993 20.8195400238037 35.0
9.3083381265 8.28715133666992 36.0
8.044767894 8.38281917572021 37.0
-2.5262790395e-09 -0.804296553134918 38.0
-2.5262785966e-09 0.113371789455414 39.0
-12.387792857 -14.5510091781616 40.0
-23.81592281 -27.5234241485596 41.0
4.9278776077 3.5162615776062 42.0
-31.108913619 -34.7394485473633 43.0
92.052766479 91.3064651489258 44.0
92.05276649 90.2343292236328 45.0
6.7273095346 3.00315618515015 46.0
2.6568330431 1.83220052719116 47.0
137.17687743 136.686645507812 48.0
131.07448231 127.367897033691 49.0
0 3.56870460510254 50.0
69.705106686 70.9871215820312 51.0
78.669161734 84.2529830932617 52.0
77.500548402 81.0479965209961 53.0
77.500548402 80.9212951660156 54.0
77.500548402 79.4541549682617 55.0
0 -0.0822644829750061 56.0
80.656571391 80.3950500488281 57.0
221.53560809 218.26057434082 58.0
248.81825375 246.469757080078 59.0
20.901626002 19.9398994445801 60.0
0 2.89454770088196 61.0
168.90515651609 159.122817993164 0.0
168.90515651609 166.531951904297 1.0
167.43504021 171.66389465332 2.0
162.64597048 168.984817504883 3.0
0 4.72804737091064 4.0
-9.16977491487e-09 6.43109130859375 5.0
-68.588820658 -64.9397583007812 6.0
79.2140531718899 76.6658477783203 7.0
56.7458563138899 45.9822425842285 8.0
22.4681968568899 19.6554565429688 9.0
90.8786468158899 88.815055847168 10.0
77.3633989458899 78.3051834106445 11.0
76.6965648058899 82.0606460571289 12.0
76.6965648058899 80.7486343383789 13.0
76.6965648058899 80.6511077880859 14.0
0 -4.2552924156189 15.0
0 -1.22108054161072 16.0
-7.965374712e-09 -0.675061523914337 17.0
61.364476936849 59.2884330749512 18.0
-61.3644769508348 -60.989917755127 19.0
-61.3644769288348 -58.8710289001465 20.0
-61.3644769378348 -56.3006782531738 21.0
82.622512807 80.3920440673828 22.0
139.172072986849 142.572463989258 23.0
108.663244583561 107.339179992676 24.0
22.9426401675613 19.8159141540527 25.0
221.794585792181 229.304321289062 26.0
221.794585792181 223.401016235352 27.0
23.7874681901807 28.2870502471924 28.0
245.582053982181 252.127151489258 29.0
245.582053982181 251.419799804688 30.0
46.847469241 48.8461303710938 31.0
9.4268572214 8.66774368286133 32.0
21.2580358733342 21.5374069213867 33.0
21.2580358733342 16.5085620880127 34.0
21.258035875 19.6609516143799 35.0
9.479636628 7.60832262039185 36.0
8.2935228277 6.76836776733398 37.0
-9.623241476e-09 0.104680478572845 38.0
-9.623241477e-09 3.05262517929077 39.0
-11.0522725879174 -18.3972225189209 40.0
-24.1504029887406 -29.0594959259033 41.0
5.02482540750825 10.7464475631714 42.0
-31.9641432549082 -34.5761222839355 43.0
106.280698989908 90.3768539428711 44.0
106.280698979908 98.7340316772461 45.0
6.7489761069 9.64444637298584 46.0
2.46849683 2.90503764152527 47.0
168.90515651 168.032791137695 48.0
162.64597049 166.707244873047 49.0
0 -5.79203367233276 50.0
79.214053172 75.3565368652344 51.0
90.878646816 83.126106262207 52.0
76.696564806 81.4167556762695 53.0
76.696564806 81.5614013671875 54.0
76.696564806 80.225212097168 55.0
0 -0.590853869915009 56.0
82.622512811 80.1401290893555 57.0
221.79458579 224.469482421875 58.0
245.58205398 249.657485961914 59.0
21.258035877 19.7219791412354 60.0
0 3.48126840591431 61.0
153.54035896 156.634017944336 0.0
153.54035896 162.650131225586 1.0
152.22469149 165.504196166992 2.0
147.43561663 163.959274291992 3.0
-2.1579126107e-10 -3.7332181930542 4.0
0 2.33964252471924 5.0
-54.97369832 -61.9161567687988 6.0
64.385589912 73.1185302734375 7.0
62.135557214 50.6164741516113 8.0
2.2500326884 21.1481437683105 9.0
85.548964364 88.3032073974609 10.0
72.996807907 76.8216171264648 11.0
87.088646184 81.814453125 12.0
87.088646184 80.3844146728516 13.0
87.088646184 80.3746185302734 14.0
0 -0.884523212909698 15.0
0 -2.72604584693909 16.0
0 -3.58486890792847 17.0
59.451790441 58.0313301086426 18.0
-59.451790451 -61.025749206543 19.0
-59.451790429 -57.6452598571777 20.0
-59.451790441 -58.3103523254395 21.0
80.28465414 77.3173370361328 22.0
143.7659406 140.878707885742 23.0
110.66066673 99.9706649780273 24.0
22.121322261 17.6918830871582 25.0
224.05059475 228.622421264648 26.0
224.05059475 225.746597290039 27.0
24.099693007 24.2206344604492 28.0
248.15028777 261.557800292969 29.0
248.15028777 256.305419921875 30.0
46.554673613 48.1017951965332 31.0
9.3929642438 9.86105442047119 32.0
20.832863689 16.9543838500977 33.0
20.832863689 18.2218990325928 34.0
20.832863689 18.0034637451172 35.0
9.152731159 12.3382711410522 36.0
8.0232986828 7.12056064605713 37.0
0 -2.50459575653076 38.0
0 -3.43044519424438 39.0
-11.248320913 -11.3123111724854 40.0
-23.708397604 -25.0745277404785 41.0
5.5423207026 4.54512882232666 42.0
-31.824503961 -34.5991401672363 43.0
80.290016502 84.5041885375977 44.0
80.290016515 85.0938949584961 45.0
6.8894830881 7.41545677185059 46.0
2.3526289764 2.72590899467468 47.0
153.54035897 162.369812011719 48.0
147.43561664 148.381958007812 49.0
0 3.94319248199463 50.0
64.385589922 75.5794219970703 51.0
85.548964374 91.4917602539062 52.0
87.088646194 81.31298828125 53.0
87.088646194 76.5849838256836 54.0
87.088646194 78.904670715332 55.0
0 3.55851602554321 56.0
80.28465415 79.340934753418 57.0
224.05059476 223.237777709961 58.0
248.15028778 246.381652832031 59.0
20.832863699 22.4873046875 60.0
0 3.09565210342407 61.0
137.17687742 143.006103515625 0.0
137.17687742 143.046401977539 1.0
135.86355256 143.427612304688 2.0
131.0744823 135.567276000977 3.0
0 -1.89350032806396 4.0
0 0.340449452400208 5.0
-59.42861149 -59.206615447998 6.0
69.705106679 70.1769332885742 7.0
65.43029471 62.4528312683105 8.0
4.2748119615 8.48783111572266 9.0
78.669161727 80.2065200805664 10.0
65.229460649 71.2585144042969 11.0
77.500548395 80.8047180175781 12.0
77.500548395 77.9000701904297 13.0
77.500548395 76.9198608398438 14.0
0 -0.64198762178421 15.0
0 0.28728449344635 16.0
0 -3.15485000610352 17.0
59.754945382 58.5015258789062 18.0
-59.754945394 -59.4383087158203 19.0
-59.754945372 -60.0649642944336 20.0
-59.754945384 -61.8801422119141 21.0
80.656571382 77.8813858032227 22.0
140.87903669 140.394561767578 23.0
106.41719675 107.357681274414 24.0
22.90053999 20.4083709716797 25.0
221.53560808 223.594223022461 26.0
221.53560808 221.987747192383 27.0
27.282645659 25.3551406860352 28.0
248.81825375 253.574676513672 29.0
248.81825375 252.408157348633 30.0
47.033013387 47.2299919128418 31.0
9.3131031344 9.18030452728271 32.0
20.901625993 20.4819431304932 33.0
20.901625993 21.1129970550537 34.0
20.901625993 20.8195400238037 35.0
9.3083381265 8.28715133666992 36.0
8.044767894 8.38281917572021 37.0
-2.5262790395e-09 -0.804296553134918 38.0
-2.5262785966e-09 0.113371789455414 39.0
-12.387792857 -14.5510091781616 40.0
-23.81592281 -27.5234241485596 41.0
4.9278776077 3.5162615776062 42.0
-31.108913619 -34.7394485473633 43.0
92.052766479 91.3064651489258 44.0
92.05276649 90.2343292236328 45.0
6.7273095346 3.00315618515015 46.0
2.6568330431 1.83220052719116 47.0
137.17687743 136.690689086914 48.0
131.07448231 127.219017028809 49.0
0 3.58974075317383 50.0
69.705106686 70.9878311157227 51.0
78.669161734 84.2529830932617 52.0
77.500548402 81.0479965209961 53.0
77.500548402 80.9212951660156 54.0
77.500548402 79.4541549682617 55.0
0 -0.0822644829750061 56.0
80.656571391 80.3950500488281 57.0
221.53560809 218.26057434082 58.0
248.81825375 246.469757080078 59.0
20.901626002 19.9398994445801 60.0
0 2.89454770088196 61.0
175.985955883785 163.913772583008 0.0
175.985955883785 169.106384277344 1.0
174.83687136 169.008483886719 2.0
170.04780159 164.93229675293 3.0
0 -2.15561604499817 4.0
-5.44831445e-09 4.85976505279541 5.0
-69.9965568706364 -67.6281051635742 6.0
81.6824462742273 73.2623443603516 7.0
50.4815024252273 52.3644142150879 8.0
31.2009438492273 22.9435348510742 9.0
87.6109575212273 90.0299530029297 10.0
74.9150300602273 76.6412124633789 11.0
80.1773899862273 81.0155563354492 12.0
80.1773899862273 77.8965225219727 13.0
80.1773899862273 80.0086059570312 14.0
4.413536655e-09 -1.00556683540344 15.0
3.2297705184e-09 -0.77076929807663 16.0
-2.71324236999999e-09 -1.07371878623962 17.0
58.6667945642508 59.3933181762695 18.0
-58.666794626841 -58.8314666748047 19.0
-58.666794609841 -58.9333190917969 20.0
-58.666794583841 -58.158863067627 21.0
80.5201165578182 78.0052185058594 22.0
145.829294952306 143.579040527344 23.0
110.346702275382 111.749877929688 24.0
23.9900580533823 20.4592990875244 25.0
226.34941150757 225.940490722656 26.0
226.34941150757 225.795135498047 27.0
27.32402915557 26.0282249450684 28.0
253.67344066757 255.521575927734 29.0
253.67344066757 254.060623168945 30.0
46.6514525951342 46.4862442016602 31.0
9.28272762140731 8.72709274291992 32.0
21.8533219739091 18.7860927581787 33.0
21.8533219739091 14.5370988845825 34.0
21.8533220176301 19.9592723846436 35.0
9.17889328053384 6.82407569885254 36.0
7.91905011503384 7.11823844909668 37.0
-5.87578609999998e-09 -0.209656417369843 38.0
-5.87578609999998e-09 -3.07396459579468 39.0
-11.6120369999205 -11.663122177124 40.0
-24.246919736319 -23.8504734039307 41.0
5.36956277955608 5.73640871047974 42.0
-32.1910388923561 -34.1623840332031 43.0
100.364691104356 93.3474044799805 44.0
100.364690774993 94.7572555541992 45.0
6.91645026235474 6.85776996612549 46.0
2.36653329335474 2.46229028701782 47.0
175.98595589 166.057220458984 48.0
170.0478016 161.040756225586 49.0
0 3.36818408966064 50.0
81.682446274 77.9660339355469 51.0
87.610957522 87.2157592773438 52.0
80.177389986 80.6327209472656 53.0
80.177389986 81.2730560302734 54.0
80.177389986 79.9518661499023 55.0
0 -0.0962209105491638 56.0
80.520116558 79.2682952880859 57.0
226.34941151 222.593368530273 58.0
253.67344067 251.147918701172 59.0
21.853322018 23.1302165985107 60.0
0 2.116774559021 61.0
147.982033057583 157.456436157227 0.0
147.982033057583 151.967163085938 1.0
146.808413813287 153.377685546875 2.0
142.01934397 150.933837890625 3.0
-1.55484310406e-09 -0.673610985279083 4.0
-1.55484309999997e-09 -2.84894132614136 5.0
-64.9693040339936 -64.3301467895508 6.0
77.6547829144777 74.1872863769531 7.0
61.2786926634778 55.4030647277832 8.0
16.3760902514777 7.20984888076782 9.0
97.5261141374777 89.7391510009766 10.0
83.6472874924777 79.6450042724609 11.0
85.2981411144777 83.8642730712891 12.0
85.2981411144777 76.4624557495117 13.0
85.2981411144777 81.5436782836914 14.0
2.3328021736e-08 -2.03496789932251 15.0
2.3328021735e-08 -1.6052393913269 16.0
5.464350236e-08 -2.72059202194214 17.0
56.2518018559467 59.0783615112305 18.0
-56.2518018797166 -57.0891418457031 19.0
-56.2518018707166 -59.2854194641113 20.0
-56.2518017707166 -58.115608215332 21.0
77.1875821073724 73.3798065185547 22.0
153.454771802806 145.008895874023 23.0
115.220596133508 115.750228881836 24.0
21.7005584165075 20.8819332122803 25.0
230.642353915167 228.381210327148 26.0
230.642353915167 229.708221435547 27.0
26.8883420391666 21.3647422790527 28.0
257.530695955167 253.23681640625 29.0
257.530695955167 255.336776733398 30.0
46.7385281475469 45.537841796875 31.0
9.28724497493877 8.4355525970459 32.0
20.9357803369809 22.6376686096191 33.0
20.9357803369809 19.0790691375732 34.0
20.935780467613 20.4264030456543 35.0
8.86054244871795 8.39873600006104 36.0
7.60121916461794 6.85336446762085 37.0
2.62465480000001e-09 1.1959273815155 38.0
2.62465480000001e-09 -1.17553234100342 39.0
-11.6034306368583 -12.5643882751465 40.0
-24.5599220844463 -26.7411251068115 41.0
4.75986482786468 1.92820739746094 42.0
-31.7734218810648 -31.6524333953857 43.0
89.4696777832616 86.4044494628906 44.0
89.4696781210649 91.9600372314453 45.0
7.05809588635643 4.443039894104 46.0
2.46335994485643 2.26466846466064 47.0
147.98203306 156.095245361328 48.0
142.01934397 153.16194152832 49.0
0 -3.73428678512573 50.0
77.654782914 76.0231323242188 51.0
97.526114137 92.0755157470703 52.0
85.298141114 81.5651702880859 53.0
85.298141114 80.2131118774414 54.0
85.298141114 79.9203643798828 55.0
5.4643502359e-08 3.07315587997437 56.0
77.187582108 79.4883346557617 57.0
230.64235391 222.583419799805 58.0
257.53069595 249.491134643555 59.0
20.935780468 19.6116352081299 60.0
2.6246548239e-09 3.64096260070801 61.0
190.647378735192 161.895294189453 0.0
190.647378735192 160.695236206055 1.0
189.4975724 169.331817626953 2.0
184.70850265 181.659439086914 3.0
-4.3192167311e-10 3.52344131469727 4.0
-8.678559515e-09 15.7490911483765 5.0
-58.1348169851429 -31.1547870635986 6.0
70.1895573370002 87.4838104248047 7.0
46.0646208670002 52.0309371948242 8.0
24.1249364700002 28.5126190185547 9.0
92.8833664690002 89.747200012207 10.0
79.8768505570002 74.7976608276367 11.0
89.3858795700002 81.074577331543 12.0
89.3858795700002 82.9573669433594 13.0
89.3858795700002 81.2618026733398 14.0
2.7568023062e-09 -2.23969578742981 15.0
0 0.438403308391571 16.0
2.18829254000001e-09 -1.25363516807556 17.0
56.1767973424546 54.1891479492188 18.0
-56.1767973155003 -59.4069366455078 19.0
-56.1767972965003 -57.4609565734863 20.0
-56.1767972815003 -56.5178642272949 21.0
78.1198582272163 75.9596328735352 22.0
151.166098771455 140.546081542969 23.0
118.755770164318 111.101417541504 24.0
25.4933099443183 17.3717384338379 25.0
229.285957000385 228.89714050293 26.0
229.285957000385 218.922439575195 27.0
26.5214315953848 31.8759117126465 28.0
255.807388600385 266.027862548828 29.0
255.807388600385 256.318328857422 30.0
47.2672013870389 45.4459190368652 31.0
9.47576333448043 9.64591884613037 32.0
21.9430609464287 20.3555603027344 33.0
21.9430609464287 21.4529819488525 34.0
21.9430609783794 20.6161727905273 35.0
9.26390241817459 9.9046049118042 36.0
7.90091200137459 6.74843502044678 37.0
-6.76092301999999e-09 1.12744498252869 38.0
-6.76092301999999e-09 -3.86343288421631 39.0
-11.3906696547501 -14.4837303161621 40.0
-24.5414105528136 -25.6143016815186 41.0
5.40322002876681 8.80830192565918 42.0
-32.6581408011668 -33.6052436828613 43.0
90.5404341372887 93.673469543457 44.0
90.5404342831668 88.9677505493164 45.0
6.3801672475 8.5949535369873 46.0
2.31548583 3.02157020568848 47.0
190.64737873 161.12158203125 48.0
184.70850266 171.928924560547 49.0
0 3.43359184265137 50.0
70.189557337 72.942626953125 51.0
92.883366469 92.9441528320312 52.0
89.38587957 81.1010208129883 53.0
89.38587957 81.7617340087891 54.0
89.38587957 79.0190811157227 55.0
2.1882925456e-09 4.86723518371582 56.0
78.119858227 78.5853805541992 57.0
229.285957 219.896591186523 58.0
255.8073886 252.756561279297 59.0
21.943060978 15.9973011016846 60.0
0 1.78565287590027 61.0
212.391738827015 153.841064453125 0.0
212.391738827015 156.152008056641 1.0
211.04319376 162.609252929688 2.0
205.5785585 183.532791137695 3.0
18.404050808 -11.9846677780151 4.0
18.4040508170453 16.5005359649658 5.0
-61.3077697580453 -106.727119445801 6.0
74.2237211591586 101.702270507812 7.0
47.5084168701586 49.3908500671387 8.0
26.7153042891586 26.0035495758057 9.0
73.6279026801586 82.7762222290039 10.0
62.1474872191586 58.7497711181641 11.0
78.3848871701586 68.9429779052734 12.0
78.3848871701586 72.1487426757812 13.0
78.3848871701585 84.5159912109375 14.0
4.213840709e-10 7.65522480010986 15.0
4.2138407099e-10 3.21506071090698 16.0
5.24329936999999e-09 4.69373941421509 17.0
56.7790807251774 59.1677894592285 18.0
-56.7790806952379 -56.4286880493164 19.0
-56.7790806772379 -56.0826797485352 20.0
-56.7790806622379 -68.0509414672852 21.0
79.0602866060699 71.6340713500977 22.0
149.501502613177 134.62646484375 23.0
119.756521836472 102.935264587402 24.0
24.5677496374717 26.2532119750977 25.0
228.561789214029 220.424560546875 26.0
228.561789214029 206.224975585938 27.0
27.4961151460291 28.7196178436279 28.0
256.057904364029 246.56787109375 29.0
256.057904364029 273.834167480469 30.0
45.391893868001 38.7327346801758 31.0
8.76348561509971 6.41286516189575 32.0
22.2812059441359 13.2257738113403 33.0
22.2812059441359 20.2075481414795 34.0
22.2812059764684 22.1812534332275 35.0
8.66141742863524 15.2447719573975 36.0
7.34531140923524 4.93151473999023 37.0
-6.99365361999996e-09 -6.4220929145813 38.0
-6.99365360999999e-09 -2.56813478469849 39.0
-12.1264598581189 -3.07913827896118 40.0
-26.0813195914196 -26.7585544586182 41.0
5.60695360219884 0.621493518352509 42.0
-34.4708309687988 -32.5315818786621 43.0
84.1913674207981 89.0402679443359 44.0
84.1913670312712 90.0791091918945 45.0
6.6020739941 6.19962120056152 46.0
2.5827328274 1.97778654098511 47.0
212.39173882 158.674331665039 48.0
205.57855851 163.197311401367 49.0
18.404050817 2.99577260017395 50.0
74.223721159 75.4420852661133 51.0
73.62790268 83.2851333618164 52.0
78.38488717 75.5351104736328 53.0
78.38488717 68.6956787109375 54.0
78.38488717 81.1668319702148 55.0
5.2432993646e-09 5.09521484375 56.0
79.060286606 76.9335174560547 57.0
228.56178922 207.424560546875 58.0
256.05790436 258.141723632812 59.0
22.281205976 22.5056571960449 60.0
0 6.23766803741455 61.0
152.01511135 162.740570068359 0.0
152.01511135 168.15510559082 1.0
150.73624439 159.719116210938 2.0
145.94717467 143.296997070312 3.0
0 -0.225555717945099 4.0
-1.3141431381e-09 -1.60792803764343 5.0
-61.351926906 -65.0940322875977 6.0
73.227952086 72.3794479370117 7.0
51.97784765 50.6285438537598 8.0
21.250104435 23.1153469085693 9.0
90.12973698 85.8495178222656 10.0
77.301623962 74.9970550537109 11.0
82.87862434 79.1162033081055 12.0
82.87862434 78.6806869506836 13.0
82.87862434 80.2984237670898 14.0
0 -0.18605238199234 15.0
0 0.416769862174988 16.0
-5.6875996278e-09 -1.38802456855774 17.0
59.796153207 57.8106422424316 18.0
-59.7961532296897 -58.6203651428223 19.0
-59.7961532076897 -61.0177040100098 20.0
-59.7961532186897 -57.3818244934082 21.0
82.21616587 78.9521865844727 22.0
146.35352463 143.477645874023 23.0
111.47855424 109.745254516602 24.0
21.472918849 19.1866321563721 25.0
228.569690518992 223.430084228516 26.0
228.569690518992 220.779678344727 27.0
24.9196510329921 28.3873729705811 28.0
253.489341548992 250.355941772461 29.0
253.489341548992 250.91374206543 30.0
46.770680272 46.1404495239258 31.0
9.2903104244 8.69324684143066 32.0
22.420012657 20.2706165313721 33.0
22.420012657 18.0167675018311 34.0
22.420012654 21.0888233184814 35.0
9.6028777931 8.00218200683594 36.0
8.2744029474 7.47547245025635 37.0
-9.7306891187e-09 -2.50349807739258 38.0
-9.7306891154e-09 -2.84876680374146 39.0
-12.512212937 -13.2917108535767 40.0
-24.94288347 -27.4303588867188 41.0
4.9331253561 3.24215149879456 42.0
-32.44445226 -32.925838470459 43.0
96.15539432 92.8194274902344 44.0
96.155394318 93.8757858276367 45.0
6.7588287829 6.33729934692383 46.0
2.4784149669 2.26625633239746 47.0
152.01511135 169.551361083984 48.0
145.94717468 144.236389160156 49.0
0 -1.02974104881287 50.0
73.227952086 74.794189453125 51.0
90.129736981 89.9113235473633 52.0
82.878624341 80.9114456176758 53.0
82.878624341 78.8574447631836 54.0
82.878624341 80.5269470214844 55.0
0 1.56181168556213 56.0
82.216165877 79.1684265136719 57.0
228.56969052 219.438079833984 58.0
253.48934155 249.097885131836 59.0
22.42001266 24.9486141204834 60.0
0 1.63914787769318 61.0
166.92464535 158.599151611328 0.0
166.92464535 147.994705200195 1.0
165.87453414 135.021545410156 2.0
161.08546345 163.482192993164 3.0
0 4.50570106506348 4.0
0 12.3102073669434 5.0
-62.994949176 -57.1477088928223 6.0
70.7556407624234 71.6711044311523 7.0
50.9264333854233 49.046760559082 8.0
19.8292073774233 6.32955551147461 9.0
81.0410764364234 84.5661468505859 10.0
66.8364521564233 66.0532684326172 11.0
83.1701834614233 73.1791152954102 12.0
83.1701834614233 84.5976257324219 13.0
83.1701834614233 74.6877670288086 14.0
0 -4.56651163101196 15.0
0 0.643130958080292 16.0
-9.49209839999999e-11 4.12973213195801 17.0
60.690582972 48.6859664916992 18.0
-60.690582990635 -60.5208930969238 19.0
-60.690582968635 -58.1220283508301 20.0
-60.690582979635 -64.3530197143555 21.0
81.286060358 81.422248840332 22.0
129.36456076 148.115982055664 23.0
96.213201018 108.916793823242 24.0
19.284544804 20.9009742736816 25.0
210.650621126642 221.899459838867 26.0
210.650621126642 217.501052856445 27.0
27.0471453746423 28.5838508605957 28.0
237.697766496642 266.224853515625 29.0
237.697766496642 249.328659057617 30.0
46.775123688 46.4126968383789 31.0
9.2600397946 8.8416223526001 32.0
20.595477386 15.0750246047974 33.0
20.595477386 20.6151237487793 34.0
20.595477381 19.7769069671631 35.0
9.1455036793 9.17498874664307 36.0
7.7203932114 7.43995475769043 37.0
-9.508426294e-09 -2.24972581863403 38.0
-9.5084262938e-09 -2.73203110694885 39.0
-11.746081492 -18.3729400634766 40.0
-23.416886055 -27.3278503417969 41.0
5.428921787 6.39367961883545 42.0
-31.085185306 -32.4104537963867 43.0
90.719675113 86.2834548950195 44.0
90.719675108 94.6495208740234 45.0
5.9850864964 7.37868213653564 46.0
2.2774254241 2.5459201335907 47.0
166.92464536 137.365997314453 48.0
161.08546346 126.97135925293 49.0
0 -12.3505821228027 50.0
70.755640762 78.9581680297852 51.0
81.041076437 80.8466339111328 52.0
83.170183462 76.2148513793945 53.0
83.170183462 80.8953247070312 54.0
83.170183462 78.5184326171875 55.0
0 8.51326751708984 56.0
81.286060365 79.9564590454102 57.0
210.65062113 211.277969360352 58.0
237.6977665 249.14372253418 59.0
20.595477387 21.4344730377197 60.0
0 1.38015127182007 61.0
209.75326329 167.277633666992 0.0
209.75326329 168.525314331055 1.0
208.43933093 156.866806030273 2.0
202.97469574 185.758682250977 3.0
13.728284255 -1.07649993896484 4.0
13.728284255 4.46328258514404 5.0
-56.95207807 -74.5553512573242 6.0
70.066394197 68.254508972168 7.0
41.555360252 50.678840637207 8.0
28.511033935 11.3005437850952 9.0
83.28096054 87.8781661987305 10.0
70.307423857 75.0391693115234 11.0
75.836651013 80.1827392578125 12.0
75.836651013 76.0610809326172 13.0
75.836651013 77.7435760498047 14.0
0 1.94851970672607 15.0
0 -1.71258807182312 16.0
-7.5911282875e-14 2.92896032333374 17.0
59.305429194 59.1812934875488 18.0
-59.305429214 -58.7015190124512 19.0
-59.305429192 -58.6372375488281 20.0
-59.305429204 -66.0722885131836 21.0
80.522346428 74.4707641601562 22.0
149.43437571 139.530868530273 23.0
111.19272854 111.497436523438 24.0
22.256610643 21.5800495147705 25.0
229.95672215 226.459075927734 26.0
229.95672215 227.606903076172 27.0
28.022503166 23.5910911560059 28.0
257.97922532 243.121612548828 29.0
257.97922532 251.300827026367 30.0
47.062569824 46.3496932983398 31.0
9.40674248 9.49158668518066 32.0
21.216917214 20.4482746124268 33.0
21.216917214 14.5653305053711 34.0
21.216917214 20.6590099334717 35.0
9.2094950057 8.82297611236572 36.0
8.07706796 6.85976552963257 37.0
-1.2115874183e-13 -4.34025955200195 38.0
-1.2115874183e-13 0.262088596820831 39.0
-11.908943733 -17.4776306152344 40.0
-25.119745562 -26.8792362213135 41.0
4.9751376311 9.49308204650879 42.0
-32.671583021 -36.1850128173828 43.0
98.451009504 94.940055847168 44.0
98.451009517 93.7212753295898 45.0
6.8054452948 4.84517860412598 46.0
2.8386070371 1.68762540817261 47.0
209.7532633 168.668914794922 48.0
202.97469575 171.3173828125 49.0
13.728284265 2.75528764724731 50.0
70.066394207 75.8960952758789 51.0
83.28096055 84.5548934936523 52.0
75.836651023 81.7414245605469 53.0
75.836651023 74.7881240844727 54.0
75.836651023 80.4707794189453 55.0
0 0.88787168264389 56.0
80.522346438 79.0495452880859 57.0
229.95672216 218.256225585938 58.0
257.97922533 242.452713012695 59.0
21.216917224 17.4710559844971 60.0
0 2.72514605522156 61.0
172.92099326 165.037796020508 0.0
172.92099326 168.324829101562 1.0
171.65917337 163.943984985352 2.0
166.87010305 167.634353637695 3.0
0 -1.51656866073608 4.0
0 0.710752546787262 5.0
-61.433736469 -60.8258743286133 6.0
71.64988361 72.4676055908203 7.0
55.511403747 52.718189239502 8.0
16.138479854 21.7563076019287 9.0
90.964224067 89.0090103149414 10.0
77.964670603 76.6576309204102 11.0
83.204517574 82.6695938110352 12.0
83.204517574 81.0991897583008 13.0
83.204517574 79.8214645385742 14.0
0 0.41631007194519 15.0
0 -0.123291730880737 16.0
0 -2.12039017677307 17.0
57.081054085 58.9081230163574 18.0
-57.081054095 -59.4183464050293 19.0
-57.081054074 -60.1552925109863 20.0
-57.081054086 -59.1037101745605 21.0
78.410579862 77.8158569335938 22.0
147.10496557 142.309036254883 23.0
113.4704548 109.823043823242 24.0
23.550208112 19.9987602233887 25.0
225.51554545 222.458511352539 26.0
225.51554545 224.446701049805 27.0
26.183374286 28.3237228393555 28.0
251.69891974 256.030578613281 29.0
251.69891974 255.153915405273 30.0
46.939969802 45.646183013916 31.0
9.3011447032 8.80996894836426 32.0
21.329525767 20.8150081634521 33.0
21.329525767 17.6436882019043 34.0
21.329525767 21.2434539794922 35.0
9.6204980995 9.45438385009766 36.0
8.3532839566 7.84531259536743 37.0
-4.929434163e-10 -1.29540920257568 38.0
-4.929434162e-10 -1.99346613883972 39.0
-11.755164802 -12.866756439209 40.0
-24.407328107 -24.8533611297607 41.0
5.0972306267 6.22848129272461 42.0
-31.964032156 -34.794548034668 43.0
95.465201279 93.9734420776367 44.0
95.465201291 94.4564437866211 45.0
6.8537856479 7.44746017456055 46.0
2.6103753436 1.20440483093262 47.0
172.92099327 165.998001098633 48.0
166.87010306 161.95622253418 49.0
0 1.02102017402649 50.0
71.64988362 74.4577789306641 51.0
90.964224077 89.7902984619141 52.0
83.204517583 81.0613327026367 53.0
83.204517583 80.7883377075195 54.0
83.204517583 78.6458358764648 55.0
0 -0.214804828166962 56.0
78.410579872 78.7178726196289 57.0
225.51554545 220.415634155273 58.0
251.69891975 249.248474121094 59.0
21.329525777 22.2093029022217 60.0
0 2.19265365600586 61.0
174.563944512647 161.488342285156 0.0
174.563944512647 163.452606201172 1.0
173.42738984 160.471694946289 2.0
168.6383201 163.194671630859 3.0
0 3.02020740509033 4.0
-5.927963602e-09 4.47275590896606 5.0
-55.1721014587397 -48.8631286621094 6.0
64.497642085589 90.4137878417969 7.0
48.162924564589 51.4999237060547 8.0
16.334717521589 23.567907333374 9.0
86.568717397589 88.5119323730469 10.0
73.380220323589 77.2399749755859 11.0
76.786569349589 79.8520050048828 12.0
76.786569349589 83.9617156982422 13.0
76.786569349589 81.7519683837891 14.0
4.303224042e-10 -2.65774583816528 15.0
0 -0.466832220554352 16.0
-3.29194132000002e-09 1.55909276008606 17.0
61.8911283464122 58.9040832519531 18.0
-61.8911283328834 -59.9628524780273 19.0
-61.8911283138834 -56.6428260803223 20.0
-61.8911283048835 -59.6584625244141 21.0
82.6656411700781 74.011344909668 22.0
154.399872617412 141.554840087891 23.0
118.517007216765 112.212196350098 24.0
25.1328476997652 21.514066696167 25.0
237.065513785295 232.400848388672 26.0
237.065513785295 224.842498779297 27.0
29.353383245295 28.3377513885498 28.0
266.418897035295 254.77978515625 29.0
266.418897035295 254.502166748047 30.0
46.6654262528997 45.9960289001465 31.0
9.27911179291156 9.52303504943848 32.0
20.7745128652191 21.149076461792 33.0
20.7745128652191 20.5866107940674 34.0
20.7745128909214 16.6605052947998 35.0
9.2980310417223 7.10479736328125 36.0
7.9714331706223 6.45343351364136 37.0
-7.90629667999999e-09 1.31264042854309 38.0
-7.90629667999999e-09 -4.07684564590454 39.0
-10.5807658844417 -12.7510681152344 40.0
-23.6276654320113 -22.9334850311279 41.0
4.79256323172414 5.16328620910645 42.0
-30.8643489788241 -35.1371955871582 43.0
85.5986928048241 85.0373992919922 44.0
85.598692653061 89.9132843017578 45.0
7.1182378535 9.20614051818848 46.0
2.4230751515 2.67674875259399 47.0
174.56394451 167.161712646484 48.0
168.63832011 149.14387512207 49.0
0 8.89645671844482 50.0
64.497642086 77.644401550293 51.0
86.568717397 89.9904174804688 52.0
76.786569349 80.2107620239258 53.0
76.786569349 81.0089111328125 54.0
76.786569349 79.4894866943359 55.0
0 1.90742945671082 56.0
82.66564117 80.8617935180664 57.0
237.06551379 217.191116333008 58.0
266.41889703 251.266067504883 59.0
20.774512891 16.2351665496826 60.0
0 1.62354445457458 61.0
179.25232168 163.998733520508 0.0
179.25232168 165.307373046875 1.0
177.92960652 160.226837158203 2.0
173.1405368 169.373291015625 3.0
0 -0.704248011112213 4.0
-4.7659178585e-10 5.98341274261475 5.0
-62.345393454 -60.3801040649414 6.0
73.484608509 77.7742691040039 7.0
48.629293876 52.6692962646484 8.0
24.85531463 23.842191696167 9.0
82.113999887 85.3835906982422 10.0
69.33096806 77.5535278320312 11.0
78.021719963 83.7888259887695 12.0
78.021719963 78.2337799072266 13.0
78.021719963 78.0484008789062 14.0
0 -0.507340848445892 15.0
0 -0.061289370059967 16.0
-4.0107191721e-09 1.26573145389557 17.0
56.89811104 58.2625160217285 18.0
-56.8981110642561 -59.4961204528809 19.0
-56.8981110422561 -60.060489654541 20.0
-56.8981110532561 -61.0581893920898 21.0
78.999212302 77.0507354736328 22.0
143.27346793 139.825286865234 23.0
108.54837405 111.342636108398 24.0
22.359366677 20.5602684020996 25.0
222.27268025 226.405319213867 26.0
222.27268025 226.170272827148 27.0
27.958197075 24.0431079864502 28.0
250.23087732 252.221939086914 29.0
250.23087732 253.366088867188 30.0
46.574721696 45.8532524108887 31.0
9.3042430108 9.25042629241943 32.0
22.101101254 22.5838832855225 33.0
22.101101254 18.2276439666748 34.0
22.101101251 20.1819248199463 35.0
9.608971595 8.35261631011963 36.0
8.328737502 7.13340425491333 37.0
-7.2195669927e-09 1.16330933570862 38.0
-7.2195669966e-09 2.72856307029724 39.0
-11.715280436 -15.7117128372192 40.0
-24.570662382 -22.7517738342285 41.0
5.2685986261 6.18556070327759 42.0
-32.432891909 -34.0849304199219 43.0
90.553968872 90.4039001464844 44.0
90.553968873 91.1977996826172 45.0
7.1516178957 7.58145904541016 46.0
2.7536288416 2.84569835662842 47.0
179.25232169 164.72395324707 48.0
173.14053681 160.746139526367 49.0
0 2.21334743499756 50.0
73.484608511 76.7437973022461 51.0
82.11399989 84.5780715942383 52.0
78.021719965 81.4580230712891 53.0
78.021719965 81.5313491821289 54.0
78.021719965 80.4742660522461 55.0
0 2.59196257591248 56.0
78.99921231 80.4106216430664 57.0
222.27268025 221.393020629883 58.0
250.23087733 246.987670898438 59.0
22.101101258 20.1114158630371 60.0
0 2.46784090995789 61.0
165.35585942 166.146713256836 0.0
165.35585942 163.769515991211 1.0
164.13882377 157.752502441406 2.0
159.34975405 158.938583374023 3.0
-1.6038711469e-13 4.4372992515564 4.0
-2.0692320027e-13 0.893424570560455 5.0
-70.527160423 -64.4718704223633 6.0
81.190871003 73.4829635620117 7.0
49.433266094 50.7289924621582 8.0
31.7576049 26.9154090881348 9.0
93.026929153 89.0005340576172 10.0
80.66219421 76.1218109130859 11.0
84.20306809 82.8613052368164 12.0
84.20306809 78.4818649291992 13.0
84.20306809 79.0971221923828 14.0
0 -1.97326350212097 15.0
0 -0.206106007099152 16.0
-9.8865157344e-13 0.0193739533424377 17.0
55.803230373 58.3192024230957 18.0
-55.803230393 -59.5762138366699 19.0
-55.803230371 -59.2066688537598 20.0
-55.803230384 -57.5769691467285 21.0
77.720026466 76.4954681396484 22.0
140.37203364 141.661087036133 23.0
105.15840454 109.751754760742 24.0
22.085270054 20.4219837188721 25.0
218.09206012 228.355743408203 26.0
218.09206012 226.287658691406 27.0
27.498065237 24.6006965637207 28.0
245.59012536 258.279693603516 29.0
245.59012536 257.355010986328 30.0
46.613954445 46.4101181030273 31.0
9.4041898947 8.53684139251709 32.0
21.916796072 18.5543308258057 33.0
21.916796072 23.1406879425049 34.0
21.916796072 20.3922443389893 35.0
10.100345406 7.27163028717041 36.0
8.7899596816 7.29723119735718 37.0
-1.0371448166e-13 1.12825620174408 38.0
-1.0371448153e-13 -2.0894181728363 39.0
-11.527318203 -9.6851224899292 40.0
-24.533526655 -25.4112186431885 41.0
5.3822841084 2.70980763435364 42.0
-32.309422838 -32.9041328430176 43.0
90.705438536 89.62353515625 44.0
90.70543855 91.6171951293945 45.0
6.6968893179 6.74580669403076 46.0
2.3987043784 2.82680153846741 47.0
165.35585943 164.871994018555 48.0
159.34975406 155.822738647461 49.0
0 3.12990713119507 50.0
81.190871013 76.6820297241211 51.0
93.026929163 85.2289428710938 52.0
84.2030681 81.6374588012695 53.0
84.2030681 80.8560943603516 54.0
84.2030681 80.1999130249023 55.0
0 1.39121532440186 56.0
77.720026476 79.9969863891602 57.0
218.09206013 223.366806030273 58.0
245.59012537 248.666473388672 59.0
21.916796082 20.858793258667 60.0
0 2.85544800758362 61.0
162.691484849725 158.741302490234 0.0
162.691484849725 161.358184814453 1.0
161.49122414 163.513137817383 2.0
156.70215441 168.851837158203 3.0
0 -0.450048744678497 4.0
-8.7976841104e-09 2.20378470420837 5.0
-69.7622129369589 -64.5264053344727 6.0
80.7799501148563 73.0988388061523 7.0
50.9440192178563 52.4340133666992 8.0
29.8359308968563 21.121395111084 9.0
91.2813247838563 86.7498779296875 10.0
78.7191235398563 74.9923324584961 11.0
82.4120558788563 83.9982528686523 12.0
82.4120558788563 80.8315658569336 13.0
82.4120558788563 81.392448425293 14.0
0 -0.616707861423492 15.0
0 -0.346517860889435 16.0
-8.368547484e-09 0.753006517887115 17.0
59.5168818108307 59.4330787658691 18.0
-59.5168818232845 -59.4487495422363 19.0
-59.5168818022845 -58.7132110595703 20.0
-59.5168818092845 -59.2624931335449 21.0
80.448872179 78.1264877319336 22.0
143.235451210831 141.671279907227 23.0
107.177747376038 110.28971862793 24.0
21.9195844870384 22.1619701385498 25.0
223.68432338945 223.083343505859 26.0
223.68432338945 221.898193359375 27.0
25.4718589214496 26.221794128418 28.0
249.15618230945 252.490997314453 29.0
249.15618230945 254.335754394531 30.0
46.988160683764 47.065372467041 31.0
9.3887762128 9.12158203125 32.0
20.9319903698768 18.3112354278564 33.0
20.9319903698768 19.6860103607178 34.0
20.9319903758815 21.305269241333 35.0
9.3896297608 8.9376277923584 36.0
8.1251510418 7.73391819000244 37.0
-9.436810703e-09 0.0641846656799316 38.0
-9.436810703e-09 -2.79352784156799 39.0
-11.4685039276422 -11.9934644699097 40.0
-24.4747206841303 -22.7480621337891 41.0
5.05244618161359 7.39779472351074 42.0
-32.0464420067136 -35.1409339904785 43.0
102.952566970714 95.9430084228516 44.0
102.952566960714 95.6218490600586 45.0
7.1179878574 7.83101749420166 46.0
2.5497763753 2.2245728969574 47.0
162.69148485 162.720642089844 48.0
156.70215442 158.841094970703 49.0
0 -0.440704643726349 50.0
80.779950115 77.9616546630859 51.0
91.281324784 89.9453506469727 52.0
82.412055878 81.404411315918 53.0
82.412055878 81.3535003662109 54.0
82.412055878 80.6638488769531 55.0
0 -0.917096197605133 56.0
80.44887218 80.6970443725586 57.0
223.68432339 220.511962890625 58.0
249.15618231 247.941421508789 59.0
20.931990376 22.6764907836914 60.0
0 2.38331913948059 61.0
135.32247141 157.941452026367 0.0
135.32247141 159.656768798828 1.0
134.09793078 163.993530273438 2.0
129.30886106 150.735626220703 3.0
0 1.89920878410339 4.0
-2.5833782583e-10 0.32428377866745 5.0
-70.786186413 -74.8748931884766 6.0
82.870709241 76.2684631347656 7.0
52.070600402 58.136589050293 8.0
30.800108836 18.562313079834 9.0
87.855568524 87.7768478393555 10.0
75.268909365 78.9088668823242 11.0
76.872778884 86.1163940429688 12.0
76.872778884 80.5132369995117 13.0
76.872778884 77.8519134521484 14.0
0 -2.60725736618042 15.0
0 -1.53382349014282 16.0
-3.3628181586e-09 2.30831623077393 17.0
51.879001735 58.3984489440918 18.0
-51.8790017591731 -58.730655670166 19.0
-51.8790017371731 -57.7160186767578 20.0
-51.8790017491731 -58.5861473083496 21.0
73.710401443 74.1280212402344 22.0
145.90703336 142.199462890625 23.0
108.1605972 101.446365356445 24.0
24.140446051 22.8856220245361 25.0
219.61743482 222.981704711914 26.0
219.61743482 223.322601318359 27.0
29.565792709 25.3659591674805 28.0
249.18322753 248.451721191406 29.0
249.18322753 249.054840087891 30.0
46.477397147 47.5719375610352 31.0
9.3858057749 8.07505702972412 32.0
21.831399699 16.4623908996582 33.0
21.831399699 19.7391948699951 34.0
21.831399696 21.6652793884277 35.0
8.8627961862 4.20557641983032 36.0
7.5830971296 7.75957012176514 37.0
-6.3574432908e-09 -1.6573634147644 38.0
-6.357443292e-09 0.347078323364258 39.0
-11.434613608 -16.2675075531006 40.0
-23.698630809 -24.7757053375244 41.0
4.8243761285 1.20134460926056 42.0
-30.931534443 -29.8115844726562 43.0
90.822141487 91.9838485717773 44.0
90.822141506 91.0496978759766 45.0
7.3259602944 5.21848344802856 46.0
2.478270858 1.86680221557617 47.0
135.32247141 167.244094848633 48.0
129.30886107 149.930450439453 49.0
0 0.0441086888313293 50.0
82.870709244 75.6241455078125 51.0
87.855568527 89.9070434570312 52.0
76.872778887 81.6404113769531 53.0
76.872778887 81.6830520629883 54.0
76.872778887 81.1380157470703 55.0
0 0.922041714191437 56.0
73.710401451 79.7252883911133 57.0
219.61743482 222.180755615234 58.0
249.18322753 247.911590576172 59.0
21.831399704 19.8793277740479 60.0
0 2.67515969276428 61.0
137.17687742 143.006408691406 0.0
137.17687742 143.034271240234 1.0
135.86355256 143.512969970703 2.0
131.0744823 135.724792480469 3.0
0 -2.07063269615173 4.0
0 0.341697096824646 5.0
-59.42861149 -59.2288932800293 6.0
69.705106679 70.1772079467773 7.0
65.43029471 62.4528312683105 8.0
4.2748119615 8.48783111572266 9.0
78.669161727 80.2065200805664 10.0
65.229460649 71.2585144042969 11.0
77.500548395 80.8047180175781 12.0
77.500548395 77.9000701904297 13.0
77.500548395 76.9198608398438 14.0
0 -0.64198762178421 15.0
0 0.28728449344635 16.0
0 -3.15485000610352 17.0
59.754945382 58.5015258789062 18.0
-59.754945394 -59.4383087158203 19.0
-59.754945372 -60.0649642944336 20.0
-59.754945384 -61.8801422119141 21.0
80.656571382 77.8813858032227 22.0
140.87903669 140.394561767578 23.0
106.41719675 107.357681274414 24.0
22.90053999 20.4083709716797 25.0
221.53560808 223.594223022461 26.0
221.53560808 221.987747192383 27.0
27.282645659 25.3551406860352 28.0
248.81825375 253.574676513672 29.0
248.81825375 252.408157348633 30.0
47.033013387 47.2299919128418 31.0
9.3131031344 9.18030452728271 32.0
20.901625993 20.4819431304932 33.0
20.901625993 21.1129970550537 34.0
20.901625993 20.8195400238037 35.0
9.3083381265 8.28715133666992 36.0
8.044767894 8.38281917572021 37.0
-2.5262790395e-09 -0.804296553134918 38.0
-2.5262785966e-09 0.113371789455414 39.0
-12.387792857 -14.5510091781616 40.0
-23.81592281 -27.5234241485596 41.0
4.9278776077 3.5162615776062 42.0
-31.108913619 -34.7394485473633 43.0
92.052766479 91.3064651489258 44.0
92.05276649 90.2343292236328 45.0
6.7273095346 3.00315618515015 46.0
2.6568330431 1.83220052719116 47.0
137.17687743 136.683746337891 48.0
131.07448231 127.475601196289 49.0
0 3.55342674255371 50.0
69.705106686 70.9866027832031 51.0
78.669161734 84.2529830932617 52.0
77.500548402 81.0479965209961 53.0
77.500548402 80.9212951660156 54.0
77.500548402 79.4541549682617 55.0
0 -0.0822644829750061 56.0
80.656571391 80.3950500488281 57.0
221.53560809 218.26057434082 58.0
248.81825375 246.469757080078 59.0
20.901626002 19.9398994445801 60.0
0 2.89454770088196 61.0
178.370645713079 159.638366699219 0.0
178.370645713079 165.346588134766 1.0
177.14637536 159.157974243164 2.0
172.35730562 169.398330688477 3.0
0 2.02182245254517 4.0
-7.44681162600001e-09 -0.449241459369659 5.0
-63.251475457858 -64.9727783203125 6.0
73.6456327600031 72.9261779785156 7.0
48.8004476010031 49.6164627075195 8.0
24.8451851590031 19.3036155700684 9.0
91.1510476730031 83.268913269043 10.0
78.5903685230031 72.7832794189453 11.0
83.1256306080031 78.8830642700195 12.0
83.1256306080031 76.7704162597656 13.0
83.1256306080031 78.706413269043 14.0
0 -3.56161689758301 15.0
0 -0.0837370157241821 16.0
-6.62518859999999e-10 -0.1464883685112 17.0
61.656162343729 59.1536598205566 18.0
-61.6561623340047 -59.7981147766113 19.0
-61.6561623140047 -59.0483207702637 20.0
-61.6561623100047 -59.7948341369629 21.0
83.2493587436495 79.5201644897461 22.0
144.109550646729 142.303085327148 23.0
110.780121750911 112.327346801758 24.0
23.5935922129112 20.0302391052246 25.0
227.358909396157 225.864028930664 26.0
227.358909396157 223.158874511719 27.0
25.4977216721575 25.0483779907227 28.0
252.856631066157 256.609436035156 29.0
252.856631066157 257.754852294922 30.0
46.4315661062145 46.9652290344238 31.0
9.31180160085552 8.58171272277832 32.0
21.5931964335741 22.6594200134277 33.0
21.5931964335741 23.2540760040283 34.0
21.5931964520011 19.9214248657227 35.0
9.80006039320884 9.52949142456055 36.0
8.50121525230884 7.59352397918701 37.0
-8.24921289999999e-09 -0.393796741962433 38.0
-8.24921289e-09 -3.26352262496948 39.0
-12.0501610420023 -13.1106815338135 40.0
-24.2807843940185 -27.2108783721924 41.0
5.32958613473594 4.14575242996216 42.0
-32.1715172913359 -32.8319625854492 43.0
98.097557992336 95.9837493896484 44.0
98.0975579243359 94.8631057739258 45.0
6.7589974881 8.08051776885986 46.0
2.4503790524 2.74138641357422 47.0
178.37064571 164.45263671875 48.0
172.35730563 161.536865234375 49.0
0 -1.27973175048828 50.0
73.64563276 75.4217987060547 51.0
91.151047673 86.9432907104492 52.0
83.125630608 80.9474945068359 53.0
83.125630608 79.590690612793 54.0
83.125630608 79.0849227905273 55.0
0 1.70550811290741 56.0
83.249358744 79.7615661621094 57.0
227.3589094 220.720001220703 58.0
252.85663107 249.135391235352 59.0
21.593196452 19.1285552978516 60.0
0 1.35459554195404 61.0
162.691484849725 158.741317749023 0.0
162.691484849725 161.357986450195 1.0
161.49122414 163.510559082031 2.0
156.70215441 168.844543457031 3.0
0 -0.449474394321442 4.0
-8.7976841104e-09 2.20417022705078 5.0
-69.7622129369589 -64.5261077880859 6.0
80.7799501148563 73.0988464355469 7.0
50.9440192178563 52.4340133666992 8.0
29.8359308968563 21.121395111084 9.0
91.2813247838563 86.7498779296875 10.0
78.7191235398563 74.9923324584961 11.0
82.4120558788563 83.9982528686523 12.0
82.4120558788563 80.8315658569336 13.0
82.4120558788563 81.392448425293 14.0
0 -0.616707861423492 15.0
0 -0.346517860889435 16.0
-8.368547484e-09 0.753006517887115 17.0
59.5168818108307 59.4330787658691 18.0
-59.5168818232845 -59.4487495422363 19.0
-59.5168818022845 -58.7132110595703 20.0
-59.5168818092845 -59.2624931335449 21.0
80.448872179 78.1264877319336 22.0
143.235451210831 141.671279907227 23.0
107.177747376038 110.28971862793 24.0
21.9195844870384 22.1619701385498 25.0
223.68432338945 223.083343505859 26.0
223.68432338945 221.898193359375 27.0
25.4718589214496 26.221794128418 28.0
249.15618230945 252.490997314453 29.0
249.15618230945 254.335754394531 30.0
46.988160683764 47.065372467041 31.0
9.3887762128 9.12158203125 32.0
20.9319903698768 18.3112354278564 33.0
20.9319903698768 19.6860103607178 34.0
20.9319903758815 21.305269241333 35.0
9.3896297608 8.9376277923584 36.0
8.1251510418 7.73391819000244 37.0
-9.436810703e-09 0.0641846656799316 38.0
-9.436810703e-09 -2.79352784156799 39.0
-11.4685039276422 -11.9934644699097 40.0
-24.4747206841303 -22.7480621337891 41.0
5.05244618161359 7.39779472351074 42.0
-32.0464420067136 -35.1409339904785 43.0
102.952566970714 95.9430084228516 44.0
102.952566960714 95.6218490600586 45.0
7.1179878574 7.83101749420166 46.0
2.5497763753 2.2245728969574 47.0
162.69148485 162.720596313477 48.0
156.70215442 158.833404541016 49.0
0 -0.439934909343719 50.0
80.779950115 77.9617309570312 51.0
91.281324784 89.9453506469727 52.0
82.412055878 81.404411315918 53.0
82.412055878 81.3535003662109 54.0
82.412055878 80.6638488769531 55.0
0 -0.917096197605133 56.0
80.44887218 80.6970443725586 57.0
223.68432339 220.511962890625 58.0
249.15618231 247.941421508789 59.0
20.931990376 22.6764907836914 60.0
0 2.38331913948059 61.0
196.44745048 161.026382446289 0.0
196.44745048 121.36873626709 1.0
194.75855662 82.9459915161133 2.0
193.6935201 57.6554222106934 3.0
-1.2719505245e-15 70.1268997192383 4.0
-4.845081145e-14 12.5025796890259 5.0
-61.921323706 -55.3584175109863 6.0
69.297993011 74.3241271972656 7.0
43.903305881 45.5045471191406 8.0
25.394687119 35.2793464660645 9.0
16.197087863 79.5253067016602 10.0
3.2702574565 71.1091537475586 11.0
0 82.7157669067383 12.0
0 64.14794921875 13.0
0 86.5423889160156 14.0
32.963121211 9.58790111541748 15.0
32.963121211 -4.84317255020142 16.0
-9.0970458399e-14 -0.601776301860809 17.0
47.121605951 58.5559539794922 18.0
-47.121605971 -57.2429237365723 19.0
-47.893861081 -62.9575233459473 20.0
-47.96831849 -55.4232063293457 21.0
72.606878809 27.1720657348633 22.0
138.6846334 51.998046875 23.0
103.08384621 94.5372543334961 24.0
21.642934366 22.7051219940186 25.0
211.29151222 213.18635559082 26.0
211.29151222 214.921569824219 27.0
23.57364519 20.8295478820801 28.0
234.86515742 258.338226318359 29.0
234.86515742 241.550003051758 30.0
44.468193981 42.6367721557617 31.0
6.0953469923 11.7058229446411 32.0
24.638560308 8.17380905151367 33.0
24.638560308 22.1183738708496 34.0
24.638560308 20.3425693511963 35.0
11.909245053 17.6514148712158 36.0
10.469357299 8.52056312561035 37.0
-1.2947191131e-13 -19.9473972320557 38.0
-1.294719113e-13 5.34555101394653 39.0
-13.010000242 -24.9991359710693 40.0
-27.451222786 -24.9937610626221 41.0
4.2526355847 1.01073729991913 42.0
-34.282292324 -27.6708030700684 43.0
75.391918829 87.0721054077148 44.0
75.391918842 93.8142471313477 45.0
5.3549116103 12.1208848953247 46.0
0.87806140739 2.6255738735199 47.0
196.44745049 146.721420288086 48.0
193.69352011 66.231819152832 49.0
0 0.109617531299591 50.0
69.297993021 76.6744995117188 51.0
16.197087873 81.40625 52.0
0 78.1261749267578 53.0
0 72.7663040161133 54.0
0 77.5921249389648 55.0
0 3.59107780456543 56.0
72.606878819 71.5482711791992 57.0
211.29151223 218.468841552734 58.0
234.86515743 242.727233886719 59.0
24.638560318 15.8048362731934 60.0
0 1.50724959373474 61.0
169.95506867 159.836486816406 0.0
169.95506867 163.548721313477 1.0
168.72780352 163.31852722168 2.0
163.9387338 166.680969238281 3.0
0 2.01036381721497 4.0
0 1.42036485671997 5.0
-63.709395325 -63.6932411193848 6.0
74.743747757 77.1054000854492 7.0
49.647684154 49.5442733764648 8.0
25.096063593 26.6819458007812 9.0
84.157955328 87.353515625 10.0
71.605337576 76.6259536743164 11.0
78.207192888 83.3800659179688 12.0
78.207192888 80.5856781005859 13.0
78.207192888 82.2062301635742 14.0
0 -0.739538133144379 15.0
0 -0.71247786283493 16.0
0 0.410266280174255 17.0
60.385664502 58.7569732666016 18.0
-60.385664522 -58.9403114318848 19.0
-60.3856645 -60.4838218688965 20.0
-60.385664513 -59.7166442871094 21.0
82.394674631 78.1226043701172 22.0
143.28204402 142.811309814453 23.0
108.50546632 111.617126464844 24.0
22.110926411 20.4024524688721 25.0
225.67671866 226.390014648438 26.0
225.67671866 224.798767089844 27.0
25.431926674 23.7103424072266 28.0
251.10864535 253.351379394531 29.0
251.10864535 252.705520629883 30.0
46.857329038 45.6667861938477 31.0
9.2811530467 8.17576789855957 32.0
22.009010108 19.075662612915 33.0
22.009010108 14.6046390533447 34.0
22.009010108 20.3169269561768 35.0
9.2275939014 6.7463755607605 36.0
7.9828812639 7.65429449081421 37.0
0 0.422620058059692 38.0
0 -3.99123430252075 39.0
-11.23936405 -11.9397468566895 40.0
-24.235955029 -28.8415069580078 41.0
5.2833205941 3.23311519622803 42.0
-32.064989144 -31.1926746368408 43.0
102.57472947 93.1945648193359 44.0
102.57472948 95.3900909423828 45.0
6.6805689588 6.98043918609619 46.0
2.4038177864 2.14020824432373 47.0
169.95506868 168.514068603516 48.0
163.93873381 160.48469543457 49.0
0 -3.10845541954041 50.0
74.743747767 76.6830749511719 51.0
84.157955338 89.7051239013672 52.0
78.207192898 81.4601593017578 53.0
78.207192898 78.2401123046875 54.0
78.207192898 79.8687438964844 55.0
0 3.52368354797363 56.0
82.394674641 80.6725921630859 57.0
225.67671867 222.741989135742 58.0
251.10864536 246.864013671875 59.0
22.009010118 23.1880702972412 60.0
0 1.69914627075195 61.0
146.413045322204 166.690139770508 0.0
146.413045322204 160.436889648438 1.0
144.773467393454 161.384231567383 2.0
139.98439756 0.624688327312469 3.0
-1.2568409789e-09 99.158821105957 4.0
-1.77351387000001e-09 20.8468151092529 5.0
-56.9722833562822 -50.4580307006836 6.0
68.6198729624878 77.4388885498047 7.0
66.7328595914878 16.7994785308838 8.0
1.88701337088781 6.38286304473877 9.0
87.2830921044878 89.2781448364258 10.0
76.3900787794878 75.8393783569336 11.0
96.1159214424878 87.4044494628906 12.0
96.1159214424878 87.4379196166992 13.0
96.1159214424878 79.5546493530273 14.0
2.4590033544e-08 -9.45604515075684 15.0
2.4590033544e-08 -3.10487985610962 16.0
5.435670129e-08 -2.53948020935059 17.0
54.5495839151751 60.5534477233887 18.0
-54.5495837337317 -59.0884590148926 19.0
-54.5495837247317 -25.4327087402344 20.0
-54.5495836257317 -54.9111099243164 21.0
75.8599347985372 77.9451141357422 22.0
137.118925257175 135.965148925781 23.0
106.247605848969 101.686943054199 24.0
19.9303705659689 26.1957454681396 25.0
212.978860054409 229.46923828125 26.0
212.978860054409 212.296310424805 27.0
28.510512802409 32.8345794677734 28.0
241.489372854409 230.102600097656 29.0
241.489372854409 229.459884643555 30.0
47.3259415178799 43.2185287475586 31.0
9.19466691800937 11.1725931167603 32.0
21.3103511728467 13.1430253982544 33.0
21.3103511728467 17.4046192169189 34.0
21.3103513023751 29.0187473297119 35.0
8.44057521743984 8.11145210266113 36.0
7.31493717563984 4.30904626846313 37.0
2.55471940000003e-09 -12.2092704772949 38.0
2.55471929999996e-09 2.38560128211975 39.0
-12.8416562308658 -18.6047325134277 40.0
-25.5290108845192 -22.6226215362549 41.0
5.91082045553982 5.70356893539429 42.0
-33.9532358977398 -34.1223182678223 43.0
66.9929202024133 97.1109848022461 44.0
66.9929205057398 97.4752883911133 45.0
8.13352858812163 9.02347087860107 46.0
2.89333904202163 2.00357580184937 47.0
146.41304532 166.685104370117 48.0
139.98439756 107.169799804688 49.0
0 4.72535037994385 50.0
68.619872963 79.2415237426758 51.0
87.283092105 77.5881118774414 52.0
96.115921442 81.4774398803711 53.0
96.115921442 75.9232177734375 54.0
96.115921442 80.3342742919922 55.0
5.4356701286e-08 -1.5921094417572 56.0
75.859934798 74.3314056396484 57.0
212.97886005 219.402954101562 58.0
241.48937285 226.226547241211 59.0
21.310351302 27.7552127838135 60.0
2.5547193311e-09 0.693950593471527 61.0
175.03132233 158.287841796875 0.0
175.03132233 163.403137207031 1.0
173.70929243 169.534255981445 2.0
168.9202227 165.995712280273 3.0
0 -1.591548204422 4.0
0 5.76281261444092 5.0
-70.249358355 -65.1135559082031 6.0
80.997655979 81.6181793212891 7.0
46.545681657 51.6586151123047 8.0
34.451974312 25.9110984802246 9.0
91.083058335 87.1587753295898 10.0
78.230204244 77.7970657348633 11.0
85.233927928 80.5519332885742 12.0
85.233927928 77.5763702392578 13.0
85.233927928 79.1738586425781 14.0
0 -0.685663878917694 15.0
0 -0.384305775165558 16.0
0 -1.33428955078125 17.0
59.599931626 59.2771682739258 18.0
-59.599931645 -59.5249061584473 19.0
-59.599931624 -60.104076385498 20.0
-59.599931636 -60.0260124206543 21.0
81.656906927 79.0201110839844 22.0
148.20433721 146.422348022461 23.0
113.69784259 110.860275268555 24.0
21.838918626 19.8066654205322 25.0
229.86124415 225.107147216797 26.0
229.86124415 229.852890014648 27.0
26.65902897 22.5583000183105 28.0
256.52027313 250.500671386719 29.0
256.52027313 254.218566894531 30.0
46.710776323 48.2051849365234 31.0
9.3012166774 9.2275390625 32.0
22.056975281 19.2189521789551 33.0
22.056975281 16.0446491241455 34.0
22.056975281 20.1404724121094 35.0
9.5720260072 9.52567577362061 36.0
8.3532525578 6.97866296768188 37.0
0 0.818329215049744 38.0
0 -1.21554493904114 39.0
-11.774688445 -12.8988943099976 40.0
-25.110778222 -26.1925735473633 41.0
5.1717526737 1.89213395118713 42.0
-32.946588345 -32.30712890625 43.0
96.438770959 92.0685806274414 44.0
96.438770972 93.1123352050781 45.0
6.8874570582 5.94141721725464 46.0
2.5828755496 1.82582378387451 47.0
175.03132234 162.490921020508 48.0
168.92022271 166.738174438477 49.0
0 -0.706529796123505 50.0
80.997655989 75.8721313476562 51.0
91.083058345 91.3995132446289 52.0
85.233927938 81.3833236694336 53.0
85.233927938 80.4926376342773 54.0
85.233927938 80.29345703125 55.0
0 0.709381520748138 56.0
81.656906937 78.8676223754883 57.0
229.86124416 220.742721557617 58.0
256.52027314 250.582168579102 59.0
22.056975291 22.6891574859619 60.0
0 0.630337297916412 61.0
153.00233927 157.949127197266 0.0
153.00233927 147.725051879883 1.0
151.67351874 161.569412231445 2.0
146.88444902 147.436065673828 3.0
0 0.810172617435455 4.0
-7.1965235375e-10 1.68558967113495 5.0
-54.424229624 -56.0901489257812 6.0
63.44270118 74.5617828369141 7.0
47.959658386 54.8575897216797 8.0
15.483042792 24.5070972442627 9.0
95.954945739 80.1057586669922 10.0
82.514787316 74.9596252441406 11.0
85.792509116 79.3929061889648 12.0
85.792509116 80.9749755859375 13.0
85.792509116 81.6438522338867 14.0
0 1.92555260658264 15.0
0 0.052378237247467 16.0
-4.4831104982e-09 3.1993350982666 17.0
56.572512365 59.4933700561523 18.0
-56.5725123885068 -58.3467254638672 19.0
-56.5725123665068 -58.4966468811035 20.0
-56.5725123785068 -60.1123542785645 21.0
77.169320136 76.3447723388672 22.0
150.9121983 145.891998291016 23.0
112.42481064 109.572807312012 24.0
24.695826524 22.7604827880859 25.0
228.08151845 226.870651245117 26.0
228.08151845 219.383544921875 27.0
28.912326713 28.8398513793945 28.0
256.99384516 255.081481933594 29.0
256.99384516 250.962020874023 30.0
46.900811149 46.8018608093262 31.0
9.2911278584 6.45509672164917 32.0
20.596807762 17.9376831054688 33.0
20.596807762 12.7218046188354 34.0
20.596807759 22.1236190795898 35.0
8.905169772 7.75627517700195 36.0
7.5478157531 7.50654315948486 37.0
-7.9529071376e-09 3.8303017616272 38.0
-7.9528710425e-09 2.06318616867065 39.0
-12.715978096 -5.993408203125 40.0
-24.772798476 -25.4389228820801 41.0
5.3998750468 2.4508204460144 42.0
-32.408430274 -30.0709743499756 43.0
95.25976099 91.5812301635742 44.0
95.259760998 88.5799026489258 45.0
6.7214250507 3.6342921257019 46.0
1.9026439252 2.64665031433105 47.0
153.00233928 153.346817016602 48.0
146.88444903 153.186111450195 49.0
0 1.30362462997437 50.0
63.442701181 75.4771728515625 51.0
95.954945741 81.7000885009766 52.0
85.792509118 81.5665893554688 53.0
85.792509118 81.8162841796875 54.0
85.792509118 80.7918014526367 55.0
0 -0.675220310688019 56.0
77.169320143 79.9252853393555 57.0
228.08151845 219.172256469727 58.0
256.99384516 246.190444946289 59.0
20.596807767 16.7493686676025 60.0
0 4.21874952316284 61.0
164.1053677 162.877426147461 0.0
164.1053677 166.316757202148 1.0
162.93102379 161.806884765625 2.0
158.14195407 152.596160888672 3.0
0 -3.99929618835449 4.0
0 2.39349436759949 5.0
-60.82281403 -66.3274230957031 6.0
71.755214483 75.7222747802734 7.0
56.430852422 51.3183403015137 8.0
15.324362051 22.8368282318115 9.0
87.69963633 86.8045806884766 10.0
74.768828951 75.8081588745117 11.0
79.602133025 81.0682907104492 12.0
79.602133025 77.8076324462891 13.0
79.602133025 78.8077774047852 14.0
0 -1.05350375175476 15.0
0 -0.24024111032486 16.0
0 -0.897970259189606 17.0
56.768719404 59.083854675293 18.0
-56.768719414 -59.4968338012695 19.0
-56.768719392 -59.8538703918457 20.0
-56.768719405 -59.3070640563965 21.0
77.85104794 76.3232803344727 22.0
145.32533065 142.850830078125 23.0
109.98151039 109.951110839844 24.0
22.420458015 20.44287109375 25.0
223.1763786 226.335510253906 26.0
223.1763786 226.017929077148 27.0
25.403324333 24.7086658477783 28.0
248.57970294 255.943176269531 29.0
248.57970294 250.351593017578 30.0
46.690974181 46.718147277832 31.0
9.3550523057 8.76643943786621 32.0
21.082328526 21.7175216674805 33.0
21.082328526 19.7561626434326 34.0
21.082328526 19.6611175537109 35.0
9.0987090848 6.8767147064209 36.0
7.8593622379 7.15653228759766 37.0
-1.5960770215e-11 0.112923443317413 38.0
-1.5960770215e-11 -0.860056817531586 39.0
-12.298075436 -17.30442237854 40.0
-25.384419706 -25.059684753418 41.0
5.3016297424 5.72132253646851 42.0
-33.267862288 -32.9192314147949 43.0
91.753319316 91.023811340332 44.0
91.753319329 91.3166122436523 45.0
6.9786044251 6.00684642791748 46.0
2.7195891394 2.57459235191345 47.0
164.10536771 165.59260559082 48.0
158.14195408 154.422500610352 49.0
0 2.17766189575195 50.0
71.755214493 75.9837875366211 51.0
87.69963634 89.7603759765625 52.0
79.602133035 80.509033203125 53.0
79.602133035 80.1979217529297 54.0
79.602133035 80.2428359985352 55.0
0 0.859796226024628 56.0
77.85104795 79.7022399902344 57.0
223.17637861 221.656234741211 58.0
248.57970295 248.039138793945 59.0
21.082328536 20.2112789154053 60.0
0 1.79017806053162 61.0
162.82061198 161.382034301758 0.0
162.82061198 164.884323120117 1.0
161.54557009 160.500946044922 2.0
156.75650037 153.650985717773 3.0
0 0.720955312252045 4.0
-2.5785874991e-14 -0.501177489757538 5.0
-63.036768998 -62.4407081604004 6.0
73.088212834 75.2378692626953 7.0
47.241213159 47.3818626403809 8.0
25.846999664 25.4585781097412 9.0
90.330234731 90.6795120239258 10.0
77.180411278 78.1233367919922 11.0
81.270169195 82.1066055297852 12.0
81.270169195 79.4502563476562 13.0
81.270169195 79.9354705810547 14.0
0 -0.226097404956818 15.0
0 -0.486578404903412 16.0
-1.0819178425e-14 -0.454637825489044 17.0
57.967662355 58.4910545349121 18.0
-57.967662375 -58.3711051940918 19.0
-57.967662353 -59.8825836181641 20.0
-57.967662365 -60.2147483825684 21.0
78.926268247 80.2490844726562 22.0
141.6194507 141.437225341797 23.0
108.39091133 109.270881652832 24.0
23.722175613 19.9356536865234 25.0
220.54571896 222.616622924805 26.0
220.54571896 219.701843261719 27.0
25.059071064 25.7856025695801 28.0
245.60479003 256.196990966797 29.0
245.60479003 255.814483642578 30.0
46.689936121 45.6534194946289 31.0
9.2833396275 9.15184879302979 32.0
20.958605872 20.040843963623 33.0
20.958605872 20.7526950836182 34.0
20.958605872 19.7976989746094 35.0
8.9453979392 7.01146411895752 36.0
7.6843433188 7.3772234916687 37.0
0 1.12191390991211 38.0
0 -4.29746437072754 39.0
-11.69159293 -11.8616256713867 40.0
-24.19870589 -24.2394371032715 41.0
5.3877026778 6.43013668060303 42.0
-32.254256478 -33.5113792419434 43.0
99.13152889 94.1791687011719 44.0
99.131528903 94.1954040527344 45.0
6.6311241228 5.84266328811646 46.0
2.4293925549 2.8503794670105 47.0
162.82061199 161.422775268555 48.0
156.75650038 153.571395874023 49.0
0 -1.52610635757446 50.0
73.088212844 77.4487075805664 51.0
90.330234741 89.3013458251953 52.0
81.270169205 81.2380981445312 53.0
81.270169205 80.8290100097656 54.0
81.270169205 79.9892578125 55.0
0 -0.119755566120148 56.0
78.926268257 80.382568359375 57.0
220.54571897 219.241226196289 58.0
245.60479004 248.428192138672 59.0
20.958605882 21.2847156524658 60.0
0 1.5748016834259 61.0
137.17687742 143.006423950195 0.0
137.17687742 143.034698486328 1.0
135.86355256 143.509613037109 2.0
131.0744823 135.718490600586 3.0
0 -2.06355714797974 4.0
0 0.341646194458008 5.0
-59.42861149 -59.2279930114746 6.0
69.705106679 70.1772155761719 7.0
65.43029471 62.4528312683105 8.0
4.2748119615 8.48783111572266 9.0
78.669161727 80.2065200805664 10.0
65.229460649 71.2585144042969 11.0
77.500548395 80.8047180175781 12.0
77.500548395 77.9000701904297 13.0
77.500548395 76.9198608398438 14.0
0 -0.64198762178421 15.0
0 0.28728449344635 16.0
0 -3.15485000610352 17.0
59.754945382 58.5015258789062 18.0
-59.754945394 -59.4383087158203 19.0
-59.754945372 -60.0649642944336 20.0
-59.754945384 -61.8801422119141 21.0
80.656571382 77.8813858032227 22.0
140.87903669 140.394561767578 23.0
106.41719675 107.357681274414 24.0
22.90053999 20.4083709716797 25.0
221.53560808 223.594223022461 26.0
221.53560808 221.987747192383 27.0
27.282645659 25.3551406860352 28.0
248.81825375 253.574676513672 29.0
248.81825375 252.408157348633 30.0
47.033013387 47.2299919128418 31.0
9.3131031344 9.18030452728271 32.0
20.901625993 20.4819431304932 33.0
20.901625993 21.1129970550537 34.0
20.901625993 20.8195400238037 35.0
9.3083381265 8.28715133666992 36.0
8.044767894 8.38281917572021 37.0
-2.5262790395e-09 -0.804296553134918 38.0
-2.5262785966e-09 0.113371789455414 39.0
-12.387792857 -14.5510091781616 40.0
-23.81592281 -27.5234241485596 41.0
4.9278776077 3.5162615776062 42.0
-31.108913619 -34.7394485473633 43.0
92.052766479 91.3064651489258 44.0
92.05276649 90.2343292236328 45.0
6.7273095346 3.00315618515015 46.0
2.6568330431 1.83220052719116 47.0
137.17687743 136.683990478516 48.0
131.07448231 127.465370178223 49.0
0 3.55485916137695 50.0
69.705106686 70.9866561889648 51.0
78.669161734 84.2529830932617 52.0
77.500548402 81.0479965209961 53.0
77.500548402 80.9212951660156 54.0
77.500548402 79.4541549682617 55.0
0 -0.0822644829750061 56.0
80.656571391 80.3950500488281 57.0
221.53560809 218.26057434082 58.0
248.81825375 246.469757080078 59.0
20.901626002 19.9398994445801 60.0
0 2.89454770088196 61.0
169.56868627 167.634338378906 0.0
169.56868627 160.667465209961 1.0
168.53855874 161.968734741211 2.0
163.74948902 145.746215820312 3.0
0 26.0863857269287 4.0
-4.575093401e-10 2.85132241249084 5.0
-62.448069677 -61.1956825256348 6.0
72.524243708 74.2431106567383 7.0
54.523489329 52.7561531066895 8.0
18.000754378 22.0112915039062 9.0
88.480360411 89.0852279663086 10.0
75.938738652 81.2172546386719 11.0
79.964873939 88.6846466064453 12.0
79.964873939 80.8084487915039 13.0
79.964873939 81.8111190795898 14.0
0 -4.21952438354492 15.0
0 -0.824238955974579 16.0
-1.1440727465e-09 -0.761316359043121 17.0
57.316007798 58.0772018432617 18.0
-57.3160078153628 -57.6231307983398 19.0
-57.3160077933628 -57.5851974487305 20.0
-57.3160078043628 -60.4259910583496 21.0
77.743773209 73.4272537231445 22.0
150.80736966 138.589111328125 23.0
117.23263361 120.32389831543 24.0
21.66227638 21.6275386810303 25.0
228.55114288 227.624206542969 26.0
228.55114288 220.881149291992 27.0
26.248086173 26.590892791748 28.0
254.79922905 233.35807800293 29.0
254.79922905 242.455047607422 30.0
46.580359594 43.958324432373 31.0
9.2976623422 8.49182415008545 32.0
20.427765409 17.3149490356445 33.0
20.427765409 12.0462398529053 34.0
20.427765406 23.5251789093018 35.0
9.147305784 6.28241872787476 36.0
8.0131455197 8.94467544555664 37.0
-8.0875027616e-09 -0.71451324224472 38.0
-8.0875033772e-09 -0.312146008014679 39.0
-11.823513895 -17.1159324645996 40.0
-24.92815741 -21.468469619751 41.0
5.0044951752 9.42869853973389 42.0
-32.175031196 -36.4483528137207 43.0
108.2289823 88.1074752807617 44.0
108.22898231 99.7563705444336 45.0
6.9583515649 8.14672756195068 46.0
2.364132446 1.73952054977417 47.0
169.56868628 164.874954223633 48.0
163.74948903 143.575866699219 49.0
0 -1.50608158111572 50.0
72.52424371 73.4009170532227 51.0
88.480360413 83.8144989013672 52.0
79.964873941 80.2979354858398 53.0
79.964873941 79.8395309448242 54.0
79.964873941 80.4273071289062 55.0
0 -0.441971957683563 56.0
77.743773217 80.8192520141602 57.0
228.55114288 221.399368286133 58.0
254.79922905 234.633880615234 59.0
20.427765413 21.0315093994141 60.0
0 3.12115263938904 61.0
181.565204996597 162.790969848633 0.0
181.565204996597 162.625717163086 1.0
180.43108149 162.142288208008 2.0
175.13749171 169.81103515625 3.0
0 -2.50072407722473 4.0
-4.347292007e-09 6.25275135040283 5.0
-68.5567689924099 -62.5234794616699 6.0
79.0817276034346 70.5057678222656 7.0
47.0158940014346 51.5044059753418 8.0
32.0658336014346 23.5325775146484 9.0
88.2568061374346 87.8085784912109 10.0
75.8124349274346 76.4428024291992 11.0
81.1804499634346 79.9860610961914 12.0
81.1804499634346 79.3312225341797 13.0
81.1804499634346 79.9496231079102 14.0
0 1.53486680984497 15.0
0 0.305219948291779 16.0
-1.006927278e-09 -0.522368252277374 17.0
58.3082037158734 59.2912292480469 18.0
-58.3082037201519 -59.6605606079102 19.0
-58.3082036991519 -59.4362983703613 20.0
-58.3082037051519 -58.0326766967773 21.0
79.8490874350364 77.3367309570312 22.0
142.694388967873 141.158798217773 23.0
106.991749992342 109.713310241699 24.0
22.9694009013418 21.1483364105225 25.0
222.543476403194 227.865936279297 26.0
222.543476403194 224.611557006836 27.0
28.0841392681937 26.5120792388916 28.0
250.627615673194 253.945846557617 29.0
250.627615673194 254.325210571289 30.0
46.6658112511848 45.3654289245605 31.0
9.3772359394 8.69370365142822 32.0
21.5408837292296 20.8130168914795 33.0
21.5408837292296 19.2078380584717 34.0
21.5408837350119 20.8727874755859 35.0
9.4327396673 6.04313468933105 36.0
8.1335852511 7.8212513923645 37.0
-8.990213113e-09 5.26758193969727 38.0
-8.990213113e-09 -0.0696725845336914 39.0
-12.2649766745759 -12.4621295928955 40.0
-25.1436170683735 -25.6628704071045 41.0
5.38967249569548 5.46975374221802 42.0
-33.1726034461955 -32.6316070556641 43.0
104.039120846195 98.0142669677734 44.0
104.039120826195 95.1773529052734 45.0
6.5359472316 4.93746757507324 46.0
2.36301427 2.7395179271698 47.0
181.565205 165.20964050293 48.0
175.13749172 164.483932495117 49.0
0 -0.572933495044708 50.0
79.081727603 77.2921981811523 51.0
88.256806137 90.8616409301758 52.0
81.180449964 81.3178634643555 53.0
81.180449964 80.2224426269531 54.0
81.180449964 80.1489105224609 55.0
0 0.146731913089752 56.0
79.849087435 79.8956680297852 57.0
222.5434764 220.964721679688 58.0
250.62761567 251.006988525391 59.0
21.540883736 22.1544666290283 60.0
0 2.6007559299469 61.0
173.206505309386 174.489318847656 0.0
173.206505309386 175.342147827148 1.0
172.12348614 167.030654907227 2.0
167.3344164 170.333312988281 3.0
0 3.43476128578186 4.0
-8.185434373e-09 -1.69502019882202 5.0
-74.852188806077 -64.7896728515625 6.0
84.8337591802694 69.1440658569336 7.0
59.7253570032694 55.046558380127 8.0
25.1084021762694 15.1080713272095 9.0
88.3420909152694 92.6110534667969 10.0
74.5724466022694 77.3556747436523 11.0
85.6210436372694 79.3623046875 12.0
85.6210436372694 82.9876022338867 13.0
85.6210436372694 81.7024078369141 14.0
2.50386292706e-09 1.63728451728821 15.0
0 0.67412656545639 16.0
1.04786730999999e-09 5.27228879928589 17.0
56.8391166767833 58.4946022033691 18.0
-56.8391166614041 -57.5986366271973 19.0
-56.8391166424041 -59.8900680541992 20.0
-56.8391166344041 -58.9126930236816 21.0
77.415187717235 71.0409317016602 22.0
143.130858913783 144.010894775391 23.0
108.118771874729 107.596138000488 24.0
21.9337686117291 20.7878856658936 25.0
220.546046628772 219.969757080078 26.0
220.546046628772 224.169525146484 27.0
28.1661767117724 26.6949501037598 28.0
248.712223338772 252.976791381836 29.0
248.712223338772 254.49772644043 30.0
46.5313248703018 45.7875747680664 31.0
9.27595566630184 9.49917984008789 32.0
20.5760710832309 16.4184055328369 33.0
20.5760710832309 21.0432605743408 34.0
20.5760711058171 19.8119792938232 35.0
9.32625065169575 10.5892295837402 36.0
8.00994083879575 7.83387327194214 37.0
-7.89117753000001e-09 -0.602889716625214 38.0
-7.89117752999998e-09 -3.00918531417847 39.0
-10.7884473862021 -14.7103567123413 40.0
-24.9346037519505 -29.6362495422363 41.0
5.23280407305784 3.61065196990967 42.0
-33.0200855135578 -32.5323219299316 43.0
88.7604245195578 84.3785934448242 44.0
88.7604243342566 84.101318359375 45.0
7.4993306027 8.98408699035645 46.0
2.4928089588 2.79538583755493 47.0
173.20650531 174.387329101562 48.0
167.33441641 180.439910888672 49.0
0 -0.986160576343536 50.0
84.83375918 79.1847457885742 51.0
88.342090915 87.8935089111328 52.0
85.621043637 79.4213943481445 53.0
85.621043637 81.681266784668 54.0
85.621043637 80.7965393066406 55.0
1.0478673047e-09 -0.257964432239532 56.0
77.415187718 80.5299911499023 57.0
220.54604663 220.434020996094 58.0
248.71222334 246.147369384766 59.0
20.576071106 17.1954975128174 60.0
0 0.725447237491608 61.0
175.429917329716 170.020202636719 0.0
175.429917329716 170.527786254883 1.0
174.16769483 166.570159912109 2.0
169.3786251 178.914855957031 3.0
0 1.2425332069397 4.0
-8.418845503e-09 2.92017936706543 5.0
-62.2980219930499 -65.1210098266602 6.0
74.3239689911748 74.1341171264648 7.0
50.6417265201747 44.8937034606934 8.0
23.6822424711748 21.9911708831787 9.0
91.6111885521748 88.8447570800781 10.0
78.6172409881748 76.1004333496094 11.0
81.4730670541748 84.6351699829102 12.0
81.4730670541748 79.9136276245117 13.0
81.4730670541748 76.9628067016602 14.0
0 -0.562500774860382 15.0
0 -0.125568509101868 16.0
-6.782706642e-09 1.70587372779846 17.0
58.6449134939135 58.1755638122559 18.0
-58.6449134982621 -60.4924545288086 19.0
-58.6449134772621 -59.8865547180176 20.0
-58.6449134802621 -58.637092590332 21.0
80.3591213961283 80.398567199707 22.0
147.928799131914 141.174102783203 23.0
114.717387142392 109.117500305176 24.0
23.9461144483919 21.6827220916748 25.0
228.287920529432 217.549682617188 26.0
228.287920529432 222.059753417969 27.0
27.4524862024324 26.062520980835 28.0
255.740406729432 253.678497314453 29.0
255.740406729432 257.662719726562 30.0
46.8957171096625 46.3721961975098 31.0
9.28635448610878 8.91452980041504 32.0
21.7142079161498 19.9010200500488 33.0
21.7142079161498 15.2899169921875 34.0
21.7142079279223 20.8412971496582 35.0
9.52342697370499 6.09104299545288 36.0
8.25922209630499 7.82858324050903 37.0
-8.885359522e-09 -3.51889657974243 38.0
-8.885359523e-09 -1.01354289054871 39.0
-11.9842642716311 -13.1648797988892 40.0
-25.2585714014956 -25.7219734191895 41.0
4.98448242641229 4.23767328262329 42.0
-32.9644901658123 -33.8026657104492 43.0
90.8236849968123 90.5956192016602 44.0
90.8236849063536 93.3806762695312 45.0
6.8791030828 5.40291404724121 46.0
2.4424148336 2.75354194641113 47.0
175.42991733 172.703338623047 48.0
169.37862511 168.553176879883 49.0
0 3.15641236305237 50.0
74.323968992 77.0097045898438 51.0
91.611188552 85.7279663085938 52.0
81.473067054 81.2490005493164 53.0
81.473067054 81.5131607055664 54.0
81.473067054 80.2349700927734 55.0
0 1.886479139328 56.0
80.359121396 78.9532623291016 57.0
228.28792053 215.42839050293 58.0
255.74040673 251.176620483398 59.0
21.714207928 22.9901733398438 60.0
0 1.40290951728821 61.0
149.575081040798 164.287887573242 0.0
149.575081040798 168.477142333984 1.0
148.2970609 156.977462768555 2.0
143.50799116 141.509552001953 3.0
0 -0.161849558353424 4.0
-6.278144422e-09 2.63408946990967 5.0
-64.9491952166332 -60.83056640625 6.0
76.3333944707161 74.4209365844727 7.0
54.5569578517161 52.0205841064453 8.0
21.7764366197161 25.4498748779297 9.0
87.8096269887161 86.5691223144531 10.0
75.0648298417161 76.5296401977539 11.0
79.1772020137161 82.8260192871094 12.0
79.1772020137161 82.000244140625 13.0
79.1772020137161 80.9393615722656 14.0
0 -1.46080541610718 15.0
0 -0.433019459247589 16.0
-4.98813483799999e-09 -2.73008894920349 17.0
56.0483458444815 60.2046165466309 18.0
-56.0483458450742 -58.8705368041992 19.0
-56.0483458240742 -60.1780014038086 20.0
-56.0483458250742 -62.4484748840332 21.0
77.3854060499503 77.5117950439453 22.0
147.403488683482 141.558685302734 23.0
112.508847751852 108.550407409668 24.0
23.1237729518519 21.322868347168 25.0
224.788894731596 225.291564941406 26.0
224.788894731596 223.918426513672 27.0
27.4920300815955 24.0779151916504 28.0
252.280924811596 256.080017089844 29.0
252.280924811596 257.343048095703 30.0
46.68347949816 45.7027778625488 31.0
9.38265989387767 8.93407154083252 32.0
21.3370602248995 21.668888092041 33.0
21.3370602248995 20.8613796234131 34.0
21.3370602385918 19.597204208374 35.0
9.60962606176085 7.79098892211914 36.0
8.36987815066085 7.37017440795898 37.0
-8.8020604e-09 0.404836714267731 38.0
-8.8020604e-09 -5.29227828979492 39.0
-11.2629740300371 -13.4446496963501 40.0
-24.0016762276353 -25.1775455474854 41.0
4.98076211473011 6.62622976303101 42.0
-31.4308604949301 -35.2885589599609 43.0
97.7547038069301 96.2006149291992 44.0
97.7547037599301 95.015754699707 45.0
7.220988082 5.65399694442749 46.0
2.680581481 1.7913122177124 47.0
149.57508104 165.886566162109 48.0
143.50799117 138.679260253906 49.0
0 -0.469508349895477 50.0
76.333394471 77.486686706543 51.0
87.809626989 90.4419708251953 52.0
79.177202014 80.778678894043 53.0
79.177202014 81.5603637695312 54.0
79.177202014 80.1599273681641 55.0
0 1.60002899169922 56.0
77.38540605 79.4752883911133 57.0
224.78889473 220.980072021484 58.0
252.28092481 252.561645507812 59.0
21.337060239 19.0434303283691 60.0
0 0.95581191778183 61.0
165.344975426446 164.556365966797 0.0
165.344975426446 165.016357421875 1.0
164.17958462 157.830581665039 2.0
159.39051485 155.804092407227 3.0
0 -3.86797952651978 4.0
-5.70272672000002e-09 3.25805354118347 5.0
-70.3683249620972 -65.6893539428711 6.0
81.9690991498402 73.2571716308594 7.0
48.4256408028402 52.4498672485352 8.0
33.5434583468402 24.9267921447754 9.0
89.2551065218402 86.1276626586914 10.0
76.4114561648402 75.4891510009766 11.0
81.1999948708402 78.138313293457 12.0
81.1999948708402 78.8213729858398 13.0
81.1999948708402 81.5171661376953 14.0
4.806758797e-09 -2.1628565788269 15.0
4.80675879706e-09 -1.08408617973328 16.0
1.112953862e-08 -1.49414086341858 17.0
61.2370771524583 59.0944404602051 18.0
-61.2370770957603 -59.3224563598633 19.0
-61.2370770797603 -59.7922744750977 20.0
-61.2370770467603 -59.6026878356934 21.0
82.6406904060109 78.4824066162109 22.0
147.013210216458 140.697723388672 23.0
112.006341873073 109.83984375 24.0
23.3564199140729 20.2012023925781 25.0
229.653900622893 226.383178710938 26.0
229.653900622893 223.825271606445 27.0
25.2693702858925 27.0924339294434 28.0
254.923270902893 254.755355834961 29.0
254.923270902893 254.677398681641 30.0
46.8770054433321 47.190502166748 31.0
9.39616815611573 8.82416820526123 32.0
21.4036133592916 20.8851928710938 33.0
21.4036133592916 14.5832643508911 34.0
21.4036134112897 21.4851703643799 35.0
9.46210315022193 7.40823698043823 36.0
8.27101564102193 6.50053882598877 37.0
-5.38583630000002e-09 0.146860003471375 38.0
-5.38583629e-09 -0.825106203556061 39.0
-11.2549325753802 -14.7519407272339 40.0
-24.3414094862238 -22.1240577697754 41.0
4.87717685406683 8.3837718963623 42.0
-31.8643112828668 -35.1824531555176 43.0
96.6767357008668 93.7697982788086 44.0
96.6767354918669 94.9240570068359 45.0
7.09087648879423 7.34370422363281 46.0
2.48035146109423 1.37342596054077 47.0
165.34497543 166.326766967773 48.0
159.39051485 152.32275390625 49.0
0 0.427283644676208 50.0
81.96909915 75.5657196044922 51.0
89.255106522 90.3668212890625 52.0
81.199994871 80.8472061157227 53.0
81.199994871 79.3540954589844 54.0
81.199994871 79.7449417114258 55.0
1.1129538618e-08 2.5918595790863 56.0
82.640690406 79.0775451660156 57.0
229.65390062 219.342575073242 58.0
254.9232709 249.008651733398 59.0
21.403613411 22.755220413208 60.0
0 2.87257099151611 61.0
152.5421472 163.445922851562 0.0
152.5421472 163.231567382812 1.0
151.35012006 157.286361694336 2.0
146.56105034 144.633255004883 3.0
0 1.62437105178833 4.0
-2.393190869e-12 2.96719193458557 5.0
-65.175441535 -66.0746307373047 6.0
75.42587961 78.3929214477539 7.0
49.676377452 50.7633743286133 8.0
25.749502149 23.7489624023438 9.0
85.989314824 90.5766448974609 10.0
72.665770655 80.3772583007812 11.0
80.175607508 81.397102355957 12.0
80.175607508 76.7245712280273 13.0
80.175607508 78.8517684936523 14.0
0 -2.24130702018738 15.0
0 0.0961032509803772 16.0
0 1.78160238265991 17.0
58.999884248 58.2966690063477 18.0
-58.999884268 -58.109691619873 19.0
-58.999884246 -59.2192611694336 20.0
-58.999884258 -58.6987915039062 21.0
80.43728154 78.0234756469727 22.0
146.46760867 139.902862548828 23.0
109.93333726 107.119361877441 24.0
21.551378254 19.4526424407959 25.0
226.90489022 226.875228881836 26.0
226.90489022 222.210739135742 27.0
26.398711847 27.6090888977051 28.0
253.30360208 256.164093017578 29.0
253.30360208 257.736877441406 30.0
46.845502664 46.5041122436523 31.0
9.3684174114 8.82255935668945 32.0
21.437397272 19.5078029632568 33.0
21.437397272 22.9383144378662 34.0
21.437397272 21.5370788574219 35.0
9.6790135414 8.09219551086426 36.0
8.4049317396 7.85404014587402 37.0
-1.6423027206e-11 -0.224362313747406 38.0
-1.6422996578e-11 0.702457964420319 39.0
-12.131493626 -10.8593378067017 40.0
-25.215095077 -29.2422828674316 41.0
5.0353531699 5.52958393096924 42.0
-32.691311892 -31.7741985321045 43.0
96.677690753 91.8984069824219 44.0
96.677690766 95.2328414916992 45.0
6.7531904103 6.80513763427734 46.0
2.6972772927 2.76006889343262 47.0
152.54214721 165.6708984375 48.0
146.56105035 149.555252075195 49.0
0 -1.92121648788452 50.0
75.42587962 76.2786254882812 51.0
85.989314833 89.0621871948242 52.0
80.175607518 80.7182769775391 53.0
80.175607518 80.2890930175781 54.0
80.175607518 80.3862380981445 55.0
0 1.60376501083374 56.0
80.43728155 80.5320281982422 57.0
226.90489023 222.438812255859 58.0
253.30360209 249.627853393555 59.0
21.437397282 21.3337497711182 60.0
0 2.39399337768555 61.0
170.920297160303 166.685638427734 0.0
170.920297160303 167.304122924805 1.0
169.749259525554 163.235076904297 2.0
164.9601897 166.98210144043 3.0
-2.4386375784e-10 -1.52636671066284 4.0
-2.58116956999999e-09 6.11654472351074 5.0
-62.5929166170547 -59.934986114502 6.0
72.5753245371915 75.1639556884766 7.0
49.3732291351915 52.1425476074219 8.0
23.2020954021915 24.1264324188232 9.0
87.9311180511915 88.4296112060547 10.0
74.8476426501915 74.5186767578125 11.0
80.7697746601915 78.3004913330078 12.0
80.7697746601915 78.6733627319336 13.0
80.7697746601915 79.3158950805664 14.0
1.475045861e-08 -1.12527132034302 15.0
1.475045861e-08 -0.562447130680084 16.0
4.618759346e-08 -0.486694991588593 17.0
56.5950425131044 58.1897354125977 18.0
-56.5950423567872 -58.544490814209 19.0
-56.5950423457872 -59.9885406494141 20.0
-56.5950422617872 -60.4796752929688 21.0
78.3264028957544 79.3713226318359 22.0
147.468700342104 142.953002929688 23.0
112.744539647631 110.467620849609 24.0
22.7279826946306 21.0113563537598 25.0
225.795103240606 229.199996948242 26.0
225.795103240606 223.962295532227 27.0
25.9413919436055 25.8728485107422 28.0
251.736495180606 255.313552856445 29.0
251.736495180606 256.075531005859 30.0
46.7667877222111 45.5988121032715 31.0
9.36831773705519 8.82804489135742 32.0
21.7313606351641 20.178689956665 33.0
21.7313606351641 21.6315898895264 34.0
21.7313607481005 20.6277122497559 35.0
9.33062134401624 9.15672779083252 36.0
8.05194773261624 7.55928134918213 37.0
9.6254839999983e-10 -0.996068060398102 38.0
9.62548400000041e-10 -2.2338924407959 39.0
-11.3711413493936 -13.619797706604 40.0
-24.6638257047245 -23.9235973358154 41.0
5.02176524385958 5.76927852630615 42.0
-32.2553188706596 -33.1297721862793 43.0
101.22122398266 98.5569458007812 44.0
101.22122351266 95.5587768554688 45.0
7.26434316821298 6.7813401222229 46.0
2.68594643401299 2.77741265296936 47.0
170.92029716 165.963745117188 48.0
164.96018971 163.515502929688 49.0
0 1.69944429397583 50.0
72.575324537 76.9591217041016 51.0
87.931118051 89.1653213500977 52.0
80.76977466 81.1744155883789 53.0
80.76977466 79.5469284057617 54.0
80.76977466 80.1421127319336 55.0
4.6187593455e-08 1.86801338195801 56.0
78.326402896 79.9541854858398 57.0
225.79510324 221.647201538086 58.0
251.73649518 250.920043945312 59.0
21.731360748 20.1066856384277 60.0
9.6254841021e-10 1.20868265628815 61.0
161.464323483046 163.383636474609 0.0
161.464323483046 152.833740234375 1.0
160.19873735 157.675628662109 2.0
155.40966762 159.423950195312 3.0
0 -2.52564096450806 4.0
-8.782604249e-09 6.43689632415771 5.0
-53.1017804526513 -57.8126068115234 6.0
65.3489096532794 69.3166961669922 7.0
52.9975025992794 50.601936340332 8.0
12.3514070542794 27.7344303131104 9.0
85.5884879572794 86.1074600219727 10.0
73.8637784712794 72.0203552246094 11.0
75.7731636902794 80.6948699951172 12.0
75.7731636902794 83.8072357177734 13.0
75.7731636902794 78.4548492431641 14.0
0 -5.08563232421875 15.0
0 -1.04890012741089 16.0
-5.436396189e-09 -5.80834102630615 17.0
52.5459279188993 58.3730163574219 18.0
-52.5459279519191 -58.6178588867188 19.0
-52.5459279309191 -61.0719718933105 20.0
-52.5459279359191 -56.0626029968262 21.0
74.4799720193728 74.7979507446289 22.0
140.252255884516 144.303253173828 23.0
107.953168718145 109.778038024902 24.0
20.087941068145 18.9377727508545 25.0
214.732227906092 223.229339599609 26.0
214.732227906092 226.594879150391 27.0
24.7549790840916 25.7979679107666 28.0
239.487206986092 254.562881469727 29.0
239.487206986092 256.168975830078 30.0
46.4226951943921 47.4500350952148 31.0
9.2972348456 9.45430088043213 32.0
21.9340440829538 17.9560146331787 33.0
21.9340440829538 21.3133792877197 34.0
21.9340440908894 20.988166809082 35.0
9.205955454 12.1505517959595 36.0
7.8661622202 6.61413812637329 37.0
-9.22286532e-09 -6.14931344985962 38.0
-9.22286532e-09 -2.11518311500549 39.0
-12.7618526829595 -8.93564510345459 40.0
-25.7379024782438 -27.8157844543457 41.0
4.63881470206616 7.281174659729 42.0
-32.9734878310662 -34.9303321838379 43.0
94.186957054052 93.4640884399414 44.0
94.1869571050662 96.5966644287109 45.0
7.3911972129 5.03342723846436 46.0
2.7138170514 1.33035409450531 47.0
161.46432349 173.436950683594 48.0
155.40966763 147.891708374023 49.0
0 -1.24117708206177 50.0
65.348909654 72.182746887207 51.0
85.588487958 82.1266555786133 52.0
75.773163691 79.9139709472656 53.0
75.773163691 76.5281066894531 54.0
75.773163691 80.4564208984375 55.0
0 -2.32747364044189 56.0
74.479972019 78.4493103027344 57.0
214.7322279 216.704940795898 58.0
239.48720699 246.550704956055 59.0
21.934044091 21.9634418487549 60.0
0 1.44568991661072 61.0
132.234641297998 161.766754150391 0.0
132.234641297998 178.146286010742 1.0
131.32885415 149.648025512695 2.0
126.53978441 149.419570922852 3.0
0 5.65979528427124 4.0
-7.843937472e-09 4.91043138504028 5.0
-51.7671612583896 -56.5757064819336 6.0
62.5966586288636 76.4231338500977 7.0
44.7514640488636 53.9211616516113 8.0
17.8451945798636 21.9192028045654 9.0
96.6107170158636 95.8165130615234 10.0
83.0245191438636 97.3708572387695 11.0
79.7814365588636 81.7599868774414 12.0
79.7814365588636 74.2051239013672 13.0
79.7814365588636 77.8654861450195 14.0
4.688145605e-10 0.936655700206757 15.0
0 0.934317350387573 16.0
2.96656128e-09 3.69208645820618 17.0
54.661430092471 58.6068267822266 18.0
-54.6614300747954 -58.5590057373047 19.0
-54.6614300557954 -58.4763984680176 20.0
-54.6614300477954 -49.6406898498535 21.0
75.9483076065848 75.4926376342773 22.0
134.442271692471 138.197906494141 23.0
100.853405483089 105.738525390625 24.0
20.8113000210888 21.5341262817383 25.0
210.390579295996 212.740173339844 26.0
210.390579295996 223.933471679688 27.0
24.9560137519964 27.0036220550537 28.0
235.346593055996 256.068786621094 29.0
235.346593055996 249.718383789062 30.0
47.307876572601 47.2309494018555 31.0
9.37346792223011 8.10899829864502 32.0
21.2868775581688 14.8815574645996 33.0
21.2868775581688 12.2124662399292 34.0
21.2868775809874 21.8078193664551 35.0
9.16618290946687 14.0629348754883 36.0
8.01659251826868 8.8354663848877 37.0
-7.55334186e-09 -0.655404150485992 38.0
-7.55334186e-09 0.75853031873703 39.0
-10.5445150943977 -16.3947162628174 40.0
-23.5905176327273 -16.11008644104 41.0
5.21585840375307 -7.61994457244873 42.0
-31.436208474553 -30.5152206420898 43.0
94.1171816025529 87.6061096191406 44.0
94.1171815165529 92.507453918457 45.0
6.44138761371154 5.01651096343994 46.0
2.5144971854 2.92206430435181 47.0
132.2346413 169.634658813477 48.0
126.53978442 140.335571289062 49.0
0 -5.28760385513306 50.0
62.596658629 73.3418350219727 51.0
96.610717016 81.0832061767578 52.0
79.781436559 82.4935455322266 53.0
79.781436559 79.4855194091797 54.0
79.781436559 78.9576644897461 55.0
2.9665612726e-09 1.56643199920654 56.0
75.948307606 75.4334411621094 57.0
210.3905793 217.055847167969 58.0
235.34659305 249.248657226562 59.0
21.286877581 23.6209506988525 60.0
0 2.61032724380493 61.0
147.982033057583 157.456436157227 0.0
147.982033057583 151.966354370117 1.0
146.808413813287 153.371154785156 2.0
142.01934397 150.885101318359 3.0
-1.55484310406e-09 -0.668092310428619 4.0
-1.55484309999997e-09 -2.84683012962341 5.0
-64.9693040339936 -64.3280944824219 6.0
77.6547829144777 74.1872634887695 7.0
61.2786926634778 55.4030647277832 8.0
16.3760902514777 7.20984888076782 9.0
97.5261141374777 89.7391510009766 10.0
83.6472874924777 79.6450042724609 11.0
85.2981411144777 83.8642730712891 12.0
85.2981411144777 76.4624557495117 13.0
85.2981411144777 81.5436782836914 14.0
2.3328021736e-08 -2.03496789932251 15.0
2.3328021735e-08 -1.6052393913269 16.0
5.464350236e-08 -2.72059202194214 17.0
56.2518018559467 59.0783615112305 18.0
-56.2518018797166 -57.0891418457031 19.0
-56.2518018707166 -59.2854194641113 20.0
-56.2518017707166 -58.115608215332 21.0
77.1875821073724 73.3798065185547 22.0
153.454771802806 145.008895874023 23.0
115.220596133508 115.750228881836 24.0
21.7005584165075 20.8819332122803 25.0
230.642353915167 228.381210327148 26.0
230.642353915167 229.708221435547 27.0
26.8883420391666 21.3647422790527 28.0
257.530695955167 253.23681640625 29.0
257.530695955167 255.336776733398 30.0
46.7385281475469 45.537841796875 31.0
9.28724497493877 8.4355525970459 32.0
20.9357803369809 22.6376686096191 33.0
20.9357803369809 19.0790691375732 34.0
20.935780467613 20.4264030456543 35.0
8.86054244871795 8.39873600006104 36.0
7.60121916461794 6.85336446762085 37.0
2.62465480000001e-09 1.1959273815155 38.0
2.62465480000001e-09 -1.17553234100342 39.0
-11.6034306368583 -12.5643882751465 40.0
-24.5599220844463 -26.7411251068115 41.0
4.75986482786468 1.92820739746094 42.0
-31.7734218810648 -31.6524333953857 43.0
89.4696777832616 86.4044494628906 44.0
89.4696781210649 91.9600372314453 45.0
7.05809588635643 4.443039894104 46.0
2.46335994485643 2.26466846466064 47.0
147.98203306 156.095489501953 48.0
142.01934397 153.122375488281 49.0
0 -3.73737192153931 50.0
77.654782914 76.0232696533203 51.0
97.526114137 92.0755157470703 52.0
85.298141114 81.5651702880859 53.0
85.298141114 80.2131118774414 54.0
85.298141114 79.9203643798828 55.0
5.4643502359e-08 3.07315587997437 56.0
77.187582108 79.4883346557617 57.0
230.64235391 222.583419799805 58.0
257.53069595 249.491134643555 59.0
20.935780468 19.6116352081299 60.0
2.6246548239e-09 3.64096260070801 61.0
166.25847831 160.948181152344 0.0
166.25847831 167.395050048828 1.0
165.00161414 162.02880859375 2.0
159.53697895 162.42204284668 3.0
0.053089762974 -2.64210915565491 4.0
0.053089762974 2.74183773994446 5.0
-65.592150936 -63.1159515380859 6.0
76.043146509 76.7854461669922 7.0
45.475155101 51.6805953979492 8.0
30.567991398 22.458776473999 9.0
83.583516136 87.1007919311523 10.0
71.170048692 75.5833358764648 11.0
78.92234323 85.8072204589844 12.0
78.92234323 80.1295166015625 13.0
78.92234323 80.4129791259766 14.0
0 0.0786507129669189 15.0
0 -0.492902100086212 16.0
0 -1.48141765594482 17.0
59.870871725 58.3560600280762 18.0
-59.870871745 -58.9556312561035 19.0
-59.870871723 -59.931037902832 20.0
-59.870871736 -57.7058982849121 21.0
81.202816233 76.0993728637695 22.0
147.39703203 142.99006652832 23.0
111.77401552 107.190093994141 24.0
22.875363859 20.6158657073975 25.0
228.59984828 221.934677124023 26.0
228.59984828 223.971496582031 27.0
24.961164621 25.2713661193848 28.0
253.56101291 257.136077880859 29.0
253.56101291 257.675354003906 30.0
46.90478904 45.8503646850586 31.0
9.3657890911 8.66880893707275 32.0
21.331944488 19.5110969543457 33.0
21.331944488 19.5986385345459 34.0
21.331944488 20.5691146850586 35.0
9.6326234192 7.01118135452271 36.0
8.3784323792 7.83701705932617 37.0
0 0.122348487377167 38.0
0 -3.48350954055786 39.0
-11.722531607 -13.6475610733032 40.0
-24.38995777 -26.4905090332031 41.0
5.175531526 2.59741854667664 42.0
-32.124582255 -32.4026985168457 43.0
102.79376804 95.1686325073242 44.0
102.79376805 95.0252838134766 45.0
7.0354296021 5.82985353469849 46.0
2.7117855098 2.16836524009705 47.0
166.25847832 163.708709716797 48.0
159.53697896 161.154357910156 49.0
0.053089772974 -1.27081179618835 50.0
76.043146519 77.7453994750977 51.0
83.583516146 88.7781829833984 52.0
78.92234324 81.4774475097656 53.0
78.92234324 81.9326934814453 54.0
78.92234324 79.5724716186523 55.0
0 1.82720327377319 56.0
81.202816243 79.3906784057617 57.0
228.59984829 220.030746459961 58.0
253.56101292 251.484039306641 59.0
21.331944498 19.093729019165 60.0
0 1.14338994026184 61.0
152.64297682 160.036682128906 0.0
152.64297682 166.12890625 1.0
151.30880137 160.873352050781 2.0
146.51973165 155.807327270508 3.0
0 1.68981385231018 4.0
-9.7575826113e-12 -0.649509370326996 5.0
-60.171253998 -64.0325241088867 6.0
71.909917528 73.189338684082 7.0
48.758386357 52.1474494934082 8.0
23.151531161 21.7900924682617 9.0
86.826849954 87.0083618164062 10.0
73.552810128 76.3201599121094 11.0
80.602393349 83.9651336669922 12.0
80.602393349 80.768196105957 13.0
80.602393349 81.0190582275391 14.0
0 -0.84912496805191 15.0
0 -0.124878108501434 16.0
-2.0478330647e-11 1.56776702404022 17.0
57.787708491 57.8376617431641 18.0
-57.787708511 -58.7761268615723 19.0
-57.787708489 -58.6178245544434 20.0
-57.787708501 -60.1469497680664 21.0
78.972809862 77.8163452148438 22.0
147.55108975 141.160171508789 23.0
114.00553285 109.683776855469 24.0
23.832830305 21.1275196075439 25.0
226.52389963 225.416320800781 26.0
226.52389963 224.487258911133 27.0
27.250651875 23.9672393798828 28.0
253.77455151 255.55924987793 29.0
253.77455151 255.986831665039 30.0
46.750312798 47.26416015625 31.0
9.3327220691 8.8815860748291 32.0
21.185101351 20.4296798706055 33.0
21.185101351 13.5004377365112 34.0
21.185101351 21.0598030090332 35.0
9.3633052591 7.69415855407715 36.0
8.0955714756 7.50305700302124 37.0
-3.2383671916e-11 -1.62402009963989 38.0
-3.2383671917e-11 -0.647268831729889 39.0
-11.546597478 -16.6982536315918 40.0
-24.700979458 -24.6478672027588 41.0
4.9555451378 6.78781080245972 42.0
-32.24502116 -33.9722900390625 43.0
95.439969452 92.4297256469727 44.0
95.439969466 95.0690383911133 45.0
6.998990478 6.08623933792114 46.0
2.5429915587 1.70197439193726 47.0
152.64297683 162.449768066406 48.0
146.51973166 150.270690917969 49.0
0 -1.06047630310059 50.0
71.909917538 74.576545715332 51.0
86.826849964 88.0537414550781 52.0
80.602393359 80.2794189453125 53.0
80.602393359 80.9284744262695 54.0
80.602393359 80.7080764770508 55.0
0 4.07350015640259 56.0
78.972809872 79.7845153808594 57.0
226.52389964 221.917831420898 58.0
253.77455152 249.932388305664 59.0
21.185101361 24.6385650634766 60.0
0 0.51330977678299 61.0
175.81360781 158.893325805664 0.0
175.81360781 160.706970214844 1.0
174.57809832 165.443878173828 2.0
169.78902859 172.824401855469 3.0
0 -1.84168672561646 4.0
-4.441045821e-10 -0.697107970714569 5.0
-70.302013814 -65.8326110839844 6.0
82.145105535 75.5527725219727 7.0
53.953306396 57.577709197998 8.0
28.191799137 26.7716903686523 9.0
87.648013961 85.9062881469727 10.0
75.393362562 76.3276138305664 11.0
84.077750112 83.7318572998047 12.0
84.077750112 78.7674789428711 13.0
84.077750112 79.0019226074219 14.0
0 0.278261780738831 15.0
0 0.284722089767456 16.0
-3.9667634274e-09 0.166833579540253 17.0
58.674600403 58.6971435546875 18.0
-58.6746004273699 -59.3303642272949 19.0
-58.6746004053699 -61.3872375488281 20.0
-58.6746004163699 -56.3875312805176 21.0
80.15239981 77.5188674926758 22.0
147.06179663 142.472213745117 23.0
112.66155193 109.822494506836 24.0
23.551182492 21.4972839355469 25.0
227.21419645 223.805694580078 26.0
227.21419645 224.638885498047 27.0
26.210031798 27.6818866729736 28.0
253.42422825 255.836517333984 29.0
253.42422825 255.309219360352 30.0
46.803232598 45.4166564941406 31.0
9.3668374492 8.8443660736084 32.0
21.477799398 20.0808639526367 33.0
21.477799398 22.6484050750732 34.0
21.477799395 20.6789817810059 35.0
9.2339436349 8.99952411651611 36.0
8.0319516275 7.82161045074463 37.0
-7.1073082031e-09 -0.440208613872528 38.0
-7.1073081315e-09 -1.45399522781372 39.0
-12.172273008 -11.7032737731934 40.0
-24.495375871 -22.2200088500977 41.0
5.3664536844 8.27184581756592 42.0
-32.511104303 -35.5179405212402 43.0
96.044139781 92.6450576782227 44.0
96.044139789 94.6486053466797 45.0
6.8385369303 5.64429235458374 46.0
2.4617582885 2.64976763725281 47.0
175.81360782 165.369049072266 48.0
169.7890286 165.885345458984 49.0
0 1.5202169418335 50.0
82.145105537 76.3880844116211 51.0
87.648013963 89.6451797485352 52.0
84.077750114 81.6148300170898 53.0
84.077750114 80.7368698120117 54.0
84.077750114 79.4060211181641 55.0
0 0.0851380228996277 56.0
80.152399818 78.9444122314453 57.0
227.21419645 220.812423706055 58.0
253.42422825 251.895477294922 59.0
21.477799403 19.8039436340332 60.0
0 1.67854022979736 61.0
152.59039391 166.497375488281 0.0
152.59039391 168.7978515625 1.0
151.39594385 158.664428710938 2.0
146.60687358 149.543151855469 3.0
0 0.909300744533539 4.0
0 -1.10960388183594 5.0
-58.513565211 -58.5779418945312 6.0
70.045552392 75.7007446289062 7.0
53.619430813 52.5391387939453 8.0
16.426121571 22.8161964416504 9.0
85.103068342 88.5177230834961 10.0
72.078306929 76.414924621582 11.0
79.84592605 84.4225921630859 12.0
79.84592605 80.5408554077148 13.0
79.84592605 80.5255661010742 14.0
0 0.176242351531982 15.0
0 0.16450560092926 16.0
0 -2.5583701133728 17.0
60.831894955 59.0853042602539 18.0
-60.831894967 -59.3462905883789 19.0
-60.831894945 -56.9251136779785 20.0
-60.831894957 -57.704761505127 21.0
82.173842737 77.7776260375977 22.0
146.44946739 141.921157836914 23.0
111.48461891 109.728927612305 24.0
22.952146044 20.7627468109131 25.0
228.62331014 221.021911621094 26.0
228.62331014 224.214797973633 27.0
26.392745468 27.4309024810791 28.0
255.01605562 255.971313476562 29.0
255.01605562 258.235321044922 30.0
46.853048662 45.9834365844727 31.0
9.3156416562 9.21270942687988 32.0
21.341947775 17.9251537322998 33.0
21.341947775 23.872278213501 34.0
21.341947774 20.0875301361084 35.0
9.1949672258 7.04130792617798 36.0
7.9492724968 7.48486709594727 37.0
-2.5024104086e-09 0.985273480415344 38.0
-2.5024109245e-09 -5.51380729675293 39.0
-11.63015152 -13.9507274627686 40.0
-24.440412558 -29.1632518768311 41.0
5.2892996002 5.00957250595093 42.0
-32.256858831 -32.4203720092773 43.0
104.94993807 97.1271286010742 44.0
104.94993808 94.9040756225586 45.0
6.8363927332 5.30925512313843 46.0
2.4111979429 2.50229525566101 47.0
152.59039392 165.352188110352 48.0
146.60687359 147.668182373047 49.0
0 -0.649416506290436 50.0
70.045552399 77.9643630981445 51.0
85.103068349 88.6096878051758 52.0
79.845926057 80.6585464477539 53.0
79.845926057 80.6819839477539 54.0
79.845926057 79.7787017822266 55.0
0 1.62002897262573 56.0
82.173842746 80.4518890380859 57.0
228.62331015 222.355941772461 58.0
255.01605562 249.946685791016 59.0
21.341947783 17.3087425231934 60.0
0 1.52439713478088 61.0
175.312286046877 168.967971801758 0.0
175.312286046877 160.23942565918 1.0
174.06990004 161.14533996582 2.0
169.2808303 171.259918212891 3.0
0 0.151144921779633 4.0
-6.11104077e-09 -2.99939703941345 5.0
-64.4514555966437 -59.8633422851562 6.0
75.0805405132529 75.9342803955078 7.0
47.2598739692529 54.2783317565918 8.0
27.8206665432529 25.0394229888916 9.0
89.6410683252529 86.729606628418 10.0
76.9350420992529 76.3420867919922 11.0
78.9241931172529 81.9688262939453 12.0
78.9241931172529 79.6647186279297 13.0
78.9241931172529 80.9153671264648 14.0
0 -1.46726846694946 15.0
0 -1.35265469551086 16.0
-6.33361835e-10 -1.98108553886414 17.0
59.4058047052288 59.2813987731934 18.0
-59.4058047028793 -59.7793273925781 19.0
-59.4058046818793 -59.1364250183105 20.0
-59.4058046838793 -58.1208915710449 21.0
81.2934113997981 77.3223495483398 22.0
145.095019229229 140.561813354492 23.0
111.915792069036 110.54133605957 24.0
23.202618791036 21.1786460876465 25.0
226.388430633753 231.157455444336 26.0
226.388430633753 223.825042724609 27.0
28.1270734247533 25.7983207702637 28.0
254.515504053753 256.004364013672 29.0
254.515504053753 255.06867980957 30.0
46.683776491229 46.5096740722656 31.0
9.2858371405437 8.80064678192139 32.0
21.887606715931 17.8186683654785 33.0
21.887606715931 12.2479686737061 34.0
21.8876067282251 20.4433555603027 35.0
9.35849906303128 6.45709609985352 36.0
8.01944723233128 7.34612894058228 37.0
-8.210452878e-09 -2.31416368484497 38.0
-8.21045287399999e-09 -3.22738599777222 39.0
-11.4815566869397 -10.1720218658447 40.0
-24.589932038993 -24.023265838623 41.0
4.87708110181073 5.67755031585693 42.0
-31.9939758597107 -34.1001472473145 43.0
92.5676851137107 89.748664855957 44.0
92.5676850737107 93.7062225341797 45.0
6.9609343856 4.96434354782104 46.0
2.5257442115 2.48931074142456 47.0
175.31228604 158.530792236328 48.0
169.28083031 166.05110168457 49.0
0 1.21560835838318 50.0
75.080540513 76.8617172241211 51.0
89.641068326 86.0822067260742 52.0
78.924193118 81.8149795532227 53.0
78.924193118 81.4288177490234 54.0
78.924193118 80.2298278808594 55.0
0 0.428759396076202 56.0
81.2934114 79.9225769042969 57.0
226.38843063 222.769989013672 58.0
254.51550406 248.287368774414 59.0
21.887606728 22.7359676361084 60.0
0 2.31894707679749 61.0
168.25705184 159.25309753418 0.0
168.25705184 165.181335449219 1.0
166.92505394 164.089904785156 2.0
162.13598422 163.729019165039 3.0
0 -9.85237789154053 4.0
-6.065795364e-10 0.55745142698288 5.0
-59.098451575 -47.5934524536133 6.0
70.781687622 79.5775756835938 7.0
51.948149445 57.6776084899902 8.0
18.833538175 19.1163120269775 9.0
88.042675719 86.2280502319336 10.0
73.997487828 74.1962203979492 11.0
75.868244934 82.5213165283203 12.0
75.868244934 75.388313293457 13.0
75.868244934 77.5743026733398 14.0
0 -1.71764492988586 15.0
0 -0.123538017272949 16.0
-4.3580546961e-09 1.72296726703644 17.0
58.35013101 59.0747451782227 18.0
-58.3501310347434 -57.3387413024902 19.0
-58.3501310127434 -62.5521621704102 20.0
-58.3501310237434 -58.8787002563477 21.0
79.509145486 80.8163833618164 22.0
143.31860988 141.008285522461 23.0
105.22087979 105.895240783691 24.0
23.461065379 20.1135196685791 25.0
222.82775538 222.622589111328 26.0
222.82775538 223.238647460938 27.0
23.635189801 25.3444747924805 28.0
246.46294518 259.759704589844 29.0
246.46294518 251.848266601562 30.0
47.271561863 46.7089157104492 31.0
9.3926228498 9.33640766143799 32.0
21.159014467 18.3085136413574 33.0
21.159014467 17.9865703582764 34.0
21.159014464 20.7084522247314 35.0
9.4933792594 8.3939905166626 36.0
8.3087865694 7.24747467041016 37.0
-7.7017154178e-09 -3.5925669670105 38.0
-7.7017170886e-09 -4.65700435638428 39.0
-12.73859098 -11.6193008422852 40.0
-25.459366641 -30.7911567687988 41.0
4.7013831644 -1.87295913696289 42.0
-32.614107298 -30.3409824371338 43.0
100.13284098 92.2936630249023 44.0
100.13284098 95.1883010864258 45.0
6.3773185425 4.92675685882568 46.0
2.3128723003 2.34215688705444 47.0
168.25705185 165.015441894531 48.0
162.13598423 154.453918457031 49.0
0 4.28343486785889 50.0
70.781687623 70.0048217773438 51.0
88.042675721 84.182487487793 52.0
75.868244936 81.6436157226562 53.0
75.868244936 77.2946472167969 54.0
75.868244936 79.792350769043 55.0
0 2.69205546379089 56.0
79.509145494 80.1337661743164 57.0
222.82775538 220.732162475586 58.0
246.46294518 248.306884765625 59.0
21.159014471 21.6225986480713 60.0
0 1.68316686153412 61.0
155.15525905 163.506729125977 0.0
155.15525905 149.892150878906 1.0
154.1244628 150.587326049805 2.0
149.33539307 150.406372070312 3.0
0 4.71155309677124 4.0
0 -4.95639801025391 5.0
-70.003426944 -75.3766021728516 6.0
82.744949211 80.581184387207 7.0
58.502875023 59.8817024230957 8.0
24.242074178 26.6077518463135 9.0
88.090973925 89.7896423339844 10.0
74.731383751 78.3926849365234 11.0
74.89268264 83.2743377685547 12.0
74.89268264 83.0963516235352 13.0
74.89268264 82.1998977661133 14.0
0 -3.54033899307251 15.0
0 -0.0879197120666504 16.0
0 1.59580886363983 17.0
54.037359791 55.6459121704102 18.0
-54.037359811 -59.389045715332 19.0
-54.037359789 -57.2445449829102 20.0
-54.037359802 -57.7747116088867 21.0
75.844202848 71.9645919799805 22.0
142.15625628 136.301284790039 23.0
108.62555201 113.205749511719 24.0
22.546866841 21.4203929901123 25.0
218.00045914 229.746124267578 26.0
218.00045914 220.142486572266 27.0
23.552534644 23.2411785125732 28.0
241.5529938 246.590530395508 29.0
241.5529938 253.643508911133 30.0
47.107978499 44.4222946166992 31.0
9.400686324 8.48046112060547 32.0
21.806843036 15.898042678833 33.0
21.806843036 13.7953996658325 34.0
21.806843036 18.4409675598145 35.0
9.1185282773 10.2734909057617 36.0
7.912128534 8.51003646850586 37.0
0 -2.67721462249756 38.0
0 0.238736093044281 39.0
-12.515977313 -16.7426204681396 40.0
-26.060022455 -27.9819736480713 41.0
5.5550243411 1.31806802749634 42.0
-34.299196395 -33.0994529724121 43.0
84.426811656 82.4141693115234 44.0
84.426811669 94.4920349121094 45.0
7.3191168417 3.26004767417908 46.0
2.2555979722 2.17806911468506 47.0
155.15525906 147.248291015625 48.0
149.33539308 137.331634521484 49.0
0 2.90389442443848 50.0
82.744949221 76.1030578613281 51.0
88.090973935 81.3979949951172 52.0
74.89268265 81.1176223754883 53.0
74.89268265 80.5804748535156 54.0
74.89268265 80.2389221191406 55.0
0 2.31572794914246 56.0
75.844202858 80.0537414550781 57.0
218.00045915 220.256118774414 58.0
241.55299381 245.757690429688 59.0
21.806843046 23.7047576904297 60.0
0 0.708923518657684 61.0
165.351029781406 157.878555297852 0.0
165.351029781406 162.381286621094 1.0
164.08770701 165.633880615234 2.0
158.62307174 155.306137084961 3.0
1.0832880939 2.47745299339294 4.0
1.08328810226321 8.18509292602539 5.0
-67.7555482903632 -61.1804313659668 6.0
78.5853610227712 75.5879516601562 7.0
49.4458210727712 51.3133583068848 8.0
29.1395399497712 23.7947292327881 9.0
89.1073980397712 83.4504318237305 10.0
76.7986695777712 74.4040222167969 11.0
81.8736442707712 80.9371109008789 12.0
81.8736442707712 78.3630065917969 13.0
81.8736442707712 80.6130523681641 14.0
3.7602291097e-09 -0.0713506937026978 15.0
1.6272515452e-09 -0.0682880282402039 16.0
9.55669260000003e-09 1.25840258598328 17.0
57.8610660729661 59.728572845459 18.0
-57.8610660316569 -60.1925582885742 19.0
-57.8610660136569 -57.8545188903809 20.0
-57.8610659926569 -58.7698059082031 21.0
80.0843260790733 77.0896530151367 22.0
147.797166008966 141.997604370117 23.0
112.755926366208 108.210830688477 24.0
23.3683622332077 21.67160987854 25.0
227.881492092813 225.019256591797 26.0
227.881492092813 226.250411987305 27.0
26.901197311813 24.5954608917236 28.0
254.782689402813 256.296020507812 29.0
254.782689402813 257.09765625 30.0
46.4226135160305 46.9759521484375 31.0
9.28952276629632 8.97862720489502 32.0
22.2232600870896 21.1368560791016 33.0
22.2232600870896 25.127046585083 34.0
22.2232601261625 21.121654510498 35.0
9.86013207565584 8.47025203704834 36.0
8.62443095355584 7.59949207305908 37.0
-6.21446807000003e-09 -2.18093037605286 38.0
-6.21446807000003e-09 -2.29818391799927 39.0
-11.6745441648284 -14.040864944458 40.0
-24.6294752288498 -25.0429153442383 41.0
5.21946170480936 6.99011611938477 42.0
-32.3541859148094 -34.2182998657227 43.0
96.8442895768094 95.5808563232422 44.0
96.8442894218094 92.9817581176758 45.0
7.12667677344732 6.46695756912231 46.0
2.47454266054732 2.56767535209656 47.0
165.35102978 162.666442871094 48.0
158.62307175 153.150115966797 49.0
1.0832881023 1.3274472951889 50.0
78.585361023 77.6819915771484 51.0
89.10739804 88.1610565185547 52.0
81.873644271 81.6924667358398 53.0
81.873644271 81.0575866699219 54.0
81.873644271 80.1906585693359 55.0
9.556692601e-09 1.85244917869568 56.0
80.084326079 79.4581756591797 57.0
227.88149209 222.508331298828 58.0
254.7826894 251.537719726562 59.0
22.223260126 18.91335105896 60.0
0 1.36672592163086 61.0
147.982033057583 157.456405639648 0.0
147.982033057583 151.964782714844 1.0
146.808413813287 153.35823059082 2.0
142.01934397 150.788269042969 3.0
-1.55484310406e-09 -0.656918823719025 4.0
-1.55484309999997e-09 -2.8427472114563 5.0
-64.9693040339936 -64.3239364624023 6.0
77.6547829144777 74.1871948242188 7.0
61.2786926634778 55.4030647277832 8.0
16.3760902514777 7.20984888076782 9.0
97.5261141374777 89.7391510009766 10.0
83.6472874924777 79.6450042724609 11.0
85.2981411144777 83.8642730712891 12.0
85.2981411144777 76.4624557495117 13.0
85.2981411144777 81.5436782836914 14.0
2.3328021736e-08 -2.03496789932251 15.0
2.3328021735e-08 -1.6052393913269 16.0
5.464350236e-08 -2.72059202194214 17.0
56.2518018559467 59.0783615112305 18.0
-56.2518018797166 -57.0891418457031 19.0
-56.2518018707166 -59.2854194641113 20.0
-56.2518017707166 -58.115608215332 21.0
77.1875821073724 73.3798065185547 22.0
153.454771802806 145.008895874023 23.0
115.220596133508 115.750228881836 24.0
21.7005584165075 20.8819332122803 25.0
230.642353915167 228.381210327148 26.0
230.642353915167 229.708221435547 27.0
26.8883420391666 21.3647422790527 28.0
257.530695955167 253.23681640625 29.0
257.530695955167 255.336776733398 30.0
46.7385281475469 45.537841796875 31.0
9.28724497493877 8.4355525970459 32.0
20.9357803369809 22.6376686096191 33.0
20.9357803369809 19.0790691375732 34.0
20.935780467613 20.4264030456543 35.0
8.86054244871795 8.39873600006104 36.0
7.60121916461794 6.85336446762085 37.0
2.62465480000001e-09 1.1959273815155 38.0
2.62465480000001e-09 -1.17553234100342 39.0
-11.6034306368583 -12.5643882751465 40.0
-24.5599220844463 -26.7411251068115 41.0
4.75986482786468 1.92820739746094 42.0
-31.7734218810648 -31.6524333953857 43.0
89.4696777832616 86.4044494628906 44.0
89.4696781210649 91.9600372314453 45.0
7.05809588635643 4.443039894104 46.0
2.46335994485643 2.26466846466064 47.0
147.98203306 156.096160888672 48.0
142.01934397 153.043350219727 49.0
0 -3.74360084533691 50.0
77.654782914 76.0235214233398 51.0
97.526114137 92.0755157470703 52.0
85.298141114 81.5651702880859 53.0
85.298141114 80.2131118774414 54.0
85.298141114 79.9203643798828 55.0
5.4643502359e-08 3.07315587997437 56.0
77.187582108 79.4883346557617 57.0
230.64235391 222.583419799805 58.0
257.53069595 249.491134643555 59.0
20.935780468 19.6116352081299 60.0
2.6246548239e-09 3.64096260070801 61.0
174.15303409 157.279235839844 0.0
174.15303409 160.435073852539 1.0
173.11283797 188.722763061523 2.0
168.32376794 179.008636474609 3.0
0 149.807388305664 4.0
0 14.2312202453613 5.0
-57.336008764 -32.736873626709 6.0
65.505991111 99.3455123901367 7.0
51.531367083 72.5224075317383 8.0
13.974624018 27.1742191314697 9.0
77.766966209 80.9453125 10.0
64.826682822 76.0166397094727 11.0
70.003233323 78.7140197753906 12.0
70.003233323 79.4134902954102 13.0
70.003233323 80.5695114135742 14.0
0 0.736366510391235 15.0
0 0.79651540517807 16.0
0 2.6685779094696 17.0
63.26119569 59.2786827087402 18.0
-63.2611957 -59.1451377868652 19.0
-63.261195678 -56.9179801940918 20.0
-63.261195691 -53.3066139221191 21.0
84.275518744 79.3789215087891 22.0
150.0799346 127.939582824707 23.0
114.97869869 117.523170471191 24.0
24.449923276 18.861665725708 25.0
234.35545335 213.315399169922 26.0
234.35545335 211.889831542969 27.0
21.176236414 31.2151126861572 28.0
255.53168978 238.876724243164 29.0
255.53168978 246.240539550781 30.0
46.559009779 48.7621726989746 31.0
9.3555729033 7.33979606628418 32.0
21.014323044 17.3204956054688 33.0
21.014323044 19.3162212371826 34.0
21.014323044 26.9709396362305 35.0
9.1868983362 7.13414335250854 36.0
7.7424378729 8.54882621765137 37.0
-1.4574267484e-11 -11.0120944976807 38.0
-1.4574267484e-11 -2.90527200698853 39.0
-11.854278764 -14.7010307312012 40.0
-24.21655474 -29.0470561981201 41.0
5.108271771 1.41790819168091 42.0
-31.861586012 -29.7483711242676 43.0
104.86114983 75.0838623046875 44.0
104.86114984 95.3180847167969 45.0
7.922600275 15.0135850906372 46.0
2.5174040466 3.46135139465332 47.0
174.1530341 165.3505859375 48.0
168.32376795 173.761642456055 49.0
0 77.0171127319336 50.0
65.505991121 75.6341094970703 51.0
77.766966219 89.1083526611328 52.0
70.003233333 81.8452758789062 53.0
70.003233333 81.9692764282227 54.0
70.003233333 81.6793365478516 55.0
0 -1.07615303993225 56.0
84.275518754 81.8201904296875 57.0
234.35545336 215.044586181641 58.0
255.53168979 248.505950927734 59.0
21.014323054 32.7886695861816 60.0
0 0.994669377803802 61.0
174.08295871 163.053634643555 0.0
174.08295871 165.361099243164 1.0
172.63640235 169.467575073242 2.0
167.84733209 167.425018310547 3.0
0 -3.52621650695801 4.0
0 9.03551864624023 5.0
-54.586883846 -63.7568321228027 6.0
65.879081909 73.8586578369141 7.0
39.584431245 47.9568519592285 8.0
26.294650658 26.5436630249023 9.0
81.552870769 85.8021392822266 10.0
67.490767806 74.5012969970703 11.0
78.738077292 79.5194702148438 12.0
78.738077292 79.4128265380859 13.0
78.738077292 76.2885284423828 14.0
0 -2.35027074813843 15.0
0 -0.296773731708527 16.0
0 -0.975928366184235 17.0
57.500481168 54.1100311279297 18.0
-57.500481181 -58.024772644043 19.0
-57.500481159 -58.8975868225098 20.0
-57.500481171 -56.6237869262695 21.0
78.283021925 81.8481979370117 22.0
150.91708409 151.231781005859 23.0
115.72638434 99.4927825927734 24.0
24.195817424 21.3602848052979 25.0
229.20010603 218.124771118164 26.0
229.20010603 220.34831237793 27.0
28.489441499 24.171012878418 28.0
257.68954753 237.497436523438 29.0
257.68954753 244.817749023438 30.0
47.137005914 45.6305198669434 31.0
9.4779848409 9.50099754333496 32.0
20.782540752 17.1448230743408 33.0
20.782540752 19.5017223358154 34.0
20.782540751 19.1953201293945 35.0
8.718446329 4.36214017868042 36.0
7.4807384003 7.19095325469971 37.0
-2.5379428361e-09 -1.10380864143372 38.0
-2.5379429728e-09 2.69718647003174 39.0
-12.029213738 -16.5282421112061 40.0
-24.841378433 -27.6668338775635 41.0
4.9820272262 4.31374454498291 42.0
-32.326548335 -33.5197792053223 43.0
81.086491092 84.6238479614258 44.0
81.086491099 77.5110244750977 45.0
6.7726610789 8.07354640960693 46.0
2.506191642 2.68825793266296 47.0
174.08295872 167.606964111328 48.0
167.8473321 161.639175415039 49.0
0 1.0118362903595 50.0
65.879081915 76.3284912109375 51.0
81.552870775 88.5764236450195 52.0
78.738077298 81.0380401611328 53.0
78.738077298 77.4749298095703 54.0
78.738077298 77.2695236206055 55.0
1.207292863e-10 0.700113236904144 56.0
78.283021934 79.5949096679688 57.0
229.20010603 222.513214111328 58.0
257.68954754 242.689498901367 59.0
20.782540759 21.5993499755859 60.0
0 2.57099866867065 61.0
137.17687742 143.006195068359 0.0
137.17687742 143.04296875 1.0
135.86355256 143.451507568359 2.0
131.0744823 135.611175537109 3.0
0 -1.94344711303711 4.0
0 0.34082692861557 5.0
-59.42861149 -59.2128486633301 6.0
69.705106679 70.1770553588867 7.0
65.43029471 62.4528312683105 8.0
4.2748119615 8.48783111572266 9.0
78.669161727 80.2065200805664 10.0
65.229460649 71.2585144042969 11.0
77.500548395 80.8047180175781 12.0
77.500548395 77.9000701904297 13.0
77.500548395 76.9198608398438 14.0
0 -0.64198762178421 15.0
0 0.28728449344635 16.0
0 -3.15485000610352 17.0
59.754945382 58.5015258789062 18.0
-59.754945394 -59.4383087158203 19.0
-59.754945372 -60.0649642944336 20.0
-59.754945384 -61.8801422119141 21.0
80.656571382 77.8813858032227 22.0
140.87903669 140.394561767578 23.0
106.41719675 107.357681274414 24.0
22.90053999 20.4083709716797 25.0
221.53560808 223.594223022461 26.0
221.53560808 221.987747192383 27.0
27.282645659 25.3551406860352 28.0
248.81825375 253.574676513672 29.0
248.81825375 252.408157348633 30.0
47.033013387 47.2299919128418 31.0
9.3131031344 9.18030452728271 32.0
20.901625993 20.4819431304932 33.0
20.901625993 21.1129970550537 34.0
20.901625993 20.8195400238037 35.0
9.3083381265 8.28715133666992 36.0
8.044767894 8.38281917572021 37.0
-2.5262790395e-09 -0.804296553134918 38.0
-2.5262785966e-09 0.113371789455414 39.0
-12.387792857 -14.5510091781616 40.0
-23.81592281 -27.5234241485596 41.0
4.9278776077 3.5162615776062 42.0
-31.108913619 -34.7394485473633 43.0
92.052766479 91.3064651489258 44.0
92.05276649 90.2343292236328 45.0
6.7273095346 3.00315618515015 46.0
2.6568330431 1.83220052719116 47.0
137.17687743 136.688690185547 48.0
131.07448231 127.290840148926 49.0
0 3.57963562011719 50.0
69.705106686 70.9875259399414 51.0
78.669161734 84.2529830932617 52.0
77.500548402 81.0479965209961 53.0
77.500548402 80.9212951660156 54.0
77.500548402 79.4541549682617 55.0
0 -0.0822644829750061 56.0
80.656571391 80.3950500488281 57.0
221.53560809 218.26057434082 58.0
248.81825375 246.469757080078 59.0
20.901626002 19.9398994445801 60.0
0 2.89454770088196 61.0
169.27010139 168.25715637207 0.0
169.27010139 162.539276123047 1.0
168.24091778 153.278991699219 2.0
163.45184806 154.595245361328 3.0
0 0.672469079494476 4.0
-1.0109135791e-11 12.9834766387939 5.0
-60.580282342 -49.4930763244629 6.0
73.161915127 78.5560531616211 7.0
57.171618964 52.1862983703613 8.0
15.990296153 25.2701835632324 9.0
85.605417207 77.7727661132812 10.0
72.318309279 71.4398193359375 11.0
85.75948681 74.6589202880859 12.0
85.75948681 82.6873016357422 13.0
85.75948681 72.3928985595703 14.0
0 -1.85192918777466 15.0
0 0.376586854457855 16.0
-1.7312707342e-11 0.0777302980422974 17.0
53.276537165 57.3253479003906 18.0
-53.276537185 -58.6821784973145 19.0
-53.276537163 -53.4306030273438 20.0
-53.276537175 -60.1398162841797 21.0
74.237173942 75.5637741088867 22.0
153.48066716 145.539642333984 23.0
116.99074502 109.855537414551 24.0
23.130346851 19.9287967681885 25.0
227.71784111 218.220672607422 26.0
227.71784111 213.761825561523 27.0
29.607142156 27.8440036773682 28.0
257.32498328 245.226104736328 29.0
257.32498328 247.878204345703 30.0
46.903289153 47.5566558837891 31.0
9.2969014174 6.83636569976807 32.0
20.960636757 18.6979789733887 33.0
20.960636757 22.3353538513184 34.0
20.960636757 17.944465637207 35.0
8.8454532874 9.34835338592529 36.0
7.4765488743 9.22372150421143 37.0
-3.5287760078e-11 2.05488443374634 38.0
-3.5287763168e-11 3.96840333938599 39.0
-11.080992706 -12.8041372299194 40.0
-24.932509321 -23.1585235595703 41.0
5.0863548628 -2.91485667228699 42.0
-32.733906015 -31.3253612518311 43.0
93.46051945 95.7999420166016 44.0
93.460519463 93.5414352416992 45.0
6.8313614437 2.56629920005798 46.0
2.5901924429 1.38604354858398 47.0
169.2701014 164.61865234375 48.0
163.45184807 164.866653442383 49.0
0 2.2274317741394 50.0
73.161915137 77.8220672607422 51.0
85.605417217 87.2107849121094 52.0
85.759486819 80.4459457397461 53.0
85.759486819 81.0049362182617 54.0
85.759486819 80.0435562133789 55.0
0 -1.23136425018311 56.0
74.237173952 76.26904296875 57.0
227.71784112 220.241577148438 58.0
257.32498329 243.803512573242 59.0
20.960636767 16.3977470397949 60.0
0 2.85984873771667 61.0
180.085945367258 163.177841186523 0.0
180.085945367258 178.182571411133 1.0
179.17267164 133.123687744141 2.0
173.70803644 29.4496059417725 3.0
3.696483199 40.0556297302246 4.0
3.69648320873752 -14.4661645889282 5.0
-62.9539317057375 -67.5780639648438 6.0
70.6641483355813 68.9834976196289 7.0
33.1718339055813 66.2462615966797 8.0
37.4923144295813 40.6153373718262 9.0
88.6246806395813 92.835319519043 10.0
77.7090824635813 79.0045852661133 11.0
82.5347931115813 87.9532928466797 12.0
82.5347931115813 89.300537109375 13.0
82.5347931115813 75.0655212402344 14.0
0 5.21911096572876 15.0
0 -1.41556692123413 16.0
-1.276183582e-09 5.00787734985352 17.0
50.1882113134989 38.3934631347656 18.0
-50.188211317872 -52.7964897155762 19.0
-50.188211296872 -57.2694892883301 20.0
-50.188211303872 -62.0642967224121 21.0
69.9279228623463 62.282527923584 22.0
140.372753808499 115.96817779541 23.0
103.744293703124 93.113410949707 24.0
20.9735713041236 25.6098766326904 25.0
210.300676674517 207.897445678711 26.0
210.300676674517 199.422073364258 27.0
22.8491056755169 26.1998558044434 28.0
233.149782344517 256.641418457031 29.0
233.149782344517 237.781646728516 30.0
46.9784938015188 44.980785369873 31.0
9.2336646853 10.2628221511841 32.0
19.7397115582126 21.3796939849854 33.0
19.7397115582126 12.8534870147705 34.0
19.7397115644073 16.3345794677734 35.0
8.3880556272 7.9157886505127 36.0
7.0794475313 6.95694828033447 37.0
-9.190539568e-09 12.8772373199463 38.0
-9.190539567e-09 1.207026720047 39.0
-12.254075936436 -9.71978378295898 40.0
-25.1957986049566 -22.1212978363037 41.0
5.91471750505109 7.69311380386353 42.0
-33.7096052321511 -37.3771820068359 43.0
89.5700401841511 98.3935775756836 44.0
89.5700401701511 95.5712738037109 45.0
6.4345157345 8.23186588287354 46.0
1.8174191808 2.67862939834595 47.0
180.08594536 169.829330444336 48.0
173.70803645 142.833618164062 49.0
3.6964832087 9.22891139984131 50.0
70.664148335 72.1745758056641 51.0
88.62468064 83.9836273193359 52.0
82.534793112 80.401237487793 53.0
82.534793112 81.3976898193359 54.0
82.534793112 81.452262878418 55.0
0 -4.67055130004883 56.0
69.927922862 74.8558120727539 57.0
210.30067667 217.961273193359 58.0
233.14978235 242.936737060547 59.0
19.739711565 6.29499578475952 60.0
0 5.59589004516602 61.0
170.083878316469 157.816253662109 0.0
170.083878316469 157.453735351562 1.0
168.788706857946 158.770797729492 2.0
163.99963705 158.431747436523 3.0
0 -2.61057186126709 4.0
-3.64718998999996e-09 8.1868953704834 5.0
-65.9445630618226 -62.1393775939941 6.0
77.4109841448792 78.0552749633789 7.0
48.2079889088791 50.7772903442383 8.0
29.2029952368793 24.4059295654297 9.0
94.7684301998792 88.8115615844727 10.0
82.0481931238792 76.8876266479492 11.0
85.4560780108792 83.5412139892578 12.0
85.4560780108792 82.4553375244141 13.0
85.4560780108792 82.5894241333008 14.0
-2.0210446233e-08 -1.72448968887329 15.0
-1.2827776325e-08 -1.04012513160706 16.0
3.671273728e-08 -1.79126453399658 17.0
58.3359280796611 59.4617156982422 18.0
-58.3359279518188 -59.5928955078125 19.0
-58.3359279398188 -59.0153999328613 20.0
-58.3359278708188 -59.1716766357422 21.0
79.8271822287511 78.0592422485352 22.0
140.369856244661 141.005126953125 23.0
106.168739595827 109.724090576172 24.0
22.3465316288267 19.7794818878174 25.0
220.197038472937 224.808822631836 26.0
220.197038472937 218.165908813477 27.0
26.0808301069373 28.6898517608643 28.0
246.277868572937 254.500579833984 29.0
246.277868572937 253.694686889648 30.0
46.6820190386153 46.8048667907715 31.0
9.40345932024295 8.98416328430176 32.0
21.491254357468 20.675594329834 33.0
21.491254357468 16.6573123931885 34.0
21.4912544526478 20.3899593353271 35.0
9.24633800707493 7.62591743469238 36.0
7.93106650837494 7.20929384231567 37.0
-8.81501959999916e-10 0.475449562072754 38.0
-8.81501960000021e-10 -5.17730569839478 39.0
-12.2413252834096 -12.8781337738037 40.0
-25.1440221070403 -29.4673118591309 41.0
5.15225554073331 4.28948593139648 42.0
-32.8317131282327 -32.128345489502 43.0
104.570770318232 97.9997863769531 44.0
104.570769928232 95.3408660888672 45.0
6.75303592774189 5.77723550796509 46.0
2.6278217995419 2.73974990844727 47.0
170.08387832 163.024688720703 48.0
163.99963706 157.68879699707 49.0
0 0.86491858959198 50.0
77.410984145 76.664192199707 51.0
94.768430199 89.9916381835938 52.0
85.456078011 81.0958709716797 53.0
85.456078011 80.9692840576172 54.0
85.456078011 80.1057052612305 55.0
3.6712737279e-08 0.654493868350983 56.0
79.827182229 81.3045654296875 57.0
220.19703847 219.398086547852 58.0
246.27786858 248.994201660156 59.0
21.491254452 24.4201393127441 60.0
0 0.813128650188446 61.0
147.982033057583 157.456268310547 0.0
147.982033057583 151.960220336914 1.0
146.808413813287 153.321578979492 2.0
142.01934397 150.524429321289 3.0
-1.55484310406e-09 -0.625551521778107 4.0
-1.55484309999997e-09 -2.83146190643311 5.0
-64.9693040339936 -64.3126525878906 6.0
77.6547829144777 74.1870574951172 7.0
61.2786926634778 55.4030647277832 8.0
16.3760902514777 7.20984888076782 9.0
97.5261141374777 89.7391510009766 10.0
83.6472874924777 79.6450042724609 11.0
85.2981411144777 83.8642730712891 12.0
85.2981411144777 76.4624557495117 13.0
85.2981411144777 81.5436782836914 14.0
2.3328021736e-08 -2.03496789932251 15.0
2.3328021735e-08 -1.6052393913269 16.0
5.464350236e-08 -2.72059202194214 17.0
56.2518018559467 59.0783615112305 18.0
-56.2518018797166 -57.0891418457031 19.0
-56.2518018707166 -59.2854194641113 20.0
-56.2518017707166 -58.115608215332 21.0
77.1875821073724 73.3798065185547 22.0
153.454771802806 145.008895874023 23.0
115.220596133508 115.750228881836 24.0
21.7005584165075 20.8819332122803 25.0
230.642353915167 228.381210327148 26.0
230.642353915167 229.708221435547 27.0
26.8883420391666 21.3647422790527 28.0
257.530695955167 253.23681640625 29.0
257.530695955167 255.336776733398 30.0
46.7385281475469 45.537841796875 31.0
9.28724497493877 8.4355525970459 32.0
20.9357803369809 22.6376686096191 33.0
20.9357803369809 19.0790691375732 34.0
20.935780467613 20.4264030456543 35.0
8.86054244871795 8.39873600006104 36.0
7.60121916461794 6.85336446762085 37.0
2.62465480000001e-09 1.1959273815155 38.0
2.62465480000001e-09 -1.17553234100342 39.0
-11.6034306368583 -12.5643882751465 40.0
-24.5599220844463 -26.7411251068115 41.0
4.75986482786468 1.92820739746094 42.0
-31.7734218810648 -31.6524333953857 43.0
89.4696777832616 86.4044494628906 44.0
89.4696781210649 91.9600372314453 45.0
7.05809588635643 4.443039894104 46.0
2.46335994485643 2.26466846466064 47.0
147.98203306 156.098007202148 48.0
142.01934397 152.82453918457 49.0
0 -3.76043605804443 50.0
77.654782914 76.0241928100586 51.0
97.526114137 92.0755157470703 52.0
85.298141114 81.5651702880859 53.0
85.298141114 80.2131118774414 54.0
85.298141114 79.9203643798828 55.0
5.4643502359e-08 3.07315587997437 56.0
77.187582108 79.4883346557617 57.0
230.64235391 222.583419799805 58.0
257.53069595 249.491134643555 59.0
20.935780468 19.6116352081299 60.0
2.6246548239e-09 3.64096260070801 61.0
137.17687742 143.006042480469 0.0
137.17687742 143.051696777344 1.0
135.86355256 143.390106201172 2.0
131.0744823 135.498626708984 3.0
0 -1.81451749801636 4.0
0 0.339907050132751 5.0
-59.42861149 -59.1968727111816 6.0
69.705106679 70.1768035888672 7.0
65.43029471 62.4528312683105 8.0
4.2748119615 8.48783111572266 9.0
78.669161727 80.2065200805664 10.0
65.229460649 71.2585144042969 11.0
77.500548395 80.8047180175781 12.0
77.500548395 77.9000701904297 13.0
77.500548395 76.9198608398438 14.0
0 -0.64198762178421 15.0
0 0.28728449344635 16.0
0 -3.15485000610352 17.0
59.754945382 58.5015258789062 18.0
-59.754945394 -59.4383087158203 19.0
-59.754945372 -60.0649642944336 20.0
-59.754945384 -61.8801422119141 21.0
80.656571382 77.8813858032227 22.0
140.87903669 140.394561767578 23.0
106.41719675 107.357681274414 24.0
22.90053999 20.4083709716797 25.0
221.53560808 223.594223022461 26.0
221.53560808 221.987747192383 27.0
27.282645659 25.3551406860352 28.0
248.81825375 253.574676513672 29.0
248.81825375 252.408157348633 30.0
47.033013387 47.2299919128418 31.0
9.3131031344 9.18030452728271 32.0
20.901625993 20.4819431304932 33.0
20.901625993 21.1129970550537 34.0
20.901625993 20.8195400238037 35.0
9.3083381265 8.28715133666992 36.0
8.044767894 8.38281917572021 37.0
-2.5262790395e-09 -0.804296553134918 38.0
-2.5262785966e-09 0.113371789455414 39.0
-12.387792857 -14.5510091781616 40.0
-23.81592281 -27.5234241485596 41.0
4.9278776077 3.5162615776062 42.0
-31.108913619 -34.7394485473633 43.0
92.052766479 91.3064651489258 44.0
92.05276649 90.2343292236328 45.0
6.7273095346 3.00315618515015 46.0
2.6568330431 1.83220052719116 47.0
137.17687743 136.693725585938 48.0
131.07448231 127.106163024902 49.0
0 3.60570907592773 50.0
69.705106686 70.9884185791016 51.0
78.669161734 84.2529830932617 52.0
77.500548402 81.0479965209961 53.0
77.500548402 80.9212951660156 54.0
77.500548402 79.4541549682617 55.0
0 -0.0822644829750061 56.0
80.656571391 80.3950500488281 57.0
221.53560809 218.26057434082 58.0
248.81825375 246.469757080078 59.0
20.901626002 19.9398994445801 60.0
0 2.89454770088196 61.0
162.85757008 168.048934936523 0.0
162.85757008 171.743576049805 1.0
161.59781867 160.362609863281 2.0
156.80874895 153.022537231445 3.0
0 -2.34056806564331 4.0
-2.092613687e-12 -2.06375503540039 5.0
-68.62114863 -67.0987167358398 6.0
79.219081044 72.4851608276367 7.0
56.750475929 52.9640197753906 8.0
22.468605105 25.1544570922852 9.0
84.115252491 86.1191101074219 10.0
70.73289663 73.9639282226562 11.0
80.512618814 80.9126052856445 12.0
80.512618814 80.520133972168 13.0
80.512618814 80.0003662109375 14.0
0 0.138406097888947 15.0
0 0.0945003628730774 16.0
-3.9005268992e-12 0.361804664134979 17.0
62.079443818 59.164306640625 18.0
-62.079443838 -60.022087097168 19.0
-62.079443816 -59.5993537902832 20.0
-62.079443829 -61.2314796447754 21.0
83.800853031 79.0967254638672 22.0
143.56518206 141.190338134766 23.0
111.08825492 110.990364074707 24.0
22.102595507 20.9055976867676 25.0
227.3660351 228.187683105469 26.0
227.3660351 220.825866699219 27.0
27.151956572 27.458568572998 28.0
254.51799168 255.203247070312 29.0
254.51799168 252.875152587891 30.0
46.445063337 45.4638214111328 31.0
9.2801016785 8.85960102081299 32.0
21.721409192 21.5966873168945 33.0
21.721409192 18.4746513366699 34.0
21.721409192 19.161958694458 35.0
9.1915613704 7.43347930908203 36.0
7.9803120314 7.94640445709229 37.0
-6.2509432785e-12 2.74730730056763 38.0
-6.2509432785e-12 -0.45326167345047 39.0
-11.568147451 -12.2604560852051 40.0
-23.95019328 -24.429651260376 41.0
5.2798150624 5.21954727172852 42.0
-31.870696421 -33.4950752258301 43.0
96.061378949 90.5321578979492 44.0
96.061378962 92.3324508666992 45.0
6.9557535068 7.72731447219849 46.0
2.4010232192 2.51868844032288 47.0
162.85757009 167.042449951172 48.0
156.80874896 153.837890625 49.0
0 0.986276388168335 50.0
79.219081054 74.1609573364258 51.0
84.115252501 86.2581100463867 52.0
80.512618824 80.298942565918 53.0
80.512618824 81.1014175415039 54.0
80.512618824 79.3614654541016 55.0
0 0.294139325618744 56.0
83.800853041 79.6522903442383 57.0
227.36603511 220.93196105957 58.0
254.51799169 249.336181640625 59.0
21.721409202 19.2826614379883 60.0
0 1.40553545951843 61.0
191.01697008 172.522674560547 0.0
191.01697008 152.431625366211 1.0
189.58354118 177.738906860352 2.0
184.79447114 186.149856567383 3.0
0 -22.4192276000977 4.0
-1.3327987697e-10 -3.04034113883972 5.0
-68.142043271 -71.3364944458008 6.0
77.809135988 71.7424011230469 7.0
62.996940906 50.9178199768066 8.0
14.812195073 16.4648342132568 9.0
90.280907649 87.7635345458984 10.0
76.556814964 77.198371887207 11.0
78.845418231 88.4357147216797 12.0
78.845418231 79.1166305541992 13.0
78.845418231 80.3032913208008 14.0
0 -1.13258099555969 15.0
0 0.176832318305969 16.0
0 3.57990503311157 17.0
58.277423471 58.944221496582 18.0
-58.277423482 -58.5585975646973 19.0
-58.27742346 -57.1651077270508 20.0
-58.277423472 -57.9702339172363 21.0
79.151883105 77.5008697509766 22.0
136.02159316 141.9638671875 23.0
103.65807874 108.820678710938 24.0
23.873551201 20.9192504882812 25.0
215.17347628 227.357177734375 26.0
215.17347628 222.578567504883 27.0
26.19532582 25.4187965393066 28.0
241.3688021 251.541870117188 29.0
241.3688021 245.628753662109 30.0
46.730193864 47.0511054992676 31.0
9.4322005595 7.00126457214355 32.0
20.874459624 21.3104610443115 33.0
20.874459624 15.2398481369019 34.0
20.874459624 18.9794425964355 35.0
9.4116623757 7.54806518554688 36.0
8.2549799295 7.53655529022217 37.0
-5.9335209366e-10 1.44418025016785 38.0
-5.9347829989e-10 0.911677241325378 39.0
-11.144049162 -19.7990436553955 40.0
-24.346216315 -24.6967487335205 41.0
5.1058567699 4.60519933700562 42.0
-31.86136454 -32.536190032959 43.0
90.516620084 89.0560455322266 44.0
90.516620096 89.8517532348633 45.0
7.3245523412 3.49604249000549 46.0
2.4288374168 2.14216756820679 47.0
191.01697009 159.136032104492 48.0
184.79447115 162.280151367188 49.0
0 2.44088864326477 50.0
77.809135998 73.3010940551758 51.0
90.280907659 81.3495941162109 52.0
78.84541824 81.5682754516602 53.0
78.84541824 80.6553192138672 54.0
78.84541824 80.0597381591797 55.0
0 -4.47466087341309 56.0
79.151883115 80.64892578125 57.0
215.17347628 222.893325805664 58.0
241.36880211 237.154342651367 59.0
20.874459634 17.8611640930176 60.0
0 -0.290941774845123 61.0
152.524167139285 161.566421508789 0.0
152.524167139285 166.802764892578 1.0
151.35996019 159.588134765625 2.0
146.57089042 149.170959472656 3.0
0 2.72538709640503 4.0
-5.13071198e-09 3.58576822280884 5.0
-59.53115247936 -58.3728713989258 6.0
70.2507919487599 73.059700012207 7.0
47.7617113627599 52.3051528930664 8.0
22.4890805857599 22.5464820861816 9.0
89.8720038897599 86.200553894043 10.0
76.7725129637599 74.7876510620117 11.0
84.1030283177599 83.5107803344727 12.0
84.1030283177599 81.3520660400391 13.0
84.1030283177599 81.384880065918 14.0
4.8380164139e-09 -1.74889588356018 15.0
2.8396886315e-09 -1.23449850082397 16.0
1.106254265e-08 -2.39599061012268 17.0
56.9242230005054 59.1188087463379 18.0
-56.9242229541398 -58.7474822998047 19.0
-56.9242229371398 -60.8387451171875 20.0
-56.9242229121398 -59.8430595397949 21.0
78.9448309813946 78.8030319213867 22.0
146.934885677505 142.000686645508 23.0
114.237803181882 114.043357849121 24.0
23.5512008828817 20.2917079925537 25.0
225.879716658569 230.664596557617 26.0
225.879716658569 224.718948364258 27.0
25.2255969145691 25.6801815032959 28.0
251.105313568569 253.560928344727 29.0
251.105313568569 254.459930419922 30.0
46.9430737789106 46.5697174072266 31.0
9.34221855687645 8.89533805847168 32.0
22.0206080690799 20.3247928619385 33.0
22.0206080690799 17.7073993682861 34.0
22.0206081119172 21.5819072723389 35.0
9.92452194042601 9.5117883682251 36.0
8.63421670852602 7.30539178848267 37.0
-5.88814005000002e-09 -5.77714157104492 38.0
-5.88814005000002e-09 -5.02064228057861 39.0
-12.0646002710699 -12.758918762207 40.0
-24.828896422166 -27.8423080444336 41.0
5.00332440529989 5.63134098052979 42.0
-32.4500067182999 -34.2524604797363 43.0
92.7446581152999 89.7777099609375 44.0
92.7446577858464 95.4271469116211 45.0
6.91667725158936 5.73279666900635 46.0
2.42208655028936 2.3217933177948 47.0
152.52416714 162.400604248047 48.0
146.57089043 142.361236572266 49.0
0 1.32731580734253 50.0
70.250791949 77.4059295654297 51.0
89.872003889 88.4692153930664 52.0
84.103028317 81.2791595458984 53.0
84.103028317 81.4317092895508 54.0
84.103028317 80.59765625 55.0
1.106254265e-08 0.0498611330986023 56.0
78.944830981 79.5637359619141 57.0
225.87971666 222.978210449219 58.0
251.10531357 249.396270751953 59.0
22.020608111 24.7376899719238 60.0
0 1.85762310028076 61.0
147.982033057583 157.456451416016 0.0
147.982033057583 151.966705322266 1.0
146.808413813287 153.373962402344 2.0
142.01934397 150.906112670898 3.0
-1.55484310406e-09 -0.67039567232132 4.0
-1.55484309999997e-09 -2.84772348403931 5.0
-64.9693040339936 -64.3290252685547 6.0
77.6547829144777 74.1872634887695 7.0
61.2786926634778 55.4030647277832 8.0
16.3760902514777 7.20984888076782 9.0
97.5261141374777 89.7391510009766 10.0
83.6472874924777 79.6450042724609 11.0
85.2981411144777 83.8642730712891 12.0
85.2981411144777 76.4624557495117 13.0
85.2981411144777 81.5436782836914 14.0
2.3328021736e-08 -2.03496789932251 15.0
2.3328021735e-08 -1.6052393913269 16.0
5.464350236e-08 -2.72059202194214 17.0
56.2518018559467 59.0783615112305 18.0
-56.2518018797166 -57.0891418457031 19.0
-56.2518018707166 -59.2854194641113 20.0
-56.2518017707166 -58.115608215332 21.0
77.1875821073724 73.3798065185547 22.0
153.454771802806 145.008895874023 23.0
115.220596133508 115.750228881836 24.0
21.7005584165075 20.8819332122803 25.0
230.642353915167 228.381210327148 26.0
230.642353915167 229.708221435547 27.0
26.8883420391666 21.3647422790527 28.0
257.530695955167 253.23681640625 29.0
257.530695955167 255.336776733398 30.0
46.7385281475469 45.537841796875 31.0
9.28724497493877 8.4355525970459 32.0
20.9357803369809 22.6376686096191 33.0
20.9357803369809 19.0790691375732 34.0
20.935780467613 20.4264030456543 35.0
8.86054244871795 8.39873600006104 36.0
7.60121916461794 6.85336446762085 37.0
2.62465480000001e-09 1.1959273815155 38.0
2.62465480000001e-09 -1.17553234100342 39.0
-11.6034306368583 -12.5643882751465 40.0
-24.5599220844463 -26.7411251068115 41.0
4.75986482786468 1.92820739746094 42.0
-31.7734218810648 -31.6524333953857 43.0
89.4696777832616 86.4044494628906 44.0
89.4696781210649 91.9600372314453 45.0
7.05809588635643 4.443039894104 46.0
2.46335994485643 2.26466846466064 47.0
147.98203306 156.095336914062 48.0
142.01934397 153.139389038086 49.0
0 -3.73605728149414 50.0
77.654782914 76.0232543945312 51.0
97.526114137 92.0755157470703 52.0
85.298141114 81.5651702880859 53.0
85.298141114 80.2131118774414 54.0
85.298141114 79.9203643798828 55.0
5.4643502359e-08 3.07315587997437 56.0
77.187582108 79.4883346557617 57.0
230.64235391 222.583419799805 58.0
257.53069595 249.491134643555 59.0
20.935780468 19.6116352081299 60.0
2.6246548239e-09 3.64096260070801 61.0
154.94214544 166.212432861328 0.0
154.94214544 168.091445922852 1.0
153.62589673 167.219589233398 2.0
148.83682701 157.139724731445 3.0
0 3.65104961395264 4.0
-4.9505104376e-10 2.73673295974731 5.0
-66.795543015 -62.8263244628906 6.0
77.888199916 78.7957305908203 7.0
55.413902211 52.0081825256348 8.0
22.474297703 26.0417633056641 9.0
95.706300325 87.4950256347656 10.0
82.344258531 76.6533813476562 11.0
83.793651688 83.0808029174805 12.0
83.793651688 80.3548049926758 13.0
83.793651688 82.2523345947266 14.0
0 0.904163360595703 15.0
0 0.118558824062347 16.0
-4.0711194556e-09 -0.081396758556366 17.0
58.272470789 59.6661109924316 18.0
-58.2724708132059 -59.9852256774902 19.0
-58.2724707912059 -60.0485877990723 20.0
-58.2724708022059 -61.9626693725586 21.0
79.078352293 78.2061920166016 22.0
146.48641082 142.542114257812 23.0
112.41539298 111.946632385254 24.0
23.128940161 20.5876731872559 25.0
225.56476313 227.659149169922 26.0
225.56476313 227.675369262695 27.0
27.020278561 24.2825508117676 28.0
252.58504169 252.99772644043 29.0
252.58504169 255.526184082031 30.0
46.640486506 48.6040306091309 31.0
9.3728432198 8.63383388519287 32.0
20.805881496 21.6275863647461 33.0
20.805881496 18.8017044067383 34.0
20.805881493 20.760196685791 35.0
9.2008137471 6.86174058914185 36.0
7.88975615 7.60456466674805 37.0
-7.2708282969e-09 0.856835246086121 38.0
-7.2708377778e-09 -1.82655334472656 39.0
-12.095975605 -15.1782827377319 40.0
-25.163325798 -24.8636951446533 41.0
4.9047260102 5.71446466445923 42.0
-32.708732529 -34.7498626708984 43.0
91.528294481 90.3782424926758 44.0
91.528294482 92.1180877685547 45.0
6.6696959989 6.33383512496948 46.0
2.6088884578 2.52103924751282 47.0
154.94214545 168.774185180664 48.0
148.83682702 151.686676025391 49.0
0 0.956853687763214 50.0
77.888199918 75.6590118408203 51.0
95.706300327 89.6363754272461 52.0
83.79365169 81.8090438842773 53.0
83.79365169 81.0633850097656 54.0
83.79365169 80.0689010620117 55.0
0 0.68585079908371 56.0
79.078352301 80.0451049804688 57.0
225.56476313 222.409576416016 58.0
252.58504169 247.320617675781 59.0
20.8058815 22.6847305297852 60.0
0 1.50079369544983 61.0
156.29145535 168.717666625977 0.0
156.29145535 168.816070556641 1.0
154.95797318 158.868072509766 2.0
150.16890346 153.175170898438 3.0
0 0.694754064083099 4.0
-1.5763668704e-13 -0.321757853031158 5.0
-59.513142545 -50.2802238464355 6.0
71.065195419 81.9145736694336 7.0
53.608585529 51.4097175598145 8.0
17.45660988 25.2165222167969 9.0
84.470516169 87.1336441040039 10.0
71.752746382 76.3091201782227 11.0
78.239808874 84.8103179931641 12.0
78.239808874 80.8271102905273 13.0
78.239808874 81.7426681518555 14.0
0 -0.275531589984894 15.0
0 0.0921301245689392 16.0
-2.7799551485e-13 -0.831766784191132 17.0
59.187696035 59.3059425354004 18.0
-59.187696055 -59.419261932373 19.0
-59.187696033 -60.0270652770996 20.0
-59.187696045 -60.0391654968262 21.0
81.324000119 79.2195053100586 22.0
142.34889543 143.373443603516 23.0
109.79940278 109.305633544922 24.0
22.592001452 19.3882579803467 25.0
223.67289556 227.238174438477 26.0
223.67289556 225.819610595703 27.0
27.935162937 24.3881702423096 28.0
251.60805851 255.880599975586 29.0
251.60805851 254.912979125977 30.0
46.395465018 47.2871246337891 31.0
9.2843642059 8.65455150604248 32.0
22.136304064 20.9917125701904 33.0
22.136304064 19.836986541748 34.0
22.136304064 21.0628108978271 35.0
9.678753604 7.50641822814941 36.0
8.4100230818 6.72224760055542 37.0
-2.862309201e-13 -3.3115177154541 38.0
-2.8623092346e-13 -1.6782693862915 39.0
-11.364842801 -16.3675079345703 40.0
-23.765676154 -24.6364936828613 41.0
5.1593674889 6.23309659957886 42.0
-31.489768091 -32.9559059143066 43.0
93.543804157 96.731071472168 44.0
93.54380417 96.1247100830078 45.0
6.9857098664 5.84714412689209 46.0
2.5929403942 1.28131520748138 47.0
156.29145536 169.883926391602 48.0
150.16890347 152.11328125 49.0
0 0.163073718547821 50.0
71.065195429 76.8097305297852 51.0
84.470516179 90.7965850830078 52.0
78.239808884 79.955078125 53.0
78.239808884 81.9208755493164 54.0
78.239808884 80.4639282226562 55.0
0 2.61783266067505 56.0
81.324000129 80.0384368896484 57.0
223.67289557 222.408294677734 58.0
251.60805852 248.074584960938 59.0
22.136304074 22.983512878418 60.0
0 2.05088663101196 61.0
137.17687742 143.006484985352 0.0
137.17687742 143.032318115234 1.0
135.86355256 143.526748657227 2.0
131.0744823 135.750274658203 3.0
0 -2.09861373901367 4.0
0 0.341894507408142 5.0
-59.42861149 -59.2324256896973 6.0
69.705106679 70.1772766113281 7.0
65.43029471 62.4528312683105 8.0
4.2748119615 8.48783111572266 9.0
78.669161727 80.2065200805664 10.0
65.229460649 71.2585144042969 11.0
77.500548395 80.8047180175781 12.0
77.500548395 77.9000701904297 13.0
77.500548395 76.9198608398438 14.0
0 -0.64198762178421 15.0
0 0.28728449344635 16.0
0 -3.15485000610352 17.0
59.754945382 58.5015258789062 18.0
-59.754945394 -59.4383087158203 19.0
-59.754945372 -60.0649642944336 20.0
-59.754945384 -61.8801422119141 21.0
80.656571382 77.8813858032227 22.0
140.87903669 140.394561767578 23.0
106.41719675 107.357681274414 24.0
22.90053999 20.4083709716797 25.0
221.53560808 223.594223022461 26.0
221.53560808 221.987747192383 27.0
27.282645659 25.3551406860352 28.0
248.81825375 253.574676513672 29.0
248.81825375 252.408157348633 30.0
47.033013387 47.2299919128418 31.0
9.3131031344 9.18030452728271 32.0
20.901625993 20.4819431304932 33.0
20.901625993 21.1129970550537 34.0
20.901625993 20.8195400238037 35.0
9.3083381265 8.28715133666992 36.0
8.044767894 8.38281917572021 37.0
-2.5262790395e-09 -0.804296553134918 38.0
-2.5262785966e-09 0.113371789455414 39.0
-12.387792857 -14.5510091781616 40.0
-23.81592281 -27.5234241485596 41.0
4.9278776077 3.5162615776062 42.0
-31.108913619 -34.7394485473633 43.0
92.052766479 91.3064651489258 44.0
92.05276649 90.2343292236328 45.0
6.7273095346 3.00315618515015 46.0
2.6568330431 1.83220052719116 47.0
137.17687743 136.682632446289 48.0
131.07448231 127.516792297363 49.0
0 3.54757404327393 50.0
69.705106686 70.9864120483398 51.0
78.669161734 84.2529830932617 52.0
77.500548402 81.0479965209961 53.0
77.500548402 80.9212951660156 54.0
77.500548402 79.4541549682617 55.0
0 -0.0822644829750061 56.0
80.656571391 80.3950500488281 57.0
221.53560809 218.26057434082 58.0
248.81825375 246.469757080078 59.0
20.901626002 19.9398994445801 60.0
0 2.89454770088196 61.0
152.524167139285 161.566421508789 0.0
152.524167139285 166.802734375 1.0
151.35996019 159.585845947266 2.0
146.57089042 149.159912109375 3.0
0 2.72630953788757 4.0
-5.13071198e-09 3.58648037910461 5.0
-59.53115247936 -58.3727226257324 6.0
70.2507919487599 73.059684753418 7.0
47.7617113627599 52.3051528930664 8.0
22.4890805857599 22.5464820861816 9.0
89.8720038897599 86.200553894043 10.0
76.7725129637599 74.7876510620117 11.0
84.1030283177599 83.5107803344727 12.0
84.1030283177599 81.3520660400391 13.0
84.1030283177599 81.384880065918 14.0
4.8380164139e-09 -1.74889588356018 15.0
2.8396886315e-09 -1.23449850082397 16.0
1.106254265e-08 -2.39599061012268 17.0
56.9242230005054 59.1188087463379 18.0
-56.9242229541398 -58.7474822998047 19.0
-56.9242229371398 -60.8387451171875 20.0
-56.9242229121398 -59.8430595397949 21.0
78.9448309813946 78.8030319213867 22.0
146.934885677505 142.000686645508 23.0
114.237803181882 114.043357849121 24.0
23.5512008828817 20.2917079925537 25.0
225.879716658569 230.664596557617 26.0
225.879716658569 224.718948364258 27.0
25.2255969145691 25.6801815032959 28.0
251.105313568569 253.560928344727 29.0
251.105313568569 254.459930419922 30.0
46.9430737789106 46.5697174072266 31.0
9.34221855687645 8.89533805847168 32.0
22.0206080690799 20.3247928619385 33.0
22.0206080690799 17.7073993682861 34.0
22.0206081119172 21.5819072723389 35.0
9.92452194042601 9.5117883682251 36.0
8.63421670852602 7.30539178848267 37.0
-5.88814005000002e-09 -5.77714157104492 38.0
-5.88814005000002e-09 -5.02064228057861 39.0
-12.0646002710699 -12.758918762207 40.0
-24.828896422166 -27.8423080444336 41.0
5.00332440529989 5.63134098052979 42.0
-32.4500067182999 -34.2524604797363 43.0
92.7446581152999 89.7777099609375 44.0
92.7446577858464 95.4271469116211 45.0
6.91667725158936 5.73279666900635 46.0
2.42208655028936 2.3217933177948 47.0
152.52416714 162.400619506836 48.0
146.57089043 142.35383605957 49.0
0 1.32746386528015 50.0
70.250791949 77.4059371948242 51.0
89.872003889 88.4692153930664 52.0
84.103028317 81.2791595458984 53.0
84.103028317 81.4317092895508 54.0
84.103028317 80.59765625 55.0
1.106254265e-08 0.0498611330986023 56.0
78.944830981 79.5637359619141 57.0
225.87971666 222.978210449219 58.0
251.10531357 249.396270751953 59.0
22.020608111 24.7376899719238 60.0
0 1.85762310028076 61.0
188.35861472 161.527542114258 0.0
188.35861472 166.743255615234 1.0
187.12495818 177.637115478516 2.0
182.33588845 173.685607910156 3.0
0 5.14164781570435 4.0
-7.012611516e-12 1.29842388629913 5.0
-53.244760868 -61.9058570861816 6.0
65.595817622 73.8034210205078 7.0
49.142738849 45.5146675109863 8.0
16.453078763 17.7125606536865 9.0
98.137238072 90.3498229980469 10.0
85.200221678 75.5721282958984 11.0
85.839947068 92.6935043334961 12.0
85.839947068 80.8045959472656 13.0
85.839947068 81.6752243041992 14.0
0 -1.43005418777466 15.0
0 -0.885065853595734 16.0
-8.7226531548e-12 -0.0886810421943665 17.0
61.636511309 55.809211730957 18.0
-61.636511329 -52.5389595031738 19.0
-61.636511307 -61.0566711425781 20.0
-61.63651132 -56.9647941589355 21.0
83.481460517 77.7001571655273 22.0
139.38782775 141.661605834961 23.0
105.13815424 110.336486816406 24.0
21.146376421 21.9361515045166 25.0
222.86928827 226.956588745117 26.0
222.86928827 216.012084960938 27.0
28.029464278 29.3300018310547 28.0
250.89875256 259.154327392578 29.0
250.89875256 247.227645874023 30.0
46.672996384 47.0265655517578 31.0
9.254354393 6.82443141937256 32.0
21.844949187 22.9647731781006 33.0
21.844949187 15.4007530212402 34.0
21.844949187 20.6262683868408 35.0
10.362363952 6.2620792388916 36.0
9.0499341022 8.03130149841309 37.0
-2.10045253e-11 -1.76586318016052 38.0
-2.1004555058e-11 1.09970664978027 39.0
-12.290568638 -16.8061084747314 40.0
-25.067807734 -24.3410568237305 41.0
4.859556457 4.55338954925537 42.0
-32.638170905 -32.7825164794922 43.0
90.298294905 94.1913681030273 44.0
90.298294918 93.3824996948242 45.0
6.5153827865 3.95994138717651 46.0
2.0175575739 1.01958823204041 47.0
188.35861473 165.483795166016 48.0
182.33588846 171.608459472656 49.0
0 0.0955412983894348 50.0
65.595817631 77.1698913574219 51.0
98.137238082 91.5490570068359 52.0
85.839947078 79.6971893310547 53.0
85.839947078 81.5385131835938 54.0
85.839947078 79.9933242797852 55.0
0 -0.120720326900482 56.0
83.481460527 78.8674545288086 57.0
222.86928828 220.218078613281 58.0
250.89875257 243.010635375977 59.0
21.844949197 22.1279582977295 60.0
0 2.65086913108826 61.0
188.11289016 159.915313720703 0.0
188.11289016 170.102005004883 1.0
187.06914844 177.421401977539 2.0
182.28007872 168.033065795898 3.0
0 4.44767189025879 4.0
-4.722767456e-14 2.86840200424194 5.0
-71.310829306 -55.7166938781738 6.0
82.963367141 77.9166259765625 7.0
51.043695763 51.1218185424805 8.0
31.919671368 21.8328685760498 9.0
83.063168831 90.4022979736328 10.0
71.358745608 73.4979248046875 11.0
80.560929915 81.213249206543 12.0
80.560929915 77.803352355957 13.0
80.560929915 83.0279541015625 14.0
0 2.83509683609009 15.0
0 -0.247417747974396 16.0
0 2.37327075004578 17.0
52.855188393 53.970401763916 18.0
-52.855188403 -52.4599380493164 19.0
-52.855188382 -60.683422088623 20.0
-52.855188394 -56.3549842834473 21.0
73.581352593 77.2407073974609 22.0
146.72933516 143.000671386719 23.0
116.2671277 111.094497680664 24.0
24.520672941 18.9487476348877 25.0
220.31068776 232.348083496094 26.0
220.31068776 226.983825683594 27.0
27.61786958 26.3444766998291 28.0
247.92855735 252.710021972656 29.0
247.92855735 254.260498046875 30.0
46.478335659 44.2851600646973 31.0
9.3221717998 9.015305519104 32.0
20.72616419 18.7731781005859 33.0
20.72616419 17.7240924835205 34.0
20.72616419 22.5669727325439 35.0
9.4555057032 7.25285673141479 36.0
8.2087649269 7.35748624801636 37.0
-1.4102704278e-13 -9.76002025604248 38.0
-1.4102704273e-13 5.53264093399048 39.0
-12.075001761 -13.839243888855 40.0
-25.580642543 -16.6018142700195 41.0
5.3808760486 10.0048122406006 42.0
-33.635633279 -36.1592674255371 43.0
87.058553694 87.5045471191406 44.0
87.058553707 91.0124816894531 45.0
7.5058741731 4.76283359527588 46.0
2.7846570648 1.48382472991943 47.0
188.11289017 164.454818725586 48.0
182.28007873 167.030212402344 49.0
0 -2.6207230091095 50.0
82.963367151 76.9311065673828 51.0
83.063168841 86.3522567749023 52.0
80.560929925 79.2899856567383 53.0
80.560929925 80.6327590942383 54.0
80.560929925 79.7011184692383 55.0
0 -0.765894949436188 56.0
73.581352603 77.6605529785156 57.0
220.31068777 221.160140991211 58.0
247.92855736 248.710693359375 59.0
20.7261642 24.8430576324463 60.0
0 6.97875690460205 61.0
178.888201318669 168.374816894531 0.0
178.888201318669 172.24870300293 1.0
177.64728756 169.221603393555 2.0
172.85821783 182.543380737305 3.0
0 0.800851047039032 4.0
-6.437998937e-09 0.218645751476288 5.0
-63.78225297 -59.7823143005371 6.0
74.6886032493594 81.4997406005859 7.0
46.5043263003595 49.9050712585449 8.0
28.1842769483595 25.407735824585 9.0
89.2862577113594 87.7333297729492 10.0
76.6038195393594 74.8431854248047 11.0
83.7078348043594 84.7164535522461 12.0
83.7078348043594 78.5349426269531 13.0
83.7078348043594 76.927116394043 14.0
0 -0.505191147327423 15.0
0 0.382724165916443 16.0
-8.3996938744e-09 -2.89570498466492 17.0
55.490699341287 58.7428016662598 18.0
-55.4906993565392 -59.6449279785156 19.0
-55.4906993355392 -59.5124130249023 20.0
-55.4906993445392 -59.8692932128906 21.0
77.056278242 76.7875671386719 22.0
143.397078489287 140.011520385742 23.0
110.918074366609 114.016265869141 24.0
22.9836760286087 20.8778285980225 25.0
220.453356737338 229.823822021484 26.0
220.453356737338 225.399703979492 27.0
26.1386657943378 25.5570697784424 28.0
246.592022537338 254.735412597656 29.0
246.592022537338 255.256484985352 30.0
46.402069793 44.3921279907227 31.0
9.2741165734 9.23030376434326 32.0
21.5655789021652 19.4150314331055 33.0
21.5655789021652 23.0353355407715 34.0
21.565578902 20.1715812683105 35.0
9.8271505547 8.97214698791504 36.0
8.6084834152 7.07053279876709 37.0
-9.678018421e-09 1.03383266925812 38.0
-9.678018421e-09 -2.7609691619873 39.0
-11.8654536422696 -13.5939950942993 40.0
-24.055723913 -25.0566234588623 41.0
5.14487075603287 5.05587673187256 42.0
-31.7612125306329 -33.0886344909668 43.0
103.576685793633 93.8842315673828 44.0
103.576685793633 93.6891784667969 45.0
6.4522704253 6.28902435302734 46.0
2.4693809398 2.55379343032837 47.0
178.88820132 166.748168945312 48.0
172.85821784 167.167190551758 49.0
0 3.71830129623413 50.0
74.688603249 73.9269790649414 51.0
89.286257711 87.9648132324219 52.0
83.707834805 81.4174194335938 53.0
83.707834805 81.5539627075195 54.0
83.707834805 80.0681838989258 55.0
0 1.85776305198669 56.0
77.056278247 80.1147613525391 57.0
220.45335674 221.236770629883 58.0
246.59202254 247.994049072266 59.0
21.565578905 19.2825412750244 60.0
0 1.08062720298767 61.0
153.45630845 157.846054077148 0.0
153.45630845 162.785614013672 1.0
152.27388972 169.747131347656 2.0
147.48481999 157.551879882812 3.0
0 0.228694140911102 4.0
-1.0348165221e-09 4.09048557281494 5.0
-67.527517278 -61.0231170654297 6.0
78.864930217 78.2646942138672 7.0
63.598385629 53.0704765319824 8.0
15.266544583 29.6661071777344 9.0
86.988488038 86.5590286254883 10.0
75.413175072 77.8539886474609 11.0
77.949070768 85.9935531616211 12.0
77.949070768 79.7432861328125 13.0
77.949070768 72.9773406982422 14.0
0 -8.07441902160645 15.0
0 -0.585292160511017 16.0
0 0.52325302362442 17.0
56.957479582 59.480712890625 18.0
-56.957479597 -59.4113616943359 19.0
-56.957479575 -58.966064453125 20.0
-56.957479587 -58.2117614746094 21.0
80.099169183 78.8520660400391 22.0
152.32242761 137.579177856445 23.0
115.72624204 112.943344116211 24.0
22.85984251 19.6951160430908 25.0
232.4215968 221.7119140625 26.0
232.4215968 223.896621704102 27.0
23.581579218 29.0908660888672 28.0
256.00317603 247.303695678711 29.0
256.00317603 249.910003662109 30.0
46.710474883 46.961597442627 31.0
9.3890758442 8.73222732543945 32.0
23.141689594 18.4091815948486 33.0
23.141689594 16.3139667510986 34.0
23.141689593 19.9925861358643 35.0
10.684884215 9.23737049102783 36.0
9.4227954612 8.00438213348389 37.0
-3.4712976217e-09 -5.89797782897949 38.0
-3.4712976217e-09 1.25412392616272 39.0
-11.889861599 -12.8110990524292 40.0
-25.252705442 -22.574462890625 41.0
5.2702049723 5.13136577606201 42.0
-32.885090471 -32.994571685791 43.0
97.705632693 94.4482803344727 44.0
97.705632704 91.5599136352539 45.0
6.7160299085 5.30523872375488 46.0
1.8921580034 2.86065030097961 47.0
153.45630846 165.56640625 48.0
147.48482 149.425201416016 49.0
0 0.0676442384719849 50.0
78.864930224 77.2593002319336 51.0
86.988488045 93.1648788452148 52.0
77.949070774 82.3914642333984 53.0
77.949070774 81.4169769287109 54.0
77.949070774 80.3136291503906 55.0
0 3.70716738700867 56.0
80.099169192 79.9301605224609 57.0
232.42159681 216.034515380859 58.0
256.00317603 247.22607421875 59.0
23.141689601 21.4480133056641 60.0
0 2.68804907798767 61.0
189.24281877 153.304092407227 0.0
189.24281877 163.583435058594 1.0
187.95025603 166.658432006836 2.0
183.16118631 192.000015258789 3.0
0 -1.33618831634521 4.0
-3.5130057022e-12 4.74383306503296 5.0
-63.809428785 -46.5229682922363 6.0
73.171654401 91.8746719360352 7.0
52.508177435 58.0366401672363 8.0
20.663476956 31.7980499267578 9.0
89.684471167 94.2119064331055 10.0
76.143017535 79.6999740600586 11.0
78.654227189 80.5734558105469 12.0
78.654227189 79.3919372558594 13.0
78.654227189 81.4614791870117 14.0
0 -2.35738492012024 15.0
0 -0.034458339214325 16.0
-3.4104055046e-12 -1.54538536071777 17.0
58.659545579 56.8751602172852 18.0
-58.659545599 -59.3209037780762 19.0
-58.659545577 -59.4521331787109 20.0
-58.659545589 -62.9230079650879 21.0
79.153149103 76.8940048217773 22.0
143.34294736 137.906433105469 23.0
109.84666404 113.695251464844 24.0
21.087354511 21.287467956543 25.0
222.49609648 222.445602416992 26.0
222.49609648 219.009521484375 27.0
25.297755294 29.5036334991455 28.0
247.79385178 254.924407958984 29.0
247.79385178 253.633453369141 30.0
46.919162042 44.7332916259766 31.0
9.3048826358 9.12325382232666 32.0
20.493603503 14.793999671936 33.0
20.493603503 13.0898857116699 34.0
20.493603503 20.1962985992432 35.0
9.2407929802 7.62399101257324 36.0
7.919669314 5.75054121017456 37.0
-9.9716074119e-12 -1.36597967147827 38.0
-9.9716074119e-12 8.58164310455322 39.0
-10.498044225 -19.6664161682129 40.0
-22.756317274 -28.3956623077393 41.0
4.9079236427 2.5391948223114 42.0
-30.174633715 -31.6249828338623 43.0
82.203103872 85.6971817016602 44.0
82.203103885 87.4307632446289 45.0
6.3674716656 5.39477443695068 46.0
2.47033766 1.18622839450836 47.0
189.24281878 160.348983764648 48.0
183.16118632 177.780044555664 49.0
0 1.9334716796875 50.0
73.171654411 77.8594818115234 51.0
89.684471177 90.0266189575195 52.0
78.654227199 78.2778930664062 53.0
78.654227199 80.8933792114258 54.0
78.654227199 80.2361450195312 55.0
0 0.811580240726471 56.0
79.153149113 81.3647766113281 57.0
222.49609648 211.3525390625 58.0
247.79385179 249.86540222168 59.0
20.493603513 23.6781959533691 60.0
0 2.83803534507751 61.0
142.212199234643 158.242401123047 0.0
142.212199234643 162.660583496094 1.0
140.95816682 164.748504638672 2.0
135.49353159 149.753005981445 3.0
0.3983086571 3.72175645828247 4.0
0.398308666200141 10.4839057922363 5.0
-67.0605747916801 -54.2794990539551 6.0
77.5104181053804 73.3832855224609 7.0
50.3997691703804 52.9961013793945 8.0
27.1106489343804 4.04772090911865 9.0
95.6405128233804 94.5696258544922 10.0
82.9776675753804 75.8003082275391 11.0
80.7589626083804 85.5527801513672 12.0
80.7589626083804 79.0862503051758 13.0
80.7589626083804 71.4978790283203 14.0
1.9406676556e-09 2.05648946762085 15.0
0 1.2619469165802 16.0
-6.86606412999999e-09 3.07014536857605 17.0
57.0688802847529 56.9949607849121 18.0
-57.0688802790706 -59.2165870666504 19.0
-57.0688802590706 -57.974365234375 20.0
-57.0688802530706 -59.2229347229004 21.0
79.4923384806974 75.0018539428711 22.0
147.444902208753 143.802124023438 23.0
114.445707400941 100.480964660645 24.0
22.5470211779412 24.4297466278076 25.0
226.937240689285 222.736999511719 26.0
226.937240689285 220.457794189453 27.0
26.4700803152851 27.5725479125977 28.0
253.407320999285 247.391662597656 29.0
253.407320999285 248.090621948242 30.0
47.2884398759555 44.9387474060059 31.0
9.35680360318831 8.19455909729004 32.0
22.4234582280403 18.1145076751709 33.0
22.4234582280403 13.6144094467163 34.0
22.4234582499587 20.69655418396 35.0
9.69072750791308 5.32759475708008 36.0
8.43755135561308 6.51123142242432 37.0
-8.63881030000001e-09 -8.93080043792725 38.0
-8.63881030000001e-09 -5.27449560165405 39.0
-11.8614972370353 -9.32962226867676 40.0
-24.5266586395832 -27.0754890441895 41.0
5.02697101575034 3.89485025405884 42.0
-32.0307519371503 -31.5535106658936 43.0
104.71402758115 100.212158203125 44.0
104.71402750115 94.9428329467773 45.0
6.4596344641 3.31813955307007 46.0
2.2386912463 2.42008018493652 47.0
142.21219923 161.704040527344 48.0
135.4935316 147.787734985352 49.0
0.3983086662 5.97553730010986 50.0
77.510418105 75.5816040039062 51.0
95.640512823 87.5701446533203 52.0
80.758962608 80.6090393066406 53.0
80.758962608 81.7780685424805 54.0
80.758962608 79.2183303833008 55.0
0 -0.774916470050812 56.0
79.492338481 79.2989044189453 57.0
226.93724069 219.722610473633 58.0
253.407321 244.706100463867 59.0
22.42345825 22.360445022583 60.0
0 4.82096004486084 61.0
182.030902908478 158.702072143555 0.0
182.030902908478 159.748474121094 1.0
180.7369681 175.198959350586 2.0
175.94789838 165.452087402344 3.0
0 -0.822106659412384 4.0
-8.5055487225e-09 12.4162473678589 5.0
-56.5248579853413 -49.1940307617188 6.0
66.4603942246947 81.7761306762695 7.0
62.6511539796947 55.082576751709 8.0
3.80924024539473 17.8995456695557 9.0
87.2863398856947 83.1842346191406 10.0
73.9589352206947 74.006477355957 11.0
81.2988143496947 80.8458480834961 12.0
81.2988143496947 79.7641296386719 13.0
81.2988143496947 83.5395431518555 14.0
0 -1.33115434646606 15.0
0 -0.582838475704193 16.0
-6.299969124e-09 1.12940430641174 17.0
56.9725534776517 58.0894393920898 18.0
-56.9725534870421 -59.2990303039551 19.0
-56.9725534660421 -61.3771324157715 20.0
-56.9725534730421 -62.3262329101562 21.0
79.3500212569175 79.4605407714844 22.0
150.245282629652 143.983062744141 23.0
116.179884984565 109.887596130371 24.0
21.4526102155646 21.1231079101562 25.0
229.595303886955 225.225112915039 26.0
229.595303886955 227.641876220703 27.0
24.1637688839555 25.2515773773193 28.0
253.759072766955 255.398513793945 29.0
253.759072766955 253.366439819336 30.0
46.3401295151344 45.927661895752 31.0
9.3990912314 8.79635620117188 32.0
22.3774677840241 18.7030410766602 33.0
22.3774677840241 8.86799144744873 34.0
22.3774677901361 19.1004734039307 35.0
9.7811686651 6.1878457069397 36.0
8.6383513445 8.77671337127686 37.0
-9.402654923e-09 -0.264768302440643 38.0
-9.402654923e-09 -6.85224056243896 39.0
-11.731173514021 -13.8486528396606 40.0
-24.8050307820312 -31.550085067749 41.0
5.32330462191228 -0.624313652515411 42.0
-32.3843285729123 -33.6930618286133 43.0
92.1879425709123 91.7188262939453 44.0
92.1879425579123 93.5702133178711 45.0
5.8801326403 4.93358898162842 46.0
2.1651973211 1.77921867370605 47.0
182.03090291 163.521026611328 48.0
175.94789839 156.667846679688 49.0
0 3.35714960098267 50.0
66.460394225 74.5578460693359 51.0
87.286339886 83.9722137451172 52.0
81.29881435 80.0632019042969 53.0
81.29881435 73.9345932006836 54.0
81.29881435 80.2585525512695 55.0
0 2.67186665534973 56.0
79.350021257 80.1596450805664 57.0
229.59530389 221.394561767578 58.0
253.75907277 249.784591674805 59.0
22.37746779 17.4587841033936 60.0
0 1.41972827911377 61.0
154.087209775025 164.572647094727 0.0
154.087209775025 165.257537841797 1.0
152.77251473 158.085220336914 2.0
147.98344498 134.186630249023 3.0
0 3.04957509040833 4.0
-6.976018033e-09 -1.35981464385986 5.0
-62.7109286838695 -66.0733642578125 6.0
73.3583489290431 71.3933029174805 7.0
61.1147634020431 50.691722869873 8.0
12.2435855270431 19.2828578948975 9.0
93.4150109450431 82.1822509765625 10.0
80.7849847610431 71.3012847900391 11.0
84.7907752690431 89.7280578613281 12.0
84.7907752690431 79.441520690918 13.0
84.7907752690431 78.5909729003906 14.0
1.6850988523e-09 -3.77514743804932 15.0
0 -2.45132374763489 16.0
5.22104629999988e-10 -2.63921022415161 17.0
58.1437922491144 59.7875862121582 18.0
-58.1437922360646 -55.6384010314941 19.0
-58.1437922160646 -62.1023330688477 20.0
-58.1437922110646 -65.8423614501953 21.0
80.9426927278765 75.0631866455078 22.0
133.032358129114 137.24934387207 23.0
101.627668681393 108.015922546387 24.0
23.932245220393 19.642032623291 25.0
213.97505086005 221.849060058594 26.0
213.97505086005 223.381546020508 27.0
24.3172031670497 25.3994960784912 28.0
238.29225403005 259.591827392578 29.0
238.29225403005 259.405792236328 30.0
46.3267666781485 48.1941757202148 31.0
9.39286044421825 10.5296545028687 32.0
22.7989005176084 21.7379035949707 33.0
22.7989005176084 21.7576217651367 34.0
22.7989005391873 21.0930480957031 35.0
10.3350462748461 8.86963176727295 36.0
9.00247180643072 6.21103858947754 37.0
-7.94938021e-09 1.92535495758057 38.0
-7.94938021e-09 -2.35086536407471 39.0
-12.2632537075323 -10.7590579986572 40.0
-23.9318346619202 -25.7616367340088 41.0
5.19403945340256 11.4188480377197 42.0
-31.9086647087026 -37.7276229858398 43.0
90.2248390667026 87.2116394042969 44.0
90.2248389887026 94.1394653320312 45.0
7.1480689636 7.03012847900391 46.0
2.5064031783 2.77427697181702 47.0
154.08720978 170.710357666016 48.0
147.98344499 143.025283813477 49.0
0 1.75157034397125 50.0
73.358348929 73.2819061279297 51.0
93.415010945 83.2170181274414 52.0
84.790775269 79.1651229858398 53.0
84.790775269 81.3897705078125 54.0
84.790775269 79.9757232666016 55.0
5.2210463501e-10 1.01836371421814 56.0
80.942692728 81.1798324584961 57.0
213.97505086 213.279205322266 58.0
238.29225403 249.094100952148 59.0
22.798900539 23.0661716461182 60.0
0 3.38156795501709 61.0
122.492444500762 168.830184936523 0.0
122.492444500762 163.895141601562 1.0
121.01375239 166.507125854492 2.0
116.22468266 146.639068603516 3.0
0 3.61375141143799 4.0
-8.7204412675e-09 -3.22320508956909 5.0
-48.662264027 -50.0952835083008 6.0
57.8690252826536 81.5259857177734 7.0
41.9501174776536 48.5476684570312 8.0
15.9189078056536 28.3375110626221 9.0
78.8416983976536 91.3793869018555 10.0
66.8546783646536 67.6533966064453 11.0
69.8693741986536 85.3102264404297 12.0
69.8693741986536 90.1217727661133 13.0
69.8693741986536 77.5349884033203 14.0
0 -0.37270313501358 15.0
0 0.0945518016815186 16.0
-6.6574927e-09 7.43443536758423 17.0
57.840129998811 60.7123069763184 18.0
-57.8401300099803 -55.4037399291992 19.0
-57.8401299879803 -58.2770347595215 20.0
-57.8401299959803 -65.0445556640625 21.0
79.626649826 80.4363250732422 22.0
143.073012921811 135.835235595703 23.0
110.831199767264 119.732498168945 24.0
22.0973223922638 19.5632400512695 25.0
222.699662751523 220.432907104492 26.0
222.699662751523 213.651062011719 27.0
26.3263536215233 23.4881248474121 28.0
249.026016371523 245.819686889648 29.0
249.026016371523 235.467971801758 30.0
47.073735752 48.1618003845215 31.0
9.3328621107 8.06484222412109 32.0
21.7865198314173 15.7455434799194 33.0
21.7865198314173 9.83489799499512 34.0
21.7865198365012 18.4934825897217 35.0
9.0203447251 6.72415113449097 36.0
7.700245251 3.6424036026001 37.0
-9.423547705e-09 -32.4079437255859 38.0
-9.423547705e-09 -0.160629570484161 39.0
-11.3561600629902 -18.3249378204346 40.0
-23.4459576020442 -31.2633533477783 41.0
5.8945740132113 1.55947828292847 42.0
-31.9086475822113 -32.2541618347168 43.0
73.6047289992113 70.7840042114258 44.0
73.6047289584886 55.2600135803223 45.0
6.5415248309 13.0380973815918 46.0
1.6279530272 3.18781232833862 47.0
122.4924445 171.360061645508 48.0
116.22468267 150.33366394043 49.0
0 -8.71527481079102 50.0
57.869025283 74.366828918457 51.0
78.841698398 79.8299560546875 52.0
69.869374198 78.5195999145508 53.0
69.869374198 81.5034942626953 54.0
69.869374198 82.9144744873047 55.0
0 3.37303924560547 56.0
79.626649827 73.5043792724609 57.0
222.69966275 215.117584228516 58.0
249.02601637 228.20100402832 59.0
21.786519837 11.8999328613281 60.0
0 2.35391712188721 61.0
162.38469945 160.608215332031 0.0
162.38469945 159.386184692383 1.0
161.07140382 157.380813598633 2.0
156.28233409 189.3349609375 3.0
0 14.5906200408936 4.0
-2.3519564874e-10 3.5825891494751 5.0
-51.951625121 -60.7122116088867 6.0
64.2212339482612 79.3839645385742 7.0
42.8630173082612 50.2891998291016 8.0
21.3582166402612 24.4324169158936 9.0
93.2544519242612 93.301887512207 10.0
79.4160535262612 77.4576950073242 11.0
83.8442717592612 81.9186019897461 12.0
83.8442717592612 79.6177444458008 13.0
83.8442717592612 80.3756790161133 14.0
0 4.48700714111328 15.0
0 0.225697994232178 16.0
-1.3961640607e-09 -3.052405834198 17.0
62.607322303 60.0581207275391 18.0
-62.6073223218918 -58.7881240844727 19.0
-62.6073223008918 -57.3753204345703 20.0
-62.6073223118918 -62.1456336975098 21.0
85.079235204 76.3406219482422 22.0
136.26480927 139.821411132812 23.0
105.20446137 111.490913391113 24.0
21.531351826 22.2858963012695 25.0
221.344044487609 227.267440795898 26.0
221.344044487609 221.270355224609 27.0
24.4819073396091 26.5734195709229 28.0
245.825951827609 255.254104614258 29.0
245.825951827609 256.236114501953 30.0
46.895413385 47.9875183105469 31.0
9.2883438486 9.48495674133301 32.0
22.471912899 34.0405082702637 33.0
22.471912899 14.0964555740356 34.0
22.471912895 18.123254776001 35.0
10.162413254 10.6498689651489 36.0
9.0132287061 7.99884557723999 37.0
-9.6130710561e-09 7.30465269088745 38.0
-9.6130710564e-09 0.173000156879425 39.0
-11.766748876 -23.4504070281982 40.0
-24.609117299 -27.5924739837646 41.0
5.1215971911 1.75763726234436 42.0
-32.22820584 -33.2091636657715 43.0
91.106149504 90.9697570800781 44.0
91.10614951 96.122314453125 45.0
7.0584821816 6.5449595451355 46.0
2.2915314816 2.45220828056335 47.0
162.38469946 168.936569213867 48.0
156.2823341 157.071701049805 49.0
0 -1.9548978805542 50.0
64.221233949 75.6350860595703 51.0
93.254451924 91.9958114624023 52.0
83.84427176 82.1744537353516 53.0
83.84427176 81.1248931884766 54.0
83.84427176 81.4801940917969 55.0
0 4.37057161331177 56.0
85.079235211 80.3185195922852 57.0
221.34404449 220.285537719727 58.0
245.82595183 246.211288452148 59.0
22.471912901 18.4660816192627 60.0
0 2.34848952293396 61.0
193.620195423445 166.634307861328 0.0
193.620195423445 165.151702880859 1.0
192.51512681 163.651779174805 2.0
187.72605706 175.645416259766 3.0
-2.3135613719e-10 1.25951719284058 4.0
-8.647187154e-09 1.55087041854858 5.0
-71.41837288423 -75.5401611328125 6.0
84.4565912553051 71.6154403686523 7.0
54.8692068283051 30.6962127685547 8.0
29.5873844273051 20.8891258239746 9.0
86.4911072633051 85.8573989868164 10.0
73.6765223503051 81.8769683837891 11.0
75.0946964873051 77.241943359375 12.0
75.0946964873051 76.8628387451172 13.0
75.0946964873051 79.9015731811523 14.0
2.6469896651e-09 -4.75218296051025 15.0
0 0.401790797710419 16.0
1.87899875000001e-09 7.24826908111572 17.0
57.2199505702574 54.1835556030273 18.0
-57.2199505459577 -56.4330558776855 19.0
-57.2199505279577 -58.5358848571777 20.0
-57.2199505139577 -58.7397308349609 21.0
77.7338775470825 80.9988098144531 22.0
153.723823269257 144.229385375977 23.0
119.612085404072 111.542259216309 24.0
22.1670579150717 23.6703205108643 25.0
231.457700806891 225.893768310547 26.0
231.457700806891 225.304977416992 27.0
27.6857023538906 20.6725406646729 28.0
259.143403166891 246.940063476562 29.0
259.143403166891 251.32763671875 30.0
46.1791815442054 48.2974662780762 31.0
9.33613497328532 9.06848526000977 32.0
20.5139270336901 25.5485877990723 33.0
20.5139270336901 25.0695781707764 34.0
20.5139270633121 23.332218170166 35.0
9.35858819441393 10.3592844009399 36.0
8.04134756511393 7.78337001800537 37.0
-7.13341441000003e-09 0.731598198413849 38.0
-7.13341440999998e-09 -5.06418037414551 39.0
-11.5848948084789 -13.2671432495117 40.0
-24.4229379770874 -25.1310577392578 41.0
5.62421696768762 6.40897083282471 42.0
-32.6573894150876 -34.5704574584961 43.0
91.6286594930876 91.076416015625 44.0
91.6286593770876 93.1831359863281 45.0
6.5093336585 11.8331708908081 46.0
2.2247797827 3.34036993980408 47.0
193.62019542 174.051391601562 48.0
187.72605707 165.526718139648 49.0
0 -5.14626026153564 50.0
84.456591256 71.5327377319336 51.0
86.491107264 81.328742980957 52.0
75.094696488 79.8986358642578 53.0
75.094696488 79.7839431762695 54.0
75.094696488 80.1530303955078 55.0
1.8789987524e-09 5.04774045944214 56.0
77.733877547 77.833869934082 57.0
231.45770081 221.005035400391 58.0
259.14340317 247.586334228516 59.0
20.513927063 16.7871913909912 60.0
0 0.494023025035858 61.0
147.982033057583 157.456390380859 0.0
147.982033057583 151.963851928711 1.0
146.808413813287 153.350814819336 2.0
142.01934397 150.732803344727 3.0
-1.55484310406e-09 -0.650561392307281 4.0
-1.55484309999997e-09 -2.84032654762268 5.0
-64.9693040339936 -64.321533203125 6.0
77.6547829144777 74.1871871948242 7.0
61.2786926634778 55.4030647277832 8.0
16.3760902514777 7.20984888076782 9.0
97.5261141374777 89.7391510009766 10.0
83.6472874924777 79.6450042724609 11.0
85.2981411144777 83.8642730712891 12.0
85.2981411144777 76.4624557495117 13.0
85.2981411144777 81.5436782836914 14.0
2.3328021736e-08 -2.03496789932251 15.0
2.3328021735e-08 -1.6052393913269 16.0
5.464350236e-08 -2.72059202194214 17.0
56.2518018559467 59.0783615112305 18.0
-56.2518018797166 -57.0891418457031 19.0
-56.2518018707166 -59.2854194641113 20.0
-56.2518017707166 -58.115608215332 21.0
77.1875821073724 73.3798065185547 22.0
153.454771802806 145.008895874023 23.0
115.220596133508 115.750228881836 24.0
21.7005584165075 20.8819332122803 25.0
230.642353915167 228.381210327148 26.0
230.642353915167 229.708221435547 27.0
26.8883420391666 21.3647422790527 28.0
257.530695955167 253.23681640625 29.0
257.530695955167 255.336776733398 30.0
46.7385281475469 45.537841796875 31.0
9.28724497493877 8.4355525970459 32.0
20.9357803369809 22.6376686096191 33.0
20.9357803369809 19.0790691375732 34.0
20.935780467613 20.4264030456543 35.0
8.86054244871795 8.39873600006104 36.0
7.60121916461794 6.85336446762085 37.0
2.62465480000001e-09 1.1959273815155 38.0
2.62465480000001e-09 -1.17553234100342 39.0
-11.6034306368583 -12.5643882751465 40.0
-24.5599220844463 -26.7411251068115 41.0
4.75986482786468 1.92820739746094 42.0
-31.7734218810648 -31.6524333953857 43.0
89.4696777832616 86.4044494628906 44.0
89.4696781210649 91.9600372314453 45.0
7.05809588635643 4.443039894104 46.0
2.46335994485643 2.26466846466064 47.0
147.98203306 156.096572875977 48.0
142.01934397 152.998229980469 49.0
0 -3.74713611602783 50.0
77.654782914 76.0236663818359 51.0
97.526114137 92.0755157470703 52.0
85.298141114 81.5651702880859 53.0
85.298141114 80.2131118774414 54.0
85.298141114 79.9203643798828 55.0
5.4643502359e-08 3.07315587997437 56.0
77.187582108 79.4883346557617 57.0
230.64235391 222.583419799805 58.0
257.53069595 249.491134643555 59.0
20.935780468 19.6116352081299 60.0
2.6246548239e-09 3.64096260070801 61.0
156.861091404888 159.86442565918 0.0
156.861091404888 164.524307250977 1.0
155.52926231 162.341293334961 2.0
150.74019254 151.753829956055 3.0
0 -8.4057445526123 4.0
-5.22601789e-09 5.28012847900391 5.0
-68.3498850872301 -62.6833190917969 6.0
79.7958029518052 75.4073104858398 7.0
55.8596434508053 52.1667633056641 8.0
23.9361595018052 20.6490592956543 9.0
88.9024303608052 84.6050186157227 10.0
76.2639219328052 74.9370651245117 11.0
81.0219682688052 82.3687744140625 12.0
81.0219682688052 79.6216354370117 13.0
81.0219682688052 79.358039855957 14.0
4.602401647e-09 0.437743544578552 15.0
4.60240164699e-09 0.799240529537201 16.0
1.080383174e-08 -0.0869020819664001 17.0
59.0643240078584 58.368766784668 18.0
-59.0643240557079 -59.7739334106445 19.0
-59.0643240377079 -60.3347778320312 20.0
-59.0643240107079 -59.6208000183105 21.0
80.5432193786501 78.8417358398438 22.0
143.967734453894 142.143432617188 23.0
107.978514844867 110.714378356934 24.0
22.5896058518672 20.7431335449219 25.0
224.510953829775 227.853851318359 26.0
224.510953829775 226.092163085938 27.0
25.5775924157753 20.7458992004395 28.0
250.088546249775 256.343200683594 29.0
250.088546249775 256.1025390625 30.0
46.7252001506423 46.4096641540527 31.0
9.32784005598338 9.00160598754883 32.0
21.4788953686902 21.6592845916748 33.0
21.4788953686902 22.7354488372803 34.0
21.4788954153122 21.0440635681152 35.0
9.4949079483962 7.08659553527832 36.0
8.2248182515962 7.61897659301758 37.0
-5.71842478999997e-09 -0.65439909696579 38.0
-5.71842479000002e-09 -3.97519254684448 39.0
-11.9173161503539 -14.1116676330566 40.0
-24.7425931554772 -27.1329803466797 41.0
4.97985589750437 1.50881910324097 42.0
-32.1971692865044 -32.2347640991211 43.0
100.134876521504 92.1187362670898 44.0
100.134876194672 94.4784088134766 45.0
7.02279715335161 5.6803994178772 46.0
2.47956480345161 2.43925404548645 47.0
156.86109141 161.436309814453 48.0
150.74019255 151.165802001953 49.0
0 -2.44583010673523 50.0
79.795802952 76.5777435302734 51.0
88.902430361 85.7161865234375 52.0
81.021968268 80.1285095214844 53.0
81.021968268 80.2013702392578 54.0
81.021968268 80.0349426269531 55.0
1.080383174e-08 0.902844250202179 56.0
80.543219379 79.0610198974609 57.0
224.51095383 221.159454345703 58.0
250.08854625 250.688598632812 59.0
21.478895416 21.6838111877441 60.0
0 0.0651013255119324 61.0
137.005895469483 169.175033569336 0.0
137.005895469483 155.581268310547 1.0
135.516270716247 161.902816772461 2.0
130.7272009 156.904541015625 3.0
-5.8296371693e-10 4.80070352554321 4.0
-2.59751367000001e-09 6.96303176879883 5.0
-67.3460199805536 -62.1502380371094 6.0
80.2726303504377 77.3187561035156 7.0
47.6696531994377 39.2517852783203 8.0
32.6029771504377 28.2792701721191 9.0
97.1258292864377 96.2259216308594 10.0
85.4824485874377 85.5156021118164 11.0
82.9360884054377 89.9001770019531 12.0
82.9360884054377 82.386100769043 13.0
82.9360884054377 84.353271484375 14.0
1.2922127485e-08 0.417059361934662 15.0
1.2922127485e-08 0.272698760032654 16.0
3.915904464e-08 0.927337288856506 17.0
63.0490300392388 59.5019035339355 18.0
-63.0490298956565 -60.3414001464844 19.0
-63.0490298836565 -58.489818572998 20.0
-63.0490298036565 -59.6975135803223 21.0
85.8557698587399 80.5869522094727 22.0
147.811586740239 146.577682495117 23.0
113.384614360298 111.430938720703 24.0
24.9993712882984 20.3570289611816 25.0
233.667356598967 222.051284790039 26.0
233.667356598967 223.795104980469 27.0
27.5232060249666 27.0389232635498 28.0
261.190562618967 248.242050170898 29.0
261.190562618967 248.609664916992 30.0
46.7921314788945 47.291919708252 31.0
9.31710317827406 8.17745494842529 32.0
22.8067400546609 21.9532279968262 33.0
22.8067400546609 20.2594451904297 34.0
22.8067401625648 18.9529666900635 35.0
10.3268204819179 7.24479913711548 36.0
9.06730788291793 9.10525798797607 37.0
4.02955999999953e-10 2.29529619216919 38.0
4.02955989999882e-10 -0.975394785404205 39.0
-11.4667637913282 -19.0702209472656 40.0
-25.5173221166293 -26.342643737793 41.0
4.72456525226468 8.42488670349121 42.0
-32.9526050203647 -33.9198150634766 43.0
94.2014653713647 95.6677398681641 44.0
94.2014649233648 94.476806640625 45.0
6.88126306931484 4.57992601394653 46.0
1.96925799252877 1.61277008056641 47.0
137.00589546 177.432083129883 48.0
130.7272009 150.917098999023 49.0
0 -0.546056807041168 50.0
80.27263035 72.9873199462891 51.0
97.125829286 89.6365280151367 52.0
82.936088406 79.0650024414062 53.0
82.936088406 80.5479049682617 54.0
82.936088406 78.9662704467773 55.0
3.9159044646e-08 1.75368356704712 56.0
85.855769859 80.5921859741211 57.0
233.6673566 222.274505615234 58.0
261.19056262 250.529968261719 59.0
22.806740163 15.0579719543457 60.0
4.0295599212e-10 2.0735502243042 61.0
194.88548137 163.457046508789 0.0
194.88548137 166.252792358398 1.0
193.76144832 174.015045166016 2.0
188.97237819 170.795471191406 3.0
0 -0.307850897312164 4.0
0 -3.59379720687866 5.0
-60.518097083 -51.1522483825684 6.0
73.194678309 79.0123825073242 7.0
43.798588725 51.6629028320312 8.0
29.396089574 24.0503444671631 9.0
88.737171249 94.4666595458984 10.0
75.043716784 73.0774612426758 11.0
83.329338092 84.8226013183594 12.0
83.329338092 80.4830932617188 13.0
83.329338092 81.79443359375 14.0
0 -2.30506205558777 15.0
0 0.150099694728851 16.0
0 3.75373411178589 17.0
56.389960647 57.6626243591309 18.0
-56.389960657 -58.700984954834 19.0
-56.389960635 -58.0731430053711 20.0
-56.389960648 -60.586311340332 21.0
77.722591005 80.1325836181641 22.0
148.65920514 142.191390991211 23.0
115.20971987 115.137847900391 24.0
24.674873731 21.498218536377 25.0
226.38179616 226.1103515625 26.0
226.38179616 225.181594848633 27.0
23.207177487 35.499195098877 28.0
249.58897366 258.49267578125 29.0
249.58897366 248.625640869141 30.0
46.230960053 43.0743942260742 31.0
9.374021486 8.20328712463379 32.0
21.332630347 23.0570945739746 33.0
21.332630347 18.6942710876465 34.0
21.332630347 21.7621898651123 35.0
9.3443311439 7.69729900360107 36.0
7.9221594824 8.49557399749756 37.0
-1.8330635987e-12 -1.99160814285278 38.0
-1.8330635967e-12 8.35755348205566 39.0
-12.560672361 -16.6681900024414 40.0
-24.51022151 -27.7770767211914 41.0
5.1357273828 4.08682012557983 42.0
-32.022359071 -32.5144309997559 43.0
85.679242326 91.7135772705078 44.0
85.679242339 91.6370620727539 45.0
6.5350517346 7.277099609375 46.0
2.8318151813 2.73348569869995 47.0
194.88548138 169.927810668945 48.0
188.9723782 177.539215087891 49.0
0 0.0973289012908936 50.0
73.194678319 76.7121047973633 51.0
88.737171259 90.3965759277344 52.0
83.329338102 78.1466751098633 53.0
83.329338102 81.4420623779297 54.0
83.329338102 80.5947875976562 55.0
0 2.95475482940674 56.0
77.722591015 80.1952590942383 57.0
226.38179617 222.889541625977 58.0
249.58897367 250.532302856445 59.0
21.332630357 19.9082889556885 60.0
0 3.18520259857178 61.0
169.27010139 168.257202148438 0.0
169.27010139 162.539306640625 1.0
168.24091778 153.276321411133 2.0
163.45184806 154.579788208008 3.0
0 0.673179090023041 4.0
-1.0109135791e-11 12.9842929840088 5.0
-60.580282342 -49.4930038452148 6.0
73.161915127 78.5560302734375 7.0
57.171618964 52.1862983703613 8.0
15.990296153 25.2701835632324 9.0
85.605417207 77.7727661132812 10.0
72.318309279 71.4398193359375 11.0
85.75948681 74.6589202880859 12.0
85.75948681 82.6873016357422 13.0
85.75948681 72.3928985595703 14.0
0 -1.85192918777466 15.0
0 0.376586854457855 16.0
-1.7312707342e-11 0.0777302980422974 17.0
53.276537165 57.3253479003906 18.0
-53.276537185 -58.6821784973145 19.0
-53.276537163 -53.4306030273438 20.0
-53.276537175 -60.1398162841797 21.0
74.237173942 75.5637741088867 22.0
153.48066716 145.539642333984 23.0
116.99074502 109.855537414551 24.0
23.130346851 19.9287967681885 25.0
227.71784111 218.220672607422 26.0
227.71784111 213.761825561523 27.0
29.607142156 27.8440036773682 28.0
257.32498328 245.226104736328 29.0
257.32498328 247.878204345703 30.0
46.903289153 47.5566558837891 31.0
9.2969014174 6.83636569976807 32.0
20.960636757 18.6979789733887 33.0
20.960636757 22.3353538513184 34.0
20.960636757 17.944465637207 35.0
8.8454532874 9.34835338592529 36.0
7.4765488743 9.22372150421143 37.0
-3.5287760078e-11 2.05488443374634 38.0
-3.5287763168e-11 3.96840333938599 39.0
-11.080992706 -12.8041372299194 40.0
-24.932509321 -23.1585235595703 41.0
5.0863548628 -2.91485667228699 42.0
-32.733906015 -31.3253612518311 43.0
93.46051945 95.7999420166016 44.0
93.460519463 93.5414352416992 45.0
6.8313614437 2.56629920005798 46.0
2.5901924429 1.38604354858398 47.0
169.2701014 164.618606567383 48.0
163.45184807 164.858612060547 49.0
0 2.22699403762817 50.0
73.161915137 77.8220825195312 51.0
85.605417217 87.2107849121094 52.0
85.759486819 80.4459457397461 53.0
85.759486819 81.0049362182617 54.0
85.759486819 80.0435562133789 55.0
0 -1.23136425018311 56.0
74.237173952 76.26904296875 57.0
227.71784112 220.241577148438 58.0
257.32498329 243.803512573242 59.0
20.960636767 16.3977470397949 60.0
0 2.85984873771667 61.0
137.17687742 143.006607055664 0.0
137.17687742 143.027954101562 1.0
135.86355256 143.557556152344 2.0
131.0744823 135.807662963867 3.0
0 -2.16135358810425 4.0
0 0.342353284358978 5.0
-59.42861149 -59.2404556274414 6.0
69.705106679 70.1773910522461 7.0
65.43029471 62.4528312683105 8.0
4.2748119615 8.48783111572266 9.0
78.669161727 80.2065200805664 10.0
65.229460649 71.2585144042969 11.0
77.500548395 80.8047180175781 12.0
77.500548395 77.9000701904297 13.0
77.500548395 76.9198608398438 14.0
0 -0.64198762178421 15.0
0 0.28728449344635 16.0
0 -3.15485000610352 17.0
59.754945382 58.5015258789062 18.0
-59.754945394 -59.4383087158203 19.0
-59.754945372 -60.0649642944336 20.0
-59.754945384 -61.8801422119141 21.0
80.656571382 77.8813858032227 22.0
140.87903669 140.394561767578 23.0
106.41719675 107.357681274414 24.0
22.90053999 20.4083709716797 25.0
221.53560808 223.594223022461 26.0
221.53560808 221.987747192383 27.0
27.282645659 25.3551406860352 28.0
248.81825375 253.574676513672 29.0
248.81825375 252.408157348633 30.0
47.033013387 47.2299919128418 31.0
9.3131031344 9.18030452728271 32.0
20.901625993 20.4819431304932 33.0
20.901625993 21.1129970550537 34.0
20.901625993 20.8195400238037 35.0
9.3083381265 8.28715133666992 36.0
8.044767894 8.38281917572021 37.0
-2.5262790395e-09 -0.804296553134918 38.0
-2.5262785966e-09 0.113371789455414 39.0
-12.387792857 -14.5510091781616 40.0
-23.81592281 -27.5234241485596 41.0
4.9278776077 3.5162615776062 42.0
-31.108913619 -34.7394485473633 43.0
92.052766479 91.3064651489258 44.0
92.05276649 90.2343292236328 45.0
6.7273095346 3.00315618515015 46.0
2.6568330431 1.83220052719116 47.0
137.17687743 136.680145263672 48.0
131.07448231 127.609191894531 49.0
0 3.5344455242157 50.0
69.705106686 70.9859237670898 51.0
78.669161734 84.2529830932617 52.0
77.500548402 81.0479965209961 53.0
77.500548402 80.9212951660156 54.0
77.500548402 79.4541549682617 55.0
0 -0.0822644829750061 56.0
80.656571391 80.3950500488281 57.0
221.53560809 218.26057434082 58.0
248.81825375 246.469757080078 59.0
20.901626002 19.9398994445801 60.0
0 2.89454770088196 61.0
190.26764164 173.481658935547 0.0
190.26764164 161.25390625 1.0
188.80017147 164.760925292969 2.0
184.01110174 178.398300170898 3.0
0 -5.99856901168823 4.0
-9.7203544719e-13 -4.00283670425415 5.0
-70.466765617 -66.1257476806641 6.0
79.734416694 79.601448059082 7.0
49.539546511 40.5537300109863 8.0
30.194870174 23.0752201080322 9.0
88.072765278 78.9701385498047 10.0
75.265717193 69.8022155761719 11.0
78.883916273 74.3051681518555 12.0
78.883916273 78.6778030395508 13.0
78.883916273 79.0331115722656 14.0
0 0.545309603214264 15.0
0 0.141220808029175 16.0
-1.8110751101e-12 -0.951424539089203 17.0
54.960348035 58.3135948181152 18.0
-54.960348055 -59.837215423584 19.0
-54.960348033 -62.5448837280273 20.0
-54.960348046 -56.575553894043 21.0
76.266156353 73.7048568725586 22.0
152.7878799 141.825241088867 23.0
114.67837866 113.446746826172 24.0
25.376636054 20.3599510192871 25.0
229.05403627 220.529403686523 26.0
229.05403627 220.607437133789 27.0
23.882189651 23.6607971191406 28.0
252.93622593 246.982482910156 29.0
252.93622593 252.573532104492 30.0
46.653085158 48.0742340087891 31.0
9.4599891472 8.84025859832764 32.0
21.305808297 22.7263202667236 33.0
21.305808297 24.729305267334 34.0
21.305808297 20.3125534057617 35.0
10.14562924 8.99026298522949 36.0
8.9236384307 5.17169380187988 37.0
-2.8998450736e-12 -4.98981332778931 38.0
-2.8998450736e-12 -3.18509578704834 39.0
-12.585091277 -15.442910194397 40.0
-25.184987025 -26.2191543579102 41.0
4.8273671346 5.03851222991943 42.0
-32.540584449 -31.6707210540771 43.0
93.106679105 89.4441986083984 44.0
93.106679118 94.4637832641602 45.0
6.3593390654 6.37011814117432 46.0
2.7141111603 2.28955936431885 47.0
190.26764165 163.732177734375 48.0
184.01110175 176.035552978516 49.0
0 -0.829241573810577 50.0
79.734416704 76.8819732666016 51.0
88.072765288 83.5243453979492 52.0
78.883916283 79.7604217529297 53.0
78.883916283 76.5640029907227 54.0
78.883916283 80.81005859375 55.0
0 0.633165657520294 56.0
76.266156363 78.5821151733398 57.0
229.05403628 218.466491699219 58.0
252.93622594 246.579849243164 59.0
21.305808307 14.0568923950195 60.0
0 5.29197406768799 61.0
177.69958826 168.569854736328 0.0
177.69958826 170.10774230957 1.0
176.33211327 161.594146728516 2.0
171.54304355 167.123657226562 3.0
0 -2.27063083648682 4.0
0 2.58953046798706 5.0
-66.787878238 -61.1720275878906 6.0
78.318521313 76.5486145019531 7.0
54.104359793 56.6209106445312 8.0
24.21416151 24.7831363677979 9.0
86.177068852 88.4899749755859 10.0
73.346185666 77.4846954345703 11.0
80.992439828 82.542106628418 12.0
80.992439828 78.454460144043 13.0
80.992439828 80.8387145996094 14.0
0 -0.155265808105469 15.0
0 -0.386723577976227 16.0
0 2.39572644233704 17.0
59.065608975 59.3639945983887 18.0
-59.065608994 -59.5033226013184 19.0
-59.065608972 -58.9887466430664 20.0
-59.065608985 -60.4299507141113 21.0
80.289761899 78.4914398193359 22.0
147.59869815 140.49983215332 23.0
112.83621692 109.161079406738 24.0
23.657389004 21.39674949646 25.0
227.88846006 224.852081298828 26.0
227.88846006 224.099212646484 27.0
26.737351965 26.9970550537109 28.0
254.62581203 251.366394042969 29.0
254.62581203 252.631256103516 30.0
46.838664829 47.8081550598145 31.0
9.3511470062 8.7345724105835 32.0
21.224152904 21.9319934844971 33.0
21.224152904 18.4335823059082 34.0
21.224152902 20.0221424102783 35.0
9.2871526916 8.00479602813721 36.0
7.9786377407 7.33905076980591 37.0
0 -2.12422180175781 38.0
0 -2.02996301651001 39.0
-12.272556288 -10.887659072876 40.0
-25.390414751 -27.5586032867432 41.0
5.0713489167 3.97365665435791 42.0
-33.115321804 -31.3177127838135 43.0
101.46918301 97.0360565185547 44.0
101.46918303 94.8101959228516 45.0
7.1329787789 5.60208225250244 46.0
2.6174668473 2.49674654006958 47.0
177.69958827 169.465545654297 48.0
171.54304356 158.510986328125 49.0
0 -0.561467945575714 50.0
78.318521323 77.935546875 51.0
86.177068862 90.2946472167969 52.0
80.992439838 80.5961227416992 53.0
80.992439838 81.4355010986328 54.0
80.992439838 81.0632781982422 55.0
0 4.07161808013916 56.0
80.289761909 79.6881103515625 57.0
227.88846007 221.953201293945 58.0
254.62581204 249.424530029297 59.0
21.224152912 20.9004421234131 60.0
0 2.4765625 61.0
177.87051421 157.23014831543 0.0
177.87051421 157.381759643555 1.0
176.58236531 160.589782714844 2.0
171.79329558 177.796325683594 3.0
0 -0.380195677280426 4.0
0 2.49001169204712 5.0
-70.152873017 -64.4183883666992 6.0
82.089076129 72.6956176757812 7.0
58.057715537 52.3443183898926 8.0
24.031360582 22.2785320281982 9.0
88.710800526 84.0865707397461 10.0
75.321215037 73.7085418701172 11.0
83.795411448 82.3859024047852 12.0
83.795411448 81.4031219482422 13.0
83.795411448 81.3030319213867 14.0
0 -1.47425818443298 15.0
0 -0.0593032836914062 16.0
0 -1.15541481971741 17.0
59.545401561 58.3959541320801 18.0
-59.545401571 -59.1994895935059 19.0
-59.545401549 -59.5394592285156 20.0
-59.545401561 -60.1432685852051 21.0
80.863464785 81.8225021362305 22.0
144.88449461 146.062438964844 23.0
109.41983303 107.950012207031 24.0
23.179052413 21.1260395050049 25.0
225.7479594 224.376724243164 26.0
225.7479594 225.282440185547 27.0
26.483806182 25.1836280822754 28.0
252.2317656 255.779174804688 29.0
252.2317656 254.070877075195 30.0
46.799795988 46.4438858032227 31.0
9.2803769684 8.83085060119629 32.0
21.318063215 19.963981628418 33.0
21.318063215 14.7355928421021 34.0
21.318063214 20.9155406951904 35.0
9.2842497141 7.14387607574463 36.0
8.0500313372 7.57214069366455 37.0
0 -2.44457674026489 38.0
0 -4.08209705352783 39.0
-12.433583366 -11.7426996231079 40.0
-24.678115021 -22.7127265930176 41.0
5.0838124933 6.58776235580444 42.0
-32.277684946 -34.478199005127 43.0
99.650235695 97.0063934326172 44.0
99.650235709 94.0319061279297 45.0
7.1092798974 6.65397119522095 46.0
2.771167161 1.48266696929932 47.0
177.87051422 159.510528564453 48.0
171.79329559 168.209167480469 49.0
0 3.0337963104248 50.0
82.089076138 76.7830963134766 51.0
88.710800536 84.2613296508789 52.0
83.795411458 81.6314163208008 53.0
83.795411458 80.487190246582 54.0
83.795411458 79.5650253295898 55.0
0 1.36744618415833 56.0
80.863464795 78.8252334594727 57.0
225.74795941 223.198928833008 58.0
252.23176561 248.064102172852 59.0
21.318063224 25.8337898254395 60.0
0 2.03378462791443 61.0
155.281458406996 164.723602294922 0.0
155.281458406996 169.633575439453 1.0
154.06458915 162.53938293457 2.0
149.27551943 146.269790649414 3.0
0 0.607545793056488 4.0
-7.0221453781e-09 6.4477391242981 5.0
-61.167999284 -61.3488464355469 6.0
72.5986798364597 72.1886825561523 7.0
44.4383293444597 51.7384071350098 8.0
28.1603504914597 22.1149864196777 9.0
87.1846541644597 85.0534133911133 10.0
74.3175940644597 73.0812225341797 11.0
78.2997631064597 82.7881546020508 12.0
78.2997631064597 79.8143997192383 13.0
78.2997631064597 80.5021514892578 14.0
0 -1.12125277519226 15.0
0 -0.232740819454193 16.0
-1.069592519e-09 -2.21503901481628 17.0
55.9944917727053 58.3006248474121 18.0
-55.9944917801895 -59.6059989929199 19.0
-55.9944917581895 -58.3044090270996 20.0
-55.9944917671895 -58.698860168457 21.0
78.198987255 78.4128799438477 22.0
142.997879917705 142.281890869141 23.0
108.606804254632 109.871170043945 24.0
23.5489866636316 19.6922092437744 25.0
221.196867183992 224.401565551758 26.0
221.196867183992 221.410217285156 27.0
26.722837163992 24.5310153961182 28.0
247.919704343992 256.898254394531 29.0
247.919704343992 259.085296630859 30.0
46.509657719 46.4189910888672 31.0
9.3083665075 9.06631278991699 32.0
22.2044954916797 20.2022380828857 33.0
22.2044954916797 23.46901512146 34.0
22.204495492 21.6495208740234 35.0
9.8280805117 10.8560190200806 36.0
8.5625755292 7.25412368774414 37.0
-9.372443382e-09 -1.03730440139771 38.0
-9.372443382e-09 0.784080862998962 39.0
-11.4604337620948 -16.5521907806396 40.0
-24.5295376665388 -21.3309574127197 41.0
4.94825125921641 6.64780855178833 42.0
-31.8549439142164 -34.8529396057129 43.0
95.5435055892164 91.192024230957 44.0
95.5435055683409 93.7182846069336 45.0
6.9889899182 7.25674057006836 46.0
2.3984148507 1.86671304702759 47.0
155.28145841 165.99772644043 48.0
149.27551944 154.608505249023 49.0
0 0.0370653867721558 50.0
72.598679836 76.538703918457 51.0
87.184654164 85.0021514892578 52.0
78.299763106 81.7453536987305 53.0
78.299763106 81.2339782714844 54.0
78.299763106 79.1463623046875 55.0
0 3.74673938751221 56.0
78.198987259 79.3663330078125 57.0
221.19686718 220.091186523438 58.0
247.91970434 249.679077148438 59.0
22.204495495 19.1708488464355 60.0
0 0.291259109973907 61.0
172.35119383 159.020690917969 0.0
172.35119383 165.14372253418 1.0
171.04252036 167.288101196289 2.0
166.25345064 164.488922119141 3.0
0 3.23561382293701 4.0
-1.6451690929e-09 1.03974580764771 5.0
-67.343509727 -65.0764465332031 6.0
77.809450777 73.6737670898438 7.0
48.520173488 45.0125274658203 8.0
29.289277284 21.7065258026123 9.0
84.481446789 85.7255249023438 10.0
71.797616593 73.9356842041016 11.0
77.801111627 83.127082824707 12.0
77.801111627 81.5435791015625 13.0
77.801111627 79.5764007568359 14.0
0 0.172659039497375 15.0
0 -0.872308194637299 16.0
0 -1.5846095085144 17.0
55.597991371 58.2832679748535 18.0
-55.597991386 -58.6324501037598 19.0
-55.597991364 -58.4351844787598 20.0
-55.597991376 -60.4586410522461 21.0
76.660753262 78.5278091430664 22.0
148.09630106 145.922897338867 23.0
112.40370209 109.140083312988 24.0
22.360263802 21.9482555389404 25.0
224.75705433 222.470092773438 26.0
224.75705433 228.308547973633 27.0
26.438768977 25.9064178466797 28.0
251.19582331 254.967498779297 29.0
251.19582331 254.843292236328 30.0
46.817069354 45.7465209960938 31.0
9.2854026436 8.63901329040527 32.0
21.062761888 19.4740371704102 33.0
21.062761888 17.5669384002686 34.0
21.062761886 20.3960399627686 35.0
9.2913647793 8.1458158493042 36.0
8.077666971 6.71571588516235 37.0
-6.187134435e-09 1.45391309261322 38.0
-6.1871447462e-09 -3.98621129989624 39.0
-11.405068051 -12.2312393188477 40.0
-23.750940197 -26.5812206268311 41.0
5.2624891682 7.09080123901367 42.0
-31.519132071 -33.1057281494141 43.0
97.076324 97.6797790527344 44.0
97.076324009 96.2917175292969 45.0
6.9328729838 4.90256690979004 46.0
2.6690883236 2.86559367179871 47.0
172.35119383 158.640472412109 48.0
166.25345065 163.674835205078 49.0
0 -2.84831929206848 50.0
77.809450781 76.0805282592773 51.0
84.481446793 87.8623199462891 52.0
77.801111632 81.5950317382812 53.0
77.801111632 81.9937591552734 54.0
77.801111632 79.4980621337891 55.0
0 1.92705988883972 56.0
76.660753271 78.6751174926758 57.0
224.75705434 221.070724487305 58.0
251.19582332 251.855819702148 59.0
21.062761894 19.4337997436523 60.0
0 1.26650261878967 61.0
177.164929737811 145.897430419922 0.0
177.164929737811 136.317611694336 1.0
175.88982533 162.760299682617 2.0
170.42519005 166.559188842773 3.0
2.9770452519 3.60111021995544 4.0
2.97704526003012 -3.37046766281128 5.0
-65.9413272866301 -67.0601959228516 6.0
77.3983446872054 67.9698181152344 7.0
45.0092029392054 53.9463882446289 8.0
32.3891417482054 31.293794631958 9.0
88.9452763112054 75.3668441772461 10.0
76.8882769402054 72.1569747924805 11.0
81.0739971372054 83.5133895874023 12.0
81.0739971372054 84.3966445922852 13.0
81.0739971372054 73.6970977783203 14.0
4.8866851593e-09 1.52226376533508 15.0
2.9722335533e-09 0.587432086467743 16.0
1.126127652e-08 1.24907660484314 17.0
58.119589377112 59.9065971374512 18.0
-58.1195893303081 -58.9923934936523 19.0
-58.1195893123081 -56.1766014099121 20.0
-58.1195892873081 -61.3385772705078 21.0
79.3386956807042 72.7973785400391 22.0
149.162636056112 148.292358398438 23.0
112.106907344575 106.704330444336 24.0
16.7110341091401 19.5246353149414 25.0
228.501331735622 227.213592529297 26.0
228.501331735622 229.274230957031 27.0
24.4810018976216 20.7159557342529 28.0
252.982333635622 252.531646728516 29.0
252.982333635622 252.748474121094 30.0
46.7869275631666 44.9259338378906 31.0
9.24433597897557 9.02713489532471 32.0
21.2191063928904 20.6559467315674 33.0
21.2191063928904 17.6619129180908 34.0
21.2191064360364 20.1464672088623 35.0
10.1662825331726 13.2780389785767 36.0
8.83780550317256 6.20301914215088 37.0
-5.84900388999996e-09 4.65851879119873 38.0
-5.84900389000002e-09 12.8796453475952 39.0
-11.6423309271541 -20.788745880127 40.0
-25.1727594668672 -22.8817882537842 41.0
5.57838051097118 4.32786178588867 42.0
-33.0743632721712 -31.8539752960205 43.0
98.7414251751712 95.1267852783203 44.0
98.7414250041712 88.6044464111328 45.0
6.47860133337289 10.9594011306763 46.0
2.52230631777289 2.98115372657776 47.0
177.16492974 131.970474243164 48.0
170.42519006 153.490020751953 49.0
2.9770452601 1.03455686569214 50.0
77.398344687 72.7456512451172 51.0
88.945276311 87.6454620361328 52.0
81.073997138 81.5939025878906 53.0
81.073997138 81.1960601806641 54.0
81.073997138 80.0464859008789 55.0
1.1261276519e-08 1.31798958778381 56.0
79.33869568 77.6551055908203 57.0
228.50133173 222.934204101562 58.0
252.98233363 246.414916992188 59.0
21.219106436 19.8000469207764 60.0
0 3.52136397361755 61.0
164.20276864 163.19841003418 0.0
164.20276864 164.499816894531 1.0
163.06270683 157.81721496582 2.0
158.27363711 146.860214233398 3.0
0 0.137821853160858 4.0
-4.7304324708e-13 6.21064901351929 5.0
-70.360866016 -68.39453125 6.0
82.45371021 75.0832748413086 7.0
54.738800388 51.278678894043 8.0
27.714909813 27.4431266784668 9.0
85.12516422 87.2895660400391 10.0
72.212095503 73.4793548583984 11.0
80.041125683 83.0516052246094 12.0
80.041125683 77.1731033325195 13.0
80.041125683 80.1431198120117 14.0
0 -1.52249312400818 15.0
0 -0.402916252613068 16.0
0 2.76452565193176 17.0
58.639912167 58.6222152709961 18.0
-58.639912187 -58.4037742614746 19.0
-58.639912165 -59.7175369262695 20.0
-58.639912177 -58.9224281311035 21.0
79.819003128 76.4960327148438 22.0
145.39530095 141.598327636719 23.0
111.93445704 111.526794433594 24.0
22.341212649 20.7231636047363 25.0
225.21430409 227.936447143555 26.0
225.21430409 224.811859130859 27.0
27.13711532 24.2449035644531 28.0
252.35141942 255.792373657227 29.0
252.35141942 256.354400634766 30.0
46.955438179 46.1532211303711 31.0
9.3240830286 9.17787170410156 32.0
21.17909094 21.5747776031494 33.0
21.17909094 20.370641708374 34.0
21.17909094 19.3668098449707 35.0
9.4871421981 6.98397445678711 36.0
8.1860699939 7.89634895324707 37.0
-2.3842303557e-11 0.717953443527222 38.0
-2.3842314775e-11 -3.21129751205444 39.0
-12.285819928 -12.0952882766724 40.0
-24.84712441 -27.1547241210938 41.0
5.3443386371 5.1508412361145 42.0
-32.887895515 -32.5927658081055 43.0
94.924355183 93.7793731689453 44.0
94.924355196 94.7898712158203 45.0
6.9459168683 5.95544195175171 46.0
2.5387236124 2.68239188194275 47.0
164.20276865 166.141250610352 48.0
158.27363712 145.87060546875 49.0
0 3.83543682098389 50.0
82.45371022 77.4585494995117 51.0
85.12516423 88.4177703857422 52.0
80.041125693 79.8197631835938 53.0
80.041125693 80.918212890625 54.0
80.041125693 80.4292373657227 55.0
0 4.10649251937866 56.0
79.819003138 79.9147872924805 57.0
225.2143041 220.197814941406 58.0
252.35141943 250.566970825195 59.0
21.17909095 19.3396854400635 60.0
0 1.98774838447571 61.0
161.88219684 158.280822753906 0.0
161.88219684 165.1845703125 1.0
160.7158787 159.91845703125 2.0
155.92680877 161.442016601562 3.0
0 -0.331950008869171 4.0
0 1.5471498966217 5.0
-63.068304676 -63.4603691101074 6.0
73.039586843 77.5407180786133 7.0
48.277384052 51.2517776489258 8.0
24.762202782 25.3994197845459 9.0
89.247766671 86.6848526000977 10.0
76.595423798 75.2800827026367 11.0
79.434258383 78.2172088623047 12.0
79.434258383 79.0038833618164 13.0
79.434258383 79.1201019287109 14.0
0 -4.28533029556274 15.0
0 0.0433433651924133 16.0
0 0.00333774089813232 17.0
59.087797573 58.7646865844727 18.0
-59.087797585 -58.7436485290527 19.0
-59.087797563 -58.3572959899902 20.0
-59.087797575 -60.6889762878418 21.0
80.902956997 79.0813064575195 22.0
144.3162998 143.185440063477 23.0
111.34231504 109.438385009766 24.0
21.508732657 21.0511131286621 25.0
225.21925681 224.150390625 26.0
225.21925681 220.809646606445 27.0
28.040931744 27.7706718444824 28.0
253.26018856 258.102935791016 29.0
253.26018856 257.372802734375 30.0
46.724913224 46.9565963745117 31.0
9.4036503757 9.0781364440918 32.0
21.815159416 19.6459693908691 33.0
21.815159416 17.6097812652588 34.0
21.815159415 20.9502716064453 35.0
9.2403216298 7.7659330368042 36.0
7.9542119034 7.52775382995605 37.0
-1.158760055e-09 -0.309494316577911 38.0
-1.1587516502e-09 -0.372691452503204 39.0
-11.868792232 -11.8866405487061 40.0
-24.302136471 -24.7554397583008 41.0
5.3914041986 4.06352806091309 42.0
-32.380084555 -32.7065086364746 43.0
95.663631366 93.3230667114258 44.0
95.663631378 94.6852874755859 45.0
6.8895931705 7.13876152038574 46.0
2.5419606567 2.59749627113342 47.0
161.88219685 163.223236083984 48.0
155.92680878 152.729614257812 49.0
0 2.01570343971252 50.0
73.039586851 76.4024963378906 51.0
89.247766679 90.3725280761719 52.0
79.434258392 80.5820770263672 53.0
79.434258392 81.1669769287109 54.0
79.434258392 79.0086212158203 55.0
0 -0.501103222370148 56.0
80.902957007 79.8517074584961 57.0
225.21925682 220.891616821289 58.0
253.26018857 249.555419921875 59.0
21.815159425 21.4329566955566 60.0
0 2.38003373146057 61.0
146.72313099 161.733428955078 0.0
146.72313099 162.831970214844 1.0
145.29244784 160.000183105469 2.0
139.86032113 145.458465576172 3.0
0 -5.06838178634644 4.0
0 7.58450508117676 5.0
-67.137093675 -64.1567764282227 6.0
77.896205617 77.1391906738281 7.0
48.783684087 54.646541595459 8.0
29.11252152 36.0154609680176 9.0
87.383489333 86.0259780883789 10.0
75.092272684 79.2085647583008 11.0
74.509772946 85.7099533081055 12.0
74.509772946 75.4863052368164 13.0
74.509772946 66.8195190429688 14.0
0 1.31463229656219 15.0
0 0.071042537689209 16.0
-4.0920264633e-11 -1.88080906867981 17.0
53.511775643 58.0321655273438 18.0
-53.511775663 -56.9864387512207 19.0
-53.511775641 -55.9021530151367 20.0
-53.511775653 -59.355411529541 21.0
73.875072594 75.882682800293 22.0
147.90164771 133.067474365234 23.0
113.57033813 107.032661437988 24.0
22.300690383 18.625675201416 25.0
221.77672032 225.232772827148 26.0
221.77672032 219.633773803711 27.0
25.446476634 26.2026271820068 28.0
247.22319696 254.798446655273 29.0
247.22319696 257.230987548828 30.0
46.472305843 46.7516288757324 31.0
9.4349234458 8.84954929351807 32.0
20.363296931 17.2399368286133 33.0
20.363296931 14.5404167175293 34.0
20.363296931 18.0513820648193 35.0
9.2306662327 7.96103954315186 36.0
7.8710867021 7.74818658828735 37.0
-1.9260925813e-11 3.55230307579041 38.0
-1.9258196653e-11 2.49777126312256 39.0
-11.155378323 -14.9308891296387 40.0
-24.695785723 -24.0677604675293 41.0
5.4303515565 7.303635597229 42.0
-32.477598697 -30.9490070343018 43.0
89.942362791 88.894157409668 44.0
89.942362804 91.1482849121094 45.0
7.2074565423 5.51829719543457 46.0
2.4713835175 2.31978058815002 47.0
146.723131 162.779174804688 48.0
139.86032114 149.138931274414 49.0
0 -2.35939502716064 50.0
77.896205627 78.7068939208984 51.0
87.383489343 83.3788986206055 52.0
74.509772956 81.2249755859375 53.0
74.509772956 79.4948043823242 54.0
74.509772956 79.3221282958984 55.0
0 0.05499666929245 56.0
73.875072604 79.2693634033203 57.0
221.77672033 220.34602355957 58.0
247.22319697 250.294876098633 59.0
20.363296941 19.592529296875 60.0
0 2.22631454467773 61.0
179.0259914 167.867523193359 0.0
179.0259914 169.914840698242 1.0
177.63719215 162.947357177734 2.0
172.84812242 159.465423583984 3.0
0 1.23682689666748 4.0
-1.6437543894e-09 8.29115676879883 5.0
-75.453213806 -57.9083786010742 6.0
86.770330124 81.298698425293 7.0
61.475266559 63.5732765197754 8.0
25.295063563 14.5845184326172 9.0
100.23620079 90.6003875732422 10.0
86.616819164 80.4937591552734 11.0
86.520684917 84.5791397094727 12.0
86.520684917 76.8081817626953 13.0
86.520684917 81.4239959716797 14.0
0 0.457417786121368 15.0
0 -0.252119600772858 16.0
0 1.77099585533142 17.0
61.738444825 59.7940216064453 18.0
-61.7384448397873 -58.776683807373 19.0
-61.7384448177873 -61.1812133789062 20.0
-61.7384448287873 -62.9905166625977 21.0
83.736352842 77.8951187133789 22.0
145.11867839 144.94075012207 23.0
112.89956417 111.735084533691 24.0
22.869741607 21.9257678985596 25.0
228.85503125 227.931289672852 26.0
228.85503125 231.5771484375 27.0
28.799322849 27.4030132293701 28.0
257.6543541 256.598205566406 29.0
257.6543541 251.171661376953 30.0
46.823592904 48.198299407959 31.0
9.2379758031 8.81124019622803 32.0
21.997908017 17.6100025177002 33.0
21.997908017 13.8290586471558 34.0
21.997908015 19.3900890350342 35.0
10.281267362 8.54693603515625 36.0
8.9920742976 7.43687105178833 37.0
-8.0125466756e-09 3.14848232269287 38.0
-8.0125466758e-09 2.45889091491699 39.0
-12.573206301 -12.8977022171021 40.0
-25.359831413 -23.8995094299316 41.0
4.6742227472 5.44534778594971 42.0
-32.345537561 -34.6331634521484 43.0
80.385801002 84.0253448486328 44.0
80.385801004 89.978401184082 45.0
6.8927810889 8.03003883361816 46.0
2.8484500481 2.74822854995728 47.0
179.02599141 168.596237182617 48.0
172.84812243 156.800552368164 49.0
0 1.13210916519165 50.0
86.770330127 75.9741973876953 51.0
100.23620079 90.2667541503906 52.0
86.52068492 82.1306762695312 53.0
86.52068492 80.6188201904297 54.0
86.52068492 80.9032669067383 55.0
0 2.46472024917603 56.0
83.73635285 79.0859222412109 57.0
228.85503125 220.663604736328 58.0
257.6543541 248.657943725586 59.0
21.997908022 17.7893772125244 60.0
0 2.13268899917603 61.0
168.25413216 163.703048706055 0.0
168.25413216 166.955123901367 1.0
166.96508107 163.168151855469 2.0
162.17601135 159.626495361328 3.0
0 -0.361185610294342 4.0
0 1.17716062068939 5.0
-63.41669957 -60.8032302856445 6.0
75.409932039 71.5934295654297 7.0
52.231959359 53.9571533203125 8.0
23.17797267 16.4676246643066 9.0
89.170029919 87.0826187133789 10.0
76.145483321 75.4902572631836 11.0
81.452853383 83.8852844238281 12.0
81.452853383 81.8851013183594 13.0
81.452853383 80.0561904907227 14.0
0 -1.22152018547058 15.0
0 -0.567316234111786 16.0
0 -0.678081095218658 17.0
56.307697478 58.8465423583984 18.0
-56.307697488 -59.506706237793 19.0
-56.307697466 -59.3068885803223 20.0
-56.307697478 -59.4495544433594 21.0
77.218718021 77.9145278930664 22.0
143.95237449 141.255584716797 23.0
109.85729556 108.105400085449 24.0
22.649492798 19.7362270355225 25.0
221.17109252 222.299957275391 26.0
221.17109252 222.729751586914 27.0
27.237550476 24.0346603393555 28.0
248.40864301 250.727874755859 29.0
248.40864301 253.511825561523 30.0
46.898277479 47.9354667663574 31.0
9.3583715969 8.58743953704834 32.0
20.911020533 22.2232913970947 33.0
20.911020533 26.2610092163086 34.0
20.911020533 21.3616542816162 35.0
9.2368244193 9.68162250518799 36.0
7.9890702266 7.69323778152466 37.0
0 -0.654069244861603 38.0
0 -5.4309778213501 39.0
-12.061718657 -12.6461620330811 40.0
-24.538454492 -26.3531341552734 41.0
4.8869322107 6.55457210540771 42.0
-31.868504413 -34.4001731872559 43.0
103.20181173 93.9459457397461 44.0
103.20181174 94.787712097168 45.0
7.1321760323 5.85695362091064 46.0
2.5841040129 2.19469809532166 47.0
168.25413217 170.172744750977 48.0
162.17601136 157.350357055664 49.0
0 -0.146920800209045 50.0
75.409932049 77.2582855224609 51.0
89.170029929 88.5906600952148 52.0
81.452853393 81.3527069091797 53.0
81.452853393 81.0411758422852 54.0
81.452853393 79.0702590942383 55.0
0 0.637151896953583 56.0
77.218718031 79.0145645141602 57.0
221.17109253 221.216323852539 58.0
248.40864302 246.466995239258 59.0
20.911020543 19.7313442230225 60.0
0 0.638004124164581 61.0
152.64297682 160.036682128906 0.0
152.64297682 166.128677368164 1.0
151.30880137 160.864929199219 2.0
146.51973165 155.787445068359 3.0
0 1.6928825378418 4.0
-9.7575826113e-12 -0.64812558889389 5.0
-60.171253998 -64.0317535400391 6.0
71.909917528 73.1893157958984 7.0
48.758386357 52.1474494934082 8.0
23.151531161 21.7900924682617 9.0
86.826849954 87.0083618164062 10.0
73.552810128 76.3201599121094 11.0
80.602393349 83.9651336669922 12.0
80.602393349 80.768196105957 13.0
80.602393349 81.0190582275391 14.0
0 -0.84912496805191 15.0
0 -0.124878108501434 16.0
-2.0478330647e-11 1.56776702404022 17.0
57.787708491 57.8376617431641 18.0
-57.787708511 -58.7761268615723 19.0
-57.787708489 -58.6178245544434 20.0
-57.787708501 -60.1469497680664 21.0
78.972809862 77.8163452148438 22.0
147.55108975 141.160171508789 23.0
114.00553285 109.683776855469 24.0
23.832830305 21.1275196075439 25.0
226.52389963 225.416320800781 26.0
226.52389963 224.487258911133 27.0
27.250651875 23.9672393798828 28.0
253.77455151 255.55924987793 29.0
253.77455151 255.986831665039 30.0
46.750312798 47.26416015625 31.0
9.3327220691 8.8815860748291 32.0
21.185101351 20.4296798706055 33.0
21.185101351 13.5004377365112 34.0
21.185101351 21.0598030090332 35.0
9.3633052591 7.69415855407715 36.0
8.0955714756 7.50305700302124 37.0
-3.2383671916e-11 -1.62402009963989 38.0
-3.2383671917e-11 -0.647268831729889 39.0
-11.546597478 -16.6982536315918 40.0
-24.700979458 -24.6478672027588 41.0
4.9555451378 6.78781080245972 42.0
-32.24502116 -33.9722900390625 43.0
95.439969452 92.4297256469727 44.0
95.439969466 95.0690383911133 45.0
6.998990478 6.08623933792114 46.0
2.5429915587 1.70197439193726 47.0
152.64297683 162.449752807617 48.0
146.51973166 150.245803833008 49.0
0 -1.06036114692688 50.0
71.909917538 74.5766372680664 51.0
86.826849964 88.0537414550781 52.0
80.602393359 80.2794189453125 53.0
80.602393359 80.9284744262695 54.0
80.602393359 80.7080764770508 55.0
0 4.07350015640259 56.0
78.972809872 79.7845153808594 57.0
226.52389964 221.917831420898 58.0
253.77455152 249.932388305664 59.0
21.185101361 24.6385650634766 60.0
0 0.51330977678299 61.0
196.25928605 170.616348266602 0.0
196.25928605 140.56169128418 1.0
194.98984575 175.298400878906 2.0
190.20077603 184.880981445312 3.0
0 4.48895025253296 4.0
-1.1025513597e-11 3.22876024246216 5.0
-62.224177208 -62.1030883789062 6.0
71.4221583 79.1224594116211 7.0
52.673291002 53.5543556213379 8.0
18.748867288 25.9505043029785 9.0
87.907677454 86.1435012817383 10.0
75.168505269 74.3984146118164 11.0
78.685257149 87.799430847168 12.0
78.685257149 79.9794845581055 13.0
78.685257149 76.2600402832031 14.0
0 -7.71416091918945 15.0
0 -0.568047344684601 16.0
-1.6271889453e-11 0.328090131282806 17.0
61.529291107 58.3423233032227 18.0
-61.529291127 -60.678882598877 19.0
-61.529291105 -59.9003715515137 20.0
-61.529291118 -59.5348892211914 21.0
83.769541051 77.5344314575195 22.0
134.1633165 136.856567382812 23.0
103.86938131 116.203414916992 24.0
22.661968589 18.5166797637939 25.0
217.93285756 227.738342285156 26.0
217.93285756 216.371383666992 27.0
26.271042688 23.033145904541 28.0
244.20390026 249.96012878418 29.0
244.20390026 247.862243652344 30.0
47.086535579 46.4017448425293 31.0
9.2366801175 8.74038982391357 32.0
22.240249923 20.5689754486084 33.0
22.240249923 15.0594282150269 34.0
22.240249923 22.4124031066895 35.0
9.9373883071 6.38421630859375 36.0
8.6141042032 7.55092334747314 37.0
-3.3061659441e-11 3.18837666511536 38.0
-3.3061659088e-11 -2.79774022102356 39.0
-11.391859936 -13.326868057251 40.0
-25.263762923 -25.4034442901611 41.0
5.5724343252 -0.886982142925262 42.0
-33.517121199 -32.901252746582 43.0
96.276116611 91.1942901611328 44.0
96.276116624 94.0519943237305 45.0
6.6539984995 3.76974868774414 46.0
2.5354246784 2.8068163394928 47.0
196.25928606 140.336059570312 48.0
190.20077604 174.783416748047 49.0
0 -3.15927267074585 50.0
71.42215831 73.8201141357422 51.0
87.907677464 88.4184341430664 52.0
78.685257158 81.4348602294922 53.0
78.685257158 79.4645843505859 54.0
78.685257158 79.3870239257812 55.0
0 1.76400089263916 56.0
83.769541061 78.8210067749023 57.0
217.93285757 215.713241577148 58.0
244.20390027 246.907348632812 59.0
22.240249933 22.404224395752 60.0
0 2.33555960655212 61.0
191.243038255948 165.662857055664 0.0
191.243038255948 167.668350219727 1.0
190.32096084 165.057495117188 2.0
185.53189109 156.271255493164 3.0
0 4.90902328491211 4.0
-5.44047916199999e-09 -1.66890668869019 5.0
-45.5635682614216 -67.9886322021484 6.0
54.3885125749757 73.5510482788086 7.0
50.0372004249757 41.3194313049316 8.0
4.35131214957568 25.2292671203613 9.0
88.7666922029757 85.1084747314453 10.0
76.3079239779757 64.9908599853516 11.0
89.3707401439757 72.9733581542969 12.0
89.3707401439757 64.0062255859375 13.0
89.3707401439757 69.6000289916992 14.0
2.0098395543e-09 -5.98749113082886 15.0
0 -2.63715839385986 16.0
2.81071386999998e-09 -0.442050874233246 17.0
50.4898133690776 49.1035003662109 18.0
-50.4898133474635 -42.1806907653809 19.0
-50.4898133284635 -60.1069602966309 20.0
-50.4898133174635 -64.052375793457 21.0
69.2838634339394 72.5167388916016 22.0
155.384083261078 121.85546875 23.0
118.602963048847 100.087715148926 24.0
22.4522169698471 28.0475234985352 25.0
224.667946701895 223.685684204102 26.0
224.667946701895 227.735916137695 27.0
21.990834925895 30.0927200317383 28.0
246.658781621895 253.58935546875 29.0
246.658781621895 257.313018798828 30.0
45.2491018492328 47.4955253601074 31.0
8.28486747263445 10.2102012634277 32.0
18.7940501162649 21.5288677215576 33.0
18.7940501162649 12.9021005630493 34.0
18.7940501427847 13.3974771499634 35.0
7.73818669866077 9.74925804138184 36.0
6.59564943346077 7.27200317382812 37.0
-7.46848417e-09 -1.12456011772156 38.0
-7.46848418000001e-09 6.2295298576355 39.0
-13.0784300497318 -16.3183097839355 40.0
-25.3852108871063 -18.1431083679199 41.0
5.8857516506464 12.5454225540161 42.0
-33.5461830611464 -36.0137214660645 43.0
79.0155630011464 93.2400131225586 44.0
79.0155627922581 87.0110168457031 45.0
7.0169525334 -0.996220767498016 46.0
2.5266999857 1.71630835533142 47.0
191.24303825 169.471435546875 48.0
185.5318911 185.305572509766 49.0
0 -0.0965622067451477 50.0
54.388512575 70.0879745483398 51.0
88.766692203 83.1926040649414 52.0
89.370740144 76.4519958496094 53.0
89.370740144 67.652587890625 54.0
89.370740144 78.296760559082 55.0
2.8107138724e-09 8.16756057739258 56.0
69.283863434 76.8240966796875 57.0
224.6679467 218.090225219727 58.0
246.65878163 249.275146484375 59.0
18.794050143 15.2432851791382 60.0
0 2.71892619132996 61.0
161.055547296419 159.463455200195 0.0
161.055547296419 162.629730224609 1.0
159.7047314 158.621063232422 2.0
154.91566166 157.988540649414 3.0
0 -1.75240874290466 4.0
-7.453676137e-09 2.51288604736328 5.0
-63.9326951505597 -63.0221405029297 6.0
74.7974153094589 76.3322601318359 7.0
54.7860929434589 52.7191047668457 8.0
20.0113223654589 22.1771850585938 9.0
91.8208956394589 85.5688705444336 10.0
78.7059598874589 73.1144485473633 11.0
82.5901756594589 82.0715713500977 12.0
82.5901756594589 80.1103744506836 13.0
82.5901756594589 80.0095672607422 14.0
1.7015838358e-09 -2.05273962020874 15.0
0 0.0374583005905151 16.0
7.6026848e-10 -2.62952876091003 17.0
56.2263147674321 58.7478218078613 18.0
-56.2263147546884 -58.8654479980469 19.0
-56.2263147346884 -59.4848327636719 20.0
-56.2263147296884 -57.7499961853027 21.0
77.9706539095294 78.8263092041016 22.0
147.778237220432 143.442260742188 23.0
114.45285942304 111.93155670166 24.0
22.4516058450402 20.9458847045898 25.0
225.748891122837 227.337097167969 26.0
225.748891122837 225.230346679688 27.0
26.8586585698372 23.5172023773193 28.0
252.607549692837 254.312774658203 29.0
252.607549692837 253.873870849609 30.0
46.8536817168521 47.0026321411133 31.0
9.34627627379195 8.66091918945312 32.0
21.7443391796791 20.7118816375732 33.0
21.7443391796791 24.2901020050049 34.0
21.7443392010203 21.3826026916504 35.0
9.27549209805957 8.16342067718506 36.0
8.02416121325957 6.9852991104126 37.0
-8.05791201999999e-09 0.73761385679245 38.0
-8.05791202000001e-09 1.02350926399231 39.0
-11.8306311023442 -14.2825622558594 40.0
-24.2073342721486 -24.4874401092529 41.0
5.38142997811576 2.44996285438538 42.0
-32.0131755107158 -32.0721092224121 43.0
95.5397704097158 92.3600845336914 44.0
95.5397703337158 93.9741744995117 45.0
6.5841793262 6.52369499206543 46.0
2.4599891433 2.56682538986206 47.0
161.05554729 163.086273193359 48.0
154.91566167 154.632369995117 49.0
0 0.563391387462616 50.0
74.79741531 77.252799987793 51.0
91.82089564 83.9520797729492 52.0
82.59017566 81.6953659057617 53.0
82.59017566 79.8443145751953 54.0
82.59017566 78.4787902832031 55.0
7.6026847694e-10 0.29662412405014 56.0
77.970653909 79.2238235473633 57.0
225.74889113 222.439300537109 58.0
252.6075497 249.560455322266 59.0
21.744339201 22.864875793457 60.0
0 1.13002014160156 61.0
153.45630845 157.846054077148 0.0
153.45630845 162.785339355469 1.0
152.27388972 169.745529174805 2.0
147.48481999 157.547988891602 3.0
0 0.228997409343719 4.0
-1.0348165221e-09 4.09080219268799 5.0
-67.527517278 -61.0228996276855 6.0
78.864930217 78.2647018432617 7.0
63.598385629 53.0704765319824 8.0
15.266544583 29.6661071777344 9.0
86.988488038 86.5590286254883 10.0
75.413175072 77.8539886474609 11.0
77.949070768 85.9935531616211 12.0
77.949070768 79.7432861328125 13.0
77.949070768 72.9773406982422 14.0
0 -8.07441902160645 15.0
0 -0.585292160511017 16.0
0 0.52325302362442 17.0
56.957479582 59.480712890625 18.0
-56.957479597 -59.4113616943359 19.0
-56.957479575 -58.966064453125 20.0
-56.957479587 -58.2117614746094 21.0
80.099169183 78.8520660400391 22.0
152.32242761 137.579177856445 23.0
115.72624204 112.943344116211 24.0
22.85984251 19.6951160430908 25.0
232.4215968 221.7119140625 26.0
232.4215968 223.896621704102 27.0
23.581579218 29.0908660888672 28.0
256.00317603 247.303695678711 29.0
256.00317603 249.910003662109 30.0
46.710474883 46.961597442627 31.0
9.3890758442 8.73222732543945 32.0
23.141689594 18.4091815948486 33.0
23.141689594 16.3139667510986 34.0
23.141689593 19.9925861358643 35.0
10.684884215 9.23737049102783 36.0
9.4227954612 8.00438213348389 37.0
-3.4712976217e-09 -5.89797782897949 38.0
-3.4712976217e-09 1.25412392616272 39.0
-11.889861599 -12.8110990524292 40.0
-25.252705442 -22.574462890625 41.0
5.2702049723 5.13136577606201 42.0
-32.885090471 -32.994571685791 43.0
97.705632693 94.4482803344727 44.0
97.705632704 91.5599136352539 45.0
6.7160299085 5.30523872375488 46.0
1.8921580034 2.86065030097961 47.0
153.45630846 165.56640625 48.0
147.48482 149.422058105469 49.0
0 0.0673092007637024 50.0
78.864930224 77.2592926025391 51.0
86.988488045 93.1648788452148 52.0
77.949070774 82.3914642333984 53.0
77.949070774 81.4169769287109 54.0
77.949070774 80.3136291503906 55.0
0 3.70716738700867 56.0
80.099169192 79.9301605224609 57.0
232.42159681 216.034515380859 58.0
256.00317603 247.22607421875 59.0
23.141689601 21.4480133056641 60.0
0 2.68804907798767 61.0
185.245458306128 159.532821655273 0.0
185.245458306128 167.833633422852 1.0
183.82623146 161.442016601562 2.0
179.03716173 170.586273193359 3.0
0 2.04981184005737 4.0
-8.7965670243e-09 -0.382828652858734 5.0
-68.514197289 -65.5893783569336 6.0
78.6385475979547 75.234245300293 7.0
58.1390437949547 33.7385101318359 8.0
20.4995038039547 22.5668029785156 9.0
94.6960532909547 84.6592102050781 10.0
82.4064072059547 71.7513427734375 11.0
87.4280572209547 71.5389709472656 12.0
87.4280572209547 69.588752746582 13.0
87.4280572209547 81.1365203857422 14.0
0 -0.444423735141754 15.0
0 -0.276923000812531 16.0
-7.144111616e-09 4.16800308227539 17.0
54.7339621658844 56.8717918395996 18.0
-54.7339621789321 -57.9688568115234 19.0
-54.7339621579321 -57.8627891540527 20.0
-54.7339621659321 -59.3919906616211 21.0
76.560310706 79.5906219482422 22.0
143.070669456884 136.409515380859 23.0
107.393556583606 115.960952758789 24.0
20.1882973886055 18.54958152771 25.0
219.630980162255 234.888473510742 26.0
219.630980162255 228.203186035156 27.0
28.3147752272555 27.6894435882568 28.0
247.945755392255 258.563385009766 29.0
247.945755392255 256.76513671875 30.0
46.475382013 44.7290382385254 31.0
9.2282724698 9.926438331604 32.0
21.8263485433898 19.3122634887695 33.0
21.8263485433898 13.1911602020264 34.0
21.826348546 19.0191383361816 35.0
9.0903726535 8.81616306304932 36.0
7.7992130176 5.50540447235107 37.0
-9.624066696e-09 -2.54686307907104 38.0
-9.624066695e-09 2.94918489456177 39.0
-11.943086791966 -17.569149017334 40.0
-23.8599887047736 -27.3843746185303 41.0
4.86263780496227 2.92028379440308 42.0
-30.9337897289623 -30.3326473236084 43.0
92.2688145729623 89.8268051147461 44.0
92.2688145496027 92.6708221435547 45.0
7.0436495545 5.05848217010498 46.0
2.4422558251 2.31274843215942 47.0
185.24545831 164.111892700195 48.0
179.03716174 173.894561767578 49.0
0 -0.408442437648773 50.0
78.638547598 76.5820541381836 51.0
94.696053291 87.012939453125 52.0
87.428057221 76.8561477661133 53.0
87.428057221 73.7853012084961 54.0
87.428057221 79.1652755737305 55.0
0 2.25526452064514 56.0
76.56031071 78.782585144043 57.0
219.63098016 223.641265869141 58.0
247.94575539 248.84228515625 59.0
21.826348547 24.085132598877 60.0
0 3.1074812412262 61.0
153.30263323 157.701889038086 0.0
153.30263323 163.485336303711 1.0
152.12155095 158.967864990234 2.0
147.33248123 152.431945800781 3.0
0 1.27010607719421 4.0
-1.036009053e-11 0.263977468013763 5.0
-63.79326684 -65.2791519165039 6.0
74.066080199 76.924430847168 7.0
49.257535433 52.0238418579102 8.0
24.808544756 23.8600730895996 9.0
93.076678886 88.3865966796875 10.0
80.167385514 77.4255065917969 11.0
85.355792054 84.5462417602539 12.0
85.355792054 79.5964279174805 13.0
85.355792054 78.1725692749023 14.0
0 -0.716274321079254 15.0
0 0.0241878032684326 16.0
-1.4869952162e-11 1.20923495292664 17.0
57.791749773 57.9205207824707 18.0
-57.791749793 -58.5988349914551 19.0
-57.791749771 -59.2294044494629 20.0
-57.791749784 -59.3059730529785 21.0
79.485008835 77.5516052246094 22.0
142.89008365 141.308029174805 23.0
108.50551575 109.606300354004 24.0
22.87624861 20.1146602630615 25.0
222.37509249 219.152282714844 26.0
222.37509249 221.40234375 27.0
27.183619735 26.7830619812012 28.0
249.55871224 253.792144775391 29.0
249.55871224 253.35041809082 30.0
46.565146504 47.2012214660645 31.0
9.3494813596 8.54714584350586 32.0
21.693259042 19.1272125244141 33.0
21.693259042 22.287784576416 34.0
21.693259042 21.2558193206787 35.0
9.4541947719 9.72253131866455 36.0
8.1468047521 7.80622625350952 37.0
-3.2054701962e-11 2.47348833084106 38.0
-3.2054684988e-11 -2.23412752151489 39.0
-11.873492658 -13.1899290084839 40.0
-24.930356219 -25.1794090270996 41.0
4.9926361721 5.86267232894897 42.0
-32.318741974 -33.5793838500977 43.0
95.276261874 95.6772155761719 44.0
95.276261887 94.6682357788086 45.0
6.8642655567 7.65038299560547 46.0
2.4654665248 2.83246159553528 47.0
153.30263324 162.748443603516 48.0
147.33248124 146.297180175781 49.0
0 -1.09081435203552 50.0
74.066080209 74.88525390625 51.0
93.076678896 90.5929107666016 52.0
85.355792064 80.7928314208984 53.0
85.355792064 81.6294174194336 54.0
85.355792064 79.7192916870117 55.0
0 2.63580918312073 56.0
79.485008845 79.080207824707 57.0
222.3750925 219.417053222656 58.0
249.55871225 248.845855712891 59.0
21.693259052 17.8646469116211 60.0
0 1.70618319511414 61.0
152.13801784 160.553802490234 0.0
152.13801784 161.833297729492 1.0
150.90884694 163.175201416016 2.0
146.11977722 156.844467163086 3.0
0 3.93334722518921 4.0
-2.558694878e-09 6.0013256072998 5.0
-70.322475265 -63.8794288635254 6.0
81.5290715321181 84.2096633911133 7.0
42.9697558131181 52.7967262268066 8.0
38.5593157191181 24.5920848846436 9.0
82.3682895711181 82.2136383056641 10.0
69.4524357701181 72.5929412841797 11.0
76.3274687121181 80.1995010375977 12.0
76.3274687121181 80.5531158447266 13.0
76.3274687121181 77.0881042480469 14.0
0 -1.70671510696411 15.0
0 -1.2283775806427 16.0
-8.046193682e-09 -1.75717973709106 17.0
61.346163736 59.012939453125 18.0
-61.3461637561771 -57.7952651977539 19.0
-61.3461637351771 -59.6315765380859 20.0
-61.3461637451771 -59.0630950927734 21.0
83.474591253 71.3625335693359 22.0
135.69447911 141.654067993164 23.0
103.72775045 108.750061035156 24.0
22.563284881 22.1817684173584 25.0
219.16907036629 217.624130249023 26.0
219.16907036629 212.050918579102 27.0
25.9949833802901 28.6721954345703 28.0
245.16405374629 248.882690429688 29.0
245.16405374629 251.868179321289 30.0
46.217623333 44.815372467041 31.0
9.3150012558 9.49200916290283 32.0
22.1284275138155 23.1292304992676 33.0
22.1284275138155 13.7601089477539 34.0
22.12842751 22.7078151702881 35.0
9.3343165979 8.2629861831665 36.0
8.1445255899 6.12377262115479 37.0
-9.7955484612e-09 -0.974577605724335 38.0
-9.7955484607e-09 -0.0193242430686951 39.0
-12.487309077 -17.6934013366699 40.0
-25.759730672 -25.1294078826904 41.0
4.6588769402 4.23417329788208 42.0
-33.26286977 -32.3006820678711 43.0
89.720286132854 96.5370330810547 44.0
89.720286156 94.5148010253906 45.0
7.1432086578 8.31339263916016 46.0
2.713426924 -0.610271394252777 47.0
152.13801784 157.163238525391 48.0
146.11977723 162.414581298828 49.0
0 -4.18948936462402 50.0
81.529071533 75.3400726318359 51.0
82.368289571 87.3412094116211 52.0
76.327468712 81.3555145263672 53.0
76.327468712 77.264778137207 54.0
76.327468712 77.3903579711914 55.0
0 -1.44669532775879 56.0
83.47459126 80.1200714111328 57.0
219.16907037 208.652069091797 58.0
245.16405375 248.349761962891 59.0
22.128427516 21.3807601928711 60.0
0 4.27111053466797 61.0
137.17687742 143.006454467773 0.0
137.17687742 143.032775878906 1.0
135.86355256 143.523300170898 2.0
131.0744823 135.743881225586 3.0
0 -2.09159421920776 4.0
0 0.341849207878113 5.0
-59.42861149 -59.2315139770508 6.0
69.705106679 70.17724609375 7.0
65.43029471 62.4528312683105 8.0
4.2748119615 8.48783111572266 9.0
78.669161727 80.2065200805664 10.0
65.229460649 71.2585144042969 11.0
77.500548395 80.8047180175781 12.0
77.500548395 77.9000701904297 13.0
77.500548395 76.9198608398438 14.0
0 -0.64198762178421 15.0
0 0.28728449344635 16.0
0 -3.15485000610352 17.0
59.754945382 58.5015258789062 18.0
-59.754945394 -59.4383087158203 19.0
-59.754945372 -60.0649642944336 20.0
-59.754945384 -61.8801422119141 21.0
80.656571382 77.8813858032227 22.0
140.87903669 140.394561767578 23.0
106.41719675 107.357681274414 24.0
22.90053999 20.4083709716797 25.0
221.53560808 223.594223022461 26.0
221.53560808 221.987747192383 27.0
27.282645659 25.3551406860352 28.0
248.81825375 253.574676513672 29.0
248.81825375 252.408157348633 30.0
47.033013387 47.2299919128418 31.0
9.3131031344 9.18030452728271 32.0
20.901625993 20.4819431304932 33.0
20.901625993 21.1129970550537 34.0
20.901625993 20.8195400238037 35.0
9.3083381265 8.28715133666992 36.0
8.044767894 8.38281917572021 37.0
-2.5262790395e-09 -0.804296553134918 38.0
-2.5262785966e-09 0.113371789455414 39.0
-12.387792857 -14.5510091781616 40.0
-23.81592281 -27.5234241485596 41.0
4.9278776077 3.5162615776062 42.0
-31.108913619 -34.7394485473633 43.0
92.052766479 91.3064651489258 44.0
92.05276649 90.2343292236328 45.0
6.7273095346 3.00315618515015 46.0
2.6568330431 1.83220052719116 47.0
137.17687743 136.682907104492 48.0
131.07448231 127.506469726562 49.0
0 3.54901957511902 50.0
69.705106686 70.986442565918 51.0
78.669161734 84.2529830932617 52.0
77.500548402 81.0479965209961 53.0
77.500548402 80.9212951660156 54.0
77.500548402 79.4541549682617 55.0
0 -0.0822644829750061 56.0
80.656571391 80.3950500488281 57.0
221.53560809 218.26057434082 58.0
248.81825375 246.469757080078 59.0
20.901626002 19.9398994445801 60.0
0 2.89454770088196 61.0
137.13846934 158.91291809082 0.0
137.13846934 150.986724853516 1.0
135.92197635 150.1650390625 2.0
131.13290634 152.210891723633 3.0
0 -1.09210205078125 4.0
0 -2.01123023033142 5.0
-62.441890661 -62.5267066955566 6.0
71.799187529 67.8026504516602 7.0
47.084906293 23.0055961608887 8.0
24.714281233 13.8530769348145 9.0
86.43888905 86.5032730102539 10.0
73.106768476 77.1076354980469 11.0
78.767521507 81.5571594238281 12.0
78.767521507 82.4412002563477 13.0
78.767521507 80.7606506347656 14.0
0 1.71697807312012 15.0
0 -0.635009825229645 16.0
0 -2.03317499160767 17.0
54.585219982 58.1784477233887 18.0
-54.5852200009043 -58.3511734008789 19.0
-54.5852199789043 -58.432933807373 20.0
-54.5852199909043 -60.0339241027832 21.0
76.313412818 77.3753967285156 22.0
141.23234166 137.775650024414 23.0
107.09010202 113.330032348633 24.0
23.091353976 21.0643310546875 25.0
217.54575449 227.284393310547 26.0
217.54575449 224.992324829102 27.0
27.455041784 21.9919414520264 28.0
245.00079627 251.160049438477 29.0
245.00079627 252.739929199219 30.0
46.882024227 49.9916877746582 31.0
9.448843687 8.53668594360352 32.0
21.728192832 17.1581039428711 33.0
21.728192832 11.1834526062012 34.0
21.728192829 19.9378547668457 35.0
9.712307849 9.15047359466553 36.0
8.4187401487 7.09975290298462 37.0
-7.328613513e-09 -2.42366600036621 38.0
-7.3286136455e-09 -4.41936159133911 39.0
-10.750384377 -13.5948066711426 40.0
-24.712875033 -25.3574619293213 41.0
5.0015155386 5.0268726348877 42.0
-32.37968635 -33.4421043395996 43.0
91.675758805 91.4501342773438 44.0
91.675758806 95.6742858886719 45.0
6.4608248496 6.4703893661499 46.0
1.6880388828 2.72294592857361 47.0
137.13846935 149.683654785156 48.0
131.13290635 130.442520141602 49.0
0 3.11720371246338 50.0
71.799187531 72.9834136962891 51.0
86.438889053 92.2690277099609 52.0
78.76752151 81.7139129638672 53.0
78.76752151 79.0022125244141 54.0
78.76752151 79.8092346191406 55.0
0 1.41558957099915 56.0
76.313412826 78.9837341308594 57.0
217.54575449 219.962600708008 58.0
245.00079628 247.450592041016 59.0
21.728192837 22.0046710968018 60.0
0 2.05504369735718 61.0
131.32341614 168.637130737305 0.0
131.32341614 159.243423461914 1.0
130.26628725 155.461029052734 2.0
125.47721753 141.508331298828 3.0
0 -16.8846244812012 4.0
-1.1931313088e-09 6.18940353393555 5.0
-47.784117128 -54.1045951843262 6.0
56.649653909 72.1643371582031 7.0
50.208874528 65.6240310668945 8.0
6.4407793798 9.73790264129639 9.0
85.735166622 92.4845809936523 10.0
73.867476036 83.8279495239258 11.0
75.083477257 106.161712646484 12.0
75.083477257 72.7754516601562 13.0
75.083477257 76.1397171020508 14.0
0 1.6951105594635 15.0
0 -1.37529897689819 16.0
-4.4620188946e-09 1.01799583435059 17.0
66.553355537 59.9158401489258 18.0
-66.5533555603686 -58.5431747436523 19.0
-66.5533555383686 -60.2339248657227 20.0
-66.5533555503686 -63.6286277770996 21.0
87.426637304 71.3088455200195 22.0
138.58906602 145.905334472656 23.0
105.34441988 102.457893371582 24.0
26.135633908 24.5328769683838 25.0
226.01570334 221.958221435547 26.0
226.01570334 221.103713989258 27.0
25.955894251 21.9739894866943 28.0
251.97159759 250.971939086914 29.0
251.97159759 250.041580200195 30.0
46.542898628 48.2532081604004 31.0
9.4492233321 6.80576801300049 32.0
20.873281758 16.5797939300537 33.0
20.873281758 11.966986656189 34.0
20.873281756 20.2482891082764 35.0
9.6298339596 10.7235174179077 36.0
8.2675355695 9.29043102264404 37.0
-8.271034304e-09 -0.60243946313858 38.0
-8.2710352079e-09 0.305939853191376 39.0
-12.089553687 -10.0653676986694 40.0
-25.051731592 -29.0459213256836 41.0
5.7708426268 4.80828857421875 42.0
-33.517053181 -33.0807685852051 43.0
106.46933719 98.3003005981445 44.0
106.4693372 90.2534790039062 45.0
5.9808473206 6.50777339935303 46.0
2.2352429495 1.693354845047 47.0
131.32341615 167.746017456055 48.0
125.47721754 150.912811279297 49.0
0 -0.565988004207611 50.0
56.649653911 71.0344085693359 51.0
85.735166624 74.5326385498047 52.0
75.083477258 75.2894668579102 53.0
75.083477258 68.7667999267578 54.0
75.083477258 76.7282867431641 55.0
0 3.10320281982422 56.0
87.426637311 78.7432098388672 57.0
226.01570334 219.242767333984 58.0
251.97159759 252.736083984375 59.0
20.873281763 20.0467071533203 60.0
0 4.09624433517456 61.0
172.92832751 162.751068115234 0.0
172.92832751 167.998458862305 1.0
171.78894318 167.349792480469 2.0
166.99987346 163.631652832031 3.0
0 0.32995218038559 4.0
-1.2877477402e-09 0.600756943225861 5.0
-69.048158693 -65.8662185668945 6.0
80.332961755 77.5103149414062 7.0
49.177364591 48.7197799682617 8.0
31.155597164 26.1453685760498 9.0
83.873112543 88.6293029785156 10.0
70.451885255 75.8163604736328 11.0
78.059056003 81.7220306396484 12.0
78.059056003 78.1977005004883 13.0
78.059056003 78.6708526611328 14.0
0 -0.783433258533478 15.0
0 0.0497881770133972 16.0
-5.644279346e-09 2.75825953483582 17.0
57.948703265 58.8503494262695 18.0
-57.948703287781 -59.971061706543 19.0
-57.948703265781 -58.1465492248535 20.0
-57.948703276781 -59.4423370361328 21.0
78.51377712 77.4166259765625 22.0
147.0574902 143.959701538086 23.0
112.59455345 107.817680358887 24.0
22.203032012 18.21875 25.0
225.571267329062 228.530731201172 26.0
225.571267329062 226.827301025391 27.0
27.1922402540623 24.4481792449951 28.0
252.763507579062 257.836578369141 29.0
252.763507579062 252.518524169922 30.0
46.557847631 44.6832466125488 31.0
9.4009572528 7.98283910751343 32.0
20.565073849 20.1297035217285 33.0
20.565073849 17.409086227417 34.0
20.565073845 20.5113697052002 35.0
9.1342989068 7.96131420135498 36.0
7.9328692318 7.65216827392578 37.0
-9.6413118754e-09 1.13102126121521 38.0
-9.6413119444e-09 -3.46826887130737 39.0
-11.764605552 -15.0157032012939 40.0
-24.722855528 -26.6903896331787 41.0
5.264062295 4.85135936737061 42.0
-32.507321484 -32.9000129699707 43.0
95.346343424 91.6533737182617 44.0
95.346343423 93.4915771484375 45.0
7.2518008897 6.6089391708374 46.0
2.6318067649 2.56226420402527 47.0
172.92832752 163.716827392578 48.0
166.99987347 159.916687011719 49.0
0 1.80834984779358 50.0
80.332961756 77.7092056274414 51.0
83.873112544 88.8013153076172 52.0
78.059056004 81.4548797607422 53.0
78.059056004 81.2985229492188 54.0
78.059056004 80.5078735351562 55.0
0 3.05376291275024 56.0
78.513777127 79.5195846557617 57.0
225.57126733 224.293655395508 58.0
252.76350758 248.963012695312 59.0
20.565073852 21.8464279174805 60.0
0 1.61009359359741 61.0
36.968321715 160.366714477539 0.0
36.968321715 163.302017211914 1.0
4.7442468957e-14 153.796630859375 2.0
7.3915904828 139.541427612305 3.0
204.05803751 -3.28587341308594 4.0
590.92071305 1.3857901096344 5.0
582.05463804 -60.6388969421387 6.0
571.80687349 75.6256408691406 7.0
528.591894373889 46.8555679321289 8.0
526.200687417322 22.7334861755371 9.0
526.38392379 91.9388122558594 10.0
524.66413002 77.0415725708008 11.0
519.25742636 81.7472839355469 12.0
93.641379642 79.3833465576172 13.0
50.240245805 82.3139343261719 14.0
13.600848744 -0.0155409574508667 15.0
0 -0.00416749715805054 16.0
0.0049798427585 3.35134220123291 17.0
0 57.2610206604004 18.0
0 -58.360221862793 19.0
0.0058198025145 -58.4182510375977 20.0
0 -59.963794708252 21.0
0.0057910890569 79.0705642700195 22.0
0 139.623809814453 23.0
0.0059149176932 112.224617004395 24.0
93.641379642 21.2510833740234 25.0
43.548168396 222.812515258789 26.0
-1.4460118017 220.299438476562 27.0
13.600848744 25.7935523986816 28.0
12.154836943 256.247619628906 29.0
-37.839371968 256.160675048828 30.0
-50.240245805 45.4642372131348 31.0
12.400873836 8.78755950927734 32.0
-50.093211246 20.1801090240479 33.0
5.5158310645 19.2806758880615 34.0
1.0007206917 21.1165599822998 35.0
-2.8415121526 8.71727466583252 36.0
0.12208759713 7.50342035293579 37.0
-3.8843438953 -2.09091186523438 38.0
-3.638521889 -1.33939743041992 39.0
-4.8637294528 -13.2849006652832 40.0
-2.1392946711 -24.6767520904541 41.0
-4.1358353588 3.91112995147705 42.0
0 -32.6285552978516 43.0
0 92.3550415039062 44.0
0 95.4485092163086 45.0
0 6.64146423339844 46.0
0 2.43197274208069 47.0
0 161.573913574219 48.0
15.15148833 141.619903564453 49.0
21.81879063 0.896890223026276 50.0
36.968321715 76.3170928955078 51.0
0.90271466161 91.0124435424805 52.0
17.443472465 81.4464111328125 53.0
0.83775624047 81.7278823852539 54.0
19.52484925 80.6322860717773 55.0
0.88545659744 3.77111339569092 56.0
-4.3653574652 78.8316421508789 57.0
-0.81921922947 218.256439208984 58.0
15.159491785 249.127014160156 59.0
0.7709054218 23.3009185791016 60.0
21.80882993 2.07745409011841 61.0
194.54488019 159.075927734375 0.0
194.54488019 167.64811706543 1.0
193.07913495 172.870574951172 2.0
188.29006523 184.538208007812 3.0
-3.4156646434e-11 1.07188034057617 4.0
-4.0391911542e-11 4.49218559265137 5.0
-71.5300739 -52.615966796875 6.0
80.461981986 76.9665756225586 7.0
58.448847751 56.5962257385254 8.0
22.013134226 5.20493221282959 9.0
81.628850304 86.8225860595703 10.0
68.37745927 78.4542083740234 11.0
81.66393832 77.2426223754883 12.0
81.66393832 80.2858047485352 13.0
81.66393832 83.1369323730469 14.0
0 -2.635422706604 15.0
0 -1.01033854484558 16.0
-5.9942699058e-11 7.24216365814209 17.0
57.911027946 59.1430130004883 18.0
-57.911027966 -59.2003288269043 19.0
-57.911027944 -56.9994735717773 20.0
-57.911027957 -58.2809448242188 21.0
79.327813461 76.30908203125 22.0
145.64584376 142.290679931641 23.0
111.92990402 112.969482421875 24.0
21.755866618 20.6269683837891 25.0
224.97365723 228.639572143555 26.0
224.97365723 221.946014404297 27.0
24.236855424 24.835916519165 28.0
249.21051267 251.029800415039 29.0
249.21051267 255.203216552734 30.0
46.760932419 43.572639465332 31.0
9.4650710176 8.89524078369141 32.0
21.416785495 15.0644464492798 33.0
21.416785495 4.17405462265015 34.0
21.416785494 20.1440906524658 35.0
10.043361444 4.83081340789795 36.0
8.754619378 6.76093864440918 37.0
0 -7.24033641815186 38.0
0 1.41309213638306 39.0
-11.210280846 -19.4171848297119 40.0
-23.224039804 -19.4423809051514 41.0
4.8599470037 5.6995267868042 42.0
-30.554353162 -35.7832946777344 43.0
89.362904497 86.7814865112305 44.0
89.36290451 95.391716003418 45.0
7.0625740537 5.25438022613525 46.0
2.6841415901 2.62718939781189 47.0
194.5448802 161.648590087891 48.0
188.29006524 176.408309936523 49.0
0 3.36089086532593 50.0
80.461981996 76.0813598632812 51.0
81.628850314 90.2776260375977 52.0
81.66393833 80.2112350463867 53.0
81.66393833 78.5968170166016 54.0
81.66393833 80.4939270019531 55.0
0 1.24942493438721 56.0
79.327813471 81.2722854614258 57.0
224.97365724 220.016830444336 58.0
249.21051268 249.139907836914 59.0
21.416785504 23.1984577178955 60.0
0 3.21474361419678 61.0
156.041740118704 164.227416992188 0.0
156.041740118704 164.024047851562 1.0
154.86377368 157.737014770508 2.0
150.07470392 152.013427734375 3.0
0 -1.46193861961365 4.0
-5.13944870000001e-09 0.00509363412857056 5.0
-59.0942599620726 -62.1293601989746 6.0
70.0105627637541 74.2977752685547 7.0
47.8191686267541 55.9872169494629 8.0
22.1913941367541 21.1517658233643 9.0
85.7399816587541 88.411750793457 10.0
73.4435332487541 75.4782943725586 11.0
79.0318684497541 82.3558502197266 12.0
79.0318684497541 81.7325057983398 13.0
79.0318684497541 82.3169860839844 14.0
4.6386986821e-09 -1.097323179245 15.0
2.7687428355e-09 -0.301229178905487 16.0
1.059187854e-08 0.950678050518036 17.0
60.5040296949568 58.7327537536621 18.0
-60.5040296486311 -59.5685386657715 19.0
-60.5040296306311 -59.1882667541504 20.0
-60.5040296066311 -62.0662460327148 21.0
82.0376625801227 78.7727203369141 22.0
149.106724886957 141.720809936523 23.0
113.003166401196 113.464294433594 24.0
23.215941281196 19.901237487793 25.0
231.144387467409 224.130065917969 26.0
231.144387467409 224.189666748047 27.0
25.0531014924086 24.041374206543 28.0
256.197488957409 257.387237548828 29.0
256.197488957409 252.98908996582 30.0
46.7530805726177 45.1330757141113 31.0
9.30285158507924 8.79094219207764 32.0
21.5336329732178 20.7552623748779 33.0
21.5336329732178 19.3896923065186 34.0
21.5336330165704 20.7399482727051 35.0
9.61546373964747 7.12272024154663 36.0
8.32611508954747 7.52221870422363 37.0
-5.93317749999999e-09 -1.12284302711487 38.0
-5.93317749999999e-09 -3.0559287071228 39.0
-11.8106011533156 -11.2067155838013 40.0
-25.2170438946546 -27.637279510498 41.0
5.12244986286174 2.83657360076904 42.0
-32.8387053674617 -31.3802356719971 43.0
101.50071291868 93.8650741577148 44.0
101.500713071462 92.6988983154297 45.0
6.6091380735435 4.83456993103027 46.0
2.4936369324435 1.68995547294617 47.0
156.04174012 162.648071289062 48.0
150.07470392 147.097885131836 49.0
0 -0.695102751255035 50.0
70.010562764 74.3794021606445 51.0
85.739981658 89.6062316894531 52.0
79.031868449 81.1115798950195 53.0
79.031868449 81.1199951171875 54.0
79.031868449 80.1611328125 55.0
1.059187854e-08 0.709782898426056 56.0
82.03766258 80.305419921875 57.0
231.14438746 221.11083984375 58.0
256.19748896 247.962997436523 59.0
21.533633016 20.1235504150391 60.0
0 2.21874952316284 61.0
150.31740659 163.973083496094 0.0
150.31740659 166.144378662109 1.0
149.17586712 154.478424072266 2.0
144.3867974 149.130065917969 3.0
0 2.26004672050476 4.0
-8.5030818286e-10 2.15266847610474 5.0
-61.140284149 -55.0037689208984 6.0
72.70335599 79.3717422485352 7.0
54.720363685 47.7660713195801 8.0
17.982992299 23.3718814849854 9.0
88.181408553 87.3057403564453 10.0
75.282658081 76.9413681030273 11.0
80.857151728 82.3808135986328 12.0
80.857151728 79.8690185546875 13.0
80.857151728 81.5380020141602 14.0
0 -2.51617741584778 15.0
0 -0.729720413684845 16.0
0 0.605931460857391 17.0
60.317631118 58.0951118469238 18.0
-60.317631132 -60.4323348999023 19.0
-60.31763111 -58.4515495300293 20.0
-60.317631122 -58.8321876525879 21.0
82.102500529 77.5609512329102 22.0
148.38733839 144.538330078125 23.0
114.84676461 109.781112670898 24.0
22.338930785 18.4482097625732 25.0
230.48983893 225.580032348633 26.0
230.48983893 225.569686889648 27.0
25.647239124 26.7828483581543 28.0
256.13707806 253.130508422852 29.0
256.13707806 252.883102416992 30.0
46.497367864 46.5226287841797 31.0
9.3663425153 8.53235054016113 32.0
21.784869403 19.2298927307129 33.0
21.784869403 20.3403854370117 34.0
21.784869402 21.4746532440186 35.0
9.4097445505 8.63479709625244 36.0
8.0900347315 6.70832967758179 37.0
-2.873877607e-09 -1.91395711898804 38.0
-2.873877607e-09 -1.12903666496277 39.0
-11.319439517 -14.8916072845459 40.0
-23.649136006 -24.7447204589844 41.0
5.2264805986 3.93278050422668 42.0
-31.381344977 -32.857982635498 43.0
100.4061621 94.2885894775391 44.0
100.40616211 94.8221130371094 45.0
6.9364250907 5.55480766296387 46.0
2.6332686323 2.15771412849426 47.0
150.3174066 164.663589477539 48.0
144.38679741 147.081817626953 49.0
0 -2.87885665893555 50.0
72.703355997 74.0565414428711 51.0
88.18140856 90.7114791870117 52.0
80.857151734 80.6112670898438 53.0
80.857151734 80.7744598388672 54.0
80.857151734 80.0585021972656 55.0
0 4.10533475875854 56.0
82.102500538 80.6623992919922 57.0
230.48983894 221.549652099609 58.0
256.13707807 249.496765136719 59.0
21.784869411 22.802661895752 60.0
0 1.01528573036194 61.0
147.982033057583 157.456283569336 0.0
147.982033057583 151.960418701172 1.0
146.808413813287 153.322540283203 2.0
142.01934397 150.531326293945 3.0
-1.55484310406e-09 -0.626340210437775 4.0
-1.55484309999997e-09 -2.83180117607117 5.0
-64.9693040339936 -64.3129272460938 6.0
77.6547829144777 74.1870651245117 7.0
61.2786926634778 55.4030647277832 8.0
16.3760902514777 7.20984888076782 9.0
97.5261141374777 89.7391510009766 10.0
83.6472874924777 79.6450042724609 11.0
85.2981411144777 83.8642730712891 12.0
85.2981411144777 76.4624557495117 13.0
85.2981411144777 81.5436782836914 14.0
2.3328021736e-08 -2.03496789932251 15.0
2.3328021735e-08 -1.6052393913269 16.0
5.464350236e-08 -2.72059202194214 17.0
56.2518018559467 59.0783615112305 18.0
-56.2518018797166 -57.0891418457031 19.0
-56.2518018707166 -59.2854194641113 20.0
-56.2518017707166 -58.115608215332 21.0
77.1875821073724 73.3798065185547 22.0
153.454771802806 145.008895874023 23.0
115.220596133508 115.750228881836 24.0
21.7005584165075 20.8819332122803 25.0
230.642353915167 228.381210327148 26.0
230.642353915167 229.708221435547 27.0
26.8883420391666 21.3647422790527 28.0
257.530695955167 253.23681640625 29.0
257.530695955167 255.336776733398 30.0
46.7385281475469 45.537841796875 31.0
9.28724497493877 8.4355525970459 32.0
20.9357803369809 22.6376686096191 33.0
20.9357803369809 19.0790691375732 34.0
20.935780467613 20.4264030456543 35.0
8.86054244871795 8.39873600006104 36.0
7.60121916461794 6.85336446762085 37.0
2.62465480000001e-09 1.1959273815155 38.0
2.62465480000001e-09 -1.17553234100342 39.0
-11.6034306368583 -12.5643882751465 40.0
-24.5599220844463 -26.7411251068115 41.0
4.75986482786468 1.92820739746094 42.0
-31.7734218810648 -31.6524333953857 43.0
89.4696777832616 86.4044494628906 44.0
89.4696781210649 91.9600372314453 45.0
7.05809588635643 4.443039894104 46.0
2.46335994485643 2.26466846466064 47.0
147.98203306 156.097946166992 48.0
142.01934397 152.830352783203 49.0
0 -3.76001691818237 50.0
77.654782914 76.0241928100586 51.0
97.526114137 92.0755157470703 52.0
85.298141114 81.5651702880859 53.0
85.298141114 80.2131118774414 54.0
85.298141114 79.9203643798828 55.0
5.4643502359e-08 3.07315587997437 56.0
77.187582108 79.4883346557617 57.0
230.64235391 222.583419799805 58.0
257.53069595 249.491134643555 59.0
20.935780468 19.6116352081299 60.0
2.6246548239e-09 3.64096260070801 61.0
178.1092976 154.399490356445 0.0
178.1092976 161.102279663086 1.0
176.7463671 176.901947021484 2.0
171.95729738 163.166961669922 3.0
0 3.73396444320679 4.0
-6.4967660651e-12 7.82714748382568 5.0
-58.696921404 -67.9038314819336 6.0
70.527717722 84.6482009887695 7.0
47.80691563 57.1935081481934 8.0
22.720802082 28.8980674743652 9.0
88.972143521 88.0253372192383 10.0
76.376668321 78.5444717407227 11.0
80.773868337 85.5110931396484 12.0
80.773868337 81.4156036376953 13.0
80.773868337 80.6563262939453 14.0
0 0.776227235794067 15.0
0 4.10664844512939 16.0
-1.2401527814e-11 5.43900346755981 17.0
58.290858783 58.123363494873 18.0
-58.290858803 -58.1011543273926 19.0
-58.290858781 -50.241039276123 20.0
-58.290858793 -56.4194526672363 21.0
80.216955727 72.9683837890625 22.0
142.92577388 131.649353027344 23.0
108.6227484 108.848281860352 24.0
22.932853154 23.8715229034424 25.0
223.14272962 227.951995849609 26.0
223.14272962 225.83283996582 27.0
28.368392321 25.9905891418457 28.0
251.51112195 254.169952392578 29.0
251.51112195 257.625946044922 30.0
46.893961157 47.6603660583496 31.0
9.4018906555 8.22420215606689 32.0
21.926096924 27.4454135894775 33.0
21.926096924 6.44482612609863 34.0
21.926096924 23.4320106506348 35.0
9.5515575796 -0.555282294750214 36.0
8.287181272 9.2640905380249 37.0
-1.9877181649e-11 7.77374267578125 38.0
-1.9877181649e-11 3.07543063163757 39.0
-12.287432724 -29.2069129943848 40.0
-25.343215388 -28.3953132629395 41.0
5.0851776519 3.63574552536011 42.0
-33.04764115 -32.0890235900879 43.0
94.710837291 93.5221405029297 44.0
94.710837304 94.8918762207031 45.0
6.8351185337 8.53389549255371 46.0
2.489241411 1.75495767593384 47.0
178.10929761 162.948822021484 48.0
171.95729739 163.839172363281 49.0
0 -0.350852072238922 50.0
70.527717732 74.4366455078125 51.0
88.972143531 90.0881881713867 52.0
80.773868347 80.4677429199219 53.0
80.773868347 79.674430847168 54.0
80.773868347 80.6995391845703 55.0
0 4.51551866531372 56.0
80.216955737 80.7464599609375 57.0
223.14272963 221.236663818359 58.0
251.51112196 254.375946044922 59.0
21.926096934 26.0170154571533 60.0
0 6.70063018798828 61.0
166.43504996 155.809127807617 0.0
166.43504996 162.220916748047 1.0
165.50339172 166.473892211914 2.0
160.03875653 174.172424316406 3.0
2.9648315608 -3.60944652557373 4.0
2.9648315608 5.7509503364563 5.0
-65.686833793 -43.0611000061035 6.0
75.217444867 82.0187149047852 7.0
45.874195116 52.9012451171875 8.0
29.343249741 21.575345993042 9.0
93.820107532 105.879104614258 10.0
80.241339409 80.06005859375 11.0
86.304266506 78.0165176391602 12.0
86.304266506 78.2705841064453 13.0
86.304266506 72.2389602661133 14.0
0 2.72473096847534 15.0
0 1.80607008934021 16.0
0 -3.73666858673096 17.0
55.679370272 55.4431419372559 18.0
-55.679370292 -64.6455917358398 19.0
-55.67937027 -52.0256271362305 20.0
-55.679370283 -55.0497093200684 21.0
77.073265076 76.9359436035156 22.0
129.45181644 145.590362548828 23.0
97.800097469 117.895927429199 24.0
21.40534584 18.0543022155762 25.0
206.52508152 226.631591796875 26.0
206.52508152 227.959259033203 27.0
26.655645479 27.2698554992676 28.0
233.18072701 255.445938110352 29.0
233.18072701 246.521682739258 30.0
46.380298789 53.5155982971191 31.0
9.332226119 8.28149509429932 32.0
21.393894783 26.0273628234863 33.0
21.393894783 12.9887676239014 34.0
21.393894783 17.6148071289062 35.0
8.5748444431 10.4764366149902 36.0
7.1656481692 6.29544448852539 37.0
0 -3.85597372055054 38.0
0 1.17383623123169 39.0
-11.986129813 -15.682240486145 40.0
-25.089079815 -27.2865104675293 41.0
5.5624522815 3.50567007064819 42.0
-33.394054469 -30.9954528808594 43.0
90.892334825 82.2317352294922 44.0
90.892334838 103.775894165039 45.0
5.8209091389 10.0716638565063 46.0
2.2535866089 2.74248719215393 47.0
166.43504997 165.067337036133 48.0
160.03875654 160.942153930664 49.0
2.9648315708 -1.22879528999329 50.0
75.217444877 73.320686340332 51.0
93.820107542 74.3947906494141 52.0
86.304266516 77.4001770019531 53.0
86.304266516 81.3580627441406 54.0
86.304266516 81.7724838256836 55.0
0 4.24825000762939 56.0
77.073265086 81.3222579956055 57.0
206.52508153 220.293426513672 58.0
233.18072702 253.527359008789 59.0
21.393894793 19.5257892608643 60.0
0 3.46240949630737 61.0
176.39373395592 163.735916137695 0.0
176.39373395592 164.427322387695 1.0
175.3559948 159.3779296875 2.0
170.56692507 160.412384033203 3.0
0 -0.949622213840485 4.0
-8.862281502e-09 3.83186149597168 5.0
-54.2307042540748 -61.5949172973633 6.0
63.3659285132617 80.8567276000977 7.0
48.2094616602617 51.8532028198242 8.0
15.1564668522617 27.6419563293457 9.0
97.7643859022618 84.108024597168 10.0
84.5028619762617 74.9425811767578 11.0
81.8250055012617 80.9078598022461 12.0
81.8250055012617 83.4761276245117 13.0
81.8250055012617 77.005241394043 14.0
0 -0.565450251102448 15.0
0 -1.72565269470215 16.0
-6.838747084e-09 -1.4163236618042 17.0
59.5677893492337 56.8830528259277 18.0
-59.5677893593926 -59.9420433044434 19.0
-59.5677893383926 -60.9268989562988 20.0
-59.5677893443926 -59.1392936706543 21.0
82.0311396137194 77.1762847900391 22.0
147.465342467234 143.561126708984 23.0
108.909708551542 108.65380859375 24.0
22.2983464765421 20.535530090332 25.0
229.49648208184 225.765533447266 26.0
229.49648208184 225.582183837891 27.0
28.3777771378405 24.5713500976562 28.0
257.87425922184 249.941040039062 29.0
257.87425922184 254.323287963867 30.0
46.8712742688432 45.7575416564941 31.0
9.3418054421 7.96539402008057 32.0
22.4633502692244 16.7593975067139 33.0
22.4633502692244 15.5215864181519 34.0
22.4633502756075 19.3990287780762 35.0
10.199780717 12.1584539413452 36.0
8.9492779877 7.02440738677979 37.0
-9.400370781e-09 -3.98684120178223 38.0
-9.400370781e-09 -4.7182822227478 39.0
-11.3551464796963 -9.29204654693604 40.0
-23.9434004627772 -26.6096115112305 41.0
5.34257862171812 1.26965880393982 42.0
-31.5385366282181 -33.1806373596191 43.0
86.8111861602181 89.4104385375977 44.0
86.8111861104062 90.4108352661133 45.0
7.1676575759 5.05861902236938 46.0
2.3111890102 1.67849206924438 47.0
176.39373395 171.864883422852 48.0
170.56692508 161.476348876953 49.0
0 0.105547428131104 50.0
63.365928513 76.3408050537109 51.0
97.764385902 88.9855194091797 52.0
81.825005501 79.6799926757812 53.0
81.825005501 81.2104873657227 54.0
81.825005501 78.9302520751953 55.0
0 -1.41538953781128 56.0
82.031139614 77.9886703491211 57.0
229.49648208 223.803741455078 58.0
257.87425922 247.524917602539 59.0
22.463350276 22.0135345458984 60.0
0 2.19289541244507 61.0
137.17687742 143.006698608398 0.0
137.17687742 143.024017333984 1.0
135.86355256 143.584884643555 2.0
131.0744823 135.858932495117 3.0
0 -2.21669626235962 4.0
0 0.342764794826508 5.0
-59.42861149 -59.2475357055664 6.0
69.705106679 70.1774826049805 7.0
65.43029471 62.4528312683105 8.0
4.2748119615 8.48783111572266 9.0
78.669161727 80.2065200805664 10.0
65.229460649 71.2585144042969 11.0
77.500548395 80.8047180175781 12.0
77.500548395 77.9000701904297 13.0
77.500548395 76.9198608398438 14.0
0 -0.64198762178421 15.0
0 0.28728449344635 16.0
0 -3.15485000610352 17.0
59.754945382 58.5015258789062 18.0
-59.754945394 -59.4383087158203 19.0
-59.754945372 -60.0649642944336 20.0
-59.754945384 -61.8801422119141 21.0
80.656571382 77.8813858032227 22.0
140.87903669 140.394561767578 23.0
106.41719675 107.357681274414 24.0
22.90053999 20.4083709716797 25.0
221.53560808 223.594223022461 26.0
221.53560808 221.987747192383 27.0
27.282645659 25.3551406860352 28.0
248.81825375 253.574676513672 29.0
248.81825375 252.408157348633 30.0
47.033013387 47.2299919128418 31.0
9.3131031344 9.18030452728271 32.0
20.901625993 20.4819431304932 33.0
20.901625993 21.1129970550537 34.0
20.901625993 20.8195400238037 35.0
9.3083381265 8.28715133666992 36.0
8.044767894 8.38281917572021 37.0
-2.5262790395e-09 -0.804296553134918 38.0
-2.5262785966e-09 0.113371789455414 39.0
-12.387792857 -14.5510091781616 40.0
-23.81592281 -27.5234241485596 41.0
4.9278776077 3.5162615776062 42.0
-31.108913619 -34.7394485473633 43.0
92.052766479 91.3064651489258 44.0
92.05276649 90.2343292236328 45.0
6.7273095346 3.00315618515015 46.0
2.6568330431 1.83220052719116 47.0
137.17687743 136.677871704102 48.0
131.07448231 127.691390991211 49.0
0 3.52269697189331 50.0
69.705106686 70.9855194091797 51.0
78.669161734 84.2529830932617 52.0
77.500548402 81.0479965209961 53.0
77.500548402 80.9212951660156 54.0
77.500548402 79.4541549682617 55.0
0 -0.0822644829750061 56.0
80.656571391 80.3950500488281 57.0
221.53560809 218.26057434082 58.0
248.81825375 246.469757080078 59.0
20.901626002 19.9398994445801 60.0
0 2.89454770088196 61.0
147.59555202 164.645812988281 0.0
147.59555202 158.001007080078 1.0
146.39025614 157.817626953125 2.0
141.60118641 139.932540893555 3.0
0 7.75734853744507 4.0
0 0.429560720920563 5.0
-64.753654227 -59.9145317077637 6.0
76.026143549 80.3137741088867 7.0
46.150106242 45.6958045959473 8.0
29.876037297 24.1751804351807 9.0
84.495648475 96.9319839477539 10.0
72.638321873 78.2974090576172 11.0
79.558471733 87.3150787353516 12.0
79.558471733 86.6168060302734 13.0
79.558471733 76.4477691650391 14.0
0 -0.0214637517929077 15.0
0 -0.23643547296524 16.0
0 0.650009214878082 17.0
60.723538287 60.4936294555664 18.0
-60.723538297 -58.8111991882324 19.0
-60.723538275 -53.197639465332 20.0
-60.723538288 -56.3383140563965 21.0
82.373400961 77.549674987793 22.0
139.61453948 147.057968139648 23.0
106.98216904 112.452606201172 24.0
21.386479709 19.2137184143066 25.0
221.98794045 222.956954956055 26.0
221.98794045 226.307723999023 27.0
28.780443619 25.3258571624756 28.0
250.76838408 248.267578125 29.0
250.76838408 249.997619628906 30.0
47.002454833 49.5958099365234 31.0
9.285912751 9.5930347442627 32.0
21.649862664 18.1959705352783 33.0
21.649862664 23.7815761566162 34.0
21.649862664 22.2283973693848 35.0
9.9459293968 6.6279730796814 36.0
8.7606888039 6.35870599746704 37.0
-2.6040799383e-10 -1.76334714889526 38.0
-2.6040792422e-10 -3.19870805740356 39.0
-11.635736569 -16.9976825714111 40.0
-25.22907137 -26.5920715332031 41.0
4.8900668851 5.66343832015991 42.0
-32.52567893 -33.843620300293 43.0
90.165309755 87.4204483032227 44.0
90.165309768 90.1847076416016 45.0
6.9789841328 4.81195688247681 46.0
2.3836761511 2.24263024330139 47.0
147.59555203 161.683502197266 48.0
141.60118642 147.434127807617 49.0
0 7.5224461555481 50.0
76.026143559 70.6722412109375 51.0
84.495648485 86.4512710571289 52.0
79.558471743 80.1537017822266 53.0
79.558471743 82.3549194335938 54.0
79.558471743 79.8503494262695 55.0
0 1.61147546768188 56.0
82.373400971 80.1528396606445 57.0
221.98794046 223.339050292969 58.0
250.76838408 244.4599609375 59.0
21.649862673 17.5413990020752 60.0
0 1.13901662826538 61.0
170.42828661 159.065307617188 0.0
170.42828661 159.948745727539 1.0
169.20185224 158.352661132812 2.0
164.41278252 155.777420043945 3.0
0 -1.45657134056091 4.0
-1.9157067912e-09 1.28356671333313 5.0
-67.875488659 -60.9502754211426 6.0
79.144750301 74.5976104736328 7.0
53.384642381 50.3271217346191 8.0
25.760107916 24.9055614471436 9.0
88.714336233 90.6856384277344 10.0
76.133213146 77.1031036376953 11.0
82.239560211 85.1240615844727 12.0
82.239560211 79.4011535644531 13.0
82.239560211 77.8806533813477 14.0
0 -1.84207630157471 15.0
0 -0.349497377872467 16.0
0 2.75881457328796 17.0
57.964099941 57.6090278625488 18.0
-57.964099956 -59.0741500854492 19.0
-57.964099934 -59.7240600585938 20.0
-57.964099946 -58.4843597412109 21.0
79.363844231 77.1597671508789 22.0
145.55931512 141.331832885742 23.0
111.00269654 111.06600189209 24.0
21.546285825 22.2725505828857 25.0
224.92315937 228.901947021484 26.0
224.92315937 224.171737670898 27.0
27.714883336 26.1833782196045 28.0
252.6380427 253.83757019043 29.0
252.6380427 255.720504760742 30.0
46.717884608 45.622013092041 31.0
9.2988496224 9.12876415252686 32.0
21.399744289 19.8968029022217 33.0
21.399744289 23.0106754302979 34.0
21.399744287 19.9864082336426 35.0
9.3889695642 7.702073097229 36.0
8.1673655693 7.0573787689209 37.0
-6.9849843898e-09 -2.71027994155884 38.0
-6.9849843355e-09 -1.06142449378967 39.0
-11.629401297 -15.39759349823 40.0
-24.492600629 -25.065788269043 41.0
4.9419372357 2.96994924545288 42.0
-32.143726257 -32.7842292785645 43.0
90.970316464 89.2125396728516 44.0
90.970316467 91.3017120361328 45.0
6.73937146 6.85662364959717 46.0
2.4545655073 2.19619560241699 47.0
170.42828662 164.859970092773 48.0
164.41278253 156.40771484375 49.0
0 -0.789117276668549 50.0
79.144750304 78.7460250854492 51.0
88.714336236 91.9713897705078 52.0
82.239560214 81.7778244018555 53.0
82.239560214 81.6386566162109 54.0
82.239560214 80.1250991821289 55.0
0 3.35901856422424 56.0
79.36384424 80.0875778198242 57.0
224.92315937 223.353530883789 58.0
252.63804271 248.181655883789 59.0
21.399744294 19.6026935577393 60.0
0 0.969676554203033 61.0
152.372681506956 166.943725585938 0.0
152.372681506956 166.741821289062 1.0
151.17254784 163.429672241211 2.0
146.38347811 147.562866210938 3.0
0 3.12316727638245 4.0
-8.0469011407e-09 6.74087285995483 5.0
-56.572494264 -46.3272323608398 6.0
67.609214459391 80.1870880126953 7.0
52.670678250391 42.0725860595703 8.0
14.938536209391 21.4882640838623 9.0
84.672193511391 92.1217422485352 10.0
72.352619779391 72.7977600097656 11.0
86.217727744391 91.5621871948242 12.0
86.217727744391 79.8793792724609 13.0
86.217727744391 80.9972839355469 14.0
0 -1.26860570907593 15.0
0 0.224732100963593 16.0
-7.295415606e-09 -1.23044967651367 17.0
58.8202653856678 58.1891021728516 18.0
-58.8202653990865 -59.1922225952148 19.0
-58.8202653770865 -60.4987640380859 20.0
-58.8202653860865 -58.2667083740234 21.0
80.56566502 80.5951385498047 22.0
152.388479447668 141.142120361328 23.0
119.568199974585 111.809814453125 24.0
21.6281370715848 20.7619113922119 25.0
232.954144473913 225.943313598633 26.0
232.954144473913 221.749908447266 27.0
26.0855591859127 28.7114238739014 28.0
259.039703663913 258.899932861328 29.0
259.039703663913 257.357818603516 30.0
46.306567672 46.1879653930664 31.0
9.2599501791 8.53433513641357 32.0
21.7453996376208 18.3843078613281 33.0
21.7453996376208 17.9027290344238 34.0
21.745399639 21.1620769500732 35.0
9.6147188781 8.18524551391602 36.0
8.3507369678 5.24706077575684 37.0
-9.645152647e-09 2.15929889678955 38.0
-9.645152648e-09 2.11360454559326 39.0
-11.6407826100432 -13.0579500198364 40.0
-25.2192415895039 -28.023380279541 41.0
5.48640030905915 2.91068553924561 42.0
-33.5139239531591 -33.4050140380859 43.0
88.9523876381592 91.9756164550781 44.0
88.9523876166797 93.0760192871094 45.0
6.4163568273 12.7913999557495 46.0
2.7002071852 3.21081113815308 47.0
152.37268151 164.343933105469 48.0
146.38347812 158.315231323242 49.0
0 1.74542450904846 50.0
67.609214459 76.3130416870117 51.0
84.672193511 86.621826171875 52.0
86.217727744 81.1737976074219 53.0
86.217727744 80.4597625732422 54.0
86.217727744 79.7098083496094 55.0
0 1.58753418922424 56.0
80.565665024 80.3560791015625 57.0
232.95414448 217.963241577148 58.0
259.03970366 251.920654296875 59.0
21.745399642 22.1938514709473 60.0
0 2.37082457542419 61.0
147.982033057583 157.456329345703 0.0
147.982033057583 151.962692260742 1.0
146.808413813287 153.341369628906 2.0
142.01934397 150.663513183594 3.0
-1.55484310406e-09 -0.642547428607941 4.0
-1.55484309999997e-09 -2.83744192123413 5.0
-64.9693040339936 -64.3185882568359 6.0
77.6547829144777 74.1871490478516 7.0
61.2786926634778 55.4030647277832 8.0
16.3760902514777 7.20984888076782 9.0
97.5261141374777 89.7391510009766 10.0
83.6472874924777 79.6450042724609 11.0
85.2981411144777 83.8642730712891 12.0
85.2981411144777 76.4624557495117 13.0
85.2981411144777 81.5436782836914 14.0
2.3328021736e-08 -2.03496789932251 15.0
2.3328021735e-08 -1.6052393913269 16.0
5.464350236e-08 -2.72059202194214 17.0
56.2518018559467 59.0783615112305 18.0
-56.2518018797166 -57.0891418457031 19.0
-56.2518018707166 -59.2854194641113 20.0
-56.2518017707166 -58.115608215332 21.0
77.1875821073724 73.3798065185547 22.0
153.454771802806 145.008895874023 23.0
115.220596133508 115.750228881836 24.0
21.7005584165075 20.8819332122803 25.0
230.642353915167 228.381210327148 26.0
230.642353915167 229.708221435547 27.0
26.8883420391666 21.3647422790527 28.0
257.530695955167 253.23681640625 29.0
257.530695955167 255.336776733398 30.0
46.7385281475469 45.537841796875 31.0
9.28724497493877 8.4355525970459 32.0
20.9357803369809 22.6376686096191 33.0
20.9357803369809 19.0790691375732 34.0
20.935780467613 20.4264030456543 35.0
8.86054244871795 8.39873600006104 36.0
7.60121916461794 6.85336446762085 37.0
2.62465480000001e-09 1.1959273815155 38.0
2.62465480000001e-09 -1.17553234100342 39.0
-11.6034306368583 -12.5643882751465 40.0
-24.5599220844463 -26.7411251068115 41.0
4.75986482786468 1.92820739746094 42.0
-31.7734218810648 -31.6524333953857 43.0
89.4696777832616 86.4044494628906 44.0
89.4696781210649 91.9600372314453 45.0
7.05809588635643 4.443039894104 46.0
2.46335994485643 2.26466846466064 47.0
147.98203306 156.097015380859 48.0
142.01934397 152.941223144531 49.0
0 -3.75158500671387 50.0
77.654782914 76.0238265991211 51.0
97.526114137 92.0755157470703 52.0
85.298141114 81.5651702880859 53.0
85.298141114 80.2131118774414 54.0
85.298141114 79.9203643798828 55.0
5.4643502359e-08 3.07315587997437 56.0
77.187582108 79.4883346557617 57.0
230.64235391 222.583419799805 58.0
257.53069595 249.491134643555 59.0
20.935780468 19.6116352081299 60.0
2.6246548239e-09 3.64096260070801 61.0
160.35063881316 167.894668579102 0.0
160.35063881316 170.422882080078 1.0
159.091024497706 171.265716552734 2.0
153.62638895 147.936340332031 3.0
4.3736320528 12.9847917556763 4.0
4.37363205541732 15.3899278640747 5.0
-66.9858621755173 -48.4729957580566 6.0
79.5663925708106 83.5473022460938 7.0
44.2486092298106 58.347225189209 8.0
35.3177833408106 22.5283107757568 9.0
88.9020717958106 88.4344940185547 10.0
76.1019445608106 79.3463745117188 11.0
85.6025560708106 83.1701049804688 12.0
85.6025560708106 75.6027145385742 13.0
85.6025560708106 77.0297622680664 14.0
3.5917564607e-08 0.888904213905334 15.0
3.5917564608e-08 -0.089235782623291 16.0
7.51975143999999e-08 -0.00610464811325073 17.0
63.1678763288058 56.9941749572754 18.0
-63.1678760782159 -59.3318901062012 19.0
-63.1678760732159 -59.9870834350586 20.0
-63.1678759362159 -54.3889389038086 21.0
83.1373687300299 76.5203704833984 22.0
136.675409366806 136.036178588867 23.0
104.930735791007 112.107666015625 24.0
21.5234025330072 22.1469116210938 25.0
219.81277809632 219.089492797852 26.0
219.81277809632 216.169067382812 27.0
24.82958511732 24.5978698730469 28.0
244.64236321632 253.917907714844 29.0
244.64236321632 254.26139831543 30.0
46.6761279145856 46.297492980957 31.0
9.35352104745502 8.80612564086914 32.0
19.969492793552 20.5294246673584 33.0
19.969492793552 19.7413063049316 34.0
19.969492968314 23.0593090057373 35.0
8.70648974130307 11.0390787124634 36.0
7.47673157460307 4.53157520294189 37.0
6.73553879999999e-09 -7.89872550964355 38.0
6.73553879999999e-09 -2.05979943275452 39.0
-11.583622860108 -20.0420265197754 40.0
-24.1605542933444 -23.3095092773438 41.0
4.68326180110882 9.88284397125244 42.0
-31.1458389500088 -35.9146423339844 43.0
99.1332674240089 94.5591201782227 44.0
99.1332662917558 94.3307189941406 45.0
7.09816358145277 3.68941688537598 46.0
2.76626398315276 2.0198974609375 47.0
160.35063881 170.955535888672 48.0
153.62638895 162.811157226562 49.0
4.3736320554 2.18092799186707 50.0
79.566392571 73.5864410400391 51.0
88.902071796 84.9199981689453 52.0
85.602556071 78.7710189819336 53.0
85.602556071 79.8957061767578 54.0
85.602556071 78.9692687988281 55.0
7.519751443e-08 4.61810159683228 56.0
83.13736873 79.282585144043 57.0
219.81277809 212.412750244141 58.0
244.64236321 251.319396972656 59.0
19.969492968 21.6904354095459 60.0
6.7355388327e-09 3.6285126209259 61.0
189.25750279 163.488204956055 0.0
189.25750279 166.181442260742 1.0
187.82716165 170.539947509766 2.0
183.03809161 206.850860595703 3.0
0 0.413111627101898 4.0
0 2.73274230957031 5.0
-62.399311056 -65.2886352539062 6.0
72.921458875 75.1139831542969 7.0
43.828508234 45.2858734130859 8.0
29.092950632 9.50836563110352 9.0
89.028145945 84.2620391845703 10.0
75.60611941 67.3173065185547 11.0
84.614631517 80.2053298950195 12.0
84.614631517 83.298225402832 13.0
84.614631517 81.3337936401367 14.0
0 1.01181495189667 15.0
0 -0.0452842712402344 16.0
0 0.899537801742554 17.0
58.141011275 58.8525886535645 18.0
-58.141011287 -58.9113655090332 19.0
-58.141011265 -58.6760482788086 20.0
-58.141011277 -58.0465393066406 21.0
78.444789496 76.6932144165039 22.0
132.21339057 144.883392333984 23.0
100.36847201 118.364273071289 24.0
20.717118768 20.1690559387207 25.0
210.65818008 224.940719604492 26.0
210.65818008 214.490280151367 27.0
26.210370561 32.077018737793 28.0
236.86855065 244.829208374023 29.0
236.86855065 230.620697021484 30.0
46.54962023 42.9310188293457 31.0
9.3526201389 9.26723289489746 32.0
20.303778211 17.4900093078613 33.0
20.303778211 17.8375415802002 34.0
20.303778211 18.9083271026611 35.0
8.7166495937 8.694899559021 36.0
7.3770673439 6.62734365463257 37.0
-8.5534957676e-10 -1.19837975502014 38.0
-8.5534957156e-10 -2.77520680427551 39.0
-11.957612527 -17.1162509918213 40.0
-25.701691042 -28.5237674713135 41.0
4.9242841396 5.85122299194336 42.0
-33.153156294 -33.4081993103027 43.0
86.373748697 96.4817657470703 44.0
86.37374871 90.4525756835938 45.0
7.2853312272 3.52993106842041 46.0
2.7162990298 2.41360569000244 47.0
189.2575028 164.891357421875 48.0
183.03809162 171.884796142578 49.0
0 -5.91379880905151 50.0
72.921458884 74.6535415649414 51.0
89.028145954 80.9882125854492 52.0
84.614631526 80.2339401245117 53.0
84.614631526 77.0661163330078 54.0
84.614631526 79.2052688598633 55.0
0 1.444899559021 56.0
78.444789506 79.1247863769531 57.0
210.65818009 217.39045715332 58.0
236.86855066 240.646514892578 59.0
20.30377822 22.639627456665 60.0
0 1.51905465126038 61.0
179.87042804 163.428466796875 0.0
179.87042804 164.804565429688 1.0
178.59360446 165.164596557617 2.0
173.80453474 166.622161865234 3.0
0 -4.75371932983398 4.0
0 8.1449031829834 5.0
-60.81057847 -59.9308395385742 6.0
72.205216369 73.5568466186523 7.0
48.182121709 52.6439170837402 8.0
24.023094651 18.2233257293701 9.0
90.954267683 86.9186859130859 10.0
78.106472107 72.9382705688477 11.0
81.446715458 82.7116394042969 12.0
81.446715458 80.4266357421875 13.0
81.446715458 81.2852783203125 14.0
0 -1.14512276649475 15.0
0 -0.0151953101158142 16.0
0 1.68383920192719 17.0
57.168582547 58.4678535461426 18.0
-57.168582557 -58.9646301269531 19.0
-57.168582535 -59.6098480224609 20.0
-57.168582548 -58.1340484619141 21.0
78.561701539 76.9391708374023 22.0
145.42241232 141.494049072266 23.0
110.78882087 110.363571166992 24.0
24.085164087 19.9674625396729 25.0
223.98411387 225.872756958008 26.0
223.98411387 220.490081787109 27.0
27.628962939 29.3433818817139 28.0
251.61307681 256.386871337891 29.0
251.61307681 256.043029785156 30.0
46.829541976 45.5860214233398 31.0
9.3325701774 9.22627067565918 32.0
21.393118983 17.5126152038574 33.0
21.393118983 19.0022487640381 34.0
21.393118982 20.2901363372803 35.0
9.6735473307 5.512864112854 36.0
8.4264532871 7.88378429412842 37.0
-5.6297110477e-10 -0.0107722878456116 38.0
-5.6297110477e-10 -2.31899356842041 39.0
-11.569348985 -14.1814813613892 40.0
-24.615150545 -26.2458248138428 41.0
4.880051929 5.54185581207275 42.0
-32.086690815 -33.4734992980957 43.0
95.193286116 90.9193878173828 44.0
95.193286128 93.8373260498047 45.0
6.8978137424 7.70694780349731 46.0
2.5886928279 2.68894600868225 47.0
179.87042805 164.607223510742 48.0
173.80453475 160.958984375 49.0
0 1.16922283172607 50.0
72.205216379 77.2485961914062 51.0
90.954267693 84.3228073120117 52.0
81.446715467 81.1558227539062 53.0
81.446715467 80.5682067871094 54.0
81.446715467 80.5592041015625 55.0
0 1.20625376701355 56.0
78.561701549 80.0138473510742 57.0
223.98411388 220.348541259766 58.0
251.61307682 250.526992797852 59.0
21.393118992 23.775857925415 60.0
0 3.22667717933655 61.0
159.87793028 159.781051635742 0.0
159.87793028 154.506362915039 1.0
158.49183143 161.291641235352 2.0
153.7027613 145.443984985352 3.0
0 -12.2476854324341 4.0
0 17.2499198913574 5.0
-67.400503702 -45.0059967041016 6.0
79.544323635 80.6306228637695 7.0
54.007987226 40.4847984313965 8.0
25.536336399 21.4741344451904 9.0
78.003783756 82.0531005859375 10.0
65.522349866 74.5462341308594 11.0
73.942849837 70.8478851318359 12.0
73.942849837 76.1288528442383 13.0
73.942849837 80.3363037109375 14.0
0 1.29038631916046 15.0
0 -1.30883169174194 16.0
0 0.870879709720612 17.0
59.105666287 59.3675117492676 18.0
-59.105666297 -59.2049674987793 19.0
-59.105666276 -57.0794258117676 20.0
-59.105666288 -58.6720848083496 21.0
80.322549265 69.9968032836914 22.0
141.84408234 137.041259765625 23.0
106.18491377 108.513023376465 24.0
22.218091295 20.1900367736816 25.0
222.16663162 224.071304321289 26.0
222.16663162 217.401962280273 27.0
24.459597793 27.3965129852295 28.0
246.62622942 243.74462890625 29.0
246.62622942 251.254974365234 30.0
47.026444908 40.9113235473633 31.0
9.2996802835 8.64844703674316 32.0
21.216882967 21.1490592956543 33.0
21.216882967 22.6075611114502 34.0
21.216882967 21.4203281402588 35.0
9.7728622608 9.60377407073975 36.0
8.5336969608 8.01455211639404 37.0
-3.3399054203e-12 -5.34731388092041 38.0
-3.339905656e-12 -2.4704065322876 39.0
-11.623479193 -16.2676811218262 40.0
-24.94518145 -21.7509174346924 41.0
5.6537678347 7.99192476272583 42.0
-33.39877357 -34.5808486938477 43.0
91.680904746 90.2656326293945 44.0
91.680904759 90.5833969116211 45.0
6.8628312506 7.33338642120361 46.0
2.7759735028 1.21577906608582 47.0
159.87793029 168.45622253418 48.0
153.70276131 146.389297485352 49.0
0 -3.07367014884949 50.0
79.544323645 76.6467514038086 51.0
78.003783766 89.6426696777344 52.0
73.942849847 76.5899963378906 53.0
73.942849847 74.6017761230469 54.0
73.942849847 80.9799118041992 55.0
0 5.45345783233643 56.0
80.322549275 79.8838424682617 57.0
222.16663163 221.840286254883 58.0
246.62622943 248.984573364258 59.0
21.216882977 19.2913799285889 60.0
0 2.36471343040466 61.0
160.40029773 161.286422729492 0.0
160.40029773 164.271453857422 1.0
159.04685921 156.979202270508 2.0
154.25778948 160.440948486328 3.0
0 1.96068620681763 4.0
-1.1106157605e-11 4.07722473144531 5.0
-67.164254488 -69.7969284057617 6.0
77.905037981 76.2602233886719 7.0
56.332002807 47.2559432983398 8.0
21.573035164 25.714204788208 9.0
86.970057114 90.5282669067383 10.0
73.950416555 75.4032516479492 11.0
81.119634047 87.77490234375 12.0
81.119634047 78.8990020751953 13.0
81.119634047 80.1012420654297 14.0
0 -0.256228864192963 15.0
0 -0.19922286272049 16.0
0 0.492870032787323 17.0
59.264416476 57.1221313476562 18.0
-59.264416496 -58.6440391540527 19.0
-59.264416474 -57.773265838623 20.0
-59.264416487 -57.5845527648926 21.0
80.633842594 78.2018508911133 22.0
145.67202515 141.996139526367 23.0
110.34261621 108.156616210938 24.0
22.971694682 19.1125392913818 25.0
226.30586775 222.863891601562 26.0
226.30586775 227.450820922852 27.0
28.061618658 29.2107601165771 28.0
254.36748642 252.690139770508 29.0
254.36748642 253.60498046875 30.0
46.518523875 45.2752456665039 31.0
9.3238344953 8.95865631103516 32.0
21.369426097 17.7103004455566 33.0
21.369426097 17.6770057678223 34.0
21.369426097 20.8870105743408 35.0
9.8534733136 7.96312856674194 36.0
8.5804788689 6.95929193496704 37.0
-3.4865499095e-11 -0.434172928333282 38.0
-3.4865475198e-11 -3.54875135421753 39.0
-12.411186885 -13.9457921981812 40.0
-24.601415118 -26.8714199066162 41.0
5.0371208571 5.8277759552002 42.0
-32.131741558 -34.3593444824219 43.0
94.496748584 90.5503082275391 44.0
94.496748597 94.2157592773438 45.0
6.7865864039 6.48163652420044 46.0
2.4610538327 2.17473244667053 47.0
160.40029774 162.612167358398 48.0
154.25778949 150.847152709961 49.0
0 2.82914710044861 50.0
77.905037991 77.1076354980469 51.0
86.970057124 87.8557891845703 52.0
81.119634057 81.9147491455078 53.0
81.119634057 81.5159759521484 54.0
81.119634057 80.1029281616211 55.0
0 1.16054677963257 56.0
80.633842604 79.8010025024414 57.0
226.30586776 219.901840209961 58.0
254.36748643 250.350433349609 59.0
21.369426107 22.7730197906494 60.0
0 1.7182629108429 61.0
205.88622948 173.514938354492 0.0
205.88622948 148.778915405273 1.0
204.34934471 164.515563964844 2.0
203.29335043 193.710494995117 3.0
-4.0865192275e-13 -30.2441349029541 4.0
-1.699540558e-11 -3.37182903289795 5.0
-54.720220948 -66.4943542480469 6.0
62.440552717 79.6623306274414 7.0
38.617790334 55.1751747131348 8.0
23.822762373 16.3021488189697 9.0
92.262519811 94.2973937988281 10.0
80.60863046 83.7443771362305 11.0
75.716709666 94.0955581665039 12.0
75.716709666 84.0410232543945 13.0
75.716709666 74.218017578125 14.0
0 1.89795613288879 15.0
0 -0.959831535816193 16.0
-2.736801062e-11 0.252919852733612 17.0
53.182406828 56.1366577148438 18.0
-53.182406848 -55.9189147949219 19.0
-53.851025897 -55.4914054870605 20.0
-53.919667878 -54.391902923584 21.0
72.319633039 79.8481750488281 22.0
140.95980522 143.012832641602 23.0
109.52526146 110.76490020752 24.0
21.76060813 23.0991744995117 25.0
213.27943827 221.519638061523 26.0
213.27943827 223.415313720703 27.0
30.326918439 27.8478965759277 28.0
243.60635672 230.311416625977 29.0
243.60635672 243.790466308594 30.0
46.890139691 42.1221771240234 31.0
9.3865843553 14.0318698883057 32.0
18.399965152 20.7162628173828 33.0
18.399965152 33.9090690612793 34.0
18.399965152 12.4397554397583 35.0
9.4062169528 14.0111961364746 36.0
8.2007042844 6.22383642196655 37.0
-4.3220675773e-11 -4.60679912567139 38.0
-4.3220675758e-11 7.74320936203003 39.0
-11.523847108 1.3075897693634 40.0
-24.285918417 -25.19167137146 41.0
4.7578124137 -2.60031032562256 42.0
-31.668135572 -24.6885070800781 43.0
92.257051498 90.1106033325195 44.0
92.257051511 92.1590347290039 45.0
7.3511518177 6.29843139648438 46.0
2.7625543742 1.09672236442566 47.0
205.88622949 185.597076416016 48.0
203.29335044 174.330703735352 49.0
0 -4.64554977416992 50.0
62.440552727 69.513557434082 51.0
92.262519821 83.9690017700195 52.0
75.716709676 80.9438781738281 53.0
75.716709676 79.6303405761719 54.0
75.716709676 80.9406585693359 55.0
0 -2.64241337776184 56.0
72.319633049 81.4288635253906 57.0
213.27943828 217.504852294922 58.0
243.60635673 245.586532592773 59.0
18.399965162 24.8256778717041 60.0
0 12.7717094421387 61.0
177.230790776723 168.276840209961 0.0
177.230790776723 170.954711914062 1.0
175.97489696 167.775146484375 2.0
171.18582724 165.013198852539 3.0
0 1.00840997695923 4.0
-8.695896658e-09 3.25687646865845 5.0
-66.6813280614721 -64.9787826538086 6.0
78.6009224736525 80.8284149169922 7.0
55.1150721596525 51.1753311157227 8.0
23.4858503146525 27.8591899871826 9.0
89.9623216806525 89.4219512939453 10.0
77.6099152396525 77.1262817382812 11.0
81.2213177876525 81.5533676147461 12.0
81.2213177876525 78.3586730957031 13.0
81.2213177876525 78.9095306396484 14.0
0 -1.47511386871338 15.0
0 -0.0684186220169067 16.0
-6.719166526e-09 0.437236607074738 17.0
60.8898286619923 58.855541229248 18.0
-60.8898286714787 -58.6323928833008 19.0
-60.8898286504787 -60.4781761169434 20.0
-60.8898286564787 -58.7246360778809 21.0
83.2160013870953 76.590950012207 22.0
146.105084467992 142.297088623047 23.0
112.87952896249 110.16837310791 24.0
22.4766954984903 19.7099113464355 25.0
229.321085853445 220.831146240234 26.0
229.321085853445 223.985000610352 27.0
26.8991514584452 25.2338886260986 28.0
256.220237313445 257.053894042969 29.0
256.220237313445 256.384429931641 30.0
46.6620169402483 45.3788032531738 31.0
9.3391522138 8.84155368804932 32.0
22.3261727304164 20.6443119049072 33.0
22.3261727304164 20.8280563354492 34.0
22.3261727370871 20.1241493225098 35.0
9.7525403042 6.9317421913147 36.0
8.4757786226 6.67255544662476 37.0
-9.36233601e-09 1.99968218803406 38.0
-9.36233601e-09 -3.0368480682373 39.0
-11.6085369957393 -12.1175355911255 40.0
-25.0228063454843 -25.3875064849854 41.0
4.98434496197705 6.57820272445679 42.0
-32.5929350063771 -33.683406829834 43.0
98.797614073377 95.0439376831055 44.0
98.7976140573771 94.74658203125 45.0
6.80137819 5.90730619430542 46.0
2.4754452337 2.69849920272827 47.0
177.23079077 167.853790283203 48.0
171.18582724 159.193328857422 49.0
0 0.369772851467133 50.0
78.600922474 75.9872894287109 51.0
89.96232168 91.9298629760742 52.0
81.221317788 81.5424041748047 53.0
81.221317788 80.8521270751953 54.0
81.221317788 80.2592697143555 55.0
0 2.4076042175293 56.0
83.216001387 78.8677368164062 57.0
229.32108586 219.342559814453 58.0
256.22023731 249.697052001953 59.0
22.326172737 20.5679149627686 60.0
0 2.27629065513611 61.0
194.50491643 158.78385925293 0.0
194.50491643 160.224014282227 1.0
193.06848243 164.345428466797 2.0
187.60384724 156.945617675781 3.0
30.398962328 -25.3855628967285 4.0
30.398962328 13.4483613967896 5.0
-44.133173312 -51.7471237182617 6.0
54.844291013 105.291076660156 7.0
30.415905688 72.8280487060547 8.0
24.428385315 69.9720993041992 9.0
76.411670882 86.7153854370117 10.0
64.501631571 74.2790374755859 11.0
78.632568296 78.4349670410156 12.0
78.632568296 77.4130706787109 13.0
78.632568296 77.451286315918 14.0
0 -9.98974323272705 15.0
0 -2.43092060089111 16.0
-1.7575985849e-13 7.48656415939331 17.0
50.561854878 50.427547454834 18.0
-50.561854898 -53.2969703674316 19.0
-50.561854876 -54.9677734375 20.0
-50.561854889 -45.9612159729004 21.0
69.24826932 82.5991744995117 22.0
151.24116966 150.298202514648 23.0
117.89321728 86.5286483764648 24.0
19.477699927 31.4649333953857 25.0
220.48943899 217.967193603516 26.0
220.48943899 225.018569946289 27.0
31.621710876 29.6233425140381 28.0
252.11114987 255.021987915039 29.0
252.11114987 259.780120849609 30.0
45.984317059 46.8003044128418 31.0
9.3844218336 8.82529544830322 32.0
18.686414421 13.2807312011719 33.0
18.686414421 17.5692443847656 34.0
18.686414421 19.0710906982422 35.0
8.2243982504 7.06270885467529 36.0
7.0804167549 4.9096565246582 37.0
-1.7080465289e-12 -11.7643432617188 38.0
-1.7080393598e-12 -3.7664623260498 39.0
-11.04040656 -13.9357299804688 40.0
-22.951277844 -27.2106456756592 41.0
5.6114485235 2.16883397102356 42.0
-31.403900297 -35.0329132080078 43.0
73.833493183 81.2718048095703 44.0
73.833493196 68.4049072265625 45.0
6.3215873537 5.36140012741089 46.0
1.8513764592 2.602618932724 47.0
194.50491644 166.47590637207 48.0
187.60384725 148.235290527344 49.0
30.398962338 -4.18765115737915 50.0
54.844291023 69.0881500244141 51.0
76.411670892 84.1355361938477 52.0
78.632568306 79.2091598510742 53.0
78.632568306 74.607177734375 54.0
78.632568306 79.8735580444336 55.0
0 10.1250905990601 56.0
69.24826933 78.8467407226562 57.0
220.489439 223.080413818359 58.0
252.11114988 256.340759277344 59.0
18.686414431 27.9230880737305 60.0
0 2.42515635490417 61.0
131.292175352041 169.580505371094 0.0
131.292175352041 182.656173706055 1.0
130.22602559 162.937957763672 2.0
124.76139035 128.88623046875 3.0
5.4278882976 0.809947073459625 4.0
5.42788830653951 13.7109050750732 5.0
-78.1983969934395 -87.278938293457 6.0
92.6068568515383 78.4707946777344 7.0
62.5250961945383 41.9572944641113 8.0
30.0817606565383 11.8795213699341 9.0
81.8214669485383 87.4131393432617 10.0
68.0064155625383 63.3590278625488 11.0
75.7700025005383 58.9240036010742 12.0
75.7700025005383 67.5720977783203 13.0
75.7700025005383 71.0859146118164 14.0
3.1868823314e-09 4.48831844329834 15.0
0 1.35277986526489 16.0
7.16567026999998e-09 0.0922404527664185 17.0
39.6398577708529 54.357292175293 18.0
-39.6398578093074 -58.1208419799805 19.0
-39.6398577903074 -51.2564544677734 20.0
-39.6398577813074 -53.0891227722168 21.0
59.4905371677941 71.4259338378906 22.0
132.528328896839 120.27131652832 23.0
100.18050262294 103.581985473633 24.0
11.8598445510488 26.3986778259277 25.0
192.018866064083 178.269561767578 26.0
192.018866064083 195.499557495117 27.0
31.1994688420826 19.689567565918 28.0
223.218334904083 233.820861816406 29.0
223.218334904083 205.715133666992 30.0
46.3085063081691 63.8246383666992 31.0
9.30386468980555 7.19010305404663 32.0
19.8506793863185 12.9460248947144 33.0
19.8506793863185 13.6501321792603 34.0
19.850679411427 22.1063594818115 35.0
8.19801294945471 7.37389850616455 36.0
7.02365271983579 6.53996753692627 37.0
-7.19083610000001e-09 -14.6133298873901 38.0
-7.19083610000001e-09 9.72542381286621 39.0
-14.1586521526537 -19.9678897857666 40.0
-28.1561169624771 -26.1425552368164 41.0
6.16095936544857 -3.18109512329102 42.0
-36.9019809359486 -25.6850032806396 43.0
60.9542831089486 73.3307800292969 44.0
60.9542829201705 51.3957366943359 45.0
7.3749640408 6.6553635597229 46.0
2.9186157837 -2.00330328941345 47.0
131.29217535 194.840347290039 48.0
124.76139036 121.299278259277 49.0
5.4278883066 3.56686043739319 50.0
92.606856852 73.3419647216797 51.0
81.821466949 82.5123596191406 52.0
75.7700025 66.3910903930664 53.0
75.7700025 65.5056457519531 54.0
75.7700025 77.4005889892578 55.0
7.1656702734e-09 9.00715923309326 56.0
59.490537167 64.4880142211914 57.0
192.01886606 208.727355957031 58.0
223.21833491 187.188812255859 59.0
19.850679411 28.0587978363037 60.0
0 35.9724349975586 61.0
147.982033057583 157.456268310547 0.0
147.982033057583 151.96012878418 1.0
146.808413813287 153.320571899414 2.0
142.01934397 150.517456054688 3.0
-1.55484310406e-09 -0.624643623828888 4.0
-1.55484309999997e-09 -2.83117961883545 5.0
-64.9693040339936 -64.312385559082 6.0
77.6547829144777 74.1870651245117 7.0
61.2786926634778 55.4030647277832 8.0
16.3760902514777 7.20984888076782 9.0
97.5261141374777 89.7391510009766 10.0
83.6472874924777 79.6450042724609 11.0
85.2981411144777 83.8642730712891 12.0
85.2981411144777 76.4624557495117 13.0
85.2981411144777 81.5436782836914 14.0
2.3328021736e-08 -2.03496789932251 15.0
2.3328021735e-08 -1.6052393913269 16.0
5.464350236e-08 -2.72059202194214 17.0
56.2518018559467 59.0783615112305 18.0
-56.2518018797166 -57.0891418457031 19.0
-56.2518018707166 -59.2854194641113 20.0
-56.2518017707166 -58.115608215332 21.0
77.1875821073724 73.3798065185547 22.0
153.454771802806 145.008895874023 23.0
115.220596133508 115.750228881836 24.0
21.7005584165075 20.8819332122803 25.0
230.642353915167 228.381210327148 26.0
230.642353915167 229.708221435547 27.0
26.8883420391666 21.3647422790527 28.0
257.530695955167 253.23681640625 29.0
257.530695955167 255.336776733398 30.0
46.7385281475469 45.537841796875 31.0
9.28724497493877 8.4355525970459 32.0
20.9357803369809 22.6376686096191 33.0
20.9357803369809 19.0790691375732 34.0
20.935780467613 20.4264030456543 35.0
8.86054244871795 8.39873600006104 36.0
7.60121916461794 6.85336446762085 37.0
2.62465480000001e-09 1.1959273815155 38.0
2.62465480000001e-09 -1.17553234100342 39.0
-11.6034306368583 -12.5643882751465 40.0
-24.5599220844463 -26.7411251068115 41.0
4.75986482786468 1.92820739746094 42.0
-31.7734218810648 -31.6524333953857 43.0
89.4696777832616 86.4044494628906 44.0
89.4696781210649 91.9600372314453 45.0
7.05809588635643 4.443039894104 46.0
2.46335994485643 2.26466846466064 47.0
147.98203306 156.098068237305 48.0
142.01934397 152.81867980957 49.0
0 -3.76089572906494 50.0
77.654782914 76.0241851806641 51.0
97.526114137 92.0755157470703 52.0
85.298141114 81.5651702880859 53.0
85.298141114 80.2131118774414 54.0
85.298141114 79.9203643798828 55.0
5.4643502359e-08 3.07315587997437 56.0
77.187582108 79.4883346557617 57.0
230.64235391 222.583419799805 58.0
257.53069595 249.491134643555 59.0
20.935780468 19.6116352081299 60.0
2.6246548239e-09 3.64096260070801 61.0
175.05640447 158.047271728516 0.0
175.05640447 162.89387512207 1.0
173.86983802 167.242279052734 2.0
168.65913708 168.698623657227 3.0
0 0.889920651912689 4.0
-4.1281766273e-10 2.32937526702881 5.0
-68.631844265 -66.3096923828125 6.0
79.017161661 75.489143371582 7.0
47.789810844 49.8381233215332 8.0
31.227350809 24.71116065979 9.0
86.753133177 86.4171981811523 10.0
73.521080011 75.5051727294922 11.0
81.326417839 81.1204376220703 12.0
81.326417839 79.945556640625 13.0
81.326417839 78.0863037109375 14.0
0 0.194886147975922 15.0
0 0.441041111946106 16.0
-1.2762600089e-09 0.898068130016327 17.0
58.179979465 59.8065986633301 18.0
-58.179979488 -59.6823425292969 19.0
-58.179979466 -59.6734619140625 20.0
-58.179979478 -60.8730735778809 21.0
79.799760093 77.9290008544922 22.0
146.48581339 141.393615722656 23.0
112.31755034 110.322265625 24.0
22.818628662 21.3721466064453 25.0
226.2855735 223.709777832031 26.0
226.2855735 219.762481689453 27.0
26.16727697 23.9745922088623 28.0
252.45285047 255.15885925293 29.0
252.45285047 257.40869140625 30.0
46.933708058 48.1185150146484 31.0
9.3144364392 8.44257068634033 32.0
21.61978061 18.908634185791 33.0
21.61978061 18.9554576873779 34.0
21.619780609 19.3043079376221 35.0
9.2718822657 7.30234718322754 36.0
8.0060640176 8.38574600219727 37.0
-1.6395844448e-09 0.38869833946228 38.0
-1.6395844447e-09 -1.42251467704773 39.0
-11.645049069 -13.5318040847778 40.0
-24.09040202 -23.3780937194824 41.0
4.9097031261 5.66740608215332 42.0
-31.543361217 -33.8258323669434 43.0
92.689325955 91.2669372558594 44.0
92.689325964 94.9345016479492 45.0
6.5748148242 7.28198480606079 46.0
2.4864486216 2.73059940338135 47.0
175.05640448 163.227355957031 48.0
168.65913709 162.038284301758 49.0
0 -1.35363626480103 50.0
79.017161668 76.3994903564453 51.0
86.753133184 90.2097473144531 52.0
81.326417847 81.2863235473633 53.0
81.326417847 81.3412857055664 54.0
81.326417847 79.7621231079102 55.0
0 -0.53716379404068 56.0
79.799760103 80.1216735839844 57.0
226.2855735 219.242980957031 58.0
252.45285048 250.191162109375 59.0
21.619780618 20.7179851531982 60.0
0 1.77052474021912 61.0
149.74300524 162.196640014648 0.0
149.74300524 164.963790893555 1.0
148.60292447 157.309143066406 2.0
143.81385475 144.935913085938 3.0
-2.0689067385e-09 0.731975972652435 4.0
-2.0695859295e-09 2.81129240989685 5.0
-59.172436536 -63.2283973693848 6.0
69.863961297 72.3412857055664 7.0
50.351923902 51.8477439880371 8.0
19.512037386 19.549036026001 9.0
89.061401272 89.3783111572266 10.0
76.646541462 77.4210739135742 11.0
82.271461616 84.4279251098633 12.0
82.271461616 80.3724746704102 13.0
82.271461616 80.786750793457 14.0
0 -1.98374891281128 15.0
0 -0.383570730686188 16.0
0 -1.99984955787659 17.0
58.587428012 58.7384834289551 18.0
-58.587428022 -59.7323417663574 19.0
-58.587428 -58.4210662841797 20.0
-58.587428012 -61.8523635864258 21.0
80.815527246 78.6102600097656 22.0
149.14495043 138.846939086914 23.0
114.74863874 109.534286499023 24.0
23.494457945 22.0825862884521 25.0
229.96047769 223.637069702148 26.0
229.96047769 225.190765380859 27.0
25.257363317 27.0872573852539 28.0
255.21784102 254.795501708984 29.0
255.21784102 254.561431884766 30.0
46.445612732 46.0470008850098 31.0
9.3080716648 8.71134281158447 32.0
22.228099224 20.3574371337891 33.0
22.228099224 19.8917560577393 34.0
22.228099224 21.5110111236572 35.0
9.7988489677 8.20902442932129 36.0
8.571063741 8.01464653015137 37.0
-1.9840740013e-13 -1.75648212432861 38.0
-2.0075265721e-13 -2.94999742507935 39.0
-11.596625147 -13.4047613143921 40.0
-24.687040939 -26.1258544921875 41.0
5.2629816457 4.64785957336426 42.0
-32.514995078 -33.0170478820801 43.0
93.869330832 92.4433822631836 44.0
93.869330845 95.4139099121094 45.0
6.9110970773 5.80423069000244 46.0
2.4109034906 1.53973054885864 47.0
149.74300525 163.100250244141 48.0
143.81385476 136.874740600586 49.0
0 -1.35773801803589 50.0
69.863961307 75.5542221069336 51.0
89.061401282 89.7482452392578 52.0
82.271461626 80.341438293457 53.0
82.271461626 80.9588012695312 54.0
82.271461626 79.8151702880859 55.0
0 2.55020761489868 56.0
80.815527256 79.0756912231445 57.0
229.9604777 218.848068237305 58.0
255.21784103 249.596008300781 59.0
22.228099234 20.0694141387939 60.0
0 1.23250555992126 61.0
147.982033057583 157.456314086914 0.0
147.982033057583 151.961029052734 1.0
146.808413813287 153.327545166016 2.0
142.01934397 150.566131591797 3.0
-1.55484310406e-09 -0.630645573139191 4.0
-1.55484309999997e-09 -2.83325052261353 5.0
-64.9693040339936 -64.3144454956055 6.0
77.6547829144777 74.1870880126953 7.0
61.2786926634778 55.4030647277832 8.0
16.3760902514777 7.20984888076782 9.0
97.5261141374777 89.7391510009766 10.0
83.6472874924777 79.6450042724609 11.0
85.2981411144777 83.8642730712891 12.0
85.2981411144777 76.4624557495117 13.0
85.2981411144777 81.5436782836914 14.0
2.3328021736e-08 -2.03496789932251 15.0
2.3328021735e-08 -1.6052393913269 16.0
5.464350236e-08 -2.72059202194214 17.0
56.2518018559467 59.0783615112305 18.0
-56.2518018797166 -57.0891418457031 19.0
-56.2518018707166 -59.2854194641113 20.0
-56.2518017707166 -58.115608215332 21.0
77.1875821073724 73.3798065185547 22.0
153.454771802806 145.008895874023 23.0
115.220596133508 115.750228881836 24.0
21.7005584165075 20.8819332122803 25.0
230.642353915167 228.381210327148 26.0
230.642353915167 229.708221435547 27.0
26.8883420391666 21.3647422790527 28.0
257.530695955167 253.23681640625 29.0
257.530695955167 255.336776733398 30.0
46.7385281475469 45.537841796875 31.0
9.28724497493877 8.4355525970459 32.0
20.9357803369809 22.6376686096191 33.0
20.9357803369809 19.0790691375732 34.0
20.935780467613 20.4264030456543 35.0
8.86054244871795 8.39873600006104 36.0
7.60121916461794 6.85336446762085 37.0
2.62465480000001e-09 1.1959273815155 38.0
2.62465480000001e-09 -1.17553234100342 39.0
-11.6034306368583 -12.5643882751465 40.0
-24.5599220844463 -26.7411251068115 41.0
4.75986482786468 1.92820739746094 42.0
-31.7734218810648 -31.6524333953857 43.0
89.4696777832616 86.4044494628906 44.0
89.4696781210649 91.9600372314453 45.0
7.05809588635643 4.443039894104 46.0
2.46335994485643 2.26466846466064 47.0
147.98203306 156.097717285156 48.0
142.01934397 152.859603881836 49.0
0 -3.75777673721313 50.0
77.654782914 76.024055480957 51.0
97.526114137 92.0755157470703 52.0
85.298141114 81.5651702880859 53.0
85.298141114 80.2131118774414 54.0
85.298141114 79.9203643798828 55.0
5.4643502359e-08 3.07315587997437 56.0
77.187582108 79.4883346557617 57.0
230.64235391 222.583419799805 58.0
257.53069595 249.491134643555 59.0
20.935780468 19.6116352081299 60.0
2.6246548239e-09 3.64096260070801 61.0
155.9515958 158.761871337891 0.0
155.9515958 161.552047729492 1.0
154.57709238 166.050643920898 2.0
149.78802215 167.815277099609 3.0
0 1.69625949859619 4.0
0 4.26820755004883 5.0
-58.820004579 -60.1047286987305 6.0
70.091633906 77.1000213623047 7.0
52.84159464 50.0321350097656 8.0
17.250039256 23.1166610717773 9.0
86.175303871 83.5700225830078 10.0
73.763020481 73.6824798583984 11.0
80.221660628 78.245491027832 12.0
80.221660628 78.6270904541016 13.0
80.221660628 79.8447570800781 14.0
0 -1.54206323623657 15.0
0 -0.581657707691193 16.0
0 -1.07288432121277 17.0
60.156580567 59.3688850402832 18.0
-60.156580577 -58.8717002868652 19.0
-60.156580555 -58.3993797302246 20.0
-60.156580568 -59.3029632568359 21.0
81.030113508 78.3415222167969 22.0
142.62343296 139.917678833008 23.0
107.55639392 107.150863647461 24.0
22.553402363 20.098575592041 25.0
223.65354648 224.046173095703 26.0
223.65354648 223.720199584961 27.0
27.857738575 23.5725154876709 28.0
251.51128506 255.02229309082 29.0
251.51128506 256.141540527344 30.0
46.64135947 46.6488189697266 31.0
9.3104949216 8.98957538604736 32.0
20.873532931 17.2766151428223 33.0
20.873532931 22.2694683074951 34.0
20.873532931 20.0050506591797 35.0
9.2636577405 6.59417009353638 36.0
7.9875683739 7.7546558380127 37.0
-5.5727330635e-11 -2.11790919303894 38.0
-5.5727326372e-11 -3.50058078765869 39.0
-11.556082016 -13.4613647460938 40.0
-24.060686796 -25.9978294372559 41.0
5.3170338411 3.93987369537354 42.0
-31.943974983 -32.7224235534668 43.0
102.27302073 95.5835952758789 44.0
102.27302075 91.9066696166992 45.0
7.3010676096 6.24856805801392 46.0
2.7150235208 2.1412034034729 47.0
155.95159581 166.415267944336 48.0
149.78802216 149.850845336914 49.0
0 2.77348327636719 50.0
70.091633916 76.5297393798828 51.0
86.175303881 87.8432540893555 52.0
80.221660638 80.0057525634766 53.0
80.221660638 78.6451034545898 54.0
80.221660638 80.6831817626953 55.0
0 2.16135334968567 56.0
81.030113518 80.0179061889648 57.0
223.65354649 220.881546020508 58.0
251.51128507 250.301864624023 59.0
20.873532941 20.8628005981445 60.0
0 1.57388949394226 61.0
193.74743458 161.006546020508 0.0
193.74743458 169.974227905273 1.0
192.26112285 170.249908447266 2.0
187.47205313 155.828231811523 3.0
0 62.9340057373047 4.0
-1.5500005758e-11 -1.46593189239502 5.0
-59.170737721 -57.3060722351074 6.0
67.981301975 69.3440704345703 7.0
50.801036555 55.084156036377 8.0
17.18026541 42.0902938842773 9.0
80.60791345 89.4135437011719 10.0
67.187792988 75.4097213745117 11.0
77.200616673 79.3903884887695 12.0
77.200616673 79.7898483276367 13.0
77.200616673 78.0370635986328 14.0
0 0.524172425270081 15.0
0 0.356610119342804 16.0
-3.0789439856e-11 5.35756587982178 17.0
61.169110979 60.0959129333496 18.0
-61.169110999 -58.0934829711914 19.0
-61.169110977 -55.479606628418 20.0
-61.16911099 -61.8921394348145 21.0
82.368088346 82.9021911621094 22.0
148.41919822 140.862213134766 23.0
115.06825702 102.917236328125 24.0
23.270491976 20.276252746582 25.0
230.78728657 224.030563354492 26.0
230.78728657 224.760726928711 27.0
24.71447785 25.1674041748047 28.0
255.50176443 257.416534423828 29.0
255.50176443 254.730758666992 30.0
46.903718362 47.8527412414551 31.0
9.3965485599 9.23358058929443 32.0
21.198977347 18.4731750488281 33.0
21.198977347 16.0892372131348 34.0
21.198977347 16.7273693084717 35.0
10.014368724 7.9159893989563 36.0
8.7283996713 8.09894180297852 37.0
-3.2342910638e-12 3.56922101974487 38.0
-3.2337165534e-12 3.24809980392456 39.0
-11.265223443 -17.1737785339355 40.0
-24.774508429 -25.0396366119385 41.0
5.4987536196 0.516071200370789 42.0
-32.730757149 -31.7689914703369 43.0
98.182308392 91.5102081298828 44.0
98.182308406 90.2717742919922 45.0
6.9879104268 3.32054257392883 46.0
2.4755071244 1.10750114917755 47.0
193.74743459 166.589050292969 48.0
187.47205314 155.339584350586 49.0
0 10.1205749511719 50.0
67.981301985 67.3926620483398 51.0
80.60791346 89.8803253173828 52.0
77.200616683 80.3043441772461 53.0
77.200616683 81.2972412109375 54.0
77.200616683 81.8302917480469 55.0
0 4.57571887969971 56.0
82.368088356 79.931526184082 57.0
230.78728658 222.874664306641 58.0
255.50176444 246.720397949219 59.0
21.198977357 22.9484462738037 60.0
0 3.50199699401855 61.0
137.17687742 143.006530761719 0.0
137.17687742 143.030563354492 1.0
135.86355256 143.538650512695 2.0
131.0744823 135.772583007812 3.0
0 -2.1230001449585 4.0
0 0.342061758041382 5.0
-59.42861149 -59.2355995178223 6.0
69.705106679 70.1772918701172 7.0
65.43029471 62.4528312683105 8.0
4.2748119615 8.48783111572266 9.0
78.669161727 80.2065200805664 10.0
65.229460649 71.2585144042969 11.0
77.500548395 80.8047180175781 12.0
77.500548395 77.9000701904297 13.0
77.500548395 76.9198608398438 14.0
0 -0.64198762178421 15.0
0 0.28728449344635 16.0
0 -3.15485000610352 17.0
59.754945382 58.5015258789062 18.0
-59.754945394 -59.4383087158203 19.0
-59.754945372 -60.0649642944336 20.0
-59.754945384 -61.8801422119141 21.0
80.656571382 77.8813858032227 22.0
140.87903669 140.394561767578 23.0
106.41719675 107.357681274414 24.0
22.90053999 20.4083709716797 25.0
221.53560808 223.594223022461 26.0
221.53560808 221.987747192383 27.0
27.282645659 25.3551406860352 28.0
248.81825375 253.574676513672 29.0
248.81825375 252.408157348633 30.0
47.033013387 47.2299919128418 31.0
9.3131031344 9.18030452728271 32.0
20.901625993 20.4819431304932 33.0
20.901625993 21.1129970550537 34.0
20.901625993 20.8195400238037 35.0
9.3083381265 8.28715133666992 36.0
8.044767894 8.38281917572021 37.0
-2.5262790395e-09 -0.804296553134918 38.0
-2.5262785966e-09 0.113371789455414 39.0
-12.387792857 -14.5510091781616 40.0
-23.81592281 -27.5234241485596 41.0
4.9278776077 3.5162615776062 42.0
-31.108913619 -34.7394485473633 43.0
92.052766479 91.3064651489258 44.0
92.05276649 90.2343292236328 45.0
6.7273095346 3.00315618515015 46.0
2.6568330431 1.83220052719116 47.0
137.17687743 136.681655883789 48.0
131.07448231 127.552749633789 49.0
0 3.54248547554016 50.0
69.705106686 70.9861755371094 51.0
78.669161734 84.2529830932617 52.0
77.500548402 81.0479965209961 53.0
77.500548402 80.9212951660156 54.0
77.500548402 79.4541549682617 55.0
0 -0.0822644829750061 56.0
80.656571391 80.3950500488281 57.0
221.53560809 218.26057434082 58.0
248.81825375 246.469757080078 59.0
20.901626002 19.9398994445801 60.0
0 2.89454770088196 61.0
163.21019338 156.904968261719 0.0
163.21019338 160.027648925781 1.0
162.13291783 158.540023803711 2.0
157.34384811 165.640472412109 3.0
0 -3.39346551895142 4.0
-2.6575017708e-10 2.82552671432495 5.0
-57.070536688 -45.0189170837402 6.0
68.377124439 81.7586975097656 7.0
42.05560288 50.7922821044922 8.0
26.32152155 24.0542373657227 9.0
85.180823624 102.963668823242 10.0
71.764497626 77.6302337646484 11.0
86.784890536 82.7948532104492 12.0
86.784890536 79.5152816772461 13.0
86.784890536 81.1663284301758 14.0
0 -1.38389801979065 15.0
0 -1.30660533905029 16.0
0 -3.19053030014038 17.0
53.845706319 59.040153503418 18.0
-53.845706333 -59.5292282104492 19.0
-53.845706312 -56.6618347167969 20.0
-53.845706324 -58.52197265625 21.0
76.472319104 77.8278503417969 22.0
137.34531213 146.705673217773 23.0
105.23721232 104.183837890625 24.0
22.970900431 20.9522037506104 25.0
213.81763124 227.177154541016 26.0
213.81763124 230.356597900391 27.0
29.690470297 21.1811180114746 28.0
243.50810155 255.470336914062 29.0
243.50810155 258.360717773438 30.0
46.273089087 44.6531562805176 31.0
9.4128103892 8.67997074127197 32.0
22.626612772 17.0356483459473 33.0
22.626612772 20.0318946838379 34.0
22.626612772 23.1199531555176 35.0
9.2333805794 5.3756685256958 36.0
8.0211882633 7.87395715713501 37.0
-8.2687492362e-10 -6.03721141815186 38.0
-8.2687492362e-10 -3.70010757446289 39.0
-11.886949603 -13.1104164123535 40.0
-24.518890588 -22.9885635375977 41.0
4.7816001544 7.0728120803833 42.0
-31.672323387 -35.900333404541 43.0
94.744469039 92.9986267089844 44.0
94.744469051 95.8764190673828 45.0
6.7767909481 6.85640478134155 46.0
1.8964593914 2.79343247413635 47.0
163.21019339 166.855224609375 48.0
157.34384812 161.147888183594 49.0
0 -2.01153326034546 50.0
68.377124448 70.8328323364258 51.0
85.180823633 79.6351699829102 52.0
86.784890544 81.0298461914062 53.0
86.784890544 77.4944152832031 54.0
86.784890544 77.8703689575195 55.0
0 4.08803749084473 56.0
76.472319113 79.5845184326172 57.0
213.81763125 222.935165405273 58.0
243.50810156 250.826385498047 59.0
22.626612781 22.0172710418701 60.0
0 2.44450354576111 61.0
147.982033057583 157.456359863281 0.0
147.982033057583 151.963317871094 1.0
146.808413813287 153.346160888672 2.0
142.01934397 150.698165893555 3.0
-1.55484310406e-09 -0.646620571613312 4.0
-1.55484309999997e-09 -2.83890986442566 5.0
-64.9693040339936 -64.3200607299805 6.0
77.6547829144777 74.1871795654297 7.0
61.2786926634778 55.4030647277832 8.0
16.3760902514777 7.20984888076782 9.0
97.5261141374777 89.7391510009766 10.0
83.6472874924777 79.6450042724609 11.0
85.2981411144777 83.8642730712891 12.0
85.2981411144777 76.4624557495117 13.0
85.2981411144777 81.5436782836914 14.0
2.3328021736e-08 -2.03496789932251 15.0
2.3328021735e-08 -1.6052393913269 16.0
5.464350236e-08 -2.72059202194214 17.0
56.2518018559467 59.0783615112305 18.0
-56.2518018797166 -57.0891418457031 19.0
-56.2518018707166 -59.2854194641113 20.0
-56.2518017707166 -58.115608215332 21.0
77.1875821073724 73.3798065185547 22.0
153.454771802806 145.008895874023 23.0
115.220596133508 115.750228881836 24.0
21.7005584165075 20.8819332122803 25.0
230.642353915167 228.381210327148 26.0
230.642353915167 229.708221435547 27.0
26.8883420391666 21.3647422790527 28.0
257.530695955167 253.23681640625 29.0
257.530695955167 255.336776733398 30.0
46.7385281475469 45.537841796875 31.0
9.28724497493877 8.4355525970459 32.0
20.9357803369809 22.6376686096191 33.0
20.9357803369809 19.0790691375732 34.0
20.935780467613 20.4264030456543 35.0
8.86054244871795 8.39873600006104 36.0
7.60121916461794 6.85336446762085 37.0
2.62465480000001e-09 1.1959273815155 38.0
2.62465480000001e-09 -1.17553234100342 39.0
-11.6034306368583 -12.5643882751465 40.0
-24.5599220844463 -26.7411251068115 41.0
4.75986482786468 1.92820739746094 42.0
-31.7734218810648 -31.6524333953857 43.0
89.4696777832616 86.4044494628906 44.0
89.4696781210649 91.9600372314453 45.0
7.05809588635643 4.443039894104 46.0
2.46335994485643 2.26466846466064 47.0
147.98203306 156.096786499023 48.0
142.01934397 152.969879150391 49.0
0 -3.74933815002441 50.0
77.654782914 76.0237503051758 51.0
97.526114137 92.0755157470703 52.0
85.298141114 81.5651702880859 53.0
85.298141114 80.2131118774414 54.0
85.298141114 79.9203643798828 55.0
5.4643502359e-08 3.07315587997437 56.0
77.187582108 79.4883346557617 57.0
230.64235391 222.583419799805 58.0
257.53069595 249.491134643555 59.0
20.935780468 19.6116352081299 60.0
2.6246548239e-09 3.64096260070801 61.0
207.33112182 151.057998657227 0.0
207.33112182 160.373062133789 1.0
206.26681161 118.335189819336 2.0
205.20177509 363.828094482422 3.0
-6.2266179105e-14 129.975357055664 4.0
-1.7242380797e-12 87.7270584106445 5.0
-79.720747063 13.1375980377197 6.0
90.113817521 96.2265014648438 7.0
57.933688317 55.4162902832031 8.0
32.180129194 44.277904510498 9.0
37.282471236 68.2573165893555 10.0
25.703815044 76.4301452636719 11.0
36.347091839 86.7657318115234 12.0
36.347091839 93.8886260986328 13.0
36.347091839 83.4907760620117 14.0
31.762962145 4.0364089012146 15.0
31.762962145 0.0255515575408936 16.0
-8.4902221915e-13 4.5118260383606 17.0
65.126256677 21.5293254852295 18.0
-65.126256697 -49.1938705444336 19.0
-65.898511807 -53.3093948364258 20.0
-65.972969217 -60.3564529418945 21.0
82.733259206 64.6162109375 22.0
151.91053792 134.790924072266 23.0
122.35766704 106.200942993164 24.0
22.605745976 23.3355350494385 25.0
234.64379713 214.928344726562 26.0
234.64379713 161.5107421875 27.0
28.665875307 16.5078525543213 28.0
263.30967245 248.261138916016 29.0
263.30967245 262.544128417969 30.0
46.542123363 39.3439865112305 31.0
9.5954381231 7.68420696258545 32.0
16.760289979 13.1359949111938 33.0
16.760289979 23.1713733673096 34.0
16.760289979 12.6223812103271 35.0
9.5567332476 10.2268171310425 36.0
8.2393408304 7.88309383392334 37.0
-6.6129727112e-12 4.61397504806519 38.0
-6.6162229666e-12 2.24675035476685 39.0
-11.41939128 -15.4827032089233 40.0
-25.032594995 -26.4645137786865 41.0
4.4565703624 4.31735038757324 42.0
-32.027102905 -32.4000282287598 43.0
91.686921833 67.3941345214844 44.0
91.686921847 120.748497009277 45.0
5.7379018289 4.17728900909424 46.0
2.1585055949 2.2637882232666 47.0
207.33112183 203.754135131836 48.0
205.2017751 138.138473510742 49.0
0 -1.336585521698 50.0
90.113817531 78.1714096069336 51.0
37.282471246 91.0510330200195 52.0
36.347091849 71.4832305908203 53.0
36.347091849 74.9620361328125 54.0
36.347091849 76.4087677001953 55.0
0 -0.0811017155647278 56.0
82.733259216 83.0273895263672 57.0
234.64379714 186.550201416016 58.0
263.30967246 251.74787902832 59.0
16.760289989 22.8362655639648 60.0
0 1.27411901950836 61.0
137.17687742 143.006134033203 0.0
137.17687742 143.046112060547 1.0
135.86355256 143.429351806641 2.0
131.0744823 135.570373535156 3.0
0 -1.89706754684448 4.0
0 0.340481162071228 5.0
-59.42861149 -59.2070846557617 6.0
69.705106679 70.1769409179688 7.0
65.43029471 62.4528312683105 8.0
4.2748119615 8.48783111572266 9.0
78.669161727 80.2065200805664 10.0
65.229460649 71.2585144042969 11.0
77.500548395 80.8047180175781 12.0
77.500548395 77.9000701904297 13.0
77.500548395 76.9198608398438 14.0
0 -0.64198762178421 15.0
0 0.28728449344635 16.0
0 -3.15485000610352 17.0
59.754945382 58.5015258789062 18.0
-59.754945394 -59.4383087158203 19.0
-59.754945372 -60.0649642944336 20.0
-59.754945384 -61.8801422119141 21.0
80.656571382 77.8813858032227 22.0
140.87903669 140.394561767578 23.0
106.41719675 107.357681274414 24.0
22.90053999 20.4083709716797 25.0
221.53560808 223.594223022461 26.0
221.53560808 221.987747192383 27.0
27.282645659 25.3551406860352 28.0
248.81825375 253.574676513672 29.0
248.81825375 252.408157348633 30.0
47.033013387 47.2299919128418 31.0
9.3131031344 9.18030452728271 32.0
20.901625993 20.4819431304932 33.0
20.901625993 21.1129970550537 34.0
20.901625993 20.8195400238037 35.0
9.3083381265 8.28715133666992 36.0
8.044767894 8.38281917572021 37.0
-2.5262790395e-09 -0.804296553134918 38.0
-2.5262785966e-09 0.113371789455414 39.0
-12.387792857 -14.5510091781616 40.0
-23.81592281 -27.5234241485596 41.0
4.9278776077 3.5162615776062 42.0
-31.108913619 -34.7394485473633 43.0
92.052766479 91.3064651489258 44.0
92.05276649 90.2343292236328 45.0
6.7273095346 3.00315618515015 46.0
2.6568330431 1.83220052719116 47.0
137.17687743 136.690536499023 48.0
131.07448231 127.224128723145 49.0
0 3.58906507492065 50.0
69.705106686 70.9878616333008 51.0
78.669161734 84.2529830932617 52.0
77.500548402 81.0479965209961 53.0
77.500548402 80.9212951660156 54.0
77.500548402 79.4541549682617 55.0
0 -0.0822644829750061 56.0
80.656571391 80.3950500488281 57.0
221.53560809 218.26057434082 58.0
248.81825375 246.469757080078 59.0
20.901626002 19.9398994445801 60.0
0 2.89454770088196 61.0
189.304518208508 170.960205078125 0.0
189.304518208508 167.69563293457 1.0
187.82453698 167.274978637695 2.0
183.03546725 183.901794433594 3.0
0 -15.534384727478 4.0
-6.9856735579e-09 6.07494783401489 5.0
-63.410598325 -62.4085960388184 6.0
72.6696130880802 80.3980102539062 7.0
43.7925234060802 53.0764694213867 8.0
28.8770896820802 27.8406219482422 9.0
87.5332455400802 84.5579071044922 10.0
75.3478096880802 77.6870422363281 11.0
78.7610531000802 83.2871170043945 12.0
78.7610531000802 79.5605316162109 13.0
78.7610531000802 83.0144119262695 14.0
0 0.913967609405518 15.0
0 0.0715634822845459 16.0
-8.2996972475e-09 0.358518540859222 17.0
55.2075620801346 58.3616905212402 18.0
-55.2075620951202 -57.9902229309082 19.0
-55.2075620731202 -56.584228515625 20.0
-55.2075620831202 -64.4827346801758 21.0
77.779042283 75.5705108642578 22.0
141.965920259135 143.463714599609 23.0
110.202033436418 109.971092224121 24.0
22.8799597924183 20.2720470428467 25.0
219.744962547016 214.326400756836 26.0
219.744962547016 212.163818359375 27.0
28.1173903310156 28.1757984161377 28.0
247.862352877016 251.879241943359 29.0
247.862352877016 249.825134277344 30.0
46.29745722 46.6723289489746 31.0
9.4363286623 9.51933860778809 32.0
22.5714802049258 20.0208301544189 33.0
22.5714802049258 22.6639537811279 34.0
22.571480205 24.005802154541 35.0
10.185368379 9.82302474975586 36.0
8.9598749127 6.81504440307617 37.0
-9.684717302e-09 2.88863563537598 38.0
-9.684717301e-09 1.16648244857788 39.0
-12.3807243210601 -8.84003734588623 40.0
-26.0749796368713 -22.7343235015869 41.0
5.29033539550013 3.5368766784668 42.0
-34.2783496834001 -33.0724830627441 43.0
108.7660487334 91.2919998168945 44.0
108.7660487334 101.049751281738 45.0
6.2330646929 6.04404783248901 46.0
2.4510011124 2.25894284248352 47.0
189.30451821 163.618545532227 48.0
183.03546726 170.559158325195 49.0
0 -3.62745809555054 50.0
72.669613088 76.8121871948242 51.0
87.53324554 90.5131759643555 52.0
78.7610531 80.9618835449219 53.0
78.7610531 80.9442367553711 54.0
78.7610531 80.0259017944336 55.0
0 4.48706579208374 56.0
77.779042288 79.1200485229492 57.0
219.74496254 220.499572753906 58.0
247.86235287 245.665054321289 59.0
22.571480208 20.6870174407959 60.0
0 3.12471985816956 61.0
171.904389231121 166.616256713867 0.0
171.904389231121 166.213485717773 1.0
170.57179709 162.469909667969 2.0
165.10716188 168.368682861328 3.0
0.39191866934 2.36859154701233 4.0
0.39191867898824 6.36576890945435 5.0
-68.3561336756982 -64.2236557006836 6.0
78.6411639729438 73.9886245727539 7.0
46.7850924509438 52.7926597595215 8.0
31.8560715229438 23.126443862915 9.0
88.4799369339438 89.8842849731445 10.0
75.7534711339438 76.729362487793 11.0
81.7694108229438 84.4457168579102 12.0
81.7694108229438 79.9000854492188 13.0
81.7694108229438 78.5475311279297 14.0
0 -1.12946653366089 15.0
0 -0.0269544124603271 16.0
-5.173094624e-09 -0.331485092639923 17.0
60.3669782016966 57.9027328491211 18.0
-60.3669782079158 -58.2627906799316 19.0
-60.3669781869158 -58.5256156921387 20.0
-60.3669781929158 -59.684196472168 21.0
81.4400825854713 77.7154769897461 22.0
145.427462812697 142.602600097656 23.0
109.693421605871 113.33984375 24.0
22.3461917368708 20.8249912261963 25.0
226.867545402243 219.398635864258 26.0
226.867545402243 223.654418945312 27.0
26.3506673472429 26.6620502471924 28.0
253.218212752243 254.722839355469 29.0
253.218212752243 257.276580810547 30.0
46.4833722894205 47.0532073974609 31.0
9.3272532778 8.6466760635376 32.0
21.0731043920947 19.0303649902344 33.0
21.0731043920947 18.4538803100586 34.0
21.0731044007393 20.0057277679443 35.0
9.3952582204 6.26320457458496 36.0
8.2148379078 7.90796852111816 37.0
-9.168946015e-09 1.63839423656464 38.0
-9.16894601400001e-09 1.38869953155518 39.0
-11.3798843634579 -17.2227268218994 40.0
-24.4111804375477 -25.2907676696777 41.0
5.02627471718653 5.53434610366821 42.0
-31.9317015352865 -33.5254783630371 43.0
98.2812755622865 96.7085266113281 44.0
98.2812755402865 94.7829437255859 45.0
6.7672017412 6.35817337036133 46.0
2.5995187044 2.94870662689209 47.0
171.90438923 166.497833251953 48.0
165.10716189 158.215087890625 49.0
0.39191867899 2.63768434524536 50.0
78.641163973 78.1464538574219 51.0
88.479936934 89.9521331787109 52.0
81.769410823 81.5143051147461 53.0
81.769410823 80.9984283447266 54.0
81.769410823 79.8731079101562 55.0
0 0.379146158695221 56.0
81.440082585 79.5886077880859 57.0
226.8675454 219.88932800293 58.0
253.21821275 249.349853515625 59.0
21.073104401 21.9800567626953 60.0
0 2.63437795639038 61.0
178.26221784 160.780715942383 0.0
178.26221784 166.832794189453 1.0
176.93820213 163.500823974609 2.0
172.14913241 168.300018310547 3.0
0 -2.88749575614929 4.0
-4.4569951226e-09 -0.0803593993186951 5.0
-62.931773827 -63.414134979248 6.0
73.1081538527101 72.7530364990234 7.0
52.3240061737101 48.4903411865234 8.0
20.7841476787101 20.2349014282227 9.0
86.1713706827101 88.0327301025391 10.0
72.8366448787101 75.6551284790039 11.0
79.8137857017101 81.6067428588867 12.0
79.8137857017101 79.0227432250977 13.0
79.8137857017101 79.6179275512695 14.0
0 -0.712208807468414 15.0
0 -0.302031457424164 16.0
-8.1385369218e-09 -1.38565325737 17.0
55.877077086 58.5077171325684 18.0
-55.8770771030652 -60.0913352966309 19.0
-55.8770770810652 -60.7361755371094 20.0
-55.8770770910652 -58.479850769043 21.0
77.477979784 78.7444915771484 22.0
145.0792666 140.275802612305 23.0
111.466747699575 112.392211914062 24.0
21.9740377495751 20.8169403076172 25.0
222.557246392358 220.375061035156 26.0
222.557246392358 222.256439208984 27.0
26.7449192693579 26.6806755065918 28.0
249.302165662358 253.390686035156 29.0
249.302165662358 253.478713989258 30.0
46.872250997 45.880485534668 31.0
9.3294542081 8.61965560913086 32.0
21.6009026988944 20.4084224700928 33.0
21.6009026988944 17.9559726715088 34.0
21.600902697 20.4492359161377 35.0
9.4711134168 7.84704303741455 36.0
8.1797954503 7.89985513687134 37.0
-9.7629943769e-09 -0.233429729938507 38.0
-9.7629943769e-09 -3.82192754745483 39.0
-11.9853726425326 -15.4433012008667 40.0
-24.944443893 -26.4558200836182 41.0
4.95325382475845 4.50580072402954 42.0
-32.5585927562584 -32.5993919372559 43.0
95.2594059692584 90.9521560668945 44.0
95.2594059593937 92.7809677124023 45.0
6.5554655389 6.64472627639771 46.0
2.4257694165 2.56620478630066 47.0
178.26221784 159.235397338867 48.0
172.14913242 160.493316650391 49.0
0 1.49891531467438 50.0
73.108153853 76.6615447998047 51.0
86.171370683 89.3563766479492 52.0
79.813785701 81.1708221435547 53.0
79.813785701 80.2668914794922 54.0
79.813785701 80.03125 55.0
0 0.979378640651703 56.0
77.47797979 80.8110504150391 57.0
222.55724639 220.991775512695 58.0
249.30216566 250.750427246094 59.0
21.600902701 21.1341400146484 60.0
0 0.754475295543671 61.0
167.398219503162 164.30517578125 0.0
167.398219503162 160.867691040039 1.0
166.21192958 157.853805541992 2.0
161.42285981 143.385589599609 3.0
0 2.46065092086792 4.0
-5.70384083999999e-09 -0.500318944454193 5.0
-54.73105681128 -58.7806091308594 6.0
65.55360645148 71.0213394165039 7.0
42.84549014848 26.5824108123779 8.0
22.70811630348 13.1857147216797 9.0
92.53702286548 88.511604309082 10.0
79.14653738348 72.1813278198242 11.0
82.01370652948 75.4453735351562 12.0
82.01370652948 79.6989288330078 13.0
82.01370652948 83.0244216918945 14.0
4.1836207177e-09 -1.79367065429688 15.0
2.4687989636e-09 -1.34356260299683 16.0
4.80564180000003e-10 0.535277366638184 17.0
56.6835814157623 57.489917755127 18.0
-56.68358148722 -59.3170585632324 19.0
-56.68358147022 -58.4264869689941 20.0
-56.68358144622 -63.1539916992188 21.0
79.2587394712108 78.0254516601562 22.0
143.664619131171 142.54426574707 23.0
109.995038246464 115.072708129883 24.0
21.8907466834636 21.3432388305664 25.0
222.923358606323 223.009765625 26.0
222.923358606323 229.539184570312 27.0
29.7463543183231 26.0443687438965 28.0
252.669712916323 245.916015625 29.0
252.669712916323 248.220855712891 30.0
46.819907914868 48.5141563415527 31.0
9.38699087039411 7.87262201309204 32.0
22.57515802484 14.8079099655151 33.0
22.57515802484 16.6938724517822 34.0
22.5751580662345 21.8195934295654 35.0
10.1633471068189 11.5607137680054 36.0
8.89275677601893 7.79754066467285 37.0
-5.99078328999999e-09 -4.22226762771606 38.0
-5.99078328999999e-09 5.9595193862915 39.0
-12.34185612011 -15.9061841964722 40.0
-26.734944673583 -25.207088470459 41.0
4.95428759069998 7.23943996429443 42.0
-34.4592230089 -32.3693923950195 43.0
93.4656577189 92.6826248168945 44.0
93.4656575549 87.560905456543 45.0
7.77480101539574 5.86192798614502 46.0
2.66136748449574 2.5920684337616 47.0
167.3982195 166.16357421875 48.0
161.42285982 152.760177612305 49.0
0 -1.19777774810791 50.0
65.553606452 75.4621047973633 51.0
92.537022865 84.0349807739258 52.0
82.01370653 79.9286804199219 53.0
82.01370653 77.5451812744141 54.0
82.01370653 80.2460021972656 55.0
4.8056418036e-10 0.532793343067169 56.0
79.258739472 80.23388671875 57.0
222.9233586 220.572250366211 58.0
252.66971292 248.091323852539 59.0
22.575158067 28.4421787261963 60.0
0 3.27370643615723 61.0
156.082093 162.214797973633 0.0
156.082093 160.569854736328 1.0
154.79700079 157.84326171875 2.0
150.00793078 150.361633300781 3.0
0 -7.53031826019287 4.0
-1.0724441244e-10 7.68039178848267 5.0
-66.37750158 -61.8665657043457 6.0
78.230632433 75.2372055053711 7.0
51.607899185 51.3782386779785 8.0
26.622733239 25.5275592803955 9.0
86.577696589 88.5163650512695 10.0
74.394369246 76.2100067138672 11.0
82.396526945 78.5201721191406 12.0
82.396526945 78.7479248046875 13.0
82.396526945 80.0785064697266 14.0
0 0.399173855781555 15.0
0 0.28646445274353 16.0
0 -0.830898106098175 17.0
58.491979535 59.7950744628906 18.0
-58.491979546 -59.38916015625 19.0
-58.491979524 -59.7015113830566 20.0
-58.491979536 -56.5009994506836 21.0
80.262375359 77.3691558837891 22.0
143.1616356 142.041641235352 23.0
107.6834575 109.393753051758 24.0
23.33670731 20.3506526947021 25.0
223.42401097 226.571258544922 26.0
223.42401097 225.745025634766 27.0
25.581073566 24.2859859466553 28.0
249.00508454 255.306991577148 29.0
249.00508454 255.668762207031 30.0
46.827142013 47.4622230529785 31.0
9.319480327 8.59048461914062 32.0
21.770395814 20.011022567749 33.0
21.770395814 18.6805019378662 34.0
21.770395813 21.0069904327393 35.0
9.5415861694 6.74989223480225 36.0
8.2112450609 7.28718757629395 37.0
-3.2105106354e-10 -0.435894191265106 38.0
-3.2105106354e-10 -1.58552122116089 39.0
-11.565285609 -17.5057582855225 40.0
-24.098123265 -24.4864330291748 41.0
5.1681339378 6.1418137550354 42.0
-31.833321194 -32.70263671875 43.0
100.79471777 98.6260604858398 44.0
100.79471779 94.2139053344727 45.0
6.9380991087 6.25279998779297 46.0
2.5851825412 2.77737593650818 47.0
156.08209301 166.277221679688 48.0
150.00793079 149.958282470703 49.0
0 -2.6074686050415 50.0
78.230632442 77.5861282348633 51.0
86.577696599 89.5719680786133 52.0
82.396526954 81.063720703125 53.0
82.396526954 79.1731033325195 54.0
82.396526954 80.0159606933594 55.0
0 1.22525072097778 56.0
80.262375369 79.4992065429688 57.0
223.42401098 222.679077148438 58.0
249.00508455 248.612365722656 59.0
21.770395823 20.8162326812744 60.0
0 0.272449195384979 61.0
150.821821044767 157.260406494141 0.0
150.821821044767 163.717010498047 1.0
149.51201117 163.032379150391 2.0
144.72294144 152.788452148438 3.0
0 0.779697835445404 4.0
-6.243268203e-09 -1.29189991950989 5.0
-59.3614038435988 -64.6644134521484 6.0
69.9021673325958 72.3038940429688 7.0
44.9092893085958 51.9030227661133 8.0
24.9928780235958 23.5916767120361 9.0
88.0796744775958 84.6522598266602 10.0
74.8871997725958 72.0070037841797 11.0
84.8673568555958 78.0580596923828 12.0
84.8673568555958 79.8587646484375 13.0
84.8673568555958 82.4039840698242 14.0
0 -1.42144227027893 15.0
0 -1.16838979721069 16.0
-3.89735760700001e-09 -1.93273425102234 17.0
58.5439442952341 59.2381706237793 18.0
-58.5439442963937 -58.8240013122559 19.0
-58.5439442753937 -59.546314239502 20.0
-58.5439442763937 -60.3496894836426 21.0
79.5589895898097 80.4983062744141 22.0
149.543323647234 144.261459350586 23.0
114.811254696543 113.141021728516 24.0
22.6746174235426 20.1010684967041 25.0
229.102313239534 228.623260498047 26.0
229.102313239534 224.030654907227 27.0
27.5192493175336 26.6806945800781 28.0
256.621562559534 255.72102355957 29.0
256.621562559534 253.14501953125 30.0
46.8549618361638 46.0376853942871 31.0
9.39631126072663 8.84011268615723 32.0
21.0150453127964 19.5434322357178 33.0
21.0150453127964 18.1887626647949 34.0
21.0150453259641 21.2907981872559 35.0
9.0464144587821 7.27692604064941 36.0
7.7613646750821 7.89680671691895 37.0
-8.78135544000001e-09 -0.982130110263824 38.0
-8.78135545e-09 -1.16997575759888 39.0
-12.0515858521969 -13.5726737976074 40.0
-25.2157212011335 -24.619556427002 41.0
4.91005240666318 5.24948692321777 42.0
-32.5973597926632 -33.66748046875 43.0
97.8074231666632 96.9871597290039 44.0
97.8074231246632 95.0500183105469 45.0
7.1343647056 6.40090560913086 46.0
2.7529340113 2.30599737167358 47.0
150.82182104 162.913131713867 48.0
144.72294145 150.35302734375 49.0
0 1.8812894821167 50.0
69.902167333 74.2294006347656 51.0
88.079674478 85.3330459594727 52.0
84.867356856 80.8157806396484 53.0
84.867356856 80.9596176147461 54.0
84.867356856 80.2074661254883 55.0
0 1.35114288330078 56.0
79.55898959 80.3739547729492 57.0
229.10231324 221.967224121094 58.0
256.62156255 249.842742919922 59.0
21.015045326 22.4666728973389 60.0
0 1.93557405471802 61.0
179.58374763 154.981033325195 0.0
179.58374763 164.433547973633 1.0
178.09586585 173.550430297852 2.0
173.30679612 162.159240722656 3.0
0 0.638141572475433 4.0
-6.9795269753e-12 -3.6782054901123 5.0
-59.189662051 -55.9802436828613 6.0
69.776022331 67.296028137207 7.0
47.475838939 52.4808235168457 8.0
22.300183382 12.1840448379517 9.0
88.821537977 88.8774795532227 10.0
75.023899228 70.5848770141602 11.0
78.866567502 85.6770172119141 12.0
78.866567502 77.1945495605469 13.0
78.866567502 78.0753402709961 14.0
0 -2.80911064147949 15.0
0 -0.294819176197052 16.0
-9.2702984087e-12 0.503480017185211 17.0
58.026308017 57.4204254150391 18.0
-58.026308037 -59.2049369812012 19.0
-58.026308015 -55.775806427002 20.0
-58.026308028 -58.9349632263184 21.0
79.719705858 75.3375701904297 22.0
140.08396526 138.524032592773 23.0
109.20985555 119.183738708496 24.0
24.393135165 22.033519744873 25.0
219.80367113 229.27326965332 26.0
219.80367113 225.539123535156 27.0
25.878908804 28.4840831756592 28.0
245.68257995 254.022094726562 29.0
245.68257995 256.024749755859 30.0
46.322755174 45.0152359008789 31.0
9.2073720392 8.82196521759033 32.0
21.69339782 20.5651607513428 33.0
21.69339782 14.8337736129761 34.0
21.69339782 21.166561126709 35.0
8.9053135346 13.4904985427856 36.0
7.6967992005 7.88865566253662 37.0
-1.9985095519e-11 -7.89371776580811 38.0
-1.9985094968e-11 -1.83516716957092 39.0
-12.532187562 -19.097936630249 40.0
-25.280158855 -27.7370700836182 41.0
4.9637216629 2.32139158248901 42.0
-32.883775836 -29.6237964630127 43.0
84.32784665 83.1050796508789 44.0
84.327846663 91.8439788818359 45.0
6.4625884563 8.64734268188477 46.0
2.556705248 2.88617825508118 47.0
179.58374764 160.66943359375 48.0
173.30679613 166.356796264648 49.0
0 -1.43274211883545 50.0
69.776022341 63.4874839782715 51.0
88.821537987 83.5063095092773 52.0
78.866567512 80.5438385009766 53.0
78.866567512 81.4388961791992 54.0
78.866567512 81.5377197265625 55.0
0 4.49287080764771 56.0
79.719705868 79.9958419799805 57.0
219.80367114 222.408248901367 58.0
245.68257996 255.153945922852 59.0
21.69339783 24.4930152893066 60.0
0 0.205057203769684 61.0
196.31403567 166.513565063477 0.0
196.31403567 163.632507324219 1.0
195.07511775 173.14990234375 2.0
190.28604803 184.374450683594 3.0
0 -0.112115919589996 4.0
-3.9842847002e-14 0.70617550611496 5.0
-61.755705255 -54.8226432800293 6.0
74.33074435 78.7782897949219 7.0
48.135322234 56.1127243041992 8.0
26.195422106 24.1540985107422 9.0
97.845664129 92.6463165283203 10.0
85.644134651 78.8717193603516 11.0
87.868061031 82.4380187988281 12.0
87.868061031 75.8274536132812 13.0
87.868061031 80.3923492431641 14.0
0 0.10986340045929 15.0
0 0.494659662246704 16.0
-7.5677341987e-14 2.7772068977356 17.0
53.958632454 56.3520812988281 18.0
-53.958632474 -58.7212104797363 19.0
-53.958632452 -60.7751960754395 20.0
-53.958632465 -60.1841773986816 21.0
74.005804484 78.3452301025391 22.0
143.86339625 143.884002685547 23.0
107.73112142 113.381958007812 24.0
22.276499509 19.7274703979492 25.0
217.86920074 229.944107055664 26.0
217.86920074 225.497573852539 27.0
28.270007267 22.215259552002 28.0
246.13920802 253.933853149414 29.0
246.13920802 249.125900268555 30.0
46.414903933 45.3497505187988 31.0
9.2986945269 8.89930629730225 32.0
20.04717201 22.5094509124756 33.0
20.04717201 17.3878841400146 34.0
20.04717201 17.322624206543 35.0
10.185917238 9.45906352996826 36.0
8.8033050615 7.84967231750488 37.0
-1.148898725e-13 1.59054923057556 38.0
-1.1488987273e-13 -1.73623752593994 39.0
-10.534834092 -11.5413331985474 40.0
-24.071630538 -24.4835777282715 41.0
4.6947731847 6.05080461502075 42.0
-31.336128039 -33.7988014221191 43.0
102.76831072 93.1435012817383 44.0
102.76831074 91.7321624755859 45.0
7.103359256 7.71426486968994 46.0
2.6180599651 1.97687458992004 47.0
196.31403568 165.46257019043 48.0
190.28604804 176.148773193359 49.0
0 11.6204395294189 50.0
74.33074436 78.6569213867188 51.0
97.845664139 92.2472686767578 52.0
87.868061041 80.4715957641602 53.0
87.868061041 79.2366790771484 54.0
87.868061041 79.3248672485352 55.0
0 -0.714002430438995 56.0
74.005804494 79.7049865722656 57.0
217.86920075 225.453018188477 58.0
246.13920803 249.588043212891 59.0
20.04717202 21.3780651092529 60.0
0 2.46380305290222 61.0
156.6713181 155.580200195312 0.0
156.6713181 153.993469238281 1.0
155.35669501 163.743011474609 2.0
150.56762529 171.040649414062 3.0
0 0.796224057674408 4.0
-8.2508583107e-11 3.01405358314514 5.0
-64.361672988 -64.9945755004883 6.0
77.036033037 71.0951232910156 7.0
58.131839861 44.9297409057617 8.0
18.904193166 20.0676231384277 9.0
93.396596857 85.5535125732422 10.0
81.116172554 75.229362487793 11.0
86.896080413 81.2087097167969 12.0
86.896080413 81.3138809204102 13.0
86.896080413 75.3482513427734 14.0
0 2.50444602966309 15.0
0 -0.301564037799835 16.0
-1.6017837051e-10 2.54906129837036 17.0
59.672261496 59.9125785827637 18.0
-59.672261517 -58.1255798339844 19.0
-59.672261495 -59.941967010498 20.0
-59.672261507 -56.4147453308105 21.0
81.072120565 72.6730422973633 22.0
146.92085713 143.486389160156 23.0
113.41120244 114.981101989746 24.0
23.793852767 22.2632541656494 25.0
227.9929777 219.011138916016 26.0
227.9929777 223.71435546875 27.0
29.555613189 29.4188594818115 28.0
257.5485909 247.107376098633 29.0
257.5485909 253.453430175781 30.0
46.498981697 44.3253707885742 31.0
9.3093496483 9.8873176574707 32.0
21.399859049 16.7217254638672 33.0
21.399859049 17.6967658996582 34.0
21.399859049 22.8812217712402 35.0
9.8833975679 12.6430921554565 36.0
8.6650482972 7.1448392868042 37.0
-2.6223804236e-10 -0.434596121311188 38.0
-2.6223832629e-10 2.10137701034546 39.0
-11.425641458 -9.90792179107666 40.0
-24.316442634 -29.3452816009521 41.0
5.1309119045 1.98286557197571 42.0
-32.120470683 -32.5556640625 43.0
83.234212436 85.1669235229492 44.0
83.234212448 88.3465042114258 45.0
6.9774120402 9.73092460632324 46.0
2.6592941 3.21241331100464 47.0
156.67131811 162.748138427734 48.0
150.5676253 153.104629516602 49.0
0 -4.18636703491211 50.0
77.036033047 73.7664031982422 51.0
93.396596867 88.487174987793 52.0
86.896080422 81.3423767089844 53.0
86.896080422 77.584358215332 54.0
86.896080422 79.9667739868164 55.0
0 -1.66871166229248 56.0
81.072120575 77.192626953125 57.0
227.99297771 219.784393310547 58.0
257.54859091 251.383544921875 59.0
21.399859059 21.5899105072021 60.0
0 2.1902916431427 61.0
173.48222006 161.578262329102 0.0
173.48222006 163.331939697266 1.0
172.22314636 167.266174316406 2.0
167.43407664 166.624298095703 3.0
0 -3.36571025848389 4.0
0 2.25831246376038 5.0
-60.925447999 -61.7270164489746 6.0
71.757340088 76.3736953735352 7.0
53.963103693 50.3312759399414 8.0
17.794236388 23.8486480712891 9.0
93.275422472 84.7403182983398 10.0
80.514570873 74.4210586547852 11.0
84.707128687 80.9948196411133 12.0
84.707128687 78.0986328125 13.0
84.707128687 76.8569869995117 14.0
0 -1.82194638252258 15.0
0 -0.0762017369270325 16.0
0 -0.500680029392242 17.0
62.017509075 59.6797676086426 18.0
-62.01750909 -59.3171920776367 19.0
-62.017509068 -60.0256080627441 20.0
-62.01750908 -60.1812858581543 21.0
83.758171021 78.2418975830078 22.0
142.62006388 139.097549438477 23.0
109.98802284 115.773361206055 24.0
22.448442191 20.7120990753174 25.0
226.37823492 228.323150634766 26.0
226.37823492 225.910629272461 27.0
26.22733819 27.6501235961914 28.0
252.60557311 254.049514770508 29.0
252.60557311 253.372268676758 30.0
46.753014065 44.893009185791 31.0
9.3756517084 9.05110168457031 32.0
21.740661938 20.3500671386719 33.0
21.740661938 22.270414352417 34.0
21.740661937 19.051908493042 35.0
9.3190110995 7.99687623977661 36.0
8.085393261 7.12882471084595 37.0
-3.0965105066e-09 3.13680648803711 38.0
-3.0965105777e-09 -1.33643507957458 39.0
-11.427791127 -14.1493148803711 40.0
-23.871954779 -24.8282051086426 41.0
5.1234589867 3.50647449493408 42.0
-31.593364576 -32.6555023193359 43.0
91.97137263 90.3484420776367 44.0
91.971372637 94.119010925293 45.0
6.7503169777 7.88937902450562 46.0
2.6459629266 2.58006477355957 47.0
173.48222007 157.88493347168 48.0
167.43407665 163.731369018555 49.0
0 0.578079879283905 50.0
71.757340094 76.4790191650391 51.0
93.275422478 87.3532485961914 52.0
84.707128693 80.56494140625 53.0
84.707128693 80.1917190551758 54.0
84.707128693 78.89599609375 55.0
0 1.92142558097839 56.0
83.75817103 79.4292144775391 57.0
226.37823492 221.616287231445 58.0
252.60557312 250.850006103516 59.0
21.740661945 17.7413444519043 60.0
0 1.75033211708069 61.0
135.23410079 169.788604736328 0.0
135.23410079 166.591918945312 1.0
133.95118823 131.948928833008 2.0
129.1621185 148.057968139648 3.0
0 5.20253276824951 4.0
-8.3907282495e-12 11.7939386367798 5.0
-69.249145189 -51.6119041442871 6.0
81.953309087 89.5547409057617 7.0
54.332931979 58.1504592895508 8.0
27.620377098 28.1583499908447 9.0
87.876710723 87.9928970336914 10.0
75.327632932 71.9663543701172 11.0
80.872710005 79.3334503173828 12.0
80.872710005 81.684455871582 13.0
80.872710005 81.6591110229492 14.0
0 -0.383741915225983 15.0
0 0.362612307071686 16.0
0 0.337352693080902 17.0
54.231342039 58.3456344604492 18.0
-54.231342059 -59.1166954040527 19.0
-54.231342037 -60.2064094543457 20.0
-54.231342049 -62.4636573791504 21.0
76.823953231 79.9689407348633 22.0
150.48951071 141.773864746094 23.0
114.2326215 113.729446411133 24.0
22.245216441 19.4365348815918 25.0
227.31346395 227.807006835938 26.0
227.31346395 228.513565063477 27.0
28.991035429 32.847339630127 28.0
256.30449939 253.661148071289 29.0
256.30449939 251.939041137695 30.0
46.944538783 45.5591087341309 31.0
9.3892131455 8.86161422729492 32.0
22.592611172 19.2100582122803 33.0
22.592611172 18.3906383514404 34.0
22.592611172 20.2865562438965 35.0
10.046873714 10.434193611145 36.0
8.7358599932 6.76989555358887 37.0
-2.5728122599e-11 0.883245170116425 38.0
-2.5720825822e-11 2.39985370635986 39.0
-10.997012032 -16.3521690368652 40.0
-22.984366475 -21.885555267334 41.0
5.5475765567 1.52570652961731 42.0
-31.06137093 -30.8829021453857 43.0
89.757376337 90.1653900146484 44.0
89.75737635 92.5040512084961 45.0
7.3919800441 12.7606048583984 46.0
2.7595251952 3.69848227500916 47.0
135.2341008 166.703994750977 48.0
129.16211851 98.2080535888672 49.0
0 -0.095609724521637 50.0
81.953309097 74.2945404052734 51.0
87.876710733 88.3643493652344 52.0
80.872710015 80.6881484985352 53.0
80.872710015 77.3513870239258 54.0
80.872710015 79.3991622924805 55.0
0 -0.992567121982574 56.0
76.823953241 81.0624237060547 57.0
227.31346396 223.354110717773 58.0
256.3044994 252.875762939453 59.0
22.592611182 20.5984325408936 60.0
0 2.09887385368347 61.0
153.3839496 169.550277709961 0.0
153.3839496 163.974609375 1.0
152.14714636 152.869018554688 2.0
147.35806642 142.920074462891 3.0
0 -0.349619448184967 4.0
0 0.376686573028564 5.0
-69.830181494 -67.1229705810547 6.0
81.84293429 70.8216781616211 7.0
51.609788383 53.0792350769043 8.0
30.233145897 23.5586624145508 9.0
88.949097643 84.9171676635742 10.0
75.989783371 78.7946472167969 11.0
80.549787033 78.1467056274414 12.0
80.549787033 79.5270309448242 13.0
80.549787033 80.7845077514648 14.0
0 -1.31488752365112 15.0
0 0.0854939818382263 16.0
0 1.60049486160278 17.0
57.680770365 56.9308776855469 18.0
-57.680770375 -58.4602355957031 19.0
-57.680770354 -57.7262916564941 20.0
-57.680770366 -59.2174644470215 21.0
79.131522101 79.1325531005859 22.0
150.99711534 143.231994628906 23.0
114.56763406 109.449287414551 24.0
23.495462498 20.9317455291748 25.0
230.12863745 223.45475769043 26.0
230.12863745 223.984512329102 27.0
25.116445817 27.5430030822754 28.0
255.24508327 255.467163085938 29.0
255.24508327 254.276016235352 30.0
46.670804864 46.0799789428711 31.0
9.3094619135 8.88829803466797 32.0
21.450751725 20.2042446136475 33.0
21.450751725 19.3287105560303 34.0
21.450751725 21.501745223999 35.0
9.6527967316 8.18502902984619 36.0
8.3376182378 7.49669218063354 37.0
-1.2511595721e-10 -4.73814058303833 38.0
-1.2485817063e-10 1.04703307151794 39.0
-11.547466424 -17.3145923614502 40.0
-23.971401994 -25.2851810455322 41.0
5.364714783 5.56867933273315 42.0
-31.850556068 -33.4314270019531 43.0
95.939698522 92.4681777954102 44.0
95.939698535 93.9376831054688 45.0
6.8161404872 7.26275014877319 46.0
2.5693554774 2.24136447906494 47.0
153.38394961 164.243041992188 48.0
147.35806643 138.976516723633 49.0
0 0.754528880119324 50.0
81.8429343 75.1002807617188 51.0
88.949097653 83.5122756958008 52.0
80.549787043 79.149040222168 53.0
80.549787043 80.9481964111328 54.0
80.549787043 79.748664855957 55.0
0 0.223032653331757 56.0
79.131522111 79.8990859985352 57.0
230.12863746 221.391220092773 58.0
255.24508328 252.145050048828 59.0
21.450751735 21.2096195220947 60.0
0 2.07180380821228 61.0
145.697862231535 156.984512329102 0.0
145.697862231535 148.199111938477 1.0
144.10927649 153.831024169922 2.0
139.32020675 149.446304321289 3.0
0 -3.38298892974854 4.0
-6.243418611e-09 0.762799501419067 5.0
-66.5229315429982 -73.4125366210938 6.0
80.3029777169938 75.8588790893555 7.0
57.3882814959938 32.2454833984375 8.0
22.9146962209938 9.18483066558838 9.0
78.7261909489938 97.0041122436523 10.0
64.7650111909938 78.0491027832031 11.0
70.4032795089938 83.0165176391602 12.0
70.4032795089938 71.7395401000977 13.0
70.4032795089938 73.2057418823242 14.0
0 3.58913087844849 15.0
0 0.624194502830505 16.0
-4.143583128e-09 6.37584733963013 17.0
65.4479607361784 61.1800270080566 18.0
-65.4479607359907 -57.8827323913574 19.0
-65.4479607149907 -50.2868537902832 20.0
-65.4479607159907 -59.5731811523438 21.0
85.4590124222956 78.4349136352539 22.0
138.887316014178 148.617279052734 23.0
107.741617360833 114.738731384277 24.0
17.1213624897231 20.7439556121826 25.0
224.34632843307 236.537902832031 26.0
224.34632843307 224.347763061523 27.0
28.9637771710698 27.0954780578613 28.0
253.31010560307 255.821624755859 29.0
253.31010560307 250.243286132812 30.0
46.3814199465322 48.1454772949219 31.0
9.5481516920282 12.3198108673096 32.0
20.0110517059947 13.9491481781006 33.0
20.0110517059947 17.147144317627 34.0
20.0110517200323 18.952657699585 35.0
8.56330740328766 6.47381591796875 36.0
7.43978935018766 6.38044309616089 37.0
-8.76161119999999e-09 -1.05077743530273 38.0
-8.7616112e-09 1.08951282501221 39.0
-13.7631795299954 -13.1583013534546 40.0
-25.078191361285 -26.0287914276123 41.0
5.40894745029484 2.79352045059204 42.0
-33.2475495079948 -32.8999671936035 43.0
101.510070149995 97.4493179321289 44.0
101.510070099995 95.9318695068359 45.0
5.759187568 4.34534931182861 46.0
1.3291043141 1.82240223884583 47.0
145.69786224 153.956130981445 48.0
139.32020676 149.490234375 49.0
0 1.87250423431396 50.0
80.302977717 76.8250885009766 51.0
78.726190949 84.0516815185547 52.0
70.403279509 77.0247039794922 53.0
70.403279509 78.2080001831055 54.0
70.403279509 78.0674743652344 55.0
0 6.8220386505127 56.0
85.459012422 80.2134246826172 57.0
224.34632843 216.688812255859 58.0
253.31010561 246.05192565918 59.0
20.01105172 24.6805572509766 60.0
0 1.81334662437439 61.0
193.620195423445 166.634338378906 0.0
193.620195423445 165.151641845703 1.0
192.51512681 163.648574829102 2.0
187.72605706 175.637313842773 3.0
-2.3135613719e-10 1.25907754898071 4.0
-8.647187154e-09 1.55174970626831 5.0
-71.41837288423 -75.5393753051758 6.0
84.4565912553051 71.6154327392578 7.0
54.8692068283051 30.6962127685547 8.0
29.5873844273051 20.8891258239746 9.0
86.4911072633051 85.8573989868164 10.0
73.6765223503051 81.8769683837891 11.0
75.0946964873051 77.241943359375 12.0
75.0946964873051 76.8628387451172 13.0
75.0946964873051 79.9015731811523 14.0
2.6469896651e-09 -4.75218296051025 15.0
0 0.401790797710419 16.0
1.87899875000001e-09 7.24826908111572 17.0
57.2199505702574 54.1835556030273 18.0
-57.2199505459577 -56.4330558776855 19.0
-57.2199505279577 -58.5358848571777 20.0
-57.2199505139577 -58.7397308349609 21.0
77.7338775470825 80.9988098144531 22.0
153.723823269257 144.229385375977 23.0
119.612085404072 111.542259216309 24.0
22.1670579150717 23.6703205108643 25.0
231.457700806891 225.893768310547 26.0
231.457700806891 225.304977416992 27.0
27.6857023538906 20.6725406646729 28.0
259.143403166891 246.940063476562 29.0
259.143403166891 251.32763671875 30.0
46.1791815442054 48.2974662780762 31.0
9.33613497328532 9.06848526000977 32.0
20.5139270336901 25.5485877990723 33.0
20.5139270336901 25.0695781707764 34.0
20.5139270633121 23.332218170166 35.0
9.35858819441393 10.3592844009399 36.0
8.04134756511393 7.78337001800537 37.0
-7.13341441000003e-09 0.731598198413849 38.0
-7.13341440999998e-09 -5.06418037414551 39.0
-11.5848948084789 -13.2671432495117 40.0
-24.4229379770874 -25.1310577392578 41.0
5.62421696768762 6.40897083282471 42.0
-32.6573894150876 -34.5704574584961 43.0
91.6286594930876 91.076416015625 44.0
91.6286593770876 93.1831359863281 45.0
6.5093336585 11.8331708908081 46.0
2.2247797827 3.34036993980408 47.0
193.62019542 174.051284790039 48.0
187.72605707 165.517044067383 49.0
0 -5.14651966094971 50.0
84.456591256 71.5327835083008 51.0
86.491107264 81.328742980957 52.0
75.094696488 79.8986358642578 53.0
75.094696488 79.7839431762695 54.0
75.094696488 80.1530303955078 55.0
1.8789987524e-09 5.04774045944214 56.0
77.733877547 77.833869934082 57.0
231.45770081 221.005035400391 58.0
259.14340317 247.586334228516 59.0
20.513927063 16.7871913909912 60.0
0 0.494023025035858 61.0
147.982033057583 157.456344604492 0.0
147.982033057583 151.962463378906 1.0
146.808413813287 153.339401245117 2.0
142.01934397 150.649459838867 3.0
-1.55484310406e-09 -0.640832006931305 4.0
-1.55484309999997e-09 -2.83684945106506 5.0
-64.9693040339936 -64.3180236816406 6.0
77.6547829144777 74.187141418457 7.0
61.2786926634778 55.4030647277832 8.0
16.3760902514777 7.20984888076782 9.0
97.5261141374777 89.7391510009766 10.0
83.6472874924777 79.6450042724609 11.0
85.2981411144777 83.8642730712891 12.0
85.2981411144777 76.4624557495117 13.0
85.2981411144777 81.5436782836914 14.0
2.3328021736e-08 -2.03496789932251 15.0
2.3328021735e-08 -1.6052393913269 16.0
5.464350236e-08 -2.72059202194214 17.0
56.2518018559467 59.0783615112305 18.0
-56.2518018797166 -57.0891418457031 19.0
-56.2518018707166 -59.2854194641113 20.0
-56.2518017707166 -58.115608215332 21.0
77.1875821073724 73.3798065185547 22.0
153.454771802806 145.008895874023 23.0
115.220596133508 115.750228881836 24.0
21.7005584165075 20.8819332122803 25.0
230.642353915167 228.381210327148 26.0
230.642353915167 229.708221435547 27.0
26.8883420391666 21.3647422790527 28.0
257.530695955167 253.23681640625 29.0
257.530695955167 255.336776733398 30.0
46.7385281475469 45.537841796875 31.0
9.28724497493877 8.4355525970459 32.0
20.9357803369809 22.6376686096191 33.0
20.9357803369809 19.0790691375732 34.0
20.935780467613 20.4264030456543 35.0
8.86054244871795 8.39873600006104 36.0
7.60121916461794 6.85336446762085 37.0
2.62465480000001e-09 1.1959273815155 38.0
2.62465480000001e-09 -1.17553234100342 39.0
-11.6034306368583 -12.5643882751465 40.0
-24.5599220844463 -26.7411251068115 41.0
4.75986482786468 1.92820739746094 42.0
-31.7734218810648 -31.6524333953857 43.0
89.4696777832616 86.4044494628906 44.0
89.4696781210649 91.9600372314453 45.0
7.05809588635643 4.443039894104 46.0
2.46335994485643 2.26466846466064 47.0
147.98203306 156.097106933594 48.0
142.01934397 152.929565429688 49.0
0 -3.75245094299316 50.0
77.654782914 76.0238418579102 51.0
97.526114137 92.0755157470703 52.0
85.298141114 81.5651702880859 53.0
85.298141114 80.2131118774414 54.0
85.298141114 79.9203643798828 55.0
5.4643502359e-08 3.07315587997437 56.0
77.187582108 79.4883346557617 57.0
230.64235391 222.583419799805 58.0
257.53069595 249.491134643555 59.0
20.935780468 19.6116352081299 60.0
2.6246548239e-09 3.64096260070801 61.0
205.575624893706 168.632400512695 0.0
205.575624894812 191.117858886719 1.0
204.247242249085 142.328414916992 2.0
198.772367854326 151.130233764648 3.0
27.1441717093241 0.854316651821136 4.0
27.144171709362 2.4410092830658 5.0
-51.805385096498 -61.219066619873 6.0
60.546244953037 81.0194931030273 7.0
39.534298563742 28.4202938079834 8.0
21.011946389317 17.0011806488037 9.0
71.79145322161 86.0324859619141 10.0
60.882191959015 74.0307388305664 11.0
76.269907110256 76.1554183959961 12.0
76.269907110519 86.6791915893555 13.0
76.269907110315 76.8275299072266 14.0
0.0061500074131 -0.0162836313247681 15.0
0.0061500074139 1.64983510971069 16.0
0.005869413351 4.77681350708008 17.0
48.571390523334 53.7779655456543 18.0
-48.55086590424 -51.4713325500488 19.0
-48.55255688308 -47.2903900146484 20.0
-48.54656708698 -54.5618782043457 21.0
69.513993820783 83.0792007446289 22.0
133.619297272011 143.934173583984 23.0
105.064876930952 100.400032043457 24.0
23.135164168772 20.4033012390137 25.0
203.133291095898 222.641540527344 26.0
203.133291087927 229.656646728516 27.0
22.640948950171 28.1990756988525 28.0
225.774240045199 253.793579101562 29.0
225.774240043429 252.587753295898 30.0
46.082158763251 46.6269187927246 31.0
9.204063416754 8.35661315917969 32.0
20.967426733831 21.4675197601318 33.0
20.967426733366 20.7275466918945 34.0
20.983028241445 22.3625431060791 35.0
8.98076094466 9.37813091278076 36.0
7.831268689316 8.69820117950439 37.0
0.00244885128 -39.1440773010254 38.0
0.00244885128 2.58822178840637 39.0
-13.508783391732 -5.12524938583374 40.0
-26.767726376461 -24.7929458618164 41.0
5.776954132834 7.14891910552979 42.0
-35.173629577678 -35.7819404602051 43.0
74.371568051357 88.8954086303711 44.0
74.2886007206808 69.5551071166992 45.0
6.4605393090326 4.42808485031128 46.0
2.725247265615 1.57077050209045 47.0
205.57562489 173.76416015625 48.0
198.77236785 125.675994873047 49.0
27.144171709 -0.214078009128571 50.0
60.546244953 76.5798263549805 51.0
71.791453222 87.4036865234375 52.0
76.26990711 79.3963470458984 53.0
76.26990711 67.4430389404297 54.0
76.26990711 69.9981460571289 55.0
0.0058694133509 -2.68988084793091 56.0
69.513993821 80.5761337280273 57.0
203.13329109 224.982849121094 58.0
225.77424004 245.844741821289 59.0
20.983028242 22.6592578887939 60.0
0.0024488512809 3.90898275375366 61.0
169.59521971 166.903472900391 0.0
169.59521971 167.952621459961 1.0
168.40750865 163.285568237305 2.0
163.61843893 157.098052978516 3.0
0 3.38961672782898 4.0
-1.0257165379e-11 2.6569652557373 5.0
-70.321732936 -60.4683456420898 6.0
82.089794872 73.3115844726562 7.0
54.330636871 51.3781890869141 8.0
27.759157991 23.9983501434326 9.0
88.78690474 87.6086578369141 10.0
75.402045057 74.8118591308594 11.0
83.928933159 82.4956665039062 12.0
83.928933159 76.9270401000977 13.0
83.928933159 80.3449935913086 14.0
0 -3.67463779449463 15.0
0 -1.4483757019043 16.0
-2.1669919121e-11 0.772549569606781 17.0
58.815983541 59.0329933166504 18.0
-58.815983561 -59.9689140319824 19.0
-58.81598354 -60.5350875854492 20.0
-58.815983552 -60.1198883056641 21.0
80.900083637 80.1489562988281 22.0
143.9215244 143.269439697266 23.0
108.05135683 110.762603759766 24.0
21.35988456 20.9867553710938 25.0
224.82160804 226.399627685547 26.0
224.82160804 223.642379760742 27.0
25.59095257 26.1356468200684 28.0
250.41256062 255.767181396484 29.0
250.41256062 255.07958984375 30.0
46.85679736 46.1918640136719 31.0
9.3675405612 9.0810022354126 32.0
22.084100076 19.4306564331055 33.0
22.084100076 20.284818649292 34.0
22.084100075 21.1912212371826 35.0
9.5808631785 6.29204130172729 36.0
8.323516867 7.22706842422485 37.0
-3.4286220901e-11 -1.32654714584351 38.0
-3.4286220897e-11 -1.3523313999176 39.0
-11.689962069 -14.0800380706787 40.0
-24.663619272 -22.3693408966064 41.0
5.1585708628 7.37772130966187 42.0
-32.249757575 -34.2683563232422 43.0
94.922983432 98.1433868408203 44.0
94.922983445 94.3908157348633 45.0
6.8346963969 5.72695112228394 46.0
2.5160545979 2.96147537231445 47.0
169.59521972 166.615036010742 48.0
163.61843894 156.794052124023 49.0
0 -0.0453022718429565 50.0
82.089794882 77.5471343994141 51.0
88.786904749 84.2161712646484 52.0
83.928933169 80.9061660766602 53.0
83.928933169 79.7558822631836 54.0
83.928933169 80.2895889282227 55.0
0 4.15162324905396 56.0
80.900083647 79.5766830444336 57.0
224.82160805 222.525970458984 58.0
250.41256063 250.849609375 59.0
22.084100085 21.9497013092041 60.0
0 2.85353088378906 61.0
172.02213121 161.764770507812 0.0
172.02213121 164.366455078125 1.0
170.7883793 164.787902832031 2.0
165.99930904 164.08235168457 3.0
0 0.310557305812836 4.0
-9.3459673047e-10 2.69911909103394 5.0
-64.035259673 -60.4401397705078 6.0
75.887396505 76.3105926513672 7.0
49.476505928 50.0294952392578 8.0
26.410890571 26.5510501861572 9.0
94.046571642 85.8565216064453 10.0
81.64523765 74.6103973388672 11.0
83.565014333 80.4717483520508 12.0
83.565014333 80.2432174682617 13.0
83.565014333 79.9203948974609 14.0
0 -0.256983935832977 15.0
0 -0.382390797138214 16.0
0 0.0277670621871948 17.0
57.761914656 59.6808204650879 18.0
-57.76191467 -59.605052947998 19.0
-57.761914648 -61.183219909668 20.0
-57.76191466 -57.1807250976562 21.0
79.724846677 78.7864456176758 22.0
144.71878658 144.886840820312 23.0
110.9180245 111.262992858887 24.0
21.999439314 19.204345703125 25.0
224.44363327 226.009780883789 26.0
224.44363327 227.127990722656 27.0
27.865304028 25.346155166626 28.0
252.3089373 254.804946899414 29.0
252.3089373 253.649536132812 30.0
46.73624401 47.5324783325195 31.0
9.4090615128 8.8231840133667 32.0
21.962932016 19.5459041595459 33.0
21.962932016 24.9945812225342 34.0
21.962932015 20.2880802154541 35.0
9.3739570869 8.48083114624023 36.0
8.0884356671 7.65444946289062 37.0
-3.5219912705e-09 1.37312960624695 38.0
-3.5219885485e-09 -2.80533361434937 39.0
-12.123212156 -13.5793170928955 40.0
-25.486362932 -25.8559169769287 41.0
5.1648775273 4.0712776184082 42.0
-33.065543994 -32.4094886779785 43.0
99.671191233 93.0017318725586 44.0
99.671191243 93.8151168823242 45.0
6.9384615782 6.5172643661499 46.0
2.5751176841 2.2311577796936 47.0
172.02213122 165.347198486328 48.0
165.99930905 160.801681518555 49.0
0 1.15293717384338 50.0
75.887396511 76.8672256469727 51.0
94.046571648 89.5045471191406 52.0
83.565014338 80.6000518798828 53.0
83.565014338 81.1125259399414 54.0
83.565014338 80.4337387084961 55.0
0 1.10032272338867 56.0
79.724846686 79.3628692626953 57.0
224.44363327 220.408233642578 58.0
252.30893731 247.758163452148 59.0
21.962932023 15.9107732772827 60.0
0 1.13124322891235 61.0
151.230920889189 159.380111694336 0.0
151.230920889189 153.250610351562 1.0
150.04163867 148.534057617188 2.0
145.25256894 145.982559204102 3.0
0 2.10366487503052 4.0
-5.7258573836e-09 10.2604665756226 5.0
-74.092081416 -44.5838623046875 6.0
86.4502074282606 78.2317504882812 7.0
55.6571452602605 51.3968353271484 8.0
30.7930621672605 19.0710697174072 9.0
94.6581067912606 87.7251968383789 10.0
82.6091826202606 76.2432098388672 11.0
88.0578862432606 79.2166290283203 12.0
88.0578862432606 81.8155822753906 13.0
88.0578862432606 82.5050354003906 14.0
0 0.724687814712524 15.0
0 0.654193878173828 16.0
-8.873524319e-09 1.42933678627014 17.0
51.7378696516438 57.6933746337891 18.0
-51.7378696798908 -47.1553649902344 19.0
-51.7378696578908 -58.7985877990723 20.0
-51.7378696678908 -59.9187278747559 21.0
73.247757852 80.1811599731445 22.0
145.888571589778 139.590484619141 23.0
113.500663277223 109.238891601562 24.0
24.9956885872231 22.5794506072998 25.0
219.136329438378 221.409622192383 26.0
219.136329438378 228.380966186523 27.0
29.7970302393776 27.4348621368408 28.0
248.933359678378 261.354705810547 29.0
248.933359678378 250.328369140625 30.0
46.170024825 54.0310134887695 31.0
9.3198022861 9.39995861053467 32.0
21.5098881889376 20.9211845397949 33.0
21.5098881889376 22.8536911010742 34.0
21.509888188 25.0573768615723 35.0
9.4290218969 14.1020479202271 36.0
8.0681923533 6.81961441040039 37.0
-9.673544458e-09 -4.42902374267578 38.0
-9.673544458e-09 0.724046766757965 39.0
-12.5007048479454 -15.0703735351562 40.0
-26.111296934 -28.1043434143066 41.0
4.69707399948379 3.61745882034302 42.0
-33.4822639523838 -32.7636222839355 43.0
90.1863208144408 96.8077392578125 44.0
90.1863208463838 95.6390838623047 45.0
6.5594383958 8.99624919891357 46.0
2.3685398472 2.0243604183197 47.0
151.23092088 168.184616088867 48.0
145.25256895 138.336273193359 49.0
0 4.08967733383179 50.0
86.450207428 77.2539596557617 51.0
94.658106792 81.4582366943359 52.0
88.057886243 79.9287872314453 53.0
88.057886243 72.3237457275391 54.0
88.057886243 79.7825164794922 55.0
0 1.97722864151001 56.0
73.247757857 80.1723556518555 57.0
219.13632944 219.170135498047 58.0
248.93335968 250.279113769531 59.0
21.509888192 22.4158344268799 60.0
0 0.0970513224601746 61.0
137.17687742 143.006530761719 0.0
137.17687742 143.030090332031 1.0
135.86355256 143.542068481445 2.0
131.0744823 135.778945922852 3.0
0 -2.12998342514038 4.0
0 0.342118084430695 5.0
-59.42861149 -59.2364273071289 6.0
69.705106679 70.1773300170898 7.0
65.43029471 62.4528312683105 8.0
4.2748119615 8.48783111572266 9.0
78.669161727 80.2065200805664 10.0
65.229460649 71.2585144042969 11.0
77.500548395 80.8047180175781 12.0
77.500548395 77.9000701904297 13.0
77.500548395 76.9198608398438 14.0
0 -0.64198762178421 15.0
0 0.28728449344635 16.0
0 -3.15485000610352 17.0
59.754945382 58.5015258789062 18.0
-59.754945394 -59.4383087158203 19.0
-59.754945372 -60.0649642944336 20.0
-59.754945384 -61.8801422119141 21.0
80.656571382 77.8813858032227 22.0
140.87903669 140.394561767578 23.0
106.41719675 107.357681274414 24.0
22.90053999 20.4083709716797 25.0
221.53560808 223.594223022461 26.0
221.53560808 221.987747192383 27.0
27.282645659 25.3551406860352 28.0
248.81825375 253.574676513672 29.0
248.81825375 252.408157348633 30.0
47.033013387 47.2299919128418 31.0
9.3131031344 9.18030452728271 32.0
20.901625993 20.4819431304932 33.0
20.901625993 21.1129970550537 34.0
20.901625993 20.8195400238037 35.0
9.3083381265 8.28715133666992 36.0
8.044767894 8.38281917572021 37.0
-2.5262790395e-09 -0.804296553134918 38.0
-2.5262785966e-09 0.113371789455414 39.0
-12.387792857 -14.5510091781616 40.0
-23.81592281 -27.5234241485596 41.0
4.9278776077 3.5162615776062 42.0
-31.108913619 -34.7394485473633 43.0
92.052766479 91.3064651489258 44.0
92.05276649 90.2343292236328 45.0
6.7273095346 3.00315618515015 46.0
2.6568330431 1.83220052719116 47.0
137.17687743 136.681350708008 48.0
131.07448231 127.562980651855 49.0
0 3.54100728034973 50.0
69.705106686 70.9861526489258 51.0
78.669161734 84.2529830932617 52.0
77.500548402 81.0479965209961 53.0
77.500548402 80.9212951660156 54.0
77.500548402 79.4541549682617 55.0
0 -0.0822644829750061 56.0
80.656571391 80.3950500488281 57.0
221.53560809 218.26057434082 58.0
248.81825375 246.469757080078 59.0
20.901626002 19.9398994445801 60.0
0 2.89454770088196 61.0
160.40029773 161.28645324707 0.0
160.40029773 164.27180480957 1.0
159.04685921 156.982894897461 2.0
154.25778948 160.458343505859 3.0
0 1.95944714546204 4.0
-1.1106157605e-11 4.07647275924683 5.0
-67.164254488 -69.7971801757812 6.0
77.905037981 76.2602233886719 7.0
56.332002807 47.2559432983398 8.0
21.573035164 25.714204788208 9.0
86.970057114 90.5282669067383 10.0
73.950416555 75.4032516479492 11.0
81.119634047 87.77490234375 12.0
81.119634047 78.8990020751953 13.0
81.119634047 80.1012420654297 14.0
0 -0.256228864192963 15.0
0 -0.19922286272049 16.0
0 0.492870032787323 17.0
59.264416476 57.1221313476562 18.0
-59.264416496 -58.6440391540527 19.0
-59.264416474 -57.773265838623 20.0
-59.264416487 -57.5845527648926 21.0
80.633842594 78.2018508911133 22.0
145.67202515 141.996139526367 23.0
110.34261621 108.156616210938 24.0
22.971694682 19.1125392913818 25.0
226.30586775 222.863891601562 26.0
226.30586775 227.450820922852 27.0
28.061618658 29.2107601165771 28.0
254.36748642 252.690139770508 29.0
254.36748642 253.60498046875 30.0
46.518523875 45.2752456665039 31.0
9.3238344953 8.95865631103516 32.0
21.369426097 17.7103004455566 33.0
21.369426097 17.6770057678223 34.0
21.369426097 20.8870105743408 35.0
9.8534733136 7.96312856674194 36.0
8.5804788689 6.95929193496704 37.0
-3.4865499095e-11 -0.434172928333282 38.0
-3.4865475198e-11 -3.54875135421753 39.0
-12.411186885 -13.9457921981812 40.0
-24.601415118 -26.8714199066162 41.0
5.0371208571 5.8277759552002 42.0
-32.131741558 -34.3593444824219 43.0
94.496748584 90.5503082275391 44.0
94.496748597 94.2157592773438 45.0
6.7865864039 6.48163652420044 46.0
2.4610538327 2.17473244667053 47.0
160.40029774 162.612213134766 48.0
154.25778949 150.858261108398 49.0
0 2.82973146438599 50.0
77.905037991 77.1076126098633 51.0
86.970057124 87.8557891845703 52.0
81.119634057 81.9147491455078 53.0
81.119634057 81.5159759521484 54.0
81.119634057 80.1029281616211 55.0
0 1.16054677963257 56.0
80.633842604 79.8010025024414 57.0
226.30586776 219.901840209961 58.0
254.36748643 250.350433349609 59.0
21.369426107 22.7730197906494 60.0
0 1.7182629108429 61.0
160.93433464 169.785308837891 0.0
160.93433464 161.603240966797 1.0
159.43483656 150.959365844727 2.0
154.64576684 148.385284423828 3.0
0 3.035719871521 4.0
0 3.18494582176208 5.0
-55.874809179 -60.5923805236816 6.0
68.144100481 78.8034744262695 7.0
43.834445368 31.9751930236816 8.0
24.309655103 21.8643894195557 9.0
80.499323107 79.911003112793 10.0
68.183861895 72.6819839477539 11.0
79.34480887 77.1422729492188 12.0
79.34480887 80.1088256835938 13.0
79.34480887 77.8283615112305 14.0
0 1.37341594696045 15.0
0 0.617354154586792 16.0
0 -2.24680805206299 17.0
61.69957869 58.5894927978516 18.0
-61.6995787 -60.0284957885742 19.0
-61.699578679 -58.125675201416 20.0
-61.699578691 -59.0461654663086 21.0
83.530823997 75.5485305786133 22.0
148.48924223 140.187484741211 23.0
116.67694672 103.751068115234 24.0
23.432331078 15.7434549331665 25.0
232.02006624 226.605712890625 26.0
232.02006624 224.345779418945 27.0
24.28243193 26.4546566009521 28.0
256.30249818 256.409210205078 29.0
256.30249818 254.630111694336 30.0
46.154613284 44.947395324707 31.0
9.2851904492 8.57789993286133 32.0
21.831245296 20.7569828033447 33.0
21.831245296 25.9506664276123 34.0
21.831245296 19.5229644775391 35.0
9.8307377028 9.88792705535889 36.0
8.5724299747 8.21868801116943 37.0
-2.1848804195e-10 3.90726327896118 38.0
-2.1848804585e-10 2.46069526672363 39.0
-12.85371231 -15.7637720108032 40.0
-26.598005719 -22.2959251403809 41.0
4.86477693 0.782196819782257 42.0
-34.25238851 -32.5598907470703 43.0
85.581078916 89.5849304199219 44.0
85.581078929 93.930534362793 45.0
6.3182011789 6.83851099014282 46.0
2.6792978575 0.508032441139221 47.0
160.93433465 163.047149658203 48.0
154.64576685 142.139129638672 49.0
0 -3.07301688194275 50.0
68.144100491 76.8133392333984 51.0
80.499323117 80.595703125 52.0
79.344808879 81.1695327758789 53.0
79.344808879 78.4633026123047 54.0
79.344808879 76.6886367797852 55.0
0 -1.76897859573364 56.0
83.530824007 80.8844299316406 57.0
232.02006625 222.967407226562 58.0
256.30249819 251.541778564453 59.0
21.831245306 15.5613679885864 60.0
0 -0.0100010633468628 61.0
170.74103843 177.27766418457 0.0
170.74103843 166.731323242188 1.0
169.44583939 156.563629150391 2.0
164.65676966 185.092620849609 3.0
0 -2.45031881332397 4.0
-1.474190968e-12 6.27417087554932 5.0
-51.87515768 -60.5354385375977 6.0
63.776670305 80.0149078369141 7.0
62.518916525 50.8745727539062 8.0
1.2577537698 26.6626567840576 9.0
77.969462414 82.2695541381836 10.0
65.6606771 80.1906356811523 11.0
79.285801375 85.6974258422852 12.0
79.285801375 78.0426177978516 13.0
79.285801375 79.0845642089844 14.0
0 -1.12983083724976 15.0
0 0.146002352237701 16.0
-2.757033131e-12 0.254054605960846 17.0
59.28862776 57.658561706543 18.0
-59.28862778 -60.4683837890625 19.0
-59.288627758 -57.047191619873 20.0
-59.28862777 -53.8515853881836 21.0
81.653455976 74.5212631225586 22.0
145.5994357 137.908767700195 23.0
112.24206043 105.716743469238 24.0
25.066460109 20.8935413360596 25.0
227.25289169 222.542068481445 26.0
227.25289169 219.388442993164 27.0
24.608081457 28.0927543640137 28.0
251.86097315 254.049331665039 29.0
251.86097315 253.692916870117 30.0
46.139016363 45.7891426086426 31.0
9.2284755494 7.82419013977051 32.0
22.364828196 18.2973766326904 33.0
22.364828196 26.0855922698975 34.0
22.364828196 22.5620956420898 35.0
10.053490182 8.99110221862793 36.0
8.7085869813 7.48154926300049 37.0
-4.4617831932e-12 -2.80508995056152 38.0
-4.4617831928e-12 -1.21680045127869 39.0
-11.771241113 -15.5092573165894 40.0
-24.911710506 -30.12766456604 41.0
4.8968708365 4.62127447128296 42.0
-32.100214882 -29.7808952331543 43.0
102.49732093 80.6413955688477 44.0
102.49732094 115.221733093262 45.0
7.3892086852 8.47246646881104 46.0
2.6837738981 2.34726142883301 47.0
170.74103844 172.790084838867 48.0
164.65676967 152.247711181641 49.0
0 -0.852011859416962 50.0
63.776670315 74.5739288330078 51.0
77.969462424 87.5405502319336 52.0
79.285801385 80.2415466308594 53.0
79.285801385 80.8439788818359 54.0
79.285801385 79.7606582641602 55.0
0 1.41208457946777 56.0
81.653455986 83.185661315918 57.0
227.2528917 218.730697631836 58.0
251.86097316 254.12060546875 59.0
22.364828206 14.8394842147827 60.0
0 1.26519131660461 61.0
161.37032887 161.834625244141 0.0
161.37032887 163.695175170898 1.0
160.10767226 161.17692565918 2.0
155.31860253 163.316299438477 3.0
0 4.75594186782837 4.0
-8.3304147943e-12 0.996659934520721 5.0
-69.510437007 -66.1367645263672 6.0
80.573838403 73.0668869018555 7.0
56.531257042 53.5855255126953 8.0
24.042581351 21.7659435272217 9.0
81.056614139 88.7799835205078 10.0
68.419215597 75.1851654052734 11.0
78.645193067 81.3330230712891 12.0
78.645193067 78.6636581420898 13.0
78.645193067 81.2270660400391 14.0
0 -1.30366945266724 15.0
0 -1.20515084266663 16.0
-1.4736127519e-11 -1.24356150627136 17.0
60.726269621 58.4571418762207 18.0
-60.726269641 -59.8049163818359 19.0
-60.726269619 -58.4447975158691 20.0
-60.726269631 -60.4258346557617 21.0
82.906722552 79.8324737548828 22.0
145.85612845 143.788146972656 23.0
112.44381923 112.683372497559 24.0
22.811642575 21.9399585723877 25.0
228.76285101 226.530731201172 26.0
228.76285101 226.742202758789 27.0
26.589453625 25.111795425415 28.0
255.35230465 252.538070678711 29.0
255.35230465 251.53727722168 30.0
46.922627764 46.4211578369141 31.0
9.361691109 8.84907245635986 32.0
22.180452911 20.2960624694824 33.0
22.180452911 17.7955379486084 34.0
22.180452911 21.2942180633545 35.0
9.6633261439 9.68622970581055 36.0
8.3821479639 7.43668174743652 37.0
-1.9263427575e-11 -3.54132032394409 38.0
-1.9263461944e-11 -3.06932258605957 39.0
-11.94108426 -14.0427684783936 40.0
-25.232480584 -27.7097778320312 41.0
5.3179015532 4.50117015838623 42.0
-32.986870895 -31.5768070220947 43.0
98.668826995 89.9813003540039 44.0
98.668827008 91.3762969970703 45.0
7.055661771 7.67901706695557 46.0
2.5125628377 2.55115938186646 47.0
161.37032888 162.063537597656 48.0
155.31860254 155.76252746582 49.0
0 3.59447240829468 50.0
80.573838413 76.0603332519531 51.0
81.056614149 89.0262451171875 52.0
78.645193077 79.9142532348633 53.0
78.645193077 81.4903411865234 54.0
78.645193077 78.118782043457 55.0
0 4.44547891616821 56.0
82.906722562 79.6491241455078 57.0
228.76285102 221.502365112305 58.0
255.35230466 250.708312988281 59.0
22.180452921 21.4082565307617 60.0
0 1.69054889678955 61.0
194.54488019 159.075942993164 0.0
194.54488019 167.64794921875 1.0
193.07913495 172.868148803711 2.0
188.29006523 184.529998779297 3.0
-3.4156646434e-11 1.07089042663574 4.0
-4.0391911542e-11 4.49304294586182 5.0
-71.5300739 -52.615837097168 6.0
80.461981986 76.9666061401367 7.0
58.448847751 56.5962257385254 8.0
22.013134226 5.20493221282959 9.0
81.628850304 86.8225860595703 10.0
68.37745927 78.4542083740234 11.0
81.66393832 77.2426223754883 12.0
81.66393832 80.2858047485352 13.0
81.66393832 83.1369323730469 14.0
0 -2.635422706604 15.0
0 -1.01033854484558 16.0
-5.9942699058e-11 7.24216365814209 17.0
57.911027946 59.1430130004883 18.0
-57.911027966 -59.2003288269043 19.0
-57.911027944 -56.9994735717773 20.0
-57.911027957 -58.2809448242188 21.0
79.327813461 76.30908203125 22.0
145.64584376 142.290679931641 23.0
111.92990402 112.969482421875 24.0
21.755866618 20.6269683837891 25.0
224.97365723 228.639572143555 26.0
224.97365723 221.946014404297 27.0
24.236855424 24.835916519165 28.0
249.21051267 251.029800415039 29.0
249.21051267 255.203216552734 30.0
46.760932419 43.572639465332 31.0
9.4650710176 8.89524078369141 32.0
21.416785495 15.0644464492798 33.0
21.416785495 4.17405462265015 34.0
21.416785494 20.1440906524658 35.0
10.043361444 4.83081340789795 36.0
8.754619378 6.76093864440918 37.0
0 -7.24033641815186 38.0
0 1.41309213638306 39.0
-11.210280846 -19.4171848297119 40.0
-23.224039804 -19.4423809051514 41.0
4.8599470037 5.6995267868042 42.0
-30.554353162 -35.7832946777344 43.0
89.362904497 86.7814865112305 44.0
89.36290451 95.391716003418 45.0
7.0625740537 5.25438022613525 46.0
2.6841415901 2.62718939781189 47.0
194.5448802 161.648651123047 48.0
188.29006524 176.402755737305 49.0
0 3.36057901382446 50.0
80.461981996 76.0813598632812 51.0
81.628850314 90.2776260375977 52.0
81.66393833 80.2112350463867 53.0
81.66393833 78.5968170166016 54.0
81.66393833 80.4939270019531 55.0
0 1.24942493438721 56.0
79.327813471 81.2722854614258 57.0
224.97365724 220.016830444336 58.0
249.21051268 249.139907836914 59.0
21.416785504 23.1984577178955 60.0
0 3.21474361419678 61.0
178.14527711 167.300155639648 0.0
178.14527711 169.241851806641 1.0
176.97369394 161.30143737793 2.0
172.18462422 165.964553833008 3.0
0 0.920322358608246 4.0
-3.5198190369e-14 -1.79222726821899 5.0
-63.097698814 -68.2848510742188 6.0
73.030436147 76.6071929931641 7.0
51.473624422 47.9730110168457 8.0
21.556811715 21.2689208984375 9.0
89.348846835 85.5426254272461 10.0
76.888181042 75.6859970092773 11.0
81.202571227 85.1503372192383 12.0
81.202571227 79.6795959472656 13.0
81.202571227 77.5959320068359 14.0
0 0.259729385375977 15.0
0 -0.275744020938873 16.0
-7.0422371007e-14 0.234831809997559 17.0
60.918002378 58.669246673584 18.0
-60.918002398 -59.9002380371094 19.0
-60.918002376 -58.0961723327637 20.0
-60.918002388 -60.683464050293 21.0
82.832528382 79.2853698730469 22.0
148.14460492 142.260559082031 23.0
113.57009826 110.086334228516 24.0
22.42650822 22.1837291717529 25.0
230.97713331 222.765075683594 26.0
230.97713331 221.299423217773 27.0
26.719082351 29.0221042633057 28.0
257.69621567 253.000701904297 29.0
257.69621567 253.067016601562 30.0
46.597201673 44.5862617492676 31.0
9.3789993569 8.85275077819824 32.0
21.914525984 21.05393409729 33.0
21.914525984 17.7514572143555 34.0
21.914525984 21.8315010070801 35.0
9.5576217055 9.72793579101562 36.0
8.2462137773 6.65504837036133 37.0
-1.3093567975e-13 -1.60119271278381 38.0
-1.3093567805e-13 0.000318825244903564 39.0
-12.082631661 -16.8722133636475 40.0
-25.174390818 -24.906867980957 41.0
5.0528315667 5.24925470352173 42.0
-32.800761994 -32.7948150634766 43.0
98.782964972 91.6522827148438 44.0
98.782964985 94.904167175293 45.0
7.0657363023 7.28465032577515 46.0
2.4762401604 1.81919288635254 47.0
178.14527712 168.751571655273 48.0
172.18462423 161.647415161133 49.0
0 0.449927687644958 50.0
73.030436157 76.4434509277344 51.0
89.348846845 89.4178314208984 52.0
81.202571237 81.2289657592773 53.0
81.202571237 81.2747955322266 54.0
81.202571237 80.5928573608398 55.0
0 2.08465814590454 56.0
82.832528392 79.1607513427734 57.0
230.97713332 219.988128662109 58.0
257.69621568 251.314666748047 59.0
21.914525994 20.7639713287354 60.0
0 0.69889622926712 61.0
183.69863006 161.255767822266 0.0
183.69863006 160.342071533203 1.0
182.30479632 149.351287841797 2.0
177.5157266 230.24201965332 3.0
0 7.41556692123413 4.0
-3.8275518074e-12 1.54130530357361 5.0
-61.298514416 -44.0417404174805 6.0
71.797770875 77.7220306396484 7.0
62.566034639 54.9362106323242 8.0
9.2317362264 25.0542888641357 9.0
94.74833664 84.5469436645508 10.0
82.634307891 76.3621063232422 11.0
84.119096645 79.5232467651367 12.0
84.119096645 78.9040374755859 13.0
84.119096645 82.4336395263672 14.0
0 2.06801557540894 15.0
0 -0.15214866399765 16.0
-5.7569911734e-12 -0.536546409130096 17.0
60.167941594 58.0651206970215 18.0
-60.167941614 -58.8423461914062 19.0
-60.167941592 -55.9011688232422 20.0
-60.167941604 -59.3753547668457 21.0
82.174665591 84.2162628173828 22.0
137.28687671 145.529373168945 23.0
101.60709881 108.382522583008 24.0
22.482562023 22.4743804931641 25.0
219.46154231 218.487823486328 26.0
219.46154231 221.18815612793 27.0
29.632212128 24.2882652282715 28.0
249.09375445 251.144226074219 29.0
249.09375445 250.702423095703 30.0
46.549794097 44.5808944702148 31.0
9.3308846469 6.75656366348267 32.0
22.006723977 14.8711614608765 33.0
22.006723977 21.0119037628174 34.0
22.006723977 21.789119720459 35.0
9.8594892316 13.2579545974731 36.0
8.5483829446 7.33737850189209 37.0
-1.1565633744e-11 -11.1506271362305 38.0
-1.1565624568e-11 0.223828911781311 39.0
-12.003147099 -11.3338050842285 40.0
-24.582430832 -29.4993286132812 41.0
5.0388204109 2.41506886482239 42.0
-32.333853794 -30.80149269104 43.0
95.413570403 91.383430480957 44.0
95.413570416 95.0140151977539 45.0
6.9879297325 5.29206657409668 46.0
2.858329224 2.33294129371643 47.0
183.69863007 162.648452758789 48.0
177.51572661 162.827285766602 49.0
0 -0.202023804187775 50.0
71.797770885 77.6056747436523 51.0
94.74833665 85.6980133056641 52.0
84.119096655 81.159423828125 53.0
84.119096655 79.3929061889648 54.0
84.119096655 80.9838790893555 55.0
0 3.04920053482056 56.0
82.174665601 78.4147872924805 57.0
219.46154232 222.234832763672 58.0
249.09375446 247.347991943359 59.0
22.006723987 24.3862419128418 60.0
0 4.85806703567505 61.0
176.69396978 172.737380981445 0.0
176.69396978 178.795013427734 1.0
175.77063211 173.082061767578 2.0
170.98156239 159.37873840332 3.0
-6.2986018063e-13 -3.09858417510986 4.0
-6.9819693682e-13 -2.50659489631653 5.0
-79.775720264 -64.054313659668 6.0
92.039848618 70.8834533691406 7.0
70.654809807 61.4159851074219 8.0
21.385038801 29.8398017883301 9.0
79.836644547 95.394660949707 10.0
64.681112026 95.8524703979492 11.0
72.353706199 90.2138671875 12.0
72.353706199 80.4745788574219 13.0
72.353706199 67.8321075439453 14.0
0 2.58032155036926 15.0
0 2.1721887588501 16.0
0 -3.55337715148926 17.0
63.006860474 53.5044059753418 18.0
-63.006860494 -58.1772117614746 19.0
-63.006860472 -47.9112701416016 20.0
-63.006860485 -54.297607421875 21.0
84.312002141 73.8552932739258 22.0
134.46802551 141.880142211914 23.0
98.125424485 106.994438171387 24.0
22.74464851 19.6945495605469 25.0
218.78002766 240.850692749023 26.0
218.78002766 236.375503540039 27.0
26.074593497 26.8674087524414 28.0
244.85462117 256.929351806641 29.0
244.85462117 244.893508911133 30.0
45.829212312 54.4556312561035 31.0
9.2504318289 9.45322513580322 32.0
21.305141647 23.6718559265137 33.0
21.305141647 16.9097061157227 34.0
21.305141647 21.2481956481934 35.0
8.903502035 7.4536919593811 36.0
7.6566485711 5.06501388549805 37.0
0 -4.89432573318481 38.0
0 -3.40336751937866 39.0
-12.35173632 -13.0760440826416 40.0
-24.649322457 -24.523738861084 41.0
5.5304438843 11.0218000411987 42.0
-33.149268543 -35.4238739013672 43.0
86.574395206 92.3511657714844 44.0
86.574395219 91.5722045898438 45.0
6.3240460839 -6.6446704864502 46.0
2.583145984 8.99392032623291 47.0
176.69396979 192.106521606445 48.0
170.9815624 144.042922973633 49.0
0 -8.49411392211914 50.0
92.039848628 79.6958312988281 51.0
79.836644557 62.717357635498 52.0
72.353706209 81.0981979370117 53.0
72.353706209 81.74169921875 54.0
72.353706209 81.0683212280273 55.0
0 0.803893744945526 56.0
84.312002151 78.9203567504883 57.0
218.78002767 230.466674804688 58.0
244.85462118 240.675048828125 59.0
21.305141657 26.7222270965576 60.0
0 3.71923160552979 61.0
147.982033057583 157.456359863281 0.0
147.982033057583 151.964233398438 1.0
146.808413813287 153.353607177734 2.0
142.01934397 150.753616333008 3.0
-1.55484310406e-09 -0.652972280979156 4.0
-1.55484309999997e-09 -2.84123921394348 5.0
-64.9693040339936 -64.3224563598633 6.0
77.6547829144777 74.1872024536133 7.0
61.2786926634778 55.4030647277832 8.0
16.3760902514777 7.20984888076782 9.0
97.5261141374777 89.7391510009766 10.0
83.6472874924777 79.6450042724609 11.0
85.2981411144777 83.8642730712891 12.0
85.2981411144777 76.4624557495117 13.0
85.2981411144777 81.5436782836914 14.0
2.3328021736e-08 -2.03496789932251 15.0
2.3328021735e-08 -1.6052393913269 16.0
5.464350236e-08 -2.72059202194214 17.0
56.2518018559467 59.0783615112305 18.0
-56.2518018797166 -57.0891418457031 19.0
-56.2518018707166 -59.2854194641113 20.0
-56.2518017707166 -58.115608215332 21.0
77.1875821073724 73.3798065185547 22.0
153.454771802806 145.008895874023 23.0
115.220596133508 115.750228881836 24.0
21.7005584165075 20.8819332122803 25.0
230.642353915167 228.381210327148 26.0
230.642353915167 229.708221435547 27.0
26.8883420391666 21.3647422790527 28.0
257.530695955167 253.23681640625 29.0
257.530695955167 255.336776733398 30.0
46.7385281475469 45.537841796875 31.0
9.28724497493877 8.4355525970459 32.0
20.9357803369809 22.6376686096191 33.0
20.9357803369809 19.0790691375732 34.0
20.935780467613 20.4264030456543 35.0
8.86054244871795 8.39873600006104 36.0
7.60121916461794 6.85336446762085 37.0
2.62465480000001e-09 1.1959273815155 38.0
2.62465480000001e-09 -1.17553234100342 39.0
-11.6034306368583 -12.5643882751465 40.0
-24.5599220844463 -26.7411251068115 41.0
4.75986482786468 1.92820739746094 42.0
-31.7734218810648 -31.6524333953857 43.0
89.4696777832616 86.4044494628906 44.0
89.4696781210649 91.9600372314453 45.0
7.05809588635643 4.443039894104 46.0
2.46335994485643 2.26466846466064 47.0
147.98203306 156.096481323242 48.0
142.01934397 153.015121459961 49.0
0 -3.74582099914551 50.0
77.654782914 76.0236129760742 51.0
97.526114137 92.0755157470703 52.0
85.298141114 81.5651702880859 53.0
85.298141114 80.2131118774414 54.0
85.298141114 79.9203643798828 55.0
5.4643502359e-08 3.07315587997437 56.0
77.187582108 79.4883346557617 57.0
230.64235391 222.583419799805 58.0
257.53069595 249.491134643555 59.0
20.935780468 19.6116352081299 60.0
2.6246548239e-09 3.64096260070801 61.0
147.982033057583 157.456436157227 0.0
147.982033057583 151.966171264648 1.0
146.808413813287 153.369369506836 2.0
142.01934397 150.871368408203 3.0
-1.55484310406e-09 -0.666505873203278 4.0
-1.55484309999997e-09 -2.84622478485107 5.0
-64.9693040339936 -64.3275375366211 6.0
77.6547829144777 74.1872482299805 7.0
61.2786926634778 55.4030647277832 8.0
16.3760902514777 7.20984888076782 9.0
97.5261141374777 89.7391510009766 10.0
83.6472874924777 79.6450042724609 11.0
85.2981411144777 83.8642730712891 12.0
85.2981411144777 76.4624557495117 13.0
85.2981411144777 81.5436782836914 14.0
2.3328021736e-08 -2.03496789932251 15.0
2.3328021735e-08 -1.6052393913269 16.0
5.464350236e-08 -2.72059202194214 17.0
56.2518018559467 59.0783615112305 18.0
-56.2518018797166 -57.0891418457031 19.0
-56.2518018707166 -59.2854194641113 20.0
-56.2518017707166 -58.115608215332 21.0
77.1875821073724 73.3798065185547 22.0
153.454771802806 145.008895874023 23.0
115.220596133508 115.750228881836 24.0
21.7005584165075 20.8819332122803 25.0
230.642353915167 228.381210327148 26.0
230.642353915167 229.708221435547 27.0
26.8883420391666 21.3647422790527 28.0
257.530695955167 253.23681640625 29.0
257.530695955167 255.336776733398 30.0
46.7385281475469 45.537841796875 31.0
9.28724497493877 8.4355525970459 32.0
20.9357803369809 22.6376686096191 33.0
20.9357803369809 19.0790691375732 34.0
20.935780467613 20.4264030456543 35.0
8.86054244871795 8.39873600006104 36.0
7.60121916461794 6.85336446762085 37.0
2.62465480000001e-09 1.1959273815155 38.0
2.62465480000001e-09 -1.17553234100342 39.0
-11.6034306368583 -12.5643882751465 40.0
-24.5599220844463 -26.7411251068115 41.0
4.75986482786468 1.92820739746094 42.0
-31.7734218810648 -31.6524333953857 43.0
89.4696777832616 86.4044494628906 44.0
89.4696781210649 91.9600372314453 45.0
7.05809588635643 4.443039894104 46.0
2.46335994485643 2.26466846466064 47.0
147.98203306 156.095626831055 48.0
142.01934397 153.111145019531 49.0
0 -3.73828077316284 50.0
77.654782914 76.0232925415039 51.0
97.526114137 92.0755157470703 52.0
85.298141114 81.5651702880859 53.0
85.298141114 80.2131118774414 54.0
85.298141114 79.9203643798828 55.0
5.4643502359e-08 3.07315587997437 56.0
77.187582108 79.4883346557617 57.0
230.64235391 222.583419799805 58.0
257.53069595 249.491134643555 59.0
20.935780468 19.6116352081299 60.0
2.6246548239e-09 3.64096260070801 61.0
173.66791269033 166.734527587891 0.0
173.66791269033 162.891311645508 1.0
172.26713182 164.300765991211 2.0
167.47806208 153.524749755859 3.0
0 -7.19111633300781 4.0
-7.875698605e-09 1.89501595497131 5.0
-73.1392608874969 -60.1633796691895 6.0
85.6971479372393 73.2032470703125 7.0
51.8825515662393 61.9264221191406 8.0
33.8145963712393 15.4562854766846 9.0
86.0546532632392 90.5621871948242 10.0
73.1403174012392 76.6735916137695 11.0
83.7981330662392 80.1997985839844 12.0
83.7981330662392 81.1996307373047 13.0
83.7981330662392 74.253532409668 14.0
1.0183230703e-09 -4.51358270645142 15.0
0 -1.80627846717834 16.0
1.9586105000002e-10 -0.531194269657135 17.0
61.5117567031305 57.2512283325195 18.0
-61.5117566918589 -53.8158302307129 19.0
-61.5117566718589 -59.2176094055176 20.0
-61.5117566668589 -58.2656135559082 21.0
83.902509463362 77.4449996948242 22.0
149.241310944131 141.347564697266 23.0
118.072750147663 107.887817382812 24.0
22.8035399266631 18.132396697998 25.0
233.143820400661 220.646759033203 26.0
233.143820400661 224.705596923828 27.0
25.1609618036607 27.3962287902832 28.0
258.304782210661 254.96305847168 29.0
258.304782210661 254.36262512207 30.0
46.173295023827 48.6827278137207 31.0
9.35744571762142 9.44705200195312 32.0
22.3907527964908 17.5944347381592 33.0
22.3907527964908 14.3833513259888 34.0
22.3907528163584 21.2651615142822 35.0
9.56967750246318 9.88761806488037 36.0
8.38147665886318 7.68716716766357 37.0
-8.15994281999999e-09 -5.07474422454834 38.0
-8.15994280999999e-09 -6.04146766662598 39.0
-11.8955725524294 -20.5972270965576 40.0
-24.9642038225962 -24.7266674041748 41.0
5.30673767266604 1.23535752296448 42.0
-32.580962627366 -30.3393878936768 43.0
91.598955117366 96.9008407592773 44.0
91.598955046366 96.2051467895508 45.0
6.9771983202 6.89241933822632 46.0
2.3118499226 2.29085612297058 47.0
173.66791269 166.923873901367 48.0
167.47806209 159.29655456543 49.0
0 -1.55586719512939 50.0
85.697147937 73.3957366943359 51.0
86.054653263 79.8920211791992 52.0
83.798133066 80.6196212768555 53.0
83.798133066 78.0019226074219 54.0
83.798133066 78.6284484863281 55.0
1.9586104823e-10 -5.52205181121826 56.0
83.902509463 78.5379028320312 57.0
233.14382041 216.375411987305 58.0
258.30478221 247.157791137695 59.0
22.390752816 23.8628978729248 60.0
0 4.11010074615479 61.0
179.103283948298 168.961120605469 0.0
179.103283948298 171.110000610352 1.0
177.79783692 169.107330322266 2.0
173.00876717 176.676986694336 3.0
0 2.60780620574951 4.0
-7.61149997e-09 -0.0143744349479675 5.0
-60.4952508334905 -59.4933166503906 6.0
71.6128596977166 74.3285522460938 7.0
49.9466633877166 53.6119499206543 8.0
21.6661963097166 17.5298595428467 9.0
84.1748046647166 90.4200973510742 10.0
71.3253228367166 78.2450180053711 11.0
79.9175488417166 85.3231811523438 12.0
79.9175488417166 80.6547012329102 13.0
79.9175488417166 81.2348861694336 14.0
1.1560271588e-09 0.705878436565399 15.0
0 0.206540107727051 16.0
4.97683930000007e-10 0.770088374614716 17.0
60.851817171209 59.6606292724609 18.0
-60.8518171595749 -59.5813522338867 19.0
-60.8518171395749 -60.354621887207 20.0
-60.8518171345749 -61.3985137939453 21.0
82.0320967984099 79.792839050293 22.0
143.289499322209 143.987319946289 23.0
108.999254885261 110.579765319824 24.0
22.0608285202613 18.849422454834 25.0
225.321596116596 226.129425048828 26.0
225.321596116596 222.805358886719 27.0
26.307492927596 27.3937110900879 28.0
251.629089046596 254.824096679688 29.0
251.629089046596 250.675506591797 30.0
46.5630254288009 45.3393211364746 31.0
9.3615102388307 8.13702201843262 32.0
21.1802796644713 20.1425113677979 33.0
21.1802796644713 16.4871234893799 34.0
21.1802796851436 20.2971172332764 35.0
9.48754273963785 7.50277614593506 36.0
8.21612593233785 6.97669744491577 37.0
-8.10757414000001e-09 -0.255769431591034 38.0
-8.10757413999999e-09 -0.629701197147369 39.0
-12.3614839637874 -12.5336265563965 40.0
-25.079665917805 -24.6759757995605 41.0
5.05033433133047 6.12962055206299 42.0
-32.8276310244305 -32.752986907959 43.0
94.8428525354305 94.0620269775391 44.0
94.8428524614305 92.8666610717773 45.0
7.1737057602 6.34871864318848 46.0
2.6697034577 1.05629897117615 47.0
179.10328395 168.883071899414 48.0
173.00876718 170.308166503906 49.0
0 -0.683293282985687 50.0
71.612859697 76.5010223388672 51.0
84.174804665 90.7347869873047 52.0
79.917548842 81.3800048828125 53.0
79.917548842 80.7949447631836 54.0
79.917548842 79.7079086303711 55.0
4.9768392924e-10 2.51922488212585 56.0
82.032096799 80.1395950317383 57.0
225.32159612 221.537628173828 58.0
251.62908905 252.365737915039 59.0
21.180279685 20.7248725891113 60.0
0 1.71327292919159 61.0
167.314577190639 171.654998779297 0.0
167.314577190639 162.438995361328 1.0
166.09927871 164.961380004883 2.0
160.63464351 114.219902038574 3.0
7.1337080016 23.098352432251 4.0
7.13370801125907 15.9946279525757 5.0
-74.0894011344591 -59.2308349609375 6.0
88.4931098961067 88.9129104614258 7.0
67.6776169881067 22.2629280090332 8.0
20.8154929081067 10.5851640701294 9.0
93.8634301471067 55.8013610839844 10.0
80.1191217241067 86.9708709716797 11.0
79.1228040811068 83.7723846435547 12.0
79.1228040811068 72.173942565918 13.0
79.1228040811068 77.2857131958008 14.0
0 4.94209623336792 15.0
0 -1.52295327186584 16.0
-5.298430263e-09 2.91290473937988 17.0
49.195458628733 41.9270095825195 18.0
-49.1954586616601 -53.9749526977539 19.0
-49.1954586406601 -46.4687805175781 20.0
-49.1954586456601 -52.6158065795898 21.0
67.9021219612451 79.1926956176758 22.0
114.63025489224 132.946319580078 23.0
86.6775151856767 104.554618835449 24.0
5.12492498270005 25.9933643341064 25.0
182.532376851277 202.824584960938 26.0
182.532376851277 214.565734863281 27.0
31.439170213277 29.1898593902588 28.0
213.971547071277 249.992752075195 29.0
213.971547071277 258.158905029297 30.0
40.1227290627147 45.2578659057617 31.0
1.7854235123 6.19486951828003 32.0
18.7066633163772 18.5409202575684 33.0
18.7066633163772 17.9886798858643 34.0
18.7066633245151 16.1070785522461 35.0
9.4591016811 7.5654411315918 36.0
8.1059226357 4.67108535766602 37.0
-9.191432868e-09 -2.16798138618469 38.0
-9.191432868e-09 -5.25315713882446 39.0
-10.1969913878301 -0.335666477680206 40.0
-22.7426672531221 -23.9016456604004 41.0
4.79959465878895 -0.28809529542923 42.0
-29.942174925589 -31.5504093170166 43.0
85.0482678706595 92.7224273681641 44.0
85.0482679245889 96.4161071777344 45.0
6.5239043858 2.14909696578979 46.0
2.8317449572 1.45713233947754 47.0
167.31457719 165.640350341797 48.0
160.63464352 139.295440673828 49.0
7.1337080113 -13.8281841278076 50.0
88.493109896 53.8064155578613 51.0
93.863430147 81.2395477294922 52.0
79.122804081 80.8477325439453 53.0
79.122804081 80.597282409668 54.0
79.122804081 62.7402572631836 55.0
0 17.6642742156982 56.0
67.902121961 81.650260925293 57.0
182.53237685 207.083755493164 58.0
213.97154707 254.508926391602 59.0
18.706663324 33.6629524230957 60.0
0 0.690270960330963 61.0
180.56120698 158.672058105469 0.0
180.56120698 160.892959594727 1.0
179.42366895 159.682327270508 2.0
174.63459922 174.058059692383 3.0
0 -1.33216953277588 4.0
-3.8299588184e-09 -0.0357697606086731 5.0
-62.788001716 -64.6817016601562 6.0
73.3667477875118 75.2163391113281 7.0
46.3407543985118 51.560604095459 8.0
27.0259933885118 23.9737758636475 9.0
87.3532325335118 85.1591186523438 10.0
73.8305626775118 72.7441024780273 11.0
79.2555855715118 81.4703140258789 12.0
79.2555855715118 80.4981307983398 13.0
79.2555855715118 80.4476852416992 14.0
0 -0.891246378421783 15.0
0 0.188278496265411 16.0
-8.7568110599e-09 3.01566290855408 17.0
58.783830954 59.0972938537598 18.0
-58.7838309732678 -59.547664642334 19.0
-58.7838309512678 -60.2056007385254 20.0
-58.7838309612678 -61.1560363769531 21.0
80.477742487 78.951545715332 22.0
144.23420647 140.728912353516 23.0
110.79537471 108.575744628906 24.0
23.122419063 21.1175670623779 25.0
224.711948963283 228.765182495117 26.0
224.711948963283 223.179641723633 27.0
25.2786230382829 26.3548107147217 28.0
249.990572003283 253.711517333984 29.0
249.990572003283 251.888534545898 30.0
46.488833305 45.8179512023926 31.0
9.2883098987 8.48523330688477 32.0
21.6939115325816 22.5071830749512 33.0
21.6939115325816 20.946569442749 34.0
21.693911529 19.082052230835 35.0
9.6231391732 8.97734260559082 36.0
8.3375640296 7.29577970504761 37.0
-9.7500530017e-09 0.439431488513947 38.0
-9.7500530026e-09 -5.56924533843994 39.0
-11.3969348871339 -13.6627254486084 40.0
-23.916263426 -26.9965324401855 41.0
5.04623185022654 6.0635814666748 42.0
-31.5495625149265 -32.5325088500977 43.0
93.3324665699265 86.1542739868164 44.0
93.3324665607527 90.8757476806641 45.0
6.9116512699 6.27865314483643 46.0
2.5160461172 1.16380739212036 47.0
180.56120698 162.805374145508 48.0
174.63459923 165.648574829102 49.0
0 0.704916059970856 50.0
73.366747787 75.3045959472656 51.0
87.353232533 87.6276550292969 52.0
79.255585571 80.3227615356445 53.0
79.255585571 81.1637725830078 54.0
79.255585571 80.3124618530273 55.0
0 1.27056169509888 56.0
80.477742494 79.8733901977539 57.0
224.71194896 222.875183105469 58.0
249.990572 252.965530395508 59.0
21.693911534 19.9257888793945 60.0
0 1.4767792224884 61.0
175.603615999326 163.70051574707 0.0
175.603615999326 163.226989746094 1.0
174.361026076965 167.951293945312 2.0
169.57195626 168.679351806641 3.0
0 2.21631646156311 4.0
-3.11035666999999e-09 0.0948862433433533 5.0
-68.4657742233803 -68.1041564941406 6.0
78.4603731578309 74.9543914794922 7.0
55.8593717988309 50.4680938720703 8.0
22.6010013588309 23.725658416748 9.0
88.6381626078309 89.4837493896484 10.0
76.1095343658309 74.349723815918 11.0
84.2457675568309 80.786865234375 12.0
84.2457675568309 78.8211517333984 13.0
84.2457675568309 80.2257080078125 14.0
1.5967694022e-08 -3.03719592094421 15.0
1.5967694022e-08 -0.6790731549263 16.0
4.10408307600001e-08 4.02962303161621 17.0
59.3427396009986 58.4497680664062 18.0
-59.3427394612463 -59.6069488525391 19.0
-59.3427394492463 -57.9750862121582 20.0
-59.3427393742463 -62.565013885498 21.0
81.4162820210354 79.7224426269531 22.0
147.563924408999 141.379867553711 23.0
111.203413653748 108.555786132812 24.0
22.6462377107483 20.5621700286865 25.0
228.980206428651 225.713012695312 26.0
228.980206428651 225.826522827148 27.0
25.000763623651 26.6047496795654 28.0
253.980970048651 253.595413208008 29.0
253.980970048651 257.869659423828 30.0
46.8273874749119 45.596263885498 31.0
9.40145704531782 9.3886547088623 32.0
22.0735426461408 19.2898845672607 33.0
22.0735426461408 19.9772872924805 34.0
22.0735427484244 21.1353893280029 35.0
9.66482077112323 9.33578968048096 36.0
8.39634724462323 8.03990364074707 37.0
-5.18780499999713e-11 -1.14138412475586 38.0
-5.18780499999713e-11 -1.43837594985962 39.0
-12.1448151781232 -14.1659688949585 40.0
-25.2639431901174 -24.3290786743164 41.0
4.94296027062577 4.6250057220459 42.0
-32.8332034700257 -33.7567710876465 43.0
92.1813686560257 92.5753555297852 44.0
92.181367865322 93.6878433227539 45.0
6.83699246819685 6.20694732666016 46.0
2.67937249829685 2.12435221672058 47.0
175.603616 161.165100097656 48.0
169.57195627 162.540100097656 49.0
0 -2.26017141342163 50.0
78.460373158 76.3995590209961 51.0
88.638162608 89.4166870117188 52.0
84.245767557 81.3833618164062 53.0
84.245767557 80.8873825073242 54.0
84.245767557 80.3582992553711 55.0
4.1040830757e-08 5.18019723892212 56.0
81.416282021 79.6352844238281 57.0
228.98020643 223.247207641602 58.0
253.98097005 251.416793823242 59.0
22.073542748 20.8257064819336 60.0
0 1.12887668609619 61.0
187.25174543 167.712448120117 0.0
187.25174543 167.243118286133 1.0
185.76652126 162.38427734375 2.0
180.97745154 172.061538696289 3.0
0 0.537262141704559 4.0
0 0.12977010011673 5.0
-75.834988292 -70.2542343139648 6.0
86.243373572 72.3633117675781 7.0
54.738453324 47.7086601257324 8.0
31.504920238 23.9303398132324 9.0
84.523370062 82.7559356689453 10.0
70.696428485 77.9505233764648 11.0
79.540379777 88.9773788452148 12.0
79.540379777 82.764518737793 13.0
79.540379777 79.8790893554688 14.0
0 -1.16056251525879 15.0
0 -0.640219509601593 16.0
0 -1.35294342041016 17.0
59.023681675 59.1393890380859 18.0
-59.023681695 -60.1408157348633 19.0
-59.023681673 -57.2784767150879 20.0
-59.023681685 -58.0735282897949 21.0
80.651479128 77.8520050048828 22.0
148.34500437 142.460784912109 23.0
110.71020082 107.796539306641 24.0
22.801924619 21.5507507324219 25.0
228.99648351 233.00553894043 26.0
228.99648351 223.427124023438 27.0
25.813950021 24.1170043945312 28.0
254.81043354 255.948516845703 29.0
254.81043354 251.362045288086 30.0
46.692061519 46.1001853942871 31.0
9.4146724968 9.8388147354126 32.0
21.627797433 14.1586351394653 33.0
21.627797433 19.4121780395508 34.0
21.627797433 20.3372936248779 35.0
9.7012600016 8.90798950195312 36.0
8.3765762092 7.49448728561401 37.0
0 -1.0698516368866 38.0
0 1.40821516513824 39.0
-12.389737588 -14.8989124298096 40.0
-26.288917956 -22.4758625030518 41.0
4.9750077173 4.85038757324219 42.0
-33.781630988 -34.1070709228516 43.0
80.472763168 82.7687072753906 44.0
80.472763181 77.7280502319336 45.0
6.7665957462 9.79705238342285 46.0
2.8617454844 1.27422618865967 47.0
187.25174544 162.551345825195 48.0
180.97745155 176.300170898438 49.0
0 -3.21843862533569 50.0
86.243373582 77.5929794311523 51.0
84.523370072 89.6717147827148 52.0
79.540379787 79.4229736328125 53.0
79.540379787 81.5679931640625 54.0
79.540379787 80.7562866210938 55.0
0 3.64743280410767 56.0
80.651479138 80.0808944702148 57.0
228.99648352 221.467956542969 58.0
254.81043355 250.751495361328 59.0
21.627797443 17.4413070678711 60.0
0 1.16197752952576 61.0
147.982033057583 157.45637512207 0.0
147.982033057583 151.964126586914 1.0
146.808413813287 153.352630615234 2.0
142.01934397 150.746673583984 3.0
-1.55484310406e-09 -0.652255833148956 4.0
-1.55484309999997e-09 -2.84094476699829 5.0
-64.9693040339936 -64.3221588134766 6.0
77.6547829144777 74.1871871948242 7.0
61.2786926634778 55.4030647277832 8.0
16.3760902514777 7.20984888076782 9.0
97.5261141374777 89.7391510009766 10.0
83.6472874924777 79.6450042724609 11.0
85.2981411144777 83.8642730712891 12.0
85.2981411144777 76.4624557495117 13.0
85.2981411144777 81.5436782836914 14.0
2.3328021736e-08 -2.03496789932251 15.0
2.3328021735e-08 -1.6052393913269 16.0
5.464350236e-08 -2.72059202194214 17.0
56.2518018559467 59.0783615112305 18.0
-56.2518018797166 -57.0891418457031 19.0
-56.2518018707166 -59.2854194641113 20.0
-56.2518017707166 -58.115608215332 21.0
77.1875821073724 73.3798065185547 22.0
153.454771802806 145.008895874023 23.0
115.220596133508 115.750228881836 24.0
21.7005584165075 20.8819332122803 25.0
230.642353915167 228.381210327148 26.0
230.642353915167 229.708221435547 27.0
26.8883420391666 21.3647422790527 28.0
257.530695955167 253.23681640625 29.0
257.530695955167 255.336776733398 30.0
46.7385281475469 45.537841796875 31.0
9.28724497493877 8.4355525970459 32.0
20.9357803369809 22.6376686096191 33.0
20.9357803369809 19.0790691375732 34.0
20.935780467613 20.4264030456543 35.0
8.86054244871795 8.39873600006104 36.0
7.60121916461794 6.85336446762085 37.0
2.62465480000001e-09 1.1959273815155 38.0
2.62465480000001e-09 -1.17553234100342 39.0
-11.6034306368583 -12.5643882751465 40.0
-24.5599220844463 -26.7411251068115 41.0
4.75986482786468 1.92820739746094 42.0
-31.7734218810648 -31.6524333953857 43.0
89.4696777832616 86.4044494628906 44.0
89.4696781210649 91.9600372314453 45.0
7.05809588635643 4.443039894104 46.0
2.46335994485643 2.26466846466064 47.0
147.98203306 156.096435546875 48.0
142.01934397 153.009490966797 49.0
0 -3.74624156951904 50.0
77.654782914 76.0236282348633 51.0
97.526114137 92.0755157470703 52.0
85.298141114 81.5651702880859 53.0
85.298141114 80.2131118774414 54.0
85.298141114 79.9203643798828 55.0
5.4643502359e-08 3.07315587997437 56.0
77.187582108 79.4883346557617 57.0
230.64235391 222.583419799805 58.0
257.53069595 249.491134643555 59.0
20.935780468 19.6116352081299 60.0
2.6246548239e-09 3.64096260070801 61.0
134.24481883 157.957214355469 0.0
134.24481883 163.595458984375 1.0
132.82307575 158.586990356445 2.0
128.03400603 77.6806793212891 3.0
0 49.2262802124023 4.0
-5.2541713817e-10 9.60797786712646 5.0
-62.648268578 -65.2880477905273 6.0
75.698873081 82.0057907104492 7.0
38.252844302 59.4929809570312 8.0
37.446028777 23.5055904388428 9.0
92.158865121 90.9868087768555 10.0
78.821465908 76.4257736206055 11.0
77.288218161 82.996208190918 12.0
77.288218161 80.1845703125 13.0
77.288218161 80.4059219360352 14.0
0 -1.5278742313385 15.0
0 -0.272740662097931 16.0
-3.8580734939e-09 3.19112825393677 17.0
61.396789336 58.2097244262695 18.0
-61.3967893605794 -59.7254524230957 19.0
-61.3967893385794 -57.906810760498 20.0
-61.3967893505794 -56.6375045776367 21.0
82.743374176 77.9580307006836 22.0
142.54028823 146.576904296875 23.0
108.73346637 107.098533630371 24.0
24.369949699 20.8150310516357 25.0
225.28366241 223.415069580078 26.0
225.28366241 225.543701171875 27.0
25.302348782 26.6648349761963 28.0
250.5860112 252.414978027344 29.0
250.5860112 253.120086669922 30.0
47.407915959 45.7993392944336 31.0
9.4760424616 8.62786388397217 32.0
21.34658483 20.2449626922607 33.0
21.34658483 10.584192276001 34.0
21.346584827 18.3100490570068 35.0
9.5680411703 6.18280506134033 36.0
8.2857930979 7.21461915969849 37.0
-6.977820572e-09 3.21312284469604 38.0
-6.9778205758e-09 6.17932653427124 39.0
-12.158339285 -21.8348827362061 40.0
-25.692887733 -24.5948123931885 41.0
5.1109415633 7.47711849212646 42.0
-33.368688116 -34.4948768615723 43.0
89.536940237 90.2981185913086 44.0
89.536940238 89.7506866455078 45.0
7.2303487883 6.78892421722412 46.0
2.488949114 1.2705819606781 47.0
134.24481884 161.660125732422 48.0
128.03400604 112.872871398926 49.0
0 -2.49492287635803 50.0
75.698873084 75.5928573608398 51.0
92.158865124 90.2025680541992 52.0
77.288218164 81.1239776611328 53.0
77.288218164 81.7022018432617 54.0
77.288218164 79.715690612793 55.0
0 6.4595193862915 56.0
82.743374183 78.4909439086914 57.0
225.28366242 217.227340698242 58.0
250.5860112 245.601760864258 59.0
21.346584834 20.4458293914795 60.0
0 4.00279664993286 61.0
156.444909094495 161.839569091797 0.0
156.444909094495 161.653442382812 1.0
155.23754101 165.262878417969 2.0
150.44847128 157.919326782227 3.0
0 2.30976414680481 4.0
-8.9481355392e-09 6.67272281646729 5.0
-57.77577294 -36.5774917602539 6.0
66.7565049451246 91.5021209716797 7.0
49.5646924781246 58.0445556640625 8.0
17.1918124661246 4.82854175567627 9.0
94.0203926851246 89.9691772460938 10.0
81.0576863641246 74.9639434814453 11.0
84.0425682181246 86.9017333984375 12.0
84.0425682181246 81.5962448120117 13.0
84.0425682181246 76.1716079711914 14.0
0 -3.2539496421814 15.0
0 -0.101828813552856 16.0
-7.133330226e-09 -4.37102746963501 17.0
53.3249561025444 53.1096801757812 18.0
-53.324956129687 -58.2780647277832 19.0
-53.324956108687 -50.8171577453613 20.0
-53.324956116687 -59.3309593200684 21.0
74.155556989 78.5148391723633 22.0
152.141419075341 141.177886962891 23.0
116.385006401676 111.784851074219 24.0
24.6283618916759 21.1199188232422 25.0
226.29697606899 215.132354736328 26.0
226.29697606899 222.101516723633 27.0
26.64660490099 21.9099025726318 28.0
252.94358096899 248.841461181641 29.0
252.94358096899 249.401565551758 30.0
47.237095487 48.7177886962891 31.0
9.3625456757 8.60711193084717 32.0
20.830600874964 20.2689380645752 33.0
20.830600874964 19.9243202209473 34.0
20.830600878 20.5518074035645 35.0
9.0430553836 9.92651653289795 36.0
7.8596073311 7.71602582931519 37.0
-9.577887965e-09 -0.329658925533295 38.0
-9.577887965e-09 1.14548230171204 39.0
-12.3225098738435 -12.1976957321167 40.0
-24.9667912583346 -25.8983516693115 41.0
5.35537586260387 7.56761360168457 42.0
-33.0497348136039 -34.5957450866699 43.0
98.3015891546039 90.6993255615234 44.0
98.3015891268181 90.4664077758789 45.0
6.7129042359 5.50776863098145 46.0
2.6635332936 0.911825060844421 47.0
156.4449091 163.393905639648 48.0
150.44847129 145.188827514648 49.0
0 1.04045760631561 50.0
66.756504945 76.4115295410156 51.0
94.020392685 85.0493698120117 52.0
84.042568218 82.0698776245117 53.0
84.042568218 79.9915924072266 54.0
84.042568218 79.5889739990234 55.0
0 2.31768703460693 56.0
74.155556992 80.8663177490234 57.0
226.29697607 215.949844360352 58.0
252.94358097 243.482604980469 59.0
20.830600879 21.7150840759277 60.0
0 1.37411832809448 61.0
181.27880342 163.901519775391 0.0
181.27880342 158.076095581055 1.0
180.07179026 171.61164855957 2.0
175.28272054 165.716613769531 3.0
0 9.02878665924072 4.0
-4.0961819131e-14 11.0166940689087 5.0
-73.893847844 -40.9224281311035 6.0
85.1207335 79.3961486816406 7.0
60.884535001 51.1230316162109 8.0
24.236198488 23.6634693145752 9.0
91.798538036 85.4269561767578 10.0
78.291608073 74.0266647338867 11.0
82.359161058 80.6502990722656 12.0
82.359161058 83.6718139648438 13.0
82.359161058 79.0970153808594 14.0
0 -0.806605160236359 15.0
0 -0.522596180438995 16.0
-7.6330452278e-14 -2.25158309936523 17.0
58.220490291 56.7839851379395 18.0
-58.220490311 -52.3093147277832 19.0
-58.220490289 -60.0571441650391 20.0
-58.220490301 -55.8244514465332 21.0
78.957086036 77.5180969238281 22.0
131.06054296 140.18180847168 23.0
99.726129996 113.035057067871 24.0
20.107377957 22.6023998260498 25.0
210.01762901 227.869766235352 26.0
210.01762901 225.881713867188 27.0
27.831755148 28.5244598388672 28.0
237.84938416 251.537567138672 29.0
237.84938416 247.640365600586 30.0
46.983341668 47.7649230957031 31.0
9.4516880729 9.29814434051514 32.0
20.736595725 18.758358001709 33.0
20.736595725 13.1751203536987 34.0
20.736595725 23.3286209106445 35.0
8.6701210208 14.7594375610352 36.0
7.4039490815 7.23575305938721 37.0
-1.2231621506e-13 -6.47165155410767 38.0
-1.2231621506e-13 1.58286809921265 39.0
-11.157888397 -19.9507312774658 40.0
-24.742701918 -23.5566711425781 41.0
5.6258182819 9.20473766326904 42.0
-33.116505879 -37.2065696716309 43.0
99.824925014 94.5742950439453 44.0
99.824925027 92.1477661132812 45.0
6.3383794196 5.71562099456787 46.0
2.5755228733 2.79513144493103 47.0
181.27880343 161.587768554688 48.0
175.28272055 159.515991210938 49.0
0 -4.52954339981079 50.0
85.12073351 75.1525650024414 51.0
91.798538046 87.8462448120117 52.0
82.359161068 80.0779800415039 53.0
82.359161068 75.6326446533203 54.0
82.359161068 76.8956298828125 55.0
0 -0.702819049358368 56.0
78.957086046 79.7414398193359 57.0
210.01762902 220.559600830078 58.0
237.84938417 246.47233581543 59.0
20.736595735 25.3904685974121 60.0
0 1.22404217720032 61.0
157.961808219909 160.42024230957 0.0
157.961808219909 163.259094238281 1.0
156.647955 160.145233154297 2.0
151.85888521 150.161758422852 3.0
0 -2.67139410972595 4.0
-4.77400117000001e-09 0.848523676395416 5.0
-64.8398559171454 -60.9600257873535 6.0
76.3705552425091 79.3763122558594 7.0
47.0920143985091 56.8492813110352 8.0
29.2785408435091 25.2860927581787 9.0
88.6745185525091 89.7431411743164 10.0
75.8372005315091 77.8119583129883 11.0
83.8087447565091 81.4020233154297 12.0
83.8087447565091 79.2996139526367 13.0
83.8087447565091 80.1181564331055 14.0
7.78111765000001e-10 -1.79699897766113 15.0
7.78111764999999e-10 -0.814296662807465 16.0
2.372650157e-08 0.206989169120789 17.0
60.6078662995504 58.3393592834473 18.0
-60.6078662107636 -58.8892860412598 19.0
-60.6078661957636 -58.501392364502 20.0
-60.6078661477636 -59.178295135498 21.0
82.0925553208403 78.9327545166016 22.0
144.93187403355 141.506317138672 23.0
111.186674674438 108.447463989258 24.0
23.840689336438 19.9385356903076 25.0
227.024429349818 226.233673095703 26.0
227.024429349818 223.504547119141 27.0
26.5455084538181 25.3960151672363 28.0
253.569937809818 257.518341064453 29.0
253.569937809818 258.653778076172 30.0
46.7881837187405 46.8427581787109 31.0
9.36499621630177 8.94511413574219 32.0
21.4846891734364 19.4050960540771 33.0
21.4846891734364 18.5549926757812 34.0
21.4846892445203 21.1903457641602 35.0
9.31401654804894 8.91484355926514 36.0
8.08399132364894 7.79282474517822 37.0
-3.33720051999997e-09 -5.73064422607422 38.0
-3.33720053000004e-09 -2.24365162849426 39.0
-12.1302714518818 -14.9110012054443 40.0
-24.7042221855809 -26.0708904266357 41.0
5.02433739725762 5.21726369857788 42.0
-32.3929829867576 -34.0537948608398 43.0
98.5130089577575 92.1992416381836 44.0
98.5130083228782 95.2472839355469 45.0
6.84103154913597 6.83663749694824 46.0
2.47691726153597 2.32861948013306 47.0
157.96180822 161.301879882812 48.0
151.85888522 150.782928466797 49.0
0 -0.0950888991355896 50.0
76.370555242 75.9614410400391 51.0
88.674518552 91.2663116455078 52.0
83.808744757 81.2053756713867 53.0
83.808744757 79.3992538452148 54.0
83.808744757 79.6827392578125 55.0
2.3726501569e-08 1.97269725799561 56.0
82.092555321 79.6980133056641 57.0
227.02442935 219.908187866211 58.0
253.56993781 251.606750488281 59.0
21.484689244 24.1281318664551 60.0
0 1.79337692260742 61.0
165.91333431 165.989074707031 0.0
165.91333431 165.382766723633 1.0
164.85491325 161.222320556641 2.0
160.06584352 160.244125366211 3.0
0 3.71303486824036 4.0
-9.5499982473e-10 1.7773734331131 5.0
-71.664264161 -65.6970596313477 6.0
85.26876738 73.8433074951172 7.0
40.407049605 42.7232055664062 8.0
44.861717775 23.7758960723877 9.0
70.69684861 85.8271331787109 10.0
59.47694646 74.3417663574219 11.0
76.283465957 81.4239730834961 12.0
76.283465957 81.6078338623047 13.0
76.283465957 79.3901519775391 14.0
0 -0.440466821193695 15.0
0 0.435059428215027 16.0
-5.0042869294e-09 0.0723074674606323 17.0
59.765125017 57.7593574523926 18.0
-59.7651250407191 -57.5140838623047 19.0
-59.7651250187191 -59.9023170471191 20.0
-59.7651250297191 -56.0849342346191 21.0
78.972612228 76.0455627441406 22.0
138.22205466 141.996627807617 23.0
99.057356638 109.940643310547 24.0
11.705180701 22.0290355682373 25.0
217.194666899784 225.849166870117 26.0
217.194666899784 222.296600341797 27.0
28.8094353527839 26.6014232635498 28.0
246.004102249784 254.182037353516 29.0
246.004102249784 253.347518920898 30.0
47.343202682 46.6320343017578 31.0
9.3218295817 8.62320041656494 32.0
19.207487203 20.0475921630859 33.0
19.207487203 16.8119983673096 34.0
19.2074872 20.6830234527588 35.0
10.212255531 8.6940450668335 36.0
8.9458556559 8.0379638671875 37.0
-8.7105039606e-09 -0.524441301822662 38.0
-8.7105038714e-09 -4.77853488922119 39.0
-10.939001042 -13.1231002807617 40.0
-25.42583068 -26.8582649230957 41.0
4.3808906211 6.42677450180054 42.0
-31.969943109 -33.678638458252 43.0
73.457270044 94.1996459960938 44.0
73.457270044 95.3253555297852 45.0
5.6573354231 6.44148111343384 46.0
1.248980579 2.15072870254517 47.0
165.91333431 166.156860351562 48.0
160.06584353 155.285873413086 49.0
0 0.809720098972321 50.0
85.268767382 77.1408233642578 51.0
70.696848611 85.3040618896484 52.0
76.283465959 80.8775939941406 53.0
76.283465959 81.2314453125 54.0
76.283465959 80.2796325683594 55.0
0 -0.597361743450165 56.0
78.972612236 79.67236328125 57.0
217.1946669 220.908843994141 58.0
246.00410225 248.17951965332 59.0
19.207487207 20.576997756958 60.0
0 1.74727439880371 61.0
155.637104339877 166.587844848633 0.0
155.637104339877 167.282989501953 1.0
154.16776254 158.562973022461 2.0
148.70312733 153.517578125 3.0
1.0646380247 -8.35963344573975 4.0
1.06463803428188 8.99030303955078 5.0
-72.0621086190819 -59.6013946533203 6.0
81.1416724867866 71.127571105957 7.0
45.0808907437866 49.1983909606934 8.0
36.0607817427866 27.0397129058838 9.0
96.3374243067866 88.5754699707031 10.0
83.4733926307866 80.3317260742188 11.0
84.8869963567866 85.8347778320312 12.0
84.8869963567866 74.3229675292969 13.0
84.8869963567866 75.0870208740234 14.0
0 -1.87260580062866 15.0
0 0.332335472106934 16.0
-5.000964814e-09 -3.16841459274292 17.0
52.97538748252 54.2587890625 18.0
-52.9753874886799 -60.3420715332031 19.0
-52.9753874676799 -59.998851776123 20.0
-52.9753874726799 -56.2374687194824 21.0
74.6083269968883 77.8691635131836 22.0
143.16526813152 138.994964599609 23.0
111.5192151444 108.341621398926 24.0
22.7148384633999 19.5548839569092 25.0
217.773595129754 224.018936157227 26.0
217.773595129754 218.673583984375 27.0
25.1968256877537 24.1853427886963 28.0
242.970420819754 250.491668701172 29.0
242.970420819754 245.156829833984 30.0
46.9983026877922 44.915599822998 31.0
9.23239878156208 9.04692268371582 32.0
21.6329395232456 17.7893486022949 33.0
21.6329395232456 16.4908485412598 34.0
21.6329395319954 18.723669052124 35.0
10.260757919 10.3768997192383 36.0
8.9385966191 4.58684825897217 37.0
-9.143116014e-09 -7.1552939414978 38.0
-9.143116015e-09 -5.90771293640137 39.0
-11.0985178068399 -13.0565195083618 40.0
-24.5551210074508 -25.7593898773193 41.0
4.84337531208882 0.10114187002182 42.0
-32.1074139234888 -34.5739898681641 43.0
94.7377572604888 88.0042724609375 44.0
94.7377571959425 94.2622756958008 45.0
6.9596509637 2.46823143959045 46.0
2.5424144271 1.05478036403656 47.0
155.63710434 164.383422851562 48.0
148.70312734 153.2255859375 49.0
1.0646380343 -7.11886167526245 50.0
81.141672487 76.8535919189453 51.0
96.337424307 89.3755645751953 52.0
84.886996356 78.3330612182617 53.0
84.886996356 81.2938690185547 54.0
84.886996356 79.3526000976562 55.0
0 3.55653500556946 56.0
74.608326997 82.4492034912109 57.0
217.77359513 219.920745849609 58.0
242.97042082 245.884155273438 59.0
21.632939532 27.8692989349365 60.0
0 2.8526463508606 61.0
147.982033057583 157.456359863281 0.0
147.982033057583 151.963821411133 1.0
146.808413813287 153.349899291992 2.0
142.01934397 150.726043701172 3.0
-1.55484310406e-09 -0.649782240390778 4.0
-1.55484309999997e-09 -2.84008049964905 5.0
-64.9693040339936 -64.3212738037109 6.0
77.6547829144777 74.1871871948242 7.0
61.2786926634778 55.4030647277832 8.0
16.3760902514777 7.20984888076782 9.0
97.5261141374777 89.7391510009766 10.0
83.6472874924777 79.6450042724609 11.0
85.2981411144777 83.8642730712891 12.0
85.2981411144777 76.4624557495117 13.0
85.2981411144777 81.5436782836914 14.0
2.3328021736e-08 -2.03496789932251 15.0
2.3328021735e-08 -1.6052393913269 16.0
5.464350236e-08 -2.72059202194214 17.0
56.2518018559467 59.0783615112305 18.0
-56.2518018797166 -57.0891418457031 19.0
-56.2518018707166 -59.2854194641113 20.0
-56.2518017707166 -58.115608215332 21.0
77.1875821073724 73.3798065185547 22.0
153.454771802806 145.008895874023 23.0
115.220596133508 115.750228881836 24.0
21.7005584165075 20.8819332122803 25.0
230.642353915167 228.381210327148 26.0
230.642353915167 229.708221435547 27.0
26.8883420391666 21.3647422790527 28.0
257.530695955167 253.23681640625 29.0
257.530695955167 255.336776733398 30.0
46.7385281475469 45.537841796875 31.0
9.28724497493877 8.4355525970459 32.0
20.9357803369809 22.6376686096191 33.0
20.9357803369809 19.0790691375732 34.0
20.935780467613 20.4264030456543 35.0
8.86054244871795 8.39873600006104 36.0
7.60121916461794 6.85336446762085 37.0
2.62465480000001e-09 1.1959273815155 38.0
2.62465480000001e-09 -1.17553234100342 39.0
-11.6034306368583 -12.5643882751465 40.0
-24.5599220844463 -26.7411251068115 41.0
4.75986482786468 1.92820739746094 42.0
-31.7734218810648 -31.6524333953857 43.0
89.4696777832616 86.4044494628906 44.0
89.4696781210649 91.9600372314453 45.0
7.05809588635643 4.443039894104 46.0
2.46335994485643 2.26466846466064 47.0
147.98203306 156.096588134766 48.0
142.01934397 152.992614746094 49.0
0 -3.7475733757019 50.0
77.654782914 76.0236663818359 51.0
97.526114137 92.0755157470703 52.0
85.298141114 81.5651702880859 53.0
85.298141114 80.2131118774414 54.0
85.298141114 79.9203643798828 55.0
5.4643502359e-08 3.07315587997437 56.0
77.187582108 79.4883346557617 57.0
230.64235391 222.583419799805 58.0
257.53069595 249.491134643555 59.0
20.935780468 19.6116352081299 60.0
2.6246548239e-09 3.64096260070801 61.0
168.22939792 161.80744934082 0.0
168.22939792 161.741287231445 1.0
166.8551853 161.987121582031 2.0
162.06611558 154.941513061523 3.0
0 -3.38102102279663 4.0
-3.4430194723e-10 1.14741683006287 5.0
-64.717524185 -64.5426864624023 6.0
75.795757451 73.5059967041016 7.0
47.811512438 51.4997253417969 8.0
27.984245006 18.8855247497559 9.0
85.495582132 90.9133224487305 10.0
72.806800886 77.6438293457031 11.0
81.388609367 82.83935546875 12.0
81.388609367 79.3970413208008 13.0
81.388609367 81.0634994506836 14.0
0 -1.68036651611328 15.0
0 -0.587034523487091 16.0
0 -0.773277342319489 17.0
57.030575553 59.3173522949219 18.0
-57.030575568 -59.4718246459961 19.0
-57.030575546 -58.5947761535645 20.0
-57.030575558 -58.781867980957 21.0
78.468031474 78.1894683837891 22.0
143.48040884 144.007369995117 23.0
107.74571813 112.145500183105 24.0
21.65787533 20.3248405456543 25.0
221.94844032 221.194931030273 26.0
221.94844032 224.770721435547 27.0
25.034615474 25.1922798156738 28.0
246.9830558 249.78190612793 29.0
246.9830558 252.521362304688 30.0
46.868159797 47.6341247558594 31.0
9.3514247421 8.4442024230957 32.0
21.437455912 17.6786956787109 33.0
21.437455912 17.721212387085 34.0
21.437455911 19.6689872741699 35.0
9.7189938178 8.14753437042236 36.0
8.4390559148 8.11818408966064 37.0
-1.6180830419e-09 -0.13116979598999 38.0
-1.6180829237e-09 -1.77909874916077 39.0
-11.673813193 -18.1659965515137 40.0
-24.220127003 -25.3308277130127 41.0
5.2382433149 7.20737838745117 42.0
-32.062704899 -33.6589088439941 43.0
100.08206036 93.2306213378906 44.0
100.08206037 95.3342514038086 45.0
7.2221543507 7.74946308135986 46.0
2.5721167859 1.37758088111877 47.0
168.22939792 164.439270019531 48.0
162.06611559 148.876495361328 49.0
0 1.03792572021484 50.0
75.795757458 74.9175415039062 51.0
85.495582138 90.5932464599609 52.0
81.388609374 80.7393035888672 53.0
81.388609374 80.5549392700195 54.0
81.388609374 80.2425765991211 55.0
0 3.7588357925415 56.0
78.468031483 80.0088272094727 57.0
221.94844033 219.556533813477 58.0
246.98305581 247.573440551758 59.0
21.43745592 21.4858741760254 60.0
0 0.854927837848663 61.0
164.84154879 161.271331787109 0.0
164.84154879 167.210983276367 1.0
163.47930919 163.794281005859 2.0
158.69023947 160.321548461914 3.0
0 0.706295907497406 4.0
-4.0468769349e-14 2.94811487197876 5.0
-58.491897564 -60.9163131713867 6.0
70.357512535 73.8166656494141 7.0
51.232337208 47.436393737793 8.0
19.125175317 23.2492370605469 9.0
87.159604405 86.8559265136719 10.0
73.533108674 75.3401489257812 11.0
80.018953901 84.8341674804688 12.0
80.018953901 82.0210113525391 13.0
80.018953901 80.3478546142578 14.0
0 -1.05221343040466 15.0
0 -0.00106298923492432 16.0
-7.4578979994e-14 0.0666258335113525 17.0
58.704611565 60.0535774230957 18.0
-58.704611585 -59.343391418457 19.0
-58.704611563 -59.738208770752 20.0
-58.704611575 -56.8117065429688 21.0
80.467223632 75.9871063232422 22.0
146.76373537 144.755950927734 23.0
113.17330386 107.855262756348 24.0
23.084633007 18.6483631134033 25.0
227.23095901 227.412673950195 26.0
227.23095901 226.498336791992 27.0
24.88227757 25.9784355163574 28.0
252.11323659 254.187118530273 29.0
252.11323659 253.853805541992 30.0
46.981384966 45.9723167419434 31.0
9.3542172718 9.00819873809814 32.0
21.762612047 19.2253379821777 33.0
21.762612047 21.2942810058594 34.0
21.762612047 20.931568145752 35.0
9.5787196354 9.75351333618164 36.0
8.3747943394 7.46340417861938 37.0
-1.1902268272e-13 0.403506875038147 38.0
-1.1902268272e-13 -1.78502511978149 39.0
-11.784172657 -10.8179597854614 40.0
-25.081070163 -26.5109004974365 41.0
4.9115430991 5.38447904586792 42.0
-32.709799887 -34.3439903259277 43.0
94.81106231 90.1027069091797 44.0
94.811062324 94.771728515625 45.0
6.8163383027 4.88591718673706 46.0
2.6631233094 1.97947812080383 47.0
164.8415488 161.909332275391 48.0
158.69023948 157.754364013672 49.0
0 2.15846681594849 50.0
70.357512545 79.6644287109375 51.0
87.159604415 87.633056640625 52.0
80.018953911 81.2675323486328 53.0
80.018953911 81.6069793701172 54.0
80.018953911 80.8037719726562 55.0
0 1.42237138748169 56.0
80.467223642 79.4640274047852 57.0
227.23095902 221.368530273438 58.0
252.1132366 249.44482421875 59.0
21.762612057 19.881010055542 60.0
0 3.32653045654297 61.0
174.68306493 161.475463867188 0.0
174.68306493 168.026153564453 1.0
173.39491555 160.259262084961 2.0
168.60584583 165.641784667969 3.0
0 3.31988906860352 4.0
-6.0679551382e-13 2.94897079467773 5.0
-60.89541916 -60.703441619873 6.0
72.773688817 78.7125244140625 7.0
45.221305118 52.774112701416 8.0
27.552383689 21.4295654296875 9.0
83.34434583 87.6958847045898 10.0
70.832916922 75.0015258789062 11.0
79.159399821 81.0632934570312 12.0
79.159399821 80.3233108520508 13.0
79.159399821 79.9422378540039 14.0
0 0.213598191738129 15.0
0 0.529818534851074 16.0
-1.1308196494e-12 0.332491755485535 17.0
61.361266496 59.5489158630371 18.0
-61.361266516 -59.0195503234863 19.0
-61.361266494 -59.786922454834 20.0
-61.361266507 -61.6598358154297 21.0
83.62356725 80.1228561401367 22.0
139.95238012 141.843566894531 23.0
105.87904897 109.555274963379 24.0
22.128026702 22.2596817016602 25.0
223.57594738 226.605941772461 26.0
223.57594738 224.994171142578 27.0
25.845937869 27.3473987579346 28.0
249.42188525 254.101257324219 29.0
249.42188525 254.581497192383 30.0
46.874844231 46.1727447509766 31.0
9.3907165758 8.84638214111328 32.0
22.262300733 19.816463470459 33.0
22.262300733 17.8013515472412 34.0
22.262300733 20.5926704406738 35.0
9.5287428384 7.75536870956421 36.0
8.2499879739 7.2976131439209 37.0
-1.8121307213e-12 1.96029734611511 38.0
-1.8121307213e-12 0.92163497209549 39.0
-11.830747483 -15.2654428482056 40.0
-25.331477775 -23.1388530731201 41.0
4.9843583267 4.84584712982178 42.0
-32.746417896 -32.8084983825684 43.0
97.804317732 94.9270095825195 44.0
97.804317745 95.4181671142578 45.0
7.2015837174 5.89092683792114 46.0
2.5857712967 2.30231428146362 47.0
174.68306494 164.92985534668 48.0
168.60584584 157.646179199219 49.0
0 -0.0948031544685364 50.0
72.773688827 77.6631240844727 51.0
83.34434584 90.7836685180664 52.0
79.159399831 81.3405609130859 53.0
79.159399831 81.0017242431641 54.0
79.159399831 80.0534515380859 55.0
0 0.0413513779640198 56.0
83.62356726 80.1179656982422 57.0
223.57594739 224.142547607422 58.0
249.42188526 249.183868408203 59.0
22.262300743 21.9204692840576 60.0
0 1.20121669769287 61.0
147.982033057583 157.456390380859 0.0
147.982033057583 151.964889526367 1.0
146.808413813287 153.359130859375 2.0
142.01934397 150.795227050781 3.0
-1.55484310406e-09 -0.657764256000519 4.0
-1.55484309999997e-09 -2.84302091598511 5.0
-64.9693040339936 -64.32421875 6.0
77.6547829144777 74.1872253417969 7.0
61.2786926634778 55.4030647277832 8.0
16.3760902514777 7.20984888076782 9.0
97.5261141374777 89.7391510009766 10.0
83.6472874924777 79.6450042724609 11.0
85.2981411144777 83.8642730712891 12.0
85.2981411144777 76.4624557495117 13.0
85.2981411144777 81.5436782836914 14.0
2.3328021736e-08 -2.03496789932251 15.0
2.3328021735e-08 -1.6052393913269 16.0
5.464350236e-08 -2.72059202194214 17.0
56.2518018559467 59.0783615112305 18.0
-56.2518018797166 -57.0891418457031 19.0
-56.2518018707166 -59.2854194641113 20.0
-56.2518017707166 -58.115608215332 21.0
77.1875821073724 73.3798065185547 22.0
153.454771802806 145.008895874023 23.0
115.220596133508 115.750228881836 24.0
21.7005584165075 20.8819332122803 25.0
230.642353915167 228.381210327148 26.0
230.642353915167 229.708221435547 27.0
26.8883420391666 21.3647422790527 28.0
257.530695955167 253.23681640625 29.0
257.530695955167 255.336776733398 30.0
46.7385281475469 45.537841796875 31.0
9.28724497493877 8.4355525970459 32.0
20.9357803369809 22.6376686096191 33.0
20.9357803369809 19.0790691375732 34.0
20.935780467613 20.4264030456543 35.0
8.86054244871795 8.39873600006104 36.0
7.60121916461794 6.85336446762085 37.0
2.62465480000001e-09 1.1959273815155 38.0
2.62465480000001e-09 -1.17553234100342 39.0
-11.6034306368583 -12.5643882751465 40.0
-24.5599220844463 -26.7411251068115 41.0
4.75986482786468 1.92820739746094 42.0
-31.7734218810648 -31.6524333953857 43.0
89.4696777832616 86.4044494628906 44.0
89.4696781210649 91.9600372314453 45.0
7.05809588635643 4.443039894104 46.0
2.46335994485643 2.26466846466064 47.0
147.98203306 156.096145629883 48.0
142.01934397 153.049011230469 49.0
0 -3.74314832687378 50.0
77.654782914 76.0234603881836 51.0
97.526114137 92.0755157470703 52.0
85.298141114 81.5651702880859 53.0
85.298141114 80.2131118774414 54.0
85.298141114 79.9203643798828 55.0
5.4643502359e-08 3.07315587997437 56.0
77.187582108 79.4883346557617 57.0
230.64235391 222.583419799805 58.0
257.53069595 249.491134643555 59.0
20.935780468 19.6116352081299 60.0
2.6246548239e-09 3.64096260070801 61.0
155.439662787599 163.337280273438 0.0
155.439662787599 165.126602172852 1.0
154.30230653 160.011566162109 2.0
149.51323678 143.671264648438 3.0
0 -0.321387112140656 4.0
-6.420204037e-09 4.96934270858765 5.0
-58.8392240852395 -60.8455009460449 6.0
68.7627721478381 75.1139755249023 7.0
54.5942969048381 49.8149223327637 8.0
14.1684752428381 27.1934108734131 9.0
86.0010369738381 89.7719345092773 10.0
73.5807551968381 77.7846832275391 11.0
80.4739642328381 81.5004959106445 12.0
80.4739642328381 81.3285064697266 13.0
80.4739642328381 78.5678024291992 14.0
3.9152147791e-09 -1.1339225769043 15.0
0 0.278671145439148 16.0
2.61172055999999e-09 -1.03324484825134 17.0
60.468305371639 58.3043632507324 18.0
-60.4683053517572 -59.2306175231934 19.0
-60.4683053317572 -60.1124000549316 20.0
-60.4683053217572 -57.6546058654785 21.0
81.682429163713 77.8001327514648 22.0
146.379765302639 143.742172241211 23.0
110.062697210799 112.264030456543 24.0
22.4296901667987 21.9177856445312 25.0
228.062194465198 223.014770507812 26.0
228.062194465198 222.323669433594 27.0
27.9351146371978 26.8501014709473 28.0
255.997309095198 257.388793945312 29.0
255.997309095198 257.382690429688 30.0
46.7240133833509 47.4519462585449 31.0
9.36354750899571 9.32487487792969 32.0
21.2141238417184 17.8097953796387 33.0
21.2141238417184 23.0975475311279 34.0
21.2141238677718 20.5346145629883 35.0
9.1469342787689 5.46794748306274 36.0
7.9580722208689 6.99554109573364 37.0
-7.47642845999998e-09 2.02303838729858 38.0
-7.47642844999999e-09 -2.50212097167969 39.0
-11.6769694638786 -12.8153533935547 40.0
-25.0182964125618 -24.042085647583 41.0
5.33947917523177 5.32516098022461 42.0
-32.9808792835318 -33.7839431762695 43.0
98.2423104375318 96.9407958984375 44.0
98.2423103395318 94.5310745239258 45.0
6.8270840345 8.05819129943848 46.0
2.6492598031 1.59819686412811 47.0
155.43966279 164.126281738281 48.0
149.51323679 143.260803222656 49.0
0 2.17118644714355 50.0
68.762772148 75.5315780639648 51.0
86.001036974 90.9975204467773 52.0
80.473964233 80.8694305419922 53.0
80.473964233 81.3702163696289 54.0
80.473964233 78.6575012207031 55.0
2.6117205574e-09 -0.555000841617584 56.0
81.682429164 79.3925399780273 57.0
228.06219446 221.316833496094 58.0
255.9973091 247.795120239258 59.0
21.214123868 21.6998443603516 60.0
0 1.56267380714417 61.0
177.78446364 156.513153076172 0.0
177.78446364 162.869171142578 1.0
176.41996462 164.206665039062 2.0
170.95532886 164.239196777344 3.0
0.67748415553 -0.559761583805084 4.0
0.67748415571 3.64633440971375 5.0
-67.093244272 -65.6517486572266 6.0
78.435453126 76.1014633178711 7.0
47.637388567 52.1387939453125 8.0
30.798064549 22.1250038146973 9.0
87.539372279 83.8738327026367 10.0
74.810642459 72.9772796630859 11.0
80.07931631 82.5814590454102 12.0
80.07931631 79.0310363769531 13.0
80.07931631 80.2618713378906 14.0
0 -3.07081294059753 15.0
0 -0.696132242679596 16.0
0 0.481258988380432 17.0
56.798777369 59.2578811645508 18.0
-56.798777379 -58.2444801330566 19.0
-56.798777358 -59.7374267578125 20.0
-56.79877737 -60.0397300720215 21.0
79.007105484 77.9094543457031 22.0
139.41872599 141.624328613281 23.0
105.91592169 108.725219726562 24.0
21.441916501 19.6990737915039 25.0
218.42583148 219.787170410156 26.0
218.42583148 221.636932373047 27.0
25.944059549 27.3495445251465 28.0
244.36989104 254.787109375 29.0
244.36989104 257.717132568359 30.0
46.971400309 46.0481796264648 31.0
9.3706955991 8.90136432647705 32.0
22.208328106 18.9254245758057 33.0
22.208328106 17.5182361602783 34.0
22.208328105 20.1714420318604 35.0
9.9893807486 8.91634082794189 36.0
8.7115742034 6.50516557693481 37.0
-4.7818052895e-10 -0.401360809803009 38.0
-4.781805745e-10 -2.8292875289917 39.0
-11.519078161 -14.9320249557495 40.0
-23.996323586 -25.023458480835 41.0
4.9469242204 7.04139471054077 42.0
-31.392111544 -34.7502136230469 43.0
90.381846896 88.4025497436523 44.0
90.381846908 94.4198608398438 45.0
6.6804054069 6.9216194152832 46.0
2.5042770677 2.66599488258362 47.0
177.78446365 159.377655029297 48.0
170.95532887 161.026489257812 49.0
0.67748416553 -2.41484403610229 50.0
78.435453135 78.5746688842773 51.0
87.539372289 85.3046493530273 52.0
80.079316319 82.0100402832031 53.0
80.079316319 80.4009704589844 54.0
80.079316319 79.7129440307617 55.0
0 2.82733631134033 56.0
79.007105494 78.9993667602539 57.0
218.42583149 220.153472900391 58.0
244.36989105 249.837310791016 59.0
22.208328115 23.438383102417 60.0
0 2.2962954044342 61.0
157.36929389 164.27229309082 0.0
157.36929389 168.359237670898 1.0
156.19034304 163.481307983398 2.0
151.40127332 149.99658203125 3.0
0 0.881423652172089 4.0
-4.4001625206e-12 1.28015303611755 5.0
-63.202481494 -62.7311019897461 6.0
73.940932648 75.2346725463867 7.0
49.21522014 48.8494338989258 8.0
24.725712498 25.1796379089355 9.0
82.071308945 87.9838638305664 10.0
69.710579191 74.2460708618164 11.0
78.268452594 81.4736785888672 12.0
78.268452594 78.1099090576172 13.0
78.268452594 78.0607223510742 14.0
0 -1.73987436294556 15.0
0 0.20755273103714 16.0
-8.6498091718e-12 2.28687357902527 17.0
55.991161147 58.3939933776855 18.0
-55.991161167 -58.2031936645508 19.0
-55.991161145 -60.6356239318848 20.0
-55.991161157 -58.4873924255371 21.0
78.143484672 77.9015045166016 22.0
143.15317945 142.212677001953 23.0
109.4842638 109.648582458496 24.0
23.643519739 20.927677154541 25.0
221.29666413 228.536636352539 26.0
221.29666413 226.047409057617 27.0
25.631460129 25.2963237762451 28.0
246.92812427 256.373504638672 29.0
246.92812427 258.433868408203 30.0
46.797891875 47.0584945678711 31.0
9.3851534707 8.83886432647705 32.0
22.152323504 20.3074417114258 33.0
22.152323504 18.4097843170166 34.0
22.152323504 20.0688018798828 35.0
9.5374703059 8.62298393249512 36.0
8.2534785553 7.0811595916748 37.0
-1.756954887e-11 -2.27432584762573 38.0
-1.7569546978e-11 -3.01892614364624 39.0
-11.801832176 -13.6235008239746 40.0
-24.540615721 -25.7453060150146 41.0
5.2053856641 7.12257862091064 42.0
-32.287741837 -33.7818374633789 43.0
101.53993446 95.6819458007812 44.0
101.53993448 93.4483795166016 45.0
6.8535260139 7.25196695327759 46.0
2.5912805594 2.4670615196228 47.0
157.3692939 164.761383056641 48.0
151.40127333 146.843017578125 49.0
0 0.856912851333618 50.0
73.940932658 76.8228073120117 51.0
82.071308955 87.3712387084961 52.0
78.268452604 81.4715957641602 53.0
78.268452604 81.4059982299805 54.0
78.268452604 79.6282348632812 55.0
0 1.05015206336975 56.0
78.143484682 78.9081726074219 57.0
221.29666414 222.131652832031 58.0
246.92812428 249.281433105469 59.0
22.152323514 20.3138408660889 60.0
0 2.08443331718445 61.0
176.29860482 168.580444335938 0.0
176.29860482 173.377548217773 1.0
175.16865358 163.713333129883 2.0
170.37958384 161.690292358398 3.0
0 3.24395108222961 4.0
0 5.51086568832397 5.0
-65.270666863 -62.9606819152832 6.0
76.762728385 76.8805313110352 7.0
42.488702145 51.4834098815918 8.0
34.274026231 19.5944557189941 9.0
89.516854999 94.0824127197266 10.0
77.663058224 73.4254837036133 11.0
78.940936199 80.7382888793945 12.0
78.940936199 79.2204360961914 13.0
78.940936199 80.9472503662109 14.0
-1.1623383031e-08 -2.45532608032227 15.0
0 -1.02141404151917 16.0
0 -1.96145725250244 17.0
55.062952957 59.437572479248 18.0
-55.062952969 -58.469108581543 19.0
-55.062952947 -58.6026077270508 20.0
-55.062952959 -59.3681259155273 21.0
76.457871931 78.7075500488281 22.0
148.98639806 139.289047241211 23.0
112.48565108 113.441032409668 24.0
20.324403424 19.6628036499023 25.0
225.44427001 226.562286376953 26.0
225.44427001 223.700317382812 27.0
29.722881407 30.1496410369873 28.0
255.16715142 253.909317016602 29.0
255.16715142 254.789642333984 30.0
46.378282561 45.0921783447266 31.0
9.4329365578 9.13247203826904 32.0
21.394918966 15.1990261077881 33.0
21.394918966 24.057279586792 34.0
21.394918964 21.1588687896729 35.0
9.3126087622 4.91349315643311 36.0
8.1259553792 7.98107290267944 37.0
-7.4149137691e-10 0.0585658550262451 38.0
-7.4150109029e-10 0.308051109313965 39.0
-11.606094901 -17.6695861816406 40.0
-24.942731654 -25.4256763458252 41.0
5.6454245049 4.46205472946167 42.0
-33.259090558 -30.9183597564697 43.0
91.727018377 88.2412948608398 44.0
91.727018389 95.6060180664062 45.0
6.4433716383 8.38315296173096 46.0
2.4124310659 2.25329637527466 47.0
176.29860483 167.677124023438 48.0
170.37958385 168.160797119141 49.0
0 -0.781177580356598 50.0
76.762728393 74.7907485961914 51.0
89.516855007 88.8126525878906 52.0
78.940936207 80.8297119140625 53.0
78.940936207 76.5449142456055 54.0
78.940936207 78.9845657348633 55.0
5.7767483422e-12 1.50308060646057 56.0
76.45787194 81.0239868164062 57.0
225.44427001 220.678131103516 58.0
255.16715143 251.577133178711 59.0
21.394918974 21.5973052978516 60.0
0 2.14086961746216 61.0
158.05603952 167.419097900391 0.0
158.05603952 162.144897460938 1.0
156.52449475 147.829681396484 2.0
151.73542499 145.744750976562 3.0
0 5.36371612548828 4.0
0 -12.1541585922241 5.0
-61.630763191 -79.925666809082 6.0
74.018161276 82.4157943725586 7.0
46.718249105 31.2298469543457 8.0
27.299912161 18.2428722381592 9.0
82.810333666 90.9979476928711 10.0
68.619616017 70.1046676635742 11.0
70.069503943 77.1744689941406 12.0
70.069503943 83.4726028442383 13.0
70.069503943 79.1923065185547 14.0
0 9.53188228607178 15.0
0 -3.81320858001709 16.0
0 1.72674071788788 17.0
58.053432727 56.0061416625977 18.0
-58.053432737 -58.0427513122559 19.0
-58.053432715 -54.4132308959961 20.0
-58.053432728 -53.3800659179688 21.0
80.382500248 76.5772247314453 22.0
148.69019065 143.49446105957 23.0
110.3901322 106.363052368164 24.0
19.532157938 19.8203716278076 25.0
229.07269091 221.509384155273 26.0
229.07269091 217.391937255859 27.0
25.772358325 16.2064552307129 28.0
254.84504924 258.373016357422 29.0
254.84504924 247.268295288086 30.0
46.821077276 54.5309410095215 31.0
9.20590207 10.2766008377075 32.0
22.329067511 16.1239490509033 33.0
22.329067511 12.5439128875732 34.0
22.329067511 20.7414588928223 35.0
10.657124263 7.92519903182983 36.0
9.468907638 7.88749694824219 37.0
-1.1372044016e-13 -4.84613418579102 38.0
-1.1372671877e-13 0.525445520877838 39.0
-10.407142258 -21.8385734558105 40.0
-22.183119932 -23.5663261413574 41.0
5.8842323768 1.86289405822754 42.0
-30.148246898 -34.6149749755859 43.0
111.15285693 95.6768493652344 44.0
111.15285694 90.0571899414062 45.0
7.0547961518 18.3261394500732 46.0
2.9488525526 2.98372411727905 47.0
158.05603953 162.360580444336 48.0
151.735425 146.356735229492 49.0
0 10.0269107818604 50.0
74.018161286 74.59228515625 51.0
82.810333676 83.3695297241211 52.0
70.069503953 74.5173645019531 53.0
70.069503953 74.0831756591797 54.0
70.069503953 78.7452087402344 55.0
0 -0.377448379993439 56.0
80.382500258 78.0791625976562 57.0
229.07269092 219.218109130859 58.0
254.84504925 248.735687255859 59.0
22.329067521 18.9575977325439 60.0
0 4.22753810882568 61.0
137.17687742 143.006317138672 0.0
137.17687742 143.037628173828 1.0
135.86355256 143.48908996582 2.0
131.0744823 135.680419921875 3.0
0 -2.02137804031372 4.0
0 0.341359198093414 5.0
-59.42861149 -59.2226219177246 6.0
69.705106679 70.1771240234375 7.0
65.43029471 62.4528312683105 8.0
4.2748119615 8.48783111572266 9.0
78.669161727 80.2065200805664 10.0
65.229460649 71.2585144042969 11.0
77.500548395 80.8047180175781 12.0
77.500548395 77.9000701904297 13.0
77.500548395 76.9198608398438 14.0
0 -0.64198762178421 15.0
0 0.28728449344635 16.0
0 -3.15485000610352 17.0
59.754945382 58.5015258789062 18.0
-59.754945394 -59.4383087158203 19.0
-59.754945372 -60.0649642944336 20.0
-59.754945384 -61.8801422119141 21.0
80.656571382 77.8813858032227 22.0
140.87903669 140.394561767578 23.0
106.41719675 107.357681274414 24.0
22.90053999 20.4083709716797 25.0
221.53560808 223.594223022461 26.0
221.53560808 221.987747192383 27.0
27.282645659 25.3551406860352 28.0
248.81825375 253.574676513672 29.0
248.81825375 252.408157348633 30.0
47.033013387 47.2299919128418 31.0
9.3131031344 9.18030452728271 32.0
20.901625993 20.4819431304932 33.0
20.901625993 21.1129970550537 34.0
20.901625993 20.8195400238037 35.0
9.3083381265 8.28715133666992 36.0
8.044767894 8.38281917572021 37.0
-2.5262790395e-09 -0.804296553134918 38.0
-2.5262785966e-09 0.113371789455414 39.0
-12.387792857 -14.5510091781616 40.0
-23.81592281 -27.5234241485596 41.0
4.9278776077 3.5162615776062 42.0
-31.108913619 -34.7394485473633 43.0
92.052766479 91.3064651489258 44.0
92.05276649 90.2343292236328 45.0
6.7273095346 3.00315618515015 46.0
2.6568330431 1.83220052719116 47.0
137.17687743 136.685623168945 48.0
131.07448231 127.40372467041 49.0
0 3.56358671188354 50.0
69.705106686 70.9869613647461 51.0
78.669161734 84.2529830932617 52.0
77.500548402 81.0479965209961 53.0
77.500548402 80.9212951660156 54.0
77.500548402 79.4541549682617 55.0
0 -0.0822644829750061 56.0
80.656571391 80.3950500488281 57.0
221.53560809 218.26057434082 58.0
248.81825375 246.469757080078 59.0
20.901626002 19.9398994445801 60.0
0 2.89454770088196 61.0
192.04060388 169.70783996582 0.0
192.04060388 160.703155517578 1.0
190.79559302 173.271377563477 2.0
186.00652329 171.783386230469 3.0
0 3.79397869110107 4.0
-1.9843393692e-12 14.0036382675171 5.0
-69.649550077 -34.2856712341309 6.0
81.966550601 99.9586791992188 7.0
53.71743788 43.9833564758301 8.0
28.249112711 16.6087741851807 9.0
96.492282225 69.7925338745117 10.0
83.940675251 74.3257675170898 11.0
88.527824268 80.3586883544922 12.0
88.527824268 77.9436492919922 13.0
88.527824268 81.6979446411133 14.0
0 -0.809234440326691 15.0
0 -0.909365117549896 16.0
-1.6293462721e-12 -0.976521074771881 17.0
55.23486981 54.1554756164551 18.0
-55.23486983 -59.069393157959 19.0
-55.234869808 -59.8050079345703 20.0
-55.23486982 -55.4216156005859 21.0
76.497123892 75.5044937133789 22.0
137.14516542 141.606430053711 23.0
104.68893665 114.713569641113 24.0
20.025103816 20.2468490600586 25.0
213.64228933 230.280944824219 26.0
213.64228933 227.063858032227 27.0
29.13985109 28.7818183898926 28.0
242.78214043 254.231292724609 29.0
242.78214043 254.240097045898 30.0
47.042364809 49.1876106262207 31.0
9.3171841719 9.16771602630615 32.0
21.262254062 18.2827396392822 33.0
21.262254062 21.0859107971191 34.0
21.262254062 19.7067832946777 35.0
9.4816960264 10.1337938308716 36.0
8.1100822919 8.65343570709229 37.0
-5.0180338057e-12 -3.94859075546265 38.0
-5.0180347893e-12 0.0306174159049988 39.0
-10.342314443 -17.6593456268311 40.0
-23.045286171 -27.89479637146 41.0
5.4437539399 3.42345881462097 42.0
-31.303272407 -31.6381015777588 43.0
97.103900357 95.3370361328125 44.0
97.10390037 95.6479644775391 45.0
6.7663603695 6.8359489440918 46.0
1.9660756734 3.30141258239746 47.0
192.04060389 162.335815429688 48.0
186.0065233 166.159439086914 49.0
0 0.0356399416923523 50.0
81.966550611 79.9598999023438 51.0
96.492282235 87.3791351318359 52.0
88.527824278 81.3931884765625 53.0
88.527824278 80.8326568603516 54.0
88.527824278 79.1809844970703 55.0
0 1.07269859313965 56.0
76.497123902 80.9557952880859 57.0
213.64228934 219.421722412109 58.0
242.78214044 248.056762695312 59.0
21.262254072 19.0842914581299 60.0
0 0.804853439331055 61.0
212.391738827015 153.841064453125 0.0
212.391738827015 156.152221679688 1.0
211.04319376 162.610855102539 2.0
205.5785585 183.54035949707 3.0
18.404050808 -11.9831399917603 4.0
18.4040508170453 16.5000095367432 5.0
-61.3077697580453 -106.728057861328 6.0
74.2237211591586 101.702270507812 7.0
47.5084168701586 49.3908500671387 8.0
26.7153042891586 26.0035495758057 9.0
73.6279026801586 82.7762222290039 10.0
62.1474872191586 58.7497711181641 11.0
78.3848871701586 68.9429779052734 12.0
78.3848871701586 72.1487426757812 13.0
78.3848871701585 84.5159912109375 14.0
4.213840709e-10 7.65522480010986 15.0
4.2138407099e-10 3.21506071090698 16.0
5.24329936999999e-09 4.69373941421509 17.0
56.7790807251774 59.1677894592285 18.0
-56.7790806952379 -56.4286880493164 19.0
-56.7790806772379 -56.0826797485352 20.0
-56.7790806622379 -68.0509414672852 21.0
79.0602866060699 71.6340713500977 22.0
149.501502613177 134.62646484375 23.0
119.756521836472 102.935264587402 24.0
24.5677496374717 26.2532119750977 25.0
228.561789214029 220.424560546875 26.0
228.561789214029 206.224975585938 27.0
27.4961151460291 28.7196178436279 28.0
256.057904364029 246.56787109375 29.0
256.057904364029 273.834167480469 30.0
45.391893868001 38.7327346801758 31.0
8.76348561509971 6.41286516189575 32.0
22.2812059441359 13.2257738113403 33.0
22.2812059441359 20.2075481414795 34.0
22.2812059764684 22.1812534332275 35.0
8.66141742863524 15.2447719573975 36.0
7.34531140923524 4.93151473999023 37.0
-6.99365361999996e-09 -6.4220929145813 38.0
-6.99365360999999e-09 -2.56813478469849 39.0
-12.1264598581189 -3.07913827896118 40.0
-26.0813195914196 -26.7585544586182 41.0
5.60695360219884 0.621493518352509 42.0
-34.4708309687988 -32.5315818786621 43.0
84.1913674207981 89.0402679443359 44.0
84.1913670312712 90.0791091918945 45.0
6.6020739941 6.19962120056152 46.0
2.5827328274 1.97778654098511 47.0
212.39173882 158.674377441406 48.0
205.57855851 163.200561523438 49.0
18.404050817 2.99647617340088 50.0
74.223721159 75.4421081542969 51.0
73.62790268 83.2851333618164 52.0
78.38488717 75.5351104736328 53.0
78.38488717 68.6956787109375 54.0
78.38488717 81.1668319702148 55.0
5.2432993646e-09 5.09521484375 56.0
79.060286606 76.9335174560547 57.0
228.56178922 207.424560546875 58.0
256.05790436 258.141723632812 59.0
22.281205976 22.5056571960449 60.0
0 6.23766803741455 61.0
178.207640723141 164.794021606445 0.0
178.207640723141 169.657516479492 1.0
176.96929637 165.466583251953 2.0
171.50466118 177.473114013672 3.0
0.011311137038 -0.465768873691559 4.0
0.011311145186 2.47690844535828 5.0
-69.815234849 -65.2861633300781 6.0
80.5036806957776 74.8213806152344 7.0
47.6899974787776 54.906551361084 8.0
32.8136832167776 25.8940162658691 9.0
81.6465450127776 90.9756698608398 10.0
68.4703389247776 76.6848754882812 11.0
78.0568804337776 82.0586166381836 12.0
78.0568804337776 78.5951385498047 13.0
78.0568804337776 79.7747039794922 14.0
0 -1.74653077125549 15.0
0 -0.232097208499908 16.0
-2.229436111e-09 -0.978347837924957 17.0
59.6751761000605 56.7601928710938 18.0
-59.6751761071664 -59.5052185058594 19.0
-59.6751760861664 -59.7838439941406 20.0
-59.6751760941664 -60.2965087890625 21.0
80.96986919 79.8902740478516 22.0
147.552067054061 143.310745239258 23.0
111.998711180076 114.614204406738 24.0
21.8624391480756 21.8418960571289 25.0
228.521936246282 222.758880615234 26.0
228.521936246282 223.840194702148 27.0
26.7706270442819 26.572301864624 28.0
255.292563296282 253.079330444336 29.0
255.292563296282 251.376190185547 30.0
46.90172438 46.5974349975586 31.0
9.4044095005 8.56643676757812 32.0
21.2946930979522 20.9325866699219 33.0
21.2946930979522 22.147403717041 34.0
21.2946931019233 20.9858531951904 35.0
9.2033128712 9.96604156494141 36.0
7.9497293985 6.93172693252563 37.0
-9.264059778e-09 -0.302216351032257 38.0
-9.264059779e-09 -2.54786014556885 39.0
-11.6972307900832 -11.6977939605713 40.0
-24.5134038461412 -25.968318939209 41.0
5.35222024894801 4.1248083114624 42.0
-32.524666555648 -32.4774208068848 43.0
98.681922190648 98.7723541259766 44.0
98.681922183648 95.1378860473633 45.0
6.9664429612 4.89138698577881 46.0
2.4247044541 2.79537320137024 47.0
178.20764072 165.184616088867 48.0
171.50466119 171.192260742188 49.0
0.011311146836 3.18514800071716 50.0
80.503680696 77.8827896118164 51.0
81.646545013 89.9194869995117 52.0
78.056880433 80.8806762695312 53.0
78.056880433 80.3594436645508 54.0
78.056880433 79.6668243408203 55.0
0 3.39220094680786 56.0
80.969869192 78.6907196044922 57.0
228.52193625 221.251800537109 58.0
255.29256329 249.256927490234 59.0
21.294693102 21.2775440216064 60.0
0 2.10299062728882 61.0
137.17687742 143.006256103516 0.0
137.17687742 143.039810180664 1.0
135.86355256 143.473709106445 2.0
131.0744823 135.652114868164 3.0
0 -1.9895703792572 4.0
0 0.341125309467316 5.0
-59.42861149 -59.2186317443848 6.0
69.705106679 70.1770935058594 7.0
65.43029471 62.4528312683105 8.0
4.2748119615 8.48783111572266 9.0
78.669161727 80.2065200805664 10.0
65.229460649 71.2585144042969 11.0
77.500548395 80.8047180175781 12.0
77.500548395 77.9000701904297 13.0
77.500548395 76.9198608398438 14.0
0 -0.64198762178421 15.0
0 0.28728449344635 16.0
0 -3.15485000610352 17.0
59.754945382 58.5015258789062 18.0
-59.754945394 -59.4383087158203 19.0
-59.754945372 -60.0649642944336 20.0
-59.754945384 -61.8801422119141 21.0
80.656571382 77.8813858032227 22.0
140.87903669 140.394561767578 23.0
106.41719675 107.357681274414 24.0
22.90053999 20.4083709716797 25.0
221.53560808 223.594223022461 26.0
221.53560808 221.987747192383 27.0
27.282645659 25.3551406860352 28.0
248.81825375 253.574676513672 29.0
248.81825375 252.408157348633 30.0
47.033013387 47.2299919128418 31.0
9.3131031344 9.18030452728271 32.0
20.901625993 20.4819431304932 33.0
20.901625993 21.1129970550537 34.0
20.901625993 20.8195400238037 35.0
9.3083381265 8.28715133666992 36.0
8.044767894 8.38281917572021 37.0
-2.5262790395e-09 -0.804296553134918 38.0
-2.5262785966e-09 0.113371789455414 39.0
-12.387792857 -14.5510091781616 40.0
-23.81592281 -27.5234241485596 41.0
4.9278776077 3.5162615776062 42.0
-31.108913619 -34.7394485473633 43.0
92.052766479 91.3064651489258 44.0
92.05276649 90.2343292236328 45.0
6.7273095346 3.00315618515015 46.0
2.6568330431 1.83220052719116 47.0
137.17687743 136.686950683594 48.0
131.07448231 127.357582092285 49.0
0 3.57017040252686 50.0
69.705106686 70.9871673583984 51.0
78.669161734 84.2529830932617 52.0
77.500548402 81.0479965209961 53.0
77.500548402 80.9212951660156 54.0
77.500548402 79.4541549682617 55.0
0 -0.0822644829750061 56.0
80.656571391 80.3950500488281 57.0
221.53560809 218.26057434082 58.0
248.81825375 246.469757080078 59.0
20.901626002 19.9398994445801 60.0
0 2.89454770088196 61.0
172.28800891 164.801223754883 0.0
172.28800891 164.864456176758 1.0
170.91200536 157.22526550293 2.0
166.12293563 155.440658569336 3.0
0 3.68533158302307 4.0
-3.9722199451e-14 1.32631063461304 5.0
-61.274900136 -66.5722885131836 6.0
74.354078879 73.93603515625 7.0
44.785720759 53.9901885986328 8.0
29.56835811 41.6500129699707 9.0
97.842406111 92.4045486450195 10.0
84.63981625 75.2420806884766 11.0
83.878115524 81.2909469604492 12.0
83.878115524 80.4497222900391 13.0
83.878115524 80.1469879150391 14.0
0 1.8238308429718 15.0
0 0.970125317573547 16.0
-7.1864597557e-14 5.05307722091675 17.0
53.752351497 60.0706024169922 18.0
-53.752351517 -57.6053810119629 19.0
-53.752351495 -58.2016563415527 20.0
-53.752351508 -64.3277282714844 21.0
76.235189332 75.9088516235352 22.0
142.42438672 142.868225097656 23.0
109.56420672 114.170463562012 24.0
20.91258406 19.1376533508301 25.0
218.65957606 222.679504394531 26.0
218.65957606 224.602615356445 27.0
27.523890894 25.2965526580811 28.0
246.18346697 251.719436645508 29.0
246.18346697 248.094360351562 30.0
46.272328162 44.8565559387207 31.0
9.4213914206 6.53609657287598 32.0
22.482837814 19.1265182495117 33.0
22.482837814 22.3240509033203 34.0
22.482837814 22.9536247253418 35.0
10.001514514 11.2502279281616 36.0
8.7844763001 6.04310512542725 37.0
-1.1406739484e-13 -6.11618709564209 38.0
-1.1406739484e-13 -4.09970092773438 39.0
-13.213612113 -14.5043878555298 40.0
-26.112887992 -29.1668395996094 41.0
4.9944217456 3.99777245521545 42.0
-33.853769223 -33.9650802612305 43.0
84.615870429 84.6892700195312 44.0
84.615870442 81.8800582885742 45.0
6.8950440142 10.3680782318115 46.0
2.8387258744 3.02762293815613 47.0
172.28800892 166.074554443359 48.0
166.12293564 151.147705078125 49.0
0 5.51293849945068 50.0
74.354078889 73.4553604125977 51.0
97.842406121 80.6607894897461 52.0
83.878115534 79.7698135375977 53.0
83.878115534 80.1583099365234 54.0
83.878115534 79.5986557006836 55.0
0 -2.21276044845581 56.0
76.235189342 76.9362411499023 57.0
218.65957607 221.116012573242 58.0
246.18346698 245.8330078125 59.0
22.482837824 22.7274360656738 60.0
0 1.96875524520874 61.0
172.88771985 159.216583251953 0.0
172.88771985 165.944061279297 1.0
171.69414474 160.36882019043 2.0
166.9050745 159.708114624023 3.0
0 -5.72295761108398 4.0
0 -0.0561079382896423 5.0
-58.09247028 -62.5938339233398 6.0
69.814779674 76.3981170654297 7.0
54.071270728 51.3997993469238 8.0
15.743508936 20.1744251251221 9.0
88.572345336 85.5068130493164 10.0
75.650769891 73.9368667602539 11.0
81.176639076 81.5606079101562 12.0
81.176639076 82.8508758544922 13.0
81.176639076 80.0913009643555 14.0
0 -0.0340642333030701 15.0
0 -0.0111111402511597 16.0
0 -0.643964231014252 17.0
57.433940052 59.1439590454102 18.0
-57.433940063 -59.4892158508301 19.0
-57.433940041 -59.8360481262207 20.0
-57.433940053 -58.4611167907715 21.0
79.506319721 77.1096954345703 22.0
147.19071558 144.219802856445 23.0
114.13610346 111.639984130859 24.0
21.614200962 21.3315982818604 25.0
226.69703531 223.701538085938 26.0
226.69703531 224.886306762695 27.0
27.794745074 27.5343494415283 28.0
254.49178039 253.573883056641 29.0
254.49178039 249.379302978516 30.0
46.852097838 46.3834075927734 31.0
9.3889927428 8.33438491821289 32.0
22.072379658 20.3859081268311 33.0
22.072379658 21.65403175354 34.0
22.072379657 20.622163772583 35.0
9.5364417676 7.99863910675049 36.0
8.2293706154 7.37063884735107 37.0
0 -0.505694448947906 38.0
0 -0.288975179195404 39.0
-11.855990362 -15.9893598556519 40.0
-24.660688049 -24.0361442565918 41.0
5.2255809617 4.37536096572876 42.0
-32.558172511 -34.2521209716797 43.0
93.36094152 89.9708099365234 44.0
93.360941533 92.5578689575195 45.0
6.9430950182 6.24269914627075 46.0
2.5203046831 2.25370144844055 47.0
172.88771986 163.539688110352 48.0
166.90507451 161.955673217773 49.0
0 -0.507585108280182 50.0
69.814779684 75.0684509277344 51.0
88.572345346 89.0556793212891 52.0
81.176639086 81.4612731933594 53.0
81.176639086 80.6294631958008 54.0
81.176639086 78.8301315307617 55.0
0 2.30692553520203 56.0
79.506319731 79.1343841552734 57.0
226.69703532 221.258407592773 58.0
254.4917804 247.444686889648 59.0
22.072379667 19.2788066864014 60.0
0 1.50705409049988 61.0
178.207640723141 164.794021606445 0.0
178.207640723141 169.657653808594 1.0
176.96929637 165.472442626953 2.0
171.50466118 177.500793457031 3.0
0.011311137038 -0.463097631931305 4.0
0.011311145186 2.4756293296814 5.0
-69.815234849 -65.2874984741211 6.0
80.5036806957776 74.821403503418 7.0
47.6899974787776 54.906551361084 8.0
32.8136832167776 25.8940162658691 9.0
81.6465450127776 90.9756698608398 10.0
68.4703389247776 76.6848754882812 11.0
78.0568804337776 82.0586166381836 12.0
78.0568804337776 78.5951385498047 13.0
78.0568804337776 79.7747039794922 14.0
0 -1.74653077125549 15.0
0 -0.232097208499908 16.0
-2.229436111e-09 -0.978347837924957 17.0
59.6751761000605 56.7601928710938 18.0
-59.6751761071664 -59.5052185058594 19.0
-59.6751760861664 -59.7838439941406 20.0
-59.6751760941664 -60.2965087890625 21.0
80.96986919 79.8902740478516 22.0
147.552067054061 143.310745239258 23.0
111.998711180076 114.614204406738 24.0
21.8624391480756 21.8418960571289 25.0
228.521936246282 222.758880615234 26.0
228.521936246282 223.840194702148 27.0
26.7706270442819 26.572301864624 28.0
255.292563296282 253.079330444336 29.0
255.292563296282 251.376190185547 30.0
46.90172438 46.5974349975586 31.0
9.4044095005 8.56643676757812 32.0
21.2946930979522 20.9325866699219 33.0
21.2946930979522 22.147403717041 34.0
21.2946931019233 20.9858531951904 35.0
9.2033128712 9.96604156494141 36.0
7.9497293985 6.93172693252563 37.0
-9.264059778e-09 -0.302216351032257 38.0
-9.264059779e-09 -2.54786014556885 39.0
-11.6972307900832 -11.6977939605713 40.0
-24.5134038461412 -25.968318939209 41.0
5.35222024894801 4.1248083114624 42.0
-32.524666555648 -32.4774208068848 43.0
98.681922190648 98.7723541259766 44.0
98.681922183648 95.1378860473633 45.0
6.9664429612 4.89138698577881 46.0
2.4247044541 2.79537320137024 47.0
178.20764072 165.184661865234 48.0
171.50466119 171.215972900391 49.0
0.011311146836 3.18645834922791 50.0
80.503680696 77.8827133178711 51.0
81.646545013 89.9194869995117 52.0
78.056880433 80.8806762695312 53.0
78.056880433 80.3594436645508 54.0
78.056880433 79.6668243408203 55.0
0 3.39220094680786 56.0
80.969869192 78.6907196044922 57.0
228.52193625 221.251800537109 58.0
255.29256329 249.256927490234 59.0
21.294693102 21.2775440216064 60.0
0 2.10299062728882 61.0
177.964838103064 164.829864501953 0.0
177.964838103064 163.596786499023 1.0
176.62178297 166.596633911133 2.0
171.83271322 172.272171020508 3.0
0 2.72491836547852 4.0
-7.23416762999999e-09 2.54375076293945 5.0
-60.2572654420891 -64.4018936157227 6.0
70.1735207063117 74.9471435546875 7.0
45.3566397863118 51.537036895752 8.0
24.8168809193118 24.6687049865723 9.0
89.4734957553117 90.0189743041992 10.0
76.8167889193117 76.5227813720703 11.0
80.7935708103117 86.5502777099609 12.0
80.7935708103117 81.4552459716797 13.0
80.7935708103117 80.4851226806641 14.0
-1.28034342414e-08 -1.90150594711304 15.0
-1.0607477515e-08 -0.401547849178314 16.0
1.72887792999999e-09 1.3115873336792 17.0
59.2653310534428 59.5427093505859 18.0
-59.2653310259676 -58.5588912963867 19.0
-59.2653310079676 -57.2177734375 20.0
-59.2653309919676 -59.3407249450684 21.0
80.6760002572194 78.4233932495117 22.0
143.174943689443 143.048248291016 23.0
106.524994911803 114.130584716797 24.0
22.1632789158035 21.0429420471191 25.0
223.850943956129 229.358016967773 26.0
223.850943956129 221.86994934082 27.0
27.0783090901289 26.4877643585205 28.0
250.929253046129 253.639266967773 29.0
250.929253046129 253.008331298828 30.0
47.0222912959646 46.581226348877 31.0
9.38093538705721 8.94614219665527 32.0
21.4106692662672 18.2610569000244 33.0
21.4106692662672 17.921012878418 34.0
21.4106693001075 21.6109447479248 35.0
9.06172507171984 9.04576778411865 36.0
7.73760472741984 7.40656137466431 37.0
-7.34448764999999e-09 -1.23897385597229 38.0
-7.34448766000001e-09 -0.269631206989288 39.0
-11.5473937859838 -16.3723964691162 40.0
-24.0146304669382 -25.3186473846436 41.0
5.24447971989312 5.57401943206787 42.0
-31.7928168860931 -32.2737274169922 43.0
97.7957010440931 95.5336990356445 44.0
97.7957009160931 94.3523864746094 45.0
7.1660698978 6.69756746292114 46.0
2.6976312976 2.19551587104797 47.0
177.9648381 164.509033203125 48.0
171.83271322 173.021179199219 49.0
0 -2.40969395637512 50.0
70.173520706 75.1387710571289 51.0
89.473495756 87.6039962768555 52.0
80.79357081 81.1569061279297 53.0
80.79357081 81.7158508300781 54.0
80.79357081 80.4649887084961 55.0
1.7288779325e-09 2.47181510925293 56.0
80.676000258 80.1251983642578 57.0
223.85094395 220.109436035156 58.0
250.92925304 249.617416381836 59.0
21.4106693 26.5850639343262 60.0
0 2.20698952674866 61.0
160.15128511078 158.534591674805 0.0
160.15128511078 163.104507446289 1.0
158.9233224 162.505676269531 2.0
154.13425266 152.234344482422 3.0
0 1.98635363578796 4.0
-7.886678219e-09 2.34495234489441 5.0
-67.5656547827196 -59.4745407104492 6.0
78.0615326040187 77.4067840576172 7.0
48.5306532060187 52.5880432128906 8.0
29.5308793980187 25.168737411499 9.0
83.2750488060187 86.0558776855469 10.0
70.7855183450187 76.5161056518555 11.0
77.4880955670187 83.0972900390625 12.0
77.4880955670187 79.5309295654297 13.0
77.4880955670187 80.9615097045898 14.0
9.3669341014e-10 0.104771852493286 15.0
0 0.000358819961547852 16.0
1.05972339999997e-10 -0.165647387504578 17.0
56.4188029775556 59.0423355102539 18.0
-56.418802967028 -58.991626739502 19.0
-56.418802947028 -61.964973449707 20.0
-56.418802942028 -59.7525177001953 21.0
78.0857561385726 78.3356552124023 22.0
149.512993834556 142.295181274414 23.0
114.485376958195 109.454330444336 24.0
22.5262006701946 21.6337203979492 25.0
227.59874997156 223.518814086914 26.0
227.59874997156 222.031402587891 27.0
26.99822137556 27.6183300018311 28.0
254.59697135156 251.821151733398 29.0
254.59697135156 252.95849609375 30.0
46.882558502054 47.7451438903809 31.0
9.31821424357425 8.59184646606445 32.0
21.6669531961589 19.6386070251465 33.0
21.6669531961589 17.6075859069824 34.0
21.6669532156271 21.5961036682129 35.0
9.46965905610154 9.04811382293701 36.0
8.19056192540154 7.77456188201904 37.0
-8.17120836e-09 0.578768253326416 38.0
-8.17120836e-09 -2.95404100418091 39.0
-12.287234381014 -11.2146425247192 40.0
-24.6855717299925 -26.5904579162598 41.0
5.10770284741556 3.75986647605896 42.0
-32.1821362250156 -32.6481437683105 43.0
95.6833331260156 90.121696472168 44.0
95.6833330550156 91.6902694702148 45.0
7.1300948452 6.27911567687988 46.0
2.5688762927 1.50638008117676 47.0
160.15128512 162.486679077148 48.0
154.13425267 149.000289916992 49.0
0 1.47512316703796 50.0
78.061532604 76.4068069458008 51.0
83.275048806 89.1537933349609 52.0
77.488095567 81.4231109619141 53.0
77.488095567 81.5558853149414 54.0
77.488095567 81.2057037353516 55.0
1.0597233881e-10 1.87951517105103 56.0
78.085756139 79.0843963623047 57.0
227.59874997 218.903121948242 58.0
254.59697135 247.691879272461 59.0
21.666953216 24.4464092254639 60.0
0 2.26404237747192 61.0
160.35063881316 167.894653320312 0.0
160.35063881316 170.422622680664 1.0
159.091024497706 171.263961791992 2.0
153.62638895 147.930557250977 3.0
4.3736320528 12.9852876663208 4.0
4.37363205541732 15.3893356323242 5.0
-66.9858621755173 -48.4729804992676 6.0
79.5663925708106 83.5473022460938 7.0
44.2486092298106 58.347225189209 8.0
35.3177833408106 22.5283107757568 9.0
88.9020717958106 88.4344940185547 10.0
76.1019445608106 79.3463745117188 11.0
85.6025560708106 83.1701049804688 12.0
85.6025560708106 75.6027145385742 13.0
85.6025560708106 77.0297622680664 14.0
3.5917564607e-08 0.888904213905334 15.0
3.5917564608e-08 -0.089235782623291 16.0
7.51975143999999e-08 -0.00610464811325073 17.0
63.1678763288058 56.9941749572754 18.0
-63.1678760782159 -59.3318901062012 19.0
-63.1678760732159 -59.9870834350586 20.0
-63.1678759362159 -54.3889389038086 21.0
83.1373687300299 76.5203704833984 22.0
136.675409366806 136.036178588867 23.0
104.930735791007 112.107666015625 24.0
21.5234025330072 22.1469116210938 25.0
219.81277809632 219.089492797852 26.0
219.81277809632 216.169067382812 27.0
24.82958511732 24.5978698730469 28.0
244.64236321632 253.917907714844 29.0
244.64236321632 254.26139831543 30.0
46.6761279145856 46.297492980957 31.0
9.35352104745502 8.80612564086914 32.0
19.969492793552 20.5294246673584 33.0
19.969492793552 19.7413063049316 34.0
19.969492968314 23.0593090057373 35.0
8.70648974130307 11.0390787124634 36.0
7.47673157460307 4.53157520294189 37.0
6.73553879999999e-09 -7.89872550964355 38.0
6.73553879999999e-09 -2.05979943275452 39.0
-11.583622860108 -20.0420265197754 40.0
-24.1605542933444 -23.3095092773438 41.0
4.68326180110882 9.88284397125244 42.0
-31.1458389500088 -35.9146423339844 43.0
99.1332674240089 94.5591201782227 44.0
99.1332662917558 94.3307189941406 45.0
7.09816358145277 3.68941688537598 46.0
2.76626398315276 2.0198974609375 47.0
160.35063881 170.955520629883 48.0
153.62638895 162.806762695312 49.0
4.3736320554 2.17990398406982 50.0
79.566392571 73.5864639282227 51.0
88.902071796 84.9199981689453 52.0
85.602556071 78.7710189819336 53.0
85.602556071 79.8957061767578 54.0
85.602556071 78.9692687988281 55.0
7.519751443e-08 4.61810159683228 56.0
83.13736873 79.282585144043 57.0
219.81277809 212.412750244141 58.0
244.64236321 251.319396972656 59.0
19.969492968 21.6904354095459 60.0
6.7355388327e-09 3.6285126209259 61.0
187.288344242038 173.49592590332 0.0
187.288344023488 169.016586303711 1.0
185.78171876 162.113174438477 2.0
184.72573848 165.822616577148 3.0
-2.870219778e-10 -2.05552387237549 4.0
-9.9984596e-09 8.90333843231201 5.0
-59.0502755547237 -40.2437057495117 6.0
69.8346210945328 81.0414810180664 7.0
67.3526956205328 34.2678108215332 8.0
2.48192547383307 27.5605373382568 9.0
61.8827258685328 78.4123840332031 10.0
50.6917348355328 75.9734802246094 11.0
69.7953541125328 84.9539566040039 12.0
69.7953541125328 82.1781005859375 13.0
69.7953541125328 78.7456283569336 14.0
5.30089582706036 -0.847216188907623 15.0
5.300895822 -1.2560977935791 16.0
-4.29341008000001e-09 1.56092715263367 17.0
55.6578439071088 56.564624786377 18.0
-55.6578438762992 -54.2142868041992 19.0
-56.3261116162992 -55.504093170166 20.0
-56.3946732152992 -57.3280296325684 21.0
74.5798067426845 76.9763107299805 22.0
131.644285720109 149.606201171875 23.0
98.172969081136 107.938621520996 24.0
20.140286001136 21.7871379852295 25.0
206.224092464076 225.852752685547 26.0
206.224092464076 224.813552856445 27.0
22.5814312330763 27.178674697876 28.0
228.805523704076 250.778274536133 29.0
228.805523704076 243.255172729492 30.0
46.1226244793008 46.6967391967773 31.0
9.49843719611231 7.27699851989746 32.0
18.185133527171 17.0185546875 33.0
18.185133527171 15.3805913925171 34.0
18.1851335690006 11.1762437820435 35.0
9.32217600277328 14.1983394622803 36.0
8.06771151317327 6.19570159912109 37.0
-6.01205212000002e-09 -1.23084020614624 38.0
-6.01205211999997e-09 -6.18098783493042 39.0
-9.80225438944965 -10.5643844604492 40.0
-23.406185656593 -21.0307083129883 41.0
5.80619682907742 6.52995014190674 42.0
-31.5484495642774 -34.4138145446777 43.0
75.1598364312774 85.7690963745117 44.0
73.9231028314474 75.7332153320312 45.0
7.30468345632153 9.32948112487793 46.0
2.54847063530697 2.94901013374329 47.0
187.28834425 174.34765625 48.0
184.72573849 158.262985229492 49.0
0 0.622473537921906 50.0
69.834621095 69.1166076660156 51.0
61.882725868 86.135124206543 52.0
69.795354113 77.444709777832 53.0
69.795354113 80.1006164550781 54.0
69.795354113 80.3380126953125 55.0
0 1.87008285522461 56.0
74.579806742 77.4955368041992 57.0
206.22409247 220.6181640625 58.0
228.8055237 246.978958129883 59.0
18.185133569 18.6738967895508 60.0
0 1.76524174213409 61.0
159.87736064 166.00927734375 0.0
159.87736064 165.713638305664 1.0
158.5706993 163.703948974609 2.0
153.10606411 147.684722900391 3.0
2.2222490814 5.9890308380127 4.0
2.2222490814 4.56553411483765 5.0
-65.437327632 -68.3894348144531 6.0
75.664329287 77.8534622192383 7.0
51.108354085 49.4421463012695 8.0
24.555975192 9.11969757080078 9.0
92.987177682 86.8309860229492 10.0
80.790065495 74.0654678344727 11.0
85.289926471 81.088737487793 12.0
85.289926471 78.2449569702148 13.0
85.289926471 81.8081588745117 14.0
0 -0.973830044269562 15.0
0 -0.5943883061409 16.0
0 -0.440052688121796 17.0
58.943466374 59.3272476196289 18.0
-58.943466384 -57.8566017150879 19.0
-58.943466362 -59.6720733642578 20.0
-58.943466374 -59.0004272460938 21.0
81.480781503 81.8764724731445 22.0
141.85278777 148.70735168457 23.0
107.68301163 100.897888183594 24.0
23.828896794 25.1163291931152 25.0
223.33356928 227.640853881836 26.0
223.33356928 227.137176513672 27.0
25.318084844 28.3211536407471 28.0
248.65165414 253.972015380859 29.0
248.65165414 253.659774780273 30.0
47.025960909 47.2988777160645 31.0
9.3008114574 8.71549701690674 32.0
22.537315119 18.0005264282227 33.0
22.537315119 22.6298713684082 34.0
22.537315119 24.9536209106445 35.0
9.3928173242 18.7687530517578 36.0
8.0031645184 7.56163263320923 37.0
0 -2.91100907325745 38.0
0 -4.46647262573242 39.0
-12.782236307 -12.1249217987061 40.0
-25.912888337 -23.6491546630859 41.0
4.7720772151 8.77756977081299 42.0
-33.197785209 -35.9213714599609 43.0
93.615217577 89.5795822143555 44.0
93.61521759 96.7800903320312 45.0
6.7475659379 7.70829248428345 46.0
1.9236503671 2.82325625419617 47.0
159.87736065 171.518341064453 48.0
153.10606412 156.628875732422 49.0
2.2222490914 -1.91006922721863 50.0
75.664329297 75.0343704223633 51.0
92.987177692 88.4841232299805 52.0
85.289926481 80.3231430053711 53.0
85.289926481 76.8518447875977 54.0
85.289926481 79.4929504394531 55.0
0 3.51923942565918 56.0
81.480781513 77.5570068359375 57.0
223.33356929 222.824584960938 58.0
248.65165415 248.192947387695 59.0
22.537315129 21.9556446075439 60.0
0 1.57640981674194 61.0
157.36548838 170.930877685547 0.0
157.36548838 172.19694519043 1.0
156.1023543 165.949951171875 2.0
151.31328458 159.548950195312 3.0
0 4.67210912704468 4.0
-1.0106839505e-09 2.16285419464111 5.0
-64.604244972 -62.8880653381348 6.0
74.535804141 78.6902160644531 7.0
51.399035256 52.4100685119629 8.0
23.136768884 24.4524040222168 9.0
89.254529497 87.1481323242188 10.0
77.000080906 75.1320343017578 11.0
81.430517812 83.490119934082 12.0
81.430517812 79.643684387207 13.0
81.430517812 81.8576049804688 14.0
0 -0.965415775775909 15.0
0 0.24268639087677 16.0
-5.067987411e-09 -0.464225113391876 17.0
57.867504373 58.9413070678711 18.0
-57.8675043956387 -58.6981735229492 19.0
-57.8675043736387 -61.060791015625 20.0
-57.8675043846387 -60.9772644042969 21.0
79.777552516 79.6585083007812 22.0
147.30026188 143.372604370117 23.0
111.12845189 111.668113708496 24.0
23.331766867 20.6868991851807 25.0
227.077814409722 228.655410766602 26.0
227.077814409722 223.216217041016 27.0
26.9601671657221 24.3390655517578 28.0
254.037981569722 254.596633911133 29.0
254.037981569722 256.463409423828 30.0
46.586943243 47.0950660705566 31.0
9.3048568884 9.49477005004883 32.0
21.910048136 21.3646202087402 33.0
21.910048136 18.4760494232178 34.0
21.910048133 20.8963356018066 35.0
9.7342415406 8.329833984375 36.0
8.4434670463 6.77964496612549 37.0
-8.8053647453e-09 -1.51850342750549 38.0
-8.8053648459e-09 -3.84855270385742 39.0
-12.426981428 -14.0024938583374 40.0
-25.144790067 -27.8206272125244 41.0
5.2038552011 4.52153968811035 42.0
-33.017419753 -31.6740703582764 43.0
94.906558984 94.7040863037109 44.0
94.906558991 94.1482849121094 45.0
6.8363352879 9.84699726104736 46.0
2.401432487 2.71115565299988 47.0
157.36548838 169.014404296875 48.0
151.31328459 147.314788818359 49.0
0 -1.50774383544922 50.0
74.535804142 76.3723678588867 51.0
89.254529498 85.858024597168 52.0
81.430517813 81.814567565918 53.0
81.430517813 82.1610107421875 54.0
81.430517813 79.9546203613281 55.0
0 1.68385219573975 56.0
79.777552523 80.0764846801758 57.0
227.07781441 223.571853637695 58.0
254.03798157 248.814178466797 59.0
21.91004814 21.4324569702148 60.0
0 2.15945339202881 61.0
159.669554822697 156.772811889648 0.0
159.669554822697 158.209609985352 1.0
158.31753785 152.619674682617 2.0
153.52846811 200.741516113281 3.0
0 1.48635315895081 4.0
-5.786220817e-09 4.9621787071228 5.0
-58.3342901777168 -63.9529838562012 6.0
69.4005103350087 77.8396377563477 7.0
42.3143384320087 48.8424949645996 8.0
27.0861719020087 23.5846633911133 9.0
90.9545392160087 88.2309951782227 10.0
78.1503668210087 81.2277526855469 11.0
79.7077952760087 79.4914093017578 12.0
79.7077952760087 80.690544128418 13.0
79.7077952760087 66.816047668457 14.0
1.606322688476e-09 -9.15841770172119 15.0
0 -2.00083494186401 16.0
-7.21941409999994e-10 3.62987232208252 17.0
55.2873157909139 58.0085601806641 18.0
-55.2873157780131 -58.4212799072266 19.0
-55.2873157580131 -58.083797454834 20.0
-55.2873157520131 -57.503475189209 21.0
76.7563081857861 72.3499374389648 22.0
136.961197236914 142.653717041016 23.0
102.428467808642 109.782516479492 24.0
21.0336781116423 21.7296619415283 25.0
213.717505425395 228.498489379883 26.0
213.717505425395 221.618499755859 27.0
28.3193584703947 26.5471248626709 28.0
242.036863895395 250.071823120117 29.0
242.036863895395 249.080764770508 30.0
47.2652214349734 47.1528053283691 31.0
9.4098325756272 6.13962888717651 32.0
21.4689924341503 15.2322454452515 33.0
21.4689924341503 16.9494571685791 34.0
21.4689924557961 21.2580909729004 35.0
9.94532784291457 11.5681495666504 36.0
8.77093395321457 8.32499027252197 37.0
-8.02466460000002e-09 -4.01570415496826 38.0
-8.02466459999999e-09 -0.14772754907608 39.0
-11.5257927435065 -16.6352653503418 40.0
-24.3481317238688 -26.554292678833 41.0
4.8099021797406 2.1361722946167 42.0
-31.6045810603406 -30.2187156677246 43.0
93.3517773503406 87.632194519043 44.0
93.3517772693406 90.4069290161133 45.0
6.4145328529 4.05588960647583 46.0
1.6486533839 0.868272364139557 47.0
159.66955482 167.535095214844 48.0
153.52846812 138.920501708984 49.0
0 0.364760458469391 50.0
69.400510335 77.3041152954102 51.0
90.954539216 84.9003067016602 52.0
79.707795276 80.2201080322266 53.0
79.707795276 81.3272857666016 54.0
79.707795276 72.0225830078125 55.0
0 11.5913343429565 56.0
76.756308186 78.6438751220703 57.0
213.71750542 221.224517822266 58.0
242.03686389 242.078567504883 59.0
21.468992456 20.3095703125 60.0
0 3.16678953170776 61.0
159.506534424871 160.841995239258 0.0
159.506534424871 165.693984985352 1.0
158.25293895 164.408706665039 2.0
153.46386921 155.01985168457 3.0
0 -2.78462028503418 4.0
-6.746147961e-09 4.93320560455322 5.0
-59.0692891178411 -61.7377548217773 6.0
70.5438716744439 74.6734161376953 7.0
45.4467888014439 50.7782096862793 8.0
25.0970828734439 25.0196056365967 9.0
89.2831841514439 88.7387161254883 10.0
76.1754963594439 78.4196395874023 11.0
84.1387935514439 79.9969863891602 12.0
84.1387935514439 76.9131011962891 13.0
84.1387935514439 78.0875778198242 14.0
2.31282605214e-09 -1.17495608329773 15.0
0 0.173320174217224 16.0
1.94412250999999e-09 -0.185572683811188 17.0
60.3271150195148 58.3110008239746 18.0
-60.3271150016658 -58.3121948242188 19.0
-60.3271149816658 -60.1042785644531 20.0
-60.3271149736658 -60.1904067993164 21.0
81.18550360412 79.2737045288086 22.0
144.099779899515 141.748229980469 23.0
110.028541339394 109.412475585938 24.0
23.2510261493936 20.4718742370605 25.0
225.285283499743 229.698059082031 26.0
225.285283499743 224.282333374023 27.0
25.8171617477429 27.1678676605225 28.0
251.102445249743 250.980819702148 29.0
251.102445249743 251.796798706055 30.0
46.9412050840225 47.4172019958496 31.0
9.29098009086743 8.61005973815918 32.0
20.8583886305233 19.0273818969727 33.0
20.8583886305233 13.4690179824829 34.0
20.8583886551186 20.7260417938232 35.0
9.28863951980661 6.78936862945557 36.0
7.97048488220661 7.88682174682617 37.0
-7.67182938999999e-09 1.11590886116028 38.0
-7.67182938999999e-09 -2.93408823013306 39.0
-11.3675668683329 -15.4006910324097 40.0
-23.8319987439715 -24.8283882141113 41.0
5.38134004273657 6.23417091369629 42.0
-31.6832072490366 -33.0877571105957 43.0
98.2327613260366 91.8390197753906 44.0
98.2327611281946 95.1000671386719 45.0
6.9541939016 7.48569297790527 46.0
2.5059499292 1.14463913440704 47.0
159.50653443 162.724700927734 48.0
153.46386922 154.10905456543 49.0
0 -2.63244128227234 50.0
70.543871675 76.0920104980469 51.0
89.283184151 92.0413513183594 52.0
84.138793552 80.9965515136719 53.0
84.138793552 79.6780624389648 54.0
84.138793552 80.1773071289062 55.0
1.9441225139e-09 0.979798257350922 56.0
81.185503604 80.8437728881836 57.0
225.2852835 222.570114135742 58.0
251.10244525 248.323760986328 59.0
20.858388655 23.9139366149902 60.0
0 2.47190237045288 61.0
166.91973447 162.646530151367 0.0
166.91973447 168.166030883789 1.0
165.76036249 165.374298095703 2.0
160.97129277 166.775146484375 3.0
0 4.90537071228027 4.0
0 -0.223446905612946 5.0
-64.193029013 -62.8290634155273 6.0
75.705222813 73.1513366699219 7.0
48.019620597 51.7409172058105 8.0
27.685602207 24.7373886108398 9.0
86.680933934 84.5756759643555 10.0
73.922318244 75.8984451293945 11.0
80.927336143 81.8754043579102 12.0
80.927336143 77.5858535766602 13.0
80.927336143 79.7336273193359 14.0
0 -1.61417746543884 15.0
0 -0.60650247335434 16.0
-1.5096567517e-14 -0.469739258289337 17.0
57.701515205 58.4437217712402 18.0
-57.701515225 -60.0666580200195 19.0
-57.701515203 -58.341625213623 20.0
-57.701515215 -58.674674987793 21.0
78.411513052 77.2468719482422 22.0
139.75595277 141.570526123047 23.0
106.50947314 112.474548339844 24.0
21.814033666 20.0787925720215 25.0
218.16746583 226.344314575195 26.0
218.16746583 226.396087646484 27.0
26.932104411 24.4077606201172 28.0
245.09957026 252.26350402832 29.0
245.09957026 251.818984985352 30.0
46.480835147 46.0074768066406 31.0
9.3271374659 8.74269104003906 32.0
20.709997827 20.44921875 33.0
20.709997827 16.3474578857422 34.0
20.709997827 20.2802639007568 35.0
8.9912851006 6.03111934661865 36.0
7.6720581139 6.10623931884766 37.0
-1.3217527922e-12 -0.955964028835297 38.0
-1.3217522116e-12 -0.503637135028839 39.0
-11.470250745 -16.1994228363037 40.0
-24.966953102 -24.7081718444824 41.0
5.2116120616 3.25950288772583 42.0
-32.783234911 -33.0631790161133 43.0
92.681553037 88.811408996582 44.0
92.68155305 92.3274993896484 45.0
7.3005037972 7.06185913085938 46.0
2.6947730607 2.66597843170166 47.0
166.91973448 167.108779907227 48.0
160.97129278 164.436050415039 49.0
0 0.090697169303894 50.0
75.705222823 76.3485488891602 51.0
86.680933944 90.4780502319336 52.0
80.927336153 81.3497619628906 53.0
80.927336153 80.0339279174805 54.0
80.927336153 80.4364547729492 55.0
0 2.42539262771606 56.0
78.411513062 79.666259765625 57.0
218.16746584 222.249710083008 58.0
245.09957027 247.500839233398 59.0
20.709997837 24.1465034484863 60.0
0 1.48754703998566 61.0
170.20658371 158.878173828125 0.0
170.20658371 164.111068725586 1.0
168.84186141 161.609466552734 2.0
164.05279168 171.039031982422 3.0
0 6.50696992874146 4.0
-1.6796992301e-13 6.32832860946655 5.0
-60.641653497 -54.2512435913086 6.0
71.920691722 79.7441329956055 7.0
54.443475797 51.9478759765625 8.0
17.477215915 21.9105186462402 9.0
87.842062929 89.8479843139648 10.0
74.49942985 78.5485382080078 11.0
80.402923361 81.1614074707031 12.0
80.402923361 77.8133850097656 13.0
80.402923361 76.0155410766602 14.0
0 -1.46089601516724 15.0
0 0.233545124530792 16.0
-3.7605504764e-13 0.000267505645751953 17.0
57.654977333 57.7353820800781 18.0
-57.654977353 -60.0115203857422 19.0
-57.654977331 -60.3126335144043 20.0
-57.654977343 -58.8246421813965 21.0
78.840369075 77.0826644897461 22.0
142.37308358 142.446899414062 23.0
107.30663572 111.930191040039 24.0
22.680733098 20.761157989502 25.0
221.21345266 229.087829589844 26.0
221.21345266 227.878890991211 27.0
27.492024189 24.9391136169434 28.0
248.70547686 255.033096313477 29.0
248.70547686 257.879089355469 30.0
46.803743586 47.2554397583008 31.0
9.4074474231 9.04789447784424 32.0
21.185391722 18.6295375823975 33.0
21.185391722 17.9288730621338 34.0
21.185391722 19.5672473907471 35.0
9.4535673835 8.38052368164062 36.0
8.1569138496 7.66783857345581 37.0
-4.7880624972e-13 -1.96472382545471 38.0
-4.7880551717e-13 -4.55815267562866 39.0
-12.092863796 -14.0839786529541 40.0
-24.254899908 -26.1865730285645 41.0
5.1452346412 7.53444814682007 42.0
-32.121083457 -34.7955741882324 43.0
104.84722134 98.935661315918 44.0
104.84722135 97.4809188842773 45.0
6.719517578 5.64507007598877 46.0
2.4229028689 2.41529393196106 47.0
170.20658372 162.718475341797 48.0
164.05279169 154.811019897461 49.0
0 2.47728967666626 50.0
71.920691732 75.9860305786133 51.0
87.842062939 92.3452835083008 52.0
80.402923371 81.0938568115234 53.0
80.402923371 80.9172439575195 54.0
80.402923371 79.3611907958984 55.0
0 0.0384770035743713 56.0
78.840369085 80.5426712036133 57.0
221.21345267 221.916412353516 58.0
248.70547687 248.380554199219 59.0
21.185391732 22.9592208862305 60.0
0 0.96477860212326 61.0
147.59555202 164.645797729492 0.0
147.59555202 158.001083374023 1.0
146.39025614 157.816375732422 2.0
141.60118641 139.930114746094 3.0
0 7.75796604156494 4.0
0 0.429585039615631 5.0
-64.753654227 -59.91455078125 6.0
76.026143549 80.3137664794922 7.0
46.150106242 45.6958045959473 8.0
29.876037297 24.1751804351807 9.0
84.495648475 96.9319839477539 10.0
72.638321873 78.2974090576172 11.0
79.558471733 87.3150787353516 12.0
79.558471733 86.6168060302734 13.0
79.558471733 76.4477691650391 14.0
0 -0.0214637517929077 15.0
0 -0.23643547296524 16.0
0 0.650009214878082 17.0
60.723538287 60.4936294555664 18.0
-60.723538297 -58.8111991882324 19.0
-60.723538275 -53.197639465332 20.0
-60.723538288 -56.3383140563965 21.0
82.373400961 77.549674987793 22.0
139.61453948 147.057968139648 23.0
106.98216904 112.452606201172 24.0
21.386479709 19.2137184143066 25.0
221.98794045 222.956954956055 26.0
221.98794045 226.307723999023 27.0
28.780443619 25.3258571624756 28.0
250.76838408 248.267578125 29.0
250.76838408 249.997619628906 30.0
47.002454833 49.5958099365234 31.0
9.285912751 9.5930347442627 32.0
21.649862664 18.1959705352783 33.0
21.649862664 23.7815761566162 34.0
21.649862664 22.2283973693848 35.0
9.9459293968 6.6279730796814 36.0
8.7606888039 6.35870599746704 37.0
-2.6040799383e-10 -1.76334714889526 38.0
-2.6040792422e-10 -3.19870805740356 39.0
-11.635736569 -16.9976825714111 40.0
-25.22907137 -26.5920715332031 41.0
4.8900668851 5.66343832015991 42.0
-32.52567893 -33.843620300293 43.0
90.165309755 87.4204483032227 44.0
90.165309768 90.1847076416016 45.0
6.9789841328 4.81195688247681 46.0
2.3836761511 2.24263024330139 47.0
147.59555203 161.683410644531 48.0
141.60118642 147.433410644531 49.0
0 7.52249002456665 50.0
76.026143559 70.6722412109375 51.0
84.495648485 86.4512710571289 52.0
79.558471743 80.1537017822266 53.0
79.558471743 82.3549194335938 54.0
79.558471743 79.8503494262695 55.0
0 1.61147546768188 56.0
82.373400971 80.1528396606445 57.0
221.98794046 223.339050292969 58.0
250.76838408 244.4599609375 59.0
21.649862673 17.5413990020752 60.0
0 1.13901662826538 61.0
168.33347034 163.97477722168 0.0
168.33347034 168.158737182617 1.0
166.98117493 159.204711914062 2.0
162.19210471 156.5029296875 3.0
0 -4.11255598068237 4.0
0 -1.4016227722168 5.0
-60.450536893 -62.1218643188477 6.0
72.098058675 78.9589309692383 7.0
52.0049694 54.6025352478027 8.0
20.093089267 28.0340404510498 9.0
87.36596195 91.0633239746094 10.0
74.007385893 76.5895690917969 11.0
80.448521552 83.6150817871094 12.0
80.448521552 76.9171600341797 13.0
80.448521552 77.9993133544922 14.0
0 -2.24201631546021 15.0
0 -0.648998200893402 16.0
0 -1.49034929275513 17.0
56.888507975 57.4160652160645 18.0
-56.888507986 -58.7110061645508 19.0
-56.888507965 -59.1994514465332 20.0
-56.888507977 -59.6584815979004 21.0
77.795382624 78.7428359985352 22.0
148.58337569 140.64533996582 23.0
114.50596064 106.097091674805 24.0
22.803985899 22.3262920379639 25.0
226.37875833 222.057281494141 26.0
226.37875833 223.905624389648 27.0
26.698097689 24.1361980438232 28.0
253.07685603 257.846282958984 29.0
253.07685603 258.512145996094 30.0
47.007156717 46.9509162902832 31.0
9.4050604318 9.44370555877686 32.0
20.906874641 19.6620101928711 33.0
20.906874641 18.4821739196777 34.0
20.90687464 19.7689437866211 35.0
9.2562446277 8.15321445465088 36.0
8.0189347443 7.52362060546875 37.0
-1.1408579462e-09 -0.26648873090744 38.0
-1.1408578188e-09 -3.71512985229492 39.0
-11.851329836 -13.5215511322021 40.0
-24.936524141 -29.1733074188232 41.0
5.115440827 4.52317476272583 42.0
-32.623255276 -32.5054588317871 43.0
98.260034141 92.7761535644531 44.0
98.260034152 93.91259765625 45.0
6.8768490861 7.74327707290649 46.0
2.6916736038 2.42230105400085 47.0
168.33347035 164.203460693359 48.0
162.19210472 153.296920776367 49.0
0 2.09337568283081 50.0
72.098058684 76.7841033935547 51.0
87.365961958 88.3346328735352 52.0
80.448521561 81.1731948852539 53.0
80.448521561 79.3433609008789 54.0
80.448521561 79.5940399169922 55.0
0 1.24563837051392 56.0
77.795382634 80.5439376831055 57.0
226.37875834 222.448471069336 58.0
253.07685604 249.853607177734 59.0
20.90687465 22.5224685668945 60.0
0 1.15553200244904 61.0
175.05640447 158.04736328125 0.0
175.05640447 162.894073486328 1.0
173.86983802 167.243835449219 2.0
168.65913708 168.704818725586 3.0
0 0.890222012996674 4.0
-4.1281766273e-10 2.32930445671082 5.0
-68.631844265 -66.3098831176758 6.0
79.017161661 75.489143371582 7.0
47.789810844 49.8381233215332 8.0
31.227350809 24.71116065979 9.0
86.753133177 86.4171981811523 10.0
73.521080011 75.5051727294922 11.0
81.326417839 81.1204376220703 12.0
81.326417839 79.945556640625 13.0
81.326417839 78.0863037109375 14.0
0 0.194886147975922 15.0
0 0.441041111946106 16.0
-1.2762600089e-09 0.898068130016327 17.0
58.179979465 59.8065986633301 18.0
-58.179979488 -59.6823425292969 19.0
-58.179979466 -59.6734619140625 20.0
-58.179979478 -60.8730735778809 21.0
79.799760093 77.9290008544922 22.0
146.48581339 141.393615722656 23.0
112.31755034 110.322265625 24.0
22.818628662 21.3721466064453 25.0
226.2855735 223.709777832031 26.0
226.2855735 219.762481689453 27.0
26.16727697 23.9745922088623 28.0
252.45285047 255.15885925293 29.0
252.45285047 257.40869140625 30.0
46.933708058 48.1185150146484 31.0
9.3144364392 8.44257068634033 32.0
21.61978061 18.908634185791 33.0
21.61978061 18.9554576873779 34.0
21.619780609 19.3043079376221 35.0
9.2718822657 7.30234718322754 36.0
8.0060640176 8.38574600219727 37.0
-1.6395844448e-09 0.38869833946228 38.0
-1.6395844447e-09 -1.42251467704773 39.0
-11.645049069 -13.5318040847778 40.0
-24.09040202 -23.3780937194824 41.0
4.9097031261 5.66740608215332 42.0
-31.543361217 -33.8258323669434 43.0
92.689325955 91.2669372558594 44.0
92.689325964 94.9345016479492 45.0
6.5748148242 7.28198480606079 46.0
2.4864486216 2.73059940338135 47.0
175.05640448 163.227355957031 48.0
168.65913709 162.042755126953 49.0
0 -1.3536810874939 50.0
79.017161668 76.3994903564453 51.0
86.753133184 90.2097473144531 52.0
81.326417847 81.2863235473633 53.0
81.326417847 81.3412857055664 54.0
81.326417847 79.7621231079102 55.0
0 -0.53716379404068 56.0
79.799760103 80.1216735839844 57.0
226.2855735 219.242980957031 58.0
252.45285048 250.191162109375 59.0
21.619780618 20.7179851531982 60.0
0 1.77052474021912 61.0
185.23231047 173.081634521484 0.0
185.23231047 170.448989868164 1.0
183.94323359 167.960510253906 2.0
179.15416387 166.392913818359 3.0
-1.2089405523e-11 0.507571518421173 4.0
-6.909495409e-10 8.51286220550537 5.0
-64.931742152 -46.8035316467285 6.0
74.996920376 81.4064712524414 7.0
50.073468428 56.8715019226074 8.0
24.92345194 26.1057147979736 9.0
94.258645082 94.2136077880859 10.0
81.933673716 72.0984344482422 11.0
84.614510425 88.234260559082 12.0
84.614510425 80.1509399414062 13.0
84.614510425 79.6584701538086 14.0
0 -4.38411283493042 15.0
0 -1.70706295967102 16.0
-1.2260981433e-09 -2.55496740341187 17.0
54.744439125 53.0308036804199 18.0
-54.744439147 -59.0440254211426 19.0
-54.744439125 -60.2427406311035 20.0
-54.744439137 -55.4530296325684 21.0
75.541516636 78.3162536621094 22.0
147.1617598 134.707534790039 23.0
113.3477 113.360260009766 24.0
23.261845564 17.3957538604736 25.0
222.70327645 227.284790039062 26.0
222.70327645 225.055419921875 27.0
28.405561077 26.361047744751 28.0
251.10883754 260.920501708984 29.0
251.10883754 251.12419128418 30.0
46.319978068 46.8136024475098 31.0
9.2251238914 8.75328826904297 32.0
20.797077493 18.0978450775146 33.0
20.797077493 23.4176120758057 34.0
20.797077492 24.8525199890137 35.0
8.8939995559 11.1411352157593 36.0
7.5988477859 7.22424650192261 37.0
-1.9797391e-09 -2.48810052871704 38.0
-1.9797390784e-09 -8.10943508148193 39.0
-12.153268295 -11.7657699584961 40.0
-24.264750008 -31.8395290374756 41.0
4.6941931407 1.64945387840271 42.0
-31.277076302 -29.650411605835 43.0
105.40686621 97.3973083496094 44.0
105.40686623 89.2252731323242 45.0
6.9800475752 7.3021936416626 46.0
2.7799446129 1.79686880111694 47.0
185.23231048 170.293365478516 48.0
179.15416388 174.168090820312 49.0
0 4.03368616104126 50.0
74.996920384 74.9470672607422 51.0
94.25864509 87.1755981445312 52.0
84.614510433 81.8534164428711 53.0
84.614510433 81.3036499023438 54.0
84.614510433 79.3001708984375 55.0
0 -0.267118155956268 56.0
75.541516645 78.8032455444336 57.0
222.70327646 220.205627441406 58.0
251.10883754 251.374908447266 59.0
20.797077501 20.7562503814697 60.0
0 2.84490990638733 61.0
137.17687742 143.006240844727 0.0
137.17687742 143.040786743164 1.0
135.86355256 143.466873168945 2.0
131.0744823 135.639465332031 3.0
0 -1.97539496421814 4.0
0 0.341026425361633 5.0
-59.42861149 -59.2168769836426 6.0
69.705106679 70.1770782470703 7.0
65.43029471 62.4528312683105 8.0
4.2748119615 8.48783111572266 9.0
78.669161727 80.2065200805664 10.0
65.229460649 71.2585144042969 11.0
77.500548395 80.8047180175781 12.0
77.500548395 77.9000701904297 13.0
77.500548395 76.9198608398438 14.0
0 -0.64198762178421 15.0
0 0.28728449344635 16.0
0 -3.15485000610352 17.0
59.754945382 58.5015258789062 18.0
-59.754945394 -59.4383087158203 19.0
-59.754945372 -60.0649642944336 20.0
-59.754945384 -61.8801422119141 21.0
80.656571382 77.8813858032227 22.0
140.87903669 140.394561767578 23.0
106.41719675 107.357681274414 24.0
22.90053999 20.4083709716797 25.0
221.53560808 223.594223022461 26.0
221.53560808 221.987747192383 27.0
27.282645659 25.3551406860352 28.0
248.81825375 253.574676513672 29.0
248.81825375 252.408157348633 30.0
47.033013387 47.2299919128418 31.0
9.3131031344 9.18030452728271 32.0
20.901625993 20.4819431304932 33.0
20.901625993 21.1129970550537 34.0
20.901625993 20.8195400238037 35.0
9.3083381265 8.28715133666992 36.0
8.044767894 8.38281917572021 37.0
-2.5262790395e-09 -0.804296553134918 38.0
-2.5262785966e-09 0.113371789455414 39.0
-12.387792857 -14.5510091781616 40.0
-23.81592281 -27.5234241485596 41.0
4.9278776077 3.5162615776062 42.0
-31.108913619 -34.7394485473633 43.0
92.052766479 91.3064651489258 44.0
92.05276649 90.2343292236328 45.0
6.7273095346 3.00315618515015 46.0
2.6568330431 1.83220052719116 47.0
137.17687743 136.687423706055 48.0
131.07448231 127.337051391602 49.0
0 3.57309532165527 50.0
69.705106686 70.9872741699219 51.0
78.669161734 84.2529830932617 52.0
77.500548402 81.0479965209961 53.0
77.500548402 80.9212951660156 54.0
77.500548402 79.4541549682617 55.0
0 -0.0822644829750061 56.0
80.656571391 80.3950500488281 57.0
221.53560809 218.26057434082 58.0
248.81825375 246.469757080078 59.0
20.901626002 19.9398994445801 60.0
0 2.89454770088196 61.0
175.77805803 165.798156738281 0.0
175.77805803 170.663879394531 1.0
174.45269644 162.692047119141 2.0
169.66362672 167.22248840332 3.0
0 -3.92404699325562 4.0
-3.8923948093e-13 4.07518672943115 5.0
-58.000557144 -58.65478515625 6.0
69.729463776 75.6072006225586 7.0
55.447244167 52.7974700927734 8.0
14.282219599 25.0312442779541 9.0
90.564740323 88.9465026855469 10.0
77.301426271 74.8580093383789 11.0
84.067740057 83.7080459594727 12.0
84.067740057 80.8328018188477 13.0
84.067740057 80.1614456176758 14.0
0 -1.3222930431366 15.0
0 -0.344551622867584 16.0
-7.2299182576e-13 0.501116693019867 17.0
58.049486605 59.7023010253906 18.0
-58.049486625 -59.2217788696289 19.0
-58.049486603 -60.2557907104492 20.0
-58.049486616 -62.7721710205078 21.0
79.089975492 79.1132431030273 22.0
138.49721424 140.447402954102 23.0
105.16517765 109.268852233887 24.0
21.59899047 21.8382625579834 25.0
217.58718974 223.894180297852 26.0
217.58718974 226.037353515625 27.0
25.770421133 27.6371021270752 28.0
243.35761088 256.538940429688 29.0
243.35761088 259.378143310547 30.0
46.794658636 45.3545455932617 31.0
9.3279929039 9.155104637146 32.0
21.040488867 20.3900718688965 33.0
21.040488867 14.7594356536865 34.0
21.040488867 20.235502243042 35.0
9.3933147598 8.45570755004883 36.0
8.1927453955 6.68218469619751 37.0
-1.1623616491e-12 0.93789130449295 38.0
-1.1623616491e-12 -1.89944624900818 39.0
-12.106718654 -12.8285684585571 40.0
-25.143037703 -22.9966411590576 41.0
5.146188256 6.58392333984375 42.0
-32.964689678 -35.0429611206055 43.0
93.347844889 91.43017578125 44.0
93.347844903 93.3094863891602 45.0
6.6339627234 9.41577529907227 46.0
2.3965423574 2.65504670143127 47.0
175.77805804 167.175750732422 48.0
169.66362673 167.820404052734 49.0
0 0.994617342948914 50.0
69.729463786 76.3901290893555 51.0
90.564740333 87.5362396240234 52.0
84.067740067 81.5448150634766 53.0
84.067740067 82.0349273681641 54.0
84.067740067 80.1372985839844 55.0
0 1.44336652755737 56.0
79.089975502 80.3284530639648 57.0
217.58718975 221.905624389648 58.0
243.35761089 250.509979248047 59.0
21.040488877 23.9585723876953 60.0
0 1.12809407711029 61.0
137.17687742 143.006546020508 0.0
137.17687742 143.028930664062 1.0
135.86355256 143.550674438477 2.0
131.0744823 135.794906616211 3.0
0 -2.14743828773499 4.0
0 0.342255353927612 5.0
-59.42861149 -59.2386283874512 6.0
69.705106679 70.1773300170898 7.0
65.43029471 62.4528312683105 8.0
4.2748119615 8.48783111572266 9.0
78.669161727 80.2065200805664 10.0
65.229460649 71.2585144042969 11.0
77.500548395 80.8047180175781 12.0
77.500548395 77.9000701904297 13.0
77.500548395 76.9198608398438 14.0
0 -0.64198762178421 15.0
0 0.28728449344635 16.0
0 -3.15485000610352 17.0
59.754945382 58.5015258789062 18.0
-59.754945394 -59.4383087158203 19.0
-59.754945372 -60.0649642944336 20.0
-59.754945384 -61.8801422119141 21.0
80.656571382 77.8813858032227 22.0
140.87903669 140.394561767578 23.0
106.41719675 107.357681274414 24.0
22.90053999 20.4083709716797 25.0
221.53560808 223.594223022461 26.0
221.53560808 221.987747192383 27.0
27.282645659 25.3551406860352 28.0
248.81825375 253.574676513672 29.0
248.81825375 252.408157348633 30.0
47.033013387 47.2299919128418 31.0
9.3131031344 9.18030452728271 32.0
20.901625993 20.4819431304932 33.0
20.901625993 21.1129970550537 34.0
20.901625993 20.8195400238037 35.0
9.3083381265 8.28715133666992 36.0
8.044767894 8.38281917572021 37.0
-2.5262790395e-09 -0.804296553134918 38.0
-2.5262785966e-09 0.113371789455414 39.0
-12.387792857 -14.5510091781616 40.0
-23.81592281 -27.5234241485596 41.0
4.9278776077 3.5162615776062 42.0
-31.108913619 -34.7394485473633 43.0
92.052766479 91.3064651489258 44.0
92.05276649 90.2343292236328 45.0
6.7273095346 3.00315618515015 46.0
2.6568330431 1.83220052719116 47.0
137.17687743 136.680679321289 48.0
131.07448231 127.58861541748 49.0
0 3.53733205795288 50.0
69.705106686 70.9860610961914 51.0
78.669161734 84.2529830932617 52.0
77.500548402 81.0479965209961 53.0
77.500548402 80.9212951660156 54.0
77.500548402 79.4541549682617 55.0
0 -0.0822644829750061 56.0
80.656571391 80.3950500488281 57.0
221.53560809 218.26057434082 58.0
248.81825375 246.469757080078 59.0
20.901626002 19.9398994445801 60.0
0 2.89454770088196 61.0
152.202548229204 157.322189331055 0.0
152.202548229204 164.971481323242 1.0
150.99109542 164.711318969727 2.0
146.20202566 151.026153564453 3.0
0 1.26391172409058 4.0
-7.53088371000001e-09 -0.803841531276703 5.0
-62.1573908402249 -61.9489936828613 6.0
72.9440734052873 72.8865509033203 7.0
46.5415529112873 52.2627944946289 8.0
26.4025204932873 19.0091743469238 9.0
88.8488293402873 86.2480926513672 10.0
75.9093493362873 74.6933212280273 11.0
81.5446026952873 85.0018539428711 12.0
81.5446026952873 81.4387893676758 13.0
81.5446026952873 81.7567443847656 14.0
3.7192044836e-09 -0.95390123128891 15.0
1.1032933818e-09 -0.267994701862335 16.0
5.53019511999997e-09 3.51214385032654 17.0
56.9125642503385 59.1101608276367 18.0
-56.912564215931 -59.7802848815918 19.0
-56.912564197931 -60.6737442016602 20.0
-56.912564177931 -61.6849098205566 21.0
78.7357316607263 79.2635955810547 22.0
143.198518706339 141.775695800781 23.0
107.867294635423 109.505264282227 24.0
23.1717073214232 22.4024963378906 25.0
221.934250368408 229.750106811523 26.0
221.934250368408 224.472885131836 27.0
26.8582280894085 25.9367046356201 28.0
248.792478458408 254.338592529297 29.0
248.792478458408 255.061828613281 30.0
47.0090673619672 45.4946746826172 31.0
9.37531978094653 8.91460418701172 32.0
21.8231674826749 21.540901184082 33.0
21.8231674826749 19.0893440246582 34.0
21.8231675198232 20.4136791229248 35.0
9.61824148415515 7.82994031906128 36.0
8.30270502755515 7.87763595581055 37.0
-6.48639794000001e-09 1.58322811126709 38.0
-6.48639794000001e-09 -2.12903046607971 39.0
-11.9894986449655 -11.1676397323608 40.0
-25.2428782237224 -24.7264995574951 41.0
5.05417011467277 4.72118091583252 42.0
-32.8960729120728 -33.7942428588867 43.0
96.4844857120728 93.1328887939453 44.0
96.4844855660728 93.153434753418 45.0
7.04105604146356 6.34788608551025 46.0
2.45800433126356 1.64446806907654 47.0
152.20254823 161.616958618164 48.0
146.20202567 149.350082397461 49.0
0 -0.502545297145844 50.0
72.944073405 75.8470306396484 51.0
88.848829341 87.7344512939453 52.0
81.544602696 81.6076049804688 53.0
81.544602696 81.570915222168 54.0
81.544602696 80.1489715576172 55.0
5.5301951236e-09 3.13362646102905 56.0
78.735731661 78.9906845092773 57.0
221.93425037 224.149429321289 58.0
248.79247845 249.284881591797 59.0
21.82316752 16.0651054382324 60.0
0 3.05733728408813 61.0
172.355090980436 154.491729736328 0.0
172.355090949466 165.270889282227 1.0
170.82001198 170.870452880859 2.0
167.24577199 200.642349243164 3.0
-6.7856787405e-10 111.297546386719 4.0
-9.998601357e-09 11.1175413131714 5.0
-87.1411577614062 -26.3115520477295 6.0
94.4993196104218 117.315338134766 7.0
74.0841731704218 57.1876258850098 8.0
20.4151464394218 48.3380126953125 9.0
64.9550990484218 93.6061553955078 10.0
54.3575233934218 53.3722496032715 11.0
74.6123992114218 80.7476577758789 12.0
74.6123992114218 76.6094207763672 13.0
74.6123992114218 54.6823348999023 14.0
0 5.46304178237915 15.0
0 5.69322776794434 16.0
-2.78330012e-09 9.04144382476807 17.0
70.5612062795934 62.6850090026855 18.0
-70.5612062801327 -61.1007308959961 19.0
-71.2129985461327 -50.7910766601562 20.0
-71.2806356701327 -53.1272583007812 21.0
87.8161691704654 77.9925994873047 22.0
138.402740412594 146.763916015625 23.0
109.331288093242 117.089317321777 24.0
18.7215428012421 21.2151737213135 25.0
226.218909580871 228.880294799805 26.0
226.218909580871 225.20735168457 27.0
21.6695657058713 17.3275852203369 28.0
247.888475290871 231.142425537109 29.0
247.888475290871 248.616668701172 30.0
47.0768019380258 44.2265434265137 31.0
9.48791469825329 3.73199081420898 32.0
16.5355335002187 8.91613674163818 33.0
16.5355335002187 23.2562580108643 34.0
16.5355334915061 20.1860370635986 35.0
7.34880285522712 7.83675241470337 36.0
5.97471621912636 6.75076055526733 37.0
-8.73526372600001e-09 -5.82499742507935 38.0
-8.735263725e-09 -4.04763793945312 39.0
-11.4255071585664 -1.17507123947144 40.0
-25.4132546591297 -27.6501083374023 41.0
4.54713599135151 0.444972932338715 42.0
-32.7915121798515 -33.8603744506836 43.0
101.118414492851 71.5796966552734 44.0
99.8816846818515 104.63516998291 45.0
6.5869072657 -1.64678645133972 46.0
2.6626125083 4.84823131561279 47.0
172.35509098 169.341079711914 48.0
167.245772 160.495315551758 49.0
0 42.9410438537598 50.0
94.49931961 74.2149810791016 51.0
64.955099048 72.200080871582 52.0
74.612399212 65.98095703125 53.0
74.612399212 77.9971389770508 54.0
74.612399212 82.692268371582 55.0
0 13.0352535247803 56.0
87.816169171 60.7117118835449 57.0
226.21890958 222.701217651367 58.0
247.88847529 246.038925170898 59.0
16.535533511 23.7425174713135 60.0
0 9.9270601272583 61.0
157.65644031 162.174331665039 0.0
157.65644031 166.539016723633 1.0
156.39682282 162.734405517578 2.0
151.6077531 155.320343017578 3.0
0 -6.04369640350342 4.0
0 8.13252735137939 5.0
-69.458886709 -61.3907661437988 6.0
81.552202036 71.298583984375 7.0
54.066281223 52.3880386352539 8.0
27.485920803 20.8511390686035 9.0
89.737972437 90.1127014160156 10.0
76.740560592 77.5459594726562 11.0
84.217148036 82.6173934936523 12.0
84.217148036 81.777946472168 13.0
84.217148036 82.3778610229492 14.0
0 -0.00140321254730225 15.0
0 -0.0232996344566345 16.0
0 0.850755751132965 17.0
58.126782676 58.2879066467285 18.0
-58.126782686 -59.4120788574219 19.0
-58.126782664 -59.4753761291504 20.0
-58.126782676 -62.13232421875 21.0
80.083504625 79.1843032836914 22.0
142.34171634 139.875381469727 23.0
107.47551412 111.708885192871 24.0
21.727588103 20.3942070007324 25.0
222.42522098 224.759689331055 26.0
222.42522098 220.163665771484 27.0
26.553868223 27.0234813690186 28.0
248.97908921 254.043685913086 29.0
248.97908921 254.650268554688 30.0
46.479927875 47.8236236572266 31.0
9.3618932703 8.95964431762695 32.0
21.95672194 20.6577892303467 33.0
21.95672194 17.3291893005371 34.0
21.956721954 20.6631240844727 35.0
9.4567433989 9.03712177276611 36.0
8.2381993183 8.19704532623291 37.0
-1.4920506262e-08 -1.36918997764587 38.0
-1.4921513217e-08 -3.44121599197388 39.0
-11.598314028 -16.9725723266602 40.0
-24.80657705 -27.9332370758057 41.0
5.2214173278 5.30205249786377 42.0
-32.48137828 -33.2649307250977 43.0
94.0566714 89.9106369018555 44.0
94.056671412 92.9812622070312 45.0
6.8309627456 6.72204399108887 46.0
2.4877884008 2.39770698547363 47.0
157.65644032 167.137603759766 48.0
151.60775311 157.791793823242 49.0
0 -2.55766296386719 50.0
81.552202045 77.4887390136719 51.0
89.737972446 90.6845397949219 52.0
84.217148045 80.5391082763672 53.0
84.217148045 79.4933776855469 54.0
84.217148045 79.9844284057617 55.0
2.3524907173e-10 -0.0422603487968445 56.0
80.083504635 79.9156494140625 57.0
222.42522098 217.496826171875 58.0
248.97908922 248.03825378418 59.0
21.956721964 21.7224063873291 60.0
0 0.433449566364288 61.0
150.29574753 157.828094482422 0.0
150.29574753 160.309814453125 1.0
148.97077664 159.970336914062 2.0
144.18170692 144.969848632812 3.0
0 -1.9567084312439 4.0
0 2.38657546043396 5.0
-61.453931685 -58.354793548584 6.0
72.164945027 80.8728637695312 7.0
46.783760075 56.6258506774902 8.0
25.381184942 27.2121410369873 9.0
86.720823861 88.4160079956055 10.0
73.885091285 77.0278549194336 11.0
79.921301763 83.9798278808594 12.0
79.921301763 80.4702224731445 13.0
79.921301763 77.7390213012695 14.0
0 -2.32883381843567 15.0
0 -0.705342471599579 16.0
0 -2.21999216079712 17.0
59.272806486 59.4338340759277 18.0
-59.272806496 -58.7741966247559 19.0
-59.272806474 -60.8734169006348 20.0
-59.272806486 -62.0842323303223 21.0
81.049502375 79.2354736328125 22.0
145.64284032 142.156188964844 23.0
110.75546897 107.378372192383 24.0
21.852686223 21.714916229248 25.0
226.69234271 226.128646850586 26.0
226.69234271 225.535491943359 27.0
25.540558355 26.4161434173584 28.0
252.23290107 251.39501953125 29.0
252.23290107 249.915390014648 30.0
46.62822853 44.3883666992188 31.0
9.4066770434 8.21288585662842 32.0
21.776695879 20.2067070007324 33.0
21.776695879 19.2272357940674 34.0
21.776695879 21.4385738372803 35.0
9.612299931 7.5541877746582 36.0
8.3406354029 7.81544160842896 37.0
0 -0.211334764957428 38.0
0 -5.97356224060059 39.0
-11.910775681 -12.4081106185913 40.0
-24.646047697 -27.7698726654053 41.0
5.3167796666 3.33823108673096 42.0
-32.567337667 -31.7139682769775 43.0
95.770986605 95.1708221435547 44.0
95.770986618 94.5535125732422 45.0
6.7218992606 5.49050426483154 46.0
2.6918377239 2.3685610294342 47.0
150.29574754 163.129943847656 48.0
144.18170693 145.618225097656 49.0
0 0.832417547702789 50.0
72.164945037 74.8575973510742 51.0
86.720823871 91.250129699707 52.0
79.921301773 80.9856033325195 53.0
79.921301773 80.4847717285156 54.0
79.921301773 79.7217102050781 55.0
0 0.53804737329483 56.0
81.049502385 79.8111877441406 57.0
226.69234272 223.118606567383 58.0
252.23290108 248.665481567383 59.0
21.776695889 17.0536575317383 60.0
0 0.990958750247955 61.0
179.203453472687 156.849548339844 0.0
179.203453472687 162.928024291992 1.0
178.03445721 163.097991943359 2.0
172.8683367 159.225952148438 3.0
0 2.20367693901062 4.0
-6.245505111e-09 -0.786801159381866 5.0
-73.2460716744736 -64.4604339599609 6.0
84.9034963326577 81.9991912841797 7.0
53.3481898566577 51.2916488647461 8.0
31.5553064756577 28.9774799346924 9.0
98.0963289366577 86.2174987792969 10.0
84.8078374576577 75.7922668457031 11.0
82.7521103596577 85.2691040039062 12.0
82.7521103596577 81.1116180419922 13.0
82.7521103596577 78.4533996582031 14.0
0 1.09654688835144 15.0
0 -0.833416759967804 16.0
-5.340848837e-09 -2.99944448471069 17.0
56.2069001211769 60.1412353515625 18.0
-56.2069001279866 -58.3474540710449 19.0
-56.2069001069866 -60.1050796508789 20.0
-56.2069001129866 -55.9419136047363 21.0
76.2401764272048 77.6256332397461 22.0
149.716225914177 131.061096191406 23.0
114.486518247721 113.026634216309 24.0
22.1071740707212 24.7782745361328 25.0
225.956402335374 224.00666809082 26.0
225.956402335374 224.770462036133 27.0
29.0842634733743 25.5518321990967 28.0
255.040665815374 255.908355712891 29.0
255.040665815374 251.235794067383 30.0
46.987305010211 45.1370086669922 31.0
9.3817391489 9.00999164581299 32.0
20.0332763144209 20.012716293335 33.0
20.0332763144209 14.4477376937866 34.0
20.0332763229495 19.3942356109619 35.0
10.305600937 6.10015106201172 36.0
8.9324250706 7.58011054992676 37.0
-9.252993654e-09 -2.5513482093811 38.0
-9.252993653e-09 -2.24348950386047 39.0
-11.8359959974933 -13.5330247879028 40.0
-24.7638933049277 -29.2619037628174 41.0
5.42883465614811 6.4380989074707 42.0
-32.8216220835481 -31.4959926605225 43.0
93.5548889625481 94.4348449707031 44.0
93.5548889425481 93.4975662231445 45.0
6.4499842098 5.68216514587402 46.0
2.2766408282 2.06433629989624 47.0
179.20345347 159.863983154297 48.0
172.86833671 173.666458129883 49.0
0 3.76059150695801 50.0
84.903496333 77.5471343994141 51.0
98.096328936 88.5695495605469 52.0
82.752110359 79.4581909179688 53.0
82.752110359 81.3844375610352 54.0
82.752110359 78.8069076538086 55.0
0 -0.48759251832962 56.0
76.240176428 79.1658248901367 57.0
225.95640234 219.110946655273 58.0
255.04066581 247.554107666016 59.0
20.033276323 29.2174987792969 60.0
0 1.69464445114136 61.0
175.83446047 159.331268310547 0.0
175.83446047 167.472793579102 1.0
174.60824964 169.244247436523 2.0
169.81917992 179.026992797852 3.0
0 2.73298025131226 4.0
-8.9959772292e-10 5.3869948387146 5.0
-64.615329602 -61.8555946350098 6.0
75.157270719 75.5880432128906 7.0
45.082002224 52.7280693054199 8.0
30.075268494 24.0262718200684 9.0
85.009278218 87.2977066040039 10.0
71.779518342 76.2696914672852 11.0
79.189524017 80.1190643310547 12.0
79.189524017 77.54296875 13.0
79.189524017 79.2548675537109 14.0
0 -1.33206820487976 15.0
0 -0.595543563365936 16.0
-4.9553559936e-09 -0.425393879413605 17.0
55.083619948 58.3408889770508 18.0
-55.0836199718431 -59.1762161254883 19.0
-55.0836199498431 -58.496654510498 20.0
-55.0836199608431 -58.4256477355957 21.0
76.727615254 79.4428634643555 22.0
143.62822175 141.994232177734 23.0
109.03599735 109.05184173584 24.0
22.755557567 21.4528102874756 25.0
220.355837019879 226.97216796875 26.0
220.355837019879 224.638900756836 27.0
26.8124841588793 24.7122249603271 28.0
247.168321169879 258.351257324219 29.0
247.168321169879 254.782272338867 30.0
46.807869018 46.1999740600586 31.0
9.3870338421 8.8903226852417 32.0
21.643995298 19.1523151397705 33.0
21.643995298 18.4990463256836 34.0
21.643995295 19.6684513092041 35.0
9.9283574993 10.518575668335 36.0
8.6414679427 7.5203857421875 37.0
-8.5543335034e-09 -0.646070539951324 38.0
-8.5543588766e-09 -3.28870582580566 39.0
-11.706428615 -12.7851705551147 40.0
-25.139974939 -26.3869647979736 41.0
5.3868171088 4.20161914825439 42.0
-33.170334999 -33.5060081481934 43.0
96.442541164 92.6611938476562 44.0
96.442541164 95.3925628662109 45.0
6.9371028106 6.44675827026367 46.0
2.6780893725 2.48320269584656 47.0
175.83446048 161.528457641602 48.0
169.81917993 164.5703125 49.0
0 3.64193224906921 50.0
75.157270721 77.483283996582 51.0
85.00927822 90.3457946777344 52.0
79.189524019 81.4577407836914 53.0
79.189524019 80.6318130493164 54.0
79.189524019 80.0802459716797 55.0
0 0.0996492505073547 56.0
76.727615262 79.8878860473633 57.0
220.35583702 223.386184692383 58.0
247.16832117 249.768112182617 59.0
21.643995302 23.6205654144287 60.0
0 1.07071089744568 61.0
156.83234799 158.746398925781 0.0
156.83234799 157.190841674805 1.0
155.53998088 158.539306640625 2.0
150.75091116 148.806823730469 3.0
0 -0.550102531909943 4.0
0 2.02057003974915 5.0
-70.351804676 -65.2748794555664 6.0
80.750074137 74.3665542602539 7.0
46.531024486 50.0589752197266 8.0
34.219049641 25.5131397247314 9.0
87.446462721 86.4352798461914 10.0
74.719546131 75.1993789672852 11.0
82.177551722 78.8154754638672 12.0
82.177551722 78.9507675170898 13.0
82.177551722 80.4209976196289 14.0
0 -3.48472356796265 15.0
0 -0.909306943416595 16.0
0 1.2164523601532 17.0
60.892314064 57.8691749572754 18.0
-60.892314074 -58.298828125 19.0
-60.892314052 -59.2076606750488 20.0
-60.892314064 -60.955394744873 21.0
81.96180066 76.8051300048828 22.0
143.16750759 141.247894287109 23.0
106.719705 110.113632202148 24.0
21.656633986 22.4772338867188 25.0
225.12930826 223.70002746582 26.0
225.12930826 225.729095458984 27.0
27.378944891 24.777551651001 28.0
252.50825316 255.221878051758 29.0
252.50825316 258.220306396484 30.0
46.715220681 46.1986274719238 31.0
9.3479103297 9.122239112854 32.0
21.069486586 20.1278533935547 33.0
21.069486586 18.8489627838135 34.0
21.069486586 21.2967205047607 35.0
9.1751953504 7.52423572540283 36.0
7.9225374016 6.81124401092529 37.0
-2.3021364033e-10 0.423498809337616 38.0
-2.3021413624e-10 -0.554073631763458 39.0
-11.468377867 -14.9679346084595 40.0
-24.199109633 -25.4148139953613 41.0
4.8824960524 4.27468967437744 42.0
-31.656519129 -33.375072479248 43.0
93.253690146 91.2723999023438 44.0
93.253690159 95.2701263427734 45.0
6.8365455229 7.46168041229248 46.0
2.5355427436 2.64920401573181 47.0
156.832348 163.685516357422 48.0
150.75091117 148.024139404297 49.0
0 -2.12741827964783 50.0
80.750074147 77.5399322509766 51.0
87.44646273 89.2044067382812 52.0
82.177551732 81.0141830444336 53.0
82.177551732 79.2189483642578 54.0
82.177551732 80.3916015625 55.0
0 2.03908276557922 56.0
81.96180067 80.5405349731445 57.0
225.12930827 220.066268920898 58.0
252.50825317 251.363693237305 59.0
21.069486596 26.268684387207 60.0
0 1.73217487335205 61.0
172.188551194307 165.528060913086 0.0
172.188551194307 166.413177490234 1.0
170.754562744856 158.973693847656 2.0
165.96549285 163.719543457031 3.0
1.5423518217e-09 5.11746263504028 4.0
1.54235182e-09 2.60798501968384 5.0
-53.6734697514663 -61.2328987121582 6.0
62.7142349561321 81.9588470458984 7.0
57.9545874841321 59.7164421081543 8.0
4.75964747213196 41.2237548828125 9.0
91.6979590331321 88.8060455322266 10.0
79.5961755931321 74.4312057495117 11.0
85.4224415601321 75.3353271484375 12.0
85.4224415601321 77.1122131347656 13.0
85.4224415601321 80.3796539306641 14.0
1.9262465079e-08 -1.37970280647278 15.0
1.9262465081e-08 -0.501269042491913 16.0
8.67058163000001e-08 -0.483811914920807 17.0
60.4769839701015 57.3510131835938 18.0
-60.4769839361981 -60.5245399475098 19.0
-60.4769839341981 -59.1617965698242 20.0
-60.4769837781981 -56.8208160400391 21.0
81.9238693380898 78.6046981811523 22.0
137.424649442254 131.302001953125 23.0
104.799675327817 118.212692260742 24.0
21.2731341928173 20.6229267120361 25.0
219.348518778614 230.859771728516 26.0
219.348518778614 216.776916503906 27.0
30.054314069614 30.6156597137451 28.0
249.402832848614 248.972351074219 29.0
249.402832848614 253.497406005859 30.0
46.6715722957763 44.5382652282715 31.0
9.23058475845532 9.22140407562256 32.0
21.4468855603989 20.1591892242432 33.0
21.4468855603989 15.893102645874 34.0
21.4468857571835 22.734411239624 35.0
8.57446183504062 7.82262134552002 36.0
7.36494536394061 7.55616569519043 37.0
8.90893710000002e-09 -7.00286340713501 38.0
8.90893710000002e-09 -4.3419885635376 39.0
-12.3819438645991 -13.9982595443726 40.0
-24.4265337158468 -27.1775741577148 41.0
4.97758282791 4.04443979263306 42.0
-31.7801274441101 -30.4395751953125 43.0
96.7576486161198 82.5388870239258 44.0
96.7576490301101 96.7024993896484 45.0
7.12333720750651 3.37304949760437 46.0
2.46072288042122 1.44243597984314 47.0
172.18855119 166.6962890625 48.0
165.96549285 166.765533447266 49.0
1.5423518218e-09 -5.80873584747314 50.0
62.714234956 73.1149749755859 51.0
91.697959033 91.7118301391602 52.0
85.42244156 79.3073348999023 53.0
85.42244156 75.8380584716797 54.0
85.42244156 80.2763061523438 55.0
8.6705816308e-08 1.91650915145874 56.0
81.923869338 78.757194519043 57.0
219.34851878 217.007873535156 58.0
249.40283285 248.407577514648 59.0
21.446885757 27.666820526123 60.0
8.9089370924e-09 2.06781053543091 61.0
158.78492799 158.511871337891 0.0
158.78492799 160.345504760742 1.0
157.55845776 161.634689331055 2.0
152.76938803 157.02604675293 3.0
0 2.5329372882843 4.0
-2.544593235e-09 0.373229801654816 5.0
-58.2326681 -59.9811553955078 6.0
69.0042065680114 79.5702362060547 7.0
45.3968947500114 51.3475608825684 8.0
23.6073118180114 23.5053615570068 9.0
95.0640429550114 87.0657730102539 10.0
82.0546797530114 74.3658676147461 11.0
85.3699658280114 81.8947525024414 12.0
85.3699658280114 78.724479675293 13.0
85.3699658280114 78.1794052124023 14.0
0 -3.1383318901062 15.0
0 -0.0864683389663696 16.0
-8.1062745031e-09 0.911620855331421 17.0
55.25467914 57.9483222961426 18.0
-55.2546791610172 -58.7306785583496 19.0
-55.2546791390172 -60.5880126953125 20.0
-55.2546791500172 -56.6350975036621 21.0
77.304837526 77.6958618164062 22.0
141.66523917 140.632919311523 23.0
108.75521162 110.377182006836 24.0
23.392796791 20.246021270752 25.0
218.970076706167 224.754730224609 26.0
218.970076706167 223.898422241211 27.0
25.798573808167 25.7207260131836 28.0
244.768650516167 255.512435913086 29.0
244.768650516167 256.97998046875 30.0
46.784455177 45.4250679016113 31.0
9.3568982446 9.14182376861572 32.0
22.0501583827241 19.82204246521 33.0
22.0501583827241 19.2031345367432 34.0
22.050158379 20.3097171783447 35.0
9.7730771557 7.7411003112793 36.0
8.5197689385 7.00214004516602 37.0
-9.7930293767e-09 -0.729664027690887 38.0
-9.793029377e-09 -2.27411699295044 39.0
-11.879575757 -14.2215900421143 40.0
-24.686485019 -25.7833938598633 41.0
5.2909415675 3.25091695785522 42.0
-32.416208745 -32.5273628234863 43.0
99.885911084 93.4572219848633 44.0
99.885911089 94.8323364257812 45.0
7.0851753283 4.62867450714111 46.0
2.7442576215 2.01078462600708 47.0
158.78492799 161.392395019531 48.0
152.76938804 148.182556152344 49.0
0 -2.55120730400085 50.0
69.004206568 75.5496444702148 51.0
95.064042955 88.6372833251953 52.0
85.369965828 81.0701217651367 53.0
85.369965828 79.9349746704102 54.0
85.369965828 80.065803527832 55.0
0 1.26070809364319 56.0
77.304837532 80.0971832275391 57.0
218.97007671 221.678405761719 58.0
244.76865052 251.536270141602 59.0
22.050158385 24.7950115203857 60.0
0 0.563408076763153 61.0
151.64624176 158.84765625 0.0
151.64624176 165.787063598633 1.0
150.38828208 164.369125366211 2.0
145.59920524 153.706726074219 3.0
-5.3252550901e-11 6.02484130859375 4.0
-5.3299257115e-11 1.3813304901123 5.0
-61.604778542 -65.460578918457 6.0
71.939024658 74.6935043334961 7.0
58.719205568 53.016902923584 8.0
13.219819079 22.5619659423828 9.0
84.159605919 86.0759582519531 10.0
71.55339741 76.109977722168 11.0
77.786502181 82.6060104370117 12.0
77.786502181 79.1815719604492 13.0
77.786502181 81.5312271118164 14.0
0 -1.94463753700256 15.0
0 -0.773880422115326 16.0
0 -1.47571706771851 17.0
60.565003059 59.0563087463379 18.0
-60.565003069 -59.5979499816895 19.0
-60.565003047 -59.8511238098145 20.0
-60.56500306 -58.535961151123 21.0
82.439979301 76.3897552490234 22.0
139.92442618 141.520965576172 23.0
106.01089807 109.384071350098 24.0
21.723927641 18.8281269073486 25.0
222.36440549 230.435638427734 26.0
222.36440549 224.566131591797 27.0
26.102021825 26.8473663330078 28.0
248.46642732 255.192993164062 29.0
248.46642732 253.590515136719 30.0
46.595926078 44.9616622924805 31.0
9.3411053986 8.85549545288086 32.0
21.874976231 19.8073902130127 33.0
21.874976231 19.0252304077148 34.0
21.874976231 21.2135524749756 35.0
9.4067347628 6.26200389862061 36.0
8.1136646441 7.41254377365112 37.0
-1.119689488e-13 -1.96453285217285 38.0
-1.1907128515e-13 -0.97162252664566 39.0
-12.131503059 -12.2143669128418 40.0
-24.869162558 -29.3370761871338 41.0
5.2760376598 3.84713768959045 42.0
-32.732582688 -33.1023139953613 43.0
93.428576185 89.0642166137695 44.0
93.428576198 95.1086196899414 45.0
6.8515603131 7.30044269561768 46.0
2.6779769278 2.33660936355591 47.0
151.64624177 159.097808837891 48.0
145.59920525 153.741394042969 49.0
0 -1.63738894462585 50.0
71.939024668 75.2205352783203 51.0
84.159605929 90.2451095581055 52.0
77.786502191 81.4206924438477 53.0
77.786502191 81.41650390625 54.0
77.786502191 80.282112121582 55.0
0 3.75088381767273 56.0
82.439979311 79.6773223876953 57.0
222.3644055 222.277633666992 58.0
248.46642733 249.351791381836 59.0
21.874976241 22.9760704040527 60.0
0 1.41318845748901 61.0
170.60319807 158.037536621094 0.0
170.60319807 159.982330322266 1.0
169.43562546 166.981063842773 2.0
164.64655573 163.061340332031 3.0
0 2.16191697120667 4.0
0 2.88785815238953 5.0
-60.14313647 -65.0493392944336 6.0
70.08138323 75.3288192749023 7.0
54.953517976 51.5064163208008 8.0
15.127865244 24.7839546203613 9.0
87.861713091 88.6258239746094 10.0
75.085335522 76.7249755859375 11.0
85.117897201 81.2533569335938 12.0
85.117897201 77.8865737915039 13.0
85.117897201 77.4321975708008 14.0
0 -3.68081378936768 15.0
0 -0.843175709247589 16.0
0 2.15332818031311 17.0
58.900063342 58.1665496826172 18.0
-58.90006336 -59.8762855529785 19.0
-58.900063338 -59.0597953796387 20.0
-58.90006335 -59.898063659668 21.0
80.909169582 78.9380569458008 22.0
147.84205711 143.088088989258 23.0
112.54285648 108.837051391602 24.0
22.127104375 19.0511856079102 25.0
228.75122671 228.351028442383 26.0
228.75122671 225.462173461914 27.0
26.689506413 25.6229476928711 28.0
255.44073313 257.707366943359 29.0
255.44073313 257.128509521484 30.0
46.777400949 46.2585906982422 31.0
9.3834268056 9.01065540313721 32.0
22.009106221 20.5918216705322 33.0
22.009106221 17.7936878204346 34.0
22.009106219 20.2432861328125 35.0
9.4215328669 8.70091915130615 36.0
8.1627662754 7.89605045318604 37.0
0 -1.60021018981934 38.0
0 -3.52996921539307 39.0
-11.463464557 -14.236798286438 40.0
-24.24925898 -26.9932613372803 41.0
5.235497529 2.70149898529053 42.0
-31.933803643 -32.8002281188965 43.0
99.652103113 95.9946823120117 44.0
99.652103126 95.9328765869141 45.0
7.0996218909 4.61811017990112 46.0
2.63877172 2.03792834281921 47.0
170.60319808 163.235748291016 48.0
164.64655574 159.644485473633 49.0
0 -4.34657192230225 50.0
70.08138324 75.7954940795898 51.0
87.861713101 89.797966003418 52.0
85.117897211 80.6523132324219 53.0
85.117897211 81.5453262329102 54.0
85.117897211 79.8261642456055 55.0
0 3.37748336791992 56.0
80.909169592 79.4521408081055 57.0
228.75122672 223.439102172852 58.0
255.44073314 250.315505981445 59.0
22.009106229 23.0748558044434 60.0
0 1.17029571533203 61.0
167.851578580291 157.624694824219 0.0
167.851578580291 157.893371582031 1.0
166.57735499 164.799667358398 2.0
161.78828525 167.081008911133 3.0
0 1.32436752319336 4.0
-6.64285253e-09 0.0777676701545715 5.0
-61.3956489183344 -65.1298904418945 6.0
71.8097325321702 76.6603546142578 7.0
48.9969343011702 53.6090507507324 8.0
22.8127982311702 23.6905727386475 9.0
89.7781064941702 89.7309112548828 10.0
76.3594286621702 77.3215179443359 11.0
84.7871816201702 85.5937728881836 12.0
84.7871816201702 81.909309387207 13.0
84.7871816201702 81.6051864624023 14.0
0 -1.08381700515747 15.0
0 -1.30826950073242 16.0
-4.41900778500001e-09 -1.9726459980011 17.0
57.8835417824982 58.042781829834 18.0
-57.8835418297554 -59.7814826965332 19.0
-57.8835418087553 -59.0279655456543 20.0
-57.8835418117554 -60.7680854797363 21.0
79.8701809153974 78.1941757202148 22.0
146.451452522456 140.681259155273 23.0
111.648017253071 107.876846313477 24.0
21.3297494220706 21.9726047515869 25.0
226.321633440581 227.088104248047 26.0
226.321633440581 225.171005249023 27.0
25.914641882581 27.543342590332 28.0
252.236275320581 255.906646728516 29.0
252.236275320581 253.653778076172 30.0
46.4206329649525 45.5576324462891 31.0
9.28241392350397 8.83930969238281 32.0
21.9866391040031 20.2806625366211 33.0
21.9866391040031 23.0100383758545 34.0
21.9866391152656 20.2263431549072 35.0
9.8363200119817 6.7127046585083 36.0
8.5244416084817 7.61852979660034 37.0
-8.913903959e-09 1.64067280292511 38.0
-8.913903959e-09 -2.53642416000366 39.0
-11.7185840903777 -13.2036571502686 40.0
-24.5302552240018 -27.2293472290039 41.0
5.05447173854186 4.57785081863403 42.0
-32.2067254496419 -32.3760643005371 43.0
97.5205461086419 91.2004699707031 44.0
97.5205460082612 94.4583282470703 45.0
6.9910232021 5.53251647949219 46.0
2.7931508456 1.15151596069336 47.0
167.85157858 159.698196411133 48.0
161.78828526 158.298217773438 49.0
0 -1.23219132423401 50.0
71.809732532 76.7780075073242 51.0
89.778106494 90.8864059448242 52.0
84.78718162 81.4463577270508 53.0
84.78718162 81.3097152709961 54.0
84.78718162 80.0934753417969 55.0
0 1.49703764915466 56.0
79.870180915 80.2734832763672 57.0
226.32163344 223.388259887695 58.0
252.23627532 255.827041625977 59.0
21.986639115 17.8986434936523 60.0
0 1.27972865104675 61.0
155.72149572 166.822128295898 0.0
155.72149572 167.83854675293 1.0
154.36209964 159.873916625977 2.0
149.57302992 154.213043212891 3.0
0 1.00091886520386 4.0
-2.1329449713e-14 4.05695390701294 5.0
-59.029042726 -64.6119079589844 6.0
69.637068519 74.0829086303711 7.0
46.760599862 46.6488189697266 8.0
22.876468647 23.217248916626 9.0
92.327884871 84.3157196044922 10.0
79.671073078 74.6905899047852 11.0
83.891089171 81.943489074707 12.0
83.891089171 79.1677627563477 13.0
83.891089171 79.104850769043 14.0
0 -3.7549262046814 15.0
0 -0.817016661167145 16.0
-4.9703502458e-14 3.94266295433044 17.0
58.600096264 57.2898139953613 18.0
-58.600096284 -57.911979675293 19.0
-58.600096262 -58.8154487609863 20.0
-58.600096275 -59.1742210388184 21.0
80.404302592 76.5683670043945 22.0
151.38892495 141.837020874023 23.0
115.86888952 105.457878112793 24.0
24.055438403 21.5884704589844 25.0
231.79322755 226.150680541992 26.0
231.79322755 226.115051269531 27.0
27.193361005 27.110803604126 28.0
258.98658857 252.017517089844 29.0
258.98658857 252.512680053711 30.0
46.823890995 45.8057518005371 31.0
9.3792601913 8.49391841888428 32.0
21.804206307 20.2965641021729 33.0
21.804206307 21.852710723877 34.0
21.804206307 20.5895042419434 35.0
9.6155620046 8.04371452331543 36.0
8.358955766 7.01451253890991 37.0
-3.1662993643e-14 -0.733949482440948 38.0
-3.1662993637e-14 -3.28981256484985 39.0
-11.807313532 -15.0730028152466 40.0
-24.829814885 -26.2302989959717 41.0
5.3128321851 6.1045503616333 42.0
-32.663139354 -33.1430435180664 43.0
101.01406899 92.7848892211914 44.0
101.014069 94.5856018066406 45.0
6.8367133115 6.12080860137939 46.0
2.5495157899 2.27223634719849 47.0
155.72149573 164.35595703125 48.0
149.57302993 148.828948974609 49.0
0 -2.56835126876831 50.0
69.637068529 74.6783981323242 51.0
92.327884881 86.7743530273438 52.0
83.891089181 80.7299652099609 53.0
83.891089181 81.3804550170898 54.0
83.891089181 80.2576141357422 55.0
0 4.9667181968689 56.0
80.404302602 79.8279647827148 57.0
231.79322756 222.422393798828 58.0
258.98658858 249.107757568359 59.0
21.804206317 21.4210338592529 60.0
0 0.899785339832306 61.0
137.374995184379 167.431610107422 0.0
137.374995184379 168.116363525391 1.0
136.3102256 147.96061706543 2.0
131.52115587 138.277587890625 3.0
0 7.75590705871582 4.0
-6.333032104e-09 -2.41415929794312 5.0
-66.2396845214069 -65.5914840698242 6.0
75.2081867339241 71.0615539550781 7.0
53.059418127924 52.6302680969238 8.0
22.148768605924 5.08159017562866 9.0
94.161286678924 89.9552001953125 10.0
80.538240061924 87.063362121582 11.0
83.674359737924 84.0154876708984 12.0
83.674359737924 73.7836608886719 13.0
83.674359737924 80.2213134765625 14.0
0 -6.39899301528931 15.0
0 -0.322709619998932 16.0
-3.915399699e-09 -1.25803756713867 17.0
59.2715141168677 54.8433036804199 18.0
-59.2715141163861 -60.0240058898926 19.0
-59.2715140963861 -59.5690765380859 20.0
-59.2715140973861 -56.798038482666 21.0
82.1956198616281 79.4920120239258 22.0
146.666070116868 141.594573974609 23.0
111.374568646085 108.022399902344 24.0
20.2017082260846 22.2541351318359 25.0
228.861689978759 227.367248535156 26.0
228.861689978759 226.758560180664 27.0
25.6355753607585 31.0771675109863 28.0
254.497265338758 243.41471862793 29.0
254.497265338758 244.169891357422 30.0
46.5601388079682 45.5739479064941 31.0
9.47121734989491 8.14004611968994 32.0
22.9241057642206 16.1393127441406 33.0
22.9241057642206 22.220308303833 34.0
22.9241057767324 19.4652290344238 35.0
10.1220978289628 6.28947973251343 36.0
8.87400961456285 6.59697437286377 37.0
-8.76623282e-09 0.150171458721161 38.0
-8.76623281999999e-09 5.52019691467285 39.0
-12.695174089693 -15.9745111465454 40.0
-24.9924547757919 -19.9606285095215 41.0
4.83749505730338 6.81323194503784 42.0
-32.3056239071034 -35.373706817627 43.0
85.5363773001034 84.8748550415039 44.0
85.5363771695882 88.1779479980469 45.0
7.4262603606 5.58299827575684 46.0
2.6530002252 2.05302095413208 47.0
137.37499518 171.981369018555 48.0
131.52115588 134.866836547852 49.0
0 3.11496019363403 50.0
75.208186734 74.8920974731445 51.0
94.161286679 83.7756881713867 52.0
83.674359738 81.9499740600586 53.0
83.674359738 78.828010559082 54.0
83.674359738 79.2952728271484 55.0
0 4.30617475509644 56.0
82.195619862 79.753288269043 57.0
228.86168998 225.364013671875 58.0
254.49726534 242.576126098633 59.0
22.924105777 17.9782180786133 60.0
0 1.27670609951019 61.0
160.01582397 160.546875 0.0
160.01582397 164.467071533203 1.0
158.85952038 160.635971069336 2.0
154.07045066 155.67985534668 3.0
0 0.333596169948578 4.0
0 4.47875308990479 5.0
-67.574651288 -59.5486755371094 6.0
78.492135886 75.3117294311523 7.0
56.626794927 51.3640251159668 8.0
21.865340949 21.9085216522217 9.0
93.017289512 92.3890075683594 10.0
79.796423925 77.6202163696289 11.0
85.771420467 87.4445419311523 12.0
85.771420467 81.7160034179688 13.0
85.771420467 82.3588027954102 14.0
0 -0.0411826372146606 15.0
0 -0.0819154977798462 16.0
0 -0.175158679485321 17.0
56.642530877 59.1359558105469 18.0
-56.642530887 -59.7608108520508 19.0
-56.642530865 -59.2791862487793 20.0
-56.642530877 -61.4495162963867 21.0
78.740122294 78.3712615966797 22.0
146.96045729 143.845291137695 23.0
111.70155071 112.202178955078 24.0
23.554718076 20.9785194396973 25.0
225.70057959 225.795425415039 26.0
225.70057959 224.478088378906 27.0
26.294236319 27.8840370178223 28.0
251.99481592 256.230682373047 29.0
251.99481592 256.483734130859 30.0
46.900787267 46.1537971496582 31.0
9.3641391652 8.89545059204102 32.0
22.097591407 22.0585689544678 33.0
22.097591407 25.7006778717041 34.0
22.097591409 20.5135154724121 35.0
9.4995605065 7.62343311309814 36.0
8.296552137 7.67874431610107 37.0
-2.0636932738e-09 1.98853230476379 38.0
-2.0636933037e-09 -1.746098279953 39.0
-12.030853227 -15.0867700576782 40.0
-24.945261012 -26.4139595031738 41.0
5.0097297805 5.06824445724487 42.0
-32.621496875 -33.2409973144531 43.0
95.892521389 94.883674621582 44.0
95.892521402 94.8365707397461 45.0
7.2248013033 6.44018507003784 46.0
2.6932152991 2.58672904968262 47.0
160.01582398 165.635559082031 48.0
154.07045067 155.25651550293 49.0
0 0.719836711883545 50.0
78.492135896 76.7763595581055 51.0
93.017289522 89.0708160400391 52.0
85.771420477 81.7571563720703 53.0
85.771420477 81.6886901855469 54.0
85.771420477 80.5338363647461 55.0
0 2.54747176170349 56.0
78.740122304 78.7477798461914 57.0
225.7005796 219.340270996094 58.0
251.99481593 252.599456787109 59.0
22.097591419 18.6205577850342 60.0
0 1.74484467506409 61.0
165.351029781406 157.878570556641 0.0
165.351029781406 162.381591796875 1.0
164.08770701 165.636184692383 2.0
158.62307174 155.311874389648 3.0
1.0832880939 2.47760844230652 4.0
1.08328810226321 8.18432426452637 5.0
-67.7555482903632 -61.1807518005371 6.0
78.5853610227712 75.5879669189453 7.0
49.4458210727712 51.3133583068848 8.0
29.1395399497712 23.7947292327881 9.0
89.1073980397712 83.4504318237305 10.0
76.7986695777712 74.4040222167969 11.0
81.8736442707712 80.9371109008789 12.0
81.8736442707712 78.3630065917969 13.0
81.8736442707712 80.6130523681641 14.0
3.7602291097e-09 -0.0713506937026978 15.0
1.6272515452e-09 -0.0682880282402039 16.0
9.55669260000003e-09 1.25840258598328 17.0
57.8610660729661 59.728572845459 18.0
-57.8610660316569 -60.1925582885742 19.0
-57.8610660136569 -57.8545188903809 20.0
-57.8610659926569 -58.7698059082031 21.0
80.0843260790733 77.0896530151367 22.0
147.797166008966 141.997604370117 23.0
112.755926366208 108.210830688477 24.0
23.3683622332077 21.67160987854 25.0
227.881492092813 225.019256591797 26.0
227.881492092813 226.250411987305 27.0
26.901197311813 24.5954608917236 28.0
254.782689402813 256.296020507812 29.0
254.782689402813 257.09765625 30.0
46.4226135160305 46.9759521484375 31.0
9.28952276629632 8.97862720489502 32.0
22.2232600870896 21.1368560791016 33.0
22.2232600870896 25.127046585083 34.0
22.2232601261625 21.121654510498 35.0
9.86013207565584 8.47025203704834 36.0
8.62443095355584 7.59949207305908 37.0
-6.21446807000003e-09 -2.18093037605286 38.0
-6.21446807000003e-09 -2.29818391799927 39.0
-11.6745441648284 -14.040864944458 40.0
-24.6294752288498 -25.0429153442383 41.0
5.21946170480936 6.99011611938477 42.0
-32.3541859148094 -34.2182998657227 43.0
96.8442895768094 95.5808563232422 44.0
96.8442894218094 92.9817581176758 45.0
7.12667677344732 6.46695756912231 46.0
2.47454266054732 2.56767535209656 47.0
165.35102978 162.666473388672 48.0
158.62307175 153.156524658203 49.0
1.0832881023 1.32762098312378 50.0
78.585361023 77.681999206543 51.0
89.10739804 88.1610565185547 52.0
81.873644271 81.6924667358398 53.0
81.873644271 81.0575866699219 54.0
81.873644271 80.1906585693359 55.0
9.556692601e-09 1.85244917869568 56.0
80.084326079 79.4581756591797 57.0
227.88149209 222.508331298828 58.0
254.7826894 251.537719726562 59.0
22.223260126 18.91335105896 60.0
0 1.36672592163086 61.0
167.553581723244 161.729125976562 0.0
167.553581723244 166.587524414062 1.0
166.34051561 152.938430786133 2.0
160.87588041 148.695068359375 3.0
0.07443217065 -1.69536685943604 4.0
0.0744321803303848 9.64665508270264 5.0
-71.9832955547494 -58.3377151489258 6.0
83.6356714486229 77.2297058105469 7.0
46.2686053676229 55.8860054016113 8.0
37.3670660806229 26.8739986419678 9.0
80.5862279426229 88.8506317138672 10.0
67.1094132356229 78.7582244873047 11.0
76.9737490936229 83.3227310180664 12.0
76.9737490936229 75.3675537109375 13.0
76.9737490936229 76.1182327270508 14.0
0 -5.86562061309814 15.0
0 -1.0022246837616 16.0
-5.408911194e-09 -2.96489357948303 17.0
52.4755535595627 55.6439476013184 18.0
-52.4755535914343 -59.2691841125488 19.0
-52.4755535704343 -60.353946685791 20.0
-52.4755535764343 -58.8188629150391 21.0
73.3996972724656 73.2803039550781 22.0
143.041160184703 149.606903076172 23.0
110.856823048379 109.340538024902 24.0
22.7097946413792 21.3275775909424 25.0
216.440857456488 223.272705078125 26.0
216.440857456488 220.47053527832 27.0
29.4519952804879 27.7737464904785 28.0
245.892852736488 248.232498168945 29.0
245.892852736488 251.146209716797 30.0
46.6444792144921 42.8906707763672 31.0
9.4470507578 7.6134295463562 32.0
20.9241436962482 16.4485282897949 33.0
20.9241436962482 15.3453779220581 34.0
20.9241437030079 26.5982818603516 35.0
9.4172577167 12.2832746505737 36.0
8.2650889788 6.61757755279541 37.0
-9.25244505399999e-09 -6.74445009231567 38.0
-9.252445055e-09 -1.6116828918457 39.0
-11.9780049362171 -18.2689743041992 40.0
-25.9524022634184 -23.8079967498779 41.0
4.92695565085238 6.14523983001709 42.0
-33.3264788503524 -34.1615676879883 43.0
88.2464570983524 86.0989761352539 44.0
88.2464570415785 89.2647171020508 45.0
7.2137318066 9.87790203094482 46.0
2.8224558435 2.58502411842346 47.0
167.55358172 158.868850708008 48.0
160.87588042 148.166061401367 49.0
0.07443218033 3.71821880340576 50.0
83.635671449 70.618278503418 51.0
80.586227943 86.6785125732422 52.0
76.973749094 80.6233673095703 53.0
76.973749094 77.1133193969727 54.0
76.973749094 78.8096694946289 55.0
0 0.0758295655250549 56.0
73.399697272 79.2366409301758 57.0
216.44085746 218.775924682617 58.0
245.89285274 249.626708984375 59.0
20.924143703 20.0493106842041 60.0
0 3.16526579856873 61.0
133.96447814 156.484268188477 0.0
133.96447814 158.69873046875 1.0
132.82417083 158.257568359375 2.0
127.35953564 158.636291503906 3.0
6.9324334406 -3.07842350006104 4.0
6.9324334407 11.8928127288818 5.0
-75.396993483 -31.5629863739014 6.0
82.579096537 49.2709197998047 7.0
56.966683588 46.1846618652344 8.0
25.612412939 13.753514289856 9.0
107.99938727 101.844917297363 10.0
95.327787904 83.2505493164062 11.0
95.787695119 86.7883453369141 12.0
95.787695119 93.5574493408203 13.0
95.787695119 49.8998031616211 14.0
0 8.48309516906738 15.0
0 3.82146835327148 16.0
0 1.09526300430298 17.0
60.825454134 57.9217948913574 18.0
-60.825454144 -61.5902099609375 19.0
-60.825454122 -35.0293159484863 20.0
-60.825454134 -61.0265808105469 21.0
82.881445175 78.1933822631836 22.0
132.51398669 123.663558959961 23.0
96.547829206 108.719528198242 24.0
19.226695444 26.3640766143799 25.0
215.39543187 231.842742919922 26.0
215.39543187 215.280090332031 27.0
24.894436475 26.0250625610352 28.0
240.28986836 219.113784790039 29.0
240.28986836 187.782821655273 30.0
46.308874158 43.5092620849609 31.0
9.1907036593 7.53676605224609 32.0
22.055991031 27.6432456970215 33.0
22.055991031 23.8381061553955 34.0
22.055991031 22.5114707946777 35.0
9.3675431257 0.207445204257965 36.0
8.1441339579 0.867900669574738 37.0
-3.3839689981e-11 -7.23617076873779 38.0
-5.0898929458e-11 5.38095569610596 39.0
-14.078492871 -22.0102939605713 40.0
-26.851151096 -18.1554660797119 41.0
4.343056661 15.1537284851074 42.0
-34.269056436 -38.4561042785645 43.0
90.280993891 71.0094757080078 44.0
90.280993904 113.503524780273 45.0
6.2510272122 5.36233043670654 46.0
2.7194826251 -1.93530201911926 47.0
133.96447815 157.435180664062 48.0
127.35953565 141.216506958008 49.0
6.9324334506 2.55555558204651 50.0
82.579096547 17.9202690124512 51.0
107.99938728 80.0791091918945 52.0
95.787695128 74.9440383911133 53.0
95.787695128 78.4908676147461 54.0
95.787695128 80.2130737304688 55.0
0 7.01712417602539 56.0
82.881445185 77.2958831787109 57.0
215.39543188 216.850936889648 58.0
240.28986837 238.672897338867 59.0
22.055991041 23.0276584625244 60.0
0 14.5080785751343 61.0
175.706422112054 159.601943969727 0.0
175.706422112054 157.137359619141 1.0
174.33733556 161.827819824219 2.0
169.54826579 153.448379516602 3.0
0 -1.86016917228699 4.0
-5.55836977e-09 9.21675491333008 5.0
-63.7346356289219 -56.0812187194824 6.0
74.7073562792267 71.0563125610352 7.0
45.9912394332267 51.9045448303223 8.0
28.7161168452268 21.938570022583 9.0
89.6519887592268 87.940803527832 10.0
77.0901447872267 77.6189880371094 11.0
85.0701401072267 87.7401580810547 12.0
85.0701401072267 80.4932403564453 13.0
85.0701401072267 79.3829727172852 14.0
1.818650972e-09 -1.04033899307251 15.0
1.8186509716e-09 -0.732830822467804 16.0
1.544446326e-08 -1.37163949012756 17.0
56.1804717526764 57.3487739562988 18.0
-56.1804718873401 -59.0750350952148 19.0
-56.1804718703401 -58.1939544677734 20.0
-56.1804718373401 -55.9028167724609 21.0
78.8305113919532 77.6848602294922 22.0
144.020082732305 145.140319824219 23.0
111.750808827882 112.757568359375 24.0
21.5375415518819 19.4748802185059 25.0
222.850594124108 229.32600402832 26.0
222.850594124108 220.734298706055 27.0
24.4508406311078 26.052713394165 28.0
247.301434754108 260.256988525391 29.0
247.301434754108 257.740753173828 30.0
46.3661002741146 45.1001472473145 31.0
9.33517575652289 9.31139469146729 32.0
22.6500395547658 21.3931980133057 33.0
22.6500395547658 5.70928478240967 34.0
22.6500396076644 20.4942874908447 35.0
9.89761356117041 6.58216190338135 36.0
8.65092282737043 7.86821031570435 37.0
-4.98269157000003e-09 7.35635805130005 38.0
-4.98269156999997e-09 8.42078018188477 39.0
-11.5960186386701 -10.5329122543335 40.0
-25.3611226063526 -25.5153541564941 41.0
5.00706462432234 4.88421964645386 42.0
-33.0957925755223 -31.4123058319092 43.0
93.8241078555223 94.6019668579102 44.0
93.8241073253902 95.5746688842773 45.0
6.57560281524712 4.55637645721436 46.0
2.34044698064712 2.27259063720703 47.0
175.70642211 160.249374389648 48.0
169.5482658 160.735504150391 49.0
0 -4.06069850921631 50.0
74.707356279 75.0306091308594 51.0
89.651988759 90.6996231079102 52.0
85.070140108 80.5338439941406 53.0
85.070140108 80.6412048339844 54.0
85.070140108 80.2041168212891 55.0
1.5444463253e-08 0.321158587932587 56.0
78.830511392 78.9087066650391 57.0
222.85059412 217.436340332031 58.0
247.30143476 252.286392211914 59.0
22.650039608 21.7111225128174 60.0
0 6.01372909545898 61.0
181.1821658 157.900985717773 0.0
181.1821658 163.289321899414 1.0
180.21342245 177.867095947266 2.0
179.11688484 139.412002563477 3.0
-4.3847347347e-09 130.802612304688 4.0
-4.3847811105e-09 2.42138719558716 5.0
-55.801199946 -54.0600128173828 6.0
64.149270125 71.4222183227539 7.0
64.149270125 62.8218231201172 8.0
0 47.848804473877 9.0
83.889955065 109.681121826172 10.0
69.908477834 74.6972274780273 11.0
83.157495096 77.0538482666016 12.0
83.157495096 74.9897918701172 13.0
83.157495096 76.3853759765625 14.0
0 -5.85904169082642 15.0
0 -1.74360251426697 16.0
0 -0.437414824962616 17.0
51.232682739 58.9941101074219 18.0
-51.232682749 -57.7681579589844 19.0
-51.232682727 -58.8051910400391 20.0
-51.232682739 -55.3772392272949 21.0
73.741860452 78.412223815918 22.0
147.28957609 138.725082397461 23.0
107.85851804 105.323516845703 24.0
21.395254059 17.7509860992432 25.0
221.03143655 225.42578125 26.0
221.03143655 216.883941650391 27.0
26.506236652 29.6016464233398 28.0
247.53767321 264.289886474609 29.0
247.53767321 254.138336181641 30.0
46.314357556 46.1159934997559 31.0
9.2082868403 9.56864452362061 32.0
22.509177703 17.2197532653809 33.0
22.509177703 19.1549186706543 34.0
22.509177707 18.1371040344238 35.0
9.8730648562 9.88022613525391 36.0
8.5779212016 5.31985950469971 37.0
-4.3848622858e-09 -3.15506982803345 38.0
-4.3848622867e-09 -2.0541410446167 39.0
-10.157002873 -16.1597652435303 40.0
-22.382747678 -28.6011276245117 41.0
4.6217172636 2.81594061851501 42.0
-29.638636697 -28.1725120544434 43.0
101.08510702 72.10986328125 44.0
101.08510703 121.046783447266 45.0
7.820850159 8.48478507995605 46.0
2.7257580952 2.95245552062988 47.0
181.18216581 173.100173950195 48.0
179.11688485 169.613510131836 49.0
0 38.6413993835449 50.0
64.149270135 76.6653366088867 51.0
83.889955075 95.2996597290039 52.0
83.157495106 76.4390411376953 53.0
83.157495106 76.7992324829102 54.0
83.157495106 80.6597595214844 55.0
0 -0.948798000812531 56.0
73.741860462 79.1239624023438 57.0
221.03143656 221.055206298828 58.0
247.53767322 252.454467773438 59.0
22.509177717 28.0094604492188 60.0
0 0.754416167736053 61.0
165.37282172 164.771591186523 0.0
165.37282172 165.59538269043 1.0
164.08621151 163.676712036133 2.0
159.29714179 167.432281494141 3.0
-1.3888822338e-14 5.86176204681396 4.0
-4.7192277145e-14 8.77202796936035 5.0
-66.005992672 -61.3398475646973 6.0
75.961399788 76.3741836547852 7.0
45.913450672 51.3402900695801 8.0
30.047949106 25.0838871002197 9.0
93.031398984 88.3005294799805 10.0
80.004953217 76.4509124755859 11.0
84.258851389 80.7977828979492 12.0
84.258851389 81.2753372192383 13.0
84.258851389 80.1710052490234 14.0
0 -2.71896481513977 15.0
0 -1.34778261184692 16.0
0 -1.89199686050415 17.0
57.540175565 58.7489356994629 18.0
-57.540175575 -58.3593559265137 19.0
-57.540175554 -57.5392189025879 20.0
-57.540175566 -55.945198059082 21.0
79.7360664 75.8618316650391 22.0
145.25207475 141.160629272461 23.0
110.43041955 107.821685791016 24.0
23.347608587 18.2015056610107 25.0
224.98814116 227.524459838867 26.0
224.98814116 226.288146972656 27.0
25.828111373 25.6544818878174 28.0
250.81625254 254.621795654297 29.0
250.81625254 255.857360839844 30.0
46.663023555 46.9519081115723 31.0
9.2981412245 8.83577060699463 32.0
22.195890824 21.7209815979004 33.0
22.195890824 18.6042194366455 34.0
22.195890824 22.0220108032227 35.0
10.027075124 8.91591453552246 36.0
8.7661727812 8.14398765563965 37.0
-3.5740541168e-17 0.823405861854553 38.0
0 -1.72273635864258 39.0
-11.826009581 -16.6547870635986 40.0
-24.366354123 -23.3950366973877 41.0
5.2004399562 7.3001971244812 42.0
-32.158958363 -33.5394248962402 43.0
102.50665687 95.3285522460938 44.0
102.50665688 94.5260162353516 45.0
7.1120677723 7.85497713088989 46.0
2.5830831519 2.62177586555481 47.0
165.37282173 161.207443237305 48.0
159.2971418 158.772521972656 49.0
0 2.92333197593689 50.0
75.961399798 76.3285980224609 51.0
93.031398994 91.6590805053711 52.0
84.258851399 81.1545104980469 53.0
84.258851399 80.7502899169922 54.0
84.258851399 80.411750793457 55.0
0 0.316005885601044 56.0
79.73606641 80.2386474609375 57.0
224.98814117 218.659759521484 58.0
250.81625255 249.927993774414 59.0
22.195890834 22.4010620117188 60.0
0 2.1990225315094 61.0
164.58328001 162.778747558594 0.0
164.58328001 164.676956176758 1.0
163.48846591 167.106384277344 2.0
158.69939618 190.850860595703 3.0
0 5.54310894012451 4.0
-2.5694952765e-12 3.57540583610535 5.0
-66.030374652 -46.684009552002 6.0
75.748186354 86.2411193847656 7.0
53.928597355 26.7707633972168 8.0
21.819588989 22.2929382324219 9.0
85.346785249 92.963508605957 10.0
72.28886925 79.7825012207031 11.0
77.060322163 80.9975433349609 12.0
77.060322163 70.2804107666016 13.0
77.060322163 77.7778778076172 14.0
0 1.55915141105652 15.0
0 0.437254726886749 16.0
-6.4982172206e-13 2.54799199104309 17.0
58.685746701 57.2248001098633 18.0
-58.685746721 -55.7280426025391 19.0
-58.685746699 -58.165454864502 20.0
-58.685746711 -61.3686294555664 21.0
80.726705539 78.8855209350586 22.0
137.12085662 137.622833251953 23.0
105.4521144 111.922592163086 24.0
23.350383004 24.5733757019043 25.0
217.84756217 223.580169677734 26.0
217.84756217 217.089416503906 27.0
24.50774448 31.1229057312012 28.0
242.35530666 254.59001159668 29.0
242.35530666 249.690078735352 30.0
46.991592628 45.1640243530273 31.0
9.4380682961 10.0436143875122 32.0
22.040958818 24.6665992736816 33.0
22.040958818 15.8842258453369 34.0
22.040958818 20.4890289306641 35.0
9.0336922681 5.73582553863525 36.0
7.8456315029 8.33577060699463 37.0
-1.6126984745e-11 0.188456356525421 38.0
-1.6126997886e-11 0.0804041624069214 39.0
-11.962981483 -18.2951908111572 40.0
-24.218896123 -24.5339126586914 41.0
5.5105523825 2.20825862884521 42.0
-32.240533973 -31.249303817749 43.0
108.77585045 92.3369369506836 44.0
108.77585047 100.005844116211 45.0
7.0644886267 4.21744251251221 46.0
2.3791763176 2.18349099159241 47.0
164.58328002 166.90608215332 48.0
158.69939619 180.135452270508 49.0
0 -2.49729037284851 50.0
75.748186364 71.5849304199219 51.0
85.346785259 89.1382522583008 52.0
77.060322173 79.5747604370117 53.0
77.060322173 75.283935546875 54.0
77.060322173 78.9232788085938 55.0
0 7.73783016204834 56.0
80.726705549 80.5338516235352 57.0
217.84756218 209.096801757812 58.0
242.35530667 246.213226318359 59.0
22.040958828 21.9083290100098 60.0
0 3.01537418365479 61.0
156.56602634 166.996856689453 0.0
156.56602634 163.216842651367 1.0
155.52811462 170.640487670898 2.0
150.7390449 151.848907470703 3.0
0 4.94050264358521 4.0
-4.2660657474e-11 -1.41317081451416 5.0
-68.731927322 -66.0900115966797 6.0
77.866121724 80.3051452636719 7.0
54.790496082 46.7500190734863 8.0
23.075625632 32.5637359619141 9.0
98.648221469 87.4646377563477 10.0
84.841879345 76.28076171875 11.0
82.413322651 81.6635131835938 12.0
82.413322651 79.5570602416992 13.0
82.413322651 81.3304672241211 14.0
0 4.93020153045654 15.0
0 0.34750235080719 16.0
-8.1178606482e-11 4.74170827865601 17.0
56.504815131 60.2819137573242 18.0
-56.504815151 -51.4048843383789 19.0
-56.504815129 -60.8882675170898 20.0
-56.504815141 -58.2508201599121 21.0
77.90135043 75.4257125854492 22.0
153.77241756 140.865768432617 23.0
117.22080763 111.613304138184 24.0
24.48983216 19.244457244873 25.0
231.673768 227.543914794922 26.0
231.673768 220.664215087891 27.0
23.990083289 29.0249137878418 28.0
255.66385129 255.848693847656 29.0
255.66385129 248.753280639648 30.0
46.75308208 46.4244575500488 31.0
9.3064218121 7.82740592956543 32.0
21.396535279 19.7544956207275 33.0
21.396535279 13.4163761138916 34.0
21.396535279 20.6565799713135 35.0
8.9771209311 11.1744432449341 36.0
7.7013958298 5.92961311340332 37.0
-1.2841912078e-10 1.2199182510376 38.0
-1.2841912089e-10 -0.85528177022934 39.0
-12.303226357 -15.8131647109985 40.0
-26.19435029 -24.2702789306641 41.0
4.9156547157 11.4817390441895 42.0
-33.638238155 -36.9418678283691 43.0
87.857086331 89.1479721069336 44.0
87.857086344 89.78125 45.0
7.2063665523 5.07337284088135 46.0
2.7195235577 2.20142197608948 47.0
156.56602635 169.271057128906 48.0
150.73904491 152.083435058594 49.0
0 1.83639597892761 50.0
77.866121733 76.2368850708008 51.0
98.648221479 90.2155609130859 52.0
82.413322661 80.6285781860352 53.0
82.413322661 81.4595336914062 54.0
82.413322661 80.2137832641602 55.0
0 0.546156346797943 56.0
77.90135044 78.7353744506836 57.0
231.67376801 220.579742431641 58.0
255.6638513 247.508041381836 59.0
21.396535289 16.7020797729492 60.0
0 4.17463970184326 61.0
161.464323483046 163.383651733398 0.0
161.464323483046 152.833740234375 1.0
160.19873735 157.677688598633 2.0
155.40966762 159.434783935547 3.0
0 -2.52708196640015 4.0
-8.782604249e-09 6.43675565719604 5.0
-53.1017804526513 -57.8128204345703 6.0
65.3489096532794 69.3166885375977 7.0
52.9975025992794 50.601936340332 8.0
12.3514070542794 27.7344303131104 9.0
85.5884879572794 86.1074600219727 10.0
73.8637784712794 72.0203552246094 11.0
75.7731636902794 80.6948699951172 12.0
75.7731636902794 83.8072357177734 13.0
75.7731636902794 78.4548492431641 14.0
0 -5.08563232421875 15.0
0 -1.04890012741089 16.0
-5.436396189e-09 -5.80834102630615 17.0
52.5459279188993 58.3730163574219 18.0
-52.5459279519191 -58.6178588867188 19.0
-52.5459279309191 -61.0719718933105 20.0
-52.5459279359191 -56.0626029968262 21.0
74.4799720193728 74.7979507446289 22.0
140.252255884516 144.303253173828 23.0
107.953168718145 109.778038024902 24.0
20.087941068145 18.9377727508545 25.0
214.732227906092 223.229339599609 26.0
214.732227906092 226.594879150391 27.0
24.7549790840916 25.7979679107666 28.0
239.487206986092 254.562881469727 29.0
239.487206986092 256.168975830078 30.0
46.4226951943921 47.4500350952148 31.0
9.2972348456 9.45430088043213 32.0
21.9340440829538 17.9560146331787 33.0
21.9340440829538 21.3133792877197 34.0
21.9340440908894 20.988166809082 35.0
9.205955454 12.1505517959595 36.0
7.8661622202 6.61413812637329 37.0
-9.22286532e-09 -6.14931344985962 38.0
-9.22286532e-09 -2.11518311500549 39.0
-12.7618526829595 -8.93564510345459 40.0
-25.7379024782438 -27.8157844543457 41.0
4.63881470206616 7.281174659729 42.0
-32.9734878310662 -34.9303321838379 43.0
94.186957054052 93.4640884399414 44.0
94.1869571050662 96.5966644287109 45.0
7.3911972129 5.03342723846436 46.0
2.7138170514 1.33035409450531 47.0
161.46432349 173.43701171875 48.0
155.40966763 147.898818969727 49.0
0 -1.2395281791687 50.0
65.348909654 72.1827087402344 51.0
85.588487958 82.1266555786133 52.0
75.773163691 79.9139709472656 53.0
75.773163691 76.5281066894531 54.0
75.773163691 80.4564208984375 55.0
0 -2.32747364044189 56.0
74.479972019 78.4493103027344 57.0
214.7322279 216.704940795898 58.0
239.48720699 246.550704956055 59.0
21.934044091 21.9634418487549 60.0
0 1.44568991661072 61.0
139.81114941 172.960586547852 0.0
139.81114941 163.743499755859 1.0
138.72371154 162.726257324219 2.0
133.25907635 139.098007202148 3.0
5.5821240868 5.601393699646 4.0
5.5821240872 8.69606971740723 5.0
-67.648715297 -46.5973510742188 6.0
77.140169597 76.7492980957031 7.0
50.26987356 52.6912727355957 8.0
26.870296028 28.2903938293457 9.0
91.660201736 74.2676239013672 10.0
79.593270253 83.2192840576172 11.0
79.152063418 87.2714614868164 12.0
79.152063418 84.8142929077148 13.0
79.152063418 80.5010604858398 14.0
0 1.75771641731262 15.0
0 -1.00091528892517 16.0
-2.4962131261e-10 -3.18909978866577 17.0
55.207716411 59.0972061157227 18.0
-55.207716432 -57.2512435913086 19.0
-55.207716411 -62.5128440856934 20.0
-55.207716423 -59.7905235290527 21.0
77.299299234 77.015266418457 22.0
144.21040881 146.919906616211 23.0
113.17796305 109.035057067871 24.0
21.090090984 21.6090049743652 25.0
221.50970805 220.214736938477 26.0
221.50970805 222.326293945312 27.0
27.369192479 25.0262985229492 28.0
248.87890054 254.409912109375 29.0
248.87890054 249.222778320312 30.0
46.402939231 46.7055397033691 31.0
9.3869740051 8.57632637023926 32.0
22.091582805 18.1158866882324 33.0
22.091582805 21.7424583435059 34.0
22.091582804 22.1324806213379 35.0
9.4773175635 10.0925178527832 36.0
8.3005898756 7.59938144683838 37.0
-5.2605100372e-10 -0.195957005023956 38.0
-5.2605100368e-10 -1.74140310287476 39.0
-11.751540186 -13.1289091110229 40.0
-23.189330043 -24.2517471313477 41.0
5.3867767012 6.74650573730469 42.0
-31.427511193 -33.0878295898438 43.0
85.859865353 91.6721267700195 44.0
85.859865365 96.119384765625 45.0
7.2040498429 3.69613766670227 46.0
2.5412092818 2.63412261009216 47.0
139.81114942 171.721771240234 48.0
133.25907636 140.233779907227 49.0
5.5821240968 3.48210573196411 50.0
77.140169605 73.6385726928711 51.0
91.660201744 85.2067413330078 52.0
79.152063426 81.7283782958984 53.0
79.152063426 80.6709747314453 54.0
79.152063426 78.910026550293 55.0
0 -1.31793212890625 56.0
77.299299244 74.585807800293 57.0
221.50970806 219.083526611328 58.0
248.87890055 249.451721191406 59.0
22.091582814 21.5800094604492 60.0
0 1.68747019767761 61.0
137.17687742 143.006500244141 0.0
137.17687742 143.031036376953 1.0
135.86355256 143.535263061523 2.0
131.0744823 135.766204833984 3.0
0 -2.11607456207275 4.0
0 0.342018187046051 5.0
-59.42861149 -59.2346534729004 6.0
69.705106679 70.1772918701172 7.0
65.43029471 62.4528312683105 8.0
4.2748119615 8.48783111572266 9.0
78.669161727 80.2065200805664 10.0
65.229460649 71.2585144042969 11.0
77.500548395 80.8047180175781 12.0
77.500548395 77.9000701904297 13.0
77.500548395 76.9198608398438 14.0
0 -0.64198762178421 15.0
0 0.28728449344635 16.0
0 -3.15485000610352 17.0
59.754945382 58.5015258789062 18.0
-59.754945394 -59.4383087158203 19.0
-59.754945372 -60.0649642944336 20.0
-59.754945384 -61.8801422119141 21.0
80.656571382 77.8813858032227 22.0
140.87903669 140.394561767578 23.0
106.41719675 107.357681274414 24.0
22.90053999 20.4083709716797 25.0
221.53560808 223.594223022461 26.0
221.53560808 221.987747192383 27.0
27.282645659 25.3551406860352 28.0
248.81825375 253.574676513672 29.0
248.81825375 252.408157348633 30.0
47.033013387 47.2299919128418 31.0
9.3131031344 9.18030452728271 32.0
20.901625993 20.4819431304932 33.0
20.901625993 21.1129970550537 34.0
20.901625993 20.8195400238037 35.0
9.3083381265 8.28715133666992 36.0
8.044767894 8.38281917572021 37.0
-2.5262790395e-09 -0.804296553134918 38.0
-2.5262785966e-09 0.113371789455414 39.0
-12.387792857 -14.5510091781616 40.0
-23.81592281 -27.5234241485596 41.0
4.9278776077 3.5162615776062 42.0
-31.108913619 -34.7394485473633 43.0
92.052766479 91.3064651489258 44.0
92.05276649 90.2343292236328 45.0
6.7273095346 3.00315618515015 46.0
2.6568330431 1.83220052719116 47.0
137.17687743 136.681884765625 48.0
131.07448231 127.542411804199 49.0
0 3.5438928604126 50.0
69.705106686 70.9862594604492 51.0
78.669161734 84.2529830932617 52.0
77.500548402 81.0479965209961 53.0
77.500548402 80.9212951660156 54.0
77.500548402 79.4541549682617 55.0
0 -0.0822644829750061 56.0
80.656571391 80.3950500488281 57.0
221.53560809 218.26057434082 58.0
248.81825375 246.469757080078 59.0
20.901626002 19.9398994445801 60.0
0 2.89454770088196 61.0
137.17687742 143.006271362305 0.0
137.17687742 143.039321899414 1.0
135.86355256 143.477111816406 2.0
131.0744823 135.658432006836 3.0
0 -1.99662303924561 4.0
0 0.341170310974121 5.0
-59.42861149 -59.2195014953613 6.0
69.705106679 70.1771087646484 7.0
65.43029471 62.4528312683105 8.0
4.2748119615 8.48783111572266 9.0
78.669161727 80.2065200805664 10.0
65.229460649 71.2585144042969 11.0
77.500548395 80.8047180175781 12.0
77.500548395 77.9000701904297 13.0
77.500548395 76.9198608398438 14.0
0 -0.64198762178421 15.0
0 0.28728449344635 16.0
0 -3.15485000610352 17.0
59.754945382 58.5015258789062 18.0
-59.754945394 -59.4383087158203 19.0
-59.754945372 -60.0649642944336 20.0
-59.754945384 -61.8801422119141 21.0
80.656571382 77.8813858032227 22.0
140.87903669 140.394561767578 23.0
106.41719675 107.357681274414 24.0
22.90053999 20.4083709716797 25.0
221.53560808 223.594223022461 26.0
221.53560808 221.987747192383 27.0
27.282645659 25.3551406860352 28.0
248.81825375 253.574676513672 29.0
248.81825375 252.408157348633 30.0
47.033013387 47.2299919128418 31.0
9.3131031344 9.18030452728271 32.0
20.901625993 20.4819431304932 33.0
20.901625993 21.1129970550537 34.0
20.901625993 20.8195400238037 35.0
9.3083381265 8.28715133666992 36.0
8.044767894 8.38281917572021 37.0
-2.5262790395e-09 -0.804296553134918 38.0
-2.5262785966e-09 0.113371789455414 39.0
-12.387792857 -14.5510091781616 40.0
-23.81592281 -27.5234241485596 41.0
4.9278776077 3.5162615776062 42.0
-31.108913619 -34.7394485473633 43.0
92.052766479 91.3064651489258 44.0
92.05276649 90.2343292236328 45.0
6.7273095346 3.00315618515015 46.0
2.6568330431 1.83220052719116 47.0
137.17687743 136.686645507812 48.0
131.07448231 127.367897033691 49.0
0 3.56870460510254 50.0
69.705106686 70.9871215820312 51.0
78.669161734 84.2529830932617 52.0
77.500548402 81.0479965209961 53.0
77.500548402 80.9212951660156 54.0
77.500548402 79.4541549682617 55.0
0 -0.0822644829750061 56.0
80.656571391 80.3950500488281 57.0
221.53560809 218.26057434082 58.0
248.81825375 246.469757080078 59.0
20.901626002 19.9398994445801 60.0
0 2.89454770088196 61.0
168.90515651609 159.122817993164 0.0
168.90515651609 166.531951904297 1.0
167.43504021 171.66389465332 2.0
162.64597048 168.984817504883 3.0
0 4.72804737091064 4.0
-9.16977491487e-09 6.43109130859375 5.0
-68.588820658 -64.9397583007812 6.0
79.2140531718899 76.6658477783203 7.0
56.7458563138899 45.9822425842285 8.0
22.4681968568899 19.6554565429688 9.0
90.8786468158899 88.815055847168 10.0
77.3633989458899 78.3051834106445 11.0
76.6965648058899 82.0606460571289 12.0
76.6965648058899 80.7486343383789 13.0
76.6965648058899 80.6511077880859 14.0
0 -4.2552924156189 15.0
0 -1.22108054161072 16.0
-7.965374712e-09 -0.675061523914337 17.0
61.364476936849 59.2884330749512 18.0
-61.3644769508348 -60.989917755127 19.0
-61.3644769288348 -58.8710289001465 20.0
-61.3644769378348 -56.3006782531738 21.0
82.622512807 80.3920440673828 22.0
139.172072986849 142.572463989258 23.0
108.663244583561 107.339179992676 24.0
22.9426401675613 19.8159141540527 25.0
221.794585792181 229.304321289062 26.0
221.794585792181 223.401016235352 27.0
23.7874681901807 28.2870502471924 28.0
245.582053982181 252.127151489258 29.0
245.582053982181 251.419799804688 30.0
46.847469241 48.8461303710938 31.0
9.4268572214 8.66774368286133 32.0
21.2580358733342 21.5374069213867 33.0
21.2580358733342 16.5085620880127 34.0
21.258035875 19.6609516143799 35.0
9.479636628 7.60832262039185 36.0
8.2935228277 6.76836776733398 37.0
-9.623241476e-09 0.104680478572845 38.0
-9.623241477e-09 3.05262517929077 39.0
-11.0522725879174 -18.3972225189209 40.0
-24.1504029887406 -29.0594959259033 41.0
5.02482540750825 10.7464475631714 42.0
-31.9641432549082 -34.5761222839355 43.0
106.280698989908 90.3768539428711 44.0
106.280698979908 98.7340316772461 45.0
6.7489761069 9.64444637298584 46.0
2.46849683 2.90503764152527 47.0
168.90515651 168.032791137695 48.0
162.64597049 166.707244873047 49.0
0 -5.79203367233276 50.0
79.214053172 75.3565368652344 51.0
90.878646816 83.126106262207 52.0
76.696564806 81.4167556762695 53.0
76.696564806 81.5614013671875 54.0
76.696564806 80.225212097168 55.0
0 -0.590853869915009 56.0
82.622512811 80.1401290893555 57.0
221.79458579 224.469482421875 58.0
245.58205398 249.657485961914 59.0
21.258035877 19.7219791412354 60.0
0 3.48126840591431 61.0
153.54035896 156.634017944336 0.0
153.54035896 162.650131225586 1.0
152.22469149 165.504196166992 2.0
147.43561663 163.959274291992 3.0
-2.1579126107e-10 -3.7332181930542 4.0
0 2.33964252471924 5.0
-54.97369832 -61.9161567687988 6.0
64.385589912 73.1185302734375 7.0
62.135557214 50.6164741516113 8.0
2.2500326884 21.1481437683105 9.0
85.548964364 88.3032073974609 10.0
72.996807907 76.8216171264648 11.0
87.088646184 81.814453125 12.0
87.088646184 80.3844146728516 13.0
87.088646184 80.3746185302734 14.0
0 -0.884523212909698 15.0
0 -2.72604584693909 16.0
0 -3.58486890792847 17.0
59.451790441 58.0313301086426 18.0
-59.451790451 -61.025749206543 19.0
-59.451790429 -57.6452598571777 20.0
-59.451790441 -58.3103523254395 21.0
80.28465414 77.3173370361328 22.0
143.7659406 140.878707885742 23.0
110.66066673 99.9706649780273 24.0
22.121322261 17.6918830871582 25.0
224.05059475 228.622421264648 26.0
224.05059475 225.746597290039 27.0
24.099693007 24.2206344604492 28.0
248.15028777 261.557800292969 29.0
248.15028777 256.305419921875 30.0
46.554673613 48.1017951965332 31.0
9.3929642438 9.86105442047119 32.0
20.832863689 16.9543838500977 33.0
20.832863689 18.2218990325928 34.0
20.832863689 18.0034637451172 35.0
9.152731159 12.3382711410522 36.0
8.0232986828 7.12056064605713 37.0
0 -2.50459575653076 38.0
0 -3.43044519424438 39.0
-11.248320913 -11.3123111724854 40.0
-23.708397604 -25.0745277404785 41.0
5.5423207026 4.54512882232666 42.0
-31.824503961 -34.5991401672363 43.0
80.290016502 84.5041885375977 44.0
80.290016515 85.0938949584961 45.0
6.8894830881 7.41545677185059 46.0
2.3526289764 2.72590899467468 47.0
153.54035897 162.369812011719 48.0
147.43561664 148.381958007812 49.0
0 3.94319248199463 50.0
64.385589922 75.5794219970703 51.0
85.548964374 91.4917602539062 52.0
87.088646194 81.31298828125 53.0
87.088646194 76.5849838256836 54.0
87.088646194 78.904670715332 55.0
0 3.55851602554321 56.0
80.28465415 79.340934753418 57.0
224.05059476 223.237777709961 58.0
248.15028778 246.381652832031 59.0
20.832863699 22.4873046875 60.0
0 3.09565210342407 61.0
137.17687742 143.006103515625 0.0
137.17687742 143.046401977539 1.0
135.86355256 143.427612304688 2.0
131.0744823 135.567276000977 3.0
0 -1.89350032806396 4.0
0 0.340449452400208 5.0
-59.42861149 -59.206615447998 6.0
69.705106679 70.1769332885742 7.0
65.43029471 62.4528312683105 8.0
4.2748119615 8.48783111572266 9.0
78.669161727 80.2065200805664 10.0
65.229460649 71.2585144042969 11.0
77.500548395 80.8047180175781 12.0
77.500548395 77.9000701904297 13.0
77.500548395 76.9198608398438 14.0
0 -0.64198762178421 15.0
0 0.28728449344635 16.0
0 -3.15485000610352 17.0
59.754945382 58.5015258789062 18.0
-59.754945394 -59.4383087158203 19.0
-59.754945372 -60.0649642944336 20.0
-59.754945384 -61.8801422119141 21.0
80.656571382 77.8813858032227 22.0
140.87903669 140.394561767578 23.0
106.41719675 107.357681274414 24.0
22.90053999 20.4083709716797 25.0
221.53560808 223.594223022461 26.0
221.53560808 221.987747192383 27.0
27.282645659 25.3551406860352 28.0
248.81825375 253.574676513672 29.0
248.81825375 252.408157348633 30.0
47.033013387 47.2299919128418 31.0
9.3131031344 9.18030452728271 32.0
20.901625993 20.4819431304932 33.0
20.901625993 21.1129970550537 34.0
20.901625993 20.8195400238037 35.0
9.3083381265 8.28715133666992 36.0
8.044767894 8.38281917572021 37.0
-2.5262790395e-09 -0.804296553134918 38.0
-2.5262785966e-09 0.113371789455414 39.0
-12.387792857 -14.5510091781616 40.0
-23.81592281 -27.5234241485596 41.0
4.9278776077 3.5162615776062 42.0
-31.108913619 -34.7394485473633 43.0
92.052766479 91.3064651489258 44.0
92.05276649 90.2343292236328 45.0
6.7273095346 3.00315618515015 46.0
2.6568330431 1.83220052719116 47.0
137.17687743 136.690689086914 48.0
131.07448231 127.219017028809 49.0
0 3.58974075317383 50.0
69.705106686 70.9878311157227 51.0
78.669161734 84.2529830932617 52.0
77.500548402 81.0479965209961 53.0
77.500548402 80.9212951660156 54.0
77.500548402 79.4541549682617 55.0
0 -0.0822644829750061 56.0
80.656571391 80.3950500488281 57.0
221.53560809 218.26057434082 58.0
248.81825375 246.469757080078 59.0
20.901626002 19.9398994445801 60.0
0 2.89454770088196 61.0
175.985955883785 163.913772583008 0.0
175.985955883785 169.106384277344 1.0
174.83687136 169.008483886719 2.0
170.04780159 164.93229675293 3.0
0 -2.15561604499817 4.0
-5.44831445e-09 4.85976505279541 5.0
-69.9965568706364 -67.6281051635742 6.0
81.6824462742273 73.2623443603516 7.0
50.4815024252273 52.3644142150879 8.0
31.2009438492273 22.9435348510742 9.0
87.6109575212273 90.0299530029297 10.0
74.9150300602273 76.6412124633789 11.0
80.1773899862273 81.0155563354492 12.0
80.1773899862273 77.8965225219727 13.0
80.1773899862273 80.0086059570312 14.0
4.413536655e-09 -1.00556683540344 15.0
3.2297705184e-09 -0.77076929807663 16.0
-2.71324236999999e-09 -1.07371878623962 17.0
58.6667945642508 59.3933181762695 18.0
-58.666794626841 -58.8314666748047 19.0
-58.666794609841 -58.9333190917969 20.0
-58.666794583841 -58.158863067627 21.0
80.5201165578182 78.0052185058594 22.0
145.829294952306 143.579040527344 23.0
110.346702275382 111.749877929688 24.0
23.9900580533823 20.4592990875244 25.0
226.34941150757 225.940490722656 26.0
226.34941150757 225.795135498047 27.0
27.32402915557 26.0282249450684 28.0
253.67344066757 255.521575927734 29.0
253.67344066757 254.060623168945 30.0
46.6514525951342 46.4862442016602 31.0
9.28272762140731 8.72709274291992 32.0
21.8533219739091 18.7860927581787 33.0
21.8533219739091 14.5370988845825 34.0
21.8533220176301 19.9592723846436 35.0
9.17889328053384 6.82407569885254 36.0
7.91905011503384 7.11823844909668 37.0
-5.87578609999998e-09 -0.209656417369843 38.0
-5.87578609999998e-09 -3.07396459579468 39.0
-11.6120369999205 -11.663122177124 40.0
-24.246919736319 -23.8504734039307 41.0
5.36956277955608 5.73640871047974 42.0
-32.1910388923561 -34.1623840332031 43.0
100.364691104356 93.3474044799805 44.0
100.364690774993 94.7572555541992 45.0
6.91645026235474 6.85776996612549 46.0
2.36653329335474 2.46229028701782 47.0
175.98595589 166.057220458984 48.0
170.0478016 161.040756225586 49.0
0 3.36818408966064 50.0
81.682446274 77.9660339355469 51.0
87.610957522 87.2157592773438 52.0
80.177389986 80.6327209472656 53.0
80.177389986 81.2730560302734 54.0
80.177389986 79.9518661499023 55.0
0 -0.0962209105491638 56.0
80.520116558 79.2682952880859 57.0
226.34941151 222.593368530273 58.0
253.67344067 251.147918701172 59.0
21.853322018 23.1302165985107 60.0
0 2.116774559021 61.0
147.982033057583 157.456436157227 0.0
147.982033057583 151.967163085938 1.0
146.808413813287 153.377685546875 2.0
142.01934397 150.933837890625 3.0
-1.55484310406e-09 -0.673610985279083 4.0
-1.55484309999997e-09 -2.84894132614136 5.0
-64.9693040339936 -64.3301467895508 6.0
77.6547829144777 74.1872863769531 7.0
61.2786926634778 55.4030647277832 8.0
16.3760902514777 7.20984888076782 9.0
97.5261141374777 89.7391510009766 10.0
83.6472874924777 79.6450042724609 11.0
85.2981411144777 83.8642730712891 12.0
85.2981411144777 76.4624557495117 13.0
85.2981411144777 81.5436782836914 14.0
2.3328021736e-08 -2.03496789932251 15.0
2.3328021735e-08 -1.6052393913269 16.0
5.464350236e-08 -2.72059202194214 17.0
56.2518018559467 59.0783615112305 18.0
-56.2518018797166 -57.0891418457031 19.0
-56.2518018707166 -59.2854194641113 20.0
-56.2518017707166 -58.115608215332 21.0
77.1875821073724 73.3798065185547 22.0
153.454771802806 145.008895874023 23.0
115.220596133508 115.750228881836 24.0
21.7005584165075 20.8819332122803 25.0
230.642353915167 228.381210327148 26.0
230.642353915167 229.708221435547 27.0
26.8883420391666 21.3647422790527 28.0
257.530695955167 253.23681640625 29.0
257.530695955167 255.336776733398 30.0
46.7385281475469 45.537841796875 31.0
9.28724497493877 8.4355525970459 32.0
20.9357803369809 22.6376686096191 33.0
20.9357803369809 19.0790691375732 34.0
20.935780467613 20.4264030456543 35.0
8.86054244871795 8.39873600006104 36.0
7.60121916461794 6.85336446762085 37.0
2.62465480000001e-09 1.1959273815155 38.0
2.62465480000001e-09 -1.17553234100342 39.0
-11.6034306368583 -12.5643882751465 40.0
-24.5599220844463 -26.7411251068115 41.0
4.75986482786468 1.92820739746094 42.0
-31.7734218810648 -31.6524333953857 43.0
89.4696777832616 86.4044494628906 44.0
89.4696781210649 91.9600372314453 45.0
7.05809588635643 4.443039894104 46.0
2.46335994485643 2.26466846466064 47.0
147.98203306 156.095245361328 48.0
142.01934397 153.16194152832 49.0
0 -3.73428678512573 50.0
77.654782914 76.0231323242188 51.0
97.526114137 92.0755157470703 52.0
85.298141114 81.5651702880859 53.0
85.298141114 80.2131118774414 54.0
85.298141114 79.9203643798828 55.0
5.4643502359e-08 3.07315587997437 56.0
77.187582108 79.4883346557617 57.0
230.64235391 222.583419799805 58.0
257.53069595 249.491134643555 59.0
20.935780468 19.6116352081299 60.0
2.6246548239e-09 3.64096260070801 61.0
190.647378735192 161.895294189453 0.0
190.647378735192 160.695236206055 1.0
189.4975724 169.331817626953 2.0
184.70850265 181.659439086914 3.0
-4.3192167311e-10 3.52344131469727 4.0
-8.678559515e-09 15.7490911483765 5.0
-58.1348169851429 -31.1547870635986 6.0
70.1895573370002 87.4838104248047 7.0
46.0646208670002 52.0309371948242 8.0
24.1249364700002 28.5126190185547 9.0
92.8833664690002 89.747200012207 10.0
79.8768505570002 74.7976608276367 11.0
89.3858795700002 81.074577331543 12.0
89.3858795700002 82.9573669433594 13.0
89.3858795700002 81.2618026733398 14.0
2.7568023062e-09 -2.23969578742981 15.0
0 0.438403308391571 16.0
2.18829254000001e-09 -1.25363516807556 17.0
56.1767973424546 54.1891479492188 18.0
-56.1767973155003 -59.4069366455078 19.0
-56.1767972965003 -57.4609565734863 20.0
-56.1767972815003 -56.5178642272949 21.0
78.1198582272163 75.9596328735352 22.0
151.166098771455 140.546081542969 23.0
118.755770164318 111.101417541504 24.0
25.4933099443183 17.3717384338379 25.0
229.285957000385 228.89714050293 26.0
229.285957000385 218.922439575195 27.0
26.5214315953848 31.8759117126465 28.0
255.807388600385 266.027862548828 29.0
255.807388600385 256.318328857422 30.0
47.2672013870389 45.4459190368652 31.0
9.47576333448043 9.64591884613037 32.0
21.9430609464287 20.3555603027344 33.0
21.9430609464287 21.4529819488525 34.0
21.9430609783794 20.6161727905273 35.0
9.26390241817459 9.9046049118042 36.0
7.90091200137459 6.74843502044678 37.0
-6.76092301999999e-09 1.12744498252869 38.0
-6.76092301999999e-09 -3.86343288421631 39.0
-11.3906696547501 -14.4837303161621 40.0
-24.5414105528136 -25.6143016815186 41.0
5.40322002876681 8.80830192565918 42.0
-32.6581408011668 -33.6052436828613 43.0
90.5404341372887 93.673469543457 44.0
90.5404342831668 88.9677505493164 45.0
6.3801672475 8.5949535369873 46.0
2.31548583 3.02157020568848 47.0
190.64737873 161.12158203125 48.0
184.70850266 171.928924560547 49.0
0 3.43359184265137 50.0
70.189557337 72.942626953125 51.0
92.883366469 92.9441528320312 52.0
89.38587957 81.1010208129883 53.0
89.38587957 81.7617340087891 54.0
89.38587957 79.0190811157227 55.0
2.1882925456e-09 4.86723518371582 56.0
78.119858227 78.5853805541992 57.0
229.285957 219.896591186523 58.0
255.8073886 252.756561279297 59.0
21.943060978 15.9973011016846 60.0
0 1.78565287590027 61.0
212.391738827015 153.841064453125 0.0
212.391738827015 156.152008056641 1.0
211.04319376 162.609252929688 2.0
205.5785585 183.532791137695 3.0
18.404050808 -11.9846677780151 4.0
18.4040508170453 16.5005359649658 5.0
-61.3077697580453 -106.727119445801 6.0
74.2237211591586 101.702270507812 7.0
47.5084168701586 49.3908500671387 8.0
26.7153042891586 26.0035495758057 9.0
73.6279026801586 82.7762222290039 10.0
62.1474872191586 58.7497711181641 11.0
78.3848871701586 68.9429779052734 12.0
78.3848871701586 72.1487426757812 13.0
78.3848871701585 84.5159912109375 14.0
4.213840709e-10 7.65522480010986 15.0
4.2138407099e-10 3.21506071090698 16.0
5.24329936999999e-09 4.69373941421509 17.0
56.7790807251774 59.1677894592285 18.0
-56.7790806952379 -56.4286880493164 19.0
-56.7790806772379 -56.0826797485352 20.0
-56.7790806622379 -68.0509414672852 21.0
79.0602866060699 71.6340713500977 22.0
149.501502613177 134.62646484375 23.0
119.756521836472 102.935264587402 24.0
24.5677496374717 26.2532119750977 25.0
228.561789214029 220.424560546875 26.0
228.561789214029 206.224975585938 27.0
27.4961151460291 28.7196178436279 28.0
256.057904364029 246.56787109375 29.0
256.057904364029 273.834167480469 30.0
45.391893868001 38.7327346801758 31.0
8.76348561509971 6.41286516189575 32.0
22.2812059441359 13.2257738113403 33.0
22.2812059441359 20.2075481414795 34.0
22.2812059764684 22.1812534332275 35.0
8.66141742863524 15.2447719573975 36.0
7.34531140923524 4.93151473999023 37.0
-6.99365361999996e-09 -6.4220929145813 38.0
-6.99365360999999e-09 -2.56813478469849 39.0
-12.1264598581189 -3.07913827896118 40.0
-26.0813195914196 -26.7585544586182 41.0
5.60695360219884 0.621493518352509 42.0
-34.4708309687988 -32.5315818786621 43.0
84.1913674207981 89.0402679443359 44.0
84.1913670312712 90.0791091918945 45.0
6.6020739941 6.19962120056152 46.0
2.5827328274 1.97778654098511 47.0
212.39173882 158.674331665039 48.0
205.57855851 163.197311401367 49.0
18.404050817 2.99577260017395 50.0
74.223721159 75.4420852661133 51.0
73.62790268 83.2851333618164 52.0
78.38488717 75.5351104736328 53.0
78.38488717 68.6956787109375 54.0
78.38488717 81.1668319702148 55.0
5.2432993646e-09 5.09521484375 56.0
79.060286606 76.9335174560547 57.0
228.56178922 207.424560546875 58.0
256.05790436 258.141723632812 59.0
22.281205976 22.5056571960449 60.0
0 6.23766803741455 61.0
152.01511135 162.740570068359 0.0
152.01511135 168.15510559082 1.0
150.73624439 159.719116210938 2.0
145.94717467 143.296997070312 3.0
0 -0.225555717945099 4.0
-1.3141431381e-09 -1.60792803764343 5.0
-61.351926906 -65.0940322875977 6.0
73.227952086 72.3794479370117 7.0
51.97784765 50.6285438537598 8.0
21.250104435 23.1153469085693 9.0
90.12973698 85.8495178222656 10.0
77.301623962 74.9970550537109 11.0
82.87862434 79.1162033081055 12.0
82.87862434 78.6806869506836 13.0
82.87862434 80.2984237670898 14.0
0 -0.18605238199234 15.0
0 0.416769862174988 16.0
-5.6875996278e-09 -1.38802456855774 17.0
59.796153207 57.8106422424316 18.0
-59.7961532296897 -58.6203651428223 19.0
-59.7961532076897 -61.0177040100098 20.0
-59.7961532186897 -57.3818244934082 21.0
82.21616587 78.9521865844727 22.0
146.35352463 143.477645874023 23.0
111.47855424 109.745254516602 24.0
21.472918849 19.1866321563721 25.0
228.569690518992 223.430084228516 26.0
228.569690518992 220.779678344727 27.0
24.9196510329921 28.3873729705811 28.0
253.489341548992 250.355941772461 29.0
253.489341548992 250.91374206543 30.0
46.770680272 46.1404495239258 31.0
9.2903104244 8.69324684143066 32.0
22.420012657 20.2706165313721 33.0
22.420012657 18.0167675018311 34.0
22.420012654 21.0888233184814 35.0
9.6028777931 8.00218200683594 36.0
8.2744029474 7.47547245025635 37.0
-9.7306891187e-09 -2.50349807739258 38.0
-9.7306891154e-09 -2.84876680374146 39.0
-12.512212937 -13.2917108535767 40.0
-24.94288347 -27.4303588867188 41.0
4.9331253561 3.24215149879456 42.0
-32.44445226 -32.925838470459 43.0
96.15539432 92.8194274902344 44.0
96.155394318 93.8757858276367 45.0
6.7588287829 6.33729934692383 46.0
2.4784149669 2.26625633239746 47.0
152.01511135 169.551361083984 48.0
145.94717468 144.236389160156 49.0
0 -1.02974104881287 50.0
73.227952086 74.794189453125 51.0
90.129736981 89.9113235473633 52.0
82.878624341 80.9114456176758 53.0
82.878624341 78.8574447631836 54.0
82.878624341 80.5269470214844 55.0
0 1.56181168556213 56.0
82.216165877 79.1684265136719 57.0
228.56969052 219.438079833984 58.0
253.48934155 249.097885131836 59.0
22.42001266 24.9486141204834 60.0
0 1.63914787769318 61.0
166.92464535 158.599151611328 0.0
166.92464535 147.994705200195 1.0
165.87453414 135.021545410156 2.0
161.08546345 163.482192993164 3.0
0 4.50570106506348 4.0
0 12.3102073669434 5.0
-62.994949176 -57.1477088928223 6.0
70.7556407624234 71.6711044311523 7.0
50.9264333854233 49.046760559082 8.0
19.8292073774233 6.32955551147461 9.0
81.0410764364234 84.5661468505859 10.0
66.8364521564233 66.0532684326172 11.0
83.1701834614233 73.1791152954102 12.0
83.1701834614233 84.5976257324219 13.0
83.1701834614233 74.6877670288086 14.0
0 -4.56651163101196 15.0
0 0.643130958080292 16.0
-9.49209839999999e-11 4.12973213195801 17.0
60.690582972 48.6859664916992 18.0
-60.690582990635 -60.5208930969238 19.0
-60.690582968635 -58.1220283508301 20.0
-60.690582979635 -64.3530197143555 21.0
81.286060358 81.422248840332 22.0
129.36456076 148.115982055664 23.0
96.213201018 108.916793823242 24.0
19.284544804 20.9009742736816 25.0
210.650621126642 221.899459838867 26.0
210.650621126642 217.501052856445 27.0
27.0471453746423 28.5838508605957 28.0
237.697766496642 266.224853515625 29.0
237.697766496642 249.328659057617 30.0
46.775123688 46.4126968383789 31.0
9.2600397946 8.8416223526001 32.0
20.595477386 15.0750246047974 33.0
20.595477386 20.6151237487793 34.0
20.595477381 19.7769069671631 35.0
9.1455036793 9.17498874664307 36.0
7.7203932114 7.43995475769043 37.0
-9.508426294e-09 -2.24972581863403 38.0
-9.5084262938e-09 -2.73203110694885 39.0
-11.746081492 -18.3729400634766 40.0
-23.416886055 -27.3278503417969 41.0
5.428921787 6.39367961883545 42.0
-31.085185306 -32.4104537963867 43.0
90.719675113 86.2834548950195 44.0
90.719675108 94.6495208740234 45.0
5.9850864964 7.37868213653564 46.0
2.2774254241 2.5459201335907 47.0
166.92464536 137.365997314453 48.0
161.08546346 126.97135925293 49.0
0 -12.3505821228027 50.0
70.755640762 78.9581680297852 51.0
81.041076437 80.8466339111328 52.0
83.170183462 76.2148513793945 53.0
83.170183462 80.8953247070312 54.0
83.170183462 78.5184326171875 55.0
0 8.51326751708984 56.0
81.286060365 79.9564590454102 57.0
210.65062113 211.277969360352 58.0
237.6977665 249.14372253418 59.0
20.595477387 21.4344730377197 60.0
0 1.38015127182007 61.0
209.75326329 167.277633666992 0.0
209.75326329 168.525314331055 1.0
208.43933093 156.866806030273 2.0
202.97469574 185.758682250977 3.0
13.728284255 -1.07649993896484 4.0
13.728284255 4.46328258514404 5.0
-56.95207807 -74.5553512573242 6.0
70.066394197 68.254508972168 7.0
41.555360252 50.678840637207 8.0
28.511033935 11.3005437850952 9.0
83.28096054 87.8781661987305 10.0
70.307423857 75.0391693115234 11.0
75.836651013 80.1827392578125 12.0
75.836651013 76.0610809326172 13.0
75.836651013 77.7435760498047 14.0
0 1.94851970672607 15.0
0 -1.71258807182312 16.0
-7.5911282875e-14 2.92896032333374 17.0
59.305429194 59.1812934875488 18.0
-59.305429214 -58.7015190124512 19.0
-59.305429192 -58.6372375488281 20.0
-59.305429204 -66.0722885131836 21.0
80.522346428 74.4707641601562 22.0
149.43437571 139.530868530273 23.0
111.19272854 111.497436523438 24.0
22.256610643 21.5800495147705 25.0
229.95672215 226.459075927734 26.0
229.95672215 227.606903076172 27.0
28.022503166 23.5910911560059 28.0
257.97922532 243.121612548828 29.0
257.97922532 251.300827026367 30.0
47.062569824 46.3496932983398 31.0
9.40674248 9.49158668518066 32.0
21.216917214 20.4482746124268 33.0
21.216917214 14.5653305053711 34.0
21.216917214 20.6590099334717 35.0
9.2094950057 8.82297611236572 36.0
8.07706796 6.85976552963257 37.0
-1.2115874183e-13 -4.34025955200195 38.0
-1.2115874183e-13 0.262088596820831 39.0
-11.908943733 -17.4776306152344 40.0
-25.119745562 -26.8792362213135 41.0
4.9751376311 9.49308204650879 42.0
-32.671583021 -36.1850128173828 43.0
98.451009504 94.940055847168 44.0
98.451009517 93.7212753295898 45.0
6.8054452948 4.84517860412598 46.0
2.8386070371 1.68762540817261 47.0
209.7532633 168.668914794922 48.0
202.97469575 171.3173828125 49.0
13.728284265 2.75528764724731 50.0
70.066394207 75.8960952758789 51.0
83.28096055 84.5548934936523 52.0
75.836651023 81.7414245605469 53.0
75.836651023 74.7881240844727 54.0
75.836651023 80.4707794189453 55.0
0 0.88787168264389 56.0
80.522346438 79.0495452880859 57.0
229.95672216 218.256225585938 58.0
257.97922533 242.452713012695 59.0
21.216917224 17.4710559844971 60.0
0 2.72514605522156 61.0
172.92099326 165.037796020508 0.0
172.92099326 168.324829101562 1.0
171.65917337 163.943984985352 2.0
166.87010305 167.634353637695 3.0
0 -1.51656866073608 4.0
0 0.710752546787262 5.0
-61.433736469 -60.8258743286133 6.0
71.64988361 72.4676055908203 7.0
55.511403747 52.718189239502 8.0
16.138479854 21.7563076019287 9.0
90.964224067 89.0090103149414 10.0
77.964670603 76.6576309204102 11.0
83.204517574 82.6695938110352 12.0
83.204517574 81.0991897583008 13.0
83.204517574 79.8214645385742 14.0
0 0.41631007194519 15.0
0 -0.123291730880737 16.0
0 -2.12039017677307 17.0
57.081054085 58.9081230163574 18.0
-57.081054095 -59.4183464050293 19.0
-57.081054074 -60.1552925109863 20.0
-57.081054086 -59.1037101745605 21.0
78.410579862 77.8158569335938 22.0
147.10496557 142.309036254883 23.0
113.4704548 109.823043823242 24.0
23.550208112 19.9987602233887 25.0
225.51554545 222.458511352539 26.0
225.51554545 224.446701049805 27.0
26.183374286 28.3237228393555 28.0
251.69891974 256.030578613281 29.0
251.69891974 255.153915405273 30.0
46.939969802 45.646183013916 31.0
9.3011447032 8.80996894836426 32.0
21.329525767 20.8150081634521 33.0
21.329525767 17.6436882019043 34.0
21.329525767 21.2434539794922 35.0
9.6204980995 9.45438385009766 36.0
8.3532839566 7.84531259536743 37.0
-4.929434163e-10 -1.29540920257568 38.0
-4.929434162e-10 -1.99346613883972 39.0
-11.755164802 -12.866756439209 40.0
-24.407328107 -24.8533611297607 41.0
5.0972306267 6.22848129272461 42.0
-31.964032156 -34.794548034668 43.0
95.465201279 93.9734420776367 44.0
95.465201291 94.4564437866211 45.0
6.8537856479 7.44746017456055 46.0
2.6103753436 1.20440483093262 47.0
172.92099327 165.998001098633 48.0
166.87010306 161.95622253418 49.0
0 1.02102017402649 50.0
71.64988362 74.4577789306641 51.0
90.964224077 89.7902984619141 52.0
83.204517583 81.0613327026367 53.0
83.204517583 80.7883377075195 54.0
83.204517583 78.6458358764648 55.0
0 -0.214804828166962 56.0
78.410579872 78.7178726196289 57.0
225.51554545 220.415634155273 58.0
251.69891975 249.248474121094 59.0
21.329525777 22.2093029022217 60.0
0 2.19265365600586 61.0
174.563944512647 161.488342285156 0.0
174.563944512647 163.452606201172 1.0
173.42738984 160.471694946289 2.0
168.6383201 163.194671630859 3.0
0 3.02020740509033 4.0
-5.927963602e-09 4.47275590896606 5.0
-55.1721014587397 -48.8631286621094 6.0
64.497642085589 90.4137878417969 7.0
48.162924564589 51.4999237060547 8.0
16.334717521589 23.567907333374 9.0
86.568717397589 88.5119323730469 10.0
73.380220323589 77.2399749755859 11.0
76.786569349589 79.8520050048828 12.0
76.786569349589 83.9617156982422 13.0
76.786569349589 81.7519683837891 14.0
4.303224042e-10 -2.65774583816528 15.0
0 -0.466832220554352 16.0
-3.29194132000002e-09 1.55909276008606 17.0
61.8911283464122 58.9040832519531 18.0
-61.8911283328834 -59.9628524780273 19.0
-61.8911283138834 -56.6428260803223 20.0
-61.8911283048835 -59.6584625244141 21.0
82.6656411700781 74.011344909668 22.0
154.399872617412 141.554840087891 23.0
118.517007216765 112.212196350098 24.0
25.1328476997652 21.514066696167 25.0
237.065513785295 232.400848388672 26.0
237.065513785295 224.842498779297 27.0
29.353383245295 28.3377513885498 28.0
266.418897035295 254.77978515625 29.0
266.418897035295 254.502166748047 30.0
46.6654262528997 45.9960289001465 31.0
9.27911179291156 9.52303504943848 32.0
20.7745128652191 21.149076461792 33.0
20.7745128652191 20.5866107940674 34.0
20.7745128909214 16.6605052947998 35.0
9.2980310417223 7.10479736328125 36.0
7.9714331706223 6.45343351364136 37.0
-7.90629667999999e-09 1.31264042854309 38.0
-7.90629667999999e-09 -4.07684564590454 39.0
-10.5807658844417 -12.7510681152344 40.0
-23.6276654320113 -22.9334850311279 41.0
4.79256323172414 5.16328620910645 42.0
-30.8643489788241 -35.1371955871582 43.0
85.5986928048241 85.0373992919922 44.0
85.598692653061 89.9132843017578 45.0
7.1182378535 9.20614051818848 46.0
2.4230751515 2.67674875259399 47.0
174.56394451 167.161712646484 48.0
168.63832011 149.14387512207 49.0
0 8.89645671844482 50.0
64.497642086 77.644401550293 51.0
86.568717397 89.9904174804688 52.0
76.786569349 80.2107620239258 53.0
76.786569349 81.0089111328125 54.0
76.786569349 79.4894866943359 55.0
0 1.90742945671082 56.0
82.66564117 80.8617935180664 57.0
237.06551379 217.191116333008 58.0
266.41889703 251.266067504883 59.0
20.774512891 16.2351665496826 60.0
0 1.62354445457458 61.0
179.25232168 163.998733520508 0.0
179.25232168 165.307373046875 1.0
177.92960652 160.226837158203 2.0
173.1405368 169.373291015625 3.0
0 -0.704248011112213 4.0
-4.7659178585e-10 5.98341274261475 5.0
-62.345393454 -60.3801040649414 6.0
73.484608509 77.7742691040039 7.0
48.629293876 52.6692962646484 8.0
24.85531463 23.842191696167 9.0
82.113999887 85.3835906982422 10.0
69.33096806 77.5535278320312 11.0
78.021719963 83.7888259887695 12.0
78.021719963 78.2337799072266 13.0
78.021719963 78.0484008789062 14.0
0 -0.507340848445892 15.0
0 -0.061289370059967 16.0
-4.0107191721e-09 1.26573145389557 17.0
56.89811104 58.2625160217285 18.0
-56.8981110642561 -59.4961204528809 19.0
-56.8981110422561 -60.060489654541 20.0
-56.8981110532561 -61.0581893920898 21.0
78.999212302 77.0507354736328 22.0
143.27346793 139.825286865234 23.0
108.54837405 111.342636108398 24.0
22.359366677 20.5602684020996 25.0
222.27268025 226.405319213867 26.0
222.27268025 226.170272827148 27.0
27.958197075 24.0431079864502 28.0
250.23087732 252.221939086914 29.0
250.23087732 253.366088867188 30.0
46.574721696 45.8532524108887 31.0
9.3042430108 9.25042629241943 32.0
22.101101254 22.5838832855225 33.0
22.101101254 18.2276439666748 34.0
22.101101251 20.1819248199463 35.0
9.608971595 8.35261631011963 36.0
8.328737502 7.13340425491333 37.0
-7.2195669927e-09 1.16330933570862 38.0
-7.2195669966e-09 2.72856307029724 39.0
-11.715280436 -15.7117128372192 40.0
-24.570662382 -22.7517738342285 41.0
5.2685986261 6.18556070327759 42.0
-32.432891909 -34.0849304199219 43.0
90.553968872 90.4039001464844 44.0
90.553968873 91.1977996826172 45.0
7.1516178957 7.58145904541016 46.0
2.7536288416 2.84569835662842 47.0
179.25232169 164.72395324707 48.0
173.14053681 160.746139526367 49.0
0 2.21334743499756 50.0
73.484608511 76.7437973022461 51.0
82.11399989 84.5780715942383 52.0
78.021719965 81.4580230712891 53.0
78.021719965 81.5313491821289 54.0
78.021719965 80.4742660522461 55.0
0 2.59196257591248 56.0
78.99921231 80.4106216430664 57.0
222.27268025 221.393020629883 58.0
250.23087733 246.987670898438 59.0
22.101101258 20.1114158630371 60.0
0 2.46784090995789 61.0
165.35585942 166.146713256836 0.0
165.35585942 163.769515991211 1.0
164.13882377 157.752502441406 2.0
159.34975405 158.938583374023 3.0
-1.6038711469e-13 4.4372992515564 4.0
-2.0692320027e-13 0.893424570560455 5.0
-70.527160423 -64.4718704223633 6.0
81.190871003 73.4829635620117 7.0
49.433266094 50.7289924621582 8.0
31.7576049 26.9154090881348 9.0
93.026929153 89.0005340576172 10.0
80.66219421 76.1218109130859 11.0
84.20306809 82.8613052368164 12.0
84.20306809 78.4818649291992 13.0
84.20306809 79.0971221923828 14.0
0 -1.97326350212097 15.0
0 -0.206106007099152 16.0
-9.8865157344e-13 0.0193739533424377 17.0
55.803230373 58.3192024230957 18.0
-55.803230393 -59.5762138366699 19.0
-55.803230371 -59.2066688537598 20.0
-55.803230384 -57.5769691467285 21.0
77.720026466 76.4954681396484 22.0
140.37203364 141.661087036133 23.0
105.15840454 109.751754760742 24.0
22.085270054 20.4219837188721 25.0
218.09206012 228.355743408203 26.0
218.09206012 226.287658691406 27.0
27.498065237 24.6006965637207 28.0
245.59012536 258.279693603516 29.0
245.59012536 257.355010986328 30.0
46.613954445 46.4101181030273 31.0
9.4041898947 8.53684139251709 32.0
21.916796072 18.5543308258057 33.0
21.916796072 23.1406879425049 34.0
21.916796072 20.3922443389893 35.0
10.100345406 7.27163028717041 36.0
8.7899596816 7.29723119735718 37.0
-1.0371448166e-13 1.12825620174408 38.0
-1.0371448153e-13 -2.0894181728363 39.0
-11.527318203 -9.6851224899292 40.0
-24.533526655 -25.4112186431885 41.0
5.3822841084 2.70980763435364 42.0
-32.309422838 -32.9041328430176 43.0
90.705438536 89.62353515625 44.0
90.70543855 91.6171951293945 45.0
6.6968893179 6.74580669403076 46.0
2.3987043784 2.82680153846741 47.0
165.35585943 164.871994018555 48.0
159.34975406 155.822738647461 49.0
0 3.12990713119507 50.0
81.190871013 76.6820297241211 51.0
93.026929163 85.2289428710938 52.0
84.2030681 81.6374588012695 53.0
84.2030681 80.8560943603516 54.0
84.2030681 80.1999130249023 55.0
0 1.39121532440186 56.0
77.720026476 79.9969863891602 57.0
218.09206013 223.366806030273 58.0
245.59012537 248.666473388672 59.0
21.916796082 20.858793258667 60.0
0 2.85544800758362 61.0
162.691484849725 158.741302490234 0.0
162.691484849725 161.358184814453 1.0
161.49122414 163.513137817383 2.0
156.70215441 168.851837158203 3.0
0 -0.450048744678497 4.0
-8.7976841104e-09 2.20378470420837 5.0
-69.7622129369589 -64.5264053344727 6.0
80.7799501148563 73.0988388061523 7.0
50.9440192178563 52.4340133666992 8.0
29.8359308968563 21.121395111084 9.0
91.2813247838563 86.7498779296875 10.0
78.7191235398563 74.9923324584961 11.0
82.4120558788563 83.9982528686523 12.0
82.4120558788563 80.8315658569336 13.0
82.4120558788563 81.392448425293 14.0
0 -0.616707861423492 15.0
0 -0.346517860889435 16.0
-8.368547484e-09 0.753006517887115 17.0
59.5168818108307 59.4330787658691 18.0
-59.5168818232845 -59.4487495422363 19.0
-59.5168818022845 -58.7132110595703 20.0
-59.5168818092845 -59.2624931335449 21.0
80.448872179 78.1264877319336 22.0
143.235451210831 141.671279907227 23.0
107.177747376038 110.28971862793 24.0
21.9195844870384 22.1619701385498 25.0
223.68432338945 223.083343505859 26.0
223.68432338945 221.898193359375 27.0
25.4718589214496 26.221794128418 28.0
249.15618230945 252.490997314453 29.0
249.15618230945 254.335754394531 30.0
46.988160683764 47.065372467041 31.0
9.3887762128 9.12158203125 32.0
20.9319903698768 18.3112354278564 33.0
20.9319903698768 19.6860103607178 34.0
20.9319903758815 21.305269241333 35.0
9.3896297608 8.9376277923584 36.0
8.1251510418 7.73391819000244 37.0
-9.436810703e-09 0.0641846656799316 38.0
-9.436810703e-09 -2.79352784156799 39.0
-11.4685039276422 -11.9934644699097 40.0
-24.4747206841303 -22.7480621337891 41.0
5.05244618161359 7.39779472351074 42.0
-32.0464420067136 -35.1409339904785 43.0
102.952566970714 95.9430084228516 44.0
102.952566960714 95.6218490600586 45.0
7.1179878574 7.83101749420166 46.0
2.5497763753 2.2245728969574 47.0
162.69148485 162.720642089844 48.0
156.70215442 158.841094970703 49.0
0 -0.440704643726349 50.0
80.779950115 77.9616546630859 51.0
91.281324784 89.9453506469727 52.0
82.412055878 81.404411315918 53.0
82.412055878 81.3535003662109 54.0
82.412055878 80.6638488769531 55.0
0 -0.917096197605133 56.0
80.44887218 80.6970443725586 57.0
223.68432339 220.511962890625 58.0
249.15618231 247.941421508789 59.0
20.931990376 22.6764907836914 60.0
0 2.38331913948059 61.0
135.32247141 157.941452026367 0.0
135.32247141 159.656768798828 1.0
134.09793078 163.993530273438 2.0
129.30886106 150.735626220703 3.0
0 1.89920878410339 4.0
-2.5833782583e-10 0.32428377866745 5.0
-70.786186413 -74.8748931884766 6.0
82.870709241 76.2684631347656 7.0
52.070600402 58.136589050293 8.0
30.800108836 18.562313079834 9.0
87.855568524 87.7768478393555 10.0
75.268909365 78.9088668823242 11.0
76.872778884 86.1163940429688 12.0
76.872778884 80.5132369995117 13.0
76.872778884 77.8519134521484 14.0
0 -2.60725736618042 15.0
0 -1.53382349014282 16.0
-3.3628181586e-09 2.30831623077393 17.0
51.879001735 58.3984489440918 18.0
-51.8790017591731 -58.730655670166 19.0
-51.8790017371731 -57.7160186767578 20.0
-51.8790017491731 -58.5861473083496 21.0
73.710401443 74.1280212402344 22.0
145.90703336 142.199462890625 23.0
108.1605972 101.446365356445 24.0
24.140446051 22.8856220245361 25.0
219.61743482 222.981704711914 26.0
219.61743482 223.322601318359 27.0
29.565792709 25.3659591674805 28.0
249.18322753 248.451721191406 29.0
249.18322753 249.054840087891 30.0
46.477397147 47.5719375610352 31.0
9.3858057749 8.07505702972412 32.0
21.831399699 16.4623908996582 33.0
21.831399699 19.7391948699951 34.0
21.831399696 21.6652793884277 35.0
8.8627961862 4.20557641983032 36.0
7.5830971296 7.75957012176514 37.0
-6.3574432908e-09 -1.6573634147644 38.0
-6.357443292e-09 0.347078323364258 39.0
-11.434613608 -16.2675075531006 40.0
-23.698630809 -24.7757053375244 41.0
4.8243761285 1.20134460926056 42.0
-30.931534443 -29.8115844726562 43.0
90.822141487 91.9838485717773 44.0
90.822141506 91.0496978759766 45.0
7.3259602944 5.21848344802856 46.0
2.478270858 1.86680221557617 47.0
135.32247141 167.244094848633 48.0
129.30886107 149.930450439453 49.0
0 0.0441086888313293 50.0
82.870709244 75.6241455078125 51.0
87.855568527 89.9070434570312 52.0
76.872778887 81.6404113769531 53.0
76.872778887 81.6830520629883 54.0
76.872778887 81.1380157470703 55.0
0 0.922041714191437 56.0
73.710401451 79.7252883911133 57.0
219.61743482 222.180755615234 58.0
249.18322753 247.911590576172 59.0
21.831399704 19.8793277740479 60.0
0 2.67515969276428 61.0
137.17687742 143.006408691406 0.0
137.17687742 143.034271240234 1.0
135.86355256 143.512969970703 2.0
131.0744823 135.724792480469 3.0
0 -2.07063269615173 4.0
0 0.341697096824646 5.0
-59.42861149 -59.2288932800293 6.0
69.705106679 70.1772079467773 7.0
65.43029471 62.4528312683105 8.0
4.2748119615 8.48783111572266 9.0
78.669161727 80.2065200805664 10.0
65.229460649 71.2585144042969 11.0
77.500548395 80.8047180175781 12.0
77.500548395 77.9000701904297 13.0
77.500548395 76.9198608398438 14.0
0 -0.64198762178421 15.0
0 0.28728449344635 16.0
0 -3.15485000610352 17.0
59.754945382 58.5015258789062 18.0
-59.754945394 -59.4383087158203 19.0
-59.754945372 -60.0649642944336 20.0
-59.754945384 -61.8801422119141 21.0
80.656571382 77.8813858032227 22.0
140.87903669 140.394561767578 23.0
106.41719675 107.357681274414 24.0
22.90053999 20.4083709716797 25.0
221.53560808 223.594223022461 26.0
221.53560808 221.987747192383 27.0
27.282645659 25.3551406860352 28.0
248.81825375 253.574676513672 29.0
248.81825375 252.408157348633 30.0
47.033013387 47.2299919128418 31.0
9.3131031344 9.18030452728271 32.0
20.901625993 20.4819431304932 33.0
20.901625993 21.1129970550537 34.0
20.901625993 20.8195400238037 35.0
9.3083381265 8.28715133666992 36.0
8.044767894 8.38281917572021 37.0
-2.5262790395e-09 -0.804296553134918 38.0
-2.5262785966e-09 0.113371789455414 39.0
-12.387792857 -14.5510091781616 40.0
-23.81592281 -27.5234241485596 41.0
4.9278776077 3.5162615776062 42.0
-31.108913619 -34.7394485473633 43.0
92.052766479 91.3064651489258 44.0
92.05276649 90.2343292236328 45.0
6.7273095346 3.00315618515015 46.0
2.6568330431 1.83220052719116 47.0
137.17687743 136.683746337891 48.0
131.07448231 127.475601196289 49.0
0 3.55342674255371 50.0
69.705106686 70.9866027832031 51.0
78.669161734 84.2529830932617 52.0
77.500548402 81.0479965209961 53.0
77.500548402 80.9212951660156 54.0
77.500548402 79.4541549682617 55.0
0 -0.0822644829750061 56.0
80.656571391 80.3950500488281 57.0
221.53560809 218.26057434082 58.0
248.81825375 246.469757080078 59.0
20.901626002 19.9398994445801 60.0
0 2.89454770088196 61.0
178.370645713079 159.638366699219 0.0
178.370645713079 165.346588134766 1.0
177.14637536 159.157974243164 2.0
172.35730562 169.398330688477 3.0
0 2.02182245254517 4.0
-7.44681162600001e-09 -0.449241459369659 5.0
-63.251475457858 -64.9727783203125 6.0
73.6456327600031 72.9261779785156 7.0
48.8004476010031 49.6164627075195 8.0
24.8451851590031 19.3036155700684 9.0
91.1510476730031 83.268913269043 10.0
78.5903685230031 72.7832794189453 11.0
83.1256306080031 78.8830642700195 12.0
83.1256306080031 76.7704162597656 13.0
83.1256306080031 78.706413269043 14.0
0 -3.56161689758301 15.0
0 -0.0837370157241821 16.0
-6.62518859999999e-10 -0.1464883685112 17.0
61.656162343729 59.1536598205566 18.0
-61.6561623340047 -59.7981147766113 19.0
-61.6561623140047 -59.0483207702637 20.0
-61.6561623100047 -59.7948341369629 21.0
83.2493587436495 79.5201644897461 22.0
144.109550646729 142.303085327148 23.0
110.780121750911 112.327346801758 24.0
23.5935922129112 20.0302391052246 25.0
227.358909396157 225.864028930664 26.0
227.358909396157 223.158874511719 27.0
25.4977216721575 25.0483779907227 28.0
252.856631066157 256.609436035156 29.0
252.856631066157 257.754852294922 30.0
46.4315661062145 46.9652290344238 31.0
9.31180160085552 8.58171272277832 32.0
21.5931964335741 22.6594200134277 33.0
21.5931964335741 23.2540760040283 34.0
21.5931964520011 19.9214248657227 35.0
9.80006039320884 9.52949142456055 36.0
8.50121525230884 7.59352397918701 37.0
-8.24921289999999e-09 -0.393796741962433 38.0
-8.24921289e-09 -3.26352262496948 39.0
-12.0501610420023 -13.1106815338135 40.0
-24.2807843940185 -27.2108783721924 41.0
5.32958613473594 4.14575242996216 42.0
-32.1715172913359 -32.8319625854492 43.0
98.097557992336 95.9837493896484 44.0
98.0975579243359 94.8631057739258 45.0
6.7589974881 8.08051776885986 46.0
2.4503790524 2.74138641357422 47.0
178.37064571 164.45263671875 48.0
172.35730563 161.536865234375 49.0
0 -1.27973175048828 50.0
73.64563276 75.4217987060547 51.0
91.151047673 86.9432907104492 52.0
83.125630608 80.9474945068359 53.0
83.125630608 79.590690612793 54.0
83.125630608 79.0849227905273 55.0
0 1.70550811290741 56.0
83.249358744 79.7615661621094 57.0
227.3589094 220.720001220703 58.0
252.85663107 249.135391235352 59.0
21.593196452 19.1285552978516 60.0
0 1.35459554195404 61.0
162.691484849725 158.741317749023 0.0
162.691484849725 161.357986450195 1.0
161.49122414 163.510559082031 2.0
156.70215441 168.844543457031 3.0
0 -0.449474394321442 4.0
-8.7976841104e-09 2.20417022705078 5.0
-69.7622129369589 -64.5261077880859 6.0
80.7799501148563 73.0988464355469 7.0
50.9440192178563 52.4340133666992 8.0
29.8359308968563 21.121395111084 9.0
91.2813247838563 86.7498779296875 10.0
78.7191235398563 74.9923324584961 11.0
82.4120558788563 83.9982528686523 12.0
82.4120558788563 80.8315658569336 13.0
82.4120558788563 81.392448425293 14.0
0 -0.616707861423492 15.0
0 -0.346517860889435 16.0
-8.368547484e-09 0.753006517887115 17.0
59.5168818108307 59.4330787658691 18.0
-59.5168818232845 -59.4487495422363 19.0
-59.5168818022845 -58.7132110595703 20.0
-59.5168818092845 -59.2624931335449 21.0
80.448872179 78.1264877319336 22.0
143.235451210831 141.671279907227 23.0
107.177747376038 110.28971862793 24.0
21.9195844870384 22.1619701385498 25.0
223.68432338945 223.083343505859 26.0
223.68432338945 221.898193359375 27.0
25.4718589214496 26.221794128418 28.0
249.15618230945 252.490997314453 29.0
249.15618230945 254.335754394531 30.0
46.988160683764 47.065372467041 31.0
9.3887762128 9.12158203125 32.0
20.9319903698768 18.3112354278564 33.0
20.9319903698768 19.6860103607178 34.0
20.9319903758815 21.305269241333 35.0
9.3896297608 8.9376277923584 36.0
8.1251510418 7.73391819000244 37.0
-9.436810703e-09 0.0641846656799316 38.0
-9.436810703e-09 -2.79352784156799 39.0
-11.4685039276422 -11.9934644699097 40.0
-24.4747206841303 -22.7480621337891 41.0
5.05244618161359 7.39779472351074 42.0
-32.0464420067136 -35.1409339904785 43.0
102.952566970714 95.9430084228516 44.0
102.952566960714 95.6218490600586 45.0
7.1179878574 7.83101749420166 46.0
2.5497763753 2.2245728969574 47.0
162.69148485 162.720596313477 48.0
156.70215442 158.833404541016 49.0
0 -0.439934909343719 50.0
80.779950115 77.9617309570312 51.0
91.281324784 89.9453506469727 52.0
82.412055878 81.404411315918 53.0
82.412055878 81.3535003662109 54.0
82.412055878 80.6638488769531 55.0
0 -0.917096197605133 56.0
80.44887218 80.6970443725586 57.0
223.68432339 220.511962890625 58.0
249.15618231 247.941421508789 59.0
20.931990376 22.6764907836914 60.0
0 2.38331913948059 61.0
196.44745048 161.026382446289 0.0
196.44745048 121.36873626709 1.0
194.75855662 82.9459915161133 2.0
193.6935201 57.6554222106934 3.0
-1.2719505245e-15 70.1268997192383 4.0
-4.845081145e-14 12.5025796890259 5.0
-61.921323706 -55.3584175109863 6.0
69.297993011 74.3241271972656 7.0
43.903305881 45.5045471191406 8.0
25.394687119 35.2793464660645 9.0
16.197087863 79.5253067016602 10.0
3.2702574565 71.1091537475586 11.0
0 82.7157669067383 12.0
0 64.14794921875 13.0
0 86.5423889160156 14.0
32.963121211 9.58790111541748 15.0
32.963121211 -4.84317255020142 16.0
-9.0970458399e-14 -0.601776301860809 17.0
47.121605951 58.5559539794922 18.0
-47.121605971 -57.2429237365723 19.0
-47.893861081 -62.9575233459473 20.0
-47.96831849 -55.4232063293457 21.0
72.606878809 27.1720657348633 22.0
138.6846334 51.998046875 23.0
103.08384621 94.5372543334961 24.0
21.642934366 22.7051219940186 25.0
211.29151222 213.18635559082 26.0
211.29151222 214.921569824219 27.0
23.57364519 20.8295478820801 28.0
234.86515742 258.338226318359 29.0
234.86515742 241.550003051758 30.0
44.468193981 42.6367721557617 31.0
6.0953469923 11.7058229446411 32.0
24.638560308 8.17380905151367 33.0
24.638560308 22.1183738708496 34.0
24.638560308 20.3425693511963 35.0
11.909245053 17.6514148712158 36.0
10.469357299 8.52056312561035 37.0
-1.2947191131e-13 -19.9473972320557 38.0
-1.294719113e-13 5.34555101394653 39.0
-13.010000242 -24.9991359710693 40.0
-27.451222786 -24.9937610626221 41.0
4.2526355847 1.01073729991913 42.0
-34.282292324 -27.6708030700684 43.0
75.391918829 87.0721054077148 44.0
75.391918842 93.8142471313477 45.0
5.3549116103 12.1208848953247 46.0
0.87806140739 2.6255738735199 47.0
196.44745049 146.721420288086 48.0
193.69352011 66.231819152832 49.0
0 0.109617531299591 50.0
69.297993021 76.6744995117188 51.0
16.197087873 81.40625 52.0
0 78.1261749267578 53.0
0 72.7663040161133 54.0
0 77.5921249389648 55.0
0 3.59107780456543 56.0
72.606878819 71.5482711791992 57.0
211.29151223 218.468841552734 58.0
234.86515743 242.727233886719 59.0
24.638560318 15.8048362731934 60.0
0 1.50724959373474 61.0
169.95506867 159.836486816406 0.0
169.95506867 163.548721313477 1.0
168.72780352 163.31852722168 2.0
163.9387338 166.680969238281 3.0
0 2.01036381721497 4.0
0 1.42036485671997 5.0
-63.709395325 -63.6932411193848 6.0
74.743747757 77.1054000854492 7.0
49.647684154 49.5442733764648 8.0
25.096063593 26.6819458007812 9.0
84.157955328 87.353515625 10.0
71.605337576 76.6259536743164 11.0
78.207192888 83.3800659179688 12.0
78.207192888 80.5856781005859 13.0
78.207192888 82.2062301635742 14.0
0 -0.739538133144379 15.0
0 -0.71247786283493 16.0
0 0.410266280174255 17.0
60.385664502 58.7569732666016 18.0
-60.385664522 -58.9403114318848 19.0
-60.3856645 -60.4838218688965 20.0
-60.385664513 -59.7166442871094 21.0
82.394674631 78.1226043701172 22.0
143.28204402 142.811309814453 23.0
108.50546632 111.617126464844 24.0
22.110926411 20.4024524688721 25.0
225.67671866 226.390014648438 26.0
225.67671866 224.798767089844 27.0
25.431926674 23.7103424072266 28.0
251.10864535 253.351379394531 29.0
251.10864535 252.705520629883 30.0
46.857329038 45.6667861938477 31.0
9.2811530467 8.17576789855957 32.0
22.009010108 19.075662612915 33.0
22.009010108 14.6046390533447 34.0
22.009010108 20.3169269561768 35.0
9.2275939014 6.7463755607605 36.0
7.9828812639 7.65429449081421 37.0
0 0.422620058059692 38.0
0 -3.99123430252075 39.0
-11.23936405 -11.9397468566895 40.0
-24.235955029 -28.8415069580078 41.0
5.2833205941 3.23311519622803 42.0
-32.064989144 -31.1926746368408 43.0
102.57472947 93.1945648193359 44.0
102.57472948 95.3900909423828 45.0
6.6805689588 6.98043918609619 46.0
2.4038177864 2.14020824432373 47.0
169.95506868 168.514068603516 48.0
163.93873381 160.48469543457 49.0
0 -3.10845541954041 50.0
74.743747767 76.6830749511719 51.0
84.157955338 89.7051239013672 52.0
78.207192898 81.4601593017578 53.0
78.207192898 78.2401123046875 54.0
78.207192898 79.8687438964844 55.0
0 3.52368354797363 56.0
82.394674641 80.6725921630859 57.0
225.67671867 222.741989135742 58.0
251.10864536 246.864013671875 59.0
22.009010118 23.1880702972412 60.0
0 1.69914627075195 61.0
146.413045322204 166.690139770508 0.0
146.413045322204 160.436889648438 1.0
144.773467393454 161.384231567383 2.0
139.98439756 0.624688327312469 3.0
-1.2568409789e-09 99.158821105957 4.0
-1.77351387000001e-09 20.8468151092529 5.0
-56.9722833562822 -50.4580307006836 6.0
68.6198729624878 77.4388885498047 7.0
66.7328595914878 16.7994785308838 8.0
1.88701337088781 6.38286304473877 9.0
87.2830921044878 89.2781448364258 10.0
76.3900787794878 75.8393783569336 11.0
96.1159214424878 87.4044494628906 12.0
96.1159214424878 87.4379196166992 13.0
96.1159214424878 79.5546493530273 14.0
2.4590033544e-08 -9.45604515075684 15.0
2.4590033544e-08 -3.10487985610962 16.0
5.435670129e-08 -2.53948020935059 17.0
54.5495839151751 60.5534477233887 18.0
-54.5495837337317 -59.0884590148926 19.0
-54.5495837247317 -25.4327087402344 20.0
-54.5495836257317 -54.9111099243164 21.0
75.8599347985372 77.9451141357422 22.0
137.118925257175 135.965148925781 23.0
106.247605848969 101.686943054199 24.0
19.9303705659689 26.1957454681396 25.0
212.978860054409 229.46923828125 26.0
212.978860054409 212.296310424805 27.0
28.510512802409 32.8345794677734 28.0
241.489372854409 230.102600097656 29.0
241.489372854409 229.459884643555 30.0
47.3259415178799 43.2185287475586 31.0
9.19466691800937 11.1725931167603 32.0
21.3103511728467 13.1430253982544 33.0
21.3103511728467 17.4046192169189 34.0
21.3103513023751 29.0187473297119 35.0
8.44057521743984 8.11145210266113 36.0
7.31493717563984 4.30904626846313 37.0
2.55471940000003e-09 -12.2092704772949 38.0
2.55471929999996e-09 2.38560128211975 39.0
-12.8416562308658 -18.6047325134277 40.0
-25.5290108845192 -22.6226215362549 41.0
5.91082045553982 5.70356893539429 42.0
-33.9532358977398 -34.1223182678223 43.0
66.9929202024133 97.1109848022461 44.0
66.9929205057398 97.4752883911133 45.0
8.13352858812163 9.02347087860107 46.0
2.89333904202163 2.00357580184937 47.0
146.41304532 166.685104370117 48.0
139.98439756 107.169799804688 49.0
0 4.72535037994385 50.0
68.619872963 79.2415237426758 51.0
87.283092105 77.5881118774414 52.0
96.115921442 81.4774398803711 53.0
96.115921442 75.9232177734375 54.0
96.115921442 80.3342742919922 55.0
5.4356701286e-08 -1.5921094417572 56.0
75.859934798 74.3314056396484 57.0
212.97886005 219.402954101562 58.0
241.48937285 226.226547241211 59.0
21.310351302 27.7552127838135 60.0
2.5547193311e-09 0.693950593471527 61.0
175.03132233 158.287841796875 0.0
175.03132233 163.403137207031 1.0
173.70929243 169.534255981445 2.0
168.9202227 165.995712280273 3.0
0 -1.591548204422 4.0
0 5.76281261444092 5.0
-70.249358355 -65.1135559082031 6.0
80.997655979 81.6181793212891 7.0
46.545681657 51.6586151123047 8.0
34.451974312 25.9110984802246 9.0
91.083058335 87.1587753295898 10.0
78.230204244 77.7970657348633 11.0
85.233927928 80.5519332885742 12.0
85.233927928 77.5763702392578 13.0
85.233927928 79.1738586425781 14.0
0 -0.685663878917694 15.0
0 -0.384305775165558 16.0
0 -1.33428955078125 17.0
59.599931626 59.2771682739258 18.0
-59.599931645 -59.5249061584473 19.0
-59.599931624 -60.104076385498 20.0
-59.599931636 -60.0260124206543 21.0
81.656906927 79.0201110839844 22.0
148.20433721 146.422348022461 23.0
113.69784259 110.860275268555 24.0
21.838918626 19.8066654205322 25.0
229.86124415 225.107147216797 26.0
229.86124415 229.852890014648 27.0
26.65902897 22.5583000183105 28.0
256.52027313 250.500671386719 29.0
256.52027313 254.218566894531 30.0
46.710776323 48.2051849365234 31.0
9.3012166774 9.2275390625 32.0
22.056975281 19.2189521789551 33.0
22.056975281 16.0446491241455 34.0
22.056975281 20.1404724121094 35.0
9.5720260072 9.52567577362061 36.0
8.3532525578 6.97866296768188 37.0
0 0.818329215049744 38.0
0 -1.21554493904114 39.0
-11.774688445 -12.8988943099976 40.0
-25.110778222 -26.1925735473633 41.0
5.1717526737 1.89213395118713 42.0
-32.946588345 -32.30712890625 43.0
96.438770959 92.0685806274414 44.0
96.438770972 93.1123352050781 45.0
6.8874570582 5.94141721725464 46.0
2.5828755496 1.82582378387451 47.0
175.03132234 162.490921020508 48.0
168.92022271 166.738174438477 49.0
0 -0.706529796123505 50.0
80.997655989 75.8721313476562 51.0
91.083058345 91.3995132446289 52.0
85.233927938 81.3833236694336 53.0
85.233927938 80.4926376342773 54.0
85.233927938 80.29345703125 55.0
0 0.709381520748138 56.0
81.656906937 78.8676223754883 57.0
229.86124416 220.742721557617 58.0
256.52027314 250.582168579102 59.0
22.056975291 22.6891574859619 60.0
0 0.630337297916412 61.0
153.00233927 157.949127197266 0.0
153.00233927 147.725051879883 1.0
151.67351874 161.569412231445 2.0
146.88444902 147.436065673828 3.0
0 0.810172617435455 4.0
-7.1965235375e-10 1.68558967113495 5.0
-54.424229624 -56.0901489257812 6.0
63.44270118 74.5617828369141 7.0
47.959658386 54.8575897216797 8.0
15.483042792 24.5070972442627 9.0
95.954945739 80.1057586669922 10.0
82.514787316 74.9596252441406 11.0
85.792509116 79.3929061889648 12.0
85.792509116 80.9749755859375 13.0
85.792509116 81.6438522338867 14.0
0 1.92555260658264 15.0
0 0.052378237247467 16.0
-4.4831104982e-09 3.1993350982666 17.0
56.572512365 59.4933700561523 18.0
-56.5725123885068 -58.3467254638672 19.0
-56.5725123665068 -58.4966468811035 20.0
-56.5725123785068 -60.1123542785645 21.0
77.169320136 76.3447723388672 22.0
150.9121983 145.891998291016 23.0
112.42481064 109.572807312012 24.0
24.695826524 22.7604827880859 25.0
228.08151845 226.870651245117 26.0
228.08151845 219.383544921875 27.0
28.912326713 28.8398513793945 28.0
256.99384516 255.081481933594 29.0
256.99384516 250.962020874023 30.0
46.900811149 46.8018608093262 31.0
9.2911278584 6.45509672164917 32.0
20.596807762 17.9376831054688 33.0
20.596807762 12.7218046188354 34.0
20.596807759 22.1236190795898 35.0
8.905169772 7.75627517700195 36.0
7.5478157531 7.50654315948486 37.0
-7.9529071376e-09 3.8303017616272 38.0
-7.9528710425e-09 2.06318616867065 39.0
-12.715978096 -5.993408203125 40.0
-24.772798476 -25.4389228820801 41.0
5.3998750468 2.4508204460144 42.0
-32.408430274 -30.0709743499756 43.0
95.25976099 91.5812301635742 44.0
95.259760998 88.5799026489258 45.0
6.7214250507 3.6342921257019 46.0
1.9026439252 2.64665031433105 47.0
153.00233928 153.346817016602 48.0
146.88444903 153.186111450195 49.0
0 1.30362462997437 50.0
63.442701181 75.4771728515625 51.0
95.954945741 81.7000885009766 52.0
85.792509118 81.5665893554688 53.0
85.792509118 81.8162841796875 54.0
85.792509118 80.7918014526367 55.0
0 -0.675220310688019 56.0
77.169320143 79.9252853393555 57.0
228.08151845 219.172256469727 58.0
256.99384516 246.190444946289 59.0
20.596807767 16.7493686676025 60.0
0 4.21874952316284 61.0
164.1053677 162.877426147461 0.0
164.1053677 166.316757202148 1.0
162.93102379 161.806884765625 2.0
158.14195407 152.596160888672 3.0
0 -3.99929618835449 4.0
0 2.39349436759949 5.0
-60.82281403 -66.3274230957031 6.0
71.755214483 75.7222747802734 7.0
56.430852422 51.3183403015137 8.0
15.324362051 22.8368282318115 9.0
87.69963633 86.8045806884766 10.0
74.768828951 75.8081588745117 11.0
79.602133025 81.0682907104492 12.0
79.602133025 77.8076324462891 13.0
79.602133025 78.8077774047852 14.0
0 -1.05350375175476 15.0
0 -0.24024111032486 16.0
0 -0.897970259189606 17.0
56.768719404 59.083854675293 18.0
-56.768719414 -59.4968338012695 19.0
-56.768719392 -59.8538703918457 20.0
-56.768719405 -59.3070640563965 21.0
77.85104794 76.3232803344727 22.0
145.32533065 142.850830078125 23.0
109.98151039 109.951110839844 24.0
22.420458015 20.44287109375 25.0
223.1763786 226.335510253906 26.0
223.1763786 226.017929077148 27.0
25.403324333 24.7086658477783 28.0
248.57970294 255.943176269531 29.0
248.57970294 250.351593017578 30.0
46.690974181 46.718147277832 31.0
9.3550523057 8.76643943786621 32.0
21.082328526 21.7175216674805 33.0
21.082328526 19.7561626434326 34.0
21.082328526 19.6611175537109 35.0
9.0987090848 6.8767147064209 36.0
7.8593622379 7.15653228759766 37.0
-1.5960770215e-11 0.112923443317413 38.0
-1.5960770215e-11 -0.860056817531586 39.0
-12.298075436 -17.30442237854 40.0
-25.384419706 -25.059684753418 41.0
5.3016297424 5.72132253646851 42.0
-33.267862288 -32.9192314147949 43.0
91.753319316 91.023811340332 44.0
91.753319329 91.3166122436523 45.0
6.9786044251 6.00684642791748 46.0
2.7195891394 2.57459235191345 47.0
164.10536771 165.59260559082 48.0
158.14195408 154.422500610352 49.0
0 2.17766189575195 50.0
71.755214493 75.9837875366211 51.0
87.69963634 89.7603759765625 52.0
79.602133035 80.509033203125 53.0
79.602133035 80.1979217529297 54.0
79.602133035 80.2428359985352 55.0
0 0.859796226024628 56.0
77.85104795 79.7022399902344 57.0
223.17637861 221.656234741211 58.0
248.57970295 248.039138793945 59.0
21.082328536 20.2112789154053 60.0
0 1.79017806053162 61.0
162.82061198 161.382034301758 0.0
162.82061198 164.884323120117 1.0
161.54557009 160.500946044922 2.0
156.75650037 153.650985717773 3.0
0 0.720955312252045 4.0
-2.5785874991e-14 -0.501177489757538 5.0
-63.036768998 -62.4407081604004 6.0
73.088212834 75.2378692626953 7.0
47.241213159 47.3818626403809 8.0
25.846999664 25.4585781097412 9.0
90.330234731 90.6795120239258 10.0
77.180411278 78.1233367919922 11.0
81.270169195 82.1066055297852 12.0
81.270169195 79.4502563476562 13.0
81.270169195 79.9354705810547 14.0
0 -0.226097404956818 15.0
0 -0.486578404903412 16.0
-1.0819178425e-14 -0.454637825489044 17.0
57.967662355 58.4910545349121 18.0
-57.967662375 -58.3711051940918 19.0
-57.967662353 -59.8825836181641 20.0
-57.967662365 -60.2147483825684 21.0
78.926268247 80.2490844726562 22.0
141.6194507 141.437225341797 23.0
108.39091133 109.270881652832 24.0
23.722175613 19.9356536865234 25.0
220.54571896 222.616622924805 26.0
220.54571896 219.701843261719 27.0
25.059071064 25.7856025695801 28.0
245.60479003 256.196990966797 29.0
245.60479003 255.814483642578 30.0
46.689936121 45.6534194946289 31.0
9.2833396275 9.15184879302979 32.0
20.958605872 20.040843963623 33.0
20.958605872 20.7526950836182 34.0
20.958605872 19.7976989746094 35.0
8.9453979392 7.01146411895752 36.0
7.6843433188 7.3772234916687 37.0
0 1.12191390991211 38.0
0 -4.29746437072754 39.0
-11.69159293 -11.8616256713867 40.0
-24.19870589 -24.2394371032715 41.0
5.3877026778 6.43013668060303 42.0
-32.254256478 -33.5113792419434 43.0
99.13152889 94.1791687011719 44.0
99.131528903 94.1954040527344 45.0
6.6311241228 5.84266328811646 46.0
2.4293925549 2.8503794670105 47.0
162.82061199 161.422775268555 48.0
156.75650038 153.571395874023 49.0
0 -1.52610635757446 50.0
73.088212844 77.4487075805664 51.0
90.330234741 89.3013458251953 52.0
81.270169205 81.2380981445312 53.0
81.270169205 80.8290100097656 54.0
81.270169205 79.9892578125 55.0
0 -0.119755566120148 56.0
78.926268257 80.382568359375 57.0
220.54571897 219.241226196289 58.0
245.60479004 248.428192138672 59.0
20.958605882 21.2847156524658 60.0
0 1.5748016834259 61.0
137.17687742 143.006423950195 0.0
137.17687742 143.034698486328 1.0
135.86355256 143.509613037109 2.0
131.0744823 135.718490600586 3.0
0 -2.06355714797974 4.0
0 0.341646194458008 5.0
-59.42861149 -59.2279930114746 6.0
69.705106679 70.1772155761719 7.0
65.43029471 62.4528312683105 8.0
4.2748119615 8.48783111572266 9.0
78.669161727 80.2065200805664 10.0
65.229460649 71.2585144042969 11.0
77.500548395 80.8047180175781 12.0
77.500548395 77.9000701904297 13.0
77.500548395 76.9198608398438 14.0
0 -0.64198762178421 15.0
0 0.28728449344635 16.0
0 -3.15485000610352 17.0
59.754945382 58.5015258789062 18.0
-59.754945394 -59.4383087158203 19.0
-59.754945372 -60.0649642944336 20.0
-59.754945384 -61.8801422119141 21.0
80.656571382 77.8813858032227 22.0
140.87903669 140.394561767578 23.0
106.41719675 107.357681274414 24.0
22.90053999 20.4083709716797 25.0
221.53560808 223.594223022461 26.0
221.53560808 221.987747192383 27.0
27.282645659 25.3551406860352 28.0
248.81825375 253.574676513672 29.0
248.81825375 252.408157348633 30.0
47.033013387 47.2299919128418 31.0
9.3131031344 9.18030452728271 32.0
20.901625993 20.4819431304932 33.0
20.901625993 21.1129970550537 34.0
20.901625993 20.8195400238037 35.0
9.3083381265 8.28715133666992 36.0
8.044767894 8.38281917572021 37.0
-2.5262790395e-09 -0.804296553134918 38.0
-2.5262785966e-09 0.113371789455414 39.0
-12.387792857 -14.5510091781616 40.0
-23.81592281 -27.5234241485596 41.0
4.9278776077 3.5162615776062 42.0
-31.108913619 -34.7394485473633 43.0
92.052766479 91.3064651489258 44.0
92.05276649 90.2343292236328 45.0
6.7273095346 3.00315618515015 46.0
2.6568330431 1.83220052719116 47.0
137.17687743 136.683990478516 48.0
131.07448231 127.465370178223 49.0
0 3.55485916137695 50.0
69.705106686 70.9866561889648 51.0
78.669161734 84.2529830932617 52.0
77.500548402 81.0479965209961 53.0
77.500548402 80.9212951660156 54.0
77.500548402 79.4541549682617 55.0
0 -0.0822644829750061 56.0
80.656571391 80.3950500488281 57.0
221.53560809 218.26057434082 58.0
248.81825375 246.469757080078 59.0
20.901626002 19.9398994445801 60.0
0 2.89454770088196 61.0
169.56868627 167.634338378906 0.0
169.56868627 160.667465209961 1.0
168.53855874 161.968734741211 2.0
163.74948902 145.746215820312 3.0
0 26.0863857269287 4.0
-4.575093401e-10 2.85132241249084 5.0
-62.448069677 -61.1956825256348 6.0
72.524243708 74.2431106567383 7.0
54.523489329 52.7561531066895 8.0
18.000754378 22.0112915039062 9.0
88.480360411 89.0852279663086 10.0
75.938738652 81.2172546386719 11.0
79.964873939 88.6846466064453 12.0
79.964873939 80.8084487915039 13.0
79.964873939 81.8111190795898 14.0
0 -4.21952438354492 15.0
0 -0.824238955974579 16.0
-1.1440727465e-09 -0.761316359043121 17.0
57.316007798 58.0772018432617 18.0
-57.3160078153628 -57.6231307983398 19.0
-57.3160077933628 -57.5851974487305 20.0
-57.3160078043628 -60.4259910583496 21.0
77.743773209 73.4272537231445 22.0
150.80736966 138.589111328125 23.0
117.23263361 120.32389831543 24.0
21.66227638 21.6275386810303 25.0
228.55114288 227.624206542969 26.0
228.55114288 220.881149291992 27.0
26.248086173 26.590892791748 28.0
254.79922905 233.35807800293 29.0
254.79922905 242.455047607422 30.0
46.580359594 43.958324432373 31.0
9.2976623422 8.49182415008545 32.0
20.427765409 17.3149490356445 33.0
20.427765409 12.0462398529053 34.0
20.427765406 23.5251789093018 35.0
9.147305784 6.28241872787476 36.0
8.0131455197 8.94467544555664 37.0
-8.0875027616e-09 -0.71451324224472 38.0
-8.0875033772e-09 -0.312146008014679 39.0
-11.823513895 -17.1159324645996 40.0
-24.92815741 -21.468469619751 41.0
5.0044951752 9.42869853973389 42.0
-32.175031196 -36.4483528137207 43.0
108.2289823 88.1074752807617 44.0
108.22898231 99.7563705444336 45.0
6.9583515649 8.14672756195068 46.0
2.364132446 1.73952054977417 47.0
169.56868628 164.874954223633 48.0
163.74948903 143.575866699219 49.0
0 -1.50608158111572 50.0
72.52424371 73.4009170532227 51.0
88.480360413 83.8144989013672 52.0
79.964873941 80.2979354858398 53.0
79.964873941 79.8395309448242 54.0
79.964873941 80.4273071289062 55.0
0 -0.441971957683563 56.0
77.743773217 80.8192520141602 57.0
228.55114288 221.399368286133 58.0
254.79922905 234.633880615234 59.0
20.427765413 21.0315093994141 60.0
0 3.12115263938904 61.0
181.565204996597 162.790969848633 0.0
181.565204996597 162.625717163086 1.0
180.43108149 162.142288208008 2.0
175.13749171 169.81103515625 3.0
0 -2.50072407722473 4.0
-4.347292007e-09 6.25275135040283 5.0
-68.5567689924099 -62.5234794616699 6.0
79.0817276034346 70.5057678222656 7.0
47.0158940014346 51.5044059753418 8.0
32.0658336014346 23.5325775146484 9.0
88.2568061374346 87.8085784912109 10.0
75.8124349274346 76.4428024291992 11.0
81.1804499634346 79.9860610961914 12.0
81.1804499634346 79.3312225341797 13.0
81.1804499634346 79.9496231079102 14.0
0 1.53486680984497 15.0
0 0.305219948291779 16.0
-1.006927278e-09 -0.522368252277374 17.0
58.3082037158734 59.2912292480469 18.0
-58.3082037201519 -59.6605606079102 19.0
-58.3082036991519 -59.4362983703613 20.0
-58.3082037051519 -58.0326766967773 21.0
79.8490874350364 77.3367309570312 22.0
142.694388967873 141.158798217773 23.0
106.991749992342 109.713310241699 24.0
22.9694009013418 21.1483364105225 25.0
222.543476403194 227.865936279297 26.0
222.543476403194 224.611557006836 27.0
28.0841392681937 26.5120792388916 28.0
250.627615673194 253.945846557617 29.0
250.627615673194 254.325210571289 30.0
46.6658112511848 45.3654289245605 31.0
9.3772359394 8.69370365142822 32.0
21.5408837292296 20.8130168914795 33.0
21.5408837292296 19.2078380584717 34.0
21.5408837350119 20.8727874755859 35.0
9.4327396673 6.04313468933105 36.0
8.1335852511 7.8212513923645 37.0
-8.990213113e-09 5.26758193969727 38.0
-8.990213113e-09 -0.0696725845336914 39.0
-12.2649766745759 -12.4621295928955 40.0
-25.1436170683735 -25.6628704071045 41.0
5.38967249569548 5.46975374221802 42.0
-33.1726034461955 -32.6316070556641 43.0
104.039120846195 98.0142669677734 44.0
104.039120826195 95.1773529052734 45.0
6.5359472316 4.93746757507324 46.0
2.36301427 2.7395179271698 47.0
181.565205 165.20964050293 48.0
175.13749172 164.483932495117 49.0
0 -0.572933495044708 50.0
79.081727603 77.2921981811523 51.0
88.256806137 90.8616409301758 52.0
81.180449964 81.3178634643555 53.0
81.180449964 80.2224426269531 54.0
81.180449964 80.1489105224609 55.0
0 0.146731913089752 56.0
79.849087435 79.8956680297852 57.0
222.5434764 220.964721679688 58.0
250.62761567 251.006988525391 59.0
21.540883736 22.1544666290283 60.0
0 2.6007559299469 61.0
173.206505309386 174.489318847656 0.0
173.206505309386 175.342147827148 1.0
172.12348614 167.030654907227 2.0
167.3344164 170.333312988281 3.0
0 3.43476128578186 4.0
-8.185434373e-09 -1.69502019882202 5.0
-74.852188806077 -64.7896728515625 6.0
84.8337591802694 69.1440658569336 7.0
59.7253570032694 55.046558380127 8.0
25.1084021762694 15.1080713272095 9.0
88.3420909152694 92.6110534667969 10.0
74.5724466022694 77.3556747436523 11.0
85.6210436372694 79.3623046875 12.0
85.6210436372694 82.9876022338867 13.0
85.6210436372694 81.7024078369141 14.0
2.50386292706e-09 1.63728451728821 15.0
0 0.67412656545639 16.0
1.04786730999999e-09 5.27228879928589 17.0
56.8391166767833 58.4946022033691 18.0
-56.8391166614041 -57.5986366271973 19.0
-56.8391166424041 -59.8900680541992 20.0
-56.8391166344041 -58.9126930236816 21.0
77.415187717235 71.0409317016602 22.0
143.130858913783 144.010894775391 23.0
108.118771874729 107.596138000488 24.0
21.9337686117291 20.7878856658936 25.0
220.546046628772 219.969757080078 26.0
220.546046628772 224.169525146484 27.0
28.1661767117724 26.6949501037598 28.0
248.712223338772 252.976791381836 29.0
248.712223338772 254.49772644043 30.0
46.5313248703018 45.7875747680664 31.0
9.27595566630184 9.49917984008789 32.0
20.5760710832309 16.4184055328369 33.0
20.5760710832309 21.0432605743408 34.0
20.5760711058171 19.8119792938232 35.0
9.32625065169575 10.5892295837402 36.0
8.00994083879575 7.83387327194214 37.0
-7.89117753000001e-09 -0.602889716625214 38.0
-7.89117752999998e-09 -3.00918531417847 39.0
-10.7884473862021 -14.7103567123413 40.0
-24.9346037519505 -29.6362495422363 41.0
5.23280407305784 3.61065196990967 42.0
-33.0200855135578 -32.5323219299316 43.0
88.7604245195578 84.3785934448242 44.0
88.7604243342566 84.101318359375 45.0
7.4993306027 8.98408699035645 46.0
2.4928089588 2.79538583755493 47.0
173.20650531 174.387329101562 48.0
167.33441641 180.439910888672 49.0
0 -0.986160576343536 50.0
84.83375918 79.1847457885742 51.0
88.342090915 87.8935089111328 52.0
85.621043637 79.4213943481445 53.0
85.621043637 81.681266784668 54.0
85.621043637 80.7965393066406 55.0
1.0478673047e-09 -0.257964432239532 56.0
77.415187718 80.5299911499023 57.0
220.54604663 220.434020996094 58.0
248.71222334 246.147369384766 59.0
20.576071106 17.1954975128174 60.0
0 0.725447237491608 61.0
175.429917329716 170.020202636719 0.0
175.429917329716 170.527786254883 1.0
174.16769483 166.570159912109 2.0
169.3786251 178.914855957031 3.0
0 1.2425332069397 4.0
-8.418845503e-09 2.92017936706543 5.0
-62.2980219930499 -65.1210098266602 6.0
74.3239689911748 74.1341171264648 7.0
50.6417265201747 44.8937034606934 8.0
23.6822424711748 21.9911708831787 9.0
91.6111885521748 88.8447570800781 10.0
78.6172409881748 76.1004333496094 11.0
81.4730670541748 84.6351699829102 12.0
81.4730670541748 79.9136276245117 13.0
81.4730670541748 76.9628067016602 14.0
0 -0.562500774860382 15.0
0 -0.125568509101868 16.0
-6.782706642e-09 1.70587372779846 17.0
58.6449134939135 58.1755638122559 18.0
-58.6449134982621 -60.4924545288086 19.0
-58.6449134772621 -59.8865547180176 20.0
-58.6449134802621 -58.637092590332 21.0
80.3591213961283 80.398567199707 22.0
147.928799131914 141.174102783203 23.0
114.717387142392 109.117500305176 24.0
23.9461144483919 21.6827220916748 25.0
228.287920529432 217.549682617188 26.0
228.287920529432 222.059753417969 27.0
27.4524862024324 26.062520980835 28.0
255.740406729432 253.678497314453 29.0
255.740406729432 257.662719726562 30.0
46.8957171096625 46.3721961975098 31.0
9.28635448610878 8.91452980041504 32.0
21.7142079161498 19.9010200500488 33.0
21.7142079161498 15.2899169921875 34.0
21.7142079279223 20.8412971496582 35.0
9.52342697370499 6.09104299545288 36.0
8.25922209630499 7.82858324050903 37.0
-8.885359522e-09 -3.51889657974243 38.0
-8.885359523e-09 -1.01354289054871 39.0
-11.9842642716311 -13.1648797988892 40.0
-25.2585714014956 -25.7219734191895 41.0
4.98448242641229 4.23767328262329 42.0
-32.9644901658123 -33.8026657104492 43.0
90.8236849968123 90.5956192016602 44.0
90.8236849063536 93.3806762695312 45.0
6.8791030828 5.40291404724121 46.0
2.4424148336 2.75354194641113 47.0
175.42991733 172.703338623047 48.0
169.37862511 168.553176879883 49.0
0 3.15641236305237 50.0
74.323968992 77.0097045898438 51.0
91.611188552 85.7279663085938 52.0
81.473067054 81.2490005493164 53.0
81.473067054 81.5131607055664 54.0
81.473067054 80.2349700927734 55.0
0 1.886479139328 56.0
80.359121396 78.9532623291016 57.0
228.28792053 215.42839050293 58.0
255.74040673 251.176620483398 59.0
21.714207928 22.9901733398438 60.0
0 1.40290951728821 61.0
149.575081040798 164.287887573242 0.0
149.575081040798 168.477142333984 1.0
148.2970609 156.977462768555 2.0
143.50799116 141.509552001953 3.0
0 -0.161849558353424 4.0
-6.278144422e-09 2.63408946990967 5.0
-64.9491952166332 -60.83056640625 6.0
76.3333944707161 74.4209365844727 7.0
54.5569578517161 52.0205841064453 8.0
21.7764366197161 25.4498748779297 9.0
87.8096269887161 86.5691223144531 10.0
75.0648298417161 76.5296401977539 11.0
79.1772020137161 82.8260192871094 12.0
79.1772020137161 82.000244140625 13.0
79.1772020137161 80.9393615722656 14.0
0 -1.46080541610718 15.0
0 -0.433019459247589 16.0
-4.98813483799999e-09 -2.73008894920349 17.0
56.0483458444815 60.2046165466309 18.0
-56.0483458450742 -58.8705368041992 19.0
-56.0483458240742 -60.1780014038086 20.0
-56.0483458250742 -62.4484748840332 21.0
77.3854060499503 77.5117950439453 22.0
147.403488683482 141.558685302734 23.0
112.508847751852 108.550407409668 24.0
23.1237729518519 21.322868347168 25.0
224.788894731596 225.291564941406 26.0
224.788894731596 223.918426513672 27.0
27.4920300815955 24.0779151916504 28.0
252.280924811596 256.080017089844 29.0
252.280924811596 257.343048095703 30.0
46.68347949816 45.7027778625488 31.0
9.38265989387767 8.93407154083252 32.0
21.3370602248995 21.668888092041 33.0
21.3370602248995 20.8613796234131 34.0
21.3370602385918 19.597204208374 35.0
9.60962606176085 7.79098892211914 36.0
8.36987815066085 7.37017440795898 37.0
-8.8020604e-09 0.404836714267731 38.0
-8.8020604e-09 -5.29227828979492 39.0
-11.2629740300371 -13.4446496963501 40.0
-24.0016762276353 -25.1775455474854 41.0
4.98076211473011 6.62622976303101 42.0
-31.4308604949301 -35.2885589599609 43.0
97.7547038069301 96.2006149291992 44.0
97.7547037599301 95.015754699707 45.0
7.220988082 5.65399694442749 46.0
2.680581481 1.7913122177124 47.0
149.57508104 165.886566162109 48.0
143.50799117 138.679260253906 49.0
0 -0.469508349895477 50.0
76.333394471 77.486686706543 51.0
87.809626989 90.4419708251953 52.0
79.177202014 80.778678894043 53.0
79.177202014 81.5603637695312 54.0
79.177202014 80.1599273681641 55.0
0 1.60002899169922 56.0
77.38540605 79.4752883911133 57.0
224.78889473 220.980072021484 58.0
252.28092481 252.561645507812 59.0
21.337060239 19.0434303283691 60.0
0 0.95581191778183 61.0
165.344975426446 164.556365966797 0.0
165.344975426446 165.016357421875 1.0
164.17958462 157.830581665039 2.0
159.39051485 155.804092407227 3.0
0 -3.86797952651978 4.0
-5.70272672000002e-09 3.25805354118347 5.0
-70.3683249620972 -65.6893539428711 6.0
81.9690991498402 73.2571716308594 7.0
48.4256408028402 52.4498672485352 8.0
33.5434583468402 24.9267921447754 9.0
89.2551065218402 86.1276626586914 10.0
76.4114561648402 75.4891510009766 11.0
81.1999948708402 78.138313293457 12.0
81.1999948708402 78.8213729858398 13.0
81.1999948708402 81.5171661376953 14.0
4.806758797e-09 -2.1628565788269 15.0
4.80675879706e-09 -1.08408617973328 16.0
1.112953862e-08 -1.49414086341858 17.0
61.2370771524583 59.0944404602051 18.0
-61.2370770957603 -59.3224563598633 19.0
-61.2370770797603 -59.7922744750977 20.0
-61.2370770467603 -59.6026878356934 21.0
82.6406904060109 78.4824066162109 22.0
147.013210216458 140.697723388672 23.0
112.006341873073 109.83984375 24.0
23.3564199140729 20.2012023925781 25.0
229.653900622893 226.383178710938 26.0
229.653900622893 223.825271606445 27.0
25.2693702858925 27.0924339294434 28.0
254.923270902893 254.755355834961 29.0
254.923270902893 254.677398681641 30.0
46.8770054433321 47.190502166748 31.0
9.39616815611573 8.82416820526123 32.0
21.4036133592916 20.8851928710938 33.0
21.4036133592916 14.5832643508911 34.0
21.4036134112897 21.4851703643799 35.0
9.46210315022193 7.40823698043823 36.0
8.27101564102193 6.50053882598877 37.0
-5.38583630000002e-09 0.146860003471375 38.0
-5.38583629e-09 -0.825106203556061 39.0
-11.2549325753802 -14.7519407272339 40.0
-24.3414094862238 -22.1240577697754 41.0
4.87717685406683 8.3837718963623 42.0
-31.8643112828668 -35.1824531555176 43.0
96.6767357008668 93.7697982788086 44.0
96.6767354918669 94.9240570068359 45.0
7.09087648879423 7.34370422363281 46.0
2.48035146109423 1.37342596054077 47.0
165.34497543 166.326766967773 48.0
159.39051485 152.32275390625 49.0
0 0.427283644676208 50.0
81.96909915 75.5657196044922 51.0
89.255106522 90.3668212890625 52.0
81.199994871 80.8472061157227 53.0
81.199994871 79.3540954589844 54.0
81.199994871 79.7449417114258 55.0
1.1129538618e-08 2.5918595790863 56.0
82.640690406 79.0775451660156 57.0
229.65390062 219.342575073242 58.0
254.9232709 249.008651733398 59.0
21.403613411 22.755220413208 60.0
0 2.87257099151611 61.0
152.5421472 163.445922851562 0.0
152.5421472 163.231567382812 1.0
151.35012006 157.286361694336 2.0
146.56105034 144.633255004883 3.0
0 1.62437105178833 4.0
-2.393190869e-12 2.96719193458557 5.0
-65.175441535 -66.0746307373047 6.0
75.42587961 78.3929214477539 7.0
49.676377452 50.7633743286133 8.0
25.749502149 23.7489624023438 9.0
85.989314824 90.5766448974609 10.0
72.665770655 80.3772583007812 11.0
80.175607508 81.397102355957 12.0
80.175607508 76.7245712280273 13.0
80.175607508 78.8517684936523 14.0
0 -2.24130702018738 15.0
0 0.0961032509803772 16.0
0 1.78160238265991 17.0
58.999884248 58.2966690063477 18.0
-58.999884268 -58.109691619873 19.0
-58.999884246 -59.2192611694336 20.0
-58.999884258 -58.6987915039062 21.0
80.43728154 78.0234756469727 22.0
146.46760867 139.902862548828 23.0
109.93333726 107.119361877441 24.0
21.551378254 19.4526424407959 25.0
226.90489022 226.875228881836 26.0
226.90489022 222.210739135742 27.0
26.398711847 27.6090888977051 28.0
253.30360208 256.164093017578 29.0
253.30360208 257.736877441406 30.0
46.845502664 46.5041122436523 31.0
9.3684174114 8.82255935668945 32.0
21.437397272 19.5078029632568 33.0
21.437397272 22.9383144378662 34.0
21.437397272 21.5370788574219 35.0
9.6790135414 8.09219551086426 36.0
8.4049317396 7.85404014587402 37.0
-1.6423027206e-11 -0.224362313747406 38.0
-1.6422996578e-11 0.702457964420319 39.0
-12.131493626 -10.8593378067017 40.0
-25.215095077 -29.2422828674316 41.0
5.0353531699 5.52958393096924 42.0
-32.691311892 -31.7741985321045 43.0
96.677690753 91.8984069824219 44.0
96.677690766 95.2328414916992 45.0
6.7531904103 6.80513763427734 46.0
2.6972772927 2.76006889343262 47.0
152.54214721 165.6708984375 48.0
146.56105035 149.555252075195 49.0
0 -1.92121648788452 50.0
75.42587962 76.2786254882812 51.0
85.989314833 89.0621871948242 52.0
80.175607518 80.7182769775391 53.0
80.175607518 80.2890930175781 54.0
80.175607518 80.3862380981445 55.0
0 1.60376501083374 56.0
80.43728155 80.5320281982422 57.0
226.90489023 222.438812255859 58.0
253.30360209 249.627853393555 59.0
21.437397282 21.3337497711182 60.0
0 2.39399337768555 61.0
170.920297160303 166.685638427734 0.0
170.920297160303 167.304122924805 1.0
169.749259525554 163.235076904297 2.0
164.9601897 166.98210144043 3.0
-2.4386375784e-10 -1.52636671066284 4.0
-2.58116956999999e-09 6.11654472351074 5.0
-62.5929166170547 -59.934986114502 6.0
72.5753245371915 75.1639556884766 7.0
49.3732291351915 52.1425476074219 8.0
23.2020954021915 24.1264324188232 9.0
87.9311180511915 88.4296112060547 10.0
74.8476426501915 74.5186767578125 11.0
80.7697746601915 78.3004913330078 12.0
80.7697746601915 78.6733627319336 13.0
80.7697746601915 79.3158950805664 14.0
1.475045861e-08 -1.12527132034302 15.0
1.475045861e-08 -0.562447130680084 16.0
4.618759346e-08 -0.486694991588593 17.0
56.5950425131044 58.1897354125977 18.0
-56.5950423567872 -58.544490814209 19.0
-56.5950423457872 -59.9885406494141 20.0
-56.5950422617872 -60.4796752929688 21.0
78.3264028957544 79.3713226318359 22.0
147.468700342104 142.953002929688 23.0
112.744539647631 110.467620849609 24.0
22.7279826946306 21.0113563537598 25.0
225.795103240606 229.199996948242 26.0
225.795103240606 223.962295532227 27.0
25.9413919436055 25.8728485107422 28.0
251.736495180606 255.313552856445 29.0
251.736495180606 256.075531005859 30.0
46.7667877222111 45.5988121032715 31.0
9.36831773705519 8.82804489135742 32.0
21.7313606351641 20.178689956665 33.0
21.7313606351641 21.6315898895264 34.0
21.7313607481005 20.6277122497559 35.0
9.33062134401624 9.15672779083252 36.0
8.05194773261624 7.55928134918213 37.0
9.6254839999983e-10 -0.996068060398102 38.0
9.62548400000041e-10 -2.2338924407959 39.0
-11.3711413493936 -13.619797706604 40.0
-24.6638257047245 -23.9235973358154 41.0
5.02176524385958 5.76927852630615 42.0
-32.2553188706596 -33.1297721862793 43.0
101.22122398266 98.5569458007812 44.0
101.22122351266 95.5587768554688 45.0
7.26434316821298 6.7813401222229 46.0
2.68594643401299 2.77741265296936 47.0
170.92029716 165.963745117188 48.0
164.96018971 163.515502929688 49.0
0 1.69944429397583 50.0
72.575324537 76.9591217041016 51.0
87.931118051 89.1653213500977 52.0
80.76977466 81.1744155883789 53.0
80.76977466 79.5469284057617 54.0
80.76977466 80.1421127319336 55.0
4.6187593455e-08 1.86801338195801 56.0
78.326402896 79.9541854858398 57.0
225.79510324 221.647201538086 58.0
251.73649518 250.920043945312 59.0
21.731360748 20.1066856384277 60.0
9.6254841021e-10 1.20868265628815 61.0
161.464323483046 163.383636474609 0.0
161.464323483046 152.833740234375 1.0
160.19873735 157.675628662109 2.0
155.40966762 159.423950195312 3.0
0 -2.52564096450806 4.0
-8.782604249e-09 6.43689632415771 5.0
-53.1017804526513 -57.8126068115234 6.0
65.3489096532794 69.3166961669922 7.0
52.9975025992794 50.601936340332 8.0
12.3514070542794 27.7344303131104 9.0
85.5884879572794 86.1074600219727 10.0
73.8637784712794 72.0203552246094 11.0
75.7731636902794 80.6948699951172 12.0
75.7731636902794 83.8072357177734 13.0
75.7731636902794 78.4548492431641 14.0
0 -5.08563232421875 15.0
0 -1.04890012741089 16.0
-5.436396189e-09 -5.80834102630615 17.0
52.5459279188993 58.3730163574219 18.0
-52.5459279519191 -58.6178588867188 19.0
-52.5459279309191 -61.0719718933105 20.0
-52.5459279359191 -56.0626029968262 21.0
74.4799720193728 74.7979507446289 22.0
140.252255884516 144.303253173828 23.0
107.953168718145 109.778038024902 24.0
20.087941068145 18.9377727508545 25.0
214.732227906092 223.229339599609 26.0
214.732227906092 226.594879150391 27.0
24.7549790840916 25.7979679107666 28.0
239.487206986092 254.562881469727 29.0
239.487206986092 256.168975830078 30.0
46.4226951943921 47.4500350952148 31.0
9.2972348456 9.45430088043213 32.0
21.9340440829538 17.9560146331787 33.0
21.9340440829538 21.3133792877197 34.0
21.9340440908894 20.988166809082 35.0
9.205955454 12.1505517959595 36.0
7.8661622202 6.61413812637329 37.0
-9.22286532e-09 -6.14931344985962 38.0
-9.22286532e-09 -2.11518311500549 39.0
-12.7618526829595 -8.93564510345459 40.0
-25.7379024782438 -27.8157844543457 41.0
4.63881470206616 7.281174659729 42.0
-32.9734878310662 -34.9303321838379 43.0
94.186957054052 93.4640884399414 44.0
94.1869571050662 96.5966644287109 45.0
7.3911972129 5.03342723846436 46.0
2.7138170514 1.33035409450531 47.0
161.46432349 173.436950683594 48.0
155.40966763 147.891708374023 49.0
0 -1.24117708206177 50.0
65.348909654 72.182746887207 51.0
85.588487958 82.1266555786133 52.0
75.773163691 79.9139709472656 53.0
75.773163691 76.5281066894531 54.0
75.773163691 80.4564208984375 55.0
0 -2.32747364044189 56.0
74.479972019 78.4493103027344 57.0
214.7322279 216.704940795898 58.0
239.48720699 246.550704956055 59.0
21.934044091 21.9634418487549 60.0
0 1.44568991661072 61.0
132.234641297998 161.766754150391 0.0
132.234641297998 178.146286010742 1.0
131.32885415 149.648025512695 2.0
126.53978441 149.419570922852 3.0
0 5.65979528427124 4.0
-7.843937472e-09 4.91043138504028 5.0
-51.7671612583896 -56.5757064819336 6.0
62.5966586288636 76.4231338500977 7.0
44.7514640488636 53.9211616516113 8.0
17.8451945798636 21.9192028045654 9.0
96.6107170158636 95.8165130615234 10.0
83.0245191438636 97.3708572387695 11.0
79.7814365588636 81.7599868774414 12.0
79.7814365588636 74.2051239013672 13.0
79.7814365588636 77.8654861450195 14.0
4.688145605e-10 0.936655700206757 15.0
0 0.934317350387573 16.0
2.96656128e-09 3.69208645820618 17.0
54.661430092471 58.6068267822266 18.0
-54.6614300747954 -58.5590057373047 19.0
-54.6614300557954 -58.4763984680176 20.0
-54.6614300477954 -49.6406898498535 21.0
75.9483076065848 75.4926376342773 22.0
134.442271692471 138.197906494141 23.0
100.853405483089 105.738525390625 24.0
20.8113000210888 21.5341262817383 25.0
210.390579295996 212.740173339844 26.0
210.390579295996 223.933471679688 27.0
24.9560137519964 27.0036220550537 28.0
235.346593055996 256.068786621094 29.0
235.346593055996 249.718383789062 30.0
47.307876572601 47.2309494018555 31.0
9.37346792223011 8.10899829864502 32.0
21.2868775581688 14.8815574645996 33.0
21.2868775581688 12.2124662399292 34.0
21.2868775809874 21.8078193664551 35.0
9.16618290946687 14.0629348754883 36.0
8.01659251826868 8.8354663848877 37.0
-7.55334186e-09 -0.655404150485992 38.0
-7.55334186e-09 0.75853031873703 39.0
-10.5445150943977 -16.3947162628174 40.0
-23.5905176327273 -16.11008644104 41.0
5.21585840375307 -7.61994457244873 42.0
-31.436208474553 -30.5152206420898 43.0
94.1171816025529 87.6061096191406 44.0
94.1171815165529 92.507453918457 45.0
6.44138761371154 5.01651096343994 46.0
2.5144971854 2.92206430435181 47.0
132.2346413 169.634658813477 48.0
126.53978442 140.335571289062 49.0
0 -5.28760385513306 50.0
62.596658629 73.3418350219727 51.0
96.610717016 81.0832061767578 52.0
79.781436559 82.4935455322266 53.0
79.781436559 79.4855194091797 54.0
79.781436559 78.9576644897461 55.0
2.9665612726e-09 1.56643199920654 56.0
75.948307606 75.4334411621094 57.0
210.3905793 217.055847167969 58.0
235.34659305 249.248657226562 59.0
21.286877581 23.6209506988525 60.0
0 2.61032724380493 61.0
147.982033057583 157.456436157227 0.0
147.982033057583 151.966354370117 1.0
146.808413813287 153.371154785156 2.0
142.01934397 150.885101318359 3.0
-1.55484310406e-09 -0.668092310428619 4.0
-1.55484309999997e-09 -2.84683012962341 5.0
-64.9693040339936 -64.3280944824219 6.0
77.6547829144777 74.1872634887695 7.0
61.2786926634778 55.4030647277832 8.0
16.3760902514777 7.20984888076782 9.0
97.5261141374777 89.7391510009766 10.0
83.6472874924777 79.6450042724609 11.0
85.2981411144777 83.8642730712891 12.0
85.2981411144777 76.4624557495117 13.0
85.2981411144777 81.5436782836914 14.0
2.3328021736e-08 -2.03496789932251 15.0
2.3328021735e-08 -1.6052393913269 16.0
5.464350236e-08 -2.72059202194214 17.0
56.2518018559467 59.0783615112305 18.0
-56.2518018797166 -57.0891418457031 19.0
-56.2518018707166 -59.2854194641113 20.0
-56.2518017707166 -58.115608215332 21.0
77.1875821073724 73.3798065185547 22.0
153.454771802806 145.008895874023 23.0
115.220596133508 115.750228881836 24.0
21.7005584165075 20.8819332122803 25.0
230.642353915167 228.381210327148 26.0
230.642353915167 229.708221435547 27.0
26.8883420391666 21.3647422790527 28.0
257.530695955167 253.23681640625 29.0
257.530695955167 255.336776733398 30.0
46.7385281475469 45.537841796875 31.0
9.28724497493877 8.4355525970459 32.0
20.9357803369809 22.6376686096191 33.0
20.9357803369809 19.0790691375732 34.0
20.935780467613 20.4264030456543 35.0
8.86054244871795 8.39873600006104 36.0
7.60121916461794 6.85336446762085 37.0
2.62465480000001e-09 1.1959273815155 38.0
2.62465480000001e-09 -1.17553234100342 39.0
-11.6034306368583 -12.5643882751465 40.0
-24.5599220844463 -26.7411251068115 41.0
4.75986482786468 1.92820739746094 42.0
-31.7734218810648 -31.6524333953857 43.0
89.4696777832616 86.4044494628906 44.0
89.4696781210649 91.9600372314453 45.0
7.05809588635643 4.443039894104 46.0
2.46335994485643 2.26466846466064 47.0
147.98203306 156.095489501953 48.0
142.01934397 153.122375488281 49.0
0 -3.73737192153931 50.0
77.654782914 76.0232696533203 51.0
97.526114137 92.0755157470703 52.0
85.298141114 81.5651702880859 53.0
85.298141114 80.2131118774414 54.0
85.298141114 79.9203643798828 55.0
5.4643502359e-08 3.07315587997437 56.0
77.187582108 79.4883346557617 57.0
230.64235391 222.583419799805 58.0
257.53069595 249.491134643555 59.0
20.935780468 19.6116352081299 60.0
2.6246548239e-09 3.64096260070801 61.0
166.25847831 160.948181152344 0.0
166.25847831 167.395050048828 1.0
165.00161414 162.02880859375 2.0
159.53697895 162.42204284668 3.0
0.053089762974 -2.64210915565491 4.0
0.053089762974 2.74183773994446 5.0
-65.592150936 -63.1159515380859 6.0
76.043146509 76.7854461669922 7.0
45.475155101 51.6805953979492 8.0
30.567991398 22.458776473999 9.0
83.583516136 87.1007919311523 10.0
71.170048692 75.5833358764648 11.0
78.92234323 85.8072204589844 12.0
78.92234323 80.1295166015625 13.0
78.92234323 80.4129791259766 14.0
0 0.0786507129669189 15.0
0 -0.492902100086212 16.0
0 -1.48141765594482 17.0
59.870871725 58.3560600280762 18.0
-59.870871745 -58.9556312561035 19.0
-59.870871723 -59.931037902832 20.0
-59.870871736 -57.7058982849121 21.0
81.202816233 76.0993728637695 22.0
147.39703203 142.99006652832 23.0
111.77401552 107.190093994141 24.0
22.875363859 20.6158657073975 25.0
228.59984828 221.934677124023 26.0
228.59984828 223.971496582031 27.0
24.961164621 25.2713661193848 28.0
253.56101291 257.136077880859 29.0
253.56101291 257.675354003906 30.0
46.90478904 45.8503646850586 31.0
9.3657890911 8.66880893707275 32.0
21.331944488 19.5110969543457 33.0
21.331944488 19.5986385345459 34.0
21.331944488 20.5691146850586 35.0
9.6326234192 7.01118135452271 36.0
8.3784323792 7.83701705932617 37.0
0 0.122348487377167 38.0
0 -3.48350954055786 39.0
-11.722531607 -13.6475610733032 40.0
-24.38995777 -26.4905090332031 41.0
5.175531526 2.59741854667664 42.0
-32.124582255 -32.4026985168457 43.0
102.79376804 95.1686325073242 44.0
102.79376805 95.0252838134766 45.0
7.0354296021 5.82985353469849 46.0
2.7117855098 2.16836524009705 47.0
166.25847832 163.708709716797 48.0
159.53697896 161.154357910156 49.0
0.053089772974 -1.27081179618835 50.0
76.043146519 77.7453994750977 51.0
83.583516146 88.7781829833984 52.0
78.92234324 81.4774475097656 53.0
78.92234324 81.9326934814453 54.0
78.92234324 79.5724716186523 55.0
0 1.82720327377319 56.0
81.202816243 79.3906784057617 57.0
228.59984829 220.030746459961 58.0
253.56101292 251.484039306641 59.0
21.331944498 19.093729019165 60.0
0 1.14338994026184 61.0
152.64297682 160.036682128906 0.0
152.64297682 166.12890625 1.0
151.30880137 160.873352050781 2.0
146.51973165 155.807327270508 3.0
0 1.68981385231018 4.0
-9.7575826113e-12 -0.649509370326996 5.0
-60.171253998 -64.0325241088867 6.0
71.909917528 73.189338684082 7.0
48.758386357 52.1474494934082 8.0
23.151531161 21.7900924682617 9.0
86.826849954 87.0083618164062 10.0
73.552810128 76.3201599121094 11.0
80.602393349 83.9651336669922 12.0
80.602393349 80.768196105957 13.0
80.602393349 81.0190582275391 14.0
0 -0.84912496805191 15.0
0 -0.124878108501434 16.0
-2.0478330647e-11 1.56776702404022 17.0
57.787708491 57.8376617431641 18.0
-57.787708511 -58.7761268615723 19.0
-57.787708489 -58.6178245544434 20.0
-57.787708501 -60.1469497680664 21.0
78.972809862 77.8163452148438 22.0
147.55108975 141.160171508789 23.0
114.00553285 109.683776855469 24.0
23.832830305 21.1275196075439 25.0
226.52389963 225.416320800781 26.0
226.52389963 224.487258911133 27.0
27.250651875 23.9672393798828 28.0
253.77455151 255.55924987793 29.0
253.77455151 255.986831665039 30.0
46.750312798 47.26416015625 31.0
9.3327220691 8.8815860748291 32.0
21.185101351 20.4296798706055 33.0
21.185101351 13.5004377365112 34.0
21.185101351 21.0598030090332 35.0
9.3633052591 7.69415855407715 36.0
8.0955714756 7.50305700302124 37.0
-3.2383671916e-11 -1.62402009963989 38.0
-3.2383671917e-11 -0.647268831729889 39.0
-11.546597478 -16.6982536315918 40.0
-24.700979458 -24.6478672027588 41.0
4.9555451378 6.78781080245972 42.0
-32.24502116 -33.9722900390625 43.0
95.439969452 92.4297256469727 44.0
95.439969466 95.0690383911133 45.0
6.998990478 6.08623933792114 46.0
2.5429915587 1.70197439193726 47.0
152.64297683 162.449768066406 48.0
146.51973166 150.270690917969 49.0
0 -1.06047630310059 50.0
71.909917538 74.576545715332 51.0
86.826849964 88.0537414550781 52.0
80.602393359 80.2794189453125 53.0
80.602393359 80.9284744262695 54.0
80.602393359 80.7080764770508 55.0
0 4.07350015640259 56.0
78.972809872 79.7845153808594 57.0
226.52389964 221.917831420898 58.0
253.77455152 249.932388305664 59.0
21.185101361 24.6385650634766 60.0
0 0.51330977678299 61.0
175.81360781 158.893325805664 0.0
175.81360781 160.706970214844 1.0
174.57809832 165.443878173828 2.0
169.78902859 172.824401855469 3.0
0 -1.84168672561646 4.0
-4.441045821e-10 -0.697107970714569 5.0
-70.302013814 -65.8326110839844 6.0
82.145105535 75.5527725219727 7.0
53.953306396 57.577709197998 8.0
28.191799137 26.7716903686523 9.0
87.648013961 85.9062881469727 10.0
75.393362562 76.3276138305664 11.0
84.077750112 83.7318572998047 12.0
84.077750112 78.7674789428711 13.0
84.077750112 79.0019226074219 14.0
0 0.278261780738831 15.0
0 0.284722089767456 16.0
-3.9667634274e-09 0.166833579540253 17.0
58.674600403 58.6971435546875 18.0
-58.6746004273699 -59.3303642272949 19.0
-58.6746004053699 -61.3872375488281 20.0
-58.6746004163699 -56.3875312805176 21.0
80.15239981 77.5188674926758 22.0
147.06179663 142.472213745117 23.0
112.66155193 109.822494506836 24.0
23.551182492 21.4972839355469 25.0
227.21419645 223.805694580078 26.0
227.21419645 224.638885498047 27.0
26.210031798 27.6818866729736 28.0
253.42422825 255.836517333984 29.0
253.42422825 255.309219360352 30.0
46.803232598 45.4166564941406 31.0
9.3668374492 8.8443660736084 32.0
21.477799398 20.0808639526367 33.0
21.477799398 22.6484050750732 34.0
21.477799395 20.6789817810059 35.0
9.2339436349 8.99952411651611 36.0
8.0319516275 7.82161045074463 37.0
-7.1073082031e-09 -0.440208613872528 38.0
-7.1073081315e-09 -1.45399522781372 39.0
-12.172273008 -11.7032737731934 40.0
-24.495375871 -22.2200088500977 41.0
5.3664536844 8.27184581756592 42.0
-32.511104303 -35.5179405212402 43.0
96.044139781 92.6450576782227 44.0
96.044139789 94.6486053466797 45.0
6.8385369303 5.64429235458374 46.0
2.4617582885 2.64976763725281 47.0
175.81360782 165.369049072266 48.0
169.7890286 165.885345458984 49.0
0 1.5202169418335 50.0
82.145105537 76.3880844116211 51.0
87.648013963 89.6451797485352 52.0
84.077750114 81.6148300170898 53.0
84.077750114 80.7368698120117 54.0
84.077750114 79.4060211181641 55.0
0 0.0851380228996277 56.0
80.152399818 78.9444122314453 57.0
227.21419645 220.812423706055 58.0
253.42422825 251.895477294922 59.0
21.477799403 19.8039436340332 60.0
0 1.67854022979736 61.0
152.59039391 166.497375488281 0.0
152.59039391 168.7978515625 1.0
151.39594385 158.664428710938 2.0
146.60687358 149.543151855469 3.0
0 0.909300744533539 4.0
0 -1.10960388183594 5.0
-58.513565211 -58.5779418945312 6.0
70.045552392 75.7007446289062 7.0
53.619430813 52.5391387939453 8.0
16.426121571 22.8161964416504 9.0
85.103068342 88.5177230834961 10.0
72.078306929 76.414924621582 11.0
79.84592605 84.4225921630859 12.0
79.84592605 80.5408554077148 13.0
79.84592605 80.5255661010742 14.0
0 0.176242351531982 15.0
0 0.16450560092926 16.0
0 -2.5583701133728 17.0
60.831894955 59.0853042602539 18.0
-60.831894967 -59.3462905883789 19.0
-60.831894945 -56.9251136779785 20.0
-60.831894957 -57.704761505127 21.0
82.173842737 77.7776260375977 22.0
146.44946739 141.921157836914 23.0
111.48461891 109.728927612305 24.0
22.952146044 20.7627468109131 25.0
228.62331014 221.021911621094 26.0
228.62331014 224.214797973633 27.0
26.392745468 27.4309024810791 28.0
255.01605562 255.971313476562 29.0
255.01605562 258.235321044922 30.0
46.853048662 45.9834365844727 31.0
9.3156416562 9.21270942687988 32.0
21.341947775 17.9251537322998 33.0
21.341947775 23.872278213501 34.0
21.341947774 20.0875301361084 35.0
9.1949672258 7.04130792617798 36.0
7.9492724968 7.48486709594727 37.0
-2.5024104086e-09 0.985273480415344 38.0
-2.5024109245e-09 -5.51380729675293 39.0
-11.63015152 -13.9507274627686 40.0
-24.440412558 -29.1632518768311 41.0
5.2892996002 5.00957250595093 42.0
-32.256858831 -32.4203720092773 43.0
104.94993807 97.1271286010742 44.0
104.94993808 94.9040756225586 45.0
6.8363927332 5.30925512313843 46.0
2.4111979429 2.50229525566101 47.0
152.59039392 165.352188110352 48.0
146.60687359 147.668182373047 49.0
0 -0.649416506290436 50.0
70.045552399 77.9643630981445 51.0
85.103068349 88.6096878051758 52.0
79.845926057 80.6585464477539 53.0
79.845926057 80.6819839477539 54.0
79.845926057 79.7787017822266 55.0
0 1.62002897262573 56.0
82.173842746 80.4518890380859 57.0
228.62331015 222.355941772461 58.0
255.01605562 249.946685791016 59.0
21.341947783 17.3087425231934 60.0
0 1.52439713478088 61.0
175.312286046877 168.967971801758 0.0
175.312286046877 160.23942565918 1.0
174.06990004 161.14533996582 2.0
169.2808303 171.259918212891 3.0
0 0.151144921779633 4.0
-6.11104077e-09 -2.99939703941345 5.0
-64.4514555966437 -59.8633422851562 6.0
75.0805405132529 75.9342803955078 7.0
47.2598739692529 54.2783317565918 8.0
27.8206665432529 25.0394229888916 9.0
89.6410683252529 86.729606628418 10.0
76.9350420992529 76.3420867919922 11.0
78.9241931172529 81.9688262939453 12.0
78.9241931172529 79.6647186279297 13.0
78.9241931172529 80.9153671264648 14.0
0 -1.46726846694946 15.0
0 -1.35265469551086 16.0
-6.33361835e-10 -1.98108553886414 17.0
59.4058047052288 59.2813987731934 18.0
-59.4058047028793 -59.7793273925781 19.0
-59.4058046818793 -59.1364250183105 20.0
-59.4058046838793 -58.1208915710449 21.0
81.2934113997981 77.3223495483398 22.0
145.095019229229 140.561813354492 23.0
111.915792069036 110.54133605957 24.0
23.202618791036 21.1786460876465 25.0
226.388430633753 231.157455444336 26.0
226.388430633753 223.825042724609 27.0
28.1270734247533 25.7983207702637 28.0
254.515504053753 256.004364013672 29.0
254.515504053753 255.06867980957 30.0
46.683776491229 46.5096740722656 31.0
9.2858371405437 8.80064678192139 32.0
21.887606715931 17.8186683654785 33.0
21.887606715931 12.2479686737061 34.0
21.8876067282251 20.4433555603027 35.0
9.35849906303128 6.45709609985352 36.0
8.01944723233128 7.34612894058228 37.0
-8.210452878e-09 -2.31416368484497 38.0
-8.21045287399999e-09 -3.22738599777222 39.0
-11.4815566869397 -10.1720218658447 40.0
-24.589932038993 -24.023265838623 41.0
4.87708110181073 5.67755031585693 42.0
-31.9939758597107 -34.1001472473145 43.0
92.5676851137107 89.748664855957 44.0
92.5676850737107 93.7062225341797 45.0
6.9609343856 4.96434354782104 46.0
2.5257442115 2.48931074142456 47.0
175.31228604 158.530792236328 48.0
169.28083031 166.05110168457 49.0
0 1.21560835838318 50.0
75.080540513 76.8617172241211 51.0
89.641068326 86.0822067260742 52.0
78.924193118 81.8149795532227 53.0
78.924193118 81.4288177490234 54.0
78.924193118 80.2298278808594 55.0
0 0.428759396076202 56.0
81.2934114 79.9225769042969 57.0
226.38843063 222.769989013672 58.0
254.51550406 248.287368774414 59.0
21.887606728 22.7359676361084 60.0
0 2.31894707679749 61.0
168.25705184 159.25309753418 0.0
168.25705184 165.181335449219 1.0
166.92505394 164.089904785156 2.0
162.13598422 163.729019165039 3.0
0 -9.85237789154053 4.0
-6.065795364e-10 0.55745142698288 5.0
-59.098451575 -47.5934524536133 6.0
70.781687622 79.5775756835938 7.0
51.948149445 57.6776084899902 8.0
18.833538175 19.1163120269775 9.0
88.042675719 86.2280502319336 10.0
73.997487828 74.1962203979492 11.0
75.868244934 82.5213165283203 12.0
75.868244934 75.388313293457 13.0
75.868244934 77.5743026733398 14.0
0 -1.71764492988586 15.0
0 -0.123538017272949 16.0
-4.3580546961e-09 1.72296726703644 17.0
58.35013101 59.0747451782227 18.0
-58.3501310347434 -57.3387413024902 19.0
-58.3501310127434 -62.5521621704102 20.0
-58.3501310237434 -58.8787002563477 21.0
79.509145486 80.8163833618164 22.0
143.31860988 141.008285522461 23.0
105.22087979 105.895240783691 24.0
23.461065379 20.1135196685791 25.0
222.82775538 222.622589111328 26.0
222.82775538 223.238647460938 27.0
23.635189801 25.3444747924805 28.0
246.46294518 259.759704589844 29.0
246.46294518 251.848266601562 30.0
47.271561863 46.7089157104492 31.0
9.3926228498 9.33640766143799 32.0
21.159014467 18.3085136413574 33.0
21.159014467 17.9865703582764 34.0
21.159014464 20.7084522247314 35.0
9.4933792594 8.3939905166626 36.0
8.3087865694 7.24747467041016 37.0
-7.7017154178e-09 -3.5925669670105 38.0
-7.7017170886e-09 -4.65700435638428 39.0
-12.73859098 -11.6193008422852 40.0
-25.459366641 -30.7911567687988 41.0
4.7013831644 -1.87295913696289 42.0
-32.614107298 -30.3409824371338 43.0
100.13284098 92.2936630249023 44.0
100.13284098 95.1883010864258 45.0
6.3773185425 4.92675685882568 46.0
2.3128723003 2.34215688705444 47.0
168.25705185 165.015441894531 48.0
162.13598423 154.453918457031 49.0
0 4.28343486785889 50.0
70.781687623 70.0048217773438 51.0
88.042675721 84.182487487793 52.0
75.868244936 81.6436157226562 53.0
75.868244936 77.2946472167969 54.0
75.868244936 79.792350769043 55.0
0 2.69205546379089 56.0
79.509145494 80.1337661743164 57.0
222.82775538 220.732162475586 58.0
246.46294518 248.306884765625 59.0
21.159014471 21.6225986480713 60.0
0 1.68316686153412 61.0
155.15525905 163.506729125977 0.0
155.15525905 149.892150878906 1.0
154.1244628 150.587326049805 2.0
149.33539307 150.406372070312 3.0
0 4.71155309677124 4.0
0 -4.95639801025391 5.0
-70.003426944 -75.3766021728516 6.0
82.744949211 80.581184387207 7.0
58.502875023 59.8817024230957 8.0
24.242074178 26.6077518463135 9.0
88.090973925 89.7896423339844 10.0
74.731383751 78.3926849365234 11.0
74.89268264 83.2743377685547 12.0
74.89268264 83.0963516235352 13.0
74.89268264 82.1998977661133 14.0
0 -3.54033899307251 15.0
0 -0.0879197120666504 16.0
0 1.59580886363983 17.0
54.037359791 55.6459121704102 18.0
-54.037359811 -59.389045715332 19.0
-54.037359789 -57.2445449829102 20.0
-54.037359802 -57.7747116088867 21.0
75.844202848 71.9645919799805 22.0
142.15625628 136.301284790039 23.0
108.62555201 113.205749511719 24.0
22.546866841 21.4203929901123 25.0
218.00045914 229.746124267578 26.0
218.00045914 220.142486572266 27.0
23.552534644 23.2411785125732 28.0
241.5529938 246.590530395508 29.0
241.5529938 253.643508911133 30.0
47.107978499 44.4222946166992 31.0
9.400686324 8.48046112060547 32.0
21.806843036 15.898042678833 33.0
21.806843036 13.7953996658325 34.0
21.806843036 18.4409675598145 35.0
9.1185282773 10.2734909057617 36.0
7.912128534 8.51003646850586 37.0
0 -2.67721462249756 38.0
0 0.238736093044281 39.0
-12.515977313 -16.7426204681396 40.0
-26.060022455 -27.9819736480713 41.0
5.5550243411 1.31806802749634 42.0
-34.299196395 -33.0994529724121 43.0
84.426811656 82.4141693115234 44.0
84.426811669 94.4920349121094 45.0
7.3191168417 3.26004767417908 46.0
2.2555979722 2.17806911468506 47.0
155.15525906 147.248291015625 48.0
149.33539308 137.331634521484 49.0
0 2.90389442443848 50.0
82.744949221 76.1030578613281 51.0
88.090973935 81.3979949951172 52.0
74.89268265 81.1176223754883 53.0
74.89268265 80.5804748535156 54.0
74.89268265 80.2389221191406 55.0
0 2.31572794914246 56.0
75.844202858 80.0537414550781 57.0
218.00045915 220.256118774414 58.0
241.55299381 245.757690429688 59.0
21.806843046 23.7047576904297 60.0
0 0.708923518657684 61.0
165.351029781406 157.878555297852 0.0
165.351029781406 162.381286621094 1.0
164.08770701 165.633880615234 2.0
158.62307174 155.306137084961 3.0
1.0832880939 2.47745299339294 4.0
1.08328810226321 8.18509292602539 5.0
-67.7555482903632 -61.1804313659668 6.0
78.5853610227712 75.5879516601562 7.0
49.4458210727712 51.3133583068848 8.0
29.1395399497712 23.7947292327881 9.0
89.1073980397712 83.4504318237305 10.0
76.7986695777712 74.4040222167969 11.0
81.8736442707712 80.9371109008789 12.0
81.8736442707712 78.3630065917969 13.0
81.8736442707712 80.6130523681641 14.0
3.7602291097e-09 -0.0713506937026978 15.0
1.6272515452e-09 -0.0682880282402039 16.0
9.55669260000003e-09 1.25840258598328 17.0
57.8610660729661 59.728572845459 18.0
-57.8610660316569 -60.1925582885742 19.0
-57.8610660136569 -57.8545188903809 20.0
-57.8610659926569 -58.7698059082031 21.0
80.0843260790733 77.0896530151367 22.0
147.797166008966 141.997604370117 23.0
112.755926366208 108.210830688477 24.0
23.3683622332077 21.67160987854 25.0
227.881492092813 225.019256591797 26.0
227.881492092813 226.250411987305 27.0
26.901197311813 24.5954608917236 28.0
254.782689402813 256.296020507812 29.0
254.782689402813 257.09765625 30.0
46.4226135160305 46.9759521484375 31.0
9.28952276629632 8.97862720489502 32.0
22.2232600870896 21.1368560791016 33.0
22.2232600870896 25.127046585083 34.0
22.2232601261625 21.121654510498 35.0
9.86013207565584 8.47025203704834 36.0
8.62443095355584 7.59949207305908 37.0
-6.21446807000003e-09 -2.18093037605286 38.0
-6.21446807000003e-09 -2.29818391799927 39.0
-11.6745441648284 -14.040864944458 40.0
-24.6294752288498 -25.0429153442383 41.0
5.21946170480936 6.99011611938477 42.0
-32.3541859148094 -34.2182998657227 43.0
96.8442895768094 95.5808563232422 44.0
96.8442894218094 92.9817581176758 45.0
7.12667677344732 6.46695756912231 46.0
2.47454266054732 2.56767535209656 47.0
165.35102978 162.666442871094 48.0
158.62307175 153.150115966797 49.0
1.0832881023 1.3274472951889 50.0
78.585361023 77.6819915771484 51.0
89.10739804 88.1610565185547 52.0
81.873644271 81.6924667358398 53.0
81.873644271 81.0575866699219 54.0
81.873644271 80.1906585693359 55.0
9.556692601e-09 1.85244917869568 56.0
80.084326079 79.4581756591797 57.0
227.88149209 222.508331298828 58.0
254.7826894 251.537719726562 59.0
22.223260126 18.91335105896 60.0
0 1.36672592163086 61.0
147.982033057583 157.456405639648 0.0
147.982033057583 151.964782714844 1.0
146.808413813287 153.35823059082 2.0
142.01934397 150.788269042969 3.0
-1.55484310406e-09 -0.656918823719025 4.0
-1.55484309999997e-09 -2.8427472114563 5.0
-64.9693040339936 -64.3239364624023 6.0
77.6547829144777 74.1871948242188 7.0
61.2786926634778 55.4030647277832 8.0
16.3760902514777 7.20984888076782 9.0
97.5261141374777 89.7391510009766 10.0
83.6472874924777 79.6450042724609 11.0
85.2981411144777 83.8642730712891 12.0
85.2981411144777 76.4624557495117 13.0
85.2981411144777 81.5436782836914 14.0
2.3328021736e-08 -2.03496789932251 15.0
2.3328021735e-08 -1.6052393913269 16.0
5.464350236e-08 -2.72059202194214 17.0
56.2518018559467 59.0783615112305 18.0
-56.2518018797166 -57.0891418457031 19.0
-56.2518018707166 -59.2854194641113 20.0
-56.2518017707166 -58.115608215332 21.0
77.1875821073724 73.3798065185547 22.0
153.454771802806 145.008895874023 23.0
115.220596133508 115.750228881836 24.0
21.7005584165075 20.8819332122803 25.0
230.642353915167 228.381210327148 26.0
230.642353915167 229.708221435547 27.0
26.8883420391666 21.3647422790527 28.0
257.530695955167 253.23681640625 29.0
257.530695955167 255.336776733398 30.0
46.7385281475469 45.537841796875 31.0
9.28724497493877 8.4355525970459 32.0
20.9357803369809 22.6376686096191 33.0
20.9357803369809 19.0790691375732 34.0
20.935780467613 20.4264030456543 35.0
8.86054244871795 8.39873600006104 36.0
7.60121916461794 6.85336446762085 37.0
2.62465480000001e-09 1.1959273815155 38.0
2.62465480000001e-09 -1.17553234100342 39.0
-11.6034306368583 -12.5643882751465 40.0
-24.5599220844463 -26.7411251068115 41.0
4.75986482786468 1.92820739746094 42.0
-31.7734218810648 -31.6524333953857 43.0
89.4696777832616 86.4044494628906 44.0
89.4696781210649 91.9600372314453 45.0
7.05809588635643 4.443039894104 46.0
2.46335994485643 2.26466846466064 47.0
147.98203306 156.096160888672 48.0
142.01934397 153.043350219727 49.0
0 -3.74360084533691 50.0
77.654782914 76.0235214233398 51.0
97.526114137 92.0755157470703 52.0
85.298141114 81.5651702880859 53.0
85.298141114 80.2131118774414 54.0
85.298141114 79.9203643798828 55.0
5.4643502359e-08 3.07315587997437 56.0
77.187582108 79.4883346557617 57.0
230.64235391 222.583419799805 58.0
257.53069595 249.491134643555 59.0
20.935780468 19.6116352081299 60.0
2.6246548239e-09 3.64096260070801 61.0
174.15303409 157.279235839844 0.0
174.15303409 160.435073852539 1.0
173.11283797 188.722763061523 2.0
168.32376794 179.008636474609 3.0
0 149.807388305664 4.0
0 14.2312202453613 5.0
-57.336008764 -32.736873626709 6.0
65.505991111 99.3455123901367 7.0
51.531367083 72.5224075317383 8.0
13.974624018 27.1742191314697 9.0
77.766966209 80.9453125 10.0
64.826682822 76.0166397094727 11.0
70.003233323 78.7140197753906 12.0
70.003233323 79.4134902954102 13.0
70.003233323 80.5695114135742 14.0
0 0.736366510391235 15.0
0 0.79651540517807 16.0
0 2.6685779094696 17.0
63.26119569 59.2786827087402 18.0
-63.2611957 -59.1451377868652 19.0
-63.261195678 -56.9179801940918 20.0
-63.261195691 -53.3066139221191 21.0
84.275518744 79.3789215087891 22.0
150.0799346 127.939582824707 23.0
114.97869869 117.523170471191 24.0
24.449923276 18.861665725708 25.0
234.35545335 213.315399169922 26.0
234.35545335 211.889831542969 27.0
21.176236414 31.2151126861572 28.0
255.53168978 238.876724243164 29.0
255.53168978 246.240539550781 30.0
46.559009779 48.7621726989746 31.0
9.3555729033 7.33979606628418 32.0
21.014323044 17.3204956054688 33.0
21.014323044 19.3162212371826 34.0
21.014323044 26.9709396362305 35.0
9.1868983362 7.13414335250854 36.0
7.7424378729 8.54882621765137 37.0
-1.4574267484e-11 -11.0120944976807 38.0
-1.4574267484e-11 -2.90527200698853 39.0
-11.854278764 -14.7010307312012 40.0
-24.21655474 -29.0470561981201 41.0
5.108271771 1.41790819168091 42.0
-31.861586012 -29.7483711242676 43.0
104.86114983 75.0838623046875 44.0
104.86114984 95.3180847167969 45.0
7.922600275 15.0135850906372 46.0
2.5174040466 3.46135139465332 47.0
174.1530341 165.3505859375 48.0
168.32376795 173.761642456055 49.0
0 77.0171127319336 50.0
65.505991121 75.6341094970703 51.0
77.766966219 89.1083526611328 52.0
70.003233333 81.8452758789062 53.0
70.003233333 81.9692764282227 54.0
70.003233333 81.6793365478516 55.0
0 -1.07615303993225 56.0
84.275518754 81.8201904296875 57.0
234.35545336 215.044586181641 58.0
255.53168979 248.505950927734 59.0
21.014323054 32.7886695861816 60.0
0 0.994669377803802 61.0
174.08295871 163.053634643555 0.0
174.08295871 165.361099243164 1.0
172.63640235 169.467575073242 2.0
167.84733209 167.425018310547 3.0
0 -3.52621650695801 4.0
0 9.03551864624023 5.0
-54.586883846 -63.7568321228027 6.0
65.879081909 73.8586578369141 7.0
39.584431245 47.9568519592285 8.0
26.294650658 26.5436630249023 9.0
81.552870769 85.8021392822266 10.0
67.490767806 74.5012969970703 11.0
78.738077292 79.5194702148438 12.0
78.738077292 79.4128265380859 13.0
78.738077292 76.2885284423828 14.0
0 -2.35027074813843 15.0
0 -0.296773731708527 16.0
0 -0.975928366184235 17.0
57.500481168 54.1100311279297 18.0
-57.500481181 -58.024772644043 19.0
-57.500481159 -58.8975868225098 20.0
-57.500481171 -56.6237869262695 21.0
78.283021925 81.8481979370117 22.0
150.91708409 151.231781005859 23.0
115.72638434 99.4927825927734 24.0
24.195817424 21.3602848052979 25.0
229.20010603 218.124771118164 26.0
229.20010603 220.34831237793 27.0
28.489441499 24.171012878418 28.0
257.68954753 237.497436523438 29.0
257.68954753 244.817749023438 30.0
47.137005914 45.6305198669434 31.0
9.4779848409 9.50099754333496 32.0
20.782540752 17.1448230743408 33.0
20.782540752 19.5017223358154 34.0
20.782540751 19.1953201293945 35.0
8.718446329 4.36214017868042 36.0
7.4807384003 7.19095325469971 37.0
-2.5379428361e-09 -1.10380864143372 38.0
-2.5379429728e-09 2.69718647003174 39.0
-12.029213738 -16.5282421112061 40.0
-24.841378433 -27.6668338775635 41.0
4.9820272262 4.31374454498291 42.0
-32.326548335 -33.5197792053223 43.0
81.086491092 84.6238479614258 44.0
81.086491099 77.5110244750977 45.0
6.7726610789 8.07354640960693 46.0
2.506191642 2.68825793266296 47.0
174.08295872 167.606964111328 48.0
167.8473321 161.639175415039 49.0
0 1.0118362903595 50.0
65.879081915 76.3284912109375 51.0
81.552870775 88.5764236450195 52.0
78.738077298 81.0380401611328 53.0
78.738077298 77.4749298095703 54.0
78.738077298 77.2695236206055 55.0
1.207292863e-10 0.700113236904144 56.0
78.283021934 79.5949096679688 57.0
229.20010603 222.513214111328 58.0
257.68954754 242.689498901367 59.0
20.782540759 21.5993499755859 60.0
0 2.57099866867065 61.0
137.17687742 143.006195068359 0.0
137.17687742 143.04296875 1.0
135.86355256 143.451507568359 2.0
131.0744823 135.611175537109 3.0
0 -1.94344711303711 4.0
0 0.34082692861557 5.0
-59.42861149 -59.2128486633301 6.0
69.705106679 70.1770553588867 7.0
65.43029471 62.4528312683105 8.0
4.2748119615 8.48783111572266 9.0
78.669161727 80.2065200805664 10.0
65.229460649 71.2585144042969 11.0
77.500548395 80.8047180175781 12.0
77.500548395 77.9000701904297 13.0
77.500548395 76.9198608398438 14.0
0 -0.64198762178421 15.0
0 0.28728449344635 16.0
0 -3.15485000610352 17.0
59.754945382 58.5015258789062 18.0
-59.754945394 -59.4383087158203 19.0
-59.754945372 -60.0649642944336 20.0
-59.754945384 -61.8801422119141 21.0
80.656571382 77.8813858032227 22.0
140.87903669 140.394561767578 23.0
106.41719675 107.357681274414 24.0
22.90053999 20.4083709716797 25.0
221.53560808 223.594223022461 26.0
221.53560808 221.987747192383 27.0
27.282645659 25.3551406860352 28.0
248.81825375 253.574676513672 29.0
248.81825375 252.408157348633 30.0
47.033013387 47.2299919128418 31.0
9.3131031344 9.18030452728271 32.0
20.901625993 20.4819431304932 33.0
20.901625993 21.1129970550537 34.0
20.901625993 20.8195400238037 35.0
9.3083381265 8.28715133666992 36.0
8.044767894 8.38281917572021 37.0
-2.5262790395e-09 -0.804296553134918 38.0
-2.5262785966e-09 0.113371789455414 39.0
-12.387792857 -14.5510091781616 40.0
-23.81592281 -27.5234241485596 41.0
4.9278776077 3.5162615776062 42.0
-31.108913619 -34.7394485473633 43.0
92.052766479 91.3064651489258 44.0
92.05276649 90.2343292236328 45.0
6.7273095346 3.00315618515015 46.0
2.6568330431 1.83220052719116 47.0
137.17687743 136.688690185547 48.0
131.07448231 127.290840148926 49.0
0 3.57963562011719 50.0
69.705106686 70.9875259399414 51.0
78.669161734 84.2529830932617 52.0
77.500548402 81.0479965209961 53.0
77.500548402 80.9212951660156 54.0
77.500548402 79.4541549682617 55.0
0 -0.0822644829750061 56.0
80.656571391 80.3950500488281 57.0
221.53560809 218.26057434082 58.0
248.81825375 246.469757080078 59.0
20.901626002 19.9398994445801 60.0
0 2.89454770088196 61.0
169.27010139 168.25715637207 0.0
169.27010139 162.539276123047 1.0
168.24091778 153.278991699219 2.0
163.45184806 154.595245361328 3.0
0 0.672469079494476 4.0
-1.0109135791e-11 12.9834766387939 5.0
-60.580282342 -49.4930763244629 6.0
73.161915127 78.5560531616211 7.0
57.171618964 52.1862983703613 8.0
15.990296153 25.2701835632324 9.0
85.605417207 77.7727661132812 10.0
72.318309279 71.4398193359375 11.0
85.75948681 74.6589202880859 12.0
85.75948681 82.6873016357422 13.0
85.75948681 72.3928985595703 14.0
0 -1.85192918777466 15.0
0 0.376586854457855 16.0
-1.7312707342e-11 0.0777302980422974 17.0
53.276537165 57.3253479003906 18.0
-53.276537185 -58.6821784973145 19.0
-53.276537163 -53.4306030273438 20.0
-53.276537175 -60.1398162841797 21.0
74.237173942 75.5637741088867 22.0
153.48066716 145.539642333984 23.0
116.99074502 109.855537414551 24.0
23.130346851 19.9287967681885 25.0
227.71784111 218.220672607422 26.0
227.71784111 213.761825561523 27.0
29.607142156 27.8440036773682 28.0
257.32498328 245.226104736328 29.0
257.32498328 247.878204345703 30.0
46.903289153 47.5566558837891 31.0
9.2969014174 6.83636569976807 32.0
20.960636757 18.6979789733887 33.0
20.960636757 22.3353538513184 34.0
20.960636757 17.944465637207 35.0
8.8454532874 9.34835338592529 36.0
7.4765488743 9.22372150421143 37.0
-3.5287760078e-11 2.05488443374634 38.0
-3.5287763168e-11 3.96840333938599 39.0
-11.080992706 -12.8041372299194 40.0
-24.932509321 -23.1585235595703 41.0
5.0863548628 -2.91485667228699 42.0
-32.733906015 -31.3253612518311 43.0
93.46051945 95.7999420166016 44.0
93.460519463 93.5414352416992 45.0
6.8313614437 2.56629920005798 46.0
2.5901924429 1.38604354858398 47.0
169.2701014 164.61865234375 48.0
163.45184807 164.866653442383 49.0
0 2.2274317741394 50.0
73.161915137 77.8220672607422 51.0
85.605417217 87.2107849121094 52.0
85.759486819 80.4459457397461 53.0
85.759486819 81.0049362182617 54.0
85.759486819 80.0435562133789 55.0
0 -1.23136425018311 56.0
74.237173952 76.26904296875 57.0
227.71784112 220.241577148438 58.0
257.32498329 243.803512573242 59.0
20.960636767 16.3977470397949 60.0
0 2.85984873771667 61.0
180.085945367258 163.177841186523 0.0
180.085945367258 178.182571411133 1.0
179.17267164 133.123687744141 2.0
173.70803644 29.4496059417725 3.0
3.696483199 40.0556297302246 4.0
3.69648320873752 -14.4661645889282 5.0
-62.9539317057375 -67.5780639648438 6.0
70.6641483355813 68.9834976196289 7.0
33.1718339055813 66.2462615966797 8.0
37.4923144295813 40.6153373718262 9.0
88.6246806395813 92.835319519043 10.0
77.7090824635813 79.0045852661133 11.0
82.5347931115813 87.9532928466797 12.0
82.5347931115813 89.300537109375 13.0
82.5347931115813 75.0655212402344 14.0
0 5.21911096572876 15.0
0 -1.41556692123413 16.0
-1.276183582e-09 5.00787734985352 17.0
50.1882113134989 38.3934631347656 18.0
-50.188211317872 -52.7964897155762 19.0
-50.188211296872 -57.2694892883301 20.0
-50.188211303872 -62.0642967224121 21.0
69.9279228623463 62.282527923584 22.0
140.372753808499 115.96817779541 23.0
103.744293703124 93.113410949707 24.0
20.9735713041236 25.6098766326904 25.0
210.300676674517 207.897445678711 26.0
210.300676674517 199.422073364258 27.0
22.8491056755169 26.1998558044434 28.0
233.149782344517 256.641418457031 29.0
233.149782344517 237.781646728516 30.0
46.9784938015188 44.980785369873 31.0
9.2336646853 10.2628221511841 32.0
19.7397115582126 21.3796939849854 33.0
19.7397115582126 12.8534870147705 34.0
19.7397115644073 16.3345794677734 35.0
8.3880556272 7.9157886505127 36.0
7.0794475313 6.95694828033447 37.0
-9.190539568e-09 12.8772373199463 38.0
-9.190539567e-09 1.207026720047 39.0
-12.254075936436 -9.71978378295898 40.0
-25.1957986049566 -22.1212978363037 41.0
5.91471750505109 7.69311380386353 42.0
-33.7096052321511 -37.3771820068359 43.0
89.5700401841511 98.3935775756836 44.0
89.5700401701511 95.5712738037109 45.0
6.4345157345 8.23186588287354 46.0
1.8174191808 2.67862939834595 47.0
180.08594536 169.829330444336 48.0
173.70803645 142.833618164062 49.0
3.6964832087 9.22891139984131 50.0
70.664148335 72.1745758056641 51.0
88.62468064 83.9836273193359 52.0
82.534793112 80.401237487793 53.0
82.534793112 81.3976898193359 54.0
82.534793112 81.452262878418 55.0
0 -4.67055130004883 56.0
69.927922862 74.8558120727539 57.0
210.30067667 217.961273193359 58.0
233.14978235 242.936737060547 59.0
19.739711565 6.29499578475952 60.0
0 5.59589004516602 61.0
170.083878316469 157.816253662109 0.0
170.083878316469 157.453735351562 1.0
168.788706857946 158.770797729492 2.0
163.99963705 158.431747436523 3.0
0 -2.61057186126709 4.0
-3.64718998999996e-09 8.1868953704834 5.0
-65.9445630618226 -62.1393775939941 6.0
77.4109841448792 78.0552749633789 7.0
48.2079889088791 50.7772903442383 8.0
29.2029952368793 24.4059295654297 9.0
94.7684301998792 88.8115615844727 10.0
82.0481931238792 76.8876266479492 11.0
85.4560780108792 83.5412139892578 12.0
85.4560780108792 82.4553375244141 13.0
85.4560780108792 82.5894241333008 14.0
-2.0210446233e-08 -1.72448968887329 15.0
-1.2827776325e-08 -1.04012513160706 16.0
3.671273728e-08 -1.79126453399658 17.0
58.3359280796611 59.4617156982422 18.0
-58.3359279518188 -59.5928955078125 19.0
-58.3359279398188 -59.0153999328613 20.0
-58.3359278708188 -59.1716766357422 21.0
79.8271822287511 78.0592422485352 22.0
140.369856244661 141.005126953125 23.0
106.168739595827 109.724090576172 24.0
22.3465316288267 19.7794818878174 25.0
220.197038472937 224.808822631836 26.0
220.197038472937 218.165908813477 27.0
26.0808301069373 28.6898517608643 28.0
246.277868572937 254.500579833984 29.0
246.277868572937 253.694686889648 30.0
46.6820190386153 46.8048667907715 31.0
9.40345932024295 8.98416328430176 32.0
21.491254357468 20.675594329834 33.0
21.491254357468 16.6573123931885 34.0
21.4912544526478 20.3899593353271 35.0
9.24633800707493 7.62591743469238 36.0
7.93106650837494 7.20929384231567 37.0
-8.81501959999916e-10 0.475449562072754 38.0
-8.81501960000021e-10 -5.17730569839478 39.0
-12.2413252834096 -12.8781337738037 40.0
-25.1440221070403 -29.4673118591309 41.0
5.15225554073331 4.28948593139648 42.0
-32.8317131282327 -32.128345489502 43.0
104.570770318232 97.9997863769531 44.0
104.570769928232 95.3408660888672 45.0
6.75303592774189 5.77723550796509 46.0
2.6278217995419 2.73974990844727 47.0
170.08387832 163.024688720703 48.0
163.99963706 157.68879699707 49.0
0 0.86491858959198 50.0
77.410984145 76.664192199707 51.0
94.768430199 89.9916381835938 52.0
85.456078011 81.0958709716797 53.0
85.456078011 80.9692840576172 54.0
85.456078011 80.1057052612305 55.0
3.6712737279e-08 0.654493868350983 56.0
79.827182229 81.3045654296875 57.0
220.19703847 219.398086547852 58.0
246.27786858 248.994201660156 59.0
21.491254452 24.4201393127441 60.0
0 0.813128650188446 61.0
147.982033057583 157.456268310547 0.0
147.982033057583 151.960220336914 1.0
146.808413813287 153.321578979492 2.0
142.01934397 150.524429321289 3.0
-1.55484310406e-09 -0.625551521778107 4.0
-1.55484309999997e-09 -2.83146190643311 5.0
-64.9693040339936 -64.3126525878906 6.0
77.6547829144777 74.1870574951172 7.0
61.2786926634778 55.4030647277832 8.0
16.3760902514777 7.20984888076782 9.0
97.5261141374777 89.7391510009766 10.0
83.6472874924777 79.6450042724609 11.0
85.2981411144777 83.8642730712891 12.0
85.2981411144777 76.4624557495117 13.0
85.2981411144777 81.5436782836914 14.0
2.3328021736e-08 -2.03496789932251 15.0
2.3328021735e-08 -1.6052393913269 16.0
5.464350236e-08 -2.72059202194214 17.0
56.2518018559467 59.0783615112305 18.0
-56.2518018797166 -57.0891418457031 19.0
-56.2518018707166 -59.2854194641113 20.0
-56.2518017707166 -58.115608215332 21.0
77.1875821073724 73.3798065185547 22.0
153.454771802806 145.008895874023 23.0
115.220596133508 115.750228881836 24.0
21.7005584165075 20.8819332122803 25.0
230.642353915167 228.381210327148 26.0
230.642353915167 229.708221435547 27.0
26.8883420391666 21.3647422790527 28.0
257.530695955167 253.23681640625 29.0
257.530695955167 255.336776733398 30.0
46.7385281475469 45.537841796875 31.0
9.28724497493877 8.4355525970459 32.0
20.9357803369809 22.6376686096191 33.0
20.9357803369809 19.0790691375732 34.0
20.935780467613 20.4264030456543 35.0
8.86054244871795 8.39873600006104 36.0
7.60121916461794 6.85336446762085 37.0
2.62465480000001e-09 1.1959273815155 38.0
2.62465480000001e-09 -1.17553234100342 39.0
-11.6034306368583 -12.5643882751465 40.0
-24.5599220844463 -26.7411251068115 41.0
4.75986482786468 1.92820739746094 42.0
-31.7734218810648 -31.6524333953857 43.0
89.4696777832616 86.4044494628906 44.0
89.4696781210649 91.9600372314453 45.0
7.05809588635643 4.443039894104 46.0
2.46335994485643 2.26466846466064 47.0
147.98203306 156.098007202148 48.0
142.01934397 152.82453918457 49.0
0 -3.76043605804443 50.0
77.654782914 76.0241928100586 51.0
97.526114137 92.0755157470703 52.0
85.298141114 81.5651702880859 53.0
85.298141114 80.2131118774414 54.0
85.298141114 79.9203643798828 55.0
5.4643502359e-08 3.07315587997437 56.0
77.187582108 79.4883346557617 57.0
230.64235391 222.583419799805 58.0
257.53069595 249.491134643555 59.0
20.935780468 19.6116352081299 60.0
2.6246548239e-09 3.64096260070801 61.0
137.17687742 143.006042480469 0.0
137.17687742 143.051696777344 1.0
135.86355256 143.390106201172 2.0
131.0744823 135.498626708984 3.0
0 -1.81451749801636 4.0
0 0.339907050132751 5.0
-59.42861149 -59.1968727111816 6.0
69.705106679 70.1768035888672 7.0
65.43029471 62.4528312683105 8.0
4.2748119615 8.48783111572266 9.0
78.669161727 80.2065200805664 10.0
65.229460649 71.2585144042969 11.0
77.500548395 80.8047180175781 12.0
77.500548395 77.9000701904297 13.0
77.500548395 76.9198608398438 14.0
0 -0.64198762178421 15.0
0 0.28728449344635 16.0
0 -3.15485000610352 17.0
59.754945382 58.5015258789062 18.0
-59.754945394 -59.4383087158203 19.0
-59.754945372 -60.0649642944336 20.0
-59.754945384 -61.8801422119141 21.0
80.656571382 77.8813858032227 22.0
140.87903669 140.394561767578 23.0
106.41719675 107.357681274414 24.0
22.90053999 20.4083709716797 25.0
221.53560808 223.594223022461 26.0
221.53560808 221.987747192383 27.0
27.282645659 25.3551406860352 28.0
248.81825375 253.574676513672 29.0
248.81825375 252.408157348633 30.0
47.033013387 47.2299919128418 31.0
9.3131031344 9.18030452728271 32.0
20.901625993 20.4819431304932 33.0
20.901625993 21.1129970550537 34.0
20.901625993 20.8195400238037 35.0
9.3083381265 8.28715133666992 36.0
8.044767894 8.38281917572021 37.0
-2.5262790395e-09 -0.804296553134918 38.0
-2.5262785966e-09 0.113371789455414 39.0
-12.387792857 -14.5510091781616 40.0
-23.81592281 -27.5234241485596 41.0
4.9278776077 3.5162615776062 42.0
-31.108913619 -34.7394485473633 43.0
92.052766479 91.3064651489258 44.0
92.05276649 90.2343292236328 45.0
6.7273095346 3.00315618515015 46.0
2.6568330431 1.83220052719116 47.0
137.17687743 136.693725585938 48.0
131.07448231 127.106163024902 49.0
0 3.60570907592773 50.0
69.705106686 70.9884185791016 51.0
78.669161734 84.2529830932617 52.0
77.500548402 81.0479965209961 53.0
77.500548402 80.9212951660156 54.0
77.500548402 79.4541549682617 55.0
0 -0.0822644829750061 56.0
80.656571391 80.3950500488281 57.0
221.53560809 218.26057434082 58.0
248.81825375 246.469757080078 59.0
20.901626002 19.9398994445801 60.0
0 2.89454770088196 61.0
162.85757008 168.048934936523 0.0
162.85757008 171.743576049805 1.0
161.59781867 160.362609863281 2.0
156.80874895 153.022537231445 3.0
0 -2.34056806564331 4.0
-2.092613687e-12 -2.06375503540039 5.0
-68.62114863 -67.0987167358398 6.0
79.219081044 72.4851608276367 7.0
56.750475929 52.9640197753906 8.0
22.468605105 25.1544570922852 9.0
84.115252491 86.1191101074219 10.0
70.73289663 73.9639282226562 11.0
80.512618814 80.9126052856445 12.0
80.512618814 80.520133972168 13.0
80.512618814 80.0003662109375 14.0
0 0.138406097888947 15.0
0 0.0945003628730774 16.0
-3.9005268992e-12 0.361804664134979 17.0
62.079443818 59.164306640625 18.0
-62.079443838 -60.022087097168 19.0
-62.079443816 -59.5993537902832 20.0
-62.079443829 -61.2314796447754 21.0
83.800853031 79.0967254638672 22.0
143.56518206 141.190338134766 23.0
111.08825492 110.990364074707 24.0
22.102595507 20.9055976867676 25.0
227.3660351 228.187683105469 26.0
227.3660351 220.825866699219 27.0
27.151956572 27.458568572998 28.0
254.51799168 255.203247070312 29.0
254.51799168 252.875152587891 30.0
46.445063337 45.4638214111328 31.0
9.2801016785 8.85960102081299 32.0
21.721409192 21.5966873168945 33.0
21.721409192 18.4746513366699 34.0
21.721409192 19.161958694458 35.0
9.1915613704 7.43347930908203 36.0
7.9803120314 7.94640445709229 37.0
-6.2509432785e-12 2.74730730056763 38.0
-6.2509432785e-12 -0.45326167345047 39.0
-11.568147451 -12.2604560852051 40.0
-23.95019328 -24.429651260376 41.0
5.2798150624 5.21954727172852 42.0
-31.870696421 -33.4950752258301 43.0
96.061378949 90.5321578979492 44.0
96.061378962 92.3324508666992 45.0
6.9557535068 7.72731447219849 46.0
2.4010232192 2.51868844032288 47.0
162.85757009 167.042449951172 48.0
156.80874896 153.837890625 49.0
0 0.986276388168335 50.0
79.219081054 74.1609573364258 51.0
84.115252501 86.2581100463867 52.0
80.512618824 80.298942565918 53.0
80.512618824 81.1014175415039 54.0
80.512618824 79.3614654541016 55.0
0 0.294139325618744 56.0
83.800853041 79.6522903442383 57.0
227.36603511 220.93196105957 58.0
254.51799169 249.336181640625 59.0
21.721409202 19.2826614379883 60.0
0 1.40553545951843 61.0
191.01697008 172.522674560547 0.0
191.01697008 152.431625366211 1.0
189.58354118 177.738906860352 2.0
184.79447114 186.149856567383 3.0
0 -22.4192276000977 4.0
-1.3327987697e-10 -3.04034113883972 5.0
-68.142043271 -71.3364944458008 6.0
77.809135988 71.7424011230469 7.0
62.996940906 50.9178199768066 8.0
14.812195073 16.4648342132568 9.0
90.280907649 87.7635345458984 10.0
76.556814964 77.198371887207 11.0
78.845418231 88.4357147216797 12.0
78.845418231 79.1166305541992 13.0
78.845418231 80.3032913208008 14.0
0 -1.13258099555969 15.0
0 0.176832318305969 16.0
0 3.57990503311157 17.0
58.277423471 58.944221496582 18.0
-58.277423482 -58.5585975646973 19.0
-58.27742346 -57.1651077270508 20.0
-58.277423472 -57.9702339172363 21.0
79.151883105 77.5008697509766 22.0
136.02159316 141.9638671875 23.0
103.65807874 108.820678710938 24.0
23.873551201 20.9192504882812 25.0
215.17347628 227.357177734375 26.0
215.17347628 222.578567504883 27.0
26.19532582 25.4187965393066 28.0
241.3688021 251.541870117188 29.0
241.3688021 245.628753662109 30.0
46.730193864 47.0511054992676 31.0
9.4322005595 7.00126457214355 32.0
20.874459624 21.3104610443115 33.0
20.874459624 15.2398481369019 34.0
20.874459624 18.9794425964355 35.0
9.4116623757 7.54806518554688 36.0
8.2549799295 7.53655529022217 37.0
-5.9335209366e-10 1.44418025016785 38.0
-5.9347829989e-10 0.911677241325378 39.0
-11.144049162 -19.7990436553955 40.0
-24.346216315 -24.6967487335205 41.0
5.1058567699 4.60519933700562 42.0
-31.86136454 -32.536190032959 43.0
90.516620084 89.0560455322266 44.0
90.516620096 89.8517532348633 45.0
7.3245523412 3.49604249000549 46.0
2.4288374168 2.14216756820679 47.0
191.01697009 159.136032104492 48.0
184.79447115 162.280151367188 49.0
0 2.44088864326477 50.0
77.809135998 73.3010940551758 51.0
90.280907659 81.3495941162109 52.0
78.84541824 81.5682754516602 53.0
78.84541824 80.6553192138672 54.0
78.84541824 80.0597381591797 55.0
0 -4.47466087341309 56.0
79.151883115 80.64892578125 57.0
215.17347628 222.893325805664 58.0
241.36880211 237.154342651367 59.0
20.874459634 17.8611640930176 60.0
0 -0.290941774845123 61.0
152.524167139285 161.566421508789 0.0
152.524167139285 166.802764892578 1.0
151.35996019 159.588134765625 2.0
146.57089042 149.170959472656 3.0
0 2.72538709640503 4.0
-5.13071198e-09 3.58576822280884 5.0
-59.53115247936 -58.3728713989258 6.0
70.2507919487599 73.059700012207 7.0
47.7617113627599 52.3051528930664 8.0
22.4890805857599 22.5464820861816 9.0
89.8720038897599 86.200553894043 10.0
76.7725129637599 74.7876510620117 11.0
84.1030283177599 83.5107803344727 12.0
84.1030283177599 81.3520660400391 13.0
84.1030283177599 81.384880065918 14.0
4.8380164139e-09 -1.74889588356018 15.0
2.8396886315e-09 -1.23449850082397 16.0
1.106254265e-08 -2.39599061012268 17.0
56.9242230005054 59.1188087463379 18.0
-56.9242229541398 -58.7474822998047 19.0
-56.9242229371398 -60.8387451171875 20.0
-56.9242229121398 -59.8430595397949 21.0
78.9448309813946 78.8030319213867 22.0
146.934885677505 142.000686645508 23.0
114.237803181882 114.043357849121 24.0
23.5512008828817 20.2917079925537 25.0
225.879716658569 230.664596557617 26.0
225.879716658569 224.718948364258 27.0
25.2255969145691 25.6801815032959 28.0
251.105313568569 253.560928344727 29.0
251.105313568569 254.459930419922 30.0
46.9430737789106 46.5697174072266 31.0
9.34221855687645 8.89533805847168 32.0
22.0206080690799 20.3247928619385 33.0
22.0206080690799 17.7073993682861 34.0
22.0206081119172 21.5819072723389 35.0
9.92452194042601 9.5117883682251 36.0
8.63421670852602 7.30539178848267 37.0
-5.88814005000002e-09 -5.77714157104492 38.0
-5.88814005000002e-09 -5.02064228057861 39.0
-12.0646002710699 -12.758918762207 40.0
-24.828896422166 -27.8423080444336 41.0
5.00332440529989 5.63134098052979 42.0
-32.4500067182999 -34.2524604797363 43.0
92.7446581152999 89.7777099609375 44.0
92.7446577858464 95.4271469116211 45.0
6.91667725158936 5.73279666900635 46.0
2.42208655028936 2.3217933177948 47.0
152.52416714 162.400604248047 48.0
146.57089043 142.361236572266 49.0
0 1.32731580734253 50.0
70.250791949 77.4059295654297 51.0
89.872003889 88.4692153930664 52.0
84.103028317 81.2791595458984 53.0
84.103028317 81.4317092895508 54.0
84.103028317 80.59765625 55.0
1.106254265e-08 0.0498611330986023 56.0
78.944830981 79.5637359619141 57.0
225.87971666 222.978210449219 58.0
251.10531357 249.396270751953 59.0
22.020608111 24.7376899719238 60.0
0 1.85762310028076 61.0
147.982033057583 157.456451416016 0.0
147.982033057583 151.966705322266 1.0
146.808413813287 153.373962402344 2.0
142.01934397 150.906112670898 3.0
-1.55484310406e-09 -0.67039567232132 4.0
-1.55484309999997e-09 -2.84772348403931 5.0
-64.9693040339936 -64.3290252685547 6.0
77.6547829144777 74.1872634887695 7.0
61.2786926634778 55.4030647277832 8.0
16.3760902514777 7.20984888076782 9.0
97.5261141374777 89.7391510009766 10.0
83.6472874924777 79.6450042724609 11.0
85.2981411144777 83.8642730712891 12.0
85.2981411144777 76.4624557495117 13.0
85.2981411144777 81.5436782836914 14.0
2.3328021736e-08 -2.03496789932251 15.0
2.3328021735e-08 -1.6052393913269 16.0
5.464350236e-08 -2.72059202194214 17.0
56.2518018559467 59.0783615112305 18.0
-56.2518018797166 -57.0891418457031 19.0
-56.2518018707166 -59.2854194641113 20.0
-56.2518017707166 -58.115608215332 21.0
77.1875821073724 73.3798065185547 22.0
153.454771802806 145.008895874023 23.0
115.220596133508 115.750228881836 24.0
21.7005584165075 20.8819332122803 25.0
230.642353915167 228.381210327148 26.0
230.642353915167 229.708221435547 27.0
26.8883420391666 21.3647422790527 28.0
257.530695955167 253.23681640625 29.0
257.530695955167 255.336776733398 30.0
46.7385281475469 45.537841796875 31.0
9.28724497493877 8.4355525970459 32.0
20.9357803369809 22.6376686096191 33.0
20.9357803369809 19.0790691375732 34.0
20.935780467613 20.4264030456543 35.0
8.86054244871795 8.39873600006104 36.0
7.60121916461794 6.85336446762085 37.0
2.62465480000001e-09 1.1959273815155 38.0
2.62465480000001e-09 -1.17553234100342 39.0
-11.6034306368583 -12.5643882751465 40.0
-24.5599220844463 -26.7411251068115 41.0
4.75986482786468 1.92820739746094 42.0
-31.7734218810648 -31.6524333953857 43.0
89.4696777832616 86.4044494628906 44.0
89.4696781210649 91.9600372314453 45.0
7.05809588635643 4.443039894104 46.0
2.46335994485643 2.26466846466064 47.0
147.98203306 156.095336914062 48.0
142.01934397 153.139389038086 49.0
0 -3.73605728149414 50.0
77.654782914 76.0232543945312 51.0
97.526114137 92.0755157470703 52.0
85.298141114 81.5651702880859 53.0
85.298141114 80.2131118774414 54.0
85.298141114 79.9203643798828 55.0
5.4643502359e-08 3.07315587997437 56.0
77.187582108 79.4883346557617 57.0
230.64235391 222.583419799805 58.0
257.53069595 249.491134643555 59.0
20.935780468 19.6116352081299 60.0
2.6246548239e-09 3.64096260070801 61.0
154.94214544 166.212432861328 0.0
154.94214544 168.091445922852 1.0
153.62589673 167.219589233398 2.0
148.83682701 157.139724731445 3.0
0 3.65104961395264 4.0
-4.9505104376e-10 2.73673295974731 5.0
-66.795543015 -62.8263244628906 6.0
77.888199916 78.7957305908203 7.0
55.413902211 52.0081825256348 8.0
22.474297703 26.0417633056641 9.0
95.706300325 87.4950256347656 10.0
82.344258531 76.6533813476562 11.0
83.793651688 83.0808029174805 12.0
83.793651688 80.3548049926758 13.0
83.793651688 82.2523345947266 14.0
0 0.904163360595703 15.0
0 0.118558824062347 16.0
-4.0711194556e-09 -0.081396758556366 17.0
58.272470789 59.6661109924316 18.0
-58.2724708132059 -59.9852256774902 19.0
-58.2724707912059 -60.0485877990723 20.0
-58.2724708022059 -61.9626693725586 21.0
79.078352293 78.2061920166016 22.0
146.48641082 142.542114257812 23.0
112.41539298 111.946632385254 24.0
23.128940161 20.5876731872559 25.0
225.56476313 227.659149169922 26.0
225.56476313 227.675369262695 27.0
27.020278561 24.2825508117676 28.0
252.58504169 252.99772644043 29.0
252.58504169 255.526184082031 30.0
46.640486506 48.6040306091309 31.0
9.3728432198 8.63383388519287 32.0
20.805881496 21.6275863647461 33.0
20.805881496 18.8017044067383 34.0
20.805881493 20.760196685791 35.0
9.2008137471 6.86174058914185 36.0
7.88975615 7.60456466674805 37.0
-7.2708282969e-09 0.856835246086121 38.0
-7.2708377778e-09 -1.82655334472656 39.0
-12.095975605 -15.1782827377319 40.0
-25.163325798 -24.8636951446533 41.0
4.9047260102 5.71446466445923 42.0
-32.708732529 -34.7498626708984 43.0
91.528294481 90.3782424926758 44.0
91.528294482 92.1180877685547 45.0
6.6696959989 6.33383512496948 46.0
2.6088884578 2.52103924751282 47.0
154.94214545 168.774185180664 48.0
148.83682702 151.686676025391 49.0
0 0.956853687763214 50.0
77.888199918 75.6590118408203 51.0
95.706300327 89.6363754272461 52.0
83.79365169 81.8090438842773 53.0
83.79365169 81.0633850097656 54.0
83.79365169 80.0689010620117 55.0
0 0.68585079908371 56.0
79.078352301 80.0451049804688 57.0
225.56476313 222.409576416016 58.0
252.58504169 247.320617675781 59.0
20.8058815 22.6847305297852 60.0
0 1.50079369544983 61.0
156.29145535 168.717666625977 0.0
156.29145535 168.816070556641 1.0
154.95797318 158.868072509766 2.0
150.16890346 153.175170898438 3.0
0 0.694754064083099 4.0
-1.5763668704e-13 -0.321757853031158 5.0
-59.513142545 -50.2802238464355 6.0
71.065195419 81.9145736694336 7.0
53.608585529 51.4097175598145 8.0
17.45660988 25.2165222167969 9.0
84.470516169 87.1336441040039 10.0
71.752746382 76.3091201782227 11.0
78.239808874 84.8103179931641 12.0
78.239808874 80.8271102905273 13.0
78.239808874 81.7426681518555 14.0
0 -0.275531589984894 15.0
0 0.0921301245689392 16.0
-2.7799551485e-13 -0.831766784191132 17.0
59.187696035 59.3059425354004 18.0
-59.187696055 -59.419261932373 19.0
-59.187696033 -60.0270652770996 20.0
-59.187696045 -60.0391654968262 21.0
81.324000119 79.2195053100586 22.0
142.34889543 143.373443603516 23.0
109.79940278 109.305633544922 24.0
22.592001452 19.3882579803467 25.0
223.67289556 227.238174438477 26.0
223.67289556 225.819610595703 27.0
27.935162937 24.3881702423096 28.0
251.60805851 255.880599975586 29.0
251.60805851 254.912979125977 30.0
46.395465018 47.2871246337891 31.0
9.2843642059 8.65455150604248 32.0
22.136304064 20.9917125701904 33.0
22.136304064 19.836986541748 34.0
22.136304064 21.0628108978271 35.0
9.678753604 7.50641822814941 36.0
8.4100230818 6.72224760055542 37.0
-2.862309201e-13 -3.3115177154541 38.0
-2.8623092346e-13 -1.6782693862915 39.0
-11.364842801 -16.3675079345703 40.0
-23.765676154 -24.6364936828613 41.0
5.1593674889 6.23309659957886 42.0
-31.489768091 -32.9559059143066 43.0
93.543804157 96.731071472168 44.0
93.54380417 96.1247100830078 45.0
6.9857098664 5.84714412689209 46.0
2.5929403942 1.28131520748138 47.0
156.29145536 169.883926391602 48.0
150.16890347 152.11328125 49.0
0 0.163073718547821 50.0
71.065195429 76.8097305297852 51.0
84.470516179 90.7965850830078 52.0
78.239808884 79.955078125 53.0
78.239808884 81.9208755493164 54.0
78.239808884 80.4639282226562 55.0
0 2.61783266067505 56.0
81.324000129 80.0384368896484 57.0
223.67289557 222.408294677734 58.0
251.60805852 248.074584960938 59.0
22.136304074 22.983512878418 60.0
0 2.05088663101196 61.0
137.17687742 143.006484985352 0.0
137.17687742 143.032318115234 1.0
135.86355256 143.526748657227 2.0
131.0744823 135.750274658203 3.0
0 -2.09861373901367 4.0
0 0.341894507408142 5.0
-59.42861149 -59.2324256896973 6.0
69.705106679 70.1772766113281 7.0
65.43029471 62.4528312683105 8.0
4.2748119615 8.48783111572266 9.0
78.669161727 80.2065200805664 10.0
65.229460649 71.2585144042969 11.0
77.500548395 80.8047180175781 12.0
77.500548395 77.9000701904297 13.0
77.500548395 76.9198608398438 14.0
0 -0.64198762178421 15.0
0 0.28728449344635 16.0
0 -3.15485000610352 17.0
59.754945382 58.5015258789062 18.0
-59.754945394 -59.4383087158203 19.0
-59.754945372 -60.0649642944336 20.0
-59.754945384 -61.8801422119141 21.0
80.656571382 77.8813858032227 22.0
140.87903669 140.394561767578 23.0
106.41719675 107.357681274414 24.0
22.90053999 20.4083709716797 25.0
221.53560808 223.594223022461 26.0
221.53560808 221.987747192383 27.0
27.282645659 25.3551406860352 28.0
248.81825375 253.574676513672 29.0
248.81825375 252.408157348633 30.0
47.033013387 47.2299919128418 31.0
9.3131031344 9.18030452728271 32.0
20.901625993 20.4819431304932 33.0
20.901625993 21.1129970550537 34.0
20.901625993 20.8195400238037 35.0
9.3083381265 8.28715133666992 36.0
8.044767894 8.38281917572021 37.0
-2.5262790395e-09 -0.804296553134918 38.0
-2.5262785966e-09 0.113371789455414 39.0
-12.387792857 -14.5510091781616 40.0
-23.81592281 -27.5234241485596 41.0
4.9278776077 3.5162615776062 42.0
-31.108913619 -34.7394485473633 43.0
92.052766479 91.3064651489258 44.0
92.05276649 90.2343292236328 45.0
6.7273095346 3.00315618515015 46.0
2.6568330431 1.83220052719116 47.0
137.17687743 136.682632446289 48.0
131.07448231 127.516792297363 49.0
0 3.54757404327393 50.0
69.705106686 70.9864120483398 51.0
78.669161734 84.2529830932617 52.0
77.500548402 81.0479965209961 53.0
77.500548402 80.9212951660156 54.0
77.500548402 79.4541549682617 55.0
0 -0.0822644829750061 56.0
80.656571391 80.3950500488281 57.0
221.53560809 218.26057434082 58.0
248.81825375 246.469757080078 59.0
20.901626002 19.9398994445801 60.0
0 2.89454770088196 61.0
152.524167139285 161.566421508789 0.0
152.524167139285 166.802734375 1.0
151.35996019 159.585845947266 2.0
146.57089042 149.159912109375 3.0
0 2.72630953788757 4.0
-5.13071198e-09 3.58648037910461 5.0
-59.53115247936 -58.3727226257324 6.0
70.2507919487599 73.059684753418 7.0
47.7617113627599 52.3051528930664 8.0
22.4890805857599 22.5464820861816 9.0
89.8720038897599 86.200553894043 10.0
76.7725129637599 74.7876510620117 11.0
84.1030283177599 83.5107803344727 12.0
84.1030283177599 81.3520660400391 13.0
84.1030283177599 81.384880065918 14.0
4.8380164139e-09 -1.74889588356018 15.0
2.8396886315e-09 -1.23449850082397 16.0
1.106254265e-08 -2.39599061012268 17.0
56.9242230005054 59.1188087463379 18.0
-56.9242229541398 -58.7474822998047 19.0
-56.9242229371398 -60.8387451171875 20.0
-56.9242229121398 -59.8430595397949 21.0
78.9448309813946 78.8030319213867 22.0
146.934885677505 142.000686645508 23.0
114.237803181882 114.043357849121 24.0
23.5512008828817 20.2917079925537 25.0
225.879716658569 230.664596557617 26.0
225.879716658569 224.718948364258 27.0
25.2255969145691 25.6801815032959 28.0
251.105313568569 253.560928344727 29.0
251.105313568569 254.459930419922 30.0
46.9430737789106 46.5697174072266 31.0
9.34221855687645 8.89533805847168 32.0
22.0206080690799 20.3247928619385 33.0
22.0206080690799 17.7073993682861 34.0
22.0206081119172 21.5819072723389 35.0
9.92452194042601 9.5117883682251 36.0
8.63421670852602 7.30539178848267 37.0
-5.88814005000002e-09 -5.77714157104492 38.0
-5.88814005000002e-09 -5.02064228057861 39.0
-12.0646002710699 -12.758918762207 40.0
-24.828896422166 -27.8423080444336 41.0
5.00332440529989 5.63134098052979 42.0
-32.4500067182999 -34.2524604797363 43.0
92.7446581152999 89.7777099609375 44.0
92.7446577858464 95.4271469116211 45.0
6.91667725158936 5.73279666900635 46.0
2.42208655028936 2.3217933177948 47.0
152.52416714 162.400619506836 48.0
146.57089043 142.35383605957 49.0
0 1.32746386528015 50.0
70.250791949 77.4059371948242 51.0
89.872003889 88.4692153930664 52.0
84.103028317 81.2791595458984 53.0
84.103028317 81.4317092895508 54.0
84.103028317 80.59765625 55.0
1.106254265e-08 0.0498611330986023 56.0
78.944830981 79.5637359619141 57.0
225.87971666 222.978210449219 58.0
251.10531357 249.396270751953 59.0
22.020608111 24.7376899719238 60.0
0 1.85762310028076 61.0
188.35861472 161.527542114258 0.0
188.35861472 166.743255615234 1.0
187.12495818 177.637115478516 2.0
182.33588845 173.685607910156 3.0
0 5.14164781570435 4.0
-7.012611516e-12 1.29842388629913 5.0
-53.244760868 -61.9058570861816 6.0
65.595817622 73.8034210205078 7.0
49.142738849 45.5146675109863 8.0
16.453078763 17.7125606536865 9.0
98.137238072 90.3498229980469 10.0
85.200221678 75.5721282958984 11.0
85.839947068 92.6935043334961 12.0
85.839947068 80.8045959472656 13.0
85.839947068 81.6752243041992 14.0
0 -1.43005418777466 15.0
0 -0.885065853595734 16.0
-8.7226531548e-12 -0.0886810421943665 17.0
61.636511309 55.809211730957 18.0
-61.636511329 -52.5389595031738 19.0
-61.636511307 -61.0566711425781 20.0
-61.63651132 -56.9647941589355 21.0
83.481460517 77.7001571655273 22.0
139.38782775 141.661605834961 23.0
105.13815424 110.336486816406 24.0
21.146376421 21.9361515045166 25.0
222.86928827 226.956588745117 26.0
222.86928827 216.012084960938 27.0
28.029464278 29.3300018310547 28.0
250.89875256 259.154327392578 29.0
250.89875256 247.227645874023 30.0
46.672996384 47.0265655517578 31.0
9.254354393 6.82443141937256 32.0
21.844949187 22.9647731781006 33.0
21.844949187 15.4007530212402 34.0
21.844949187 20.6262683868408 35.0
10.362363952 6.2620792388916 36.0
9.0499341022 8.03130149841309 37.0
-2.10045253e-11 -1.76586318016052 38.0
-2.1004555058e-11 1.09970664978027 39.0
-12.290568638 -16.8061084747314 40.0
-25.067807734 -24.3410568237305 41.0
4.859556457 4.55338954925537 42.0
-32.638170905 -32.7825164794922 43.0
90.298294905 94.1913681030273 44.0
90.298294918 93.3824996948242 45.0
6.5153827865 3.95994138717651 46.0
2.0175575739 1.01958823204041 47.0
188.35861473 165.483795166016 48.0
182.33588846 171.608459472656 49.0
0 0.0955412983894348 50.0
65.595817631 77.1698913574219 51.0
98.137238082 91.5490570068359 52.0
85.839947078 79.6971893310547 53.0
85.839947078 81.5385131835938 54.0
85.839947078 79.9933242797852 55.0
0 -0.120720326900482 56.0
83.481460527 78.8674545288086 57.0
222.86928828 220.218078613281 58.0
250.89875257 243.010635375977 59.0
21.844949197 22.1279582977295 60.0
0 2.65086913108826 61.0
188.11289016 159.915313720703 0.0
188.11289016 170.102005004883 1.0
187.06914844 177.421401977539 2.0
182.28007872 168.033065795898 3.0
0 4.44767189025879 4.0
-4.722767456e-14 2.86840200424194 5.0
-71.310829306 -55.7166938781738 6.0
82.963367141 77.9166259765625 7.0
51.043695763 51.1218185424805 8.0
31.919671368 21.8328685760498 9.0
83.063168831 90.4022979736328 10.0
71.358745608 73.4979248046875 11.0
80.560929915 81.213249206543 12.0
80.560929915 77.803352355957 13.0
80.560929915 83.0279541015625 14.0
0 2.83509683609009 15.0
0 -0.247417747974396 16.0
0 2.37327075004578 17.0
52.855188393 53.970401763916 18.0
-52.855188403 -52.4599380493164 19.0
-52.855188382 -60.683422088623 20.0
-52.855188394 -56.3549842834473 21.0
73.581352593 77.2407073974609 22.0
146.72933516 143.000671386719 23.0
116.2671277 111.094497680664 24.0
24.520672941 18.9487476348877 25.0
220.31068776 232.348083496094 26.0
220.31068776 226.983825683594 27.0
27.61786958 26.3444766998291 28.0
247.92855735 252.710021972656 29.0
247.92855735 254.260498046875 30.0
46.478335659 44.2851600646973 31.0
9.3221717998 9.015305519104 32.0
20.72616419 18.7731781005859 33.0
20.72616419 17.7240924835205 34.0
20.72616419 22.5669727325439 35.0
9.4555057032 7.25285673141479 36.0
8.2087649269 7.35748624801636 37.0
-1.4102704278e-13 -9.76002025604248 38.0
-1.4102704273e-13 5.53264093399048 39.0
-12.075001761 -13.839243888855 40.0
-25.580642543 -16.6018142700195 41.0
5.3808760486 10.0048122406006 42.0
-33.635633279 -36.1592674255371 43.0
87.058553694 87.5045471191406 44.0
87.058553707 91.0124816894531 45.0
7.5058741731 4.76283359527588 46.0
2.7846570648 1.48382472991943 47.0
188.11289017 164.454818725586 48.0
182.28007873 167.030212402344 49.0
0 -2.6207230091095 50.0
82.963367151 76.9311065673828 51.0
83.063168841 86.3522567749023 52.0
80.560929925 79.2899856567383 53.0
80.560929925 80.6327590942383 54.0
80.560929925 79.7011184692383 55.0
0 -0.765894949436188 56.0
73.581352603 77.6605529785156 57.0
220.31068777 221.160140991211 58.0
247.92855736 248.710693359375 59.0
20.7261642 24.8430576324463 60.0
0 6.97875690460205 61.0
178.888201318669 168.374816894531 0.0
178.888201318669 172.24870300293 1.0
177.64728756 169.221603393555 2.0
172.85821783 182.543380737305 3.0
0 0.800851047039032 4.0
-6.437998937e-09 0.218645751476288 5.0
-63.78225297 -59.7823143005371 6.0
74.6886032493594 81.4997406005859 7.0
46.5043263003595 49.9050712585449 8.0
28.1842769483595 25.407735824585 9.0
89.2862577113594 87.7333297729492 10.0
76.6038195393594 74.8431854248047 11.0
83.7078348043594 84.7164535522461 12.0
83.7078348043594 78.5349426269531 13.0
83.7078348043594 76.927116394043 14.0
0 -0.505191147327423 15.0
0 0.382724165916443 16.0
-8.3996938744e-09 -2.89570498466492 17.0
55.490699341287 58.7428016662598 18.0
-55.4906993565392 -59.6449279785156 19.0
-55.4906993355392 -59.5124130249023 20.0
-55.4906993445392 -59.8692932128906 21.0
77.056278242 76.7875671386719 22.0
143.397078489287 140.011520385742 23.0
110.918074366609 114.016265869141 24.0
22.9836760286087 20.8778285980225 25.0
220.453356737338 229.823822021484 26.0
220.453356737338 225.399703979492 27.0
26.1386657943378 25.5570697784424 28.0
246.592022537338 254.735412597656 29.0
246.592022537338 255.256484985352 30.0
46.402069793 44.3921279907227 31.0
9.2741165734 9.23030376434326 32.0
21.5655789021652 19.4150314331055 33.0
21.5655789021652 23.0353355407715 34.0
21.565578902 20.1715812683105 35.0
9.8271505547 8.97214698791504 36.0
8.6084834152 7.07053279876709 37.0
-9.678018421e-09 1.03383266925812 38.0
-9.678018421e-09 -2.7609691619873 39.0
-11.8654536422696 -13.5939950942993 40.0
-24.055723913 -25.0566234588623 41.0
5.14487075603287 5.05587673187256 42.0
-31.7612125306329 -33.0886344909668 43.0
103.576685793633 93.8842315673828 44.0
103.576685793633 93.6891784667969 45.0
6.4522704253 6.28902435302734 46.0
2.4693809398 2.55379343032837 47.0
178.88820132 166.748168945312 48.0
172.85821784 167.167190551758 49.0
0 3.71830129623413 50.0
74.688603249 73.9269790649414 51.0
89.286257711 87.9648132324219 52.0
83.707834805 81.4174194335938 53.0
83.707834805 81.5539627075195 54.0
83.707834805 80.0681838989258 55.0
0 1.85776305198669 56.0
77.056278247 80.1147613525391 57.0
220.45335674 221.236770629883 58.0
246.59202254 247.994049072266 59.0
21.565578905 19.2825412750244 60.0
0 1.08062720298767 61.0
153.45630845 157.846054077148 0.0
153.45630845 162.785614013672 1.0
152.27388972 169.747131347656 2.0
147.48481999 157.551879882812 3.0
0 0.228694140911102 4.0
-1.0348165221e-09 4.09048557281494 5.0
-67.527517278 -61.0231170654297 6.0
78.864930217 78.2646942138672 7.0
63.598385629 53.0704765319824 8.0
15.266544583 29.6661071777344 9.0
86.988488038 86.5590286254883 10.0
75.413175072 77.8539886474609 11.0
77.949070768 85.9935531616211 12.0
77.949070768 79.7432861328125 13.0
77.949070768 72.9773406982422 14.0
0 -8.07441902160645 15.0
0 -0.585292160511017 16.0
0 0.52325302362442 17.0
56.957479582 59.480712890625 18.0
-56.957479597 -59.4113616943359 19.0
-56.957479575 -58.966064453125 20.0
-56.957479587 -58.2117614746094 21.0
80.099169183 78.8520660400391 22.0
152.32242761 137.579177856445 23.0
115.72624204 112.943344116211 24.0
22.85984251 19.6951160430908 25.0
232.4215968 221.7119140625 26.0
232.4215968 223.896621704102 27.0
23.581579218 29.0908660888672 28.0
256.00317603 247.303695678711 29.0
256.00317603 249.910003662109 30.0
46.710474883 46.961597442627 31.0
9.3890758442 8.73222732543945 32.0
23.141689594 18.4091815948486 33.0
23.141689594 16.3139667510986 34.0
23.141689593 19.9925861358643 35.0
10.684884215 9.23737049102783 36.0
9.4227954612 8.00438213348389 37.0
-3.4712976217e-09 -5.89797782897949 38.0
-3.4712976217e-09 1.25412392616272 39.0
-11.889861599 -12.8110990524292 40.0
-25.252705442 -22.574462890625 41.0
5.2702049723 5.13136577606201 42.0
-32.885090471 -32.994571685791 43.0
97.705632693 94.4482803344727 44.0
97.705632704 91.5599136352539 45.0
6.7160299085 5.30523872375488 46.0
1.8921580034 2.86065030097961 47.0
153.45630846 165.56640625 48.0
147.48482 149.425201416016 49.0
0 0.0676442384719849 50.0
78.864930224 77.2593002319336 51.0
86.988488045 93.1648788452148 52.0
77.949070774 82.3914642333984 53.0
77.949070774 81.4169769287109 54.0
77.949070774 80.3136291503906 55.0
0 3.70716738700867 56.0
80.099169192 79.9301605224609 57.0
232.42159681 216.034515380859 58.0
256.00317603 247.22607421875 59.0
23.141689601 21.4480133056641 60.0
0 2.68804907798767 61.0
189.24281877 153.304092407227 0.0
189.24281877 163.583435058594 1.0
187.95025603 166.658432006836 2.0
183.16118631 192.000015258789 3.0
0 -1.33618831634521 4.0
-3.5130057022e-12 4.74383306503296 5.0
-63.809428785 -46.5229682922363 6.0
73.171654401 91.8746719360352 7.0
52.508177435 58.0366401672363 8.0
20.663476956 31.7980499267578 9.0
89.684471167 94.2119064331055 10.0
76.143017535 79.6999740600586 11.0
78.654227189 80.5734558105469 12.0
78.654227189 79.3919372558594 13.0
78.654227189 81.4614791870117 14.0
0 -2.35738492012024 15.0
0 -0.034458339214325 16.0
-3.4104055046e-12 -1.54538536071777 17.0
58.659545579 56.8751602172852 18.0
-58.659545599 -59.3209037780762 19.0
-58.659545577 -59.4521331787109 20.0
-58.659545589 -62.9230079650879 21.0
79.153149103 76.8940048217773 22.0
143.34294736 137.906433105469 23.0
109.84666404 113.695251464844 24.0
21.087354511 21.287467956543 25.0
222.49609648 222.445602416992 26.0
222.49609648 219.009521484375 27.0
25.297755294 29.5036334991455 28.0
247.79385178 254.924407958984 29.0
247.79385178 253.633453369141 30.0
46.919162042 44.7332916259766 31.0
9.3048826358 9.12325382232666 32.0
20.493603503 14.793999671936 33.0
20.493603503 13.0898857116699 34.0
20.493603503 20.1962985992432 35.0
9.2407929802 7.62399101257324 36.0
7.919669314 5.75054121017456 37.0
-9.9716074119e-12 -1.36597967147827 38.0
-9.9716074119e-12 8.58164310455322 39.0
-10.498044225 -19.6664161682129 40.0
-22.756317274 -28.3956623077393 41.0
4.9079236427 2.5391948223114 42.0
-30.174633715 -31.6249828338623 43.0
82.203103872 85.6971817016602 44.0
82.203103885 87.4307632446289 45.0
6.3674716656 5.39477443695068 46.0
2.47033766 1.18622839450836 47.0
189.24281878 160.348983764648 48.0
183.16118632 177.780044555664 49.0
0 1.9334716796875 50.0
73.171654411 77.8594818115234 51.0
89.684471177 90.0266189575195 52.0
78.654227199 78.2778930664062 53.0
78.654227199 80.8933792114258 54.0
78.654227199 80.2361450195312 55.0
0 0.811580240726471 56.0
79.153149113 81.3647766113281 57.0
222.49609648 211.3525390625 58.0
247.79385179 249.86540222168 59.0
20.493603513 23.6781959533691 60.0
0 2.83803534507751 61.0
142.212199234643 158.242401123047 0.0
142.212199234643 162.660583496094 1.0
140.95816682 164.748504638672 2.0
135.49353159 149.753005981445 3.0
0.3983086571 3.72175645828247 4.0
0.398308666200141 10.4839057922363 5.0
-67.0605747916801 -54.2794990539551 6.0
77.5104181053804 73.3832855224609 7.0
50.3997691703804 52.9961013793945 8.0
27.1106489343804 4.04772090911865 9.0
95.6405128233804 94.5696258544922 10.0
82.9776675753804 75.8003082275391 11.0
80.7589626083804 85.5527801513672 12.0
80.7589626083804 79.0862503051758 13.0
80.7589626083804 71.4978790283203 14.0
1.9406676556e-09 2.05648946762085 15.0
0 1.2619469165802 16.0
-6.86606412999999e-09 3.07014536857605 17.0
57.0688802847529 56.9949607849121 18.0
-57.0688802790706 -59.2165870666504 19.0
-57.0688802590706 -57.974365234375 20.0
-57.0688802530706 -59.2229347229004 21.0
79.4923384806974 75.0018539428711 22.0
147.444902208753 143.802124023438 23.0
114.445707400941 100.480964660645 24.0
22.5470211779412 24.4297466278076 25.0
226.937240689285 222.736999511719 26.0
226.937240689285 220.457794189453 27.0
26.4700803152851 27.5725479125977 28.0
253.407320999285 247.391662597656 29.0
253.407320999285 248.090621948242 30.0
47.2884398759555 44.9387474060059 31.0
9.35680360318831 8.19455909729004 32.0
22.4234582280403 18.1145076751709 33.0
22.4234582280403 13.6144094467163 34.0
22.4234582499587 20.69655418396 35.0
9.69072750791308 5.32759475708008 36.0
8.43755135561308 6.51123142242432 37.0
-8.63881030000001e-09 -8.93080043792725 38.0
-8.63881030000001e-09 -5.27449560165405 39.0
-11.8614972370353 -9.32962226867676 40.0
-24.5266586395832 -27.0754890441895 41.0
5.02697101575034 3.89485025405884 42.0
-32.0307519371503 -31.5535106658936 43.0
104.71402758115 100.212158203125 44.0
104.71402750115 94.9428329467773 45.0
6.4596344641 3.31813955307007 46.0
2.2386912463 2.42008018493652 47.0
142.21219923 161.704040527344 48.0
135.4935316 147.787734985352 49.0
0.3983086662 5.97553730010986 50.0
77.510418105 75.5816040039062 51.0
95.640512823 87.5701446533203 52.0
80.758962608 80.6090393066406 53.0
80.758962608 81.7780685424805 54.0
80.758962608 79.2183303833008 55.0
0 -0.774916470050812 56.0
79.492338481 79.2989044189453 57.0
226.93724069 219.722610473633 58.0
253.407321 244.706100463867 59.0
22.42345825 22.360445022583 60.0
0 4.82096004486084 61.0
182.030902908478 158.702072143555 0.0
182.030902908478 159.748474121094 1.0
180.7369681 175.198959350586 2.0
175.94789838 165.452087402344 3.0
0 -0.822106659412384 4.0
-8.5055487225e-09 12.4162473678589 5.0
-56.5248579853413 -49.1940307617188 6.0
66.4603942246947 81.7761306762695 7.0
62.6511539796947 55.082576751709 8.0
3.80924024539473 17.8995456695557 9.0
87.2863398856947 83.1842346191406 10.0
73.9589352206947 74.006477355957 11.0
81.2988143496947 80.8458480834961 12.0
81.2988143496947 79.7641296386719 13.0
81.2988143496947 83.5395431518555 14.0
0 -1.33115434646606 15.0
0 -0.582838475704193 16.0
-6.299969124e-09 1.12940430641174 17.0
56.9725534776517 58.0894393920898 18.0
-56.9725534870421 -59.2990303039551 19.0
-56.9725534660421 -61.3771324157715 20.0
-56.9725534730421 -62.3262329101562 21.0
79.3500212569175 79.4605407714844 22.0
150.245282629652 143.983062744141 23.0
116.179884984565 109.887596130371 24.0
21.4526102155646 21.1231079101562 25.0
229.595303886955 225.225112915039 26.0
229.595303886955 227.641876220703 27.0
24.1637688839555 25.2515773773193 28.0
253.759072766955 255.398513793945 29.0
253.759072766955 253.366439819336 30.0
46.3401295151344 45.927661895752 31.0
9.3990912314 8.79635620117188 32.0
22.3774677840241 18.7030410766602 33.0
22.3774677840241 8.86799144744873 34.0
22.3774677901361 19.1004734039307 35.0
9.7811686651 6.1878457069397 36.0
8.6383513445 8.77671337127686 37.0
-9.402654923e-09 -0.264768302440643 38.0
-9.402654923e-09 -6.85224056243896 39.0
-11.731173514021 -13.8486528396606 40.0
-24.8050307820312 -31.550085067749 41.0
5.32330462191228 -0.624313652515411 42.0
-32.3843285729123 -33.6930618286133 43.0
92.1879425709123 91.7188262939453 44.0
92.1879425579123 93.5702133178711 45.0
5.8801326403 4.93358898162842 46.0
2.1651973211 1.77921867370605 47.0
182.03090291 163.521026611328 48.0
175.94789839 156.667846679688 49.0
0 3.35714960098267 50.0
66.460394225 74.5578460693359 51.0
87.286339886 83.9722137451172 52.0
81.29881435 80.0632019042969 53.0
81.29881435 73.9345932006836 54.0
81.29881435 80.2585525512695 55.0
0 2.67186665534973 56.0
79.350021257 80.1596450805664 57.0
229.59530389 221.394561767578 58.0
253.75907277 249.784591674805 59.0
22.37746779 17.4587841033936 60.0
0 1.41972827911377 61.0
154.087209775025 164.572647094727 0.0
154.087209775025 165.257537841797 1.0
152.77251473 158.085220336914 2.0
147.98344498 134.186630249023 3.0
0 3.04957509040833 4.0
-6.976018033e-09 -1.35981464385986 5.0
-62.7109286838695 -66.0733642578125 6.0
73.3583489290431 71.3933029174805 7.0
61.1147634020431 50.691722869873 8.0
12.2435855270431 19.2828578948975 9.0
93.4150109450431 82.1822509765625 10.0
80.7849847610431 71.3012847900391 11.0
84.7907752690431 89.7280578613281 12.0
84.7907752690431 79.441520690918 13.0
84.7907752690431 78.5909729003906 14.0
1.6850988523e-09 -3.77514743804932 15.0
0 -2.45132374763489 16.0
5.22104629999988e-10 -2.63921022415161 17.0
58.1437922491144 59.7875862121582 18.0
-58.1437922360646 -55.6384010314941 19.0
-58.1437922160646 -62.1023330688477 20.0
-58.1437922110646 -65.8423614501953 21.0
80.9426927278765 75.0631866455078 22.0
133.032358129114 137.24934387207 23.0
101.627668681393 108.015922546387 24.0
23.932245220393 19.642032623291 25.0
213.97505086005 221.849060058594 26.0
213.97505086005 223.381546020508 27.0
24.3172031670497 25.3994960784912 28.0
238.29225403005 259.591827392578 29.0
238.29225403005 259.405792236328 30.0
46.3267666781485 48.1941757202148 31.0
9.39286044421825 10.5296545028687 32.0
22.7989005176084 21.7379035949707 33.0
22.7989005176084 21.7576217651367 34.0
22.7989005391873 21.0930480957031 35.0
10.3350462748461 8.86963176727295 36.0
9.00247180643072 6.21103858947754 37.0
-7.94938021e-09 1.92535495758057 38.0
-7.94938021e-09 -2.35086536407471 39.0
-12.2632537075323 -10.7590579986572 40.0
-23.9318346619202 -25.7616367340088 41.0
5.19403945340256 11.4188480377197 42.0
-31.9086647087026 -37.7276229858398 43.0
90.2248390667026 87.2116394042969 44.0
90.2248389887026 94.1394653320312 45.0
7.1480689636 7.03012847900391 46.0
2.5064031783 2.77427697181702 47.0
154.08720978 170.710357666016 48.0
147.98344499 143.025283813477 49.0
0 1.75157034397125 50.0
73.358348929 73.2819061279297 51.0
93.415010945 83.2170181274414 52.0
84.790775269 79.1651229858398 53.0
84.790775269 81.3897705078125 54.0
84.790775269 79.9757232666016 55.0
5.2210463501e-10 1.01836371421814 56.0
80.942692728 81.1798324584961 57.0
213.97505086 213.279205322266 58.0
238.29225403 249.094100952148 59.0
22.798900539 23.0661716461182 60.0
0 3.38156795501709 61.0
122.492444500762 168.830184936523 0.0
122.492444500762 163.895141601562 1.0
121.01375239 166.507125854492 2.0
116.22468266 146.639068603516 3.0
0 3.61375141143799 4.0
-8.7204412675e-09 -3.22320508956909 5.0
-48.662264027 -50.0952835083008 6.0
57.8690252826536 81.5259857177734 7.0
41.9501174776536 48.5476684570312 8.0
15.9189078056536 28.3375110626221 9.0
78.8416983976536 91.3793869018555 10.0
66.8546783646536 67.6533966064453 11.0
69.8693741986536 85.3102264404297 12.0
69.8693741986536 90.1217727661133 13.0
69.8693741986536 77.5349884033203 14.0
0 -0.37270313501358 15.0
0 0.0945518016815186 16.0
-6.6574927e-09 7.43443536758423 17.0
57.840129998811 60.7123069763184 18.0
-57.8401300099803 -55.4037399291992 19.0
-57.8401299879803 -58.2770347595215 20.0
-57.8401299959803 -65.0445556640625 21.0
79.626649826 80.4363250732422 22.0
143.073012921811 135.835235595703 23.0
110.831199767264 119.732498168945 24.0
22.0973223922638 19.5632400512695 25.0
222.699662751523 220.432907104492 26.0
222.699662751523 213.651062011719 27.0
26.3263536215233 23.4881248474121 28.0
249.026016371523 245.819686889648 29.0
249.026016371523 235.467971801758 30.0
47.073735752 48.1618003845215 31.0
9.3328621107 8.06484222412109 32.0
21.7865198314173 15.7455434799194 33.0
21.7865198314173 9.83489799499512 34.0
21.7865198365012 18.4934825897217 35.0
9.0203447251 6.72415113449097 36.0
7.700245251 3.6424036026001 37.0
-9.423547705e-09 -32.4079437255859 38.0
-9.423547705e-09 -0.160629570484161 39.0
-11.3561600629902 -18.3249378204346 40.0
-23.4459576020442 -31.2633533477783 41.0
5.8945740132113 1.55947828292847 42.0
-31.9086475822113 -32.2541618347168 43.0
73.6047289992113 70.7840042114258 44.0
73.6047289584886 55.2600135803223 45.0
6.5415248309 13.0380973815918 46.0
1.6279530272 3.18781232833862 47.0
122.4924445 171.360061645508 48.0
116.22468267 150.33366394043 49.0
0 -8.71527481079102 50.0
57.869025283 74.366828918457 51.0
78.841698398 79.8299560546875 52.0
69.869374198 78.5195999145508 53.0
69.869374198 81.5034942626953 54.0
69.869374198 82.9144744873047 55.0
0 3.37303924560547 56.0
79.626649827 73.5043792724609 57.0
222.69966275 215.117584228516 58.0
249.02601637 228.20100402832 59.0
21.786519837 11.8999328613281 60.0
0 2.35391712188721 61.0
162.38469945 160.608215332031 0.0
162.38469945 159.386184692383 1.0
161.07140382 157.380813598633 2.0
156.28233409 189.3349609375 3.0
0 14.5906200408936 4.0
-2.3519564874e-10 3.5825891494751 5.0
-51.951625121 -60.7122116088867 6.0
64.2212339482612 79.3839645385742 7.0
42.8630173082612 50.2891998291016 8.0
21.3582166402612 24.4324169158936 9.0
93.2544519242612 93.301887512207 10.0
79.4160535262612 77.4576950073242 11.0
83.8442717592612 81.9186019897461 12.0
83.8442717592612 79.6177444458008 13.0
83.8442717592612 80.3756790161133 14.0
0 4.48700714111328 15.0
0 0.225697994232178 16.0
-1.3961640607e-09 -3.052405834198 17.0
62.607322303 60.0581207275391 18.0
-62.6073223218918 -58.7881240844727 19.0
-62.6073223008918 -57.3753204345703 20.0
-62.6073223118918 -62.1456336975098 21.0
85.079235204 76.3406219482422 22.0
136.26480927 139.821411132812 23.0
105.20446137 111.490913391113 24.0
21.531351826 22.2858963012695 25.0
221.344044487609 227.267440795898 26.0
221.344044487609 221.270355224609 27.0
24.4819073396091 26.5734195709229 28.0
245.825951827609 255.254104614258 29.0
245.825951827609 256.236114501953 30.0
46.895413385 47.9875183105469 31.0
9.2883438486 9.48495674133301 32.0
22.471912899 34.0405082702637 33.0
22.471912899 14.0964555740356 34.0
22.471912895 18.123254776001 35.0
10.162413254 10.6498689651489 36.0
9.0132287061 7.99884557723999 37.0
-9.6130710561e-09 7.30465269088745 38.0
-9.6130710564e-09 0.173000156879425 39.0
-11.766748876 -23.4504070281982 40.0
-24.609117299 -27.5924739837646 41.0
5.1215971911 1.75763726234436 42.0
-32.22820584 -33.2091636657715 43.0
91.106149504 90.9697570800781 44.0
91.10614951 96.122314453125 45.0
7.0584821816 6.5449595451355 46.0
2.2915314816 2.45220828056335 47.0
162.38469946 168.936569213867 48.0
156.2823341 157.071701049805 49.0
0 -1.9548978805542 50.0
64.221233949 75.6350860595703 51.0
93.254451924 91.9958114624023 52.0
83.84427176 82.1744537353516 53.0
83.84427176 81.1248931884766 54.0
83.84427176 81.4801940917969 55.0
0 4.37057161331177 56.0
85.079235211 80.3185195922852 57.0
221.34404449 220.285537719727 58.0
245.82595183 246.211288452148 59.0
22.471912901 18.4660816192627 60.0
0 2.34848952293396 61.0
193.620195423445 166.634307861328 0.0
193.620195423445 165.151702880859 1.0
192.51512681 163.651779174805 2.0
187.72605706 175.645416259766 3.0
-2.3135613719e-10 1.25951719284058 4.0
-8.647187154e-09 1.55087041854858 5.0
-71.41837288423 -75.5401611328125 6.0
84.4565912553051 71.6154403686523 7.0
54.8692068283051 30.6962127685547 8.0
29.5873844273051 20.8891258239746 9.0
86.4911072633051 85.8573989868164 10.0
73.6765223503051 81.8769683837891 11.0
75.0946964873051 77.241943359375 12.0
75.0946964873051 76.8628387451172 13.0
75.0946964873051 79.9015731811523 14.0
2.6469896651e-09 -4.75218296051025 15.0
0 0.401790797710419 16.0
1.87899875000001e-09 7.24826908111572 17.0
57.2199505702574 54.1835556030273 18.0
-57.2199505459577 -56.4330558776855 19.0
-57.2199505279577 -58.5358848571777 20.0
-57.2199505139577 -58.7397308349609 21.0
77.7338775470825 80.9988098144531 22.0
153.723823269257 144.229385375977 23.0
119.612085404072 111.542259216309 24.0
22.1670579150717 23.6703205108643 25.0
231.457700806891 225.893768310547 26.0
231.457700806891 225.304977416992 27.0
27.6857023538906 20.6725406646729 28.0
259.143403166891 246.940063476562 29.0
259.143403166891 251.32763671875 30.0
46.1791815442054 48.2974662780762 31.0
9.33613497328532 9.06848526000977 32.0
20.5139270336901 25.5485877990723 33.0
20.5139270336901 25.0695781707764 34.0
20.5139270633121 23.332218170166 35.0
9.35858819441393 10.3592844009399 36.0
8.04134756511393 7.78337001800537 37.0
-7.13341441000003e-09 0.731598198413849 38.0
-7.13341440999998e-09 -5.06418037414551 39.0
-11.5848948084789 -13.2671432495117 40.0
-24.4229379770874 -25.1310577392578 41.0
5.62421696768762 6.40897083282471 42.0
-32.6573894150876 -34.5704574584961 43.0
91.6286594930876 91.076416015625 44.0
91.6286593770876 93.1831359863281 45.0
6.5093336585 11.8331708908081 46.0
2.2247797827 3.34036993980408 47.0
193.62019542 174.051391601562 48.0
187.72605707 165.526718139648 49.0
0 -5.14626026153564 50.0
84.456591256 71.5327377319336 51.0
86.491107264 81.328742980957 52.0
75.094696488 79.8986358642578 53.0
75.094696488 79.7839431762695 54.0
75.094696488 80.1530303955078 55.0
1.8789987524e-09 5.04774045944214 56.0
77.733877547 77.833869934082 57.0
231.45770081 221.005035400391 58.0
259.14340317 247.586334228516 59.0
20.513927063 16.7871913909912 60.0
0 0.494023025035858 61.0
147.982033057583 157.456390380859 0.0
147.982033057583 151.963851928711 1.0
146.808413813287 153.350814819336 2.0
142.01934397 150.732803344727 3.0
-1.55484310406e-09 -0.650561392307281 4.0
-1.55484309999997e-09 -2.84032654762268 5.0
-64.9693040339936 -64.321533203125 6.0
77.6547829144777 74.1871871948242 7.0
61.2786926634778 55.4030647277832 8.0
16.3760902514777 7.20984888076782 9.0
97.5261141374777 89.7391510009766 10.0
83.6472874924777 79.6450042724609 11.0
85.2981411144777 83.8642730712891 12.0
85.2981411144777 76.4624557495117 13.0
85.2981411144777 81.5436782836914 14.0
2.3328021736e-08 -2.03496789932251 15.0
2.3328021735e-08 -1.6052393913269 16.0
5.464350236e-08 -2.72059202194214 17.0
56.2518018559467 59.0783615112305 18.0
-56.2518018797166 -57.0891418457031 19.0
-56.2518018707166 -59.2854194641113 20.0
-56.2518017707166 -58.115608215332 21.0
77.1875821073724 73.3798065185547 22.0
153.454771802806 145.008895874023 23.0
115.220596133508 115.750228881836 24.0
21.7005584165075 20.8819332122803 25.0
230.642353915167 228.381210327148 26.0
230.642353915167 229.708221435547 27.0
26.8883420391666 21.3647422790527 28.0
257.530695955167 253.23681640625 29.0
257.530695955167 255.336776733398 30.0
46.7385281475469 45.537841796875 31.0
9.28724497493877 8.4355525970459 32.0
20.9357803369809 22.6376686096191 33.0
20.9357803369809 19.0790691375732 34.0
20.935780467613 20.4264030456543 35.0
8.86054244871795 8.39873600006104 36.0
7.60121916461794 6.85336446762085 37.0
2.62465480000001e-09 1.1959273815155 38.0
2.62465480000001e-09 -1.17553234100342 39.0
-11.6034306368583 -12.5643882751465 40.0
-24.5599220844463 -26.7411251068115 41.0
4.75986482786468 1.92820739746094 42.0
-31.7734218810648 -31.6524333953857 43.0
89.4696777832616 86.4044494628906 44.0
89.4696781210649 91.9600372314453 45.0
7.05809588635643 4.443039894104 46.0
2.46335994485643 2.26466846466064 47.0
147.98203306 156.096572875977 48.0
142.01934397 152.998229980469 49.0
0 -3.74713611602783 50.0
77.654782914 76.0236663818359 51.0
97.526114137 92.0755157470703 52.0
85.298141114 81.5651702880859 53.0
85.298141114 80.2131118774414 54.0
85.298141114 79.9203643798828 55.0
5.4643502359e-08 3.07315587997437 56.0
77.187582108 79.4883346557617 57.0
230.64235391 222.583419799805 58.0
257.53069595 249.491134643555 59.0
20.935780468 19.6116352081299 60.0
2.6246548239e-09 3.64096260070801 61.0
156.861091404888 159.86442565918 0.0
156.861091404888 164.524307250977 1.0
155.52926231 162.341293334961 2.0
150.74019254 151.753829956055 3.0
0 -8.4057445526123 4.0
-5.22601789e-09 5.28012847900391 5.0
-68.3498850872301 -62.6833190917969 6.0
79.7958029518052 75.4073104858398 7.0
55.8596434508053 52.1667633056641 8.0
23.9361595018052 20.6490592956543 9.0
88.9024303608052 84.6050186157227 10.0
76.2639219328052 74.9370651245117 11.0
81.0219682688052 82.3687744140625 12.0
81.0219682688052 79.6216354370117 13.0
81.0219682688052 79.358039855957 14.0
4.602401647e-09 0.437743544578552 15.0
4.60240164699e-09 0.799240529537201 16.0
1.080383174e-08 -0.0869020819664001 17.0
59.0643240078584 58.368766784668 18.0
-59.0643240557079 -59.7739334106445 19.0
-59.0643240377079 -60.3347778320312 20.0
-59.0643240107079 -59.6208000183105 21.0
80.5432193786501 78.8417358398438 22.0
143.967734453894 142.143432617188 23.0
107.978514844867 110.714378356934 24.0
22.5896058518672 20.7431335449219 25.0
224.510953829775 227.853851318359 26.0
224.510953829775 226.092163085938 27.0
25.5775924157753 20.7458992004395 28.0
250.088546249775 256.343200683594 29.0
250.088546249775 256.1025390625 30.0
46.7252001506423 46.4096641540527 31.0
9.32784005598338 9.00160598754883 32.0
21.4788953686902 21.6592845916748 33.0
21.4788953686902 22.7354488372803 34.0
21.4788954153122 21.0440635681152 35.0
9.4949079483962 7.08659553527832 36.0
8.2248182515962 7.61897659301758 37.0
-5.71842478999997e-09 -0.65439909696579 38.0
-5.71842479000002e-09 -3.97519254684448 39.0
-11.9173161503539 -14.1116676330566 40.0
-24.7425931554772 -27.1329803466797 41.0
4.97985589750437 1.50881910324097 42.0
-32.1971692865044 -32.2347640991211 43.0
100.134876521504 92.1187362670898 44.0
100.134876194672 94.4784088134766 45.0
7.02279715335161 5.6803994178772 46.0
2.47956480345161 2.43925404548645 47.0
156.86109141 161.436309814453 48.0
150.74019255 151.165802001953 49.0
0 -2.44583010673523 50.0
79.795802952 76.5777435302734 51.0
88.902430361 85.7161865234375 52.0
81.021968268 80.1285095214844 53.0
81.021968268 80.2013702392578 54.0
81.021968268 80.0349426269531 55.0
1.080383174e-08 0.902844250202179 56.0
80.543219379 79.0610198974609 57.0
224.51095383 221.159454345703 58.0
250.08854625 250.688598632812 59.0
21.478895416 21.6838111877441 60.0
0 0.0651013255119324 61.0
137.005895469483 169.175033569336 0.0
137.005895469483 155.581268310547 1.0
135.516270716247 161.902816772461 2.0
130.7272009 156.904541015625 3.0
-5.8296371693e-10 4.80070352554321 4.0
-2.59751367000001e-09 6.96303176879883 5.0
-67.3460199805536 -62.1502380371094 6.0
80.2726303504377 77.3187561035156 7.0
47.6696531994377 39.2517852783203 8.0
32.6029771504377 28.2792701721191 9.0
97.1258292864377 96.2259216308594 10.0
85.4824485874377 85.5156021118164 11.0
82.9360884054377 89.9001770019531 12.0
82.9360884054377 82.386100769043 13.0
82.9360884054377 84.353271484375 14.0
1.2922127485e-08 0.417059361934662 15.0
1.2922127485e-08 0.272698760032654 16.0
3.915904464e-08 0.927337288856506 17.0
63.0490300392388 59.5019035339355 18.0
-63.0490298956565 -60.3414001464844 19.0
-63.0490298836565 -58.489818572998 20.0
-63.0490298036565 -59.6975135803223 21.0
85.8557698587399 80.5869522094727 22.0
147.811586740239 146.577682495117 23.0
113.384614360298 111.430938720703 24.0
24.9993712882984 20.3570289611816 25.0
233.667356598967 222.051284790039 26.0
233.667356598967 223.795104980469 27.0
27.5232060249666 27.0389232635498 28.0
261.190562618967 248.242050170898 29.0
261.190562618967 248.609664916992 30.0
46.7921314788945 47.291919708252 31.0
9.31710317827406 8.17745494842529 32.0
22.8067400546609 21.9532279968262 33.0
22.8067400546609 20.2594451904297 34.0
22.8067401625648 18.9529666900635 35.0
10.3268204819179 7.24479913711548 36.0
9.06730788291793 9.10525798797607 37.0
4.02955999999953e-10 2.29529619216919 38.0
4.02955989999882e-10 -0.975394785404205 39.0
-11.4667637913282 -19.0702209472656 40.0
-25.5173221166293 -26.342643737793 41.0
4.72456525226468 8.42488670349121 42.0
-32.9526050203647 -33.9198150634766 43.0
94.2014653713647 95.6677398681641 44.0
94.2014649233648 94.476806640625 45.0
6.88126306931484 4.57992601394653 46.0
1.96925799252877 1.61277008056641 47.0
137.00589546 177.432083129883 48.0
130.7272009 150.917098999023 49.0
0 -0.546056807041168 50.0
80.27263035 72.9873199462891 51.0
97.125829286 89.6365280151367 52.0
82.936088406 79.0650024414062 53.0
82.936088406 80.5479049682617 54.0
82.936088406 78.9662704467773 55.0
3.9159044646e-08 1.75368356704712 56.0
85.855769859 80.5921859741211 57.0
233.6673566 222.274505615234 58.0
261.19056262 250.529968261719 59.0
22.806740163 15.0579719543457 60.0
4.0295599212e-10 2.0735502243042 61.0
194.88548137 163.457046508789 0.0
194.88548137 166.252792358398 1.0
193.76144832 174.015045166016 2.0
188.97237819 170.795471191406 3.0
0 -0.307850897312164 4.0
0 -3.59379720687866 5.0
-60.518097083 -51.1522483825684 6.0
73.194678309 79.0123825073242 7.0
43.798588725 51.6629028320312 8.0
29.396089574 24.0503444671631 9.0
88.737171249 94.4666595458984 10.0
75.043716784 73.0774612426758 11.0
83.329338092 84.8226013183594 12.0
83.329338092 80.4830932617188 13.0
83.329338092 81.79443359375 14.0
0 -2.30506205558777 15.0
0 0.150099694728851 16.0
0 3.75373411178589 17.0
56.389960647 57.6626243591309 18.0
-56.389960657 -58.700984954834 19.0
-56.389960635 -58.0731430053711 20.0
-56.389960648 -60.586311340332 21.0
77.722591005 80.1325836181641 22.0
148.65920514 142.191390991211 23.0
115.20971987 115.137847900391 24.0
24.674873731 21.498218536377 25.0
226.38179616 226.1103515625 26.0
226.38179616 225.181594848633 27.0
23.207177487 35.499195098877 28.0
249.58897366 258.49267578125 29.0
249.58897366 248.625640869141 30.0
46.230960053 43.0743942260742 31.0
9.374021486 8.20328712463379 32.0
21.332630347 23.0570945739746 33.0
21.332630347 18.6942710876465 34.0
21.332630347 21.7621898651123 35.0
9.3443311439 7.69729900360107 36.0
7.9221594824 8.49557399749756 37.0
-1.8330635987e-12 -1.99160814285278 38.0
-1.8330635967e-12 8.35755348205566 39.0
-12.560672361 -16.6681900024414 40.0
-24.51022151 -27.7770767211914 41.0
5.1357273828 4.08682012557983 42.0
-32.022359071 -32.5144309997559 43.0
85.679242326 91.7135772705078 44.0
85.679242339 91.6370620727539 45.0
6.5350517346 7.277099609375 46.0
2.8318151813 2.73348569869995 47.0
194.88548138 169.927810668945 48.0
188.9723782 177.539215087891 49.0
0 0.0973289012908936 50.0
73.194678319 76.7121047973633 51.0
88.737171259 90.3965759277344 52.0
83.329338102 78.1466751098633 53.0
83.329338102 81.4420623779297 54.0
83.329338102 80.5947875976562 55.0
0 2.95475482940674 56.0
77.722591015 80.1952590942383 57.0
226.38179617 222.889541625977 58.0
249.58897367 250.532302856445 59.0
21.332630357 19.9082889556885 60.0
0 3.18520259857178 61.0
169.27010139 168.257202148438 0.0
169.27010139 162.539306640625 1.0
168.24091778 153.276321411133 2.0
163.45184806 154.579788208008 3.0
0 0.673179090023041 4.0
-1.0109135791e-11 12.9842929840088 5.0
-60.580282342 -49.4930038452148 6.0
73.161915127 78.5560302734375 7.0
57.171618964 52.1862983703613 8.0
15.990296153 25.2701835632324 9.0
85.605417207 77.7727661132812 10.0
72.318309279 71.4398193359375 11.0
85.75948681 74.6589202880859 12.0
85.75948681 82.6873016357422 13.0
85.75948681 72.3928985595703 14.0
0 -1.85192918777466 15.0
0 0.376586854457855 16.0
-1.7312707342e-11 0.0777302980422974 17.0
53.276537165 57.3253479003906 18.0
-53.276537185 -58.6821784973145 19.0
-53.276537163 -53.4306030273438 20.0
-53.276537175 -60.1398162841797 21.0
74.237173942 75.5637741088867 22.0
153.48066716 145.539642333984 23.0
116.99074502 109.855537414551 24.0
23.130346851 19.9287967681885 25.0
227.71784111 218.220672607422 26.0
227.71784111 213.761825561523 27.0
29.607142156 27.8440036773682 28.0
257.32498328 245.226104736328 29.0
257.32498328 247.878204345703 30.0
46.903289153 47.5566558837891 31.0
9.2969014174 6.83636569976807 32.0
20.960636757 18.6979789733887 33.0
20.960636757 22.3353538513184 34.0
20.960636757 17.944465637207 35.0
8.8454532874 9.34835338592529 36.0
7.4765488743 9.22372150421143 37.0
-3.5287760078e-11 2.05488443374634 38.0
-3.5287763168e-11 3.96840333938599 39.0
-11.080992706 -12.8041372299194 40.0
-24.932509321 -23.1585235595703 41.0
5.0863548628 -2.91485667228699 42.0
-32.733906015 -31.3253612518311 43.0
93.46051945 95.7999420166016 44.0
93.460519463 93.5414352416992 45.0
6.8313614437 2.56629920005798 46.0
2.5901924429 1.38604354858398 47.0
169.2701014 164.618606567383 48.0
163.45184807 164.858612060547 49.0
0 2.22699403762817 50.0
73.161915137 77.8220825195312 51.0
85.605417217 87.2107849121094 52.0
85.759486819 80.4459457397461 53.0
85.759486819 81.0049362182617 54.0
85.759486819 80.0435562133789 55.0
0 -1.23136425018311 56.0
74.237173952 76.26904296875 57.0
227.71784112 220.241577148438 58.0
257.32498329 243.803512573242 59.0
20.960636767 16.3977470397949 60.0
0 2.85984873771667 61.0
137.17687742 143.006607055664 0.0
137.17687742 143.027954101562 1.0
135.86355256 143.557556152344 2.0
131.0744823 135.807662963867 3.0
0 -2.16135358810425 4.0
0 0.342353284358978 5.0
-59.42861149 -59.2404556274414 6.0
69.705106679 70.1773910522461 7.0
65.43029471 62.4528312683105 8.0
4.2748119615 8.48783111572266 9.0
78.669161727 80.2065200805664 10.0
65.229460649 71.2585144042969 11.0
77.500548395 80.8047180175781 12.0
77.500548395 77.9000701904297 13.0
77.500548395 76.9198608398438 14.0
0 -0.64198762178421 15.0
0 0.28728449344635 16.0
0 -3.15485000610352 17.0
59.754945382 58.5015258789062 18.0
-59.754945394 -59.4383087158203 19.0
-59.754945372 -60.0649642944336 20.0
-59.754945384 -61.8801422119141 21.0
80.656571382 77.8813858032227 22.0
140.87903669 140.394561767578 23.0
106.41719675 107.357681274414 24.0
22.90053999 20.4083709716797 25.0
221.53560808 223.594223022461 26.0
221.53560808 221.987747192383 27.0
27.282645659 25.3551406860352 28.0
248.81825375 253.574676513672 29.0
248.81825375 252.408157348633 30.0
47.033013387 47.2299919128418 31.0
9.3131031344 9.18030452728271 32.0
20.901625993 20.4819431304932 33.0
20.901625993 21.1129970550537 34.0
20.901625993 20.8195400238037 35.0
9.3083381265 8.28715133666992 36.0
8.044767894 8.38281917572021 37.0
-2.5262790395e-09 -0.804296553134918 38.0
-2.5262785966e-09 0.113371789455414 39.0
-12.387792857 -14.5510091781616 40.0
-23.81592281 -27.5234241485596 41.0
4.9278776077 3.5162615776062 42.0
-31.108913619 -34.7394485473633 43.0
92.052766479 91.3064651489258 44.0
92.05276649 90.2343292236328 45.0
6.7273095346 3.00315618515015 46.0
2.6568330431 1.83220052719116 47.0
137.17687743 136.680145263672 48.0
131.07448231 127.609191894531 49.0
0 3.5344455242157 50.0
69.705106686 70.9859237670898 51.0
78.669161734 84.2529830932617 52.0
77.500548402 81.0479965209961 53.0
77.500548402 80.9212951660156 54.0
77.500548402 79.4541549682617 55.0
0 -0.0822644829750061 56.0
80.656571391 80.3950500488281 57.0
221.53560809 218.26057434082 58.0
248.81825375 246.469757080078 59.0
20.901626002 19.9398994445801 60.0
0 2.89454770088196 61.0
190.26764164 173.481658935547 0.0
190.26764164 161.25390625 1.0
188.80017147 164.760925292969 2.0
184.01110174 178.398300170898 3.0
0 -5.99856901168823 4.0
-9.7203544719e-13 -4.00283670425415 5.0
-70.466765617 -66.1257476806641 6.0
79.734416694 79.601448059082 7.0
49.539546511 40.5537300109863 8.0
30.194870174 23.0752201080322 9.0
88.072765278 78.9701385498047 10.0
75.265717193 69.8022155761719 11.0
78.883916273 74.3051681518555 12.0
78.883916273 78.6778030395508 13.0
78.883916273 79.0331115722656 14.0
0 0.545309603214264 15.0
0 0.141220808029175 16.0
-1.8110751101e-12 -0.951424539089203 17.0
54.960348035 58.3135948181152 18.0
-54.960348055 -59.837215423584 19.0
-54.960348033 -62.5448837280273 20.0
-54.960348046 -56.575553894043 21.0
76.266156353 73.7048568725586 22.0
152.7878799 141.825241088867 23.0
114.67837866 113.446746826172 24.0
25.376636054 20.3599510192871 25.0
229.05403627 220.529403686523 26.0
229.05403627 220.607437133789 27.0
23.882189651 23.6607971191406 28.0
252.93622593 246.982482910156 29.0
252.93622593 252.573532104492 30.0
46.653085158 48.0742340087891 31.0
9.4599891472 8.84025859832764 32.0
21.305808297 22.7263202667236 33.0
21.305808297 24.729305267334 34.0
21.305808297 20.3125534057617 35.0
10.14562924 8.99026298522949 36.0
8.9236384307 5.17169380187988 37.0
-2.8998450736e-12 -4.98981332778931 38.0
-2.8998450736e-12 -3.18509578704834 39.0
-12.585091277 -15.442910194397 40.0
-25.184987025 -26.2191543579102 41.0
4.8273671346 5.03851222991943 42.0
-32.540584449 -31.6707210540771 43.0
93.106679105 89.4441986083984 44.0
93.106679118 94.4637832641602 45.0
6.3593390654 6.37011814117432 46.0
2.7141111603 2.28955936431885 47.0
190.26764165 163.732177734375 48.0
184.01110175 176.035552978516 49.0
0 -0.829241573810577 50.0
79.734416704 76.8819732666016 51.0
88.072765288 83.5243453979492 52.0
78.883916283 79.7604217529297 53.0
78.883916283 76.5640029907227 54.0
78.883916283 80.81005859375 55.0
0 0.633165657520294 56.0
76.266156363 78.5821151733398 57.0
229.05403628 218.466491699219 58.0
252.93622594 246.579849243164 59.0
21.305808307 14.0568923950195 60.0
0 5.29197406768799 61.0
177.69958826 168.569854736328 0.0
177.69958826 170.10774230957 1.0
176.33211327 161.594146728516 2.0
171.54304355 167.123657226562 3.0
0 -2.27063083648682 4.0
0 2.58953046798706 5.0
-66.787878238 -61.1720275878906 6.0
78.318521313 76.5486145019531 7.0
54.104359793 56.6209106445312 8.0
24.21416151 24.7831363677979 9.0
86.177068852 88.4899749755859 10.0
73.346185666 77.4846954345703 11.0
80.992439828 82.542106628418 12.0
80.992439828 78.454460144043 13.0
80.992439828 80.8387145996094 14.0
0 -0.155265808105469 15.0
0 -0.386723577976227 16.0
0 2.39572644233704 17.0
59.065608975 59.3639945983887 18.0
-59.065608994 -59.5033226013184 19.0
-59.065608972 -58.9887466430664 20.0
-59.065608985 -60.4299507141113 21.0
80.289761899 78.4914398193359 22.0
147.59869815 140.49983215332 23.0
112.83621692 109.161079406738 24.0
23.657389004 21.39674949646 25.0
227.88846006 224.852081298828 26.0
227.88846006 224.099212646484 27.0
26.737351965 26.9970550537109 28.0
254.62581203 251.366394042969 29.0
254.62581203 252.631256103516 30.0
46.838664829 47.8081550598145 31.0
9.3511470062 8.7345724105835 32.0
21.224152904 21.9319934844971 33.0
21.224152904 18.4335823059082 34.0
21.224152902 20.0221424102783 35.0
9.2871526916 8.00479602813721 36.0
7.9786377407 7.33905076980591 37.0
0 -2.12422180175781 38.0
0 -2.02996301651001 39.0
-12.272556288 -10.887659072876 40.0
-25.390414751 -27.5586032867432 41.0
5.0713489167 3.97365665435791 42.0
-33.115321804 -31.3177127838135 43.0
101.46918301 97.0360565185547 44.0
101.46918303 94.8101959228516 45.0
7.1329787789 5.60208225250244 46.0
2.6174668473 2.49674654006958 47.0
177.69958827 169.465545654297 48.0
171.54304356 158.510986328125 49.0
0 -0.561467945575714 50.0
78.318521323 77.935546875 51.0
86.177068862 90.2946472167969 52.0
80.992439838 80.5961227416992 53.0
80.992439838 81.4355010986328 54.0
80.992439838 81.0632781982422 55.0
0 4.07161808013916 56.0
80.289761909 79.6881103515625 57.0
227.88846007 221.953201293945 58.0
254.62581204 249.424530029297 59.0
21.224152912 20.9004421234131 60.0
0 2.4765625 61.0
177.87051421 157.23014831543 0.0
177.87051421 157.381759643555 1.0
176.58236531 160.589782714844 2.0
171.79329558 177.796325683594 3.0
0 -0.380195677280426 4.0
0 2.49001169204712 5.0
-70.152873017 -64.4183883666992 6.0
82.089076129 72.6956176757812 7.0
58.057715537 52.3443183898926 8.0
24.031360582 22.2785320281982 9.0
88.710800526 84.0865707397461 10.0
75.321215037 73.7085418701172 11.0
83.795411448 82.3859024047852 12.0
83.795411448 81.4031219482422 13.0
83.795411448 81.3030319213867 14.0
0 -1.47425818443298 15.0
0 -0.0593032836914062 16.0
0 -1.15541481971741 17.0
59.545401561 58.3959541320801 18.0
-59.545401571 -59.1994895935059 19.0
-59.545401549 -59.5394592285156 20.0
-59.545401561 -60.1432685852051 21.0
80.863464785 81.8225021362305 22.0
144.88449461 146.062438964844 23.0
109.41983303 107.950012207031 24.0
23.179052413 21.1260395050049 25.0
225.7479594 224.376724243164 26.0
225.7479594 225.282440185547 27.0
26.483806182 25.1836280822754 28.0
252.2317656 255.779174804688 29.0
252.2317656 254.070877075195 30.0
46.799795988 46.4438858032227 31.0
9.2803769684 8.83085060119629 32.0
21.318063215 19.963981628418 33.0
21.318063215 14.7355928421021 34.0
21.318063214 20.9155406951904 35.0
9.2842497141 7.14387607574463 36.0
8.0500313372 7.57214069366455 37.0
0 -2.44457674026489 38.0
0 -4.08209705352783 39.0
-12.433583366 -11.7426996231079 40.0
-24.678115021 -22.7127265930176 41.0
5.0838124933 6.58776235580444 42.0
-32.277684946 -34.478199005127 43.0
99.650235695 97.0063934326172 44.0
99.650235709 94.0319061279297 45.0
7.1092798974 6.65397119522095 46.0
2.771167161 1.48266696929932 47.0
177.87051422 159.510528564453 48.0
171.79329559 168.209167480469 49.0
0 3.0337963104248 50.0
82.089076138 76.7830963134766 51.0
88.710800536 84.2613296508789 52.0
83.795411458 81.6314163208008 53.0
83.795411458 80.487190246582 54.0
83.795411458 79.5650253295898 55.0
0 1.36744618415833 56.0
80.863464795 78.8252334594727 57.0
225.74795941 223.198928833008 58.0
252.23176561 248.064102172852 59.0
21.318063224 25.8337898254395 60.0
0 2.03378462791443 61.0
155.281458406996 164.723602294922 0.0
155.281458406996 169.633575439453 1.0
154.06458915 162.53938293457 2.0
149.27551943 146.269790649414 3.0
0 0.607545793056488 4.0
-7.0221453781e-09 6.4477391242981 5.0
-61.167999284 -61.3488464355469 6.0
72.5986798364597 72.1886825561523 7.0
44.4383293444597 51.7384071350098 8.0
28.1603504914597 22.1149864196777 9.0
87.1846541644597 85.0534133911133 10.0
74.3175940644597 73.0812225341797 11.0
78.2997631064597 82.7881546020508 12.0
78.2997631064597 79.8143997192383 13.0
78.2997631064597 80.5021514892578 14.0
0 -1.12125277519226 15.0
0 -0.232740819454193 16.0
-1.069592519e-09 -2.21503901481628 17.0
55.9944917727053 58.3006248474121 18.0
-55.9944917801895 -59.6059989929199 19.0
-55.9944917581895 -58.3044090270996 20.0
-55.9944917671895 -58.698860168457 21.0
78.198987255 78.4128799438477 22.0
142.997879917705 142.281890869141 23.0
108.606804254632 109.871170043945 24.0
23.5489866636316 19.6922092437744 25.0
221.196867183992 224.401565551758 26.0
221.196867183992 221.410217285156 27.0
26.722837163992 24.5310153961182 28.0
247.919704343992 256.898254394531 29.0
247.919704343992 259.085296630859 30.0
46.509657719 46.4189910888672 31.0
9.3083665075 9.06631278991699 32.0
22.2044954916797 20.2022380828857 33.0
22.2044954916797 23.46901512146 34.0
22.204495492 21.6495208740234 35.0
9.8280805117 10.8560190200806 36.0
8.5625755292 7.25412368774414 37.0
-9.372443382e-09 -1.03730440139771 38.0
-9.372443382e-09 0.784080862998962 39.0
-11.4604337620948 -16.5521907806396 40.0
-24.5295376665388 -21.3309574127197 41.0
4.94825125921641 6.64780855178833 42.0
-31.8549439142164 -34.8529396057129 43.0
95.5435055892164 91.192024230957 44.0
95.5435055683409 93.7182846069336 45.0
6.9889899182 7.25674057006836 46.0
2.3984148507 1.86671304702759 47.0
155.28145841 165.99772644043 48.0
149.27551944 154.608505249023 49.0
0 0.0370653867721558 50.0
72.598679836 76.538703918457 51.0
87.184654164 85.0021514892578 52.0
78.299763106 81.7453536987305 53.0
78.299763106 81.2339782714844 54.0
78.299763106 79.1463623046875 55.0
0 3.74673938751221 56.0
78.198987259 79.3663330078125 57.0
221.19686718 220.091186523438 58.0
247.91970434 249.679077148438 59.0
22.204495495 19.1708488464355 60.0
0 0.291259109973907 61.0
172.35119383 159.020690917969 0.0
172.35119383 165.14372253418 1.0
171.04252036 167.288101196289 2.0
166.25345064 164.488922119141 3.0
0 3.23561382293701 4.0
-1.6451690929e-09 1.03974580764771 5.0
-67.343509727 -65.0764465332031 6.0
77.809450777 73.6737670898438 7.0
48.520173488 45.0125274658203 8.0
29.289277284 21.7065258026123 9.0
84.481446789 85.7255249023438 10.0
71.797616593 73.9356842041016 11.0
77.801111627 83.127082824707 12.0
77.801111627 81.5435791015625 13.0
77.801111627 79.5764007568359 14.0
0 0.172659039497375 15.0
0 -0.872308194637299 16.0
0 -1.5846095085144 17.0
55.597991371 58.2832679748535 18.0
-55.597991386 -58.6324501037598 19.0
-55.597991364 -58.4351844787598 20.0
-55.597991376 -60.4586410522461 21.0
76.660753262 78.5278091430664 22.0
148.09630106 145.922897338867 23.0
112.40370209 109.140083312988 24.0
22.360263802 21.9482555389404 25.0
224.75705433 222.470092773438 26.0
224.75705433 228.308547973633 27.0
26.438768977 25.9064178466797 28.0
251.19582331 254.967498779297 29.0
251.19582331 254.843292236328 30.0
46.817069354 45.7465209960938 31.0
9.2854026436 8.63901329040527 32.0
21.062761888 19.4740371704102 33.0
21.062761888 17.5669384002686 34.0
21.062761886 20.3960399627686 35.0
9.2913647793 8.1458158493042 36.0
8.077666971 6.71571588516235 37.0
-6.187134435e-09 1.45391309261322 38.0
-6.1871447462e-09 -3.98621129989624 39.0
-11.405068051 -12.2312393188477 40.0
-23.750940197 -26.5812206268311 41.0
5.2624891682 7.09080123901367 42.0
-31.519132071 -33.1057281494141 43.0
97.076324 97.6797790527344 44.0
97.076324009 96.2917175292969 45.0
6.9328729838 4.90256690979004 46.0
2.6690883236 2.86559367179871 47.0
172.35119383 158.640472412109 48.0
166.25345065 163.674835205078 49.0
0 -2.84831929206848 50.0
77.809450781 76.0805282592773 51.0
84.481446793 87.8623199462891 52.0
77.801111632 81.5950317382812 53.0
77.801111632 81.9937591552734 54.0
77.801111632 79.4980621337891 55.0
0 1.92705988883972 56.0
76.660753271 78.6751174926758 57.0
224.75705434 221.070724487305 58.0
251.19582332 251.855819702148 59.0
21.062761894 19.4337997436523 60.0
0 1.26650261878967 61.0
177.164929737811 145.897430419922 0.0
177.164929737811 136.317611694336 1.0
175.88982533 162.760299682617 2.0
170.42519005 166.559188842773 3.0
2.9770452519 3.60111021995544 4.0
2.97704526003012 -3.37046766281128 5.0
-65.9413272866301 -67.0601959228516 6.0
77.3983446872054 67.9698181152344 7.0
45.0092029392054 53.9463882446289 8.0
32.3891417482054 31.293794631958 9.0
88.9452763112054 75.3668441772461 10.0
76.8882769402054 72.1569747924805 11.0
81.0739971372054 83.5133895874023 12.0
81.0739971372054 84.3966445922852 13.0
81.0739971372054 73.6970977783203 14.0
4.8866851593e-09 1.52226376533508 15.0
2.9722335533e-09 0.587432086467743 16.0
1.126127652e-08 1.24907660484314 17.0
58.119589377112 59.9065971374512 18.0
-58.1195893303081 -58.9923934936523 19.0
-58.1195893123081 -56.1766014099121 20.0
-58.1195892873081 -61.3385772705078 21.0
79.3386956807042 72.7973785400391 22.0
149.162636056112 148.292358398438 23.0
112.106907344575 106.704330444336 24.0
16.7110341091401 19.5246353149414 25.0
228.501331735622 227.213592529297 26.0
228.501331735622 229.274230957031 27.0
24.4810018976216 20.7159557342529 28.0
252.982333635622 252.531646728516 29.0
252.982333635622 252.748474121094 30.0
46.7869275631666 44.9259338378906 31.0
9.24433597897557 9.02713489532471 32.0
21.2191063928904 20.6559467315674 33.0
21.2191063928904 17.6619129180908 34.0
21.2191064360364 20.1464672088623 35.0
10.1662825331726 13.2780389785767 36.0
8.83780550317256 6.20301914215088 37.0
-5.84900388999996e-09 4.65851879119873 38.0
-5.84900389000002e-09 12.8796453475952 39.0
-11.6423309271541 -20.788745880127 40.0
-25.1727594668672 -22.8817882537842 41.0
5.57838051097118 4.32786178588867 42.0
-33.0743632721712 -31.8539752960205 43.0
98.7414251751712 95.1267852783203 44.0
98.7414250041712 88.6044464111328 45.0
6.47860133337289 10.9594011306763 46.0
2.52230631777289 2.98115372657776 47.0
177.16492974 131.970474243164 48.0
170.42519006 153.490020751953 49.0
2.9770452601 1.03455686569214 50.0
77.398344687 72.7456512451172 51.0
88.945276311 87.6454620361328 52.0
81.073997138 81.5939025878906 53.0
81.073997138 81.1960601806641 54.0
81.073997138 80.0464859008789 55.0
1.1261276519e-08 1.31798958778381 56.0
79.33869568 77.6551055908203 57.0
228.50133173 222.934204101562 58.0
252.98233363 246.414916992188 59.0
21.219106436 19.8000469207764 60.0
0 3.52136397361755 61.0
164.20276864 163.19841003418 0.0
164.20276864 164.499816894531 1.0
163.06270683 157.81721496582 2.0
158.27363711 146.860214233398 3.0
0 0.137821853160858 4.0
-4.7304324708e-13 6.21064901351929 5.0
-70.360866016 -68.39453125 6.0
82.45371021 75.0832748413086 7.0
54.738800388 51.278678894043 8.0
27.714909813 27.4431266784668 9.0
85.12516422 87.2895660400391 10.0
72.212095503 73.4793548583984 11.0
80.041125683 83.0516052246094 12.0
80.041125683 77.1731033325195 13.0
80.041125683 80.1431198120117 14.0
0 -1.52249312400818 15.0
0 -0.402916252613068 16.0
0 2.76452565193176 17.0
58.639912167 58.6222152709961 18.0
-58.639912187 -58.4037742614746 19.0
-58.639912165 -59.7175369262695 20.0
-58.639912177 -58.9224281311035 21.0
79.819003128 76.4960327148438 22.0
145.39530095 141.598327636719 23.0
111.93445704 111.526794433594 24.0
22.341212649 20.7231636047363 25.0
225.21430409 227.936447143555 26.0
225.21430409 224.811859130859 27.0
27.13711532 24.2449035644531 28.0
252.35141942 255.792373657227 29.0
252.35141942 256.354400634766 30.0
46.955438179 46.1532211303711 31.0
9.3240830286 9.17787170410156 32.0
21.17909094 21.5747776031494 33.0
21.17909094 20.370641708374 34.0
21.17909094 19.3668098449707 35.0
9.4871421981 6.98397445678711 36.0
8.1860699939 7.89634895324707 37.0
-2.3842303557e-11 0.717953443527222 38.0
-2.3842314775e-11 -3.21129751205444 39.0
-12.285819928 -12.0952882766724 40.0
-24.84712441 -27.1547241210938 41.0
5.3443386371 5.1508412361145 42.0
-32.887895515 -32.5927658081055 43.0
94.924355183 93.7793731689453 44.0
94.924355196 94.7898712158203 45.0
6.9459168683 5.95544195175171 46.0
2.5387236124 2.68239188194275 47.0
164.20276865 166.141250610352 48.0
158.27363712 145.87060546875 49.0
0 3.83543682098389 50.0
82.45371022 77.4585494995117 51.0
85.12516423 88.4177703857422 52.0
80.041125693 79.8197631835938 53.0
80.041125693 80.918212890625 54.0
80.041125693 80.4292373657227 55.0
0 4.10649251937866 56.0
79.819003138 79.9147872924805 57.0
225.2143041 220.197814941406 58.0
252.35141943 250.566970825195 59.0
21.17909095 19.3396854400635 60.0
0 1.98774838447571 61.0
161.88219684 158.280822753906 0.0
161.88219684 165.1845703125 1.0
160.7158787 159.91845703125 2.0
155.92680877 161.442016601562 3.0
0 -0.331950008869171 4.0
0 1.5471498966217 5.0
-63.068304676 -63.4603691101074 6.0
73.039586843 77.5407180786133 7.0
48.277384052 51.2517776489258 8.0
24.762202782 25.3994197845459 9.0
89.247766671 86.6848526000977 10.0
76.595423798 75.2800827026367 11.0
79.434258383 78.2172088623047 12.0
79.434258383 79.0038833618164 13.0
79.434258383 79.1201019287109 14.0
0 -4.28533029556274 15.0
0 0.0433433651924133 16.0
0 0.00333774089813232 17.0
59.087797573 58.7646865844727 18.0
-59.087797585 -58.7436485290527 19.0
-59.087797563 -58.3572959899902 20.0
-59.087797575 -60.6889762878418 21.0
80.902956997 79.0813064575195 22.0
144.3162998 143.185440063477 23.0
111.34231504 109.438385009766 24.0
21.508732657 21.0511131286621 25.0
225.21925681 224.150390625 26.0
225.21925681 220.809646606445 27.0
28.040931744 27.7706718444824 28.0
253.26018856 258.102935791016 29.0
253.26018856 257.372802734375 30.0
46.724913224 46.9565963745117 31.0
9.4036503757 9.0781364440918 32.0
21.815159416 19.6459693908691 33.0
21.815159416 17.6097812652588 34.0
21.815159415 20.9502716064453 35.0
9.2403216298 7.7659330368042 36.0
7.9542119034 7.52775382995605 37.0
-1.158760055e-09 -0.309494316577911 38.0
-1.1587516502e-09 -0.372691452503204 39.0
-11.868792232 -11.8866405487061 40.0
-24.302136471 -24.7554397583008 41.0
5.3914041986 4.06352806091309 42.0
-32.380084555 -32.7065086364746 43.0
95.663631366 93.3230667114258 44.0
95.663631378 94.6852874755859 45.0
6.8895931705 7.13876152038574 46.0
2.5419606567 2.59749627113342 47.0
161.88219685 163.223236083984 48.0
155.92680878 152.729614257812 49.0
0 2.01570343971252 50.0
73.039586851 76.4024963378906 51.0
89.247766679 90.3725280761719 52.0
79.434258392 80.5820770263672 53.0
79.434258392 81.1669769287109 54.0
79.434258392 79.0086212158203 55.0
0 -0.501103222370148 56.0
80.902957007 79.8517074584961 57.0
225.21925682 220.891616821289 58.0
253.26018857 249.555419921875 59.0
21.815159425 21.4329566955566 60.0
0 2.38003373146057 61.0
146.72313099 161.733428955078 0.0
146.72313099 162.831970214844 1.0
145.29244784 160.000183105469 2.0
139.86032113 145.458465576172 3.0
0 -5.06838178634644 4.0
0 7.58450508117676 5.0
-67.137093675 -64.1567764282227 6.0
77.896205617 77.1391906738281 7.0
48.783684087 54.646541595459 8.0
29.11252152 36.0154609680176 9.0
87.383489333 86.0259780883789 10.0
75.092272684 79.2085647583008 11.0
74.509772946 85.7099533081055 12.0
74.509772946 75.4863052368164 13.0
74.509772946 66.8195190429688 14.0
0 1.31463229656219 15.0
0 0.071042537689209 16.0
-4.0920264633e-11 -1.88080906867981 17.0
53.511775643 58.0321655273438 18.0
-53.511775663 -56.9864387512207 19.0
-53.511775641 -55.9021530151367 20.0
-53.511775653 -59.355411529541 21.0
73.875072594 75.882682800293 22.0
147.90164771 133.067474365234 23.0
113.57033813 107.032661437988 24.0
22.300690383 18.625675201416 25.0
221.77672032 225.232772827148 26.0
221.77672032 219.633773803711 27.0
25.446476634 26.2026271820068 28.0
247.22319696 254.798446655273 29.0
247.22319696 257.230987548828 30.0
46.472305843 46.7516288757324 31.0
9.4349234458 8.84954929351807 32.0
20.363296931 17.2399368286133 33.0
20.363296931 14.5404167175293 34.0
20.363296931 18.0513820648193 35.0
9.2306662327 7.96103954315186 36.0
7.8710867021 7.74818658828735 37.0
-1.9260925813e-11 3.55230307579041 38.0
-1.9258196653e-11 2.49777126312256 39.0
-11.155378323 -14.9308891296387 40.0
-24.695785723 -24.0677604675293 41.0
5.4303515565 7.303635597229 42.0
-32.477598697 -30.9490070343018 43.0
89.942362791 88.894157409668 44.0
89.942362804 91.1482849121094 45.0
7.2074565423 5.51829719543457 46.0
2.4713835175 2.31978058815002 47.0
146.723131 162.779174804688 48.0
139.86032114 149.138931274414 49.0
0 -2.35939502716064 50.0
77.896205627 78.7068939208984 51.0
87.383489343 83.3788986206055 52.0
74.509772956 81.2249755859375 53.0
74.509772956 79.4948043823242 54.0
74.509772956 79.3221282958984 55.0
0 0.05499666929245 56.0
73.875072604 79.2693634033203 57.0
221.77672033 220.34602355957 58.0
247.22319697 250.294876098633 59.0
20.363296941 19.592529296875 60.0
0 2.22631454467773 61.0
179.0259914 167.867523193359 0.0
179.0259914 169.914840698242 1.0
177.63719215 162.947357177734 2.0
172.84812242 159.465423583984 3.0
0 1.23682689666748 4.0
-1.6437543894e-09 8.29115676879883 5.0
-75.453213806 -57.9083786010742 6.0
86.770330124 81.298698425293 7.0
61.475266559 63.5732765197754 8.0
25.295063563 14.5845184326172 9.0
100.23620079 90.6003875732422 10.0
86.616819164 80.4937591552734 11.0
86.520684917 84.5791397094727 12.0
86.520684917 76.8081817626953 13.0
86.520684917 81.4239959716797 14.0
0 0.457417786121368 15.0
0 -0.252119600772858 16.0
0 1.77099585533142 17.0
61.738444825 59.7940216064453 18.0
-61.7384448397873 -58.776683807373 19.0
-61.7384448177873 -61.1812133789062 20.0
-61.7384448287873 -62.9905166625977 21.0
83.736352842 77.8951187133789 22.0
145.11867839 144.94075012207 23.0
112.89956417 111.735084533691 24.0
22.869741607 21.9257678985596 25.0
228.85503125 227.931289672852 26.0
228.85503125 231.5771484375 27.0
28.799322849 27.4030132293701 28.0
257.6543541 256.598205566406 29.0
257.6543541 251.171661376953 30.0
46.823592904 48.198299407959 31.0
9.2379758031 8.81124019622803 32.0
21.997908017 17.6100025177002 33.0
21.997908017 13.8290586471558 34.0
21.997908015 19.3900890350342 35.0
10.281267362 8.54693603515625 36.0
8.9920742976 7.43687105178833 37.0
-8.0125466756e-09 3.14848232269287 38.0
-8.0125466758e-09 2.45889091491699 39.0
-12.573206301 -12.8977022171021 40.0
-25.359831413 -23.8995094299316 41.0
4.6742227472 5.44534778594971 42.0
-32.345537561 -34.6331634521484 43.0
80.385801002 84.0253448486328 44.0
80.385801004 89.978401184082 45.0
6.8927810889 8.03003883361816 46.0
2.8484500481 2.74822854995728 47.0
179.02599141 168.596237182617 48.0
172.84812243 156.800552368164 49.0
0 1.13210916519165 50.0
86.770330127 75.9741973876953 51.0
100.23620079 90.2667541503906 52.0
86.52068492 82.1306762695312 53.0
86.52068492 80.6188201904297 54.0
86.52068492 80.9032669067383 55.0
0 2.46472024917603 56.0
83.73635285 79.0859222412109 57.0
228.85503125 220.663604736328 58.0
257.6543541 248.657943725586 59.0
21.997908022 17.7893772125244 60.0
0 2.13268899917603 61.0
168.25413216 163.703048706055 0.0
168.25413216 166.955123901367 1.0
166.96508107 163.168151855469 2.0
162.17601135 159.626495361328 3.0
0 -0.361185610294342 4.0
0 1.17716062068939 5.0
-63.41669957 -60.8032302856445 6.0
75.409932039 71.5934295654297 7.0
52.231959359 53.9571533203125 8.0
23.17797267 16.4676246643066 9.0
89.170029919 87.0826187133789 10.0
76.145483321 75.4902572631836 11.0
81.452853383 83.8852844238281 12.0
81.452853383 81.8851013183594 13.0
81.452853383 80.0561904907227 14.0
0 -1.22152018547058 15.0
0 -0.567316234111786 16.0
0 -0.678081095218658 17.0
56.307697478 58.8465423583984 18.0
-56.307697488 -59.506706237793 19.0
-56.307697466 -59.3068885803223 20.0
-56.307697478 -59.4495544433594 21.0
77.218718021 77.9145278930664 22.0
143.95237449 141.255584716797 23.0
109.85729556 108.105400085449 24.0
22.649492798 19.7362270355225 25.0
221.17109252 222.299957275391 26.0
221.17109252 222.729751586914 27.0
27.237550476 24.0346603393555 28.0
248.40864301 250.727874755859 29.0
248.40864301 253.511825561523 30.0
46.898277479 47.9354667663574 31.0
9.3583715969 8.58743953704834 32.0
20.911020533 22.2232913970947 33.0
20.911020533 26.2610092163086 34.0
20.911020533 21.3616542816162 35.0
9.2368244193 9.68162250518799 36.0
7.9890702266 7.69323778152466 37.0
0 -0.654069244861603 38.0
0 -5.4309778213501 39.0
-12.061718657 -12.6461620330811 40.0
-24.538454492 -26.3531341552734 41.0
4.8869322107 6.55457210540771 42.0
-31.868504413 -34.4001731872559 43.0
103.20181173 93.9459457397461 44.0
103.20181174 94.787712097168 45.0
7.1321760323 5.85695362091064 46.0
2.5841040129 2.19469809532166 47.0
168.25413217 170.172744750977 48.0
162.17601136 157.350357055664 49.0
0 -0.146920800209045 50.0
75.409932049 77.2582855224609 51.0
89.170029929 88.5906600952148 52.0
81.452853393 81.3527069091797 53.0
81.452853393 81.0411758422852 54.0
81.452853393 79.0702590942383 55.0
0 0.637151896953583 56.0
77.218718031 79.0145645141602 57.0
221.17109253 221.216323852539 58.0
248.40864302 246.466995239258 59.0
20.911020543 19.7313442230225 60.0
0 0.638004124164581 61.0
152.64297682 160.036682128906 0.0
152.64297682 166.128677368164 1.0
151.30880137 160.864929199219 2.0
146.51973165 155.787445068359 3.0
0 1.6928825378418 4.0
-9.7575826113e-12 -0.64812558889389 5.0
-60.171253998 -64.0317535400391 6.0
71.909917528 73.1893157958984 7.0
48.758386357 52.1474494934082 8.0
23.151531161 21.7900924682617 9.0
86.826849954 87.0083618164062 10.0
73.552810128 76.3201599121094 11.0
80.602393349 83.9651336669922 12.0
80.602393349 80.768196105957 13.0
80.602393349 81.0190582275391 14.0
0 -0.84912496805191 15.0
0 -0.124878108501434 16.0
-2.0478330647e-11 1.56776702404022 17.0
57.787708491 57.8376617431641 18.0
-57.787708511 -58.7761268615723 19.0
-57.787708489 -58.6178245544434 20.0
-57.787708501 -60.1469497680664 21.0
78.972809862 77.8163452148438 22.0
147.55108975 141.160171508789 23.0
114.00553285 109.683776855469 24.0
23.832830305 21.1275196075439 25.0
226.52389963 225.416320800781 26.0
226.52389963 224.487258911133 27.0
27.250651875 23.9672393798828 28.0
253.77455151 255.55924987793 29.0
253.77455151 255.986831665039 30.0
46.750312798 47.26416015625 31.0
9.3327220691 8.8815860748291 32.0
21.185101351 20.4296798706055 33.0
21.185101351 13.5004377365112 34.0
21.185101351 21.0598030090332 35.0
9.3633052591 7.69415855407715 36.0
8.0955714756 7.50305700302124 37.0
-3.2383671916e-11 -1.62402009963989 38.0
-3.2383671917e-11 -0.647268831729889 39.0
-11.546597478 -16.6982536315918 40.0
-24.700979458 -24.6478672027588 41.0
4.9555451378 6.78781080245972 42.0
-32.24502116 -33.9722900390625 43.0
95.439969452 92.4297256469727 44.0
95.439969466 95.0690383911133 45.0
6.998990478 6.08623933792114 46.0
2.5429915587 1.70197439193726 47.0
152.64297683 162.449752807617 48.0
146.51973166 150.245803833008 49.0
0 -1.06036114692688 50.0
71.909917538 74.5766372680664 51.0
86.826849964 88.0537414550781 52.0
80.602393359 80.2794189453125 53.0
80.602393359 80.9284744262695 54.0
80.602393359 80.7080764770508 55.0
0 4.07350015640259 56.0
78.972809872 79.7845153808594 57.0
226.52389964 221.917831420898 58.0
253.77455152 249.932388305664 59.0
21.185101361 24.6385650634766 60.0
0 0.51330977678299 61.0
196.25928605 170.616348266602 0.0
196.25928605 140.56169128418 1.0
194.98984575 175.298400878906 2.0
190.20077603 184.880981445312 3.0
0 4.48895025253296 4.0
-1.1025513597e-11 3.22876024246216 5.0
-62.224177208 -62.1030883789062 6.0
71.4221583 79.1224594116211 7.0
52.673291002 53.5543556213379 8.0
18.748867288 25.9505043029785 9.0
87.907677454 86.1435012817383 10.0
75.168505269 74.3984146118164 11.0
78.685257149 87.799430847168 12.0
78.685257149 79.9794845581055 13.0
78.685257149 76.2600402832031 14.0
0 -7.71416091918945 15.0
0 -0.568047344684601 16.0
-1.6271889453e-11 0.328090131282806 17.0
61.529291107 58.3423233032227 18.0
-61.529291127 -60.678882598877 19.0
-61.529291105 -59.9003715515137 20.0
-61.529291118 -59.5348892211914 21.0
83.769541051 77.5344314575195 22.0
134.1633165 136.856567382812 23.0
103.86938131 116.203414916992 24.0
22.661968589 18.5166797637939 25.0
217.93285756 227.738342285156 26.0
217.93285756 216.371383666992 27.0
26.271042688 23.033145904541 28.0
244.20390026 249.96012878418 29.0
244.20390026 247.862243652344 30.0
47.086535579 46.4017448425293 31.0
9.2366801175 8.74038982391357 32.0
22.240249923 20.5689754486084 33.0
22.240249923 15.0594282150269 34.0
22.240249923 22.4124031066895 35.0
9.9373883071 6.38421630859375 36.0
8.6141042032 7.55092334747314 37.0
-3.3061659441e-11 3.18837666511536 38.0
-3.3061659088e-11 -2.79774022102356 39.0
-11.391859936 -13.326868057251 40.0
-25.263762923 -25.4034442901611 41.0
5.5724343252 -0.886982142925262 42.0
-33.517121199 -32.901252746582 43.0
96.276116611 91.1942901611328 44.0
96.276116624 94.0519943237305 45.0
6.6539984995 3.76974868774414 46.0
2.5354246784 2.8068163394928 47.0
196.25928606 140.336059570312 48.0
190.20077604 174.783416748047 49.0
0 -3.15927267074585 50.0
71.42215831 73.8201141357422 51.0
87.907677464 88.4184341430664 52.0
78.685257158 81.4348602294922 53.0
78.685257158 79.4645843505859 54.0
78.685257158 79.3870239257812 55.0
0 1.76400089263916 56.0
83.769541061 78.8210067749023 57.0
217.93285757 215.713241577148 58.0
244.20390027 246.907348632812 59.0
22.240249933 22.404224395752 60.0
0 2.33555960655212 61.0
191.243038255948 165.662857055664 0.0
191.243038255948 167.668350219727 1.0
190.32096084 165.057495117188 2.0
185.53189109 156.271255493164 3.0
0 4.90902328491211 4.0
-5.44047916199999e-09 -1.66890668869019 5.0
-45.5635682614216 -67.9886322021484 6.0
54.3885125749757 73.5510482788086 7.0
50.0372004249757 41.3194313049316 8.0
4.35131214957568 25.2292671203613 9.0
88.7666922029757 85.1084747314453 10.0
76.3079239779757 64.9908599853516 11.0
89.3707401439757 72.9733581542969 12.0
89.3707401439757 64.0062255859375 13.0
89.3707401439757 69.6000289916992 14.0
2.0098395543e-09 -5.98749113082886 15.0
0 -2.63715839385986 16.0
2.81071386999998e-09 -0.442050874233246 17.0
50.4898133690776 49.1035003662109 18.0
-50.4898133474635 -42.1806907653809 19.0
-50.4898133284635 -60.1069602966309 20.0
-50.4898133174635 -64.052375793457 21.0
69.2838634339394 72.5167388916016 22.0
155.384083261078 121.85546875 23.0
118.602963048847 100.087715148926 24.0
22.4522169698471 28.0475234985352 25.0
224.667946701895 223.685684204102 26.0
224.667946701895 227.735916137695 27.0
21.990834925895 30.0927200317383 28.0
246.658781621895 253.58935546875 29.0
246.658781621895 257.313018798828 30.0
45.2491018492328 47.4955253601074 31.0
8.28486747263445 10.2102012634277 32.0
18.7940501162649 21.5288677215576 33.0
18.7940501162649 12.9021005630493 34.0
18.7940501427847 13.3974771499634 35.0
7.73818669866077 9.74925804138184 36.0
6.59564943346077 7.27200317382812 37.0
-7.46848417e-09 -1.12456011772156 38.0
-7.46848418000001e-09 6.2295298576355 39.0
-13.0784300497318 -16.3183097839355 40.0
-25.3852108871063 -18.1431083679199 41.0
5.8857516506464 12.5454225540161 42.0
-33.5461830611464 -36.0137214660645 43.0
79.0155630011464 93.2400131225586 44.0
79.0155627922581 87.0110168457031 45.0
7.0169525334 -0.996220767498016 46.0
2.5266999857 1.71630835533142 47.0
191.24303825 169.471435546875 48.0
185.5318911 185.305572509766 49.0
0 -0.0965622067451477 50.0
54.388512575 70.0879745483398 51.0
88.766692203 83.1926040649414 52.0
89.370740144 76.4519958496094 53.0
89.370740144 67.652587890625 54.0
89.370740144 78.296760559082 55.0
2.8107138724e-09 8.16756057739258 56.0
69.283863434 76.8240966796875 57.0
224.6679467 218.090225219727 58.0
246.65878163 249.275146484375 59.0
18.794050143 15.2432851791382 60.0
0 2.71892619132996 61.0
161.055547296419 159.463455200195 0.0
161.055547296419 162.629730224609 1.0
159.7047314 158.621063232422 2.0
154.91566166 157.988540649414 3.0
0 -1.75240874290466 4.0
-7.453676137e-09 2.51288604736328 5.0
-63.9326951505597 -63.0221405029297 6.0
74.7974153094589 76.3322601318359 7.0
54.7860929434589 52.7191047668457 8.0
20.0113223654589 22.1771850585938 9.0
91.8208956394589 85.5688705444336 10.0
78.7059598874589 73.1144485473633 11.0
82.5901756594589 82.0715713500977 12.0
82.5901756594589 80.1103744506836 13.0
82.5901756594589 80.0095672607422 14.0
1.7015838358e-09 -2.05273962020874 15.0
0 0.0374583005905151 16.0
7.6026848e-10 -2.62952876091003 17.0
56.2263147674321 58.7478218078613 18.0
-56.2263147546884 -58.8654479980469 19.0
-56.2263147346884 -59.4848327636719 20.0
-56.2263147296884 -57.7499961853027 21.0
77.9706539095294 78.8263092041016 22.0
147.778237220432 143.442260742188 23.0
114.45285942304 111.93155670166 24.0
22.4516058450402 20.9458847045898 25.0
225.748891122837 227.337097167969 26.0
225.748891122837 225.230346679688 27.0
26.8586585698372 23.5172023773193 28.0
252.607549692837 254.312774658203 29.0
252.607549692837 253.873870849609 30.0
46.8536817168521 47.0026321411133 31.0
9.34627627379195 8.66091918945312 32.0
21.7443391796791 20.7118816375732 33.0
21.7443391796791 24.2901020050049 34.0
21.7443392010203 21.3826026916504 35.0
9.27549209805957 8.16342067718506 36.0
8.02416121325957 6.9852991104126 37.0
-8.05791201999999e-09 0.73761385679245 38.0
-8.05791202000001e-09 1.02350926399231 39.0
-11.8306311023442 -14.2825622558594 40.0
-24.2073342721486 -24.4874401092529 41.0
5.38142997811576 2.44996285438538 42.0
-32.0131755107158 -32.0721092224121 43.0
95.5397704097158 92.3600845336914 44.0
95.5397703337158 93.9741744995117 45.0
6.5841793262 6.52369499206543 46.0
2.4599891433 2.56682538986206 47.0
161.05554729 163.086273193359 48.0
154.91566167 154.632369995117 49.0
0 0.563391387462616 50.0
74.79741531 77.252799987793 51.0
91.82089564 83.9520797729492 52.0
82.59017566 81.6953659057617 53.0
82.59017566 79.8443145751953 54.0
82.59017566 78.4787902832031 55.0
7.6026847694e-10 0.29662412405014 56.0
77.970653909 79.2238235473633 57.0
225.74889113 222.439300537109 58.0
252.6075497 249.560455322266 59.0
21.744339201 22.864875793457 60.0
0 1.13002014160156 61.0
153.45630845 157.846054077148 0.0
153.45630845 162.785339355469 1.0
152.27388972 169.745529174805 2.0
147.48481999 157.547988891602 3.0
0 0.228997409343719 4.0
-1.0348165221e-09 4.09080219268799 5.0
-67.527517278 -61.0228996276855 6.0
78.864930217 78.2647018432617 7.0
63.598385629 53.0704765319824 8.0
15.266544583 29.6661071777344 9.0
86.988488038 86.5590286254883 10.0
75.413175072 77.8539886474609 11.0
77.949070768 85.9935531616211 12.0
77.949070768 79.7432861328125 13.0
77.949070768 72.9773406982422 14.0
0 -8.07441902160645 15.0
0 -0.585292160511017 16.0
0 0.52325302362442 17.0
56.957479582 59.480712890625 18.0
-56.957479597 -59.4113616943359 19.0
-56.957479575 -58.966064453125 20.0
-56.957479587 -58.2117614746094 21.0
80.099169183 78.8520660400391 22.0
152.32242761 137.579177856445 23.0
115.72624204 112.943344116211 24.0
22.85984251 19.6951160430908 25.0
232.4215968 221.7119140625 26.0
232.4215968 223.896621704102 27.0
23.581579218 29.0908660888672 28.0
256.00317603 247.303695678711 29.0
256.00317603 249.910003662109 30.0
46.710474883 46.961597442627 31.0
9.3890758442 8.73222732543945 32.0
23.141689594 18.4091815948486 33.0
23.141689594 16.3139667510986 34.0
23.141689593 19.9925861358643 35.0
10.684884215 9.23737049102783 36.0
9.4227954612 8.00438213348389 37.0
-3.4712976217e-09 -5.89797782897949 38.0
-3.4712976217e-09 1.25412392616272 39.0
-11.889861599 -12.8110990524292 40.0
-25.252705442 -22.574462890625 41.0
5.2702049723 5.13136577606201 42.0
-32.885090471 -32.994571685791 43.0
97.705632693 94.4482803344727 44.0
97.705632704 91.5599136352539 45.0
6.7160299085 5.30523872375488 46.0
1.8921580034 2.86065030097961 47.0
153.45630846 165.56640625 48.0
147.48482 149.422058105469 49.0
0 0.0673092007637024 50.0
78.864930224 77.2592926025391 51.0
86.988488045 93.1648788452148 52.0
77.949070774 82.3914642333984 53.0
77.949070774 81.4169769287109 54.0
77.949070774 80.3136291503906 55.0
0 3.70716738700867 56.0
80.099169192 79.9301605224609 57.0
232.42159681 216.034515380859 58.0
256.00317603 247.22607421875 59.0
23.141689601 21.4480133056641 60.0
0 2.68804907798767 61.0
185.245458306128 159.532821655273 0.0
185.245458306128 167.833633422852 1.0
183.82623146 161.442016601562 2.0
179.03716173 170.586273193359 3.0
0 2.04981184005737 4.0
-8.7965670243e-09 -0.382828652858734 5.0
-68.514197289 -65.5893783569336 6.0
78.6385475979547 75.234245300293 7.0
58.1390437949547 33.7385101318359 8.0
20.4995038039547 22.5668029785156 9.0
94.6960532909547 84.6592102050781 10.0
82.4064072059547 71.7513427734375 11.0
87.4280572209547 71.5389709472656 12.0
87.4280572209547 69.588752746582 13.0
87.4280572209547 81.1365203857422 14.0
0 -0.444423735141754 15.0
0 -0.276923000812531 16.0
-7.144111616e-09 4.16800308227539 17.0
54.7339621658844 56.8717918395996 18.0
-54.7339621789321 -57.9688568115234 19.0
-54.7339621579321 -57.8627891540527 20.0
-54.7339621659321 -59.3919906616211 21.0
76.560310706 79.5906219482422 22.0
143.070669456884 136.409515380859 23.0
107.393556583606 115.960952758789 24.0
20.1882973886055 18.54958152771 25.0
219.630980162255 234.888473510742 26.0
219.630980162255 228.203186035156 27.0
28.3147752272555 27.6894435882568 28.0
247.945755392255 258.563385009766 29.0
247.945755392255 256.76513671875 30.0
46.475382013 44.7290382385254 31.0
9.2282724698 9.926438331604 32.0
21.8263485433898 19.3122634887695 33.0
21.8263485433898 13.1911602020264 34.0
21.826348546 19.0191383361816 35.0
9.0903726535 8.81616306304932 36.0
7.7992130176 5.50540447235107 37.0
-9.624066696e-09 -2.54686307907104 38.0
-9.624066695e-09 2.94918489456177 39.0
-11.943086791966 -17.569149017334 40.0
-23.8599887047736 -27.3843746185303 41.0
4.86263780496227 2.92028379440308 42.0
-30.9337897289623 -30.3326473236084 43.0
92.2688145729623 89.8268051147461 44.0
92.2688145496027 92.6708221435547 45.0
7.0436495545 5.05848217010498 46.0
2.4422558251 2.31274843215942 47.0
185.24545831 164.111892700195 48.0
179.03716174 173.894561767578 49.0
0 -0.408442437648773 50.0
78.638547598 76.5820541381836 51.0
94.696053291 87.012939453125 52.0
87.428057221 76.8561477661133 53.0
87.428057221 73.7853012084961 54.0
87.428057221 79.1652755737305 55.0
0 2.25526452064514 56.0
76.56031071 78.782585144043 57.0
219.63098016 223.641265869141 58.0
247.94575539 248.84228515625 59.0
21.826348547 24.085132598877 60.0
0 3.1074812412262 61.0
153.30263323 157.701889038086 0.0
153.30263323 163.485336303711 1.0
152.12155095 158.967864990234 2.0
147.33248123 152.431945800781 3.0
0 1.27010607719421 4.0
-1.036009053e-11 0.263977468013763 5.0
-63.79326684 -65.2791519165039 6.0
74.066080199 76.924430847168 7.0
49.257535433 52.0238418579102 8.0
24.808544756 23.8600730895996 9.0
93.076678886 88.3865966796875 10.0
80.167385514 77.4255065917969 11.0
85.355792054 84.5462417602539 12.0
85.355792054 79.5964279174805 13.0
85.355792054 78.1725692749023 14.0
0 -0.716274321079254 15.0
0 0.0241878032684326 16.0
-1.4869952162e-11 1.20923495292664 17.0
57.791749773 57.9205207824707 18.0
-57.791749793 -58.5988349914551 19.0
-57.791749771 -59.2294044494629 20.0
-57.791749784 -59.3059730529785 21.0
79.485008835 77.5516052246094 22.0
142.89008365 141.308029174805 23.0
108.50551575 109.606300354004 24.0
22.87624861 20.1146602630615 25.0
222.37509249 219.152282714844 26.0
222.37509249 221.40234375 27.0
27.183619735 26.7830619812012 28.0
249.55871224 253.792144775391 29.0
249.55871224 253.35041809082 30.0
46.565146504 47.2012214660645 31.0
9.3494813596 8.54714584350586 32.0
21.693259042 19.1272125244141 33.0
21.693259042 22.287784576416 34.0
21.693259042 21.2558193206787 35.0
9.4541947719 9.72253131866455 36.0
8.1468047521 7.80622625350952 37.0
-3.2054701962e-11 2.47348833084106 38.0
-3.2054684988e-11 -2.23412752151489 39.0
-11.873492658 -13.1899290084839 40.0
-24.930356219 -25.1794090270996 41.0
4.9926361721 5.86267232894897 42.0
-32.318741974 -33.5793838500977 43.0
95.276261874 95.6772155761719 44.0
95.276261887 94.6682357788086 45.0
6.8642655567 7.65038299560547 46.0
2.4654665248 2.83246159553528 47.0
153.30263324 162.748443603516 48.0
147.33248124 146.297180175781 49.0
0 -1.09081435203552 50.0
74.066080209 74.88525390625 51.0
93.076678896 90.5929107666016 52.0
85.355792064 80.7928314208984 53.0
85.355792064 81.6294174194336 54.0
85.355792064 79.7192916870117 55.0
0 2.63580918312073 56.0
79.485008845 79.080207824707 57.0
222.3750925 219.417053222656 58.0
249.55871225 248.845855712891 59.0
21.693259052 17.8646469116211 60.0
0 1.70618319511414 61.0
152.13801784 160.553802490234 0.0
152.13801784 161.833297729492 1.0
150.90884694 163.175201416016 2.0
146.11977722 156.844467163086 3.0
0 3.93334722518921 4.0
-2.558694878e-09 6.0013256072998 5.0
-70.322475265 -63.8794288635254 6.0
81.5290715321181 84.2096633911133 7.0
42.9697558131181 52.7967262268066 8.0
38.5593157191181 24.5920848846436 9.0
82.3682895711181 82.2136383056641 10.0
69.4524357701181 72.5929412841797 11.0
76.3274687121181 80.1995010375977 12.0
76.3274687121181 80.5531158447266 13.0
76.3274687121181 77.0881042480469 14.0
0 -1.70671510696411 15.0
0 -1.2283775806427 16.0
-8.046193682e-09 -1.75717973709106 17.0
61.346163736 59.012939453125 18.0
-61.3461637561771 -57.7952651977539 19.0
-61.3461637351771 -59.6315765380859 20.0
-61.3461637451771 -59.0630950927734 21.0
83.474591253 71.3625335693359 22.0
135.69447911 141.654067993164 23.0
103.72775045 108.750061035156 24.0
22.563284881 22.1817684173584 25.0
219.16907036629 217.624130249023 26.0
219.16907036629 212.050918579102 27.0
25.9949833802901 28.6721954345703 28.0
245.16405374629 248.882690429688 29.0
245.16405374629 251.868179321289 30.0
46.217623333 44.815372467041 31.0
9.3150012558 9.49200916290283 32.0
22.1284275138155 23.1292304992676 33.0
22.1284275138155 13.7601089477539 34.0
22.12842751 22.7078151702881 35.0
9.3343165979 8.2629861831665 36.0
8.1445255899 6.12377262115479 37.0
-9.7955484612e-09 -0.974577605724335 38.0
-9.7955484607e-09 -0.0193242430686951 39.0
-12.487309077 -17.6934013366699 40.0
-25.759730672 -25.1294078826904 41.0
4.6588769402 4.23417329788208 42.0
-33.26286977 -32.3006820678711 43.0
89.720286132854 96.5370330810547 44.0
89.720286156 94.5148010253906 45.0
7.1432086578 8.31339263916016 46.0
2.713426924 -0.610271394252777 47.0
152.13801784 157.163238525391 48.0
146.11977723 162.414581298828 49.0
0 -4.18948936462402 50.0
81.529071533 75.3400726318359 51.0
82.368289571 87.3412094116211 52.0
76.327468712 81.3555145263672 53.0
76.327468712 77.264778137207 54.0
76.327468712 77.3903579711914 55.0
0 -1.44669532775879 56.0
83.47459126 80.1200714111328 57.0
219.16907037 208.652069091797 58.0
245.16405375 248.349761962891 59.0
22.128427516 21.3807601928711 60.0
0 4.27111053466797 61.0
137.17687742 143.006454467773 0.0
137.17687742 143.032775878906 1.0
135.86355256 143.523300170898 2.0
131.0744823 135.743881225586 3.0
0 -2.09159421920776 4.0
0 0.341849207878113 5.0
-59.42861149 -59.2315139770508 6.0
69.705106679 70.17724609375 7.0
65.43029471 62.4528312683105 8.0
4.2748119615 8.48783111572266 9.0
78.669161727 80.2065200805664 10.0
65.229460649 71.2585144042969 11.0
77.500548395 80.8047180175781 12.0
77.500548395 77.9000701904297 13.0
77.500548395 76.9198608398438 14.0
0 -0.64198762178421 15.0
0 0.28728449344635 16.0
0 -3.15485000610352 17.0
59.754945382 58.5015258789062 18.0
-59.754945394 -59.4383087158203 19.0
-59.754945372 -60.0649642944336 20.0
-59.754945384 -61.8801422119141 21.0
80.656571382 77.8813858032227 22.0
140.87903669 140.394561767578 23.0
106.41719675 107.357681274414 24.0
22.90053999 20.4083709716797 25.0
221.53560808 223.594223022461 26.0
221.53560808 221.987747192383 27.0
27.282645659 25.3551406860352 28.0
248.81825375 253.574676513672 29.0
248.81825375 252.408157348633 30.0
47.033013387 47.2299919128418 31.0
9.3131031344 9.18030452728271 32.0
20.901625993 20.4819431304932 33.0
20.901625993 21.1129970550537 34.0
20.901625993 20.8195400238037 35.0
9.3083381265 8.28715133666992 36.0
8.044767894 8.38281917572021 37.0
-2.5262790395e-09 -0.804296553134918 38.0
-2.5262785966e-09 0.113371789455414 39.0
-12.387792857 -14.5510091781616 40.0
-23.81592281 -27.5234241485596 41.0
4.9278776077 3.5162615776062 42.0
-31.108913619 -34.7394485473633 43.0
92.052766479 91.3064651489258 44.0
92.05276649 90.2343292236328 45.0
6.7273095346 3.00315618515015 46.0
2.6568330431 1.83220052719116 47.0
137.17687743 136.682907104492 48.0
131.07448231 127.506469726562 49.0
0 3.54901957511902 50.0
69.705106686 70.986442565918 51.0
78.669161734 84.2529830932617 52.0
77.500548402 81.0479965209961 53.0
77.500548402 80.9212951660156 54.0
77.500548402 79.4541549682617 55.0
0 -0.0822644829750061 56.0
80.656571391 80.3950500488281 57.0
221.53560809 218.26057434082 58.0
248.81825375 246.469757080078 59.0
20.901626002 19.9398994445801 60.0
0 2.89454770088196 61.0
137.13846934 158.91291809082 0.0
137.13846934 150.986724853516 1.0
135.92197635 150.1650390625 2.0
131.13290634 152.210891723633 3.0
0 -1.09210205078125 4.0
0 -2.01123023033142 5.0
-62.441890661 -62.5267066955566 6.0
71.799187529 67.8026504516602 7.0
47.084906293 23.0055961608887 8.0
24.714281233 13.8530769348145 9.0
86.43888905 86.5032730102539 10.0
73.106768476 77.1076354980469 11.0
78.767521507 81.5571594238281 12.0
78.767521507 82.4412002563477 13.0
78.767521507 80.7606506347656 14.0
0 1.71697807312012 15.0
0 -0.635009825229645 16.0
0 -2.03317499160767 17.0
54.585219982 58.1784477233887 18.0
-54.5852200009043 -58.3511734008789 19.0
-54.5852199789043 -58.432933807373 20.0
-54.5852199909043 -60.0339241027832 21.0
76.313412818 77.3753967285156 22.0
141.23234166 137.775650024414 23.0
107.09010202 113.330032348633 24.0
23.091353976 21.0643310546875 25.0
217.54575449 227.284393310547 26.0
217.54575449 224.992324829102 27.0
27.455041784 21.9919414520264 28.0
245.00079627 251.160049438477 29.0
245.00079627 252.739929199219 30.0
46.882024227 49.9916877746582 31.0
9.448843687 8.53668594360352 32.0
21.728192832 17.1581039428711 33.0
21.728192832 11.1834526062012 34.0
21.728192829 19.9378547668457 35.0
9.712307849 9.15047359466553 36.0
8.4187401487 7.09975290298462 37.0
-7.328613513e-09 -2.42366600036621 38.0
-7.3286136455e-09 -4.41936159133911 39.0
-10.750384377 -13.5948066711426 40.0
-24.712875033 -25.3574619293213 41.0
5.0015155386 5.0268726348877 42.0
-32.37968635 -33.4421043395996 43.0
91.675758805 91.4501342773438 44.0
91.675758806 95.6742858886719 45.0
6.4608248496 6.4703893661499 46.0
1.6880388828 2.72294592857361 47.0
137.13846935 149.683654785156 48.0
131.13290635 130.442520141602 49.0
0 3.11720371246338 50.0
71.799187531 72.9834136962891 51.0
86.438889053 92.2690277099609 52.0
78.76752151 81.7139129638672 53.0
78.76752151 79.0022125244141 54.0
78.76752151 79.8092346191406 55.0
0 1.41558957099915 56.0
76.313412826 78.9837341308594 57.0
217.54575449 219.962600708008 58.0
245.00079628 247.450592041016 59.0
21.728192837 22.0046710968018 60.0
0 2.05504369735718 61.0
131.32341614 168.637130737305 0.0
131.32341614 159.243423461914 1.0
130.26628725 155.461029052734 2.0
125.47721753 141.508331298828 3.0
0 -16.8846244812012 4.0
-1.1931313088e-09 6.18940353393555 5.0
-47.784117128 -54.1045951843262 6.0
56.649653909 72.1643371582031 7.0
50.208874528 65.6240310668945 8.0
6.4407793798 9.73790264129639 9.0
85.735166622 92.4845809936523 10.0
73.867476036 83.8279495239258 11.0
75.083477257 106.161712646484 12.0
75.083477257 72.7754516601562 13.0
75.083477257 76.1397171020508 14.0
0 1.6951105594635 15.0
0 -1.37529897689819 16.0
-4.4620188946e-09 1.01799583435059 17.0
66.553355537 59.9158401489258 18.0
-66.5533555603686 -58.5431747436523 19.0
-66.5533555383686 -60.2339248657227 20.0
-66.5533555503686 -63.6286277770996 21.0
87.426637304 71.3088455200195 22.0
138.58906602 145.905334472656 23.0
105.34441988 102.457893371582 24.0
26.135633908 24.5328769683838 25.0
226.01570334 221.958221435547 26.0
226.01570334 221.103713989258 27.0
25.955894251 21.9739894866943 28.0
251.97159759 250.971939086914 29.0
251.97159759 250.041580200195 30.0
46.542898628 48.2532081604004 31.0
9.4492233321 6.80576801300049 32.0
20.873281758 16.5797939300537 33.0
20.873281758 11.966986656189 34.0
20.873281756 20.2482891082764 35.0
9.6298339596 10.7235174179077 36.0
8.2675355695 9.29043102264404 37.0
-8.271034304e-09 -0.60243946313858 38.0
-8.2710352079e-09 0.305939853191376 39.0
-12.089553687 -10.0653676986694 40.0
-25.051731592 -29.0459213256836 41.0
5.7708426268 4.80828857421875 42.0
-33.517053181 -33.0807685852051 43.0
106.46933719 98.3003005981445 44.0
106.4693372 90.2534790039062 45.0
5.9808473206 6.50777339935303 46.0
2.2352429495 1.693354845047 47.0
131.32341615 167.746017456055 48.0
125.47721754 150.912811279297 49.0
0 -0.565988004207611 50.0
56.649653911 71.0344085693359 51.0
85.735166624 74.5326385498047 52.0
75.083477258 75.2894668579102 53.0
75.083477258 68.7667999267578 54.0
75.083477258 76.7282867431641 55.0
0 3.10320281982422 56.0
87.426637311 78.7432098388672 57.0
226.01570334 219.242767333984 58.0
251.97159759 252.736083984375 59.0
20.873281763 20.0467071533203 60.0
0 4.09624433517456 61.0
172.92832751 162.751068115234 0.0
172.92832751 167.998458862305 1.0
171.78894318 167.349792480469 2.0
166.99987346 163.631652832031 3.0
0 0.32995218038559 4.0
-1.2877477402e-09 0.600756943225861 5.0
-69.048158693 -65.8662185668945 6.0
80.332961755 77.5103149414062 7.0
49.177364591 48.7197799682617 8.0
31.155597164 26.1453685760498 9.0
83.873112543 88.6293029785156 10.0
70.451885255 75.8163604736328 11.0
78.059056003 81.7220306396484 12.0
78.059056003 78.1977005004883 13.0
78.059056003 78.6708526611328 14.0
0 -0.783433258533478 15.0
0 0.0497881770133972 16.0
-5.644279346e-09 2.75825953483582 17.0
57.948703265 58.8503494262695 18.0
-57.948703287781 -59.971061706543 19.0
-57.948703265781 -58.1465492248535 20.0
-57.948703276781 -59.4423370361328 21.0
78.51377712 77.4166259765625 22.0
147.0574902 143.959701538086 23.0
112.59455345 107.817680358887 24.0
22.203032012 18.21875 25.0
225.571267329062 228.530731201172 26.0
225.571267329062 226.827301025391 27.0
27.1922402540623 24.4481792449951 28.0
252.763507579062 257.836578369141 29.0
252.763507579062 252.518524169922 30.0
46.557847631 44.6832466125488 31.0
9.4009572528 7.98283910751343 32.0
20.565073849 20.1297035217285 33.0
20.565073849 17.409086227417 34.0
20.565073845 20.5113697052002 35.0
9.1342989068 7.96131420135498 36.0
7.9328692318 7.65216827392578 37.0
-9.6413118754e-09 1.13102126121521 38.0
-9.6413119444e-09 -3.46826887130737 39.0
-11.764605552 -15.0157032012939 40.0
-24.722855528 -26.6903896331787 41.0
5.264062295 4.85135936737061 42.0
-32.507321484 -32.9000129699707 43.0
95.346343424 91.6533737182617 44.0
95.346343423 93.4915771484375 45.0
7.2518008897 6.6089391708374 46.0
2.6318067649 2.56226420402527 47.0
172.92832752 163.716827392578 48.0
166.99987347 159.916687011719 49.0
0 1.80834984779358 50.0
80.332961756 77.7092056274414 51.0
83.873112544 88.8013153076172 52.0
78.059056004 81.4548797607422 53.0
78.059056004 81.2985229492188 54.0
78.059056004 80.5078735351562 55.0
0 3.05376291275024 56.0
78.513777127 79.5195846557617 57.0
225.57126733 224.293655395508 58.0
252.76350758 248.963012695312 59.0
20.565073852 21.8464279174805 60.0
0 1.61009359359741 61.0
36.968321715 160.366714477539 0.0
36.968321715 163.302017211914 1.0
4.7442468957e-14 153.796630859375 2.0
7.3915904828 139.541427612305 3.0
204.05803751 -3.28587341308594 4.0
590.92071305 1.3857901096344 5.0
582.05463804 -60.6388969421387 6.0
571.80687349 75.6256408691406 7.0
528.591894373889 46.8555679321289 8.0
526.200687417322 22.7334861755371 9.0
526.38392379 91.9388122558594 10.0
524.66413002 77.0415725708008 11.0
519.25742636 81.7472839355469 12.0
93.641379642 79.3833465576172 13.0
50.240245805 82.3139343261719 14.0
13.600848744 -0.0155409574508667 15.0
0 -0.00416749715805054 16.0
0.0049798427585 3.35134220123291 17.0
0 57.2610206604004 18.0
0 -58.360221862793 19.0
0.0058198025145 -58.4182510375977 20.0
0 -59.963794708252 21.0
0.0057910890569 79.0705642700195 22.0
0 139.623809814453 23.0
0.0059149176932 112.224617004395 24.0
93.641379642 21.2510833740234 25.0
43.548168396 222.812515258789 26.0
-1.4460118017 220.299438476562 27.0
13.600848744 25.7935523986816 28.0
12.154836943 256.247619628906 29.0
-37.839371968 256.160675048828 30.0
-50.240245805 45.4642372131348 31.0
12.400873836 8.78755950927734 32.0
-50.093211246 20.1801090240479 33.0
5.5158310645 19.2806758880615 34.0
1.0007206917 21.1165599822998 35.0
-2.8415121526 8.71727466583252 36.0
0.12208759713 7.50342035293579 37.0
-3.8843438953 -2.09091186523438 38.0
-3.638521889 -1.33939743041992 39.0
-4.8637294528 -13.2849006652832 40.0
-2.1392946711 -24.6767520904541 41.0
-4.1358353588 3.91112995147705 42.0
0 -32.6285552978516 43.0
0 92.3550415039062 44.0
0 95.4485092163086 45.0
0 6.64146423339844 46.0
0 2.43197274208069 47.0
0 161.573913574219 48.0
15.15148833 141.619903564453 49.0
21.81879063 0.896890223026276 50.0
36.968321715 76.3170928955078 51.0
0.90271466161 91.0124435424805 52.0
17.443472465 81.4464111328125 53.0
0.83775624047 81.7278823852539 54.0
19.52484925 80.6322860717773 55.0
0.88545659744 3.77111339569092 56.0
-4.3653574652 78.8316421508789 57.0
-0.81921922947 218.256439208984 58.0
15.159491785 249.127014160156 59.0
0.7709054218 23.3009185791016 60.0
21.80882993 2.07745409011841 61.0
194.54488019 159.075927734375 0.0
194.54488019 167.64811706543 1.0
193.07913495 172.870574951172 2.0
188.29006523 184.538208007812 3.0
-3.4156646434e-11 1.07188034057617 4.0
-4.0391911542e-11 4.49218559265137 5.0
-71.5300739 -52.615966796875 6.0
80.461981986 76.9665756225586 7.0
58.448847751 56.5962257385254 8.0
22.013134226 5.20493221282959 9.0
81.628850304 86.8225860595703 10.0
68.37745927 78.4542083740234 11.0
81.66393832 77.2426223754883 12.0
81.66393832 80.2858047485352 13.0
81.66393832 83.1369323730469 14.0
0 -2.635422706604 15.0
0 -1.01033854484558 16.0
-5.9942699058e-11 7.24216365814209 17.0
57.911027946 59.1430130004883 18.0
-57.911027966 -59.2003288269043 19.0
-57.911027944 -56.9994735717773 20.0
-57.911027957 -58.2809448242188 21.0
79.327813461 76.30908203125 22.0
145.64584376 142.290679931641 23.0
111.92990402 112.969482421875 24.0
21.755866618 20.6269683837891 25.0
224.97365723 228.639572143555 26.0
224.97365723 221.946014404297 27.0
24.236855424 24.835916519165 28.0
249.21051267 251.029800415039 29.0
249.21051267 255.203216552734 30.0
46.760932419 43.572639465332 31.0
9.4650710176 8.89524078369141 32.0
21.416785495 15.0644464492798 33.0
21.416785495 4.17405462265015 34.0
21.416785494 20.1440906524658 35.0
10.043361444 4.83081340789795 36.0
8.754619378 6.76093864440918 37.0
0 -7.24033641815186 38.0
0 1.41309213638306 39.0
-11.210280846 -19.4171848297119 40.0
-23.224039804 -19.4423809051514 41.0
4.8599470037 5.6995267868042 42.0
-30.554353162 -35.7832946777344 43.0
89.362904497 86.7814865112305 44.0
89.36290451 95.391716003418 45.0
7.0625740537 5.25438022613525 46.0
2.6841415901 2.62718939781189 47.0
194.5448802 161.648590087891 48.0
188.29006524 176.408309936523 49.0
0 3.36089086532593 50.0
80.461981996 76.0813598632812 51.0
81.628850314 90.2776260375977 52.0
81.66393833 80.2112350463867 53.0
81.66393833 78.5968170166016 54.0
81.66393833 80.4939270019531 55.0
0 1.24942493438721 56.0
79.327813471 81.2722854614258 57.0
224.97365724 220.016830444336 58.0
249.21051268 249.139907836914 59.0
21.416785504 23.1984577178955 60.0
0 3.21474361419678 61.0
156.041740118704 164.227416992188 0.0
156.041740118704 164.024047851562 1.0
154.86377368 157.737014770508 2.0
150.07470392 152.013427734375 3.0
0 -1.46193861961365 4.0
-5.13944870000001e-09 0.00509363412857056 5.0
-59.0942599620726 -62.1293601989746 6.0
70.0105627637541 74.2977752685547 7.0
47.8191686267541 55.9872169494629 8.0
22.1913941367541 21.1517658233643 9.0
85.7399816587541 88.411750793457 10.0
73.4435332487541 75.4782943725586 11.0
79.0318684497541 82.3558502197266 12.0
79.0318684497541 81.7325057983398 13.0
79.0318684497541 82.3169860839844 14.0
4.6386986821e-09 -1.097323179245 15.0
2.7687428355e-09 -0.301229178905487 16.0
1.059187854e-08 0.950678050518036 17.0
60.5040296949568 58.7327537536621 18.0
-60.5040296486311 -59.5685386657715 19.0
-60.5040296306311 -59.1882667541504 20.0
-60.5040296066311 -62.0662460327148 21.0
82.0376625801227 78.7727203369141 22.0
149.106724886957 141.720809936523 23.0
113.003166401196 113.464294433594 24.0
23.215941281196 19.901237487793 25.0
231.144387467409 224.130065917969 26.0
231.144387467409 224.189666748047 27.0
25.0531014924086 24.041374206543 28.0
256.197488957409 257.387237548828 29.0
256.197488957409 252.98908996582 30.0
46.7530805726177 45.1330757141113 31.0
9.30285158507924 8.79094219207764 32.0
21.5336329732178 20.7552623748779 33.0
21.5336329732178 19.3896923065186 34.0
21.5336330165704 20.7399482727051 35.0
9.61546373964747 7.12272024154663 36.0
8.32611508954747 7.52221870422363 37.0
-5.93317749999999e-09 -1.12284302711487 38.0
-5.93317749999999e-09 -3.0559287071228 39.0
-11.8106011533156 -11.2067155838013 40.0
-25.2170438946546 -27.637279510498 41.0
5.12244986286174 2.83657360076904 42.0
-32.8387053674617 -31.3802356719971 43.0
101.50071291868 93.8650741577148 44.0
101.500713071462 92.6988983154297 45.0
6.6091380735435 4.83456993103027 46.0
2.4936369324435 1.68995547294617 47.0
156.04174012 162.648071289062 48.0
150.07470392 147.097885131836 49.0
0 -0.695102751255035 50.0
70.010562764 74.3794021606445 51.0
85.739981658 89.6062316894531 52.0
79.031868449 81.1115798950195 53.0
79.031868449 81.1199951171875 54.0
79.031868449 80.1611328125 55.0
1.059187854e-08 0.709782898426056 56.0
82.03766258 80.305419921875 57.0
231.14438746 221.11083984375 58.0
256.19748896 247.962997436523 59.0
21.533633016 20.1235504150391 60.0
0 2.21874952316284 61.0
150.31740659 163.973083496094 0.0
150.31740659 166.144378662109 1.0
149.17586712 154.478424072266 2.0
144.3867974 149.130065917969 3.0
0 2.26004672050476 4.0
-8.5030818286e-10 2.15266847610474 5.0
-61.140284149 -55.0037689208984 6.0
72.70335599 79.3717422485352 7.0
54.720363685 47.7660713195801 8.0
17.982992299 23.3718814849854 9.0
88.181408553 87.3057403564453 10.0
75.282658081 76.9413681030273 11.0
80.857151728 82.3808135986328 12.0
80.857151728 79.8690185546875 13.0
80.857151728 81.5380020141602 14.0
0 -2.51617741584778 15.0
0 -0.729720413684845 16.0
0 0.605931460857391 17.0
60.317631118 58.0951118469238 18.0
-60.317631132 -60.4323348999023 19.0
-60.31763111 -58.4515495300293 20.0
-60.317631122 -58.8321876525879 21.0
82.102500529 77.5609512329102 22.0
148.38733839 144.538330078125 23.0
114.84676461 109.781112670898 24.0
22.338930785 18.4482097625732 25.0
230.48983893 225.580032348633 26.0
230.48983893 225.569686889648 27.0
25.647239124 26.7828483581543 28.0
256.13707806 253.130508422852 29.0
256.13707806 252.883102416992 30.0
46.497367864 46.5226287841797 31.0
9.3663425153 8.53235054016113 32.0
21.784869403 19.2298927307129 33.0
21.784869403 20.3403854370117 34.0
21.784869402 21.4746532440186 35.0
9.4097445505 8.63479709625244 36.0
8.0900347315 6.70832967758179 37.0
-2.873877607e-09 -1.91395711898804 38.0
-2.873877607e-09 -1.12903666496277 39.0
-11.319439517 -14.8916072845459 40.0
-23.649136006 -24.7447204589844 41.0
5.2264805986 3.93278050422668 42.0
-31.381344977 -32.857982635498 43.0
100.4061621 94.2885894775391 44.0
100.40616211 94.8221130371094 45.0
6.9364250907 5.55480766296387 46.0
2.6332686323 2.15771412849426 47.0
150.3174066 164.663589477539 48.0
144.38679741 147.081817626953 49.0
0 -2.87885665893555 50.0
72.703355997 74.0565414428711 51.0
88.18140856 90.7114791870117 52.0
80.857151734 80.6112670898438 53.0
80.857151734 80.7744598388672 54.0
80.857151734 80.0585021972656 55.0
0 4.10533475875854 56.0
82.102500538 80.6623992919922 57.0
230.48983894 221.549652099609 58.0
256.13707807 249.496765136719 59.0
21.784869411 22.802661895752 60.0
0 1.01528573036194 61.0
147.982033057583 157.456283569336 0.0
147.982033057583 151.960418701172 1.0
146.808413813287 153.322540283203 2.0
142.01934397 150.531326293945 3.0
-1.55484310406e-09 -0.626340210437775 4.0
-1.55484309999997e-09 -2.83180117607117 5.0
-64.9693040339936 -64.3129272460938 6.0
77.6547829144777 74.1870651245117 7.0
61.2786926634778 55.4030647277832 8.0
16.3760902514777 7.20984888076782 9.0
97.5261141374777 89.7391510009766 10.0
83.6472874924777 79.6450042724609 11.0
85.2981411144777 83.8642730712891 12.0
85.2981411144777 76.4624557495117 13.0
85.2981411144777 81.5436782836914 14.0
2.3328021736e-08 -2.03496789932251 15.0
2.3328021735e-08 -1.6052393913269 16.0
5.464350236e-08 -2.72059202194214 17.0
56.2518018559467 59.0783615112305 18.0
-56.2518018797166 -57.0891418457031 19.0
-56.2518018707166 -59.2854194641113 20.0
-56.2518017707166 -58.115608215332 21.0
77.1875821073724 73.3798065185547 22.0
153.454771802806 145.008895874023 23.0
115.220596133508 115.750228881836 24.0
21.7005584165075 20.8819332122803 25.0
230.642353915167 228.381210327148 26.0
230.642353915167 229.708221435547 27.0
26.8883420391666 21.3647422790527 28.0
257.530695955167 253.23681640625 29.0
257.530695955167 255.336776733398 30.0
46.7385281475469 45.537841796875 31.0
9.28724497493877 8.4355525970459 32.0
20.9357803369809 22.6376686096191 33.0
20.9357803369809 19.0790691375732 34.0
20.935780467613 20.4264030456543 35.0
8.86054244871795 8.39873600006104 36.0
7.60121916461794 6.85336446762085 37.0
2.62465480000001e-09 1.1959273815155 38.0
2.62465480000001e-09 -1.17553234100342 39.0
-11.6034306368583 -12.5643882751465 40.0
-24.5599220844463 -26.7411251068115 41.0
4.75986482786468 1.92820739746094 42.0
-31.7734218810648 -31.6524333953857 43.0
89.4696777832616 86.4044494628906 44.0
89.4696781210649 91.9600372314453 45.0
7.05809588635643 4.443039894104 46.0
2.46335994485643 2.26466846466064 47.0
147.98203306 156.097946166992 48.0
142.01934397 152.830352783203 49.0
0 -3.76001691818237 50.0
77.654782914 76.0241928100586 51.0
97.526114137 92.0755157470703 52.0
85.298141114 81.5651702880859 53.0
85.298141114 80.2131118774414 54.0
85.298141114 79.9203643798828 55.0
5.4643502359e-08 3.07315587997437 56.0
77.187582108 79.4883346557617 57.0
230.64235391 222.583419799805 58.0
257.53069595 249.491134643555 59.0
20.935780468 19.6116352081299 60.0
2.6246548239e-09 3.64096260070801 61.0
178.1092976 154.399490356445 0.0
178.1092976 161.102279663086 1.0
176.7463671 176.901947021484 2.0
171.95729738 163.166961669922 3.0
0 3.73396444320679 4.0
-6.4967660651e-12 7.82714748382568 5.0
-58.696921404 -67.9038314819336 6.0
70.527717722 84.6482009887695 7.0
47.80691563 57.1935081481934 8.0
22.720802082 28.8980674743652 9.0
88.972143521 88.0253372192383 10.0
76.376668321 78.5444717407227 11.0
80.773868337 85.5110931396484 12.0
80.773868337 81.4156036376953 13.0
80.773868337 80.6563262939453 14.0
0 0.776227235794067 15.0
0 4.10664844512939 16.0
-1.2401527814e-11 5.43900346755981 17.0
58.290858783 58.123363494873 18.0
-58.290858803 -58.1011543273926 19.0
-58.290858781 -50.241039276123 20.0
-58.290858793 -56.4194526672363 21.0
80.216955727 72.9683837890625 22.0
142.92577388 131.649353027344 23.0
108.6227484 108.848281860352 24.0
22.932853154 23.8715229034424 25.0
223.14272962 227.951995849609 26.0
223.14272962 225.83283996582 27.0
28.368392321 25.9905891418457 28.0
251.51112195 254.169952392578 29.0
251.51112195 257.625946044922 30.0
46.893961157 47.6603660583496 31.0
9.4018906555 8.22420215606689 32.0
21.926096924 27.4454135894775 33.0
21.926096924 6.44482612609863 34.0
21.926096924 23.4320106506348 35.0
9.5515575796 -0.555282294750214 36.0
8.287181272 9.2640905380249 37.0
-1.9877181649e-11 7.77374267578125 38.0
-1.9877181649e-11 3.07543063163757 39.0
-12.287432724 -29.2069129943848 40.0
-25.343215388 -28.3953132629395 41.0
5.0851776519 3.63574552536011 42.0
-33.04764115 -32.0890235900879 43.0
94.710837291 93.5221405029297 44.0
94.710837304 94.8918762207031 45.0
6.8351185337 8.53389549255371 46.0
2.489241411 1.75495767593384 47.0
178.10929761 162.948822021484 48.0
171.95729739 163.839172363281 49.0
0 -0.350852072238922 50.0
70.527717732 74.4366455078125 51.0
88.972143531 90.0881881713867 52.0
80.773868347 80.4677429199219 53.0
80.773868347 79.674430847168 54.0
80.773868347 80.6995391845703 55.0
0 4.51551866531372 56.0
80.216955737 80.7464599609375 57.0
223.14272963 221.236663818359 58.0
251.51112196 254.375946044922 59.0
21.926096934 26.0170154571533 60.0
0 6.70063018798828 61.0
166.43504996 155.809127807617 0.0
166.43504996 162.220916748047 1.0
165.50339172 166.473892211914 2.0
160.03875653 174.172424316406 3.0
2.9648315608 -3.60944652557373 4.0
2.9648315608 5.7509503364563 5.0
-65.686833793 -43.0611000061035 6.0
75.217444867 82.0187149047852 7.0
45.874195116 52.9012451171875 8.0
29.343249741 21.575345993042 9.0
93.820107532 105.879104614258 10.0
80.241339409 80.06005859375 11.0
86.304266506 78.0165176391602 12.0
86.304266506 78.2705841064453 13.0
86.304266506 72.2389602661133 14.0
0 2.72473096847534 15.0
0 1.80607008934021 16.0
0 -3.73666858673096 17.0
55.679370272 55.4431419372559 18.0
-55.679370292 -64.6455917358398 19.0
-55.67937027 -52.0256271362305 20.0
-55.679370283 -55.0497093200684 21.0
77.073265076 76.9359436035156 22.0
129.45181644 145.590362548828 23.0
97.800097469 117.895927429199 24.0
21.40534584 18.0543022155762 25.0
206.52508152 226.631591796875 26.0
206.52508152 227.959259033203 27.0
26.655645479 27.2698554992676 28.0
233.18072701 255.445938110352 29.0
233.18072701 246.521682739258 30.0
46.380298789 53.5155982971191 31.0
9.332226119 8.28149509429932 32.0
21.393894783 26.0273628234863 33.0
21.393894783 12.9887676239014 34.0
21.393894783 17.6148071289062 35.0
8.5748444431 10.4764366149902 36.0
7.1656481692 6.29544448852539 37.0
0 -3.85597372055054 38.0
0 1.17383623123169 39.0
-11.986129813 -15.682240486145 40.0
-25.089079815 -27.2865104675293 41.0
5.5624522815 3.50567007064819 42.0
-33.394054469 -30.9954528808594 43.0
90.892334825 82.2317352294922 44.0
90.892334838 103.775894165039 45.0
5.8209091389 10.0716638565063 46.0
2.2535866089 2.74248719215393 47.0
166.43504997 165.067337036133 48.0
160.03875654 160.942153930664 49.0
2.9648315708 -1.22879528999329 50.0
75.217444877 73.320686340332 51.0
93.820107542 74.3947906494141 52.0
86.304266516 77.4001770019531 53.0
86.304266516 81.3580627441406 54.0
86.304266516 81.7724838256836 55.0
0 4.24825000762939 56.0
77.073265086 81.3222579956055 57.0
206.52508153 220.293426513672 58.0
233.18072702 253.527359008789 59.0
21.393894793 19.5257892608643 60.0
0 3.46240949630737 61.0
176.39373395592 163.735916137695 0.0
176.39373395592 164.427322387695 1.0
175.3559948 159.3779296875 2.0
170.56692507 160.412384033203 3.0
0 -0.949622213840485 4.0
-8.862281502e-09 3.83186149597168 5.0
-54.2307042540748 -61.5949172973633 6.0
63.3659285132617 80.8567276000977 7.0
48.2094616602617 51.8532028198242 8.0
15.1564668522617 27.6419563293457 9.0
97.7643859022618 84.108024597168 10.0
84.5028619762617 74.9425811767578 11.0
81.8250055012617 80.9078598022461 12.0
81.8250055012617 83.4761276245117 13.0
81.8250055012617 77.005241394043 14.0
0 -0.565450251102448 15.0
0 -1.72565269470215 16.0
-6.838747084e-09 -1.4163236618042 17.0
59.5677893492337 56.8830528259277 18.0
-59.5677893593926 -59.9420433044434 19.0
-59.5677893383926 -60.9268989562988 20.0
-59.5677893443926 -59.1392936706543 21.0
82.0311396137194 77.1762847900391 22.0
147.465342467234 143.561126708984 23.0
108.909708551542 108.65380859375 24.0
22.2983464765421 20.535530090332 25.0
229.49648208184 225.765533447266 26.0
229.49648208184 225.582183837891 27.0
28.3777771378405 24.5713500976562 28.0
257.87425922184 249.941040039062 29.0
257.87425922184 254.323287963867 30.0
46.8712742688432 45.7575416564941 31.0
9.3418054421 7.96539402008057 32.0
22.4633502692244 16.7593975067139 33.0
22.4633502692244 15.5215864181519 34.0
22.4633502756075 19.3990287780762 35.0
10.199780717 12.1584539413452 36.0
8.9492779877 7.02440738677979 37.0
-9.400370781e-09 -3.98684120178223 38.0
-9.400370781e-09 -4.7182822227478 39.0
-11.3551464796963 -9.29204654693604 40.0
-23.9434004627772 -26.6096115112305 41.0
5.34257862171812 1.26965880393982 42.0
-31.5385366282181 -33.1806373596191 43.0
86.8111861602181 89.4104385375977 44.0
86.8111861104062 90.4108352661133 45.0
7.1676575759 5.05861902236938 46.0
2.3111890102 1.67849206924438 47.0
176.39373395 171.864883422852 48.0
170.56692508 161.476348876953 49.0
0 0.105547428131104 50.0
63.365928513 76.3408050537109 51.0
97.764385902 88.9855194091797 52.0
81.825005501 79.6799926757812 53.0
81.825005501 81.2104873657227 54.0
81.825005501 78.9302520751953 55.0
0 -1.41538953781128 56.0
82.031139614 77.9886703491211 57.0
229.49648208 223.803741455078 58.0
257.87425922 247.524917602539 59.0
22.463350276 22.0135345458984 60.0
0 2.19289541244507 61.0
137.17687742 143.006698608398 0.0
137.17687742 143.024017333984 1.0
135.86355256 143.584884643555 2.0
131.0744823 135.858932495117 3.0
0 -2.21669626235962 4.0
0 0.342764794826508 5.0
-59.42861149 -59.2475357055664 6.0
69.705106679 70.1774826049805 7.0
65.43029471 62.4528312683105 8.0
4.2748119615 8.48783111572266 9.0
78.669161727 80.2065200805664 10.0
65.229460649 71.2585144042969 11.0
77.500548395 80.8047180175781 12.0
77.500548395 77.9000701904297 13.0
77.500548395 76.9198608398438 14.0
0 -0.64198762178421 15.0
0 0.28728449344635 16.0
0 -3.15485000610352 17.0
59.754945382 58.5015258789062 18.0
-59.754945394 -59.4383087158203 19.0
-59.754945372 -60.0649642944336 20.0
-59.754945384 -61.8801422119141 21.0
80.656571382 77.8813858032227 22.0
140.87903669 140.394561767578 23.0
106.41719675 107.357681274414 24.0
22.90053999 20.4083709716797 25.0
221.53560808 223.594223022461 26.0
221.53560808 221.987747192383 27.0
27.282645659 25.3551406860352 28.0
248.81825375 253.574676513672 29.0
248.81825375 252.408157348633 30.0
47.033013387 47.2299919128418 31.0
9.3131031344 9.18030452728271 32.0
20.901625993 20.4819431304932 33.0
20.901625993 21.1129970550537 34.0
20.901625993 20.8195400238037 35.0
9.3083381265 8.28715133666992 36.0
8.044767894 8.38281917572021 37.0
-2.5262790395e-09 -0.804296553134918 38.0
-2.5262785966e-09 0.113371789455414 39.0
-12.387792857 -14.5510091781616 40.0
-23.81592281 -27.5234241485596 41.0
4.9278776077 3.5162615776062 42.0
-31.108913619 -34.7394485473633 43.0
92.052766479 91.3064651489258 44.0
92.05276649 90.2343292236328 45.0
6.7273095346 3.00315618515015 46.0
2.6568330431 1.83220052719116 47.0
137.17687743 136.677871704102 48.0
131.07448231 127.691390991211 49.0
0 3.52269697189331 50.0
69.705106686 70.9855194091797 51.0
78.669161734 84.2529830932617 52.0
77.500548402 81.0479965209961 53.0
77.500548402 80.9212951660156 54.0
77.500548402 79.4541549682617 55.0
0 -0.0822644829750061 56.0
80.656571391 80.3950500488281 57.0
221.53560809 218.26057434082 58.0
248.81825375 246.469757080078 59.0
20.901626002 19.9398994445801 60.0
0 2.89454770088196 61.0
147.59555202 164.645812988281 0.0
147.59555202 158.001007080078 1.0
146.39025614 157.817626953125 2.0
141.60118641 139.932540893555 3.0
0 7.75734853744507 4.0
0 0.429560720920563 5.0
-64.753654227 -59.9145317077637 6.0
76.026143549 80.3137741088867 7.0
46.150106242 45.6958045959473 8.0
29.876037297 24.1751804351807 9.0
84.495648475 96.9319839477539 10.0
72.638321873 78.2974090576172 11.0
79.558471733 87.3150787353516 12.0
79.558471733 86.6168060302734 13.0
79.558471733 76.4477691650391 14.0
0 -0.0214637517929077 15.0
0 -0.23643547296524 16.0
0 0.650009214878082 17.0
60.723538287 60.4936294555664 18.0
-60.723538297 -58.8111991882324 19.0
-60.723538275 -53.197639465332 20.0
-60.723538288 -56.3383140563965 21.0
82.373400961 77.549674987793 22.0
139.61453948 147.057968139648 23.0
106.98216904 112.452606201172 24.0
21.386479709 19.2137184143066 25.0
221.98794045 222.956954956055 26.0
221.98794045 226.307723999023 27.0
28.780443619 25.3258571624756 28.0
250.76838408 248.267578125 29.0
250.76838408 249.997619628906 30.0
47.002454833 49.5958099365234 31.0
9.285912751 9.5930347442627 32.0
21.649862664 18.1959705352783 33.0
21.649862664 23.7815761566162 34.0
21.649862664 22.2283973693848 35.0
9.9459293968 6.6279730796814 36.0
8.7606888039 6.35870599746704 37.0
-2.6040799383e-10 -1.76334714889526 38.0
-2.6040792422e-10 -3.19870805740356 39.0
-11.635736569 -16.9976825714111 40.0
-25.22907137 -26.5920715332031 41.0
4.8900668851 5.66343832015991 42.0
-32.52567893 -33.843620300293 43.0
90.165309755 87.4204483032227 44.0
90.165309768 90.1847076416016 45.0
6.9789841328 4.81195688247681 46.0
2.3836761511 2.24263024330139 47.0
147.59555203 161.683502197266 48.0
141.60118642 147.434127807617 49.0
0 7.5224461555481 50.0
76.026143559 70.6722412109375 51.0
84.495648485 86.4512710571289 52.0
79.558471743 80.1537017822266 53.0
79.558471743 82.3549194335938 54.0
79.558471743 79.8503494262695 55.0
0 1.61147546768188 56.0
82.373400971 80.1528396606445 57.0
221.98794046 223.339050292969 58.0
250.76838408 244.4599609375 59.0
21.649862673 17.5413990020752 60.0
0 1.13901662826538 61.0
170.42828661 159.065307617188 0.0
170.42828661 159.948745727539 1.0
169.20185224 158.352661132812 2.0
164.41278252 155.777420043945 3.0
0 -1.45657134056091 4.0
-1.9157067912e-09 1.28356671333313 5.0
-67.875488659 -60.9502754211426 6.0
79.144750301 74.5976104736328 7.0
53.384642381 50.3271217346191 8.0
25.760107916 24.9055614471436 9.0
88.714336233 90.6856384277344 10.0
76.133213146 77.1031036376953 11.0
82.239560211 85.1240615844727 12.0
82.239560211 79.4011535644531 13.0
82.239560211 77.8806533813477 14.0
0 -1.84207630157471 15.0
0 -0.349497377872467 16.0
0 2.75881457328796 17.0
57.964099941 57.6090278625488 18.0
-57.964099956 -59.0741500854492 19.0
-57.964099934 -59.7240600585938 20.0
-57.964099946 -58.4843597412109 21.0
79.363844231 77.1597671508789 22.0
145.55931512 141.331832885742 23.0
111.00269654 111.06600189209 24.0
21.546285825 22.2725505828857 25.0
224.92315937 228.901947021484 26.0
224.92315937 224.171737670898 27.0
27.714883336 26.1833782196045 28.0
252.6380427 253.83757019043 29.0
252.6380427 255.720504760742 30.0
46.717884608 45.622013092041 31.0
9.2988496224 9.12876415252686 32.0
21.399744289 19.8968029022217 33.0
21.399744289 23.0106754302979 34.0
21.399744287 19.9864082336426 35.0
9.3889695642 7.702073097229 36.0
8.1673655693 7.0573787689209 37.0
-6.9849843898e-09 -2.71027994155884 38.0
-6.9849843355e-09 -1.06142449378967 39.0
-11.629401297 -15.39759349823 40.0
-24.492600629 -25.065788269043 41.0
4.9419372357 2.96994924545288 42.0
-32.143726257 -32.7842292785645 43.0
90.970316464 89.2125396728516 44.0
90.970316467 91.3017120361328 45.0
6.73937146 6.85662364959717 46.0
2.4545655073 2.19619560241699 47.0
170.42828662 164.859970092773 48.0
164.41278253 156.40771484375 49.0
0 -0.789117276668549 50.0
79.144750304 78.7460250854492 51.0
88.714336236 91.9713897705078 52.0
82.239560214 81.7778244018555 53.0
82.239560214 81.6386566162109 54.0
82.239560214 80.1250991821289 55.0
0 3.35901856422424 56.0
79.36384424 80.0875778198242 57.0
224.92315937 223.353530883789 58.0
252.63804271 248.181655883789 59.0
21.399744294 19.6026935577393 60.0
0 0.969676554203033 61.0
152.372681506956 166.943725585938 0.0
152.372681506956 166.741821289062 1.0
151.17254784 163.429672241211 2.0
146.38347811 147.562866210938 3.0
0 3.12316727638245 4.0
-8.0469011407e-09 6.74087285995483 5.0
-56.572494264 -46.3272323608398 6.0
67.609214459391 80.1870880126953 7.0
52.670678250391 42.0725860595703 8.0
14.938536209391 21.4882640838623 9.0
84.672193511391 92.1217422485352 10.0
72.352619779391 72.7977600097656 11.0
86.217727744391 91.5621871948242 12.0
86.217727744391 79.8793792724609 13.0
86.217727744391 80.9972839355469 14.0
0 -1.26860570907593 15.0
0 0.224732100963593 16.0
-7.295415606e-09 -1.23044967651367 17.0
58.8202653856678 58.1891021728516 18.0
-58.8202653990865 -59.1922225952148 19.0
-58.8202653770865 -60.4987640380859 20.0
-58.8202653860865 -58.2667083740234 21.0
80.56566502 80.5951385498047 22.0
152.388479447668 141.142120361328 23.0
119.568199974585 111.809814453125 24.0
21.6281370715848 20.7619113922119 25.0
232.954144473913 225.943313598633 26.0
232.954144473913 221.749908447266 27.0
26.0855591859127 28.7114238739014 28.0
259.039703663913 258.899932861328 29.0
259.039703663913 257.357818603516 30.0
46.306567672 46.1879653930664 31.0
9.2599501791 8.53433513641357 32.0
21.7453996376208 18.3843078613281 33.0
21.7453996376208 17.9027290344238 34.0
21.745399639 21.1620769500732 35.0
9.6147188781 8.18524551391602 36.0
8.3507369678 5.24706077575684 37.0
-9.645152647e-09 2.15929889678955 38.0
-9.645152648e-09 2.11360454559326 39.0
-11.6407826100432 -13.0579500198364 40.0
-25.2192415895039 -28.023380279541 41.0
5.48640030905915 2.91068553924561 42.0
-33.5139239531591 -33.4050140380859 43.0
88.9523876381592 91.9756164550781 44.0
88.9523876166797 93.0760192871094 45.0
6.4163568273 12.7913999557495 46.0
2.7002071852 3.21081113815308 47.0
152.37268151 164.343933105469 48.0
146.38347812 158.315231323242 49.0
0 1.74542450904846 50.0
67.609214459 76.3130416870117 51.0
84.672193511 86.621826171875 52.0
86.217727744 81.1737976074219 53.0
86.217727744 80.4597625732422 54.0
86.217727744 79.7098083496094 55.0
0 1.58753418922424 56.0
80.565665024 80.3560791015625 57.0
232.95414448 217.963241577148 58.0
259.03970366 251.920654296875 59.0
21.745399642 22.1938514709473 60.0
0 2.37082457542419 61.0
147.982033057583 157.456329345703 0.0
147.982033057583 151.962692260742 1.0
146.808413813287 153.341369628906 2.0
142.01934397 150.663513183594 3.0
-1.55484310406e-09 -0.642547428607941 4.0
-1.55484309999997e-09 -2.83744192123413 5.0
-64.9693040339936 -64.3185882568359 6.0
77.6547829144777 74.1871490478516 7.0
61.2786926634778 55.4030647277832 8.0
16.3760902514777 7.20984888076782 9.0
97.5261141374777 89.7391510009766 10.0
83.6472874924777 79.6450042724609 11.0
85.2981411144777 83.8642730712891 12.0
85.2981411144777 76.4624557495117 13.0
85.2981411144777 81.5436782836914 14.0
2.3328021736e-08 -2.03496789932251 15.0
2.3328021735e-08 -1.6052393913269 16.0
5.464350236e-08 -2.72059202194214 17.0
56.2518018559467 59.0783615112305 18.0
-56.2518018797166 -57.0891418457031 19.0
-56.2518018707166 -59.2854194641113 20.0
-56.2518017707166 -58.115608215332 21.0
77.1875821073724 73.3798065185547 22.0
153.454771802806 145.008895874023 23.0
115.220596133508 115.750228881836 24.0
21.7005584165075 20.8819332122803 25.0
230.642353915167 228.381210327148 26.0
230.642353915167 229.708221435547 27.0
26.8883420391666 21.3647422790527 28.0
257.530695955167 253.23681640625 29.0
257.530695955167 255.336776733398 30.0
46.7385281475469 45.537841796875 31.0
9.28724497493877 8.4355525970459 32.0
20.9357803369809 22.6376686096191 33.0
20.9357803369809 19.0790691375732 34.0
20.935780467613 20.4264030456543 35.0
8.86054244871795 8.39873600006104 36.0
7.60121916461794 6.85336446762085 37.0
2.62465480000001e-09 1.1959273815155 38.0
2.62465480000001e-09 -1.17553234100342 39.0
-11.6034306368583 -12.5643882751465 40.0
-24.5599220844463 -26.7411251068115 41.0
4.75986482786468 1.92820739746094 42.0
-31.7734218810648 -31.6524333953857 43.0
89.4696777832616 86.4044494628906 44.0
89.4696781210649 91.9600372314453 45.0
7.05809588635643 4.443039894104 46.0
2.46335994485643 2.26466846466064 47.0
147.98203306 156.097015380859 48.0
142.01934397 152.941223144531 49.0
0 -3.75158500671387 50.0
77.654782914 76.0238265991211 51.0
97.526114137 92.0755157470703 52.0
85.298141114 81.5651702880859 53.0
85.298141114 80.2131118774414 54.0
85.298141114 79.9203643798828 55.0
5.4643502359e-08 3.07315587997437 56.0
77.187582108 79.4883346557617 57.0
230.64235391 222.583419799805 58.0
257.53069595 249.491134643555 59.0
20.935780468 19.6116352081299 60.0
2.6246548239e-09 3.64096260070801 61.0
160.35063881316 167.894668579102 0.0
160.35063881316 170.422882080078 1.0
159.091024497706 171.265716552734 2.0
153.62638895 147.936340332031 3.0
4.3736320528 12.9847917556763 4.0
4.37363205541732 15.3899278640747 5.0
-66.9858621755173 -48.4729957580566 6.0
79.5663925708106 83.5473022460938 7.0
44.2486092298106 58.347225189209 8.0
35.3177833408106 22.5283107757568 9.0
88.9020717958106 88.4344940185547 10.0
76.1019445608106 79.3463745117188 11.0
85.6025560708106 83.1701049804688 12.0
85.6025560708106 75.6027145385742 13.0
85.6025560708106 77.0297622680664 14.0
3.5917564607e-08 0.888904213905334 15.0
3.5917564608e-08 -0.089235782623291 16.0
7.51975143999999e-08 -0.00610464811325073 17.0
63.1678763288058 56.9941749572754 18.0
-63.1678760782159 -59.3318901062012 19.0
-63.1678760732159 -59.9870834350586 20.0
-63.1678759362159 -54.3889389038086 21.0
83.1373687300299 76.5203704833984 22.0
136.675409366806 136.036178588867 23.0
104.930735791007 112.107666015625 24.0
21.5234025330072 22.1469116210938 25.0
219.81277809632 219.089492797852 26.0
219.81277809632 216.169067382812 27.0
24.82958511732 24.5978698730469 28.0
244.64236321632 253.917907714844 29.0
244.64236321632 254.26139831543 30.0
46.6761279145856 46.297492980957 31.0
9.35352104745502 8.80612564086914 32.0
19.969492793552 20.5294246673584 33.0
19.969492793552 19.7413063049316 34.0
19.969492968314 23.0593090057373 35.0
8.70648974130307 11.0390787124634 36.0
7.47673157460307 4.53157520294189 37.0
6.73553879999999e-09 -7.89872550964355 38.0
6.73553879999999e-09 -2.05979943275452 39.0
-11.583622860108 -20.0420265197754 40.0
-24.1605542933444 -23.3095092773438 41.0
4.68326180110882 9.88284397125244 42.0
-31.1458389500088 -35.9146423339844 43.0
99.1332674240089 94.5591201782227 44.0
99.1332662917558 94.3307189941406 45.0
7.09816358145277 3.68941688537598 46.0
2.76626398315276 2.0198974609375 47.0
160.35063881 170.955535888672 48.0
153.62638895 162.811157226562 49.0
4.3736320554 2.18092799186707 50.0
79.566392571 73.5864410400391 51.0
88.902071796 84.9199981689453 52.0
85.602556071 78.7710189819336 53.0
85.602556071 79.8957061767578 54.0
85.602556071 78.9692687988281 55.0
7.519751443e-08 4.61810159683228 56.0
83.13736873 79.282585144043 57.0
219.81277809 212.412750244141 58.0
244.64236321 251.319396972656 59.0
19.969492968 21.6904354095459 60.0
6.7355388327e-09 3.6285126209259 61.0
189.25750279 163.488204956055 0.0
189.25750279 166.181442260742 1.0
187.82716165 170.539947509766 2.0
183.03809161 206.850860595703 3.0
0 0.413111627101898 4.0
0 2.73274230957031 5.0
-62.399311056 -65.2886352539062 6.0
72.921458875 75.1139831542969 7.0
43.828508234 45.2858734130859 8.0
29.092950632 9.50836563110352 9.0
89.028145945 84.2620391845703 10.0
75.60611941 67.3173065185547 11.0
84.614631517 80.2053298950195 12.0
84.614631517 83.298225402832 13.0
84.614631517 81.3337936401367 14.0
0 1.01181495189667 15.0
0 -0.0452842712402344 16.0
0 0.899537801742554 17.0
58.141011275 58.8525886535645 18.0
-58.141011287 -58.9113655090332 19.0
-58.141011265 -58.6760482788086 20.0
-58.141011277 -58.0465393066406 21.0
78.444789496 76.6932144165039 22.0
132.21339057 144.883392333984 23.0
100.36847201 118.364273071289 24.0
20.717118768 20.1690559387207 25.0
210.65818008 224.940719604492 26.0
210.65818008 214.490280151367 27.0
26.210370561 32.077018737793 28.0
236.86855065 244.829208374023 29.0
236.86855065 230.620697021484 30.0
46.54962023 42.9310188293457 31.0
9.3526201389 9.26723289489746 32.0
20.303778211 17.4900093078613 33.0
20.303778211 17.8375415802002 34.0
20.303778211 18.9083271026611 35.0
8.7166495937 8.694899559021 36.0
7.3770673439 6.62734365463257 37.0
-8.5534957676e-10 -1.19837975502014 38.0
-8.5534957156e-10 -2.77520680427551 39.0
-11.957612527 -17.1162509918213 40.0
-25.701691042 -28.5237674713135 41.0
4.9242841396 5.85122299194336 42.0
-33.153156294 -33.4081993103027 43.0
86.373748697 96.4817657470703 44.0
86.37374871 90.4525756835938 45.0
7.2853312272 3.52993106842041 46.0
2.7162990298 2.41360569000244 47.0
189.2575028 164.891357421875 48.0
183.03809162 171.884796142578 49.0
0 -5.91379880905151 50.0
72.921458884 74.6535415649414 51.0
89.028145954 80.9882125854492 52.0
84.614631526 80.2339401245117 53.0
84.614631526 77.0661163330078 54.0
84.614631526 79.2052688598633 55.0
0 1.444899559021 56.0
78.444789506 79.1247863769531 57.0
210.65818009 217.39045715332 58.0
236.86855066 240.646514892578 59.0
20.30377822 22.639627456665 60.0
0 1.51905465126038 61.0
179.87042804 163.428466796875 0.0
179.87042804 164.804565429688 1.0
178.59360446 165.164596557617 2.0
173.80453474 166.622161865234 3.0
0 -4.75371932983398 4.0
0 8.1449031829834 5.0
-60.81057847 -59.9308395385742 6.0
72.205216369 73.5568466186523 7.0
48.182121709 52.6439170837402 8.0
24.023094651 18.2233257293701 9.0
90.954267683 86.9186859130859 10.0
78.106472107 72.9382705688477 11.0
81.446715458 82.7116394042969 12.0
81.446715458 80.4266357421875 13.0
81.446715458 81.2852783203125 14.0
0 -1.14512276649475 15.0
0 -0.0151953101158142 16.0
0 1.68383920192719 17.0
57.168582547 58.4678535461426 18.0
-57.168582557 -58.9646301269531 19.0
-57.168582535 -59.6098480224609 20.0
-57.168582548 -58.1340484619141 21.0
78.561701539 76.9391708374023 22.0
145.42241232 141.494049072266 23.0
110.78882087 110.363571166992 24.0
24.085164087 19.9674625396729 25.0
223.98411387 225.872756958008 26.0
223.98411387 220.490081787109 27.0
27.628962939 29.3433818817139 28.0
251.61307681 256.386871337891 29.0
251.61307681 256.043029785156 30.0
46.829541976 45.5860214233398 31.0
9.3325701774 9.22627067565918 32.0
21.393118983 17.5126152038574 33.0
21.393118983 19.0022487640381 34.0
21.393118982 20.2901363372803 35.0
9.6735473307 5.512864112854 36.0
8.4264532871 7.88378429412842 37.0
-5.6297110477e-10 -0.0107722878456116 38.0
-5.6297110477e-10 -2.31899356842041 39.0
-11.569348985 -14.1814813613892 40.0
-24.615150545 -26.2458248138428 41.0
4.880051929 5.54185581207275 42.0
-32.086690815 -33.4734992980957 43.0
95.193286116 90.9193878173828 44.0
95.193286128 93.8373260498047 45.0
6.8978137424 7.70694780349731 46.0
2.5886928279 2.68894600868225 47.0
179.87042805 164.607223510742 48.0
173.80453475 160.958984375 49.0
0 1.16922283172607 50.0
72.205216379 77.2485961914062 51.0
90.954267693 84.3228073120117 52.0
81.446715467 81.1558227539062 53.0
81.446715467 80.5682067871094 54.0
81.446715467 80.5592041015625 55.0
0 1.20625376701355 56.0
78.561701549 80.0138473510742 57.0
223.98411388 220.348541259766 58.0
251.61307682 250.526992797852 59.0
21.393118992 23.775857925415 60.0
0 3.22667717933655 61.0
159.87793028 159.781051635742 0.0
159.87793028 154.506362915039 1.0
158.49183143 161.291641235352 2.0
153.7027613 145.443984985352 3.0
0 -12.2476854324341 4.0
0 17.2499198913574 5.0
-67.400503702 -45.0059967041016 6.0
79.544323635 80.6306228637695 7.0
54.007987226 40.4847984313965 8.0
25.536336399 21.4741344451904 9.0
78.003783756 82.0531005859375 10.0
65.522349866 74.5462341308594 11.0
73.942849837 70.8478851318359 12.0
73.942849837 76.1288528442383 13.0
73.942849837 80.3363037109375 14.0
0 1.29038631916046 15.0
0 -1.30883169174194 16.0
0 0.870879709720612 17.0
59.105666287 59.3675117492676 18.0
-59.105666297 -59.2049674987793 19.0
-59.105666276 -57.0794258117676 20.0
-59.105666288 -58.6720848083496 21.0
80.322549265 69.9968032836914 22.0
141.84408234 137.041259765625 23.0
106.18491377 108.513023376465 24.0
22.218091295 20.1900367736816 25.0
222.16663162 224.071304321289 26.0
222.16663162 217.401962280273 27.0
24.459597793 27.3965129852295 28.0
246.62622942 243.74462890625 29.0
246.62622942 251.254974365234 30.0
47.026444908 40.9113235473633 31.0
9.2996802835 8.64844703674316 32.0
21.216882967 21.1490592956543 33.0
21.216882967 22.6075611114502 34.0
21.216882967 21.4203281402588 35.0
9.7728622608 9.60377407073975 36.0
8.5336969608 8.01455211639404 37.0
-3.3399054203e-12 -5.34731388092041 38.0
-3.339905656e-12 -2.4704065322876 39.0
-11.623479193 -16.2676811218262 40.0
-24.94518145 -21.7509174346924 41.0
5.6537678347 7.99192476272583 42.0
-33.39877357 -34.5808486938477 43.0
91.680904746 90.2656326293945 44.0
91.680904759 90.5833969116211 45.0
6.8628312506 7.33338642120361 46.0
2.7759735028 1.21577906608582 47.0
159.87793029 168.45622253418 48.0
153.70276131 146.389297485352 49.0
0 -3.07367014884949 50.0
79.544323645 76.6467514038086 51.0
78.003783766 89.6426696777344 52.0
73.942849847 76.5899963378906 53.0
73.942849847 74.6017761230469 54.0
73.942849847 80.9799118041992 55.0
0 5.45345783233643 56.0
80.322549275 79.8838424682617 57.0
222.16663163 221.840286254883 58.0
246.62622943 248.984573364258 59.0
21.216882977 19.2913799285889 60.0
0 2.36471343040466 61.0
160.40029773 161.286422729492 0.0
160.40029773 164.271453857422 1.0
159.04685921 156.979202270508 2.0
154.25778948 160.440948486328 3.0
0 1.96068620681763 4.0
-1.1106157605e-11 4.07722473144531 5.0
-67.164254488 -69.7969284057617 6.0
77.905037981 76.2602233886719 7.0
56.332002807 47.2559432983398 8.0
21.573035164 25.714204788208 9.0
86.970057114 90.5282669067383 10.0
73.950416555 75.4032516479492 11.0
81.119634047 87.77490234375 12.0
81.119634047 78.8990020751953 13.0
81.119634047 80.1012420654297 14.0
0 -0.256228864192963 15.0
0 -0.19922286272049 16.0
0 0.492870032787323 17.0
59.264416476 57.1221313476562 18.0
-59.264416496 -58.6440391540527 19.0
-59.264416474 -57.773265838623 20.0
-59.264416487 -57.5845527648926 21.0
80.633842594 78.2018508911133 22.0
145.67202515 141.996139526367 23.0
110.34261621 108.156616210938 24.0
22.971694682 19.1125392913818 25.0
226.30586775 222.863891601562 26.0
226.30586775 227.450820922852 27.0
28.061618658 29.2107601165771 28.0
254.36748642 252.690139770508 29.0
254.36748642 253.60498046875 30.0
46.518523875 45.2752456665039 31.0
9.3238344953 8.95865631103516 32.0
21.369426097 17.7103004455566 33.0
21.369426097 17.6770057678223 34.0
21.369426097 20.8870105743408 35.0
9.8534733136 7.96312856674194 36.0
8.5804788689 6.95929193496704 37.0
-3.4865499095e-11 -0.434172928333282 38.0
-3.4865475198e-11 -3.54875135421753 39.0
-12.411186885 -13.9457921981812 40.0
-24.601415118 -26.8714199066162 41.0
5.0371208571 5.8277759552002 42.0
-32.131741558 -34.3593444824219 43.0
94.496748584 90.5503082275391 44.0
94.496748597 94.2157592773438 45.0
6.7865864039 6.48163652420044 46.0
2.4610538327 2.17473244667053 47.0
160.40029774 162.612167358398 48.0
154.25778949 150.847152709961 49.0
0 2.82914710044861 50.0
77.905037991 77.1076354980469 51.0
86.970057124 87.8557891845703 52.0
81.119634057 81.9147491455078 53.0
81.119634057 81.5159759521484 54.0
81.119634057 80.1029281616211 55.0
0 1.16054677963257 56.0
80.633842604 79.8010025024414 57.0
226.30586776 219.901840209961 58.0
254.36748643 250.350433349609 59.0
21.369426107 22.7730197906494 60.0
0 1.7182629108429 61.0
205.88622948 173.514938354492 0.0
205.88622948 148.778915405273 1.0
204.34934471 164.515563964844 2.0
203.29335043 193.710494995117 3.0
-4.0865192275e-13 -30.2441349029541 4.0
-1.699540558e-11 -3.37182903289795 5.0
-54.720220948 -66.4943542480469 6.0
62.440552717 79.6623306274414 7.0
38.617790334 55.1751747131348 8.0
23.822762373 16.3021488189697 9.0
92.262519811 94.2973937988281 10.0
80.60863046 83.7443771362305 11.0
75.716709666 94.0955581665039 12.0
75.716709666 84.0410232543945 13.0
75.716709666 74.218017578125 14.0
0 1.89795613288879 15.0
0 -0.959831535816193 16.0
-2.736801062e-11 0.252919852733612 17.0
53.182406828 56.1366577148438 18.0
-53.182406848 -55.9189147949219 19.0
-53.851025897 -55.4914054870605 20.0
-53.919667878 -54.391902923584 21.0
72.319633039 79.8481750488281 22.0
140.95980522 143.012832641602 23.0
109.52526146 110.76490020752 24.0
21.76060813 23.0991744995117 25.0
213.27943827 221.519638061523 26.0
213.27943827 223.415313720703 27.0
30.326918439 27.8478965759277 28.0
243.60635672 230.311416625977 29.0
243.60635672 243.790466308594 30.0
46.890139691 42.1221771240234 31.0
9.3865843553 14.0318698883057 32.0
18.399965152 20.7162628173828 33.0
18.399965152 33.9090690612793 34.0
18.399965152 12.4397554397583 35.0
9.4062169528 14.0111961364746 36.0
8.2007042844 6.22383642196655 37.0
-4.3220675773e-11 -4.60679912567139 38.0
-4.3220675758e-11 7.74320936203003 39.0
-11.523847108 1.3075897693634 40.0
-24.285918417 -25.19167137146 41.0
4.7578124137 -2.60031032562256 42.0
-31.668135572 -24.6885070800781 43.0
92.257051498 90.1106033325195 44.0
92.257051511 92.1590347290039 45.0
7.3511518177 6.29843139648438 46.0
2.7625543742 1.09672236442566 47.0
205.88622949 185.597076416016 48.0
203.29335044 174.330703735352 49.0
0 -4.64554977416992 50.0
62.440552727 69.513557434082 51.0
92.262519821 83.9690017700195 52.0
75.716709676 80.9438781738281 53.0
75.716709676 79.6303405761719 54.0
75.716709676 80.9406585693359 55.0
0 -2.64241337776184 56.0
72.319633049 81.4288635253906 57.0
213.27943828 217.504852294922 58.0
243.60635673 245.586532592773 59.0
18.399965162 24.8256778717041 60.0
0 12.7717094421387 61.0
177.230790776723 168.276840209961 0.0
177.230790776723 170.954711914062 1.0
175.97489696 167.775146484375 2.0
171.18582724 165.013198852539 3.0
0 1.00840997695923 4.0
-8.695896658e-09 3.25687646865845 5.0
-66.6813280614721 -64.9787826538086 6.0
78.6009224736525 80.8284149169922 7.0
55.1150721596525 51.1753311157227 8.0
23.4858503146525 27.8591899871826 9.0
89.9623216806525 89.4219512939453 10.0
77.6099152396525 77.1262817382812 11.0
81.2213177876525 81.5533676147461 12.0
81.2213177876525 78.3586730957031 13.0
81.2213177876525 78.9095306396484 14.0
0 -1.47511386871338 15.0
0 -0.0684186220169067 16.0
-6.719166526e-09 0.437236607074738 17.0
60.8898286619923 58.855541229248 18.0
-60.8898286714787 -58.6323928833008 19.0
-60.8898286504787 -60.4781761169434 20.0
-60.8898286564787 -58.7246360778809 21.0
83.2160013870953 76.590950012207 22.0
146.105084467992 142.297088623047 23.0
112.87952896249 110.16837310791 24.0
22.4766954984903 19.7099113464355 25.0
229.321085853445 220.831146240234 26.0
229.321085853445 223.985000610352 27.0
26.8991514584452 25.2338886260986 28.0
256.220237313445 257.053894042969 29.0
256.220237313445 256.384429931641 30.0
46.6620169402483 45.3788032531738 31.0
9.3391522138 8.84155368804932 32.0
22.3261727304164 20.6443119049072 33.0
22.3261727304164 20.8280563354492 34.0
22.3261727370871 20.1241493225098 35.0
9.7525403042 6.9317421913147 36.0
8.4757786226 6.67255544662476 37.0
-9.36233601e-09 1.99968218803406 38.0
-9.36233601e-09 -3.0368480682373 39.0
-11.6085369957393 -12.1175355911255 40.0
-25.0228063454843 -25.3875064849854 41.0
4.98434496197705 6.57820272445679 42.0
-32.5929350063771 -33.683406829834 43.0
98.797614073377 95.0439376831055 44.0
98.7976140573771 94.74658203125 45.0
6.80137819 5.90730619430542 46.0
2.4754452337 2.69849920272827 47.0
177.23079077 167.853790283203 48.0
171.18582724 159.193328857422 49.0
0 0.369772851467133 50.0
78.600922474 75.9872894287109 51.0
89.96232168 91.9298629760742 52.0
81.221317788 81.5424041748047 53.0
81.221317788 80.8521270751953 54.0
81.221317788 80.2592697143555 55.0
0 2.4076042175293 56.0
83.216001387 78.8677368164062 57.0
229.32108586 219.342559814453 58.0
256.22023731 249.697052001953 59.0
22.326172737 20.5679149627686 60.0
0 2.27629065513611 61.0
194.50491643 158.78385925293 0.0
194.50491643 160.224014282227 1.0
193.06848243 164.345428466797 2.0
187.60384724 156.945617675781 3.0
30.398962328 -25.3855628967285 4.0
30.398962328 13.4483613967896 5.0
-44.133173312 -51.7471237182617 6.0
54.844291013 105.291076660156 7.0
30.415905688 72.8280487060547 8.0
24.428385315 69.9720993041992 9.0
76.411670882 86.7153854370117 10.0
64.501631571 74.2790374755859 11.0
78.632568296 78.4349670410156 12.0
78.632568296 77.4130706787109 13.0
78.632568296 77.451286315918 14.0
0 -9.98974323272705 15.0
0 -2.43092060089111 16.0
-1.7575985849e-13 7.48656415939331 17.0
50.561854878 50.427547454834 18.0
-50.561854898 -53.2969703674316 19.0
-50.561854876 -54.9677734375 20.0
-50.561854889 -45.9612159729004 21.0
69.24826932 82.5991744995117 22.0
151.24116966 150.298202514648 23.0
117.89321728 86.5286483764648 24.0
19.477699927 31.4649333953857 25.0
220.48943899 217.967193603516 26.0
220.48943899 225.018569946289 27.0
31.621710876 29.6233425140381 28.0
252.11114987 255.021987915039 29.0
252.11114987 259.780120849609 30.0
45.984317059 46.8003044128418 31.0
9.3844218336 8.82529544830322 32.0
18.686414421 13.2807312011719 33.0
18.686414421 17.5692443847656 34.0
18.686414421 19.0710906982422 35.0
8.2243982504 7.06270885467529 36.0
7.0804167549 4.9096565246582 37.0
-1.7080465289e-12 -11.7643432617188 38.0
-1.7080393598e-12 -3.7664623260498 39.0
-11.04040656 -13.9357299804688 40.0
-22.951277844 -27.2106456756592 41.0
5.6114485235 2.16883397102356 42.0
-31.403900297 -35.0329132080078 43.0
73.833493183 81.2718048095703 44.0
73.833493196 68.4049072265625 45.0
6.3215873537 5.36140012741089 46.0
1.8513764592 2.602618932724 47.0
194.50491644 166.47590637207 48.0
187.60384725 148.235290527344 49.0
30.398962338 -4.18765115737915 50.0
54.844291023 69.0881500244141 51.0
76.411670892 84.1355361938477 52.0
78.632568306 79.2091598510742 53.0
78.632568306 74.607177734375 54.0
78.632568306 79.8735580444336 55.0
0 10.1250905990601 56.0
69.24826933 78.8467407226562 57.0
220.489439 223.080413818359 58.0
252.11114988 256.340759277344 59.0
18.686414431 27.9230880737305 60.0
0 2.42515635490417 61.0
131.292175352041 169.580505371094 0.0
131.292175352041 182.656173706055 1.0
130.22602559 162.937957763672 2.0
124.76139035 128.88623046875 3.0
5.4278882976 0.809947073459625 4.0
5.42788830653951 13.7109050750732 5.0
-78.1983969934395 -87.278938293457 6.0
92.6068568515383 78.4707946777344 7.0
62.5250961945383 41.9572944641113 8.0
30.0817606565383 11.8795213699341 9.0
81.8214669485383 87.4131393432617 10.0
68.0064155625383 63.3590278625488 11.0
75.7700025005383 58.9240036010742 12.0
75.7700025005383 67.5720977783203 13.0
75.7700025005383 71.0859146118164 14.0
3.1868823314e-09 4.48831844329834 15.0
0 1.35277986526489 16.0
7.16567026999998e-09 0.0922404527664185 17.0
39.6398577708529 54.357292175293 18.0
-39.6398578093074 -58.1208419799805 19.0
-39.6398577903074 -51.2564544677734 20.0
-39.6398577813074 -53.0891227722168 21.0
59.4905371677941 71.4259338378906 22.0
132.528328896839 120.27131652832 23.0
100.18050262294 103.581985473633 24.0
11.8598445510488 26.3986778259277 25.0
192.018866064083 178.269561767578 26.0
192.018866064083 195.499557495117 27.0
31.1994688420826 19.689567565918 28.0
223.218334904083 233.820861816406 29.0
223.218334904083 205.715133666992 30.0
46.3085063081691 63.8246383666992 31.0
9.30386468980555 7.19010305404663 32.0
19.8506793863185 12.9460248947144 33.0
19.8506793863185 13.6501321792603 34.0
19.850679411427 22.1063594818115 35.0
8.19801294945471 7.37389850616455 36.0
7.02365271983579 6.53996753692627 37.0
-7.19083610000001e-09 -14.6133298873901 38.0
-7.19083610000001e-09 9.72542381286621 39.0
-14.1586521526537 -19.9678897857666 40.0
-28.1561169624771 -26.1425552368164 41.0
6.16095936544857 -3.18109512329102 42.0
-36.9019809359486 -25.6850032806396 43.0
60.9542831089486 73.3307800292969 44.0
60.9542829201705 51.3957366943359 45.0
7.3749640408 6.6553635597229 46.0
2.9186157837 -2.00330328941345 47.0
131.29217535 194.840347290039 48.0
124.76139036 121.299278259277 49.0
5.4278883066 3.56686043739319 50.0
92.606856852 73.3419647216797 51.0
81.821466949 82.5123596191406 52.0
75.7700025 66.3910903930664 53.0
75.7700025 65.5056457519531 54.0
75.7700025 77.4005889892578 55.0
7.1656702734e-09 9.00715923309326 56.0
59.490537167 64.4880142211914 57.0
192.01886606 208.727355957031 58.0
223.21833491 187.188812255859 59.0
19.850679411 28.0587978363037 60.0
0 35.9724349975586 61.0
147.982033057583 157.456268310547 0.0
147.982033057583 151.96012878418 1.0
146.808413813287 153.320571899414 2.0
142.01934397 150.517456054688 3.0
-1.55484310406e-09 -0.624643623828888 4.0
-1.55484309999997e-09 -2.83117961883545 5.0
-64.9693040339936 -64.312385559082 6.0
77.6547829144777 74.1870651245117 7.0
61.2786926634778 55.4030647277832 8.0
16.3760902514777 7.20984888076782 9.0
97.5261141374777 89.7391510009766 10.0
83.6472874924777 79.6450042724609 11.0
85.2981411144777 83.8642730712891 12.0
85.2981411144777 76.4624557495117 13.0
85.2981411144777 81.5436782836914 14.0
2.3328021736e-08 -2.03496789932251 15.0
2.3328021735e-08 -1.6052393913269 16.0
5.464350236e-08 -2.72059202194214 17.0
56.2518018559467 59.0783615112305 18.0
-56.2518018797166 -57.0891418457031 19.0
-56.2518018707166 -59.2854194641113 20.0
-56.2518017707166 -58.115608215332 21.0
77.1875821073724 73.3798065185547 22.0
153.454771802806 145.008895874023 23.0
115.220596133508 115.750228881836 24.0
21.7005584165075 20.8819332122803 25.0
230.642353915167 228.381210327148 26.0
230.642353915167 229.708221435547 27.0
26.8883420391666 21.3647422790527 28.0
257.530695955167 253.23681640625 29.0
257.530695955167 255.336776733398 30.0
46.7385281475469 45.537841796875 31.0
9.28724497493877 8.4355525970459 32.0
20.9357803369809 22.6376686096191 33.0
20.9357803369809 19.0790691375732 34.0
20.935780467613 20.4264030456543 35.0
8.86054244871795 8.39873600006104 36.0
7.60121916461794 6.85336446762085 37.0
2.62465480000001e-09 1.1959273815155 38.0
2.62465480000001e-09 -1.17553234100342 39.0
-11.6034306368583 -12.5643882751465 40.0
-24.5599220844463 -26.7411251068115 41.0
4.75986482786468 1.92820739746094 42.0
-31.7734218810648 -31.6524333953857 43.0
89.4696777832616 86.4044494628906 44.0
89.4696781210649 91.9600372314453 45.0
7.05809588635643 4.443039894104 46.0
2.46335994485643 2.26466846466064 47.0
147.98203306 156.098068237305 48.0
142.01934397 152.81867980957 49.0
0 -3.76089572906494 50.0
77.654782914 76.0241851806641 51.0
97.526114137 92.0755157470703 52.0
85.298141114 81.5651702880859 53.0
85.298141114 80.2131118774414 54.0
85.298141114 79.9203643798828 55.0
5.4643502359e-08 3.07315587997437 56.0
77.187582108 79.4883346557617 57.0
230.64235391 222.583419799805 58.0
257.53069595 249.491134643555 59.0
20.935780468 19.6116352081299 60.0
2.6246548239e-09 3.64096260070801 61.0
175.05640447 158.047271728516 0.0
175.05640447 162.89387512207 1.0
173.86983802 167.242279052734 2.0
168.65913708 168.698623657227 3.0
0 0.889920651912689 4.0
-4.1281766273e-10 2.32937526702881 5.0
-68.631844265 -66.3096923828125 6.0
79.017161661 75.489143371582 7.0
47.789810844 49.8381233215332 8.0
31.227350809 24.71116065979 9.0
86.753133177 86.4171981811523 10.0
73.521080011 75.5051727294922 11.0
81.326417839 81.1204376220703 12.0
81.326417839 79.945556640625 13.0
81.326417839 78.0863037109375 14.0
0 0.194886147975922 15.0
0 0.441041111946106 16.0
-1.2762600089e-09 0.898068130016327 17.0
58.179979465 59.8065986633301 18.0
-58.179979488 -59.6823425292969 19.0
-58.179979466 -59.6734619140625 20.0
-58.179979478 -60.8730735778809 21.0
79.799760093 77.9290008544922 22.0
146.48581339 141.393615722656 23.0
112.31755034 110.322265625 24.0
22.818628662 21.3721466064453 25.0
226.2855735 223.709777832031 26.0
226.2855735 219.762481689453 27.0
26.16727697 23.9745922088623 28.0
252.45285047 255.15885925293 29.0
252.45285047 257.40869140625 30.0
46.933708058 48.1185150146484 31.0
9.3144364392 8.44257068634033 32.0
21.61978061 18.908634185791 33.0
21.61978061 18.9554576873779 34.0
21.619780609 19.3043079376221 35.0
9.2718822657 7.30234718322754 36.0
8.0060640176 8.38574600219727 37.0
-1.6395844448e-09 0.38869833946228 38.0
-1.6395844447e-09 -1.42251467704773 39.0
-11.645049069 -13.5318040847778 40.0
-24.09040202 -23.3780937194824 41.0
4.9097031261 5.66740608215332 42.0
-31.543361217 -33.8258323669434 43.0
92.689325955 91.2669372558594 44.0
92.689325964 94.9345016479492 45.0
6.5748148242 7.28198480606079 46.0
2.4864486216 2.73059940338135 47.0
175.05640448 163.227355957031 48.0
168.65913709 162.038284301758 49.0
0 -1.35363626480103 50.0
79.017161668 76.3994903564453 51.0
86.753133184 90.2097473144531 52.0
81.326417847 81.2863235473633 53.0
81.326417847 81.3412857055664 54.0
81.326417847 79.7621231079102 55.0
0 -0.53716379404068 56.0
79.799760103 80.1216735839844 57.0
226.2855735 219.242980957031 58.0
252.45285048 250.191162109375 59.0
21.619780618 20.7179851531982 60.0
0 1.77052474021912 61.0
149.74300524 162.196640014648 0.0
149.74300524 164.963790893555 1.0
148.60292447 157.309143066406 2.0
143.81385475 144.935913085938 3.0
-2.0689067385e-09 0.731975972652435 4.0
-2.0695859295e-09 2.81129240989685 5.0
-59.172436536 -63.2283973693848 6.0
69.863961297 72.3412857055664 7.0
50.351923902 51.8477439880371 8.0
19.512037386 19.549036026001 9.0
89.061401272 89.3783111572266 10.0
76.646541462 77.4210739135742 11.0
82.271461616 84.4279251098633 12.0
82.271461616 80.3724746704102 13.0
82.271461616 80.786750793457 14.0
0 -1.98374891281128 15.0
0 -0.383570730686188 16.0
0 -1.99984955787659 17.0
58.587428012 58.7384834289551 18.0
-58.587428022 -59.7323417663574 19.0
-58.587428 -58.4210662841797 20.0
-58.587428012 -61.8523635864258 21.0
80.815527246 78.6102600097656 22.0
149.14495043 138.846939086914 23.0
114.74863874 109.534286499023 24.0
23.494457945 22.0825862884521 25.0
229.96047769 223.637069702148 26.0
229.96047769 225.190765380859 27.0
25.257363317 27.0872573852539 28.0
255.21784102 254.795501708984 29.0
255.21784102 254.561431884766 30.0
46.445612732 46.0470008850098 31.0
9.3080716648 8.71134281158447 32.0
22.228099224 20.3574371337891 33.0
22.228099224 19.8917560577393 34.0
22.228099224 21.5110111236572 35.0
9.7988489677 8.20902442932129 36.0
8.571063741 8.01464653015137 37.0
-1.9840740013e-13 -1.75648212432861 38.0
-2.0075265721e-13 -2.94999742507935 39.0
-11.596625147 -13.4047613143921 40.0
-24.687040939 -26.1258544921875 41.0
5.2629816457 4.64785957336426 42.0
-32.514995078 -33.0170478820801 43.0
93.869330832 92.4433822631836 44.0
93.869330845 95.4139099121094 45.0
6.9110970773 5.80423069000244 46.0
2.4109034906 1.53973054885864 47.0
149.74300525 163.100250244141 48.0
143.81385476 136.874740600586 49.0
0 -1.35773801803589 50.0
69.863961307 75.5542221069336 51.0
89.061401282 89.7482452392578 52.0
82.271461626 80.341438293457 53.0
82.271461626 80.9588012695312 54.0
82.271461626 79.8151702880859 55.0
0 2.55020761489868 56.0
80.815527256 79.0756912231445 57.0
229.9604777 218.848068237305 58.0
255.21784103 249.596008300781 59.0
22.228099234 20.0694141387939 60.0
0 1.23250555992126 61.0
147.982033057583 157.456314086914 0.0
147.982033057583 151.961029052734 1.0
146.808413813287 153.327545166016 2.0
142.01934397 150.566131591797 3.0
-1.55484310406e-09 -0.630645573139191 4.0
-1.55484309999997e-09 -2.83325052261353 5.0
-64.9693040339936 -64.3144454956055 6.0
77.6547829144777 74.1870880126953 7.0
61.2786926634778 55.4030647277832 8.0
16.3760902514777 7.20984888076782 9.0
97.5261141374777 89.7391510009766 10.0
83.6472874924777 79.6450042724609 11.0
85.2981411144777 83.8642730712891 12.0
85.2981411144777 76.4624557495117 13.0
85.2981411144777 81.5436782836914 14.0
2.3328021736e-08 -2.03496789932251 15.0
2.3328021735e-08 -1.6052393913269 16.0
5.464350236e-08 -2.72059202194214 17.0
56.2518018559467 59.0783615112305 18.0
-56.2518018797166 -57.0891418457031 19.0
-56.2518018707166 -59.2854194641113 20.0
-56.2518017707166 -58.115608215332 21.0
77.1875821073724 73.3798065185547 22.0
153.454771802806 145.008895874023 23.0
115.220596133508 115.750228881836 24.0
21.7005584165075 20.8819332122803 25.0
230.642353915167 228.381210327148 26.0
230.642353915167 229.708221435547 27.0
26.8883420391666 21.3647422790527 28.0
257.530695955167 253.23681640625 29.0
257.530695955167 255.336776733398 30.0
46.7385281475469 45.537841796875 31.0
9.28724497493877 8.4355525970459 32.0
20.9357803369809 22.6376686096191 33.0
20.9357803369809 19.0790691375732 34.0
20.935780467613 20.4264030456543 35.0
8.86054244871795 8.39873600006104 36.0
7.60121916461794 6.85336446762085 37.0
2.62465480000001e-09 1.1959273815155 38.0
2.62465480000001e-09 -1.17553234100342 39.0
-11.6034306368583 -12.5643882751465 40.0
-24.5599220844463 -26.7411251068115 41.0
4.75986482786468 1.92820739746094 42.0
-31.7734218810648 -31.6524333953857 43.0
89.4696777832616 86.4044494628906 44.0
89.4696781210649 91.9600372314453 45.0
7.05809588635643 4.443039894104 46.0
2.46335994485643 2.26466846466064 47.0
147.98203306 156.097717285156 48.0
142.01934397 152.859603881836 49.0
0 -3.75777673721313 50.0
77.654782914 76.024055480957 51.0
97.526114137 92.0755157470703 52.0
85.298141114 81.5651702880859 53.0
85.298141114 80.2131118774414 54.0
85.298141114 79.9203643798828 55.0
5.4643502359e-08 3.07315587997437 56.0
77.187582108 79.4883346557617 57.0
230.64235391 222.583419799805 58.0
257.53069595 249.491134643555 59.0
20.935780468 19.6116352081299 60.0
2.6246548239e-09 3.64096260070801 61.0
155.9515958 158.761871337891 0.0
155.9515958 161.552047729492 1.0
154.57709238 166.050643920898 2.0
149.78802215 167.815277099609 3.0
0 1.69625949859619 4.0
0 4.26820755004883 5.0
-58.820004579 -60.1047286987305 6.0
70.091633906 77.1000213623047 7.0
52.84159464 50.0321350097656 8.0
17.250039256 23.1166610717773 9.0
86.175303871 83.5700225830078 10.0
73.763020481 73.6824798583984 11.0
80.221660628 78.245491027832 12.0
80.221660628 78.6270904541016 13.0
80.221660628 79.8447570800781 14.0
0 -1.54206323623657 15.0
0 -0.581657707691193 16.0
0 -1.07288432121277 17.0
60.156580567 59.3688850402832 18.0
-60.156580577 -58.8717002868652 19.0
-60.156580555 -58.3993797302246 20.0
-60.156580568 -59.3029632568359 21.0
81.030113508 78.3415222167969 22.0
142.62343296 139.917678833008 23.0
107.55639392 107.150863647461 24.0
22.553402363 20.098575592041 25.0
223.65354648 224.046173095703 26.0
223.65354648 223.720199584961 27.0
27.857738575 23.5725154876709 28.0
251.51128506 255.02229309082 29.0
251.51128506 256.141540527344 30.0
46.64135947 46.6488189697266 31.0
9.3104949216 8.98957538604736 32.0
20.873532931 17.2766151428223 33.0
20.873532931 22.2694683074951 34.0
20.873532931 20.0050506591797 35.0
9.2636577405 6.59417009353638 36.0
7.9875683739 7.7546558380127 37.0
-5.5727330635e-11 -2.11790919303894 38.0
-5.5727326372e-11 -3.50058078765869 39.0
-11.556082016 -13.4613647460938 40.0
-24.060686796 -25.9978294372559 41.0
5.3170338411 3.93987369537354 42.0
-31.943974983 -32.7224235534668 43.0
102.27302073 95.5835952758789 44.0
102.27302075 91.9066696166992 45.0
7.3010676096 6.24856805801392 46.0
2.7150235208 2.1412034034729 47.0
155.95159581 166.415267944336 48.0
149.78802216 149.850845336914 49.0
0 2.77348327636719 50.0
70.091633916 76.5297393798828 51.0
86.175303881 87.8432540893555 52.0
80.221660638 80.0057525634766 53.0
80.221660638 78.6451034545898 54.0
80.221660638 80.6831817626953 55.0
0 2.16135334968567 56.0
81.030113518 80.0179061889648 57.0
223.65354649 220.881546020508 58.0
251.51128507 250.301864624023 59.0
20.873532941 20.8628005981445 60.0
0 1.57388949394226 61.0
193.74743458 161.006546020508 0.0
193.74743458 169.974227905273 1.0
192.26112285 170.249908447266 2.0
187.47205313 155.828231811523 3.0
0 62.9340057373047 4.0
-1.5500005758e-11 -1.46593189239502 5.0
-59.170737721 -57.3060722351074 6.0
67.981301975 69.3440704345703 7.0
50.801036555 55.084156036377 8.0
17.18026541 42.0902938842773 9.0
80.60791345 89.4135437011719 10.0
67.187792988 75.4097213745117 11.0
77.200616673 79.3903884887695 12.0
77.200616673 79.7898483276367 13.0
77.200616673 78.0370635986328 14.0
0 0.524172425270081 15.0
0 0.356610119342804 16.0
-3.0789439856e-11 5.35756587982178 17.0
61.169110979 60.0959129333496 18.0
-61.169110999 -58.0934829711914 19.0
-61.169110977 -55.479606628418 20.0
-61.16911099 -61.8921394348145 21.0
82.368088346 82.9021911621094 22.0
148.41919822 140.862213134766 23.0
115.06825702 102.917236328125 24.0
23.270491976 20.276252746582 25.0
230.78728657 224.030563354492 26.0
230.78728657 224.760726928711 27.0
24.71447785 25.1674041748047 28.0
255.50176443 257.416534423828 29.0
255.50176443 254.730758666992 30.0
46.903718362 47.8527412414551 31.0
9.3965485599 9.23358058929443 32.0
21.198977347 18.4731750488281 33.0
21.198977347 16.0892372131348 34.0
21.198977347 16.7273693084717 35.0
10.014368724 7.9159893989563 36.0
8.7283996713 8.09894180297852 37.0
-3.2342910638e-12 3.56922101974487 38.0
-3.2337165534e-12 3.24809980392456 39.0
-11.265223443 -17.1737785339355 40.0
-24.774508429 -25.0396366119385 41.0
5.4987536196 0.516071200370789 42.0
-32.730757149 -31.7689914703369 43.0
98.182308392 91.5102081298828 44.0
98.182308406 90.2717742919922 45.0
6.9879104268 3.32054257392883 46.0
2.4755071244 1.10750114917755 47.0
193.74743459 166.589050292969 48.0
187.47205314 155.339584350586 49.0
0 10.1205749511719 50.0
67.981301985 67.3926620483398 51.0
80.60791346 89.8803253173828 52.0
77.200616683 80.3043441772461 53.0
77.200616683 81.2972412109375 54.0
77.200616683 81.8302917480469 55.0
0 4.57571887969971 56.0
82.368088356 79.931526184082 57.0
230.78728658 222.874664306641 58.0
255.50176444 246.720397949219 59.0
21.198977357 22.9484462738037 60.0
0 3.50199699401855 61.0
137.17687742 143.006530761719 0.0
137.17687742 143.030563354492 1.0
135.86355256 143.538650512695 2.0
131.0744823 135.772583007812 3.0
0 -2.1230001449585 4.0
0 0.342061758041382 5.0
-59.42861149 -59.2355995178223 6.0
69.705106679 70.1772918701172 7.0
65.43029471 62.4528312683105 8.0
4.2748119615 8.48783111572266 9.0
78.669161727 80.2065200805664 10.0
65.229460649 71.2585144042969 11.0
77.500548395 80.8047180175781 12.0
77.500548395 77.9000701904297 13.0
77.500548395 76.9198608398438 14.0
0 -0.64198762178421 15.0
0 0.28728449344635 16.0
0 -3.15485000610352 17.0
59.754945382 58.5015258789062 18.0
-59.754945394 -59.4383087158203 19.0
-59.754945372 -60.0649642944336 20.0
-59.754945384 -61.8801422119141 21.0
80.656571382 77.8813858032227 22.0
140.87903669 140.394561767578 23.0
106.41719675 107.357681274414 24.0
22.90053999 20.4083709716797 25.0
221.53560808 223.594223022461 26.0
221.53560808 221.987747192383 27.0
27.282645659 25.3551406860352 28.0
248.81825375 253.574676513672 29.0
248.81825375 252.408157348633 30.0
47.033013387 47.2299919128418 31.0
9.3131031344 9.18030452728271 32.0
20.901625993 20.4819431304932 33.0
20.901625993 21.1129970550537 34.0
20.901625993 20.8195400238037 35.0
9.3083381265 8.28715133666992 36.0
8.044767894 8.38281917572021 37.0
-2.5262790395e-09 -0.804296553134918 38.0
-2.5262785966e-09 0.113371789455414 39.0
-12.387792857 -14.5510091781616 40.0
-23.81592281 -27.5234241485596 41.0
4.9278776077 3.5162615776062 42.0
-31.108913619 -34.7394485473633 43.0
92.052766479 91.3064651489258 44.0
92.05276649 90.2343292236328 45.0
6.7273095346 3.00315618515015 46.0
2.6568330431 1.83220052719116 47.0
137.17687743 136.681655883789 48.0
131.07448231 127.552749633789 49.0
0 3.54248547554016 50.0
69.705106686 70.9861755371094 51.0
78.669161734 84.2529830932617 52.0
77.500548402 81.0479965209961 53.0
77.500548402 80.9212951660156 54.0
77.500548402 79.4541549682617 55.0
0 -0.0822644829750061 56.0
80.656571391 80.3950500488281 57.0
221.53560809 218.26057434082 58.0
248.81825375 246.469757080078 59.0
20.901626002 19.9398994445801 60.0
0 2.89454770088196 61.0
163.21019338 156.904968261719 0.0
163.21019338 160.027648925781 1.0
162.13291783 158.540023803711 2.0
157.34384811 165.640472412109 3.0
0 -3.39346551895142 4.0
-2.6575017708e-10 2.82552671432495 5.0
-57.070536688 -45.0189170837402 6.0
68.377124439 81.7586975097656 7.0
42.05560288 50.7922821044922 8.0
26.32152155 24.0542373657227 9.0
85.180823624 102.963668823242 10.0
71.764497626 77.6302337646484 11.0
86.784890536 82.7948532104492 12.0
86.784890536 79.5152816772461 13.0
86.784890536 81.1663284301758 14.0
0 -1.38389801979065 15.0
0 -1.30660533905029 16.0
0 -3.19053030014038 17.0
53.845706319 59.040153503418 18.0
-53.845706333 -59.5292282104492 19.0
-53.845706312 -56.6618347167969 20.0
-53.845706324 -58.52197265625 21.0
76.472319104 77.8278503417969 22.0
137.34531213 146.705673217773 23.0
105.23721232 104.183837890625 24.0
22.970900431 20.9522037506104 25.0
213.81763124 227.177154541016 26.0
213.81763124 230.356597900391 27.0
29.690470297 21.1811180114746 28.0
243.50810155 255.470336914062 29.0
243.50810155 258.360717773438 30.0
46.273089087 44.6531562805176 31.0
9.4128103892 8.67997074127197 32.0
22.626612772 17.0356483459473 33.0
22.626612772 20.0318946838379 34.0
22.626612772 23.1199531555176 35.0
9.2333805794 5.3756685256958 36.0
8.0211882633 7.87395715713501 37.0
-8.2687492362e-10 -6.03721141815186 38.0
-8.2687492362e-10 -3.70010757446289 39.0
-11.886949603 -13.1104164123535 40.0
-24.518890588 -22.9885635375977 41.0
4.7816001544 7.0728120803833 42.0
-31.672323387 -35.900333404541 43.0
94.744469039 92.9986267089844 44.0
94.744469051 95.8764190673828 45.0
6.7767909481 6.85640478134155 46.0
1.8964593914 2.79343247413635 47.0
163.21019339 166.855224609375 48.0
157.34384812 161.147888183594 49.0
0 -2.01153326034546 50.0
68.377124448 70.8328323364258 51.0
85.180823633 79.6351699829102 52.0
86.784890544 81.0298461914062 53.0
86.784890544 77.4944152832031 54.0
86.784890544 77.8703689575195 55.0
0 4.08803749084473 56.0
76.472319113 79.5845184326172 57.0
213.81763125 222.935165405273 58.0
243.50810156 250.826385498047 59.0
22.626612781 22.0172710418701 60.0
0 2.44450354576111 61.0
147.982033057583 157.456359863281 0.0
147.982033057583 151.963317871094 1.0
146.808413813287 153.346160888672 2.0
142.01934397 150.698165893555 3.0
-1.55484310406e-09 -0.646620571613312 4.0
-1.55484309999997e-09 -2.83890986442566 5.0
-64.9693040339936 -64.3200607299805 6.0
77.6547829144777 74.1871795654297 7.0
61.2786926634778 55.4030647277832 8.0
16.3760902514777 7.20984888076782 9.0
97.5261141374777 89.7391510009766 10.0
83.6472874924777 79.6450042724609 11.0
85.2981411144777 83.8642730712891 12.0
85.2981411144777 76.4624557495117 13.0
85.2981411144777 81.5436782836914 14.0
2.3328021736e-08 -2.03496789932251 15.0
2.3328021735e-08 -1.6052393913269 16.0
5.464350236e-08 -2.72059202194214 17.0
56.2518018559467 59.0783615112305 18.0
-56.2518018797166 -57.0891418457031 19.0
-56.2518018707166 -59.2854194641113 20.0
-56.2518017707166 -58.115608215332 21.0
77.1875821073724 73.3798065185547 22.0
153.454771802806 145.008895874023 23.0
115.220596133508 115.750228881836 24.0
21.7005584165075 20.8819332122803 25.0
230.642353915167 228.381210327148 26.0
230.642353915167 229.708221435547 27.0
26.8883420391666 21.3647422790527 28.0
257.530695955167 253.23681640625 29.0
257.530695955167 255.336776733398 30.0
46.7385281475469 45.537841796875 31.0
9.28724497493877 8.4355525970459 32.0
20.9357803369809 22.6376686096191 33.0
20.9357803369809 19.0790691375732 34.0
20.935780467613 20.4264030456543 35.0
8.86054244871795 8.39873600006104 36.0
7.60121916461794 6.85336446762085 37.0
2.62465480000001e-09 1.1959273815155 38.0
2.62465480000001e-09 -1.17553234100342 39.0
-11.6034306368583 -12.5643882751465 40.0
-24.5599220844463 -26.7411251068115 41.0
4.75986482786468 1.92820739746094 42.0
-31.7734218810648 -31.6524333953857 43.0
89.4696777832616 86.4044494628906 44.0
89.4696781210649 91.9600372314453 45.0
7.05809588635643 4.443039894104 46.0
2.46335994485643 2.26466846466064 47.0
147.98203306 156.096786499023 48.0
142.01934397 152.969879150391 49.0
0 -3.74933815002441 50.0
77.654782914 76.0237503051758 51.0
97.526114137 92.0755157470703 52.0
85.298141114 81.5651702880859 53.0
85.298141114 80.2131118774414 54.0
85.298141114 79.9203643798828 55.0
5.4643502359e-08 3.07315587997437 56.0
77.187582108 79.4883346557617 57.0
230.64235391 222.583419799805 58.0
257.53069595 249.491134643555 59.0
20.935780468 19.6116352081299 60.0
2.6246548239e-09 3.64096260070801 61.0
207.33112182 151.057998657227 0.0
207.33112182 160.373062133789 1.0
206.26681161 118.335189819336 2.0
205.20177509 363.828094482422 3.0
-6.2266179105e-14 129.975357055664 4.0
-1.7242380797e-12 87.7270584106445 5.0
-79.720747063 13.1375980377197 6.0
90.113817521 96.2265014648438 7.0
57.933688317 55.4162902832031 8.0
32.180129194 44.277904510498 9.0
37.282471236 68.2573165893555 10.0
25.703815044 76.4301452636719 11.0
36.347091839 86.7657318115234 12.0
36.347091839 93.8886260986328 13.0
36.347091839 83.4907760620117 14.0
31.762962145 4.0364089012146 15.0
31.762962145 0.0255515575408936 16.0
-8.4902221915e-13 4.5118260383606 17.0
65.126256677 21.5293254852295 18.0
-65.126256697 -49.1938705444336 19.0
-65.898511807 -53.3093948364258 20.0
-65.972969217 -60.3564529418945 21.0
82.733259206 64.6162109375 22.0
151.91053792 134.790924072266 23.0
122.35766704 106.200942993164 24.0
22.605745976 23.3355350494385 25.0
234.64379713 214.928344726562 26.0
234.64379713 161.5107421875 27.0
28.665875307 16.5078525543213 28.0
263.30967245 248.261138916016 29.0
263.30967245 262.544128417969 30.0
46.542123363 39.3439865112305 31.0
9.5954381231 7.68420696258545 32.0
16.760289979 13.1359949111938 33.0
16.760289979 23.1713733673096 34.0
16.760289979 12.6223812103271 35.0
9.5567332476 10.2268171310425 36.0
8.2393408304 7.88309383392334 37.0
-6.6129727112e-12 4.61397504806519 38.0
-6.6162229666e-12 2.24675035476685 39.0
-11.41939128 -15.4827032089233 40.0
-25.032594995 -26.4645137786865 41.0
4.4565703624 4.31735038757324 42.0
-32.027102905 -32.4000282287598 43.0
91.686921833 67.3941345214844 44.0
91.686921847 120.748497009277 45.0
5.7379018289 4.17728900909424 46.0
2.1585055949 2.2637882232666 47.0
207.33112183 203.754135131836 48.0
205.2017751 138.138473510742 49.0
0 -1.336585521698 50.0
90.113817531 78.1714096069336 51.0
37.282471246 91.0510330200195 52.0
36.347091849 71.4832305908203 53.0
36.347091849 74.9620361328125 54.0
36.347091849 76.4087677001953 55.0
0 -0.0811017155647278 56.0
82.733259216 83.0273895263672 57.0
234.64379714 186.550201416016 58.0
263.30967246 251.74787902832 59.0
16.760289989 22.8362655639648 60.0
0 1.27411901950836 61.0
137.17687742 143.006134033203 0.0
137.17687742 143.046112060547 1.0
135.86355256 143.429351806641 2.0
131.0744823 135.570373535156 3.0
0 -1.89706754684448 4.0
0 0.340481162071228 5.0
-59.42861149 -59.2070846557617 6.0
69.705106679 70.1769409179688 7.0
65.43029471 62.4528312683105 8.0
4.2748119615 8.48783111572266 9.0
78.669161727 80.2065200805664 10.0
65.229460649 71.2585144042969 11.0
77.500548395 80.8047180175781 12.0
77.500548395 77.9000701904297 13.0
77.500548395 76.9198608398438 14.0
0 -0.64198762178421 15.0
0 0.28728449344635 16.0
0 -3.15485000610352 17.0
59.754945382 58.5015258789062 18.0
-59.754945394 -59.4383087158203 19.0
-59.754945372 -60.0649642944336 20.0
-59.754945384 -61.8801422119141 21.0
80.656571382 77.8813858032227 22.0
140.87903669 140.394561767578 23.0
106.41719675 107.357681274414 24.0
22.90053999 20.4083709716797 25.0
221.53560808 223.594223022461 26.0
221.53560808 221.987747192383 27.0
27.282645659 25.3551406860352 28.0
248.81825375 253.574676513672 29.0
248.81825375 252.408157348633 30.0
47.033013387 47.2299919128418 31.0
9.3131031344 9.18030452728271 32.0
20.901625993 20.4819431304932 33.0
20.901625993 21.1129970550537 34.0
20.901625993 20.8195400238037 35.0
9.3083381265 8.28715133666992 36.0
8.044767894 8.38281917572021 37.0
-2.5262790395e-09 -0.804296553134918 38.0
-2.5262785966e-09 0.113371789455414 39.0
-12.387792857 -14.5510091781616 40.0
-23.81592281 -27.5234241485596 41.0
4.9278776077 3.5162615776062 42.0
-31.108913619 -34.7394485473633 43.0
92.052766479 91.3064651489258 44.0
92.05276649 90.2343292236328 45.0
6.7273095346 3.00315618515015 46.0
2.6568330431 1.83220052719116 47.0
137.17687743 136.690536499023 48.0
131.07448231 127.224128723145 49.0
0 3.58906507492065 50.0
69.705106686 70.9878616333008 51.0
78.669161734 84.2529830932617 52.0
77.500548402 81.0479965209961 53.0
77.500548402 80.9212951660156 54.0
77.500548402 79.4541549682617 55.0
0 -0.0822644829750061 56.0
80.656571391 80.3950500488281 57.0
221.53560809 218.26057434082 58.0
248.81825375 246.469757080078 59.0
20.901626002 19.9398994445801 60.0
0 2.89454770088196 61.0
189.304518208508 170.960205078125 0.0
189.304518208508 167.69563293457 1.0
187.82453698 167.274978637695 2.0
183.03546725 183.901794433594 3.0
0 -15.534384727478 4.0
-6.9856735579e-09 6.07494783401489 5.0
-63.410598325 -62.4085960388184 6.0
72.6696130880802 80.3980102539062 7.0
43.7925234060802 53.0764694213867 8.0
28.8770896820802 27.8406219482422 9.0
87.5332455400802 84.5579071044922 10.0
75.3478096880802 77.6870422363281 11.0
78.7610531000802 83.2871170043945 12.0
78.7610531000802 79.5605316162109 13.0
78.7610531000802 83.0144119262695 14.0
0 0.913967609405518 15.0
0 0.0715634822845459 16.0
-8.2996972475e-09 0.358518540859222 17.0
55.2075620801346 58.3616905212402 18.0
-55.2075620951202 -57.9902229309082 19.0
-55.2075620731202 -56.584228515625 20.0
-55.2075620831202 -64.4827346801758 21.0
77.779042283 75.5705108642578 22.0
141.965920259135 143.463714599609 23.0
110.202033436418 109.971092224121 24.0
22.8799597924183 20.2720470428467 25.0
219.744962547016 214.326400756836 26.0
219.744962547016 212.163818359375 27.0
28.1173903310156 28.1757984161377 28.0
247.862352877016 251.879241943359 29.0
247.862352877016 249.825134277344 30.0
46.29745722 46.6723289489746 31.0
9.4363286623 9.51933860778809 32.0
22.5714802049258 20.0208301544189 33.0
22.5714802049258 22.6639537811279 34.0
22.571480205 24.005802154541 35.0
10.185368379 9.82302474975586 36.0
8.9598749127 6.81504440307617 37.0
-9.684717302e-09 2.88863563537598 38.0
-9.684717301e-09 1.16648244857788 39.0
-12.3807243210601 -8.84003734588623 40.0
-26.0749796368713 -22.7343235015869 41.0
5.29033539550013 3.5368766784668 42.0
-34.2783496834001 -33.0724830627441 43.0
108.7660487334 91.2919998168945 44.0
108.7660487334 101.049751281738 45.0
6.2330646929 6.04404783248901 46.0
2.4510011124 2.25894284248352 47.0
189.30451821 163.618545532227 48.0
183.03546726 170.559158325195 49.0
0 -3.62745809555054 50.0
72.669613088 76.8121871948242 51.0
87.53324554 90.5131759643555 52.0
78.7610531 80.9618835449219 53.0
78.7610531 80.9442367553711 54.0
78.7610531 80.0259017944336 55.0
0 4.48706579208374 56.0
77.779042288 79.1200485229492 57.0
219.74496254 220.499572753906 58.0
247.86235287 245.665054321289 59.0
22.571480208 20.6870174407959 60.0
0 3.12471985816956 61.0
171.904389231121 166.616256713867 0.0
171.904389231121 166.213485717773 1.0
170.57179709 162.469909667969 2.0
165.10716188 168.368682861328 3.0
0.39191866934 2.36859154701233 4.0
0.39191867898824 6.36576890945435 5.0
-68.3561336756982 -64.2236557006836 6.0
78.6411639729438 73.9886245727539 7.0
46.7850924509438 52.7926597595215 8.0
31.8560715229438 23.126443862915 9.0
88.4799369339438 89.8842849731445 10.0
75.7534711339438 76.729362487793 11.0
81.7694108229438 84.4457168579102 12.0
81.7694108229438 79.9000854492188 13.0
81.7694108229438 78.5475311279297 14.0
0 -1.12946653366089 15.0
0 -0.0269544124603271 16.0
-5.173094624e-09 -0.331485092639923 17.0
60.3669782016966 57.9027328491211 18.0
-60.3669782079158 -58.2627906799316 19.0
-60.3669781869158 -58.5256156921387 20.0
-60.3669781929158 -59.684196472168 21.0
81.4400825854713 77.7154769897461 22.0
145.427462812697 142.602600097656 23.0
109.693421605871 113.33984375 24.0
22.3461917368708 20.8249912261963 25.0
226.867545402243 219.398635864258 26.0
226.867545402243 223.654418945312 27.0
26.3506673472429 26.6620502471924 28.0
253.218212752243 254.722839355469 29.0
253.218212752243 257.276580810547 30.0
46.4833722894205 47.0532073974609 31.0
9.3272532778 8.6466760635376 32.0
21.0731043920947 19.0303649902344 33.0
21.0731043920947 18.4538803100586 34.0
21.0731044007393 20.0057277679443 35.0
9.3952582204 6.26320457458496 36.0
8.2148379078 7.90796852111816 37.0
-9.168946015e-09 1.63839423656464 38.0
-9.16894601400001e-09 1.38869953155518 39.0
-11.3798843634579 -17.2227268218994 40.0
-24.4111804375477 -25.2907676696777 41.0
5.02627471718653 5.53434610366821 42.0
-31.9317015352865 -33.5254783630371 43.0
98.2812755622865 96.7085266113281 44.0
98.2812755402865 94.7829437255859 45.0
6.7672017412 6.35817337036133 46.0
2.5995187044 2.94870662689209 47.0
171.90438923 166.497833251953 48.0
165.10716189 158.215087890625 49.0
0.39191867899 2.63768434524536 50.0
78.641163973 78.1464538574219 51.0
88.479936934 89.9521331787109 52.0
81.769410823 81.5143051147461 53.0
81.769410823 80.9984283447266 54.0
81.769410823 79.8731079101562 55.0
0 0.379146158695221 56.0
81.440082585 79.5886077880859 57.0
226.8675454 219.88932800293 58.0
253.21821275 249.349853515625 59.0
21.073104401 21.9800567626953 60.0
0 2.63437795639038 61.0
178.26221784 160.780715942383 0.0
178.26221784 166.832794189453 1.0
176.93820213 163.500823974609 2.0
172.14913241 168.300018310547 3.0
0 -2.88749575614929 4.0
-4.4569951226e-09 -0.0803593993186951 5.0
-62.931773827 -63.414134979248 6.0
73.1081538527101 72.7530364990234 7.0
52.3240061737101 48.4903411865234 8.0
20.7841476787101 20.2349014282227 9.0
86.1713706827101 88.0327301025391 10.0
72.8366448787101 75.6551284790039 11.0
79.8137857017101 81.6067428588867 12.0
79.8137857017101 79.0227432250977 13.0
79.8137857017101 79.6179275512695 14.0
0 -0.712208807468414 15.0
0 -0.302031457424164 16.0
-8.1385369218e-09 -1.38565325737 17.0
55.877077086 58.5077171325684 18.0
-55.8770771030652 -60.0913352966309 19.0
-55.8770770810652 -60.7361755371094 20.0
-55.8770770910652 -58.479850769043 21.0
77.477979784 78.7444915771484 22.0
145.0792666 140.275802612305 23.0
111.466747699575 112.392211914062 24.0
21.9740377495751 20.8169403076172 25.0
222.557246392358 220.375061035156 26.0
222.557246392358 222.256439208984 27.0
26.7449192693579 26.6806755065918 28.0
249.302165662358 253.390686035156 29.0
249.302165662358 253.478713989258 30.0
46.872250997 45.880485534668 31.0
9.3294542081 8.61965560913086 32.0
21.6009026988944 20.4084224700928 33.0
21.6009026988944 17.9559726715088 34.0
21.600902697 20.4492359161377 35.0
9.4711134168 7.84704303741455 36.0
8.1797954503 7.89985513687134 37.0
-9.7629943769e-09 -0.233429729938507 38.0
-9.7629943769e-09 -3.82192754745483 39.0
-11.9853726425326 -15.4433012008667 40.0
-24.944443893 -26.4558200836182 41.0
4.95325382475845 4.50580072402954 42.0
-32.5585927562584 -32.5993919372559 43.0
95.2594059692584 90.9521560668945 44.0
95.2594059593937 92.7809677124023 45.0
6.5554655389 6.64472627639771 46.0
2.4257694165 2.56620478630066 47.0
178.26221784 159.235397338867 48.0
172.14913242 160.493316650391 49.0
0 1.49891531467438 50.0
73.108153853 76.6615447998047 51.0
86.171370683 89.3563766479492 52.0
79.813785701 81.1708221435547 53.0
79.813785701 80.2668914794922 54.0
79.813785701 80.03125 55.0
0 0.979378640651703 56.0
77.47797979 80.8110504150391 57.0
222.55724639 220.991775512695 58.0
249.30216566 250.750427246094 59.0
21.600902701 21.1341400146484 60.0
0 0.754475295543671 61.0
167.398219503162 164.30517578125 0.0
167.398219503162 160.867691040039 1.0
166.21192958 157.853805541992 2.0
161.42285981 143.385589599609 3.0
0 2.46065092086792 4.0
-5.70384083999999e-09 -0.500318944454193 5.0
-54.73105681128 -58.7806091308594 6.0
65.55360645148 71.0213394165039 7.0
42.84549014848 26.5824108123779 8.0
22.70811630348 13.1857147216797 9.0
92.53702286548 88.511604309082 10.0
79.14653738348 72.1813278198242 11.0
82.01370652948 75.4453735351562 12.0
82.01370652948 79.6989288330078 13.0
82.01370652948 83.0244216918945 14.0
4.1836207177e-09 -1.79367065429688 15.0
2.4687989636e-09 -1.34356260299683 16.0
4.80564180000003e-10 0.535277366638184 17.0
56.6835814157623 57.489917755127 18.0
-56.68358148722 -59.3170585632324 19.0
-56.68358147022 -58.4264869689941 20.0
-56.68358144622 -63.1539916992188 21.0
79.2587394712108 78.0254516601562 22.0
143.664619131171 142.54426574707 23.0
109.995038246464 115.072708129883 24.0
21.8907466834636 21.3432388305664 25.0
222.923358606323 223.009765625 26.0
222.923358606323 229.539184570312 27.0
29.7463543183231 26.0443687438965 28.0
252.669712916323 245.916015625 29.0
252.669712916323 248.220855712891 30.0
46.819907914868 48.5141563415527 31.0
9.38699087039411 7.87262201309204 32.0
22.57515802484 14.8079099655151 33.0
22.57515802484 16.6938724517822 34.0
22.5751580662345 21.8195934295654 35.0
10.1633471068189 11.5607137680054 36.0
8.89275677601893 7.79754066467285 37.0
-5.99078328999999e-09 -4.22226762771606 38.0
-5.99078328999999e-09 5.9595193862915 39.0
-12.34185612011 -15.9061841964722 40.0
-26.734944673583 -25.207088470459 41.0
4.95428759069998 7.23943996429443 42.0
-34.4592230089 -32.3693923950195 43.0
93.4656577189 92.6826248168945 44.0
93.4656575549 87.560905456543 45.0
7.77480101539574 5.86192798614502 46.0
2.66136748449574 2.5920684337616 47.0
167.3982195 166.16357421875 48.0
161.42285982 152.760177612305 49.0
0 -1.19777774810791 50.0
65.553606452 75.4621047973633 51.0
92.537022865 84.0349807739258 52.0
82.01370653 79.9286804199219 53.0
82.01370653 77.5451812744141 54.0
82.01370653 80.2460021972656 55.0
4.8056418036e-10 0.532793343067169 56.0
79.258739472 80.23388671875 57.0
222.9233586 220.572250366211 58.0
252.66971292 248.091323852539 59.0
22.575158067 28.4421787261963 60.0
0 3.27370643615723 61.0
156.082093 162.214797973633 0.0
156.082093 160.569854736328 1.0
154.79700079 157.84326171875 2.0
150.00793078 150.361633300781 3.0
0 -7.53031826019287 4.0
-1.0724441244e-10 7.68039178848267 5.0
-66.37750158 -61.8665657043457 6.0
78.230632433 75.2372055053711 7.0
51.607899185 51.3782386779785 8.0
26.622733239 25.5275592803955 9.0
86.577696589 88.5163650512695 10.0
74.394369246 76.2100067138672 11.0
82.396526945 78.5201721191406 12.0
82.396526945 78.7479248046875 13.0
82.396526945 80.0785064697266 14.0
0 0.399173855781555 15.0
0 0.28646445274353 16.0
0 -0.830898106098175 17.0
58.491979535 59.7950744628906 18.0
-58.491979546 -59.38916015625 19.0
-58.491979524 -59.7015113830566 20.0
-58.491979536 -56.5009994506836 21.0
80.262375359 77.3691558837891 22.0
143.1616356 142.041641235352 23.0
107.6834575 109.393753051758 24.0
23.33670731 20.3506526947021 25.0
223.42401097 226.571258544922 26.0
223.42401097 225.745025634766 27.0
25.581073566 24.2859859466553 28.0
249.00508454 255.306991577148 29.0
249.00508454 255.668762207031 30.0
46.827142013 47.4622230529785 31.0
9.319480327 8.59048461914062 32.0
21.770395814 20.011022567749 33.0
21.770395814 18.6805019378662 34.0
21.770395813 21.0069904327393 35.0
9.5415861694 6.74989223480225 36.0
8.2112450609 7.28718757629395 37.0
-3.2105106354e-10 -0.435894191265106 38.0
-3.2105106354e-10 -1.58552122116089 39.0
-11.565285609 -17.5057582855225 40.0
-24.098123265 -24.4864330291748 41.0
5.1681339378 6.1418137550354 42.0
-31.833321194 -32.70263671875 43.0
100.79471777 98.6260604858398 44.0
100.79471779 94.2139053344727 45.0
6.9380991087 6.25279998779297 46.0
2.5851825412 2.77737593650818 47.0
156.08209301 166.277221679688 48.0
150.00793079 149.958282470703 49.0
0 -2.6074686050415 50.0
78.230632442 77.5861282348633 51.0
86.577696599 89.5719680786133 52.0
82.396526954 81.063720703125 53.0
82.396526954 79.1731033325195 54.0
82.396526954 80.0159606933594 55.0
0 1.22525072097778 56.0
80.262375369 79.4992065429688 57.0
223.42401098 222.679077148438 58.0
249.00508455 248.612365722656 59.0
21.770395823 20.8162326812744 60.0
0 0.272449195384979 61.0
150.821821044767 157.260406494141 0.0
150.821821044767 163.717010498047 1.0
149.51201117 163.032379150391 2.0
144.72294144 152.788452148438 3.0
0 0.779697835445404 4.0
-6.243268203e-09 -1.29189991950989 5.0
-59.3614038435988 -64.6644134521484 6.0
69.9021673325958 72.3038940429688 7.0
44.9092893085958 51.9030227661133 8.0
24.9928780235958 23.5916767120361 9.0
88.0796744775958 84.6522598266602 10.0
74.8871997725958 72.0070037841797 11.0
84.8673568555958 78.0580596923828 12.0
84.8673568555958 79.8587646484375 13.0
84.8673568555958 82.4039840698242 14.0
0 -1.42144227027893 15.0
0 -1.16838979721069 16.0
-3.89735760700001e-09 -1.93273425102234 17.0
58.5439442952341 59.2381706237793 18.0
-58.5439442963937 -58.8240013122559 19.0
-58.5439442753937 -59.546314239502 20.0
-58.5439442763937 -60.3496894836426 21.0
79.5589895898097 80.4983062744141 22.0
149.543323647234 144.261459350586 23.0
114.811254696543 113.141021728516 24.0
22.6746174235426 20.1010684967041 25.0
229.102313239534 228.623260498047 26.0
229.102313239534 224.030654907227 27.0
27.5192493175336 26.6806945800781 28.0
256.621562559534 255.72102355957 29.0
256.621562559534 253.14501953125 30.0
46.8549618361638 46.0376853942871 31.0
9.39631126072663 8.84011268615723 32.0
21.0150453127964 19.5434322357178 33.0
21.0150453127964 18.1887626647949 34.0
21.0150453259641 21.2907981872559 35.0
9.0464144587821 7.27692604064941 36.0
7.7613646750821 7.89680671691895 37.0
-8.78135544000001e-09 -0.982130110263824 38.0
-8.78135545e-09 -1.16997575759888 39.0
-12.0515858521969 -13.5726737976074 40.0
-25.2157212011335 -24.619556427002 41.0
4.91005240666318 5.24948692321777 42.0
-32.5973597926632 -33.66748046875 43.0
97.8074231666632 96.9871597290039 44.0
97.8074231246632 95.0500183105469 45.0
7.1343647056 6.40090560913086 46.0
2.7529340113 2.30599737167358 47.0
150.82182104 162.913131713867 48.0
144.72294145 150.35302734375 49.0
0 1.8812894821167 50.0
69.902167333 74.2294006347656 51.0
88.079674478 85.3330459594727 52.0
84.867356856 80.8157806396484 53.0
84.867356856 80.9596176147461 54.0
84.867356856 80.2074661254883 55.0
0 1.35114288330078 56.0
79.55898959 80.3739547729492 57.0
229.10231324 221.967224121094 58.0
256.62156255 249.842742919922 59.0
21.015045326 22.4666728973389 60.0
0 1.93557405471802 61.0
179.58374763 154.981033325195 0.0
179.58374763 164.433547973633 1.0
178.09586585 173.550430297852 2.0
173.30679612 162.159240722656 3.0
0 0.638141572475433 4.0
-6.9795269753e-12 -3.6782054901123 5.0
-59.189662051 -55.9802436828613 6.0
69.776022331 67.296028137207 7.0
47.475838939 52.4808235168457 8.0
22.300183382 12.1840448379517 9.0
88.821537977 88.8774795532227 10.0
75.023899228 70.5848770141602 11.0
78.866567502 85.6770172119141 12.0
78.866567502 77.1945495605469 13.0
78.866567502 78.0753402709961 14.0
0 -2.80911064147949 15.0
0 -0.294819176197052 16.0
-9.2702984087e-12 0.503480017185211 17.0
58.026308017 57.4204254150391 18.0
-58.026308037 -59.2049369812012 19.0
-58.026308015 -55.775806427002 20.0
-58.026308028 -58.9349632263184 21.0
79.719705858 75.3375701904297 22.0
140.08396526 138.524032592773 23.0
109.20985555 119.183738708496 24.0
24.393135165 22.033519744873 25.0
219.80367113 229.27326965332 26.0
219.80367113 225.539123535156 27.0
25.878908804 28.4840831756592 28.0
245.68257995 254.022094726562 29.0
245.68257995 256.024749755859 30.0
46.322755174 45.0152359008789 31.0
9.2073720392 8.82196521759033 32.0
21.69339782 20.5651607513428 33.0
21.69339782 14.8337736129761 34.0
21.69339782 21.166561126709 35.0
8.9053135346 13.4904985427856 36.0
7.6967992005 7.88865566253662 37.0
-1.9985095519e-11 -7.89371776580811 38.0
-1.9985094968e-11 -1.83516716957092 39.0
-12.532187562 -19.097936630249 40.0
-25.280158855 -27.7370700836182 41.0
4.9637216629 2.32139158248901 42.0
-32.883775836 -29.6237964630127 43.0
84.32784665 83.1050796508789 44.0
84.327846663 91.8439788818359 45.0
6.4625884563 8.64734268188477 46.0
2.556705248 2.88617825508118 47.0
179.58374764 160.66943359375 48.0
173.30679613 166.356796264648 49.0
0 -1.43274211883545 50.0
69.776022341 63.4874839782715 51.0
88.821537987 83.5063095092773 52.0
78.866567512 80.5438385009766 53.0
78.866567512 81.4388961791992 54.0
78.866567512 81.5377197265625 55.0
0 4.49287080764771 56.0
79.719705868 79.9958419799805 57.0
219.80367114 222.408248901367 58.0
245.68257996 255.153945922852 59.0
21.69339783 24.4930152893066 60.0
0 0.205057203769684 61.0
196.31403567 166.513565063477 0.0
196.31403567 163.632507324219 1.0
195.07511775 173.14990234375 2.0
190.28604803 184.374450683594 3.0
0 -0.112115919589996 4.0
-3.9842847002e-14 0.70617550611496 5.0
-61.755705255 -54.8226432800293 6.0
74.33074435 78.7782897949219 7.0
48.135322234 56.1127243041992 8.0
26.195422106 24.1540985107422 9.0
97.845664129 92.6463165283203 10.0
85.644134651 78.8717193603516 11.0
87.868061031 82.4380187988281 12.0
87.868061031 75.8274536132812 13.0
87.868061031 80.3923492431641 14.0
0 0.10986340045929 15.0
0 0.494659662246704 16.0
-7.5677341987e-14 2.7772068977356 17.0
53.958632454 56.3520812988281 18.0
-53.958632474 -58.7212104797363 19.0
-53.958632452 -60.7751960754395 20.0
-53.958632465 -60.1841773986816 21.0
74.005804484 78.3452301025391 22.0
143.86339625 143.884002685547 23.0
107.73112142 113.381958007812 24.0
22.276499509 19.7274703979492 25.0
217.86920074 229.944107055664 26.0
217.86920074 225.497573852539 27.0
28.270007267 22.215259552002 28.0
246.13920802 253.933853149414 29.0
246.13920802 249.125900268555 30.0
46.414903933 45.3497505187988 31.0
9.2986945269 8.89930629730225 32.0
20.04717201 22.5094509124756 33.0
20.04717201 17.3878841400146 34.0
20.04717201 17.322624206543 35.0
10.185917238 9.45906352996826 36.0
8.8033050615 7.84967231750488 37.0
-1.148898725e-13 1.59054923057556 38.0
-1.1488987273e-13 -1.73623752593994 39.0
-10.534834092 -11.5413331985474 40.0
-24.071630538 -24.4835777282715 41.0
4.6947731847 6.05080461502075 42.0
-31.336128039 -33.7988014221191 43.0
102.76831072 93.1435012817383 44.0
102.76831074 91.7321624755859 45.0
7.103359256 7.71426486968994 46.0
2.6180599651 1.97687458992004 47.0
196.31403568 165.46257019043 48.0
190.28604804 176.148773193359 49.0
0 11.6204395294189 50.0
74.33074436 78.6569213867188 51.0
97.845664139 92.2472686767578 52.0
87.868061041 80.4715957641602 53.0
87.868061041 79.2366790771484 54.0
87.868061041 79.3248672485352 55.0
0 -0.714002430438995 56.0
74.005804494 79.7049865722656 57.0
217.86920075 225.453018188477 58.0
246.13920803 249.588043212891 59.0
20.04717202 21.3780651092529 60.0
0 2.46380305290222 61.0
156.6713181 155.580200195312 0.0
156.6713181 153.993469238281 1.0
155.35669501 163.743011474609 2.0
150.56762529 171.040649414062 3.0
0 0.796224057674408 4.0
-8.2508583107e-11 3.01405358314514 5.0
-64.361672988 -64.9945755004883 6.0
77.036033037 71.0951232910156 7.0
58.131839861 44.9297409057617 8.0
18.904193166 20.0676231384277 9.0
93.396596857 85.5535125732422 10.0
81.116172554 75.229362487793 11.0
86.896080413 81.2087097167969 12.0
86.896080413 81.3138809204102 13.0
86.896080413 75.3482513427734 14.0
0 2.50444602966309 15.0
0 -0.301564037799835 16.0
-1.6017837051e-10 2.54906129837036 17.0
59.672261496 59.9125785827637 18.0
-59.672261517 -58.1255798339844 19.0
-59.672261495 -59.941967010498 20.0
-59.672261507 -56.4147453308105 21.0
81.072120565 72.6730422973633 22.0
146.92085713 143.486389160156 23.0
113.41120244 114.981101989746 24.0
23.793852767 22.2632541656494 25.0
227.9929777 219.011138916016 26.0
227.9929777 223.71435546875 27.0
29.555613189 29.4188594818115 28.0
257.5485909 247.107376098633 29.0
257.5485909 253.453430175781 30.0
46.498981697 44.3253707885742 31.0
9.3093496483 9.8873176574707 32.0
21.399859049 16.7217254638672 33.0
21.399859049 17.6967658996582 34.0
21.399859049 22.8812217712402 35.0
9.8833975679 12.6430921554565 36.0
8.6650482972 7.1448392868042 37.0
-2.6223804236e-10 -0.434596121311188 38.0
-2.6223832629e-10 2.10137701034546 39.0
-11.425641458 -9.90792179107666 40.0
-24.316442634 -29.3452816009521 41.0
5.1309119045 1.98286557197571 42.0
-32.120470683 -32.5556640625 43.0
83.234212436 85.1669235229492 44.0
83.234212448 88.3465042114258 45.0
6.9774120402 9.73092460632324 46.0
2.6592941 3.21241331100464 47.0
156.67131811 162.748138427734 48.0
150.5676253 153.104629516602 49.0
0 -4.18636703491211 50.0
77.036033047 73.7664031982422 51.0
93.396596867 88.487174987793 52.0
86.896080422 81.3423767089844 53.0
86.896080422 77.584358215332 54.0
86.896080422 79.9667739868164 55.0
0 -1.66871166229248 56.0
81.072120575 77.192626953125 57.0
227.99297771 219.784393310547 58.0
257.54859091 251.383544921875 59.0
21.399859059 21.5899105072021 60.0
0 2.1902916431427 61.0
173.48222006 161.578262329102 0.0
173.48222006 163.331939697266 1.0
172.22314636 167.266174316406 2.0
167.43407664 166.624298095703 3.0
0 -3.36571025848389 4.0
0 2.25831246376038 5.0
-60.925447999 -61.7270164489746 6.0
71.757340088 76.3736953735352 7.0
53.963103693 50.3312759399414 8.0
17.794236388 23.8486480712891 9.0
93.275422472 84.7403182983398 10.0
80.514570873 74.4210586547852 11.0
84.707128687 80.9948196411133 12.0
84.707128687 78.0986328125 13.0
84.707128687 76.8569869995117 14.0
0 -1.82194638252258 15.0
0 -0.0762017369270325 16.0
0 -0.500680029392242 17.0
62.017509075 59.6797676086426 18.0
-62.01750909 -59.3171920776367 19.0
-62.017509068 -60.0256080627441 20.0
-62.01750908 -60.1812858581543 21.0
83.758171021 78.2418975830078 22.0
142.62006388 139.097549438477 23.0
109.98802284 115.773361206055 24.0
22.448442191 20.7120990753174 25.0
226.37823492 228.323150634766 26.0
226.37823492 225.910629272461 27.0
26.22733819 27.6501235961914 28.0
252.60557311 254.049514770508 29.0
252.60557311 253.372268676758 30.0
46.753014065 44.893009185791 31.0
9.3756517084 9.05110168457031 32.0
21.740661938 20.3500671386719 33.0
21.740661938 22.270414352417 34.0
21.740661937 19.051908493042 35.0
9.3190110995 7.99687623977661 36.0
8.085393261 7.12882471084595 37.0
-3.0965105066e-09 3.13680648803711 38.0
-3.0965105777e-09 -1.33643507957458 39.0
-11.427791127 -14.1493148803711 40.0
-23.871954779 -24.8282051086426 41.0
5.1234589867 3.50647449493408 42.0
-31.593364576 -32.6555023193359 43.0
91.97137263 90.3484420776367 44.0
91.971372637 94.119010925293 45.0
6.7503169777 7.88937902450562 46.0
2.6459629266 2.58006477355957 47.0
173.48222007 157.88493347168 48.0
167.43407665 163.731369018555 49.0
0 0.578079879283905 50.0
71.757340094 76.4790191650391 51.0
93.275422478 87.3532485961914 52.0
84.707128693 80.56494140625 53.0
84.707128693 80.1917190551758 54.0
84.707128693 78.89599609375 55.0
0 1.92142558097839 56.0
83.75817103 79.4292144775391 57.0
226.37823492 221.616287231445 58.0
252.60557312 250.850006103516 59.0
21.740661945 17.7413444519043 60.0
0 1.75033211708069 61.0
135.23410079 169.788604736328 0.0
135.23410079 166.591918945312 1.0
133.95118823 131.948928833008 2.0
129.1621185 148.057968139648 3.0
0 5.20253276824951 4.0
-8.3907282495e-12 11.7939386367798 5.0
-69.249145189 -51.6119041442871 6.0
81.953309087 89.5547409057617 7.0
54.332931979 58.1504592895508 8.0
27.620377098 28.1583499908447 9.0
87.876710723 87.9928970336914 10.0
75.327632932 71.9663543701172 11.0
80.872710005 79.3334503173828 12.0
80.872710005 81.684455871582 13.0
80.872710005 81.6591110229492 14.0
0 -0.383741915225983 15.0
0 0.362612307071686 16.0
0 0.337352693080902 17.0
54.231342039 58.3456344604492 18.0
-54.231342059 -59.1166954040527 19.0
-54.231342037 -60.2064094543457 20.0
-54.231342049 -62.4636573791504 21.0
76.823953231 79.9689407348633 22.0
150.48951071 141.773864746094 23.0
114.2326215 113.729446411133 24.0
22.245216441 19.4365348815918 25.0
227.31346395 227.807006835938 26.0
227.31346395 228.513565063477 27.0
28.991035429 32.847339630127 28.0
256.30449939 253.661148071289 29.0
256.30449939 251.939041137695 30.0
46.944538783 45.5591087341309 31.0
9.3892131455 8.86161422729492 32.0
22.592611172 19.2100582122803 33.0
22.592611172 18.3906383514404 34.0
22.592611172 20.2865562438965 35.0
10.046873714 10.434193611145 36.0
8.7358599932 6.76989555358887 37.0
-2.5728122599e-11 0.883245170116425 38.0
-2.5720825822e-11 2.39985370635986 39.0
-10.997012032 -16.3521690368652 40.0
-22.984366475 -21.885555267334 41.0
5.5475765567 1.52570652961731 42.0
-31.06137093 -30.8829021453857 43.0
89.757376337 90.1653900146484 44.0
89.75737635 92.5040512084961 45.0
7.3919800441 12.7606048583984 46.0
2.7595251952 3.69848227500916 47.0
135.2341008 166.703994750977 48.0
129.16211851 98.2080535888672 49.0
0 -0.095609724521637 50.0
81.953309097 74.2945404052734 51.0
87.876710733 88.3643493652344 52.0
80.872710015 80.6881484985352 53.0
80.872710015 77.3513870239258 54.0
80.872710015 79.3991622924805 55.0
0 -0.992567121982574 56.0
76.823953241 81.0624237060547 57.0
227.31346396 223.354110717773 58.0
256.3044994 252.875762939453 59.0
22.592611182 20.5984325408936 60.0
0 2.09887385368347 61.0
153.3839496 169.550277709961 0.0
153.3839496 163.974609375 1.0
152.14714636 152.869018554688 2.0
147.35806642 142.920074462891 3.0
0 -0.349619448184967 4.0
0 0.376686573028564 5.0
-69.830181494 -67.1229705810547 6.0
81.84293429 70.8216781616211 7.0
51.609788383 53.0792350769043 8.0
30.233145897 23.5586624145508 9.0
88.949097643 84.9171676635742 10.0
75.989783371 78.7946472167969 11.0
80.549787033 78.1467056274414 12.0
80.549787033 79.5270309448242 13.0
80.549787033 80.7845077514648 14.0
0 -1.31488752365112 15.0
0 0.0854939818382263 16.0
0 1.60049486160278 17.0
57.680770365 56.9308776855469 18.0
-57.680770375 -58.4602355957031 19.0
-57.680770354 -57.7262916564941 20.0
-57.680770366 -59.2174644470215 21.0
79.131522101 79.1325531005859 22.0
150.99711534 143.231994628906 23.0
114.56763406 109.449287414551 24.0
23.495462498 20.9317455291748 25.0
230.12863745 223.45475769043 26.0
230.12863745 223.984512329102 27.0
25.116445817 27.5430030822754 28.0
255.24508327 255.467163085938 29.0
255.24508327 254.276016235352 30.0
46.670804864 46.0799789428711 31.0
9.3094619135 8.88829803466797 32.0
21.450751725 20.2042446136475 33.0
21.450751725 19.3287105560303 34.0
21.450751725 21.501745223999 35.0
9.6527967316 8.18502902984619 36.0
8.3376182378 7.49669218063354 37.0
-1.2511595721e-10 -4.73814058303833 38.0
-1.2485817063e-10 1.04703307151794 39.0
-11.547466424 -17.3145923614502 40.0
-23.971401994 -25.2851810455322 41.0
5.364714783 5.56867933273315 42.0
-31.850556068 -33.4314270019531 43.0
95.939698522 92.4681777954102 44.0
95.939698535 93.9376831054688 45.0
6.8161404872 7.26275014877319 46.0
2.5693554774 2.24136447906494 47.0
153.38394961 164.243041992188 48.0
147.35806643 138.976516723633 49.0
0 0.754528880119324 50.0
81.8429343 75.1002807617188 51.0
88.949097653 83.5122756958008 52.0
80.549787043 79.149040222168 53.0
80.549787043 80.9481964111328 54.0
80.549787043 79.748664855957 55.0
0 0.223032653331757 56.0
79.131522111 79.8990859985352 57.0
230.12863746 221.391220092773 58.0
255.24508328 252.145050048828 59.0
21.450751735 21.2096195220947 60.0
0 2.07180380821228 61.0
145.697862231535 156.984512329102 0.0
145.697862231535 148.199111938477 1.0
144.10927649 153.831024169922 2.0
139.32020675 149.446304321289 3.0
0 -3.38298892974854 4.0
-6.243418611e-09 0.762799501419067 5.0
-66.5229315429982 -73.4125366210938 6.0
80.3029777169938 75.8588790893555 7.0
57.3882814959938 32.2454833984375 8.0
22.9146962209938 9.18483066558838 9.0
78.7261909489938 97.0041122436523 10.0
64.7650111909938 78.0491027832031 11.0
70.4032795089938 83.0165176391602 12.0
70.4032795089938 71.7395401000977 13.0
70.4032795089938 73.2057418823242 14.0
0 3.58913087844849 15.0
0 0.624194502830505 16.0
-4.143583128e-09 6.37584733963013 17.0
65.4479607361784 61.1800270080566 18.0
-65.4479607359907 -57.8827323913574 19.0
-65.4479607149907 -50.2868537902832 20.0
-65.4479607159907 -59.5731811523438 21.0
85.4590124222956 78.4349136352539 22.0
138.887316014178 148.617279052734 23.0
107.741617360833 114.738731384277 24.0
17.1213624897231 20.7439556121826 25.0
224.34632843307 236.537902832031 26.0
224.34632843307 224.347763061523 27.0
28.9637771710698 27.0954780578613 28.0
253.31010560307 255.821624755859 29.0
253.31010560307 250.243286132812 30.0
46.3814199465322 48.1454772949219 31.0
9.5481516920282 12.3198108673096 32.0
20.0110517059947 13.9491481781006 33.0
20.0110517059947 17.147144317627 34.0
20.0110517200323 18.952657699585 35.0
8.56330740328766 6.47381591796875 36.0
7.43978935018766 6.38044309616089 37.0
-8.76161119999999e-09 -1.05077743530273 38.0
-8.7616112e-09 1.08951282501221 39.0
-13.7631795299954 -13.1583013534546 40.0
-25.078191361285 -26.0287914276123 41.0
5.40894745029484 2.79352045059204 42.0
-33.2475495079948 -32.8999671936035 43.0
101.510070149995 97.4493179321289 44.0
101.510070099995 95.9318695068359 45.0
5.759187568 4.34534931182861 46.0
1.3291043141 1.82240223884583 47.0
145.69786224 153.956130981445 48.0
139.32020676 149.490234375 49.0
0 1.87250423431396 50.0
80.302977717 76.8250885009766 51.0
78.726190949 84.0516815185547 52.0
70.403279509 77.0247039794922 53.0
70.403279509 78.2080001831055 54.0
70.403279509 78.0674743652344 55.0
0 6.8220386505127 56.0
85.459012422 80.2134246826172 57.0
224.34632843 216.688812255859 58.0
253.31010561 246.05192565918 59.0
20.01105172 24.6805572509766 60.0
0 1.81334662437439 61.0
193.620195423445 166.634338378906 0.0
193.620195423445 165.151641845703 1.0
192.51512681 163.648574829102 2.0
187.72605706 175.637313842773 3.0
-2.3135613719e-10 1.25907754898071 4.0
-8.647187154e-09 1.55174970626831 5.0
-71.41837288423 -75.5393753051758 6.0
84.4565912553051 71.6154327392578 7.0
54.8692068283051 30.6962127685547 8.0
29.5873844273051 20.8891258239746 9.0
86.4911072633051 85.8573989868164 10.0
73.6765223503051 81.8769683837891 11.0
75.0946964873051 77.241943359375 12.0
75.0946964873051 76.8628387451172 13.0
75.0946964873051 79.9015731811523 14.0
2.6469896651e-09 -4.75218296051025 15.0
0 0.401790797710419 16.0
1.87899875000001e-09 7.24826908111572 17.0
57.2199505702574 54.1835556030273 18.0
-57.2199505459577 -56.4330558776855 19.0
-57.2199505279577 -58.5358848571777 20.0
-57.2199505139577 -58.7397308349609 21.0
77.7338775470825 80.9988098144531 22.0
153.723823269257 144.229385375977 23.0
119.612085404072 111.542259216309 24.0
22.1670579150717 23.6703205108643 25.0
231.457700806891 225.893768310547 26.0
231.457700806891 225.304977416992 27.0
27.6857023538906 20.6725406646729 28.0
259.143403166891 246.940063476562 29.0
259.143403166891 251.32763671875 30.0
46.1791815442054 48.2974662780762 31.0
9.33613497328532 9.06848526000977 32.0
20.5139270336901 25.5485877990723 33.0
20.5139270336901 25.0695781707764 34.0
20.5139270633121 23.332218170166 35.0
9.35858819441393 10.3592844009399 36.0
8.04134756511393 7.78337001800537 37.0
-7.13341441000003e-09 0.731598198413849 38.0
-7.13341440999998e-09 -5.06418037414551 39.0
-11.5848948084789 -13.2671432495117 40.0
-24.4229379770874 -25.1310577392578 41.0
5.62421696768762 6.40897083282471 42.0
-32.6573894150876 -34.5704574584961 43.0
91.6286594930876 91.076416015625 44.0
91.6286593770876 93.1831359863281 45.0
6.5093336585 11.8331708908081 46.0
2.2247797827 3.34036993980408 47.0
193.62019542 174.051284790039 48.0
187.72605707 165.517044067383 49.0
0 -5.14651966094971 50.0
84.456591256 71.5327835083008 51.0
86.491107264 81.328742980957 52.0
75.094696488 79.8986358642578 53.0
75.094696488 79.7839431762695 54.0
75.094696488 80.1530303955078 55.0
1.8789987524e-09 5.04774045944214 56.0
77.733877547 77.833869934082 57.0
231.45770081 221.005035400391 58.0
259.14340317 247.586334228516 59.0
20.513927063 16.7871913909912 60.0
0 0.494023025035858 61.0
147.982033057583 157.456344604492 0.0
147.982033057583 151.962463378906 1.0
146.808413813287 153.339401245117 2.0
142.01934397 150.649459838867 3.0
-1.55484310406e-09 -0.640832006931305 4.0
-1.55484309999997e-09 -2.83684945106506 5.0
-64.9693040339936 -64.3180236816406 6.0
77.6547829144777 74.187141418457 7.0
61.2786926634778 55.4030647277832 8.0
16.3760902514777 7.20984888076782 9.0
97.5261141374777 89.7391510009766 10.0
83.6472874924777 79.6450042724609 11.0
85.2981411144777 83.8642730712891 12.0
85.2981411144777 76.4624557495117 13.0
85.2981411144777 81.5436782836914 14.0
2.3328021736e-08 -2.03496789932251 15.0
2.3328021735e-08 -1.6052393913269 16.0
5.464350236e-08 -2.72059202194214 17.0
56.2518018559467 59.0783615112305 18.0
-56.2518018797166 -57.0891418457031 19.0
-56.2518018707166 -59.2854194641113 20.0
-56.2518017707166 -58.115608215332 21.0
77.1875821073724 73.3798065185547 22.0
153.454771802806 145.008895874023 23.0
115.220596133508 115.750228881836 24.0
21.7005584165075 20.8819332122803 25.0
230.642353915167 228.381210327148 26.0
230.642353915167 229.708221435547 27.0
26.8883420391666 21.3647422790527 28.0
257.530695955167 253.23681640625 29.0
257.530695955167 255.336776733398 30.0
46.7385281475469 45.537841796875 31.0
9.28724497493877 8.4355525970459 32.0
20.9357803369809 22.6376686096191 33.0
20.9357803369809 19.0790691375732 34.0
20.935780467613 20.4264030456543 35.0
8.86054244871795 8.39873600006104 36.0
7.60121916461794 6.85336446762085 37.0
2.62465480000001e-09 1.1959273815155 38.0
2.62465480000001e-09 -1.17553234100342 39.0
-11.6034306368583 -12.5643882751465 40.0
-24.5599220844463 -26.7411251068115 41.0
4.75986482786468 1.92820739746094 42.0
-31.7734218810648 -31.6524333953857 43.0
89.4696777832616 86.4044494628906 44.0
89.4696781210649 91.9600372314453 45.0
7.05809588635643 4.443039894104 46.0
2.46335994485643 2.26466846466064 47.0
147.98203306 156.097106933594 48.0
142.01934397 152.929565429688 49.0
0 -3.75245094299316 50.0
77.654782914 76.0238418579102 51.0
97.526114137 92.0755157470703 52.0
85.298141114 81.5651702880859 53.0
85.298141114 80.2131118774414 54.0
85.298141114 79.9203643798828 55.0
5.4643502359e-08 3.07315587997437 56.0
77.187582108 79.4883346557617 57.0
230.64235391 222.583419799805 58.0
257.53069595 249.491134643555 59.0
20.935780468 19.6116352081299 60.0
2.6246548239e-09 3.64096260070801 61.0
205.575624893706 168.632400512695 0.0
205.575624894812 191.117858886719 1.0
204.247242249085 142.328414916992 2.0
198.772367854326 151.130233764648 3.0
27.1441717093241 0.854316651821136 4.0
27.144171709362 2.4410092830658 5.0
-51.805385096498 -61.219066619873 6.0
60.546244953037 81.0194931030273 7.0
39.534298563742 28.4202938079834 8.0
21.011946389317 17.0011806488037 9.0
71.79145322161 86.0324859619141 10.0
60.882191959015 74.0307388305664 11.0
76.269907110256 76.1554183959961 12.0
76.269907110519 86.6791915893555 13.0
76.269907110315 76.8275299072266 14.0
0.0061500074131 -0.0162836313247681 15.0
0.0061500074139 1.64983510971069 16.0
0.005869413351 4.77681350708008 17.0
48.571390523334 53.7779655456543 18.0
-48.55086590424 -51.4713325500488 19.0
-48.55255688308 -47.2903900146484 20.0
-48.54656708698 -54.5618782043457 21.0
69.513993820783 83.0792007446289 22.0
133.619297272011 143.934173583984 23.0
105.064876930952 100.400032043457 24.0
23.135164168772 20.4033012390137 25.0
203.133291095898 222.641540527344 26.0
203.133291087927 229.656646728516 27.0
22.640948950171 28.1990756988525 28.0
225.774240045199 253.793579101562 29.0
225.774240043429 252.587753295898 30.0
46.082158763251 46.6269187927246 31.0
9.204063416754 8.35661315917969 32.0
20.967426733831 21.4675197601318 33.0
20.967426733366 20.7275466918945 34.0
20.983028241445 22.3625431060791 35.0
8.98076094466 9.37813091278076 36.0
7.831268689316 8.69820117950439 37.0
0.00244885128 -39.1440773010254 38.0
0.00244885128 2.58822178840637 39.0
-13.508783391732 -5.12524938583374 40.0
-26.767726376461 -24.7929458618164 41.0
5.776954132834 7.14891910552979 42.0
-35.173629577678 -35.7819404602051 43.0
74.371568051357 88.8954086303711 44.0
74.2886007206808 69.5551071166992 45.0
6.4605393090326 4.42808485031128 46.0
2.725247265615 1.57077050209045 47.0
205.57562489 173.76416015625 48.0
198.77236785 125.675994873047 49.0
27.144171709 -0.214078009128571 50.0
60.546244953 76.5798263549805 51.0
71.791453222 87.4036865234375 52.0
76.26990711 79.3963470458984 53.0
76.26990711 67.4430389404297 54.0
76.26990711 69.9981460571289 55.0
0.0058694133509 -2.68988084793091 56.0
69.513993821 80.5761337280273 57.0
203.13329109 224.982849121094 58.0
225.77424004 245.844741821289 59.0
20.983028242 22.6592578887939 60.0
0.0024488512809 3.90898275375366 61.0
169.59521971 166.903472900391 0.0
169.59521971 167.952621459961 1.0
168.40750865 163.285568237305 2.0
163.61843893 157.098052978516 3.0
0 3.38961672782898 4.0
-1.0257165379e-11 2.6569652557373 5.0
-70.321732936 -60.4683456420898 6.0
82.089794872 73.3115844726562 7.0
54.330636871 51.3781890869141 8.0
27.759157991 23.9983501434326 9.0
88.78690474 87.6086578369141 10.0
75.402045057 74.8118591308594 11.0
83.928933159 82.4956665039062 12.0
83.928933159 76.9270401000977 13.0
83.928933159 80.3449935913086 14.0
0 -3.67463779449463 15.0
0 -1.4483757019043 16.0
-2.1669919121e-11 0.772549569606781 17.0
58.815983541 59.0329933166504 18.0
-58.815983561 -59.9689140319824 19.0
-58.81598354 -60.5350875854492 20.0
-58.815983552 -60.1198883056641 21.0
80.900083637 80.1489562988281 22.0
143.9215244 143.269439697266 23.0
108.05135683 110.762603759766 24.0
21.35988456 20.9867553710938 25.0
224.82160804 226.399627685547 26.0
224.82160804 223.642379760742 27.0
25.59095257 26.1356468200684 28.0
250.41256062 255.767181396484 29.0
250.41256062 255.07958984375 30.0
46.85679736 46.1918640136719 31.0
9.3675405612 9.0810022354126 32.0
22.084100076 19.4306564331055 33.0
22.084100076 20.284818649292 34.0
22.084100075 21.1912212371826 35.0
9.5808631785 6.29204130172729 36.0
8.323516867 7.22706842422485 37.0
-3.4286220901e-11 -1.32654714584351 38.0
-3.4286220897e-11 -1.3523313999176 39.0
-11.689962069 -14.0800380706787 40.0
-24.663619272 -22.3693408966064 41.0
5.1585708628 7.37772130966187 42.0
-32.249757575 -34.2683563232422 43.0
94.922983432 98.1433868408203 44.0
94.922983445 94.3908157348633 45.0
6.8346963969 5.72695112228394 46.0
2.5160545979 2.96147537231445 47.0
169.59521972 166.615036010742 48.0
163.61843894 156.794052124023 49.0
0 -0.0453022718429565 50.0
82.089794882 77.5471343994141 51.0
88.786904749 84.2161712646484 52.0
83.928933169 80.9061660766602 53.0
83.928933169 79.7558822631836 54.0
83.928933169 80.2895889282227 55.0
0 4.15162324905396 56.0
80.900083647 79.5766830444336 57.0
224.82160805 222.525970458984 58.0
250.41256063 250.849609375 59.0
22.084100085 21.9497013092041 60.0
0 2.85353088378906 61.0
172.02213121 161.764770507812 0.0
172.02213121 164.366455078125 1.0
170.7883793 164.787902832031 2.0
165.99930904 164.08235168457 3.0
0 0.310557305812836 4.0
-9.3459673047e-10 2.69911909103394 5.0
-64.035259673 -60.4401397705078 6.0
75.887396505 76.3105926513672 7.0
49.476505928 50.0294952392578 8.0
26.410890571 26.5510501861572 9.0
94.046571642 85.8565216064453 10.0
81.64523765 74.6103973388672 11.0
83.565014333 80.4717483520508 12.0
83.565014333 80.2432174682617 13.0
83.565014333 79.9203948974609 14.0
0 -0.256983935832977 15.0
0 -0.382390797138214 16.0
0 0.0277670621871948 17.0
57.761914656 59.6808204650879 18.0
-57.76191467 -59.605052947998 19.0
-57.761914648 -61.183219909668 20.0
-57.76191466 -57.1807250976562 21.0
79.724846677 78.7864456176758 22.0
144.71878658 144.886840820312 23.0
110.9180245 111.262992858887 24.0
21.999439314 19.204345703125 25.0
224.44363327 226.009780883789 26.0
224.44363327 227.127990722656 27.0
27.865304028 25.346155166626 28.0
252.3089373 254.804946899414 29.0
252.3089373 253.649536132812 30.0
46.73624401 47.5324783325195 31.0
9.4090615128 8.8231840133667 32.0
21.962932016 19.5459041595459 33.0
21.962932016 24.9945812225342 34.0
21.962932015 20.2880802154541 35.0
9.3739570869 8.48083114624023 36.0
8.0884356671 7.65444946289062 37.0
-3.5219912705e-09 1.37312960624695 38.0
-3.5219885485e-09 -2.80533361434937 39.0
-12.123212156 -13.5793170928955 40.0
-25.486362932 -25.8559169769287 41.0
5.1648775273 4.0712776184082 42.0
-33.065543994 -32.4094886779785 43.0
99.671191233 93.0017318725586 44.0
99.671191243 93.8151168823242 45.0
6.9384615782 6.5172643661499 46.0
2.5751176841 2.2311577796936 47.0
172.02213122 165.347198486328 48.0
165.99930905 160.801681518555 49.0
0 1.15293717384338 50.0
75.887396511 76.8672256469727 51.0
94.046571648 89.5045471191406 52.0
83.565014338 80.6000518798828 53.0
83.565014338 81.1125259399414 54.0
83.565014338 80.4337387084961 55.0
0 1.10032272338867 56.0
79.724846686 79.3628692626953 57.0
224.44363327 220.408233642578 58.0
252.30893731 247.758163452148 59.0
21.962932023 15.9107732772827 60.0
0 1.13124322891235 61.0
151.230920889189 159.380111694336 0.0
151.230920889189 153.250610351562 1.0
150.04163867 148.534057617188 2.0
145.25256894 145.982559204102 3.0
0 2.10366487503052 4.0
-5.7258573836e-09 10.2604665756226 5.0
-74.092081416 -44.5838623046875 6.0
86.4502074282606 78.2317504882812 7.0
55.6571452602605 51.3968353271484 8.0
30.7930621672605 19.0710697174072 9.0
94.6581067912606 87.7251968383789 10.0
82.6091826202606 76.2432098388672 11.0
88.0578862432606 79.2166290283203 12.0
88.0578862432606 81.8155822753906 13.0
88.0578862432606 82.5050354003906 14.0
0 0.724687814712524 15.0
0 0.654193878173828 16.0
-8.873524319e-09 1.42933678627014 17.0
51.7378696516438 57.6933746337891 18.0
-51.7378696798908 -47.1553649902344 19.0
-51.7378696578908 -58.7985877990723 20.0
-51.7378696678908 -59.9187278747559 21.0
73.247757852 80.1811599731445 22.0
145.888571589778 139.590484619141 23.0
113.500663277223 109.238891601562 24.0
24.9956885872231 22.5794506072998 25.0
219.136329438378 221.409622192383 26.0
219.136329438378 228.380966186523 27.0
29.7970302393776 27.4348621368408 28.0
248.933359678378 261.354705810547 29.0
248.933359678378 250.328369140625 30.0
46.170024825 54.0310134887695 31.0
9.3198022861 9.39995861053467 32.0
21.5098881889376 20.9211845397949 33.0
21.5098881889376 22.8536911010742 34.0
21.509888188 25.0573768615723 35.0
9.4290218969 14.1020479202271 36.0
8.0681923533 6.81961441040039 37.0
-9.673544458e-09 -4.42902374267578 38.0
-9.673544458e-09 0.724046766757965 39.0
-12.5007048479454 -15.0703735351562 40.0
-26.111296934 -28.1043434143066 41.0
4.69707399948379 3.61745882034302 42.0
-33.4822639523838 -32.7636222839355 43.0
90.1863208144408 96.8077392578125 44.0
90.1863208463838 95.6390838623047 45.0
6.5594383958 8.99624919891357 46.0
2.3685398472 2.0243604183197 47.0
151.23092088 168.184616088867 48.0
145.25256895 138.336273193359 49.0
0 4.08967733383179 50.0
86.450207428 77.2539596557617 51.0
94.658106792 81.4582366943359 52.0
88.057886243 79.9287872314453 53.0
88.057886243 72.3237457275391 54.0
88.057886243 79.7825164794922 55.0
0 1.97722864151001 56.0
73.247757857 80.1723556518555 57.0
219.13632944 219.170135498047 58.0
248.93335968 250.279113769531 59.0
21.509888192 22.4158344268799 60.0
0 0.0970513224601746 61.0
137.17687742 143.006530761719 0.0
137.17687742 143.030090332031 1.0
135.86355256 143.542068481445 2.0
131.0744823 135.778945922852 3.0
0 -2.12998342514038 4.0
0 0.342118084430695 5.0
-59.42861149 -59.2364273071289 6.0
69.705106679 70.1773300170898 7.0
65.43029471 62.4528312683105 8.0
4.2748119615 8.48783111572266 9.0
78.669161727 80.2065200805664 10.0
65.229460649 71.2585144042969 11.0
77.500548395 80.8047180175781 12.0
77.500548395 77.9000701904297 13.0
77.500548395 76.9198608398438 14.0
0 -0.64198762178421 15.0
0 0.28728449344635 16.0
0 -3.15485000610352 17.0
59.754945382 58.5015258789062 18.0
-59.754945394 -59.4383087158203 19.0
-59.754945372 -60.0649642944336 20.0
-59.754945384 -61.8801422119141 21.0
80.656571382 77.8813858032227 22.0
140.87903669 140.394561767578 23.0
106.41719675 107.357681274414 24.0
22.90053999 20.4083709716797 25.0
221.53560808 223.594223022461 26.0
221.53560808 221.987747192383 27.0
27.282645659 25.3551406860352 28.0
248.81825375 253.574676513672 29.0
248.81825375 252.408157348633 30.0
47.033013387 47.2299919128418 31.0
9.3131031344 9.18030452728271 32.0
20.901625993 20.4819431304932 33.0
20.901625993 21.1129970550537 34.0
20.901625993 20.8195400238037 35.0
9.3083381265 8.28715133666992 36.0
8.044767894 8.38281917572021 37.0
-2.5262790395e-09 -0.804296553134918 38.0
-2.5262785966e-09 0.113371789455414 39.0
-12.387792857 -14.5510091781616 40.0
-23.81592281 -27.5234241485596 41.0
4.9278776077 3.5162615776062 42.0
-31.108913619 -34.7394485473633 43.0
92.052766479 91.3064651489258 44.0
92.05276649 90.2343292236328 45.0
6.7273095346 3.00315618515015 46.0
2.6568330431 1.83220052719116 47.0
137.17687743 136.681350708008 48.0
131.07448231 127.562980651855 49.0
0 3.54100728034973 50.0
69.705106686 70.9861526489258 51.0
78.669161734 84.2529830932617 52.0
77.500548402 81.0479965209961 53.0
77.500548402 80.9212951660156 54.0
77.500548402 79.4541549682617 55.0
0 -0.0822644829750061 56.0
80.656571391 80.3950500488281 57.0
221.53560809 218.26057434082 58.0
248.81825375 246.469757080078 59.0
20.901626002 19.9398994445801 60.0
0 2.89454770088196 61.0
160.40029773 161.28645324707 0.0
160.40029773 164.27180480957 1.0
159.04685921 156.982894897461 2.0
154.25778948 160.458343505859 3.0
0 1.95944714546204 4.0
-1.1106157605e-11 4.07647275924683 5.0
-67.164254488 -69.7971801757812 6.0
77.905037981 76.2602233886719 7.0
56.332002807 47.2559432983398 8.0
21.573035164 25.714204788208 9.0
86.970057114 90.5282669067383 10.0
73.950416555 75.4032516479492 11.0
81.119634047 87.77490234375 12.0
81.119634047 78.8990020751953 13.0
81.119634047 80.1012420654297 14.0
0 -0.256228864192963 15.0
0 -0.19922286272049 16.0
0 0.492870032787323 17.0
59.264416476 57.1221313476562 18.0
-59.264416496 -58.6440391540527 19.0
-59.264416474 -57.773265838623 20.0
-59.264416487 -57.5845527648926 21.0
80.633842594 78.2018508911133 22.0
145.67202515 141.996139526367 23.0
110.34261621 108.156616210938 24.0
22.971694682 19.1125392913818 25.0
226.30586775 222.863891601562 26.0
226.30586775 227.450820922852 27.0
28.061618658 29.2107601165771 28.0
254.36748642 252.690139770508 29.0
254.36748642 253.60498046875 30.0
46.518523875 45.2752456665039 31.0
9.3238344953 8.95865631103516 32.0
21.369426097 17.7103004455566 33.0
21.369426097 17.6770057678223 34.0
21.369426097 20.8870105743408 35.0
9.8534733136 7.96312856674194 36.0
8.5804788689 6.95929193496704 37.0
-3.4865499095e-11 -0.434172928333282 38.0
-3.4865475198e-11 -3.54875135421753 39.0
-12.411186885 -13.9457921981812 40.0
-24.601415118 -26.8714199066162 41.0
5.0371208571 5.8277759552002 42.0
-32.131741558 -34.3593444824219 43.0
94.496748584 90.5503082275391 44.0
94.496748597 94.2157592773438 45.0
6.7865864039 6.48163652420044 46.0
2.4610538327 2.17473244667053 47.0
160.40029774 162.612213134766 48.0
154.25778949 150.858261108398 49.0
0 2.82973146438599 50.0
77.905037991 77.1076126098633 51.0
86.970057124 87.8557891845703 52.0
81.119634057 81.9147491455078 53.0
81.119634057 81.5159759521484 54.0
81.119634057 80.1029281616211 55.0
0 1.16054677963257 56.0
80.633842604 79.8010025024414 57.0
226.30586776 219.901840209961 58.0
254.36748643 250.350433349609 59.0
21.369426107 22.7730197906494 60.0
0 1.7182629108429 61.0
160.93433464 169.785308837891 0.0
160.93433464 161.603240966797 1.0
159.43483656 150.959365844727 2.0
154.64576684 148.385284423828 3.0
0 3.035719871521 4.0
0 3.18494582176208 5.0
-55.874809179 -60.5923805236816 6.0
68.144100481 78.8034744262695 7.0
43.834445368 31.9751930236816 8.0
24.309655103 21.8643894195557 9.0
80.499323107 79.911003112793 10.0
68.183861895 72.6819839477539 11.0
79.34480887 77.1422729492188 12.0
79.34480887 80.1088256835938 13.0
79.34480887 77.8283615112305 14.0
0 1.37341594696045 15.0
0 0.617354154586792 16.0
0 -2.24680805206299 17.0
61.69957869 58.5894927978516 18.0
-61.6995787 -60.0284957885742 19.0
-61.699578679 -58.125675201416 20.0
-61.699578691 -59.0461654663086 21.0
83.530823997 75.5485305786133 22.0
148.48924223 140.187484741211 23.0
116.67694672 103.751068115234 24.0
23.432331078 15.7434549331665 25.0
232.02006624 226.605712890625 26.0
232.02006624 224.345779418945 27.0
24.28243193 26.4546566009521 28.0
256.30249818 256.409210205078 29.0
256.30249818 254.630111694336 30.0
46.154613284 44.947395324707 31.0
9.2851904492 8.57789993286133 32.0
21.831245296 20.7569828033447 33.0
21.831245296 25.9506664276123 34.0
21.831245296 19.5229644775391 35.0
9.8307377028 9.88792705535889 36.0
8.5724299747 8.21868801116943 37.0
-2.1848804195e-10 3.90726327896118 38.0
-2.1848804585e-10 2.46069526672363 39.0
-12.85371231 -15.7637720108032 40.0
-26.598005719 -22.2959251403809 41.0
4.86477693 0.782196819782257 42.0
-34.25238851 -32.5598907470703 43.0
85.581078916 89.5849304199219 44.0
85.581078929 93.930534362793 45.0
6.3182011789 6.83851099014282 46.0
2.6792978575 0.508032441139221 47.0
160.93433465 163.047149658203 48.0
154.64576685 142.139129638672 49.0
0 -3.07301688194275 50.0
68.144100491 76.8133392333984 51.0
80.499323117 80.595703125 52.0
79.344808879 81.1695327758789 53.0
79.344808879 78.4633026123047 54.0
79.344808879 76.6886367797852 55.0
0 -1.76897859573364 56.0
83.530824007 80.8844299316406 57.0
232.02006625 222.967407226562 58.0
256.30249819 251.541778564453 59.0
21.831245306 15.5613679885864 60.0
0 -0.0100010633468628 61.0
170.74103843 177.27766418457 0.0
170.74103843 166.731323242188 1.0
169.44583939 156.563629150391 2.0
164.65676966 185.092620849609 3.0
0 -2.45031881332397 4.0
-1.474190968e-12 6.27417087554932 5.0
-51.87515768 -60.5354385375977 6.0
63.776670305 80.0149078369141 7.0
62.518916525 50.8745727539062 8.0
1.2577537698 26.6626567840576 9.0
77.969462414 82.2695541381836 10.0
65.6606771 80.1906356811523 11.0
79.285801375 85.6974258422852 12.0
79.285801375 78.0426177978516 13.0
79.285801375 79.0845642089844 14.0
0 -1.12983083724976 15.0
0 0.146002352237701 16.0
-2.757033131e-12 0.254054605960846 17.0
59.28862776 57.658561706543 18.0
-59.28862778 -60.4683837890625 19.0
-59.288627758 -57.047191619873 20.0
-59.28862777 -53.8515853881836 21.0
81.653455976 74.5212631225586 22.0
145.5994357 137.908767700195 23.0
112.24206043 105.716743469238 24.0
25.066460109 20.8935413360596 25.0
227.25289169 222.542068481445 26.0
227.25289169 219.388442993164 27.0
24.608081457 28.0927543640137 28.0
251.86097315 254.049331665039 29.0
251.86097315 253.692916870117 30.0
46.139016363 45.7891426086426 31.0
9.2284755494 7.82419013977051 32.0
22.364828196 18.2973766326904 33.0
22.364828196 26.0855922698975 34.0
22.364828196 22.5620956420898 35.0
10.053490182 8.99110221862793 36.0
8.7085869813 7.48154926300049 37.0
-4.4617831932e-12 -2.80508995056152 38.0
-4.4617831928e-12 -1.21680045127869 39.0
-11.771241113 -15.5092573165894 40.0
-24.911710506 -30.12766456604 41.0
4.8968708365 4.62127447128296 42.0
-32.100214882 -29.7808952331543 43.0
102.49732093 80.6413955688477 44.0
102.49732094 115.221733093262 45.0
7.3892086852 8.47246646881104 46.0
2.6837738981 2.34726142883301 47.0
170.74103844 172.790084838867 48.0
164.65676967 152.247711181641 49.0
0 -0.852011859416962 50.0
63.776670315 74.5739288330078 51.0
77.969462424 87.5405502319336 52.0
79.285801385 80.2415466308594 53.0
79.285801385 80.8439788818359 54.0
79.285801385 79.7606582641602 55.0
0 1.41208457946777 56.0
81.653455986 83.185661315918 57.0
227.2528917 218.730697631836 58.0
251.86097316 254.12060546875 59.0
22.364828206 14.8394842147827 60.0
0 1.26519131660461 61.0
161.37032887 161.834625244141 0.0
161.37032887 163.695175170898 1.0
160.10767226 161.17692565918 2.0
155.31860253 163.316299438477 3.0
0 4.75594186782837 4.0
-8.3304147943e-12 0.996659934520721 5.0
-69.510437007 -66.1367645263672 6.0
80.573838403 73.0668869018555 7.0
56.531257042 53.5855255126953 8.0
24.042581351 21.7659435272217 9.0
81.056614139 88.7799835205078 10.0
68.419215597 75.1851654052734 11.0
78.645193067 81.3330230712891 12.0
78.645193067 78.6636581420898 13.0
78.645193067 81.2270660400391 14.0
0 -1.30366945266724 15.0
0 -1.20515084266663 16.0
-1.4736127519e-11 -1.24356150627136 17.0
60.726269621 58.4571418762207 18.0
-60.726269641 -59.8049163818359 19.0
-60.726269619 -58.4447975158691 20.0
-60.726269631 -60.4258346557617 21.0
82.906722552 79.8324737548828 22.0
145.85612845 143.788146972656 23.0
112.44381923 112.683372497559 24.0
22.811642575 21.9399585723877 25.0
228.76285101 226.530731201172 26.0
228.76285101 226.742202758789 27.0
26.589453625 25.111795425415 28.0
255.35230465 252.538070678711 29.0
255.35230465 251.53727722168 30.0
46.922627764 46.4211578369141 31.0
9.361691109 8.84907245635986 32.0
22.180452911 20.2960624694824 33.0
22.180452911 17.7955379486084 34.0
22.180452911 21.2942180633545 35.0
9.6633261439 9.68622970581055 36.0
8.3821479639 7.43668174743652 37.0
-1.9263427575e-11 -3.54132032394409 38.0
-1.9263461944e-11 -3.06932258605957 39.0
-11.94108426 -14.0427684783936 40.0
-25.232480584 -27.7097778320312 41.0
5.3179015532 4.50117015838623 42.0
-32.986870895 -31.5768070220947 43.0
98.668826995 89.9813003540039 44.0
98.668827008 91.3762969970703 45.0
7.055661771 7.67901706695557 46.0
2.5125628377 2.55115938186646 47.0
161.37032888 162.063537597656 48.0
155.31860254 155.76252746582 49.0
0 3.59447240829468 50.0
80.573838413 76.0603332519531 51.0
81.056614149 89.0262451171875 52.0
78.645193077 79.9142532348633 53.0
78.645193077 81.4903411865234 54.0
78.645193077 78.118782043457 55.0
0 4.44547891616821 56.0
82.906722562 79.6491241455078 57.0
228.76285102 221.502365112305 58.0
255.35230466 250.708312988281 59.0
22.180452921 21.4082565307617 60.0
0 1.69054889678955 61.0
194.54488019 159.075942993164 0.0
194.54488019 167.64794921875 1.0
193.07913495 172.868148803711 2.0
188.29006523 184.529998779297 3.0
-3.4156646434e-11 1.07089042663574 4.0
-4.0391911542e-11 4.49304294586182 5.0
-71.5300739 -52.615837097168 6.0
80.461981986 76.9666061401367 7.0
58.448847751 56.5962257385254 8.0
22.013134226 5.20493221282959 9.0
81.628850304 86.8225860595703 10.0
68.37745927 78.4542083740234 11.0
81.66393832 77.2426223754883 12.0
81.66393832 80.2858047485352 13.0
81.66393832 83.1369323730469 14.0
0 -2.635422706604 15.0
0 -1.01033854484558 16.0
-5.9942699058e-11 7.24216365814209 17.0
57.911027946 59.1430130004883 18.0
-57.911027966 -59.2003288269043 19.0
-57.911027944 -56.9994735717773 20.0
-57.911027957 -58.2809448242188 21.0
79.327813461 76.30908203125 22.0
145.64584376 142.290679931641 23.0
111.92990402 112.969482421875 24.0
21.755866618 20.6269683837891 25.0
224.97365723 228.639572143555 26.0
224.97365723 221.946014404297 27.0
24.236855424 24.835916519165 28.0
249.21051267 251.029800415039 29.0
249.21051267 255.203216552734 30.0
46.760932419 43.572639465332 31.0
9.4650710176 8.89524078369141 32.0
21.416785495 15.0644464492798 33.0
21.416785495 4.17405462265015 34.0
21.416785494 20.1440906524658 35.0
10.043361444 4.83081340789795 36.0
8.754619378 6.76093864440918 37.0
0 -7.24033641815186 38.0
0 1.41309213638306 39.0
-11.210280846 -19.4171848297119 40.0
-23.224039804 -19.4423809051514 41.0
4.8599470037 5.6995267868042 42.0
-30.554353162 -35.7832946777344 43.0
89.362904497 86.7814865112305 44.0
89.36290451 95.391716003418 45.0
7.0625740537 5.25438022613525 46.0
2.6841415901 2.62718939781189 47.0
194.5448802 161.648651123047 48.0
188.29006524 176.402755737305 49.0
0 3.36057901382446 50.0
80.461981996 76.0813598632812 51.0
81.628850314 90.2776260375977 52.0
81.66393833 80.2112350463867 53.0
81.66393833 78.5968170166016 54.0
81.66393833 80.4939270019531 55.0
0 1.24942493438721 56.0
79.327813471 81.2722854614258 57.0
224.97365724 220.016830444336 58.0
249.21051268 249.139907836914 59.0
21.416785504 23.1984577178955 60.0
0 3.21474361419678 61.0
178.14527711 167.300155639648 0.0
178.14527711 169.241851806641 1.0
176.97369394 161.30143737793 2.0
172.18462422 165.964553833008 3.0
0 0.920322358608246 4.0
-3.5198190369e-14 -1.79222726821899 5.0
-63.097698814 -68.2848510742188 6.0
73.030436147 76.6071929931641 7.0
51.473624422 47.9730110168457 8.0
21.556811715 21.2689208984375 9.0
89.348846835 85.5426254272461 10.0
76.888181042 75.6859970092773 11.0
81.202571227 85.1503372192383 12.0
81.202571227 79.6795959472656 13.0
81.202571227 77.5959320068359 14.0
0 0.259729385375977 15.0
0 -0.275744020938873 16.0
-7.0422371007e-14 0.234831809997559 17.0
60.918002378 58.669246673584 18.0
-60.918002398 -59.9002380371094 19.0
-60.918002376 -58.0961723327637 20.0
-60.918002388 -60.683464050293 21.0
82.832528382 79.2853698730469 22.0
148.14460492 142.260559082031 23.0
113.57009826 110.086334228516 24.0
22.42650822 22.1837291717529 25.0
230.97713331 222.765075683594 26.0
230.97713331 221.299423217773 27.0
26.719082351 29.0221042633057 28.0
257.69621567 253.000701904297 29.0
257.69621567 253.067016601562 30.0
46.597201673 44.5862617492676 31.0
9.3789993569 8.85275077819824 32.0
21.914525984 21.05393409729 33.0
21.914525984 17.7514572143555 34.0
21.914525984 21.8315010070801 35.0
9.5576217055 9.72793579101562 36.0
8.2462137773 6.65504837036133 37.0
-1.3093567975e-13 -1.60119271278381 38.0
-1.3093567805e-13 0.000318825244903564 39.0
-12.082631661 -16.8722133636475 40.0
-25.174390818 -24.906867980957 41.0
5.0528315667 5.24925470352173 42.0
-32.800761994 -32.7948150634766 43.0
98.782964972 91.6522827148438 44.0
98.782964985 94.904167175293 45.0
7.0657363023 7.28465032577515 46.0
2.4762401604 1.81919288635254 47.0
178.14527712 168.751571655273 48.0
172.18462423 161.647415161133 49.0
0 0.449927687644958 50.0
73.030436157 76.4434509277344 51.0
89.348846845 89.4178314208984 52.0
81.202571237 81.2289657592773 53.0
81.202571237 81.2747955322266 54.0
81.202571237 80.5928573608398 55.0
0 2.08465814590454 56.0
82.832528392 79.1607513427734 57.0
230.97713332 219.988128662109 58.0
257.69621568 251.314666748047 59.0
21.914525994 20.7639713287354 60.0
0 0.69889622926712 61.0
183.69863006 161.255767822266 0.0
183.69863006 160.342071533203 1.0
182.30479632 149.351287841797 2.0
177.5157266 230.24201965332 3.0
0 7.41556692123413 4.0
-3.8275518074e-12 1.54130530357361 5.0
-61.298514416 -44.0417404174805 6.0
71.797770875 77.7220306396484 7.0
62.566034639 54.9362106323242 8.0
9.2317362264 25.0542888641357 9.0
94.74833664 84.5469436645508 10.0
82.634307891 76.3621063232422 11.0
84.119096645 79.5232467651367 12.0
84.119096645 78.9040374755859 13.0
84.119096645 82.4336395263672 14.0
0 2.06801557540894 15.0
0 -0.15214866399765 16.0
-5.7569911734e-12 -0.536546409130096 17.0
60.167941594 58.0651206970215 18.0
-60.167941614 -58.8423461914062 19.0
-60.167941592 -55.9011688232422 20.0
-60.167941604 -59.3753547668457 21.0
82.174665591 84.2162628173828 22.0
137.28687671 145.529373168945 23.0
101.60709881 108.382522583008 24.0
22.482562023 22.4743804931641 25.0
219.46154231 218.487823486328 26.0
219.46154231 221.18815612793 27.0
29.632212128 24.2882652282715 28.0
249.09375445 251.144226074219 29.0
249.09375445 250.702423095703 30.0
46.549794097 44.5808944702148 31.0
9.3308846469 6.75656366348267 32.0
22.006723977 14.8711614608765 33.0
22.006723977 21.0119037628174 34.0
22.006723977 21.789119720459 35.0
9.8594892316 13.2579545974731 36.0
8.5483829446 7.33737850189209 37.0
-1.1565633744e-11 -11.1506271362305 38.0
-1.1565624568e-11 0.223828911781311 39.0
-12.003147099 -11.3338050842285 40.0
-24.582430832 -29.4993286132812 41.0
5.0388204109 2.41506886482239 42.0
-32.333853794 -30.80149269104 43.0
95.413570403 91.383430480957 44.0
95.413570416 95.0140151977539 45.0
6.9879297325 5.29206657409668 46.0
2.858329224 2.33294129371643 47.0
183.69863007 162.648452758789 48.0
177.51572661 162.827285766602 49.0
0 -0.202023804187775 50.0
71.797770885 77.6056747436523 51.0
94.74833665 85.6980133056641 52.0
84.119096655 81.159423828125 53.0
84.119096655 79.3929061889648 54.0
84.119096655 80.9838790893555 55.0
0 3.04920053482056 56.0
82.174665601 78.4147872924805 57.0
219.46154232 222.234832763672 58.0
249.09375446 247.347991943359 59.0
22.006723987 24.3862419128418 60.0
0 4.85806703567505 61.0
176.69396978 172.737380981445 0.0
176.69396978 178.795013427734 1.0
175.77063211 173.082061767578 2.0
170.98156239 159.37873840332 3.0
-6.2986018063e-13 -3.09858417510986 4.0
-6.9819693682e-13 -2.50659489631653 5.0
-79.775720264 -64.054313659668 6.0
92.039848618 70.8834533691406 7.0
70.654809807 61.4159851074219 8.0
21.385038801 29.8398017883301 9.0
79.836644547 95.394660949707 10.0
64.681112026 95.8524703979492 11.0
72.353706199 90.2138671875 12.0
72.353706199 80.4745788574219 13.0
72.353706199 67.8321075439453 14.0
0 2.58032155036926 15.0
0 2.1721887588501 16.0
0 -3.55337715148926 17.0
63.006860474 53.5044059753418 18.0
-63.006860494 -58.1772117614746 19.0
-63.006860472 -47.9112701416016 20.0
-63.006860485 -54.297607421875 21.0
84.312002141 73.8552932739258 22.0
134.46802551 141.880142211914 23.0
98.125424485 106.994438171387 24.0
22.74464851 19.6945495605469 25.0
218.78002766 240.850692749023 26.0
218.78002766 236.375503540039 27.0
26.074593497 26.8674087524414 28.0
244.85462117 256.929351806641 29.0
244.85462117 244.893508911133 30.0
45.829212312 54.4556312561035 31.0
9.2504318289 9.45322513580322 32.0
21.305141647 23.6718559265137 33.0
21.305141647 16.9097061157227 34.0
21.305141647 21.2481956481934 35.0
8.903502035 7.4536919593811 36.0
7.6566485711 5.06501388549805 37.0
0 -4.89432573318481 38.0
0 -3.40336751937866 39.0
-12.35173632 -13.0760440826416 40.0
-24.649322457 -24.523738861084 41.0
5.5304438843 11.0218000411987 42.0
-33.149268543 -35.4238739013672 43.0
86.574395206 92.3511657714844 44.0
86.574395219 91.5722045898438 45.0
6.3240460839 -6.6446704864502 46.0
2.583145984 8.99392032623291 47.0
176.69396979 192.106521606445 48.0
170.9815624 144.042922973633 49.0
0 -8.49411392211914 50.0
92.039848628 79.6958312988281 51.0
79.836644557 62.717357635498 52.0
72.353706209 81.0981979370117 53.0
72.353706209 81.74169921875 54.0
72.353706209 81.0683212280273 55.0
0 0.803893744945526 56.0
84.312002151 78.9203567504883 57.0
218.78002767 230.466674804688 58.0
244.85462118 240.675048828125 59.0
21.305141657 26.7222270965576 60.0
0 3.71923160552979 61.0
147.982033057583 157.456359863281 0.0
147.982033057583 151.964233398438 1.0
146.808413813287 153.353607177734 2.0
142.01934397 150.753616333008 3.0
-1.55484310406e-09 -0.652972280979156 4.0
-1.55484309999997e-09 -2.84123921394348 5.0
-64.9693040339936 -64.3224563598633 6.0
77.6547829144777 74.1872024536133 7.0
61.2786926634778 55.4030647277832 8.0
16.3760902514777 7.20984888076782 9.0
97.5261141374777 89.7391510009766 10.0
83.6472874924777 79.6450042724609 11.0
85.2981411144777 83.8642730712891 12.0
85.2981411144777 76.4624557495117 13.0
85.2981411144777 81.5436782836914 14.0
2.3328021736e-08 -2.03496789932251 15.0
2.3328021735e-08 -1.6052393913269 16.0
5.464350236e-08 -2.72059202194214 17.0
56.2518018559467 59.0783615112305 18.0
-56.2518018797166 -57.0891418457031 19.0
-56.2518018707166 -59.2854194641113 20.0
-56.2518017707166 -58.115608215332 21.0
77.1875821073724 73.3798065185547 22.0
153.454771802806 145.008895874023 23.0
115.220596133508 115.750228881836 24.0
21.7005584165075 20.8819332122803 25.0
230.642353915167 228.381210327148 26.0
230.642353915167 229.708221435547 27.0
26.8883420391666 21.3647422790527 28.0
257.530695955167 253.23681640625 29.0
257.530695955167 255.336776733398 30.0
46.7385281475469 45.537841796875 31.0
9.28724497493877 8.4355525970459 32.0
20.9357803369809 22.6376686096191 33.0
20.9357803369809 19.0790691375732 34.0
20.935780467613 20.4264030456543 35.0
8.86054244871795 8.39873600006104 36.0
7.60121916461794 6.85336446762085 37.0
2.62465480000001e-09 1.1959273815155 38.0
2.62465480000001e-09 -1.17553234100342 39.0
-11.6034306368583 -12.5643882751465 40.0
-24.5599220844463 -26.7411251068115 41.0
4.75986482786468 1.92820739746094 42.0
-31.7734218810648 -31.6524333953857 43.0
89.4696777832616 86.4044494628906 44.0
89.4696781210649 91.9600372314453 45.0
7.05809588635643 4.443039894104 46.0
2.46335994485643 2.26466846466064 47.0
147.98203306 156.096481323242 48.0
142.01934397 153.015121459961 49.0
0 -3.74582099914551 50.0
77.654782914 76.0236129760742 51.0
97.526114137 92.0755157470703 52.0
85.298141114 81.5651702880859 53.0
85.298141114 80.2131118774414 54.0
85.298141114 79.9203643798828 55.0
5.4643502359e-08 3.07315587997437 56.0
77.187582108 79.4883346557617 57.0
230.64235391 222.583419799805 58.0
257.53069595 249.491134643555 59.0
20.935780468 19.6116352081299 60.0
2.6246548239e-09 3.64096260070801 61.0
147.982033057583 157.456436157227 0.0
147.982033057583 151.966171264648 1.0
146.808413813287 153.369369506836 2.0
142.01934397 150.871368408203 3.0
-1.55484310406e-09 -0.666505873203278 4.0
-1.55484309999997e-09 -2.84622478485107 5.0
-64.9693040339936 -64.3275375366211 6.0
77.6547829144777 74.1872482299805 7.0
61.2786926634778 55.4030647277832 8.0
16.3760902514777 7.20984888076782 9.0
97.5261141374777 89.7391510009766 10.0
83.6472874924777 79.6450042724609 11.0
85.2981411144777 83.8642730712891 12.0
85.2981411144777 76.4624557495117 13.0
85.2981411144777 81.5436782836914 14.0
2.3328021736e-08 -2.03496789932251 15.0
2.3328021735e-08 -1.6052393913269 16.0
5.464350236e-08 -2.72059202194214 17.0
56.2518018559467 59.0783615112305 18.0
-56.2518018797166 -57.0891418457031 19.0
-56.2518018707166 -59.2854194641113 20.0
-56.2518017707166 -58.115608215332 21.0
77.1875821073724 73.3798065185547 22.0
153.454771802806 145.008895874023 23.0
115.220596133508 115.750228881836 24.0
21.7005584165075 20.8819332122803 25.0
230.642353915167 228.381210327148 26.0
230.642353915167 229.708221435547 27.0
26.8883420391666 21.3647422790527 28.0
257.530695955167 253.23681640625 29.0
257.530695955167 255.336776733398 30.0
46.7385281475469 45.537841796875 31.0
9.28724497493877 8.4355525970459 32.0
20.9357803369809 22.6376686096191 33.0
20.9357803369809 19.0790691375732 34.0
20.935780467613 20.4264030456543 35.0
8.86054244871795 8.39873600006104 36.0
7.60121916461794 6.85336446762085 37.0
2.62465480000001e-09 1.1959273815155 38.0
2.62465480000001e-09 -1.17553234100342 39.0
-11.6034306368583 -12.5643882751465 40.0
-24.5599220844463 -26.7411251068115 41.0
4.75986482786468 1.92820739746094 42.0
-31.7734218810648 -31.6524333953857 43.0
89.4696777832616 86.4044494628906 44.0
89.4696781210649 91.9600372314453 45.0
7.05809588635643 4.443039894104 46.0
2.46335994485643 2.26466846466064 47.0
147.98203306 156.095626831055 48.0
142.01934397 153.111145019531 49.0
0 -3.73828077316284 50.0
77.654782914 76.0232925415039 51.0
97.526114137 92.0755157470703 52.0
85.298141114 81.5651702880859 53.0
85.298141114 80.2131118774414 54.0
85.298141114 79.9203643798828 55.0
5.4643502359e-08 3.07315587997437 56.0
77.187582108 79.4883346557617 57.0
230.64235391 222.583419799805 58.0
257.53069595 249.491134643555 59.0
20.935780468 19.6116352081299 60.0
2.6246548239e-09 3.64096260070801 61.0
173.66791269033 166.734527587891 0.0
173.66791269033 162.891311645508 1.0
172.26713182 164.300765991211 2.0
167.47806208 153.524749755859 3.0
0 -7.19111633300781 4.0
-7.875698605e-09 1.89501595497131 5.0
-73.1392608874969 -60.1633796691895 6.0
85.6971479372393 73.2032470703125 7.0
51.8825515662393 61.9264221191406 8.0
33.8145963712393 15.4562854766846 9.0
86.0546532632392 90.5621871948242 10.0
73.1403174012392 76.6735916137695 11.0
83.7981330662392 80.1997985839844 12.0
83.7981330662392 81.1996307373047 13.0
83.7981330662392 74.253532409668 14.0
1.0183230703e-09 -4.51358270645142 15.0
0 -1.80627846717834 16.0
1.9586105000002e-10 -0.531194269657135 17.0
61.5117567031305 57.2512283325195 18.0
-61.5117566918589 -53.8158302307129 19.0
-61.5117566718589 -59.2176094055176 20.0
-61.5117566668589 -58.2656135559082 21.0
83.902509463362 77.4449996948242 22.0
149.241310944131 141.347564697266 23.0
118.072750147663 107.887817382812 24.0
22.8035399266631 18.132396697998 25.0
233.143820400661 220.646759033203 26.0
233.143820400661 224.705596923828 27.0
25.1609618036607 27.3962287902832 28.0
258.304782210661 254.96305847168 29.0
258.304782210661 254.36262512207 30.0
46.173295023827 48.6827278137207 31.0
9.35744571762142 9.44705200195312 32.0
22.3907527964908 17.5944347381592 33.0
22.3907527964908 14.3833513259888 34.0
22.3907528163584 21.2651615142822 35.0
9.56967750246318 9.88761806488037 36.0
8.38147665886318 7.68716716766357 37.0
-8.15994281999999e-09 -5.07474422454834 38.0
-8.15994280999999e-09 -6.04146766662598 39.0
-11.8955725524294 -20.5972270965576 40.0
-24.9642038225962 -24.7266674041748 41.0
5.30673767266604 1.23535752296448 42.0
-32.580962627366 -30.3393878936768 43.0
91.598955117366 96.9008407592773 44.0
91.598955046366 96.2051467895508 45.0
6.9771983202 6.89241933822632 46.0
2.3118499226 2.29085612297058 47.0
173.66791269 166.923873901367 48.0
167.47806209 159.29655456543 49.0
0 -1.55586719512939 50.0
85.697147937 73.3957366943359 51.0
86.054653263 79.8920211791992 52.0
83.798133066 80.6196212768555 53.0
83.798133066 78.0019226074219 54.0
83.798133066 78.6284484863281 55.0
1.9586104823e-10 -5.52205181121826 56.0
83.902509463 78.5379028320312 57.0
233.14382041 216.375411987305 58.0
258.30478221 247.157791137695 59.0
22.390752816 23.8628978729248 60.0
0 4.11010074615479 61.0
179.103283948298 168.961120605469 0.0
179.103283948298 171.110000610352 1.0
177.79783692 169.107330322266 2.0
173.00876717 176.676986694336 3.0
0 2.60780620574951 4.0
-7.61149997e-09 -0.0143744349479675 5.0
-60.4952508334905 -59.4933166503906 6.0
71.6128596977166 74.3285522460938 7.0
49.9466633877166 53.6119499206543 8.0
21.6661963097166 17.5298595428467 9.0
84.1748046647166 90.4200973510742 10.0
71.3253228367166 78.2450180053711 11.0
79.9175488417166 85.3231811523438 12.0
79.9175488417166 80.6547012329102 13.0
79.9175488417166 81.2348861694336 14.0
1.1560271588e-09 0.705878436565399 15.0
0 0.206540107727051 16.0
4.97683930000007e-10 0.770088374614716 17.0
60.851817171209 59.6606292724609 18.0
-60.8518171595749 -59.5813522338867 19.0
-60.8518171395749 -60.354621887207 20.0
-60.8518171345749 -61.3985137939453 21.0
82.0320967984099 79.792839050293 22.0
143.289499322209 143.987319946289 23.0
108.999254885261 110.579765319824 24.0
22.0608285202613 18.849422454834 25.0
225.321596116596 226.129425048828 26.0
225.321596116596 222.805358886719 27.0
26.307492927596 27.3937110900879 28.0
251.629089046596 254.824096679688 29.0
251.629089046596 250.675506591797 30.0
46.5630254288009 45.3393211364746 31.0
9.3615102388307 8.13702201843262 32.0
21.1802796644713 20.1425113677979 33.0
21.1802796644713 16.4871234893799 34.0
21.1802796851436 20.2971172332764 35.0
9.48754273963785 7.50277614593506 36.0
8.21612593233785 6.97669744491577 37.0
-8.10757414000001e-09 -0.255769431591034 38.0
-8.10757413999999e-09 -0.629701197147369 39.0
-12.3614839637874 -12.5336265563965 40.0
-25.079665917805 -24.6759757995605 41.0
5.05033433133047 6.12962055206299 42.0
-32.8276310244305 -32.752986907959 43.0
94.8428525354305 94.0620269775391 44.0
94.8428524614305 92.8666610717773 45.0
7.1737057602 6.34871864318848 46.0
2.6697034577 1.05629897117615 47.0
179.10328395 168.883071899414 48.0
173.00876718 170.308166503906 49.0
0 -0.683293282985687 50.0
71.612859697 76.5010223388672 51.0
84.174804665 90.7347869873047 52.0
79.917548842 81.3800048828125 53.0
79.917548842 80.7949447631836 54.0
79.917548842 79.7079086303711 55.0
4.9768392924e-10 2.51922488212585 56.0
82.032096799 80.1395950317383 57.0
225.32159612 221.537628173828 58.0
251.62908905 252.365737915039 59.0
21.180279685 20.7248725891113 60.0
0 1.71327292919159 61.0
167.314577190639 171.654998779297 0.0
167.314577190639 162.438995361328 1.0
166.09927871 164.961380004883 2.0
160.63464351 114.219902038574 3.0
7.1337080016 23.098352432251 4.0
7.13370801125907 15.9946279525757 5.0
-74.0894011344591 -59.2308349609375 6.0
88.4931098961067 88.9129104614258 7.0
67.6776169881067 22.2629280090332 8.0
20.8154929081067 10.5851640701294 9.0
93.8634301471067 55.8013610839844 10.0
80.1191217241067 86.9708709716797 11.0
79.1228040811068 83.7723846435547 12.0
79.1228040811068 72.173942565918 13.0
79.1228040811068 77.2857131958008 14.0
0 4.94209623336792 15.0
0 -1.52295327186584 16.0
-5.298430263e-09 2.91290473937988 17.0
49.195458628733 41.9270095825195 18.0
-49.1954586616601 -53.9749526977539 19.0
-49.1954586406601 -46.4687805175781 20.0
-49.1954586456601 -52.6158065795898 21.0
67.9021219612451 79.1926956176758 22.0
114.63025489224 132.946319580078 23.0
86.6775151856767 104.554618835449 24.0
5.12492498270005 25.9933643341064 25.0
182.532376851277 202.824584960938 26.0
182.532376851277 214.565734863281 27.0
31.439170213277 29.1898593902588 28.0
213.971547071277 249.992752075195 29.0
213.971547071277 258.158905029297 30.0
40.1227290627147 45.2578659057617 31.0
1.7854235123 6.19486951828003 32.0
18.7066633163772 18.5409202575684 33.0
18.7066633163772 17.9886798858643 34.0
18.7066633245151 16.1070785522461 35.0
9.4591016811 7.5654411315918 36.0
8.1059226357 4.67108535766602 37.0
-9.191432868e-09 -2.16798138618469 38.0
-9.191432868e-09 -5.25315713882446 39.0
-10.1969913878301 -0.335666477680206 40.0
-22.7426672531221 -23.9016456604004 41.0
4.79959465878895 -0.28809529542923 42.0
-29.942174925589 -31.5504093170166 43.0
85.0482678706595 92.7224273681641 44.0
85.0482679245889 96.4161071777344 45.0
6.5239043858 2.14909696578979 46.0
2.8317449572 1.45713233947754 47.0
167.31457719 165.640350341797 48.0
160.63464352 139.295440673828 49.0
7.1337080113 -13.8281841278076 50.0
88.493109896 53.8064155578613 51.0
93.863430147 81.2395477294922 52.0
79.122804081 80.8477325439453 53.0
79.122804081 80.597282409668 54.0
79.122804081 62.7402572631836 55.0
0 17.6642742156982 56.0
67.902121961 81.650260925293 57.0
182.53237685 207.083755493164 58.0
213.97154707 254.508926391602 59.0
18.706663324 33.6629524230957 60.0
0 0.690270960330963 61.0
180.56120698 158.672058105469 0.0
180.56120698 160.892959594727 1.0
179.42366895 159.682327270508 2.0
174.63459922 174.058059692383 3.0
0 -1.33216953277588 4.0
-3.8299588184e-09 -0.0357697606086731 5.0
-62.788001716 -64.6817016601562 6.0
73.3667477875118 75.2163391113281 7.0
46.3407543985118 51.560604095459 8.0
27.0259933885118 23.9737758636475 9.0
87.3532325335118 85.1591186523438 10.0
73.8305626775118 72.7441024780273 11.0
79.2555855715118 81.4703140258789 12.0
79.2555855715118 80.4981307983398 13.0
79.2555855715118 80.4476852416992 14.0
0 -0.891246378421783 15.0
0 0.188278496265411 16.0
-8.7568110599e-09 3.01566290855408 17.0
58.783830954 59.0972938537598 18.0
-58.7838309732678 -59.547664642334 19.0
-58.7838309512678 -60.2056007385254 20.0
-58.7838309612678 -61.1560363769531 21.0
80.477742487 78.951545715332 22.0
144.23420647 140.728912353516 23.0
110.79537471 108.575744628906 24.0
23.122419063 21.1175670623779 25.0
224.711948963283 228.765182495117 26.0
224.711948963283 223.179641723633 27.0
25.2786230382829 26.3548107147217 28.0
249.990572003283 253.711517333984 29.0
249.990572003283 251.888534545898 30.0
46.488833305 45.8179512023926 31.0
9.2883098987 8.48523330688477 32.0
21.6939115325816 22.5071830749512 33.0
21.6939115325816 20.946569442749 34.0
21.693911529 19.082052230835 35.0
9.6231391732 8.97734260559082 36.0
8.3375640296 7.29577970504761 37.0
-9.7500530017e-09 0.439431488513947 38.0
-9.7500530026e-09 -5.56924533843994 39.0
-11.3969348871339 -13.6627254486084 40.0
-23.916263426 -26.9965324401855 41.0
5.04623185022654 6.0635814666748 42.0
-31.5495625149265 -32.5325088500977 43.0
93.3324665699265 86.1542739868164 44.0
93.3324665607527 90.8757476806641 45.0
6.9116512699 6.27865314483643 46.0
2.5160461172 1.16380739212036 47.0
180.56120698 162.805374145508 48.0
174.63459923 165.648574829102 49.0
0 0.704916059970856 50.0
73.366747787 75.3045959472656 51.0
87.353232533 87.6276550292969 52.0
79.255585571 80.3227615356445 53.0
79.255585571 81.1637725830078 54.0
79.255585571 80.3124618530273 55.0
0 1.27056169509888 56.0
80.477742494 79.8733901977539 57.0
224.71194896 222.875183105469 58.0
249.990572 252.965530395508 59.0
21.693911534 19.9257888793945 60.0
0 1.4767792224884 61.0
175.603615999326 163.70051574707 0.0
175.603615999326 163.226989746094 1.0
174.361026076965 167.951293945312 2.0
169.57195626 168.679351806641 3.0
0 2.21631646156311 4.0
-3.11035666999999e-09 0.0948862433433533 5.0
-68.4657742233803 -68.1041564941406 6.0
78.4603731578309 74.9543914794922 7.0
55.8593717988309 50.4680938720703 8.0
22.6010013588309 23.725658416748 9.0
88.6381626078309 89.4837493896484 10.0
76.1095343658309 74.349723815918 11.0
84.2457675568309 80.786865234375 12.0
84.2457675568309 78.8211517333984 13.0
84.2457675568309 80.2257080078125 14.0
1.5967694022e-08 -3.03719592094421 15.0
1.5967694022e-08 -0.6790731549263 16.0
4.10408307600001e-08 4.02962303161621 17.0
59.3427396009986 58.4497680664062 18.0
-59.3427394612463 -59.6069488525391 19.0
-59.3427394492463 -57.9750862121582 20.0
-59.3427393742463 -62.565013885498 21.0
81.4162820210354 79.7224426269531 22.0
147.563924408999 141.379867553711 23.0
111.203413653748 108.555786132812 24.0
22.6462377107483 20.5621700286865 25.0
228.980206428651 225.713012695312 26.0
228.980206428651 225.826522827148 27.0
25.000763623651 26.6047496795654 28.0
253.980970048651 253.595413208008 29.0
253.980970048651 257.869659423828 30.0
46.8273874749119 45.596263885498 31.0
9.40145704531782 9.3886547088623 32.0
22.0735426461408 19.2898845672607 33.0
22.0735426461408 19.9772872924805 34.0
22.0735427484244 21.1353893280029 35.0
9.66482077112323 9.33578968048096 36.0
8.39634724462323 8.03990364074707 37.0
-5.18780499999713e-11 -1.14138412475586 38.0
-5.18780499999713e-11 -1.43837594985962 39.0
-12.1448151781232 -14.1659688949585 40.0
-25.2639431901174 -24.3290786743164 41.0
4.94296027062577 4.6250057220459 42.0
-32.8332034700257 -33.7567710876465 43.0
92.1813686560257 92.5753555297852 44.0
92.181367865322 93.6878433227539 45.0
6.83699246819685 6.20694732666016 46.0
2.67937249829685 2.12435221672058 47.0
175.603616 161.165100097656 48.0
169.57195627 162.540100097656 49.0
0 -2.26017141342163 50.0
78.460373158 76.3995590209961 51.0
88.638162608 89.4166870117188 52.0
84.245767557 81.3833618164062 53.0
84.245767557 80.8873825073242 54.0
84.245767557 80.3582992553711 55.0
4.1040830757e-08 5.18019723892212 56.0
81.416282021 79.6352844238281 57.0
228.98020643 223.247207641602 58.0
253.98097005 251.416793823242 59.0
22.073542748 20.8257064819336 60.0
0 1.12887668609619 61.0
187.25174543 167.712448120117 0.0
187.25174543 167.243118286133 1.0
185.76652126 162.38427734375 2.0
180.97745154 172.061538696289 3.0
0 0.537262141704559 4.0
0 0.12977010011673 5.0
-75.834988292 -70.2542343139648 6.0
86.243373572 72.3633117675781 7.0
54.738453324 47.7086601257324 8.0
31.504920238 23.9303398132324 9.0
84.523370062 82.7559356689453 10.0
70.696428485 77.9505233764648 11.0
79.540379777 88.9773788452148 12.0
79.540379777 82.764518737793 13.0
79.540379777 79.8790893554688 14.0
0 -1.16056251525879 15.0
0 -0.640219509601593 16.0
0 -1.35294342041016 17.0
59.023681675 59.1393890380859 18.0
-59.023681695 -60.1408157348633 19.0
-59.023681673 -57.2784767150879 20.0
-59.023681685 -58.0735282897949 21.0
80.651479128 77.8520050048828 22.0
148.34500437 142.460784912109 23.0
110.71020082 107.796539306641 24.0
22.801924619 21.5507507324219 25.0
228.99648351 233.00553894043 26.0
228.99648351 223.427124023438 27.0
25.813950021 24.1170043945312 28.0
254.81043354 255.948516845703 29.0
254.81043354 251.362045288086 30.0
46.692061519 46.1001853942871 31.0
9.4146724968 9.8388147354126 32.0
21.627797433 14.1586351394653 33.0
21.627797433 19.4121780395508 34.0
21.627797433 20.3372936248779 35.0
9.7012600016 8.90798950195312 36.0
8.3765762092 7.49448728561401 37.0
0 -1.0698516368866 38.0
0 1.40821516513824 39.0
-12.389737588 -14.8989124298096 40.0
-26.288917956 -22.4758625030518 41.0
4.9750077173 4.85038757324219 42.0
-33.781630988 -34.1070709228516 43.0
80.472763168 82.7687072753906 44.0
80.472763181 77.7280502319336 45.0
6.7665957462 9.79705238342285 46.0
2.8617454844 1.27422618865967 47.0
187.25174544 162.551345825195 48.0
180.97745155 176.300170898438 49.0
0 -3.21843862533569 50.0
86.243373582 77.5929794311523 51.0
84.523370072 89.6717147827148 52.0
79.540379787 79.4229736328125 53.0
79.540379787 81.5679931640625 54.0
79.540379787 80.7562866210938 55.0
0 3.64743280410767 56.0
80.651479138 80.0808944702148 57.0
228.99648352 221.467956542969 58.0
254.81043355 250.751495361328 59.0
21.627797443 17.4413070678711 60.0
0 1.16197752952576 61.0
147.982033057583 157.45637512207 0.0
147.982033057583 151.964126586914 1.0
146.808413813287 153.352630615234 2.0
142.01934397 150.746673583984 3.0
-1.55484310406e-09 -0.652255833148956 4.0
-1.55484309999997e-09 -2.84094476699829 5.0
-64.9693040339936 -64.3221588134766 6.0
77.6547829144777 74.1871871948242 7.0
61.2786926634778 55.4030647277832 8.0
16.3760902514777 7.20984888076782 9.0
97.5261141374777 89.7391510009766 10.0
83.6472874924777 79.6450042724609 11.0
85.2981411144777 83.8642730712891 12.0
85.2981411144777 76.4624557495117 13.0
85.2981411144777 81.5436782836914 14.0
2.3328021736e-08 -2.03496789932251 15.0
2.3328021735e-08 -1.6052393913269 16.0
5.464350236e-08 -2.72059202194214 17.0
56.2518018559467 59.0783615112305 18.0
-56.2518018797166 -57.0891418457031 19.0
-56.2518018707166 -59.2854194641113 20.0
-56.2518017707166 -58.115608215332 21.0
77.1875821073724 73.3798065185547 22.0
153.454771802806 145.008895874023 23.0
115.220596133508 115.750228881836 24.0
21.7005584165075 20.8819332122803 25.0
230.642353915167 228.381210327148 26.0
230.642353915167 229.708221435547 27.0
26.8883420391666 21.3647422790527 28.0
257.530695955167 253.23681640625 29.0
257.530695955167 255.336776733398 30.0
46.7385281475469 45.537841796875 31.0
9.28724497493877 8.4355525970459 32.0
20.9357803369809 22.6376686096191 33.0
20.9357803369809 19.0790691375732 34.0
20.935780467613 20.4264030456543 35.0
8.86054244871795 8.39873600006104 36.0
7.60121916461794 6.85336446762085 37.0
2.62465480000001e-09 1.1959273815155 38.0
2.62465480000001e-09 -1.17553234100342 39.0
-11.6034306368583 -12.5643882751465 40.0
-24.5599220844463 -26.7411251068115 41.0
4.75986482786468 1.92820739746094 42.0
-31.7734218810648 -31.6524333953857 43.0
89.4696777832616 86.4044494628906 44.0
89.4696781210649 91.9600372314453 45.0
7.05809588635643 4.443039894104 46.0
2.46335994485643 2.26466846466064 47.0
147.98203306 156.096435546875 48.0
142.01934397 153.009490966797 49.0
0 -3.74624156951904 50.0
77.654782914 76.0236282348633 51.0
97.526114137 92.0755157470703 52.0
85.298141114 81.5651702880859 53.0
85.298141114 80.2131118774414 54.0
85.298141114 79.9203643798828 55.0
5.4643502359e-08 3.07315587997437 56.0
77.187582108 79.4883346557617 57.0
230.64235391 222.583419799805 58.0
257.53069595 249.491134643555 59.0
20.935780468 19.6116352081299 60.0
2.6246548239e-09 3.64096260070801 61.0
134.24481883 157.957214355469 0.0
134.24481883 163.595458984375 1.0
132.82307575 158.586990356445 2.0
128.03400603 77.6806793212891 3.0
0 49.2262802124023 4.0
-5.2541713817e-10 9.60797786712646 5.0
-62.648268578 -65.2880477905273 6.0
75.698873081 82.0057907104492 7.0
38.252844302 59.4929809570312 8.0
37.446028777 23.5055904388428 9.0
92.158865121 90.9868087768555 10.0
78.821465908 76.4257736206055 11.0
77.288218161 82.996208190918 12.0
77.288218161 80.1845703125 13.0
77.288218161 80.4059219360352 14.0
0 -1.5278742313385 15.0
0 -0.272740662097931 16.0
-3.8580734939e-09 3.19112825393677 17.0
61.396789336 58.2097244262695 18.0
-61.3967893605794 -59.7254524230957 19.0
-61.3967893385794 -57.906810760498 20.0
-61.3967893505794 -56.6375045776367 21.0
82.743374176 77.9580307006836 22.0
142.54028823 146.576904296875 23.0
108.73346637 107.098533630371 24.0
24.369949699 20.8150310516357 25.0
225.28366241 223.415069580078 26.0
225.28366241 225.543701171875 27.0
25.302348782 26.6648349761963 28.0
250.5860112 252.414978027344 29.0
250.5860112 253.120086669922 30.0
47.407915959 45.7993392944336 31.0
9.4760424616 8.62786388397217 32.0
21.34658483 20.2449626922607 33.0
21.34658483 10.584192276001 34.0
21.346584827 18.3100490570068 35.0
9.5680411703 6.18280506134033 36.0
8.2857930979 7.21461915969849 37.0
-6.977820572e-09 3.21312284469604 38.0
-6.9778205758e-09 6.17932653427124 39.0
-12.158339285 -21.8348827362061 40.0
-25.692887733 -24.5948123931885 41.0
5.1109415633 7.47711849212646 42.0
-33.368688116 -34.4948768615723 43.0
89.536940237 90.2981185913086 44.0
89.536940238 89.7506866455078 45.0
7.2303487883 6.78892421722412 46.0
2.488949114 1.2705819606781 47.0
134.24481884 161.660125732422 48.0
128.03400604 112.872871398926 49.0
0 -2.49492287635803 50.0
75.698873084 75.5928573608398 51.0
92.158865124 90.2025680541992 52.0
77.288218164 81.1239776611328 53.0
77.288218164 81.7022018432617 54.0
77.288218164 79.715690612793 55.0
0 6.4595193862915 56.0
82.743374183 78.4909439086914 57.0
225.28366242 217.227340698242 58.0
250.5860112 245.601760864258 59.0
21.346584834 20.4458293914795 60.0
0 4.00279664993286 61.0
156.444909094495 161.839569091797 0.0
156.444909094495 161.653442382812 1.0
155.23754101 165.262878417969 2.0
150.44847128 157.919326782227 3.0
0 2.30976414680481 4.0
-8.9481355392e-09 6.67272281646729 5.0
-57.77577294 -36.5774917602539 6.0
66.7565049451246 91.5021209716797 7.0
49.5646924781246 58.0445556640625 8.0
17.1918124661246 4.82854175567627 9.0
94.0203926851246 89.9691772460938 10.0
81.0576863641246 74.9639434814453 11.0
84.0425682181246 86.9017333984375 12.0
84.0425682181246 81.5962448120117 13.0
84.0425682181246 76.1716079711914 14.0
0 -3.2539496421814 15.0
0 -0.101828813552856 16.0
-7.133330226e-09 -4.37102746963501 17.0
53.3249561025444 53.1096801757812 18.0
-53.324956129687 -58.2780647277832 19.0
-53.324956108687 -50.8171577453613 20.0
-53.324956116687 -59.3309593200684 21.0
74.155556989 78.5148391723633 22.0
152.141419075341 141.177886962891 23.0
116.385006401676 111.784851074219 24.0
24.6283618916759 21.1199188232422 25.0
226.29697606899 215.132354736328 26.0
226.29697606899 222.101516723633 27.0
26.64660490099 21.9099025726318 28.0
252.94358096899 248.841461181641 29.0
252.94358096899 249.401565551758 30.0
47.237095487 48.7177886962891 31.0
9.3625456757 8.60711193084717 32.0
20.830600874964 20.2689380645752 33.0
20.830600874964 19.9243202209473 34.0
20.830600878 20.5518074035645 35.0
9.0430553836 9.92651653289795 36.0
7.8596073311 7.71602582931519 37.0
-9.577887965e-09 -0.329658925533295 38.0
-9.577887965e-09 1.14548230171204 39.0
-12.3225098738435 -12.1976957321167 40.0
-24.9667912583346 -25.8983516693115 41.0
5.35537586260387 7.56761360168457 42.0
-33.0497348136039 -34.5957450866699 43.0
98.3015891546039 90.6993255615234 44.0
98.3015891268181 90.4664077758789 45.0
6.7129042359 5.50776863098145 46.0
2.6635332936 0.911825060844421 47.0
156.4449091 163.393905639648 48.0
150.44847129 145.188827514648 49.0
0 1.04045760631561 50.0
66.756504945 76.4115295410156 51.0
94.020392685 85.0493698120117 52.0
84.042568218 82.0698776245117 53.0
84.042568218 79.9915924072266 54.0
84.042568218 79.5889739990234 55.0
0 2.31768703460693 56.0
74.155556992 80.8663177490234 57.0
226.29697607 215.949844360352 58.0
252.94358097 243.482604980469 59.0
20.830600879 21.7150840759277 60.0
0 1.37411832809448 61.0
181.27880342 163.901519775391 0.0
181.27880342 158.076095581055 1.0
180.07179026 171.61164855957 2.0
175.28272054 165.716613769531 3.0
0 9.02878665924072 4.0
-4.0961819131e-14 11.0166940689087 5.0
-73.893847844 -40.9224281311035 6.0
85.1207335 79.3961486816406 7.0
60.884535001 51.1230316162109 8.0
24.236198488 23.6634693145752 9.0
91.798538036 85.4269561767578 10.0
78.291608073 74.0266647338867 11.0
82.359161058 80.6502990722656 12.0
82.359161058 83.6718139648438 13.0
82.359161058 79.0970153808594 14.0
0 -0.806605160236359 15.0
0 -0.522596180438995 16.0
-7.6330452278e-14 -2.25158309936523 17.0
58.220490291 56.7839851379395 18.0
-58.220490311 -52.3093147277832 19.0
-58.220490289 -60.0571441650391 20.0
-58.220490301 -55.8244514465332 21.0
78.957086036 77.5180969238281 22.0
131.06054296 140.18180847168 23.0
99.726129996 113.035057067871 24.0
20.107377957 22.6023998260498 25.0
210.01762901 227.869766235352 26.0
210.01762901 225.881713867188 27.0
27.831755148 28.5244598388672 28.0
237.84938416 251.537567138672 29.0
237.84938416 247.640365600586 30.0
46.983341668 47.7649230957031 31.0
9.4516880729 9.29814434051514 32.0
20.736595725 18.758358001709 33.0
20.736595725 13.1751203536987 34.0
20.736595725 23.3286209106445 35.0
8.6701210208 14.7594375610352 36.0
7.4039490815 7.23575305938721 37.0
-1.2231621506e-13 -6.47165155410767 38.0
-1.2231621506e-13 1.58286809921265 39.0
-11.157888397 -19.9507312774658 40.0
-24.742701918 -23.5566711425781 41.0
5.6258182819 9.20473766326904 42.0
-33.116505879 -37.2065696716309 43.0
99.824925014 94.5742950439453 44.0
99.824925027 92.1477661132812 45.0
6.3383794196 5.71562099456787 46.0
2.5755228733 2.79513144493103 47.0
181.27880343 161.587768554688 48.0
175.28272055 159.515991210938 49.0
0 -4.52954339981079 50.0
85.12073351 75.1525650024414 51.0
91.798538046 87.8462448120117 52.0
82.359161068 80.0779800415039 53.0
82.359161068 75.6326446533203 54.0
82.359161068 76.8956298828125 55.0
0 -0.702819049358368 56.0
78.957086046 79.7414398193359 57.0
210.01762902 220.559600830078 58.0
237.84938417 246.47233581543 59.0
20.736595735 25.3904685974121 60.0
0 1.22404217720032 61.0
157.961808219909 160.42024230957 0.0
157.961808219909 163.259094238281 1.0
156.647955 160.145233154297 2.0
151.85888521 150.161758422852 3.0
0 -2.67139410972595 4.0
-4.77400117000001e-09 0.848523676395416 5.0
-64.8398559171454 -60.9600257873535 6.0
76.3705552425091 79.3763122558594 7.0
47.0920143985091 56.8492813110352 8.0
29.2785408435091 25.2860927581787 9.0
88.6745185525091 89.7431411743164 10.0
75.8372005315091 77.8119583129883 11.0
83.8087447565091 81.4020233154297 12.0
83.8087447565091 79.2996139526367 13.0
83.8087447565091 80.1181564331055 14.0
7.78111765000001e-10 -1.79699897766113 15.0
7.78111764999999e-10 -0.814296662807465 16.0
2.372650157e-08 0.206989169120789 17.0
60.6078662995504 58.3393592834473 18.0
-60.6078662107636 -58.8892860412598 19.0
-60.6078661957636 -58.501392364502 20.0
-60.6078661477636 -59.178295135498 21.0
82.0925553208403 78.9327545166016 22.0
144.93187403355 141.506317138672 23.0
111.186674674438 108.447463989258 24.0
23.840689336438 19.9385356903076 25.0
227.024429349818 226.233673095703 26.0
227.024429349818 223.504547119141 27.0
26.5455084538181 25.3960151672363 28.0
253.569937809818 257.518341064453 29.0
253.569937809818 258.653778076172 30.0
46.7881837187405 46.8427581787109 31.0
9.36499621630177 8.94511413574219 32.0
21.4846891734364 19.4050960540771 33.0
21.4846891734364 18.5549926757812 34.0
21.4846892445203 21.1903457641602 35.0
9.31401654804894 8.91484355926514 36.0
8.08399132364894 7.79282474517822 37.0
-3.33720051999997e-09 -5.73064422607422 38.0
-3.33720053000004e-09 -2.24365162849426 39.0
-12.1302714518818 -14.9110012054443 40.0
-24.7042221855809 -26.0708904266357 41.0
5.02433739725762 5.21726369857788 42.0
-32.3929829867576 -34.0537948608398 43.0
98.5130089577575 92.1992416381836 44.0
98.5130083228782 95.2472839355469 45.0
6.84103154913597 6.83663749694824 46.0
2.47691726153597 2.32861948013306 47.0
157.96180822 161.301879882812 48.0
151.85888522 150.782928466797 49.0
0 -0.0950888991355896 50.0
76.370555242 75.9614410400391 51.0
88.674518552 91.2663116455078 52.0
83.808744757 81.2053756713867 53.0
83.808744757 79.3992538452148 54.0
83.808744757 79.6827392578125 55.0
2.3726501569e-08 1.97269725799561 56.0
82.092555321 79.6980133056641 57.0
227.02442935 219.908187866211 58.0
253.56993781 251.606750488281 59.0
21.484689244 24.1281318664551 60.0
0 1.79337692260742 61.0
165.91333431 165.989074707031 0.0
165.91333431 165.382766723633 1.0
164.85491325 161.222320556641 2.0
160.06584352 160.244125366211 3.0
0 3.71303486824036 4.0
-9.5499982473e-10 1.7773734331131 5.0
-71.664264161 -65.6970596313477 6.0
85.26876738 73.8433074951172 7.0
40.407049605 42.7232055664062 8.0
44.861717775 23.7758960723877 9.0
70.69684861 85.8271331787109 10.0
59.47694646 74.3417663574219 11.0
76.283465957 81.4239730834961 12.0
76.283465957 81.6078338623047 13.0
76.283465957 79.3901519775391 14.0
0 -0.440466821193695 15.0
0 0.435059428215027 16.0
-5.0042869294e-09 0.0723074674606323 17.0
59.765125017 57.7593574523926 18.0
-59.7651250407191 -57.5140838623047 19.0
-59.7651250187191 -59.9023170471191 20.0
-59.7651250297191 -56.0849342346191 21.0
78.972612228 76.0455627441406 22.0
138.22205466 141.996627807617 23.0
99.057356638 109.940643310547 24.0
11.705180701 22.0290355682373 25.0
217.194666899784 225.849166870117 26.0
217.194666899784 222.296600341797 27.0
28.8094353527839 26.6014232635498 28.0
246.004102249784 254.182037353516 29.0
246.004102249784 253.347518920898 30.0
47.343202682 46.6320343017578 31.0
9.3218295817 8.62320041656494 32.0
19.207487203 20.0475921630859 33.0
19.207487203 16.8119983673096 34.0
19.2074872 20.6830234527588 35.0
10.212255531 8.6940450668335 36.0
8.9458556559 8.0379638671875 37.0
-8.7105039606e-09 -0.524441301822662 38.0
-8.7105038714e-09 -4.77853488922119 39.0
-10.939001042 -13.1231002807617 40.0
-25.42583068 -26.8582649230957 41.0
4.3808906211 6.42677450180054 42.0
-31.969943109 -33.678638458252 43.0
73.457270044 94.1996459960938 44.0
73.457270044 95.3253555297852 45.0
5.6573354231 6.44148111343384 46.0
1.248980579 2.15072870254517 47.0
165.91333431 166.156860351562 48.0
160.06584353 155.285873413086 49.0
0 0.809720098972321 50.0
85.268767382 77.1408233642578 51.0
70.696848611 85.3040618896484 52.0
76.283465959 80.8775939941406 53.0
76.283465959 81.2314453125 54.0
76.283465959 80.2796325683594 55.0
0 -0.597361743450165 56.0
78.972612236 79.67236328125 57.0
217.1946669 220.908843994141 58.0
246.00410225 248.17951965332 59.0
19.207487207 20.576997756958 60.0
0 1.74727439880371 61.0
155.637104339877 166.587844848633 0.0
155.637104339877 167.282989501953 1.0
154.16776254 158.562973022461 2.0
148.70312733 153.517578125 3.0
1.0646380247 -8.35963344573975 4.0
1.06463803428188 8.99030303955078 5.0
-72.0621086190819 -59.6013946533203 6.0
81.1416724867866 71.127571105957 7.0
45.0808907437866 49.1983909606934 8.0
36.0607817427866 27.0397129058838 9.0
96.3374243067866 88.5754699707031 10.0
83.4733926307866 80.3317260742188 11.0
84.8869963567866 85.8347778320312 12.0
84.8869963567866 74.3229675292969 13.0
84.8869963567866 75.0870208740234 14.0
0 -1.87260580062866 15.0
0 0.332335472106934 16.0
-5.000964814e-09 -3.16841459274292 17.0
52.97538748252 54.2587890625 18.0
-52.9753874886799 -60.3420715332031 19.0
-52.9753874676799 -59.998851776123 20.0
-52.9753874726799 -56.2374687194824 21.0
74.6083269968883 77.8691635131836 22.0
143.16526813152 138.994964599609 23.0
111.5192151444 108.341621398926 24.0
22.7148384633999 19.5548839569092 25.0
217.773595129754 224.018936157227 26.0
217.773595129754 218.673583984375 27.0
25.1968256877537 24.1853427886963 28.0
242.970420819754 250.491668701172 29.0
242.970420819754 245.156829833984 30.0
46.9983026877922 44.915599822998 31.0
9.23239878156208 9.04692268371582 32.0
21.6329395232456 17.7893486022949 33.0
21.6329395232456 16.4908485412598 34.0
21.6329395319954 18.723669052124 35.0
10.260757919 10.3768997192383 36.0
8.9385966191 4.58684825897217 37.0
-9.143116014e-09 -7.1552939414978 38.0
-9.143116015e-09 -5.90771293640137 39.0
-11.0985178068399 -13.0565195083618 40.0
-24.5551210074508 -25.7593898773193 41.0
4.84337531208882 0.10114187002182 42.0
-32.1074139234888 -34.5739898681641 43.0
94.7377572604888 88.0042724609375 44.0
94.7377571959425 94.2622756958008 45.0
6.9596509637 2.46823143959045 46.0
2.5424144271 1.05478036403656 47.0
155.63710434 164.383422851562 48.0
148.70312734 153.2255859375 49.0
1.0646380343 -7.11886167526245 50.0
81.141672487 76.8535919189453 51.0
96.337424307 89.3755645751953 52.0
84.886996356 78.3330612182617 53.0
84.886996356 81.2938690185547 54.0
84.886996356 79.3526000976562 55.0
0 3.55653500556946 56.0
74.608326997 82.4492034912109 57.0
217.77359513 219.920745849609 58.0
242.97042082 245.884155273438 59.0
21.632939532 27.8692989349365 60.0
0 2.8526463508606 61.0
147.982033057583 157.456359863281 0.0
147.982033057583 151.963821411133 1.0
146.808413813287 153.349899291992 2.0
142.01934397 150.726043701172 3.0
-1.55484310406e-09 -0.649782240390778 4.0
-1.55484309999997e-09 -2.84008049964905 5.0
-64.9693040339936 -64.3212738037109 6.0
77.6547829144777 74.1871871948242 7.0
61.2786926634778 55.4030647277832 8.0
16.3760902514777 7.20984888076782 9.0
97.5261141374777 89.7391510009766 10.0
83.6472874924777 79.6450042724609 11.0
85.2981411144777 83.8642730712891 12.0
85.2981411144777 76.4624557495117 13.0
85.2981411144777 81.5436782836914 14.0
2.3328021736e-08 -2.03496789932251 15.0
2.3328021735e-08 -1.6052393913269 16.0
5.464350236e-08 -2.72059202194214 17.0
56.2518018559467 59.0783615112305 18.0
-56.2518018797166 -57.0891418457031 19.0
-56.2518018707166 -59.2854194641113 20.0
-56.2518017707166 -58.115608215332 21.0
77.1875821073724 73.3798065185547 22.0
153.454771802806 145.008895874023 23.0
115.220596133508 115.750228881836 24.0
21.7005584165075 20.8819332122803 25.0
230.642353915167 228.381210327148 26.0
230.642353915167 229.708221435547 27.0
26.8883420391666 21.3647422790527 28.0
257.530695955167 253.23681640625 29.0
257.530695955167 255.336776733398 30.0
46.7385281475469 45.537841796875 31.0
9.28724497493877 8.4355525970459 32.0
20.9357803369809 22.6376686096191 33.0
20.9357803369809 19.0790691375732 34.0
20.935780467613 20.4264030456543 35.0
8.86054244871795 8.39873600006104 36.0
7.60121916461794 6.85336446762085 37.0
2.62465480000001e-09 1.1959273815155 38.0
2.62465480000001e-09 -1.17553234100342 39.0
-11.6034306368583 -12.5643882751465 40.0
-24.5599220844463 -26.7411251068115 41.0
4.75986482786468 1.92820739746094 42.0
-31.7734218810648 -31.6524333953857 43.0
89.4696777832616 86.4044494628906 44.0
89.4696781210649 91.9600372314453 45.0
7.05809588635643 4.443039894104 46.0
2.46335994485643 2.26466846466064 47.0
147.98203306 156.096588134766 48.0
142.01934397 152.992614746094 49.0
0 -3.7475733757019 50.0
77.654782914 76.0236663818359 51.0
97.526114137 92.0755157470703 52.0
85.298141114 81.5651702880859 53.0
85.298141114 80.2131118774414 54.0
85.298141114 79.9203643798828 55.0
5.4643502359e-08 3.07315587997437 56.0
77.187582108 79.4883346557617 57.0
230.64235391 222.583419799805 58.0
257.53069595 249.491134643555 59.0
20.935780468 19.6116352081299 60.0
2.6246548239e-09 3.64096260070801 61.0
168.22939792 161.80744934082 0.0
168.22939792 161.741287231445 1.0
166.8551853 161.987121582031 2.0
162.06611558 154.941513061523 3.0
0 -3.38102102279663 4.0
-3.4430194723e-10 1.14741683006287 5.0
-64.717524185 -64.5426864624023 6.0
75.795757451 73.5059967041016 7.0
47.811512438 51.4997253417969 8.0
27.984245006 18.8855247497559 9.0
85.495582132 90.9133224487305 10.0
72.806800886 77.6438293457031 11.0
81.388609367 82.83935546875 12.0
81.388609367 79.3970413208008 13.0
81.388609367 81.0634994506836 14.0
0 -1.68036651611328 15.0
0 -0.587034523487091 16.0
0 -0.773277342319489 17.0
57.030575553 59.3173522949219 18.0
-57.030575568 -59.4718246459961 19.0
-57.030575546 -58.5947761535645 20.0
-57.030575558 -58.781867980957 21.0
78.468031474 78.1894683837891 22.0
143.48040884 144.007369995117 23.0
107.74571813 112.145500183105 24.0
21.65787533 20.3248405456543 25.0
221.94844032 221.194931030273 26.0
221.94844032 224.770721435547 27.0
25.034615474 25.1922798156738 28.0
246.9830558 249.78190612793 29.0
246.9830558 252.521362304688 30.0
46.868159797 47.6341247558594 31.0
9.3514247421 8.4442024230957 32.0
21.437455912 17.6786956787109 33.0
21.437455912 17.721212387085 34.0
21.437455911 19.6689872741699 35.0
9.7189938178 8.14753437042236 36.0
8.4390559148 8.11818408966064 37.0
-1.6180830419e-09 -0.13116979598999 38.0
-1.6180829237e-09 -1.77909874916077 39.0
-11.673813193 -18.1659965515137 40.0
-24.220127003 -25.3308277130127 41.0
5.2382433149 7.20737838745117 42.0
-32.062704899 -33.6589088439941 43.0
100.08206036 93.2306213378906 44.0
100.08206037 95.3342514038086 45.0
7.2221543507 7.74946308135986 46.0
2.5721167859 1.37758088111877 47.0
168.22939792 164.439270019531 48.0
162.06611559 148.876495361328 49.0
0 1.03792572021484 50.0
75.795757458 74.9175415039062 51.0
85.495582138 90.5932464599609 52.0
81.388609374 80.7393035888672 53.0
81.388609374 80.5549392700195 54.0
81.388609374 80.2425765991211 55.0
0 3.7588357925415 56.0
78.468031483 80.0088272094727 57.0
221.94844033 219.556533813477 58.0
246.98305581 247.573440551758 59.0
21.43745592 21.4858741760254 60.0
0 0.854927837848663 61.0
164.84154879 161.271331787109 0.0
164.84154879 167.210983276367 1.0
163.47930919 163.794281005859 2.0
158.69023947 160.321548461914 3.0
0 0.706295907497406 4.0
-4.0468769349e-14 2.94811487197876 5.0
-58.491897564 -60.9163131713867 6.0
70.357512535 73.8166656494141 7.0
51.232337208 47.436393737793 8.0
19.125175317 23.2492370605469 9.0
87.159604405 86.8559265136719 10.0
73.533108674 75.3401489257812 11.0
80.018953901 84.8341674804688 12.0
80.018953901 82.0210113525391 13.0
80.018953901 80.3478546142578 14.0
0 -1.05221343040466 15.0
0 -0.00106298923492432 16.0
-7.4578979994e-14 0.0666258335113525 17.0
58.704611565 60.0535774230957 18.0
-58.704611585 -59.343391418457 19.0
-58.704611563 -59.738208770752 20.0
-58.704611575 -56.8117065429688 21.0
80.467223632 75.9871063232422 22.0
146.76373537 144.755950927734 23.0
113.17330386 107.855262756348 24.0
23.084633007 18.6483631134033 25.0
227.23095901 227.412673950195 26.0
227.23095901 226.498336791992 27.0
24.88227757 25.9784355163574 28.0
252.11323659 254.187118530273 29.0
252.11323659 253.853805541992 30.0
46.981384966 45.9723167419434 31.0
9.3542172718 9.00819873809814 32.0
21.762612047 19.2253379821777 33.0
21.762612047 21.2942810058594 34.0
21.762612047 20.931568145752 35.0
9.5787196354 9.75351333618164 36.0
8.3747943394 7.46340417861938 37.0
-1.1902268272e-13 0.403506875038147 38.0
-1.1902268272e-13 -1.78502511978149 39.0
-11.784172657 -10.8179597854614 40.0
-25.081070163 -26.5109004974365 41.0
4.9115430991 5.38447904586792 42.0
-32.709799887 -34.3439903259277 43.0
94.81106231 90.1027069091797 44.0
94.811062324 94.771728515625 45.0
6.8163383027 4.88591718673706 46.0
2.6631233094 1.97947812080383 47.0
164.8415488 161.909332275391 48.0
158.69023948 157.754364013672 49.0
0 2.15846681594849 50.0
70.357512545 79.6644287109375 51.0
87.159604415 87.633056640625 52.0
80.018953911 81.2675323486328 53.0
80.018953911 81.6069793701172 54.0
80.018953911 80.8037719726562 55.0
0 1.42237138748169 56.0
80.467223642 79.4640274047852 57.0
227.23095902 221.368530273438 58.0
252.1132366 249.44482421875 59.0
21.762612057 19.881010055542 60.0
0 3.32653045654297 61.0
174.68306493 161.475463867188 0.0
174.68306493 168.026153564453 1.0
173.39491555 160.259262084961 2.0
168.60584583 165.641784667969 3.0
0 3.31988906860352 4.0
-6.0679551382e-13 2.94897079467773 5.0
-60.89541916 -60.703441619873 6.0
72.773688817 78.7125244140625 7.0
45.221305118 52.774112701416 8.0
27.552383689 21.4295654296875 9.0
83.34434583 87.6958847045898 10.0
70.832916922 75.0015258789062 11.0
79.159399821 81.0632934570312 12.0
79.159399821 80.3233108520508 13.0
79.159399821 79.9422378540039 14.0
0 0.213598191738129 15.0
0 0.529818534851074 16.0
-1.1308196494e-12 0.332491755485535 17.0
61.361266496 59.5489158630371 18.0
-61.361266516 -59.0195503234863 19.0
-61.361266494 -59.786922454834 20.0
-61.361266507 -61.6598358154297 21.0
83.62356725 80.1228561401367 22.0
139.95238012 141.843566894531 23.0
105.87904897 109.555274963379 24.0
22.128026702 22.2596817016602 25.0
223.57594738 226.605941772461 26.0
223.57594738 224.994171142578 27.0
25.845937869 27.3473987579346 28.0
249.42188525 254.101257324219 29.0
249.42188525 254.581497192383 30.0
46.874844231 46.1727447509766 31.0
9.3907165758 8.84638214111328 32.0
22.262300733 19.816463470459 33.0
22.262300733 17.8013515472412 34.0
22.262300733 20.5926704406738 35.0
9.5287428384 7.75536870956421 36.0
8.2499879739 7.2976131439209 37.0
-1.8121307213e-12 1.96029734611511 38.0
-1.8121307213e-12 0.92163497209549 39.0
-11.830747483 -15.2654428482056 40.0
-25.331477775 -23.1388530731201 41.0
4.9843583267 4.84584712982178 42.0
-32.746417896 -32.8084983825684 43.0
97.804317732 94.9270095825195 44.0
97.804317745 95.4181671142578 45.0
7.2015837174 5.89092683792114 46.0
2.5857712967 2.30231428146362 47.0
174.68306494 164.92985534668 48.0
168.60584584 157.646179199219 49.0
0 -0.0948031544685364 50.0
72.773688827 77.6631240844727 51.0
83.34434584 90.7836685180664 52.0
79.159399831 81.3405609130859 53.0
79.159399831 81.0017242431641 54.0
79.159399831 80.0534515380859 55.0
0 0.0413513779640198 56.0
83.62356726 80.1179656982422 57.0
223.57594739 224.142547607422 58.0
249.42188526 249.183868408203 59.0
22.262300743 21.9204692840576 60.0
0 1.20121669769287 61.0
147.982033057583 157.456390380859 0.0
147.982033057583 151.964889526367 1.0
146.808413813287 153.359130859375 2.0
142.01934397 150.795227050781 3.0
-1.55484310406e-09 -0.657764256000519 4.0
-1.55484309999997e-09 -2.84302091598511 5.0
-64.9693040339936 -64.32421875 6.0
77.6547829144777 74.1872253417969 7.0
61.2786926634778 55.4030647277832 8.0
16.3760902514777 7.20984888076782 9.0
97.5261141374777 89.7391510009766 10.0
83.6472874924777 79.6450042724609 11.0
85.2981411144777 83.8642730712891 12.0
85.2981411144777 76.4624557495117 13.0
85.2981411144777 81.5436782836914 14.0
2.3328021736e-08 -2.03496789932251 15.0
2.3328021735e-08 -1.6052393913269 16.0
5.464350236e-08 -2.72059202194214 17.0
56.2518018559467 59.0783615112305 18.0
-56.2518018797166 -57.0891418457031 19.0
-56.2518018707166 -59.2854194641113 20.0
-56.2518017707166 -58.115608215332 21.0
77.1875821073724 73.3798065185547 22.0
153.454771802806 145.008895874023 23.0
115.220596133508 115.750228881836 24.0
21.7005584165075 20.8819332122803 25.0
230.642353915167 228.381210327148 26.0
230.642353915167 229.708221435547 27.0
26.8883420391666 21.3647422790527 28.0
257.530695955167 253.23681640625 29.0
257.530695955167 255.336776733398 30.0
46.7385281475469 45.537841796875 31.0
9.28724497493877 8.4355525970459 32.0
20.9357803369809 22.6376686096191 33.0
20.9357803369809 19.0790691375732 34.0
20.935780467613 20.4264030456543 35.0
8.86054244871795 8.39873600006104 36.0
7.60121916461794 6.85336446762085 37.0
2.62465480000001e-09 1.1959273815155 38.0
2.62465480000001e-09 -1.17553234100342 39.0
-11.6034306368583 -12.5643882751465 40.0
-24.5599220844463 -26.7411251068115 41.0
4.75986482786468 1.92820739746094 42.0
-31.7734218810648 -31.6524333953857 43.0
89.4696777832616 86.4044494628906 44.0
89.4696781210649 91.9600372314453 45.0
7.05809588635643 4.443039894104 46.0
2.46335994485643 2.26466846466064 47.0
147.98203306 156.096145629883 48.0
142.01934397 153.049011230469 49.0
0 -3.74314832687378 50.0
77.654782914 76.0234603881836 51.0
97.526114137 92.0755157470703 52.0
85.298141114 81.5651702880859 53.0
85.298141114 80.2131118774414 54.0
85.298141114 79.9203643798828 55.0
5.4643502359e-08 3.07315587997437 56.0
77.187582108 79.4883346557617 57.0
230.64235391 222.583419799805 58.0
257.53069595 249.491134643555 59.0
20.935780468 19.6116352081299 60.0
2.6246548239e-09 3.64096260070801 61.0
155.439662787599 163.337280273438 0.0
155.439662787599 165.126602172852 1.0
154.30230653 160.011566162109 2.0
149.51323678 143.671264648438 3.0
0 -0.321387112140656 4.0
-6.420204037e-09 4.96934270858765 5.0
-58.8392240852395 -60.8455009460449 6.0
68.7627721478381 75.1139755249023 7.0
54.5942969048381 49.8149223327637 8.0
14.1684752428381 27.1934108734131 9.0
86.0010369738381 89.7719345092773 10.0
73.5807551968381 77.7846832275391 11.0
80.4739642328381 81.5004959106445 12.0
80.4739642328381 81.3285064697266 13.0
80.4739642328381 78.5678024291992 14.0
3.9152147791e-09 -1.1339225769043 15.0
0 0.278671145439148 16.0
2.61172055999999e-09 -1.03324484825134 17.0
60.468305371639 58.3043632507324 18.0
-60.4683053517572 -59.2306175231934 19.0
-60.4683053317572 -60.1124000549316 20.0
-60.4683053217572 -57.6546058654785 21.0
81.682429163713 77.8001327514648 22.0
146.379765302639 143.742172241211 23.0
110.062697210799 112.264030456543 24.0
22.4296901667987 21.9177856445312 25.0
228.062194465198 223.014770507812 26.0
228.062194465198 222.323669433594 27.0
27.9351146371978 26.8501014709473 28.0
255.997309095198 257.388793945312 29.0
255.997309095198 257.382690429688 30.0
46.7240133833509 47.4519462585449 31.0
9.36354750899571 9.32487487792969 32.0
21.2141238417184 17.8097953796387 33.0
21.2141238417184 23.0975475311279 34.0
21.2141238677718 20.5346145629883 35.0
9.1469342787689 5.46794748306274 36.0
7.9580722208689 6.99554109573364 37.0
-7.47642845999998e-09 2.02303838729858 38.0
-7.47642844999999e-09 -2.50212097167969 39.0
-11.6769694638786 -12.8153533935547 40.0
-25.0182964125618 -24.042085647583 41.0
5.33947917523177 5.32516098022461 42.0
-32.9808792835318 -33.7839431762695 43.0
98.2423104375318 96.9407958984375 44.0
98.2423103395318 94.5310745239258 45.0
6.8270840345 8.05819129943848 46.0
2.6492598031 1.59819686412811 47.0
155.43966279 164.126281738281 48.0
149.51323679 143.260803222656 49.0
0 2.17118644714355 50.0
68.762772148 75.5315780639648 51.0
86.001036974 90.9975204467773 52.0
80.473964233 80.8694305419922 53.0
80.473964233 81.3702163696289 54.0
80.473964233 78.6575012207031 55.0
2.6117205574e-09 -0.555000841617584 56.0
81.682429164 79.3925399780273 57.0
228.06219446 221.316833496094 58.0
255.9973091 247.795120239258 59.0
21.214123868 21.6998443603516 60.0
0 1.56267380714417 61.0
177.78446364 156.513153076172 0.0
177.78446364 162.869171142578 1.0
176.41996462 164.206665039062 2.0
170.95532886 164.239196777344 3.0
0.67748415553 -0.559761583805084 4.0
0.67748415571 3.64633440971375 5.0
-67.093244272 -65.6517486572266 6.0
78.435453126 76.1014633178711 7.0
47.637388567 52.1387939453125 8.0
30.798064549 22.1250038146973 9.0
87.539372279 83.8738327026367 10.0
74.810642459 72.9772796630859 11.0
80.07931631 82.5814590454102 12.0
80.07931631 79.0310363769531 13.0
80.07931631 80.2618713378906 14.0
0 -3.07081294059753 15.0
0 -0.696132242679596 16.0
0 0.481258988380432 17.0
56.798777369 59.2578811645508 18.0
-56.798777379 -58.2444801330566 19.0
-56.798777358 -59.7374267578125 20.0
-56.79877737 -60.0397300720215 21.0
79.007105484 77.9094543457031 22.0
139.41872599 141.624328613281 23.0
105.91592169 108.725219726562 24.0
21.441916501 19.6990737915039 25.0
218.42583148 219.787170410156 26.0
218.42583148 221.636932373047 27.0
25.944059549 27.3495445251465 28.0
244.36989104 254.787109375 29.0
244.36989104 257.717132568359 30.0
46.971400309 46.0481796264648 31.0
9.3706955991 8.90136432647705 32.0
22.208328106 18.9254245758057 33.0
22.208328106 17.5182361602783 34.0
22.208328105 20.1714420318604 35.0
9.9893807486 8.91634082794189 36.0
8.7115742034 6.50516557693481 37.0
-4.7818052895e-10 -0.401360809803009 38.0
-4.781805745e-10 -2.8292875289917 39.0
-11.519078161 -14.9320249557495 40.0
-23.996323586 -25.023458480835 41.0
4.9469242204 7.04139471054077 42.0
-31.392111544 -34.7502136230469 43.0
90.381846896 88.4025497436523 44.0
90.381846908 94.4198608398438 45.0
6.6804054069 6.9216194152832 46.0
2.5042770677 2.66599488258362 47.0
177.78446365 159.377655029297 48.0
170.95532887 161.026489257812 49.0
0.67748416553 -2.41484403610229 50.0
78.435453135 78.5746688842773 51.0
87.539372289 85.3046493530273 52.0
80.079316319 82.0100402832031 53.0
80.079316319 80.4009704589844 54.0
80.079316319 79.7129440307617 55.0
0 2.82733631134033 56.0
79.007105494 78.9993667602539 57.0
218.42583149 220.153472900391 58.0
244.36989105 249.837310791016 59.0
22.208328115 23.438383102417 60.0
0 2.2962954044342 61.0
157.36929389 164.27229309082 0.0
157.36929389 168.359237670898 1.0
156.19034304 163.481307983398 2.0
151.40127332 149.99658203125 3.0
0 0.881423652172089 4.0
-4.4001625206e-12 1.28015303611755 5.0
-63.202481494 -62.7311019897461 6.0
73.940932648 75.2346725463867 7.0
49.21522014 48.8494338989258 8.0
24.725712498 25.1796379089355 9.0
82.071308945 87.9838638305664 10.0
69.710579191 74.2460708618164 11.0
78.268452594 81.4736785888672 12.0
78.268452594 78.1099090576172 13.0
78.268452594 78.0607223510742 14.0
0 -1.73987436294556 15.0
0 0.20755273103714 16.0
-8.6498091718e-12 2.28687357902527 17.0
55.991161147 58.3939933776855 18.0
-55.991161167 -58.2031936645508 19.0
-55.991161145 -60.6356239318848 20.0
-55.991161157 -58.4873924255371 21.0
78.143484672 77.9015045166016 22.0
143.15317945 142.212677001953 23.0
109.4842638 109.648582458496 24.0
23.643519739 20.927677154541 25.0
221.29666413 228.536636352539 26.0
221.29666413 226.047409057617 27.0
25.631460129 25.2963237762451 28.0
246.92812427 256.373504638672 29.0
246.92812427 258.433868408203 30.0
46.797891875 47.0584945678711 31.0
9.3851534707 8.83886432647705 32.0
22.152323504 20.3074417114258 33.0
22.152323504 18.4097843170166 34.0
22.152323504 20.0688018798828 35.0
9.5374703059 8.62298393249512 36.0
8.2534785553 7.0811595916748 37.0
-1.756954887e-11 -2.27432584762573 38.0
-1.7569546978e-11 -3.01892614364624 39.0
-11.801832176 -13.6235008239746 40.0
-24.540615721 -25.7453060150146 41.0
5.2053856641 7.12257862091064 42.0
-32.287741837 -33.7818374633789 43.0
101.53993446 95.6819458007812 44.0
101.53993448 93.4483795166016 45.0
6.8535260139 7.25196695327759 46.0
2.5912805594 2.4670615196228 47.0
157.3692939 164.761383056641 48.0
151.40127333 146.843017578125 49.0
0 0.856912851333618 50.0
73.940932658 76.8228073120117 51.0
82.071308955 87.3712387084961 52.0
78.268452604 81.4715957641602 53.0
78.268452604 81.4059982299805 54.0
78.268452604 79.6282348632812 55.0
0 1.05015206336975 56.0
78.143484682 78.9081726074219 57.0
221.29666414 222.131652832031 58.0
246.92812428 249.281433105469 59.0
22.152323514 20.3138408660889 60.0
0 2.08443331718445 61.0
176.29860482 168.580444335938 0.0
176.29860482 173.377548217773 1.0
175.16865358 163.713333129883 2.0
170.37958384 161.690292358398 3.0
0 3.24395108222961 4.0
0 5.51086568832397 5.0
-65.270666863 -62.9606819152832 6.0
76.762728385 76.8805313110352 7.0
42.488702145 51.4834098815918 8.0
34.274026231 19.5944557189941 9.0
89.516854999 94.0824127197266 10.0
77.663058224 73.4254837036133 11.0
78.940936199 80.7382888793945 12.0
78.940936199 79.2204360961914 13.0
78.940936199 80.9472503662109 14.0
-1.1623383031e-08 -2.45532608032227 15.0
0 -1.02141404151917 16.0
0 -1.96145725250244 17.0
55.062952957 59.437572479248 18.0
-55.062952969 -58.469108581543 19.0
-55.062952947 -58.6026077270508 20.0
-55.062952959 -59.3681259155273 21.0
76.457871931 78.7075500488281 22.0
148.98639806 139.289047241211 23.0
112.48565108 113.441032409668 24.0
20.324403424 19.6628036499023 25.0
225.44427001 226.562286376953 26.0
225.44427001 223.700317382812 27.0
29.722881407 30.1496410369873 28.0
255.16715142 253.909317016602 29.0
255.16715142 254.789642333984 30.0
46.378282561 45.0921783447266 31.0
9.4329365578 9.13247203826904 32.0
21.394918966 15.1990261077881 33.0
21.394918966 24.057279586792 34.0
21.394918964 21.1588687896729 35.0
9.3126087622 4.91349315643311 36.0
8.1259553792 7.98107290267944 37.0
-7.4149137691e-10 0.0585658550262451 38.0
-7.4150109029e-10 0.308051109313965 39.0
-11.606094901 -17.6695861816406 40.0
-24.942731654 -25.4256763458252 41.0
5.6454245049 4.46205472946167 42.0
-33.259090558 -30.9183597564697 43.0
91.727018377 88.2412948608398 44.0
91.727018389 95.6060180664062 45.0
6.4433716383 8.38315296173096 46.0
2.4124310659 2.25329637527466 47.0
176.29860483 167.677124023438 48.0
170.37958385 168.160797119141 49.0
0 -0.781177580356598 50.0
76.762728393 74.7907485961914 51.0
89.516855007 88.8126525878906 52.0
78.940936207 80.8297119140625 53.0
78.940936207 76.5449142456055 54.0
78.940936207 78.9845657348633 55.0
5.7767483422e-12 1.50308060646057 56.0
76.45787194 81.0239868164062 57.0
225.44427001 220.678131103516 58.0
255.16715143 251.577133178711 59.0
21.394918974 21.5973052978516 60.0
0 2.14086961746216 61.0
158.05603952 167.419097900391 0.0
158.05603952 162.144897460938 1.0
156.52449475 147.829681396484 2.0
151.73542499 145.744750976562 3.0
0 5.36371612548828 4.0
0 -12.1541585922241 5.0
-61.630763191 -79.925666809082 6.0
74.018161276 82.4157943725586 7.0
46.718249105 31.2298469543457 8.0
27.299912161 18.2428722381592 9.0
82.810333666 90.9979476928711 10.0
68.619616017 70.1046676635742 11.0
70.069503943 77.1744689941406 12.0
70.069503943 83.4726028442383 13.0
70.069503943 79.1923065185547 14.0
0 9.53188228607178 15.0
0 -3.81320858001709 16.0
0 1.72674071788788 17.0
58.053432727 56.0061416625977 18.0
-58.053432737 -58.0427513122559 19.0
-58.053432715 -54.4132308959961 20.0
-58.053432728 -53.3800659179688 21.0
80.382500248 76.5772247314453 22.0
148.69019065 143.49446105957 23.0
110.3901322 106.363052368164 24.0
19.532157938 19.8203716278076 25.0
229.07269091 221.509384155273 26.0
229.07269091 217.391937255859 27.0
25.772358325 16.2064552307129 28.0
254.84504924 258.373016357422 29.0
254.84504924 247.268295288086 30.0
46.821077276 54.5309410095215 31.0
9.20590207 10.2766008377075 32.0
22.329067511 16.1239490509033 33.0
22.329067511 12.5439128875732 34.0
22.329067511 20.7414588928223 35.0
10.657124263 7.92519903182983 36.0
9.468907638 7.88749694824219 37.0
-1.1372044016e-13 -4.84613418579102 38.0
-1.1372671877e-13 0.525445520877838 39.0
-10.407142258 -21.8385734558105 40.0
-22.183119932 -23.5663261413574 41.0
5.8842323768 1.86289405822754 42.0
-30.148246898 -34.6149749755859 43.0
111.15285693 95.6768493652344 44.0
111.15285694 90.0571899414062 45.0
7.0547961518 18.3261394500732 46.0
2.9488525526 2.98372411727905 47.0
158.05603953 162.360580444336 48.0
151.735425 146.356735229492 49.0
0 10.0269107818604 50.0
74.018161286 74.59228515625 51.0
82.810333676 83.3695297241211 52.0
70.069503953 74.5173645019531 53.0
70.069503953 74.0831756591797 54.0
70.069503953 78.7452087402344 55.0
0 -0.377448379993439 56.0
80.382500258 78.0791625976562 57.0
229.07269092 219.218109130859 58.0
254.84504925 248.735687255859 59.0
22.329067521 18.9575977325439 60.0
0 4.22753810882568 61.0
137.17687742 143.006317138672 0.0
137.17687742 143.037628173828 1.0
135.86355256 143.48908996582 2.0
131.0744823 135.680419921875 3.0
0 -2.02137804031372 4.0
0 0.341359198093414 5.0
-59.42861149 -59.2226219177246 6.0
69.705106679 70.1771240234375 7.0
65.43029471 62.4528312683105 8.0
4.2748119615 8.48783111572266 9.0
78.669161727 80.2065200805664 10.0
65.229460649 71.2585144042969 11.0
77.500548395 80.8047180175781 12.0
77.500548395 77.9000701904297 13.0
77.500548395 76.9198608398438 14.0
0 -0.64198762178421 15.0
0 0.28728449344635 16.0
0 -3.15485000610352 17.0
59.754945382 58.5015258789062 18.0
-59.754945394 -59.4383087158203 19.0
-59.754945372 -60.0649642944336 20.0
-59.754945384 -61.8801422119141 21.0
80.656571382 77.8813858032227 22.0
140.87903669 140.394561767578 23.0
106.41719675 107.357681274414 24.0
22.90053999 20.4083709716797 25.0
221.53560808 223.594223022461 26.0
221.53560808 221.987747192383 27.0
27.282645659 25.3551406860352 28.0
248.81825375 253.574676513672 29.0
248.81825375 252.408157348633 30.0
47.033013387 47.2299919128418 31.0
9.3131031344 9.18030452728271 32.0
20.901625993 20.4819431304932 33.0
20.901625993 21.1129970550537 34.0
20.901625993 20.8195400238037 35.0
9.3083381265 8.28715133666992 36.0
8.044767894 8.38281917572021 37.0
-2.5262790395e-09 -0.804296553134918 38.0
-2.5262785966e-09 0.113371789455414 39.0
-12.387792857 -14.5510091781616 40.0
-23.81592281 -27.5234241485596 41.0
4.9278776077 3.5162615776062 42.0
-31.108913619 -34.7394485473633 43.0
92.052766479 91.3064651489258 44.0
92.05276649 90.2343292236328 45.0
6.7273095346 3.00315618515015 46.0
2.6568330431 1.83220052719116 47.0
137.17687743 136.685623168945 48.0
131.07448231 127.40372467041 49.0
0 3.56358671188354 50.0
69.705106686 70.9869613647461 51.0
78.669161734 84.2529830932617 52.0
77.500548402 81.0479965209961 53.0
77.500548402 80.9212951660156 54.0
77.500548402 79.4541549682617 55.0
0 -0.0822644829750061 56.0
80.656571391 80.3950500488281 57.0
221.53560809 218.26057434082 58.0
248.81825375 246.469757080078 59.0
20.901626002 19.9398994445801 60.0
0 2.89454770088196 61.0
192.04060388 169.70783996582 0.0
192.04060388 160.703155517578 1.0
190.79559302 173.271377563477 2.0
186.00652329 171.783386230469 3.0
0 3.79397869110107 4.0
-1.9843393692e-12 14.0036382675171 5.0
-69.649550077 -34.2856712341309 6.0
81.966550601 99.9586791992188 7.0
53.71743788 43.9833564758301 8.0
28.249112711 16.6087741851807 9.0
96.492282225 69.7925338745117 10.0
83.940675251 74.3257675170898 11.0
88.527824268 80.3586883544922 12.0
88.527824268 77.9436492919922 13.0
88.527824268 81.6979446411133 14.0
0 -0.809234440326691 15.0
0 -0.909365117549896 16.0
-1.6293462721e-12 -0.976521074771881 17.0
55.23486981 54.1554756164551 18.0
-55.23486983 -59.069393157959 19.0
-55.234869808 -59.8050079345703 20.0
-55.23486982 -55.4216156005859 21.0
76.497123892 75.5044937133789 22.0
137.14516542 141.606430053711 23.0
104.68893665 114.713569641113 24.0
20.025103816 20.2468490600586 25.0
213.64228933 230.280944824219 26.0
213.64228933 227.063858032227 27.0
29.13985109 28.7818183898926 28.0
242.78214043 254.231292724609 29.0
242.78214043 254.240097045898 30.0
47.042364809 49.1876106262207 31.0
9.3171841719 9.16771602630615 32.0
21.262254062 18.2827396392822 33.0
21.262254062 21.0859107971191 34.0
21.262254062 19.7067832946777 35.0
9.4816960264 10.1337938308716 36.0
8.1100822919 8.65343570709229 37.0
-5.0180338057e-12 -3.94859075546265 38.0
-5.0180347893e-12 0.0306174159049988 39.0
-10.342314443 -17.6593456268311 40.0
-23.045286171 -27.89479637146 41.0
5.4437539399 3.42345881462097 42.0
-31.303272407 -31.6381015777588 43.0
97.103900357 95.3370361328125 44.0
97.10390037 95.6479644775391 45.0
6.7663603695 6.8359489440918 46.0
1.9660756734 3.30141258239746 47.0
192.04060389 162.335815429688 48.0
186.0065233 166.159439086914 49.0
0 0.0356399416923523 50.0
81.966550611 79.9598999023438 51.0
96.492282235 87.3791351318359 52.0
88.527824278 81.3931884765625 53.0
88.527824278 80.8326568603516 54.0
88.527824278 79.1809844970703 55.0
0 1.07269859313965 56.0
76.497123902 80.9557952880859 57.0
213.64228934 219.421722412109 58.0
242.78214044 248.056762695312 59.0
21.262254072 19.0842914581299 60.0
0 0.804853439331055 61.0
212.391738827015 153.841064453125 0.0
212.391738827015 156.152221679688 1.0
211.04319376 162.610855102539 2.0
205.5785585 183.54035949707 3.0
18.404050808 -11.9831399917603 4.0
18.4040508170453 16.5000095367432 5.0
-61.3077697580453 -106.728057861328 6.0
74.2237211591586 101.702270507812 7.0
47.5084168701586 49.3908500671387 8.0
26.7153042891586 26.0035495758057 9.0
73.6279026801586 82.7762222290039 10.0
62.1474872191586 58.7497711181641 11.0
78.3848871701586 68.9429779052734 12.0
78.3848871701586 72.1487426757812 13.0
78.3848871701585 84.5159912109375 14.0
4.213840709e-10 7.65522480010986 15.0
4.2138407099e-10 3.21506071090698 16.0
5.24329936999999e-09 4.69373941421509 17.0
56.7790807251774 59.1677894592285 18.0
-56.7790806952379 -56.4286880493164 19.0
-56.7790806772379 -56.0826797485352 20.0
-56.7790806622379 -68.0509414672852 21.0
79.0602866060699 71.6340713500977 22.0
149.501502613177 134.62646484375 23.0
119.756521836472 102.935264587402 24.0
24.5677496374717 26.2532119750977 25.0
228.561789214029 220.424560546875 26.0
228.561789214029 206.224975585938 27.0
27.4961151460291 28.7196178436279 28.0
256.057904364029 246.56787109375 29.0
256.057904364029 273.834167480469 30.0
45.391893868001 38.7327346801758 31.0
8.76348561509971 6.41286516189575 32.0
22.2812059441359 13.2257738113403 33.0
22.2812059441359 20.2075481414795 34.0
22.2812059764684 22.1812534332275 35.0
8.66141742863524 15.2447719573975 36.0
7.34531140923524 4.93151473999023 37.0
-6.99365361999996e-09 -6.4220929145813 38.0
-6.99365360999999e-09 -2.56813478469849 39.0
-12.1264598581189 -3.07913827896118 40.0
-26.0813195914196 -26.7585544586182 41.0
5.60695360219884 0.621493518352509 42.0
-34.4708309687988 -32.5315818786621 43.0
84.1913674207981 89.0402679443359 44.0
84.1913670312712 90.0791091918945 45.0
6.6020739941 6.19962120056152 46.0
2.5827328274 1.97778654098511 47.0
212.39173882 158.674377441406 48.0
205.57855851 163.200561523438 49.0
18.404050817 2.99647617340088 50.0
74.223721159 75.4421081542969 51.0
73.62790268 83.2851333618164 52.0
78.38488717 75.5351104736328 53.0
78.38488717 68.6956787109375 54.0
78.38488717 81.1668319702148 55.0
5.2432993646e-09 5.09521484375 56.0
79.060286606 76.9335174560547 57.0
228.56178922 207.424560546875 58.0
256.05790436 258.141723632812 59.0
22.281205976 22.5056571960449 60.0
0 6.23766803741455 61.0
178.207640723141 164.794021606445 0.0
178.207640723141 169.657516479492 1.0
176.96929637 165.466583251953 2.0
171.50466118 177.473114013672 3.0
0.011311137038 -0.465768873691559 4.0
0.011311145186 2.47690844535828 5.0
-69.815234849 -65.2861633300781 6.0
80.5036806957776 74.8213806152344 7.0
47.6899974787776 54.906551361084 8.0
32.8136832167776 25.8940162658691 9.0
81.6465450127776 90.9756698608398 10.0
68.4703389247776 76.6848754882812 11.0
78.0568804337776 82.0586166381836 12.0
78.0568804337776 78.5951385498047 13.0
78.0568804337776 79.7747039794922 14.0
0 -1.74653077125549 15.0
0 -0.232097208499908 16.0
-2.229436111e-09 -0.978347837924957 17.0
59.6751761000605 56.7601928710938 18.0
-59.6751761071664 -59.5052185058594 19.0
-59.6751760861664 -59.7838439941406 20.0
-59.6751760941664 -60.2965087890625 21.0
80.96986919 79.8902740478516 22.0
147.552067054061 143.310745239258 23.0
111.998711180076 114.614204406738 24.0
21.8624391480756 21.8418960571289 25.0
228.521936246282 222.758880615234 26.0
228.521936246282 223.840194702148 27.0
26.7706270442819 26.572301864624 28.0
255.292563296282 253.079330444336 29.0
255.292563296282 251.376190185547 30.0
46.90172438 46.5974349975586 31.0
9.4044095005 8.56643676757812 32.0
21.2946930979522 20.9325866699219 33.0
21.2946930979522 22.147403717041 34.0
21.2946931019233 20.9858531951904 35.0
9.2033128712 9.96604156494141 36.0
7.9497293985 6.93172693252563 37.0
-9.264059778e-09 -0.302216351032257 38.0
-9.264059779e-09 -2.54786014556885 39.0
-11.6972307900832 -11.6977939605713 40.0
-24.5134038461412 -25.968318939209 41.0
5.35222024894801 4.1248083114624 42.0
-32.524666555648 -32.4774208068848 43.0
98.681922190648 98.7723541259766 44.0
98.681922183648 95.1378860473633 45.0
6.9664429612 4.89138698577881 46.0
2.4247044541 2.79537320137024 47.0
178.20764072 165.184616088867 48.0
171.50466119 171.192260742188 49.0
0.011311146836 3.18514800071716 50.0
80.503680696 77.8827896118164 51.0
81.646545013 89.9194869995117 52.0
78.056880433 80.8806762695312 53.0
78.056880433 80.3594436645508 54.0
78.056880433 79.6668243408203 55.0
0 3.39220094680786 56.0
80.969869192 78.6907196044922 57.0
228.52193625 221.251800537109 58.0
255.29256329 249.256927490234 59.0
21.294693102 21.2775440216064 60.0
0 2.10299062728882 61.0
137.17687742 143.006256103516 0.0
137.17687742 143.039810180664 1.0
135.86355256 143.473709106445 2.0
131.0744823 135.652114868164 3.0
0 -1.9895703792572 4.0
0 0.341125309467316 5.0
-59.42861149 -59.2186317443848 6.0
69.705106679 70.1770935058594 7.0
65.43029471 62.4528312683105 8.0
4.2748119615 8.48783111572266 9.0
78.669161727 80.2065200805664 10.0
65.229460649 71.2585144042969 11.0
77.500548395 80.8047180175781 12.0
77.500548395 77.9000701904297 13.0
77.500548395 76.9198608398438 14.0
0 -0.64198762178421 15.0
0 0.28728449344635 16.0
0 -3.15485000610352 17.0
59.754945382 58.5015258789062 18.0
-59.754945394 -59.4383087158203 19.0
-59.754945372 -60.0649642944336 20.0
-59.754945384 -61.8801422119141 21.0
80.656571382 77.8813858032227 22.0
140.87903669 140.394561767578 23.0
106.41719675 107.357681274414 24.0
22.90053999 20.4083709716797 25.0
221.53560808 223.594223022461 26.0
221.53560808 221.987747192383 27.0
27.282645659 25.3551406860352 28.0
248.81825375 253.574676513672 29.0
248.81825375 252.408157348633 30.0
47.033013387 47.2299919128418 31.0
9.3131031344 9.18030452728271 32.0
20.901625993 20.4819431304932 33.0
20.901625993 21.1129970550537 34.0
20.901625993 20.8195400238037 35.0
9.3083381265 8.28715133666992 36.0
8.044767894 8.38281917572021 37.0
-2.5262790395e-09 -0.804296553134918 38.0
-2.5262785966e-09 0.113371789455414 39.0
-12.387792857 -14.5510091781616 40.0
-23.81592281 -27.5234241485596 41.0
4.9278776077 3.5162615776062 42.0
-31.108913619 -34.7394485473633 43.0
92.052766479 91.3064651489258 44.0
92.05276649 90.2343292236328 45.0
6.7273095346 3.00315618515015 46.0
2.6568330431 1.83220052719116 47.0
137.17687743 136.686950683594 48.0
131.07448231 127.357582092285 49.0
0 3.57017040252686 50.0
69.705106686 70.9871673583984 51.0
78.669161734 84.2529830932617 52.0
77.500548402 81.0479965209961 53.0
77.500548402 80.9212951660156 54.0
77.500548402 79.4541549682617 55.0
0 -0.0822644829750061 56.0
80.656571391 80.3950500488281 57.0
221.53560809 218.26057434082 58.0
248.81825375 246.469757080078 59.0
20.901626002 19.9398994445801 60.0
0 2.89454770088196 61.0
172.28800891 164.801223754883 0.0
172.28800891 164.864456176758 1.0
170.91200536 157.22526550293 2.0
166.12293563 155.440658569336 3.0
0 3.68533158302307 4.0
-3.9722199451e-14 1.32631063461304 5.0
-61.274900136 -66.5722885131836 6.0
74.354078879 73.93603515625 7.0
44.785720759 53.9901885986328 8.0
29.56835811 41.6500129699707 9.0
97.842406111 92.4045486450195 10.0
84.63981625 75.2420806884766 11.0
83.878115524 81.2909469604492 12.0
83.878115524 80.4497222900391 13.0
83.878115524 80.1469879150391 14.0
0 1.8238308429718 15.0
0 0.970125317573547 16.0
-7.1864597557e-14 5.05307722091675 17.0
53.752351497 60.0706024169922 18.0
-53.752351517 -57.6053810119629 19.0
-53.752351495 -58.2016563415527 20.0
-53.752351508 -64.3277282714844 21.0
76.235189332 75.9088516235352 22.0
142.42438672 142.868225097656 23.0
109.56420672 114.170463562012 24.0
20.91258406 19.1376533508301 25.0
218.65957606 222.679504394531 26.0
218.65957606 224.602615356445 27.0
27.523890894 25.2965526580811 28.0
246.18346697 251.719436645508 29.0
246.18346697 248.094360351562 30.0
46.272328162 44.8565559387207 31.0
9.4213914206 6.53609657287598 32.0
22.482837814 19.1265182495117 33.0
22.482837814 22.3240509033203 34.0
22.482837814 22.9536247253418 35.0
10.001514514 11.2502279281616 36.0
8.7844763001 6.04310512542725 37.0
-1.1406739484e-13 -6.11618709564209 38.0
-1.1406739484e-13 -4.09970092773438 39.0
-13.213612113 -14.5043878555298 40.0
-26.112887992 -29.1668395996094 41.0
4.9944217456 3.99777245521545 42.0
-33.853769223 -33.9650802612305 43.0
84.615870429 84.6892700195312 44.0
84.615870442 81.8800582885742 45.0
6.8950440142 10.3680782318115 46.0
2.8387258744 3.02762293815613 47.0
172.28800892 166.074554443359 48.0
166.12293564 151.147705078125 49.0
0 5.51293849945068 50.0
74.354078889 73.4553604125977 51.0
97.842406121 80.6607894897461 52.0
83.878115534 79.7698135375977 53.0
83.878115534 80.1583099365234 54.0
83.878115534 79.5986557006836 55.0
0 -2.21276044845581 56.0
76.235189342 76.9362411499023 57.0
218.65957607 221.116012573242 58.0
246.18346698 245.8330078125 59.0
22.482837824 22.7274360656738 60.0
0 1.96875524520874 61.0
172.88771985 159.216583251953 0.0
172.88771985 165.944061279297 1.0
171.69414474 160.36882019043 2.0
166.9050745 159.708114624023 3.0
0 -5.72295761108398 4.0
0 -0.0561079382896423 5.0
-58.09247028 -62.5938339233398 6.0
69.814779674 76.3981170654297 7.0
54.071270728 51.3997993469238 8.0
15.743508936 20.1744251251221 9.0
88.572345336 85.5068130493164 10.0
75.650769891 73.9368667602539 11.0
81.176639076 81.5606079101562 12.0
81.176639076 82.8508758544922 13.0
81.176639076 80.0913009643555 14.0
0 -0.0340642333030701 15.0
0 -0.0111111402511597 16.0
0 -0.643964231014252 17.0
57.433940052 59.1439590454102 18.0
-57.433940063 -59.4892158508301 19.0
-57.433940041 -59.8360481262207 20.0
-57.433940053 -58.4611167907715 21.0
79.506319721 77.1096954345703 22.0
147.19071558 144.219802856445 23.0
114.13610346 111.639984130859 24.0
21.614200962 21.3315982818604 25.0
226.69703531 223.701538085938 26.0
226.69703531 224.886306762695 27.0
27.794745074 27.5343494415283 28.0
254.49178039 253.573883056641 29.0
254.49178039 249.379302978516 30.0
46.852097838 46.3834075927734 31.0
9.3889927428 8.33438491821289 32.0
22.072379658 20.3859081268311 33.0
22.072379658 21.65403175354 34.0
22.072379657 20.622163772583 35.0
9.5364417676 7.99863910675049 36.0
8.2293706154 7.37063884735107 37.0
0 -0.505694448947906 38.0
0 -0.288975179195404 39.0
-11.855990362 -15.9893598556519 40.0
-24.660688049 -24.0361442565918 41.0
5.2255809617 4.37536096572876 42.0
-32.558172511 -34.2521209716797 43.0
93.36094152 89.9708099365234 44.0
93.360941533 92.5578689575195 45.0
6.9430950182 6.24269914627075 46.0
2.5203046831 2.25370144844055 47.0
172.88771986 163.539688110352 48.0
166.90507451 161.955673217773 49.0
0 -0.507585108280182 50.0
69.814779684 75.0684509277344 51.0
88.572345346 89.0556793212891 52.0
81.176639086 81.4612731933594 53.0
81.176639086 80.6294631958008 54.0
81.176639086 78.8301315307617 55.0
0 2.30692553520203 56.0
79.506319731 79.1343841552734 57.0
226.69703532 221.258407592773 58.0
254.4917804 247.444686889648 59.0
22.072379667 19.2788066864014 60.0
0 1.50705409049988 61.0
178.207640723141 164.794021606445 0.0
178.207640723141 169.657653808594 1.0
176.96929637 165.472442626953 2.0
171.50466118 177.500793457031 3.0
0.011311137038 -0.463097631931305 4.0
0.011311145186 2.4756293296814 5.0
-69.815234849 -65.2874984741211 6.0
80.5036806957776 74.821403503418 7.0
47.6899974787776 54.906551361084 8.0
32.8136832167776 25.8940162658691 9.0
81.6465450127776 90.9756698608398 10.0
68.4703389247776 76.6848754882812 11.0
78.0568804337776 82.0586166381836 12.0
78.0568804337776 78.5951385498047 13.0
78.0568804337776 79.7747039794922 14.0
0 -1.74653077125549 15.0
0 -0.232097208499908 16.0
-2.229436111e-09 -0.978347837924957 17.0
59.6751761000605 56.7601928710938 18.0
-59.6751761071664 -59.5052185058594 19.0
-59.6751760861664 -59.7838439941406 20.0
-59.6751760941664 -60.2965087890625 21.0
80.96986919 79.8902740478516 22.0
147.552067054061 143.310745239258 23.0
111.998711180076 114.614204406738 24.0
21.8624391480756 21.8418960571289 25.0
228.521936246282 222.758880615234 26.0
228.521936246282 223.840194702148 27.0
26.7706270442819 26.572301864624 28.0
255.292563296282 253.079330444336 29.0
255.292563296282 251.376190185547 30.0
46.90172438 46.5974349975586 31.0
9.4044095005 8.56643676757812 32.0
21.2946930979522 20.9325866699219 33.0
21.2946930979522 22.147403717041 34.0
21.2946931019233 20.9858531951904 35.0
9.2033128712 9.96604156494141 36.0
7.9497293985 6.93172693252563 37.0
-9.264059778e-09 -0.302216351032257 38.0
-9.264059779e-09 -2.54786014556885 39.0
-11.6972307900832 -11.6977939605713 40.0
-24.5134038461412 -25.968318939209 41.0
5.35222024894801 4.1248083114624 42.0
-32.524666555648 -32.4774208068848 43.0
98.681922190648 98.7723541259766 44.0
98.681922183648 95.1378860473633 45.0
6.9664429612 4.89138698577881 46.0
2.4247044541 2.79537320137024 47.0
178.20764072 165.184661865234 48.0
171.50466119 171.215972900391 49.0
0.011311146836 3.18645834922791 50.0
80.503680696 77.8827133178711 51.0
81.646545013 89.9194869995117 52.0
78.056880433 80.8806762695312 53.0
78.056880433 80.3594436645508 54.0
78.056880433 79.6668243408203 55.0
0 3.39220094680786 56.0
80.969869192 78.6907196044922 57.0
228.52193625 221.251800537109 58.0
255.29256329 249.256927490234 59.0
21.294693102 21.2775440216064 60.0
0 2.10299062728882 61.0
177.964838103064 164.829864501953 0.0
177.964838103064 163.596786499023 1.0
176.62178297 166.596633911133 2.0
171.83271322 172.272171020508 3.0
0 2.72491836547852 4.0
-7.23416762999999e-09 2.54375076293945 5.0
-60.2572654420891 -64.4018936157227 6.0
70.1735207063117 74.9471435546875 7.0
45.3566397863118 51.537036895752 8.0
24.8168809193118 24.6687049865723 9.0
89.4734957553117 90.0189743041992 10.0
76.8167889193117 76.5227813720703 11.0
80.7935708103117 86.5502777099609 12.0
80.7935708103117 81.4552459716797 13.0
80.7935708103117 80.4851226806641 14.0
-1.28034342414e-08 -1.90150594711304 15.0
-1.0607477515e-08 -0.401547849178314 16.0
1.72887792999999e-09 1.3115873336792 17.0
59.2653310534428 59.5427093505859 18.0
-59.2653310259676 -58.5588912963867 19.0
-59.2653310079676 -57.2177734375 20.0
-59.2653309919676 -59.3407249450684 21.0
80.6760002572194 78.4233932495117 22.0
143.174943689443 143.048248291016 23.0
106.524994911803 114.130584716797 24.0
22.1632789158035 21.0429420471191 25.0
223.850943956129 229.358016967773 26.0
223.850943956129 221.86994934082 27.0
27.0783090901289 26.4877643585205 28.0
250.929253046129 253.639266967773 29.0
250.929253046129 253.008331298828 30.0
47.0222912959646 46.581226348877 31.0
9.38093538705721 8.94614219665527 32.0
21.4106692662672 18.2610569000244 33.0
21.4106692662672 17.921012878418 34.0
21.4106693001075 21.6109447479248 35.0
9.06172507171984 9.04576778411865 36.0
7.73760472741984 7.40656137466431 37.0
-7.34448764999999e-09 -1.23897385597229 38.0
-7.34448766000001e-09 -0.269631206989288 39.0
-11.5473937859838 -16.3723964691162 40.0
-24.0146304669382 -25.3186473846436 41.0
5.24447971989312 5.57401943206787 42.0
-31.7928168860931 -32.2737274169922 43.0
97.7957010440931 95.5336990356445 44.0
97.7957009160931 94.3523864746094 45.0
7.1660698978 6.69756746292114 46.0
2.6976312976 2.19551587104797 47.0
177.9648381 164.509033203125 48.0
171.83271322 173.021179199219 49.0
0 -2.40969395637512 50.0
70.173520706 75.1387710571289 51.0
89.473495756 87.6039962768555 52.0
80.79357081 81.1569061279297 53.0
80.79357081 81.7158508300781 54.0
80.79357081 80.4649887084961 55.0
1.7288779325e-09 2.47181510925293 56.0
80.676000258 80.1251983642578 57.0
223.85094395 220.109436035156 58.0
250.92925304 249.617416381836 59.0
21.4106693 26.5850639343262 60.0
0 2.20698952674866 61.0
160.15128511078 158.534591674805 0.0
160.15128511078 163.104507446289 1.0
158.9233224 162.505676269531 2.0
154.13425266 152.234344482422 3.0
0 1.98635363578796 4.0
-7.886678219e-09 2.34495234489441 5.0
-67.5656547827196 -59.4745407104492 6.0
78.0615326040187 77.4067840576172 7.0
48.5306532060187 52.5880432128906 8.0
29.5308793980187 25.168737411499 9.0
83.2750488060187 86.0558776855469 10.0
70.7855183450187 76.5161056518555 11.0
77.4880955670187 83.0972900390625 12.0
77.4880955670187 79.5309295654297 13.0
77.4880955670187 80.9615097045898 14.0
9.3669341014e-10 0.104771852493286 15.0
0 0.000358819961547852 16.0
1.05972339999997e-10 -0.165647387504578 17.0
56.4188029775556 59.0423355102539 18.0
-56.418802967028 -58.991626739502 19.0
-56.418802947028 -61.964973449707 20.0
-56.418802942028 -59.7525177001953 21.0
78.0857561385726 78.3356552124023 22.0
149.512993834556 142.295181274414 23.0
114.485376958195 109.454330444336 24.0
22.5262006701946 21.6337203979492 25.0
227.59874997156 223.518814086914 26.0
227.59874997156 222.031402587891 27.0
26.99822137556 27.6183300018311 28.0
254.59697135156 251.821151733398 29.0
254.59697135156 252.95849609375 30.0
46.882558502054 47.7451438903809 31.0
9.31821424357425 8.59184646606445 32.0
21.6669531961589 19.6386070251465 33.0
21.6669531961589 17.6075859069824 34.0
21.6669532156271 21.5961036682129 35.0
9.46965905610154 9.04811382293701 36.0
8.19056192540154 7.77456188201904 37.0
-8.17120836e-09 0.578768253326416 38.0
-8.17120836e-09 -2.95404100418091 39.0
-12.287234381014 -11.2146425247192 40.0
-24.6855717299925 -26.5904579162598 41.0
5.10770284741556 3.75986647605896 42.0
-32.1821362250156 -32.6481437683105 43.0
95.6833331260156 90.121696472168 44.0
95.6833330550156 91.6902694702148 45.0
7.1300948452 6.27911567687988 46.0
2.5688762927 1.50638008117676 47.0
160.15128512 162.486679077148 48.0
154.13425267 149.000289916992 49.0
0 1.47512316703796 50.0
78.061532604 76.4068069458008 51.0
83.275048806 89.1537933349609 52.0
77.488095567 81.4231109619141 53.0
77.488095567 81.5558853149414 54.0
77.488095567 81.2057037353516 55.0
1.0597233881e-10 1.87951517105103 56.0
78.085756139 79.0843963623047 57.0
227.59874997 218.903121948242 58.0
254.59697135 247.691879272461 59.0
21.666953216 24.4464092254639 60.0
0 2.26404237747192 61.0
160.35063881316 167.894653320312 0.0
160.35063881316 170.422622680664 1.0
159.091024497706 171.263961791992 2.0
153.62638895 147.930557250977 3.0
4.3736320528 12.9852876663208 4.0
4.37363205541732 15.3893356323242 5.0
-66.9858621755173 -48.4729804992676 6.0
79.5663925708106 83.5473022460938 7.0
44.2486092298106 58.347225189209 8.0
35.3177833408106 22.5283107757568 9.0
88.9020717958106 88.4344940185547 10.0
76.1019445608106 79.3463745117188 11.0
85.6025560708106 83.1701049804688 12.0
85.6025560708106 75.6027145385742 13.0
85.6025560708106 77.0297622680664 14.0
3.5917564607e-08 0.888904213905334 15.0
3.5917564608e-08 -0.089235782623291 16.0
7.51975143999999e-08 -0.00610464811325073 17.0
63.1678763288058 56.9941749572754 18.0
-63.1678760782159 -59.3318901062012 19.0
-63.1678760732159 -59.9870834350586 20.0
-63.1678759362159 -54.3889389038086 21.0
83.1373687300299 76.5203704833984 22.0
136.675409366806 136.036178588867 23.0
104.930735791007 112.107666015625 24.0
21.5234025330072 22.1469116210938 25.0
219.81277809632 219.089492797852 26.0
219.81277809632 216.169067382812 27.0
24.82958511732 24.5978698730469 28.0
244.64236321632 253.917907714844 29.0
244.64236321632 254.26139831543 30.0
46.6761279145856 46.297492980957 31.0
9.35352104745502 8.80612564086914 32.0
19.969492793552 20.5294246673584 33.0
19.969492793552 19.7413063049316 34.0
19.969492968314 23.0593090057373 35.0
8.70648974130307 11.0390787124634 36.0
7.47673157460307 4.53157520294189 37.0
6.73553879999999e-09 -7.89872550964355 38.0
6.73553879999999e-09 -2.05979943275452 39.0
-11.583622860108 -20.0420265197754 40.0
-24.1605542933444 -23.3095092773438 41.0
4.68326180110882 9.88284397125244 42.0
-31.1458389500088 -35.9146423339844 43.0
99.1332674240089 94.5591201782227 44.0
99.1332662917558 94.3307189941406 45.0
7.09816358145277 3.68941688537598 46.0
2.76626398315276 2.0198974609375 47.0
160.35063881 170.955520629883 48.0
153.62638895 162.806762695312 49.0
4.3736320554 2.17990398406982 50.0
79.566392571 73.5864639282227 51.0
88.902071796 84.9199981689453 52.0
85.602556071 78.7710189819336 53.0
85.602556071 79.8957061767578 54.0
85.602556071 78.9692687988281 55.0
7.519751443e-08 4.61810159683228 56.0
83.13736873 79.282585144043 57.0
219.81277809 212.412750244141 58.0
244.64236321 251.319396972656 59.0
19.969492968 21.6904354095459 60.0
6.7355388327e-09 3.6285126209259 61.0
187.288344242038 173.49592590332 0.0
187.288344023488 169.016586303711 1.0
185.78171876 162.113174438477 2.0
184.72573848 165.822616577148 3.0
-2.870219778e-10 -2.05552387237549 4.0
-9.9984596e-09 8.90333843231201 5.0
-59.0502755547237 -40.2437057495117 6.0
69.8346210945328 81.0414810180664 7.0
67.3526956205328 34.2678108215332 8.0
2.48192547383307 27.5605373382568 9.0
61.8827258685328 78.4123840332031 10.0
50.6917348355328 75.9734802246094 11.0
69.7953541125328 84.9539566040039 12.0
69.7953541125328 82.1781005859375 13.0
69.7953541125328 78.7456283569336 14.0
5.30089582706036 -0.847216188907623 15.0
5.300895822 -1.2560977935791 16.0
-4.29341008000001e-09 1.56092715263367 17.0
55.6578439071088 56.564624786377 18.0
-55.6578438762992 -54.2142868041992 19.0
-56.3261116162992 -55.504093170166 20.0
-56.3946732152992 -57.3280296325684 21.0
74.5798067426845 76.9763107299805 22.0
131.644285720109 149.606201171875 23.0
98.172969081136 107.938621520996 24.0
20.140286001136 21.7871379852295 25.0
206.224092464076 225.852752685547 26.0
206.224092464076 224.813552856445 27.0
22.5814312330763 27.178674697876 28.0
228.805523704076 250.778274536133 29.0
228.805523704076 243.255172729492 30.0
46.1226244793008 46.6967391967773 31.0
9.49843719611231 7.27699851989746 32.0
18.185133527171 17.0185546875 33.0
18.185133527171 15.3805913925171 34.0
18.1851335690006 11.1762437820435 35.0
9.32217600277328 14.1983394622803 36.0
8.06771151317327 6.19570159912109 37.0
-6.01205212000002e-09 -1.23084020614624 38.0
-6.01205211999997e-09 -6.18098783493042 39.0
-9.80225438944965 -10.5643844604492 40.0
-23.406185656593 -21.0307083129883 41.0
5.80619682907742 6.52995014190674 42.0
-31.5484495642774 -34.4138145446777 43.0
75.1598364312774 85.7690963745117 44.0
73.9231028314474 75.7332153320312 45.0
7.30468345632153 9.32948112487793 46.0
2.54847063530697 2.94901013374329 47.0
187.28834425 174.34765625 48.0
184.72573849 158.262985229492 49.0
0 0.622473537921906 50.0
69.834621095 69.1166076660156 51.0
61.882725868 86.135124206543 52.0
69.795354113 77.444709777832 53.0
69.795354113 80.1006164550781 54.0
69.795354113 80.3380126953125 55.0
0 1.87008285522461 56.0
74.579806742 77.4955368041992 57.0
206.22409247 220.6181640625 58.0
228.8055237 246.978958129883 59.0
18.185133569 18.6738967895508 60.0
0 1.76524174213409 61.0
159.87736064 166.00927734375 0.0
159.87736064 165.713638305664 1.0
158.5706993 163.703948974609 2.0
153.10606411 147.684722900391 3.0
2.2222490814 5.9890308380127 4.0
2.2222490814 4.56553411483765 5.0
-65.437327632 -68.3894348144531 6.0
75.664329287 77.8534622192383 7.0
51.108354085 49.4421463012695 8.0
24.555975192 9.11969757080078 9.0
92.987177682 86.8309860229492 10.0
80.790065495 74.0654678344727 11.0
85.289926471 81.088737487793 12.0
85.289926471 78.2449569702148 13.0
85.289926471 81.8081588745117 14.0
0 -0.973830044269562 15.0
0 -0.5943883061409 16.0
0 -0.440052688121796 17.0
58.943466374 59.3272476196289 18.0
-58.943466384 -57.8566017150879 19.0
-58.943466362 -59.6720733642578 20.0
-58.943466374 -59.0004272460938 21.0
81.480781503 81.8764724731445 22.0
141.85278777 148.70735168457 23.0
107.68301163 100.897888183594 24.0
23.828896794 25.1163291931152 25.0
223.33356928 227.640853881836 26.0
223.33356928 227.137176513672 27.0
25.318084844 28.3211536407471 28.0
248.65165414 253.972015380859 29.0
248.65165414 253.659774780273 30.0
47.025960909 47.2988777160645 31.0
9.3008114574 8.71549701690674 32.0
22.537315119 18.0005264282227 33.0
22.537315119 22.6298713684082 34.0
22.537315119 24.9536209106445 35.0
9.3928173242 18.7687530517578 36.0
8.0031645184 7.56163263320923 37.0
0 -2.91100907325745 38.0
0 -4.46647262573242 39.0
-12.782236307 -12.1249217987061 40.0
-25.912888337 -23.6491546630859 41.0
4.7720772151 8.77756977081299 42.0
-33.197785209 -35.9213714599609 43.0
93.615217577 89.5795822143555 44.0
93.61521759 96.7800903320312 45.0
6.7475659379 7.70829248428345 46.0
1.9236503671 2.82325625419617 47.0
159.87736065 171.518341064453 48.0
153.10606412 156.628875732422 49.0
2.2222490914 -1.91006922721863 50.0
75.664329297 75.0343704223633 51.0
92.987177692 88.4841232299805 52.0
85.289926481 80.3231430053711 53.0
85.289926481 76.8518447875977 54.0
85.289926481 79.4929504394531 55.0
0 3.51923942565918 56.0
81.480781513 77.5570068359375 57.0
223.33356929 222.824584960938 58.0
248.65165415 248.192947387695 59.0
22.537315129 21.9556446075439 60.0
0 1.57640981674194 61.0
157.36548838 170.930877685547 0.0
157.36548838 172.19694519043 1.0
156.1023543 165.949951171875 2.0
151.31328458 159.548950195312 3.0
0 4.67210912704468 4.0
-1.0106839505e-09 2.16285419464111 5.0
-64.604244972 -62.8880653381348 6.0
74.535804141 78.6902160644531 7.0
51.399035256 52.4100685119629 8.0
23.136768884 24.4524040222168 9.0
89.254529497 87.1481323242188 10.0
77.000080906 75.1320343017578 11.0
81.430517812 83.490119934082 12.0
81.430517812 79.643684387207 13.0
81.430517812 81.8576049804688 14.0
0 -0.965415775775909 15.0
0 0.24268639087677 16.0
-5.067987411e-09 -0.464225113391876 17.0
57.867504373 58.9413070678711 18.0
-57.8675043956387 -58.6981735229492 19.0
-57.8675043736387 -61.060791015625 20.0
-57.8675043846387 -60.9772644042969 21.0
79.777552516 79.6585083007812 22.0
147.30026188 143.372604370117 23.0
111.12845189 111.668113708496 24.0
23.331766867 20.6868991851807 25.0
227.077814409722 228.655410766602 26.0
227.077814409722 223.216217041016 27.0
26.9601671657221 24.3390655517578 28.0
254.037981569722 254.596633911133 29.0
254.037981569722 256.463409423828 30.0
46.586943243 47.0950660705566 31.0
9.3048568884 9.49477005004883 32.0
21.910048136 21.3646202087402 33.0
21.910048136 18.4760494232178 34.0
21.910048133 20.8963356018066 35.0
9.7342415406 8.329833984375 36.0
8.4434670463 6.77964496612549 37.0
-8.8053647453e-09 -1.51850342750549 38.0
-8.8053648459e-09 -3.84855270385742 39.0
-12.426981428 -14.0024938583374 40.0
-25.144790067 -27.8206272125244 41.0
5.2038552011 4.52153968811035 42.0
-33.017419753 -31.6740703582764 43.0
94.906558984 94.7040863037109 44.0
94.906558991 94.1482849121094 45.0
6.8363352879 9.84699726104736 46.0
2.401432487 2.71115565299988 47.0
157.36548838 169.014404296875 48.0
151.31328459 147.314788818359 49.0
0 -1.50774383544922 50.0
74.535804142 76.3723678588867 51.0
89.254529498 85.858024597168 52.0
81.430517813 81.814567565918 53.0
81.430517813 82.1610107421875 54.0
81.430517813 79.9546203613281 55.0
0 1.68385219573975 56.0
79.777552523 80.0764846801758 57.0
227.07781441 223.571853637695 58.0
254.03798157 248.814178466797 59.0
21.91004814 21.4324569702148 60.0
0 2.15945339202881 61.0
159.669554822697 156.772811889648 0.0
159.669554822697 158.209609985352 1.0
158.31753785 152.619674682617 2.0
153.52846811 200.741516113281 3.0
0 1.48635315895081 4.0
-5.786220817e-09 4.9621787071228 5.0
-58.3342901777168 -63.9529838562012 6.0
69.4005103350087 77.8396377563477 7.0
42.3143384320087 48.8424949645996 8.0
27.0861719020087 23.5846633911133 9.0
90.9545392160087 88.2309951782227 10.0
78.1503668210087 81.2277526855469 11.0
79.7077952760087 79.4914093017578 12.0
79.7077952760087 80.690544128418 13.0
79.7077952760087 66.816047668457 14.0
1.606322688476e-09 -9.15841770172119 15.0
0 -2.00083494186401 16.0
-7.21941409999994e-10 3.62987232208252 17.0
55.2873157909139 58.0085601806641 18.0
-55.2873157780131 -58.4212799072266 19.0
-55.2873157580131 -58.083797454834 20.0
-55.2873157520131 -57.503475189209 21.0
76.7563081857861 72.3499374389648 22.0
136.961197236914 142.653717041016 23.0
102.428467808642 109.782516479492 24.0
21.0336781116423 21.7296619415283 25.0
213.717505425395 228.498489379883 26.0
213.717505425395 221.618499755859 27.0
28.3193584703947 26.5471248626709 28.0
242.036863895395 250.071823120117 29.0
242.036863895395 249.080764770508 30.0
47.2652214349734 47.1528053283691 31.0
9.4098325756272 6.13962888717651 32.0
21.4689924341503 15.2322454452515 33.0
21.4689924341503 16.9494571685791 34.0
21.4689924557961 21.2580909729004 35.0
9.94532784291457 11.5681495666504 36.0
8.77093395321457 8.32499027252197 37.0
-8.02466460000002e-09 -4.01570415496826 38.0
-8.02466459999999e-09 -0.14772754907608 39.0
-11.5257927435065 -16.6352653503418 40.0
-24.3481317238688 -26.554292678833 41.0
4.8099021797406 2.1361722946167 42.0
-31.6045810603406 -30.2187156677246 43.0
93.3517773503406 87.632194519043 44.0
93.3517772693406 90.4069290161133 45.0
6.4145328529 4.05588960647583 46.0
1.6486533839 0.868272364139557 47.0
159.66955482 167.535095214844 48.0
153.52846812 138.920501708984 49.0
0 0.364760458469391 50.0
69.400510335 77.3041152954102 51.0
90.954539216 84.9003067016602 52.0
79.707795276 80.2201080322266 53.0
79.707795276 81.3272857666016 54.0
79.707795276 72.0225830078125 55.0
0 11.5913343429565 56.0
76.756308186 78.6438751220703 57.0
213.71750542 221.224517822266 58.0
242.03686389 242.078567504883 59.0
21.468992456 20.3095703125 60.0
0 3.16678953170776 61.0
159.506534424871 160.841995239258 0.0
159.506534424871 165.693984985352 1.0
158.25293895 164.408706665039 2.0
153.46386921 155.01985168457 3.0
0 -2.78462028503418 4.0
-6.746147961e-09 4.93320560455322 5.0
-59.0692891178411 -61.7377548217773 6.0
70.5438716744439 74.6734161376953 7.0
45.4467888014439 50.7782096862793 8.0
25.0970828734439 25.0196056365967 9.0
89.2831841514439 88.7387161254883 10.0
76.1754963594439 78.4196395874023 11.0
84.1387935514439 79.9969863891602 12.0
84.1387935514439 76.9131011962891 13.0
84.1387935514439 78.0875778198242 14.0
2.31282605214e-09 -1.17495608329773 15.0
0 0.173320174217224 16.0
1.94412250999999e-09 -0.185572683811188 17.0
60.3271150195148 58.3110008239746 18.0
-60.3271150016658 -58.3121948242188 19.0
-60.3271149816658 -60.1042785644531 20.0
-60.3271149736658 -60.1904067993164 21.0
81.18550360412 79.2737045288086 22.0
144.099779899515 141.748229980469 23.0
110.028541339394 109.412475585938 24.0
23.2510261493936 20.4718742370605 25.0
225.285283499743 229.698059082031 26.0
225.285283499743 224.282333374023 27.0
25.8171617477429 27.1678676605225 28.0
251.102445249743 250.980819702148 29.0
251.102445249743 251.796798706055 30.0
46.9412050840225 47.4172019958496 31.0
9.29098009086743 8.61005973815918 32.0
20.8583886305233 19.0273818969727 33.0
20.8583886305233 13.4690179824829 34.0
20.8583886551186 20.7260417938232 35.0
9.28863951980661 6.78936862945557 36.0
7.97048488220661 7.88682174682617 37.0
-7.67182938999999e-09 1.11590886116028 38.0
-7.67182938999999e-09 -2.93408823013306 39.0
-11.3675668683329 -15.4006910324097 40.0
-23.8319987439715 -24.8283882141113 41.0
5.38134004273657 6.23417091369629 42.0
-31.6832072490366 -33.0877571105957 43.0
98.2327613260366 91.8390197753906 44.0
98.2327611281946 95.1000671386719 45.0
6.9541939016 7.48569297790527 46.0
2.5059499292 1.14463913440704 47.0
159.50653443 162.724700927734 48.0
153.46386922 154.10905456543 49.0
0 -2.63244128227234 50.0
70.543871675 76.0920104980469 51.0
89.283184151 92.0413513183594 52.0
84.138793552 80.9965515136719 53.0
84.138793552 79.6780624389648 54.0
84.138793552 80.1773071289062 55.0
1.9441225139e-09 0.979798257350922 56.0
81.185503604 80.8437728881836 57.0
225.2852835 222.570114135742 58.0
251.10244525 248.323760986328 59.0
20.858388655 23.9139366149902 60.0
0 2.47190237045288 61.0
166.91973447 162.646530151367 0.0
166.91973447 168.166030883789 1.0
165.76036249 165.374298095703 2.0
160.97129277 166.775146484375 3.0
0 4.90537071228027 4.0
0 -0.223446905612946 5.0
-64.193029013 -62.8290634155273 6.0
75.705222813 73.1513366699219 7.0
48.019620597 51.7409172058105 8.0
27.685602207 24.7373886108398 9.0
86.680933934 84.5756759643555 10.0
73.922318244 75.8984451293945 11.0
80.927336143 81.8754043579102 12.0
80.927336143 77.5858535766602 13.0
80.927336143 79.7336273193359 14.0
0 -1.61417746543884 15.0
0 -0.60650247335434 16.0
-1.5096567517e-14 -0.469739258289337 17.0
57.701515205 58.4437217712402 18.0
-57.701515225 -60.0666580200195 19.0
-57.701515203 -58.341625213623 20.0
-57.701515215 -58.674674987793 21.0
78.411513052 77.2468719482422 22.0
139.75595277 141.570526123047 23.0
106.50947314 112.474548339844 24.0
21.814033666 20.0787925720215 25.0
218.16746583 226.344314575195 26.0
218.16746583 226.396087646484 27.0
26.932104411 24.4077606201172 28.0
245.09957026 252.26350402832 29.0
245.09957026 251.818984985352 30.0
46.480835147 46.0074768066406 31.0
9.3271374659 8.74269104003906 32.0
20.709997827 20.44921875 33.0
20.709997827 16.3474578857422 34.0
20.709997827 20.2802639007568 35.0
8.9912851006 6.03111934661865 36.0
7.6720581139 6.10623931884766 37.0
-1.3217527922e-12 -0.955964028835297 38.0
-1.3217522116e-12 -0.503637135028839 39.0
-11.470250745 -16.1994228363037 40.0
-24.966953102 -24.7081718444824 41.0
5.2116120616 3.25950288772583 42.0
-32.783234911 -33.0631790161133 43.0
92.681553037 88.811408996582 44.0
92.68155305 92.3274993896484 45.0
7.3005037972 7.06185913085938 46.0
2.6947730607 2.66597843170166 47.0
166.91973448 167.108779907227 48.0
160.97129278 164.436050415039 49.0
0 0.090697169303894 50.0
75.705222823 76.3485488891602 51.0
86.680933944 90.4780502319336 52.0
80.927336153 81.3497619628906 53.0
80.927336153 80.0339279174805 54.0
80.927336153 80.4364547729492 55.0
0 2.42539262771606 56.0
78.411513062 79.666259765625 57.0
218.16746584 222.249710083008 58.0
245.09957027 247.500839233398 59.0
20.709997837 24.1465034484863 60.0
0 1.48754703998566 61.0
170.20658371 158.878173828125 0.0
170.20658371 164.111068725586 1.0
168.84186141 161.609466552734 2.0
164.05279168 171.039031982422 3.0
0 6.50696992874146 4.0
-1.6796992301e-13 6.32832860946655 5.0
-60.641653497 -54.2512435913086 6.0
71.920691722 79.7441329956055 7.0
54.443475797 51.9478759765625 8.0
17.477215915 21.9105186462402 9.0
87.842062929 89.8479843139648 10.0
74.49942985 78.5485382080078 11.0
80.402923361 81.1614074707031 12.0
80.402923361 77.8133850097656 13.0
80.402923361 76.0155410766602 14.0
0 -1.46089601516724 15.0
0 0.233545124530792 16.0
-3.7605504764e-13 0.000267505645751953 17.0
57.654977333 57.7353820800781 18.0
-57.654977353 -60.0115203857422 19.0
-57.654977331 -60.3126335144043 20.0
-57.654977343 -58.8246421813965 21.0
78.840369075 77.0826644897461 22.0
142.37308358 142.446899414062 23.0
107.30663572 111.930191040039 24.0
22.680733098 20.761157989502 25.0
221.21345266 229.087829589844 26.0
221.21345266 227.878890991211 27.0
27.492024189 24.9391136169434 28.0
248.70547686 255.033096313477 29.0
248.70547686 257.879089355469 30.0
46.803743586 47.2554397583008 31.0
9.4074474231 9.04789447784424 32.0
21.185391722 18.6295375823975 33.0
21.185391722 17.9288730621338 34.0
21.185391722 19.5672473907471 35.0
9.4535673835 8.38052368164062 36.0
8.1569138496 7.66783857345581 37.0
-4.7880624972e-13 -1.96472382545471 38.0
-4.7880551717e-13 -4.55815267562866 39.0
-12.092863796 -14.0839786529541 40.0
-24.254899908 -26.1865730285645 41.0
5.1452346412 7.53444814682007 42.0
-32.121083457 -34.7955741882324 43.0
104.84722134 98.935661315918 44.0
104.84722135 97.4809188842773 45.0
6.719517578 5.64507007598877 46.0
2.4229028689 2.41529393196106 47.0
170.20658372 162.718475341797 48.0
164.05279169 154.811019897461 49.0
0 2.47728967666626 50.0
71.920691732 75.9860305786133 51.0
87.842062939 92.3452835083008 52.0
80.402923371 81.0938568115234 53.0
80.402923371 80.9172439575195 54.0
80.402923371 79.3611907958984 55.0
0 0.0384770035743713 56.0
78.840369085 80.5426712036133 57.0
221.21345267 221.916412353516 58.0
248.70547687 248.380554199219 59.0
21.185391732 22.9592208862305 60.0
0 0.96477860212326 61.0
147.59555202 164.645797729492 0.0
147.59555202 158.001083374023 1.0
146.39025614 157.816375732422 2.0
141.60118641 139.930114746094 3.0
0 7.75796604156494 4.0
0 0.429585039615631 5.0
-64.753654227 -59.91455078125 6.0
76.026143549 80.3137664794922 7.0
46.150106242 45.6958045959473 8.0
29.876037297 24.1751804351807 9.0
84.495648475 96.9319839477539 10.0
72.638321873 78.2974090576172 11.0
79.558471733 87.3150787353516 12.0
79.558471733 86.6168060302734 13.0
79.558471733 76.4477691650391 14.0
0 -0.0214637517929077 15.0
0 -0.23643547296524 16.0
0 0.650009214878082 17.0
60.723538287 60.4936294555664 18.0
-60.723538297 -58.8111991882324 19.0
-60.723538275 -53.197639465332 20.0
-60.723538288 -56.3383140563965 21.0
82.373400961 77.549674987793 22.0
139.61453948 147.057968139648 23.0
106.98216904 112.452606201172 24.0
21.386479709 19.2137184143066 25.0
221.98794045 222.956954956055 26.0
221.98794045 226.307723999023 27.0
28.780443619 25.3258571624756 28.0
250.76838408 248.267578125 29.0
250.76838408 249.997619628906 30.0
47.002454833 49.5958099365234 31.0
9.285912751 9.5930347442627 32.0
21.649862664 18.1959705352783 33.0
21.649862664 23.7815761566162 34.0
21.649862664 22.2283973693848 35.0
9.9459293968 6.6279730796814 36.0
8.7606888039 6.35870599746704 37.0
-2.6040799383e-10 -1.76334714889526 38.0
-2.6040792422e-10 -3.19870805740356 39.0
-11.635736569 -16.9976825714111 40.0
-25.22907137 -26.5920715332031 41.0
4.8900668851 5.66343832015991 42.0
-32.52567893 -33.843620300293 43.0
90.165309755 87.4204483032227 44.0
90.165309768 90.1847076416016 45.0
6.9789841328 4.81195688247681 46.0
2.3836761511 2.24263024330139 47.0
147.59555203 161.683410644531 48.0
141.60118642 147.433410644531 49.0
0 7.52249002456665 50.0
76.026143559 70.6722412109375 51.0
84.495648485 86.4512710571289 52.0
79.558471743 80.1537017822266 53.0
79.558471743 82.3549194335938 54.0
79.558471743 79.8503494262695 55.0
0 1.61147546768188 56.0
82.373400971 80.1528396606445 57.0
221.98794046 223.339050292969 58.0
250.76838408 244.4599609375 59.0
21.649862673 17.5413990020752 60.0
0 1.13901662826538 61.0
168.33347034 163.97477722168 0.0
168.33347034 168.158737182617 1.0
166.98117493 159.204711914062 2.0
162.19210471 156.5029296875 3.0
0 -4.11255598068237 4.0
0 -1.4016227722168 5.0
-60.450536893 -62.1218643188477 6.0
72.098058675 78.9589309692383 7.0
52.0049694 54.6025352478027 8.0
20.093089267 28.0340404510498 9.0
87.36596195 91.0633239746094 10.0
74.007385893 76.5895690917969 11.0
80.448521552 83.6150817871094 12.0
80.448521552 76.9171600341797 13.0
80.448521552 77.9993133544922 14.0
0 -2.24201631546021 15.0
0 -0.648998200893402 16.0
0 -1.49034929275513 17.0
56.888507975 57.4160652160645 18.0
-56.888507986 -58.7110061645508 19.0
-56.888507965 -59.1994514465332 20.0
-56.888507977 -59.6584815979004 21.0
77.795382624 78.7428359985352 22.0
148.58337569 140.64533996582 23.0
114.50596064 106.097091674805 24.0
22.803985899 22.3262920379639 25.0
226.37875833 222.057281494141 26.0
226.37875833 223.905624389648 27.0
26.698097689 24.1361980438232 28.0
253.07685603 257.846282958984 29.0
253.07685603 258.512145996094 30.0
47.007156717 46.9509162902832 31.0
9.4050604318 9.44370555877686 32.0
20.906874641 19.6620101928711 33.0
20.906874641 18.4821739196777 34.0
20.90687464 19.7689437866211 35.0
9.2562446277 8.15321445465088 36.0
8.0189347443 7.52362060546875 37.0
-1.1408579462e-09 -0.26648873090744 38.0
-1.1408578188e-09 -3.71512985229492 39.0
-11.851329836 -13.5215511322021 40.0
-24.936524141 -29.1733074188232 41.0
5.115440827 4.52317476272583 42.0
-32.623255276 -32.5054588317871 43.0
98.260034141 92.7761535644531 44.0
98.260034152 93.91259765625 45.0
6.8768490861 7.74327707290649 46.0
2.6916736038 2.42230105400085 47.0
168.33347035 164.203460693359 48.0
162.19210472 153.296920776367 49.0
0 2.09337568283081 50.0
72.098058684 76.7841033935547 51.0
87.365961958 88.3346328735352 52.0
80.448521561 81.1731948852539 53.0
80.448521561 79.3433609008789 54.0
80.448521561 79.5940399169922 55.0
0 1.24563837051392 56.0
77.795382634 80.5439376831055 57.0
226.37875834 222.448471069336 58.0
253.07685604 249.853607177734 59.0
20.90687465 22.5224685668945 60.0
0 1.15553200244904 61.0
175.05640447 158.04736328125 0.0
175.05640447 162.894073486328 1.0
173.86983802 167.243835449219 2.0
168.65913708 168.704818725586 3.0
0 0.890222012996674 4.0
-4.1281766273e-10 2.32930445671082 5.0
-68.631844265 -66.3098831176758 6.0
79.017161661 75.489143371582 7.0
47.789810844 49.8381233215332 8.0
31.227350809 24.71116065979 9.0
86.753133177 86.4171981811523 10.0
73.521080011 75.5051727294922 11.0
81.326417839 81.1204376220703 12.0
81.326417839 79.945556640625 13.0
81.326417839 78.0863037109375 14.0
0 0.194886147975922 15.0
0 0.441041111946106 16.0
-1.2762600089e-09 0.898068130016327 17.0
58.179979465 59.8065986633301 18.0
-58.179979488 -59.6823425292969 19.0
-58.179979466 -59.6734619140625 20.0
-58.179979478 -60.8730735778809 21.0
79.799760093 77.9290008544922 22.0
146.48581339 141.393615722656 23.0
112.31755034 110.322265625 24.0
22.818628662 21.3721466064453 25.0
226.2855735 223.709777832031 26.0
226.2855735 219.762481689453 27.0
26.16727697 23.9745922088623 28.0
252.45285047 255.15885925293 29.0
252.45285047 257.40869140625 30.0
46.933708058 48.1185150146484 31.0
9.3144364392 8.44257068634033 32.0
21.61978061 18.908634185791 33.0
21.61978061 18.9554576873779 34.0
21.619780609 19.3043079376221 35.0
9.2718822657 7.30234718322754 36.0
8.0060640176 8.38574600219727 37.0
-1.6395844448e-09 0.38869833946228 38.0
-1.6395844447e-09 -1.42251467704773 39.0
-11.645049069 -13.5318040847778 40.0
-24.09040202 -23.3780937194824 41.0
4.9097031261 5.66740608215332 42.0
-31.543361217 -33.8258323669434 43.0
92.689325955 91.2669372558594 44.0
92.689325964 94.9345016479492 45.0
6.5748148242 7.28198480606079 46.0
2.4864486216 2.73059940338135 47.0
175.05640448 163.227355957031 48.0
168.65913709 162.042755126953 49.0
0 -1.3536810874939 50.0
79.017161668 76.3994903564453 51.0
86.753133184 90.2097473144531 52.0
81.326417847 81.2863235473633 53.0
81.326417847 81.3412857055664 54.0
81.326417847 79.7621231079102 55.0
0 -0.53716379404068 56.0
79.799760103 80.1216735839844 57.0
226.2855735 219.242980957031 58.0
252.45285048 250.191162109375 59.0
21.619780618 20.7179851531982 60.0
0 1.77052474021912 61.0
185.23231047 173.081634521484 0.0
185.23231047 170.448989868164 1.0
183.94323359 167.960510253906 2.0
179.15416387 166.392913818359 3.0
-1.2089405523e-11 0.507571518421173 4.0
-6.909495409e-10 8.51286220550537 5.0
-64.931742152 -46.8035316467285 6.0
74.996920376 81.4064712524414 7.0
50.073468428 56.8715019226074 8.0
24.92345194 26.1057147979736 9.0
94.258645082 94.2136077880859 10.0
81.933673716 72.0984344482422 11.0
84.614510425 88.234260559082 12.0
84.614510425 80.1509399414062 13.0
84.614510425 79.6584701538086 14.0
0 -4.38411283493042 15.0
0 -1.70706295967102 16.0
-1.2260981433e-09 -2.55496740341187 17.0
54.744439125 53.0308036804199 18.0
-54.744439147 -59.0440254211426 19.0
-54.744439125 -60.2427406311035 20.0
-54.744439137 -55.4530296325684 21.0
75.541516636 78.3162536621094 22.0
147.1617598 134.707534790039 23.0
113.3477 113.360260009766 24.0
23.261845564 17.3957538604736 25.0
222.70327645 227.284790039062 26.0
222.70327645 225.055419921875 27.0
28.405561077 26.361047744751 28.0
251.10883754 260.920501708984 29.0
251.10883754 251.12419128418 30.0
46.319978068 46.8136024475098 31.0
9.2251238914 8.75328826904297 32.0
20.797077493 18.0978450775146 33.0
20.797077493 23.4176120758057 34.0
20.797077492 24.8525199890137 35.0
8.8939995559 11.1411352157593 36.0
7.5988477859 7.22424650192261 37.0
-1.9797391e-09 -2.48810052871704 38.0
-1.9797390784e-09 -8.10943508148193 39.0
-12.153268295 -11.7657699584961 40.0
-24.264750008 -31.8395290374756 41.0
4.6941931407 1.64945387840271 42.0
-31.277076302 -29.650411605835 43.0
105.40686621 97.3973083496094 44.0
105.40686623 89.2252731323242 45.0
6.9800475752 7.3021936416626 46.0
2.7799446129 1.79686880111694 47.0
185.23231048 170.293365478516 48.0
179.15416388 174.168090820312 49.0
0 4.03368616104126 50.0
74.996920384 74.9470672607422 51.0
94.25864509 87.1755981445312 52.0
84.614510433 81.8534164428711 53.0
84.614510433 81.3036499023438 54.0
84.614510433 79.3001708984375 55.0
0 -0.267118155956268 56.0
75.541516645 78.8032455444336 57.0
222.70327646 220.205627441406 58.0
251.10883754 251.374908447266 59.0
20.797077501 20.7562503814697 60.0
0 2.84490990638733 61.0
137.17687742 143.006240844727 0.0
137.17687742 143.040786743164 1.0
135.86355256 143.466873168945 2.0
131.0744823 135.639465332031 3.0
0 -1.97539496421814 4.0
0 0.341026425361633 5.0
-59.42861149 -59.2168769836426 6.0
69.705106679 70.1770782470703 7.0
65.43029471 62.4528312683105 8.0
4.2748119615 8.48783111572266 9.0
78.669161727 80.2065200805664 10.0
65.229460649 71.2585144042969 11.0
77.500548395 80.8047180175781 12.0
77.500548395 77.9000701904297 13.0
77.500548395 76.9198608398438 14.0
0 -0.64198762178421 15.0
0 0.28728449344635 16.0
0 -3.15485000610352 17.0
59.754945382 58.5015258789062 18.0
-59.754945394 -59.4383087158203 19.0
-59.754945372 -60.0649642944336 20.0
-59.754945384 -61.8801422119141 21.0
80.656571382 77.8813858032227 22.0
140.87903669 140.394561767578 23.0
106.41719675 107.357681274414 24.0
22.90053999 20.4083709716797 25.0
221.53560808 223.594223022461 26.0
221.53560808 221.987747192383 27.0
27.282645659 25.3551406860352 28.0
248.81825375 253.574676513672 29.0
248.81825375 252.408157348633 30.0
47.033013387 47.2299919128418 31.0
9.3131031344 9.18030452728271 32.0
20.901625993 20.4819431304932 33.0
20.901625993 21.1129970550537 34.0
20.901625993 20.8195400238037 35.0
9.3083381265 8.28715133666992 36.0
8.044767894 8.38281917572021 37.0
-2.5262790395e-09 -0.804296553134918 38.0
-2.5262785966e-09 0.113371789455414 39.0
-12.387792857 -14.5510091781616 40.0
-23.81592281 -27.5234241485596 41.0
4.9278776077 3.5162615776062 42.0
-31.108913619 -34.7394485473633 43.0
92.052766479 91.3064651489258 44.0
92.05276649 90.2343292236328 45.0
6.7273095346 3.00315618515015 46.0
2.6568330431 1.83220052719116 47.0
137.17687743 136.687423706055 48.0
131.07448231 127.337051391602 49.0
0 3.57309532165527 50.0
69.705106686 70.9872741699219 51.0
78.669161734 84.2529830932617 52.0
77.500548402 81.0479965209961 53.0
77.500548402 80.9212951660156 54.0
77.500548402 79.4541549682617 55.0
0 -0.0822644829750061 56.0
80.656571391 80.3950500488281 57.0
221.53560809 218.26057434082 58.0
248.81825375 246.469757080078 59.0
20.901626002 19.9398994445801 60.0
0 2.89454770088196 61.0
175.77805803 165.798156738281 0.0
175.77805803 170.663879394531 1.0
174.45269644 162.692047119141 2.0
169.66362672 167.22248840332 3.0
0 -3.92404699325562 4.0
-3.8923948093e-13 4.07518672943115 5.0
-58.000557144 -58.65478515625 6.0
69.729463776 75.6072006225586 7.0
55.447244167 52.7974700927734 8.0
14.282219599 25.0312442779541 9.0
90.564740323 88.9465026855469 10.0
77.301426271 74.8580093383789 11.0
84.067740057 83.7080459594727 12.0
84.067740057 80.8328018188477 13.0
84.067740057 80.1614456176758 14.0
0 -1.3222930431366 15.0
0 -0.344551622867584 16.0
-7.2299182576e-13 0.501116693019867 17.0
58.049486605 59.7023010253906 18.0
-58.049486625 -59.2217788696289 19.0
-58.049486603 -60.2557907104492 20.0
-58.049486616 -62.7721710205078 21.0
79.089975492 79.1132431030273 22.0
138.49721424 140.447402954102 23.0
105.16517765 109.268852233887 24.0
21.59899047 21.8382625579834 25.0
217.58718974 223.894180297852 26.0
217.58718974 226.037353515625 27.0
25.770421133 27.6371021270752 28.0
243.35761088 256.538940429688 29.0
243.35761088 259.378143310547 30.0
46.794658636 45.3545455932617 31.0
9.3279929039 9.155104637146 32.0
21.040488867 20.3900718688965 33.0
21.040488867 14.7594356536865 34.0
21.040488867 20.235502243042 35.0
9.3933147598 8.45570755004883 36.0
8.1927453955 6.68218469619751 37.0
-1.1623616491e-12 0.93789130449295 38.0
-1.1623616491e-12 -1.89944624900818 39.0
-12.106718654 -12.8285684585571 40.0
-25.143037703 -22.9966411590576 41.0
5.146188256 6.58392333984375 42.0
-32.964689678 -35.0429611206055 43.0
93.347844889 91.43017578125 44.0
93.347844903 93.3094863891602 45.0
6.6339627234 9.41577529907227 46.0
2.3965423574 2.65504670143127 47.0
175.77805804 167.175750732422 48.0
169.66362673 167.820404052734 49.0
0 0.994617342948914 50.0
69.729463786 76.3901290893555 51.0
90.564740333 87.5362396240234 52.0
84.067740067 81.5448150634766 53.0
84.067740067 82.0349273681641 54.0
84.067740067 80.1372985839844 55.0
0 1.44336652755737 56.0
79.089975502 80.3284530639648 57.0
217.58718975 221.905624389648 58.0
243.35761089 250.509979248047 59.0
21.040488877 23.9585723876953 60.0
0 1.12809407711029 61.0
137.17687742 143.006546020508 0.0
137.17687742 143.028930664062 1.0
135.86355256 143.550674438477 2.0
131.0744823 135.794906616211 3.0
0 -2.14743828773499 4.0
0 0.342255353927612 5.0
-59.42861149 -59.2386283874512 6.0
69.705106679 70.1773300170898 7.0
65.43029471 62.4528312683105 8.0
4.2748119615 8.48783111572266 9.0
78.669161727 80.2065200805664 10.0
65.229460649 71.2585144042969 11.0
77.500548395 80.8047180175781 12.0
77.500548395 77.9000701904297 13.0
77.500548395 76.9198608398438 14.0
0 -0.64198762178421 15.0
0 0.28728449344635 16.0
0 -3.15485000610352 17.0
59.754945382 58.5015258789062 18.0
-59.754945394 -59.4383087158203 19.0
-59.754945372 -60.0649642944336 20.0
-59.754945384 -61.8801422119141 21.0
80.656571382 77.8813858032227 22.0
140.87903669 140.394561767578 23.0
106.41719675 107.357681274414 24.0
22.90053999 20.4083709716797 25.0
221.53560808 223.594223022461 26.0
221.53560808 221.987747192383 27.0
27.282645659 25.3551406860352 28.0
248.81825375 253.574676513672 29.0
248.81825375 252.408157348633 30.0
47.033013387 47.2299919128418 31.0
9.3131031344 9.18030452728271 32.0
20.901625993 20.4819431304932 33.0
20.901625993 21.1129970550537 34.0
20.901625993 20.8195400238037 35.0
9.3083381265 8.28715133666992 36.0
8.044767894 8.38281917572021 37.0
-2.5262790395e-09 -0.804296553134918 38.0
-2.5262785966e-09 0.113371789455414 39.0
-12.387792857 -14.5510091781616 40.0
-23.81592281 -27.5234241485596 41.0
4.9278776077 3.5162615776062 42.0
-31.108913619 -34.7394485473633 43.0
92.052766479 91.3064651489258 44.0
92.05276649 90.2343292236328 45.0
6.7273095346 3.00315618515015 46.0
2.6568330431 1.83220052719116 47.0
137.17687743 136.680679321289 48.0
131.07448231 127.58861541748 49.0
0 3.53733205795288 50.0
69.705106686 70.9860610961914 51.0
78.669161734 84.2529830932617 52.0
77.500548402 81.0479965209961 53.0
77.500548402 80.9212951660156 54.0
77.500548402 79.4541549682617 55.0
0 -0.0822644829750061 56.0
80.656571391 80.3950500488281 57.0
221.53560809 218.26057434082 58.0
248.81825375 246.469757080078 59.0
20.901626002 19.9398994445801 60.0
0 2.89454770088196 61.0
152.202548229204 157.322189331055 0.0
152.202548229204 164.971481323242 1.0
150.99109542 164.711318969727 2.0
146.20202566 151.026153564453 3.0
0 1.26391172409058 4.0
-7.53088371000001e-09 -0.803841531276703 5.0
-62.1573908402249 -61.9489936828613 6.0
72.9440734052873 72.8865509033203 7.0
46.5415529112873 52.2627944946289 8.0
26.4025204932873 19.0091743469238 9.0
88.8488293402873 86.2480926513672 10.0
75.9093493362873 74.6933212280273 11.0
81.5446026952873 85.0018539428711 12.0
81.5446026952873 81.4387893676758 13.0
81.5446026952873 81.7567443847656 14.0
3.7192044836e-09 -0.95390123128891 15.0
1.1032933818e-09 -0.267994701862335 16.0
5.53019511999997e-09 3.51214385032654 17.0
56.9125642503385 59.1101608276367 18.0
-56.912564215931 -59.7802848815918 19.0
-56.912564197931 -60.6737442016602 20.0
-56.912564177931 -61.6849098205566 21.0
78.7357316607263 79.2635955810547 22.0
143.198518706339 141.775695800781 23.0
107.867294635423 109.505264282227 24.0
23.1717073214232 22.4024963378906 25.0
221.934250368408 229.750106811523 26.0
221.934250368408 224.472885131836 27.0
26.8582280894085 25.9367046356201 28.0
248.792478458408 254.338592529297 29.0
248.792478458408 255.061828613281 30.0
47.0090673619672 45.4946746826172 31.0
9.37531978094653 8.91460418701172 32.0
21.8231674826749 21.540901184082 33.0
21.8231674826749 19.0893440246582 34.0
21.8231675198232 20.4136791229248 35.0
9.61824148415515 7.82994031906128 36.0
8.30270502755515 7.87763595581055 37.0
-6.48639794000001e-09 1.58322811126709 38.0
-6.48639794000001e-09 -2.12903046607971 39.0
-11.9894986449655 -11.1676397323608 40.0
-25.2428782237224 -24.7264995574951 41.0
5.05417011467277 4.72118091583252 42.0
-32.8960729120728 -33.7942428588867 43.0
96.4844857120728 93.1328887939453 44.0
96.4844855660728 93.153434753418 45.0
7.04105604146356 6.34788608551025 46.0
2.45800433126356 1.64446806907654 47.0
152.20254823 161.616958618164 48.0
146.20202567 149.350082397461 49.0
0 -0.502545297145844 50.0
72.944073405 75.8470306396484 51.0
88.848829341 87.7344512939453 52.0
81.544602696 81.6076049804688 53.0
81.544602696 81.570915222168 54.0
81.544602696 80.1489715576172 55.0
5.5301951236e-09 3.13362646102905 56.0
78.735731661 78.9906845092773 57.0
221.93425037 224.149429321289 58.0
248.79247845 249.284881591797 59.0
21.82316752 16.0651054382324 60.0
0 3.05733728408813 61.0
172.355090980436 154.491729736328 0.0
172.355090949466 165.270889282227 1.0
170.82001198 170.870452880859 2.0
167.24577199 200.642349243164 3.0
-6.7856787405e-10 111.297546386719 4.0
-9.998601357e-09 11.1175413131714 5.0
-87.1411577614062 -26.3115520477295 6.0
94.4993196104218 117.315338134766 7.0
74.0841731704218 57.1876258850098 8.0
20.4151464394218 48.3380126953125 9.0
64.9550990484218 93.6061553955078 10.0
54.3575233934218 53.3722496032715 11.0
74.6123992114218 80.7476577758789 12.0
74.6123992114218 76.6094207763672 13.0
74.6123992114218 54.6823348999023 14.0
0 5.46304178237915 15.0
0 5.69322776794434 16.0
-2.78330012e-09 9.04144382476807 17.0
70.5612062795934 62.6850090026855 18.0
-70.5612062801327 -61.1007308959961 19.0
-71.2129985461327 -50.7910766601562 20.0
-71.2806356701327 -53.1272583007812 21.0
87.8161691704654 77.9925994873047 22.0
138.402740412594 146.763916015625 23.0
109.331288093242 117.089317321777 24.0
18.7215428012421 21.2151737213135 25.0
226.218909580871 228.880294799805 26.0
226.218909580871 225.20735168457 27.0
21.6695657058713 17.3275852203369 28.0
247.888475290871 231.142425537109 29.0
247.888475290871 248.616668701172 30.0
47.0768019380258 44.2265434265137 31.0
9.48791469825329 3.73199081420898 32.0
16.5355335002187 8.91613674163818 33.0
16.5355335002187 23.2562580108643 34.0
16.5355334915061 20.1860370635986 35.0
7.34880285522712 7.83675241470337 36.0
5.97471621912636 6.75076055526733 37.0
-8.73526372600001e-09 -5.82499742507935 38.0
-8.735263725e-09 -4.04763793945312 39.0
-11.4255071585664 -1.17507123947144 40.0
-25.4132546591297 -27.6501083374023 41.0
4.54713599135151 0.444972932338715 42.0
-32.7915121798515 -33.8603744506836 43.0
101.118414492851 71.5796966552734 44.0
99.8816846818515 104.63516998291 45.0
6.5869072657 -1.64678645133972 46.0
2.6626125083 4.84823131561279 47.0
172.35509098 169.341079711914 48.0
167.245772 160.495315551758 49.0
0 42.9410438537598 50.0
94.49931961 74.2149810791016 51.0
64.955099048 72.200080871582 52.0
74.612399212 65.98095703125 53.0
74.612399212 77.9971389770508 54.0
74.612399212 82.692268371582 55.0
0 13.0352535247803 56.0
87.816169171 60.7117118835449 57.0
226.21890958 222.701217651367 58.0
247.88847529 246.038925170898 59.0
16.535533511 23.7425174713135 60.0
0 9.9270601272583 61.0
157.65644031 162.174331665039 0.0
157.65644031 166.539016723633 1.0
156.39682282 162.734405517578 2.0
151.6077531 155.320343017578 3.0
0 -6.04369640350342 4.0
0 8.13252735137939 5.0
-69.458886709 -61.3907661437988 6.0
81.552202036 71.298583984375 7.0
54.066281223 52.3880386352539 8.0
27.485920803 20.8511390686035 9.0
89.737972437 90.1127014160156 10.0
76.740560592 77.5459594726562 11.0
84.217148036 82.6173934936523 12.0
84.217148036 81.777946472168 13.0
84.217148036 82.3778610229492 14.0
0 -0.00140321254730225 15.0
0 -0.0232996344566345 16.0
0 0.850755751132965 17.0
58.126782676 58.2879066467285 18.0
-58.126782686 -59.4120788574219 19.0
-58.126782664 -59.4753761291504 20.0
-58.126782676 -62.13232421875 21.0
80.083504625 79.1843032836914 22.0
142.34171634 139.875381469727 23.0
107.47551412 111.708885192871 24.0
21.727588103 20.3942070007324 25.0
222.42522098 224.759689331055 26.0
222.42522098 220.163665771484 27.0
26.553868223 27.0234813690186 28.0
248.97908921 254.043685913086 29.0
248.97908921 254.650268554688 30.0
46.479927875 47.8236236572266 31.0
9.3618932703 8.95964431762695 32.0
21.95672194 20.6577892303467 33.0
21.95672194 17.3291893005371 34.0
21.956721954 20.6631240844727 35.0
9.4567433989 9.03712177276611 36.0
8.2381993183 8.19704532623291 37.0
-1.4920506262e-08 -1.36918997764587 38.0
-1.4921513217e-08 -3.44121599197388 39.0
-11.598314028 -16.9725723266602 40.0
-24.80657705 -27.9332370758057 41.0
5.2214173278 5.30205249786377 42.0
-32.48137828 -33.2649307250977 43.0
94.0566714 89.9106369018555 44.0
94.056671412 92.9812622070312 45.0
6.8309627456 6.72204399108887 46.0
2.4877884008 2.39770698547363 47.0
157.65644032 167.137603759766 48.0
151.60775311 157.791793823242 49.0
0 -2.55766296386719 50.0
81.552202045 77.4887390136719 51.0
89.737972446 90.6845397949219 52.0
84.217148045 80.5391082763672 53.0
84.217148045 79.4933776855469 54.0
84.217148045 79.9844284057617 55.0
2.3524907173e-10 -0.0422603487968445 56.0
80.083504635 79.9156494140625 57.0
222.42522098 217.496826171875 58.0
248.97908922 248.03825378418 59.0
21.956721964 21.7224063873291 60.0
0 0.433449566364288 61.0
150.29574753 157.828094482422 0.0
150.29574753 160.309814453125 1.0
148.97077664 159.970336914062 2.0
144.18170692 144.969848632812 3.0
0 -1.9567084312439 4.0
0 2.38657546043396 5.0
-61.453931685 -58.354793548584 6.0
72.164945027 80.8728637695312 7.0
46.783760075 56.6258506774902 8.0
25.381184942 27.2121410369873 9.0
86.720823861 88.4160079956055 10.0
73.885091285 77.0278549194336 11.0
79.921301763 83.9798278808594 12.0
79.921301763 80.4702224731445 13.0
79.921301763 77.7390213012695 14.0
0 -2.32883381843567 15.0
0 -0.705342471599579 16.0
0 -2.21999216079712 17.0
59.272806486 59.4338340759277 18.0
-59.272806496 -58.7741966247559 19.0
-59.272806474 -60.8734169006348 20.0
-59.272806486 -62.0842323303223 21.0
81.049502375 79.2354736328125 22.0
145.64284032 142.156188964844 23.0
110.75546897 107.378372192383 24.0
21.852686223 21.714916229248 25.0
226.69234271 226.128646850586 26.0
226.69234271 225.535491943359 27.0
25.540558355 26.4161434173584 28.0
252.23290107 251.39501953125 29.0
252.23290107 249.915390014648 30.0
46.62822853 44.3883666992188 31.0
9.4066770434 8.21288585662842 32.0
21.776695879 20.2067070007324 33.0
21.776695879 19.2272357940674 34.0
21.776695879 21.4385738372803 35.0
9.612299931 7.5541877746582 36.0
8.3406354029 7.81544160842896 37.0
0 -0.211334764957428 38.0
0 -5.97356224060059 39.0
-11.910775681 -12.4081106185913 40.0
-24.646047697 -27.7698726654053 41.0
5.3167796666 3.33823108673096 42.0
-32.567337667 -31.7139682769775 43.0
95.770986605 95.1708221435547 44.0
95.770986618 94.5535125732422 45.0
6.7218992606 5.49050426483154 46.0
2.6918377239 2.3685610294342 47.0
150.29574754 163.129943847656 48.0
144.18170693 145.618225097656 49.0
0 0.832417547702789 50.0
72.164945037 74.8575973510742 51.0
86.720823871 91.250129699707 52.0
79.921301773 80.9856033325195 53.0
79.921301773 80.4847717285156 54.0
79.921301773 79.7217102050781 55.0
0 0.53804737329483 56.0
81.049502385 79.8111877441406 57.0
226.69234272 223.118606567383 58.0
252.23290108 248.665481567383 59.0
21.776695889 17.0536575317383 60.0
0 0.990958750247955 61.0
179.203453472687 156.849548339844 0.0
179.203453472687 162.928024291992 1.0
178.03445721 163.097991943359 2.0
172.8683367 159.225952148438 3.0
0 2.20367693901062 4.0
-6.245505111e-09 -0.786801159381866 5.0
-73.2460716744736 -64.4604339599609 6.0
84.9034963326577 81.9991912841797 7.0
53.3481898566577 51.2916488647461 8.0
31.5553064756577 28.9774799346924 9.0
98.0963289366577 86.2174987792969 10.0
84.8078374576577 75.7922668457031 11.0
82.7521103596577 85.2691040039062 12.0
82.7521103596577 81.1116180419922 13.0
82.7521103596577 78.4533996582031 14.0
0 1.09654688835144 15.0
0 -0.833416759967804 16.0
-5.340848837e-09 -2.99944448471069 17.0
56.2069001211769 60.1412353515625 18.0
-56.2069001279866 -58.3474540710449 19.0
-56.2069001069866 -60.1050796508789 20.0
-56.2069001129866 -55.9419136047363 21.0
76.2401764272048 77.6256332397461 22.0
149.716225914177 131.061096191406 23.0
114.486518247721 113.026634216309 24.0
22.1071740707212 24.7782745361328 25.0
225.956402335374 224.00666809082 26.0
225.956402335374 224.770462036133 27.0
29.0842634733743 25.5518321990967 28.0
255.040665815374 255.908355712891 29.0
255.040665815374 251.235794067383 30.0
46.987305010211 45.1370086669922 31.0
9.3817391489 9.00999164581299 32.0
20.0332763144209 20.012716293335 33.0
20.0332763144209 14.4477376937866 34.0
20.0332763229495 19.3942356109619 35.0
10.305600937 6.10015106201172 36.0
8.9324250706 7.58011054992676 37.0
-9.252993654e-09 -2.5513482093811 38.0
-9.252993653e-09 -2.24348950386047 39.0
-11.8359959974933 -13.5330247879028 40.0
-24.7638933049277 -29.2619037628174 41.0
5.42883465614811 6.4380989074707 42.0
-32.8216220835481 -31.4959926605225 43.0
93.5548889625481 94.4348449707031 44.0
93.5548889425481 93.4975662231445 45.0
6.4499842098 5.68216514587402 46.0
2.2766408282 2.06433629989624 47.0
179.20345347 159.863983154297 48.0
172.86833671 173.666458129883 49.0
0 3.76059150695801 50.0
84.903496333 77.5471343994141 51.0
98.096328936 88.5695495605469 52.0
82.752110359 79.4581909179688 53.0
82.752110359 81.3844375610352 54.0
82.752110359 78.8069076538086 55.0
0 -0.48759251832962 56.0
76.240176428 79.1658248901367 57.0
225.95640234 219.110946655273 58.0
255.04066581 247.554107666016 59.0
20.033276323 29.2174987792969 60.0
0 1.69464445114136 61.0
175.83446047 159.331268310547 0.0
175.83446047 167.472793579102 1.0
174.60824964 169.244247436523 2.0
169.81917992 179.026992797852 3.0
0 2.73298025131226 4.0
-8.9959772292e-10 5.3869948387146 5.0
-64.615329602 -61.8555946350098 6.0
75.157270719 75.5880432128906 7.0
45.082002224 52.7280693054199 8.0
30.075268494 24.0262718200684 9.0
85.009278218 87.2977066040039 10.0
71.779518342 76.2696914672852 11.0
79.189524017 80.1190643310547 12.0
79.189524017 77.54296875 13.0
79.189524017 79.2548675537109 14.0
0 -1.33206820487976 15.0
0 -0.595543563365936 16.0
-4.9553559936e-09 -0.425393879413605 17.0
55.083619948 58.3408889770508 18.0
-55.0836199718431 -59.176212310791 19.0
-55.0836199498431 -58.4966430664062 20.0
-55.0836199608431 -58.4256324768066 21.0
76.727615254 79.4428787231445 22.0
143.62822175 141.994247436523 23.0
109.03599735 109.051864624023 24.0
22.755557567 21.4527816772461 25.0
220.355837019879 226.972106933594 26.0
220.355837019879 224.638931274414 27.0
26.8124841588793 24.7122192382812 28.0
247.168321169879 258.351257324219 29.0
247.168321169879 254.782272338867 30.0
46.807869018 46.1999740600586 31.0
9.3870338421 8.8903226852417 32.0
21.643995298 19.1522941589355 33.0
21.643995298 18.4990673065186 34.0
21.643995295 19.6684551239014 35.0
9.9283574993 10.5185642242432 36.0
8.6414679427 7.52037668228149 37.0
-8.5543335034e-09 -0.646056950092316 38.0
-8.5543588766e-09 -3.28868103027344 39.0
-11.706428615 -12.7851419448853 40.0
-25.139974939 -26.3869705200195 41.0
5.3868171088 4.20161914825439 42.0
-33.170334999 -33.5060081481934 43.0
96.442541164 92.6611938476562 44.0
96.442541164 95.3925628662109 45.0
6.9371028106 6.44675827026367 46.0
2.6780893725 2.48320269584656 47.0
175.83446048 161.528457641602 48.0
169.81917993 164.5703125 49.0
0 3.64193224906921 50.0
75.157270721 77.483283996582 51.0
85.00927822 90.3457946777344 52.0
79.189524019 81.4577407836914 53.0
79.189524019 80.6318130493164 54.0
79.189524019 80.0802459716797 55.0
0 0.0996492505073547 56.0
76.727615262 79.8878707885742 57.0
220.35583702 223.386199951172 58.0
247.16832117 249.768112182617 59.0
21.643995302 23.620584487915 60.0
0 1.07069671154022 61.0
};
\addlegendentry{$R^2$=0.963}
\end{axis}

\end{tikzpicture}
}}
        % \includegraphics[width=0.45\textwidth]{figures/}
        \caption{Model results using the losses associated with the predicted flows injected at the nodes and transported by the pipelines and compressors in the colombian 63-node network.}
        \label{fig:col_base_f_results}
\end{figure}


In this stage, a t-test was performed to compare the computation times between the optimization model and the GNN-based model. The results yielded a T-statistic of $47.25$, $198$ degrees of freedom and a p-value of $4.46 \times 10^{-110}$. These values confirm with high statistical confidence that the GNN-based model significantly outperforms the optimization model in terms of computational efficiency when predicting flows across the network.


\begin{table}[htbp]
\centering
\begin{tabular}{|c|p{0.8cm}|p{0.55cm}|c|c|c|c|}
    \hline
    Method & \multicolumn{2}{|c|}{Experiment} & Node Value & Edge Value & Balance Value & Time \\ \hline
           & \centering N & \centering E &  &  &  &  \\ \hline
    Opt    & \makebox[0.8cm]{\centering \checkmark} &   & 11.41 ± 49.23 & 63.52 ± 81.62 & -2.15 ± 16.49 & 5.01 ± 5.59 \\ \hline
    GNN    & \makebox[0.8cm]{\centering \checkmark} &   & 11.38 ± 49.13 & 0.91 ± 1.26   & -2.19 ± 58.56 & 0.13 ± 0.07 \\ \hline
    Opt    & \makebox[0.8cm]{\centering \checkmark} & \makebox[0.55cm]{\centering \checkmark} & 11.41 ± 49.23 & 63.52 ± 81.62 & -2.15 ± 16.50 & 5.01 ± 5.59 \\ \hline
    GNN    & \makebox[0.8cm]{\centering \checkmark} & \makebox[0.55cm]{\centering \checkmark} & 11.40 ± 48.97 & 63.45 ± 81.38 & -2.16 ± 16.60 & 0.14 ± 0.08 \\ \hline
\end{tabular}
\caption{Comparison of mean and standard deviation values for nodal flows, edge flows, nodal balance, and prediction time between the optimization model (Opt) and the GNN-based model, across 100 samples. The 'N' column refers to experiments considering the nodal loss, and the 'E' column refers to experiments considering the edge loss.}
\label{tab:lineal_col_results}
\end{table}


The \cref{tab:lineal_col_results} presents a comparison between the optimization model (Opt) and the GNN-based model across four experiments, which vary based on the method used (Opt or GNN) and the inclusion of nodal loss (N) and edge loss (E). The comparison focuses on nodal flows, edge flows, nodal balance, and prediction time, with the mean and standard deviation calculated from 100 samples.


In terms of nodal flows, both the optimization model and the GNN-based model show almost identical results across all experiments. For both models, the mean nodal flow remains consistent at around $11.41$, with very similar standard deviations: $49.23$ for the optimization model and $49.13$ and $48.97$ for the GNN model in the experiments considering only nodal loss and both nodal and edge losses, respectively. This consistency highlights that both models perform similarly in predicting nodal flows.


However, the edge flow values present a more notable distinction between the two models. In the experiments without edge loss, the GNN-based model significantly outperforms the optimization model, achieving a much lower mean edge flow of $0.91 \pm 1.26$, compared to the optimization model’s $63.52 \pm 81.62$. This demonstrates the GNN’s improved capability to predict edge flows with reduced variability. In the experiments that include both nodal and edge losses, the edge flow predictions for both models converge, with the GNN reporting $63.45 \pm 81.38$ and the optimization model showing $63.52 \pm 81.62$.



Regarding the nodal balance, the models display similar mean results. In the experiments without edge loss, the optimization model reports a mean balance of $-2.15 \pm 16.49$, while the GNN produces $-2.19 \pm 58.56$. In the experiments considering both nodal and edge losses, the mean balance results become more consistent, with the GNN model showing $-2.16 \pm 16.60$ and the optimization model remaining at $-2.15 \pm 16.50$. The higher variability observed in the GNN’s predictions during the nodal-only experiments is reduced when edge loss is included, indicating improved consistency.

Finally, the prediction time demonstrates the GNN’s computational efficiency. In all experiments, the GNN completes predictions significantly faster than the optimization model. In the nodal-only experiments, the GNN requires just $0.13$ seconds on average, compared to $5.01$ seconds for the optimization model. In the nodal and edge loss experiments, the GNN’s prediction time increases only slightly to $0.14$ seconds, whereas the optimization model maintains the same $5.01$ seconds. This significant reduction in prediction time underscores the GNN’s efficiency in handling large-scale network predictions.

\section{Discussion and conclusions}

In this section, we analyze and summarize the performance of the proposed GNN-basd model in predicting the operational parameters of natural gas transportation systems. The evaluation was conducted on two test cases: a simplified 8-node network and a more complex real-world system with 63 nodes representing the Colombian natural gas system. The results demonstrate the GNN model's ability to deliver accurate predictions with significant improvements in computational efficiency compared to traditional optimization methods.


The 8-node network provided a controlled environment to test the GNN's predictive capabilities. The initial experiment, where only nodal losses were considered, resulted in a MSE of 1.98. Although the model accurately identified the generating node, its predictions were more concentrated, lacking the dispersion seen in the actual values. This behavior suggests that while the GNN could capture certain trends in gas generation at the nodes, its predictive distribution did not fully align with real-world variability. However, the prediction of edge flows showed significant deviation from the actual values due to the exclusion of edge-related losses in the training process. This underperformance was expected, given this initial model's lack of focus on edge dynamics.


The introduction of edge-related losses in the second stage of the experiment significantly improved the GNN's performance. With optimized hyperparameters, the GNN achieved an overall loss of 2.02, reflecting its enhanced ability to predict both nodal and edge flows. Furthermore, the MSE associated with flow balancing at nodes dropped dramatically, from 284,764 to 1,719, highlighting the importance of including edge losses in the training objective. The comparison between scatter plots in \cref{fig:dummy_base_results} and \cref{fig:dummy_base_f_results} confirmed this improvement, with an $R^2$ of 0.999 for edge flow predictions, nearly aligning with the actual data.



 The results presented in \cref{tab:lineal_dummy_results} illustrate the excellent performance of the GNN-based model, making a trade-off in terms of computational speed and prediction accuracy. The GNN-based model demonstrated advantages over the optimization model regarding computational efficiency, as confirmed by the t-test results. When comparing the time required to generate predictions using node losses only, the GNN-based model outperformed the optimization model. The t-test produced a T-statistic of $14.94$, with $198$ degrees of freedom and a p-value of $1.32 \times 10^{-34}$, indicating a computational advantage for the GNN. On average, the GNN required $0.144$ seconds to make predictions in the Base experiment, compared to $0.792$ seconds for the optimization model. Similar results were observed in the Base-f experiment, where the GNN took $0.145$ seconds, while the optimization model maintained the same prediction time of $0.792$ seconds.

In the second evaluation, considering both node and edge losses, the GNN-based model again outperformed the optimization model. The t-test revealed a T-statistic of $14.81$, with $198$ degrees of freedom and a p-value of $3.47 \times 10^{-34}$, further reinforcing the computational advantage of the GNN. Despite the additional complexity introduced by edge flows, the GNN completed predictions faster than the optimization model, maintaining an average time of $0.144$ seconds in the Base experiment and $0.145$ seconds in the Base-f experiment, compared to the optimization model's consistent time of $0.792$ seconds.




The performance of the GNN-based model was evaluated on the more complex 63-node Colombian natural gas system. Initially, the model was optimized using only the nodal loss, resulting in a total loss of $11.38$ for nodal flows. However, its ability to capture the gas balance was less accurate, with a balance loss of $-2.19 \pm 58.56$. This is due to the exclusion of edge flows during optimization, similar to the first experiment on the 8-node network.

Despite this limitation, the GNN demonstrated strong predictive accuracy for nodal flows, with an $R^2$ value of 0.996, as seen in \cref{fig:lineal_col_base_results}. This high correlation shows the GNN's ability to predict nodal gas injections effectively, identifying which nodes had active injections. However, since edge flows were not considered in the loss function, predictions for edge flows were less accurate.

The computational efficiency of the GNN-based model was significant. As shown in \cref{tab:lineal_col_results}, the GNN completed predictions in an average of $0.13$ seconds, compared to $5.01$ seconds for the optimization model. The t-test confirmed this advantage, with a T-statistic of $47.29$ and a p-value of $4.92 \times 10^{-110}$.

In the second experiment, where the loss function included edge flows, the GNN-based model's performance improved significantly for edge predictions. \cref{fig:col_base_f_results} shows that including edge-related losses allowed the GNN to achieve an $R^2$ value of 0.996 for edge flows, closely matching the actual data. This improvement highlights the importance of accounting for edge flows to enhance overall model accuracy.

The comparison between the optimization and GNN-based models in \cref{tab:lineal_col_results} shows that both approaches yielded nearly identical mean and standard deviation values for nodal and edge flows. However, the GNN model achieved these results with much greater computational efficiency. The GNN consistently required less time to generate predictions, averaging $0.14$ seconds in the Base-f experiment, compared to the optimization model's $5.01$ seconds. This efficiency makes the GNN a suitable choice for real-time or large-scale applications.

