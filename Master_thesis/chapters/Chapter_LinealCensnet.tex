\chapter{Power System - Censnet} \label{cap:mpcc}

\section{Preliminaries}

\subsection{Graph definition}


A \textbf{graph} $G$ is a mathematical structure that represents a set of interconnected objects. These objects are known as \textbf{vertices} (or \textbf{nodes}), denoted by the set $V(G)$, and the connections between them are called \textbf{edges} (or \textbf{arcs}), denoted by the set $E(G)$. Formally, a graph is defined as an ordered pair $G = (V, E)$, where $V(G)$ is a non-empty set of vertices, and $E(G) \subseteq \{(u, v) \mid u, v \in V(G), u \neq v\}$ is a set of edges, where each edge connects two distinct vertices.

Graphs can be categorized based on the properties of their edges. An \textbf{undirected graph} has edges that do not have a direction, so the pair $(u, v) = (v, u)$ represents an edge that simply connects vertices $u$ and $v$. In contrast, in a \textbf{directed graph} (or \textbf{digraph}), each edge $(u, v) \in E(G)$ has a direction, meaning it goes from vertex $u$ to vertex $v$. This implies that $(u, v) \neq (v, u)$ unless $u = v$

 %
% \begin{figure}
%     \begin{center}
%         \setlength\figurewidth{.5\textwidth}        
%             \setlength\figureheight{0.4\textwidth}
%         \resizebox{\figurewidth}{\figureheight}{\input{figures/Chapter_LinealCensnet/graph_def}}
%         % \input{figures/Chapter_LinealCensnet/graph_def}
%         % \includegraphics[width=0.95\textwidth]{figures/}
%     \end{center}
%     \caption{Directed graph}\label{fig:graph_definition}
% \end{figure}
%
\begin{figure}
    \centering
        \setlength\figurewidth{.5\textwidth}        
        \setlength\figureheight{0.4\textwidth} 
        \subfloat [Undirected graph] {\label{fig:undirected_graph_def}\resizebox{\figurewidth}{\figureheight}{\begin{tikzpicture}[>=Stealth, node distance=5cm]
    
    % Define styles for the nodes and edges
    \tikzstyle{node_style}=[circle, draw, minimum size=.9cm, inner sep=0pt, font=\normalsize]
    \tikzstyle{edge_style}=[draw, -, thick]

    % Nodes
    \node[node_style] (n1) {1};
    \node[node_style] (n2) [right of=n1] {2};
    \node[node_style] (n3) [right of=n2] {3};
    \node[node_style] (n4) [below of=n2] {4};

    % Edges with labels
    \draw[edge_style] (n1) edge node[above] {A} (n2);
    \draw[edge_style] (n4) edge node[left] {B} (n1);
    \draw[edge_style] (n4) edge node[right] {C} (n2);
    \draw[edge_style] (n3) edge node[right] {D} (n4);
    \draw[edge_style] (n1) to[out=30, in=150] node[above] {E} (n3);
    \draw[edge_style] (n3) edge node[above] {F} (n2);

\end{tikzpicture}
}}
        \subfloat [Directed graph] {\label{fig:direc_graph_def}\resizebox{\figurewidth}{\figureheight}{\begin{tikzpicture}[>=Stealth, node distance=5cm]
    
    % Define styles for the nodes and edges
    \tikzstyle{node_style}=[circle, draw, minimum size=.9cm, inner sep=0pt, font=\normalsize]
    \tikzstyle{edge_style}=[draw, ->, thick]

    % Nodes
    \node[node_style] (n1) {1};
    \node[node_style] (n2) [right of=n1] {2};
    \node[node_style] (n3) [right of=n2] {3};
    \node[node_style] (n4) [below of=n2] {4};

    % Edges with labels
    \draw[edge_style] (n1) edge node[above] {A} (n2);
    \draw[edge_style] (n4) edge node[left] {B} (n1);
    \draw[edge_style] (n4) edge node[right] {C} (n2);
    \draw[edge_style] (n3) edge node[right] {D} (n4);
    % \draw[edge_style] (n1) to[out=30, in=150] node[above] {E} (n3);
    % \draw[edge_style] (n3) edge node[above] {F} (n2);

\end{tikzpicture}
}}
        % \includegraphics[width=0.45\textwidth]{figures/}
    \caption{Types of graphs}\label{fig:graph_definition}
\end{figure}

In \Cref{fig:graph_definition}, two graphs are represented, each composed of four nodes labeled $1$, $2$, $3$, and $4$,, and six edges labeled $A$, $B$, $C$, $D$, $E$, and $F$. The difference between them lies in the type of graph they represent. For example, in \Cref{fig:undirected_graph_def}, the edge c shows a connection between nodes 2 and 4. However, in \Cref{fig:direc_graph_def}, this connection provides additional information: a direction, which, in the context of this study, could represent the direction of a specific element, such as electric power or flow.


A graph can be represented in various ways using matrices, each capturing different aspects of the graph's structure. The two most common matrix representations are the \textbf{adjacency matrix} and the \textbf{incidence matrix}.

The \textbf{adjacency matrix} of a graph is a square matrix used to represent the connections between vertices. For a graph $G$ with $n$ vertices, the adjacency matrix $A$ is an $n \times n$ matrix where the entry $a_{ij}$ is defined as follows:

\[
a_{ij} =
\begin{cases}
1 & \text{if there is an edge from vertex } i \text{ to vertex } j, \\
0 & \text{otherwise}.
\end{cases}
\]

For a directed graph, the adjacency matrix captures the direction of the edges. Below is the adjacency matrix for the directed graph shown earlier:

\[
A = \begin{pmatrix}
0 & 1 & 0 & 1 \\
1 & 0 & 0 & 1 \\
0 & 1 & 0 & 1 \\
1 & 1 & 1 & 0
\end{pmatrix}
\]


The \textbf{incidence matrix} of a graph represents the relationship between vertices and edges. For a graph $G$ with $n$ vertices and $m$ edges, the incidence matrix $I$ is an $n \times m$ matrix where the entry $i_{ij}$ is defined as follows:

\[
i_{ij} =
\begin{cases}
1 & \text{if vertex } i \text{ is the starting point of edge } j \text{ in a directed graph}, \\
-1 & \text{if vertex } i \text{ is the endpoint of edge } j \text{ in a directed graph}, \\
0 & \text{if vertex } i \text{ is not connected to edge } j.
\end{cases}
\]

For the directed graph previously described, the incidence matrix is given by:

\[
I = \begin{pmatrix}
1 & -1 & 0 & 0 \\
-1 & 0 & -1 & 0 \\
0 & 0 & 0 & 1  \\
0 & 1 & 1 & -1 
\end{pmatrix}
\]

\subsection{Neural networks}

\subsubsection{Multi-Layered Perceptrons}


A Multilayer Perceptron (MLP) is a fundamental type of artificial neural network, often regarded as one of the building blocks of deep learning. At its core, an MLP consists of multiple layers of nodes, or neurons, where each layer is fully connected to the next one. The architecture typically includes an input layer, one or more hidden layers, and an output layer. The neurons in each layer are connected to the neurons in the subsequent layer through weighted connections, which are the key parameters learned during the training process. One of the most significant properties of an MLP is its ability to function as a universal approximator. This means that, given sufficient neurons in the hidden layers, an MLP can approximate any continuous function to an arbitrary degree of accuracy, provided the network is trained properly.

Mathematically, an MLP can be defined as follows. Let $\mathbf{x} \in \mathbb{R}^n$ represent the input vector, where $n$ is the number of features. The output of each neuron in the first hidden layer is calculated as:

\[
\mathbf{z}^{(1)} = \sigma\left(\mathbf{W}^{(1)}\mathbf{x} + \mathbf{b}^{(1)}\right)
\]

where $\mathbf{W}^{(1)} \in \mathbb{R}^{m_1 \times n}$ is the weight matrix for the first hidden layer, with $m_1$ being the number of neurons in this layer, $\mathbf{b}^{(1)} \in \mathbb{R}^{m_1}$ is the bias vector, and $\sigma(\cdot)$ is the activation function, typically a non-linear function such as the ReLU (Rectified Linear Unit) or sigmoid function.

This process is repeated for each subsequent hidden layer $k$, where the output of the $k$-th layer is given by:

\[
\mathbf{z}^{(k)} = \sigma\left(\mathbf{W}^{(k)}\mathbf{z}^{(k-1)} + \mathbf{b}^{(k)}\right)
\]

Here, $\mathbf{W}^{(k)} \in \mathbb{R}^{m_k \times m_{k-1}}$ represents the weight matrix connecting layer $k-1$ to layer $k$, $\mathbf{b}^{(k)} \in \mathbb{R}^{m_k}$ is the bias vector for layer $k$, and $\mathbf{z}^{(k-1)}$ is the output of the previous layer.

Finally, the output layer produces the final prediction $\mathbf{\hat{y}}$:

\[
\mathbf{\hat{y}} = \sigma\left(\mathbf{W}^{(L)}\mathbf{z}^{(L-1)} + \mathbf{b}^{(L)}\right)
\]

where $L$ denotes the number of layers in the network, including the input and output layers. Depending on the nature of the problem (e.g., classification or regression), the activation function $\sigma(\cdot)$ used in the output layer can vary, with softmax being common in multi-class classification problems, and a linear activation for regression tasks.

The entire MLP is trained using a process called backpropagation, combined with an optimization algorithm like gradient descent, to minimize a loss function $J(\mathbf{y}, \mathbf{\hat{y}})$, which measures the difference between the true outputs $\mathbf{y}$ and the predicted outputs $\mathbf{\hat{y}}$.


\begin{figure}
    \centering
    \setlength\figurewidth{1\textwidth}        
    \setlength\figureheight{0.5\textwidth}
    \resizebox{\figurewidth}{\figureheight}{\begin{tikzpicture}[shorten >=1pt, ->, draw=black!50, node distance=1.5cm and 3.5cm, align=center]

    % Styles
    \tikzstyle{input} = [circle, draw, fill=green!50, minimum size=2em]
    \tikzstyle{hidden} = [circle, draw, fill=blue!50, minimum size=2em]
    \tikzstyle{output} = [circle, draw, fill=red!50, minimum size=2em]
    \tikzstyle{connection} = [->, thick]

    % Network Stage Labels
    \node[align=center] at (0,-0.4) {Input \\ Layer};
    \node[align=center] at (6,0.4) {Hidden \\ Layers};
    \node[align=center] at (12,-1.2) {Output \\ Layer};

    % Input Layer
    \foreach \i in {1,2,3}
        \node[input] (I\i) at (0,-\i*1.5) {$x_\i$};

    % Hidden Layer 1
    \foreach \i in {1,2,3,4}
        \node[hidden] (H1\i) at (3,-\i*1.5+0.75) {$z^{(1)}_\i$};

    % Hidden Layer 2
    \foreach \i in {1,2,3,4}
        \node[hidden] (H2\i) at (6,-\i*1.5+0.75) {$z^{(2)}_\i$};

    % Hidden Layer 3
    \foreach \i in {1,2,3,4}
        \node[hidden] (H3\i) at (9,-\i*1.5+0.75) {$z^{(3)}_\i$};

    % Output Layer
    \foreach \i in {1,2}
        \node[output] (O\i) at (12,-\i*1.5-0.75) {$\hat{y}_\i$};

    % Connections from Input to Hidden Layer 1
    \foreach \i in {1,2,3}
        \foreach \j in {1,2,3,4}
            \draw[connection] (I\i) -- (H1\j);

    % Connections from Hidden Layer 1 to Hidden Layer 2
    \foreach \i in {1,2,3,4}
        \foreach \j in {1,2,3,4}
            \draw[connection] (H1\i) -- (H2\j);

    % Connections from Hidden Layer 2 to Hidden Layer 3
    \foreach \i in {1,2,3,4}
        \foreach \j in {1,2,3,4}
            \draw[connection] (H2\i) -- (H3\j);

    % Connections from Hidden Layer 3 to Output Layer
    \foreach \i in {1,2,3,4}
        \foreach \j in {1,2}
            \draw[connection] (H3\i) -- (O\j);

\end{tikzpicture}
}
    % \begin{tikzpicture}[shorten >=1pt, ->, draw=black!50, node distance=1.5cm and 3.5cm, align=center]

    % Styles
    \tikzstyle{input} = [circle, draw, fill=green!50, minimum size=2em]
    \tikzstyle{hidden} = [circle, draw, fill=blue!50, minimum size=2em]
    \tikzstyle{output} = [circle, draw, fill=red!50, minimum size=2em]
    \tikzstyle{connection} = [->, thick]

    % Network Stage Labels
    \node[align=center] at (0,-0.4) {Input \\ Layer};
    \node[align=center] at (6,0.4) {Hidden \\ Layers};
    \node[align=center] at (12,-1.2) {Output \\ Layer};

    % Input Layer
    \foreach \i in {1,2,3}
        \node[input] (I\i) at (0,-\i*1.5) {$x_\i$};

    % Hidden Layer 1
    \foreach \i in {1,2,3,4}
        \node[hidden] (H1\i) at (3,-\i*1.5+0.75) {$z^{(1)}_\i$};

    % Hidden Layer 2
    \foreach \i in {1,2,3,4}
        \node[hidden] (H2\i) at (6,-\i*1.5+0.75) {$z^{(2)}_\i$};

    % Hidden Layer 3
    \foreach \i in {1,2,3,4}
        \node[hidden] (H3\i) at (9,-\i*1.5+0.75) {$z^{(3)}_\i$};

    % Output Layer
    \foreach \i in {1,2}
        \node[output] (O\i) at (12,-\i*1.5-0.75) {$\hat{y}_\i$};

    % Connections from Input to Hidden Layer 1
    \foreach \i in {1,2,3}
        \foreach \j in {1,2,3,4}
            \draw[connection] (I\i) -- (H1\j);

    % Connections from Hidden Layer 1 to Hidden Layer 2
    \foreach \i in {1,2,3,4}
        \foreach \j in {1,2,3,4}
            \draw[connection] (H1\i) -- (H2\j);

    % Connections from Hidden Layer 2 to Hidden Layer 3
    \foreach \i in {1,2,3,4}
        \foreach \j in {1,2,3,4}
            \draw[connection] (H2\i) -- (H3\j);

    % Connections from Hidden Layer 3 to Output Layer
    \foreach \i in {1,2,3,4}
        \foreach \j in {1,2}
            \draw[connection] (H3\i) -- (O\j);

\end{tikzpicture}

    \caption{}\label{fig:}
\end{figure}


\subsubsection{Graph Neural Networks}

In recent years, \textbf{Graph Neural Networks (GNNs)} have emerged as a powerful tool in machine learning, particularly for tasks involving data that can be naturally represented as graphs. Graphs are a universal data structure that can model various systems in numerous fields, including social networks, biological networks, knowledge graphs, and physical systems. Because of their ability to represent relationships and interactions between entities, graphs are used extensively to model complex structures where the data points are not independent but interconnected.

The importance of GNNs lies in their ability to directly operate on graph-structured data, extending the success of neural networks from grid-like data structures, such as images and sequences, to more general and irregular structures. Traditional neural networks, like Convolutional Neural Networks (CNNs) or Recurrent Neural Networks (RNNs), are designed to work with data that has a fixed structure. However, many real-world problems involve data that can be better described by graphs, where nodes represent entities and edges represent relationships between those entities. 


\textbf{Graph Neural Networks} can be broadly defined as a class of neural networks designed to perform inference on data described by graphs. Formally, let $G = (V, E)$ represent a graph, where $V$ is the set of nodes (or vertices) and $E$ is the set of edges. Each node $v \in V$ can be associated with a feature vector $\mathbf{x}_v$, and each edge $(u, v) \in E$ may have an associated weight or feature vector $\mathbf{e}_{uv}$. The goal of a GNN is to learn a representation for each node (or sometimes for the entire graph) by aggregating and transforming the feature information from the node's local neighborhood in the graph.


In the GNN framework, the process of message passing is understood as a series of iterations in which each node updates its representation by exchanging information with its neighbors. To move from the abstract concept to a practical implementation, it is necessary to define the specific functions used for updating and aggregating node features.

The basic message passing operation, which simplifies the original GNN model proposed by [], is expressed by the following equation:
\[
\mathbf{h}_u^{(k)} = \sigma\left( \mathbf{W}_{\text{self}}^{(k)} \mathbf{h}_u^{(k-1)} + \mathbf{W}_{\text{neigh}}^{(k)} \sum_{v \in \mathcal{N}(u)} \mathbf{h}_v^{(k-1)} + \mathbf{b}^{(k)} \right)
\]

In this equation:
\begin{itemize}
    \item \( \mathbf{h}_u^{(k)} \) represents the updated feature vector of node \( u \) at layer \( k \).
    \item The term \( \mathbf{W}_{\text{self}}^{(k)} \mathbf{h}_u^{(k-1)} \) applies a transformation to the node's own feature vector from the previous layer, enabling the node to retain and modify its self-information. 
    \item The term \( \mathbf{W}_{\text{neigh}}^{(k)} \sum_{v \in \mathcal{N}(u)} \mathbf{h}_v^{(k-1)} \) aggregates the feature vectors of the neighboring nodes \( v \) in the set \( \mathcal{N}(u) \), and then applies a transformation via the weight matrix \( \mathbf{W}_{\text{neigh}}^{(k)} \) 
    \item \( \mathbf{b}^{(k)} \) is a bias term that can be added to the weighted sum, though it is sometimes omitted for simplicity. 
    \item  The non-linear function \( \sigma(\cdot) \), such as ReLU or tanh, is applied elementwise to introduce non-linearity into the model, which is essential for capturing complex patterns in the data.

\end{itemize}

% . This aggregation step is crucial as it allows the node to incorporate information from its local neighborhood.


In the context of Graph Neural Networks (GNNs), a Graph Convolutional Network (GCN) is a specialized model that applies the concept of convolution, widely used in image processing, to graphs. First introduced by Thomas Kipf and Max Welling in 2017, GCNs offer a method to perform deep learning on graph-structured data by extending traditional convolution operations to the irregular domain of graphs.

The fundamental idea behind GCNs is to create a spectral filter that operates on graph data. The filter's purpose is to combine features from a node's local neighborhood, taking into account the graph's structure. This process is mathematically formalized in the following way:

\[
\mathbf{H} = \sigma\left( \tilde{\mathbf{D}}^{-\frac{1}{2}} \tilde{\mathbf{A}} \tilde{\mathbf{D}}^{-\frac{1}{2}} \mathbf{X} \mathbf{\Theta} \right)
\]

Where:
\begin{itemize}
    \item \( \mathbf{H} \) represents the matrix of node representations after applying the GCN layer. Each row \( \mathbf{h}_u \) in \( \mathbf{H} \) corresponds to the updated feature vector for node \( u \). 
    \item \( \mathbf{X} \) is the matrix of input node features, where each row \( \mathbf{x}_u \) corresponds to the feature vector for node \( u \) before applying the GCN layer. 
    \item \( \sigma(\cdot) \) denotes a non-linear activation function, such as ReLU, applied elementwise to introduce non-linearity into the model. 
    \item \( \tilde{\mathbf{A}} \) is the adjacency matrix of the graph, with added self-loops to account for the node itself in the aggregation. 
    \item \( \tilde{\mathbf{D}} \) is the degree matrix of the graph, modified to include the self-loops. The degree matrix is diagonal, with each diagonal entry \( \tilde{D}_{ii} \) representing the degree of node \( i \) in the graph. 
    \item \( \mathbf{\Theta} \) is a matrix of trainable parameters, which is learned during the training process to optimize the model's performance.


\end{itemize}

The expression \( \tilde{\mathbf{D}}^{-\frac{1}{2}} \tilde{\mathbf{A}} \tilde{\mathbf{D}}^{-\frac{1}{2}} \) is a normalized version of the adjacency matrix, ensuring that the eigenvalues of the operation are bounded between 0 and 1. This normalization step is crucial as it prevents issues such as exploding or vanishing gradients during the training of deep networks.

Specifically, the adjacency matrix \( \tilde{\mathbf{A}} \) is defined as:

\[
\tilde{\mathbf{A}} = \mathbf{A} + \mathbf{I}
\]

where \( \mathbf{A} \) is the original adjacency matrix, and \( \mathbf{I} \) is the identity matrix. The identity matrix \( \mathbf{I} \) ensures that each node considers its own features when aggregating information from its neighbors.

The degree matrix \( \tilde{\mathbf{D}} \) is defined as:

\[
\tilde{D}_{ii} = \sum_{j \in V} \tilde{A}_{ij}
\]

where \( V \) represents the set of all nodes in the graph. The diagonal entries of \( \tilde{\mathbf{D}} \) correspond to the degree of each node, adjusted to account for the added self-loops.




\section{Linear formulation of the natural gas system} \label{sec:LinealCensnet_formulation}


Natural gas is a widely used energy resource, particularly for electricity generation. The natural gas system consists of a network of production centers, pipelines, compressor stations, storage facilities, and distribution points that ensure reliable gas delivery from producers to consumers. Mathematically, this system can be represented as a directed graph defined as $\mathcal{G}_f = \left\{\mathcal{N}_f, \mathcal{E}_f\right\}$ where $\mathcal{N}_f$ is the set of units within the gas system, and $ \mathcal{E}_f$ is the set of different elements linking them. This set of units includes gas supply nodes or wells $\mathcal{W} \subset \mathcal{N}_{f}$, gas demand nodes or users $\mathcal{U} \subset \mathcal{N}_{f}$, and gas storage facilities $\mathcal{S} \subset \mathcal{N}_{f}$. Similarly, the set of directed gas adjacency edges $\mathcal{A} = \left\{(n,m) \mid n,m\in\mathcal{N}_f \right\} \subset \mathcal{E}$ delineates the network structure through two kinds of transmission elements: transport pipelines $\mathcal{P} = \left\{p=(n,m) \mid n,m\in\mathcal{N}_f \right\}$ and compressing stations $\mathcal{C} = \left\{c=(n,m) \mid n,m\in\mathcal{N}_f \right\}$, so that $\mathcal{P}\cup\mathcal{C}=\mathcal{A}$ and $\mathcal{P}\cap\mathcal{C}=\emptyset$.


Natural gas transportation requires coordination to manage the flow through the different elements to maintain safe operating ranges. In optimizing this network, mathematical models minimize overall operating costs associated with the various stages of natural gas transportation, compression, storage, and handling unsupplied demand, ensuring compliance with technical and physical constraints. The function is expressed as:

\begin{equation} \label{eq:obj_func_integrated}
\begin{split}
\min_{\mathcal{P}, \mathcal{F}} \quad  \sum_{w \in \mathcal{W}} C_{w}^t {f_{w}^t} + \sum_{p \in \mathcal{P}} C_{p}^t {f_{p}^t} + \sum_{c \in \mathcal{C}} C_{c}^t {f_{c}^t} + \\ \sum_{u \in \mathcal{U}} C_{u}^{t} {f_{u}^{t}} + \quad \sum_{s \in \mathcal{S}} C_{s+}^{t} {f_{s+}^{t}}  + \sum_{s \in \mathcal{S}} C_{s-}^{t} {f_{s-}^{t}} + \sum_{s \in \mathcal{S}} C_{s}^{t} {V_{s}^{t}}
\end{split}
\end{equation}

The term $\sum_{w \in \mathcal{W}} C_{w}^t {f_{w}^t}$ represents the total cost of gas production at the wells, where $C_{w}^t$ denotes the cost per unit flow of gas at a specific well $w$ during time period $t$, and $f_{w}^t$ corresponds to the flow of gas from well $w$. Similarly, the transportation of gas through pipelines is captured by the term $\sum_{p \in \mathcal{P}} C_{p}^t {f_{p}^t}$, where $C_{p}^t$ is the cost per unit flow through pipeline $p$ during time period $t$, and $f_{p}^t$ represents the flow of gas through pipeline $p$. In addition, the total cost associated with gas compression at compressor stations is accounted for by $\sum_{c \in \mathcal{C}} C_{c}^t {f_{c}^t}$, where $C_{c}^t$ is the cost per unit flow at compressor station $c$ during time period $t$, and $f_{c}^t$ is the flow of gas through compressor station $c$.

Beyond production, transportation, and compression, the model also considers the costs related to unmet gas demand. The term $\sum_{u \in \mathcal{U}} C_{u}^{t} {f_{u}^{t}}$ reflects the penalty cost associated with unsupplied gas demand, where $C_{u}^{t}$ is the penalty cost per unit of unsupplied gas at location $u$ during time period $t$, and $f_{u}^{t}$ represents the volume of unmet demand. Additionally, the model includes costs related to storage operations. The term $\sum_{s \in \mathcal{S}} C_{s+}^{t} {f_{s+}^{t}}$ represents the cost of injecting gas into storage facilities, where $C_{s+}^{t}$ is the cost per unit flow into storage during time period $t$, and $f_{s+}^{t}$ denotes the flow into storage at facility $s$. Conversely, the cost of withdrawing gas from storage is captured by $\sum_{s \in \mathcal{S}} C_{s-}^{t} {f_{s-}^{t}}$, where $C_{s-}^{t}$ is the cost per unit flow out of storage during time period $t$, and $f_{s-}^{t}$ represents the flow out of storage at facility $s$. Finally, the model accounts for the storage cost itself through the term $\sum_{s \in \mathcal{S}} C_{s}^{t} {V_{s}^{t}}$, where $C_{s}^{t}$ is the cost per unit volume of gas stored at facility $s$ during time period $t$, and $V_{s}^{t}$ denotes the volume of gas stored. 


% Lastly, \Cref{eq:weymouth_cons}, known as the Weymouth equation, summarizes the physical behavior of gas flow through pipelines by relating the gas flow through the pipeline $f_{p}^t$ to the pressures at the ends of the pipeline $\pi_{n}^t, \pi_{m}^t \ \forall \ p = (n,m) \in\mathcal{P}$. The Weymouth equation defines a nonlinear, nonconvex, disjunctive flow-pressure relationship that hampers the optimization of the gas transport system.
% \begin{subequations}
\begin{alignat}{4}
    \underline{f_{w}^t} \leq f_{w}^t \leq \overline{f_{w}^t} &\quad \forall \ w \in \mathcal{W} \label{eq:well_limits} \\
    -\overline{f_{p}^t} \leq f_{p}^t \leq \overline{f_{p}^t} &\quad \forall \ p \in \mathcal{P} \label{eq:pipe_limits} \\
    % \underline{\pi_{n}^t} \leq \pi_{n}^t \leq \overline{\pi_{n}^t} &\quad \forall \ n \in \mathcal{N}_f \label{eq:press_limit} \\
    % \pi_{m}^t \leq \beta_{c}^t{\pi_{n}^t} &\quad \forall c=(n,m) \in \mathcal{C} \label{eq:comp_ratio} \\
    0 \leq f_{u}^{t} \leq \overline{f_{u}^{t}} &\quad \forall \ u \in \mathcal{U} \label{eq:dem_limit_gas} \\
    \sum_{m:(m,n)\in\mathcal{A}}{f_{m}^t} = \sum_{m':(n,m')\in\mathcal{A}}{f_{m'}^t} &\quad \forall \ n \in \mathcal{N}_f \label{eq:gas_balance} \\
    0 \leq f_{s+}^t \leq V_{0s} - \underline{V_s} &\quad \forall \ s \in \mathcal{S} \label{eq:sto_limit1} \\ 
    0 \leq f_{s-}^t \leq \overline{V_s} - V_{0s} &\quad \forall \ s \in \mathcal{S} \label{eq:sto_limit2} \\ 
    V_{s}^t = V_{s}^{t-1} + f_{s-}^{t-1} - f_{s+}^{t-1} &\quad \forall \ s \in \mathcal{S} \label{eq:sto_time}\\
    % sgn(f_{p}^t)(f_{p}^t)^2 = K_{nm}((\pi_{n}^t)^2-(\pi_{m}^t)^2) &\quad \forall \ p =(n,m) \in\mathcal{P} \label{eq:weymouth_cons}
\end{alignat}

The constraint set models the gas transportation system: \Cref{eq:well_limits} forces each production well to inject the flow $f_{w}^t$ over the technical minimum $\underline{f_{w}^t}$ and under the maximum capacity $\overline{f_{w}^t}$. \Cref{eq:pipe_limits} upper-bounds the gas flow through pipelines $f_{p}^t$ to the structural capacity $\overline{f_{p}^t}$. \Cref{eq:dem_limit_gas} ensures that the unsupplied demand $f_{u}^{t}$ is lower than the corresponding user demand $\overline{f_{u}^{t}}$. The nodal gas balance in \Cref{eq:gas_balance} guarantees that the gas entering the node $n$ equals the gas leaving it. \Cref{eq:sto_limit1,eq:sto_limit2} limit the gas injection $f_{s+}$  and extraction $f_{s-}$ rates at storage facilities according to the feasible operating range determined by the currently stored volume $V_{s}^t$, respectively. In turn, \Cref{eq:sto_time} balances the gas storage unit such that gas volume at operation period $t$ $V_{s}^t$ equals the volume from period $V_{s}^{t-1}$ plus the difference between injected $f_{s+}^{t-1}$ and extracted $f_{s+}^{t-1}$ gas flow, a fundamental constraint for modeling the dynamics of gas storage over time. 


